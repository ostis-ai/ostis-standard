\begin{SCn}
\scnsectionheader{\currentname}
\begin{scnsubstruct}
\begin{scnrelfromlist}{дочерний раздел}
%TODO: check by human--->
\scnitem{Предметная область и онтология действий, задач, планов, протоколов и методов, реализуемых ostis-системой, а также внутренних агентов, выполняющих эти действия}
\scnitem{Предметная область и онтология Базового языка программирования ostis-систем}
\scnitem{Предметная область и онтология искусственных нейронных сетей и соответствующая им предметная область и онтология действий по обработке искусственных нейронных сетей}
%<---TODO: check by human
\end{scnrelfromlist}
\scnheader{решатель задач ostis-системы}
\scnidtf{совокупность всех навыков, которыми обладает ostis-система на текущий момент времени}
\scnrelto{семейство подмножеств}{навык}
\scntext{note}{Предлагаемый в рамках \textit{Технологии OSTIS} подход к построению решателей задач позволяет обеспечить их модифицируемость, что, в свою очередь, позволяет \textit{ostis-системе} при необходимости легко приобретать новые \textit{навыки}, модифицировать (совершенствовать) уже имеющиеся, и даже избавляться от некоторых навыков с целью повышения производительности системы. Таким образом, имеет смысл говорить не о жестко фиксированном решателе задач, который разрабатывается один раз при создании первой версии системы и далее не меняется, а о совокупности навыков, фиксированной в каждый текущий момент времени, но постоянно эволюционирующей.}\scnsuperset{объединенный решатель задач ostis-системы}
\scnidtf{полный решатель задач ostis-системы}
\scnidtf{интегрированный решатель задач ostis-системы}
\scnidtf{решатель задач ostis-системы, реализующий все ее функциональные возможности, как основные, так и вспомогательные}
\scntext{explanation}{В общем случае \textit{объединенный решатель задач ostis-системы}, решает задачи, связанные с:
\begin{scnitemize}
\item{обеспечением основных функциональных возможностей системы (например, решение явно сформулированных задач по требованию пользователя);}
\item{обеспечением корректности и оптимизацией работы самой ostis-системы (перманентно на протяжении всего жизненного цикла ostis-системы);}
\item{обеспечением повышения квалификации конечных пользователей и разработчиков ostis-системы;}
\item{обеспечением автоматизации развития и управления развитием ostis-системы.}
\end{scnitemize}
}
\scnsuperset{гибридный решатель задач ostis-системы}
\scnidtf{решатель задач ostis-системы, реализующий две и более модели решения задач}
\scnheader{машина обработки знаний}
\scnsubset{sc-агент}
\scntext{explanation}{Под \textit{машиной обработки знаний} будем понимать совокупность интерпретаторов всех \textit{навыков}, составляющих некоторый \textit{решатель задач}. С учетом многоагентного подхода к обработке информации, используемого в рамках Технологии OSTIS, \textit{машина обработки знаний} представляет собой \textit{sc-агент} (чаще всего -- \textit{неатомарный sc-агент}), в состав которого входят более простые sc-агенты, обеспечивающие интерпретацию соответствующего множества \textit{методов}. Таким образом, \textit{машина обработки знаний} в общем случае представляет собой иерархическую систему \textit{sc-агентов}.}
\scnheader{решатель задач ostis-системы}
\scnhaselement{Решатель задач Метасистемы IMS.ostis}
\scnsuperset{решатель задач вспомогательной ostis-системы}
\scnsuperset{решатель задач интерфейса компьютерной системы}
\begin{scnsubdividing}
%TODO: check by human--->
\scnitem{решатель задач пользовательского интерфейса компьютерной системы}
\scnitem{решатель задач интерфейса компьютерной системы с другими компьютерными системами}
\scnitem{решатель задач интерфейса компьютерной системы с окружающей средой}
%<---TODO: check by human
\end{scnsubdividing}
\scnsuperset{решатель задач ostis-подсистемы поддержки проектирования компонентов определенного класса}
\scnsuperset{решатель задач ostis-подсистемы поддержки проектирования баз знаний}
\scnsuperset{решатель задач повышения качества базы знаний}
\scnsuperset{решатель задач верификации базы знаний}
\scnsuperset{решатель задач поиска и устранения некорректностей в базе знаний}
\scnsuperset{решатель задач поиска и устранения неполноты}
\scnsuperset{решатель задач оптимизации структуры базы знаний}
\scnsuperset{решатель задач выявления и устранения информационного мусора}
\scnsuperset{решатель задач ostis-подсистемы поддержки проектирования решателей задач ostis-систем}
\begin{scnsubdividing}
%TODO: check by human--->
\scnitem{решатель задач ostis-подсистемы поддержки проектирования программ обработки знаний}
\scnitem{решатель задач ostis-подсистемы поддержки проектирования агентов обработки знаний}
%<---TODO: check by human
\end{scnsubdividing}
\scnsuperset{решатель задач подсистемы управления проектирования компьютерных систем и их компонентов}
\scnsuperset{решатель задач самостоятельной ostis-системы}
\scnheader{решатель задач ostis-системы}
\scnsuperset{решатель задач с использованием хранимых методов}
\scnidtf{решатель, способный решать задачи тех классов, для которых на данный момент времени известен соответствующий метод решения}
\scnsuperset{решатель задач на основе нейросетевых моделей}
\scnsuperset{решатель задач на основе генетических алгоритмов}
\scnsuperset{решатель задач на основе императивных программ}
\scnsuperset{решатель задач на основе процедурных программ}
\scnsuperset{решатель задач на основе объектно-ориентированных программ}
\scnsuperset{решатель задач на основе декларативных программ}
\scnsuperset{решатель задач на основе логических программ}
\scnsuperset{решатель задач на основе функциональных программ}
\scnsuperset{решатель задач в условиях, когда метод решения задач данного класса в текущий момент времени не известен}
\scnidtf{решатель, реализующий стратегии решения задач, позволяющие породить метод решения задачи, который в текущий момент времени не известен ostis-системе}
\scnidtf{решатель, использующий для решения задач метаметоды, соответствующие более общим классам задач по отношению к заданной}
\scnidtf{решатель задач, позволяющий породить метод, который является частным по отношению какому-либо известному ostis-системе методу и интерпретируется соответствующей машиной обработки знаний}
\scnsuperset{решатель, реализующий стратегию поиска путей решения задачи в глубину}
\scnsuperset{решатель, реализующий стратегию поиска путей решения задачи в ширину}
\scnsuperset{решатель, реализующий стратегию проб и ошибок}
\scnsuperset{решатель, реализующий стратегию разбиения задачи на подзадачи}
\scnsuperset{решатель, реализующий стратегию решения задач по аналогии}
\scnsuperset{решатель, реализующий концепцию интеллектуального пакета программ}
\scnheader{машина обработки знаний}
\scnsuperset{машина логического вывода}
\scnsuperset{машина дедуктивного вывода}
\scnsuperset{машина прямого дедуктивного вывода}
\scnsuperset{машина обратного дедуктивного вывода}
\scnsuperset{машина индуктивного вывода}
\scnsuperset{машина абдуктивного вывода}
\scnsuperset{машина нечеткого вывода}
\scnsuperset{машина вывода на основе логики умолчаний}
\scnsuperset{машина логического вывода с учетом фактора времени}
\scnheader{решатель задач ostis-системы}
\scnsuperset{решатель задач информационного поиска}
\begin{scnsubdividing}
%TODO: check by human--->
\scnitem{решатель задач поиска информации, удовлетворяющей заданным критериям}
\scnitem{решатель задач поиска информации, не удовлетворяющей заданным критериям}
%<---TODO: check by human
\end{scnsubdividing}
\scnsuperset{решатель явно сформулированных задач}
\scnidtf{решатель задач, для которых явно сформулирована цель}
\scnsuperset{решатель задач поиска или вычисления значений заданного множества величин}
\scnsuperset{решатель задач установления истинности заданного логического высказывания в рамках заданной формальной теории}
\scnsuperset{решатель задач формирования доказательства заданного высказывания в рамках заданной формальной теории}
\scnsuperset{машина верификации ответа на указанную задачу}
\scnsuperset{машина верификации решения указанной задачи}
\scnsuperset{машина верификации доказательства заданного высказывания в рамках заданной формальной теории}
\scnsuperset{решатель задач классификации сущностей}
\scnsuperset{машина соотнесения сущности с одним из заданного множества классов}
\scnsuperset{машина разделения множества сущностей на классы по заданному множеству признаков}
\scnsuperset{решатель задач синтеза информационных конструкций}
\scnsuperset{решатель задач синтеза естественно-языковых текстов}
\scnsuperset{решатель задач синтеза изображений}
\scnsuperset{решатель задач синтеза сигналов}
\scnsuperset{решатель задач синтеза речи}
\scnsuperset{решатель задач анализа информационных конструкций}
\scnsuperset{решатель задач анализа естественно-языковых текстов}
\scnsuperset{решатель задач понимания естественно-языковых текстов}
\scnsuperset{решатель задач верификации естественно-языковых текстов}
\scnsuperset{решатель задач анализа изображений}
\scnsuperset{решатель задач сегментации изображений}
\scnsuperset{решатель задач понимания изображений}
\scnsuperset{решатель задач анализа сигналов}
\scnsuperset{решатель задач анализа речи}
\scnsuperset{решатель задач понимания речи}
\bigskip
\end{scnsubstruct}
\end{SCn}
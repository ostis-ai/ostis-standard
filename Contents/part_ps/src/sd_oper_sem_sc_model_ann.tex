\usepackage{scn}
\begin{SCn}
    \scnsectionheader{Предметная область и онтология операционной семантики sc-моделей искусственных нейронных сетей}
    \begin{scnsubstruct}

        \scnsegmentheader{Операционная семантика sc-моделей искусственных нейронных сетей, используемых в ostis-системах}
        \begin{scnsubstruct}

            \scnheader{Предметная область нейросетевых методов}
            \scnidtf{Предметная область искусственных нейронных сетей}
            \begin{scnrelfromset}{дочерняя предметная область}
                \scnitem{Предметная область нейросетевых методов SCP}
                \scnitem{Предметная область нейросетевых методов Python}
                \scnitem{Предметная область нейросетевых методов C++}
            \end{scnrelfromset}
            \scniselement{предметная область}
            \begin{scnhaselementrole}{максимальный класс объектов исследования}
            {нейросетевой метод}
            \end{scnhaselementrole}
            \begin{scnhaselementrolelist}{класс объектов исследования}
                \scnitem{нейросетевой метод}
                \scnitem{действие интерпретации нейросетевого метода}
                \scnitem{ориентированное множество чисел}
                \scnitem{матрица}
                \scnitem{действие вычисления взвешенной суммы всех нейронов слоя}
                \scnitem{действие вычисления функции активации всех нейронов слоя}
                \scnitem{действие интерпретации сверточного слоя}
                \scnitem{действие интерпретации пулинг слоя}
            \end{scnhaselementrolelist}

            \scnheader{нейросетевой метод}
            %\scndefinition{\textbf{\textit{нейросетевой метод}} --- метод }
            \scntext{примечание}{в случае описания \textbf{\textit{нейросетевого метода}} на внешнем языке, такой метод описывается в соответствующей предметной области, в рамках которой также специфицируется \textbf{\textit{действие интерпретации}} данного \textbf{\textit{нейросетевого метода}}}
            \begin{scnindent}
            \scntext{примечание}{внешний язык --- язык, который не является внутренним языком ostis-системы, т.е. не Язык SCP}
            \end{scnindent}

            \scnheader{действие интерпретации нейросетевого метода}
            \scntext{примечание}{действию соответствует агент, реализованный на соответствующем языке программирования}
                \begin{scnrelfromset}{декомпозиция}
                    \scnitem{действие интерпретации слоя и.н.с.} %возможно тут включение надо, а не декомпозиция или что-то типо того
                    \begin{scnindent}
                        \begin{scnrelfromset}{декомпозиция}
                            \scnitem{\textbf{действие вычисления взвешенной суммы всех нейронов слоя}}
                            \scnitem{\textbf{действие вычисления функции активации всех нейронов слоя}}
                            \scnitem{\textbf{действие интерпретации сверточного слоя}}
                            \scnitem{\textbf{действие интерпретации пулинг слоя}}
                        \end{scnrelfromset}
                    \end{scnindent}
                \end{scnrelfromset}


            \scnheader{ориентированное множество чисел}
            \scnidtf{ормножество чисел}
            \scnrelto{включение}{число}
            \scnrelto{включение}{ориентированное множество}
            \scnrelto{первый домен}{строковое представление ормножества чисел*}

            \scnheader{матрица}
            \scndefinition{\textbf{\textit{матрица}} --- \textit{ориентированное множество} \textit{ориентированных множеств} чисел равной мощности }

            \scnheader{действие вычисления взвешенной суммы всех нейронов слоя}
            \begin{scnrelfromset}{отношения характеризующие действие}
                \scnitem{входной вектор\scnrolesign}
                \begin{scnindent}
                    \scniselement{ролевое отношение}
                    \scnrelfrom{первый домен}{действие по обработке и.н.с.}
                    \scnrelfrom{второй домен}{матрица}
                \end{scnindent}
                \scnitem{матрица весовых коэффициентов нейронов слоя\scnrolesign}
                \begin{scnindent}
                    \scniselement{ролевое отношение}
                    \scnrelfrom{первый домен}{действие по обработке и.н.с.}
                    \scnrelfrom{второй домен}{матрица}
                \end{scnindent}
                \scnitem{результат\scnrolesign}
                \begin{scnindent}
                    \scniselement{ролевое отношение}
                    \scnrelfrom{первый домен}{действие по обработке и.н.с.}
                    \scnrelfrom{второй домен}{взвешенная сумма нейронов соответствующего слоя}
                    \begin{scnindent}
                    \scnsubset{\textbf{\textit{ориентированное множество чисел}}} %или включение или принадлежность, не уверен
                    \end{scnindent}
                \end{scnindent}
            \end{scnrelfromset}
            \scnrelfrom{описание примера}{\scnfileimage[40em]{Contents/part_ps/src/images/sd_ps/sd_ann/action_weighted_sum.png}}
            \begin{scnindent}
            \scntext{примечание}{Пример спецификации действия вычисления взвешенной суммы всех нейронов слоя для слоя с двумя нейронами и входным вектором размерностью 2}
            \end{scnindent}

            \bigskip
        \end{scnsubstruct}

        \scnendsegmentcomment{ Операционная семантика sc-моделей искусственных нейронных сетей, используемых в ostis-системах}

        \bigskip
    \end{scnsubstruct}

    \scnendcurrentsectioncomment

\end{SCn}
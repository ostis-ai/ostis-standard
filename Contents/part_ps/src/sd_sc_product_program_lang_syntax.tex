\begin{SCn}
\scnsectionheader{Предметная область и онтология синтаксиса sc-языков продукционного программирования}
\begin{scnsubstruct}
\scniselement{раздел базы знаний}
\scnhaselementrole{ключевой sc-элемент}{Предметная область синтаксиса sc-языков продукционного программирования}

\scnheader{Предметная область синтаксиса sc-языков продукционного программирования}
\scniselement{предметная область}
\begin{scnhaselementrolelist}{максимальный класс объектов исследования}
    \scnitem{язык продукционного программирования}
    \scnitem{продукция}
\end{scnhaselementrolelist}

\begin{scnrelfromlist}{соавтор}
    \scnitem{Орлов М.К.}
    \scnitem{Зотов Н.В.}
\end{scnrelfromlist}

\begin{scnrelfromvector}{библиография}
    \scnitem{\scncite{AIHandbookMM}}
\end{scnrelfromvector}

\scnheader{продукция}
\scnnote{В ряде интеллектуальных систем используются комбинации сетевых и продукционных моделей представления знаний. В таких моделях декларативные знания описываются в сетевом компоненте модели, а процедурные знания --- в продукционном. В этом случае говорят о работе продукционной системы над семантической сетью. Процедурные знания позволяют системе узнать, как можно использовать те или иные декларативные знания, в частности, знания о закономерностях той части действительности, в которой \scnqq{живет} интеллектуальная система, для получения нужных системе результатов или тех результатов, которые ожидает от нее пользователь.}
\scnexplanation{В общем виде под продукцией понимается выражение следующего вида: ($i$); $Q$; $P$; $A$ $\Rightarrow$ $B$; $N$.}
\begin{scnindent}
    \scntext{подстрока}{i}
    \begin{scnindent}
        \scnexplanation{$i$ --- имя продукции, с помощью которого данная продукция выделяется из всего множества продукций. В качестве имени может выступать некоторая лексема, отражающая суть данной продукции (например, \scnqq{покупка книги} или \scnqq{набор кода замка}), или порядковый номер продукции в их множестве, хранящемся в памяти системы. }
    \end{scnindent}
    \scntext{подстрока}{Q}
    \begin{scnindent}
        \scnexplanation{Элемент $Q$ характеризует сферу применения продукции или же контекст.}
    \end{scnindent}
    \scntext{подстрока}{P}
    \begin{scnindent}
        \scnexplanation{Элемент $P$ есть условие применимости ядра продукции. Обычно $P$ представляет собой логическое выражение (как правило, предикат). Когда $P$ принимает значение \scnqqi{истина}, ядро продукции активизируется. Если $P$ ложно, то ядро продукции не может быть использовано.}
        \begin{scnindent}
            \scnnote{Если в продукции \scnqqi{НАЛИЧИЕ ДЕНЕГ; ЕСЛИ ХОЧЕШЬ КУПИТЬ ВЕЩЬ X, ТО ЗАПЛАТИ В КАССУ ЕЕ СТОИМОСТЬ И ОТДАЙ ЧЕК ПРОДАВЦУ} условие применимости ядра продукции ложно, то есть денег нет, то применить ядро продукции невозможно.}
        \end{scnindent}
    \end{scnindent}
    \scntext{подстрока}{$A$ $\Rightarrow$ $B$}
    \begin{scnindent}
        \scnexplanation{$A$ $\Rightarrow$ $B$ --- ядро продукции. Интерпретация ядра продукции может быть различной и зависит от того, что стоит слева и справа от знака секвенции $\Rightarrow$. Обычное прочтение ядра продукции выглядит так: ЕСЛИ $A$, ТО $B$, более сложные конструкции ядра допускают в правой части альтернативный выбор, например, ЕСЛИ $A$, ТО $B_1$ ИНАЧЕ $B_2$.}
        \scnexplanation{Секвенция $\Rightarrow$ может истолковываться в обычном логическом смысле как знак логического следования $B$ из истинного $A$ (если $A$ не является истинным выражением, то о $B$ ничего сказать нельзя). Возможны и другие интерпретации ядра продукции, например $A$ описывает некоторое условие, необходимое для того, чтобы можно было совершить действие $B$.}
    \end{scnindent}
    \scntext{подстрока}{N}
    \begin{scnindent}
        \scnexplanation{Элемент $N$ описывает постусловия продукции. Они актуализируются только в том случае, если ядро продукции реализовалось. Постусловия продукции описывают действия и процедуры, которые необходимо выполнить после реализации В. Выполнение $N$ может происходить не сразу после реализации ядра продукции.}
        \begin{scnindent}
            \scnnote{После покупки некоторой вещи в магазине необходимо в описи товаров, имеющихся в этом магазине, уменьшить количество вещей такого типа на единицу.}
        \end{scnindent}
        \scnrelfrom{смотрите}{\scncite{AIHandbookMM}}
    \end{scnindent}
\end{scnindent}
\scnnote{Если в памяти системы хранится некоторый набор продукций, то они образуют систему продукций. В системе продукций должны быть заданы специальные процедуры управления продукциями, с помощью которых происходит актуализация продукций и выбор для выполнения той или иной продукции из числа актуализированных.}

\scnheader{программа на продукционном языке программирования}
\scnsubset{метод}
\begin{scnrelfromset}{обобщенная декомпозиция}
    \scnitem{продукция}
    \scnitem{определение глобальной переменной}
    \scnitem{оператор управления макрогенерацией}
    \scnitem{вставка текста на алгоритмическом языке программирования}
\end{scnrelfromset}
\begin{scnindent}
    \scnnote{Основной единицей продукционного языка программирования, определяющей его лицо и его возможности, является продукция.}
\end{scnindent}

\bigskip
\end{scnsubstruct}
\scnendsegmentcomment{Предметная область и онтология синтаксиса sc-языков продукционного программирования}

\end{SCn}

\begin{SCn}
\scnsectionheader{Предметная область и онтология sc-языков продукционного программирования}
\begin{scnsubstruct}

\scnheader{Предметная область sc-языков продукционного программирования}
\scniselement{предметная область}
\begin{scnhaselementrolelist}{максимальный класс объектов исследования}
    \scnitem{язык продукционного программирования}
    \scnitem{продукция}
\end{scnhaselementrolelist}

\begin{scnrelfromlist}{соавтор}
    \scnitem{Орлов М.К.}
    \scnitem{Зотов Н.В.}
\end{scnrelfromlist}

\scnheader{продукция}
\scniselement{средство представления знаний}
\scnexplanation{Продукции наряду с фреймами являются наиболее популярными средствами представления знаний в интеллектуальных компьютерных системах.}
\begin{scnrelfromlist}{преимущества}
    \scnfileitem{Продукции близки к  логическим моделям, что позволяет организовывать на них эффективные процедуры вывода.}
    \scnfileitem{Продукции более наглядно отражают знания, чем классические логические модели.}
    \scnfileitem{В продукциях отсутствуют жесткие ограничения, характерные для логических исчислений, что дает возможность изменять интерпретацию элементов продукции.}
\end{scnrelfromlist}
\scnnote{Так как в основе \textit{Технологии OSTIS} лежит \textit{многоагентный подход}, который позволяет легко интерпретировать любую информацию, то можно сделать вывод, что продукционный подход легко интегрируется в \textit{решатели задач} \textit{ostis-систем} в виде \textit{sc-агентов}.  Интеграция \textit{Технологии OSTIS} и продукционного подхода позволяет объединить статические и динамические знания в рамках единого формализма, а также получить их графическое представление.}

\bigskip
\end{scnsubstruct}
\scnendsegmentcomment{Предметная область и онтология sc-языков продукционного программирования}

\end{SCn}

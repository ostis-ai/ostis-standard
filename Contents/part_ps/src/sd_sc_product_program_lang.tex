\begin{SCn}
\scnsectionheader{Предметная область и онтология sc-языков продукционного программирования}
\begin{scnsubstruct}

\scnheader{Предметная область sc-языков продукционного программирования}
\scniselement{предметная область}
\begin{scnrelfromlist}{соавтор}
    \scnitem{Орлов М. К.}
    \scnitem{Зотов Н. В.}
\end{scnrelfromlist}

\scnheader{продукция}
\scniselement{средство представления знаний}
\begin{scnrelfromset}{характеристики}
    \scnitem{близкость к логическим моделям}
    \begin{scnindent}
        \scnrelto{условие}{организация логического вывода на продукциях}
    \end{scnindent}
    \scnitem{наглядное отображение знаний}
    \begin{scnindent}
        \scnrelfrom{более наглядный}{наглядность отображения знаний логических моделей}
    \end{scnindent}
    \scnitem{отсутствие жестких ограничений, характерных для логических исчислений}
    \begin{scnindent}
        \scnrelto{условие}{изменение интерпретации элементов продукции}
    \end{scnindent}
\end{scnrelfromset}
\scnrelto{объект интеграции}{интеграция продукционного подхода}
\begin{scnindent}
    \scnrelto{простота интеграции}{решатель задач ostis-системы}
    \scnrelfrom{условие интеграции}{многоагентный подход Технологии OSTIS}
    \scnrelfrom{форма интеграции}{sc-агент}
    \scnnote{Интеграция \textit{Технологии OSTIS} и продукционного подхода позволяет объединить статические и динамические знания в рамках единого формализма, а также получить их графическое представление.}
\end{scnindent}

\bigskip
\end{scnsubstruct}
\scnendsegmentcomment{Предметная область и онтология sc-языков продукционного программирования}

\end{SCn}

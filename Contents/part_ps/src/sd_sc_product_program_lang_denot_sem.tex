\begin{SCn}
\scnsectionheader{Предметная область и онтология денотационной семантики sc-языков продукционного программирования}
\begin{scnsubstruct}

\scnheader{Предметная область денотационной семантики sc-языков продукционного программирования}
\scniselement{предметная область}
\begin{scnrelfromlist}{соавтор}
    \scnitem{Орлов М. К.}
    \scnitem{Зотов Н. В.}
\end{scnrelfromlist}

\begin{scnrelfromvector}{библиография}
    \scnitem{\scncite{Brownston1985}}
    \scnitem{\scncite{CLIPS}}
\end{scnrelfromvector}

\scnheader{язык продукционного программирования}
\scnhaselement{OPS5}
\begin{scnindent}
    \scnrelfrom{база данных}{база данных языка OPS5}
    \begin{scnindent}
        \scnidtf{рабочая память}
        \scnidtf{working memory}
        \scnrelfrom{составная часть}{объект}
        \begin{scnindent}
            \scnrelfrom{составная часть}{атрибут}
            \scnnote{Объект вместе с парами <атрибут -- значение> называется элементом рабочей памяти.}
            \scnrelfrom{библиографический источник}{\scncite{Brownston1985}}
        \end{scnindent}
        \scnrelfrom{составная часть}{явные данные структуры управления}
        \begin{scnindent}
            \scnrelfrom{вмд представления}{экземпляр целевой структуры данных}
        \end{scnindent}
    \end{scnindent}
    \scnrelfrom{разработчики}{разработчики OPS5}
    \begin{scnindent}
        \scnrelfrom{поставленная задача}{максимально высокая эффективность продукционной программы}
    \end{scnindent}
    \scnrelfrom{интерпретатор}{интерпретатор OPS5}
    \begin{scnindent}
        \scnrelfrom{порождаемое множество}{конфликтное множество}
        \begin{scnindent}
            \scnrelfrom{элемент множества}{элемент конфликтного множества}
            \begin{scnindent}
                \scnrelfrom{элемент множества}{элемент конфликтного множества}
                    \begin{scnindent}
                        \scnhaselementrole{1}{имя продукции}
                        \scnhaselementrole{2}{список элементов рабочей памяти, которые являются означиваниями для образцов продукции}
                    \end{scnindent}
            \end{scnindent}
        \end{scnindent}
    \end{scnindent}
    \scnrelfrom{продукция}{продукция OPS5}
    \begin{scnindent}
        \scnhaselementrole{1}{символ P}
        \scnhaselementrole{2}{имя продукции}
        \scnhaselementrole{3}{левая часть продукции}
        \scnhaselementrole{4}{символа -->}
        \scnhaselementrole{5}{правая часть продукции}
        \scnnote{Связывание переменных, возникающее в результате сопоставления с шаблоном в левой части, используется в правой части для ссылки на данные, подлежащие модификации}
    \end{scnindent}
    \begin{scnrelfromset}{выделенные типы действия}
        \scnitem{действие MAKE}
        \begin{scnindent}
            \scnnote{создает новый элемент рабочей памяти}
        \end{scnindent}
        \scnitem{действие MODIFY}
        \begin{scnindent}
            \scnnote{изменяет один или несколько значений атрибутов у существующего элемента рабочей памяти}
        \end{scnindent}
        \scnitem{действие REMOVE}
        \begin{scnindent}
            \scnnote{удаляет элемент рабочей памяти}
        \end{scnindent}
    \end{scnrelfromset}
    \begin{scnrelfromlist}{пример правила}
        \scnheader{Рисунок. Пример правила продукции на OPS5}
        \scneqimage[30em]{Contents/part_ps/src/images/sd_sc_product_program_lang_denot_sem/ops5_production_rule_example.png}
    \end{scnrelfromlist}
    \begin{scnindent}
        \scnnote{Данные в рабочей памяти структурированы и переменные появляются в угловых скобках. Название структуры данных, такое как "goal"{} (цель) и "physical-object"{} (физический объект), является первым буквальным в условиях; поля структуры начинаются с "\textasciicircum"{}. На негативное состояние  указывает "{}--"{}.}
        \scnnote{Если несколько объектов подвешены к потолку, каждый с другой лестницей рядом, поддерживающей обезьяну с пустыми руками, конфликтный набор будет содержать столько же продукций правил продукции, полученных из одной и той же продукции "Holds::Object-Ceiling"{}. На этапе разрешения конфликта позже будет выбрано, какие продукции запускать.}
        \scnnote{Когда обезьяна держит подвешенный объект, статус цели устанавливается на «удовлетворено», и то же производственное правило больше не может применяться, поскольку его первое условие не выполняется.}
    \end{scnindent}
    \scnrelfrom{область применения продукционного правила}{все продукции структур данных, которые соответствуют условиям и соответствуют привязкам переменных}
\end{scnindent}

\scnheader{CLIPS}
\scnrelfrom{используемая модель представления знаний}{продукционная модель представления знаний}
\begin{scnrelfromlist}{составной элемент}
    \scnitem{база фактов CLIPS}
    \begin{scnindent}
        \scnidtf{fact base}
        \scnnote{База фактов представляет исходное описание задачи.}
    \end{scnindent}
    \scnitem{база правил CLIPS}
    \begin{scnindent}
        \scnidtf{rule base}
        \scnnote{База правил содержит операторы, которые преобразуют состояния проблемы, приводя его к решению --- целевому состоянию.}
    \end{scnindent}
    \scnitem{механизм логического вывода CLIPS}
    \begin{scnindent}
        \scnrelfrom{алгоритм работы}{сопоставление фактов из базы фактов и правил из базы правил и выяснение, какие из правил можно активизировать}
        \begin{scnindent}
            \scnrelfrom{режим выполнения}{цикличное выполнение}
            \scnrelfrom{составной элемент}{цикл алгоритма механизма логического вывода CLIPS}
                \begin{scnindent}
                    \scnidtf{продукционный цикл}
                    \scnidtf{цикл распознавания действия}
                    \begin{scnrelfromset}{фазы цикла}
                        \scnitem{сопоставление фактов и правил}
                        \scnitem{выбор правила, подлежащего активизации}
                        \scnitem{выполнение действий, предписанных активным правилом}
                    \end{scnrelfromset}
                \end{scnindent}
        \end{scnindent}
    \end{scnindent}
\end{scnrelfromlist}
\scnrelfrom{форма представления информации}{факт CLIPS}
\begin{scnindent}
    \scnnote{Каждый факт представляет фрагмент информации, который был помещен в текущий список фактов, называемый fact-list.}
    \scnrelto{использует}{правило CLIPS}
    \scnrelfrom{свойство}{уникальность факта CLIPS}
    \begin{scnindent}
        \scnnote{Если при добавлении нового факта к списку обнаруживается, что он полностью совпадает с одним из уже включенных в список фактов, то эта операция игнорируется.}
    \end{scnindent}
    \begin{scnrelfromset}{формат записи}
        \scnitem{запись индексом}
        \begin{scnindent}
            \scnnote{Всякий раз, когда факт добавляется (изменяется), ему присваивается уникальный целочисленный индекс.}
        \end{scnindent}
        \scnitem{запись адресом}
    \end{scnrelfromset}
    \scnrelfrom{идентификатор факта}{идентификатор факта CLIPS}
    \begin{scnindent}
        \scnnote{Идентификатор факта --- это короткая запись для отображения факта на экране. Она состоит из символа f и записанного через тире индекса факта.}
    \end{scnindent}
    \begin{scnrelfromset}{формат представления}
        \scnitem{позиционное представление}
        \begin{scnindent}
            \scnnote{Позиционные факты состоят из выражения символьного типа, за которым следует последовательность (возможно, пустая) из полей, разделенных пробелами. Вся запись заключается в скобки. Обычно первое поле определяет "отношение"{}, которое применяется к оставшимся полям.}
        \end{scnindent}
        \scnitem{позиционное представление}
    \end{scnrelfromset}
\end{scnindent}
\scnrelfrom{библиографический источник}{\scncite{CLIPS}}

\bigskip
\end{scnsubstruct}
\scnendsegmentcomment{Предметная область и онтология денотационной семантики sc-языков продукционного программирования}

\end{SCn}

\begin{SCn}
\scnsectionheader{Предметная область и онтология денотационной семантики sc-языков продукционного программирования}
\begin{scnsubstruct}

\scnheader{Предметная область денотационной семантики sc-языков продукционного программирования}
\scniselement{предметная область}
\begin{scnhaselementrolelist}{максимальный класс объектов исследования}
    \scnitem{язык продукционного программирования}
    \scnitem{продукция}
\end{scnhaselementrolelist}

\begin{scnrelfromlist}{соавтор}
    \scnitem{Орлов М.К.}
    \scnitem{Зотов Н.В.}
\end{scnrelfromlist}

\begin{scnrelfromvector}{библиография}
    \scnitem{\scncite{Brownston1985}}
    \scnitem{\scncite{CLIPS}}
\end{scnrelfromvector}

\scnheader{OPS5}
\scniselement{язык продукционного программирования}
\scnnote{База данных в языке называется рабочей памятью (working memory) и состоит из нескольких сотен объектов, каждый из которых имеет свой набор атрибутов. Объект вместе с парами <атрибут -- значение> называется элементом рабочей памяти.}
\scnrelfrom{смотрите}{\scncite{Brownston1985}}
\scnnote{Основная задача, которую поставили перед собой разработчики языка \textit{OPS5}, добиться максимально высокой эффективности выполнения продукционной программы. Интерпретатор системы порождает конфликтное множество, каждый элемент которого представляет собой пару <имя продукции, список элементов рабочей памяти, которые являются означиваниями для образцов продукции>. Каждая продукция в \textit{OPS5} состоит из символа Р, имени продукции, левой части, символа --> и правой части.}
\scnnote{В \textit{OPS5} выделены три типа действий:
 \begin{itemize}
    \item{MAKE --- создает новый элемент рабочей памяти;}
    \item{MODIFY --- изменяет один или несколько значений атрибутов у существующего элемента рабочей памяти;}
    \item{REMOVE --- удаляет элемент рабочей памяти.}
 \end{itemize}}
\scnrelfrom{изображение}{Рисунок. Пример правила продукции на OPS5}
\begin{scnindent}
    \scneqimage[30em]{Contents/part_ps/src/images/sd_sc_product_program_lang_denot_sem/ops5_production_rule_example.png}
    \scnnote{Данные в рабочей памяти структурированы и переменные появляются в угловых скобках. Название структуры данных, такое как \scnqq{goal} (цель) и \scnqq{physical-object} (физический объект), является первым буквальным в условиях; поля структуры начинаются с \scnqq{\textasciicircum}. На негативное состояние указывает \scnqq{--}.}
    \scnnote{Если несколько объектов подвешены к потолку, каждый с другой лестницей рядом, поддерживающей обезьяну с пустыми руками, конфликтный набор будет содержать столько же продукций правил продукции, полученных из одной и той же продукции \scnqq{Holds::Object-Ceiling}. На этапе разрешения конфликта позже будет выбрано, какие продукции запускать.}
    \scnnote{Когда обезьяна держит подвешенный объект, статус цели устанавливается на \scnqqi{удовлетворено}, и то же производственное правило больше не может применяться, поскольку его первое условие не выполняется.}
\end{scnindent}
\scnnote{Продукционные правила в \textit{OPS5} применяются ко всем продукциям структур данных, которые соответствуют условиям и соответствуют привязкам переменных.}

\scnheader{CLIPS}
\scnexplanation{\textbf{\textit{CLIPS}} использует продукционную модель представления знаний и поэтому содержит три основных элемента:
\begin{itemize}
    \item{база фактов (fact base);}
    \item{база правил (rule base);}
    \item{механизм логического вывода.}
\end{itemize}
База фактов представляет исходное описание задачи. База правил содержит операторы, которые преобразуют состояния проблемы, приводя его к решению --- целевому состоянию.}
\scnnote{Механизм логического вывода \textit{CLIPS} сопоставляет факты из базы фактов и правила из базы правил и выясняет, какие из правил можно активизировать. Это выполняется циклически, причем каждый цикл (так называемый продукционный цикл или цикл распознавания действия) состоит из трех основных фаз:
\begin{itemize}
    \item{сопоставление фактов и правил;}
    \item{выбор правила, подлежащего активизации;}
    \item{выполнение действий, предписанных активным (\scnqqi{зажженным}) правилом.}
\end{itemize}}
\scnnote{Факты --- это одна из основных форм представления информации в системе \textit{CLIPS}. Каждый факт представляет фрагмент информации, который был помещен в текущий список фактов, называемый fact-list. Факт представляет собой основную единицу данных, используемую правилами.

Если при добавлении нового факта к списку обнаруживается, что он полностью совпадает с одним из уже включенных в список фактов, то эта операция игнорируется.

Факт может описываться индексом или адресом. Всякий раз, когда факт добавляется (изменяется), ему присваивается уникальный целочисленный индекс. Факт также может задаваться при помощи адреса.

Идентификатор факта --- это короткая запись для отображения факта на экране. Она состоит из символа f и записанного через тире индекса факта. Существует два формата представления фактов: позиционный и непозиционный. Позиционные факты состоят из выражения символьного типа, за которым следует последовательность (возможно, пустая) из полей, разделенных пробелами. Вся запись заключается в скобки. Обычно первое поле определяет \scnqq{отношение}, которое применяется к оставшимся полям.}
\scnrelfrom{смотрите}{\scncite{CLIPS}}

\bigskip
\end{scnsubstruct}
\scnendsegmentcomment{Предметная область и онтология денотационной семантики sc-языков продукционного программирования}

\end{SCn}

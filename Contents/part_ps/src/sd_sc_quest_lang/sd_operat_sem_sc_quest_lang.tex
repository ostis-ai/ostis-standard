\begin{SCn}

\scnsectionheader{Предметная область и онтология операционной семантики Языка вопросов для ostis-систем}
\begin{scnsubstruct}
    \scniselement{раздел базы знаний}
    \scnhaselementrole{ключевой sc-элемент}{Предметная область операционной семантики Языка вопросов для ostis-систем}

    \scnheader{Предметная область операционной семантики Языка вопросов для ostis-систем}
	\scniselement{предметная область}
    \begin{scnhaselementrolelist}{класс объектов исследования}
        \scnitem{вопрос}
        \scnitem{ответ на вопрос}
        \scnitem{знак в рамках заданного вопроса}
        \scnitem{основной знак в рамках заданного вопроса}
        \scnitem{неосновной знак в рамках заданного вопроса}
        \scnitem{отношение в рамках заданного вопроса}
        \scnitem{базовое отношение в рамках заданного вопроса}
    \end{scnhaselementrolelist}
   
    \scnheader{вопрос}
    \scntext{примечание}{Каждому классу \textit{вопросов} должен соответствовать определенный \textit{коллектив sc-агентов}, реализующих поиск или синтез из \textit{базы знаний} \textit{ostis-системы} соответствующих ответов на поставленные \textit{вопросы}. Следует отметить, что в зависимости от степени наполненности \textit{базы знаний} \textit{ответы} могут содержаться в \textit{базе знаний} либо отсутствовать в текущей версии \textit{базы знаний}. В случае наличия в \textit{базе знаний} \textit{ответа на} поставленный \textit{вопрос} информационная потребность пользователя реализуется \textit{информационно-поисковыми sc-агентами}, в противном случае --- в зависимости от \textit{классов вопросов} формирование ответов осуществляется специализированными \textit{sc-агентами}, которые в процессе работы дополнительно выполняют вычислительные задачи либо осуществляют синтез на основе \textit{логического вывода} или других \textit{моделей решения задач}.} 
    \begin{scnindent}
    	\begin{scnrelfromset}{смотрите}
    		\scnitem{Смысловое представление логических формул и высказываний в различного вида логиках}
    	\end{scnrelfromset}
    \end{scnindent}
    
    \scnheader{интерпретатор Языка вопросов для ostis-систем}
    \scniselement{неатомарный sc-агент}
    \begin{scnrelfromset}{декомпозиция абстрактного sc-агента}
        \scnitem{Абстрактный sc-агент поиска ответа на заданный вопрос}
        \begin{scnindent}
            \begin{scnrelfromset}{декомпозиция абстрактного sc-агента}
                \scnitem{Абстрактный sc-агент поиска семантической окрестности \textit{основного знака}}
                \scnitem{Абстрактный sc-агент поиска ответа на вопрос, требующий раскрытия в ответе \textit{отношения состава} для \textit{основного знака}}
                \scnitem{Абстрактный sc-агент поиска ответа на вопрос, требующий раскрытия в ответе \textit{теоретико-множественного отношения} для \textit{основного знака}}
                \scnitem{Абстрактный sc-агент поиска ответа на вопрос, требующий раскрытия в ответе \textit{отношения состояния} для \textit{основного знака}}	
                \scnitem{Абстрактный sc-агент поиска ответа на вопрос, требующий раскрытия в ответе \textit{отношения действия} для \textit{основного знака}}	
                \scnitem{Абстрактный sc-агент поиска ответа на вопрос, требующий раскрытия в ответе \textit{темпорального отношения} для \textit{основного знака}}
                \scnitem{Абстрактный sc-агент поиска ответа на вопрос, требующий раскрытия в ответе \textit{пространственного отношения} для \textit{основного знака}}
                \scnitem{Абстрактный sc-агент поиска ответа на вопрос, требующий раскрытия в ответе \textit{количественного отношения} для \textit{основного знака}}
                \scnitem{Абстрактный sc-агент поиска ответа на вопрос, требующий раскрытия в ответе \textit{качественного отношения} для \textit{основного знака}}
                \scnitem{Абстрактный sc-агент поиска ответа на вопрос, требующий раскрытия в ответе \textit{отношения описания} для \textit{основного знака}}
                \scnitem{Абстрактный sc-агент поиска ответа на вопрос, требующий раскрытия в ответе \textit{отношения определения} для \textit{основного знака}}
                \scnitem{Абстрактный sc-агент поиска ответа на вопрос, требующий раскрытия в ответе \textit{отношения причины} для \textit{основного знака}}
                \scnitem{Абстрактный sc-агент поиска ответа на вопрос, требующий раскрытия в ответе \textit{отношения следствия} для \textit{основного знака}}
                \scnitem{Абстрактный sc-агент поиска ответа на вопрос, требующий раскрытия в ответе \textit{отношения детализации} для \textit{основного знака}}
            \end{scnrelfromset}
        \end{scnindent}
        \scnitem{Абстрактный sc-агент синтеза ответа на заданный вопрос}
    \end{scnrelfromset}
    \scntext{примечание}{Все \textit{sc-агенты}, выводящие \textit{ответы на} поставленные \textit{вопросы}, формируют \textit{коллектив sc-агентов} --- \textbf{\textit{интерпретатор Языка вопросов для ostis-систем}}, с помощью которого можно интерпретировать любые классы \textit{вопросов}. \textit{интерпретатор Языка вопросов для ostis-систем} может быть реализован по-разному: в виде \textit{коллектива scp-агентов} или \textit{платформенно-зависимых sc-агентов}.}

\end{scnsubstruct}

\scntext{заключение}{Перечислим основные положения:
\begin{itemize}
    \item информационная потребность \textit{пользователей ostis-системы} может быть выражена в виде \textit{вопросов}, а удовлетворение этой информационной потребности --- в виде \textit{ответов на} заданные \textit{вопросы};
    \item вывод \textit{ответов на} заданные \textit{вопросы} \textit{пользователем ostis-системы} может быть осуществлен путем поиска \textit{знаний} в текущем состоянии \textit{базы знаний} этой \textit{ostis-системы}, либо синтеза новых знаний, отсутствующих в \textit{базе знаний} этой \textit{ostis-системы};
    \item каждый \textit{вопрос} может быть представлен в виде некоторой \textit{спецификации задачи}, инициированной \textit{пользователем ostis-системы} для удовлетворения своей информационной потребности, а \textit{ответ на} этот \textit{вопрос} --- в виде \textit{семантической окрестности} \textit{основного знака в рамках заданного вопроса};
    \item каждому \textit{вопросу} может быть сопоставлен соответствующий класс \textit{вопросов} в \textit{Семантической классификации вопросов};
    \item для синтеза отсутствующих \textit{ответов на} поставленные \textit{вопросы} могут быть использованы различные \textit{модели решения задач}, в том числе \textit{логические модели решения задач};
    \item \textit{ответы на} поставленные \textit{вопросы} могут быть транслированы в \textit{естественно-языковой текст} и визуализированы при помощи соответствующих \textit{естественно-языковых интерфейсов} для удобства выдачи информации любому пользователю.
\end{itemize}}
\bigskip
\scnendcurrentsectioncomment
\end{SCn}

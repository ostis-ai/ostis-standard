\begin{SCn}

\scnsectionheader{Предметная область и онтология синтаксиса Языка вопросов для ostis-систем}
\begin{scnsubstruct}

    \scnheader{Предметная область синтаксиса Языка вопросов для ostis-систем}
	\scniselement{предметная область}
    \begin{scnhaselementrole}{максимальный класс объектов исследования}
        {синтаксис Языка вопросов для ostis-систем}
    \end{scnhaselementrole}
    
    \scnheader{синтаксис Языка вопросов для ostis-систем}
    \scntext{примечание}{\textit{Язык вопросов для ostis-систем} относится к семейству семантических совместимых языков --- \textit{sc-языков}, и предназначен для формального описания поискового предписания \textit{ostis-систем} с целью удовлетворения информационной потребности \textit{пользователя}. Поэтому \textbf{\textit{Синтаксис Языка вопросов для ostis-систем}}, как и \textit{синтаксис} любого другого \textit{sc-языка}, является \textit{Синтаксисом SC-кода}. Такой подход позволяет:
        \begin{itemize}
            \item унифицировать форму представления \textit{вопросов} и \textit{знаний}, с помощью которых строятся ответы на поставленные \textit{вопросы};
            \item использовать минимум средств для интерпретации заданных \textit{вопросов пользователей};
            \item сводить формирование ответов на большую часть заданных \textit{вопросов} к поиску информации в текущем состоянии \textit{базы знаний ostis-системы}.
        \end{itemize}}

\end{scnsubstruct}
\scnendcurrentsectioncomment 
\end{SCn}

\begin{SCn}
\scnsectionheader{Предметная область и онтология операционной семантики Базового языка программирования ostis-систем}
\begin{scnsubstruct}
	
\scnheader{Предметная область операционной семантики языка SCP}
\scniselement{предметная область}
\scnhaselementrole{класс объектов исследования}{Абстрактная scp-машина}
\begin{scnhaselementrolelist}{ключевой объект исследования}
	\scnitem{Абстрактный sc-агент создания scp-процессов}
	\scnitem{Абстрактный sc-агент интерпретации scp-операторов}
	\scnitem{Абстрактный sc-агент синхронизации процесса интерпретации scp-программ}
	\scnitem{Абстрактный sc-агент уничтожения scp-процессов}
	\scnitem{Абстрактный sc-агент синхронизации событий в sc-памяти и ее реализации}
	\scnitem{Абстрактный sc-агент трансляции сформированной спецификации события в sc-памяти во внутреннее представление}
	\scnitem{Абстрактный sc-агент обработки события в sc-памяти, инициирующего агентную scp-программу}
\end{scnhaselementrolelist}

\scntext{введение}{Преимущества предложенного многоагентного подхода к обработке информации могут работать не только на платформенно-независимом уровне, но и на более низких уровнях. Так, в частности, интерпретатор \textit{Базового языка программирования ostis-систем} также предлагается строить как \textit{неатомарный абстрактный sc-агент}, обеспечивающий интерпретацию методов, описанных на \textit{Языке SCP}. Таким образом, такой интерпретатор входит в общую иерархию агентов \textit{ostis-системы} и является \textit{абстрактным sc-агентом, не реализуемым на Языке SCP}.
	\\В общем случае вариантов реализации таких интерпретаторов может быть много. В рамках \textit{Стандарта OSTIS} один из них предлагается в качестве стандартного и называется \textit{Абстрактной scp-машиной}.}

\scnheader{Абстрактная scp-машина}
%TODO: check by human--->
\begin{scnrelfromset}{декомпозиция абстрактного sc-агента}
	\scnitem{Абстрактный sc-агент создания scp-процессов}
	\scnitem{Абстрактный sc-агент интерпретации scp-операторов}
	\scnitem{Абстрактный sc-агент синхронизации процесса интерпретации scp-программ}
	\scnitem{Абстрактный sc-агент уничтожения scp-процессов}
	\scnitem{Абстрактный sc-агент синхронизации событий в sc-памяти и ее реализации}
\end{scnrelfromset}
%<---TODO: check by human


\scnheader{Абстрактный sc-агент создания scp-процессов}
\scntext{пояснение}{Задачей \textit{Абстрактного} \textit{sc-агента создания scp-процессов} является создание \textit{scp-процессов}, соответствующих заданной \textit{scp-программе}. Данный \textit{\mbox{sc-агент}} активируется при появлении в \textit{sc-памяти} \textit{инициированного действия}, принадлежащего классу \textit{действие интерпретации scp-программы}.\\
	После проверки \textit{sc-агентом} условия инициирования выполняется создание \textit{scp-процесса} с учетом конкретных параметров интерпретации \textit{\mbox{scp-программы}}, после чего осуществляется поиск\textit{начального оператора \scnrolesign \mbox{scp-процесса}} и добавление его во множество \textit{настоящих сущностей}.}

\scnheader{Абстрактный sc-агент интерпретации scp-операторов}
\scntext{пояснение}{Задачей \textit{Абстрактного sc-агента интерпретации scp-операторов} является интерпретация операторов \textit{scp-программы}, то есть выполнение в \textit{sc-памяти} действий, описываемых конкретным \textit{\mbox{scp-оператором}}. Данный \textit{sc-агент} активируется при появлении в \textit{sc-памяти} \textit{scp-оператора}, принадлежащего классу \textit{настоящих сущностей}. После выполнения действия, описываемого \textit{scp-оператором}, \textit{scp-оператор} добавляется во множество \textit{прошлых сущностей}. В случае, когда семантика действия, описываемого \textit{\mbox{scp-оператором}}, предполагает возможность ветвления \textit{scp-программы} после выполнения данного \textit{\mbox{scp-оператора}}, то используется одно из подмножеств класса \textit{выполненных действий --- безуспешно выполненное действие} или \textit{успешно выполненное действие}.}

\scnheader{Абстрактный sc-агент синхронизации процесса интерпретации scp-программ}
\scntext{пояснение}{Задачей \textit{Абстрактного sc-агента синхронизации процесса интерпретации scp-программ} является обеспечение переходов между \textit{scp-операторами} в рамках одного \textit{scp-процесса}. Данный \textit{sc-агент} активизируется при добавлении некоторого \textit{scp-оператора} во множество \textit{прошлых сущностей}. Далее осуществляется переход по \textit{sc-дуге}, принадлежащей отношению \textit{последовательность действий*} (или более частным отношениям, в случае, если \textit{\mbox{scp-оператор}} был добавлен во множество \textit{успешно выполненных действий} или \textit{безуспешно выполненных действий}). При этом очередной \textit{scp-оператор} становится \textit{настоящей сущностью} (активным \textit{scp-оператором}) в том случае, если хотя бы один \textit{scp-оператор}, связанный с ним входящими \textit{sc-дугами}, принадлежащими отношению \textit{последовательность действий*} (или более частным отношениям), стал \textit{прошлой сущностью} (или, соответственно, подмножеством прошлых сущностей). В случае, когда необходимо дождаться завершения выполнения всех предыдущих операторов, для синхронизации используется оператор класса \textit{конъюнкция предшествующих операторов}.}

\scnheader{Абстрактный sc-агент уничтожения scp-процессов}
\scntext{пояснение}{Задачей \textit{Абстрактного sc-агента уничтожения scp-процессов} является уничтожение \textit{scp-процесса}, т. е. удаление из \textit{sc-памяти} всех составляющих его \textit{sc-элементов}. Данный \textit{sc-агент} активируется при появлении в \textit{sc-памяти} \textit{scp-процесса}, принадлежащего множеству \textit{прошлых сущностей}.\\
	При этом уничтожаемый \textit{scp-процесс} необязательно должен быть полностью сформирован. Необходимость уничтожения не до конца сформированного \textit{scp-процесса} может возникнуть в случае, если при создании \textit{scp-процесса} возникли проблемы, не позволяющие продолжить создание \textit{scp-процесса} и его выполнение.}

\scnheader{Абстрактный sc-агент синхронизации событий в sc-памяти и ее реализации}
\scntext{пояснение}{Задачей \textit{Абстрактного sc-агента синхронизации событий в sc-памяти и ее реализации} является обеспечение работы \textit{неатомарных sc-агентов}, реализованных на \textit{языке SCP}.}
\begin{scnreltoset}{декомпозиция абстрактного sc-агента}
	%TODO: check by human--->
	\scnitem{Абстрактный sc-агент трансляции сформированной спецификации события в sc-памяти во внутреннее представление}
	\scnitem{Абстрактный sc-агент обработки события в sc-памяти, инициирующего агентную scp-программу}
	%<---TODO: check by human
\end{scnreltoset}

\scnheader{Абстрактный sc-агент трансляции сформированной спецификации события в sc-памяти во внутреннее представление}
\scntext{пояснение}{Задачей \textit{\textbf{Абстрактного sc-агента трансляции сформированной спецификации события в sc-памяти во внутреннее представление}} является трансляция ориентированных пар, описывающих \textit{первичное условие инициирования*} некоторого \textit{\mbox{sc-агента}} во внутреннее представление элементарных событий на уровне \textit{\mbox{sc-хранилища}}, при условии, что этот \textit{sc-агент} реализован на платформенно-независимом уровне (с использованием \textit{языка SCP}). Условием инициирования данного \textit{sc-агента} является появление в \textit{\mbox{sc-памяти}} нового элемента множества \textit{активных sc-агентов}, для которого будет найдена и протранслирована соответствующая ориентированная пара.}

\scnheader{Абстрактный sc-агент обработки события в sc-памяти, инициирующего агентную scp-программу}
\scntext{пояснение}{Задачей \textit{Абстрактного sc-агента обработки события в sc-памяти, инициирующего агентную \mbox{scp-программу}}, является поиск \textit{агентной scp-программы}, входящей во множество \textit{программ sc-агента*} для каждого \textit{sc-агента}, первичное условие инициирования которого соответствует событию, произошедшему в \textit{sc-памяти}, а также генерация и инициирование действия, направленного на интерпретацию этой программы. В результате работы данного \textit{sc-агента} в \textit{sc-памяти} появляется \textit{инициированное действие}, принадлежащее классу \textit{действие} \textit{интерпретации scp-программы.}}

\end{scnsubstruct}
\end{SCn}

\begin{SCn}
\scnsectionheader{Предметная область и онтология решателей задач ostis-систем}
\begin{scnsubstruct}
\begin{scnrelfromlist}{дочерний раздел}
	\scnitem{Предметная область и онтология действий, задач, планов, протоколов и методов, реализуемых ostis-системой, а также внутренних агентов, выполняющих эти действия}
	\scnitem{Предметная область и онтология Базового языка программирования ostis-систем}
	\scnitem{Предметная область и онтология искусственных нейронных сетей и соответствующая им предметная область и онтология действий по обработке искусственных нейронных сетей}
\end{scnrelfromlist}

\scnheader{решатель задач ostis-системы}
\scnidtf{совокупность всех навыков, которыми обладает ostis-система на текущий момент времени}
\scnrelto{семейство подмножеств}{навык}
\scntext{примечание}{Предлагаемый в рамках \textit{Технологии OSTIS} подход к построению решателей задач позволяет обеспечить их модифицируемость, что, в свою очередь, позволяет \textit{ostis-системе} при необходимости легко приобретать новые \textit{навыки}, модифицировать (совершенствовать) уже имеющиеся и даже избавляться от некоторых навыков с целью повышения производительности системы. Таким образом, имеет смысл говорить не о жестко фиксированном решателе задач, который разрабатывается один раз при создании первой версии системы и далее не меняется, а о совокупности навыков, фиксированной в каждый текущий момент времени, но постоянно эволюционирующей.}
\scnsuperset{объединенный решатель задач ostis-системы}
	\begin{scnindent}
		\scnidtf{полный решатель задач ostis-системы}
		\scnidtf{интегрированный решатель задач ostis-системы}
		\scnidtf{решатель задач ostis-системы, реализующий все ее функциональные возможности, как основные, так и вспомогательные}
		\scntext{пояснение}{В общем случае \textit{объединенный решатель задач ostis-системы} решает задачи, связанные с:
		\begin{scnitemize}
			\item обеспечением основных функциональных возможностей системы (например, решение явно сформулированных задач по требованию пользователя);
			\item обеспечением корректности и оптимизацией работы самой ostis-системы (перманентно на протяжении всего жизненного цикла ostis-системы);
			\item обеспечением повышения квалификации конечных пользователей и разработчиков ostis-системы;
			\item обеспечением автоматизации развития и управления развитием ostis-системы.
		\end{scnitemize}
		}
	\end{scnindent}
\scnsuperset{гибридный решатель задач ostis-системы}
	\begin{scnindent}
		\scnidtf{решатель задач ostis-системы, реализующий две и более модели решения задач}
	\end{scnindent}
		
\scnheader{машина обработки знаний}
\scnsubset{sc-агент}
\scntext{пояснение}{Под \textit{машиной обработки знаний} будем понимать совокупность интерпретаторов всех \textit{навыков}, составляющих некоторый \textit{решатель задач}. С учетом многоагентного подхода к обработке информации, используемого в рамках Технологии OSTIS, \textit{машина обработки знаний} представляет собой \textit{sc-агент} (чаще всего --- \textit{неатомарный sc-агент}), в состав которого входят более простые sc-агенты, обеспечивающие интерпретацию соответствующего множества \textit{методов}. Таким образом, \textit{машина обработки знаний} в общем случае представляет собой иерархическую систему \textit{sc-агентов}.}

\scnheader{решатель задач ostis-системы}
\scnhaselement{Решатель задач Метасистемы IMS.ostis}
\scnsuperset{решатель задач вспомогательной ostis-системы}
	\begin{scnindent}
		\scnsuperset{решатель задач интерфейса компьютерной системы}
			\begin{scnindent}
				\begin{scnsubdividing}
					%TODO: check by human--->
					\scnitem{решатель задач пользовательского интерфейса компьютерной системы}
					\scnitem{решатель задач интерфейса компьютерной системы с другими компьютерными системами}
					\scnitem{решатель задач интерфейса компьютерной системы с окружающей средой}
					%<---TODO: check by human
				\end{scnsubdividing}
			\end{scnindent}
		\scnsuperset{решатель задач ostis-подсистемы поддержки проектирования компонентов определенного класса}
		\begin{scnindent}
			\scnsuperset{решатель задач ostis-подсистемы поддержки проектирования баз знаний}
				\begin{scnindent}
					\scnsuperset{решатель задач повышения качества базы знаний}
						\begin{scnindent}
							\scnsuperset{решатель задач верификации базы знаний}
								\begin{scnindent}
									\scnsuperset{решатель задач поиска и устранения некорректностей в базе знаний}
									\scnsuperset{решатель задач поиска и устранения неполноты}
								\end{scnindent}
							\scnsuperset{решатель задач оптимизации структуры базы знаний}
							\scnsuperset{решатель задач выявления и устранения информационного мусора}
						\end{scnindent}
				\end{scnindent}
			\scnsuperset{решатель задач ostis-подсистемы поддержки проектирования решателей задач ostis-систем}
			\begin{scnindent}
					\begin{scnsubdividing}
						%TODO: check by human--->
						\scnitem{решатель задач ostis-подсистемы поддержки проектирования программ обработки знаний}
						\scnitem{решатель задач ostis-подсистемы поддержки проектирования агентов обработки знаний}
						%<---TODO: check by human
					\end{scnsubdividing}
				\end{scnindent}
		\end{scnindent}
		\scnsuperset{решатель задач подсистемы управления проектирования компьютерных систем и их компонентов}
	\end{scnindent}
\scnsuperset{решатель задач самостоятельной ostis-системы}

\scnheader{решатель задач ostis-системы}
\scnsuperset{решатель задач с использованием хранимых методов}
	\begin{scnindent}
		\scnidtf{решатель, способный решать задачи тех классов, для которых на данный момент времени известен соответствующий метод решения}
		\scnsuperset{решатель задач на основе нейросетевых моделей}
		\scnsuperset{решатель задач на основе генетических алгоритмов}
		\scnsuperset{решатель задач на основе императивных программ}
			\begin{scnindent}
				\scnsuperset{решатель задач на основе процедурных программ}
				\scnsuperset{решатель задач на основе объектно-ориентированных программ}
			\end{scnindent}
		\scnsuperset{решатель задач на основе декларативных программ}
			\begin{scnindent}
				\scnsuperset{решатель задач на основе логических программ}
				\scnsuperset{решатель задач на основе функциональных программ}
			\end{scnindent}
	\end{scnindent}
\scnsuperset{решатель задач в условиях, когда метод решения задач данного класса в текущий момент времени не известен}
	\begin{scnindent}
		\scnidtf{решатель, реализующий стратегии решения задач, позволяющие породить метод решения задачи, который в текущий момент времени не известен ostis-системе}
		\scnidtf{решатель, использующий для решения задач метаметоды, соответствующие более общим классам задач по отношению к заданной}
		\scnidtf{решатель задач, позволяющий породить метод, который является частным по отношению к какому-либо известному ostis-системе методу и интерпретируется соответствующей машиной обработки знаний}
		\scnsuperset{решатель, реализующий стратегию поиска путей решения задачи в глубину}
		\scnsuperset{решатель, реализующий стратегию поиска путей решения задачи в ширину}
		\scnsuperset{решатель, реализующий стратегию проб и ошибок}
		\scnsuperset{решатель, реализующий стратегию разбиения задачи на подзадачи}
		\scnsuperset{решатель, реализующий стратегию решения задач по аналогии}
		\scnsuperset{решатель, реализующий концепцию интеллектуального пакета программ}
	\end{scnindent}

\scnheader{машина обработки знаний}
\scnsuperset{машина логического вывода}
\begin{scnindent}
	\scnsuperset{машина дедуктивного вывода}
		\begin{scnindent}
			\scnsuperset{машина прямого дедуктивного вывода}
			\scnsuperset{машина обратного дедуктивного вывода}
		\end{scnindent}
	\scnsuperset{машина индуктивного вывода}
	\scnsuperset{машина абдуктивного вывода}
	\scnsuperset{машина нечеткого вывода}
	\scnsuperset{машина вывода на основе логики умолчаний}
	\scnsuperset{машина логического вывода с учетом фактора времени}
\end{scnindent}

\scnheader{решатель задач ostis-системы}
\scnsuperset{решатель задач информационного поиска}
	\begin{scnindent}
		\begin{scnsubdividing}
			%TODO: check by human--->
			\scnitem{решатель задач поиска информации, удовлетворяющей заданным критериям}
			\scnitem{решатель задач поиска информации, не удовлетворяющей заданным критериям}
			%<---TODO: check by human
		\end{scnsubdividing}
	\end{scnindent}
\scnsuperset{решатель явно сформулированных задач}
	\begin{scnindent}
		\scnidtf{решатель задач, для которых явно сформулирована цель}
		\scnsuperset{решатель задач поиска или вычисления значений заданного множества величин}
		\scnsuperset{решатель задач установления истинности заданного логического высказывания в рамках заданной формальной теории}
		\scnsuperset{решатель задач формирования доказательства заданного высказывания в рамках заданной формальной теории}
		\scnsuperset{машина верификации ответа на указанную задачу}
		\scnsuperset{машина верификации решения указанной задачи}
			\begin{scnindent}
				\scnsuperset{машина верификации доказательства заданного высказывания в рамках заданной формальной теории}
			\end{scnindent}
	\end{scnindent}
\scnsuperset{решатель задач классификации сущностей}
	\begin{scnindent}
		\scnsuperset{машина соотнесения сущности с одним из заданного множества классов}
		\scnsuperset{машина разделения множества сущностей на классы по заданному множеству признаков}
	\end{scnindent}
\scnsuperset{решатель задач синтеза информационных конструкций}
	\begin{scnindent}
		\scnsuperset{решатель задач синтеза естественно-языковых текстов}
		\scnsuperset{решатель задач синтеза изображений}
		\scnsuperset{решатель задач синтеза сигналов}
		\begin{scnindent}
			\scnsuperset{решатель задач синтеза речи}
		\end{scnindent}
	\end{scnindent}
\scnsuperset{решатель задач анализа информационных конструкций}
	\begin{scnindent}
		\scnsuperset{решатель задач анализа естественно-языковых текстов}
			\begin{scnindent}
				\scnsuperset{решатель задач понимания естественно-языковых текстов}
				\scnsuperset{решатель задач верификации естественно-языковых текстов}
			\end{scnindent}
		\scnsuperset{решатель задач анализа изображений}
			\begin{scnindent}
				\scnsuperset{решатель задач сегментации изображений}
				\scnsuperset{решатель задач понимания изображений}
			\end{scnindent}
		\scnsuperset{решатель задач анализа сигналов}
			\begin{scnindent}
				\scnsuperset{решатель задач анализа речи}
					\begin{scnindent}
						\scnsuperset{решатель задач понимания речи}
					\end{scnindent}
			\end{scnindent}
	\end{scnindent}

\bigskip
\end{scnsubstruct}
\end{SCn}

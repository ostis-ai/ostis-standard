\begin{SCn}
\scnsectionheader{Предметная область и онтология решателей задач ostis-систем}
\begin{scnsubstruct}
\begin{scnrelfromlist}{дочерняя предметная область и онтология}
	\scnitem{Предметная область и онтология действий, задач, планов, протоколов и методов, реализуемых ostis-системой, а также внутренних агентов, выполняющих эти действия}
	\scnitem{Предметная область и онтология Базового языка программирования ostis-систем}
	\scnitem{Предметная область и онтология искусственных нейронных сетей и соответствующая им предметная область и онтология действий по обработке искусственных нейронных сетей}
\end{scnrelfromlist}

\scnheader{Предметная область решателей задач ostis-систем}
\begin{scnrelfromlist}{автор}
	\scnitem{Шункевич Д.В.}
\end{scnrelfromlist}
\begin{scnhaselementrolelist}{ключевой знак}
	\scnitem{Язык SCP}
	\scnitem{Абстрактная scp-машина}
\end{scnhaselementrolelist}
\begin{scnhaselementrolelist}{класс объектов исследования}
	\scnitem{действие в sc-памяти}
	\scnitem{действие в sc-памяти, инициируемое вопросом}
	\scnitem{действие редактирования базы знаний}
	\scnitem{задача, решаемая в sc-памяти}
	\scnitem{класс логически атомарных действий}
	\scnitem{sc-агент}
	\scnitem{абстрактный sc-агент}
	\scnitem{атомарный абстрактный sc-агент}
	\scnitem{неатомарный абстрактный sc-агент}
	\scnitem{абстрактный sc-агент, реализуемый на Языке SCP}
	\scnitem{абстрактный sc-агент, не реализуемый на Языке SCP}	
	\scnitem{тип блокировки}
	\scnitem{транзакция в sc-памяти}	
	\scnitem{scp-оператор}
	\scnitem{решатель задач ostis-системы}
	\scnitem{машина обработки знаний}
\end{scnhaselementrolelist}
\begin{scnhaselementrolelist}{исследуемое отношение}
	\scnitem{блокировка*}
	\scnitem{планируемая блокировка*}
	\scnitem{приоритет блокировки*}
	\scnitem{удаляемые sc-элементы*}
	\scnitem{параметр scp-программы\scnrolesign}
	\scnitem{scp-операнд\scnrolesign}
\end{scnhaselementrolelist}
\begin{scnrelfromlist}{библиографическая ссылка}
	\scnitem{\scncite{Kolesnikov2001}}
	\scnitem{\scncite{Pratt2002}}
	\scnitem{\scncite{Gladkov2006}}
	\scnitem{\scncite{Emelyanov2003}}
	\scnitem{\scncite{Berkinblit1993}}
	\scnitem{\scncite{Golovko2001}}
	\scnitem{\scncite{Gorban1996}}
	\scnitem{\scncite{Vagin2008}}
	\scnitem{\scncite{Khulick2001}}
	\scnitem{\scncite{Polya1975}}
	\scnitem{\scncite{Batyrshin2001}}
	\scnitem{\scncite{Demenkov2005}}
	\scnitem{\scncite{Pospelov1989}}
	\scnitem{\scncite{Reiter1980}}
	\scnitem{\scncite{Eremeev1997}}
	\scnitem{\scncite{Kachro1988}}
	\scnitem{\scncite{Ephymov1982}}
	\scnitem{\scncite{Raghovsky2011}}
	\scnitem{\scncite{Podkholzyn2008}}
	\scnitem{\scncite{Khurbatov2016}}
	\scnitem{\scncite{Vladimirov2010}}
	\scnitem{\scncite{AIRefBookP11990}}
	\scnitem{\scncite{Jackson1998}}
	\scnitem{\scncite{W3C}}
	\scnitem{\scncite{RDF}}
	\scnitem{\scncite{OWL}}
	\scnitem{\scncite{SPARQL}}
	\scnitem{\scncite{Neo4j}}
	\scnitem{\scncite{OWLImplementations}}
	\scnitem{\scncite{Gribova2015a}}
	\scnitem{\scncite{Gribova2011}}
	\scnitem{\scncite{Phylyppov2016}}
	\scnitem{\scncite{Borisov2014}}
	\scnitem{\scncite{Dutta1993}}
	\scnitem{\scncite{Pau1990}}
	\scnitem{\scncite{Wooldridge2009}}
	\scnitem{\scncite{Weyns2007}}
	\scnitem{\scncite{ACL}}
	\scnitem{\scncite{Finin1994}}
	\scnitem{\scncite{KIF}}
	\scnitem{\scncite{Hartung2008}}
	\scnitem{\scncite{Sims2008}}
	\scnitem{\scncite{Excelente-Toledo2004}}
	\scnitem{\scncite{NagendraPrasad1999}}
	\scnitem{\scncite{Vasconcelos2009}}
	\scnitem{\scncite{Rumbell2012}}
	\scnitem{\scncite{Gorodetsky2015}}
	\scnitem{\scncite{Bordini2007}}
	\scnitem{\scncite{Castillo2014}}
	\scnitem{\scncite{EVE}}
	\scnitem{\scncite{GAMA}}
	\scnitem{\scncite{GOAL}}
	\scnitem{\scncite{Evertsz2004}}
	\scnitem{\scncite{JADE}}
	\scnitem{\scncite{Boissier2013}}
	\scnitem{\scncite{Omicini1999}}
	\scnitem{\scncite{Jagannathan1989}}
	\scnitem{\scncite{Pospelov1986}}
	\scnitem{\scncite{Dijkstra2002}}
	\scnitem{\scncite{Hoare1983}}
	\scnitem{\scncite{Chatterjee2022}}
	\scnitem{\scncite{Narinjani2004}}
	\scnitem{\scncite{Cao2010}}
	\scnitem{\scncite{Cao2014}}
	\scnitem{\scncite{Pavel2015}}
	\scnitem{\scncite{Altshuller2010}}
	\scnitem{\scncite{Shhedrovickij1995}}
	\scnitem{\scncite{Sapatyj1986}}
	\scnitem{\scncite{Moldovan1985}}
	\scnitem{\scncite{Letichevskij2003}}
	\scnitem{\scncite{Letichevskij2012}}
\end{scnrelfromlist}

\scntext{аннотация}{В предметной области формулированы актуальные проблемы текущего состояния технологий разработки гибридных решателей задач, предложен подход к их решению на основе Технологии OSTIS. Сформулированы принципы построения решателя задач как иерархической системы навыков, основанной на многоагентном подходе, приведены онтологии агентов и выполняемых ими действий. Сформулированы принципы синхронизации деятельности агентов, а также разработана онтология базового языка программирования для реализации программ агентов и модель интерпретатора такого языка.}

\begin{scnrelfromvector}{введение}
	\scnfileitem{Одним из ключевых компонентов \textit{интеллектуальной системы}, обеспечивающим возможность решать широкий круг \textit{задач}, является \textit{решатель задач}. Их особенностью по сравнению с другими современными \textit{программными системами} является необходимость решать \textit{задачи} в условиях, когда необходимые сведения не локализованы явно в \textit{базе знаний} \textit{интеллектуальной системы} и должны быть найдены в процессе решения \textit{задачи} на основании каких-либо критериев.}
	\scnfileitem{Говоря другими словами, если в традиционных системах при решении задачи всегда подразумевается, что есть некоторые локализованные исходные данные (\scnqq{дано}) и некоторое описание желаемого результата (\scnqq{что требуется}), то в \textit{интеллектуальной системе} в качестве исходных данных при решении большого числа \textit{задач} выступает вся имеющаяся на текущий момент в системе информация, то есть вся \textit{база знаний}. Кроме того, при невозможности решения задачи в текущем состоянии базы знаний интеллектуальная система должна иметь возможность понять, чего именно не хватает для продолжения процесса решения и попытаться добыть недостающие сведения во внешней среде (например, запросить у пользователя).}
	\scnfileitem{К настоящему времени в рамках различных направлений \textit{Искусственного интеллекта} разработано большое количество различных \textit{моделей решения задач}, каждая из которых позволяет решать задачи определенного класса. Расширение областей применения \textit{интеллектуальных систем} требует от них возможности решать так называемые \textit{комплексные задачи}, решение каждой из которых требует комбинирования нескольких моделей решения задач, при этом априори неизвестно, в каком порядке и сколько раз будет применяться та или иная модель. \textit{решатели задач}, в рамках которых комбинируются несколько \textit{моделей решения задач}, получили название \textit{гибридных решателей задач}, а интеллектуальные системы, в рамках которых комбинируются различные \textit{виды знаний} и различные \textit{модели решения задач} --- \textit{гибридных интеллектуальных систем}.}
	\begin{scnindent}
		\begin{scnrelfromset}{смотрите}
			\scnitem{\scncite{Kolesnikov2001}}
		\end{scnrelfromset}
	\end{scnindent}
	\scnfileitem{Повышение эффективности разработки и эксплуатации \textit{гибридных интеллектуальных систем} требует унификации моделей представления различных \textit{видов знаний} и \textit{моделей обработки знаний}, которая бы позволила легко интегрировать на ее основе компоненты, соответствующие различным моделям решения задач.}
\end{scnrelfromvector}

\scnsectionheader{Современное состояние, проблемы в области разработки гибридных решателей задач и предлагаемый подход к их решению}
\begin{scnsubstruct}
	\begin{scnreltovector}{конкатенация сегментов}
		\scnitem{Современное состояние технологий разработки решателей задач и требования, предъявляемые к гибридным решателям задач}
		\scnitem{Предлагаемый подход к разработке гибридных решателей задач ostis-систем и обработке информации в ostis-системах}
	\end{scnreltovector}

\scnsectionheader{Современное состояние технологий разработки решателей задач и требования, предъявляемые к гибридным решателям задач}
\begin{scnsubstruct}
	\scnheader{решение задач}
	\begin{scnsubdividing}
		\scnitem{решение задач с использованием хранимых программ}  
		\scnitem{решение задач в условиях, когда программа решения не известна}
	\end{scnsubdividing}
	
	\scnheader{решение задач с использованием хранимых программ}
	\scntext{определение}{\textit{решение задач с использованием хранимых программ} --- это решение задач, в котором предполагается, что в системе заранее присутствует программа решения задачи заданного класса и решение сводится к поиску такой программы и интерпретации ее на заданных входных данных.}
	\scntext{пример}{К системам, ориентированным на такой подход к решению задач, относятся в том числе системы, использующие:
		\begin{itemize}
			\item программы, написанные на языках программирования, относящихся как к императивной, так и к декларативной парадигме, в том числе логических и функциональных;
			\item реализации генетических алгоритмов;
			\item нейросетевые модели обработки знаний.
		\end{itemize}}
	\begin{scnindent}
		\begin{scnrelfromset}{смотрите}
			\scnitem{\scncite{Pratt2002}}
			\scnitem{\scncite{Emelyanov2003}}
			\scnitem{\scncite{Gladkov2006}}
			\scnitem{\scncite{Berkinblit1993}}
			\scnitem{\scncite{Gorban1996}}
			\scnitem{\scncite{Golovko2001}}
		\end{scnrelfromset}
	\end{scnindent}
	\scntext{примечание}{Следует отметить, что даже в случае использования хранимой \textit{программы} решение \textit{задачи} далеко не всегда тривиально, поскольку, во-первых, требуется найти такую хранимую \textit{программу} на основе некоторой спецификации, во-вторых, обеспечить ее интерпретацию.}
	
	\scnheader{решение задач в условиях, когда программа решения не известна}
	\scntext{определение}{\textit{решение задач в условиях, когда программа решения не известна} --- решение задач, в котором предполагается, что в системе необязательно присутствует готовая \textit{программа} решения для \textit{класса задач}, которому принадлежит некоторая сформулированная задача, подлежащая решению. В связи с этим необходимо применять дополнительные методы поиска путей решения задачи, не рассчитанные на какой-либо узкий \textit{класс задач} (например, разбиение задачи на подзадачи, методы поиска решений в глубину и ширину, метод случайного поиска решения и метод проб и ошибок, метод деления пополам и другие), а также различные модели \textit{логического вывода}: классические дедуктивные, индуктивные, абдуктивные; модели, основанные на \textit{нечетких логиках}, \textit{логике умолчаний}, \textit{темпоральной логике}, и многие другие.}
	\begin{scnindent}
		\begin{scnrelfromset}{смотрите}
			\scnitem{\scncite{Eremeev1997}}
			\scnitem{\scncite{Reiter1980}}
			\scnitem{\scncite{Batyrshin2001}}
			\scnitem{\scncite{Pospelov1989}}
			\scnitem{\scncite{Vagin2008}}
			\scnitem{\scncite{Polya1975}}
			\scnitem{\scncite{Khulick2001}}
			\scnitem{\scncite{Vagin2008}}
			\scnitem{\scncite{Demenkov2005}}
		\end{scnrelfromset}
	\end{scnindent}
	
	\scnheader{решатель задач}
	\scnhaselement{STRIPS}
	\scnhaselement{GPS}
	\scnhaselement{QA3}
	\scnhaselement{ППР}
	\scnhaselement{ПРИЗ}
	\begin{scnindent}
		\begin{scnrelfromset}{смотрите}
			\scnitem{\scncite{Kachro1988}}
		\end{scnrelfromset}
	\end{scnindent}
	\scnhaselement{Компьютерный решатель математических задач}
	\begin{scnindent}
		\begin{scnrelfromset}{смотрите}
			\scnitem{\scncite{Podkholzyn2008}}
		\end{scnrelfromset}
	\end{scnindent}
	\scnhaselement{Решатель задач по планиметрии НИЦ ЭВТ}
	\begin{scnindent}
		\begin{scnrelfromset}{смотрите}
			\scnitem{\scncite{Khurbatov2016}}
		\end{scnrelfromset}
	\end{scnindent}
	\scnhaselement{Программный комплекс \scnqq{УДАВ}}
	\begin{scnindent}
		\begin{scnrelfromset}{смотрите}
			\scnitem{\scncite{Vladimirov2010}}
		\end{scnrelfromset}
	\end{scnindent}
	\scntext{примечание}{Подробный обзор \textit{решателей задач}, разработанных в период до 1982 года, таких как \textit{GPS}, \textit{STRIPS}, \textit{QA3}, \textit{ПРИЗ}, \textit{ППР} приведен в книге \textit{Ephymov1982}. Среди современных работ, исследующих вопросы применения \textit{моделей решения задач}, не ориентированных на конкретную предметную область, можно выделить \textit{Raghovsky2011}. Среди наиболее заметных представителей класса \textit{интеллектуальных решателей задач}, разработанных в более поздний период, можно отметить \textit{Компьютерный решатель математических задач}, \textit{Решатель задач по планиметрии НИЦ ЭВТ}, \textit{Программный комплекс \scnqqi{УДАВ}}.} 
	\begin{scnindent}
		\begin{scnrelfromset}{смотрите}
			\scnitem{\scncite{Ephymov1982}}
			\scnitem{\scncite{Raghovsky2011}}
		\end{scnrelfromset}
	\end{scnindent}
	\scntext{примечание}{Отдельного внимания заслуживают популярные в настоящее время \textit{системы компьютерной алгебры}, такие как \textit{Wolfram Mathematica}, \textit{Maple}, \textit{MathCAD} и другие. Указанные программные комплексы обладают мощной функциональностью как для проведения различного рода вычислений и экспериментов, так и для построения на их основе систем различного назначения, например обучающих. Более подробно возможности применения систем данного семейства для решения \textit{задач} в рамках \textit{Экосистемы OSTIS} рассмотрены в \textit{Интеграция инструментов компьютерной алгебры в приложения OSTIS}.}
	\begin{scnindent}
		\begin{scnrelfromset}{смотрите}
			\scnitem{Интеграция инструментов компьютерной алгебры в приложения OSTIS}
		\end{scnrelfromset}
	\end{scnindent}
	\scntext{примечание}{Однако при всем многообразии решаемых рассмотренными системами \textit{задач} множество \textit{классов задач} ограничивается имеющимся в системе набором жестко заданных приемов и алгоритмов решения \textit{задач}, явно используемых при решении той или иной \textit{задачи}. В то же время построение сложных систем, например, систем комплексной автоматизации, невозможно без обеспечения согласованного использования различных \textit{видов знаний} и \textit{моделей решения задач} в рамках одной системы при решении одной и той же \textit{комплексной задачи}. Кроме того, становится актуальной \textit{задача} поддержки такой системы в состоянии, соответствующем текущему уровню развития технологий, дополнения ее более совершенными \textit{моделями} и \textit{методами решения задач}. При этом очевидно, что подобная реконфигурация системы должна осуществляться \textit{непосредственно в процессе эксплуатации системы}, а не требовать каждый раз, например, полной остановки всего производства или отдельных его частей.}

	\scnheader{гибридный решатель задач}
	\begin{scnrelfromlist}{требование}
		\scnfileitem{В каждый момент времени \textit{решатель задач} должен обеспечивать решение задач из оговоренного класса за оговоренное время, при этом результат решения задачи должен удовлетворять некоторым известным требованиям. Другими словами, как и в случае современных \textit{компьютерных систем}, корректность результатов решения задач на этапе разработки системы должна верифицироваться специальными методами, в том числе для этого могут быть использованы такие современные подходы, как \textit{unit-тестирование}, \textit{тестирование методом \scnqqi{черного ящика}} и другие.}
		\begin{scnindent}
			\begin{scnsubdividing}
				\scnfileitem{Для явно сформулированных \textit{задач} система всегда должна давать какой-либо ответ за оговоренное время, при этом ответ может быть отрицательным (система не смогла решить поставленную задачу), возможно, с объяснением причин, по которым решение в текущий момент оказалось невозможным. Одним из факторов безуспешности решения является выход за рамки установленного промежутка времени.}
				\scnfileitem{Если явно сформулированная \textit{задача} решена, то все \textit{информационные процессы}, направленные на ее решение, должны быть уничтожены. Особенно актуальным данное требование становится в ситуации, когда для решения одной и той же задачи параллельно используются сразу несколько подходов и заранее неизвестно, какой из них приведет к результату раньше других.}
				\scnfileitem{После решения задачи вся временная информация, сгенерированная в процессе решения этой \textit{задачи} и имеющая ценность только в контексте решения указанной \textit{задачи}, должна быть удалена из памяти.}
			\end{scnsubdividing}
		\end{scnindent}
		\scnfileitem{\textit{\textbf{гибридный решатель}} должен обеспечивать возможность \textbf{согласованного использования различных моделей решения задач} при решении одной и той же \textit{комплексной задачи} в случае необходимости.}
		\scnfileitem{\textit{решатель задач} должен быть легко \textbf{модифицируемым}, то есть трудоемкость внесения изменений в уже разработанный \textit{решатель задач} должна быть минимальна. Путями повышения модифицируемости \textit{решателя задач} являются обеспечение локальности вносимых изменений, в том числе --- за счет стратификации \textit{решателя задач} на независимые уровни и обеспечение максимальной независимости компонентов \textit{решателя задач} друг от друга, а также наличие готовых компонентов, которые могут быть встроены в \textit{решатель задач} при необходимости. При этом внесение изменений должно осуществляться \textit{непосредственно в процессе эксплуатации системы}.}
		\scnfileitem{Для того чтобы \textit{интеллектуальная система} имела возможность анализировать и оптимизировать имеющийся \textit{решатель задач}, интегрировать в его состав новые компоненты (в том числе самостоятельно), оценивать важность тех или иных компонентов и применимость их для решения той или иной задачи, спецификация \textit{решателя задач} должна быть описана языком, понятным системе, например, при помощи тех же средств, что и обрабатываемые \textit{знания}. Другими словами, \textit{интеллектуальная система} и, соответственно, \textit{решатель задач} должны обладать \textit{рефлексивностью}.}
	\end{scnrelfromlist}

	\scnheader{модель решения задач}
	\scntext{проблемы текущего состояния}{Несмотря на то что в настоящее время существует большое число \textit{моделей решения задач}, многие из которых реализованы и успешно используются на практике в различных системах, остается актуальной проблема низкой согласованности принципов, лежащих в основе реализации таких моделей, и отсутствия единой унифицированной основы для реализации и интеграции различных \textit{моделей решения задач}, что приводит к тому, что:
		\begin{itemize}
			\item затруднена возможность одновременного использования различных \textit{моделей решения задач} в рамках одной системы при решении одной и той же комплексной задачи; практически невозможно комбинировать различные модели с целью решения \textit{задачи}, для которой априори отсутствует \textit{алгоритм} ее решения;
			\item практически невозможно использовать технические решения, реализованные в одной системе, в других системах, то есть возможности использования компонентного подхода при построении \textit{решателей задач} сильно ограничены. Как следствие, велико количество дублирований аналогичных решений в разных системах;
			\item фактически отсутствуют комплексные методики и средства построения \textit{решателей задач}, которые бы обеспечивали возможность проектирования, реализации и отладки \textit{решателей задач} различного вида.
		\end{itemize}}
	\begin{scnindent}
		\begin{scnrelfromlist}{следствие}
			\scnfileitem{Высокая трудоемкость разработки каждого \textit{решателя задач}, увеличение сроков их разработки, а значит, и увеличение затрат на разработку и поддержку соответствующих \textit{интеллектуальных систем}.}
			\scnfileitem{Высокая трудоемкость внесения изменений в уже разработанные \textit{решатели задач}, то есть отсутствует или сильно затруднена возможность дополнения уже разработанного \textit{решателя задач} новыми компонентами и внесения изменений в уже существующие компоненты в процессе эксплуатации системы. Таким образом, высока трудоемкость поддержки разработанных \textit{решателей задач}.}
			\scnfileitem{Высокий уровень профессиональных требований к разработчикам \textit{решателей задач}, что обусловлено, в частности:
			\begin{itemize}
				\item Высокой сложностью существующих формализмов в области решения \textit{задач}, рассчитанных на их интерпретацию \textit{компьютерной системой}, а не человеком;
				\item Отсутствием возможности рассматривать разрабатываемые \textit{решатели задач} на разных уровнях детализации, выделения на каждом уровне достаточно независимых компонентов, что затрудняет процесс проектирования, тестирования и отладки таких \textit{решателей задач}, а также снижает эффективность попыток объединения разработчиков \textit{решателей задач} в коллективы по причине увеличения накладных расходов на согласование их деятельности;
				\item Низким уровнем информационной поддержки разработчиков и автоматизации их \textit{деятельности}.
			\end{itemize}}
		\end{scnrelfromlist}
		\begin{scnrelfromvector}{решение проблем}
			\scnfileitem{Для решения перечисленных проблем необходимо разработать комплекс моделей, методики и средств разработки \textit{гибридных решателей задач}, удовлетворяющих перечисленным ранее требованиям.}
			\scnfileitem{Исторически сложились два основных подхода к построению \textit{решателей задач} \textit{интеллектуальных компьютерных систем}.}
			\scnfileitem{Первый подход предполагает наличие в системе фиксированного \textit{решателя задач} (например, машины логического вывода), к которому впоследствии добавляется \textit{база знаний}, наполнение которой определяется \textit{предметной областью}, в которой должна работать система. Такие системы получили название \scnqq{пустых} \textit{экспертных систем} или \scnqq{оболочек} (expert system shells). Данный подход, как правило, использовался для разработки относительно несложных систем и в настоящее время не имеет широкого применения.}
			\begin{scnindent}
				\begin{scnrelfromset}{смотрите}
					\scnitem{\scncite{Jackson1998}}
					\scnitem{\scncite{AIRefBookP11990}}
				\end{scnrelfromset}
			\end{scnindent}
			\scnfileitem{Второй подход, широко используемый в настоящее время, предполагает наличие программных средств доступа к информации, хранящейся в некоторой базе, совместимых с различными популярными \textit{языками программирования}. Данный подход широко используется, например, в системах, построенных на основе стандартов \textit{W3C}, таких как \textit{RDF}, \textit{OWL}, \textit{SPARQL}, а также \textit{графовых с.у.б.д.}, таких как \textit{Neo4j}. Структура \textit{решателя задач}, построенного на базе таких средств, определяется разработчиком в каждом конкретном случае и не фиксируется какими-либо стандартами. Такой подход обладает большей гибкостью, но отсутствие унификации в структуре и процессе разработки \textit{решателей задач} приводит к отсутствию совместимости компонентов \textit{решателей задач}, созданных разными разработчиками, большому количеству дублирований одних и тех же решений, повышению накладных расходов в процессе разработки и поддержки \textit{решателя задач}. Также существует большое количество реализаций так называемых \textit{ризонеров} (semantic reasoners), осуществляющих \textit{логический вывод} на \textit{онтологиях}, представленных в формате \textit{OWL 2}, а также средств разработки и редактирования таких \textit{онтологий}. Полный список таких средств, признанных консорциумом \textit{W3C}, можно найти на сайте \textit{OWLImplementations}. Как видно из приведенной на нем таблицы, подавляющее большинство средств способно осуществлять только прямой \textit{логический вывод} на основе \textit{отношений}, описанных в \textit{онтологии}.}
			\begin{scnindent}
				\begin{scnrelfromset}{смотрите}
					\scnitem{\scncite{W3C}}
					\scnitem{\scncite{OWL}}
					\scnitem{\scncite{RDF}}
					\scnitem{\scncite{SPARQL}}
					\scnitem{\scncite{Neo4j}}
					\scnitem{\scncite{OWLImplementations}}
				\end{scnrelfromset}
			\end{scnindent}
			\scnfileitem{Среди комплексных подходов к построению \textit{решателей задач}, разрабатываемых русскоязычными авторами, можно выделить проект \textit{IACPaaS}, активно развивающийся в настоящее время. Целью данного проекта является разработка облачной платформы для построения на ее основе \textit{интеллектуальных сервисов} различного назначения. В данном проекте активно используются \textit{библиотеки многократно используемых компонентов интеллектуальных систем}. Конкретно для построения \textit{решателей задач}, а также \textit{пользовательских интерфейсов} таких систем используется \textit{многоагентный подход}. Несмотря на близость некоторых технологических решений, реализуемых в проекте \textit{IACPaaS} и в рамках \textit{Технологии OSTIS}, основной целью указанного проекта является предоставление пользователю большого числа разнородных сервисов, выбор которых осуществляется самим пользователем, в то время как одним из ключевых принципов \textit{Технологии OSTIS} является разработка общей формальной основы для интеграции различных \textit{моделей решения задач} с целью их комбинирования при решении одной и той же \textit{комплексной задачи}.}
			\begin{scnindent}
				\begin{scnrelfromset}{смотрите}
					\scnitem{\scncite{Gribova2015a}}
					\scnitem{\scncite{Gribova2011}}
				\end{scnrelfromset}
			\end{scnindent}
			\scnfileitem{Задачи интеграции различных подходов, в том числе связанных с решением задач, исследуются также в работе \textit{Phylyppov2016} и других работах тех же авторов.}
			\begin{scnindent}
				\begin{scnrelfromset}{смотрите}
					\scnitem{\scncite{Phylyppov2016}}
				\end{scnrelfromset}
			\end{scnindent}
			\scnfileitem{Компонентному проектированию \textit{интеллектуальных систем, основанных на знаниях}, посвящена работа \textit{Borisov2014}, в которой обосновывается необходимость накопления и повторного использования различных компонентов \textit{интеллектуальных систем}, предлагаются возможные решения данной проблемы с использованием \textit{онтологий}.}
			\begin{scnindent}
				\begin{scnrelfromset}{смотрите}
					\scnitem{\scncite{Borisov2014}}
				\end{scnrelfromset}
			\end{scnindent}
			\scnfileitem{Состояние работ англоязычных авторов, посвященных вопросам решения задач в \textit{системах, основанных на знаниях}, и актуальных на момент начала 1990-х годах, отражено в обзорных публикациях \textit{Dutta1993}, \textit{Pau1990}. Более поздние англоязычные работы в данной области в основном ориентированы на решение конкретных частных \textit{задач} в системах, построенных на основе стандартов \textit{W3C}.}
			\begin{scnindent}
				\begin{scnrelfromset}{смотрите}
					\scnitem{\scncite{Dutta1993}}
					\scnitem{\scncite{Pau1990}}
				\end{scnrelfromset}
			\end{scnindent}
			\scnfileitem{Таким образом, можно сказать, что существует ряд конкретных разработок в направлении построения \textit{комплексных технологий разработки интеллектуальных систем} различных классов, в том числе с использованием \textit{библиотек многократно используемых компонентов}, однако проблема разработки комплексной технологии построения \textit{гибридных решателей задач} в рамках рассмотренных подходов не решена. Во многом это обусловлено отсутствием унифицированной формальной основы для представления любых \textit{видов знаний}, в том числе различного рода программ, отсутствием строгих принципов, регламентирующих процесс построения \textit{решателей задач}, а также средств поддержки разработчиков таких \textit{решателей задач} и их компонентов.}
		\end{scnrelfromvector}
	\end{scnindent}
	\end{scnsubstruct}

	\scnsectionheader{Предлагаемый подход к разработке гибридных решателей задач ostis-систем и обработке информации в ostis-системах}
	\begin{scnsubstruct}

	\scnheader{гибридный решатель задач}
	\begin{scnrelfromlist}{принципы лежащие в основе}
		\scnfileitem{В качестве основы для построения модели гибридного \textit{решателя задач} предлагается использовать \textit{многоагентный подход}. Данный подход позволяет обеспечить основу для построения параллельных асинхронных систем, имеющих распределенную архитектуру, повысить модифицируемость и производительность разработанных \textit{решателей задач}.}
		\scnfileitem{Процесс решения любой \textit{задачи} предлагается декомпозировать на \textit{логически атомарные действия}, что также позволит обеспечить совместимость и модифицируемость \textit{решателей задач}.}
		\scnfileitem{\textit{решатель задач} (как объединенный, так и \textit{решатель задач} частного вида) предлагается рассматривать как иерархическую систему, состоящую из нескольких взаимосвязанных уровней. Такой подход позволяет обеспечить возможность проектирования, отладки и верификации компонентов на разных уровнях независимо от других уровней, что существенно упрощает задачу построения \textit{решателя задач} за счет снижения накладных расходов.}
		\scnfileitem{Предлагается записывать \textit{всю} информацию о решателе и решаемых им задачах при помощи \textit{SC-кода} в той же \textit{базе знаний}, что и собственно предметные \textit{знания} системы.}
		\begin{scnindent}
			\begin{scnrelfromlist}{включение}
				\scnfileitem{\textit{Спецификация агентов}, входящих в состав \textit{решателя задач}.}
				\scnfileitem{\textit{Спецификация методов}, интерпретируемых \textit{агентами} \textit{решателя задач}.}
				\scnfileitem{Спецификация всех \textit{информационных процессов}, выполняемых агентами в \textit{семантической памяти}, в том числе --- конструкции, обеспечивающие синхронизацию выполнения параллельных процессов.}
				\scnfileitem{Спецификация всех \textit{задач}, на решение которых направлены указанные \textit{информационные процессы}.}
			\end{scnrelfromlist}
			\scntext{примечание}{Описание всей указанной информации в единой семантической  памяти позволит, во-первых, обеспечить независимость разрабатываемых \textit{решателей задач} от \textit{ostis-платформы}, во-вторых, обеспечить возможность системы анализировать происходящие в ней процессы, оптимизировать и синхронизировать их выполнение, то есть обеспечить \textit{рефлексивность} проектируемых \textit{интеллектуальных систем}.}
			\begin{scnindent}
				\begin{scnrelfromset}{смотрите}
					\scnitem{Универсальная модель интерпретации логико-семантических моделей ostis-систем}
				\end{scnrelfromset}
			\end{scnindent}
		\end{scnindent}
	\end{scnrelfromlist}
	
	\scnheader{многоагентный подход к обработке информации}
	\scntext{примечание}{Ориентация на \textit{многоагентный подход} как основа для построения \textit{гибридных решателей задач} обусловлена следующими основными преимуществами и принципами такого подхода.}
	\begin{scnrelfromlist}{принципы лежащие в основе}
		\scnfileitem{Автономность (независимость) \textit{агентов} в рамках такой системы, что позволяет локализовать изменения, вносимые в \textit{решатель задач} при его эволюции, и снизить соответствующие трудозатраты, а также обеспечить устойчивость такой системы к отказам некоторых агентов.}
		\scnfileitem{Децентрализация обработки, то есть отсутствие единого контролирующего центра, что также позволяет локализовать вносимые в \textit{решатель задач} изменения.}
		\scnfileitem{Возможность параллельной работы разных \textit{информационных процессов}, соответствующих как одному \textit{агенту}, так и разным агентам, как следствие, --- возможность распределенного решения задач. Однако возможность параллельного выполнения \textit{информационных процессов} подразумевает наличие средств синхронизации такого выполнения, разработка которых является отдельной задачей и подробно рассматривается ниже.}
		\scnfileitem{Активность \textit{агентов} и \textit{многоагентной системы} в целом, дающая возможность при общении с такой системой не указывать явно способ решения поставленной \textit{задачи}, а формулировать задачу в \textbf{декларативном ключе}.}
	\end{scnrelfromlist}
	\begin{scnrelfromset}{смотрите}
		\scnitem{\scncite{Wooldridge2009}}
	\end{scnrelfromset}
	
	\scnheader{многоагентная система}
	\scnsuperset{модель агента}
	\begin{scnindent}
		\scntext{примечание}{\textit{модель агента} входит в состав системы и включает классификацию \textit{агентов} и набор понятий, характеризующих каждый агент в рамках системы. В настоящее время наиболее популярной является модель \textit{BDI} (belief-desire-intention), в рамках которой предполагается описывать на соответствующих языках \scnqq{убеждения}, \scnqq{желания} и \scnqq{намерения} каждого агента системы.}
	\end{scnindent}
	\scnsuperset{модель среды}
	\begin{scnindent}
		\scntext{определение}{\textit{модель коммуникации агентов} --- это модель, в рамках которой находятся агенты, на события в которой они реагируют и в рамках которой могут осуществлять некоторые преобразования.}
		\begin{scnrelfromset}{смотрите}
			\scnitem{\scncite{Weyns2007}}
			\begin{scnindent}
				\scntext{примечание}{Приводится обзор разновидностей сред для многоагентных систем.}
			\end{scnindent}
		\end{scnrelfromset}
	\end{scnindent}
	\scnsuperset{модель коммуникации агентов}
	\begin{scnindent}
		\scntext{определение}{\textit{модель коммуникации агентов} --- это модель, в рамках которой уточняется язык взаимодействия \textit{агентов} (структура и классификация сообщений) и способ передачи сообщений между \textit{агентами}.}
	\end{scnindent}	

	\scnheader{модель коммуникации агентов}
	\scnsuperset{принципы обмена сообщениями между агентами}
	\begin{scnindent}
		\scntext{определение}{принципы обмена сообщениями между агентами --- это принципы, описывающие то, каким образом эти сообщения передаются от \textit{агента} к \textit{агенту}.}
	\end{scnindent}
	\scnsuperset{классификация, семантика и прагматика сообщениями между агентами}
	\begin{scnindent}
		\scntext{определение}{\textit{классификация, семантика и прагматика сообщениями между агентами} --- \textit{смысл} передаваемой информации и цель такого взаимодействия.}
		\scntext{примечание}{В настоящее время стандартами, описывающими структуру передаваемых агентами сообщений, являются \textit{Agent Communication Language} (\textit{ACL}), разработанный сообществом \textit{FIPA}, язык \textit{KQML}. Указанные стандарты уточняют базовые компоненты каждого сообщения (кодировка, язык сообщения, используемую онтологию понятий, получателя, отправителя и так далее), не ограничивая при этом \textit{смысл} сообщения в целом. Также для коммуникации между агентами используется язык \textit{KIF}, предназначенный для обмена \textit{знаниями} между любыми программными компонентами.}
		\begin{scnindent}
			\begin{scnrelfromset}{смотрите}
				\scnitem{\scncite{ACL}}
				\scnitem{\scncite{KIF}}
				\scnitem{\scncite{Finin1994}}
			\end{scnrelfromset}
		\end{scnindent}
	\end{scnindent}
	\scnsuperset{принципы координации деятельности агентов}
	\begin{scnindent}
		\scntext{примечание}{В литературе рассматривается большое число вариантов координации деятельности \textit{агентов}. В работе \textit{Hartung2008} предлагается выделить агенты более высокого уровня (\textit{метаагенты}), \textit{задачей} которых является сбор информации от \textit{агентов} нижнего уровня и их координация, схожие идеи высказываются в работе \textit{Sims2008}. В работах \textit{Excelente-Toledo2004}, \textit{NagendraPrasad1999} предлагаются варианты автоматического выбора оптимального механизма координации \textit{агентов} для достижения общей цели. Предлагаются также социально-психологические модели координации деятельности \textit{агентов}, например, на основе некоторых общих \scnqq{законов} или эмоций. В работе \textit{Gorodetsky2015} предложен вариант онтологии коллективного поведения автономных \textit{агентов}.}
		\begin{scnindent}
			\begin{scnrelfromset}{смотрите}	
				\scnitem{\scncite{Gorodetsky2015}}
				\scnitem{\scncite{NagendraPrasad1999}}
				\scnitem{\scncite{Excelente-Toledo2004}}	
				\scnitem{\scncite{Sims2008}}
				\scnitem{\scncite{Hartung2008}}
				\scnitem{\scncite{Rumbell2012}}
				\scnitem{\scncite{Vasconcelos2009}}
			\end{scnrelfromset}
		\end{scnindent}
	\end{scnindent}
	
	\scnheader{многоагентная система}
	\begin{scnrelfromlist}{проблемы текущего состояния}
		\scnfileitem{Жесткая ориентация большинства средств на модель \textit{BDI} приводит к существенным накладным расходам, связанным с необходимостью выражения конкретной практической \textit{задачи} в системе понятий \textit{BDI}. В то же время ориентация на модель \textit{BDI} неявно провоцирует искусственное разделение языков, описывающих собственно компоненты \textit{BDI} и знания \textit{агента} о внешней среде, что приводит к отсутствию \textit{унификации представления} и, соответственно, дополнительным накладным расходам.}
		\scnfileitem{Большинство современных средств построения \textit{многоагентных систем} ориентированы на представление \textit{знаний} \textit{агента} при помощи узкоспециализированных языков, зачастую не предназначенных для представления \textit{знаний} в широком смысле. Речь при этом идет как о знаниях агента о себе самом, так и \textit{знаниях} о внешней среде. В некоторых подходах вначале строится онтология, для создания которой, однако, часто используются средства с низкой выразительной способностью, не предназначенные для построения \textit{онтологий}. В конечном итоге такой подход приводит к сильной ограниченности возможностей построенных \textit{многоагентных систем} и их несовместимости.}
		\begin{scnindent}
			\begin{scnrelfromset}{смотрите}
				\scnitem{\scncite{Evertsz2004}}
				\scnitem{\scncite{JADE}}
			\end{scnrelfromset}
		\end{scnindent}
		\scnfileitem{Абсолютное большинство современных средств предполагает, что взаимодействие \textit{агентов} осуществляется путем обмена сообщениями непосредственно от \textit{агента} к \textit{агенту} или посредством специальных коммуникационных центров, например, в случае взаимодействия \textit{агентов} в глобальной сети. Такой подход обладает существенным недостатком, связанным с тем, что в этом случае каждый \textit{агент} системы должен иметь достаточно полную информацию о других агентах в системе, что приводит к дополнительным затратам ресурсов, кроме того, добавление или удаление одного или нескольких \textit{агентов} приводит к необходимости оповещения об этом других \textit{агентов}. Данная проблема решается путем организации общения агентов по принципу \scnqq{доски объявлений}, предполагающему, что сообщения помещаются в некоторую общую для всех агентов область, при этом каждый \textit{агент} в общем случае может не знать, какому из агентов адресовано сообщение и от какого из \textit{агентов} получено то или иное сообщение. Кроме того, в построенной таким образом системе легче обеспечивается параллельное решение несвязанных друг с другом \textit{задач}, поскольку сообщения, относящиеся к одной \textit{задаче}, будут игнорироваться агентами, решающими другую задачу. Однако данный подход не исключает проблему, связанную с необходимостью разработки специализированного языка взаимодействия \textit{агентов}, который в общем случае не связан с языком, на котором описываются \textit{знания} \textit{агента} о решаемых \textit{задачах} и окружающей среде.}
		\begin{scnindent}
			\begin{scnrelfromset}{смотрите}
				\scnitem{\scncite{Omicini1999}}
				\scnitem{\scncite{Jagannathan1989}}
			\end{scnrelfromset}
		\end{scnindent}
		\scnfileitem{Многие средства построения \textit{многоагентных систем} построены таким образом, что логический уровень взаимодействия \textit{агентов} жестко привязан к физическому уровню реализации \textit{многоагентной системы}. Например, при передаче сообщений от агента к агенту разработчику \textit{многоагентной системы} необходимо помимо семантически значимой информации указывать ip-адрес компьютера, на котором расположен \textit{агент-получатель}, кодировку, с помощью которой закодирован текст сообщения, и другую техническую информацию, обусловленную исключительно особенностями текущей реализации средств.}
		\scnfileitem{В большинстве подходов среда, с которой взаимодействуют \textit{агенты}, уточняется отдельно разработчиком для каждой \textit{многоагентной системы}, что с одной стороны, расширяет возможности применения соответствующих средств, но, с другой стороны, приводит к существенным накладным расходам и несовместимости таких многоагентных систем. Кроме того, в ряде случаев разработчик также обязан учитывать особенности технической реализации средств разработки в плане их стыковки с предполагаемой средой, в роли которой может выступать, например, локальная или глобальная сеть.}
	\end{scnrelfromlist}
	\begin{scnrelfromset}{смотрите}
		\scnitem{\scncite{Bordini2007}}
		\scnitem{\scncite{Castillo2014}}
		\scnitem{\scncite{EVE}}
		\scnitem{\scncite{Boissier2013}}
		\scnitem{\scncite{GOAL}}
		\scnitem{\scncite{Evertsz2004}}
		\scnitem{\scncite{JADE}}
		\scnitem{\scncite{GAMA}}
	\end{scnrelfromset}
	\begin{scnrelfromlist}{принципы устранения недостатков}
		\scnfileitem{Коммуникацию агентов предлагается осуществлять по принципу \textit{\scnqq{доски объявлений}}, однако в отличие от классического подхода в роли сообщений выступают спецификации в общей семантической памяти выполняемых \textit{агентами} \textit{действий}, направленных на решение каких-либо задач, а в роли среды коммуникации выступает сама эта \textit{семантическая память}.}
		\begin{scnrelfromlist}{следствие}
			\scnfileitem{Исключить необходимость разработки специализированного языка для обмена сообщениями.}
			\scnfileitem{Обеспечить \scnqq{обезличенность} общения, то есть каждый из \textit{агентов} в общем случае не знает, какие еще агенты есть в системе, кем сформулирован и кому адресован тот или иной запрос. Таким образом, добавление или удаление агентов в систему не приводит к изменениям в других \textit{агентах}, что обеспечивает модифицируемость всей системы.}
			\scnfileitem{Агентам, в том числе конечному пользователю, формулировать задачи в \textit{декларативном ключе}, то есть не указывать для каждой задачи способ ее решения. Таким образом, агенту заранее не нужно знать, каким образом система решит ту или иную задачу, достаточно лишь специфицировать конечный результат.}
			\scnfileitem{Сделать средства коммуникации \textit{агентов} и синхронизации их деятельности более понятными разработчику и пользователю системы, не требующими изучения специальных низкоуровневых типов данных и форматов сообщений. Таким образом повышается доступность предлагаемых решений широкому кругу разработчиков.}
		\end{scnrelfromlist}
	\end{scnrelfromlist}
	
	\scnheader{принцип \scnqq{доски объявлений}}
	\scntext{описание}{Следует отметить, что такой подход позволяет при необходимости организовать обмен сообщениями между \textit{агентами} напрямую и, таким образом, может являться основой для моделирования многоагентных систем, предполагающих другие способы взаимодействия между \textit{агентами}.
		\begin{itemize}
		\item в роли внешней среды для агентов выступает та же \textit{семантическая память}, в которой формулируются задачи и посредством которой осуществляется взаимодействие \textit{агентов}. Такой подход обеспечивает унификацию среды для различных систем \textit{агентов}, что, в свою очередь, обеспечивает их совместимость;
		\item спецификация каждого агента описывается средствами \textit{SC-кода} в \textit{базе знаний}, что позволяет:
			\begin{itemize}
			\item минимизировать число специализированных средств, необходимых для спецификации агентов, как языковых, так и инструментальных;
			\item с одной стороны --- минимизировать необходимую в общем случае спецификацию агента, которая включает условие его инициирования и \textit{программу}, описывающую алгоритм работы \textit{агента}, с другой стороны --- обеспечить возможность произвольного расширения спецификации для каждого конкретного случая, в том числе возможность реализации модели \textit{BDI} и других;
			\end{itemize}
		\item синхронизацию деятельности \textit{агентов} предполагается осуществлять на уровне выполняемых ими процессов, направленных на решений тех или иных задач в \textit{семантической памяти}. Таким образом, каждый агент трактуется как некий абстрактный процессор, способный решать задачи определенного класса. При таком подходе необходимо решить задачу обеспечения взаимодействия параллельных асинхронных процессов в общей \textit{семантической памяти}, для решения которой можно заимствовать и адаптировать решения, применяемые в традиционной \textit{линейной памяти}. При этом вводится дополнительный класс агентов --- \textit{метаагенты}, задачей которых является решение возникающих проблемных ситуаций, таких как \textit{взаимоблокировки};
		\item каждый \textit{информационный процесс} в любой момент времени имеет ассоциативный доступ к необходимым фрагментам \textit{базы знаний}, хранящейся в семантической памяти, за исключением фрагментов, заблокированных другими процессами в соответствии  с рассмотренным ниже механизмом синхронизации. Таким образом, с одной стороны, исключается необходимость хранения каждым агентом информации о внешней среде, с другой стороны, каждый \textit{агент}, как и в классических \textit{многоагентных системах}, обладает только частью всей информации, необходимой для решения задачи.
		\end{itemize}}
		
	\scnheader{гибридный решатель задач}
	\scntext{примечание}{Важно отметить, что в общем случае невозможно априори предсказать, какие именно знания, модели и способы решения задач понадобятся системе для решения конкретной задачи. В связи с этим необходимо обеспечить, с одной стороны, возможность доступа ко всем необходимым фрагментам \textit{базы знаний} (в пределе --- ко всей \textit{базе знаний}), с другой стороны --- иметь возможность локализовать область поиска пути решения \textit{задачи}, например, рамками одной \textit{предметной области}.}
	\scntext{примечание}{Каждый из \textit{агентов} обладает набором ключевых элементов (как правило, понятий), которые он использует в качестве отправных точек при ассоциативном поиске в рамках \textit{базы знаний}. Набор таких элементов для каждого \textit{агента} уточняется на этапах проектирования \textit{решателя задач}. Уменьшение числа ключевых элементов \textit{агента} делает его более универсальным, однако снижает эффективность его работы за счет необходимости выполнения дополнительных операций.}
	\scntext{предлагаемый подход}{Кроме \textit{многоагентного подхода}, в основу принципов решения задачи в рамках \textit{Технологии OSTIS} предлагается положить ряд идей, связанных с концепцией \textit{ситуационного управления}, рассмотренной в работе \textit{Д.А. Поспелова}.}
	\begin{scnindent}
	\begin{scnrelfromset}{смотрите}
		\scnitem{\scncite{Pospelov1986}}
	\end{scnrelfromset}
	\scntext{проблема}{До настоящего времени попытки реализации указанной концепции, несмотря на ее актуальность и востребованность, сводились к частным решениям для конкретных \textit{классов задач} и, к сожалению, не получили широкого распространения. В значительной степени это обусловлено отсутствием универсальной унифицированной основы, которая бы позволила на ее базе создавать языки ситуационного управления в применении к конкретным предметным областям и, что еще более важно, повторно использовать фрагменты описаний на таких языках.}
	\begin{scnindent}
		\scntext{решение}{Данную проблему можно решить используя предлагаемый в рамках \textit{Технологии OSTIS} \textit{SC-код} и семейство \textit{онтологий верхнего уровня}, разработанных на его основе.}
		\begin{scnindent}
			\scnrelfrom{смотрите}{Технология OSTIS}
			\begin{scnindent}
				\begin{scnrelfromlist}{принципы лежащие в основе}
					\scnfileitem{\textit{SC-код} как базовый язык для описания любой информации в \textit{базе знаний} и, соответственно, для построения языков ситуационного управления на его основе.}
					\scnfileitem{\textit{Базовая денотационная семантика \textit{SC-кода}}, которая позволяет обеспечить возможность формального уточнения всех используемых понятий в виде формального набора \textit{онтологий}, что позволяет обеспечить совместимость разрабатываемых систем и возможность повторного использования их компонентов.}
					\scnfileitem{\textit{агентно-ориентированный подход} к обработке информации, предполагающий реакцию \textit{агентов} на возникновение в \textit{базе знаний} определенных \textit{ситуаций} и \textit{событий}.}
				\end{scnrelfromlist}
			\end{scnindent}
		\end{scnindent}	
	\end{scnindent}
	\end{scnindent}

	\scnheader{обработка знаний в ostis-системах}
	\begin{scnrelfromlist}{достоинства}
		\scnfileitem{Поскольку обработка осуществляется \textit{агентами}, которые обмениваются сообщениями только через общую память, добавление нового агента или исключение (деактивация) одного или нескольких существующих \textit{агентов}, как правило, не приводит к изменениям в других \textit{агентах}, поскольку агенты не обмениваются сообщениями напрямую.}
		\scnfileitem{Инициирование \textit{агентов} осуществляется децентрализованно и чаще всего независимо друг от друга, таким образом, даже существенное расширение числа агентов в рамках одной системы не приводит к ухудшению ее производительности.}
		\scnfileitem{Спецификации \textit{агентов} и, как будет показано ниже, их программы могут быть записаны на том же языке, что и обрабатываемые знания, что существенно сокращает перечень специализированных средств, предназначенных для проектирования таких \textit{агентов} и их коллективов, а также их анализа, верификации и оптимизации, и упрощает разработку системы за счет использования более универсальных компонентов.}
	\end{scnrelfromlist}

\end{scnsubstruct}

\end{scnsubstruct}

\scnheader{решатель задач ostis-системы}
\scnidtf{совокупность всех навыков, которыми обладает ostis-система на текущий момент времени}
\scnidtf{иерархическая система навыков, которыми обладает ostis-система}
\scnrelto{семейство подмножеств}{навык}
\scntext{примечание}{С учетом того тезиса, что существуют \textit{методы} интепретации других \textit{методов} и, следовательно, иерархия \textit{методов}, а также, соответственно, иерархия \textit{навыков}, можно уточнить и понятие решателя задач, как \uline{иерархической системы навыков}. Таким образом, определим \textit{решатель задач ostis-системы} определяется как совокупность всех \textit{навыков}, которыми обладает ostis-система на текущий момент времени.}
\scntext{примечание}{Предлагаемый в рамках \textit{Технологии OSTIS} подход к построению решателей задач позволяет обеспечить их модифицируемость, что, в свою очередь, позволяет \textit{ostis-системе} при необходимости легко приобретать новые \textit{навыки}, модифицировать (совершенствовать) уже имеющиеся и даже избавляться от некоторых навыков с целью повышения производительности системы. Таким образом, имеет смысл говорить не о жестко фиксированном решателе задач, который разрабатывается один раз при создании первой версии системы и далее не меняется, а о совокупности навыков, фиксированной в каждый текущий момент времени, но постоянно эволюционирующей.}
\scnsuperset{объединенный решатель задач ostis-системы}
\begin{scnindent}
	\scnidtf{полный решатель задач ostis-системы}
	\scnidtf{интегрированный решатель задач ostis-системы}
	\scnidtf{решатель задач ostis-системы, реализующий все ее функциональные возможности, как основные, так и вспомогательные}
	\scntext{пояснение}{В общем случае \textit{объединенный решатель задач ostis-системы} решает задачи, связанные с:
		\begin{itemize}
			\item обеспечением основных функциональных возможностей системы (например, решение явно сформулированных задач по требованию пользователя);
			\item обеспечением корректности и оптимизацией работы самой ostis-системы (перманентно на протяжении всего жизненного цикла ostis-системы);
			\item обеспечением повышения квалификации конечных пользователей и разработчиков ostis-системы;
			\item обеспечением автоматизации развития и управления развитием ostis-системы.
		\end{itemize}}
\end{scnindent}
\scnsuperset{гибридный решатель задач ostis-системы}
\begin{scnindent}
	\scnidtf{решатель задач ostis-системы, реализующий две и более модели решения задач}
\end{scnindent}

\scnheader{машина обработки знаний}
\scnsubset{sc-агент}
\scntext{пояснение}{Под \textit{машиной обработки знаний} будем понимать совокупность интерпретаторов всех \textit{навыков}, составляющих некоторый \textit{решатель задач}. С учетом многоагентного подхода к обработке информации, используемого в рамках Технологии OSTIS, \textit{машина обработки знаний} представляет собой \textit{sc-агент} (чаще всего --- \textit{неатомарный sc-агент}), в состав которого входят более простые sc-агенты, обеспечивающие интерпретацию соответствующего множества \textit{методов}. Таким образом, \textit{машина обработки знаний} в общем случае представляет собой иерархическую систему \textit{sc-агентов}.}

\scnheader{решатель задач ostis-системы}
\scnhaselement{Решатель задач Метасистемы OSTIS}
\scnsuperset{решатель задач вспомогательной ostis-системы}
\begin{scnindent}
	\scnsuperset{решатель задач интерфейса компьютерной системы}
		\begin{scnindent}
			\begin{scnsubdividing}
				\scnitem{решатель задач пользовательского интерфейса компьютерной системы}
				\scnitem{решатель задач интерфейса компьютерной системы с другими компьютерными системами}
				\scnitem{решатель задач интерфейса компьютерной системы с окружающей средой}
			\end{scnsubdividing}
		\end{scnindent}
	\scnsuperset{решатель задач ostis-подсистемы поддержки проектирования компонентов определенного класса}
	\begin{scnindent}
		\scnsuperset{решатель задач ostis-подсистемы поддержки проектирования баз знаний}
			\begin{scnindent}
				\scnsuperset{решатель задач повышения качества базы знаний}
					\begin{scnindent}
						\scnsuperset{решатель задач верификации базы знаний}
							\begin{scnindent}
								\scnsuperset{решатель задач поиска и устранения некорректностей в базе знаний}
								\scnsuperset{решатель задач поиска и устранения неполноты}
							\end{scnindent}
						\scnsuperset{решатель задач оптимизации структуры базы знаний}
						\scnsuperset{решатель задач выявления и устранения информационного мусора}
					\end{scnindent}
			\end{scnindent}
		\scnsuperset{решатель задач ostis-подсистемы поддержки проектирования решателей задач ostis-систем}
		\begin{scnindent}
			\begin{scnsubdividing}
				\scnitem{решатель задач ostis-подсистемы поддержки проектирования программ обработки знаний}
				\scnitem{решатель задач ostis-подсистемы поддержки проектирования агентов обработки знаний}
			\end{scnsubdividing}
		\end{scnindent}
	\end{scnindent}
	\scnsuperset{решатель задач подсистемы управления проектирования компьютерных систем и их компонентов}
\end{scnindent}
\scnsuperset{решатель задач самостоятельной ostis-системы}

\scnheader{Классификация решателей задач ostis-систем по типу интерпретируемой модели решения задач}
\begin{scnsubstruct}
\scnheader{решатель задач ostis-системы}
\scnsuperset{решатель задач с использованием хранимых методов}
\begin{scnindent}
	\scnidtf{решатель, способный решать задачи тех классов, для которых на данный момент времени известен соответствующий метод решения}
	\scnsuperset{решатель задач на основе нейросетевых моделей}
	\scnsuperset{решатель задач на основе генетических алгоритмов}
	\scnsuperset{решатель задач на основе императивных программ}
	\begin{scnindent}
		\scnsuperset{решатель задач на основе процедурных программ}
		\scnsuperset{решатель задач на основе объектно-ориентированных программ}
	\end{scnindent}
	\scnsuperset{решатель задач на основе декларативных программ}
	\begin{scnindent}
		\scnsuperset{решатель задач на основе логических программ}
		\scnsuperset{решатель задач на основе функциональных программ}
	\end{scnindent}
\end{scnindent}
\scnsuperset{решатель задач в условиях, когда метод решения задач данного класса в текущий момент времени не известен}
\begin{scnindent}
	\scnidtf{решатель, реализующий стратегии решения задач, позволяющие породить метод решения задачи, который в текущий момент времени не известен ostis-системе}
	\scnidtf{решатель, использующий для решения задач метаметоды, соответствующие более общим классам задач по отношению к заданной}
	\scnidtf{решатель задач, позволяющий породить метод, который является частным по отношению к какому-либо известному ostis-системе методу и интерпретируется соответствующей машиной обработки знаний}
	\scnsuperset{решатель, реализующий стратегию поиска путей решения задачи в глубину}
	\scnsuperset{решатель, реализующий стратегию поиска путей решения задачи в ширину}
	\scnsuperset{решатель, реализующий стратегию проб и ошибок}
	\scnsuperset{решатель, реализующий стратегию разбиения задачи на подзадачи}
	\scnsuperset{решатель, реализующий стратегию решения задач по аналогии}
	\scnsuperset{решатель, реализующий концепцию интеллектуального пакета программ}
\end{scnindent}
\end{scnsubstruct}

\scnheader{Классификация машин обработки знаний, которые в общем случае могут соответствовать одним и тем же фрагментам базы знаний, но при этом в совокупности с ними образовывать разные навыки и соответственно разные решатели задач}
\begin{scnsubstruct}
\scnheader{машина обработки знаний}
\scnsuperset{машина логического вывода}
\begin{scnindent}
	\scnsuperset{машина дедуктивного вывода}
	\begin{scnindent}
		\scnsuperset{машина прямого дедуктивного вывода}
		\scnsuperset{машина обратного дедуктивного вывода}
	\end{scnindent}
	\scnsuperset{машина индуктивного вывода}
	\scnsuperset{машина абдуктивного вывода}
	\scnsuperset{машина нечеткого вывода}
	\scnsuperset{машина вывода на основе логики умолчаний}
	\scnsuperset{машина логического вывода с учетом фактора времени}
\end{scnindent}
\end{scnsubstruct}

\scnheader{Классификация решателей задач ostis-систем по типу решаемой задачи (цели решения задачи)}
\begin{scnsubstruct}
\scnheader{решатель задач ostis-системы}
\scnsuperset{решатель задач информационного поиска}
\begin{scnindent}
	\begin{scnsubdividing}
		\scnitem{решатель задач поиска информации, удовлетворяющей заданным критериям}
		\scnitem{решатель задач поиска информации, не удовлетворяющей заданным критериям}
	\end{scnsubdividing}
\end{scnindent}
\scnsuperset{решатель явно сформулированных задач}
\begin{scnindent}
	\scnidtf{решатель задач, для которых явно сформулирована цель}
	\scnsuperset{решатель задач поиска или вычисления значений заданного множества величин}
	\scnsuperset{решатель задач установления истинности заданного логического высказывания в рамках заданной формальной теории}
	\scnsuperset{решатель задач формирования доказательства заданного высказывания в рамках заданной формальной теории}
	\scnsuperset{машина верификации ответа на указанную задачу}
	\scnsuperset{машина верификации решения указанной задачи}
	\begin{scnindent}
		\scnsuperset{машина верификации доказательства заданного высказывания в рамках заданной формальной теории}
	\end{scnindent}
\end{scnindent}
\scnsuperset{решатель задач классификации сущностей}
\begin{scnindent}
	\scnsuperset{машина соотнесения сущности с одним из заданного множества классов}
	\scnsuperset{машина разделения множества сущностей на классы по заданному множеству признаков}
\end{scnindent}
\scnsuperset{решатель задач синтеза информационных конструкций}
\begin{scnindent}
	\scnsuperset{решатель задач синтеза естественно-языковых текстов}
	\scnsuperset{решатель задач синтеза изображений}
	\scnsuperset{решатель задач синтеза сигналов}
	\begin{scnindent}
		\scnsuperset{решатель задач синтеза речи}
	\end{scnindent}
\end{scnindent}
\scnsuperset{решатель задач анализа информационных конструкций}
\begin{scnindent}
	\scnsuperset{решатель задач анализа естественно-языковых текстов}
	\begin{scnindent}
		\scnsuperset{решатель задач понимания естественно-языковых текстов}
		\scnsuperset{решатель задач верификации естественно-языковых текстов}
	\end{scnindent}
	\scnsuperset{решатель задач анализа изображений}
	\begin{scnindent}
		\scnsuperset{решатель задач сегментации изображений}
		\scnsuperset{решатель задач понимания изображений}
	\end{scnindent}
	\scnsuperset{решатель задач анализа сигналов}
	\begin{scnindent}
		\scnsuperset{решатель задач анализа речи}
		\begin{scnindent}
			\scnsuperset{решатель задач понимания речи}
		\end{scnindent}
	\end{scnindent}
\end{scnindent}
\end{scnsubstruct}

\scnheader{Язык SCP} 
\scntext{примечание}{\textit{Язык SCP} позволяет установить границу между логико-семантической моделью \textit{ostis-системы} и \textit{ostis-платформой}. В связи с этим будем считать платформенно-независимыми \textit{абстрактные sc-агенты}, реализованные на \textit{Языке SCP} или более высокоуровневых языках на его основе, а платформенно-зависимыми \textit{абстрактные sc-агенты}, которые реализованы на уровне \textit{ostis-платформы} (например, с целью повышения их производительности). В то же время существует ряд \textit{абстрактных sc-агентов}, которые принципиально не могут быть реализованы на \textit{Языке SCP}.}


\scnsectionheader{Актуальные проблемы и перспективы развития технологий разработки гибридных решателей задач}
\begin{scnsubstruct}
\scntext{примечание}{В предметной области был детально рассмотрен подход к построению \textit{решателей задач}, позволяющий решить ряд фундаментальных проблем в области построения \textit{решателей задач}, таких как обеспечение совместимости различных \textit{решателей задач} и их компонентов, а также обеспечение обучаемости (модифицируемости и рефлексивности) самих \textit{решателей задач}. В то же время существует существует ряд проблем, остающихся актуальными и требующих решения.}
\begin{scnsubdividing}
\scnfileitem{Первая проблема связана с отсутствием достаточно строгой формализованной классификации задач, решаемых интеллектуальными системами, отсутствием унификации описания задач и классов задач, описания целей, хода и результата решения задачи, методов решения задач, связей между классами задач и методами решения задач данного класса. Решение данной проблемы, с одной стороны, позволит обеспечить возможность глубокой интеграции всевозможных \textit{моделей решения задач} различных классов и возможность облегчить процесс интеграции новых моделей решения задач в интеллектуальную систему, а с другой стороны, станет предпосылкой для решения других проблем, описанных ниже.}
\scnfileitem{Вторая проблема заключается в том, что на настоящий момент основное внимание в области разработки \textit{гибридных решателей задач} уделено снижению трудоемкости интеграции различных компонентов решателя задач в \textit{интеллектуальную систему} и реализации возможности накопления многократно используемых компонентов \textit{решателей задач}, однако в общем случае не говорится о том, как конкретно \textit{интеллектуальная система} будет применять те или иные компоненты при решении задач конкретных классов. Таким образом, построение общего плана решения задачи, то есть выбор методов решения задач, определение порядка их применения и выбор исходных данных (аргументов) для применения того или иного метода, фактически определяется разработчиком на этапе проектирования системы или на этапе ее эволюции в процессе эксплуатации. Предпосылкой для решения данной проблемы является решение ранее рассмотренной проблемы унификации представления задач различных классов и методов их решения. Решение же рассматриваемой проблемы предполагает разработку комплекса \textit{стратегий решения задач} (или \textit{метаметодов решения задач}), которые позволят \textit{интеллектуальной системе} самостоятельно формировать план решения задачи с учетом имеющихся в системе методов решения задач и, при возможности, даже запрашивать недостающие для решения задачи компоненты в соответствующих библиотеках. Следует отметить, что попытки разработки универсальных высокоуровневых подходов к решению задач предпринимались еще на заре развития \textit{Искусственного интеллекта}, в 1950-60ые годы, однако не увенчались успехов и вскоре прекратились. Во многом это связано с отсутствием на тот момент унифицированных моделей представления и обработки знаний, которые в настоящий момент предлагаются в рамках \textit{Технологии OSTIS}.}
\scnfileitem{Еще одна актуальная проблема, тесно связанная с рассмотренными выше, заключается в том, что интеллектуальные системы часто вынуждены решать задачи в условиях так называемых не-факторов, то есть неполноты описания задачи и возможных путей ее решения, нечеткости и некорректности имеющихся знаний, отсутствия критериев для оценки оптимальности полученного решения и т.д. В особенности это актуально при решении поведенческих задач, связанных с изменением состояния объектов среды, внешней по отношению к интеллектуальной системе. Для решения задач в подобных условиях интеллектуальная система должна не только обладать достаточным набором компонентов решателя задач, реализующих модели решения задач в условиях наличия не-факторов (нечеткие логические модели, модели машинного обучения, генетические алгоритмы и так далее), но и реализовывать \textit{стратегии решения задач}, которые бы позволили принимать решения и формировать \textit{план решения задачи} в такого рода условиях.}
\begin{scnindent}
	\begin{scnrelfromset}{смотрите}
		\scnitem{\scncite{Narinjani2004}}
	\end{scnrelfromset}
\end{scnindent}
\scnfileitem{В случае же распределенного коллектива интеллектуальных систем важнейшей проблемой является не просто обеспечение возможности решения задач таким коллективом в текущий момент времени, а перманентная поддержка семантической совместимости и, как следствие, интероперабельности систем, входящих в такой коллектив на протяжении всего их жизненного цикла. Очевидно, что каждая из систем, входящих в такой коллектив, и, соответственно, ее \textit{решатель задач} может эволюционировать независимо от других систем, но при этом всегда должна сохраняться \textit{интероперабельность} между системами, в противном случае решение задач в таком коллективе станет невозможным. Решение данной проблемы предполагает разработку методов перманентного анализа \textit{семантической совместимости} распределенного коллектива взаимодействующих интеллектуальных систем, выявления и устранения проблем.}
\end{scnsubdividing}
\begin{scnindent}
	\begin{scnrelfromlist}{решение проблем}
		\scnfileitem{Необходимо разработать комплексную онтологию действий, задач и методов их решения, а также онтологию \textit{гибридных решателей задач} на основе которой уточнить понятие решателя и его архитектуру. На основе первой версии \textit{Глобальной предметной области действий и задач и соответствующей ей онтологии методов и технологий} предлагается разработать комплексную онтологию действий и задач, решаемых \textit{ostis-системами}.}
		\begin{scnindent}
			\begin{scnrelfromset}{смотрите}
				\scnitem{Формализация понятий действия, задачи, метода, средства, навыка и технологии}
			\end{scnrelfromset}
		\end{scnindent}
		\scnfileitem{Необходимо разработать комплекс унифицированных обобщенных стратегий (метаметодов) решения задач в интеллектуальных системах, позволяющий интеллектуальной системе самостоятельно формировать план решения задачи с учетом имеющихся в системе \textit{методов решения задач}. В основу разрабатываемых стратегий кроме опыта аналогичных работ предлагается внести также некоторые общеметодологические идеи, связанные с \textit{Теорией бихевиоризма} и набирающими популярность идеями ее применения в информатике, ТРИЗ, а также \textit{СМД-методологией}, предложенной школой Г. П. Щедровицкого.}
		\begin{scnindent}
			\begin{scnrelfromset}{смотрите}
				\scnitem{\scncite{Pavel2015}}
				\scnitem{\scncite{Altshuller2010}}
				\scnitem{\scncite{Cao2010}}
				\scnitem{\scncite{Cao2014}}
				\scnitem{\scncite{Shhedrovickij1995}}
			\end{scnrelfromset}
		\end{scnindent}
		\scnfileitem{Необходимо разработать онтологическую модель формирования плана решения задачи и управления процессом решения задач в гибридных решателях задач в условиях различных не-факторов и отсутствия четких критериев оценки оптимальности полученного решения. Для разработки данной модели предлагается адаптировать теорию \textit{ситуационного управления}, и реализовать ее в контексте семантической теории \textit{решателей задач}, разрабатываемой в рамках \textit{Технологии OSTIS}.}
		\begin{scnindent}
			\begin{scnrelfromset}{смотрите}
				\scnitem{\scncite{Pospelov1986}}
			\end{scnrelfromset}
		\end{scnindent}
		\scnfileitem{Необходимо разработать комплексную онтологическую модель управления информационными процессами решения задач в интеллектуальных системах, построенных на базе унифицированных семантических моделей представления и обработки информации.}
		\scnfileitem{Необходимо разработать онтологическую модель платформы интерпретации унифицированных семантических моделей представления и обработки информации (\textit{ostis-платформы}).}
		\begin{scnindent}
			\begin{scnrelfromset}{смотрите}
				\scnitem{Универсальная модель интерпретации логико-семантических моделей ostis-систем}
			\end{scnrelfromset}
		\end{scnindent}
		\scnfileitem{Необходимо разработать комплексную иерархическую модель \textit{гибридного решателя задач}, основанную на многоагентном подходе и учитывающую необходимость решения задач как в рамках одиночных интеллектуальных систем, так и в рамках \uline{распределенных} \uline{коллективов интероперабельных} \uline{интеллектуальных систем}.}
		\scnfileitem{Необходимо разработать комплекс методов анализа качества \textit{гибридных решателей задач} и их компонентов.}
		\scnfileitem{Необходимо разработать комплекс методик и средств поддержки проектирования \textit{гибридных решателей задач}.}
		\begin{scnindent}
			\begin{scnrelfromset}{смотрите}
				\scnitem{Методика и средства компонентного проектирования решателей задач ostis-систем}
			\end{scnrelfromset}
		\end{scnindent}
	\end{scnrelfromlist}
	\scntext{примечание}{Рассмотренные проблемы связаны в первую очередь с процессом решения конкретной задачи интеллектуальной системой. В то же время очевидно, что в каждый момент времени интеллектуальная система вынуждена параллельно решать несколько задач, которые могут быть связаны как с непосредственным функциональным назначением системы, так и с обеспечением жизнедеятельности и эволюции самой системы. Во втором случае имеются в виду, в частности, задачи, связанные с актуализацией имеющихся у нее сведений о внешнем мире, поиском и устранением ошибок в базе знаний, оптимизацией структуры \textit{базы знаний} и \textit{решателя задач} системы, поиском и устранением информационного мусора и многие другие. При этом разные задачи могут иметь разный приоритет, который может меняться в зависимости от ситуации даже в процессе ее решения. В то же время, в ситуации, когда априори не известно, какой из возможных способов решения задачи окажется наиболее эффективным, может оказаться целесообразным параллельное использование нескольких подходов к решению одной и той же задачи. Таким образом, актуальной является проблема организации управления информационными процессами решения задач в интеллектуальной системе и взаимодействия параллельно выполняемых информационных процессов с учетом приоритетности процессов, возможности отслеживать текущее состояние \textit{информационных процессов}, порождать, приостанавливать и уничтожать информационные процессы. Для решения данной проблемы целесообразно заимствовать решения, широко используемые в традиционных компьютерных системах, в частности, реализуемые в современных операционных системах, и адаптировать их к специфике решения задач в интеллектуальных системах. Важно отметить, что реализация модели управления информационными процессами на основе общих унифицированных моделей обработки информации, предлагаемых в рамках \textit{Технологии OSTIS}, позволит сделать одни информационные процессы объектом анализа других информационных процессов, что, в свою очередь, даст возможность анализировать ход решения задачи непосредственно в процессе решения, оценивать эффективность тех или иных методов решения задач, накапливать наиболее удачные решения для применения в дальнейшем для решения аналогичных задач и многое другое.}
	\begin{scnrelfromvector}{примечание}
		\scnfileitem{Решение перечисленных проблем позволит разработать принципиально новую иерархическую модель \textit{гибридного решателя задач}, обладающую рядом существенных преимуществ, которая, в свою очередь, должна будет интерпретироваться на каких-либо платформах. Без унификации требований к \textit{ostis-платформе} и четкого разделения платформенно-независимой модели системы (и в частности решателя) и \textit{ostis-платформы} невозможно говорить о реализации модели \textit{решателя задач}, реализующей рассмотренные выше идеи. Это приведет к необходимости дублирования одних и тех же компонентов модели для разных платформ, значительно усложнит интеграцию компонентов \textit{решателя задач}, поскольку потребует учета при такой интеграции особенностей каждой \textit{ostis-платформы}. Кроме того, четкое разделение уровня модели системы и уровня \textit{ostis-платформы} даст возможность независимо друг от друга развивать различные платформы и модели интеллектуальных систем. Таким образом, предлагается сформулировать унифицированные требования к \textit{ostis-платформе}, а также построить общую модель такой \textit{ostis-платформы}, удовлетворяющую указанным требованиям.}
		\begin{scnindent}
			\begin{scnrelfromset}{смотрите}
				\scnitem{Универсальная модель интерпретации логико-семантических моделей ostis-систем}
			\end{scnrelfromset}
		\end{scnindent}
		\scnfileitem{С другой стороны, как уже было сказано, \textit{решатель задач} представляет собой сложную систему, ориентированную на работу со знаниями, а не с данными, в отличие от современных программных систем, в которых изначально известно, где конкретно локализованы нужные данные и в какой форме они представлены. В связи с этим, применение для разработки интеллектуальных систем современных программно-аппаратных платформ, ориентированных на адресный доступ к хранящимся в памяти данным, не всегда оказывается эффективным, поскольку при разработке интеллектуальных систем фактически приходится моделировать нелинейную память на базе линейной. Повышение эффективности решения задач интеллектуальными системами требует разработки специализированных платформ, в том числе аппаратных, ориентированных на унифицированные семантические модели представления и обработки информации. В качестве основы для таких разработок предлагается использовать предложенную в рамках \textit{Технологии OSTIS} общую концепцию \textit{ассоциативного семантического компьютера}, \textit{семантической памяти} и базового языка программирования, ориентированного на обработку информации в такой памяти, и дополнить их идеями \textit{волновых языков программирования}, \textit{инсерционного программирования} и других подходов, направленными на повышение эффективности обработки знаний, в том числе на аппаратном уровне.}
		\begin{scnindent}
			\begin{scnrelfromset}{смотрите}
				\scnitem{Ассоциативные семантические компьютеры для ostis-систем}
			\end{scnrelfromset}
		\end{scnindent}
	\end{scnrelfromvector}
\end{scnindent}
\end{scnsubstruct}

\bigskip

\scnheader{Принципы решения задач распределенными коллективами ostis-систем}
\scntext{примечание}{Разработка \textit{решателей задач} \textit{интеллектуальных систем} на настоящий момент как правило рассматриваются в контексте одиночных (самостоятельных) интеллектуальных систем, функционирующих в некоторой среде (частью которой является и пользователь, если он есть). В то же время очевидна тенденция современных информационных технологий к переходу от одиночных систем к коллективам распределенных взаимодействующих компьютерных систем, в частности, к распределенному хранению данных и распределенным вычислениям.}
\begin{scnsubdividing}
	\scnfileitem{В случае интеллектуальных компьютерных систем важнейшим свойством систем, входящих в такие коллективы, становится \uline{\textit{интероперабельность}}, то есть способность системы к согласованному взаимодействию с другими подобными системами с целью решения каких-либо задач. Таким образом, особо актуальным является переход от разработки \textit{решателей задач} отдельно взятых интеллектуальных систем к решателям задач взаимодействующих \textit{интероперабельных интеллектуальных систем}, включая разработку принципов решения задач в таких распределенных коллективах с учетом решения всех обозначенных выше проблем.}
	\scnfileitem{Важно отметить, что полностью отказаться от распределенности при решении задач даже в сравнительно простых прикладных системах нельзя, поскольку часто интеллектуальные системы вынуждены использовать различные датчики и эффекторы, которые с точки зрения общей архитектуры являются некоторыми внешними модулями (внешними агентами) и, таким образом, привносят распределенность в общую архитектуру системы.
		\\Для решения данной проблемы предлагается рассмотреть такую систему взаимодействующих \textit{интеллектуальных компьютерных систем} как \textit{многоагентную систему} и уточнить принцип поведения \textit{агентов} в такой системе.}
	\scnfileitem{Таким образом, можно говорить о двух видах многоагентных систем в рамках \textit{Технологии OSTIS}:
		\begin{itemize}
			\item внутренняя система sc-агентов над общей sc-памятью в рамках некоторой ostis-системы;
			\item распределенная система ostis-систем в рамках Экосистемы OSTIS.
		\end{itemize}}
	\scnfileitem{В обоих случаях можно говорить об \uline{иерархии агентов}:
		\begin{itemize}
			\item в рамках внутренней системы sc-агентов выделяются \textit{атомарные абстрактные sc-агенты} и \textit{неатомарные абстрактные sc-агенты}, кроме того существует иерархия sc-агентов с точки зрения языка интерпретации методов;
			\item в рамках \textit{Экосистемы OSTIS} выделяются как \textit{индивидуальные ostis-системы}, так и \textit{коллективные ostis-системы}, которые в свою очередь могут состоять как из \textit{индивидуальных ostis-систем}, так и \textit{коллективных ostis-систем}.
		\end{itemize}}
	\begin{scnindent}
		\begin{scnrelfromset}{смотрите}
			\scnitem{Семантически совместимые ostis-системы}
			\scnitem{Внутренние агенты, выполняющие действия в sc-памяти}
		\end{scnrelfromset}
	\end{scnindent}
	\scnfileitem{Ключевым отличием \textit{распределенной системы ostis-систем} от \textit{внутренней системы sc-агентов} в рамках \textit{индивидуальной ostis-системы} является отсутствие общей памяти, хранящей общую для всех \textit{sc-агентов} \textit{базу знаний} и выступающей в роли среды для коммуникации \textit{sc-агентов}. В общем случае в качестве средства коммуникации между агентами в рамках выделенных систем агентов может использоваться:
		\begin{itemize}
			\item Общая нераспределенная (монолитная) память, как в случае \textit{sc-агентов} над \textit{sc-памятью};
			\item Общая распределенная память. В этом случае с логической точки зрения агенты могут считать, что по-прежнему работают над общей памятью, в рамках которой хранится вся доступная база знаний, однако реально \textit{база знаний} будет распределена между несколькими \textit{ostis-системами} и выполняемые преобразования должны будут синхронизироваться между этими ostis-системами;
			\item Специализированные каналы связи. Очевидно, что при решении задачи в распределенном коллективе \textit{ostis-систем} должны существовать языковые и технические средства, позволяющие осуществлять передачу сообщений от одной \textit{ostis-системы} к другой.
		\end{itemize}}
	\scnfileitem{Все перечисленные средства коммуникации в зависимости от класса решаемой задачи, требуемых для ее решения \textit{знаний} и \textit{навыков}, а также существующего (доступного) в данный момент набора \textit{ostis-систем} могут \uline{комбинироваться}.}
	\scnfileitem{В основу решения задач в рамках \textit{распределенного коллектива ostis-систем} предлагается положить идею максимально возможной \uline{унификации} и \uline{конвергенции} принципов решения задач в рамках \textit{индивидуальной ostis-системы} и \textit{распределенного коллектива ostis-систем}. Такой подход обладает следующим важным достоинством: если общие принципы решения задач не зависят от того, какой конкретно набор \textit{ostis-систе}м участвует в решении той или иной задачи, то становится возможным легко переходить от \textit{индивидуальной ostis-системы} к \textit{распределенному коллективу ostis-систем} при ее усложнении без необходимости существенно пересматривать коллектив \textit{агентов}, входящих в состав такой \textit{ostis-системы} и заново продумывать используемый подход к решению задач того или иного класса.}
	\scnfileitem{Для перехода от \textit{индивидуальной ostis-системы} к \textit{коллективной ostis-системе} достаточно выполнить несколько шагов.}
	\begin{scnindent}
		\begin{scnsubdividing}
			\scnfileitem{Разделить множество классов задач, решаемых данной \textit{ostis-системой}, на семейство подмножеств, каждое из которых обладает некоторой логической целостностью, критерии которой в общем случае определяются разработчиком. При этом указанные подмножества могут пересекаться, но при объединении должны давать исходное множество, таким образом необходимо построить одно из возможных \textit{покрытий*} для множества классов задач, решаемых данной \textit{ostis-системой}.}
			\scnfileitem{Для каждого из выделенных подмножеств необходимо сформировать множество \textit{знаний} и \textit{навыков}, необходимых для решения задач данного множества классов. При этом в общем случае может оказаться необходимым пересмотр иерархии навыков и соответствующих им sc-агентов, в частности, преобразование некоторых атомарных sc-агентов в неатомарные. Теоретически избежать такой ситуации невозможно, однако подобные ситуации можно практически исключить на этапе проектирования решателей задач индивидуальных ostis-систем, делая иерархию агентов достаточно глубокой и ставя в соответствие \textit{атомарным sc-агентам} такие \textit{классы задач}, разделение которых на подклассы с практической точки зрения не имеет смысла. 
				\\Аналогичная ситуация может возникнуть и при выделении фрагментов \textit{базы знаний}. В этом случае может потребоваться пересмотр иерархии \textit{предметных областей} и \textit{онтологий} и, возможно, выделение новых предметных областей. Как и в случае с \textit{решателями задач}, избежать такой ситуации на практике возможно в случае, если иерархия предметных областей будет достаточно глубокой для того, чтобы выделение более частных предметных областей было практически нецелесообразным.}
			\scnfileitem{Каждое сформированное таким образом множество знаний и навыков становится соответственно базой знаний и решателем задач новой ostis-системы, которая будет способна реализовать только часть функциональных возможностей исходной ostis-системы.}
		\end{scnsubdividing}
		\scntext{примечание}{Такое разделение может выполняться итерационно и для полученных \textit{ostis-систем} в общем случае неограниченное количество раз, создавая на каждой итерации новое \scnqq{поколение} ostis-систем, полученное путем декомпозиции исходной \textit{ostis-системы}.}
	\end{scnindent}
	\scnfileitem{Таким образом, предлагаемая идея унификации принципов решения задач в \textit{ostis-системах} любого рода позволяет 
		\begin{itemize}
			\item с практической точки зрения снять ограничение на расширение функциональных возможностей (обучение) не только \textit{индивидуальной ostis-системы}, но и \textit{коллективной ostis-системы}, позволяя, таким образом, постоянно наращивать функциональные возможности \textit{Экосистемы OSTIS} в целом. 
			\item с теоретической (архитектурной) точки зрения говорить о \uline{фрактальном} характере не только внутренней организации \textit{ostis-систем} но и \textit{коллективов ostis-систем}, что, в свою очередь, позволяет обеспечить возможность наследования и других принципов построения \textit{индивидуальных ostis-систем} в \textit{распределенных коллективах ostis-систем}, включая, например, методику проектирования \textit{ostis-систем} и их компонентов и соответствующие средства, а также принципы синхронизации соответствующих sc-агентам параллельных \textit{информационных процессов}.
		\end{itemize}}
	\scnfileitem{В основе взаимодействия \textit{sc-агентов} в рамках \textit{индивидуальной ostis-системы} лежит уточненный принцип \scnqq{доски объявлений} при котором агенты взаимодействуют посредством общей для них sc-памяти. Для реализации той же идеи в случае \textit{распределенной коллективной ostis-системы} необходимо выбрать какую-либо sc-память для выполнения данной роли.}
	\scnfileitem{При решении задач в \textit{распределенном коллективе ostis-систем} возможны два варианта организации взаимодействия \textit{агентов} (которыми являются и сами \textit{ostis-системы})}
	\begin{scnindent}
		\begin{scnsubdividing}
			\scnfileitem{Если решаемая задача достаточно сложная и требует частого обращения к нескольким отдельным \textit{ostis-системам}, то целесообразно путем объединения раздельных ostis-систем создать \textit{временную ostis-систему}, где все sc-агенты, входившие в состав исходных \textit{ostis-систем}, становятся внутренними, и принципы организации их взаимодействия известны. В этом случае существенно снижаются затраты на решение задачи, но появляются накладные расходы на создание таких \textit{временных ostis-систем}. Таким образом, необходимо отдельно разработать критерии на основании которых будет приниматься решение о целесообразности такого объединения. Отметим, что для того, чтобы иметь возможность сохранить результат и ход решения задачи для последующего применения целесообразно осуществлять объединение \textit{ostis-систем} на базе одной из \textit{ostis-систем}, входящих в такое объединение, а не создавать совершенно новую \textit{ostis-систему}. При этом в такую систему будут копироваться знания и навыки из объединяемых систем, а сами эти объединяемые системы могут вообще никак не меняться. Тогда после решения задачи из исходной \textit{ostis-системы} необходимо будет исключить те \textit{навыки} и \textit{знания}, которые были нужны только для решения данной задачи.}
			\scnfileitem{Важно отметить, что описанная интеграция \textit{ostis-систем} благодаря особенностям их архитектуры выполняется значительно проще, чем в других компьютерных системах, поскольку принципы построения и баз знаний, и \textit{решателей задач} \textit{ostis-систем} изначально предполагают возможность неограниченного расширения имеющихся в системе знаний и навыков без необходимости внесения изменений в уже имеющуюся \textit{базу знаний} и \textit{решатель задач}. Таким образом, интеграция двух ostis-систем при условии их семантической совместимости сводится к обычному теоретико-множественному объединению их \textit{баз знаний} и \textit{решаталей задач} и последующему исключению продублированных компонентов. Благодаря этому создание таких временных ostis-систем может выполняться \uline{автоматически}, что делает применение такого подхода к организации решения задач целесообразным во многих случаях.}
			\scnfileitem{Другой возможный вариант предполагает, что в качестве среды для взаимодействия \textit{sc-агентов} (как внешних, так и внутренних, внешняя \textit{ostis-система} с точки зрения процесса решения задачи также рассматривается как \textit{sc-агент}) выбирается \textit{sc-память} одной из \textit{ostis-систем}, входящих в состав \textit{коллектива ostis-систем}. Предлагаются следующие критерии выбора этой sc-памяти:
			\begin{itemize}
				\item Если задача решается неоднократно в рамках некоторого \textit{ostis-сообщества} (\textit{сообщества ostis-систем} и их пользователей), то для координации действий \textit{sc-агентов} выбирается \textit{sc-память} \textit{корпоративной ostis-системы} для данного \textit{ostis-сообщества};
				\item Если \textit{коллектив ostis-систем} для решения данной задачи формируется временно (разово), то для координации действий \textit{sc-агентов} выбирается \textit{sc-память} той \textit{ostis-системы}, которая инициировала решение данной задач.
			\end{itemize}}
		   	\scnfileitem{Недостатком данного варианта является наличие затрат на коммуникацию между \textit{ostis-системами}. Если по каким-либо причинам эти затраты велики (например, из-за низкого качества соединения между системами), то более целесообразно использовать первый из предложенных вариантов.}
		\end{scnsubdividing}
		\begin{scnrelfromset}{смотрите}
			\scnitem{Предлагаемый подход к разработке гибридных решателей задач ostis-систем и обработке информации в ostis-системах}
			\scnitem{Иерархическая система взаимодействующих ostis-сообществ}
		\end{scnrelfromset}
	\end{scnindent}
	\scnfileitem{В любом из предложенных вариантов в конечном итоге определяется некоторая конкретная sc-память, которая становится средой для взаимодействия агентов, осуществляющих решение задачи, по изложенным принципам. Тогда можно уточнить понятие \textit{sc-агента} как компонента решателя задач в контексте распределенного решения задач \textit{коллективом ostis-систем} и считать sc-агентом не только компонент \textit{решателя задач индивидуальной ostis-системы}, но и любую ostis-систему, входящую в постоянный либо временный \textit{коллектив ostis-систем}, решающих какие-либо задачи, поскольку принципы взаимодействия \textit{ostis-систем} в таком коллективе полностью совпадают с принципами взаимодействия \textit{sc-агентов} в составе \textit{решателя задач} \textit{индивидуальной ostis-системы}.}
	\begin{scnindent}
		\begin{scnrelfromset}{смотрите}
			\scnitem{Предлагаемый подход к разработке гибридных решателей задач ostis-систем и обработке информации в ostis-системах}
		\end{scnrelfromset}
	\end{scnindent}
	\scnfileitem{Таким образом, можно говорить о фрактальной иерархической структуре распределенного \textit{гибридного решателя задач}, в рамках которой выделяется два варианта иерархии \textit{sc-агентов}.}
		\begin{scnindent}
		\begin{scnsubdividing}	
			\scnfileitem{Иерархия sc-агентов с точки зрения уровня \textit{языков представления методов}, на которых представлены соответствующие этим sc-агентам методы. В рамках этой иерархии в свою очередь можно выделить три уровня, имеющих важные отличия.}
			\begin{scnindent}
				\begin{scnsubdividing}
				\scnfileitem{Уровень \textit{sc-агентов} \textit{ostis-платформы}, обеспечивающий интерпретацию методов платформенно-независимого уровня в рамках \textit{индивидуальной ostis-системы}, в рамках которого может выделяться иерархия языков представления методов уровня ostis-платформы и соответствующих средств их интерпретации.}
				\scnfileitem{Уровень \textit{платформенно-независимых sc-агентов} в рамках \textit{индивидуальной ostis-системы}, в рамках которого может выделяться иерархия платформенно-независимых языков представления методов.}
				\scnfileitem{Уровень \textit{распределенных коллективов ostis-систем}, на котором также можно говорить о \textit{языках представления методов} и их иерархии, но при этом в общем случае даже отдельные методы могут физически храниться распределенно в разных ostis-системах. Например, можно говорить о \textit{языке представления методов} для финансовой деятельности крупных предприятий, но при этом целесообразно выделять подъязыки для описания деятельности отделов различных категорий и иметь отдельные ostis-системы для обслуживания каждого из отделов.}
				\end{scnsubdividing}
			\end{scnindent}
			\scnfileitem{Иерархия sc-агентов с точки зрения атомарности/неатомарности в рамках \uline{одного} \textit{языка представления методов}.}
			\begin{scnindent}
				\scntext{примечание}{Формирование такой иерархии может быть целесообразным на любом уровне языка \textit{языка представления методов}.}
				\begin{scnsubdividing}
				\scnfileitem{\textit{атомарные платформенно-зависимые sc-агенты} и \textit{неатомарные платформенно-зависимые sc-агенты} на уровне \textit{ostis-платформы}.}
				\scnfileitem{\textit{атомарные платформенно-независимые sc-агенты} и \textit{неатомарные платформенно-независимые sc-агенты} на платформенно-независимом уровне в рамках индивидуальной ostis-системы.}
				\scnfileitem{\textit{индивидуальные ostis-системы} и \textit{коллективны ostis-систем} на уровне решения задач в рамках \textit{Экосистемы OSTIS}.}
			   \end{scnsubdividing}
			\end{scnindent}
		\end{scnsubdividing}	
		\end{scnindent}
	\scnfileitem{Дальнейшее развитие представленных принципов решения задач распределенными коллективами ostis-систем предполагает:
		\begin{itemize}
			\item Разработку формальных критериев для оценки целесообразности или нецелесообразности формирования временных индивидуальных ostis-систем.
			\item Разработку языка и принципов обмена сообщениями между ostis-системами, входящими в коллектив ostis-систем, решающий какую-либо задачу. Несмотря на то, что с логической точки зрения каждая ostis-система трактуется как sc-агент и принципы их взаимодействия остаются теми же, реализация, например, возможности реагирования на события в базе знаний и внесения изменений в эту базу знаний для внутренних sc-агентов и внешних ostis-систем будет отличаться и требует уточнения.
		\end{itemize}}
\end{scnsubdividing}

\end{scnsubstruct}

\scntext{заключение}{В данной предметной области рассмотрены актуальные на сегодняшний день проблемы в области разработки \textit{гибридных решателей задач} и предложен общий подход к построению \textit{гибридных решателей задач}, который решает такие проблемы, как обеспечение совместимости и модифицируемости \textit{решателей задач}, а также создает предпосылки к решению других актуальных проблем.}
\begin{scnindent}
	\scnrelfrom{смотрите}{Актуальные проблемы и перспективы развития технологий разработки гибридных решателей задач}
	\begin{scnrelfromlist}{направления развития}
		\scnfileitem{Более тесно и полно интегрировать идеи ситуационного управления в предлагаемый подход.}
		\scnfileitem{Доработать предложенный механизм блокировок, в частности, минимизировать число классов блокировок, учесть и реализовать идеи реализации lock-free алгоритмов.}
		\scnfileitem{Исключить необходимость введения \textit{sc-метаагентов} и \textit{scp-метапрограмм}.}
		\scnfileitem{Доработать \textit{Язык SCP} до того, чтобы иметь возможность описывать в рамках \textit{scp-программ} рецепторное и эффекторное взаимодействие \textit{ostis-систем}.}
		\scnfileitem{При разработке \textit{Абстрактной scp-машины} учесть принципы построения волновых языков программирования и идеи инсерционного программирования и моделирования.}
		\begin{scnindent}
			\begin{scnrelfromset}{смотрите}
				\scnitem{\scncite{Letichevskij2003}}
				\scnitem{\scncite{Letichevskij2012}}
				\scnitem{\scncite{Moldovan1985}}
				\scnitem{\scncite{Sapatyj1986}}
			\end{scnrelfromset}
		\end{scnindent}
	\end{scnrelfromlist}
\end{scnindent}

\end{SCn}

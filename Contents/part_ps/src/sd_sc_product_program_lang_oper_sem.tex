\begin{SCn}
\scnsectionheader{Предметная область и онтология операционной семантики sc-языков продукционного программирования}
\begin{scnsubstruct}

\scnheader{Предметная область операционной семантики sc-языков продукционного программирования}
\scniselement{предметная область}
\begin{scnhaselementrolelist}{максимальный класс объектов исследования}
    \scnitem{язык продукционного программирования}
    \scnitem{продукция}
\end{scnhaselementrolelist}

\begin{scnrelfromlist}{соавтор}
    \scnitem{Орлов М.К.}
    \scnitem{Зотов Н.В.}
\end{scnrelfromlist}

\begin{scnrelfromvector}{библиография}
    \scnitem{\scncite{Forgy1982}}
    \scnitem{\scncite{Vladimirov2010}}
\end{scnrelfromvector}

\scnheader{Алгоритм Rete}
\scnexplanation{\textbf{\textit{Алгоритм Rete}} содержит обобщение логики функционала, ответственного за связь данных (фактов) и алгоритма (продукций) в системах сопоставления с образцом (вид систем: системы основанные на правилах). Продукция состоит из одного или нескольких условий и набора действий, выполняемых если актуальный набор фактов соответствует одному из условий. Условия накладываются на атрибуты фактов, включая их типы и идентификаторы.}
\begin{scnrelfromset}{преимущества}
    \scnfileitem{\textit{Алгоритм Rete} уменьшает или исключает избыточность условий за счет объединения узлов.}
    \scnfileitem{\textit{Алгоритм Rete} сохраняет частичные соответствия между фактами при слиянии разных типов фактов. Это позволяет избежать полного вычисления всех фактов при любом изменении в рабочей памяти продукционной системы. Система работает только с самими изменениями.}
    \scnfileitem{\textit{Алгоритм Rete} позволяет эффективно высвобождать память при удалении фактов.}
\end{scnrelfromset}
\scnnote{\textit{Алгоритм Rete} широко используется для реализации сопоставления с образцом в системах с циклом сопоставление-решение-действие для генерации и логического вывода.}
\scnrelfrom{смотрите}{\cite{Forgy1982}}

\scnheader{Алгоритм Rete II}
\begin{scnrelfromset}{преимущества}
    \scnfileitem{Повышена общая производительность сети включая хешированную память для больших массивов данных.}
    \scnfileitem{Добавлен алгоритм обратного вывода, работающий на той же сети.}
    \scnfileitem{Скорость обратного вывода по сравнению с \textit{Rete I} повышена значительно.}
\end{scnrelfromset}

\scnheader{Алгоритм Rete-NT}
\begin{scnindent}
    \begin{scnrelfromset}{преимущества}
        \scnfileitem{Алгоритм был признан в 10 раз более быстрым, чем его предшественник \textit{Алгоритм Rete II}}
    \end{scnrelfromset}
\end{scnindent}

\scnheader{Миварный подход}
\scnnote{\textbf{\textit{Миварный подход}} объединяет и другие научные области компьютерных наук, информатики и дискретной математики, включая: базы данных, экспертные системы, системы логического вывода на основе развития продукций, теорию графов, матрицы, параллельное выполнение программ на кластерах, проектирование новых архитектур компьютеров, массовое суммирование чисел, техническую защиту информации и информационную безопасность, гносеологию (частично и в плане создания новой наиболее мощной модели данных на основе "тройки"{} "вещь-свойство-отношение"{}), сервисно-ориентированные архитектуры, компьютерные сети, информационные инфраструктуры, теоретическую робототехнику, многоагентные системы и некоторые другие.}
\scnrelfrom{смотрите}{\cite{Vladimirov2010}}
\scnnote{\textit{Миварный подход} объединяет две основные технологии накопления данных и обработки информации:
\begin{itemize}
    \item{миварное информационное пространство: накопление данных на основе эволюционной самоорганизующейся миварной модели данных с изменяющейся структурой в теории баз данных,}
    \item{миварные сети: обработка информации на основе развития продукционного подхода к логическому выводу с учетом включения возможности автоматического конструирования алгоритмов для "решателей задач"{} и традиционной вычислительной обработки, а также с использованием идей отношений, правил и процедур, которые теперь принято относить к сервисно-ориентированным архитектурам и многоагентным системам.}
\end{itemize}}
\scnnote{Суть \textit{миварного подхода} в объединении баз данных и систем логико-вычислительной обработки в единые эволюционно развивающиеся системы, позволяющие собрать воедино все различные научные разработки на основе сервисно-ориентированных архитектур и технологий интеллектуальных агентов --- многоагентных систем.}
\scnexplanation{Миварные сети основаны на продукционном подходе "если, то…"{} с переходом к более сложной структуре правил с предусловиями, условиями, ограничениями, действиями и последействиями. Это позволяет записывать все причинно-следственные отношения, включая и все возможные формы предикатов и подобных логических выражений. Значение предикатов и поиска истинных выражений не отрицается, а только создается возможность и для их реализации, и для реализации всех возможных других представлений правил в виде: сервисов, процедур, продукций, подпрограмм и так далее. Такой подход позволяет работать одновременно с разными описаниями предметных областей, прибавляя к предикатам и продукции, и нейросети, и генетические алгоритмы, и традиционные вычислительные процедуры, и все другие в виде универсальных миварных отношений, которые представляются и хранятся перед обработкой в миварном пространстве.}

\bigskip
\end{scnsubstruct}
\scnendsegmentcomment{Предметная область и онтология операционной семантики sc-языков продукционного программирования}

\end{SCn}

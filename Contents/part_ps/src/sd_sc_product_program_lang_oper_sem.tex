\begin{SCn}
\scnsectionheader{Предметная область и онтология операционной семантики sc-языков продукционного программирования}
\begin{scnsubstruct}

\scnheader{Предметная область операционной семантики sc-языков продукционного программирования}
\scniselement{предметная область}
\begin{scnrelfromlist}{соавтор}
    \scnitem{Орлов М. К.}
    \scnitem{Зотов Н. В.}
\end{scnrelfromlist}

\begin{scnrelfromvector}{библиография}
    \scnitem{\scncite{Forgy1982}}
    \scnitem{\scncite{Vladimirov2010}}
\end{scnrelfromvector}

\scnheader{Алгоритм Rete}
\scnnote{Содержит обобщение логики функционала, ответственного за связь данных (фактов) и алгоритма (продукций) в системах сопоставления с образцом (вид систем: системы основанные на правилах). Продукция состоит из одного или нескольких условий и набора действий, выполняемых если актуальный набор фактов соответствует одному из условий. Условия накладываются на атрибуты фактов, включая их типы и идентификаторы.}
\begin{scnrelfromset}{характеристики}
    \scnfileitem{Уменьшает или исключает избыточность условий за счет объединения узлов.}
    \scnfileitem{Сохраняет частичные соответствия между фактами при слиянии разных типов фактов. Это позволяет избежать полного вычисления всех фактов при любом изменении в рабочей памяти продукционной системы. Система работает только с самими изменениями.}
    \scnfileitem{Позволяет эффективно высвобождать память при удалении фактов.}
\end{scnrelfromset}
\scnrelfrom{решаемая задача}{реализация сопоставления с образцом в системах с циклом сопоставление-решение-действие для генерации и логического вывода}
\scnrelfrom{библиографический источник}{\cite{Forgy1982}}
\scnrelfrom{улучшенная версия}{Алгоритм Rete II}
\begin{scnindent}
    \begin{scnrelfromset}{отличительные параметры}
        \scnfileitem{Повышена общая производительность сети включая хешированную память для больших массивов данных.}
        \scnfileitem{Добавлен алгоритм обратного вывода, работающий на той же сети.}
        \scnfileitem{Скорость обратного вывода по сравнению с \textit{Rete I} повышена значительно.}
    \end{scnrelfromset}
    \scnrelfrom{улучшенная версия}{Алгоритм Rete-NT}
    \begin{scnindent}
        \begin{scnrelfromset}{отличительные параметры}
            \scnfileitem{Алгоритм был признан в 10 раз более быстрым, чем его \textit{Алгоритм Rete II}}
        \end{scnrelfromset}
    \end{scnindent}
\end{scnindent}

\scnheader{Миварный подход}
\begin{scnrelfromset}{объединяемые области}
    \scnitem{база данных}
    \scnitem{экспертная система}
    \scnitem{система логического вывода на основе развития продукций}
    \scnitem{теория графов}
    \scnitem{матрица}
    \scnitem{параллельное выполнение программ на кластерах}
    \scnitem{проектирование новых архитектур компьютеров}
    \scnitem{массовое суммирование чисел}
    \scnitem{техническая защита информации}
    \scnitem{информационная безопасность}
    \scnitem{гносеологию}
    \scnitem{сервисно-ориентированная архитектура}
    \scnitem{компьютерная сеть}
    \scnitem{информационная инфраструктура}
    \scnitem{теоретическая робототехника}
    \scnitem{многоагентная система}
\end{scnrelfromset}
\scnrelfrom{библиографический источник}{\cite{Vladimirov2010}}
\begin{scnrelfromset}{объединяемые технологии накопления данных и обработки информации}
    \scnitem{миварное информационное пространство}
    \begin{scnindent}
        \scnnote{Накопление данных на основе эволюционной самоорганизующейся миварной модели данных с изменяющейся структурой в теории баз данных.}
    \end{scnindent}
    \scnitem{миварные сети}
    \begin{scnindent}
        \scnnote{Обработка информации на основе развития продукционного подхода к логическому выводу с учетом включения возможности автоматического конструирования алгоритмов для "решателей задач"{} и традиционной вычислительной обработки, а также с использованием идей отношений, правил и процедур, которые теперь принято относить к сервисно-ориентированным архитектурам и многоагентным системам.}
    \end{scnindent}
\end{scnrelfromset}
\scntext{суть подхода}{объединение баз данных и систем логико-вычислительной обработки в единые эволюционно развивающиеся системы, позволяющие собрать воедино все различные научные разработки на основе сервисно-ориентированных архитектур и технологий интеллектуальных агентов --- многоагентных систем}
\scnrelfrom{используемый подход}{продукционный подход}
\scntext{отличия от продукционного подхода}{Переход к более сложной структуре правил с предусловиями, условиями, ограничениями, действиями и последействиями. Это позволяет записывать все причинно-следственные отношения, включая и все возможные формы предикатов и подобных логических выражений. Значение предикатов и поиска истинных выражений не отрицается, а только создается возможность и для их реализации, и для реализации всех возможных других представлений правил в виде: сервисов, процедур, продукций, подпрограмм и так далее. Такой подход позволяет работать одновременно с разными описаниями предметных областей, прибавляя к предикатам и продукции, и нейросети, и генетические алгоритмы, и традиционные вычислительные процедуры, и все другие в виде универсальных миварных отношений, которые представляются и хранятся перед обработкой в миварном пространстве.}

\bigskip
\end{scnsubstruct}
\scnendsegmentcomment{Предметная область и онтология операционной семантики sc-языков продукционного программирования}

\end{SCn}

\begin{SCn}
\scnsectionheader{\currentname}
\begin{scnsubstruct}
\scnheader{Предметная область операционной семантики языка SCP}
\scniselement{предметная область}
\scnhaselementrole{класс объектов исследования}{Абстрактная scp-машина}
\begin{scnhaselementrolelist}{ключевой объект исследования}
\scnitem{Абстрактный sc-агент создания scp-процессов;}
\scnitem{Абстрактный sc-агент интерпретации scp-операторов;}
\scnitem{Абстрактный sc-агент синхронизации процесса интерпретации scp-программ;}
\scnitem{Абстрактный sc-агент уничтожения scp-процессов;}
\scnitem{Абстрактный sc-агент синхронизации событий в sc-памяти и ее реализации;}
\scnitem{Абстрактный sc-агент трансляции сформированной спецификации события в sc-памяти во внутреннее представление;}
\scnitem{Абстрактный sc-агент обработки события в sc-памяти, инициирующего агентную scp-программу}
\end{scnhaselementrolelist}
\scnheader{Абстрактная scp-машина}
\begin{scnreltoset}{декомпозиция абстрактного sc-агента}
%TODO: check by human--->
\scnitem{Абстрактный sc-агент создания scp-процессов}
\scnitem{Абстрактный sc-агент интерпретации scp-операторов}
\scnitem{Абстрактный sc-агент синхронизации процесса интерпретации scp-программ}
\scnitem{Абстрактный sc-агент уничтожения scp-процессов}
\scnitem{Абстрактный sc-агент синхронизации событий в sc-памяти и ее реализации}
%<---TODO: check by human
\end{scnreltoset}
\scnheader{Абстрактный sc-агент создания scp-процессов}
\scntext{пояснение}{Задачей \textit{Абстрактного} \textit{sc-агента создания scp-процессов}является создание \textit{scp-процессов}, соответствующих заданной\textit{scp-программе}. Данный \textit{\mbox{sc-агент}} активируется при появлениив \textit{sc-памяти} \textit{инициированного действия}, принадлежащегоклассу \textit{действие интерпретации scp-программы}.После проверки \textit{sc-агентом} условия инициирования выполняетсясоздание \textit{scp-процесса} с учетов конкретных параметровинтерпретации \textit{\mbox{scp-программы}}, после чего осуществляется поиск\textit{начального оператора\scnrolesign \mbox{scp-процесса}} и добавление его во множество\textit{настоящих сущностей}.}\scnheader{Абстрактный sc-агент интерпретации scp-операторов}
\scntext{пояснение}{Задачей \textit{Абстрактного sc-агента интерпретации scp-операторов}является собственно интерпретация операторов \textit{scp-программы}, тоесть выполнение в \textit{sc-памяти} действий, описываемых конкретным\textit{\mbox{scp-оператором}}. Данный \textit{sc-агент} активируется при появлениив \textit{sc-памяти} \textit{scp-оператора}, принадлежащего классу\textit{настоящих сущностей}. После выполнения действия, описываемого\textit{scp-оператором}, \textit{scp-оператор} добавляется во множество\textit{прошлых сущностей}. В случае когда семантика действия,описываемого \textit{\mbox{scp-оператором}}, предполагает возможность ветвления\textit{scp-программы} после выполнения данного \textit{\mbox{scp-оператора}}, тоиспользуется одно из подмножеств класса \textit{выполненных действий --безуспешно выполненное действие} или \textit{успешно выполненноедействие}.}\scnheader{Абстрактный sc-агент синхронизации процесса интерпретации scp-программ}
\scntext{пояснение}{Задачей \textit{Абстрактного sc-агента синхронизации процессаинтерпретации scp-программ} является обеспечение переходов между\textit{scp-операторами} в рамках одного \textit{scp-процесса}. Данный\textit{sc-агент} активизируется при добавлении некоторого\textit{scp-оператора} во множество \textit{прошлых сущностей}. Далееосуществляется переход по \textit{sc-дуге}, принадлежащей отношению\textit{последовательность действий*} (или более частным отношениям, вслучае, если \textit{\mbox{scp-оператор}} был добавлен во множество \textit{успешновыполненных действий} или \textit{безуспешно выполненных действий}). Приэтом очередной \textit{scp-оператор} становится \textit{настоящей сущностью}(активным \textit{scp-оператором}) в том случае, если хотя бы один\textit{scp-оператор}, связанный с ним входящими \textit{sc-дугами},принадлежащими отношению \textit{последовательность действий*} (или болеечастным отношениям), стал \textit{прошлой сущностью} (или, соответственно,подмножеством прошлых сущностей). В случае, когда необходимо дождатьсязавершения выполнения всех предыдущих операторов, для синхронизациииспользуется оператор класса \textit{конъюнкция предшествующихоператоров}.}\scnheader{Абстрактный sc-агент уничтожения scp-процессов}
\scntext{пояснение}{Задачей \textit{Абстрактного sc-агента уничтожения scp-процессов}является уничтожение \textit{scp-процесса}, т. е. удаление из\textit{sc-памяти} всех \textit{sc-элементов}, его составляющих. Данный\textit{sc-агент} активируется при появлении в \textit{sc-памяти}\textit{scp-процесса}, принадлежащего множеству \textit{прошлых сущностей}.При этом уничтожаемый \textit{scp-процесс} необязательно должен бытьполностью сформирован. Необходимость уничтожения не до концасформированного \textit{scp-процесса} может возникнуть в случае, если присоздании \textit{scp-процесса} возникли проблемы, не позволяющиепродолжить создание \textit{scp-процесса} и его выполнение.}\scnheader{Абстрактный sc-агент синхронизации событий в sc-памяти и ее реализации}
\scntext{пояснение}{Задачей \textit{Абстрактного sc-агента синхронизации событий вsc-памяти и ее реализации} является обеспечение работы \textit{неатомарныхsc-агентов}, реализованных на \textit{языке SCP}.}\begin{scnreltoset}{декомпозиция абстрактного sc-агента}
%TODO: check by human--->
\scnitem{Абстрактный sc-агент трансляции сформированной спецификации события в sc-памяти во внутреннее представление}
\scnitem{Абстрактный sc-агент обработки события в sc-памяти,инициирующего агентную scp-программу}
%<---TODO: check by human
\end{scnreltoset}
\scnheader{Абстрактный sc-агент трансляции сформированной спецификации события в sc-памяти во внутреннее представление}
\scntext{пояснение}{Задачей \textit{\textbf{Абстрактного sc-агента трансляции сформированной спецификации события в sc-памяти во внутреннее представление}}является трансляция ориентированных пар, описывающих \textit{первичноеусловие инициирования*} некоторого \textit{\mbox{sc-агента}} во внутреннеепредставление элементарных событий на уровне \textit{\mbox{sc-хранилища}}, приусловии, что этот \textit{sc-агент} реализован на платформенно-независимомуровне (с использованием \textit{языка SCP}). Условием инициированияданного \textit{sc-агента} является появление в \textit{\mbox{sc-памяти}} новогоэлемента множества \textit{активных sc-агентов}, для которого будетнайдена и протранслирована соответствующая ориентированная пара.}\scnheader{Абстрактный sc-агент обработки события в sc-памяти,инициирующего агентную scp-программу}
\scntext{пояснение}{Задачей \textit{Абстрактного sc-агента обработки события в sc-памяти,инициирующего агентную \mbox{scp-программу}}, является поиск \textit{агентнойscp-программы}, входящей во множество \textit{программ sc-агента*} длякаждого \textit{sc-агента}, первичное условие инициирования которогосоответствует событию, произошедшему в \textit{sc-памяти}, а такжегенерация и инициирование действия, направленного на интерпретацию этойпрограммы. В результате работы данного \textit{sc-агента} в\textit{sc-памяти} появляется \textit{инициированное действие},принадлежащее классу \textit{действие} \textit{интерпретации scp-программы.}}
\end{scnsubstruct}
\end{SCn}
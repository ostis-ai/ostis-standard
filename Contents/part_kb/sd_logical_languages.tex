\begin{SCn}

    \scnsectionheader{\currentname}
    
    \scnstartsubstruct
    
    \scnheader{Предметная область логических языков}
    \scniselement{предметная область}
    \scnsdmainclasssingle{логический язык}
    %\scnsdclass{логический язык}
    \scnsdrelation{логический язык*}
    
    \scnheader{логический язык}
    \scnidtf{формальный язык}
    \scnsubset{язык}
    \scndefinition{искусственный язык логики, предназначенный для воспроизведения логических форм контекстов естественного языка, а также выражения логических законов и способов правильных рассуждений в логических теориях, строящихся в данном языке.}
    
    \scnheader{логический язык*}
    \scnidtf{быть языком логики*} 
    \scnidtf{быть языком логической модели*}
    
    \bigskip
    \scnendstruct
    
    \scnstartsubstruct
    
    \scnheader{Предметная область языка логики высказываний}
    \scniselement{предметная область}
    \scnsdmainclasssingle{Язык логики высказываний}
    %\scnsdclass{Язык логики высказываний}
    
    \scnheader{Язык логики высказываний}
    \scniselement{логический язык}
    \scnsubset{язык представления методов}
    \scndefinition{формальный язык, предназначенный для анализа логической структуры сложных высказываний.}
    
    \bigskip
    \scnendstruct
    
    \scnstartsubstruct
    
    \scnheader{Предметная область языка логики предикатов}
    \scniselement{предметная область}
    \scnsdmainclasssingle{Язык логики предикатов}
    %\scnsdclass{Язык логики предикатов}
    
    \scnheader{Язык логики предикатов}
    \scniselement{логический язык}
    \scnsubset{язык представления методов}
    \scndefinition{формальный язык, предназначенный для анализа логической структуры простых высказываний.}
    
    \bigskip
    \scnendstruct
    
    \scnstartsubstruct
    
    \scnheader{Предметная область языка логического вывода}
    \scniselement{предметная область}
    \scnsdmainclasssingle{логический вывод}
    \scnsdclass{правило вывода;процесс логического вывода;аксиомная схема;предпосылка;заключение}
    \scnsdrelation{выводимость*;правила вывода*;равносильные преобразования*;аксиомные схемы логики*;применение аксиомной схемы*}
    
    \scnheader{предпосылка}
    \scndefinition{исходное суждение логического вывода}
    
    \scnheader{заключение}
    \scndefinition{новое суждение логического вывода}
    
    \scnheader{логический вывод}
    \scndefinition{рассуждение, которое описывает переход от предпосылок к заключениям.}
    
    \scnheader{процесс логического вывода}
    \scnsubset{процесс}
    \scndefinition{процесс рассуждения, в ходе которого осуществляется переход от некоторых предпосылок к заключениям.}
    
    \scnheader{выводимость*}
    \scndefinition{отношение, существующее между предпосылками и заключением рассуждения.}
    \scnexplanation{Над формулами исчисления с помощью правил вывода задается отношение выводимости.}
    
    \scnheader{правило вывода}
    \scnidtf{правило преобразования некоторой формальной системы}
    \scndefinition{допустимые способы переходов от некоторой совокупности утверждений, называемых посылками, к некоторому определённому утверждению — заключению.}
    
    \scnheader{правила вывода*}
    \scnidtf{быть правилами вывода логики*} 
    \scnidtf{быть правилами вывода логической модели*}
    \scndefinition{отношение, существующее между логикой и правилом вывода.}
    
    \scnheader{равносильные преобразования*}
    \scndefinition{отношение, существующее между логикой и преобразованием формул логики, которые принимают одинаковые логические значения при любом наборе значений входящих в формулы элементарных высказываний.}
    
    \scnheader{аксиомная схема}
    \scndefinition{формула, верная без доказательства, переменные которой понимаются как произвольные формулы.}
    
    \scnheader{аксиомные схемы логики*}
    \scndefinition{отношение, существующее между логикой и аксиомной схемой.}
    
    \scnheader{применение аксиомной схемы*}
    \scndefinition{отношение, существующее между формулой  с аксиомной схемой и результатом применения аксимной схемы.}
    
    \scnheader{следует отличать*}
    \scnhaselementset{логический вывод
    	;процесс логического вывода
    	;выводимость*
    }
    \scnhaselementset{правило вывода
    	;правила вывода*
    }
    \scnhaselementset{правила вывода*
    	;равносильные преобразования*
    }
    
    \scnheader{Примеры логических выводов}
    \scneqtoset{\scgfileitem{figures/sd_logical_languages/example_logical_conclusion_1.png}\\
    	\scnaddlevel{1}
    	\scnrelfrom{описание примера}{Пример логического вывода в логике высказываний, который является доказательством формулы (A$\rightarrow$A).}
    	\scnaddlevel{-1}
    	;
    	\scgfileitem{figures/sd_logical_languages/example_logical_conclusion_2.png}\\
    	\scnaddlevel{1}
    	\scnrelfrom{описание примера}{Пример логического вывода в логике высказываний, который является доказательством формулы (A $\vdash$ (A$\rightarrow$B).}
    	\scnaddlevel{-1}
    }
    
    \bigskip
    \scnendstruct
    
    \scnstartsubstruct
    
    \scnheader{Предметная область логических моделей решения задач}
    \scniselement{предметная область}
    \scnsdmainclasssingle{логическая модель решения задач}
    \scnsdclass{логический метод решения задач;предикат;логика высказываний;логика предикатов;нечеткая логика;темпоральная логика}
    
    \scnheader{логическая модель решения задач}
    \scnrelto{включение}{метод решения задач}
    \scnidtf{логика}
    \scnidtf{метаметод интерпретации соответствующего класса логических методов}
    
    \scnheader{логический метод решения задач}
    \scnrelto{включение}{метод решения задач}
    \scnidtf{логический вывод}
    \scnexplanation{\textit{вид знаний}, хранимых в \textit{памяти кибернетической системы} и содержащих информацию, которой достаточно либо для сведения каждой \textit{задачи} из соответствующего \textit{класса логических задач} к \textit{полной системе подзадач*}, решение которых гарантирует решение исходной \textit{задачи}, \uline{либо} для окончательного решения этой \textit{задачи} из указанного \textit{класса логических задач}}
    
    \scnheader{логика высказываний}
    \scniselement{логическая модель решения задач}
    \scnidtf{раздел символической логики, изучающий сложные высказывания, образованные из простых, и их взаимоотношения.}
    \scnexplanation{Правила вывода: Modus ponens и правило обобщения, правила введения и удаления для пропозициональных связок, правила противоречия и сведения к противоречию.}
    \scnrelfromset{правила вывода}{Modus ponens*
    	;правило монотонности*
    }
    \scnrelfrom{равносильные преобразования}{Множество равносильных преобразований логики высказываний}
    \scnrelfrom{логический язык}{Язык логики высказываний}
    \scnrelfrom{аксиомные схемы логики}{Множество аксиомных схем логики высказываний}
    
    \scnheader{Множество равносильных преобразований логики высказываний}
    \scnhaselement{ассоциативность*}
    \scnhaselement{коммутативность*}
    \scnhaselement{дистрибутивность*}
    \scnhaselement{идемпотентность*}
    \scnhaselement{двойное отрицание*}
    \scnhaselement{правило де Моргана*}
    \scnhaselement{свойство констант*}
    \scnhaselement{закон противоречия*}
    \scnhaselement{закон исключения третьего*}
    \scnhaselement{выражение связок*}
    \scnhaselement{поглощение*}
    \scnhaselement{склеивание*}
    \scnhaselement{правило перевертывания*}
    
    \scnheader{Множество аксиомных схем логики высказываний}
    \scnhaselement{аксиомня схема введения импликации}
    \scnhaselement{аксиомня схема введения конъюнкции}
    \scnhaselement{аксиомня схема введения дизъюнкции}
    \scnhaselement{аксиомня схема введения отрицания}
    \scnhaselement{аксиомня схема введения эквивалентности}
    \scnhaselement{аксиомня схема удаления импликации}
    \scnhaselement{аксиомня схема удаления конъюнкции}
    \scnhaselement{аксиомня схема удаления дизъюнкции}
    \scnhaselement{аксиомня схема удаления отрицания}
    \scnhaselement{аксиомня схема удаления эквивалентности}
    \scnhaselement{аксиомня схема введения эквивалентности}
    \scnhaselement{аксиомня схема удаления импликации}
    
    \scnheader{логика предикатов}
    \scniselement{логическая модель решения задач}
    \scnidtf{раздел символической логики, изучающий рассуждения и другие языковые контексты с учётом внутренней структуры входящих в них простых высказываний, при этом выражения языка трактуются функционально, то есть как знаки некоторых функций или же как знаки аргументов этих функций.}
    \scnrelfromset{правила вывода}{Modus ponens*
    	;правило удаления существования*
    	;правило введения всеобщности*
    }
    \scnrelfrom{равносильные преобразования}{Множество равносильных преобразований логики предикатов}
    \scnrelfrom{логический язык}{Язык логики предикатов}
    \scnrelfrom{аксиомные схемы логики}{Множество аксиомных схем логики предикатов}
    
    \scnheader{Множество равносильных преобразований логики предикатов}
    \scnrelfrom{включение}{Множество равносильных преобразований логики высказываний}
    \scnhaselement{коммутативность кванторов*}
    \scnhaselement{дистрибутивность кванторов*}
    \scnhaselement{вынос констант*}
    \scnhaselement{двойственность кванторов*}
    
    \scnheader{Множество аксиомных схем логики предикатов}
    \scnrelfrom{включение}{Множество аксиомных схем логики высказываний}
    \scnhaselement{аксиомня схема введения существования}
    \scnhaselement{аксиомня схема удаления всеобщности}
    
    \scnheader{нечеткая логика}
    \scniselement{логическая модель решения задач}
    \scnidtf{раздел многозначной логики, который базируется на обобщении классической логики и теории нечётких множеств для формализации нечётких знаний, характеризуемых лингвистической неопределённостью.}
    % \scnrelfromset{правила вывода}{Modus ponens*;правило удаления существования*;правило введения всеобщности*}
    \scnrelfrom{равносильные преобразования}{Множество равносильных преобразований нечеткой логики}
    \scnrelfrom{логический язык}{Язык нечеткой логики}
    \scnrelfrom{аксиомные схемы логики}{Множество аксиомных схем нечеткой логики}
    
    \scnheader{нечеткое множество}
    \scnexplanation{\textbf{\textit{нечеткое множество}} – это \textit{множество}, которое представляет собой совокупность элементов произвольной природы, относительно которых нельзя точно утверждать – обладают ли эти элементы некоторым характеристическим свойством, которое используется для задания этого нечеткого множества. Принадлежность элементов такому множеству указывается при помощи \textit{нечетких позитивных sc-дуг принадлежности}.}
    
    \scnheader{темпоральная логика}
    \scniselement{логическая модель решения задач}
    \scnidtf{раздел неклассической логики, в рамках которого изучаются свойства высказываний с истинностными значениями, изменяющимися во времени.}
    % \scnrelfromset{правила вывода}{Modus ponens*;правило удаления существования*;правило введения всеобщности*}
    \scnrelfrom{равносильные преобразования}{Множество равносильных преобразований темпоральной логики}
    \scnrelfrom{логический язык}{Язык темпоральной логики}
    \scnrelfrom{аксиомные схемы логики}{Множество аксиомных схем темпоральной логики}
    
    \bigskip
    \scnendstruct
    
    \scnendcurrentsectioncomment
    
    \end{SCn}
\begin{SCn}
\scnsegmentheader{Уточнение понятий план сложного действия, классы действий, класса задач, метода}
\begin{scnsubstruct}
\scniselement{сегмент базы знаний}
\newpage\scnheader{план сложного действия}
\scnidtf{план}
\scnidtf{план выполнения сложного действия}
\scnidtf{план решения \textit{сложной задачи}}
\scnidtf{план выполнения действия}
\scnidtf{спецификация выполнения действия}
\scnidtf{декомпозиция выполняемого действия на систему последовательно/параллельно выполняемых поддействий*}
\scnidtf{описание того, как может быть выполнено соответствующее сложное действие}
\scnidtf{спецификация соответствующего действия, уточняющая то, \uline{как} предполагается выполнять это действие}
\scnidtf{план решения задачи (выполнения сложного действия) путем описания последовательности выполнения поддействий с описанием того, как передается управление от одних поддействий другим, как осуществляется распараллеливание, как организуется выполнение циклов}
\scntext{definition}{вид спецификации \textit{сложного действия}, представляющий собой систему \textit{задач}, \textit{интерпретация} которой (предполагающая решение указанных \textit{задач} в определенной последовательности) обеспечивает выполнение специфицируемого \textit{сложного действия}}\scnsubset{знание}
\scntext{explanation}{Каждый \textit{план} представляет собой \textit{семантическую окрестность, ключевым sc-элементом\scnrolesign} является \textit{действие}, для которого дополнительно детализируется предполагаемый процесс его выполнения. Основная задача такой детализации -- локализация области базы знаний, в которой предполагается работать, а также набора агентов, необходимого для выполнения  описываемого действия. При этом детализация не обязательно должна быть доведена до уровня элементарных действий, цель составления плана -- уточнение подхода к решению той или иной задачи, не всегда предполагающее составления подробного пошагового решения.При описании \textit{плана} может быть использован как процедурный, так и декларативный подход. В случае процедурного подхода для соответствующего \textit{действия} указывает его декомпозиция на более частные поддействия, а также необходимая спецификация этих поддействий. В случае декларативного подхода указывается набор подцелей (например, при помощи логических утверждений), достижение которых необходимо для выполнения рассматриваемого \textit{действия}. На практике оба рассмотренных подхода можно комбинировать.В общем случае \textit{план} может содержать и переменные, например в случае, когда часть плана задается в виде цикла (многократного повторения некоторого набора действий). Также план может содержать константы, значение которых в настоящий момент не установлено и станет известно, например, только после выполнения предшествующих ему \textit{действий}.Каждый \textit{план} может быть задан заранее как часть спецификации \textit{действия}, т.е. \textit{задачи}, а может формироваться \textit{субъектов} уже собственно в процессе выполнения \textit{действия}, например, в случае использования стратегии разбиения задачи над подзадачи. В первом случае \textit{план} \textit{включается*} в \textit{задачу}, соответствующую тому же действию.}\begin{scnsubdividing}
\scnitem{процедурный план сложного действия\\\scnidtf{декомпозиция \textit{сложного действия} на множество последовательно и/или параллельно выполняемых \textit{поддействий}}
}
\scnitem{непроцедурный план сложного действия\\\scnidtf{декомпозиция исходной \textit{задачи}, соответствующей заданному \textit{сложному действию}, на иерархическую систему и/или подзадач}
}
\end{scnsubdividing}
\scnheader{процедурный план сложного действия}
\scntext{note}{В \textit{процедурном плане выполнения сложного действия} соответствующие \textit{поддействия*} декомпозируемого \textit{сложного действия} представляются специфицирующими их \textit{задачами}. Но, кроме такого рода \textit{задач}, в \textit{процедурный план выполнения сложного действия} входят также \textit{задачи}, которые специфицируют \textit{действия}, обеспечивающие:\begin{scnitemize}
\item синхронизацию выполнения \textit{поддействий*} заданного \textit{сложного действия};\item передачу управления указанным \textit{поддействиям*} (а точнее, соответствующим им \textit{задачам}), т.е. инициирование указанных \textit{поддействий*} (и соответствующих им \textit{задач}).\end{scnitemize}
}\scnheader{действие управления интерпретацией процедурного плана сложного действия}
\scnrelboth{семантически близкий знак}{задача управления интерпретацией процедурного плана сложного действия}
\scnsuperset{безусловная передача управления от одного поддействия к другому}
\scnsuperset{инициирование заданного поддействия при возникновении в базе знаний ситуации или события заданного вида}
\scnsuperset{инициирование заданного множества поддействий при успешном завершении выполнения \uline{всех} поддействий другого заданного множества}
\scnsuperset{инициирование заданного множества поддействий при успешном завершении выполнения \uline{по крайней мере одного} поддействия другого заданного множества}
\scntext{note}{Выделенные классы \textit{действий управления интерпретацией процедурного плана сложного действия} дают возможность реализовать различные виды параллелизма, если это позволяет задача.}\scnheader{класс действий}
\scnrelto{семейство подклассов}{действие}
\scnidtftext{explanation}{\uline{максимальное} множество аналогичных (похожих в определенном смысле) действий, для которого существует (но не обязательно известных в текущий момент) по крайней мере один \textit{метод} (или средство), обеспечивающий выполнение \uline{любого} действия из указанного множества действий}
\scnidtf{множество однотипных действий}
\scnsuperset{класс элементарных действий}
\scnsuperset{класс легковыполнимых сложных действий}
\scntext{note}{Тот факт, что каждому выделяемому \textit{классу действий} соответствует по крайней мере один общий для них \textit{метод} выполнения этих \textit{действий}, означает то, что речь идет о \uline{семантической} кластеризации}
\end{scnsubstruct}
\end{SCn}
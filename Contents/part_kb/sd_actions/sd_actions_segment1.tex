\begin{SCn}
    \scnsegmentheader{Уточнение понятия воздействия и понятия действия. Типология воздействий и действий}
    \begin{scnsubstruct}
        \scniselement{сегмент базы знаний}
        
        \scnheader{воздействие}
        \scnidtf{\textit{процесс} воздействия одной сущности (или некоторого множества \textit{сущностей}) на другую \textit{сущность} (или на некоторое множество других \textit{сущностей})}
        \scnidtf{\textit{процесс}, в котором могут быть явно выделены хотя бы одна воздействующая сущность (\textit{субъект воздействия\scnrolesign}) и хотя бы одна \textit{сущность}, на которую осуществляется воздействие (\textit{субъект воздействия\scnrolesign})}
        
        \scnheader{воздействие}
        \scnsubset{процесс}
        \begin{scnindent}
        	\scnidtf{динамическая структура}
        \end{scnindent}
        \scntext{примечание}{Поскольку \textit{воздействия} являются частным видом \textit{процессов}, воздействиями наследуются все свойства \textit{процессов}. Смотрите Раздел \textit{Предметная область и онтология структур}). В частности, используются все \textit{параметры}, заданные на множестве \textit{процессов}, например, \textit{длительность*}, \textit{момент начала процесса*}, \textit{момент завершения процесса\scnsupergroupsign}}
        
        \scnheader{процесс}
        \scnrelfrom{покрытие}{длительность\scnsupergroupsign}
    	\begin{scnindent}
            \begin{scneqtoset}
                \scnitem{краткосрочный процесс}
                \scnitem{среднесрочный процесс}
                \scnitem{долгосрочный процесс}
                \scnitem{перманентный процесс}
            \end{scneqtoset}
            \scntext{примечание}{Длительность различных процессов можно уточнять до любой необходимой точности, используя различные единицы измерения длительности (с точностью до секунд, минут, часов, дней, месяцев, лет, столетий и т.д.). Кроме того, можно ссылаться на процессы, длительность (время жизни) которых соизмерима с рассматриваемым процессом.}
        \end{scnindent}
        
        \scnheader{действие}
        \scnsubset{воздействие}
        \begin{scnindent}
        	\scnsubset{процесс}
        \end{scnindent}
        \scnidtf{\textit{воздействие}, в котором \textit{воздействующая сущность\scnrolesign} осуществляет \textit{воздействие} осознанно, целенаправленно}
        \scnidtf{целенаправленный процесс, выполняемый одним или несколькими субъектами (кибернетическими системами) с возможным применением некоторых инструментов}
        \scnidtf{акция}
        \scnidtf{акт}
        \scnidtf{операция}
        \scnidtf{\uline{процесс} воздействия некоторой (возможно, коллективной) сущности (субъекта воздействия) на некоторую одну или несколько сущностей (объектов воздействия - исходных объектов (аргументов) или целевых (создаваемых или модифицируемых) объектов)}
        \scnidtf{осознанное воздействие}
        \scnidtf{активное воздействие}
        \scnidtf{целеноправленный (осознанный) процесс, выполняемый (управляемый, реализуемый) неким субъектом}
        \scnidtf{акция реализации некоторого замысла}
        \scnidtf{преднамеренная акция}
        \scnidtf{работа}
        \scnidtf{процесс выполнения некоторой задачи}
        \scnidtf{дело}
        \scnidtf{целостный фрагмент некоторой деятельности}
        \scnidtf{целенаправленный процесс, управляемый некоторым субъектом (возможно, коллективным)}
        \scnidtf{целенаправленный процесс, выполняемый некоторым субъектом (исполнителем) над некоторыми объектами}
        \scntext{примечание}{Каждое \textit{действие}, выполняемое тем или иным \textit{субъектом}, трактуется как процесс решения некоторой задачи, т.е. процесс достижения заданной цели в заданных условиях, и, следовательно, выполняется целеноправленно. При \textit{этом} явное указание \textit{действия} и его связи с конкретной задачей может не всегда присутствовать в памяти. Некоторые \textit{задачи} могут решаться определенными агентами перманентно, например, оптимизация \textit{базы знаний}, поиск некорректностей и т.д. и для подобных задач не всегда есть необходимость явно вводить \textit{структуру}, являющуюся формулировкой \textit{задачи}.\\
        	Каждое \textit{действие} может обозначать сколь угодно малое преобразование, осуществляемое во внешней среде либо в памяти некоторой \textit{кибернетической системы}, однако в памяти явно вводятся знаки только тех \textit{действий}, для которых есть небходимость явно хранить в памяти их спецификацию в течение некоторого времени.При выполнении \textit{действия} можно выделить этапы:
            \begin{scnitemize}
                \item построение \textit{плана действия}, декомпозиция (детализация) действия в виде системы его \textit{поддействий};
                \item выполнение построенного плана \textit{действия}
            \end{scnitemize}}
        \scntext{правило идентификации экземпляров}{Экземпляры класса \textit{действий} в рамках \textit{Русского языка} именуются по следующим правилам:
            \begin{scnitemize}
                \item в начале идентификатора пишется слово \scnqqi{\textit{Действие}} и ставится точка;
                \item далее с прописной буквы идет либо содержащее глагол совершенного вида в инфинитиве описание сути того, что требуется получить в результате выполнения действия, либо вопросительное предложение, являющееся спецификацией запрашиваемой (ответной) информации.
            \end{scnitemize}
            Например:\\
            \textit{Действие. Сформировать полную семантическую окрестность понятия треугольник}\\
            \textit{Действие. Верифицировать Раздел. Предметная область sc-элементов}}
            
        \scnheader{следует отличать*}
        \begin{scnhaselementset}
            \scnitem{действие}
        	\begin{scnindent}
                \scntext{примечание}{Каждому \textit{действию} становится в соответствие \textit{кибернетическая система}, являющаяся \textit{субъектом} этого \textit{действия}. Указанный \textit{субъект} может быть либо \textit{индивидуальной}, либо \textit{коллективной кибернетической системой}.}
            \end{scnindent}
            \scnitem{воздействие}
        	\begin{scnindent}
                \scnsuperset{действие}
                \scnsubset{процесс}
                \scntext{примечание}{Сущностью, осущетвляющей воздействие на какой-либо объект, может быть не только \textit{кибернетическая система}, но также, например, и пассивный инструмент, управляемый некоторой \textit{кибернетической системой}.}
            \end{scnindent}
            \scnitem{деятельность}
        \end{scnhaselementset}
        
        \scnheader{параметр, заданный на множестве* (воздействие)}
        \scnidtf{признак классификации воздействий}
        \scnhaselement{осознанность воздействия\scnsupergroupsign}
        \begin{scnindent}
	        \scnidtf{Наличие субъекта (индивида), который запланировал и реализовал воздействие, т.е. обеспечил управление выполнением этого воздействия}
	        \begin{scneqtoset}
	            \scnitem{неосознанное действие}
	            \scnitem{действие}
	            \begin{scnindent}
	            	\scnidtf{осознанное действие}
	            \end{scnindent}
	        \end{scneqtoset}
        \end{scnindent}
        
        \scnheader{параметр, заданный на множестве* (действие)}
        \scnidtf{признак классификации действий}
        \scnhaselement{место выполнения действия\scnsupergroupsign}
        \scnhaselement{функциональная сложность действия\scnsupergroupsign}
        \scnhaselement{многоагентность выполнения действия\scnsupergroupsign}
        \scnhaselement{текущее состояние действия\scnsupergroupsign}
        \scnhaselement{приоритет действия\scnsupergroupsign}
        
        \scnheader{действие}
        \scnrelfrom{разбиение}{место выполнения действия\scnsupergroupsign}
        \begin{scnindent}
	        \begin{scneqtoset}
	            \scnitem{действие, выполняемое в памяти субъекта действия}
	            \scnitem{действие, выполняемое во внешней среде субъекта действия}
	            \scnitem{рецепторное действие}
	            \scnitem{эффекторное действие}
	        \end{scneqtoset}
        \end{scnindent}
        \scnrelfrom{разбиение}{функциональная сложность действия\scnsupergroupsign}
        \begin{scnindent}
	        \begin{scneqtoset}
	            \scnitem{элементарное действие}
	            \begin{scnindent}
	            	\scnidtf{действие, выполняемое индивидуальной кибернетической системой}
	                \scntext{пояснение}{Элементарное действие выполняется одним индивидуальным субъектом и является либо элементарным действием, выполняемым в памяти этого субъекта (элементарным действием его процессора), либо элементарным действием одного из его эффекторов.}
	            \end{scnindent}
	            \scnitem{сложное действие}
	        	\begin{scnindent}
	                \scnidtf{неэлементарное действие}
	                \scnidtf{действие выполнение которого требует декомпозиции этого действия на множество его \uline{поддействий}, т.е. частных действий более низкого уровня}
	                \scntext{примечание}{Декомпозиция сложного действия на поддействия может иметь весьма сложный иерархический вид с большим числом уровней иерархии, т.е. поддействиями \textit{сложного действия} могут также \textit{сложные действия}. Уровень сложности действия можно определять (1) общим числом его поддействий и (2) числом уровней иерархии этих поддействий.}
	                \scntext{примечание}{Другим примером может служить запись одной и той же процедурной программы на языке программирования более высокого уровня и на языке программирования более низкого уровня. В данном случае элементарность действий строго определяется на уровне языка.}
	                \scntext{примечание}{Темпоральные соотношения между \textit{поддействиями} сложного \textit{действия} могут быть самые различные, но в пройстейшем случае \textit{сложное действие} представляет собой строгую последовательность \textit{действий} более низкого уровня иерархии.}
	                \scnidtf{система более простых действий (\textit{поддействий}), которые могут выполняться как последовательно, так и параллельно}
	                \scntext{примечание}{В состав \textit{сложного действия} могут входить не только \textit{собственно поддействия} этого \textit{сложного действия}, но также и специальные \textit{поддействия}, осуществляющие \uline{управление} процессом выполнения \textit{сложного действия}, и, в частности, \textit{поддействия}, осуществляющие инициирование поддействий, передачу управления \textit{поддействиям}.}
	                \scnidtf{неатомарное действие}
	                \scnidtf{неэлементарное \textit{действие}}
	                \scnidtf{\textit{действие}, выполнение которого сводится в общем случае к последовательно-параллельному выполнению некоторого множества \textit{поддействий}}
	                \scnidtf{\textit{действие}, декомпозируемое на множество более простых \textit{действий} (\textit{поддействий}), обеспечивающих выполнение исходного (заданного) \textit{действия}}
	                \begin{scnrelfromset}{покрытие}
	                    \scnitem{легко выполнимое сложное действие}
	                	\begin{scnindent}
	                        \scnidtf{сложное действие, для выполнения которого известен соответствующий \textit{метод} и соответствующие этому методу исходные данные, а также (для действий, выполняемых во внешней среде) имеются в наличии все необходимые исходные объекты (расходные материалы и комплектация), а также средства (инструменты)}
	                    \end{scnindent}
	                    \scnitem{интеллектуальное действие}
	                	\begin{scnindent}
	                        \scnidtf{трудно выполнимое сложное действие}
	                        \scnidtf{сложное действие, для выполнения которого в текущий момент либо неизвестен соответствующий \textit{метод}, либо возможные \textit{методы} известны, но отсутствуют условия их применения.}
	                        \scnsuperset{действие, у которого цель известна, на зачада не совсем точна}
	                        \begin{scnindent}
		                        \scnsuperset{действие, направленное на выявление противоречий в базе знаний}
		                        \begin{scnindent}
		                        	\scntext{примечание}{Это действие декомпозируется на несколько самостоятельных поддействий, каждое из которых выявляет (локализует) противоречия (ошибки) конкретного формализуемого вида, для которого в базе знаний существует точное определение}
		                        \end{scnindent}
		                    \end{scnindent}
	                        \scnsuperset{действие, для которого априори не известен метод, обеспечивающий его выполнение}
	                        \begin{scnindent}
	                        	\scntext{примечание}{Соответствующий метод либо не найден, либо его вообще нет в памяти.}
	                        \end{scnindent}
	                    \end{scnindent}
	                \end{scnrelfromset}
	            \end{scnindent}
	        \end{scneqtoset}
	    \end{scnindent}
        \scnrelfrom{разбиение}{многоагентность выполнения действия\scnsupergroupsign}
        \begin{scnindent}
	        \begin{scneqtoset}
	            \scnitem{индивидуальное действие}
	        	\begin{scnindent}
	            	\scnidtf{действие, выполняемое одним субъектом (агентом)}
	                \scnidtf{действие, выполняемое индивидуальной кибернетической системой}
	                \scnsuperset{индивидуальное действие, выполняемое человеком}
	                \scnsuperset{индивидуальное действие, выполняемое компьютерной системой}
	            \end{scnindent}
	            \scnitem{\textit{коллективное} действие}
	        	\begin{scnindent}
	            	\scnidtf{действие, выполняемое коллективом субъектов (многоагентной системой)}
	                \scnidtf{действие, выполняемое коллективом кибернетических систем (коллективом субъектов)}
	                \scnsuperset{действие, выполняемое коллективом людей}
	                \scnsuperset{действие, выполняемое коллективом индивидуальных компьютерных систем}
	                \scnsuperset{действие, выполняемое коллективом людей и индивидуальных компьютерных систем}
	                \begin{scnindent}
		                \scnsuperset{действие, выполняемое Экосистемой OSTIS}
		                \scnsuperset{действие, выполняемое одним человеком во взаимодействии с одной индивидуальной компьютерной системой}
		            \end{scnindent}
	            \end{scnindent}
        	\end{scneqtoset}
        \end{scnindent}
        \scnrelfrom{разбиение}{текущее состояние действия\scnsupergroupsign}
        \begin{scnindent}
	        \begin{scneqtoset}
		            %TODO: check by human--->
		            \scnitem{планируемое действие}
		            \begin{scnindent}
		                \scnidtf{запланированное, но не инициированное действие}
		                \scnidtf{будущее действие}
		                \scnidtf{действие, которое планируется выполнить в будущем}
		                \scntext{пояснение}{Во множество \textit{запланированных, но не инициированных действий} входят \textit{действия}, начать выполнение которых запланировано на какой-либо момент в будущем.}
		            \end{scnindent}
		            \scnitem{инициированное действие}
		            \begin{scnindent}
		                \scnidtf{инициированное, но не выполняемое действие}
		                \scnidtf{действие, подлежащее выполнению}
		                \scnidtf{действие, включенное в план}
		                \scnidtf{действие, ожидающее начала своего выполнения}
		                \scntext{пояснение}{Во множество \textit{инициированных действий} входят \textit{действия}, выполнение которых инициировано в результате какого-либо события.В общем случае, \textit{действия} могут быть инициированы по следующим причинам:
		                    \begin{scnitemize}
		                        \item явно путем проведения соответствующей \textit{sc-дуги принадлежности} каким-либо \textit{субъектом (заказчиком*)}. В случае действия в \textit{sc-памяти}, оно может быть инициировано как внутренним \textit{sc-агентом} системы, так и пользователем при помощи соответствующего пользовательского интерфейса. При этом, спецификация действия может быть сформирована одним \textit{sc-агентом}, а собственно добавление во множество \textit{инициированных действий} может быть осуществлено позже другим \textit{sc-агентом}
		                        \item в результате того, что одно или несколько \textit{действий}, предшествовавших данному в рамках некоторой декомпозиции, стали \textit{прошлыми сущностями} (процедурный подход)
		                        \item в результате того, что в \textit{памяти} системы появилась конструкция, соответствующая некоторому условию инициирования \textit{sc-агента}, который должен выполнить данное \textit{действие} (декларативный подход).
		                    \end{scnitemize}
		                    Следует отметить, что декларативный и процедурный подходы можно рассматривать как две крайности, использование только одной из которых не является удобным и целесообразным. При этом, например, принципы инициирования по процедурному подходу могут быть полностью сведены к набору декларативных условий инициирования, но как было сказано, это не всегда удобно и наиболее рациональным будет комбинировать оба подхода в зависимости от ситуации.По сути, попадание некоторого \textit{действия} во множество \textit{инициированных действий} говорит о том, что, по мнению некоторого \textit{субъекта} (заказчика, инициатора), оно готово к выполнению и должно быть выполнено, то есть спецификация данного \textit{действия} по мнению данного \textit{субъекта} сформирована в степени, достаточной для решения поставленной \textit{задачи} и существует некоторый другой \textit{субъект} (исполнитель), который может приступать к выполнению \textit{действия}. Однако стоит отметить, что с точки зрения \textit{исполнителя} такая спецификация \textit{действия} в общем случае может оказаться недостаточной или некорректной.}
		            \end{scnindent}
		            \scnitem{выполняемое действие}
		            \begin{scnindent}
		                \scnidtf{активное действие}
		                \scnidtf{непосредственно выполняемое действие}
		                \scnidtf{действие, выполняемое в текущий момент}
		                \scnidtf{настоящее действие}
		                \scniselement{неосновное понятие}
		                \scnsubset{настоящая сущность}
		                \scntext{пояснение}{Во множество \textit{выполняемых действий} входят \textit{действия}, к выполнению которых приступил какой-либо из соответствующих \textit{субъектов}.Попадание \textit{действия} в данное множество говорит о следующем:
		                    \begin{scnitemize}
		                        \item рассматриваемое \textit{действие} уже попало во множество \textit{инициированных действий}.
		                        \item существует как минимум один \textit{субъект}, условие инициирования которого соответствует спецификации данного \textit{действия}.
		                    \end{scnitemize}
		                    После того, как собственно процесс выполнения завершился, \textit{действие} должно быть удалено из множества \textit{выполняемых действий} и добавлено во множество \textit{выполненных действий} или какое-либо из его подмножеств.Понятие \textit{выполняемое действие} является неосновным, и вместо того, чтобы относить конкретные действия к данному классу, их относят к классу \textit{настоящих сущностей}.}
		            \end{scnindent}
		            \scnitem{прерванное действие}
		            \begin{scnindent}
		                \scnidtf{действие, ожидающее продолжения своего выполнения}
		                \scnidtf{отложенное действие}
		                \scnidtf{приостановленное действие}
		                \scntext{пояснение}{Во множество \textit{прерванных действий} входят \textit{действия}, которые уже были инициированы, однако их выполнение невозможно по каким-либо причинам, например в случае, когда у исполнителя в данный момент есть более приоритетные задачи.}
		            \end{scnindent}
		            \scnitem{выполненное действие}
		            \begin{scnindent}
		                \scnidtf{завершенное действие}
		                \scnidtf{прошлое действие}
		                \scniselement{неосновное понятие}
		                \scnsubset{прошлая сущность}
		                \scntext{пояснение}{Во множество \textit{выполненных действий} попадают \textit{действия}, выполнение которых с \uline{точки зрения \textit{субъекта}}, осуществлявшего их выполнение. Таким образом, понятие \textit{выполненного действия} является относительным, поскольку с точки зрения разных субъектов одно и то же действие может считаться выполненным или все еще выполняющимся.\\
		                	В зависимости от результатов конкретного процесса выполнения, рассматриваемое \textit{действие} может стать элементом одного из подмножеств множества \textit{выполненных действий}.Понятие \textit{выполненное действие} является неосновным, и вместо того, чтобы относить конкретные \textit{действия} к данному классу, их относят к классу \textit{прошлых сущностей}.}
			                \begin{scnsubdividing}
			                    \scnitem{успешно выполненное действие}
			                    \begin{scnindent}
			                        \scntext{пояснение}{Во множество \textit{успешно выполненных действий} попадают \textit{действия}, выполнение которых успешно завершено с точки зрения \textit{субъекта}, осуществлявшего их выполнение, т.е. достигнута поставленная цель, например, получены решение и ответ какой-либо задачи, успешно преобразована какая-либо конструкция и т.д. Очень важно отметить, что в общем случае выделить критерии успешности или безуспешности выполнения действий того или иного класса \uline{невозможно}, поскольку эти критерии, во-первых, зависят от контекста, во-вторых, могут быть разными с точки зрения разных субъектов. Однозначно критерии успешности выполнения действий могут быть сформулированы для некоторых частных классов действий, например, классов операторов некоторого процедурного языка программирования (например, \textit{scp-операторов}).\\
			                         Таким образом, понятие \textit{успешно выполненное действие} является относительным.\\
			                         Если действие было выполнено успешно, то, в случае действия по генерации каких-либо знаний, к \textit{действию} при помощи связки отношения \textit{результат*} приписывается \textit{sc-конструкция}, описывающая результат выполнения указанного действия. В случае, когда действие направлено на какие-либо изменения базы знаний, \textit{sc-конструкция}, описывающая результат действия, формируется в соответствии с правилами описания истории изменений базы знаний.\\
			                         В случае, когда успешное выполнение \textit{действия} приводит к изменению какой-либо конструкции в \textit{sc-памяти}, которое необходимо занести в историю изменений базы знаний или использовать для демонстрации протокола решения задачи, то генерируется соответствующая связка отношения \textit{результат*}, связывающая задачу и \textit{sc-конструкцию}, описывающую данное изменение.}
			              		\end{scnindent}
			              		%слишком большой отступ
			                    \scnitem{безуспешно выполненное действие}
			                    \begin{scnindent}
			                        \scntext{пояснение}{Во множество \textit{безуспешно выполненных действий} попадают \textit{действия}, выполнение которых не было успешно завершено с точки зрения \textit{субъекта}, осуществлявшего их выполнение, по каким-либо причинам.Можно выделить две основные причины, по которым может сложиться указанная ситуация:
			                            \begin{scnitemize}
			                                \item соответствующая \textit{задача} сформулирована некорректно
			                                \item формулировка соответствующей \textit{задачи} корректна и понятна системе, однако решение данной задачи в текущий момент не может быть получено за удовлетворительное с точки зрения заказчика или исполнителя сроки.
			                            \end{scnitemize}
			                            Для конкретизации факта некорректности формулировки задачи можно выделить ряд более частных классов \textit{безуспешно выполненных действий}, например:
			                            \begin{scnitemize}
			                                \item действие, спецификация которого противоречит другим знаниям системы (например, не выполняется неравенство треугольника)
			                                \item действие, при спецификации которого использованы понятия, неизвестные системе
			                                \item действие, выполнение которого невозможно из-за недостаточности данных (например, найти площадь треугольника по двум сторонам)
			                                \item и другие.
			                            \end{scnitemize}
			                            Для конкретизации факта безуспешности выполнения некоторого действия в системе могут также использоваться дополнительные подмножества данного множества, при необходимости снабженные естественно-языковыми или формальными комментариями.}
			                        \scnsuperset{действие, выполненное с ошибкой}
			                        \begin{scnindent}
			                        	\scntext{пояснение}{Во множество \textit{действий, выполненных с ошибкой}, попадают \textit{действия}, выполнение которых не было успешно завершено с точки зрения \textit{субъекта}, осуществлявшего их выполнение, по причине возникновения какой-либо ошибки, например, некорректности спецификации данного \textit{действия} или нарушения её целостности каким-либо \textit{субъектом} (в случае \textit{действия в sc-памяти}).}
			                        \end{scnindent}
			                  \end{scnindent}
					\end{scnsubdividing}
					\end{scnindent}
		            \scnitem{отмененное действие}
	        \end{scneqtoset}
	    \end{scnindent}
        \scnrelfrom{покрытие}{приоритет действия\scnsupergroupsign}
        \begin{scnindent}
	        \begin{scneqtoset}
	            \scnitem{действие с очень высоким приоритетом}
	            \scnitem{действие с высоким приоритетом}
	            \scnitem{действие со средним приоритетом}
	            \scnitem{действие с низким приоритетом}
	            \scnitem{действие с очень низким приоритетом}
	        \end{scneqtoset}
        \end{scnindent}
        	
        \scnheader{действие, выполняемое в памяти субъекта действия}
        \scnidtf{информационное действие}
        \scnidtf{действие, выполняемое в памяти}
        \scnidtf{действие кибернетической системы, направленное на обработку информации, хранимой в её памяти}
        \scnsuperset{действие, выполняемое кибернетической системой в собственной памяти и направленное на организацию её деятельности во внешней среде}
        \begin{scnindent}
        	\scnidtf{действие, выполняемое кибернетической системой в её памяти и направленное на организацию её деятельности во внешней среде и в конечном счете --- на сенсо-моторную координацию деятельности её эффекторов}
        \end{scnindent}
        \scntext{пояснение}{Результатом выполнения \textit{действия, выполняемого в памяти субъекта действия} является в общем случае некоторое новое состояние памяти информационной системы (не обязательно \textit{sc-памяти}), достигнутое исключительно путем преобразования информации, хранящееся в памяти системы, то есть либо посредством генерации новых знаний на основе уже имеющихся, либо посредством удаления знаний, по каким-либо причинам ставших ненужными. Следует отметить, что если речь идет об изменении состояния \textit{sc-памяти}, то любое преобразование информации можно свести к ряду элементарных действий генерации, удаления или изменения инцидентности \textit{sc-элементов} друг относительно друга.}
        
        \scnheader{действие, выполняемое во внешней среде субъекта действия}
        \scnidtf{действие, выполняемое кибернетической системой в её внешней среде и осуществляемое (на самом низком уровне) эффекторами этой кибернетической системы}
        \scnidtf{поведенческое действие}
        \scntext{пояснение}{В случае \textit{действия, выполняемого во внешней среде субъекта действия} результатом его выполнения будет новое состояние внешней среды. Очень важно отметить, что под внешней средой в данном случае понимаются также и компоненты системы, внешние с точки зрения памяти, то есть не являющиеся хранимыми в ней информационными конструкциями. К таким компонентам можно отнести, например, различные манипуляторы и прочие средства воздействия системы на внешний мир, то есть к поведенческим задачам можно отнести изменение состояния механической конечности робота или непосредственно вывод некоторой информации на экран для восприятия пользователем.}
        
        \scnheader{поддействие*}
        \scnidtf{быть поддействием заданного \textit{сложного действия*}}
        \begin{scnsubdividing}
            \scnitem{собственное поддействие*}
            \scnitem{управляющее поддействие*}
            \begin{scnindent}
                \scntext{пояснение}{\textit{поддействие*}, которое осуществляет инициирование выполнения очередного \textit{поддействия*} (или сразу некоторого множества \textit{поддействий*}) при соблюдении некоторого априори известного условия.В общем случае таким условием является возникновение некоторой ситуации или события. В частности, это может быть:
                    \begin{scnitemize}
                        \item ситуация или событие в обрабатываемой \textit{базе знаний} (это называют условной передачей управления)
                        \item факт успешного завершения некоторого \textit{поддействия} (это безусловная передача управления)
                        \item факт успешного завершения \uline{по крайней мере одного} из заданного множества \textit{поддействий}
                        \item факт успешного завершения \uline{всех} \textit{поддействий} из заданного множества \textit{поддействий}.
                    \end{scnitemize}}
                \scntext{примечание}{На самом деле при выполнении \textit{сложных действий} многообразие условий инициирования (передачи управления) поддействиям может быть значительно шире}
           \end{scnindent}
        \end{scnsubdividing}
        \bigskip
    \end{scnsubstruct}
    \scnendsegmentcomment{Уточнение понятия воздействия и понятия действия. Типология воздействий и действий}
\end{SCn}

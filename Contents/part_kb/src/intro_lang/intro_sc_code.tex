\begin{SCn}
    \scnsectionheader{Предметная область и онтология внутреннего языка ostis-систем}
    \begin{scnsubstruct}
       	\begin{scnrelfromset}{автор}
       		\scnitem{Голенков В.В.}
       		\scnitem{Ивашенко В.П.}
       	\end{scnrelfromset}
        \begin{scnreltovector}{конкатенация сегментов}
            \scnitem{Основные положения внутреннего языка ostis-систем}
            \scnitem{Описание Ядра SC-кода}
            \scnitem{SC-код как синтаксическое расширение Ядра SC-кода}
            \scnitem{Использование SC-кода для формального описания собственного синтаксиса}
            \scnitem{Уточнение смысла выделенных классов sc-элементов и связей между этими классами}
            \scnitem{Структура базовой семантической спецификации sc-элемента}
            \scnitem{Онтологическая формализация Базовой денотационной семантики SC-кода}
			\scnitem{Смысловое пространство ostis-систем}
        \end{scnreltovector}
        \begin{scnrelfromlist}{ключевое понятие}
           	\scnitem{sc-элемент}
           	\scnitem{обозначение sc-множества}
           	\scnitem{обозначение sc-связки}
           	\scnitem{обозначение sc-класса}		
           	\scnitem{обозначение sc-структуры}		
           	\scnitem{обозначение внешней сущности}			
           	\scnitem{sc-константа}
           	\scnitem{sc-переменная}				
           	\scnitem{sc-множество}
           	\scnitem{sc-связка}
           	\scnitem{sc-класс}						
           	\scnitem{sc-структура}						
           	\scnitem{внешняя сущность}			
        \end{scnrelfromlist}
        \begin{scnrelfromlist}{ключевое знание}
           	\scnitem{Базовая денотационная семантика SC-кода}
        \end{scnrelfromlist}
        \begin{scnrelfromlist}{библиографическая ссылка}
           	\scnitem{\scncite{Narinjani2000}}
           	\scnitem{\scncite{Ivashenko2022}}
           	\scnitem{\scncite{Ivashenko2020String}}
           	\scnitem{\scncite{Collatz1966}}
           	\scnitem{\scncite{Ivashenko2017}}
           	\scnitem{\scncite{Ivashenko2014}}
           	\scnitem{\scncite{Bohm1993}}
           	\scnitem{\scncite{Bohm2002}}
           	\scnitem{\scncite{Nalimov1995}}
           	\scnitem{\scncite{Nalimov1989}}
           	\scnitem{\scncite{Nalimov1979}}
           	\scnitem{\scncite{Martynov2004}}
           	\scnitem{\scncite{Manin2016}}
           	\scnitem{\scncite{Melchuk2016}}
           	\scnitem{\scncite{Harris1992}}
           	\scnitem{\scncite{Alt1995}}
           	\scnitem{\scncite{Study1905}}
           	\scnitem{\scncite{Kostrikin1997}}
           	\scnitem{\scncite{Lowe2001}}
           	\scnitem{\scncite{Manin2014}}
           	\scnitem{\scncite{Martynov2009}}
           	\scnitem{\scncite{Gordey2014}}
        \end{scnrelfromlist}
        \begin{scnrelfromvector}{введение}
           	\scnfileitem{Поскольку все элементы \textit{информационных конструкций} являются обозначениями описываемых сущностей и, в том числе, обозначениями различных выделяемых классов \textit{sc-элементов}, можно явно ввести различные семантически значимые и синтаксически выделяемые классы \textit{sc-элементов} и на основе этого явно описать средствами \textit{SC-кода} \textit{Базовую денотационную семантику} и \textit{Синтаксис SC-кода}.}
           	\begin{scnindent}
           		\begin{scnrelfromset}{смотрите}
           			\scnitem{Денотационная семантика Ядра SC-кода}
           			\scnitem{Синтаксис Ядра SC-кода}
           		\end{scnrelfromset}
           	\end{scnindent}
           	\scnfileitem{\textit{Синтаксис SC-кода} задается семейством классов синтаксический выделяемых \textit{sc-элементов}. Элементы, принадлежащие каждому синтаксически выделяемому классу \textit{sc-элементов} должны иметь одинаковые синтаксические признаки (синтаксические метки). При этом очевидно, что \textit{Синтаксис SC-кода} существенно упростится, если синтаксически выделяемые классы sс-элементов будут одновременно иметь и четкую семантическую интерпретацию.}
           	\scnfileitem{Таким образом, формализацию \textit{Синтаксиса SC-кода} целесообразно осуществлять после формализации \textit{Базовой денотационной семантики SC-кода}. Путем синтаксического выделения тех семантически выделенных классов \textit{sc-элементов}, которые:
           		\begin{scnitemize} 
           			\item Во-первых, необходимы для кодирования sc-конструкций в памяти ostis-систем (в sc-памяти).
           			\item Во-вторых, обеспечивают максимально возможное упрощение обработки sc-конструкций (например, упрощение анализа часто проверяемых семантических характеристик обрабатываемых \textit{sc-элементов}).
           		\end{scnitemize}}
        \end{scnrelfromvector}
        
        \scnsegmentheader{Основные положения внутреннего языка ostis-систем}
\begin{scnsubstruct}
    \scnheader{SC-код}
    \scnidtf{Язык унифицированного смыслового представления знаний в памяти \textit{интеллектуальных компьютерных систем}}
    \scnidtf{Внутренний язык \textit{ostis-систем}}
    \scnrelto{внутренний язык}{ostis-система}
    \scntext{эпиграф}{Информация содержится не в самих знаках, а в конфигурации связей между ними.}
    \scntext{эпиграф}{Он вскочил на коня и поскакал во все стороны.}
    \scntext{основной внешний идентификатор sc-элемента}{\textbf{SC-код}}
    \scniselement{имя собственное}
    \scntext{часто используемый неосновной внешний идентификатор sc-элемента}{sc-текст}
    \scniselement{имя нарицательное}
    \scniselement{абстрактный язык}
    \scniselement{графовый язык}
    \scnidtf{Универсальный язык, обеспечивающий внутреннее представление и структуризацию \uline{всех}(!), используемых ostis-системой в процессе своего функционирования.}
    \scnidtf{Универсальный язык, являющийся результатом унификации (уточнения) синтаксиса и денотационной семантики семантических сетей.}
    \scntext{пояснение}{Универсальность SC-кода обеспечивается и тем, что элементами текстов SC-кода могут быть знаки описываемых сущностей \uline{любого} вида, в том числе, и  знаки связей между описываемыми сущностями и/или их знаками.}
    \scntext{следствие}{Тексты SC-кода являются графовыми структурами расширенного вида, в которых знаки описываемых связей могут соединять не только вершины (узлы) графовой структуры, но и знаки других связей.}
    \scnidtf{Базовый универсальный язык внутреннего представления знаний в памяти ostis-систем.}
    \scnidtf{Базовый внутренний язык ostis-систем.}
    \scnidtf{Максимальный внутренний язык ostis-систем, по отношению к которому все остальные (специализированные) внутренние языки являются его подъязыками (подмножествами)}
    \scnidtf{Множество всевозможных текстов SC-кода}
    \scniselement{имя собственное}
    \scnidtf{текст SC-кода}
    \scniselement{имя нарицательное}
    \begin{scnrelfromvector}{принципы, лежащие в основе}
        \scnfileitem{\textit{Знаки} (обозначения) всех \textit{сущностей}, описываемых в \textit{sc-текстах} (текстах \textit{\textbf{SC-кода}}) представляются в виде синтаксически элементарных (атомарных) фрагментов \textit{sc-текстов} и, следовательно, не имеющих внутренней структуры, не состоящих из более простых фрагментов \textit{текста}, как, например, имена (термины), которые представляют \textit{знаки} описываемых \textit{сущностей} в привычных \textit{языках} и состоят из \textit{букв}.}
        \scnfileitem{\textit{Имена} (термины), \textit{естественно-языковые тексты} и другие информационные конструкции, не являющиеся \textit{sc-текстами}, могут входить в состав \textit{sc-текста}, но только как \textit{файлы}, описываемые (специфицируемые) \textit{sc-текстами}. Таким образом, в состав базы знаний \textit{интеллектуальной компьютерной системы}, построенной на основе \textit{\textbf{SC-кода}}, могут входить \textit{имена} (термины), обозначающие некоторые описываемые \textit{сущности} и представленные соответствующими \textit{файлами}. Каждый \mbox{\textit{sc-элемент}} будем называть внутренним обозначением некоторой \textit{сущности}, а \textit{имя} этой \textit{сущности}, представленное соответствующим файлом, будем называть \textit{внешним идентификатором} (внешним обозначением) этой \textit{сущности}. При этом каждый именуемый (идентифицируемый) \textbf{\textit{sc-элемент}} связывается дугой, принадлежащей отношению \scnqqi{быть \textit{\textbf{внешним идентификатором*}}}, с \textit{узлом}, содержимым которого является \textit{файл} идентификатора (в частности, \textit{имени}), обозначающего ту же \textit{сущность}, что и указанный выше \textit{sc-элемент}. \textit{Внешним идентификатором} может быть не только \textit{имя} (термин), но и иероглиф, пиктограмма, озвученное имя, жест.
        	\\Особо отметим, что \textit{внешние идентификаторы} описываемых \textit{сущностей} в \textit{интеллектуальной компьютерной системе}, построенной на основе \textit{\textbf{SC-кода}}, используются только (1) для анализа информации, поступающей в эту систему из вне из различных источников, и ввода (понимания и погружения) этой информации в \textit{базу знаний}, а также (2) для синтеза различных \textit{сообщений}, адресуемых различным субъектам (в т.ч. пользователям).}
        \scnfileitem{Тексты \textit{\textbf{SC-кода}} (\textit{sc-тексты}) имеют в общем случае нелинейную (графовую) структуру, поскольку (1) \textit{знак} каждой описываемой сущности входит в состав \textit{sc-текста} однократно и (2) каждый такой \textit{знак} может быть инцидентен неограниченному числу других \textit{знаков}, поскольку каждая описываемая \textit{сущность} может быть связана неограниченным числом связей с другими описываемыми \textit{сущностями}.}
        \scnfileitem{\textit{База знаний}, представленная текстом \textit{\textbf{SC-кода}}, является \textit{графовой структурой} специального вида, алфавит элементов которой включает в себя множество \textit{узлов}, множество \textit{ребер}, множество \textit{дуг}, множество \textit{базовых дуг} --- дуг специально выделенного типа, обеспечивающих структуризацию \textit{баз знаний}, а также множество специальных \textit{узлов}, каждый из которых имеет содержимое, являющееся \textit{файлом}, хранящимся в памяти \textit{интеллектуальной компьютерной системы}. Структурная особенность данной \textit{графовой структуры} заключается в том, что ее \textit{дуги} и \textit{ребра} могут связывать не только \textit{узел} с \textit{узлом}, но и \textit{узел} с \textit{ребром} или \textit{дугой}, \textit{ребро} или \textit{дугу} с другим \textit{ребром} или \textit{дугой}.}
        \scnfileitem{\uline{Все элементы} (\textit{sc-элементы}) указанной выше \textit{графовой структуры} (текста \textit{\textbf{SC-кода}}), т.е. все ее узлы (\textit{sc-узлы}), ребра (\textit{sc-ребра}) и дуги (\textit{sc-дуги}) являются обозначениями различных сущностей. При этом ребро является обозначением бинарной неориентированной связки между двумя сущностями, каждая из которых либо представлена в рассматриваемой графовой структуре соответствующим знаком, либо является самим этим знаком. Дуга является обозначением бинарной ориентированной связки между двумя сущностями. Дуга специального вида (\textit{\textbf{базовая дуга}}) является знаком связи между узлом, обозначающим некоторое множество элементов рассматриваемой графовой структуры, и одним из элементов этой графовой структуры, который принадлежит указанному множеству. Узел, имеющий содержимое (узел, для которого содержимое существует, но может в текущий момент быть неизвестным) является знаком файла, который является содержимым этого узла. Узел, не являющийся знаком файла, может обозначать какой-либо материальный объект, первичный абстрактный объект(например, число, точку в некотором абстрактном пространстве), какую-либо бинарную связь, какое-либо множество (в частности, понятие, структуру, ситуацию, событие, процесс). При этом сущности, обозначаемые элементами рассматриваемой графовой структуры, могут быть постоянными (существующими всегда) и временными (сущностями, которым соответствует отрезок времени их существования).
        	\\Кроме того, сущности, обозначаемые элементами рассматриваемой графовой структуры, могут быть константными (конкретными) сущностями и переменными (произвольными) сущностями. Каждому элементу рассматриваемой графовой структуры, являющемуся обозначением переменной сущности, ставится в соответствие область возможных значений этого обозначения. Область возможных значений каждого переменного ребра является подмножеством множества всевозможных константных ребер, область возможных значений каждой переменной дуги является подмножеством множества всевозможных константных дуг, область возможных значений каждого переменного узла является подмножеством множества всевозможных константных узлов.}
        \scnfileitem{В рассматриваемой графовой структуре, являющейся представлением базы знаний в \textit{\textbf{SC-коде}}, могут, но не должны существовать разные элементы графовой структуры, обозначающие одну и ту же сущность. Если пара таких элементов обнаруживается, то эти элементы склеиваются (отождествляются). Таким образом, синонимия внутренних обозначений в базе знаний интеллектуальной компьютерной системы, построенной на основе \textit{\textbf{SC-кода}}, запрещена. При этом синонимия внешних обозначений считается нормальным явлением. Формально это означает, что из некоторых элементов рассматриваемой графовой структуры выходит несколько дуг, принадлежащих отношению \scnqqi{быть \textit{\textbf{внешним идентификатором*}}}.
        	\\Из всех указанных дуг, принадлежащих отношению \scnqqi{быть \textit{\textbf{внешним идентификатором*}}} и выходящих из одного элемента рассматриваемой графовой структуры, обязательно выделяется одна (очень редко две) путем включения их в число дуг, принадлежащих отношению \scnqqi{быть \textit{\textbf{основным внешним идентификатором*}}}. Это означает, что указываемый таким образом внешний идентификатор не является омонимичным, т.е. не может быть использован как внешний идентификатор, соответствующий другому элементу рассматриваемой графовой структуры.}
        \scnfileitem{Кроме файлов, представляющих различные внешние обозначения (имена, иероглифы, пиктограммы), в памяти интеллектуальной компьютерной системе, построенной на основе \textit{\textbf{SC-кода}}, могут хранится файлы различных текстов (книг, статей, документов, примечаний, комментариев, пояснений, чертежей, рисунков, схем, фотографий, видео-материалов, аудио-материалов).}
        \scnfileitem{\uline{Любую сущность}, требующую описания, в тексте \textit{\textbf{SC-кода}} можно обозначить в виде \textit{sc-элемента}. Это являетс яодним из факторов, обеспечивающих универсальность \textit{\textbf{SC-кода}}. Особо подчеркнем, что sc-элементы являются не просто обозначениями различных описываемых сущностей, а обозначениями, которые являются элементарными (атомарными) фрагментами знаковой конструкции, т.е. фрагментами, детализация структуры которых не требуется для \scnqq{прочтения} и понимания этой знаковой конструкции.}
        \scnfileitem{Текст \textit{\textbf{SC-кода}}, как и любая другая графовой структура, является абстрактным математическим объектом, не требующим детализации (уточнения) его кодирования в памяти компьютерной системы (например, в виде матрицы смежности, матрицы инцидентности, списковой структуры). Но такая детализация потребуется для технической реализации памяти, в которой хранятся и обрабатываются \textit{sc-тексты}.}
        \scnfileitem{Важнейшим дополнительным свойством \textit{\textbf{SC-кода}} является то,что он удобен не просто для внутреннего представления знаний в памяти интеллектуальной компьютерной системы, но также и для внутреннего представления информации в памяти компьютеров, специально предназначенных для интерпретации семантических моделей интеллектуальных компьютерных систем. Т.е., \textit{\textbf{SC-код}} определяет синтаксические, семантические и функциональные принципы организации памяти компьютеров нового поколения, ориентированных на реализацию интеллектуальных компьютерных систем, --- принципы организации графодинамической ассоциативной семантической памяти.}
        \scnfileitem{\textit{\textbf{SC-код}} рассматривается нами как объединение трех его подъязыков, в число которых входит \textit{\textbf{Ядро SC-кода}}, подъязык \textit{\textbf{SC-кода}}, обеспечивающий представление текстов \textit{\textbf{SC-кода}} (\textit{sc-текстов}) в форме орграфов классического вида, являющихся подразбиениями текстов \textit{\textbf{Ядра SC-кода}} и, соответственно, использующих \uline{явное} представление пар инцидентности элементов sc-текстов (sc-элементов), синтаксическое \textit{\textbf{Расширение Ядра SC-кода}}, обеспечивающее представление в памяти ostis-системы информационных конструкций инородного для \textit{\textbf{SC-кода}} вида.}
    \end{scnrelfromvector}

    \scnheader{абстрактный язык}
    \scnidtf{Язык, для которого не уточняется способ представления символов (синтаксически элементарных фрагментов), входящих в состав текстов этого языка, а задается только \uline{алфавит*} этих символов, то есть семейство классов символов, считающихся синтаксически эквивалентными друг другу.}
    \begin{scnindent}
        \scntext{примечание}{Каждому абстрактному языку можно поставить в соответствие целое семейство \textit{реальных языков}, обеспечивающих \uline{изоморфное} реальное представление текстов указанного абстрактного языка путем уточнения способов представления (изображения, кодирования) символов, входящих в состав этих текстов, а также путем уточнения правил установления синтаксической эквивалентности этих символов. Очевидно, что во всём остальном синтаксис и денотационная семантика указанных реальных языков полностью совпадает с синтаксисом и денотационной семантикой соответствующего абстрактного языка.}
        \begin{scnindent}
            \scntext{примечание}{Для \textit{SC-кода} как абстрактного языка необходима разработка как минимум трех синтаксически и семантически эквивалентных ему реальных языков: (1) язык кодирования текстов \textit{SC-кода} в памяти традиционных компьютеров; (2) язык кодирования текстов \textit{SC-кода} в семантической ассоциативной памяти; (3) \textit{Ядро SCg-кода} --- язык, синтаксически и семантически эквивалентный \textit{SC-коду} и обеспечивающий графическое представление текстов \textit{SC-кода}.}
        \end{scnindent}
    \end{scnindent}
    
    \scnheader{графовый язык}
    \begin{scnrelfromvector}{быть заданным}
        \scnfileitem{множество входящих в язык элементарных фрагментов (символов), которое, в свою очередь, состоит:
            \begin{scnitemize}
                \item из множества узлов (вершин), возможно, синтаксически разного вида;
                \item из множества связок, которые также могут принадлежать разным синтаксически выделяемым классам.
            \end{scnitemize}}
        \scnfileitem{в общем случае несколько отношений инцидентности связок с компонентами этих связок (при этом указанными компонентами в общем случае могут быть не только вершины, но и другие связки).}
    \end{scnrelfromvector}
    
    \scnheader{SC-код}
    \scntext{примечание}{Следует особо подчеркнуть, что  унификация и максимально возможное упрощение  \textbf{\textit{синтаксиса}} и \textbf{\textit{денотационной семантики}} внутреннего языка интеллектуальных компьютерных систем прежде всего необходимы потому, что подавляющий объем \textbf{\textit{знаний}}, хранимых в составе  базы знаний интеллектуальной компьютерной системы, представляют собой \textbf{\textit{метазнания}}, описывающие свойства других знаний.
    	\\К \textit{метазнаниям}, в частности, следует отнести и различного вида логические высказывания и всевозможного вида программы, описания методов (навыков), обеспечивающих решение различных классов задач. Необходимо исключить зависимость формы представляемого знания от вида этого знания.
    	\\Форма (структура) внутреннего представления знания любого вида должна зависеть \uline{только}(!) от смысла этого знания. Более того, конструктивное (формальное) развитие теории интеллектуальных компьютерных систем невозможно без уточнения (унификации, стандартизации) и обеспечения семантической совместимости различных видов знаний, хранимых в базе знаний интеллектуальной компьютерной  системы.  Очевидно, что многообразие форм представления семантически эквивалентных знаний делает разработку общей теории  интеллектуальных компьютерных систем практически невозможной.}
    \scntext{примечание}{\textit{SC-код} является одним из возможных вариантов \textit{смыслового представления знаний}.}
    \begin{scnindent}
        \scnrelfrom{смотрите}{Принципы, лежащие в основе смыслового представления информации}
    \end{scnindent}
            
    \scnheader{SC-пространство}
    \scntext{примечание}{Понятие \textit{SC-пространства} наряду с понятием \textit{SC-кода} играет важнейшую роль для уточнения и формализации понятия смысла информационных конструкций, для унификации смыслового представления информации и для максимально возможного исключения субъективизма в трактовке понятия смысла. Смысл информационной конструкции в конечном счете определяется (1) конфигурацией смыслового представления этой конструкции и (2) и местоположением (контекстом) смыслового представления указанной информационной конструкции в рамках смыслового пространства, т.е. в рамках объединенного смыслового представления \uline{всевозможных} информационных конструкций, либо в рамках объединенного смыслового представления информации, накопленной к заданному моменту времени некоторым индивидуальным субъектом или коллективом субъектов. Таким объединенным смысловым представлением информации, в частности, является смысловое представление глобальной базы всех знаний, накопленных человечеством к текущему моменту.}\scntext{пояснение}{Объединение (вместилище) \uline{всевозможных} унифицированных семантических сетей (текстов SC-кода)}\scntext{примечание}{При теоретико-множественном объединении текстов \textit{SC-кода} семантически эквивалентные (синонимичные) элементы (синтаксически элементарные фрагменты) этих текстов считаются совпадающими элементами и при объединении указанных текстов склеиваются.}\scnrelto{объединение}{SC-код}
    \scnidtf{Унифицированное смысловое пространство}
    \scntext{достоинство}{Важнейшим достоинством \textit{SC-пространства} является возможность уточнения таких понятий, как понятие аналогичности (сходства и отличия) различных описываемых внешних сущностей, аналогичности унифицированных семантических сетей (текстов \textit{SC-кода}), понятие семантической близости описываемых сущностей (в том числе, и текстов \textit{SC-кода}).}
    
    \bigskip
\end{scnsubstruct}
\scnsourcecommentinline{Завершили Сегмент \scnqqi{Основные положения внутреннего языка ostis-систем}}

        \begin{SCn}
\scnsuperset{фрагмент знака, представленный файлом ostis-системы}
\scnaddlevel{1}
\scnsuperset{синтаксически элементарный фрагмент информационной конструкции, представленный файлом ostis-системы}
\scnaddlevel{-1}

\scnsuperset{знак, представленный файлом ostis-системы}
\scnaddlevel{1}
\scnsuperset{sc-идентификатор, представленный файлом ostis-системы}
\scnaddlevel{-1}

\scnsuperset{\textbf{знаковая конструкция, представленная файлом ostis-системы}}
\scnaddlevel{1}
\scnsuperset{текст, представленный файлом ostis-системы}
    \scnaddlevel{1}
    \scnsuperset{знание, представленное файлом ostis-системы}
    \scnaddlevel{-1}
\scnaddlevel{-1}

\scnheader{знаковая конструкция, представленная файлом ostis-системы}
\scnsuperset{sc.s-файл ostis-системы}
\scnaddlevel{1}
\scnidtf{конструкция SCs-кода, хранимая в памяти ostis-системы в виде содержимого некоторого SC-узла}
\scnidtf{файл ostis-системы, являющийся sc.s-конструкцией}
\scnaddlevel{-1}
\scnsuperset{sc-идентификатор, представленный файлом ostis-системы}
\scnaddlevel{1}
\scnidtf{файл ostis-системы, являющийся sc-идентификатором}
\scnidtf{(sc-идентификатор $\cap$ файл ostis-системы)}
\scnreltoset{пересечение}{sc-идентификатор;файл ostis-системы}
\scnaddlevel{-1}
\scnsuperset{sc.g-файл ostis-системы}
\scnaddlevel{1}
\scnreltoset{пересечение}{sc.g-конструкция;файл ostis-системы}
\scnaddlevel{-1}
\scnsuperset{sc.n-файл ostis-системы}
\scnaddlevel{1}
\scnreltoset{пересечение}{sc.n-конструкция;файл ostis-системы}
\scnaddlevel{-1}

\scnsuperset{ея-файл ostis-системы}
\scnaddlevel{1}
\scnreltoset{пересечение}{ея-конструкция\\
    \scnaddlevel{1}
    \scnidtf{естественно-языковая конструкция}
    \scnidtf{конструкция естественного языка}
    \scnsuperset{ея-текст}
    \scnaddlevel{-1}
;файл ostis-системы}
\scnidtf{конструкция естественного языка, представленная в виде файла ostis-системы}
\scnsubset{Русский язык}
    \scnaddlevel{1}
    \scnidtf{конструкция Русского языка}
    \scnaddlevel{-1}
\scnsubset{Английский язык}
\scnidtf{естественно-языковая конструкция, являющаяся содержимм sc-узла, обозначающего эту конструкцию}
\scnaddlevel{-1}

\scnheader{файл ostis-системы}
\scnsubdividing{файл ostis-системы, предполагающий одномерную визуализацию хранимой информации
;файл ostis-системы, предполагающий двухмерную визуализацию хранимой информации
;файл ostis-системы, предполагающий трехмерную визуализацию хранимой информации}
\scnsubdividing{файл ostis-системы, представляющий статистическую информацию
;файл ostis-системы, представляющий динамическую информацию\\
\scnaddlevel{1}
\scnsuperset{видео-файл ostis-системы}
\scnsuperset{аудио-файл ostis-системы}
\scnaddlevel{-1}}
\scnsubdividing{файл-экземпляр ostis-системы\\
\scnaddlevel{1}
\scnidtf{\textit{sc-узел}, обозначающий конкретное вхождение \textit{информационной конструкции}, структура которой представлена содержимым этого \textit{sc-узла}}
\scnaddlevel{-1}
;файл-образец ostis-системы\\
\scnaddlevel{1}
\scnidtf{класс \textit{синтаксически эквивалентных} \textit{файлов-экземпляров} \textit{ostis-системы}}
\scnidtf{множество всевозможных \textit{файлов-экземпляров ostis-системы}, которые являются \textit{синтаксически эквивалентными} копиями содержимого заданного \textit{sc-узла}}
\scnaddlevel{-1}
}
\scnsubdividing{сформированный файл ostis-системы\\
\scnaddlevel{1}
\scnidtf{\textit{файл ostis-системы}, представленный \textit{sc-узлом}, имеющим сформированное (полностью построенное) содержимое}
\scnaddlevel{-1}
;несформированный файл ostis-системы\\
\scnaddlevel{1}
\scnidtf{\textit{файл ostis-системы}, представленный \textit{sc-узлом}, содержимое которого либо полностью отсутствует, либо сформировано \uline{частично}}
\scnaddlevel{-1}}
\scnnote{\textit{файл ostis-системы} можеть быть \uline{электронной копией} и знаком (!) внешней \textit{информационной конструкции}, которая может быть:
\begin{scnitemize}
\item общедоступный в сети Internet информационным ресурсом;
\item документом, опублиеованным на бумажном носители в виде какой-либо книги или статьи.
\end{scnitemize}
Кроме того , \textit{файл ostis-системы} может быть просто \uline{обозначением} указанной внешней \textit{информационной конструкции} (т.е. может быть \textit{sc-узлом}, обозначающим \textit{внешнюю информационную конструкцию}, но не имеющим содержимого). Такой \textit{sc-узел} используется формальной спецификации (средствами \textit{SC-кода}) соответствующего обозначаемого им информационного ресурса.
}
\scnheader{ея-файл ostis-системы}
\scnidtf{естественно-языковой \textit{файл ostis-системы}}
\scnexplanation{\textit{файл ostis-системы}, представляющий собой \textit{sc-узел}, содержимым которого является "электронная"{} форма \textit{информационной конструкции} (чаще всего, \textit{текста}) одного из \textit{естественных языков}}
\scnidtf{структурно выделенный ея-текст, хранимый в памяти ostis-системы (в sc-памяти)}
\scnrelfromvector{правила оформления}{
\scnfileitem{При оформлении текстов в естественно-языковых файлах (ея-файлах) \textit{ostis-систем} используются обычные разделители (точки в аббревиатурах и между предложениями\char59 круглые скобки, запятые, пробелы\char59), а также специальный символ $\square$, используемый \textit{в ея-файлах ostis-систем} как разделитель и целый ряд  следующих ограничителей, позволяющих выделять некоторые фрагменты ея-текстов:
\begin{scnitemize}
\item подчеркивание выделяет логически важные фрагменты в предложениях\char59
\item цитатные кавычки являются ограничителем цитат\char59
\item прямые кавычки ограничивают иносказательные термины, метафоры\char59
\item жирным курсивом стандартного размера выделяются идентификаторы \textit{sc-элементов} базы знаний, являющиеся ключевыми для заданного контекста\char59
\item жирным курсивом стандартного размера выделяются также условные внешние идентификаторы (обозначения) условно вводимых \textit{sc-элементов}, например, условные обозначения \textit{sc-элементов} произвольного структурного типа ($\bm{e_i}$, $\bm{e_j}$, ...), условные обозначения \textit{sc-узлов} ($\bm{v_i}$, $\bm{v_j}$, ...), условные обозначения \textit{sc-коннекторов} ($\bm{c_i}$, $\bm{c_j}$, ...), \textit{sc-дуг}, \textit{sc-связок}, не являющихся \textit{sc-коннекторами}, \textit{sc-структур} и так далее\char59
\item нежирным курсивом стандартного размера выделяются \textit{sc-идентификаторы} \textit{sc-элементов} \textit{базы знаний}, не являющиеся ключевыми для данного ея-текста.
\end{scnitemize}}
;\scnfileitem{Если в ея-тексте идентификаторы sc-элементов базы знаний (чаще всего -- \textit{простые sc-идентификаторы}), выделенные курсивом одинаковой жирности, следуют друг за другом, то число пробелов между разными идентификаторами должно быть увеличено. Если при этом выделенный sc-идентификатор включает в себя другие sc-идентификаторы, на которые желательно отдельно сослаться, то такая ссылка оформляется в виде дполнительной фразы типа "Смотрите также ..."{}}
;\scnfileitem{Текст \textit{ея-файла ostis-системы} может иметь \uline{любые} вставки не являющиеся естественно-языковыми текстами, в том числе, и фрагменты, являющиеся формальными внешними текстами представления знаний для ostis-систем (текстами SCg-кода, SCs-кода, SCn-кода). При этом указанные формальные фрагменты (вставки) могут быть как транслируемыми на внутренний язык ostis-системы (SC-код) и погружаемыми в состав ее базы знаний (т.е. фактически являться sc-текстами), так и нетранслируемыми формальными фрагментами, которые входят в состав базы знаний ostis-системы в виде содержимого соответствующих файлов. Все указанные выше "вставки" в ея-файл ostis-системы оформляются как ссылки на соответствующие sc-тексты или файлы ostis-системы. Каждая такая ссылка представляет собой \textit{sc-идентификатор} соответствующего sc-текста или файла ostis-системы и выделяется в ея-файле жирным курсивом стандартного размера со стандартным расстоянием между символами.
Таким образом, если в естественно-языковой \textit{файл ostis-системы}, необходимо "вставить" информационную конструкцию иного рода (\textit{sc.g-текст}, рисунок, таблицу, изображение), то (1) указанная конструкция оформляется как отдельный файл (2) которому приписывается имя (название), построенное по установленным правилам, и (3) на который в указанном естественно-языковом файле делается ссылка.
В естественно-языковых \textit{файлах ostis-систем} можно делать ссылки не только на другие \textit{файлы ostis-системы}, но и на \uline{именуемые} (!) фрагменты базы знаний, которые во внешнем представлении базы знаний оформляются в виде именуемых (идентифицируемых) sc.n-контуров.
Файлы и sc-тексты, на которые делаются ссылки из ея-файла, во внешнем представлении (при визуализации) базы знаний размещаются после указанного ея-файла в порядке первого их упоминания в этом ея-файле, если, конечно, на эти файлы или sc-тексты не было ссылок из ранее представленных ея-файлов.}
;\scnfileitem{В состав \textit{ея-файла ostis-системы} могут входить ссылки на любые идентифицированные (именованные) \textit{информационные конструкции} (Internet-ресурсы, библиографические источники, различные документы). Для этого достаточно указывать соответствующие \textit{sc-идентификаторы}.}
;\scnfileitem{В ея-текстах \uline{все} sc-основные идентификаторы описываемых в базе знаний сущностей должны быть выделены жирным или нежирным курсивом и могут быть представлены в любом склонении и спряжении.}
;\scnfileitem{В ея-текстах используются только основные идентификаторы (термины). Используемые синонимы явно указываются как неосновные идентификаторы.}
;\scnfileitem{Основная часть (содержимого текста) \textit{ея-файла ostis-системы} оформляется стандартным печатным шрифтом.}
}

\scnsourcecomment{Завершили перечень правил оформления содержимого \textit{ея-файлов ostis-систем}}

\scnheader{ея-файл ostis-системы}
\scnnote{Выделенные курсивом в \textit{ея-файле ostis-системы} \textit{sc-идентификаторы sc-элементов} могут являться \uline{ар\-гу\-мен\-та\-ми} различного вида \uline{запросов} к \textit{базе знаний ostis-системы} и, в первую очередь запросов типа "Что это такое"{}, предполагающих выделение из \uline{текущего} состояния \textit{базы знаний ostis-системы} семантической окрестности (спецификации) указываемого \textit{sc-элемента}, содержащей основную информацию о сущности, обозначаемой \textit{этим sc-элементом}.}


\end{SCn}
        \begin{SCn}
	
\scnsegmentheader{Структуризация баз знаний ostis-систем}

\scnstartsubstruct

\scnsubset{сегмент базы знаний}
\scnidtf{Структурная типология знаний ostis-системы}
\scntext{введение}{\textit{База знаний ostis-системы} имеет достаточно развитую иерархическую структуру. База знаний делится на разделы. Разделы бывают атомарными и неатомарными. Неатомарный раздел состоит из сегментов. Атомарные разделы не имеют сегментов. Разделы \textit{базы знаний ostis-системы} могут иметь самое различное назначение. Так, например, \textit{База знаний Метасистемы IMS.ostis} включает в себя:
\begin{scnitemize}
\item \textit{раздел}, содержащий текущее состояние постоянно пополняемого и совершенствуемого \textit{Стандарта} Технологии OSTIS;
\item \textit{раздел}, посвящённый описанию \textit{конечных пользователей и разработчиков} \textit{Метасистемы IMS.ostis}; 
\item \textit{раздел}, посвящённый описанию \textit{история эксплуатации  Метасистемы IMS.ostis};
\item \textit{раздел}, посвящённый описанию \textit{истории эволюции  Метасистемы IMS.ostis} (в т.ч. истории эволюции и её \textit{база знаний}); 
\item \textit{раздел} посвящённый описанию \textit{интеллектуальных компьютерных систем}, разработанных(порождённых) с помощью \textit{Метасистемы IMS.ostis}.
\end{scnitemize}}

\scnheader{выделенный фрагмент базы знаний}
\scnidtf{фрагмент базы знаний, для которого в \textit{базе знаний} вводится знак, обозначающий этот фрагмент, т.е. являющийся знаком множества \uline{всех} знаков, входящих в состав этого фрагмента. При представлении фрагмента базы знаний на внешних языках (SCg-коде, SCn-коде) указанный знак выделенного фрагмента базы знаний представляется либо в виде sc.g-контура, либо в виде пары фигурных скобок, ограничивающих текст обозначаемого фрагмента базы знаний}
\scnidtf{выделенный фрагмент базы знаний}
\scnaddlevel{1}
\scniselement{сокращённые sc-идентификатор} 
\scnaddlevel{-1}
\scnidtf{явно структурно выделенный фрагмент базы знаний}
\scnnote{явное выделение фрагмента базы знаний осуществляется:
\begin{scnitemize}
\item в \textit{SC-коде} путем введения \textit{знака}, обозначающего \textit{множество} \uline{всех} знаков, входящих в состав \textit{выделенного фрагмента базы знаний};
\item в \textit{SCg-коде} с помощью \textit{sc.g-контура}, ограничивающего \textit{sc.g-представление} \scnbigspace \textit{выделенного фрагмента базы знаний};
\item в \textit{SCn-коде} с помощью пары фигурных скобок, ограничивающих \textit{sc.n-представление} \scnbigspace \textit{выделенного фрагмента базы знаний}.
\end{scnitemize}}
\scnsubdividing{семейство разделов базы знаний
\scnaddlevel{1}
\scnidtf{семантически целостное множество разделов базы знаний, имеющих достаточно сильные семантические связи между собой}
\scnaddlevel{-1}
;раздел базы знаний
\scnaddlevel{1}
\scnidtf{Основной (по семантической значимости) вид выделенных фрагментов баз знаний}\\
\scnidtf{раздел базы знаний ostis-системы}
\scnidtf{модуль (блок) базы знаний}
\newpage
\scnnote{В общем случае многим \textit{разделам базы знаний} ставятся в соответствие такие тексты, как \uline{предисловие},  \uline{введение}, \uline{заключение}, \uline{аннотация}, \uline{оглавление}, упражнения.\\ Некоторые из этих текстов могут иметь статус разделов.}
\scnsubdividing{неатомарный раздел базы знаний\\
\scnaddlevel{1}
\scnidtf{раздел базы знаний, состоящий из сегментов, декомпозируемый на сегменты} 
\scnnote{В общем случае \textit{неатомарный раздел базы знаний} может иметь неограниченное число \textit{сегментов}. \textit{Сегменты базы знаний} не могут состоять из других сегментов (подсегментов). В этом смысле \textit{сегменты базы знаний} имеют атомарный характер.}
\scnaddlevel{-1}
;атомарный раздел базы знаний 
\scnaddlevel{1}
\scnidtf{раздел базы знаний, не содержащий сегментов}
\scnaddlevel{-1}}
\scnaddlevel{-1}
;сегмент базы знаний
\scnaddlevel{1}
\scnidtf{сегмент раздела базы знаний}
\scnidtf{сегмент базы знаний ostis-системы}
\scnidtf{структурно выделяемое sc-знание ostis-системы, структурный уровень которого ниже уровня разделов базы знаний}
\scnnote{Сегменты базы знаний не могут иметь иерархической структуры, т.е. не могут состоять из сегментов более низкого структурного уровня.}
\scnnote{Сегменты базы знаний входят в состав неатомарных разделов базы знаний.}
\scnaddlevel{-1}
;выделенный фрагмент сегмента или атомарного раздела базы знаний\\
\scnaddlevel{1}
\scnsubdividing{выделенный фрагмент атомарного раздела базы знаний;выделенный фрагмент сегмента базы знаний}
\scnsubset{sc-знание}
\scnnote{Каждый \textit{выделенный фрагмент базы знаний} представляет собой структурно оформленное (структурно выделенное) \textit{знание}, хранимое в \textit{базе знаний} \textit{интеллектуальной компьютерной системы}, (точнее, в \textit{базе знаний ostis-системы}) и представленное в формализованном виде на \textit{внутреннем языке представления знаний} (в \textit{SC-коде}) Представленное таким образом \textit{знание} будем называть \textit{sc-знанием}.}
\scnaddlevel{1}
\scnidtfexp{
	\textit{информационная конструкция}, которая:
	\begin{scnitemize}
		\item принадлежит \textit{SC-коду};
		\item является синтактически корректный;
		\item обладает семантической целостностью -- отсутствием \textit{информационных дыр} ("недомолвок"{}), препятствующих её пониманию;
		\item имеет нетривиальный \textit{объём информации} -- количество описываемых сущностей (в том числе, \textit{связей} и \textit{классов});
		\item имеет достаточно высокое качество по другим характеристикам, в частности, достаточно высокую ценность.
\end{scnitemize}}
\scnnote{Типология \textit{sc-знаний} по объему представленной информации определённым образом коррелирует с типологией \textit{выделенных фрагментов баз знаний} -- объём информации, содержащейся в \textit{разделах баз знаний}, должен быть приблизительно одинаковым, объём, содержащейся в разделе базы знаний, должен быть ниже объёма информации, содержащегося в любом семействе разделов базы знаний, и должен быть выше объёма информации, содержащегося в любом \textit{сегменте базы знаний}. При этом в процессе эволюции базы знаний \textit{раздел базы знаний} может преобразоваться в семейства разделов, а \textit{сегмент базы знаний} может преобразоваться в \textit{раздел базы знаний}.}
\scnaddlevel{-2}}

\scnheader{раздел базы знаний}
\scnnote{Различные \textit{предметные области}, различные \textit{онтологии}, а также предметные области, объединённые с соответствующими им онтологиями, являются важнейшими видами \textit{знаний ostis-систем}, обеспечивающими логически стройную систематизацию \textit{sc-знаний ostis-систем} и, соответственно семантическую структуризацию \textit{баз знаний}. При этом указанные типы знаний обычно представляются в виде \textit{разделов базы знаний}, иерархия которых, задаваемая Отношением \textit{частная предметная область*} или Отношением \textit{частная предметная область и онтология*}, соответствует логико-семантической иерархии \textit{предметных областей}. Но, кроме \textit{разделов базы знаний}, являющихся \textit{предметными областями}, \textit{онтологиями}, \textit{предметными областями}, объединенными с соответствующими им онтологиями, вводится и целый ряд других семантических типов \textit{разделов базы знаний}, определяемых характером соотношения разделов базы знаний с используемыми (рассматриваемыми) предметными областями и онтологиями.}
\scnsuperset{предметная область}
\scnsuperset{фрагмент предметной области}
\scnsuperset{интегрированная онтология}
\scnsuperset{частная онтология}
\scnsuperset{предметная область и онтология}
\scnaddlevel{1}
\scnidtf{предметная область, объединенная (интегрированная) с соответствующей ей онтологией}
\scnidtf{предметная область вместе с онтологией, которая её специфицирует}
\scnaddlevel{-1}
\scnnote{Если каждому \textit{разделу базы знаний} ostis-системы будет четко соответствовать его семантический тип, то к "синтаксическим"{} связям между \textit{разделами базы знаний} добавится большое количество "осмысленных"{} (семантические интерпретируемых) связей, определяющих "семантическое местоположение"{} ("семантические координаты"{}) каждого \textit{раздела базы знаний} во множестве всех разделов, входящих в состав базы знаний ostis-системы.}
\scnnote{В основе представления \textit{базы знаний ostis-системы} лежат развитые средства семантической структуризации баз знаний и семантической систематизации баз знаний \textit{ostis-систем}. Можно выделить следующие уровни систематизации элементов и фрагментов смыслового пространства, построенного на основе \textit{SC-кода}:
\begin{scnitemize}
\item уровень знаков всевозможных сущностей (уровень \textit{sc-элементов});
\item уровень вводимых \textit{понятий}, обозначающих ключевые (исследуемые) в предметных областях классы сущностей;
\item уровень \textit{высказываний}, описывающих закономерности (свойства) экземпляров исследуемых классов сущностей (исследуемых понятий);
\item уровень \textit{предметных областей}, \textit{онтологий} и разделов, семантический тип которых известен.
\end{scnitemize}}

\scnidtf{раздел б.з.}
\scnaddlevel{1}
\scniselement{сокращенный sc-идентификатор}
\scnaddlevel{-1}
\scnexplanation{Множество \textit{разделов баз знаний} имеет:
	\begin{scnitemize}
		\item богатую семантическую типологию;
		\item богатый набор отношений, описывающих семантические связи между разделами.
	\end{scnitemize}
	Синтаксически \textit{разделы баз знаний} могут пересекаться (иметь общие элементы), но никогда один раздел не может включаться (полностью входить в состав) другого раздела. В этом смысле понятие подраздела (точнее, \textit{частного раздела}*) имеет не "синтаксический"{} смысл, а семантический -- глубокое наследование свойств при достаточно большой степени "независимости"{} друг от друга}
\scnnote{Важным свойством \textit{раздела базы знаний} \scnbigspace \textit{ostis-системы} является его семантическая целостность -- наличие достаточно стабильного набора классов исследуемых сущностей (\textit{объектов исследования}) и достаточно стабильного семейства \textit{отношений} и семейства \textit{параметров}, заданных на различных классах объектов исследования, а также семейства \textit{классов структур} использующих указанные выше понятия (введенные \textit{классы объектов исследования}, введенные \textit{отношения} и \textit{классы структур}). Такая целостность дает возможность развивать \textit{раздел базы знаний}, не выходя "за рамки"{} используемой системы \textit{понятий}. Это позволяет развивать каждый \textit{раздел базы знаний} в известной мере \uline{независимо} от других разделов, что существенно повышает \uline{гибкость} и \uline{стратифицированность} \textit{базы знаний}.}
\newpage
\scnexplanation{Основными свойствами \textit{раздела базы знаний} как структурно \textit{выделяемого фрагмента базы знаний} являются следующие:
	\begin{scnitemize}
		\item семантическая целостность -- наличие четкого критерия, позволяющего установить для каждого конкретного знания то, включать или не включать это знание в состав данного раздела;
		\item потенциальная возможность эволюционировать в достаточной степени независимо от других разделов, но при условии соблюдения всех требований, обеспечивающих постоянную поддержку \uline{семантической совместимости} данного раздела со всеми остальными семантически смежными разделами.
\end{scnitemize}}
\scnaddlevel{1}
\scntext{следовательно}{Грамотная декомпозиция \textit{базы знаний} на разделы, основанная на четкой стратификации процесса эволюции накапливаемой человечеством общечеловеческой объединенной \textit{базы знаний}, сутью которой является \uline{минимизация} трудоемкости усилий по согласованию и обеспечению с авторами других разделов \textit{семантической совместимости} со смежными разделами, создает предпосылки высоких темпов эволюции \textit{базы знаний} в целом.}
\scnaddlevel{-1}
\scnnote{Любой \textit{раздел базы знаний} не является структурной частью другого раздела. Каждый раздел самодостаточен и целостен. Это обеспечивается тем, что в состав \textit{титульной спецификации} каждого раздела входит \textit{семантическая окрестность},описывающая связи специфицируемого раздела \uline{со всеми} семантически близкими ему разделами.\\
	При этом разные разделы могут иметь разный семантический тип:
	\begin{scnitemize}
		\item раздел может быть предметной областью, интегрированной со всеми ее онтологиями;
		\item раздел может быть просто предметной областью;
		\item раздел может быть какой-либо онтологией чего угодно (не обязательно предметной области);
		\item и т.д.
\end{scnitemize}}
\scnnote{Если в \textit{титульную спецификацию} каждого раздела будет входить семантическая спецификация каждого раздела, включающая его семантические связи со всеми семантически близкими разделами, то последовательность (порядок) разделов в "линейном"{} исходном тексте, публикуемом в качестве очередной версии Стандарта OSTIS, может быть в достаточной степени \uline{произвольной}.}


\scnheader{семейство разделов базы знаний}
\scnidtf{множество семантически связанных друг с другом разделов базы знаний}
\scnidtf{кластер разделов базы знаний}
\scnnote{Семантические связи между разделами, входящими в состав семейства разделов, представляются в рамках \textit{титульных спецификаций разделов}, каждая из которых является специальной частью соответствующего (специфицируемого) раздела, входящего в состав семейства разделов.
	
	Для каждого \textit{раздела базы знаний} в рамках его титульной спецификации формируется \textit{семантическая окрестность} его связей со всеми семантически близкими ему разделами (и, прежде всего, с теми разделами, которые входят в состав тех семейств разделов, в которые входит заданный раздел). При этом акцентируется внимание именно на семантических связях между разделами. Так, например, вместо структурной ("синтаксической"{}) связи ``быть подразделом*''{} (т.е. быть частью заданного раздела) вводится связь ``быть \textit{дочерним разделом}*''{}. \\
	Данная связь указывает направление наследования свойств исследуемых объектов заданного раздела от разделов, исследующих более общие классы объектов.
	Порядок (последовательность) разделов в рамках \textit{семейства разделов базы знаний} при наличии \uline{явно представленных} семантических связей между разделами, входящими в семейство разделов, может быть достаточно \uline{произвольным}, что очень важно, например, при формировании оглавления очередной издаваемой версии \textit{Стандарта OSTIS}. Таким образом, трактовка \textit{Стандарта OSTIS}, а также всех издаваемых версий этого Стандарта как \textit{семейства разделов базы знаний} \scnbigskip \textit{Метасистемы IMS.ostis} обеспечивает высокий уровень гибкости \textit{Стандарта OSTIS}, а также легкость "переиздаваемости"{} его версий.}


\scnheader{сегмент или атомарный раздел базы знаний}
\scnnote{Простейшей формой \textit{сегмента} или \textit{атомарного раздела базы знаний} является просто последовательность \textit{файлов ostis-системы}. Некоторые из этих файлов могут быть идентифицированными (именованными), если на них ссылаются другие файлы, а некоторые из них могут быть связаны с другими файлами различными отношениями (в частности, один файл может быть пояснением другого). Кроме того, некоторые из этих файлов могут быть формально специфицированы (например, указаны соответствующие им ключевые \textit{sc-элементы}).\\
	В самом простом случае \textit{сегмент} или \textit{атомарный раздел базы знаний} может быть \textit{структурой}, состоящей из \uline{одного} (!) \textit{sc-узла}, обозначающего \textit{файл ostis-системы} (чаще всего, \textit{ея-файл ostis-системы}). Т.е. сам \textit{файл ostis-системы} может быть \textit{знанием ostis-системы}, но не может быть структурно \textit{выделяемым} \textit{фрагментом базы знаний} ostis-системы. При этом \textit{sc-узел}, обозначающий \textit{файл ostis-системы}, являющийся \textit{знанием}, может быть единственным \textit{sc-элементом} структурно выделяемого \textit{знания ostis-системы}.}
\scnnote{Для наглядного отображения (визуализации) \textit{сегмента} или \textit{атомарного раздела базы знаний ostis-системы} целесообразно представить указанное \textit{sc-знание} в виде конкатенации (последовательности) таких sc-знаний, которые, во-первых, были бы достаточно крупными и логико-семантически значимыми для соответствующего \textit{сегмента} или \textit{атомарного раздела базы знаний ostis-системы} и, во-вторых, для которых существовал бы алгоритм \uline{однозначного} (!) размещения (на экране) внешнего представления этих \textit{sc-знаний} (в \textit{SCg-коде} или в \textit{SCn-коде}).\\
	Однозначность здесь означает наличие легко усваиваемого пользователями стандартного \uline{стиля визуализации} \textit{sc-знаний} и заключается в том, что многократная визуализация одного и того же \textit{sc-знания} с помощью указанного алгоритма должна приводить к синтаксически эквивалентным, а в случае \textit{SCg-кода} и к геометрически конгруэнтным текстам. Очевидно, что для произвольных \textit{sc-знаний} большого объёма такого алгоритма не существует, но для \textit{sc-знаний}, содержащих описание собственной структуры и семантической типологии собственных фрагментов, разработка такого алгоритма вполне реальна при наличии достаточного количества указанных \textit{метазнаний} о структуре отображаемых (визуализируемых) \textit{sc-знаний}.}


\scnheader{sc-идентификатор выделенного фрагмента базы знаний}
\scnidtf{название (имя) выделенного фрагмента базы знаний}
\scnexplanation{Не следует путать объект описания (спецификации) и само описание. Поэтому в \textit{sc-идентфикаторе} фрагмента базы знаний должны присутствовать слова, указывающие на семантический или структурный тип именуемого фрагмента базы знаний (описание, спецификация, анализ, сравнительный анализ, сравнение, определение, раздел, предметная область, онтология и т.п.).
	
	Таким образом, \textit{sc-идентификатор выделенного фрагмента базы знаний} ostis-системы должен иметь \uline{явное} (!) указание на то, что он является обозначением именно фрагмента базы знаний, а не того, что описывается в этом фрагменте.}
\scnnote{Мы не будем использовать такой изменчивый для нас способ идентификации разделов \textit{Стандарта OSTIS}, как нумерацию этих разделов, поскольку, например, в разных издаваемых официальных версиях \textit{Стандарта OSTIS} одному и тому же разделу \textit{Стандарта OSTIS} могут соответствовать разные номера.}


\scnheader{выделенный фрагмент базы знаний}
\scntext{основной sc-идентификатор}{выделенный фрагмент базы знаний}
\scnaddlevel{1}
\scntext{используемая аббревиатура}{выделенный фр-нт б.з.}
\scnaddlevel{-1}
\scnsubdividing{именованный фрагмент базы знаний\\
	\scnaddlevel{1}
	\scnidtf{\textit{выделенный фрагмент базы знаний}, имеющий \textit{sc-идентификатор} (имя, название)}
	\scnnote{\textit{Именованными фрагментами баз знаний} могут быть только структурно \textit{выделенные фрагменты баз знаний}}
	\scnnote{Все \textit{семейства разделов баз знаний}, все \textit{разделы баз знаний} и все \textit{сегменты баз знаний} должны быть именованными}
	\scnsuperset{семейство разделов базы знаний}
	\scnsuperset{раздел базы знаний}
	\scnsuperset{сегмент базы знаний}
	\scnaddlevel{-1}
	;неименованный фрагмент базы знаний\\
	\scnaddlevel{1}
	\scnidtf{\textit{выделенный фрагмент базы знаний}, \uline{не} имеющий \textit{sc-идентификатора} (имени, названия)}
	\newpage
	\scnnote{\textit{неименованными фрагментами баз знаний} могут быть только \textit{выделенные фрагменты сегментов баз знаний} либо выделенные фрагменты таких \textit{разделов баз знаний}, которые не состоят из \textit{сегментов}}
	\scnaddlevel{-1}}

\scnheader{титульная спецификация выделенного фрагмента базы знаний}
\scnexplanation{\textit{Титульная спецификация выделенного фрагмента базы знаний} ostis-системы представляет собой \textit{структуру}, описывающую свойства специфицируемого знания и включающую в себя: 
	\begin{scnitemize}
		\item связи принадлежности специфицируемого знания соответствующим классам \textit{знаний ostis-систем};
		\item связи, указывающие логически предшествующее и логически следующее \textit{знание ostis-системы};
		\item связь, описывающую декомпозицию специфицируемого знания на последовательность знаний более низкого структурного уровня (декомпозицию разделов на сегменты);
		\item различного вида связи с другими \textit{знаниями ostis-систем}, которые сами "целиком"{} входят в состав спецификации специфицируемого знания (такими знаниями могут быть аннотации, предисловия, введения, оглавления, заключения);
		\item различного вида связи с другими \textit{знаниями ostis-систем}, которые сами не входят в состав спецификации специфицируемого знания (такого рода связями могут быть связи \textit{семантической близости} специфицируемого знания с другими знаниями, связи \textit{семантической эквивалентности}, связи\textit{семантического включения}, связи \textit{противоречивости знаний});
		\item связи, указывающие различного вида \textit{ключевые sc-элементы} (ключевые знаки), соответствующие специфицируемому знанию;
		\item связи специфицируемого знания с авторским коллективом, коллективом рецензентов, с датой его последнего обновления;
		\item для каждого нового целостного фрагмента, вводимого в состав \textit{базы знаний}, в истории эволюции этой \textit{базы знаний} указываются:
		\begin{scnitemizeii}
			\item \textit{автор*} или \textit{авторы*} первой версии этого фрагмента;
			\item отметка времени появления (дата-час-минута) всех версий этого фрагмента (в том числе и окончательно утверждённой, согласованной версии, которая, собственно, и становится фрагментом, включенным в согласованную часть базы знаний);
			\item \textit{рецензии*} (замечания к доработке) всех предварительных версий разрабатываемого \textit{фрагмента базы знаний};
			\item \textit{авторы*} всех указанных рецензий;
			\item отметка времени появления всех указанных рецензий;
			\item события по одобрению, утверждению различных предварительных версий разрабатываемого \textit{фрагмента базы знаний} различными рецензентами и экспертами с указанием отметки времени появления этих событий;
			\item темпоральная последовательность предварительных версий.
		\end{scnitemizeii}
	\end{scnitemize}
}
\scnnote{Знак такой спецификации явно не вводится, а сама эта спецификация непосредственно входит в состав специфицируемого фрагмента и включает в себя аннотацию, предисловие, авторов, ключевые знаки, декомпозицию специфицируемого фрагмента базы знаний и прочее}
\scnsubdividing{титульная спецификация раздела базы знаний;
	титульная спецификация семейства разделов базы знаний;
	титульная спецификация сегмента базы знаний;
	титульная спецификация выделенного фрагмента сегмента или атомарного раздела базы знаний}
\scnexplanation{\textit{Титульная спецификация выделенного фрагмента базы знаний} содержит общую информацию об этом фрагменте, является непосредственно \uline{частью} специфицируемого фрагмента \textit{базы знаний} и при этом сама \uline{не является} явно \textit{выделенным фрагментом базы знаний}}
\scniselement{спецификация}
\scnidtf{основная \textit{метаинформация} (основное \textit{метазнание}) о \textit{выделенном фрагменте базы знаний} -- о его структуре, \textit{авторах\scnrolesign}, \textit{ключевых знаках\scnrolesign} и т.д.}
\scnexplanation{\uline{неявно} \textit{выделяемый фрагмент базы знаний}, который:
	\begin{scnitemize}
		\item не имеет "собственного"{} ограничителя ("собственного"{} контура или "собственных"{} ограничивающих фигурных скобок);
		\item является семантической спецификацией соответствующего \uline{явно} выделяемого фрагмента базы знаний;
		\item является непосредственной \uline{частью} специфицируемого фрагмента базы знаний
\end{scnitemize}}
\scnnote{В \textit{sc.n-тексте} титульная спецификация фрагмента базы знаний размещается сразу после фигурной скобки, открывающей этот фрагмент}

\scnheader{титульная спецификация раздела базы знаний}
\scnnote{\textit{титульная спецификация раздела базы знаний} должна включать в себя достаточно подробное описание семантических свойств этого раздела и, в частности, подробное описание его связей с другими семантически близкими разделами. Это необходимо для обеспечения автономности разделов баз знаний.}

\scnheader{титульная спецификация семейства разделов базы знаний}
\scnnote{Если \textit{разделы базы знаний} являются семантически \uline{ключевыми} \textit{выделенными фрагментами баз знаний}, определяющими спецификацию систем используемых понятий и направления наследования свойств, то \textit{семейства разделов баз знаний} являются \uline{ключевыми} для структуризации виртуальной \textit{Базы знаний Экосистемы OSTIS}, для обмена \textit{знаниями} между различными субъектами \textit{Экосистемы OSTIS}.\\
	Поэтому типология \textit{семейств разделов баз знаний} и качество \textit{титульной спецификации семейств разделов баз знаний} имеют большое значение.}

\scnheader{титульная спецификация выделенного фрагмента базы знаний}
\scnrelfrom{множество используемых понятий}{Множество понятий используемых в титульных спецификациях выделенных фрагментов баз знаний}
\scnaddlevel{1}
\scnsuperset{класс выделенных фрагментов}
\scnaddlevel{1}
\scnhaselement{семейство разделов базы знаний}
\scnhaselement{раздел базы знаний}
\scnhaselement{неатомарный раздел базы знаний}
\scnhaselement{атомарный раздел базы знаний}
\scnhaselement{сегмент базы знаний}
\scnhaselement{выделенный фрагмент сегмента или атомарного раздела базы знаний}
\scnhaselement{выделенный фрагмент атомарного раздела базы знаний}
\scnhaselement{выделенный фрагмент сегманта базы знаний}
\scnaddlevel{-1}
\scnsuperset{отношение, связывающее выделенные фрагменты баз знаний с персонами}
\scnaddlevel{1}
\scnhaselement{автор*}
\scnhaselement{рецензент*}
\scnhaselement{эксперт*}
\scnhaselement{технический редактор*}
\scnhaselement{консультант*}
\scnaddlevel{1}
\scnidtf{активный участник обсуждения вопросов, рассматриваемых в специфицируемом фрагменте базы знаний*}
\scnaddlevel{-2}
\scnsuperset{отношение, связывающее выделенные фрагменты баз знаний с ея-файлами}
\scnaddlevel{1}
\scnhaselement{аннотация*}
\scnhaselement{предисловие*}
\scnaddlevel{1}
\scnidtf{Бинарное ориентированное отношение, каждая пара которого связывает:
	\begin{scnitemize}
		\item знак некоторого информационного ресурса (в частности, раздела базы знаний или раздела опубликованного документа);
		\item знак информационной конструкции, описывающей цели создания указанного информационного ресурса, предысторию его создания, планируемые направления дальнейшего развития, состав авторов и др.
\end{scnitemize}}
\scnaddlevel{-1}
\scnhaselement{введение*}
\scnhaselement{эпиграф*}
\scnhaselement{заключение*}
\scnhaselement{рассматриваемый вопрос*}
\scnhaselement{основные положения*}
\scnhaselement{вопрос для самопроверки*}
\scnhaselement{упражнение*}
\scnaddlevel{1}
\scnidtf{задача*}
\scnidtf{самостоятельная (индивидуальная) работа*}
\scnaddlevel{-1}
\scnhaselement{коллективный проект*}
\scnhaselement{неосновной sc-идентификатор*}
\scnaddlevel{1}
\scnnote{неосновным sc-идентификатором, в частности, может быть альтернативное название выделенного (специфицируемого) фрагмента базы знаний}
\scnaddlevel{-1}
\scnhaselement{часто используемый sc-идентификатор*}
\scnhaselement{сокращенный sc-идентификатор*}
\scnhaselement{библиографический источник, отражающий аналогичную точку зрения*}
\scnhaselement{библиографический источник, отражающий альтернативную точку зрения*}
\scnhaselement{библиографический источник, дополняющий данную точку зрения*}
\scnhaselement{сокращение*}
\scnaddlevel{1}
\scnidtf{используемое сокращение*}
\scnidtf{сокращение, используемое в специфицируемом фрагменте базы знаний при построении sc-идентификаторов, а также при оформлении ея-файлов*}
\scnidtf{Бинарное ориентированное отношение, каждая пара которого связывает естественно-языковую фразу с ее сокращенной записью*}
\scnaddlevel{-2}
\scnsuperset{отношение, описывающее структурные или семантические связи и между выделенными фрагментами баз знаний}
\scnaddlevel{1}
\scnhaselement{конкатенация сегментов*}
\scnhaselement{предыдущий сегмент*}
\scnaddlevel{1}
\scnidtf{предыдущий сегмент в рамках соответствующего раздела*}
\scnaddlevel{-1}
\scnhaselement{следующий сегмент*}

\scnhaselement{дочерний фрагмент базы знаний*}
\scnaddlevel{1}
\scnsuperset{дочерний раздел базы знаний*}
\scnsuperset{дочерняя предметная область*}
\scnsuperset{дочерняя предметная область и онтология*}
\scnnote{Для фрагмента базы знаний важно указать не только дочерние по отношению к нему фрагменты базы знаний, но и те фрагменты базы знаний, по отношению к которым данный фрагмент базы знаний является дочерним}
\scnaddlevel{-2}

\scnsuperset{отношение, описывающее ролевой статус знаков, входящих в состав выделенных фрагментов баз знаний}
\scnaddlevel{1}
\scnhaselement{ключевой знак*}
\scnhaselement{ключевой знак первого плана*}
\scnhaselement{ключевой знак второго плана*}
\scnhaselement{ключевой объект исследования*}
\scnhaselement{ключевое понятие*}
\scnhaselement{ключевой класс объектов исследования*}
\scnhaselement{исследуемое отношение*}
\scnhaselement{исследуемый параметр*}
\scnhaselement{исследуемый класс структур*}
\scnaddlevel{-2}

\bigskip

\scnendstruct \scnendsegmentcomment{Структуризация баз знаний ostis-систем}

\end{SCn}

        \begin{SCn}

\vspace{-2\baselineskip}
\scnsegmentheader{Описание правил оформления внешнего представления знаний ostis-систем}

\scnstartsubstruct
\scnheader{внешнее представление знаний ostis-системы}
\scnexplanation{внешнее представление некоторого фрагмента базы знаний ostis-системы, используемое для ввода новой информации в состав базы знаний ostis-системы или для вывода (отображения) запрашиваемого фрагмента базы знаний}
\scnnote{Способ представления исходных текстов баз знаний ostis-систем должен быть максимально возможным образом использован и для вывода (отображения) запрашиваемых пользователем фрагментов баз знаний, особенно, если запрашиваются достаточно большие фрагменты баз знаний, которые необходимо не только представлять, но и структурировать унифицированным образом. Очевидно, что для пользователей желательно, чтобы и для ввода информации в ostis-систему, и для ее вывода использовались одни и те же языковые средства и правила оформления}
\scnnote{Требования, предъявляемые к оформлению внешних текстов знаний ostis-систем (sc-знаний) носят достаточно противоречивый характер -- с одной стороны, речь идет о формальных текстах, легко воспринимаемых (понимаемых, транслируемых) ostis-системами, а, с другой стороны, желательно, чтобы эти же формальные тексты легко воспринимались (понимались) широким кругом людей и не требовали для этого от них длительной подготовки. Отметим при этом, что работа с формальными текстами требует от человека достаточно высокой культуры \uline{точного} мышления (математической культуры).

Отметим также, что использование формальных языков является важнейшим и необходимым этапом эволюции человеческой деятельности в любой области (в математике, в физике, в технике).

Тем не менее, проблема создания универсального языка представления исходных текстов различного вида знаний, который был бы достаточно удобен как для интеллектуальных компьютерных систем, так и для \uline{широкого} круга разработчиков баз знаний и экспертов, требует конкретного решения.}

\scnreltovector{требования}{\scnfileitem{Стиль и характер оформления внешнего представления sc-знаний должен обеспечить возможность интуитивного понимания смысла текста при отсутствии понимания различного рода синтаксических деталей. Для этого:
\begin{scnitemize}
\item формальный текст должен максимально возможным образом использовать привычную для широкого круга специалистов терминологию\char59
\item структуризация, форматирование формальных текстов также должны опираться на сформировавшиеся традиции\char59
\item внешнее представление (внешний текст) sc-знания должен включать в себя такое количество отображаемых ея-файлов, прочтения которых было бы достаточно для понимания смысла представляемого sc-знания, а также для понимания формальных средств его представления
\end{scnitemize}};
\scnfileitem{Формальный текст (как внутреннего, так и внешнего представления sc-знаний) должен включать в себя средства для уточнения смысла используемых знаков и соответствующих им терминов, а также смысла некоторых фрагментов формального текста. Для этого в формальный язык вводятся естественно-языковые файлы, отображаемые в исходных текстах и поясняющие используемые термины, а также комментирующие или даже полностью переводящие на естественный язык различные фрагменты формального представления \textit{базы знаний}.};
\scnfileitem{Все используемые в базе знаний ostis-системы (в том числе, и в её ея-файлах) внешние идентификаторы \textit{sc-элементов} (термины, имена, условные обозначения) должны быть формально специфицированы средствами \textit{SC-кода}. Подчеркнем, что здесь речь идет о спецификации не самих sc-элементов, а их внешних идентификаторов (в первую очередь, простых sc.s-идентификаторов) -- их происхождение, использование, авторство и т.д.};
\scnfileitem{Все отношения, параметры и другие понятия, используемые в формальных текстах должны быть пояснены в соответствующих формальных онтологиях. Первое упоминание во внешнем тексте каждого такого понятия должно быть кратко пояснено с помощью поясняющего ея-файла, а также сделана ссылка на раздел базы знаний, в которых приведена подробная и формальная спецификация указанного понятия с дополнительным указанием номера этого раздела с помощью нетранслируемого комментария.};
\scnfileitem{Аналогичным образом в отображаемом внешнем представлении sc-знания поясняются и комментируются все \uline{первые} (в рамках этого внешнего представления) использования средств формального представления знаний со ссылками на разделы базы знаний, где указанные языковые средства подробно описываются. С самого начала внешнего представления большого структурированного фрагмента базы знаний (каковым, в частности, является Стандарт OSTIS) с помощью нетранслируемых комментариев, не входящих в состав базы знаний, либо с помощью ея-файлов ostis-систем необходимо пояснять все нюансы формализации со ссылкой на ближайший раздел и сегмент, где это будет подробнее рассмотрено.}
\newpage;
\scnfileitem{При описании формальных средств должны быть приведены конкретные \uline{примеры} со ссылкой на раздел или сегмент, где этот пример будет рассмотрен подробнее (например, на соответствующую предметную область и онтологию)};
\scnfileitem{Все комментарии и примечания, которые можно представить средствами \textit{SCn-кода} или \textit{SCg-кода}, нужно оформлять именно так. Нетранслируемыми комментариями  \uline{не стоит увлекаться}.};
\scnfileitem{В формальных текстах и в естественно-языковых файлах, входящих в состав \textit{базы знаний \mbox{ostis-системы}}, для идентификации (именования) \textit{sc-элементов} должны использоваться только те термины, которые являются \uline{основными}(!) внешними идентификаторами соответствующих \textit{sc-элементов}, выделяемыми жирным и нежирным курсивом. При этом, если идентификаторы (названия, имена) разделов, сегментов базы знаний находятся в позиции \uline{заголовков} указанных фрагментов базы знаний, то они оформляются жирным курсивом с увеличенным расстоянием между символами, а заголовки разделов дополнительно выделяются увеличенным размером символов.};
\scnfileitem{В согласованной (общепризнанной) части \textit{базы знаний} противоречия трактуются как выявленные ошибки в \textit{базе знаний}, подлежащие устранению. Но в истории эволюции
базы знаний противоречия могут присутствовать как противоречия разных точек зрения разных авторов. Заметим при этом, что разные точки зрения далеко не всегда являются противоречивыми (взаимоисключающими). Они могут просто дополнять друг друга, описывать исследуемые сущности с разных "ракурсов". Умение видеть противоречия только там, где, они действительно есть, и умение локализовать эти противоречия (выделить их суть) -- это необходимые навыки для разработки \uline{практически полезных} \textit{баз знаний}.}}

\scnheader{sc.n-представление знаний ostis-системы}
\scnrelfromvector{правила оформления}{\scnfileitem{В исходных текстах баз знаний \textit{интеллектуальных компьютерный систем}, построенных по \textit{Технологии OSTIS} (т.е. ostis-систем) используются следующие языки:
	\begin{scnitemize}
	\item Различные \textit{естественные языки} (\textit{\scnkeyword{Русский язык}}, \textit{\scnkeyword{Английский язык}} и др.)
	\item \textit{\scnkeyword{SCg-код}} (Semantic Code graphical), являющийся \textit{универсальным формальным языком} графического представления \textit{семантических сетей}
	\item \textit{\scnkeyword{SCs-код}} (Semantic Code string), являющийся \textit{универсальным формальным языком} линейного представления \textit{семантических сетей} в виде \textit{строк} (цепочек символов)
	\item \textit{\scnkeyword{SCn-код}} (Semantic Code Natural) являющийся \textit{универсальным формальным языком} представления \textit{семантических сетей} в виде структурированных форматированных \textit{текстов} (строк, размещенных на плоскости.
	\end{scnitemize}	
\textit{Синтаксис} и \textit{денотационная семантика} указанных \textit{формальных языков} подробно рассмотрены в Разделах \nameref{intro_scg}, \nameref{intro_scs}, \nameref{intro_scn}.};
\scnfileitem{Основным языком внешнего представления \textit{баз знаний ostis-систем} является \textit{SCn-код}, рассмотренный в разделе \nameref{intro_scn}. Но в состав текста \textit{SCn-кода} могут входить тексты и других языков (тексты \textit{SCg-кода}, тексты \textit{SCs-кода}, тексты естественных языков, тексты различных искусственнных языков), а также различного рода нетекстовые информационные конструкции (рисунки, таблицы, чертежи, графики, фотографии). Указанные "инородные"{} для \textit{SCn-кода} информационные конструкции, а также описываемые тексты самого \textit{SCn-кода} оформляются во внешнем представлении базы знаний либо как нетранслируемые, но специфицируемые файлы, либо как транслируемые инородные для \textit{SCn-кода} информационные конструкции, ограниченные, соответственно, либо \textit{sc.n-рамками} (квадратными скобками), либо \textit{sc.n-контурами} (фигурными скобками).};
\scnfileitem{Структуризация внешнего представления баз знаний ostis-систем является полным отражением структуризации внутреннего представления \textit{баз знаний ostis-систем}.};
\scnfileitem{В случае, если осуществляется внешнее представление \uline{полного} текста указываемого сложноструктурированного фрагмента базы знаний, как, например, внешнее представление семейства разделов под названием ``\textit{Стандарт OSTIS}'' вместе со всеми его разделами, то последовательность и иерархическая структура отображения разделов и сегментов указанного сложноструктурированного фрагмента базы знаний в точности соответствует иерархической структуре представляемого (отображаемого, визуализируемого) семейства разделов базы знаний.};
\scnfileitem{База знаний каждой \textit{ostis-системы} представляет собой иерархическую систему разделов, к которым должны "привязываться"{} исходные тексты каждой новой информации, вводимой в \textit{базу знаний}, и, прежде всего, исходные тексты достаточно крупных фрагментов \textit{баз знаний}. К таким исходным текстам, в частности, относится и данная публикуемая версия Стандарта OSTIS. Очевидно при этом, что нумерация разделов баз знаний не может быть стабильна. Кроме того, при оформлении исходного текста крупного фрагмента базы знаний, обладающего достаточной целостностью и по научно-технической значимости достигшего уровня монографии или диссертации, желательно иметь собственную (свою, локальную) нумерацию разделов при сохранении их иерархической структуры. Это означает, что номера разделов имеет смысл использовать только в рамках каждого исходного вводимого текста и не должны использоваться в самой базе знаний. Таким образом, ссылаться на разделы базы знаний следует по \uline{названию} разделов.};
\scnfileitem{Внешнее представление каждого структурно \textit{выделяемого фрагмента базы знаний} ostis-системы за исключением sc-знаний нижнего уровня начинается с \uline{заголовка} (имени, названия) этого фрагмента. Указанный заголовок есть не что иное, как простой sc-идентификатор sc-узла, обозначающего представляемый фрагмент базы знаний (представляемое sc-знание). Рассматриваемый заголовок оформляется жирным курсивом с \uline{увеличенным расстоянием между символами}. При этом заголовки \uline{внешнего представления} разделов базы знаний имеют дополнительно \uline{увеличенный размер шрифта}. Заголовок внешнего представления sc-знания размещается с первого символа строчки.
После заголовка представляемого фрагмента базы знаний \uline{с новой строчки} размещается (1) sc.s-коннектор вида ``$\supset$='' (2) следом за ним на следующей строчке левая фигурная скобка (открывающая фигурная скобка).};
\scnfileitem{Внешнее представление каждого раздела базы знаний начинается \uline{с новой страницы}. Соответственно, сегмент базы знаний может начинаться \uline{не} с новой страницы.};
\scnfileitem{Внешнее представление каждого неструктурируемого выделяемого фрагмента базы знаний (атомарного раздела базы знаний, сегмента базы знаний) оформляется в виде \textit{sc.n-предложения}, связывающего sc-идентификатор (имя, название) представляемого фрагмента базы знаний, оформленный в виде \uline{заголовка} внешнего текста этого фрагмента (признаком чего является жирный курсив с увеличенным расстоянием между символами), с \textit{sc.n-контуром}, который фигурными скобками ограничивает sc.n-изображение (sc.n-визуализацию) представляемого фрагмента базы знаний.\\
В самом простом случае представляемым неструктурируемым выделяемым фрагментом базы знаний является \uline{один} ея-файл ostis-системы.};
\scnfileitem{Внешний текст \textit{базы знаний} может иметь самый разный объем и может касаться только \uline{одного раздела} или сегмента, а может включать в себя материалы \uline{нескольких разделов} или сегментов. Если представляемый (отображаемый) фрагмент базы знаний является sc-знанием нижнего уровня иерархии, т.е. частью \uline{неструктурированного} фрагмента базы знаний (например, частью сегмента базы знаний), то при его представлении необходимо указать, частью какого неструктурированного фрагмента базы знаний представляемый фрагмент является.\\
Для исходного текста sc-знания нижнего уровня эта информация необходима для того, чтобы знать, в какой фрагмент базы знаний требуется включить, "погрузить"{} данное вводимое \textit{sc-знание}.};
\scnfileitem{Виды выделений во внешнем представлении знаний ostis-системы:
\begin{scnitemize}
    \item с помощью шрифта
    \begin{scnitemizeii}
        \item вид шрифта (печатный, курсив)
        \item размер шрифта (стандартный, увеличенный)
        \item расстояние между символами (стандартное, увеличенное)
    \end{scnitemizeii}
    \item подчеркиванием
    \item с помощью символьных ограничителей (скобок различного вида)
\end{scnitemize}};
\scnfileitem{Выделяемые объекты:
\begin{scnitemize}
    \item основные термины (имена, идентификаторы)
    \item специфицируемые файлы
    \begin{scnitemizeii}
        \item нетранслируемые в базу знаний файлы
        \item транслируемые в базу знаний файлы
    \end{scnitemizeii}
    \item нетранслируемые комментарии к внешнему тексту
    \item цитаты (и короткие, и длинные)
    \item ключевые фрагменты ея-текста
    \item метафорические термины
\end{scnitemize}}
\newpage;
\scnfileitem{На любой странице внешнего текста при распечатке делается разметка тонкими вертикальными табуляционными линиями для четкой визуализации длины отступа от левого края страницы. Особенно это важно при переходе на новую страницу.}} 
\scnaddlevel{1}
\scnsourcecommentpar{Завершили перечень правил оформления внешнего представления знаний ostis-системы} 
\scnaddlevel{-1}

\scnheader{sc.n-представление выделенного фрагмента базы знаний}
\scnreltovector{обобщенная конкатенация}{признак начала sc.n-представления выделенного фрагмента базы знаний\\
	\scnaddlevel{1}
	\scnexplanation{Данным признаком является \textit{нетранслируемый комментарий}, размещенный по всей длине \textit{строчки} и состоящий
	\begin{scnitemize}
	\item либо из слова "Раздел"{} (изображенного жирным печатным шрифтом такого же размера и с таким же расстоянием между символами, что и в \textit{заголовке sc.n-представления раздела базы знаний}) и последующих "звездочек"{} до конца \textit{строчки};
	\item либо из слова "Сегмент"{} (изображенного жирным печатным шрифтом такого же размера и с таким же расстоянием между символами, что и в \textit{заголовке sc.n-представления сегмента базы знаний}) и последующих "звездочек"{} до конца \textit{строчки};
	\item либо из одних "звездочек"{} до конца \textit{строчки}.
	\end{scnitemize}}
	\scnaddlevel{-1}
;заголовок sc.n-представления выделенного фрагмента базы знаний
;связка "$\supset$="{}\\
	\scnaddlevel{1}
	\scnidtf{связка, связывающая \textit{заголовок sc.n-представления выделенного фрагмента базы знаний} с самим \textit{sc.n-текстом} этого фрагмента}
	\scnrelfrom{размещение}{с начала новой строчки}
	\scnaddlevel{-1}
;фигурная скобка, открывающая sc.n-текст выделенного фрагмента базы знаний\\
	\scnaddlevel{1}
	\scnrelfrom{размещение}{первый символ новой строчки}
	\scnaddlevel{-1}
;собственно sc.n-текст выделенного фрагмента базы знаний\\
	\scnaddlevel{1}
	\scnnote{Данный текст может занимать большое количество \textit{строчек}}
	\scnaddlevel{-1}
;фигурная скобка, закрывающая sc.n-текст выделенного фрагмента базы знаний\\
	\scnaddlevel{1}
	\scnrelfrom{размещение}{первый символ новой строчки}
	\scnaddlevel{1}
	\scnidtf{первый символ \textit{строчки}, которая следует после последней \textit{строчки sc.n-текста выделенного фрагмента базы знаний}}
	\scnaddlevel{-2}
;дополнительный признак завершения sc.n-представления выделенного фрагмента базы знаний\\
	\scnaddlevel{1}
	\scnexplanation{Данный признак используется только при представлении выделенных фрагментов сегментов баз знаний и выделенных фрагментов неструктурированных разделов баз знаний, которые не декомпозируются на сегменты. При этом рассматриваемый признак изображается как нетранслируемый комментарий, имеющий длину в половину строчки и состоящий из "звездочек"{}. В случае, если в sc.n-представлении сразу за \textit{выделенным фрагментом базы знаний} следует другой \textit{выделенный фрагмент базы знаний}, допускается не указывать упомянутый признак завершения sc.n-представления \textit{выделенного фрагмента базы знаний}.}
	\scnaddlevel{-1}}
\scnnote{Отличия в sc.n-представлении различного вида \textit{выделенных фрагментов баз знаний} могут заключаться в следующем:
	\begin{scnitemize}
	\item размер шрифта и расстояние между символами в заголовке \textit{sc.n-представления выделенного фрагмента базы знаний} могут быть различны;
	\item для выделенных фрагментов сегментов и фрагментов, не содержащих сегменты, заголовок \mbox{sc.n-представления} фрагмента базы знаний и, соответственно, связка "$\supset$="{} \uline{могут отсутствовать}, т.е. такие фрагменты баз знаний могут быть неименованными.
	\end{scnitemize}}

\newpage
\scnheader{заголовок sc.n-представления выделенного фрагмента базы знаний}
\scnsubset{sc-идентификатор выделенного фрагмента базы знаний}
\scnsubset{жирный курсив возможно увеличенного размера и с увеличенным расстоянием между символами}
	\scnaddlevel{1}
	\scniselement{шрифт\scnsupergroupsign}
	\scnaddlevel{-1}
\scnrelfrom{размещение}{с начала новой строчки}
\scnaddlevel{1}
\scnidtf{с первого символа новой строчки}
\scnidtf{от первой табуляционной линии размещения sc.n-текста}
\scnaddlevel{-1}

\scnheader{титульная спецификация \uline{публикации} семейства разделов базы знаний}
\scnnote{Имеется в виду \uline{публикация} (издание) \textit{семейства разделов базы знаний} в виде \textit{sc.n-представления семейства разделов базы знаний}}
\scnnote{В простейшем случае публикуемое \textit{семейство разделов базы знаний} может состоять из одного \textit{раздела}. Например, это может быть публикация \textit{раздела базы знаний} в виде статьи.}

\scnheader{нетранслируемый комментарий к внешнему тексту}
\scnidtf{нетранслируемый комментарий к внешнему тексту отображаемого фрагмента базы знаний ostis-системы}
\scnexplanation{естественно-языковой текст, который ограничен слева наклонной чертой и звездочкой ``/*'' и справа -- звездочкой и наклонной чертой ``*/'' и который может находиться в \uline{любом} месте внешнего текста}
\scnnote{При загрузке и трансляции \uline{исходного} текста \textit{базы знаний ostis-системы} все входящие в него нетранслируемые комментарии игнорируются -- и в том случае, если эти комментарии входят в состав изображения какого-либо файла ostis-системы (в частности, естественно-языкового файла), и в том случае, если эти комментарии входят в состав формального текста, транслируемого в \textit{SC-код}.\\
При трансляции внутреннего текста (sc-текста) базы знаний ostis-системы во внешнее представление (например, в sc.n-текст) некоторые нетранслируемые комментарии могут автоматически генерироваться.\\
Например, нетранслируемые комментарии, предшествующие заголовкам внешнего представления структурно выделяемых фрагментов баз знаний ostis-систем (разделов, сегментов), нетранслируемые комментарии к правым (закрывающим) фигурным и квадратным скобкам, если ограничиваемые этими скобками выражения размещаются на нескольких страницах.

Приведем несколько конкретных примеров:\\
\scnsourcecomment{представление данного файла будет продолжено на следующей странице}\\
\scnsourcecomment{продолжение представления файла}\\
\scnsourcecomment{представление данного sc-текста будет продолжено на следующей странице}\\
\scnsourcecomment{продолжение представления sc-текста}\\
\scnsourcecomment{sc-текст (файл), обозначаемый данным sc-узлом, представлен на следующей странице (на следующих страницах)}

Примеры нетранслируемых комментариев к закрывающим скобкам:\\
\scnsourcecomment{Завершили Раздел ``\nameref{intro_rules}''}\\
\scnsourcecomment{Завершили Сегмент ``\textit{Описание правил оформления внешнего представления знаний ostis-систем}''}.}

\scnnote{Нетранслируемые комментарии к внешнему тексту базы знаний непосредственно в состав базы не входят. Указанными комментариями \uline{не следует злоупотреблять}, они должны касаться оформления внешнего текста, но не смысла представляемой базы знаний.\\
Содержательные комментарии к базе знаний должны оформляться в виде файлов, входящих в состав базы знаний.}

\bigskip

\scnendstruct \scnendsegmentcomment{Описание правил оформления внешнего представления знаний ostis-систем}

\end{SCn}
        \begin{SCn}

\scnheader{раздел базы знаний}
\scnidtf{раздел б.з.}
	\scnaddlevel{1}
	\scniselement{сокращенный sc-идентификатор}
	\scnaddlevel{-1}
\scnexplanation{Множество \textit{разделов баз знаний} имеет:
	\begin{scnitemize}
	\item богатую семантическую типологию;
	\item богатый набор отношений, описывающих семантические связи между разделами.
	\end{scnitemize}
	Синтаксически \textit{разделы баз знаний} могут пересекаться (иметь общие элементы), но никогда один раздел не может включаться (полностью входить в состав) другого раздела. В этом смысле понятие подраздела (точнее, \textit{частного раздела}*) имеет не "синтаксический"{} смысл, а семантический -- глубокое наследование свойств при достаточно большой степени "независимости"{} друг от друга}
\scnnote{Важным свойством \textit{раздела базы знаний} \scnbigspace \textit{ostis-системы} является его семантическая целостность -- наличие достаточно стабильного набора классов исследуемых сущностей (\textit{объектов исследования}) и достаточно стабильного семейства \textit{отношений} и семейства \textit{параметров}, заданных на различных классах объектов исследования, а также семейства \textit{классов структур} использующих указанные выше понятия (введенные \textit{классы объектов исследования}, введенные \textit{отношения} и \textit{классы структур}). Такая целостность дает возможность развивать \textit{раздел базы знаний}, не выходя "за рамки"{} используемой системы \textit{понятий}. Это позволяет развивать каждый \textit{раздел базы знаний} в известной мере \uline{независимо} от других разделов, что существенно повышает \uline{гибкость} и \uline{стратифицированность} \textit{базы знаний}.}	
\scnexplanation{Основными свойствами \textit{раздела базы знаний} как структурно \textit{выделяемого фрагмента базы знаний} являются:
	\begin{scnitemize}
	\item семантическая целостность -- наличие четкого критерия, позволяющего установить для каждого конкретного знания то, включать или не включать это знание в состав данного раздела;
	\item потенциальная возможность эволюционировать в достаточной степени независимо от других разделов, но при условии соблюдения всех требований, обеспечивающих постоянную поддержку \uline{семантической совместимости} данного раздела со всеми остальными семантически смежными разделами.
	\end{scnitemize}}
	\scnaddlevel{1}
	\scntext{следовательно}{Грамотная декомпозиция \textit{базы знаний} на разделы, основанная на четкой стратификации процесса эволюции накапливаемой человечеством общечеловеческой объединенной \textit{базы знаний}, сутью которой является \uline{минимизация} трудоемкости усилий по согласованию и обеспечению с авторами других разделов \textit{семантической совместимости} со смежными разделами, создает предпосылки высоких темпов эволюции \textit{базы знаний} в целом.}
	\scnaddlevel{-1}
\scnnote{Любой \textit{раздел базы знаний} не является структурной частью другого раздела. Каждый раздел самодостаточен и целостен. Это обеспечивается тем, что в состав \textit{титульной спецификации} каждого раздела входит \textit{семантическая окрестность},описывающая связи специфицируемого раздела \uline{со всеми} семантически близкими ему разделами.\\
При этом разные разделы могут иметь разный семантический тип:
	\begin{scnitemize}
	\item раздел может быть предметной областью, интегрированной со всеми ее онтологиями;
	\item раздел может быть просто предметной областью;
	\item раздел может быть какой-либо онтологией чего угодно (не обязательно предметной области);
	\item и т.д.
	\end{scnitemize}}
\scnnote{Если в \textit{титульную спецификацию} каждого раздела будет входить семантическая спецификация каждого раздела, включающая его семантические связи со всеми семантически близкими разделами, то последовательность (порядок) разделов в "линейном"{} исходном тексте, публикуемом в качестве очередной версии Стандарта OSTIS, может быть в достаточной степени \uline{произвольной}.}


\scnheader{семейство разделов базы знаний}
\scnidtf{множество семантически связанных друг с другом разделов базы знаний}
\scnidtf{кластер разделов базы знаний}
\scnnote{Семантические связи между разделами, входящими в состав семейства разделов, представляются в рамках \textit{титульных спецификаций разделов}, каждая из которых является специальной частью соответствующего (специфицируемого) раздела, входящего в состав семейства разделов.

Для каждого \textit{раздела базы знаний} в рамках его титульной спецификации формируется \textit{семантическая окрестность} его связей со всеми семантически близкими ему разделами (и, прежде всего, с теми разделами, которые входят в состав тех семейств разделов, в которые входит заданный раздел). При этом акцентируется внимание именно на семантических связях между разделами. Так, например, вместо структурной ("синтаксической"{}) связи "быть подразделом*"{} (т.е. быть частью заданного раздела) вводится связь "быть \textit{частным разделом}*"{}. \\
Данная связь указывает направление наследования свойств исследуемых объектов заданного раздела от разделов, исследующих более общие классы объектов.
Порядок (последовательность) разделов в рамках \textit{семейства разделов базы знаний} при наличии \uline{явно представленных} семантических связей между разделами, входящими в семейство разделов, может быть достаточно \uline{произвольным}, что очень важно, например, при формировании оглавления очередной издаваемой версии \textit{Стандарта OSTIS}. Таким образом, трактовка \textit{Стандарта OSTIS}, а также всех издаваемых версий этого Стандарта как \textit{семейства разделов базы знаний} \scnbigskip \textit{Метасистемы IMS.ostis} обеспечивает высокий уровень гибкости \textit{Стандарта OSTIS}, а также легкость "переиздаваемости"{} его версий.}


\scnheader{сегмент или атомарный раздел базы знаний}
\scnnote{Простейшей формой \textit{сегмента} или \textit{атомарного раздела базы знаний} является просто последовательность \textit{файлов ostis-системы}. Некоторые из этих файлов могут быть идентифицированными (именованными), если на них ссылаются другие файлы, а некоторые из них могут быть связаны с другими файлами различными отношениями (в частности, один файл может быть пояснением другого). Кроме того, некоторые из этих файлов могут быть формально специфицированы (например, указаны соответствующие им ключевые \textit{sc-элементы}).\\
В самом простом случае \textit{сегмент} или \textit{атомарный раздел базы знаний} может быть \textit{sc-структурой}, состоящей из \uline{одного} (!) \textit{sc-узла}, обозначающего \textit{файл ostis-системы} (чаще всего, \textit{ея-файл ostis-системы}). Т.е. сам \textit{файл ostis-системы} может быть \textit{знанием ostis-системы}, но не может быть структурно \textit{выделяемым} \textit{фрагментом базы знаний} ostis-системы. При этом \textit{sc-узел}, обозначающий \textit{файл ostis-системы}, являющийся \textit{знанием}, может быть единственным \textit{sc-элементом} структурно выделяемого \textit{знания ostis-системы}.}
\scnnote{Для наглядного отображения (визуализации) \textit{сегмента} или \textit{атомарного раздела базы знаний ostis-систем\textit{ы} целесообразно представить указанное \textit{sc-знание} в виде конкатенации (последовательности) таких }sc-знаний, которые, во-первых, были бы достаточно крупными и логико-семантически значимыми для соответствующего \textit{сегмента} или \textit{атомарного раздела базы знаний ostis-системы} и, во-вторых, для которых существовал бы алгоритм \uline{однозначного} (!) размещения (на экране) внешнего представления этих \textit{sc-знаний} (в \textit{SCg-коде} или в \textit{SCn-коде}).\\
Однозначность здесь означает наличие легко усваиваемого пользователями стандартного \uline{стиля визуализации} \textit{sc-знаний} и заключается в том, что многократная визуализация одного и того же \textit{sc-знания} с помощью указанного алгоритма должна приводить к синтаксически эквивалентным, а в случае \textit{SCg-кода} и к геометрически конгруэнтным текстам. Очевидно, что для произвольных \textit{sc-знаний} большого объёма такого алгоритма не существует, но для \textit{sc-знаний}, содержащих описание собственной структуры и семантической типологии собственных фрагментов, разработка такого алгоритма вполне реальна при наличии достаточного количества указанных \textit{метазнаний} о структуре отображаемых (визуализируемых) \textit{sc-знаний}.}


\scnheader{sc-идентификатор выделенного фрагмента базы знаний}
\scnidtf{название (имя) выделенного фрагмента базы знаний}
\scnexplanation{Не следует путать объект описания (спецификации) и само описание. Поэтому в \textit{sc-идентфикаторе} фрагмента базы знаний должны присутствовать слова, указывающие на семантический или структурный тип именуемого фрагмента базы знаний (описание, спецификация, анализ, сравнительный анализ, сравнение, определение, раздел, предметная область, онтология и т.п.).

Таким образом, \textit{sc-идентификатор выделенного фрагмента базы знаний} ostis-системы должен иметь \uline{явное} (!) указание на то, что он является обозначением именно фрагмента базы знаний, а не того, что описывается в этом фрагменте.}
\scnnote{Мы не будем использовать такой изменчивый для нас способ идентификации разделов \textit{Стандарта OSTIS}, как нумерацию этих разделов, поскольку, например, в разных издаваемых официальных версиях \textit{Стандарта OSTIS} одному и тому же разделу \textit{Стандарта OSTIS} могут соответствовать разные номера.}


\scnheader{выделенный фрагмент базы знаний}
\scnrelfrom{основной sc-идентификатор}{\scnfilelong{выделенный фрагмент базы знаний}}
	\scnaddlevel{1}
	\scnrelfrom{используемая аббревиатура}{\scnfilelong{выделенный фр-нт б.з.}}
	\scnaddlevel{-1}
\scnsubdividing{именованный фрагмент базы знаний\\
	\scnaddlevel{1}
	\scnidtf{\textit{выделенный фрагмент базы знаний}, имеющий \textit{sc-идентификатор} (имя, название)}
	\scnnote{\textit{Именованными фрагментами баз знаний} могут быть только структурно \textit{выделенные фрагменты баз знаний}}
	\scnnote{Все \textit{семейства разделов баз знаний}, все \textit{разделы баз знаний} и все \textit{сегменты баз знаний} должны быть именованными}
	\scnsuperset{семейство разделов базы знаний}
	\scnsuperset{раздел базы знаний}
	\scnsuperset{сегмент базы знаний}
	\scnaddlevel{-1}
;неименованный фрагмент базы знаний\\
	\scnaddlevel{1}
	\scnidtf{\textit{выделенный фрагмент базы знаний}, \uline{не} имеющий \textit{sc-идентификатор} (имени, названия)}
	\scnnote{\textit{неименованными фрагментами баз знаний} могут быть только \textit{выделенные фрагменты сегментов баз знаний} либовыделенные фрагменты таких \textit{разделов баз знаний}, которые не состоят из \textit{сегментов}}
	\scnaddlevel{-1}}

\end{SCn}
        \begin{SCn}
	
\scnheader{титульная спецификация выделенного фрагмента базы знаний}
\scnexplanation{\textit{Титульная спецификация выделенного фрагмента базы знаний} ostis-системы представляет собой \textit{sc-структуру}, описывающую свойства специфицируемого знания и включающую в себя: 
	\begin{scnitemize}
		\item связи принадлежности специфицируемого знания соответствующим классам \textit{знаний ostis-систем};
		\item связи, указывающие логически предшествующее и логически следующее \textit{знание ostis-системы};
		\item связь, описывающую декомпозицию специфицируемого знания на последовательность знаний более низкого структурного уровня (декомпозицию разделов на сегменты);
		\item различного вида связи с другими \textit{знаниями ostis-систем}, которые сами "целиком"\ входят в состав спецификации специфицируемого знания (такими знаниями могут быть аннотации, предисловия, введения, оглавления, заключения);
		\item различного вида связи с другими \textit{знаниями ostis-систем}, которые сами не входят в состав спецификации специфицируемого знания (такого рода связями могут быть связи \textit{семантической близости} специфицируемого знания с другими знаниями, связи \textit{семантической эквивалентности}, связи\textit{семантического включения}, связи \textit{противоречивости знаний});
		\item связи, указывающие различного вида \textit{ключевые sc-элементы} (ключевые знаки), соответствующие специфицируемому знанию;
		\item связи специфицируемого знания с авторским коллективом, коллективом рецензентов, с датой его последнего обновления;
		\item для каждого нового целостного фрагмента, вводимого в состав \textit{базы знаний}, в истории эволюции этой \textit{базы знаний} указываются:
		\begin{scnitemizeii}
			\item \textit{автор*} или \textit{авторы*} первой версии этого фрагмента;
			\item отметка времени появления (дата-час-минута) всех версий этого фрагмента (в том числе и окончательно утверждённой, согласованной версии, которая, собственно, и становится фрагментом, включенным в согласованную часть базы знаний);
			\item \textit{рецензии*} (замечания к доработке) всех предварительных версий разрабатываемого \textit{фрагмента базы знаний};
			\item \textit{авторы*} всех указанных рецензий;
			\item отметка времени появления всех указанных рецензий;
			\item события по одобрению, утверждению различных предварительных версий разрабатываемого \textit{фрагмента базы знаний} различными рецензентами и экспертами с указанием отметки времени появления этих событий;
			\item темпоральная последовательность предварительных версий.
		\end{scnitemizeii}
	\end{scnitemize}
}

\scnheader{титульная спецификация выделенного фрагмента базы знаний}
\scnnote{Знак такой спецификации явно не вводится, а сама эта спецификация непосредственно входит в состав специфицируемого фрагмента и включает в себя аннотацию, предисловие, авторов, ключевые знаки, декомпозицию специфицируемого фрагмента базы знаний и прочее}

\scnsubdividing{титульная спецификация раздела базы знаний;
титульная спецификация семейства разделов базы знаний;
титульная спецификация сегмента базы знаний;
титульная спецификация выделенного фрашмента сегмента или атомарного раздела базы знаний}

\scnheader{титульная спецификация выделенного фрагмента базы знаний}
\scnexplanation{\textit{Титульная спецификация выделенного фрагмента базы знаний} содержит общую информацию об этом фрагменте, является непосредственно \uline{частью} специфицируемого фрагмента \textit{базы знаний} и при этом сама \uline{не является} явно \textit{выделенным фрагментом базы знаний}}
\scnhaselement{спецификация}
\scnidtf{основная \textit{метаинформация} (основное \textit{метазнание}) о \textit{выделенном фрагменте базы знаний} -- о его структуре, \textit{авторах\scnrolesign}, \textit{ключевых знаках\scnrolesign} и т.д.}

\scnheader{титульная спецификация выделенного фрагмента базы знаний}
\scnexplanation{\uline{неявно} \textit{выделяемый фрагмент базы знаний}, который:
\begin{scnitemize}
	\item не имеет "собственного" ограничителя ("собственного" контура или "собственных" ограничивающих фигурных скобок);
	\item является семантической спецификацией соответствующего \uline{явно} выделяемого фрагмента базы знаний;
	\item является непосредственной \uline{частью} специфицируемого фрагмента базы знаний
\end{scnitemize}}
\scnnote{В \textit{sc.n-тексте} титульная спецификация фрагмента базы знаний размещается сразу после фигурной скобки, открывающей этот фрагмент}

\scnheader{титульная спецификация раздела базы знаний}
\scnnote{\textit{тиутульная спецификация раздела базы знаний} должна включать в себя достаточно подробное описание семантических свойств этого раздела и, в частности, подробное описание его связей с другими семантически близкими разделами. Это необходимо для обеспечения автономности разделов баз знаний.}

\scnheader{титульная спецификация семейства разделов базы знаний}
\scnnote{Если \textit{разделы базы знаний} являются семантически \uline{ключевыми} \textit{выделенными фрагментами баз знаний}, определяющими спецификацию систем используемых понятий и направления наследования свойств, то \textit{семейства разделов баз знаний} являются \uline{ключевыми} для структуризации виртуальной \textit{Базы знаний Экосистемы OSTIS}, для обмена \textit{знаниями} между различными субъектами \textit{Экосистемы OSTIS}.\\
Поэтому типология \textit{семейств разделов баз знаний} и качество \textit{титульной спецификации семейств разделов баз знаний} имеют большое значение.}

\scnheader{титульная спецификация выделенного фрагмента базы знаний}
\scnrelfrom{множество используемых понятий}{Множество понятий используемых в титльных спецификациях выделенных фрагментов баз знаний}
\scnaddlevel{1}
\scnsuperset{класс выделенных фрагментов}
\scnaddlevel{1}
\scnhaselement{семейство разделов базы знаний}
\scnhaselement{раздел базы знаний}
\scnhaselement{неатомарный раздел базы знаний}
\scnhaselement{атомарный раздел базы знаний}
\scnhaselement{сегмент базы знаний}
\scnhaselement{выделенный фрагмент сегмента или атомарного раздела базы знаний}
\scnhaselement{выделенный фрагмент атомарного раздела базы знаний}
\scnhaselement{выделенный фрагмент сегманта базы знаний}
\scnaddlevel{-1}
\scnsuperset{отношение, связывающее выделенные фрагменты баз знаний с персонами}
\scnaddlevel{1}
\scnhaselement{автор*}
\scnhaselement{рецензент*}
\scnhaselement{эксперт*}
\scnhaselement{технический редактор*}
\scnhaselement{консультант*}
\scnaddlevel{1}
\scnidtf{активный участник обсуждения вопросов, рассматриваемых в специфицируемом фрагменте базы знаний*}
\scnaddlevel{-1}
\scnsuperset{отношение, связывающее выделенные фрагменты баз знаний с ея-файлами}
\scnhaselement{аннотация*}
\scnhaselement{предисловие*}
\scnaddlevel{1}
\scnidtf{Бинарное ориентированное отношение, каждая пара которого связывает:
\begin{scnitemize}
	\item знак некоторого информационного ресурса (в частности, раздела базы знаний или раздела опубликованного документа)
	\item знак информационной конструкции, описывающей цели создания указанного информационного ресурса, предысторию его создания, планируемые направления дальнейшего развития, состав авторов и др.
\end{scnitemize}}
\scnaddlevel{-1}
\scnhaselement{введение*}
\scnhaselement{эпиграф*}
\scnhaselement{заключение*}
\scnhaselement{рассматриваемый вопрос*}
\scnhaselement{основные положения*}
\scnhaselement{вопрос для самопроверки*}
\scnhaselement{упражнение*}
\scnaddlevel{1}
\scnidtf{задача*}
\scnidtf{самостоятельная (индивидуальная) работа*}
\scnaddlevel{-1}
\scnhaselement{коллективный проект}

\scnhaselement{неосновной sc-идентификатор*}
\scnaddlevel{1}
\scnnote{неосновным sc-идентификатором, в частности, может быть альтернативное название выделенного (специфицируемого) фрагмента базы знаний}
\scnaddlevel{-1}
\scnhaselement{часто используемый sc-идентификатор*}
\scnhaselement{сокращенный sc-идентификатор}
\scnhaselement{используемое сокращение*}
\scnaddlevel{1}
\scnidtf{сокращение, используемое в специфицируемом фрагменте базы знаний при построении sc-идентификаторов, а также при оформлении ея-файлов*}
\scnaddlevel{-1}
\scnhaselement{библиографический источник, отражающий аналогичную точку зрения*}
\scnhaselement{библиографический источник, отражающий альтернативную точку зрения*}
\scnhaselement{библиографический источник, дополняющий данную точку зрения*}
\scnaddlevel{-1}
\scnhaselement{сокращение*}
\scnaddlevel{1}
\scnidtf{Бинарное ориентированное отношение, каждая пара которого связывает естественно-языковую фразу с ее сокращенной записью*}
\scnaddlevel{-1}
\scnsuperset{отношение, описывающее структурные или семантические связи и между выделенными фрагментами баз знаний}
\scnaddlevel{1}
\scnhaselement{конкатенация сегментов*}
\scnhaselement{предыдущий сегмент*}
\scnaddlevel{1}
\scnidtf{предыдущий сегмент в рамках соответствующего раздела*}
\scnaddlevel{-1}
\scnhaselement{следующий сегмент*}

\scnhaselement{частный фрагмент базы знаний*}
\scnaddlevel{1}
\scnsuperset{частный раздел базы знаний*}
\scnsuperset{частная предметная область*}
\scnsuperset{частная предметная область и онтология*}
\scnnote{Для фрагмента базы знаний важно указать не только частные по отношению к нему фрагменту базы знаний, но и те фрагменты базы знаний, по отношению к которым данный фрагмент базы знаний является частным}

\scnsuperset{отношение, описывающее ролевой статус знаков, входящих в состав выделенных фрагментов баз знаний}
\scnaddlevel{1}
\scnhaselement{ключевой знак*}
\scnhaselement{ключевой знак первого плана*}
\scnhaselement{ключевой знак второго плана*}
\scnhaselement{ключевой объект исследования*}
\scnhaselement{ключевое понятие*}
\scnhaselement{ключевой класс объектов исследования*}
\scnhaselement{исследуемое отношение*}
\scnhaselement{исследуемый параметр*}
\scnhaselement{исследуемый класс структур*}
\scnaddlevel{-4}

\scnendstruct \scninlinesourcecommentpar{Завершили Сегмент ``Структуризация баз знаний ostis-систем''}


\end{SCn}


        \begin{SCn}

\scnsegmentheader{Описание правил оформления внешнего представления знаний ostis-систем}

\scnstartsubstruct

\scnheader{внешнее представление знаний ostis-системы}
\scnexplanation{внешнее представление некоторого фрагмента базы знаний ostis-системы, используемое для ввода новой информации в состав базы знаний ostis-системы или для вывода (отображения) запрашиваемого фрагмента базы знаний}
\scnnote{Способ представления исходных текстов баз знаний ostis-систем должен быть максимально возможным образом использован и для вывода (отображения) запрашиваемых пользователем фрагментов баз знаний, особенно, если запрашиваются достаточно большие фрагменты баз знаний, которые необходимо не только представлять, но и структурировать унифицированным образом. Очевидно, что для пользователей желательно, чтобы и для ввода информации в ostis-систему, и для ее вывода использовались одни и те же языковые средства и правила оформления}
\scnnote{Требования, предъявляемые к оформлению внешних текстов знаний ostis-систем (sc-знаний) носят достаточно противоречивый характер -- с одной стороны, речь идет о формальных текстах, легко воспринимаемых (понимаемых, транслируемых) ostis-системами, а, с другой стороны, желательно, чтобы эти же формальные тексты легко воспринимались (понимались) широким кругом людей и не требовали для этого от них длительной подготовки. Отметим при этом, что работа с формальными текстами требует от человека достаточно высокой культуры \uline{точного} мышления (математической культуры).

Отметим также, что использование формальных языков является важнейшим и необходимым этапом эволюции человеческой деятельности в любой области (в математике, в физике, в технике).

Тем не менее, проблема создания универсального языка представления исходных текстов различного вида знаний, который был бы достаточно удобен как для интеллектуальных компьютерных систем, так и для \uline{широкого} круга разработчиков баз знаний и экспертов, требует конкретного решения.}

\filemodetrue
\scnreltovector{требования}{Стиль и характер оформления внешнего представления sc-знаний должен обеспечить возможность интуитивного понимания смысла текста при отсутствии понимания различного рода синтаксических деталей. Для этого:
\begin{scnitemize}
\item формальный текст должен максимально возможным образом использовать привычную для широкого круга специалистов терминологию\char59
\item структуризация, форматирование формальных текстов также должны опираться на сформировавшиеся традиции\char59
\item внешнее представление (внешний текст) sc-знания должен включать в себя такое количество отображаемых ея-файлов, прочтения которых было бы достаточно для понимания смысла представляемого sc-знания, а также для понимания формальных средств его представления
\end{scnitemize};
Формальный текст (как внутреннего, так и внешнего представления sc-знаний) должен включать в себя средства для уточнения смысла используемых знаков и соответствующих им терминов, а также смысла некоторых фрагментов формального текста. Для этого в формальный язык вводятся естественно-языковые файлы, отображаемые в исходных текстах и поясняющие используемые термины, а также комментирующие или даже полностью переводящие на естественный язык различные фрагменты формального представления \textit{базы знаний}.;
Все используемые в базе знаний ostis-системы (в том числе, и в её ея-файлах) внешние идентификаторы \textit{sc-элементов} (термины, имена, условные обозначения) должны быть формально специфицированы средствами \textit{SC-кода}. Подчеркнем, что здесь речь идет о спецификации не самих sc-элементов, а их внешних идентификаторов (в первую очередь, простых sc.s-идентификаторов) -- их происхождение, использование, авторство и т.д.;
Все отношения, параметры и другие понятия, используемые в формальных текстах должны быть пояснены в соответствующих формальных онтологиях. Первое упоминание во внешнем тексте каждого такого понятия должно быть кратко пояснено с помощью поясняющего ея-файла, а также сделана ссылка на раздел базы знаний, в которых приведена подробная и формальная спецификация указанного понятия с дополнительным указанием номера этого раздела с помощью нетранслируемого комментария.;
Аналогичным образом в отображаемом внешнем представлении sc-знания поясняются и комментируются все \uline{первые} (в рамках этого внешнего представления) использования средств формального представления знаний со ссылками на разделы базы знаний, где указанные языковые средства подробно описываются. С самого начала внешнего представления большого структурированного фрагмента базы знаний (каковым, в частности, является Стандарт OSTIS) с помощью нетранслируемых комментариев, не входящих в состав базы знаний, либо с помощью ея-файлов ostis-систем необходимо пояснять все нюансы формализации со ссылкой на ближайший раздел и сегмент, где это будет подробнее рассмотрено.;
При описании формальных средств должны быть приведены конкретные \uline{примеры} со ссылкой на раздел или сегмент, где этот пример будет рассмотрен подробнее (например, на соответствующую предметную область и онтологию);
Все комментарии и примечания, которые можно представить средствами \textit{SCn-кода} или \textit{SCg-кода}, нужно оформлять именно так. Нетранслируемыми комментариями  \uline{не стоит увлекаться}.;
В формальных текстах и в естественно-языковых файлах, входящих в состав \textit{базы знаний ostis-системы}, для идентификации (именования) \textit{sc-элементов} должны использоваться только те термины, которые являются \uline{основными}(!) внешними идентификаторами соответствующих \textit{sc-элементов}, выделяемыми жирным и нежирным курсивом. При этом, если идентификаторы (названия, имена) разделов, сегментов базы знаний находятся в позиции \uline{заголовков} указанных фрагментов базы знаний, то они оформляются жирным курсивом с увеличенным расстоянием между символами, а заголовки разделов дополнительно выделяются увеличенным размером символов.;
В согласованной (общепризнанной) части \textit{базы знаний} противоречия трактуются как выявленные ошибки в \textit{базе знаний}, подлежащие устранению. Но в истории эволюции
базы знаний противоречия могут присутствовать как противоречия разных точек зрения разных авторов. Заметим при этом, что разные точки зрения далеко не всегда являются противоречивыми (взаимоисключающими). Они могут просто дополнять друг друга, описывать исследуемые сущности с разных "ракурсов". Умение видеть противоречия только там, где, они действительно есть, и умение локализовать эти противоречия (выделить их суть) -- это необходимые навыки для разработки \uline{практически полезных} \textit{баз знаний}.}
\filemodefalse

\scnheader{sc.n-представление знаний ostis-системы}
\filemodetrue
\scnrelfromvector{правила оформления}{В исходных текстах баз знаний \textit{интеллектуальных компьютерный систем}, построенных по \textit{Технологии OSTIS} (т.е. ostis-систем) \scnsourcecomment{в том числе, в тексте данной монографии} используются следующие языки:
	\begin{scnitemize}
	\item Различные \textit{естественные языки} (\textit{Русский язык}, \textit{Английский язык} и др.)
	\item \textit{SCg-код} (Semantic Code graphical), являющийся \textit{универсальным формальным языком} графического представления \textit{семантических сетей}
	\item \textit{SCs-код} (Semantic Code string), являющийся \textit{универсальным формальным языком} линейного представления \textit{семантических сетей} в виде \textit{строк} (цепочек символов)
	\item \textit{SCn-код} (Semantic Code Natural) являющийся \textit{универсальным формальным языком} представления \textit{семантических сетей} в виде структурированных форматированных \textit{текстов} (строк, размещенных на плоскости.
	\end{scnitemize}	
\textit{Синтаксис} и \textit{денотационная семантика} указанных \textit{формальных языков подробно рассмотрены в Разделе ***}
;Основным языком внешнего представления \textit{баз знаний ostis-систем} является \textit{SCn-код}, рассмотренный в разделе  \textit{Введение в язык структурированного представления баз знаний ostis-систем}. Но в состав текста \textit{SCn-кода} могут входить тексты и других языков (тексты \textit{SCg-кода}, тексты \textit{SCs-кода}, тексты естественных языков, тексты различных искусственнных языков), а также различного рода нетекстовые информационные конструкции (рисунки, таблицы, чертежи, графики, фотографии). Указанные "инородные"{} для \textit{SCn-кода} информационные конструкции, а также описываемые тексты самого \textit{SCn-кода} оформляются во внешнем представлении базы знаний либо как нетранслируемые, но специфицируемые файлы, либо как транслируемые инородные для \textit{SCn-кода} информационные конструкции, ограниченные, соответственно, либо \textit{sc.n-рамками} (квадратными скобками), либо \textit{sc.n-контурами} (фигурными скобками).
;Структуризация внешнего представления баз знаний ostis-систем является полным отражением структуризации внутреннего представления \textit{баз знаний ostis-систем}.
;В случае, если осуществляется внешнее представление \uline{полного} текста указываемого сложноструктурированного фрагмента базы знаний, как, например, внешнее представление семейства разделов под названием \textit{``Документация Стандарт OSTIS''} вместе со всеми его разделами, то последовательность и иерархическая структура отображения разделов и сегментов указанного сложноструктурированного фрагмента базы знаний в точности соответствует иерархической структуре представляемого (отображаемого, визуализируемого) семейства разделов базы знаний.
;База знаний каждой \textit{ostis-системы} представляет собой иерархическую систему разделов, к которым должны "привязываться"{} исходные тексты каждой новой информации, вводимой в \textit{базу знаний}, и, прежде всего, исходные тексты достаточно крупных фрагментов \textit{баз знаний}. \scnsourcecomment{К таким исходным текстам, в частности, относится и данная монография.} Очевидно при этом, что нумерация разделов баз знаний не может быть стабильна. Кроме того, при оформлении исходного текста крупного фрагмента базы знаний, обладающего достаточной целостностью и по научно-технической значимости достигшего уровня монографии или диссертации, желательно иметь собственную (свою, локальную) нумерацию разделов при сохранении их иерархической структуры. Это означает, что номера разделов имеет смысл использовать только в рамках каждого  исходного вводимого текста и не должны использоваться в самой базе знаний. Таким образом, ссылаться на разделы базы знаний следует по \uline{названию} разделов.
;Внешнее представление каждого структурно \textit{выделяемого фрагмента базы знаний} ostis-системы за исключением sc-знаний нижнего уровня начинается с \uline{заголовка} (имени, названия) этого фрагмента. Указанный заголовок есть не что иное, как простой sc-идентификатор sc-узла, обозначающего представляемый фрагмент базы знаний (представляемое sc-знание). Рассматриваемый заголовок оформляется жирным курсивом с \uline{увеличенным расстоянием между символами}. При этом заголовки \uline{внешнего представления} разделов базы знаний имеют дополнительно \uline{увеличенный размер шрифта}. Заголовок внешнего представления sc-знания размещается с первого символа строчки.
После заголовка представляемого фрагмента базы знаний \uline{с новой строчки} размещается (1) sc.s-коннектор вида ``$\supset$='' (2) следом за ним на следующей строчке левая фигурная скобка (открывающая фигурная скобка).
;Внешнее представление каждого раздела базы знаний начинается \uline{с новой страницы}. Соответственно, сегмент базы знаний может начинаться \uline{не} с новой страницы.\\
;Внешнее представление каждого неструктурируемого выделяемого фрагмента базы знаний (атомарного раздела базы знаний, сегмента базы знаний) оформляется в виде \textit{sc.n-предложения}, связывающего sc-идентификатор (имя, название) представляемого фрагмента базы знаний, оформленный в виде \uline{заголовка} внешнего текста этого фрагмента (признаком чего является жирный курсив с увеличенным расстоянием между символами), с \textit{sc.n-контуром}, который фигурными скобками ограничивает sc.n-изображение (sc.n-визуализацию) представляемого фрагмента базы знаний.\\
В самом простом случае представляемым неструктурируемым выделяемым фрагментом базы знаний является \uline{один} ея-файл ostis-системы.
;Внешний текст \textit{базы знаний} может иметь самый разный объем и может касаться только \uline{одного раздела} или сегмента, а может включать в себя материалы \uline{нескольких разделов} или сегментов. Если представляемый (отображаемый) фрагмент базы знаний является sc-знанием нижнего уровня иерархии, т.е. частью \uline{неструктурированного} фрагмента базы знаний (например, частью сегмента базы знаний), то при его представлении необходимо указать, частью какого неструктурированного фрагмента базы знаний представляемый фрагмент является.\\
Для исходного текста sc-знания нижнего уровня эта информация необходима для того, чтобы знать, в какой фрагмент базы знаний требуется включить, "погрузить"{} данное вводимое \textit{sc-знание}.
;Виды выделений во внешнем представлении знаний ostis-системы:
\begin{scnitemize}
    \item с помощью шрифта
    \begin{scnitemizeii}
        \item вид шрифта (печатный, курсив)
        \item размер шрифта (стандартный, увеличенный)
        \item расстояние между символами (стандартное, увеличенное)
    \end{scnitemizeii}
    \item подчеркиванием
    \item с помощью символьных ограничителей (скобок различного вида)
\end{scnitemize}
;Выделяемые объекты:
\begin{scnitemize}
    \item основные термины (имена, идентификаторы)
    \item специфицируемые файлы
    \begin{scnitemizeii}
        \item нетранслируемые в базу знаний файлы
        \item транслируемые в базу знаний файлы
    \end{scnitemizeii}
    \item нетранслируемые комментарии к внешнему тексту
    \item цитаты (и короткие, и длинные)
    \item ключевые фрагменты ея-текста
    \item метафорические термины
\end{scnitemize}
;На любой странице внешнего текста при распечатке делается разметка тонкими вертикальными табуляционными линиями для четкой визуализации длины отступа от левого края страницы. Особенно это важно при переходе на новую страницу.} \scnsourcecomment{Завершили перечень правил оформления внешнего представления знаний ostis-системы} 
\filemodefalse

\scnheader{sc.n-представление выделенного фрагмента базы знаний}
\scnrelfromvector{обобщенная конкатенация}{признак начала sc.n-представления выделенного фрагмента базы знаний\\
	\scnaddlevel{1}
	\scnexplanation{Данным признаком является \textit{нетранслируемый комментарий}, размещенный по всей длине \textit{строчки} и состоящий
	\begin{scnitemize}
	\item либо из слова "Раздел"{} (изображенного жирным печатным шрифтом такого же размера и с таким же расстоянием между символами, что и в \textit{заголовке sc.n-представления раздела базы знаний}) и последующих "звездочек"{} до конца \textit{строчки};
	\item либо из слова "Сегмент"{} (изображенного жирным печатным шрифтом такого же размера и с таким же расстоянием между символами, что и в \textit{заголовке sc.n-представления сегмента базы знаний}) и последующих "звездочек"{} до конца \textit{строчки};
	\item либо из одних "звездочек"{} до конца \textit{строчки}.
	\end{scnitemize}}
	\scnaddlevel{-1}
;заголовок sc.n-представления выделенного фрагмента базы знаний
;связка "$\supset$="{}\\
	\scnaddlevel{1}
	\scnidtf{связка, связывающая \textit{заголовок sc.n-представления выделенного фрагмента базы знаний} с самим \textit{sc.n-текстом} этого фрагмента}
	\scnrelfrom{размещение}{с начала новой строчки}
	\scnaddlevel{-1}
;фигурная скобка, открывающая sc.n-текст выделенного фрагмента базы знаний\\
	\scnaddlevel{1}
	\scnrelfrom{размещение}{первый символ новой строчки}
	\scnaddlevel{-1}
;собственно sc.n-текст выделенного фрагмента базы знаний\\
	\scnaddlevel{1}
	\scnnote{Данный текст может занимать большое количество \textit{строчек}}
	\scnaddlevel{-1}
;фигурная скобка, закрывающая sc.n-текст выделенного фрагмента базы знаний\\
	\scnaddlevel{1}
	\scnrelfrom{размещение}{первый символ новой строчки}
	\scnidtf{первый символ \textit{строчки}, которая следует после последней \textit{строчки sc.n-текста выделенного фрагмента базы знаний}}
	\scnaddlevel{-1}
;дополнительный признак завершения sc.n-представления выделенного фрагмента базы знаний\\
	\scnaddlevel{1}
	\scnexplanation{Данный признак используется только при представлении выделенных фрагментов сегментов баз знаний и выделенных фрагментов неструктурированных разделов баз знаний, которые не декомпозируются на сегменты. При этом рассматриваемый признак изображается как нетранслируемый комментарий, имеющий длину в половину строчки и состоящий из "звездочек"{}.}
	\scnaddlevel{-1}}
\scnnote{Отличия в sc.n-представлении различного вида \textit{выделенных фрагментов баз знаний} заключается в следующем:
	\begin{scnitemize}
	\item размер шрифта и расстояние между символами в заголовке \textit{sc.n-представления выделенного фрагмента базы знаний} различны:
	\begin{scnitemizeii}
	\item самый большой -- у разделов;
	\item поменьше -- у сегментов;
	\item самый маленький (чуть больше стандартного) -- у выделенных фрагментов сегментов и фрагментов разделов, не содержащих сегменты;
	\end{scnitemizeii}
	\item для выделенных фрагментов сегментов и фрагментов, не содержащих сегменты, заголовок sc.n-представления фрагмента базы знаний и, соответственно, связка "$\supset$="{} \uline{могут отсутствовать}, т.е. такие фрагменты баз знаний могут быть неименованными.
	\end{scnitemize}}


\scnheader{заголовок sc.n-представления выделенного фрагмента базы знаний}
\scnsubset{sc-идентификатор выделенного фрагмента базы знаний}
\scnsubset{жирный курсив возможно увеличенного размера и с увеличенным расстоянием между символами}
	\scnaddlevel{1}
	\scniselement{шрифт\scnsupergroupsign}
	\scnaddlevel{-1}
\scnrelfrom{размещение}{с начала новой строчки}
\scnidtf{с первого символа новой строчки}
\scnidtf{от первой табуляционной линии размещения sc.n-текста}


\scnheader{титульная спецификация \uline{публикации} семейства разделов базы знаний}
\scnnote{Имеется в виду \uline{публикация} (издание) \textit{семейства разделов базы знаний} в виде \textit{sc.n-представления семейства разделов базы знаний}}
\scnnote{В простейшем случае публикуемое \textit{семейство разделов базы знаний} может состоять из одного \textit{раздела}. Например, это может быть публикация \textit{раздела базы знаний} в виде статьи}

\scnheader{нетранслируемый комментарий к внешнему тексту}
\scnidtf{нетранслируемый комментарий к внешнему тексту отображаемого фрагмента базы знаний ostis-системы}
\scnexplanation{естественно-языковой текст, который ограничен слева наклонной чертой и звездочкой ``/*'' и справа -- звездочкой и наклонной чертой ``*/'' и который может находиться в \uline{любом} месте внешнего текста}
\scnnote{При загрузке и трансляции \uline{исходного} текста \textit{базы знаний ostis-системы} все входящие в него нетранслируемые комментарии игнорируются -- и в том случае, если эти комментарии входят в состав изображения какого-либо файла ostis-системы (в частности, естественно-языкового файла), и в том случае, если эти комментарии входят в состав формального текста, транслируемого в \textit{SC-код}.\\
При трансляции внутреннего текста (sc-текста) базы знаний ostis-системы во внешнее представление (например, в sc.n-текст) некоторые нетранслируемые комментарии могут автоматически генерироваться.\\
Например, нетранслируемые комментарии, предшествующие заголовкам внешнего представления структурно выделяемых фрагментов баз знаний ostis-систем (разделов, сегментов, начал неатомарных разделов, завершений неатомарных фрагментов), нетранслируемые комментарии к правым (закрывающим) фигурным и квадратным скобкам, если ограничиваемые этими скобками выражения размещаются на нескольких страницах, нетранслируемые комментарии к sc.s-идентификаторам, являющиеся ссылками на номера разделов и сегментов, где подробно специфицируется сущность, именуемая комментируемым sc-идентификатором.\\
Приведем несколько конкретных примеров:\\
\scnsourcecomment{представление данного файла будет продолжено на следующей странице}\\
\scnsourcecomment{продолжение представления файла}\\
\scnsourcecomment{представление данного sc-текста будет продолжено на следующей странице}\\
\scnsourcecomment{продолжение представления sc-текста}\\
\scnsourcecomment{sc-текст (файл), обозначаемый данным sc-узлом, представлен на следующей странице (на следующих страницах)}\\
Примерами нетранслируемых комментариев к закрывающим скобкам:
\scnsourcecomment{Завершили раздел i.j.k}\\
\scnsourcecomment{Завершили начальный сегмент раздела i.j.k}\\
\scnsourcecomment{Завершили примечание к i.j.k}}

\scnheader{комментарий к внешнему тексту базы знаний}
\scnnote{Нетранслируемые комментарии к внешнему тексту базы знаний непосредственно в состав базы не входят. Указанными комментариями \uline{не следует злоупотреблять}, они должны касаться оформления внешнего текста, но не смысла представляемой базы знаний.\\
Содержательные комментарии к базе знаний должны оформляться в виде файлов, входящих в состав базы знаний.}

\scnendstruct 
\end{SCn}
		\scnsegmentheader{Смысловое пространство ostis-систем}
\begin{scnsubstruct}

    \begin{scnrelfromlist}{ключевое понятие}
		\scnitem{обобщенная sc-связка}
		\scnitem{обобщенное sc-отношение}
		\scnitem{бинарное sc-отношение}
		\scnitem{слотовое sc-отношение}
		\scnitem{sc-структура*}
		\scnitem{элементарно представленный элемент\scnrolesign}
		\scnitem{полносвязно представленный элемент\scnrolesign}
		\scnitem{полностью представленный элемент\scnrolesign}
		\scnitem{sc-связка\scnrolesign}
		\scnitem{sc-отношение\scnrolesign}
		\scnitem{sc-класс\scnrolesign}
		\scnitem{сущностное замыкание*}
		\scnitem{содержательное замыкание*}
		\scnitem{sc-отношение сходства по слотовым отношениям*}
		\scnitem{sc-отношение семантического сходства по слотовым отношениям*}
		\scnitem{связная sc-структура*}
		\scnitem{семантическое сходство sc-структур*}
		\scnitem{семантическое непрерывное сходство sc-структур*}
		\scnitem{ключевой запрос\scnrolesign}
		\scnitem{минимальный ключевой запрос\scnrolesign}
		\scnitem{полная семантическая окрестность элемента*}
		\scnitem{интроспективный ключевой элемент\scnrolesign}
		\scnitem{топологическое пространство}
		\scnitem{топологическое пространство замыкания инцидентности коннекторов}
		\scnitem{топологическое пространство синтаксического замыкания}
		\scnitem{топологическое пространство сущностного замыкания}
		\scnitem{топологическое пространство содержательного замыкания}
		\scnitem{метрика}
		\scnitem{семантическая метрика}
		\scnitem{метрическое пространство}
		\scnitem{метрическое конечное синтаксическое пространство}
		\scnitem{метрическое конечное семантическое пространство}
		\scnitem{псевдометрика}
		\scnitem{псевдометрическое пространство}
		\scnitem{псевдометрическое конечное семантическое пространство}
	\end{scnrelfromlist}

\begin{scnrelfromvector}{примечание}
	\scnfileitem{Важнейшим достоинством \textbf{\textit{SC-пространства}} является возможность уточнения таких понятий, как понятие аналогичности (сходства и отличия) различных описываемых \scnqq{внешних} сущностей, аналогичности унифицированных \textit{семантических сетей} (текстов \textbf{\textit{SC-кода}}), понятие семантической близости описываемых сущностей (в том числе, и текстов \textbf{\textit{SC-кода}}).}
	\scnfileitem{Следует отметить, что в силу абстрактности языков модели \textit{унифицированного семантического представления знаний} и условности выбора меток элементов их текстов, нельзя исключить, что объединение двух произвольных текстов таких языков не будет текстом языка модели \textit{унифицированного семантического представления знаний}. Чтобы избежать результатов подобных эклектических объединений с точки зрения синтаксиса или семантики, для абстрактных языков следует рассматривать множество \scnqqi{смысловых пространств}. Однако, для конкретных (реальных) языков может оказаться достаточным рассмотрение одного \scnqqi{смыслового пространства}.}
    \scnfileitem{Далее рассмотрим:
    \begin{scnitemize}
        \item возможность перехода от sc-текстов к графовым структурам и от них к топологическому пространству;
        \item возможность перехода от sc-текстов к графовым структурам и от них к многообразию (топологическому пространству);
        \item возможность перехода от sc-текстов к графовым структурам и от них к метрическому пространству.
    \end{scnitemize}}
    \scnfileitem{Чтобы исследовать структурные свойства \textbf{\textit{SC-пространства}}, можно использовать уже разработанные модели пространств и связь их известными топологическими моделями, например, такими как \textit{графы}. При этом изначально не будем принимать в расчет динамические особенности, связанные с обработкой знаний, однако позже будет показано, что учет динамики в процессах обработки и при становлении знаний является необходимым для вычисления семантической метрики, являющейся одним из определяющих признаков знаний.
    Обратимся к исследованию структурно-топологических свойств пространства.}
    \scnfileitem{Структурно-топологические свойства могут свидетельствовать о синтаксических или семантических зависимостях обозначений текстов языка, позволяющих упростить работу со структурами за счет перехода к более простым структурам на уровнях управления данными или знаниями в характерных задачах управления для \textit{библиотеки многократно используемых компонентов ostis-систем}.}
    \begin{scnindent}
    	\scnrelfrom{смотрите}{Комплексная библиотека многократно используемых семантически совместимых компонентов ostis-систем}
    \end{scnindent}
    \scnfileitem{На множестве элементов, образующих \textbf{\textit{SC-пространство}}, можно изучать топологические свойства и рассматривать \textbf{\textit{SC-пространство}} как топологическое пространство. Следует заметить, что, несмотря на то, что \textbf{\textit{SC-код}} ориентирован на смысловое представление знаний, в силу наличия \textit{не-факторов}, не все смыслы могут быть представлены в некоторый момент времени и не будет известна структура соответствующего представления. Поэтому структурно-топологические свойства текстов \textit{языка представления знаний} скорее определяют синтаксическое пространство, нежели семантическое (смысловое). Хотя оба могут приближаться друг к другу по мере устранения неопределенностей, вызванных \textit{не-факторами}.}
    \scnfileitem{Рассмотрим следующие виды \textit{топологических пространств}:
    \begin{scnitemize}
        \item \textit{топологическое пространство замыкания инцидентности коннекторов};
        \item \textit{топологическое пространство синтаксического замыкания};
        \item \textit{топологическое пространство сущностного замыкания};
        \item \textit{топологическое пространство содержательного замыкания}.
    \end{scnitemize}}
\end{scnrelfromvector}
	
	\scnheader{топологическое пространство}
	\scntext{пояснение}{\textit{топологическое пространство} --- \textit{множество} с определенным над ним \textit{множеством} (семейством) (открытых) подмножеств, включая само \textit{множество} и \textit{пустое множество}. Для любого \textit{подмножества} семейства результат объединения принадлежит \textit{семейству множеств}, а для любого конечного \textit{подмножества семейства} результат пересечения также принадлежит \textit{семейству множеств}. Дополнения множеств семейства до наибольшего из множеств называются \textit{замкнутыми множествами}.}

	\scnheader{обобщенная sc-связка}
	\scnidtf{непустое sc-множество}

	\scnheader{обобщенное sc-отношение}
	\scnidtf{sc-множество непустых sc-множеств}
	\scnexplanation{обобщенное sc-отношение --- sc-множество обобщенных sc-связок.}

	\scnheader{бинарное sc-отношение}
	\scnexplanation{Бинарное sc-отношение --- sc-множество sc-пар (или обобщенных sc-связок, которым существуют две различные принадлежности sc-элементов или одного и того же sc-элемента).}

	\scnheader{узловая sc-пара}
	\scnexplanation{узловая sc-пара --- sc-пара, которая не может быть обозначена sc-дугой принадлежности (позитивной, негативной или нечеткой).}

    \scnheader{явление принадлежности}
    \scnexplanation{явление принадлежности --- множество явлений, каждое из которых является слотовым sc-отношением, которому постоянно непринадлежат sc-дуги постоянной непринадлежности.}

    \scnheader{становление*}
    \scnexplanation{становление* --- бинарное sc-отношение между событиями (состояниями) или явлениями.}

	\scnheader{непосредственно прежде\scnrolesign}
	\scnrelfrom{первый домен}{становление*}
	\scnrelfrom{второй домен}{установленное событие или явление}

	\scnheader{непосредственно после\scnrolesign}
	\scnrelfrom{первый домен}{становление*}
	\scnrelfrom{второй домен}{устанавливающее событие или явление}

    \scnheader{продолжительность*}
    \scnexplanation{продолжительность* --- транзитивное замыкание sc-отношения становления.}

	\scnheader{раньше\scnrolesign}
	\scnrelfrom{первый домен}{продолжительность*}
	\scnrelfrom{второй домен}{раннее событие или явление}

	\scnheader{позже\scnrolesign}
	\scnrelfrom{первый домен}{продолжительность*}
	\scnrelfrom{второй домен}{позднее событие или явление}

	\scnheader{слотовое sc-отношение}
	\scnexplanation{Слотовое sc-отношение --- бинарное sc-отношение (sc-множество (ориентированных) sc-пар), элементы которого не являются узловыми sc-парами.}

	\scnheader{sc-структура*}
	\scnexplanation{sc-структура* --- sc-множество, в котором есть непустое sc-подмножество-носитель (множество первичных элементов sc-структуры*).}

	\scnheader{sc-структура\scnrolesign}
	\scnrelfrom{первый домен}{sc-структура*}
	\scnrelfrom{второй домен}{непустое sc-множество}

	\scnheader{носитель sc-структуры\scnrolesign}
	\scnrelfrom{первый домен}{sc-структура*}
	\scnrelfrom{второй домен}{непустое sc-множество}

	\scnheader{элементарно представленное sc-множество\scnrolesign}
	\scnidtf{элементарно представленный элемент\scnrolesign}
	\scnexplanation{Элементарно представленный элемент\scnrolesign --- элемент sc-структуры*, sc-множество, все элементы которого являются элементами sc-структуры*.}

	\scnheader{полносвязно представленное sc-множество\scnrolesign}
	\scnidtf{полносвязно представленный элемент\scnrolesign}
	\scnexplanation{полносвязно представленный элемент\scnrolesign --- элемент sc-структуры*, sc-множество, все элементы и все принадлежности которому являются элементами sc-структуры*, или sc-дуга, являющаяся элементарно представленным элементом\scnrolesign этой sc-структуры*.}

	\scnheader{полностью представленное sc-множество\scnrolesign}
	\scnidtf{полностью представленный элемент\scnrolesign}
	\scnexplanation{Полностью представленный элемент\scnrolesign --- полносвязно представленный элемент\scnrolesign sc-структуры*, с любым элементом, не являющимся sc-дугой, выходящей из него, связанный принадлежащей этой sc-структуре* sc-дугой принадлежности или sc-дугой непринадлежности.}

	\scnheader{sc-связка\scnrolesign}
	\scnexplanation{sc-связка\scnrolesign --- полносвязно представленный элемент\scnrolesign sc-структуры*, принадлежащий sc-отношению\scnrolesign этой sc-структуры*, являющийся sc-связкой.}

	\scnheader{sc-отношение\scnrolesign}
	\scnexplanation{sc-отношение\scnrolesign --- полносвязно представленный элемент\scnrolesign sc-структуры*, sc-отношение, все элементы которого являются sc-связками\scnrolesign этой sc-структуры*.}

	\scnheader{sc-класс\scnrolesign}
	\scnexplanation{sc-класс\scnrolesign --- полносвязно представленный элемент\scnrolesign sc-структуры*, все элементы которого являются элементами sc-структуры*, не являющийся ни sc-отношением\scnrolesign, ни sc-связкой\scnrolesign этой sc-структуры*.}

	\scnheader{сущностное замыкание*}
	\scnexplanation{Сущностное замыкание* --- наименьшее надмножество* (структура*), в котором каждый элемент является элементарно представленным\scnrolesign.}

	\scnheader{сущностное замыкание\scnrolesign}
	\scnrelfrom{первый домен}{сущностное замыкание*}
	\scnrelfrom{второй домен}{сущностное замыкание}

	\scnheader{носитель сущностного замыкания\scnrolesign}
	\scnrelfrom{первый домен}{сущностное замыкание*}
	\scnrelfrom{второй домен}{непустое sc-множество}

	\scnheader{содержательное замыкание*}
	\scnexplanation{содержательное замыкание* --- наименьшее надмножество* (структура*), в котором каждый элемент является полносвязно представленным\scnrolesign}

	\scnheader{содержательное замыкание\scnrolesign}
	\scnrelfrom{первый домен}{содержательное замыкание*}
	\scnrelfrom{второй домен}{содержательное замыкание}

	\scnheader{носитель содержательного замыкания\scnrolesign}
	\scnrelfrom{первый домен}{содержательное замыкание*}
	\scnrelfrom{второй домен}{непустое sc-множество}

	\scnheader{sc-отношение сходства по слотовым отношениям*}
	\scnexplanation{sc-отношение сходства по слотовым sc-отношениям* --- sc-отношение, являющееся рефлексивным по этим слотовым отношениям, то есть для любого элемента, входящего в связку этого sc-отношения под одним из слотовых sc-отношений, найдется связка этого sc-отношения, в которую он входит под каждым из этих слотовых sc-отношений.}

	\scnheader{sc-отношение сходства по слотовым отношениям\scnrolesign}
	\scnrelfrom{первый домен}{sc-отношение сходства по слотовым отношениям*}
	\scnrelfrom{второй домен}{sc-отношение сходства по слотовым отношениям}

	\scnheader{слотовые отношения сходства sc-отношения\scnrolesign}
	\scnrelfrom{первый домен}{sc-отношение сходства по слотовым отношениям*}
	\scnrelfrom{второй домен}{слотовые отношения сходства sc-отношения}

	\scnheader{sc-отношение семантического сходства по слотовым отношениям*}
	\scnexplanation{sc-отношение семантического сходства по слотовым отношениям* --- sc-отношение сходства по слотовым sc-отношениям* si и sj, в котором каждый элемент под слотовым sc-отношением si, может быть преобразован к элементу синтаксического типа элемента под слотовым sc-отношением sj; два инцидентных sc-элемента под слотовым sc-отношением si, в рамках этого sc-отношения семантического сходства соответствуют инцидентным элементам соответственно под слотовым sc-отношением sj.}

	\scnheader{sc-отношение семантического сходства по слотовым отношениям\scnrolesign}
	\scnrelfrom{первый домен}{sc-отношение семантического сходства по слотовым отношениям*}
	\scnrelfrom{второй домен}{sc-отношение семантического сходства по слотовым отношениям}

	\scnheader{слотовые отношения семантического сходства sc-отношения\scnrolesign}
	\scnrelfrom{первый домен}{sc-отношение семантического сходства по слотовым отношениям*}
	\scnrelfrom{второй домен}{слотовые отношения семантического сходства sc-отношения}

	\scnheader{связная sc-структура*}
	\scnexplanation{Связная sc-структура* --- sc-структура*, являющаяся связной.}

	\scnheader{связная sc-структура\scnrolesign}
	\scnrelfrom{первый домен}{связная sc-структура*}
	\scnrelfrom{второй домен}{связное непустое sc-множество}

	\scnheader{носитель связной sc-структуры\scnrolesign}
	\scnrelfrom{первый домен}{связная sc-структура*}
	\scnrelfrom{второй домен}{непустое sc-множество}

	\scnheader{семантическое сходство sc-структур*}
	\scnidtf{семантическое подобие sc-структур*}
	\scnexplanation{Семантическое сходство sc-структур* --- связывает sc-множество sc-структур* с sc-структурой* sc-отношением семантического сходства по слотовым sc-отношениям si, sj так, что для каждой sc-структуры* из sc-множества найдется ее элемент и связка этого sc-отношения сходства, в которую он входит под слотовым sc-отношением si, а под слотовым sc-отношением sj входит элемент sc-структуры*, также для каждого элемента sc-структуры найдется связка этого sc-отношения сходства, в которую он входит под слотовым sc-отношением sj, а под слотовым sc-отношением si входит элемент sc-структуры* из sc-множества.}

	\scnheader{sc-отношение семантического сходства sc-структур\scnrolesign}
	\scnrelfrom{первый домен}{семантическое сходство sc-структур*}
	\scnrelfrom{второй домен}{sc-отношение семантического сходства по слотовым отношениям*}

	\scnheader{семантическое сходство sc-структур\scnrolesign}
	\scnrelfrom{первый домен}{семантическое сходство sc-структур*}
	\scnrelfrom{второй домен}{sc-структура семантического сходства sc-структур*}

	\scnheader{sc-структура семантического сходства sc-структур\scnrolesign}
	\scnrelfrom{первый домен}{sc-структура семантического сходства sc-структур*}
	\scnrelfrom{второй домен}{sc-структура семантического сходства sc-структур}

	\scnheader{множество семантически сходных sc-структур\scnrolesign}
	\scnrelfrom{первый домен}{sc-структура семантического сходства sc-структур*}
	\scnrelfrom{второй домен}{множество семантически сходных sc-структур}

	\scnheader{семантическое непрерывное сходство sc-структур*}
	\scnidtf{семантическое непрерывное подобие sc-структур*}
	\scnexplanation{Семантическое непрерывное сходство sc-структур* --- связывает sc-множество sc-структур* со связной sc-структурой* sc-отношением семантического сходства по слотовым sc-отношениям si, sj так, что для каждой sc-структуры* из sc-множества найдется ее элемент и связка этого sc-отношения сходства, в которую он входит под слотовым sc-отношением si, а под слотовым sc-отношением sj входит элемент связной sc-структуры*, также для каждого элемента связной sc-структуры найдется связка этого sc-отношения сходства, в которую он входит под слотовым sc-отношением sj, а под слотовым sc-отношением si входит элемент sc-структуры* из sc-множества.}

	\scnheader{sc-отношение семантического непрерывного сходства sc-структур\scnrolesign}
	\scnrelfrom{первый домен}{семантическое непрерывное сходство sc-структур*}
	\scnrelfrom{второй домен}{sc-отношение семантического непрерывного сходства по слотовым отношениям*}

	\scnheader{семантическое непрерывное сходство sc-структур\scnrolesign}
	\scnrelfrom{первый домен}{семантическое непрерывное сходство sc-структур*}
	\scnrelfrom{второй домен}{sc-структура семантического непрерывного сходства sc-структур*}

	\scnheader{sc-структура семантического непрерывного сходства sc-структур\scnrolesign}
	\scnrelfrom{первый домен}{sc-структура семантического непрерывного сходства sc-структур*}
	\scnrelfrom{второй домен}{sc-структура семантического непрерывного сходства sc-структур}

	\scnheader{множество семантически непрерывно сходных sc-структур\scnrolesign}
	\scnrelfrom{первый домен}{sc-структура семантического непрерывного сходства sc-структур*}
	\scnrelfrom{второй домен}{множество семантически непрерывно сходных sc-структур}

	\scnheader{ключевой запрос\scnrolesign}
	\scnrelfrom{первый домен}{ключевой запрос*}
	\scnrelfrom{второй домен}{ключевой запрос}
	\scnexplanation{Ключевой запрос\scnrolesign --- поисковый-проверочный запрос (от одного известного элемента), который выполняется хотя бы от одного элемента и не выполняется хотя бы от одного элемента.}

	\scnheader{элемент ключевого запроса\scnrolesign}
	\scnrelfrom{первый домен}{ключевой запрос*}
	\scnrelfrom{второй домен}{элемент ключевого запроса}

	\scnheader{минимальный ключевой запрос\scnrolesign}
	\scnsubset{ключевой запрос\scnrolesign}
	\scnexplanation{Минимальный ключевой запрос --- ключевой запрос, который находит sc-подмножества множеств элементов, находимых всеми другими ключевыми запросами, которые имеют те же области известных элементов выполнимости и невыполнимости.}

	\scnheader{элемент минимального ключевого запроса\scnrolesign}
	\scnrelfrom{первый домен}{минимальный ключевой запрос*}
	\scnrelfrom{второй домен}{элемент минимального ключевого запроса}

	\scnheader{полная семантическая окрестность элемента*}
	\scnexplanation{Полная семантическая окрестность элемента* --- все элементы, находимые выполнимыми минимальными ключевыми запросами от этого элемента (c учетом дизъюнктивного поиска и отрицания поиска).}

	\scnheader{полная семантическая окрестность элемента\scnrolesign}
	\scnrelfrom{первый домен}{полная семантическая окрестность элемента*}
	\scnrelfrom{второй домен}{полная семантическая окрестность элемента}

	\scnheader{элемент полной семантической окрестности\scnrolesign}
	\scnrelfrom{первый домен}{полная семантическая окрестность элемента*}
	\scnrelfrom{второй домен}{элемент полной семантической окрестности}

	\scnheader{интроспективный ключевой элемент\scnrolesign}
	\scnexplanation{интроспективный (базовый) ключевой элемент --- элемент множества (из класса наименьших таких множеств) элементов такого, что любая полная семантическая окрестность любого элемента является sc-подмножеством объединения их полных семантических окрестностей.}

	\scnheader{топологическое пространство замыкания инцидентности коннекторов}
	\scnexplanation{Топологическое пространство замыкания инцидентности коннекторов на множестве sc-элементов --- топологическое пространство, все замкнутые множества которого содержат все sc-элементы этого множества, до которых есть маршрут по ориентированным связкам отношения инцидентности коннекторов.}
	\scntext{примечание}{В общем случае не удовлетворяет аксиоме отделимости по Тихонову. Прагматика рассмотрения таких пространств обуславливается операциями удаления sc-элементов и коннекторов, которым они инцидентны. Удаление sc-элемента требует удаления всех коннекторов, замыканию любой открытой окрестности которых он принадлежит.}

	\scnheader{топологическое подпространство замыкания инцидентности коннекторов\scnrolesign}
	\scnrelfrom{первый домен}{включение топологических пространств замыкания инцидентности коннекторов*}
	\scnrelfrom{второй домен}{топологическое пространство замыкания инцидентности коннекторов}

	\scnheader{топологическое надпространство замыкания инцидентности коннекторов\scnrolesign}
	\scnrelfrom{первый домен}{включение топологических пространств замыкания инцидентности коннекторов*}
	\scnrelfrom{второй домен}{топологическое пространство замыкания инцидентности коннекторов}

	\scnheader{топологическое пространство синтаксического замыкания}
	\scnexplanation{Топологическое пространство синтаксического замыкания на множестве sc-элементов --- топологическое пространство, все замкнутые множества которого содержат все sc-элементы этого множества, до которых есть маршрут по ориентированным связкам отношения инцидентности.}
	\scntext{примечание}{В общем случае не удовлетворяет аксиоме отделимости по Колмогорову. В качестве основы замкнутых множеств топологического пространства можно выделить синтаксическое замыкание, однако в силу возможности проведения дуг из любого sc-узла в любой в итоговом случае (в итоге процесса устранения не-факторов) такое пространство является тривиальным (антидискретным) пространством. Отношение объединения топологических пространств синтаксического замыкания алгебраически не замкнуто на множестве топологических пространств синтаксического замыкания. По той же причине для любого неполного топологического пространства синтаксического замыкания можно рассмотреть топологическое пространство синтаксического замыкания, носитель которого является надмножеством носителя первого и которое не сохраняет замкнутые множества. В этом смысле топология на основе синтаксического замыкания не является устойчивой относительно процессов становления знаний и ее рассмотрение прагматически не оправдывается. Топология полного же топологического пространства синтаксического замыкания антидискретна (тривиальна). Таким образом, у полного топологического пространства синтаксического замыкания все топологические подпространства синтаксического замыкания обладают антидискретной (тривиальной) топологией.}

	\scnheader{топологическое пространство сущностного замыкания}
	\scnexplanation{Топологическое пространство сущностного замыкания на множестве sc-элементов --- топологическое пространство, все замкнутые множества которого являются сущностными замыканиями.}
	\scntext{примечание}{В общем случае не удовлетворяет аксиоме отделимости по Тихонову. В качестве носителя топологического (под)пространства можно выделить сущностное замыкание. Топологическое пространство сущностного замыкания сохраняет замкнутые множества любых топологических пространств сущностного замыкания, носитель которых является подмножеством его носителя и сущностным замыканием. Такие пространства образуют структуру топологических подпространств-топологических надпространств сущностного замыкания. Топология пространств в этой структуре разнообразна.}

	\scnheader{топологическое подпространство сущностного замыкания\scnrolesign}
	\scnrelfrom{первый домен}{включение топологических пространств сущностного замыкания*}
	\scnrelfrom{второй домен}{топологическое пространство сущностного замыкания}

	\scnheader{топологическое надпространство сущностного замыкания\scnrolesign}
	\scnrelfrom{первый домен}{включение топологических пространств сущностного замыкания*}
	\scnrelfrom{второй домен}{топологическое пространство сущностного замыкания}

	\scnheader{топологическое пространство содержательного замыкания}
	\scnexplanation{Топологическое пространство содержательного замыкания на множестве sc-элементов --- топологическое пространство, все замкнутые множества которого являются содержательными замыканиями.}
	\scntext{примечание}{В общем случае не удовлетворяет аксиоме отделимости по Тихонову. В качестве носителя топологического (под)пространства можно выделить содержательное замыкание. Топологическое пространство содержательного замыкания сохраняет замкнутые множества любых топологических пространств содержательного замыкания, носитель которых является подмножеством его носителя и содержательным замыканием. Такие пространства образуют структуру топологических подпространств-топологических надпространств содержательного замыкания. Топология пространств в этой структуре разнообразна.}
	
	\scnheader{топологическое подпространство содержательного замыкания\scnrolesign}
	\scnrelfrom{первый домен}{включение топологических пространств содержательного замыкания*}
	\scnrelfrom{второй домен}{топологическое пространство содержательного замыкания}


	\scnheader{топологическое надпространство содержательного замыкания\scnrolesign}
	\scnrelfrom{первый домен}{включение топологических пространств содержательного замыкания*}
	\scnrelfrom{второй домен}{топологическое пространство содержательного замыкания}
    \scntext{примечание}{Возможен переход от sc-структур к многообразиям и топологическим пространствам путем сведения sc-структур к графовым структурам.
        \\Для более сложных структур таких, как полная семантическая окрестность, топологические свойства подлежат дальнейшему изучению.
        \\Далее можно рассмотреть метрические пространства, в частности --- конечные подпространства с полностью представленными sc-элементами. }
        \begin{scnindent}
        	\scnrelfrom{смотрите}{\scncite{Ivashenko2022}}
        \end{scnindent}


	\scnheader{метрика}
	\scnexplanation{Метрика --- функция двух аргументов, принимающая значения на (линейно) упорядоченном носителе группы, неотрицательна, равна нейтральному элмененту (нулю) только при равенстве аргументов, симметрична, удовлетворяет неравенству треугольника.}

	\scnheader{метрическое пространство}
	\scnexplanation{Метрическое пространство --- множество, с определенной на нем функцией двух аргументов, являющейся метрикой, принимающей значения на упорядоченном носителе группы.}

	\scnheader{семантическая метрика}
	\scnidtf{семантическая близость}
	\scnexplanation{Семантическая метрика --- метрика, определенная на знаках и выражающая количественно близость их значений.}

	\scnheader{метрическое конечное синтаксическое пространство}
	\scnexplanation{Метрическое конечное синтаксическое пространство SC-кода --- метрическое пространство с конечным носителем, элементами которого являются обозначения (sc-элементы), а значение метрики может быть определено через отношения инцидентности элементов без учета их семантического типа.}

	\scnheader{метрическое конечное синтаксическое подпространство\scnrolesign}
	\scnrelfrom{первый домен}{включение метрических конечных синтаксических пространств*}
	\scnrelfrom{второй домен}{метрическое конечное синтаксическое пространство}

	\scnheader{метрическое конечное синтаксическое надпространство\scnrolesign}
	\scnrelfrom{первый домен}{включение метрических конечных синтаксических пространств*}
	\scnrelfrom{второй домен}{метрическое конечное синтаксическое пространство}

	\scnheader{метрическое конечное семантическое пространство}
	\scnexplanation{Метрическое конечное семантическое пространство SC-кода --- метрическое пространство с конечным носителем, элементами которого являются обозначения (sc-элементы), а значение метрики не может быть определено через отношения инцидентности элементов без учета их семантического типа.}

	\scnheader{метрическое конечное семантическое подпространство\scnrolesign}
	\scnrelfrom{первый домен}{включение метрических конечных семантических пространств*}
	\scnrelfrom{второй домен}{метрическое конечное семантическое пространство}

	\scnheader{метрическое конечное семантическое надпространство\scnrolesign}
	\scnrelfrom{первый домен}{включение метрических конечных семантических пространств*}
	\scnrelfrom{второй домен}{метрическое конечное семантическое пространство}

    \scntext{примечание}{Метрическое конечное синтаксическое пространство может быть построено в соответствии с моделью обработки строк и определениями метрики.}
	\begin{scnindent}
		\begin{scnrelfromset}{смотрите}
			\scnitem{\scncite{Ivashenko2022}}
			\scnitem{\scncite{Ivashenko2020String}}
		\end{scnrelfromset}
	\end{scnindent}
	
	\scnheader{псевдометрика}
	\scntext{пояснение}{Псевдометрика --- функция двух аргументов, принимающая значения на (линейно) упорядоченном носителе группы, неотрицательна, симметрична, удовлетворяет неравенству треугольника.}

	\scnheader{псевдометрическое пространство}
	\scntext{пояснение}{псевдометрическое пространство --- множество, с определенной на нем функцией двух аргументов, являющейся псевдометрикой, принимающей значения на упорядоченном носителе группы.}
	\begin{scnindent}
		\scnrelfrom{смотрите}{\scncite{Collatz1966}}
	\end{scnindent}

	\scnheader{псевдометрическое конечное семантическое пространство}
	\scntext{пояснение}{Псевдометрическое конечное семантическое пространство SC-кода --- псевдометрическое пространство с конечным носителем, элементами которого являются обозначения (sc-элементы), а значение псевдометрики не может быть определено через отношения инцидентности элементов без учета их семантического типа.}
    \begin{scnrelfromvector}{примечание}
    	\scnfileitem{В силу неполноты выразительных средств для представления изменяющихся со временем знаний, отсутствия определенной пространственно-временной модели, наличия семантически неопределенных или слабоопределенных обозначений в текстах да и наличия недоопределенности самих текстов описанного в предыдущих разделах языка, на данном этапе в этом описании затруднительно предложить какую-либо модель метрического пространства для более сложных структур, учитывающих не-факторы, связанные с пространством-временем.}
        \scnfileitem{Предложенные модели полагались на представление, способное выразить семантику переменных обозначений и операционную семантику расширенными средствами алфавита. Для построения подобных моделей, кроме расширенных средств алфавита, предлагается полагаться на модели, описывающие процессы интеграции и становления знаний на средства спецификации знаний, ориентированные на рассмотрение финитных структур, что позволяет перейти к рассмотрению сложных метрических соотношений в рамках метамодели смыслового пространства.}
		\begin{scnindent}
			\begin{scnrelfromset}{смотрите}
				\scnitem{\scncite{Ivashenko2022}}
				\scnitem{\scncite{Ivashenko2017}}
			\end{scnrelfromset}
		\end{scnindent}
        \scnfileitem{В разных науках исследователи затрагивали вопросы касающиеся смыслов и их размещения и взаимосвязи. Можно выделить следующие работы, которые соотносятся с тремя подходами: экстериорный подход, интериорные подходы на основании количественных и структурно-динамических признаков.}
        \begin{scnindent}
        	\begin{scnrelfromset}{смотрите}
        		\scnitem{\scncite{Bohm1993}}
        		\scnitem{\scncite{Nalimov1995}}
        		\scnitem{\scncite{Bohm2002}}
        		\scnitem{\scncite{Martynov2004}}
        		\scnitem{\scncite{Nalimov1979}}
        		\scnitem{\scncite{Nalimov1989}}
        	\end{scnrelfromset}
        \end{scnindent}
        \scnfileitem{В современных работах в технических науках, возможно, наиболее близкими понятиями являются понятия, выражающие смысл термина \scnqqi{семантическое пространство} (интериорный подход).
        Общим во многих подходах к работе с \scnqqi{семантическим пространством} является рассмотрение словоформ или лексем (множеств словоформ) и их признаков. В литературе встречаются следующие подходы:
        \begin{scnitemize}
            \item подход на основе семантических осей и пространства признаков (бинарных $\scnleftcurlbrace 0,1\scnrightcurlbrace \upperscore{n}$, монополярных $\scnleftsquarebrace 0;1\scnrightsquarebrace\upperscore{n}$, биполярных $\scnleftsquarebrace -1;1\scnrightsquarebrace\upperscore{n}$);
            \item подход на основе семантических осей и нейронного кодирования места в поле смыслов (слова и словосочетания имеют области (подмножества) значений, связываясь другими частями речи как включением и пересечением, тексты соответствуют пути связанных областей, бинарное кодирование групп нейронов, распознающих смыслы);
            \item подход на основе модели \scnqqi{смысл-текст} (отражение неполноты семантических шкал и анализ синтагм и поверхностно-синтаксической структуры);
            \item нейролингвистические данные отражает процессы синтеза и восприятия речи в нейронных сетях (сеть лексического синтеза), близка к модели \scnqqi{смысл-текст};
            \item модели, построенные на основе статического анализа (корпусов) текстов (модель векторного пространства).
        \end{scnitemize}}
    	\begin{scnindent}
    		\begin{scnrelfromset}{смотрите}
    			\scnitem{\scncite{Manin2016}}
    			\scnitem{\scncite{Melchuk2016}}
    			\scnitem{\scncite{Harris1992}}
    		\end{scnrelfromset}
    	\end{scnindent}
        \scnfileitem{Статистический подход к обработке естественного языка противопоставляется интуиции и коммуникативному опыту ученых. В основе подхода лежит семантическая статистическая гипотеза, что смысл слов (лексем) определяется контекстом использования (его статистическим образом) в языке (с коммуникативной структурой).}
		\begin{scnindent}
			\scnrelfrom{смотрите}{\scncite{Manin2016}}
		\end{scnindent}
        \scnfileitem{Модель векторного пространства семантики. Модель рассматривается для двух случаев: большого словаря ($N \leq M $) и задачи информационного поиска ($M \leq N $). $M$ --- размер словаря, $N$ --- количество контекстов.}
        \begin{scnindent}
        	\scnrelfrom{смотрите}{\scncite{Manin2016}}
        \end{scnindent}
		\scnfileitem{На основе статистики строится матрица размерности $ M \times N $ частот $ p\underscore{ij} $ появления лексемы (слова) $ w\underscore{i} $ в документе (контексте, подтексты, которые могут перекрываться) $ c\underscore{j} $ . 
			\\ $$ x\underscore{ij} \eq \max{\scnleftcurlbrace 0\scnrightcurlbrace \cup \scnleftcurlbrace \log (\frac{p\upperscore{ij}}{(\Sigma \upperscore{j} p\upperscore{ij}) * ( \Sigma \upperscore{i} p\upperscore{ij}) }) \scnrightcurlbrace} $$ }
		\scnfileitem{В знаменателе --- оценки вероятности слова и контекста соответственно.
			\\В случае невырожденной матрицы $r \eq N$ каждая такая матрица задает точку в грассманиане $N$-мерных подпространств $M$-мерного пространства ($ N \leq M $).
			\\В случае невырожденной матрицы $r \eq M$ каждая такая матрица задает точку в грассманиане $M$-мерных подпространств $N$-мерного пространства ($ M \leq N $). }
		\scnfileitem{Каждый текст --- точка в грассманиане, соответствующем проективному пространству $P\upperscore{M-1} \eq Gr(\langle 1,M \rangle )$, относительно одного выделенного контекста. Для всех контекстов получая ориентированную $N$-ку, в соответствии с порядком контекстов в текстах, можно построить маршрут (путь), соединяя геодезическими соседние точки в $N$-ке. Для двух текстов $T$ и $T\scnrolesign$ это будут две ломанные, между которыми можно вычислить метрику Фреше, используя метрику Фубини-Штуди в $P\upperscore{M-1}$, для этого следует параметризовать пути $\Gamma( T )$ и $\Gamma( T\scnrolesign )$ через $t$ ($\gamma\in\Gamma( T )\upperscore{\scnleftsquarebrace 0;1\scnrightsquarebrace }$, $\gamma \scnrolesign\in\Gamma( T\scnrolesign)\upperscore{\scnleftsquarebrace 0;1\scnrightsquarebrace }$):  $$ \delta( \langle \Gamma ( T ),\Gamma ( T\scnrolesign)\rangle ) \eq\inf\upperscore{\gamma,\gamma\scnrolesign}\max{t\in\scnleftsquarebrace 0;1\scnrightsquarebrace }(  \scnleftcurlbrace d\upperscore{FS}( \langle \gamma( t) ,\gamma\scnrolesign( t) \rangle ) \scnrightcurlbrace ). $$}
        \begin{scnindent}
        	\begin{scnrelfromset}{смотрите}
        		\scnitem{\scncite{Alt1995}}
        		\scnitem{\scncite{Study1905}}
        		\scnitem{\scncite{Harris1992}}
        	\end{scnrelfromset}
        \end{scnindent}
        \scnfileitem{Другой способ задать линейный порядок --- это рассмотреть фильтрацию в $\mathbb{R}\upperscore{M}$, заданную расширяющимися контекстами. В итоге для текста получаем точки (флаги) во флаговом многообразии. Для флаговых многообразий тоже можно вычислить метрику Фубини-Штуди.}
        \begin{scnindent}
        	\begin{scnrelfromset}{смотрите}
        		\scnitem{\scncite{Kostrikin1997}}
        		\scnitem{\scncite{Study1905}}
        	\end{scnrelfromset}
        \end{scnindent}
        \scnfileitem{Этот порядок соответствует временному измерению (процессу коммуникации во времени), что может быть существенным. Другой порядок может быть не зависимым от этого, например алфавитный или порядок в соответствии с законом Ципфа.}
        \begin{scnindent}
        	\begin{scnrelfromset}{смотрите}
        		\scnitem{\scncite{Lowe2001}}
        		\scnitem{\scncite{Manin2014}}
        	\end{scnrelfromset}
        \end{scnindent}
	\end{scnrelfromvector}

	\scnheader{Таблица. Сравнение подходов к построению \scnqqi{семантических пространств}}
	\scnrelfrom{описание примера}{\scnfileimage[35em]{Contents/part_kb/src/images/sd_lang/comparison_table.png}}
	\scntext{примечание}{Вопросы соотнесения смыслов, их формализации, развития языков в пространстве и времени.}
	\begin{scnrelfromset}{смотрите}
		\scnitem{\scncite{Gordey2014}}
		\scnitem{\scncite{Martynov2009}}
		\scnitem{\scncite{Martynov2004}}
	\end{scnrelfromset}

\end{scnsubstruct}


        \begin{scnrelfromvector}{заключение}
            \scnfileitem{\textit{SC-код} может быть использован в качестве метаязыка для описания \textit{собственной} денотационной семантики и синтаксиса.}
            \scnfileitem{Аспекты представления знаний в памяти \textit{интеллектуальных компьютерных систем}, которые требуют особой аккуратности.}
            \begin{scnindent}
            	\begin{scnrelfromvector}{разбиение}
	                \scnfileitem{Константный и переменный характер денотационной семантики знаков, хранимых в памяти \textit{интеллектуальных компьютерных систем}.}
	                \scnfileitem{Динамический характер знаковых конструкций, хранимых в памяти, обусловленный либо выполняемыми в памяти \textit{информационными процессами}, либо динамическим характером структур внешних объектов, описываемых этими знаковыми конструкциями.}
	                \scnfileitem{Временный характер существования внешних описываемых объектов и временный характер существования различных конфигураций знаковых конструкций и даже самих \textit{знаков, хранимых в памяти}.}
	                \scnfileitem{Наличие информационного мусора (излишеств) в хранимых знаниях.}
	                \scnfileitem{Неизвестность (отсутствие в памяти) востребованной информации различного вида, неполнота знаний, их недостаточность для решения актуальных задач.}
	                \scnfileitem{Синтаксическая и семантическая некорректность, неадекватность и,  в частности, противоречивость некоторых имеющихся знаний.}
                \end{scnrelfromvector}
                \scntext{примечание}{Последние из перечисленных аспектов представления знаний называют \textit{не-факторами представления знаний}}
                \begin{scnindent}
                	\scnrelfrom{смотрите}{\scncite{Narinjani2000}}
                \end{scnindent}
            \end{scnindent}
        \end{scnrelfromvector}
        \scnrelfrom{пример}{Типология \textit{sc-конструкций} с точки зрения \textit{Денотационной семантики и Синтаксиса SC-кода}}

        \scnheader{Типология \textit{sc-конструкций} с точки зрения \textit{Денотационной семантики и Синтаксиса SC-кода}}
        \begin{scneqtoset}
            \scnitem{sc-множество}
			\begin{scnindent}
				\scnidtf{\textit{sc-конструкция}}
				\scnidtf{информационная конструкция, принадлежащая \textit{SC-коду}}
				\scnsuperset{sc-структура}
				\begin{scnindent}
					\scnsuperset{sc-текст}
					\begin{scnindent}
						\scnidtftext{часто используемый sc-идентификатор}{\textit{SC-код}}
						\begin{scnindent}
							\scniselement{имя собственное}
						\end{scnindent} 
						\scnidtf{синтаксически целостная и синтаксически корректная (правильно построенная) информационная конструкция SC-кода}
						\scnidtf{Класс (Множество всевозможных) sc-текстов}
						\scnsuperset{sc-знание}
						\begin{scnindent}
							\scnidtf{семантически целостный и семантически корректный \textit{sc-текст}, являющийся адекватным фрагментом соответствующей \textit{предметной области} или ее спецификации (онтологии)}
						\end{scnindent} 
					\end{scnindent} 
				\end{scnindent}
			\end{scnindent}
        \end{scneqtoset}
\end{scnsubstruct}
\scnendcurrentsectioncomment

\end{SCn}
\scsubsubsection{Пункт 21.1.1. Предметная область и онтология синтаксиса внутреннего языка ostis-систем}
\label{sd_sc_code_syntax}
\scsubsubsection{Пункт 21.1.2. Предметная область и онтология базовой денотационной семантики внутреннего языка ostis-систем}
\label{sd_sc_code_semantic}

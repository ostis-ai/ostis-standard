\scnsegmentheader{Смысловое пространство ostis-систем}
\begin{scnsubstruct}

    \begin{scnrelfromlist}{ключевое понятие}
		\scnitem{обобщенная sc-связка}
		\scnitem{обобщенное sc-отношение}
		\scnitem{бинарное sc-отношение}
		\scnitem{слотовое sc-отношение}
		\scnitem{sc-структура*}
		\scnitem{элементарно представленный элемент\scnrolesign}
		\scnitem{полносвязно представленный элемент\scnrolesign}
		\scnitem{полностью представленный элемент\scnrolesign}
		\scnitem{sc-связка\scnrolesign}
		\scnitem{sc-отношение\scnrolesign}
		\scnitem{sc-класс\scnrolesign}
		\scnitem{сущностное замыкание*}
		\scnitem{содержательное замыкание*}
		\scnitem{sc-отношение сходства по слотовым отношениям*}
		\scnitem{sc-отношение семантического сходства по слотовым отношениям*}
		\scnitem{связная sc-структура*}
		\scnitem{семантическое сходство sc-структур*}
		\scnitem{семантическое непрерывное сходство sc-структур*}
		\scnitem{ключевой запрос\scnrolesign}
		\scnitem{минимальный ключевой запрос\scnrolesign}
		\scnitem{полная семантическая окрестность элемента*}
		\scnitem{интроспективный ключевой элемент\scnrolesign}
		\scnitem{топологическое пространство}
		\scnitem{топологическое пространство замыкания инцидентности коннекторов}
		\scnitem{топологическое пространство синтаксического замыкания}
		\scnitem{топологическое пространство сущностного замыкания}
		\scnitem{топологическое пространство содержательного замыкания}
		\scnitem{метрика}
		\scnitem{семантическая метрика}
		\scnitem{метрическое пространство}
		\scnitem{метрическое конечное синтаксическое пространство}
		\scnitem{метрическое конечное семантическое пространство}
		\scnitem{псевдометрика}
		\scnitem{псевдометрическое пространство}
		\scnitem{псевдометрическое конечное семантическое пространство}
	\end{scnrelfromlist}

\begin{scnrelfromvector}{примечание}
	\scnfileitem{Важнейшим достоинством \textbf{\textit{SC-пространства}} является возможность уточнения таких понятий, как понятие аналогичности (сходства и отличия) различных описываемых \scnqq{внешних} сущностей, аналогичности унифицированных \textit{семантических сетей} (текстов \textbf{\textit{SC-кода}}), понятие семантической близости описываемых сущностей (в том числе, и текстов \textbf{\textit{SC-кода}}).}
	\scnfileitem{Следует отметить, что в силу абстрактности языков модели \textit{унифицированного семантического представления знаний} и условности выбора меток элементов их текстов, нельзя исключить, что объединение двух произвольных текстов таких языков не будет текстом языка модели \textit{унифицированного семантического представления знаний}. Чтобы избежать результатов подобных эклектических объединений с точки зрения синтаксиса или семантики, для абстрактных языков следует рассматривать множество \scnqqi{смысловых пространств}. Однако, для конкретных (реальных) языков может оказаться достаточным рассмотрение одного \scnqqi{смыслового пространства}.}
    \scnfileitem{Далее рассмотрим:
    \begin{scnitemize}
        \item возможность перехода от sc-текстов к графовым структурам и от них к топологическому пространству;
        \item возможность перехода от sc-текстов к графовым структурам и от них к многообразию (топологическому пространству);
        \item возможность перехода от sc-текстов к графовым структурам и от них к метрическому пространству.
    \end{scnitemize}}
    \scnfileitem{Чтобы исследовать структурные свойства \textbf{\textit{SC-пространства}}, можно использовать уже разработанные модели пространств и связь их известными топологическими моделями, например, такими как \textit{графы}. При этом изначально не будем принимать в расчет динамические особенности, связанные с обработкой знаний, однако позже будет показано, что учет динамики в процессах обработки и при становлении знаний является необходимым для вычисления семантической метрики, являющейся одним из определяющих признаков знаний.
    Обратимся к исследованию структурно-топологических свойств пространства.}
    \scnfileitem{Структурно-топологические свойства могут свидетельствовать о синтаксических или семантических зависимостях обозначений текстов языка, позволяющих упростить работу со структурами за счет перехода к более простым структурам на уровнях управления данными или знаниями в характерных задачах управления для \textit{библиотеки многократно используемых компонентов ostis-систем}.}
    \begin{scnindent}
    	\scnrelfrom{смотрите}{Комплексная библиотека многократно используемых семантически совместимых компонентов ostis-систем}
    \end{scnindent}
    \scnfileitem{На множестве элементов, образующих \textbf{\textit{SC-пространство}}, можно изучать топологические свойства и рассматривать \textbf{\textit{SC-пространство}} как топологическое пространство. Следует заметить, что, несмотря на то, что \textbf{\textit{SC-код}} ориентирован на смысловое представление знаний, в силу наличия \textit{не-факторов}, не все смыслы могут быть представлены в некоторый момент времени и не будет известна структура соответствующего представления. Поэтому структурно-топологические свойства текстов \textit{языка представления знаний} скорее определяют синтаксическое пространство, нежели семантическое (смысловое). Хотя оба могут приближаться друг к другу по мере устранения неопределенностей, вызванных \textit{не-факторами}.}
    \scnfileitem{Рассмотрим следующие виды \textit{топологических пространств}:
    \begin{scnitemize}
        \item \textit{топологическое пространство замыкания инцидентности коннекторов};
        \item \textit{топологическое пространство синтаксического замыкания};
        \item \textit{топологическое пространство сущностного замыкания};
        \item \textit{топологическое пространство содержательного замыкания}.
    \end{scnitemize}}
\end{scnrelfromvector}
	
	\scnheader{топологическое пространство}
	\scntext{пояснение}{\textit{топологическое пространство} --- \textit{множество} с определенным над ним \textit{множеством} (семейством) (открытых) подмножеств, включая само \textit{множество} и \textit{пустое множество}. Для любого \textit{подмножества} семейства результат объединения принадлежит \textit{семейству множеств}, а для любого конечного \textit{подмножества семейства} результат пересечения также принадлежит \textit{семейству множеств}. Дополнения множеств семейства до наибольшего из множеств называются \textit{замкнутыми множествами}.}

	\scnheader{обобщенная sc-связка}
	\scnidtf{непустое sc-множество}

	\scnheader{обобщенное sc-отношение}
	\scnidtf{sc-множество непустых sc-множеств}
	\scnexplanation{обобщенное sc-отношение --- sc-множество обобщенных sc-связок.}

	\scnheader{бинарное sc-отношение}
	\scnexplanation{Бинарное sc-отношение --- sc-множество sc-пар (или обобщенных sc-связок, которым существуют две различные принадлежности sc-элементов или одного и того же sc-элемента).}

	\scnheader{узловая sc-пара}
	\scnexplanation{узловая sc-пара --- sc-пара, которая не может быть обозначена sc-дугой принадлежности (позитивной, негативной или нечеткой).}

    \scnheader{явление принадлежности}
    \scnexplanation{явление принадлежности --- множество явлений, каждое из которых является слотовым sc-отношением, которому постоянно непринадлежат sc-дуги постоянной непринадлежности.}

    \scnheader{становление*}
    \scnexplanation{становление* --- бинарное sc-отношение между событиями (состояниями) или явлениями.}

	\scnheader{непосредственно прежде\scnrolesign}
	\scnrelfrom{первый домен}{становление*}
	\scnrelfrom{второй домен}{установленное событие или явление}

	\scnheader{непосредственно после\scnrolesign}
	\scnrelfrom{первый домен}{становление*}
	\scnrelfrom{второй домен}{устанавливающее событие или явление}

    \scnheader{продолжительность*}
    \scnexplanation{продолжительность* --- транзитивное замыкание sc-отношения становления.}

	\scnheader{раньше\scnrolesign}
	\scnrelfrom{первый домен}{продолжительность*}
	\scnrelfrom{второй домен}{раннее событие или явление}

	\scnheader{позже\scnrolesign}
	\scnrelfrom{первый домен}{продолжительность*}
	\scnrelfrom{второй домен}{позднее событие или явление}

	\scnheader{слотовое sc-отношение}
	\scnexplanation{Слотовое sc-отношение --- бинарное sc-отношение (sc-множество (ориентированных) sc-пар), элементы которого не являются узловыми sc-парами.}

	\scnheader{sc-структура*}
	\scnexplanation{sc-структура* --- sc-множество, в котором есть непустое sc-подмножество-носитель (множество первичных элементов sc-структуры*).}

	\scnheader{sc-структура\scnrolesign}
	\scnrelfrom{первый домен}{sc-структура*}
	\scnrelfrom{второй домен}{непустое sc-множество}

	\scnheader{носитель sc-структуры\scnrolesign}
	\scnrelfrom{первый домен}{sc-структура*}
	\scnrelfrom{второй домен}{непустое sc-множество}

	\scnheader{элементарно представленное sc-множество\scnrolesign}
	\scnidtf{элементарно представленный элемент\scnrolesign}
	\scnexplanation{Элементарно представленный элемент\scnrolesign --- элемент sc-структуры*, sc-множество, все элементы которого являются элементами sc-структуры*.}

	\scnheader{полносвязно представленное sc-множество\scnrolesign}
	\scnidtf{полносвязно представленный элемент\scnrolesign}
	\scnexplanation{полносвязно представленный элемент\scnrolesign --- элемент sc-структуры*, sc-множество, все элементы и все принадлежности которому являются элементами sc-структуры*, или sc-дуга, являющаяся элементарно представленным элементом\scnrolesign этой sc-структуры*.}

	\scnheader{полностью представленное sc-множество\scnrolesign}
	\scnidtf{полностью представленный элемент\scnrolesign}
	\scnexplanation{Полностью представленный элемент\scnrolesign --- полносвязно представленный элемент\scnrolesign sc-структуры*, с любым элементом, не являющимся sc-дугой, выходящей из него, связанный принадлежащей этой sc-структуре* sc-дугой принадлежности или sc-дугой непринадлежности.}

	\scnheader{sc-связка\scnrolesign}
	\scnexplanation{sc-связка\scnrolesign --- полносвязно представленный элемент\scnrolesign sc-структуры*, принадлежащий sc-отношению\scnrolesign этой sc-структуры*, являющийся sc-связкой.}

	\scnheader{sc-отношение\scnrolesign}
	\scnexplanation{sc-отношение\scnrolesign --- полносвязно представленный элемент\scnrolesign sc-структуры*, sc-отношение, все элементы которого являются sc-связками\scnrolesign этой sc-структуры*.}

	\scnheader{sc-класс\scnrolesign}
	\scnexplanation{sc-класс\scnrolesign --- полносвязно представленный элемент\scnrolesign sc-структуры*, все элементы которого являются элементами sc-структуры*, не являющийся ни sc-отношением\scnrolesign, ни sc-связкой\scnrolesign этой sc-структуры*.}

	\scnheader{сущностное замыкание*}
	\scnexplanation{Сущностное замыкание* --- наименьшее надмножество* (структура*), в котором каждый элемент является элементарно представленным\scnrolesign.}

	\scnheader{сущностное замыкание\scnrolesign}
	\scnrelfrom{первый домен}{сущностное замыкание*}
	\scnrelfrom{второй домен}{сущностное замыкание}

	\scnheader{носитель сущностного замыкания\scnrolesign}
	\scnrelfrom{первый домен}{сущностное замыкание*}
	\scnrelfrom{второй домен}{непустое sc-множество}

	\scnheader{содержательное замыкание*}
	\scnexplanation{содержательное замыкание* --- наименьшее надмножество* (структура*), в котором каждый элемент является полносвязно представленным\scnrolesign}

	\scnheader{содержательное замыкание\scnrolesign}
	\scnrelfrom{первый домен}{содержательное замыкание*}
	\scnrelfrom{второй домен}{содержательное замыкание}

	\scnheader{носитель содержательного замыкания\scnrolesign}
	\scnrelfrom{первый домен}{содержательное замыкание*}
	\scnrelfrom{второй домен}{непустое sc-множество}

	\scnheader{sc-отношение сходства по слотовым отношениям*}
	\scnexplanation{sc-отношение сходства по слотовым sc-отношениям* --- sc-отношение, являющееся рефлексивным по этим слотовым отношениям, то есть для любого элемента, входящего в связку этого sc-отношения под одним из слотовых sc-отношений, найдется связка этого sc-отношения, в которую он входит под каждым из этих слотовых sc-отношений.}

	\scnheader{sc-отношение сходства по слотовым отношениям\scnrolesign}
	\scnrelfrom{первый домен}{sc-отношение сходства по слотовым отношениям*}
	\scnrelfrom{второй домен}{sc-отношение сходства по слотовым отношениям}

	\scnheader{слотовые отношения сходства sc-отношения\scnrolesign}
	\scnrelfrom{первый домен}{sc-отношение сходства по слотовым отношениям*}
	\scnrelfrom{второй домен}{слотовые отношения сходства sc-отношения}

	\scnheader{sc-отношение семантического сходства по слотовым отношениям*}
	\scnexplanation{sc-отношение семантического сходства по слотовым отношениям* --- sc-отношение сходства по слотовым sc-отношениям* si и sj, в котором каждый элемент под слотовым sc-отношением si, может быть преобразован к элементу синтаксического типа элемента под слотовым sc-отношением sj; два инцидентных sc-элемента под слотовым sc-отношением si, в рамках этого sc-отношения семантического сходства соответствуют инцидентным элементам соответственно под слотовым sc-отношением sj.}

	\scnheader{sc-отношение семантического сходства по слотовым отношениям\scnrolesign}
	\scnrelfrom{первый домен}{sc-отношение семантического сходства по слотовым отношениям*}
	\scnrelfrom{второй домен}{sc-отношение семантического сходства по слотовым отношениям}

	\scnheader{слотовые отношения семантического сходства sc-отношения\scnrolesign}
	\scnrelfrom{первый домен}{sc-отношение семантического сходства по слотовым отношениям*}
	\scnrelfrom{второй домен}{слотовые отношения семантического сходства sc-отношения}

	\scnheader{связная sc-структура*}
	\scnexplanation{Связная sc-структура* --- sc-структура*, являющаяся связной.}

	\scnheader{связная sc-структура\scnrolesign}
	\scnrelfrom{первый домен}{связная sc-структура*}
	\scnrelfrom{второй домен}{связное непустое sc-множество}

	\scnheader{носитель связной sc-структуры\scnrolesign}
	\scnrelfrom{первый домен}{связная sc-структура*}
	\scnrelfrom{второй домен}{непустое sc-множество}

	\scnheader{семантическое сходство sc-структур*}
	\scnidtf{семантическое подобие sc-структур*}
	\scnexplanation{Семантическое сходство sc-структур* --- связывает sc-множество sc-структур* с sc-структурой* sc-отношением семантического сходства по слотовым sc-отношениям si, sj так, что для каждой sc-структуры* из sc-множества найдется ее элемент и связка этого sc-отношения сходства, в которую он входит под слотовым sc-отношением si, а под слотовым sc-отношением sj входит элемент sc-структуры*, также для каждого элемента sc-структуры найдется связка этого sc-отношения сходства, в которую он входит под слотовым sc-отношением sj, а под слотовым sc-отношением si входит элемент sc-структуры* из sc-множества.}

	\scnheader{sc-отношение семантического сходства sc-структур\scnrolesign}
	\scnrelfrom{первый домен}{семантическое сходство sc-структур*}
	\scnrelfrom{второй домен}{sc-отношение семантического сходства по слотовым отношениям*}

	\scnheader{семантическое сходство sc-структур\scnrolesign}
	\scnrelfrom{первый домен}{семантическое сходство sc-структур*}
	\scnrelfrom{второй домен}{sc-структура семантического сходства sc-структур*}

	\scnheader{sc-структура семантического сходства sc-структур\scnrolesign}
	\scnrelfrom{первый домен}{sc-структура семантического сходства sc-структур*}
	\scnrelfrom{второй домен}{sc-структура семантического сходства sc-структур}

	\scnheader{множество семантически сходных sc-структур\scnrolesign}
	\scnrelfrom{первый домен}{sc-структура семантического сходства sc-структур*}
	\scnrelfrom{второй домен}{множество семантически сходных sc-структур}

	\scnheader{семантическое непрерывное сходство sc-структур*}
	\scnidtf{семантическое непрерывное подобие sc-структур*}
	\scnexplanation{Семантическое непрерывное сходство sc-структур* --- связывает sc-множество sc-структур* со связной sc-структурой* sc-отношением семантического сходства по слотовым sc-отношениям si, sj так, что для каждой sc-структуры* из sc-множества найдется ее элемент и связка этого sc-отношения сходства, в которую он входит под слотовым sc-отношением si, а под слотовым sc-отношением sj входит элемент связной sc-структуры*, также для каждого элемента связной sc-структуры найдется связка этого sc-отношения сходства, в которую он входит под слотовым sc-отношением sj, а под слотовым sc-отношением si входит элемент sc-структуры* из sc-множества.}

	\scnheader{sc-отношение семантического непрерывного сходства sc-структур\scnrolesign}
	\scnrelfrom{первый домен}{семантическое непрерывное сходство sc-структур*}
	\scnrelfrom{второй домен}{sc-отношение семантического непрерывного сходства по слотовым отношениям*}

	\scnheader{семантическое непрерывное сходство sc-структур\scnrolesign}
	\scnrelfrom{первый домен}{семантическое непрерывное сходство sc-структур*}
	\scnrelfrom{второй домен}{sc-структура семантического непрерывного сходства sc-структур*}

	\scnheader{sc-структура семантического непрерывного сходства sc-структур\scnrolesign}
	\scnrelfrom{первый домен}{sc-структура семантического непрерывного сходства sc-структур*}
	\scnrelfrom{второй домен}{sc-структура семантического непрерывного сходства sc-структур}

	\scnheader{множество семантически непрерывно сходных sc-структур\scnrolesign}
	\scnrelfrom{первый домен}{sc-структура семантического непрерывного сходства sc-структур*}
	\scnrelfrom{второй домен}{множество семантически непрерывно сходных sc-структур}

	\scnheader{ключевой запрос\scnrolesign}
	\scnrelfrom{первый домен}{ключевой запрос*}
	\scnrelfrom{второй домен}{ключевой запрос}
	\scnexplanation{Ключевой запрос\scnrolesign --- поисковый-проверочный запрос (от одного известного элемента), который выполняется хотя бы от одного элемента и не выполняется хотя бы от одного элемента.}

	\scnheader{элемент ключевого запроса\scnrolesign}
	\scnrelfrom{первый домен}{ключевой запрос*}
	\scnrelfrom{второй домен}{элемент ключевого запроса}

	\scnheader{минимальный ключевой запрос\scnrolesign}
	\scnsubset{ключевой запрос\scnrolesign}
	\scnexplanation{Минимальный ключевой запрос --- ключевой запрос, который находит sc-подмножества множеств элементов, находимых всеми другими ключевыми запросами, которые имеют те же области известных элементов выполнимости и невыполнимости.}

	\scnheader{элемент минимального ключевого запроса\scnrolesign}
	\scnrelfrom{первый домен}{минимальный ключевой запрос*}
	\scnrelfrom{второй домен}{элемент минимального ключевого запроса}

	\scnheader{полная семантическая окрестность элемента*}
	\scnexplanation{Полная семантическая окрестность элемента* --- все элементы, находимые выполнимыми минимальными ключевыми запросами от этого элемента (c учетом дизъюнктивного поиска и отрицания поиска).}

	\scnheader{полная семантическая окрестность элемента\scnrolesign}
	\scnrelfrom{первый домен}{полная семантическая окрестность элемента*}
	\scnrelfrom{второй домен}{полная семантическая окрестность элемента}

	\scnheader{элемент полной семантической окрестности\scnrolesign}
	\scnrelfrom{первый домен}{полная семантическая окрестность элемента*}
	\scnrelfrom{второй домен}{элемент полной семантической окрестности}

	\scnheader{интроспективный ключевой элемент\scnrolesign}
	\scnexplanation{интроспективный (базовый) ключевой элемент --- элемент множества (из класса наименьших таких множеств) элементов такого, что любая полная семантическая окрестность любого элемента является sc-подмножеством объединения их полных семантических окрестностей.}

	\scnheader{топологическое пространство замыкания инцидентности коннекторов}
	\scnexplanation{Топологическое пространство замыкания инцидентности коннекторов на множестве sc-элементов --- топологическое пространство, все замкнутые множества которого содержат все sc-элементы этого множества, до которых есть маршрут по ориентированным связкам отношения инцидентности коннекторов.}
	\scntext{примечание}{В общем случае не удовлетворяет аксиоме отделимости по Тихонову. Прагматика рассмотрения таких пространств обуславливается операциями удаления sc-элементов и коннекторов, которым они инцидентны. Удаление sc-элемента требует удаления всех коннекторов, замыканию любой открытой окрестности которых он принадлежит.}

	\scnheader{топологическое подпространство замыкания инцидентности коннекторов\scnrolesign}
	\scnrelfrom{первый домен}{включение топологических пространств замыкания инцидентности коннекторов*}
	\scnrelfrom{второй домен}{топологическое пространство замыкания инцидентности коннекторов}

	\scnheader{топологическое надпространство замыкания инцидентности коннекторов\scnrolesign}
	\scnrelfrom{первый домен}{включение топологических пространств замыкания инцидентности коннекторов*}
	\scnrelfrom{второй домен}{топологическое пространство замыкания инцидентности коннекторов}

	\scnheader{топологическое пространство синтаксического замыкания}
	\scnexplanation{Топологическое пространство синтаксического замыкания на множестве sc-элементов --- топологическое пространство, все замкнутые множества которого содержат все sc-элементы этого множества, до которых есть маршрут по ориентированным связкам отношения инцидентности.}
	\scntext{примечание}{В общем случае не удовлетворяет аксиоме отделимости по Колмогорову. В качестве основы замкнутых множеств топологического пространства можно выделить синтаксическое замыкание, однако в силу возможности проведения дуг из любого sc-узла в любой в итоговом случае (в итоге процесса устранения не-факторов) такое пространство является тривиальным (антидискретным) пространством. Отношение объединения топологических пространств синтаксического замыкания алгебраически не замкнуто на множестве топологических пространств синтаксического замыкания. По той же причине для любого неполного топологического пространства синтаксического замыкания можно рассмотреть топологическое пространство синтаксического замыкания, носитель которого является надмножеством носителя первого и которое не сохраняет замкнутые множества. В этом смысле топология на основе синтаксического замыкания не является устойчивой относительно процессов становления знаний и ее рассмотрение прагматически не оправдывается. Топология полного же топологического пространства синтаксического замыкания антидискретна (тривиальна). Таким образом, у полного топологического пространства синтаксического замыкания все топологические подпространства синтаксического замыкания обладают антидискретной (тривиальной) топологией.}

	\scnheader{топологическое пространство сущностного замыкания}
	\scnexplanation{Топологическое пространство сущностного замыкания на множестве sc-элементов --- топологическое пространство, все замкнутые множества которого являются сущностными замыканиями.}
	\scntext{примечание}{В общем случае не удовлетворяет аксиоме отделимости по Тихонову. В качестве носителя топологического (под)пространства можно выделить сущностное замыкание. Топологическое пространство сущностного замыкания сохраняет замкнутые множества любых топологических пространств сущностного замыкания, носитель которых является подмножеством его носителя и сущностным замыканием. Такие пространства образуют структуру топологических подпространств-топологических надпространств сущностного замыкания. Топология пространств в этой структуре разнообразна.}

	\scnheader{топологическое подпространство сущностного замыкания\scnrolesign}
	\scnrelfrom{первый домен}{включение топологических пространств сущностного замыкания*}
	\scnrelfrom{второй домен}{топологическое пространство сущностного замыкания}

	\scnheader{топологическое надпространство сущностного замыкания\scnrolesign}
	\scnrelfrom{первый домен}{включение топологических пространств сущностного замыкания*}
	\scnrelfrom{второй домен}{топологическое пространство сущностного замыкания}

	\scnheader{топологическое пространство содержательного замыкания}
	\scnexplanation{Топологическое пространство содержательного замыкания на множестве sc-элементов --- топологическое пространство, все замкнутые множества которого являются содержательными замыканиями.}
	\scntext{примечание}{В общем случае не удовлетворяет аксиоме отделимости по Тихонову. В качестве носителя топологического (под)пространства можно выделить содержательное замыкание. Топологическое пространство содержательного замыкания сохраняет замкнутые множества любых топологических пространств содержательного замыкания, носитель которых является подмножеством его носителя и содержательным замыканием. Такие пространства образуют структуру топологических подпространств-топологических надпространств содержательного замыкания. Топология пространств в этой структуре разнообразна.}
	
	\scnheader{топологическое подпространство содержательного замыкания\scnrolesign}
	\scnrelfrom{первый домен}{включение топологических пространств содержательного замыкания*}
	\scnrelfrom{второй домен}{топологическое пространство содержательного замыкания}


	\scnheader{топологическое надпространство содержательного замыкания\scnrolesign}
	\scnrelfrom{первый домен}{включение топологических пространств содержательного замыкания*}
	\scnrelfrom{второй домен}{топологическое пространство содержательного замыкания}
    \scntext{примечание}{Возможен переход от sc-структур к многообразиям и топологическим пространствам путем сведения sc-структур к графовым структурам.
        \\Для более сложных структур таких, как полная семантическая окрестность, топологические свойства подлежат дальнейшему изучению.
        \\Далее можно рассмотреть метрические пространства, в частности --- конечные подпространства с полностью представленными sc-элементами. }
        \begin{scnindent}
        	\scnrelfrom{смотрите}{\scncite{Ivashenko2022}}
        \end{scnindent}


	\scnheader{метрика}
	\scnexplanation{Метрика --- функция двух аргументов, принимающая значения на (линейно) упорядоченном носителе группы, неотрицательна, равна нейтральному элмененту (нулю) только при равенстве аргументов, симметрична, удовлетворяет неравенству треугольника.}

	\scnheader{метрическое пространство}
	\scnexplanation{Метрическое пространство --- множество, с определенной на нем функцией двух аргументов, являющейся метрикой, принимающей значения на упорядоченном носителе группы.}

	\scnheader{семантическая метрика}
	\scnidtf{семантическая близость}
	\scnexplanation{Семантическая метрика --- метрика, определенная на знаках и выражающая количественно близость их значений.}

	\scnheader{метрическое конечное синтаксическое пространство}
	\scnexplanation{Метрическое конечное синтаксическое пространство SC-кода --- метрическое пространство с конечным носителем, элементами которого являются обозначения (sc-элементы), а значение метрики может быть определено через отношения инцидентности элементов без учета их семантического типа.}

	\scnheader{метрическое конечное синтаксическое подпространство\scnrolesign}
	\scnrelfrom{первый домен}{включение метрических конечных синтаксических пространств*}
	\scnrelfrom{второй домен}{метрическое конечное синтаксическое пространство}

	\scnheader{метрическое конечное синтаксическое надпространство\scnrolesign}
	\scnrelfrom{первый домен}{включение метрических конечных синтаксических пространств*}
	\scnrelfrom{второй домен}{метрическое конечное синтаксическое пространство}

	\scnheader{метрическое конечное семантическое пространство}
	\scnexplanation{Метрическое конечное семантическое пространство SC-кода --- метрическое пространство с конечным носителем, элементами которого являются обозначения (sc-элементы), а значение метрики не может быть определено через отношения инцидентности элементов без учета их семантического типа.}

	\scnheader{метрическое конечное семантическое подпространство\scnrolesign}
	\scnrelfrom{первый домен}{включение метрических конечных семантических пространств*}
	\scnrelfrom{второй домен}{метрическое конечное семантическое пространство}

	\scnheader{метрическое конечное семантическое надпространство\scnrolesign}
	\scnrelfrom{первый домен}{включение метрических конечных семантических пространств*}
	\scnrelfrom{второй домен}{метрическое конечное семантическое пространство}

    \scntext{примечание}{Метрическое конечное синтаксическое пространство может быть построено в соответствии с моделью обработки строк и определениями метрики.}
	\begin{scnindent}
		\begin{scnrelfromset}{смотрите}
			\scnitem{\scncite{Ivashenko2022}}
			\scnitem{\scncite{Ivashenko2020String}}
		\end{scnrelfromset}
	\end{scnindent}
	
	\scnheader{псевдометрика}
	\scntext{пояснение}{Псевдометрика --- функция двух аргументов, принимающая значения на (линейно) упорядоченном носителе группы, неотрицательна, симметрична, удовлетворяет неравенству треугольника.}

	\scnheader{псевдометрическое пространство}
	\scntext{пояснение}{псевдометрическое пространство --- множество, с определенной на нем функцией двух аргументов, являющейся псевдометрикой, принимающей значения на упорядоченном носителе группы.}
	\begin{scnindent}
		\scnrelfrom{смотрите}{\scncite{Collatz1966}}
	\end{scnindent}

	\scnheader{псевдометрическое конечное семантическое пространство}
	\scntext{пояснение}{Псевдометрическое конечное семантическое пространство SC-кода --- псевдометрическое пространство с конечным носителем, элементами которого являются обозначения (sc-элементы), а значение псевдометрики не может быть определено через отношения инцидентности элементов без учета их семантического типа.}
    \begin{scnrelfromvector}{примечание}
    	\scnfileitem{В силу неполноты выразительных средств для представления изменяющихся со временем знаний, отсутствия определенной пространственно-временной модели, наличия семантически неопределенных или слабоопределенных обозначений в текстах да и наличия недоопределенности самих текстов описанного в предыдущих разделах языка, на данном этапе в этом описании затруднительно предложить какую-либо модель метрического пространства для более сложных структур, учитывающих не-факторы, связанные с пространством-временем.}
        \scnfileitem{Предложенные модели полагались на представление, способное выразить семантику переменных обозначений и операционную семантику расширенными средствами алфавита. Для построения подобных моделей, кроме расширенных средств алфавита, предлагается полагаться на модели, описывающие процессы интеграции и становления знаний на средства спецификации знаний, ориентированные на рассмотрение финитных структур, что позволяет перейти к рассмотрению сложных метрических соотношений в рамках метамодели смыслового пространства.}
		\begin{scnindent}
			\begin{scnrelfromset}{смотрите}
				\scnitem{\scncite{Ivashenko2022}}
				\scnitem{\scncite{Ivashenko2017}}
			\end{scnrelfromset}
		\end{scnindent}
        \scnfileitem{В разных науках исследователи затрагивали вопросы касающиеся смыслов и их размещения и взаимосвязи. Можно выделить следующие работы, которые соотносятся с тремя подходами: экстериорный подход, интериорные подходы на основании количественных и структурно-динамических признаков.}
        \begin{scnindent}
        	\begin{scnrelfromset}{смотрите}
        		\scnitem{\scncite{Bohm1993}}
        		\scnitem{\scncite{Nalimov1995}}
        		\scnitem{\scncite{Bohm2002}}
        		\scnitem{\scncite{Martynov2004}}
        		\scnitem{\scncite{Nalimov1979}}
        		\scnitem{\scncite{Nalimov1989}}
        	\end{scnrelfromset}
        \end{scnindent}
        \scnfileitem{В современных работах в технических науках, возможно, наиболее близкими понятиями являются понятия, выражающие смысл термина \scnqqi{семантическое пространство} (интериорный подход).
        Общим во многих подходах к работе с \scnqqi{семантическим пространством} является рассмотрение словоформ или лексем (множеств словоформ) и их признаков. В литературе встречаются следующие подходы:
        \begin{scnitemize}
            \item подход на основе семантических осей и пространства признаков (бинарных $\scnleftcurlbrace 0,1\scnrightcurlbrace \upperscore{n}$, монополярных $\scnleftsquarebrace 0;1\scnrightsquarebrace\upperscore{n}$, биполярных $\scnleftsquarebrace -1;1\scnrightsquarebrace\upperscore{n}$);
            \item подход на основе семантических осей и нейронного кодирования места в поле смыслов (слова и словосочетания имеют области (подмножества) значений, связываясь другими частями речи как включением и пересечением, тексты соответствуют пути связанных областей, бинарное кодирование групп нейронов, распознающих смыслы);
            \item подход на основе модели \scnqqi{смысл-текст} (отражение неполноты семантических шкал и анализ синтагм и поверхностно-синтаксической структуры);
            \item нейролингвистические данные отражает процессы синтеза и восприятия речи в нейронных сетях (сеть лексического синтеза), близка к модели \scnqqi{смысл-текст};
            \item модели, построенные на основе статического анализа (корпусов) текстов (модель векторного пространства).
        \end{scnitemize}}
    	\begin{scnindent}
    		\begin{scnrelfromset}{смотрите}
    			\scnitem{\scncite{Manin2016}}
    			\scnitem{\scncite{Melchuk2016}}
    			\scnitem{\scncite{Harris1992}}
    		\end{scnrelfromset}
    	\end{scnindent}
        \scnfileitem{Статистический подход к обработке естественного языка противопоставляется интуиции и коммуникативному опыту ученых. В основе подхода лежит семантическая статистическая гипотеза, что смысл слов (лексем) определяется контекстом использования (его статистическим образом) в языке (с коммуникативной структурой).}
		\begin{scnindent}
			\scnrelfrom{смотрите}{\scncite{Manin2016}}
		\end{scnindent}
        \scnfileitem{Модель векторного пространства семантики. Модель рассматривается для двух случаев: большого словаря ($N \leq M $) и задачи информационного поиска ($M \leq N $). $M$ --- размер словаря, $N$ --- количество контекстов.}
        \begin{scnindent}
        	\scnrelfrom{смотрите}{\scncite{Manin2016}}
        \end{scnindent}
		\scnfileitem{На основе статистики строится матрица размерности $ M \times N $ частот $ p\underscore{ij} $ появления лексемы (слова) $ w\underscore{i} $ в документе (контексте, подтексты, которые могут перекрываться) $ c\underscore{j} $ . 
			\\ $$ x\underscore{ij} \eq \max{\scnleftcurlbrace 0\scnrightcurlbrace \cup \scnleftcurlbrace \log (\frac{p\upperscore{ij}}{(\Sigma \upperscore{j} p\upperscore{ij}) * ( \Sigma \upperscore{i} p\upperscore{ij}) }) \scnrightcurlbrace} $$ }
		\scnfileitem{В знаменателе --- оценки вероятности слова и контекста соответственно.
			\\В случае невырожденной матрицы $r \eq N$ каждая такая матрица задает точку в грассманиане $N$-мерных подпространств $M$-мерного пространства ($ N \leq M $).
			\\В случае невырожденной матрицы $r \eq M$ каждая такая матрица задает точку в грассманиане $M$-мерных подпространств $N$-мерного пространства ($ M \leq N $). }
		\scnfileitem{Каждый текст --- точка в грассманиане, соответствующем проективному пространству $P\upperscore{M-1} \eq Gr(\langle 1,M \rangle )$, относительно одного выделенного контекста. Для всех контекстов получая ориентированную $N$-ку, в соответствии с порядком контекстов в текстах, можно построить маршрут (путь), соединяя геодезическими соседние точки в $N$-ке. Для двух текстов $T$ и $T\scnrolesign$ это будут две ломанные, между которыми можно вычислить метрику Фреше, используя метрику Фубини-Штуди в $P\upperscore{M-1}$, для этого следует параметризовать пути $\Gamma( T )$ и $\Gamma( T\scnrolesign )$ через $t$ ($\gamma\in\Gamma( T )\upperscore{\scnleftsquarebrace 0;1\scnrightsquarebrace }$, $\gamma \scnrolesign\in\Gamma( T\scnrolesign)\upperscore{\scnleftsquarebrace 0;1\scnrightsquarebrace }$):  $$ \delta( \langle \Gamma ( T ),\Gamma ( T\scnrolesign)\rangle ) \eq\inf\upperscore{\gamma,\gamma\scnrolesign}\max{t\in\scnleftsquarebrace 0;1\scnrightsquarebrace }(  \scnleftcurlbrace d\upperscore{FS}( \langle \gamma( t) ,\gamma\scnrolesign( t) \rangle ) \scnrightcurlbrace ). }
        \begin{scnindent}
        	\begin{scnrelfromset}{смотрите}
        		\scnitem{\scncite{Alt1995}}
        		\scnitem{\scncite{Study1905}}
        		\scnitem{\scncite{Harris1992}}
        	\end{scnrelfromset}
        \end{scnindent}
        \scnfileitem{Другой способ задать линейный порядок --- это рассмотреть фильтрацию в $\mathbb{R}\upperscore{M}$, заданную расширяющимися контекстами. В итоге для текста получаем точки (флаги) во флаговом многообразии. Для флаговых многообразий тоже можно вычислить метрику Фубини-Штуди.}
        \begin{scnindent}
        	\begin{scnrelfromset}{смотрите}
        		\scnitem{\scncite{Kostrikin1997}}
        		\scnitem{\scncite{Study1905}}
        	\end{scnrelfromset}
        \end{scnindent}
        \scnfileitem{Этот порядок соответствует временному измерению (процессу коммуникации во времени), что может быть существенным. Другой порядок может быть не зависимым от этого, например алфавитный или порядок в соответствии с законом Ципфа.}
        \begin{scnindent}
        	\begin{scnrelfromset}{смотрите}
        		\scnitem{\scncite{Lowe2001}}
        		\scnitem{\scncite{Manin2014}}
        	\end{scnrelfromset}
        \end{scnindent}
	\end{scnrelfromvector}

	\scnheader{Таблица. Сравнение подходов к построению \scnqqi{семантических пространств}}
	\scnrelfrom{описание примера}{\scnfileimage[35em]{Contents/part_kb/src/images/sd_lang/comparison_table.png}}
	\scntext{примечание}{Вопросы соотнесения смыслов, их формализации, развития языков в пространстве и времени.}
	\begin{scnrelfromset}{смотрите}
		\scnitem{\scncite{Gordey2014}}
		\scnitem{\scncite{Martynov2009}}
		\scnitem{\scncite{Martynov2004}}
	\end{scnrelfromset}

\end{scnsubstruct}

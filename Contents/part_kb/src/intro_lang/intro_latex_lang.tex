\scnheader{Язык представления исходных текстов баз знаний на основе языка LaTeX}
\scntext{пояснение}{Одним из удобных вариантов представления исходных текстов баз знаний для различных областей научно-технической деятельности является Язык представления исходных текстов баз знаний на основе языка LaTeX. В его основу положен язык LaTeX, так как:
    \begin{scnitemize}
        \item LaTeX это популярный общепринятый язык для записи научно-технических текстов;
        \item LaTeX представляет собой достаточно мощный и легко расширяемый язык;
        \item LaTeX это достаточно строгий и формальный язык, что позволяет реализовать для него транслятор исходных текстов в базу знаний интеллектуальной системы.
    \end{scnitemize}}
\scntext{цель}{Язык представления исходных текстов баз знаний на основе языка LaTeX был разработан для:
    \begin{scnitemize}
        \item формирования читабельного текста для публикации всего текста \textit{Стандарта OSTIS} или фрагментов в виде печатных изданий;
        \item возможности иметь формальный исходный текст \textit{Стандарта OSTIS}, который может быть протранслирован в базу знаний.
    \end{scnitemize}}
\scntext{примечание}{Предлагаемый набор команд условно называется scn-tex, поскольку в его основу положена идея того, чтобы разработчик писал исходный текст максимально приближенно к тому, как он будет видеть результат компиляции этого исходного текста в SCn-коде и при этом максимально был избавлен от необходимости учитывать особенности языка LaTeX в работе.}
\begin{scnindent}
	\scnrelfrom{смотрите}{\scncite{Ostis-scn-latex-plugin2023}}
\end{scnindent}
\begin{scnrelfromlist}{принципы}
	\scnfileitem{Весь исходный текст стандарта формируется исключительно с использованием набора команд scn-tex.}
	\scnfileitem{Запрещается использовать любые другие команды для форматирования текста, изменения шрифта, вставки внешних файлов и т.д.}
	\scnfileitem{В рамках естественно-языковых фрагментов, входящих в состав стандарта, допускается использование команд LaTeX для вставки специальных символов и математических формул.}
	\scnfileitem{Для добавления файлов изображений в текст стандарта используются только команды scn-tex.}
	\scnfileitem{Используются для формирования нумерованных и маркированных списков, добавления закрывающих и открывающих скобок различного вида (кроме круглых) используются только команды scn-tex.}
	\scnfileitem{Для выделения курсивом идентификаторов в рамках естественно-языковых фрагментов, входящих в состав стандарта, используется только команда \textbackslash textit\{\}.}
	\scnfileitem{Для выделения полужирным курсивом используется комбинация команд \textbackslash textbf\{\textbackslash textit\{\}\}.}
\end{scnrelfromlist}
\scntext{примечание}{Каждая команда из набора scn-tex начинается с префикса \textbackslash scn, после которого идет имя команды, примерно описывающее то, как связан текущий отображаемый фрагмент текста с описываемой сущностью.
    \\Например:
    \begin{scnitemize}
        \item \textbackslash scnrelfrom --- дуга ориентированного отношения, которая выходит из описываемой сущности в другую сущность
        \item \textbackslash scnrelto --- дуга ориентированного отношения, которая входит в описываемую сущность
        \item \textbackslash scnnote --- естественно-языковое примечание к описываемой сущности
    \end{scnitemize}
    Полный перечень команд можно увидеть в файле scn.sty, а примеры использования команд каждого типа --- в исходных текстах стандарта.}
\begin{scnindent}
	\scnrelfrom{смотрите}{\scncite{Ostis-scn-latex-plugin2023}}
\end{scnindent}
\scntext{примечание}{Для формирования отступов для корректного отображения sc.n-текстов используется окружение \textbackslash begin\{scnindent\} --- \textbackslash end\{scnindent\}. После смещения на определенное число уровней вправо следует смещение на то же число уровней влево.}
\begin{scnrelfromlist}{пример}
\scnfileitem{входная конструкция}
    {
        \begin{verbatim}
        \scnheader{кибернетическая система}
        \scnsuperset{компьютерная система}
        \begin{scnindent}
            \scnidtf{искусственная кибернетическая система}
            \scnsuperset{ostis-система}
            \begin{scnindent}
                \scnidtf{компьютерная система, построенная по технологии OSTIS на основе
                интерпретации спроектированной логико-семантическая sc-модель этой системы}
            \end{scnindent}
        \end{scnindent}
        \end{verbatim}
    }
\scnfileitem{выходная конструкция}
	{
	\scnheader{кибернетическая система}
	\scnsuperset{компьютерная система}
	\begin{scnindent}
	    \scnidtf{искусственная кибернетическая система}
	    \scnsuperset{ostis-система}
	    \begin{scnindent}
	        \scnidtf{компьютерная система, построенная по технологии OSTIS на основе интерпретации спроектированной логико-семантическая sc-модель этой системы}
	    \end{scnindent}
	\end{scnindent}
	}
\end{scnrelfromlist}

\scnheader{scn-tex}
\scntext{примечание}{Команды из набора scn-tex делятся на классы. К чаще всего используемым классам относятся окружения (environments) и списки (lists), которые являются частным случаем окружения. Окружения могут быть вложенными.}
\begin{scnrelfromlist}{пример}
	\scnfileitem{входная конструкция}
	    {
		\begin{verbatim}
		\scnheader{Логико-семантическая модель Метасистемы IMS.ostis}
		\scnrelfrom{примечание}{
			\begin{scnset}
			\scnfilelong{IMS.ostis}
			\scnrelto{сокращение}{\scnfilelong{Метасистема IMS.ostis}}
			\begin{scnindent}
			\scnrelto{сокращение}{\scnfilelong{Intelligent MetaSystem of Open Semantic Technology
			 for Intelligent Systems}}
			\end{scnindent}
			\end{scnset}
			}
		\scnidtf{Логико-семантическая модель интеллектуального ostis-портала научно-технических
		 знаний по Технологии OSTIS}
		\end{verbatim}
		}
\scnfileitem{выходная конструкция}
	{
		\scnheader{Логико-семантическая модель Метасистемы IMS.ostis}
		\scnrelfrom{примечание}{
			\begin{scnset}
			\scnfilelong{IMS.ostis}
			\scnrelto{сокращение}{\scnfilelong{Метасистема IMS.ostis}}
			\begin{scnindent}
			\scnrelto{сокращение}{\scnfilelong{Intelligent MetaSystem of Open Semantic Technology for Intelligent Systems}}
			\end{scnindent}
			\end{scnset}
			}
		\scnidtf{Логико-семантическая модель интеллектуального ostis-портала научно-технических знаний по Технологии OSTIS}
	}
\end{scnrelfromlist}
\scntext{примечание}{Таким образом scn-tex позволяет записать практически любую синтаксически корректную конструкцию SCn-кода.}

\scnheader{tex2scs-translator}
\scntext{примечание}{Для трансляции исходного текста в базу знаний был разработан tex2scs-translator, который переводит фрагменты scn-tex в SCs-код. Каждой scn-tex команде соответствует определенная синтаксическая конструкция SCs-кода. Таким образом, весь исходный текст Стандарта OSTIS, записанный в scn-tex, может быть протранслирован в базу знаний любой ostis-системы, которая поддерживает сборку базы знаний из sc.s-файлов. Далее рассмотрим пример работы транслятора.}
\begin{scnindent}
	\scnrelfrom{смотрите}{\scncite{Ostis-tex2scs-translator2023}}
\end{scnindent}
\begin{scnrelfromlist}{пример}
	\scnfileitem{входная конструкция}
	{
		\begin{verbatim}
			\scnheader{множество}
			\begin{scnrelfromset}{разбиение}
				\scnitem{конечное множество}
				\scnitem{бесконечное множество}
			\end{scnrelfromset}
		\end{verbatim}
	}
	\scnfileitem{выходная конструкция}
	{
		\begin{verbatim}
			.system_element_0
		=> nrel_subdividing: {
			.system_element_1;
			.system_element_2
		};;

		.system_element_1 => nrel_main_idtf: [конечное множество] (* <- lang_ru;;
		 => nrel_format: format_html;; *);;
		.system_element_1 <- sc_node;;
		
		nrel_subdividing => nrel_main_idtf: [разбиение*] (* <- lang_ru;; =>
		nrel_format: format_html;; *);;
		nrel_subdividing <- sc_node_norole_relation;;

		.system_element_2 => nrel_main_idtf: [бесконечное множество] (* <- lang_ru;;
		 => nrel_format: format_html;; *);;
		.system_element_2 <- sc_node;;
		
		.system_element_0 => nrel_main_idtf: [множество] (* <- lang_ru;; => nrel_format:
		 format_html;; *);;
		.system_element_0 <- sc_node;;

		\end{verbatim}
	}
\end{scnrelfromlist}

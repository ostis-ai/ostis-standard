\begin{SCn}
\scnsectionheader{Раздел. Предметная область и онтология базовых спецификаций}

% Если найдется, то добавлю
% \begin{scnrelfromlist}{эпиграф}
% 	\scnfileitem{}
% \end{scnrelfromlist}

\begin{scnrelfromlist}{предисловие}
	\scnfileitem{В настоящее время из-за большого количества информации появляется необходимость в её структуризации, но из-за недобросовестного оношения к этому делу базы знаний становятся сложно обрабатываемыми. Причиной это является то, что нет единого наглядного примера или некоторой инструкции по спецификации понятий в базе знаний.}
\end{scnrelfromlist}

\begin{scnrelfromlist}{аннотация}
	\scnfileitem{В рамках данного раздела базы знаний описаны базовые спецификации основых классом сущностей базы знаний, уточнены констукции для базовой спецификации сущностей и приведены примеры хорошо описанных сущностей, на которые можно ровняться в процессе дополнения базы знаний новыми понятиями.}
\end{scnrelfromlist}

\begin{scnrelfromlist}{основная цель}
	\scnfileitem{Создание единой формы спецификации сущностей, которые входят в основные классы, существующие в базе знаний}
\end{scnrelfromlist}

\begin{scnrelfromlist}{автор}
	\scnitem{Петрочук К.Д}
	\scnitem{Гракова Н.В.}
	\scnitem{Голенков В.В.}
\end{scnrelfromlist}

\begin{scnrelfromlist}{рассматриваемый вопрос}
	\scnfileitem{Что такое базовая спецификация?}
	\scnfileitem{Для каких классов сущностей можно описать базовые спецификации?}
	\scnfileitem{Какие конструкции должны входить в базовую спецификацию различных классов сущностей?}
\end{scnrelfromlist}

\begin{scnrelfromlist}{рассматриваемый вопрос}
	\scnitem{Раздел. Предметная область семантических окрестностей}
\end{scnrelfromlist}

\begin{scnrelfromlist}{ключевое понятие}
	\scnitem{базовая спецификация}
	\scnitem{класс сущностей, имеющих унифицированную базовую спецификацию}
	\scnitem{базовая спецификация раздела базы знаний}
	\scnitem{базовая спецификация предметной области}
	\scnitem{базовая спецификация проекта}
	\scnitem{базовая спецификация многократно используемого компонента}
	\scnitem{базовая спецификация ostis-системы}
	\scnitem{базовая спецификация персоны}
	\scnitem{базовая спецификация агента}
	\scnitem{базовая сепецификация отношения}
	\scnitem{базовая спецификация*}
	\scnitem{обобщенная базовая спецификация*}
\end{scnrelfromlist}

\begin{scnsubstruct}

\scnheader{Предметная область базовых спецификацией}
\scniselement{предметная область}
		\begin{scnrelto}{частная предметная область}
			{Предметная область семантических окресностей}
		\end{scnrelto}
        \begin{scnhaselementrole}{максимальный класс объектов исследования}
            {базовая спецификация}
        \end{scnhaselementrole}
        \begin{scnhaselementrolelist}{класс объектов исследования}
			\scnitem{класс сущностей, имеющих унифицированную базовую спецификацию}
			\scnitem{базовая спецификация раздела базы знаний}
			\scnitem{базовая спецификация предметной области}
			\scnitem{базовая спецификация проекта}
			\scnitem{базовая спецификация многократно используемого компонента}
			\scnitem{базовая спецификация ostis-системы}
			\scnitem{базовая спецификация персоны}
			\scnitem{базовая спецификация агента}
			\scnitem{базовая сепецификация отношения}
			
        \end{scnhaselementrolelist}
		\begin{scnhaselementrolelist}{исследуемое отношение}
            \scnitem{базовая спецификация*}
			\scnitem{обобщенная базовая спецификация*}
        \end{scnhaselementrolelist}

	\scnheader{класс сущностей, имеющих унифицированную базовую спецификацию}
	\scnidtf{класс, для всех сущностей которого можно выделить общий набор свойств, необходимых для базового описания каждой сущности данного класса}
	\scntext{примечание}{В некоторых классах сущностей можно выделать подклассы, для которых могут быть описаны дополнительные базовые спецификации, которые будут характерны тольно для конкретного подкласса}
	\scnhaselement{базовая спецификация}
	\scnhaselement{раздел базы знаний}
	\scnhaselement{предметная область}
	\scnhaselement{проект}
	\scnhaselement{многократно используемый компонент}
	\scnhaselement{ostis-система}
	\scnhaselement{персональная информация}
	\scnhaselement{агент}
	\scnhaselement{отношение}

    \scnheader{базовая спецификация}
	\scnsubset{спецификация}
	\scnidtf{набор основных свойств некоторых сущностей, принадлежащих одному классу}
	\scnidtf{минимальный набор свойств, который необходим для описания каждой сущности, принадлежащей одному класса}
	\scntext{примечание}{У каждой сущности, принадлежащей определённому классу, должен быть описан базовый набор свойст, характерных данному классу}
	\scntext{примечание}{Базовые спецификации содержат рекомендуемый минимум свойст для спецификации сущностей каждого класса, но не являются строго обязательными. На них следует опираться во время спецификации сущностей для упрощения процесса спецификации и для того, чтобы информазия в базе знаний была структурирована и не происходило наполнение базы знаний разными понятиями, имеющими одинаковый смысл.}
	\scnsuperset{базовая спецификация базовой спецификации}
	\scnsuperset{базовая спецификация раздела базы знаний}
	\scnsuperset{базовая спецификация предметной области}
	\scnsuperset{базовая спецификация проекта}
	\scnsuperset{базовая спецификация многократно используемого компонента}
	\scnsuperset{базовая спецификация ostis-системы}
	\scnsuperset{базовая спецификация персоны}
    \scnsuperset{базовая спецификация агента}
    \scnsuperset{базовая спецификация отношения}

% \\\\\\\\\\\\ БАЗОВАЯ СПЕЦИФИКАЦИЯ
	\scnheader{базовая спецификация базовой спецификации}
	\begin{scnrelto}{обобщенная базовая спецификация}
		{базовая спецификация}
	\end{scnrelto}
	\begin{scnrelfromvector}{обобщенная декомпозиция}
		\scnitem{указание обобщенной базовой спецификации}
		\scnitem{указание обобщенной декомпозиции}
		\scnitem{указание примера}
	\end{scnrelfromvector}
	\scnrelfrom{пример}{базовая спецификация базовой спецификации предметной области}
		\begin{scnindent}
			\begin{scnrelto}{базовая спецификация}
				{базовая спецификация предметной области}
			\end{scnrelto}
		\end{scnindent}
	
	\scnheader{указание обобщенной базовой спецификации}
	\begin{scnrelfromset}{обобщенная декомпозиция}
		\scnitem{знак специфицируемого объекта}
		\scnitem{класс сущностей, имеющих унифицированную базовую спецификацию}
		\scnitem{дуга, связывающая класс сущностей, имеющих унифицированную базовую спецификацию, со специфицируемым объектом}
		\scnitem{дуга принадлежности отношению \textit{обобщeнная базовая спецификация*}}
		\scnitem{обобщeнная базовая спецификация*}
	\end{scnrelfromset}

	\scnheader{обобщeнная базовая спецификация*}
	\scnsubset{отношение}
	\scnsubset{бинарное отношение}
	\scnsubset{неролевое отношение}
	\scnsubset{ориентированное отношение}
	\scnsubset{антисимметричное отношение}
	\scnsubset{антитранзитивное отношение}
	\scnsubset{антирефлексивное отношение}
	\begin{scnrelfrom}{первый домен}
		{класс сущностей, имеющих унифицированную базовую спецификацию}
	\end{scnrelfrom}
	\begin{scnrelfrom}{второй домен}
		{базовая спецификация}
	\end{scnrelfrom}
	\begin{scnrelfromset}{область определения}
		\scnitem{класс сущностей, имеющих унифицированную базовую спецификацию}
		\scnitem{базовая спецификация}
	\end{scnrelfromset}
	\scntext{определение}{\textit{обобщeнная базовая спецификация*} - бинарное неролевое отношение, связывающее базовую спецификацию и с конкретным классом, имеющим базовую спецификацию. Оно обозначает, что у этого класса есть определённый набор базовых свойств для спецификации сущностей данного класса.}
		\begin{scnindent}
			\begin{scnrelfromlist}{используемые константы}
				\scnitem{отношение}
				\scnitem{бинарное отношение}
				\scnitem{неролевое отношение}
				\scnitem{базовая спецификация}
				\scnitem{класс сущностей, имеющих унифицированную базовую спецификацию}
				\scnitem{свойство}
				\scnitem{сущность}
				\scnitem{спецификация}
			\end{scnrelfromlist}
		\end{scnindent}

	\scnheader{базовая спецификация*}
	\scnsubset{отношение}
	\scnsubset{бинарное отношение}
	\scnsubset{неролевое отношение}
	\scnsubset{ориентированное отношение}
	\scnsubset{антисимметричное отношение}
	\scnsubset{антитранзитивное отношение}
	\scnsubset{антирефлексивное отношение}
	\begin{scnrelfrom}{первый домен}
		{класс сущностей, имеющих унифицированную базовую спецификацию}
	\end{scnrelfrom}
	\begin{scnrelfrom}{второй домен}
		{базовая спецификация}
	\end{scnrelfrom}
	\begin{scnrelfromset}{область определения}
		\scnitem{класс сущностей, имеющих унифицированную базовую спецификацию}
		\scnitem{базовая спецификация}
	\end{scnrelfromset}
	\scntext{определение}{\textit{базовая спецификация*} - бинарное неролевое отношение, связывающее некоторую сущность класса сущностей, имеющих унифицированную базовую спецификацию, и базовую спецификацию этой сущности. Данной отношение показывает, какая часть спецификации является базовой спецификацией для данной сущности.}
		\begin{scnindent}
			\begin{scnrelfromlist}{используемые константы}
				\scnitem{отношение}
				\scnitem{бинарное отношение}
				\scnitem{неролевое отношение}
				\scnitem{базовая спецификация}
				\scnitem{класс сущностей, имеющих унифицированную базовую спецификацию}
				\scnitem{сущность}
				\scnitem{спецификация}
			\end{scnrelfromlist}
		\end{scnindent}
	
	% есть вопросы относительно правильности выделения второго элемента
	\scnheader{указание обобщенной декомпозиции}
	\begin{scnrelfromset}{обобщенная декомпозиция}
		\scnitem{знак специфицируемого объекта}
		\scnitem{знак объекта обобщенной декомпозиции}
		\scnitem{дуга, связывающая специфицируемый объект с объектом обобщенной декомпозиции}
		\scnitem{дуга принадлежности отношению \textit{обобщенная декомпозиция*}}
		\scnitem{обобщенная декомпозиция*}
	\end{scnrelfromset}

	\scnheader{указание примера}
	\begin{scnrelfromset}{обобщенная декомпозиция}
		\scnitem{знак специфицируемого объекта}
		\scnitem{пример}
		\scnitem{дуга, связывающая специфицируемый объект с примером}
		\scnitem{дуга принадлежности отношению \textit{пример*}}
		\scnitem{пример*}
	\end{scnrelfromset}

% есть предположение, что данное отношение должно быть ролевым, так как среди всего множества объектов некоторого множества необходимо выделить один (или несколько) которые являются "примером для всех остальных"
	\scnheader{пример*}
	\scnsubset{отношение}
	\scnsubset{бинарное отношение}
	\scnsubset{неролевое отношение} 
	\scnsubset{ориентированное отношение}
	\scnsubset{антисимметричное отношение}
	\scnsubset{антитранзитивное отношение}
	\scnsubset{антирефлексивное отношение}
	\begin{scnrelfrom}{первый домен}
		{сущность}
	\end{scnrelfrom}
	\begin{scnrelfrom}{второй домен}
		{пример}
	\end{scnrelfrom}
	\begin{scnrelfromset}{область определения}
		\scnitem{сущность}
		\scnitem{пример}
	\end{scnrelfromset}
	\scntext{определение}{\textit{пример*} - бинарное неролевое отношение, связывающее некоторую сущность базы знаний и её конкретный пример данной сущности. Обозначающее что данный пример является наилучше передающим все свойства описываемой сущности.}
		\begin{scnindent}
			\begin{scnrelfromlist}{используемые константы}
				\scnitem{отношение}
				\scnitem{бинарное отношение}
				\scnitem{неролевое отношение}
				\scnitem{пример}
				\scnitem{свойство}
				\scnitem{сущность}
				\scnitem{спецификация}
			\end{scnrelfromlist}
		\end{scnindent}

% \\\\\\\\\\\\\\\\\ РАЗДЕЛ
	\scnheader{базовая спецификация раздела базы знаний}
	\begin{scnrelto}{обобщенная базовая спецификация}
		{раздел базы знаний}
	\end{scnrelto}
	\begin{scnrelfromvector}{обобщенная декомпозиция}
		\scnitem{указание эпиграфа}
		\scnitem{указание аннотации}
		\scnitem{указание предисловия}
		\scnitem{указание основной цели}
		\scnitem{указание автора}
		\scnitem{указание подразделов}
		\scnitem{указание предметной области}
		\scnitem{указание рассматриваемого вопроса}
		\scnitem{указание ключевых элементов}
	\end{scnrelfromvector}
	% \scnrelfrom{пример}{базовая спецификация раздела предметной области и онтологии}
	% 	\begin{scnindent}
	% 		\begin{scnrelto}{базовая спецификация}
	% 			{Раздел. Предметная область и онтология}
	% 		\end{scnrelto}
	% 	\end{scnindent}

	\scnheader{указание эпиграфа}
	\begin{scnrelfromset}{обобщенная декомпозиция}
		\scnitem{знак специфицируемого объекта}
		\scnitem{эпиграф}
		\scnitem{дуга, связывающая специфицируемый объект с эпиграфом}
		\scnitem{дуга принадлежности отношению \textit{эпиграф*}}
		\scnitem{эпиграф*}
	\end{scnrelfromset}

 % тут и в нескольких следующих отношениях в общем случаем может быть не только раздел, но и другие части базы знаний, но я не знаю, какие классы выделить
	\scnheader{эпиграф*}
	\scnsubset{отношение}
	\scnsubset{бинарное отношение}
	\scnsubset{неролевое отношение}
	\scnsubset{ориентированное отношение}
	\scnsubset{антисимметричное отношение}
	\scnsubset{антитранзитивное отношение}
	\scnsubset{антирефлексивное отношение}
	\begin{scnrelfrom}{первый домен}
		{раздел базы знаний}
	\end{scnrelfrom}
	\begin{scnrelfrom}{второй домен}
		{эпиграф}
	\end{scnrelfrom}
	\begin{scnrelfromset}{область определения}
		\scnitem{раздел базы знаний}
		\scnitem{эпиграф}
	\end{scnrelfromset}
	\scntext{определение}{\textit{эпиграф*} - бинарное неролевое отношение, связывающее раздел базы знаний и эпиграф, обозначающее, что данное высказывание является эпиграформ у данному разделу и передаю основную мысль автора данного раздела.}
		\begin{scnindent}
			\begin{scnrelfromlist}{используемые константы}
				\scnitem{отношение}
				\scnitem{бинарное отношение}
				\scnitem{неролевое отношение}
				\scnitem{эпиграф}
				\scnitem{раздел базы знаний}
				\scnitem{высказывание}
				\scnitem{мысль}
				\scnitem{автор}
			\end{scnrelfromlist}
		\end{scnindent}

	\scnheader{указание аннотации}
	\begin{scnrelfromset}{обобщенная декомпозиция}
		\scnitem{знак специфицируемого объекта}
		\scnitem{аннотация}
		\scnitem{дуга, связывающая специфицируемый объект с аннотацией}
		\scnitem{дуга принадлежности отношению \textit{аннотация*}}
		\scnitem{аннотация*}
	\end{scnrelfromset}

	\scnheader{аннотация*}
	\scnsubset{отношение}
	\scnsubset{бинарное отношение}
	\scnsubset{неролевое отношение}
	\scnsubset{ориентированное отношение}
	\scnsubset{антисимметричное отношение}
	\scnsubset{антитранзитивное отношение}
	\scnsubset{антирефлексивное отношение}
	\begin{scnrelfrom}{первый домен}
		{раздел базы знаний}
	\end{scnrelfrom}
	\begin{scnrelfrom}{второй домен}
		{аннотация}
	\end{scnrelfrom}
	\begin{scnrelfromset}{область определения}
		\scnitem{раздел базы знаний}
		\scnitem{аннотация}
	\end{scnrelfromset}
	\scntext{определение}{\textit{аннотация*} - бинарное неролевое отношение, связывающее раздел базы знаний и аннотация, обозначающее, что данный фрагмент текста является аннотацией к данному разделу и содержит краткое описание данного раздела.}
		\begin{scnindent}
			\begin{scnrelfromlist}{используемые константы}
				\scnitem{отношение}
				\scnitem{бинарное отношение}
				\scnitem{неролевое отношение}
				\scnitem{раздел базы знаний}
				\scnitem{аннотация}
				\scnitem{описание}
				\scnitem{текст}
			\end{scnrelfromlist}
		\end{scnindent}

	\scnheader{указание предисловия}
	\begin{scnrelfromset}{обобщенная декомпозиция}
		\scnitem{знак специфицируемого объекта}
		\scnitem{предисловие}
		\scnitem{дуга, связывающая специфицируемый объект с предисловием}
		\scnitem{дуга принадлежности отношению \textit{предисловие*}}
		\scnitem{предисловие*}
	\end{scnrelfromset}

	\scnheader{предисловие*}
	\scnsubset{отношение}
	\scnsubset{бинарное отношение}
	\scnsubset{неролевое отношение}
	\scnsubset{ориентированное отношение}
	\scnsubset{антисимметричное отношение}
	\scnsubset{антитранзитивное отношение}
	\scnsubset{антирефлексивное отношение}
	\begin{scnrelfrom}{первый домен}
		{раздел базы знаний}
	\end{scnrelfrom}
	\begin{scnrelfrom}{второй домен}
		{предисловие}
	\end{scnrelfrom}
	\begin{scnrelfromset}{область определения}
		\scnitem{предисловие}
		\scnitem{раздел базы знаний}
	\end{scnrelfromset}
	\scntext{определение}{\textit{предисловие*} - бинарное неролевое отношение, связывающее раздел базы знаний и предисловие, обозначающее, что данный фрагмент текста является предисловием к данному разделу.}
		\begin{scnindent}
			\begin{scnrelfromlist}{используемые константы}
				\scnitem{отношение}
				\scnitem{бинарное отношение}
				\scnitem{неролевое отношение}
				\scnitem{предисловие}
				\scnitem{раздел базы знаний}
			\end{scnrelfromlist}
		\end{scnindent}

	\scnheader{указание основной цели}
	\begin{scnrelfromset}{обобщенная декомпозиция}
		\scnitem{знак специфицируемого объекта}
		\scnitem{цель}
		\scnitem{дуга, связывающая специфицируемый объект с целью}
		\scnitem{дуга принадлежности отношению \textit{основная цель*}}
		\scnitem{основная цель*}
	\end{scnrelfromset}

	\scnheader{основная цель*}
	\scnsubset{отношение}
	\scnsubset{бинарное отношение}
	\scnsubset{неролевое отношение}
	\scnsubset{ориентированное отношение}
	\scnsubset{антисимметричное отношение}
	\scnsubset{антитранзитивное отношение}
	\scnsubset{антирефлексивное отношение}
	\begin{scnrelfrom}{первый домен}
		{раздел базы знаний}
	\end{scnrelfrom}
	\begin{scnrelfrom}{второй домен}
		{цель}
	\end{scnrelfrom}
	\begin{scnrelfromset}{область определения}
		\scnitem{раздел базы знаний}
		\scnitem{цель}
	\end{scnrelfromset}
	\scntext{определение}{\textit{основная цель*} - бинарное неролевое отношение, связывающее раздел базы знаний и некоторую цель, обозначающее, что данная цель является основной для данного раздела.}
		\begin{scnindent}
			\begin{scnrelfromlist}{используемые константы}
				\scnitem{отношение}
				\scnitem{бинарное отношение}
				\scnitem{неролевое отношение}
				\scnitem{раздел базы знаний}
				\scnitem{цель}
			\end{scnrelfromlist}
		\end{scnindent}

	\scnheader{указание автора}
	\begin{scnrelfromset}{обобщенная декомпозиция}
		\scnitem{знак специфицируемого объекта}
		\scnitem{персона}
		\scnitem{дуга, связывающая специфицируемый объект с персоной}
		\scnitem{дуга, связывающая специфицируемый объект с }
		\scnitem{дуга принадлежности отношению \textit{автор*}}
		\scnitem{автор*}
	\end{scnrelfromset}

	\scnheader{автор*}
	\scnsubset{отношение}
	\scnsubset{бинарное отношение}
	\scnsubset{неролевое отношение}
	\scnsubset{ориентированное отношение}
	\scnsubset{антисимметричное отношение}
	\scnsubset{антитранзитивное отношение}
	\scnsubset{антирефлексивное отношение}
	\begin{scnrelfrom}{первый домен}
		{сущность}
	\end{scnrelfrom}
	\begin{scnrelfrom}{второй домен}
		{персона}
	\end{scnrelfrom}
	\begin{scnrelfromset}{область определения}
		\scnitem{сущность}
	\end{scnrelfromset}
	\scntext{определение}{\textit{автор*} - бинарное неролевое отношение, связывающее некоторую сущность и персону, обозначающее, что данная персона является автором данной сущности.}
		\begin{scnindent}
			\begin{scnrelfromlist}{используемые константы}
				\scnitem{отношение}
				\scnitem{бинарное отношение}
				\scnitem{неролевое отношение}
				\scnitem{сущность}
				\scnitem{персона}
				\scnitem{автор}
			\end{scnrelfromlist}
		\end{scnindent}

	\scnheader{указание подразделов}
	\begin{scnrelfromset}{обобщенная декомпозиция}
		\scnitem{знак специфицируемого объекта}
		\scnitem{раздел базы знаний}
		\scnitem{дуга, связывающая специфицируемый объект с разделом базы знаний}
		\scnitem{дуга принадлежности отношению \textit{декомпозиция раздела*}}
		\scnitem{декомпозиция раздела*}
	\end{scnrelfromset}

	\scnheader{декомпозиция раздела*}
	\scnsubset{отношение}
	\scnsubset{бинарное отношение}
	\scnsubset{неролевое отношение}
	\scnsubset{ориентированное отношение}
	\scnsubset{антисимметричное отношение}
	\scnsubset{антитранзитивное отношение}
	\scnsubset{антирефлексивное отношение}
	\begin{scnrelfrom}{первый домен}
		{раздел базы знаний}
	\end{scnrelfrom}
	\begin{scnrelfrom}{второй домен}
		{раздел базы знаний}
	\end{scnrelfrom}
	\begin{scnrelfrom}{область определения}
		{раздел базы знаний}
	\end{scnrelfrom}
	\scntext{определение}{\textit{декомпозиция раздела*} - бинарное неролевое отношение, связывающее два раздела базы знаний, обозначающее, что один раздел входит в состав другого раздела.}
		\begin{scnindent}
			\begin{scnrelfromlist}{используемые константы}
				\scnitem{отношение}
				\scnitem{бинарное отношение}
				\scnitem{неролевое отношение}
				\scnitem{раздел базы знаний}
			\end{scnrelfromlist}
		\end{scnindent}

	\scnheader{указание предметной области}
	\begin{scnrelfromset}{обобщенная декомпозиция}
		\scnitem{знак специфицируемого объекта}
		\scnitem{предметная область}
		\scnitem{дуга, связывающая специфицируемый объект с предметной областью}
	\end{scnrelfromset}

	\scnheader{указание рассматриваемого вопроса}
	\begin{scnrelfromset}{обобщенная декомпозиция}
		\scnitem{знак специфицируемого объекта}
		\scnitem{вопрос}
		\scnitem{дуга, связывающая специфицируемый объект с вопросом}
		\scnitem{дуга принадлежности отношению \textit{рассматриваемый вопрос*}}
		\scnitem{рассматриваемый вопрос*}
	\end{scnrelfromset}

	\scnheader{рассматриваемый вопрос*}
	\scnsubset{отношение}
	\scnsubset{бинарное отношение}
	\scnsubset{неролевое отношение}
	\scnsubset{ориентированное отношение}
	\scnsubset{антисимметричное отношение}
	\scnsubset{антитранзитивное отношение}
	\scnsubset{антирефлексивное отношение}
	\begin{scnrelfrom}{первый домен}
		{раздел базы знаний}
	\end{scnrelfrom}
	\begin{scnrelfrom}{второй домен}
		{вопрос}
	\end{scnrelfrom}
	\begin{scnrelfromset}{область определения}
		\scnitem{раздел баз знаний}
		\scnitem{вопрос}
	\end{scnrelfromset}
	\scntext{определение}{\textit{рассматриваемый вопрос*} - бинарное неролевое отношение, связывающее раздел базы знаний и вопрос, обозначающее, что некоторый вопрос рассматривается в данном разделе базы знаний.}
		\begin{scnindent}
			\begin{scnrelfromlist}{используемые константы}
				\scnitem{отношение}
				\scnitem{бинарное отношение}
				\scnitem{неролевое отношение}
				\scnitem{раздел баз знаний}
				\scnitem{вопрос}
			\end{scnrelfromlist}
		\end{scnindent}
	
	\scnheader{указание ключевых элементов}
	\begin{scnrelfromset}{обобщенная декомпозиция}
		\scnitem{знак специфицируемого объекта}
		\scnitem{сущность}
		\scnitem{дуга, связывающая специфицируемый объект с сущностью}
		\scnitem{дуга принадлежности отношению \textit{ключевой sc-элемент'}}
		\scnitem{ключевой sc-элемент'}
	\end{scnrelfromset}

	\scnheader{ключевой sc-элемент'}
	\scnsubset{отношение}
	\scnsubset{бинарное отношение}
	\scnsubset{ролевое отношение}
	\scnsubset{ориентированное отношение}
	\scnsubset{антисимметричное отношение}
	\scnsubset{антитранзитивное отношение}
	\scnsubset{антирефлексивное отношение}
	\begin{scnrelfrom}{первый домен}
		{раздел базы знаний}
	\end{scnrelfrom}
	\begin{scnrelfrom}{второй домен}
		{сущность}
	\end{scnrelfrom}
	\begin{scnrelfromset}{область определения}
		\scnitem{раздел базы знаний}
		\scnitem{сущность}
	\end{scnrelfromset}
	\scntext{определение}{\textit{ключевой sc-элемент'} - бинарное ролевое отношение, связывающее раздел базы знаний и сущности, принадлежащие данному разделу базы знаний, обозначающее, что данные сущности являются ключевыми элементами данного раздела.}
		\begin{scnindent}
			\begin{scnrelfromlist}{используемые константы}
				\scnitem{отношение}
				\scnitem{бинарное отношение}
				\scnitem{ролевое отношение}
				\scnitem{раздел базы знаний}
				\scnitem{сущность}
			\end{scnrelfromlist}
		\end{scnindent}
	
% \\\\\\\\\\\\\\\ ПРЕДМЕТНАЯ ОБЛАСТЬ
	\scnheader{базовая спецификация предметной области}
	\begin{scnrelto}{обобщенная базовая спецификация}
		{предметная область}
	\end{scnrelto}
	\begin{scnrelfromvector}{обобщенная декомпозиция}
		\scnitem{указание частных предметных областей}
		\scnitem{указание максимального класса объектов исследования}
		\scnitem{указание классов объектов исследования}
		\scnitem{указание исследуемых отношений}
	\end{scnrelfromvector}
	% \scnrelfrom{пример}{базовая спецификация предметной области}
	% \begin{scnindent}
	% 	\begin{scnrelto}{базовая спецификация}
	% 		{Предметная область}
	% 	\end{scnrelto}
	% \end{scnindent}

	\scnheader{указание частных предметных областей}
	\begin{scnrelfromset}{обобщенная декомпозиция}
		\scnitem{знак специфицируемого объекта}
		\scnitem{предметная область}
		\scnitem{дуга, связывающая специфицируемый объект с предметной областью}
		\scnitem{дуга принадлежности отношению \textit{частная предметная область*}}
		\scnitem{частная предметная область*}
	\end{scnrelfromset}

	\scnheader{частная предметная область*}
	\scnsubset{отношение}
	\scnsubset{бинарное отношение}
	\scnsubset{неролевое отношение}
	\scnsubset{ориентированное отношение}
	\scnsubset{антисимметричное отношение}
	\scnsubset{антитранзитивное отношение}
	\scnsubset{антирефлексивное отношение}
	\begin{scnrelfrom}{первый домен}
		{предметная область}
	\end{scnrelfrom}
	\begin{scnrelfrom}{второй домен}
		{предметная область}
	\end{scnrelfrom}
	\begin{scnrelfrom}{область определения}
		{предметная область}
	\end{scnrelfrom}
	\scntext{определение}{\textit{частная прдметная область*} - бинарное неролевое отношение, связывающее две предметные области, обозначающее, что одна предметная область входит в состав второй предметной области.}
		\begin{scnindent}
			\begin{scnrelfromlist}{используемые константы}
				\scnitem{отношение}
				\scnitem{бинарное отношение}
				\scnitem{неролевое отношение}
				\scnitem{предметная область}
			\end{scnrelfromlist}
		\end{scnindent}

	\scnheader{указание максимального класса объектов исследования}
	\begin{scnrelfromset}{обобщенная декомпозиция}
		\scnitem{знак специфицируемого объекта}
		\scnitem{класс сущностей}
		\scnitem{дуга, связывающая специфицируемый объект с классом сущностей}
		\scnitem{дуга принадлежности отношению \textit{максимальный класс объектов исследования'}}
		\scnitem{максимальный класс объектов исследования'}
	\end{scnrelfromset}

	\scnheader{максимальный класс объектов исследования'}
	\scnsubset{отношение}
	\scnsubset{бинарное отношение}
	\scnsubset{ролевое отношение}
	\scnsubset{ориентированное отношение}
	\scnsubset{антисимметричное отношение}
	\scnsubset{антитранзитивное отношение}
	\scnsubset{антирефлексивное отношение}
	\begin{scnrelfrom}{первый домен}
		{предметная область}
	\end{scnrelfrom}
	\begin{scnrelfrom}{второй домен}
		{класс сущностей}
	\end{scnrelfrom}
	\begin{scnrelfromset}{область определения}
		\scnitem{предметная область}
		\scnitem{класс сущностей}
	\end{scnrelfromset}
	\scntext{определение}{\textit{максимальный класс объектов исследования'} - бинарное ролевое отношение, связывающее предметную область и класс сущностей, принадлежащий данной предметной области. Указывающее, что в рамках данной предметной области это класс является максимальным классом объектов исследования данной предметной области, то есть это такое исследуемое понятие’, для которого в рамках данной предметной области не существует другого исследуемого понятия', которое бы являлось надмножеством для данного.}
		\begin{scnindent}
			\begin{scnrelfromlist}{используемые константы}
				\scnitem{отношение}
				\scnitem{бинарное отношение}
				\scnitem{ролевое отношение}
				\scnitem{предметная область}
				\scnitem{класс сущностей}
				\scnitem{множество}
				\scnitem{исследуемое понятие'}
				\scnitem{надмножество}
			\end{scnrelfromlist}
		\end{scnindent}

	\scnheader{указание классов объектов исследования}
	\begin{scnrelfromset}{обобщенная декомпозиция}
		\scnitem{знак специфицируемого объекта}
		\scnitem{класс сущностей}
		\scnitem{дуга, связывающая специфицируемый объект с классом сущностей}
		\scnitem{дуга принадлежности отношению \textit{класс объектов исследования'}}
		\scnitem{класс объектов исследования'}
	\end{scnrelfromset}

	\scnheader{класс объектов исследования'}
	\scnsubset{отношение}
	\scnsubset{бинарное отношение}
	\scnsubset{ролевое отношение}
	\scnsubset{ориентированное отношение}
	\scnsubset{антисимметричное отношение}
	\scnsubset{антитранзитивное отношение}
	\scnsubset{антирефлексивное отношение}
	\begin{scnrelfrom}{первый домен}
		{предметная область}
	\end{scnrelfrom}
	\begin{scnrelfrom}{второй домен}
		{класс сущностей}
	\end{scnrelfrom}
	\begin{scnrelfromset}{область определения}
		\scnitem{предметная область}
		\scnitem{класс сущностей}
	\end{scnrelfromset}
	\scntext{определение}{\textit{класс объектов исследования'} - бинарное ролевое отношение, связывающее предметную область и класс сущностей, принадлежащий данной предметной области. Указывающее, что в рамках данной предметной области существует другое исследуемого понятия', которое бы являлось надмножеством для данного.}
		\begin{scnindent}
			\begin{scnrelfromlist}{используемые константы}
				\scnitem{отношение}
				\scnitem{бинарное отношение}
				\scnitem{ролевое отношение}
				\scnitem{предметная область}
				\scnitem{класс сущностей}
				\scnitem{множество}
				\scnitem{исследуемое понятие'}
				\scnitem{надмножество}
			\end{scnrelfromlist}
		\end{scnindent}
	
	\scnheader{указание исследуемых отношений}
	\begin{scnrelfromset}{обобщенная декомпозиция}
		\scnitem{знак специфицируемого объекта}
		\scnitem{отношение}
		\scnitem{дуга, связывающая специфицируемый объект с отношением}
		\scnitem{дуга принадлежности отношению \textit{исследуемое отношение'}}
		\scnitem{исследуемое отношение'}
	\end{scnrelfromset}

	\scnheader{исследуемое отношение'}
	\scnsubset{отношение}
	\scnsubset{бинарное отношение}
	\scnsubset{неролевое отношение}
	\scnsubset{ориентированное отношение}
	\scnsubset{антисимметричное отношение}
	\scnsubset{антитранзитивное отношение}
	\scnsubset{антирефлексивное отношение}
	\begin{scnrelfrom}{первый домен}
		{предметная область}
	\end{scnrelfrom}
	\begin{scnrelfrom}{второй домен}
		{отношение}
	\end{scnrelfrom}
	\begin{scnrelfromset}{область определения}
		\scnitem{предметная область}
		\scnitem{отношение}
	\end{scnrelfromset}
	\scntext{определение}{\textit{исследуемое отношение'} - бинарное ролевое отношение, указывающее в рамках предметной области на связки, являющееся исследуемым отношением данной предметной области, то есть таким отношением, все связки которого являются элементами этой предметной области. При этом элементы таких связок также входят в данную предметную область, но в общем случае могут не являться элементами исследуемых понятий' данной предметной области.}
		\begin{scnindent}
			\begin{scnrelfromlist}{используемые константы}
				\scnitem{отношение}
				\scnitem{бинарное отношение}
				\scnitem{ролевое отношение}
				\scnitem{предметная область}
				\scnitem{отношение}
				\scnitem{множество}
				\scnitem{исследуемое понятие'}
				\scnitem{связка}
			\end{scnrelfromlist}
		\end{scnindent}	

% \\\\\\\\\\\\\\ ПРОЕКТ
	\scnheader{базовая спецификация проекта}
	\begin{scnrelto}{обобщенная базовая спецификация}
		{проект}
	\end{scnrelto}
	\begin{scnrelfromvector}{обобщенная декомпозиция}
		\scnitem{указание автора}
		\scnitem{указание задачи}
		\scnitem{указание назначения}
		\scnitem{указание компонентов}
		\scnitem{указание способ установки}
		\scnitem{указание конечного продукта}
	\end{scnrelfromvector}

	\scnheader{указание задачи}
	\begin{scnrelfromset}{обобщенная декомпозиция}
		\scnitem{знак специфицируемого объекта}
		\scnitem{задача}
		\scnitem{дуга, связывающая специфицируемый объект с задачей}
		\scnitem{дуга принадлежности отношению \textit{задача*}}
		\scnitem{задача*}
	\end{scnrelfromset}

	\scnheader{задача*}
	\scnsubset{отношение}
	\scnsubset{бинарное отношение}
	\scnsubset{неролевое отношение}
	\scnsubset{ориентированное отношение}
	\scnsubset{антисимметричное отношение}
	\scnsubset{антитранзитивное отношение}
	\scnsubset{антирефлексивное отношение}
	\begin{scnrelfrom}{первый домен}
		{проект}
	\end{scnrelfrom}
	\begin{scnrelfrom}{второй домен}
		{задача}
	\end{scnrelfrom}
	\begin{scnrelfromset}{область определения}
		\scnitem{проект}
		\scnitem{задача}
	\end{scnrelfromset}
	\scntext{определение}{\textit{задача*} - бинарное неролевое отношение, связывающее проект и задачу, указывающее на то, какие конктретные задачи решает данный проект.}
		\begin{scnindent}
			\begin{scnrelfromlist}{используемые константы}
				\scnitem{отношение}
				\scnitem{бинарное отношение}
				\scnitem{неролевое отношение}
				\scnitem{задача}
				\scnitem{проект}
			\end{scnrelfromlist}
		\end{scnindent}
	
	\scnheader{указание назначения}
	\begin{scnrelfromset}{обобщенная декомпозиция}
		\scnitem{знак специфицируемого объекта}
		\scnitem{назначение}
		\scnitem{дуга, связывающая специфицируемый объект с назначением}
		\scnitem{дуга принадлежности отношению \textit{назначение*}}
		\scnitem{назначение*}
	\end{scnrelfromset}

	\scnheader{назначение*}
	\scnsubset{отношение}
	\scnsubset{бинарное отношение}
	\scnsubset{неролевое отношение}
	\scnsubset{ориентированное отношение}
	\scnsubset{антисимметричное отношение}
	\scnsubset{антитранзитивное отношение}
	\scnsubset{антирефлексивное отношение}
	\begin{scnrelfrom}{первый домен}
		{сущность}
	\end{scnrelfrom}
	\begin{scnrelfrom}{второй домен}
		{назначение}
	\end{scnrelfrom}
	\begin{scnrelfromset}{область определения}
		\scnitem{сущность}
		\scnitem{назначение}
	\end{scnrelfromset}
	\scntext{определение}{\textit{назначение*} - бинарное неролевое отношение, связывающее некоторую сущность базы знаний и назначение, показывающее для чего была создана данная сущность.}
		\begin{scnindent}
			\begin{scnrelfromlist}{используемые константы}
				\scnitem{отношение}
				\scnitem{бинарное отношение}
				\scnitem{неролевое отношение}
				\scnitem{сущность}
				\scnitem{назначение}
			\end{scnrelfromlist}
		\end{scnindent}
	
	\scnheader{указание компонентов}
	\begin{scnrelfromset}{обобщенная декомпозиция}
		\scnitem{знак специфицируемого объекта}
		\scnitem{сущность}
		\scnitem{дуга, связывающая специфицируемый объект с сущностью}
		\scnitem{дуга принадлежности отношению \textit{компонент*}}
		\scnitem{компонент*}
	\end{scnrelfromset}

	\scnheader{компонент*}
	\scnsubset{отношение}
	\scnsubset{бинарное отношение}
	\scnsubset{неролевое отношение}
	\scnsubset{ориентированное отношение}
	\scnsubset{антисимметричное отношение}
	\scnsubset{антитранзитивное отношение}
	\scnsubset{антирефлексивное отношение}
	\begin{scnrelfrom}{первый домен}
		{сущность}
	\end{scnrelfrom}
	\begin{scnrelfrom}{второй домен}
		{сущность}
	\end{scnrelfrom}
	\begin{scnrelfrom}{область определения}
		{сущность}
	\end{scnrelfrom}
	\scntext{определение}{\textit{компонент*} - бинарное неролевое отношение, связывающее сущности и обозначающее, что одна сущность является компонетом другой сущности.}
		\begin{scnindent}
			\begin{scnrelfromlist}{используемые константы}
				\scnitem{отношение}
				\scnitem{бинарное отношение}
				\scnitem{неролевое отношение}
				\scnitem{компонент}
				\scnitem{сущность}
			\end{scnrelfromlist}
		\end{scnindent}
	
	\scnheader{указание способа установки}
	\begin{scnrelfromset}{обобщенная декомпозиция}
		\scnitem{знак специфицируемого объекта}
		\scnitem{инструкция}
		\scnitem{дуга, связывающая специфицируемый объект с инструкцией}
		\scnitem{дуга принадлежности отношению \textit{способ установки*}}
		\scnitem{способ установки*}
	\end{scnrelfromset}

	% данное отношение используется в могократно используемом компоненте, можно ли считать, что множество кногократно используемых компонентов включается во множество проектов?
	\scnheader{способ установки*}
	\scnsubset{отношение}
	\scnsubset{бинарное отношение}
	\scnsubset{неролевое отношение}
	\scnsubset{ориентированное отношение}
	\scnsubset{антисимметричное отношение}
	\scnsubset{антитранзитивное отношение}
	\scnsubset{антирефлексивное отношение}
	\begin{scnrelfrom}{первый домен}
		{проект}
	\end{scnrelfrom}
	\begin{scnrelfrom}{второй домен}
		{инструкция}
	\end{scnrelfrom}
	\begin{scnrelfromset}{область определения}
		\scnitem{проект}
		\scnitem{инструкция}
	\end{scnrelfromset}
	\scntext{определение}{\textit{способ установки*} - бинарное неролевое отношение, связывающее проект и инструкцию к установке, показывающее, каким образом необходимо устанавливать проект.}
		\begin{scnindent}
			\begin{scnrelfromlist}{используемые константы}
				\scnitem{отношение}
				\scnitem{бинарное отношение}
				\scnitem{неролевое отношение}
				\scnitem{проект}
				\scnitem{инструкция}
			\end{scnrelfromlist}
		\end{scnindent}
	
	\scnheader{указание конечного продукта}
	\begin{scnrelfromset}{обобщенная декомпозиция}
		\scnitem{знак специфицируемого объекта}
		\scnitem{продукт}
		\scnitem{дуга, связывающая специфицируемый объект с продуктом}
		\scnitem{дуга принадлежности отношению \textit{продукт*}}
		\scnitem{продукт*}
	\end{scnrelfromset}

	% данное отношение используется в ostis-системе, как коректно обозначить первый домен? можно же туда множество добавлять, но не уверена, как это правильнее сделать
	\scnheader{способ установки*}
	\scnsubset{отношение}
	\scnsubset{бинарное отношение}
	\scnsubset{неролевое отношение}
	\scnsubset{ориентированное отношение}
	\scnsubset{антисимметричное отношение}
	\scnsubset{антитранзитивное отношение}
	\scnsubset{антирефлексивное отношение}
	\begin{scnrelfrom}{первый домен}
		{проект}
	\end{scnrelfrom}
	\begin{scnrelfrom}{второй домен}
		{продукт}
	\end{scnrelfrom}
	\begin{scnrelfromset}{область определения}
		\scnitem{проект}
		\scnitem{конечный продукт}
	\end{scnrelfromset}
	\scntext{определение}{\textit{продукт*} - бинарное неролевое отношение, связывающее проект и некоторый продукт, показывающее, что является итоговым продуктом после реализации проекта.}
		\begin{scnindent}
			\begin{scnrelfromlist}{используемые константы}
				\scnitem{отношение}
				\scnitem{бинарное отношение}
				\scnitem{неролевое отношение}
				\scnitem{проект}
				\scnitem{продукт}
				\scnitem{реализация}
			\end{scnrelfromlist}
		\end{scnindent}
	

% \\\\\\\\\\\\\ МНОГОКРАТНО ИСПОЛЬЗУЕМЫЙ КОМПОНЕНТ
	\scnheader{базовая спецификация многократно используемого компонента}
	\begin{scnrelto}{обобщенная базовая спецификация}
		{многократно используемый компонент}
	\end{scnrelto}
	\begin{scnrelfromvector}{обобщенная декомпозиция}
		\scnitem{указание назначения}
		\scnitem{указание автора}
		\scnitem{указание ссылки}
		\scnitem{указание используемого языка представления методов}
		\scnitem{указание зависимостей}
		\scnitem{указание частей}
		\scnitem{указание способа установки}
	\end{scnrelfromvector}

	\scnheader{указание ссылки}
	\begin{scnrelfromset}{обобщенная декомпозиция}
		\scnitem{знак специфицируемого объекта}
		\scnitem{интернет-ссылка}
		\scnitem{дуга, связывающая специфицируемый объект со ссылкой}
		\scnitem{дуга принадлежности отношению \textit{url*}}
		\scnitem{url*}
	\end{scnrelfromset}

	\scnheader{url*}
	\scnsubset{отношение}
	\scnsubset{бинарное отношение}
	\scnsubset{неролевое отношение}
	\scnsubset{ориентированное отношение}
	\scnsubset{антисимметричное отношение}
	\scnsubset{антитранзитивное отношение}
	\scnsubset{антирефлексивное отношение}
	\begin{scnrelfrom}{первый домен}
		{сущность}
	\end{scnrelfrom}
	\begin{scnrelfrom}{второй домен}
		{интернет-ссылка}
	\end{scnrelfrom}
	\begin{scnrelfromset}{область определения}
		\scnitem{сущность}
		\scnitem{интернет-ссылка}
	\end{scnrelfromset}
	\scntext{определение}{\textit{url*} - бинарное неролевое отношение, связывающее некоторую сущность с её интернет-ссылкой, используется для обозначения местонахождения исходных данный сущности.}
		\begin{scnindent}
			\begin{scnrelfromlist}{используемые константы}
				\scnitem{отношение}
				\scnitem{бинарное отношение}
				\scnitem{неролевое отношение}
				\scnitem{сущность}
				\scnitem{интернет-ссылка}
				\scnitem{местонахождение}
			\end{scnrelfromlist}
		\end{scnindent}
	
	\scnheader{указание используемого языка представления методов}
	\begin{scnrelfromset}{обобщенная декомпозиция}
		\scnitem{знак специфицируемого объекта}
		\scnitem{язык программирования}
		\scnitem{дуга, связывающая специфицируемый объект с языком программирования}
		\scnitem{дуга принадлежности отношению \textit{язык представления методов*}}
		\scnitem{язык представаления методов*}
	\end{scnrelfromset}

	\scnheader{язык представаления методов*}
	\scnsubset{отношение}
	\scnsubset{бинарное отношение}
	\scnsubset{неролевое отношение}
	\scnsubset{ориентированное отношение}
	\scnsubset{антисимметричное отношение}
	\scnsubset{антитранзитивное отношение}
	\scnsubset{антирефлексивное отношение}
	\begin{scnrelfrom}{первый домен}
		{многократно используемый компонент}
	\end{scnrelfrom}
	\begin{scnrelfrom}{второй домен}
		{язык программирования}
	\end{scnrelfrom}
	\begin{scnrelfromset}{область определения}
		\scnitem{многократно используемый компонент}
		\scnitem{язык программирования}
	\end{scnrelfromset}
	\scntext{определение}{\textit{язык представаления методов*} - бинарное неролевое отношение, связывающее многократно используемый компонент и языки программирования, указывающее на каком языке программирования реализованы методы в данном многократно используемом компоненте.}
		\begin{scnindent}
			\begin{scnrelfromlist}{используемые константы}
				\scnitem{отношение}
				\scnitem{бинарное отношение}
				\scnitem{неролевое отношение}
				\scnitem{многократно используемый компонент}
				\scnitem{язык программирования}
				\scnitem{метод}
			\end{scnrelfromlist}
		\end{scnindent}
	
	\scnheader{указание зависимостей}
	\begin{scnrelfromset}{обобщенная декомпозиция}
		\scnitem{знак специфицируемого объекта}
		\scnitem{зависимость}
		\scnitem{дуга, связывающая специфицируемый объект с }
		\scnitem{дуга принадлежности отношению \textit{зависимости*}}
		\scnitem{зависимости*}
	\end{scnrelfromset}

	\scnheader{зависимости*}
	\scnsubset{отношение}
	\scnsubset{бинарное отношение}
	\scnsubset{неролевое отношение}
	\scnsubset{ориентированное отношение}
	\scnsubset{антисимметричное отношение}
	\scnsubset{антитранзитивное отношение}
	\scnsubset{антирефлексивное отношение}
	\begin{scnrelfrom}{первый домен}
		{многократно используемый компонент}
	\end{scnrelfrom}
	\begin{scnrelfrom}{второй домен}
		{зависимость}
	\end{scnrelfrom}
	\begin{scnrelfromset}{область определения}
		\scnitem{многократно используемый компонент}
		\scnitem{зависимость}
	\end{scnrelfromset}
	\scntext{определение}{\textit{зависимость*} - бинарное неролевое отношение, связывающее многократно используемый компонент и некоторые зависимости, указывающее на то, какие зависимости необходимы для правильной установки и работы многократно используемого компонента.}
		\begin{scnindent}
			\begin{scnrelfromlist}{используемые константы}
				\scnitem{отношение}
				\scnitem{бинарное отношение}
				\scnitem{неролевое отношение}
				\scnitem{многократно используемый компонент}
				\scnitem{зависимость}
				\scnitem{установка}
				\scnitem{работа}
			\end{scnrelfromlist}
		\end{scnindent}
	
	\scnheader{указание частей}
	\begin{scnrelfromset}{обобщенная декомпозиция}
		\scnitem{знак специфицируемого объекта}
		\scnitem{сущность}
		\scnitem{дуга, связывающая специфицируемый объект с сущность, которая является его частью}
		\scnitem{дуга принадлежности отношению \textit{часть*}}
		\scnitem{часть*}
	\end{scnrelfromset}

	\scnheader{часть*}
	\scnsubset{отношение}
	\scnsubset{бинарное отношение}
	\scnsubset{неролевое отношение}
	\scnsubset{ориентированное отношение}
	\scnsubset{антисимметричное отношение}
	\scnsubset{антитранзитивное отношение}
	\scnsubset{антирефлексивное отношение}
	\begin{scnrelfrom}{первый домен}
		{сущность}
	\end{scnrelfrom}
	\begin{scnrelfrom}{второй домен}
		{сущность}
	\end{scnrelfrom}
	\begin{scnrelfrom}{область определения}
		{сущность}
	\end{scnrelfrom}
	\scntext{определение}{\textit{часть*} - бинарное неролевое отношение, связывающее две сущности и показывающее, что она сущность является частью другой сущности.}
		\begin{scnindent}
			\begin{scnrelfromlist}{используемые константы}
				\scnitem{отношение}
				\scnitem{бинарное отношение}
				\scnitem{неролевое отношение}
				\scnitem{сущность}
				\scnitem{часть}
			\end{scnrelfromlist}
		\end{scnindent}


% \\\\\\\\\\\\\\\\\\\\\ OSTIS-СИСТЕМА
% тут потом ещё будут дополнительно описаны базовые спецификации для различных видов ostis-систем (производственных, медицинских, обучающих и тд + метасистемы)
	\scnheader{базовая спецификация ostis-системы}
	\begin{scnrelto}{обобщенная базовая спецификация}
		{ostis-система}
	\end{scnrelto}
	\begin{scnrelfromvector}{обобщенная декомпозиция}
		\scnitem{указание декомпозиции}
		\scnitem{указание принципов реализации}
		\scnitem{указание подсистем}
		\scnitem{указание ссылка}
		\scnitem{указание конечного продукта}
	\end{scnrelfromvector}
	\scnrelfrom{пример}{базовая спецификация Метасистемы OSTIS}
	\begin{scnindent}
		\begin{scnrelto}{базовая спецификация}
			{Метасистема OSTIS}
		\end{scnrelto}
	\end{scnindent}
	\scntext{примечание}{Каждая ostis-система должна иметь базовую спецификацию, имеющую:
		\begin{itemize}
			\item предметонезависимый аспект;
			\item специализированный аспект.
		\end{itemize}}

	% тут есть вопросы с формализацией
	\scnheader{указание декомпозиции} 
	\begin{scnrelfromset}{обобщенная декомпозиция}
		\scnitem{знак специфицируемого объекта}
		\scnitem{знак объекта декомпозиции}
		\scnitem{дуга, связывающая специфицируемый объект с объектом декомпозиции}
		\scnitem{дуга принадлежности отношению \textit{декомпозиция*}}
		\scnitem{декомпозиция*}
	\end{scnrelfromset}
	
	\scnheader{указание принципов реализации}
	\begin{scnrelfromset}{обобщенная декомпозиция}
		\scnitem{знак специфицируемого объекта}
		\scnitem{принцип реализации}
		\scnitem{дуга, связывающая специфицируемый объект с принипом реализации}
		\scnitem{дуга принадлежности отношению \textit{принципы реализации*}}
		\scnitem{принципы реализации*}
	\end{scnrelfromset}

	\scnheader{принципы реализации*}
	\scnsubset{отношение}
	\scnsubset{бинарное отношение}
	\scnsubset{неролевое отношение}
	\scnsubset{ориентированное отношение}
	\scnsubset{антисимметричное отношение}
	\scnsubset{антитранзитивное отношение}
	\scnsubset{антирефлексивное отношение}
	\begin{scnrelfrom}{первый домен}
		{система}
	\end{scnrelfrom}
	\begin{scnrelfrom}{второй домен}
		{принцип реализации}
	\end{scnrelfrom}
	\begin{scnrelfromset}{область определения}
		\scnitem{система}
		\scnitem{принцип реализвации}
	\end{scnrelfromset}
	\scntext{определение}{\textit{принципы реализации*} - бинарное неролевое отношение, связывающее систему и конкретные принципы реализвации, которое позволяет уточнить различные моменты из реализации системы.}
		\begin{scnindent}
			\begin{scnrelfromlist}{используемые константы}
				\scnitem{отношение}
				\scnitem{бинарное отношение}
				\scnitem{неролевое отношение}
				\scnitem{принципы реализации}
				\scnitem{система}
				\scnitem{реализация}
				\scnitem{момент}
			\end{scnrelfromlist}
		\end{scnindent}
	
	\scnheader{указание подсистем}
	\begin{scnrelfromset}{обобщенная декомпозиция}
		\scnitem{знак специфицируемого объекта}
		\scnitem{система}
		\scnitem{дуга, связывающая специфицируемый объект с системой}
		\scnitem{дуга принадлежности отношению \textit{подсистема*}}
		\scnitem{подсистема*}
	\end{scnrelfromset}

	\scnheader{подсистема*}
	\scnsubset{отношение}
	\scnsubset{бинарное отношение}
	\scnsubset{неролевое отношение}
	\scnsubset{ориентированное отношение}
	\scnsubset{антисимметричное отношение}
	\scnsubset{антитранзитивное отношение}
	\scnsubset{антирефлексивное отношение}
	\begin{scnrelfrom}{первый домен}
		{система}
	\end{scnrelfrom}
	\begin{scnrelfrom}{второй домен}
		{система}
	\end{scnrelfrom}
	\begin{scnrelfrom}{область определения}
		{система}
	\end{scnrelfrom}
	\scntext{определение}{\textit{подсистема*} - бинарное неролевое отношение, связывающее две системы, обозначающее что одна система входит в состав другой системы.}
		\begin{scnindent}
			\begin{scnrelfromlist}{используемые константы}
				\scnitem{отношение}
				\scnitem{бинарное отношение}
				\scnitem{неролевое отношение}
				\scnitem{система}
				\scnitem{входить в состав*}
			\end{scnrelfromlist}
		\end{scnindent}


% \\\\\\\\\\\\\\\ ПЕРСОНА
	\scnheader{базовая спецификация персоны}
	\begin{scnrelto}{обобщенная базовая спецификация}
		{персона}
	\end{scnrelto}
	\begin{scnrelfromvector}{обобщенная декомпозиция}
		\scnitem{указание ФИО}
		\scnitem{указание контактной информации}
		\scnitem{указание роли}
		\scnitem{указание проектов} % для указания проектов над которыми работает, но есть отношение автор у разных сущностей, что выполняем примерно такую же роль, стоит ли вводить проекты у персоны?
	\end{scnrelfromvector}

	\scnheader{указание ФИО}
	\begin{scnrelfromset}{обобщенная декомпозиция}
		\scnitem{знак специфицируемого объекта}
		\scnitem{ФИО}
		\scnitem{дуга, связывающая специфицируемый объект с ФИО}
		\scnitem{дуга принадлежности отношению \textit{ФИО*}}
		\scnitem{ФИО*}
	\end{scnrelfromset}

	\scnheader{ФИО*}
	\scnsubset{отношение}
	\scnsubset{бинарное отношение}
	\scnsubset{неролевое отношение}
	\scnsubset{ориентированное отношение}
	\scnsubset{антисимметричное отношение}
	\scnsubset{антитранзитивное отношение}
	\scnsubset{антирефлексивное отношение}
	\begin{scnrelfrom}{первый домен}
		{персона}
	\end{scnrelfrom}
	\begin{scnrelfrom}{второй домен}
		{ФИО}
	\end{scnrelfrom}
	\begin{scnrelfromset}{область определения}
		\scnitem{персона}
		\scnitem{ФИО}
	\end{scnrelfromset}
	\scntext{определение}{\textit{ФИО*} - бинарное неролевое отношение, связывающее персону с фамилией, именем и отчеством, указывающее на то, как зовут конкретную персону.}
		\begin{scnindent}
			\begin{scnrelfromlist}{используемые константы}
				\scnitem{отношение}
				\scnitem{бинарное отношение}
				\scnitem{неролевое отношение}
				\scnitem{персона}
				\scnitem{ФИО}
			\end{scnrelfromlist}
		\end{scnindent}

	\scnheader{указание контактной информации}
	\begin{scnrelfromset}{обобщенная декомпозиция}
		\scnitem{знак специфицируемого объекта}
		\scnitem{контактная информация}
		\scnitem{дуга, связывающая специфицируемый объект с контактной информацией}
		\scnitem{дуга принадлежности отношению \textit{контактная информация*}}
		\scnitem{контактная информация*}
	\end{scnrelfromset}

	\scnheader{контактная информация*}
	\scnsubset{отношение}
	\scnsubset{бинарное отношение}
	\scnsubset{неролевое отношение}
	\scnsubset{ориентированное отношение}
	\scnsubset{антисимметричное отношение}
	\scnsubset{антитранзитивное отношение}
	\scnsubset{антирефлексивное отношение}
	\begin{scnrelfrom}{первый домен}
		{персона}
	\end{scnrelfrom}
	\begin{scnrelfrom}{второй домен}
		{контактная информация}
	\end{scnrelfrom}
	\begin{scnrelfromset}{область определения}
		\scnitem{персона}
		\scnitem{контактная информация}
	\end{scnrelfromset}
	\scntext{определение}{\textit{контактная информация*} - бинарное неролевое отношение, связывающее персону с её контактной информацией, указывающее на то, используя какую контактную информацию можно связаться с данной конкретной персоной.}
		\begin{scnindent}
			\begin{scnrelfromlist}{используемые константы}
				\scnitem{отношение}
				\scnitem{бинарное отношение}
				\scnitem{неролевое отношение}
				\scnitem{персона}
				\scnitem{контактная информация}
			\end{scnrelfromlist}
		\end{scnindent}
	
	\scnheader{указание роли}
	\begin{scnrelfromset}{обобщенная декомпозиция}
		\scnitem{знак специфицируемого объекта}
		\scnitem{роль}
		\scnitem{дуга, связывающая специфицируемый объект с ролью}
		\scnitem{дуга принадлежности отношению \textit{роль*}}
		\scnitem{роль*}
	\end{scnrelfromset}

	\scnheader{роль*}
	\scnsubset{отношение}
	\scnsubset{бинарное отношение}
	\scnsubset{неролевое отношение}
	\scnsubset{ориентированное отношение}
	\scnsubset{антисимметричное отношение}
	\scnsubset{антитранзитивное отношение}
	\scnsubset{антирефлексивное отношение}
	\begin{scnrelfrom}{первый домен}
		{персона}
	\end{scnrelfrom}
	\begin{scnrelfrom}{второй домен}
		{роль}
	\end{scnrelfrom}
	\begin{scnrelfromset}{область определения}
		\scnitem{персона}
		\scnitem{роль}
	\end{scnrelfromset}
	\scntext{определение}{\textit{роль*} - бинарное неролевое отношение, связывающее персону и её роль в рамках сообщества, указывающее на то, чем данная персона занимается в рамках сообщества.}
		\begin{scnindent}
			\begin{scnrelfromlist}{используемые константы}
				\scnitem{отношение}
				\scnitem{бинарное отношение}
				\scnitem{неролевое отношение}
				\scnitem{персона}
				\scnitem{роль}
			\end{scnrelfromlist}
		\end{scnindent}
	
	\scnheader{указание проектов}
	\begin{scnrelfromset}{обобщенная декомпозиция}
		\scnitem{знак специфицируемого объекта}
		\scnitem{проект}
		\scnitem{дуга, связывающая специфицируемый объект с проектом}
		\scnitem{дуга принадлежности отношению \textit{проект*}}
		\scnitem{проект*}
	\end{scnrelfromset}


% \\\\\\\\\\\\\\ АГЕНТ
    \scnheader{базовая спецификация агента}
	\begin{scnrelto}{обобщенная базовая спецификация}
		{агент}
	\end{scnrelto}
	\begin{scnrelfromvector}{обобщенная декомпозиция}
		\scnitem{указание входных данных}
		\scnitem{указание выходных данных}
		\scnitem{указание назначения}
		\scnitem{указание документации}
	\end{scnrelfromvector}

	\scnheader{указание входных данных}
	\begin{scnrelfromset}{обобщенная декомпозиция}
		\scnitem{знак специфицируемого объекта}
		\scnitem{данные}
		\scnitem{дуга, связывающая специфицируемый объект с данными}
		\scnitem{дуга принадлежности отношению \textit{входные данные*}}
		\scnitem{входные двнные*}
	\end{scnrelfromset}

	\scnheader{входные данные*}
	\scnsubset{отношение}
	\scnsubset{бинарное отношение}
	\scnsubset{неролевое отношение}
	\scnsubset{ориентированное отношение}
	\scnsubset{антисимметричное отношение}
	\scnsubset{антитранзитивное отношение}
	\scnsubset{антирефлексивное отношение}
	\begin{scnrelfrom}{первый домен}
		{агент}
	\end{scnrelfrom}
	\begin{scnrelfrom}{второй домен}
		{данные}
	\end{scnrelfrom}
	\begin{scnrelfromset}{область определения}
		\scnitem{агент}
		\scnitem{данные}
	\end{scnrelfromset}
	\scntext{определение}{\textit{входные данные*} - бинарное неролевое отношение, связывающее агент и некоторые данные, показывающее, какие данные, необходимые для работы, агент получает на входе.}
		\begin{scnindent}
			\begin{scnrelfromlist}{используемые константы}
				\scnitem{отношение}
				\scnitem{бинарное отношение}
				\scnitem{неролевое отношение}
				\scnitem{данные}
				\scnitem{агент}
				\scnitem{вход}
			\end{scnrelfromlist}
		\end{scnindent}
	
	\scnheader{указание выходных данных}
	\begin{scnrelfromset}{обобщенная декомпозиция}
		\scnitem{знак специфицируемого объекта}
		\scnitem{данные}
		\scnitem{дуга, связывающая специфицируемый объект с данными}
		\scnitem{дуга принадлежности отношению \textit{выходные данные*}}
		\scnitem{выходные данные*}
	\end{scnrelfromset}

	\scnheader{выходные данные*}
	\scnsubset{отношение}
	\scnsubset{бинарное отношение}
	\scnsubset{неролевое отношение}
	\scnsubset{ориентированное отношение}
	\scnsubset{антисимметричное отношение}
	\scnsubset{антитранзитивное отношение}
	\scnsubset{антирефлексивное отношение}
	\begin{scnrelfrom}{первый домен}
		{агент}
	\end{scnrelfrom}
	\begin{scnrelfrom}{второй домен}
		{данные}
	\end{scnrelfrom}
	\begin{scnrelfromset}{область определения}
		\scnitem{агент}
		\scnitem{данные}
	\end{scnrelfromset}
	\scntext{определение}{\textit{выходные данные*} - бинарное неролевое отношение, связывающее агент и некоторые данные, показывающее, какие данные данные будут получены на выходе работы агента.}
		\begin{scnindent}
			\begin{scnrelfromlist}{используемые константы}
				\scnitem{отношение}
				\scnitem{бинарное отношение}
				\scnitem{неролевое отношение}
				\scnitem{данные}
				\scnitem{агент}
				\scnitem{выход}
			\end{scnrelfromlist}
		\end{scnindent}
	
	\scnheader{указание документации}
	\begin{scnrelfromset}{обобщенная декомпозиция}
		\scnitem{знак специфицируемого объекта}
		\scnitem{документация}
		\scnitem{дуга, связывающая специфицируемый объект с документацией}
		\scnitem{дуга принадлежности отношению \textit{документация*}}
		\scnitem{документация*}
	\end{scnrelfromset}

	\scnheader{документация*}
	\scnsubset{отношение}
	\scnsubset{бинарное отношение}
	\scnsubset{неролевое отношение}
	\scnsubset{ориентированное отношение}
	\scnsubset{антисимметричное отношение}
	\scnsubset{антитранзитивное отношение}
	\scnsubset{антирефлексивное отношение}
	\begin{scnrelfrom}{первый домен}
		{агент}
	\end{scnrelfrom}
	\begin{scnrelfrom}{второй домен}
		{документация}
	\end{scnrelfrom}
	\begin{scnrelfromset}{область определения}
		\scnitem{агент}
		\scnitem{документация}
	\end{scnrelfromset}
	\scntext{определение}{\textit{документация*} - бинарное неролевое отношение, связывающее проект и его документацию. необходимое для связывания агента с его технической документацией, касающейся реализации и работы агента.}
		\begin{scnindent}
			\begin{scnrelfromlist}{используемые константы}
				\scnitem{отношение}
				\scnitem{бинарное отношение}
				\scnitem{неролевое отношение}
				\scnitem{агент}
				\scnitem{документация}
				\scnitem{реализация}
				\scnitem{работа}
			\end{scnrelfromlist}
		\end{scnindent}


% \\\\\\\\\\\\\\\\\\\ ОТНОШЕНИЕ
	\scnheader{базовая спецификация отношения}
	\begin{scnrelto}{обобщенная базовая спецификация}
		{отношение}
	\end{scnrelto}
	\begin{scnrelfromvector}{обобщенная декомпозиция}
		\scnitem{указание свойств отношения}
		\scnitem{указание первого домена}
		\scnitem{указание второго домена}
		\scnitem{указание области определения}
		\scnitem{указание определения}
	\end{scnrelfromvector}
		
	\scnheader{указание свойств отношения}
	\begin{scnrelfromset}{обобщенная декомпозиция}
		\scnitem{знак специфицируемого объекта}
		\scnitem{класс отношений}
		\scnitem{дуга принадлежности классу отношений}
	\end{scnrelfromset}

	\scnheader{указание первого домена}
	\begin{scnrelfromset}{обобщенная декомпозиция}
		\scnitem{знак специфицируемого объекта}
		\scnitem{класс сущностей}
		\scnitem{дуга, связывающая специфицируемый объект с классом сущностей}
		\scnitem{дуга принадлежности отношению \textit{первый домен*}}
		\scnitem{первый домен*}
	\end{scnrelfromset}
	
	\scnheader{указание второго домена}
	\begin{scnrelfromset}{обобщенная декомпозиция}
		\scnitem{знак специфицируемого объекта}
		\scnitem{класс сущностей}
		\scnitem{дуга, связывающая специфицируемый объект с классом сущностей}
		\scnitem{дуга принадлежности отношению \textit{второй домен*}}
		\scnitem{второй домен*}
	\end{scnrelfromset}

	\scnheader{указание области определения}
	\begin{scnrelfromset}{обобщенная декомпозиция}
		\scnitem{знак специфицируемого объекта}
		\scnitem{область определения}
		\scnitem{дуга, связывающая специфицируемый объект с областью определения}
		\scnitem{дуга принадлежности отношению \textit{область определения*}}
		\scnitem{область определения*}
	\end{scnrelfromset}

	\scnheader{указание определения}
	\begin{scnrelfromset}{обобщенная декомпозиция}
		\scnitem{знак специфицируемого объекта}
		\scnitem{определение}
		\scnitem{дуга, связывающая специфицируемый объект с определением}
		\scnitem{дуга принадлежности отношению \textit{определение*}}
		\scnitem{определение*}
	\end{scnrelfromset}

	\scnheader{определение*}
	\scnsubset{отношение}
	\scnsubset{бинарное отношение}
	\scnsubset{неролевое отношение}
	\scnsubset{ориентированное отношение}
	\scnsubset{антисимметричное отношение}
	\scnsubset{антитранзитивное отношение}
	\scnsubset{антирефлексивное отношение}
	\begin{scnrelfrom}{первый домен}
		{сущность}
	\end{scnrelfrom}
	\begin{scnrelfrom}{второй домен}
		{определение}
	\end{scnrelfrom}
	\begin{scnrelfromset}{область определения}
		\scnitem{сущность}
		\scnitem{определение}
	\end{scnrelfromset}
	\scntext{определение}{\textit{определение*} - бинарное неролевое отношение, связывающее сущность и её определение. Позволяет дать опеределение конкретной сущности и передать некоторую информацию о ней.}
		\begin{scnindent}
			\begin{scnrelfromlist}{используемые константы}
				\scnitem{отношение}
				\scnitem{бинарное отношение}
				\scnitem{неролевое отношение}
				\scnitem{сущность}
				\scnitem{определение}
				\scnitem{информация}
			\end{scnrelfromlist}
		\end{scnindent}
	
\end{scnsubstruct}
\end{SCn}

\begin{SCn}
% пересмотреть, убрать название, нужно чтобы предисловие относилось к Пункту, а дальше шла структура с ПрО
% или выделение подпунктов и где уже будет находится раздел (спецификация обобщённой базовой спецификации)

\scnheader{Подпункт 23.1.1.1 Формализация предметной области и онтологии базовых спецификаций}
\begin{scnsubstruct}

\scnsectionheader{Раздел. Предметная область и онтология базовых спецификаций}

% Если найдется, то добавлю
% \begin{scnrelfromlist}{эпиграф}
% 	\scnfileitem{}
% \end{scnrelfromlist}

\begin{scnrelfromlist}{предисловие}
	\scnfileitem{В настоящее время из-за большого количества информации появляется необходимость в её структуризации. Однако, из-за недостаточного внимания к этому процессу базы знаний становятся трудными для обработки. Это связано с отсутствием единого наглядного примера и чёткой инструкции по спецификации понятий в базе знаний.}
\end{scnrelfromlist}

\begin{scnrelfromlist}{аннотация}
	\scnfileitem{В данном разделе базы знаний представлены базовые спецификации ключевых классов сущностей, уточнены элементы базовых спецификаций этих сущностей и приведены примеры элементов этих спецификаций, которые могут служить образцом при дополнении базы знаний новыми понятиями.}
\end{scnrelfromlist}

\begin{scnrelfromlist}{основная цель}
	\scnfileitem{Унификация спецификации сущностей, принадлежащих ключевым классам в базе знаний.}
\end{scnrelfromlist}

\begin{scnrelfromlist}{автор}
	\scnitem{Петрочук К.Д}
	\scnitem{Гракова Н.В.}
\end{scnrelfromlist}

\begin{scnrelfromlist}{рассматриваемый вопрос}
	\scnfileitem{Что такое базовая спецификация?}
	\scnfileitem{Для каких классов сущностей можно описать базовые спецификации?}
	\scnfileitem{Какие конструкции должны входить в базовую спецификацию различных классов сущностей?}
\end{scnrelfromlist}

\begin{scnrelfromlist}{рассматриваемый вопрос}
	\scnitem{Раздел. Предметная область семантических окрестностей}
\end{scnrelfromlist}

\begin{scnrelfromlist}{ключевое понятие}
	\scnitem{базовая спецификация}
	\scnitem{класс сущностей, имеющих унифицированную базовую спецификацию}
	\scnitem{базовая спецификация раздела базы знаний}
	\scnitem{базовая спецификация предметной области}
	\scnitem{базовая спецификация проекта}
	\scnitem{базовая спецификация многократно используемого компонента}
	\scnitem{базовая спецификация ostis-системы}
	\scnitem{базовая спецификация персоны}
	\scnitem{базовая спецификация агента}
	\scnitem{базовая спецификация отношения}
	\scnitem{базовая спецификация класса сущностей}
	\scnitem{базовая спецификация библиографического источника}
	\scnitem{базовая спецификация*}
	\scnitem{обобщенная базовая спецификация*}
\end{scnrelfromlist}

\scnheader{Предметная область базовых спецификацией}
\scniselement{предметная область}
		\begin{scnrelto}{частная предметная область}
			{Предметная область семантических окресностей}
		\end{scnrelto}
        \begin{scnhaselementrole}{максимальный класс объектов исследования}
            {базовая спецификация}
        \end{scnhaselementrole}
        \begin{scnhaselementrolelist}{класс объектов исследования}
			\scnitem{класс сущностей, имеющих унифицированную базовую спецификацию}
			\scnitem{базовая спецификация понятия базовой спецификации}
			\scnitem{базовая спецификация раздела базы знаний}
			\scnitem{базовая спецификация предметной области}
			\scnitem{базовая спецификация проекта}
			\scnitem{базовая спецификация многократно используемого компонента}
			\scnitem{базовая спецификация ostis-системы}
			\scnitem{базовая спецификация персоны}
			\scnitem{базовая спецификация агента}
			\scnitem{базовая спецификация отношения}
			\scnitem{базовая спецификация класса сущностей}
			\scnitem{базовая спецификация библиографического источника}
			
        \end{scnhaselementrolelist}
		\begin{scnhaselementrolelist}{исследуемое отношение}
            \scnitem{базовая спецификация*}
			\scnitem{обобщенная базовая спецификация*}
        \end{scnhaselementrolelist}
\end{scnsubstruct}

\scnheader{Подпункт 23.1.1.2 Формализация основных понятий в рамках предметной области базовых спецификаций}
\begin{scnsubstruct}
	\scnheader{класс сущностей, имеющих унифицированную базовую спецификацию}
	\scnidtf{класс, для всех сущностей которого можно выделить общий набор свойств, необходимых для базового описания каждой сущности данного класса}
	\scntext{примечание}{В некоторых классах сущностей можно выделить подклассы, для которых могут быть описаны дополнительные базовые спецификации, которые будут характерны тольно для конкретного подкласса}
	\scnhaselement{базовая спецификация}
	\scnhaselement{раздел базы знаний}
	\scnhaselement{предметная область}
	\scnhaselement{проект}
	\scnhaselement{многократно используемый компонент}
	\scnhaselement{ostis-система}
	\scnhaselement{персона}
	\scnhaselement{агент}
	\scnhaselement{отношение}
	\scnhaselement{класс сущностей}
	\scnhaselement{библиографический источник}

	\scnheader{базовая спецификация}
	\scnsubset{спецификация}
	\scnidtf{набор основных свойств некоторых сущностей, принадлежащих одному классу}
	\scnidtf{минимальный набор свойств, который необходим для описания каждой сущности, принадлежащей одному классу}
	\scntext{примечание}{У каждой сущности, принадлежащей определённому классу, должен быть описан базовый набор свойст, характерных данному классу}
	\scntext{примечание}{Базовые спецификации описывают рекомендуемый минимум свойств для сущностей каждого класса, но не являются строго обязательными. Их следует использовать при спецификации сущностей для упрощения процесса разработки и структурирования информации в базе знаний, чтобы избежать появления в ней различных понятий с одинаковым значением.}
	\scnsuperset{базовая спецификация понятия базовой спецификации}
	\scnsuperset{базовая спецификация раздела базы знаний}
	\scnsuperset{базовая спецификация предметной области}
	\scnsuperset{базовая спецификация проекта}
	\scnsuperset{базовая спецификация многократно используемого компонента}
	\scnsuperset{базовая спецификация ostis-системы}
	\scnsuperset{базовая спецификация персоны}
    \scnsuperset{базовая спецификация агента}
    \scnsuperset{базовая спецификация отношения}
	\scnsuperset{базовая спецификация класса сущностей}
	\scnsuperset{базовая спецификация библиографического источника}

% \\\\\\\\\\\\ БАЗОВАЯ СПЕЦИФИКАЦИЯ
	\scnheader{базовая спецификация понятия базовой спецификации}
	\scntext{примечание}{Базовая спецификация понятия базовой спецификации отражает набор свойств, которые должны быть описаны в каждой базовой спецификации}
	\begin{scnrelto}{обобщенная базовая спецификация}
		{базовая спецификация}
	\end{scnrelto}
	\begin{scnrelfromvector}{обобщенная декомпозиция}
		\scnitem{спецификация обобщенной базовой спецификации}
		\scnitem{спецификация обобщенной декомпозиции}
		\scnitem{спецификация примера}
	\end{scnrelfromvector}
	\scnhaselementrole{пример}{базовая спецификация понятия базовой спецификации персоны}
		\begin{scnindent}
			\begin{scnrelto}{базовая спецификация}
				{базовая спецификация персоны}
			\end{scnrelto}
		\end{scnindent}
	
	\scnheader{спецификация обобщенной базовой спецификации}
	\begin{scnrelfromset}{обобщенная декомпозиция}
		\scnitem{знак специфицируемого объекта}
		\scnitem{класс сущностей, имеющих унифицированную базовую спецификацию}
		\scnitem{дуга, связывающая класс сущностей, имеющих унифицированную базовую спецификацию, со специфицируемым объектом}
		\scnitem{дуга принадлежности отношению \textit{обобщeнная базовая спецификация*}}
		\scnitem{обобщeнная базовая спецификация*}
	\end{scnrelfromset}
	\scnrelfrom{шаблон}{шаблон для описания обобщённой базовой спецификации}
	\begin{scnindent}
		\scnrelfrom{иллюстрация}{\scnfileimage[30em]{Contents/part_kb/src/images/sd_basic_specification/generalized_specification.png}}
		\begin{scnindent}
			\scnidtf{SCg-текст. Шаблон для описания обобщённой базовой спецификации}
		\end{scnindent}
	\end{scnindent}

	\scnheader{обобщeнная базовая спецификация*}
	\scnsubset{отношение}
	\scnsubset{бинарное отношение}
	\scnsubset{неролевое отношение}
	\scnsubset{ориентированное отношение}
	\scnsubset{антисимметричное отношение}
	\scnsubset{антитранзитивное отношение}
	\scnsubset{антирефлексивное отношение}
	\begin{scnrelfrom}{первый домен}
		{класс сущностей, имеющих унифицированную базовую спецификацию}
	\end{scnrelfrom}
	\begin{scnrelfrom}{второй домен}
		{базовая спецификация}
	\end{scnrelfrom}
	\begin{scnrelfromset}{область определения}
		\scnitem{класс сущностей, имеющих унифицированную базовую спецификацию}
		\scnitem{базовая спецификация}
	\end{scnrelfromset}
	\scntext{определение}{\textit{обобщeнная базовая спецификация*} - бинарное ориентированное неролевое отношение, связывающее конкретный класс и его базовую спецификацию, имеющим базовую спецификацию, и обозначающее, что у этого класса есть определённый набор базовых свойств для спецификации сущностей данного класса.}
		\begin{scnindent}
			\begin{scnrelfromlist}{используемые константы}
				\scnitem{отношение}
				\scnitem{ориентированное отношение}
				\scnitem{бинарное отношение}
				\scnitem{неролевое отношение}
				\scnitem{базовая спецификация}
				\scnitem{класс сущностей, имеющих унифицированную базовую спецификацию}
				\scnitem{свойство}
				\scnitem{сущность}
				\scnitem{спецификация}
			\end{scnrelfromlist}
		\end{scnindent}

	\scnheader{базовая спецификация*}
	\scnsubset{отношение}
	\scnsubset{бинарное отношение}
	\scnsubset{неролевое отношение}
	\scnsubset{ориентированное отношение}
	\scnsubset{антисимметричное отношение}
	\scnsubset{антитранзитивное отношение}
	\scnsubset{антирефлексивное отношение}
	\begin{scnrelfrom}{первый домен}
		{класс сущностей, имеющих унифицированную базовую спецификацию}
	\end{scnrelfrom}
	\begin{scnrelfrom}{второй домен}
		{базовая спецификация}
	\end{scnrelfrom}
	\begin{scnrelfromset}{область определения}
		\scnitem{класс сущностей, имеющих унифицированную базовую спецификацию}
		\scnitem{базовая спецификация}
	\end{scnrelfromset}
	\scntext{определение}{\textit{базовая спецификация*} - бинарное ориентированное неролевое отношение, связывающее некоторую сущность класса сущностей, имеющих унифицированную базовую спецификацию, и базовую спецификацию этой сущности, и показывающее, какая часть спецификации является базовой спецификацией для данной сущности.}
		\begin{scnindent}
			\begin{scnrelfromlist}{используемые константы}
				\scnitem{отношение}
				\scnitem{ориентированное отношение}
				\scnitem{бинарное отношение}
				\scnitem{неролевое отношение}
				\scnitem{базовая спецификация}
				\scnitem{класс сущностей, имеющих унифицированную базовую спецификацию}
				\scnitem{сущность}
				\scnitem{спецификация}
			\end{scnrelfromlist}
		\end{scnindent}
	
	% есть вопросы относительно правильности выделения второго элемента
	\scnheader{спецификация обобщенной декомпозиции}
	\begin{scnrelfromset}{обобщенная декомпозиция}
		\scnitem{знак специфицируемого объекта}
		\scnitem{знак объекта обобщенной декомпозиции}
		\scnitem{дуга, связывающая специфицируемый объект с объектом обобщенной декомпозиции}
		\scnitem{дуга принадлежности отношению \textit{обобщенная декомпозиция*}}
		\scnitem{обобщенная декомпозиция*}
	\end{scnrelfromset}

	\scnheader{спецификация примера}
	\begin{scnrelfromset}{обобщенная декомпозиция}
		\scnitem{знак специфицируемого объекта}
		\scnitem{пример}
		\scnitem{дуга, связывающая специфицируемый объект с примером}
		\scnitem{дуга принадлежности отношению \textit{пример\scnrolesign}}
		\scnitem{пример\scnrolesign}
	\end{scnrelfromset}
	\scnrelfrom{шаблон}{шаблон для описания примера}
	\begin{scnindent}
		\scnrelfrom{иллюстрация}{\scnfileimage[20em]{Contents/part_kb/src/images/sd_basic_specification/example.png}}
		\begin{scnindent}
			\scnidtf{SCg-текст. Шаблон для описания примера}
		\end{scnindent}
	\end{scnindent}

	\scnheader{пример\scnrolesign}
	\scnsubset{отношение}
	\scnsubset{бинарное отношение}
	\scnsubset{ролевое отношение} 
	\scnsubset{ориентированное отношение}
	\scnsubset{антисимметричное отношение}
	\scnsubset{антитранзитивное отношение}
	\scnsubset{антирефлексивное отношение}
	\begin{scnrelfrom}{первый домен}
		{класс сущностей}
	\end{scnrelfrom}
	\begin{scnrelfrom}{второй домен}
		{сущность}
	\end{scnrelfrom}
	\begin{scnrelfromset}{область определения}
		\scnitem{класс сущностей}
		\scnitem{сущность}
	\end{scnrelfromset}
	\scntext{определение}{\textit{пример\scnrolesign} - бинарное ориентированное ролевое отношение, связывающее некоторый класс сущностей базы знаний и экземпляр этого класса, и обозначающее, что указанный экземпляр наилучшим образом передаёт все свойства описываемого класса.}
		\begin{scnindent}
			\begin{scnrelfromlist}{используемые константы}
				\scnitem{отношение}
				\scnitem{ориентированное отношение}
				\scnitem{бинарное отношение}
				\scnitem{ролевое отношение}
				\scnitem{сущность}
				\scnitem{свойство}
				\scnitem{класс сущностей}
				\scnitem{спецификация}
			\end{scnrelfromlist}
		\end{scnindent}
\end{scnsubstruct}

\scnheader{Подпункт 23.1.1.3 Формализация базовой спецификации раздела}
\begin{scnsubstruct}
% \\\\\\\\\\\\\\\\\ РАЗДЕЛ
	\scnheader{базовая спецификация раздела базы знаний}
	\scntext{примечание}{Базовая спецификация раздела базы знаний отражает набор свойств, которые должны быть описаны для каждого раздела базы знаний}
	\begin{scnrelto}{обобщенная базовая спецификация}
		{раздел базы знаний}
	\end{scnrelto}
	\begin{scnrelfromvector}{обобщенная декомпозиция}
		\scnitem{спецификация эпиграфа}
		\scnitem{спецификация аннотации}
		\scnitem{спецификация предисловия}
		\scnitem{спецификация основной цели}
		\scnitem{спецификация автора}
		\scnitem{спецификация подразделов}
		\scnitem{спецификация предметной области}
		\scnitem{спецификация рассматриваемого вопроса}
		\scnitem{спецификация ключевых элементов}
	\end{scnrelfromvector}
	% \scnhaselementrole{пример}{базовая спецификация раздела предметной области и онтологии}
	% 	\begin{scnindent}
	% 		\begin{scnrelto}{базовая спецификация}
	% 			{Раздел. Предметная область и онтология треугольников}
	% 		\end{scnrelto}
	% 	\end{scnindent}

	\scnheader{спецификация эпиграфа}
	\begin{scnrelfromset}{обобщенная декомпозиция}
		\scnitem{знак специфицируемого объекта}
		\scnitem{эпиграф}
		\scnitem{дуга, связывающая специфицируемый объект с эпиграфом}
		\scnitem{дуга принадлежности отношению \textit{эпиграф*}}
		\scnitem{эпиграф*}
	\end{scnrelfromset}
	\scnrelfrom{шаблон}{шаблон для описания эпиграфа}
	\begin{scnindent}
		\scnrelfrom{иллюстрация}{\scnfileimage[20em]{Contents/part_kb/src/images/sd_basic_specification/epigraph.png}}
		\begin{scnindent}
			\scnidtf{SCg-текст. Шаблон для описания эпиграфа}
		\end{scnindent}
	\end{scnindent}

 	\scnheader{эпиграф*}
	\scnsubset{отношение}
	\scnsubset{бинарное отношение}
	\scnsubset{неролевое отношение}
	\scnsubset{ориентированное отношение}
	\scnsubset{антисимметричное отношение}
	\scnsubset{антитранзитивное отношение}
	\scnsubset{антирефлексивное отношение}
	\begin{scnrelfrom}{первый домен}
		{фрагмент базы знаний}
	\end{scnrelfrom}
	\begin{scnrelfrom}{второй домен}
		{sc-текст}
	\end{scnrelfrom}
	\begin{scnrelfromset}{область определения}
		\scnitem{фрагмент базы знаний}
		\scnitem{sc-текст}
	\end{scnrelfromset}
	\scntext{определение}{\textit{эпиграф*} - это бинарное ориентированное неролевое отношение, связывающее некоторый фрагмент базы знаний и sc-текст, являющийся эпиграфом к данному разделу.}
		\begin{scnindent}
			\begin{scnrelfromlist}{используемые константы}
				\scnitem{отношение}
				\scnitem{ориентированное отношение}
				\scnitem{бинарное отношение}
				\scnitem{неролевое отношение}
				\scnitem{эпиграф}
				\scnitem{фрагмент базы знаний}
				\scnitem{высказывание}
				\scnitem{мысль}
				\scnitem{автор}
				\scnitem{sc-текст}
			\end{scnrelfromlist}
		\end{scnindent}

	\scnheader{спецификация аннотации}
	\begin{scnrelfromset}{обобщенная декомпозиция}
		\scnitem{знак специфицируемого объекта}
		\scnitem{аннотация}
		\scnitem{дуга, связывающая специфицируемый объект с аннотацией}
		\scnitem{дуга принадлежности отношению \textit{аннотация*}}
		\scnitem{аннотация*}
	\end{scnrelfromset}
	\scnrelfrom{шаблон}{шаблон для описания аннотации}
	\begin{scnindent}
		\scnrelfrom{иллюстрация}{\scnfileimage[20em]{Contents/part_kb/src/images/sd_basic_specification/abstract.png}}
		\begin{scnindent}
			\scnidtf{SCg-текст. Шаблон для описания аннотации}
		\end{scnindent}
	\end{scnindent}

	\scnheader{аннотация*}
	\scnsubset{отношение}
	\scnsubset{бинарное отношение}
	\scnsubset{неролевое отношение}
	\scnsubset{ориентированное отношение}
	\scnsubset{антисимметричное отношение}
	\scnsubset{антитранзитивное отношение}
	\scnsubset{антирефлексивное отношение}
	\begin{scnrelfrom}{первый домен}
		{фрагмент базы знаний}
	\end{scnrelfrom}
	\begin{scnrelfrom}{второй домен}
		{sc-текст}
	\end{scnrelfrom}
	\begin{scnrelfromset}{область определения}
		\scnitem{фрагмент базы знаний}
		\scnitem{sc-текст}
	\end{scnrelfromset}
	\scntext{определение}{\textit{аннотация*} - бинарное ориентированное неролевое отношение, связывающее фрагмент базы знаний и некоторый sc-текст, являющийся аннотацией к данному разделу, то есть содержащий краткое описание данного раздела.}
		\begin{scnindent}
			\begin{scnrelfromlist}{используемые константы}
				\scnitem{отношение}
				\scnitem{ориентированное отношение}
				\scnitem{бинарное отношение}
				\scnitem{неролевое отношение}
				\scnitem{фрагмент базы знаний}
				\scnitem{аннотация}
				\scnitem{описание}
				\scnitem{sc-текст}
			\end{scnrelfromlist}
		\end{scnindent}

	\scnheader{спецификация предисловия}
	\begin{scnrelfromset}{обобщенная декомпозиция}
		\scnitem{знак специфицируемого объекта}
		\scnitem{предисловие}
		\scnitem{дуга, связывающая специфицируемый объект с предисловием}
		\scnitem{дуга принадлежности отношению \textit{предисловие*}}
		\scnitem{предисловие*}
	\end{scnrelfromset}
	\scnrelfrom{шаблон}{шаблон для описания предисловия}
	\begin{scnindent}
		\scnrelfrom{иллюстрация}{\scnfileimage[20em]{Contents/part_kb/src/images/sd_basic_specification/preface.png}}
		\begin{scnindent}
			\scnidtf{SCg-текст. Шаблон для описания предисловия}
		\end{scnindent}
	\end{scnindent}

	\scnheader{предисловие*}
	\scnsubset{отношение}
	\scnsubset{бинарное отношение}
	\scnsubset{неролевое отношение}
	\scnsubset{ориентированное отношение}
	\scnsubset{антисимметричное отношение}
	\scnsubset{антитранзитивное отношение}
	\scnsubset{антирефлексивное отношение}
	\begin{scnrelfrom}{первый домен}
		{фрагмент базы знаний}
	\end{scnrelfrom}
	\begin{scnrelfrom}{второй домен}
		{sc-текст}
	\end{scnrelfrom}
	\begin{scnrelfromset}{область определения}
		\scnitem{sc-текст}
		\scnitem{фрагмент базы знаний}
	\end{scnrelfromset}
	\scntext{определение}{\textit{предисловие*} - бинарное ориентированное неролевое отношение, связывающее фрагмент базы знаний и некоторый sc-текст, являющийся предисловием к данному разделу.}
		\begin{scnindent}
			\begin{scnrelfromlist}{используемые константы}
				\scnitem{отношение}
				\scnitem{ориентированное отношение}
				\scnitem{бинарное отношение}
				\scnitem{неролевое отношение}
				\scnitem{предисловие}
				\scnitem{фрагмент базы знаний}
				\scnitem{sc-текст}
			\end{scnrelfromlist}
		\end{scnindent}

	\scnheader{спецификация основной цели}
	\begin{scnrelfromset}{обобщенная декомпозиция}
		\scnitem{знак специфицируемого объекта}
		\scnitem{цель}
		\scnitem{дуга, связывающая специфицируемый объект с целью}
		\scnitem{дуга принадлежности отношению \textit{основная цель*}}
		\scnitem{основная цель*}
	\end{scnrelfromset}
	\scnrelfrom{шаблон}{шаблон для описания основной цели}
	\begin{scnindent}
		\scnrelfrom{иллюстрация}{\scnfileimage[20em]{Contents/part_kb/src/images/sd_basic_specification/goal.png}}
		\begin{scnindent}
			\scnidtf{SCg-текст. Шаблон для описания основной цели}
		\end{scnindent}
	\end{scnindent}

	\scnheader{основная цель*}
	\scnsubset{отношение}
	\scnsubset{бинарное отношение}
	\scnsubset{неролевое отношение}
	\scnsubset{ориентированное отношение}
	\scnsubset{антисимметричное отношение}
	\scnsubset{антитранзитивное отношение}
	\scnsubset{антирефлексивное отношение}
	\begin{scnrelfrom}{первый домен}
		{фрагмент базы знаний}
	\end{scnrelfrom}
	\begin{scnrelfrom}{второй домен}
		{sc-текст}
	\end{scnrelfrom}
	\begin{scnrelfromset}{область определения}
		\scnitem{фрагмент базы знаний}
		\scnitem{sc-текст}
	\end{scnrelfromset}
	\scntext{определение}{\textit{основная цель*} - бинарное ориентированное неролевое отношение, связывающее фрагмент базы знаний и некоторый sc-текст, являющийся основной целью данного раздела.}
		\begin{scnindent}
			\begin{scnrelfromlist}{используемые константы}
				\scnitem{отношение}
				\scnitem{ориентированное отношение}
				\scnitem{бинарное отношение}
				\scnitem{неролевое отношение}
				\scnitem{фрагмент базы знаний}
				\scnitem{цель}
				\scnitem{sc-текст}
			\end{scnrelfromlist}
		\end{scnindent}

	\scnheader{спецификация автора}
	\begin{scnrelfromset}{обобщенная декомпозиция}
		\scnitem{знак специфицируемого объекта}
		\scnitem{персона}
		\scnitem{дуга, связывающая специфицируемый объект с персоной}
		\scnitem{дуга, связывающая специфицируемый объект с }
		\scnitem{дуга принадлежности отношению \textit{автор*}}
		\scnitem{автор*}
	\end{scnrelfromset}
	\scnrelfrom{шаблон}{шаблон для описания автора}
	\begin{scnindent}
		\scnrelfrom{иллюстрация}{\scnfileimage[20em]{Contents/part_kb/src/images/sd_basic_specification/author.png}}
		\begin{scnindent}
			\scnidtf{SCg-текст. Шаблон для описания автора}
		\end{scnindent}
	\end{scnindent}

	\scnheader{автор*}
	\scnsubset{отношение}
	\scnsubset{бинарное отношение}
	\scnsubset{неролевое отношение}
	\scnsubset{ориентированное отношение}
	\scnsubset{антисимметричное отношение}
	\scnsubset{антитранзитивное отношение}
	\scnsubset{антирефлексивное отношение}
	\begin{scnrelfrom}{первый домен}
		{сущность}
	\end{scnrelfrom}
	\begin{scnrelfrom}{второй домен}
		{персона}
	\end{scnrelfrom}
	\begin{scnrelfromset}{область определения}
		\scnitem{сущность}
	\end{scnrelfromset}
	\scntext{определение}{\textit{автор*} - бинарное ориентированное неролевое отношение, связывающее некоторую сущность и персону, являющуюся автором данной сущности.}
		\begin{scnindent}
			\begin{scnrelfromlist}{используемые константы}
				\scnitem{отношение}
				\scnitem{ориентированное отношение}
				\scnitem{бинарное отношение}
				\scnitem{неролевое отношение}
				\scnitem{сущность}
				\scnitem{персона}
				\scnitem{автор}
			\end{scnrelfromlist}
		\end{scnindent}

	\scnheader{спецификация подразделов}
	\begin{scnrelfromset}{обобщенная декомпозиция}
		\scnitem{знак специфицируемого объекта}
		\scnitem{раздел базы знаний}
		\scnitem{дуга, связывающая специфицируемый объект с разделом базы знаний}
		\scnitem{дуга принадлежности отношению \textit{декомпозиция раздела*}}
		\scnitem{декомпозиция раздела*}
	\end{scnrelfromset}
	\scnrelfrom{шаблон}{шаблон для описания подразделов}
	\begin{scnindent}
		\scnrelfrom{иллюстрация}{\scnfileimage[20em]{Contents/part_kb/src/images/sd_basic_specification/subsection.png}}
		\begin{scnindent}
			\scnidtf{SCg-текст. Шаблон для описания подразделов}
		\end{scnindent}
	\end{scnindent}

	\scnheader{декомпозиция раздела*}
	\scnsubset{отношение}
	\scnsubset{бинарное отношение}
	\scnsubset{неролевое отношение}
	\scnsubset{ориентированное отношение}
	\scnsubset{антисимметричное отношение}
	\scnsubset{антитранзитивное отношение}
	\scnsubset{антирефлексивное отношение}
	\begin{scnrelfrom}{первый домен}
		{раздел базы знаний}
	\end{scnrelfrom}
	\begin{scnrelfrom}{второй домен}
		{раздел базы знаний}
	\end{scnrelfrom}
	\begin{scnrelfrom}{область определения}
		{раздел базы знаний}
	\end{scnrelfrom}
	\scntext{определение}{\textit{декомпозиция раздела*} - бинарное ориентированное неролевое отношение, связывающее два раздела базы знаний, обозначающее, что один раздел входит в состав другого раздела.}
		\begin{scnindent}
			\begin{scnrelfromlist}{используемые константы}
				\scnitem{отношение}
				\scnitem{ориентированное отношение}
				\scnitem{бинарное отношение}
				\scnitem{неролевое отношение}
				\scnitem{раздел базы знаний}
			\end{scnrelfromlist}
		\end{scnindent}

	\scnheader{спецификация предметной области}
	\begin{scnrelfromset}{обобщенная декомпозиция}
		\scnitem{знак специфицируемого объекта}
		\scnitem{предметная область}
		\scnitem{дуга, связывающая специфицируемый объект с предметной областью}
	\end{scnrelfromset}

	\scnheader{спецификация рассматриваемого вопроса}
	\begin{scnrelfromset}{обобщенная декомпозиция}
		\scnitem{знак специфицируемого объекта}
		\scnitem{вопрос}
		\scnitem{дуга, связывающая специфицируемый объект с вопросом}
		\scnitem{дуга принадлежности отношению \textit{рассматриваемый вопрос*}}
		\scnitem{рассматриваемый вопрос*}
	\end{scnrelfromset}
	\scnrelfrom{шаблон}{шаблон для описания рассматриваемого вопроса}
	\begin{scnindent}
		\scnrelfrom{иллюстрация}{\scnfileimage[20em]{Contents/part_kb/src/images/sd_basic_specification/issue.png}}
		\begin{scnindent}
			\scnidtf{SCg-текст. Шаблон для описания рассматриваемого вопроса}
		\end{scnindent}
	\end{scnindent}

	\scnheader{рассматриваемый вопрос*}
	\scnsubset{отношение}
	\scnsubset{бинарное отношение}
	\scnsubset{неролевое отношение}
	\scnsubset{ориентированное отношение}
	\scnsubset{антисимметричное отношение}
	\scnsubset{антитранзитивное отношение}
	\scnsubset{антирефлексивное отношение}
	\begin{scnrelfrom}{первый домен}
		{раздел базы знаний}
	\end{scnrelfrom}
	\begin{scnrelfrom}{второй домен}
		{вопрос}
	\end{scnrelfrom}
	\begin{scnrelfromset}{область определения}
		\scnitem{раздел баз знаний}
		\scnitem{вопрос}
	\end{scnrelfromset}
	\scntext{определение}{\textit{рассматриваемый вопрос*} - бинарное ориентированное неролевое отношение, связывающее раздел базы знаний и вопрос, рассматриваемый в данном разделе базы знаний.}
		\begin{scnindent}
			\begin{scnrelfromlist}{используемые константы}
				\scnitem{отношение}
				\scnitem{ориентированное отношение}
				\scnitem{бинарное отношение}
				\scnitem{неролевое отношение}
				\scnitem{раздел баз знаний}
				\scnitem{вопрос}
			\end{scnrelfromlist}
		\end{scnindent}
	
	\scnheader{спецификация ключевых элементов}
	\begin{scnrelfromset}{обобщенная декомпозиция}
		\scnitem{знак специфицируемого объекта}
		\scnitem{сущность}
		\scnitem{дуга, связывающая специфицируемый объект с сущностью}
		\scnitem{дуга принадлежности отношению \textit{ключевой sc-элемент\scnrolesign}}
		\scnitem{ключевой sc-элемент\scnrolesign}
	\end{scnrelfromset}
	\scnrelfrom{шаблон}{шаблон для описания ключевых sc-элементов}
	\begin{scnindent}
		\scnrelfrom{иллюстрация}{\scnfileimage[20em]{Contents/part_kb/src/images/sd_basic_specification/key_sc_element.png}}
		\begin{scnindent}
			\scnidtf{SCg-текст. Шаблон для описания ключевых sc-элементов}
		\end{scnindent}
	\end{scnindent}

	\scnheader{ключевой sc-элемент\scnrolesign}
	\scnsubset{отношение}
	\scnsubset{бинарное отношение}
	\scnsubset{ролевое отношение}
	\scnsubset{ориентированное отношение}
	\scnsubset{антисимметричное отношение}
	\scnsubset{антитранзитивное отношение}
	\scnsubset{антирефлексивное отношение}
	\begin{scnrelfrom}{первый домен}
		{раздел базы знаний}
	\end{scnrelfrom}
	\begin{scnrelfrom}{второй домен}
		{сущность}
	\end{scnrelfrom}
	\begin{scnrelfromset}{область определения}
		\scnitem{раздел базы знаний}
		\scnitem{сущность}
	\end{scnrelfromset}
	\scntext{определение}{\textit{ключевой sc-элемент\scnrolesign} - бинарное ориентированное ролевое отношение, связывающее раздел базы знаний и сущность, принадлежащую данному разделу базы знаний и являющуюся ключевым элементом данного раздела.}
		\begin{scnindent}
			\begin{scnrelfromlist}{используемые константы}
				\scnitem{отношение}
				\scnitem{ориентированное отношение}
				\scnitem{бинарное отношение}
				\scnitem{ролевое отношение}
				\scnitem{раздел базы знаний}
				\scnitem{сущность}
			\end{scnrelfromlist}
		\end{scnindent}
\end{scnsubstruct}

\scnheader{Подпункт 23.1.1.4 Формализация базовой спецификации предметной области}
\begin{scnsubstruct}
% \\\\\\\\\\\\\\\ ПРЕДМЕТНАЯ ОБЛАСТЬ
	\scnheader{базовая спецификация предметной области}
	\scntext{примечание}{Базовая спецификация предметной области отражает набор свойств, которые должны быть описаны для каждой предметной области}
	\begin{scnrelto}{обобщенная базовая спецификация}
		{предметная область}
	\end{scnrelto}
	\begin{scnrelfromvector}{обобщенная декомпозиция}
		\scnitem{спецификация частных предметных областей}
		\scnitem{спецификация максимального класса объектов исследования}
		\scnitem{спецификация классов объектов исследования}
		\scnitem{спецификация исследуемых отношений}
	\end{scnrelfromvector}
	% \scnhaselementrole{пример}{базовая спецификация предметной области}
	% \begin{scnindent}
	% 	\begin{scnrelto}{базовая спецификация}
	% 		{Предметная область треугольников}
	% 	\end{scnrelto}
	% \end{scnindent}

	\scnheader{спецификация частных предметных областей}
	\begin{scnrelfromset}{обобщенная декомпозиция}
		\scnitem{знак специфицируемого объекта}
		\scnitem{предметная область}
		\scnitem{дуга, связывающая специфицируемый объект с предметной областью}
		\scnitem{дуга принадлежности отношению \textit{частная предметная область*}}
		\scnitem{частная предметная область*}
	\end{scnrelfromset}
	\scnrelfrom{шаблон}{шаблон для описания частных предметных областей}
	\begin{scnindent}
		\scnrelfrom{иллюстрация}{\scnfileimage[20em]{Contents/part_kb/src/images/sd_basic_specification/subject_area.png}}
		\begin{scnindent}
			\scnidtf{SCg-текст. Шаблон для описания частных предметных областей}
		\end{scnindent}
	\end{scnindent}

	\scnheader{частная предметная область*}
	\scnsubset{отношение}
	\scnsubset{бинарное отношение}
	\scnsubset{неролевое отношение}
	\scnsubset{ориентированное отношение}
	\scnsubset{антисимметричное отношение}
	\scnsubset{антитранзитивное отношение}
	\scnsubset{антирефлексивное отношение}
	\begin{scnrelfrom}{первый домен}
		{предметная область}
	\end{scnrelfrom}
	\begin{scnrelfrom}{второй домен}
		{предметная область}
	\end{scnrelfrom}
	\begin{scnrelfrom}{область определения}
		{предметная область}
	\end{scnrelfrom}
	\scntext{определение}{\textit{частная предметная область*} - бинарное ориентированное неролевое отношение, с помощью которого задаётся иерархия предметных областей путем перехода от менее детального к более детальному рассмотрению соответствующих исследуемых понятий.}
		\begin{scnindent}
			\begin{scnrelfromlist}{используемые константы}
				\scnitem{отношение}
				\scnitem{ориентированное отношение}
				\scnitem{бинарное отношение}
				\scnitem{неролевое отношение}
				\scnitem{предметная область}
			\end{scnrelfromlist}
		\end{scnindent}

	\scnheader{спецификация максимального класса объектов исследования}
	\begin{scnrelfromset}{обобщенная декомпозиция}
		\scnitem{знак специфицируемого объекта}
		\scnitem{класс сущностей}
		\scnitem{дуга, связывающая специфицируемый объект с классом сущностей}
		\scnitem{дуга принадлежности отношению \textit{максимальный класс объектов исследования\scnrolesign}}
		\scnitem{максимальный класс объектов исследования\scnrolesign}
	\end{scnrelfromset}
	\scnrelfrom{шаблон}{шаблон для описания максимального класса объектов исследования}
	\begin{scnindent}
		\scnrelfrom{иллюстрация}{\scnfileimage[20em]{Contents/part_kb/src/images/sd_basic_specification/maximum_class.png}}
		\begin{scnindent}
			\scnidtf{SCg-текст. Шаблон для описания максимального класса объектов исследовния}
		\end{scnindent}
	\end{scnindent}

	\scnheader{максимальный класс объектов исследования\scnrolesign}
	\scnsubset{отношение}
	\scnsubset{бинарное отношение}
	\scnsubset{ролевое отношение}
	\scnsubset{ориентированное отношение}
	\scnsubset{антисимметричное отношение}
	\scnsubset{антитранзитивное отношение}
	\scnsubset{антирефлексивное отношение}
	\begin{scnrelfrom}{первый домен}
		{предметная область}
	\end{scnrelfrom}
	\begin{scnrelfrom}{второй домен}
		{класс сущностей}
	\end{scnrelfrom}
	\begin{scnrelfromset}{область определения}
		\scnitem{предметная область}
		\scnitem{класс сущностей}
	\end{scnrelfromset}
	\scntext{определение}{\textit{максимальный класс объектов исследования\scnrolesign} - это ролевое бинарное отношение, указывающее в рамках предметной области на множество, являющееся максимальным классом объектов исследования данной предметной области, то есть на такое исследуемое понятие\scnrolesign, для которого в рамках данной предметной области не существует другого \scnrolesign исследуемого понятия\scnrolesign, которое бы являлось надмножеством для данного.}
		\begin{scnindent}
			\begin{scnrelfromlist}{используемые константы}
				\scnitem{отношение}
				\scnitem{ориентированное отношение}
				\scnitem{бинарное отношение}
				\scnitem{ролевое отношение}
				\scnitem{предметная область}
				\scnitem{класс сущностей}
				\scnitem{множество}
				\scnitem{исследуемое понятие\scnrolesign}
				\scnitem{надмножество}
			\end{scnrelfromlist}
		\end{scnindent}

	\scnheader{спецификация классов объектов исследования}
	\begin{scnrelfromset}{обобщенная декомпозиция}
		\scnitem{знак специфицируемого объекта}
		\scnitem{класс сущностей}
		\scnitem{дуга, связывающая специфицируемый объект с классом сущностей}
		\scnitem{дуга принадлежности отношению \textit{класс объектов исследования\scnrolesign}}
		\scnitem{класс объектов исследования\scnrolesign}
	\end{scnrelfromset}
	\scnrelfrom{шаблон}{шаблон для описания немаксиамльных классов объектов исследования}
	\begin{scnindent}
		\scnrelfrom{иллюстрация}{\scnfileimage[20em]{Contents/part_kb/src/images/sd_basic_specification/non_maximal_class.png}}
		\begin{scnindent}
			\scnidtf{SCg-текст. Шаблон для описания немаксимальных классов объектов исследования}
		\end{scnindent}
	\end{scnindent}

	\scnheader{класс объектов исследования\scnrolesign}
	\scnsubset{отношение}
	\scnsubset{бинарное отношение}
	\scnsubset{ролевое отношение}
	\scnsubset{ориентированное отношение}
	\scnsubset{антисимметричное отношение}
	\scnsubset{антитранзитивное отношение}
	\scnsubset{антирефлексивное отношение}
	\begin{scnrelfrom}{первый домен}
		{предметная область}
	\end{scnrelfrom}
	\begin{scnrelfrom}{второй домен}
		{класс сущностей}
	\end{scnrelfrom}
	\begin{scnrelfromset}{область определения}
		\scnitem{предметная область}
		\scnitem{класс сущностей}
	\end{scnrelfromset}
	\scntext{определение}{\textit{класс объектов исследования\scnrolesign} - бинарное ориентированное ролевое отношение, связывающее предметную область и класс сущностей, принадлежащий данной предметной области, указывающее, что в рамках данной предметной области существует другое исследуемое понятие\scnrolesign, которое бы являлось надмножеством для данного.}
		\begin{scnindent}
			\begin{scnrelfromlist}{используемые константы}
				\scnitem{отношение}
				\scnitem{ориентированное отношение}
				\scnitem{бинарное отношение}
				\scnitem{ролевое отношение}
				\scnitem{предметная область}
				\scnitem{класс сущностей}
				\scnitem{множество}
				\scnitem{исследуемое понятие\scnrolesign}
				\scnitem{надмножество}
			\end{scnrelfromlist}
		\end{scnindent}
	
	\scnheader{спецификация исследуемых отношений}
	\begin{scnrelfromset}{обобщенная декомпозиция}
		\scnitem{знак специфицируемого объекта}
		\scnitem{отношение}
		\scnitem{дуга, связывающая специфицируемый объект с отношением}
		\scnitem{дуга принадлежности отношению \textit{исследуемое отношение\scnrolesign}}
		\scnitem{исследуемое отношение\scnrolesign}
	\end{scnrelfromset}
	\scnrelfrom{шаблон}{шаблон для описания исследуемых отношений}
	\begin{scnindent}
		\scnrelfrom{иллюстрация}{\scnfileimage[20em]{Contents/part_kb/src/images/sd_basic_specification/studied_relation.png}}
		\begin{scnindent}
			\scnidtf{SCg-текст. Шаблон для описания исследуемых отношений}
		\end{scnindent}
	\end{scnindent}

	\scnheader{исследуемое отношение\scnrolesign}
	\scnsubset{отношение}
	\scnsubset{бинарное отношение}
	\scnsubset{неролевое отношение}
	\scnsubset{ориентированное отношение}
	\scnsubset{антисимметричное отношение}
	\scnsubset{антитранзитивное отношение}
	\scnsubset{антирефлексивное отношение}
	\begin{scnrelfrom}{первый домен}
		{предметная область}
	\end{scnrelfrom}
	\begin{scnrelfrom}{второй домен}
		{отношение}
	\end{scnrelfrom}
	\begin{scnrelfromset}{область определения}
		\scnitem{предметная область}
		\scnitem{отношение}
	\end{scnrelfromset}
	\scntext{определение}{\textit{исследуемое отношение\scnrolesign} - бинарное ориентированное ролевое отношение, указывающее в рамках предметной области на связки, являющееся исследуемым отношением данной предметной области, то есть таким отношением, все связки которого являются элементами этой предметной области. При этом элементы таких связок также входят в данную предметную область, но в общем случае могут не являться элементами исследуемых понятий\scnrolesign данной предметной области.}
		\begin{scnindent}
			\begin{scnrelfromlist}{используемые константы}
				\scnitem{отношение}
				\scnitem{ориентированное отношение}
				\scnitem{бинарное отношение}
				\scnitem{ролевое отношение}
				\scnitem{предметная область}
				\scnitem{отношение}
				\scnitem{множество}
				\scnitem{исследуемое понятие\scnrolesign}
				\scnitem{связка}
			\end{scnrelfromlist}
		\end{scnindent}	
\end{scnsubstruct}

\scnheader{Подпункт 23.1.1.5 Формализация базовой спецификации проекта}
\begin{scnsubstruct}
% \\\\\\\\\\\\\\ ПРОЕКТ
	\scnheader{базовая спецификация проекта}
	\scntext{примечание}{Базовая спецификация проекта отражает набор свойств, которые должны быть описаны для каждого проекта}
	\begin{scnrelto}{обобщенная базовая спецификация}
		{проект}
	\end{scnrelto}
	\begin{scnrelfromvector}{обобщенная декомпозиция}
		\scnitem{спецификация автора}
		\scnitem{спецификация задачи}
		\scnitem{спецификация назначения}
		\scnitem{спецификация компонентов}
		\scnitem{спецификация способ установки}
		\scnitem{спецификация конечного продукта}
	\end{scnrelfromvector}

	\scnheader{спецификация задачи}
	\begin{scnrelfromset}{обобщенная декомпозиция}
		\scnitem{знак специфицируемого объекта}
		\scnitem{задача}
		\scnitem{дуга, связывающая специфицируемый объект с задачей}
		\scnitem{дуга принадлежности отношению \textit{задача*}}
		\scnitem{задача*}
	\end{scnrelfromset}
	\scnrelfrom{шаблон}{шаблон для описания задачи}
	\begin{scnindent}
		\scnrelfrom{иллюстрация}{\scnfileimage[20em]{Contents/part_kb/src/images/sd_basic_specification/task.png}}
		\begin{scnindent}
			\scnidtf{SCg-текст. Шаблон для описания задачи}
		\end{scnindent}
	\end{scnindent}

	\scnheader{задача*}
	\scnsubset{отношение}
	\scnsubset{бинарное отношение}
	\scnsubset{неролевое отношение}
	\scnsubset{ориентированное отношение}
	\scnsubset{антисимметричное отношение}
	\scnsubset{антитранзитивное отношение}
	\scnsubset{антирефлексивное отношение}
	\begin{scnrelfrom}{первый домен}
		{проект}
	\end{scnrelfrom}
	\begin{scnrelfrom}{второй домен}
		{задача}
	\end{scnrelfrom}
	\begin{scnrelfromset}{область определения}
		\scnitem{проект}
		\scnitem{задача}
	\end{scnrelfromset}
	\scntext{определение}{\textit{задача*} - бинарное ориентированное неролевое отношение, связывающее проект и задачу, решаемую в данном проекте.}
		\begin{scnindent}
			\begin{scnrelfromlist}{используемые константы}
				\scnitem{отношение}
				\scnitem{ориентированное отношение}
				\scnitem{бинарное отношение}
				\scnitem{неролевое отношение}
				\scnitem{задача}
				\scnitem{проект}
			\end{scnrelfromlist}
		\end{scnindent}
	
	\scnheader{спецификация назначения}
	\begin{scnrelfromset}{обобщенная декомпозиция}
		\scnitem{знак специфицируемого объекта}
		\scnitem{назначение}
		\scnitem{дуга, связывающая специфицируемый объект с назначением}
		\scnitem{дуга принадлежности отношению \textit{назначение*}}
		\scnitem{назначение*}
	\end{scnrelfromset}
	\scnrelfrom{шаблон}{шаблон для описания назначения}
	\begin{scnindent}
		\scnrelfrom{иллюстрация}{\scnfileimage[20em]{Contents/part_kb/src/images/sd_basic_specification/purpose.png}}
		\begin{scnindent}
			\scnidtf{SCg-текст. Шаблон для описания назначения}
		\end{scnindent}
	\end{scnindent}

	\scnheader{назначение*}
	\scnsubset{отношение}
	\scnsubset{бинарное отношение}
	\scnsubset{неролевое отношение}
	\scnsubset{ориентированное отношение}
	\scnsubset{антисимметричное отношение}
	\scnsubset{антитранзитивное отношение}
	\scnsubset{антирефлексивное отношение}
	\begin{scnrelfrom}{первый домен}
		{сущность}
	\end{scnrelfrom}
	\begin{scnrelfrom}{второй домен}
		{sc-текст}
	\end{scnrelfrom}
	\begin{scnrelfromset}{область определения}
		\scnitem{сущность}
		\scnitem{sc-текст}
	\end{scnrelfromset}
	\scntext{определение}{\textit{назначение*} - бинарное ориентированное неролевое отношение, связывающее некоторую сущность базы знаний и sc-текст, показывающий, для чего была создана данная сущность.}
		\begin{scnindent}
			\begin{scnrelfromlist}{используемые константы}
				\scnitem{отношение}
				\scnitem{ориентированное отношение}
				\scnitem{бинарное отношение}
				\scnitem{неролевое отношение}
				\scnitem{сущность}
				\scnitem{sc-текст}
			\end{scnrelfromlist}
		\end{scnindent}
	
	\scnheader{спецификация компонентов}
	\begin{scnrelfromset}{обобщенная декомпозиция}
		\scnitem{знак специфицируемого объекта}
		\scnitem{сущность}
		\scnitem{дуга, связывающая специфицируемый объект с сущностью}
		\scnitem{дуга принадлежности отношению \textit{компонент*}}
		\scnitem{компонент*}
	\end{scnrelfromset}
	\scnrelfrom{шаблон}{шаблон для описания компонентов}
	\begin{scnindent}
		\scnrelfrom{иллюстрация}{\scnfileimage[20em]{Contents/part_kb/src/images/sd_basic_specification/component.png}}
		\begin{scnindent}
			\scnidtf{SCg-текст. Шаблон для описания компонентов}
		\end{scnindent}
	\end{scnindent}

	\scnheader{компонент*}
	\scnsubset{отношение}
	\scnsubset{бинарное отношение}
	\scnsubset{неролевое отношение}
	\scnsubset{ориентированное отношение}
	\scnsubset{антисимметричное отношение}
	\scnsubset{антитранзитивное отношение}
	\scnsubset{антирефлексивное отношение}
	\begin{scnrelfrom}{первый домен}
		{сущность}
	\end{scnrelfrom}
	\begin{scnrelfrom}{второй домен}
		{сущность}
	\end{scnrelfrom}
	\begin{scnrelfrom}{область определения}
		{сущность}
	\end{scnrelfrom}
	\scntext{определение}{\textit{компонент*} - бинарное ориентированное неролевое отношение, связывающее сущности и обозначающее, что одна сущность является компонетом другой сущности.}
		\begin{scnindent}
			\begin{scnrelfromlist}{используемые константы}
				\scnitem{отношение}
				\scnitem{ориентированное отношение}
				\scnitem{бинарное отношение}
				\scnitem{неролевое отношение}
				\scnitem{компонент}
				\scnitem{сущность}
			\end{scnrelfromlist}
		\end{scnindent}
	
	\scnheader{спецификация способа установки}
	\begin{scnrelfromset}{обобщенная декомпозиция}
		\scnitem{знак специфицируемого объекта}
		\scnitem{инструкция}
		\scnitem{дуга, связывающая специфицируемый объект с инструкцией}
		\scnitem{дуга принадлежности отношению \textit{способ установки*}}
		\scnitem{способ установки*}
	\end{scnrelfromset}

	% данное отношение используется в могократно используемом компоненте, можно ли считать, что множество кногократно используемых компонентов включается во множество проектов?
	\scnheader{способ установки*}
	\scnsubset{отношение}
	\scnsubset{бинарное отношение}
	\scnsubset{неролевое отношение}
	\scnsubset{ориентированное отношение}
	\scnsubset{антисимметричное отношение}
	\scnsubset{антитранзитивное отношение}
	\scnsubset{антирефлексивное отношение}
	\begin{scnrelfrom}{первый домен}
		{проект}
	\end{scnrelfrom}
	\begin{scnrelfrom}{второй домен}
		{инструкция}
	\end{scnrelfrom}
	\begin{scnrelfromset}{область определения}
		\scnitem{проект}
		\scnitem{инструкция}
	\end{scnrelfromset}
	\scntext{определение}{\textit{способ установки*} - бинарное ориентированное неролевое отношение, связывающее проект и инструкцию к установке, показывающее, каким образом необходимо устанавливать проект.}
		\begin{scnindent}
			\begin{scnrelfromlist}{используемые константы}
				\scnitem{отношение}
				\scnitem{ориентированное отношение}
				\scnitem{бинарное отношение}
				\scnitem{неролевое отношение}
				\scnitem{проект}
				\scnitem{инструкция}
			\end{scnrelfromlist}
		\end{scnindent}
	
	\scnheader{спецификация конечного продукта}
	\begin{scnrelfromset}{обобщенная декомпозиция}
		\scnitem{знак специфицируемого объекта}
		\scnitem{продукт}
		\scnitem{дуга, связывающая специфицируемый объект с продуктом}
		\scnitem{дуга принадлежности отношению \textit{продукт*}}
		\scnitem{продукт*}
	\end{scnrelfromset}
	\scnrelfrom{шаблон}{шаблон для описания конечного продукта}
	\begin{scnindent}
		\scnrelfrom{иллюстрация}{\scnfileimage[20em]{Contents/part_kb/src/images/sd_basic_specification/final_product.png}}
		\begin{scnindent}
			\scnidtf{SCg-текст. Шаблон для описания конечного продукта}
		\end{scnindent}
	\end{scnindent}

	% данное отношение используется в ostis-системе, как коректно обозначить первый домен? можно же туда множество добавлять, но не уверена, как это правильнее сделать
	\scnheader{продукт*}
	\scnsubset{отношение}
	\scnsubset{бинарное отношение}
	\scnsubset{неролевое отношение}
	\scnsubset{ориентированное отношение}
	\scnsubset{антисимметричное отношение}
	\scnsubset{антитранзитивное отношение}
	\scnsubset{антирефлексивное отношение}
	\begin{scnrelfrom}{первый домен}
		{проект}
	\end{scnrelfrom}
	\begin{scnrelfrom}{второй домен}
		{продукт}
	\end{scnrelfrom}
	\begin{scnrelfromset}{область определения}
		\scnitem{проект}
		\scnitem{конечный продукт}
	\end{scnrelfromset}
	\scntext{определение}{\textit{продукт*} - бинарное ориентированное неролевое отношение, связывающее проект и некоторый продукт, показывающее, что является итоговым продуктом после реализации проекта.}
		\begin{scnindent}
			\begin{scnrelfromlist}{используемые константы}
				\scnitem{отношение}
				\scnitem{ориентированное отношение}
				\scnitem{бинарное отношение}
				\scnitem{неролевое отношение}
				\scnitem{проект}
				\scnitem{продукт}
				\scnitem{реализация}
			\end{scnrelfromlist}
		\end{scnindent}
\end{scnsubstruct}

\scnheader{Подпункт 23.1.1.6 Формализация базовой спецификации многократно используемого компонента}
\begin{scnsubstruct}
% \\\\\\\\\\\\\ МНОГОКРАТНО ИСПОЛЬЗУЕМЫЙ КОМПОНЕНТ
	\scnheader{базовая спецификация многократно используемого компонента}
	\scntext{примечание}{Базовая спецификация многократно используемого компонента отражает набор свойств, которые должны быть описаны для каждого многократно используемого компонента}
	\begin{scnrelto}{обобщенная базовая спецификация}
		{многократно используемый компонент}
	\end{scnrelto}
	\begin{scnrelfromvector}{обобщенная декомпозиция}
		\scnitem{спецификация назначения}
		\scnitem{спецификация автора}
		\scnitem{спецификация ссылки}
		\scnitem{спецификация используемого языка представления методов}
		\scnitem{спецификация зависимостей}
		\scnitem{спецификация частей}
		\scnitem{спецификация способа установки}
	\end{scnrelfromvector}

	\scnheader{спецификация ссылки}
	\begin{scnrelfromset}{обобщенная декомпозиция}
		\scnitem{знак специфицируемого объекта}
		\scnitem{интернет-ссылка}
		\scnitem{дуга, связывающая специфицируемый объект со ссылкой}
		\scnitem{дуга принадлежности отношению \textit{url*}}
		\scnitem{url*}
	\end{scnrelfromset}
	\scnrelfrom{шаблон}{шаблон для описания интернет-ссылки}
	\begin{scnindent}
		\scnrelfrom{иллюстрация}{\scnfileimage[20em]{Contents/part_kb/src/images/sd_basic_specification/url.png}}
		\begin{scnindent}
			\scnidtf{SCg-текст. Шаблон для описания интернет-ссылки}
		\end{scnindent}
	\end{scnindent}

	\scnheader{url*}
	\scnsubset{отношение}
	\scnsubset{бинарное отношение}
	\scnsubset{неролевое отношение}
	\scnsubset{ориентированное отношение}
	\scnsubset{антисимметричное отношение}
	\scnsubset{антитранзитивное отношение}
	\scnsubset{антирефлексивное отношение}
	\begin{scnrelfrom}{первый домен}
		{сущность}
	\end{scnrelfrom}
	\begin{scnrelfrom}{второй домен}
		{интернет-ссылка}
	\end{scnrelfrom}
	\begin{scnrelfromset}{область определения}
		\scnitem{сущность}
		\scnitem{интернет-ссылка}
	\end{scnrelfromset}
	\scntext{определение}{\textit{url*} - бинарное ориентированное неролевое отношение, связывающее некоторую сущность с её интернет-ссылкой, описывающей местонахождения исходных данных сущности.}
		\begin{scnindent}
			\begin{scnrelfromlist}{используемые константы}
				\scnitem{отношение}
				\scnitem{ориентированное отношение}
				\scnitem{бинарное отношение}
				\scnitem{неролевое отношение}
				\scnitem{сущность}
				\scnitem{интернет-ссылка}
				\scnitem{местонахождение}
			\end{scnrelfromlist}
		\end{scnindent}
	
	\scnheader{спецификация используемого языка представления методов}
	\begin{scnrelfromset}{обобщенная декомпозиция}
		\scnitem{знак специфицируемого объекта}
		\scnitem{язык программирования}
		\scnitem{дуга, связывающая специфицируемый объект с языком программирования}
		\scnitem{дуга принадлежности отношению \textit{язык представления методов*}}
		\scnitem{язык представаления методов*}
	\end{scnrelfromset}
	\scnrelfrom{шаблон}{шаблон для описания используемого языка представления методов}
	\begin{scnindent}
		\scnrelfrom{иллюстрация}{\scnfileimage[20em]{Contents/part_kb/src/images/sd_basic_specification/language.png}}
		\begin{scnindent}
			\scnidtf{SCg-текст. Шаблон для описания используемого языка представления методов}
		\end{scnindent}
	\end{scnindent}

	\scnheader{язык представления методов*}
	\scnsubset{отношение}
	\scnsubset{бинарное отношение}
	\scnsubset{неролевое отношение}
	\scnsubset{ориентированное отношение}
	\scnsubset{антисимметричное отношение}
	\scnsubset{антитранзитивное отношение}
	\scnsubset{антирефлексивное отношение}
	\begin{scnrelfrom}{первый домен}
		{многократно используемый компонент}
	\end{scnrelfrom}
	\begin{scnrelfrom}{второй домен}
		{язык программирования}
	\end{scnrelfrom}
	\begin{scnrelfromset}{область определения}
		\scnitem{многократно используемый компонент}
		\scnitem{язык программирования}
	\end{scnrelfromset}
	\scntext{определение}{\textit{язык представаления методов*} - бинарное ориентированное неролевое отношение, связывающее многократно используемый компонент и язык программирования, на котором реализованы методы в данном многократно используемом компоненте.}
		\begin{scnindent}
			\begin{scnrelfromlist}{используемые константы}
				\scnitem{отношение}
				\scnitem{ориентированное отношение}
				\scnitem{бинарное отношение}
				\scnitem{неролевое отношение}
				\scnitem{многократно используемый компонент}
				\scnitem{язык программирования}
				\scnitem{метод}
			\end{scnrelfromlist}
		\end{scnindent}
	
	\scnheader{спецификация зависимостей}
	\begin{scnrelfromset}{обобщенная декомпозиция}
		\scnitem{знак специфицируемого объекта}
		\scnitem{многократно используемый компонент}
		\scnitem{дуга, связывающая специфицируемый объект с }
		\scnitem{дуга принадлежности отношению \textit{зависимости*}}
		\scnitem{зависимости*}
	\end{scnrelfromset}
	\scnrelfrom{шаблон}{шаблон для описания зависимостей}
	\begin{scnindent}
		\scnrelfrom{иллюстрация}{\scnfileimage[20em]{Contents/part_kb/src/images/sd_basic_specification/dependence.png}}
		\begin{scnindent}
			\scnidtf{SCg-текст. Шаблон для описания зависимостей}
		\end{scnindent}
	\end{scnindent}

	\scnheader{зависимости*}
	\scnsubset{отношение}
	\scnsubset{бинарное отношение}
	\scnsubset{неролевое отношение}
	\scnsubset{ориентированное отношение}
	\scnsubset{антисимметричное отношение}
	\scnsubset{антитранзитивное отношение}
	\scnsubset{антирефлексивное отношение}
	\begin{scnrelfrom}{первый домен}
		{многократно используемый компонент}
	\end{scnrelfrom}
	\begin{scnrelfrom}{второй домен}
		{многократно используемый компонент}
	\end{scnrelfrom}
	\begin{scnrelfrom}{область определения}
		{многократно используемый компонент}
	\end{scnrelfrom}
	\scntext{определение}{\textit{зависимость*} - бинарное ориентированное неролевое отношение, связывающее многократно используемый компонент и другой многократно используемый компонент, являющийся зависимостью данного компонента, то есть компонентом необходимым для правильной установки и работы данного компонента.}
		\begin{scnindent}
			\begin{scnrelfromlist}{используемые константы}
				\scnitem{отношение}
				\scnitem{ориентированное отношение}
				\scnitem{бинарное отношение}
				\scnitem{неролевое отношение}
				\scnitem{многократно используемый компонент}
				\scnitem{зависимость}
				\scnitem{установка}
				\scnitem{работа}
			\end{scnrelfromlist}
		\end{scnindent}
	
	\scnheader{спецификация частей}
	\begin{scnrelfromset}{обобщенная декомпозиция}
		\scnitem{знак специфицируемого объекта}
		\scnitem{сущность}
		\scnitem{дуга, связывающая специфицируемый объект с сущностью, которая является его частью}
		\scnitem{дуга принадлежности отношению \textit{часть*}}
		\scnitem{часть*}
	\end{scnrelfromset}
	\scnrelfrom{шаблон}{шаблон для описания частей}
	\begin{scnindent}
		\scnrelfrom{иллюстрация}{\scnfileimage[20em]{Contents/part_kb/src/images/sd_basic_specification/part.png}}
		\begin{scnindent}
			\scnidtf{SCg-текст. Шаблон для описания частей}
		\end{scnindent}
	\end{scnindent}

	\scnheader{часть*}
	\scnsubset{отношение}
	\scnsubset{бинарное отношение}
	\scnsubset{неролевое отношение}
	\scnsubset{ориентированное отношение}
	\scnsubset{антисимметричное отношение}
	\scnsubset{антитранзитивное отношение}
	\scnsubset{антирефлексивное отношение}
	\begin{scnrelfrom}{первый домен}
		{сущность}
	\end{scnrelfrom}
	\begin{scnrelfrom}{второй домен}
		{сущность}
	\end{scnrelfrom}
	\begin{scnrelfrom}{область определения}
		{сущность}
	\end{scnrelfrom}
	\scntext{определение}{\textit{часть*} - бинарное ориентированное неролевое отношение, связывающее две сущности и показывающее, что она сущность является частью другой сущности.}
		\begin{scnindent}
			\begin{scnrelfromlist}{используемые константы}
				\scnitem{отношение}
				\scnitem{ориентированное отношение}
				\scnitem{бинарное отношение}
				\scnitem{неролевое отношение}
				\scnitem{сущность}
				\scnitem{часть}
			\end{scnrelfromlist}
		\end{scnindent}
\end{scnsubstruct}

\scnheader{Подпункт 23.1.1.7 Формализация базовой спецификации ostis-системы}
\begin{scnsubstruct}
% \\\\\\\\\\\\\\\\\\\\\ OSTIS-СИСТЕМА
% тут потом ещё будут дополнительно описаны базовые спецификации для различных видов ostis-систем (производственных, медицинских, обучающих и тд + метасистемы)
	\scnheader{базовая спецификация ostis-системы}
	\scntext{примечание}{Базовая спецификация ostis-системы отражает набор свойств, которые должны быть описаны для каждой ostis-системы}
	\begin{scnrelto}{обобщенная базовая спецификация}
		{ostis-система}
	\end{scnrelto}
	\begin{scnrelfromvector}{обобщенная декомпозиция}
		\scnitem{спецификация декомпозиции}
		\scnitem{спецификация принципов реализации}
		\scnitem{спецификация подсистем}
		\scnitem{спецификация ссылка}
		\scnitem{спецификация конечного продукта}
	\end{scnrelfromvector}
	\scnhaselementrole{пример}{базовая спецификация Метасистемы OSTIS}
	\begin{scnindent}
		\begin{scnrelto}{базовая спецификация}
			{Метасистема OSTIS}
		\end{scnrelto}
	\end{scnindent}
	\scntext{примечание}{Каждая ostis-система должна иметь базовую спецификацию, имеющую:
		\begin{itemize}
			\item предметонезависимый аспект;
			\item специализированный аспект.
		\end{itemize}}

	% тут есть вопросы с формализацией
	\scnheader{спецификация декомпозиции} 
	\begin{scnrelfromset}{обобщенная декомпозиция}
		\scnitem{знак специфицируемого объекта}
		\scnitem{знак объекта декомпозиции}
		\scnitem{дуга, связывающая специфицируемый объект с объектом декомпозиции}
		\scnitem{дуга принадлежности отношению \textit{декомпозиция*}}
		\scnitem{декомпозиция*}
	\end{scnrelfromset}
	
	\scnheader{спецификация принципов реализации}
	\begin{scnrelfromset}{обобщенная декомпозиция}
		\scnitem{знак специфицируемого объекта}
		\scnitem{принцип реализации}
		\scnitem{дуга, связывающая специфицируемый объект с принципом реализации}
		\scnitem{дуга принадлежности отношению \textit{принципы реализации*}}
		\scnitem{принципы реализации*}
	\end{scnrelfromset}

	\scnheader{принципы реализации*}
	\scnsubset{отношение}
	\scnsubset{бинарное отношение}
	\scnsubset{неролевое отношение}
	\scnsubset{ориентированное отношение}
	\scnsubset{антисимметричное отношение}
	\scnsubset{антитранзитивное отношение}
	\scnsubset{антирефлексивное отношение}
	\begin{scnrelfrom}{первый домен}
		{система}
	\end{scnrelfrom}
	\begin{scnrelfrom}{второй домен}
		{файл ostis-системы}
	\end{scnrelfrom}
	\begin{scnrelfromset}{область определения}
		\scnitem{система}
		\scnitem{файл ostis-системы}
	\end{scnrelfromset}
	\scntext{определение}{\textit{принципы реализации*} - бинарное ориентированное неролевое отношение, связывающее систему и множество файлов ostis-системы, представляющие собой описания, уточняющие различные моменты реализации системы.}
		\begin{scnindent}
			\begin{scnrelfromlist}{используемые константы}
				\scnitem{отношение}
				\scnitem{ориентированное отношение}
				\scnitem{бинарное отношение}
				\scnitem{неролевое отношение}
				\scnitem{принципы реализации}
				\scnitem{файл ostis-системы}
				\scnitem{реализация}
				\scnitem{момент}
			\end{scnrelfromlist}
		\end{scnindent}
	
	\scnheader{спецификация подсистем}
	\begin{scnrelfromset}{обобщенная декомпозиция}
		\scnitem{знак специфицируемого объекта}
		\scnitem{система}
		\scnitem{дуга, связывающая специфицируемый объект с системой}
		\scnitem{дуга принадлежности отношению \textit{подсистема*}}
		\scnitem{подсистема*}
	\end{scnrelfromset}
	\scnrelfrom{шаблон}{шаблон для описания подсистем}
	\begin{scnindent}
		\scnrelfrom{иллюстрация}{\scnfileimage[20em]{Contents/part_kb/src/images/sd_basic_specification/subsystem.png}}
		\begin{scnindent}
			\scnidtf{SCg-текст. Шаблон для описания подсистем}
		\end{scnindent}
	\end{scnindent}

	\scnheader{подсистема*}
	\scnsubset{отношение}
	\scnsubset{бинарное отношение}
	\scnsubset{неролевое отношение}
	\scnsubset{ориентированное отношение}
	\scnsubset{антисимметричное отношение}
	\scnsubset{антитранзитивное отношение}
	\scnsubset{антирефлексивное отношение}
	\begin{scnrelfrom}{первый домен}
		{система}
	\end{scnrelfrom}
	\begin{scnrelfrom}{второй домен}
		{система}
	\end{scnrelfrom}
	\begin{scnrelfrom}{область определения}
		{система}
	\end{scnrelfrom}
	\scntext{определение}{\textit{подсистема*} - бинарное ориентированное неролевое отношение, связывающее систему и другую систему, входящую в состав данной системы.}
		\begin{scnindent}
			\begin{scnrelfromlist}{используемые константы}
				\scnitem{отношение}
				\scnitem{ориентированное отношение}
				\scnitem{бинарное отношение}
				\scnitem{неролевое отношение}
				\scnitem{система}
				\scnitem{входить в состав*}
			\end{scnrelfromlist}
		\end{scnindent}
\end{scnsubstruct}

\scnheader{Подпункт 23.1.1.8 Формализация базовой спецификации персоны}
\begin{scnsubstruct}
% \\\\\\\\\\\\\\\ ПЕРСОНА
	\scnheader{базовая спецификация персоны}
	\scntext{примечание}{Базовая спецификация персоны отражает набор свойств, которые должны быть описаны для каждой персоны}
	\begin{scnrelto}{обобщенная базовая спецификация}
		{персона}
	\end{scnrelto}
	\begin{scnrelfromvector}{обобщенная декомпозиция}
		\scnitem{спецификация ФИО}
		\scnitem{спецификация контактной информации}
		\scnitem{спецификация роли}
	\end{scnrelfromvector}

	\scnheader{спецификация ФИО}
	\begin{scnrelfromset}{обобщенная декомпозиция}
		\scnitem{знак специфицируемого объекта}
		\scnitem{ФИО}
		\scnitem{дуга, связывающая специфицируемый объект с ФИО}
		\scnitem{дуга принадлежности отношению \textit{ФИО*}}
		\scnitem{ФИО*}
	\end{scnrelfromset}

	\scnheader{ФИО*}
	\scnsubset{отношение}
	\scnsubset{бинарное отношение}
	\scnsubset{неролевое отношение}
	\scnsubset{ориентированное отношение}
	\scnsubset{антисимметричное отношение}
	\scnsubset{антитранзитивное отношение}
	\scnsubset{антирефлексивное отношение}
	\begin{scnrelfrom}{первый домен}
		{персона}
	\end{scnrelfrom}
	\begin{scnrelfrom}{второй домен}
		{файл ostis-системы}
	\end{scnrelfrom}
	\begin{scnrelfromset}{область определения}
		\scnitem{персона}
		\scnitem{файл ostis-системы}
	\end{scnrelfromset}
	\scntext{определение}{\textit{ФИО*} - - бинарное ориентированное неролевое отношение, связывающее персону и файл ostis-системы с фамилией, именем и отчеством этой персоны.}
		\begin{scnindent}
			\begin{scnrelfromlist}{используемые константы}
				\scnitem{отношение}
				\scnitem{ориентированное отношение}
				\scnitem{бинарное отношение}
				\scnitem{неролевое отношение}
				\scnitem{персона}
				\scnitem{файл ostis-системы}
			\end{scnrelfromlist}
		\end{scnindent}

	\scnheader{спецификация контактной информации}
	\begin{scnrelfromset}{обобщенная декомпозиция}
		\scnitem{знак специфицируемого объекта}
		\scnitem{контактная информация}
		\scnitem{дуга, связывающая специфицируемый объект с контактной информацией}
		\scnitem{дуга принадлежности отношению \textit{контактная информация*}}
		\scnitem{контактная информация*}
	\end{scnrelfromset}

	\scnheader{контактная информация*}
	\scnsubset{отношение}
	\scnsubset{бинарное отношение}
	\scnsubset{неролевое отношение}
	\scnsubset{ориентированное отношение}
	\scnsubset{антисимметричное отношение}
	\scnsubset{антитранзитивное отношение}
	\scnsubset{антирефлексивное отношение}
	\begin{scnrelfrom}{первый домен}
		{персона}
	\end{scnrelfrom}
	\begin{scnrelfrom}{второй домен}
		{файл ostis-системы}
	\end{scnrelfrom}
	\begin{scnrelfromset}{область определения}
		\scnitem{персона}
		\scnitem{файл ostis-системы}
	\end{scnrelfromset}
	\scntext{определение}{\textit{контактная информация*} - бинарное ориентированное неролевое отношение, связывающее персону и файл ostis-системы с контактной информацией данной персоны.}
		\begin{scnindent}
			\begin{scnrelfromlist}{используемые константы}
				\scnitem{отношение}
				\scnitem{ориентированное отношение}
				\scnitem{бинарное отношение}
				\scnitem{неролевое отношение}
				\scnitem{персона}
				\scnitem{контактная информация}
				\scnitem{файл ostis-системы}
			\end{scnrelfromlist}
		\end{scnindent}
	
	\scnheader{спецификация роли}
	\begin{scnrelfromset}{обобщенная декомпозиция}
		\scnitem{знак специфицируемого объекта}
		\scnitem{роль}
		\scnitem{дуга, связывающая специфицируемый объект с ролью}
		\scnitem{дуга принадлежности отношению \textit{роль*}}
		\scnitem{роль*}
	\end{scnrelfromset}
	\scnrelfrom{шаблон}{шаблон для описания роли}
	\begin{scnindent}
		\scnrelfrom{иллюстрация}{\scnfileimage[20em]{Contents/part_kb/src/images/sd_basic_specification/role.png}}
		\begin{scnindent}
			\scnidtf{SCg-текст. Шаблон для описания роли}
		\end{scnindent}
	\end{scnindent}

	\scnheader{роль*}
	\scnsubset{отношение}
	\scnsubset{бинарное отношение}
	\scnsubset{неролевое отношение}
	\scnsubset{ориентированное отношение}
	\scnsubset{антисимметричное отношение}
	\scnsubset{антитранзитивное отношение}
	\scnsubset{антирефлексивное отношение}
	\begin{scnrelfrom}{первый домен}
		{персона}
	\end{scnrelfrom}
	\begin{scnrelfrom}{второй домен}
		{sc-текст}
	\end{scnrelfrom}
	\begin{scnrelfromset}{область определения}
		\scnitem{персона}
		\scnitem{sc-текст}
	\end{scnrelfromset}
	\scntext{определение}{\textit{роль*} - бинарное ориентированное неролевое отношение, связывающее персону и sc-текст, обозначающий её роль в рамках сообщества.}
		\begin{scnindent}
			\begin{scnrelfromlist}{используемые константы}
				\scnitem{отношение}
				\scnitem{ориентированное отношение}
				\scnitem{бинарное отношение}
				\scnitem{неролевое отношение}
				\scnitem{персона}
				\scnitem{роль}
				\scnitem{sc-текст}
			\end{scnrelfromlist}
		\end{scnindent}
\end{scnsubstruct}

\scnheader{Подпункт 23.1.1.9 Формализация базовой спецификации агента}
\begin{scnsubstruct}
% \\\\\\\\\\\\\\ АГЕНТ
    \scnheader{базовая спецификация агента}
	\scntext{примечание}{Базовая спецификация агента отражает набор свойств, которые должны быть описаны для каждого агента}
	\begin{scnrelto}{обобщенная базовая спецификация}
		{агент}
	\end{scnrelto}
	\begin{scnrelfromvector}{обобщенная декомпозиция}
		\scnitem{спецификация выходных данных}
		\scnitem{спецификация назначения}
	\end{scnrelfromvector}
	
	\scnheader{спецификация выходных данных}
	\begin{scnrelfromset}{обобщенная декомпозиция}
		\scnitem{знак специфицируемого объекта}
		\scnitem{данные}
		\scnitem{дуга, связывающая специфицируемый объект с данными}
		\scnitem{дуга принадлежности отношению \textit{выходные данные*}}
		\scnitem{выходные данные*}
	\end{scnrelfromset}
	\scnrelfrom{шаблон}{шаблон для описания выходных данных}
	\begin{scnindent}
		\scnrelfrom{иллюстрация}{\scnfileimage[20em]{Contents/part_kb/src/images/sd_basic_specification/output_data.png}}
		\begin{scnindent}
			\scnidtf{SCg-текст. Шаблон для описания выходных данных}
		\end{scnindent}
	\end{scnindent}

	\scnheader{выходные данные*}
	\scnsubset{отношение}
	\scnsubset{бинарное отношение}
	\scnsubset{неролевое отношение}
	\scnsubset{ориентированное отношение}
	\scnsubset{антисимметричное отношение}
	\scnsubset{антитранзитивное отношение}
	\scnsubset{антирефлексивное отношение}
	\begin{scnrelfrom}{первый домен}
		{агент}
	\end{scnrelfrom}
	\begin{scnrelfrom}{второй домен}
		{данные}
	\end{scnrelfrom}
	\begin{scnrelfromset}{область определения}
		\scnitem{агент}
		\scnitem{данные}
	\end{scnrelfromset}
	\scntext{определение}{\textit{выходные данные*} - бинарное ориентированное неролевое отношение, связывающее агент и некоторые данные, показывающее, какие данные данные будут получены на выходе работы агента.}
		\begin{scnindent}
			\begin{scnrelfromlist}{используемые константы}
				\scnitem{отношение}
				\scnitem{ориентированное отношение}
				\scnitem{бинарное отношение}
				\scnitem{неролевое отношение}
				\scnitem{данные}
				\scnitem{агент}
				\scnitem{выход}
			\end{scnrelfromlist}
		\end{scnindent}
\end{scnsubstruct}

\scnheader{Подпункт 23.1.1.10 Формализация базовой спецификации отношения}
\begin{scnsubstruct}
% \\\\\\\\\\\\\\\\\\\ ОТНОШЕНИЕ
	\scnheader{базовая спецификация отношения}
	\scntext{примечание}{Базовая спецификация отношения отражает набор свойств, которые должны быть описаны для каждого отношения}
	\begin{scnrelto}{обобщенная базовая спецификация}
		{отношение}
	\end{scnrelto}
	\begin{scnrelfromvector}{обобщенная декомпозиция}
		\scnitem{спецификация свойств отношения}
		\scnitem{спецификация первого домена}
		\scnitem{спецификация второго домена}
		\scnitem{спецификация области определения}
		\scnitem{спецификация определения}
	\end{scnrelfromvector}
		
	% вопрос: каким образом тут должен выглядеть шаблон?
	\scnheader{спецификация свойств отношения}
	\begin{scnrelfromset}{обобщенная декомпозиция}
		\scnitem{знак специфицируемого объекта}
		\scnitem{класс отношений}
		\scnitem{дуга принадлежности классу отношений}
	\end{scnrelfromset}

	\scnheader{спецификация первого домена}
	\begin{scnrelfromset}{обобщенная декомпозиция}
		\scnitem{знак специфицируемого объекта}
		\scnitem{класс сущностей}
		\scnitem{дуга, связывающая специфицируемый объект с классом сущностей}
		\scnitem{дуга принадлежности отношению \textit{первый домен*}}
		\scnitem{первый домен*}
	\end{scnrelfromset}
	\scnrelfrom{шаблон}{шаблон для описания первого домена}
	\begin{scnindent}
		\scnrelfrom{иллюстрация}{\scnfileimage[20em]{Contents/part_kb/src/images/sd_basic_specification/first_domain.png}}
		\begin{scnindent}
			\scnidtf{SCg-текст. Шаблон для описания первого домена}
		\end{scnindent}
	\end{scnindent}
	
	\scnheader{спецификация второго домена}
	\begin{scnrelfromset}{обобщенная декомпозиция}
		\scnitem{знак специфицируемого объекта}
		\scnitem{класс сущностей}
		\scnitem{дуга, связывающая специфицируемый объект с классом сущностей}
		\scnitem{дуга принадлежности отношению \textit{второй домен*}}
		\scnitem{второй домен*}
	\end{scnrelfromset}
	\scnrelfrom{шаблон}{шаблон для описания второго домена}
	\begin{scnindent}
		\scnrelfrom{иллюстрация}{\scnfileimage[20em]{Contents/part_kb/src/images/sd_basic_specification/second_domain.png}}
		\begin{scnindent}
			\scnidtf{SCg-текст. Шаблон для описания второго домена}
		\end{scnindent}
	\end{scnindent}

	\scnheader{спецификация области определения}
	\begin{scnrelfromset}{обобщенная декомпозиция}
		\scnitem{знак специфицируемого объекта}
		\scnitem{область определения}
		\scnitem{дуга, связывающая специфицируемый объект с областью определения}
		\scnitem{дуга принадлежности отношению \textit{область определения*}}
		\scnitem{область определения*}
	\end{scnrelfromset}
	\scnrelfrom{шаблон}{шаблон для описания области определения}
	\begin{scnindent}
		\scnrelfrom{иллюстрация}{\scnfileimage[20em]{Contents/part_kb/src/images/sd_basic_specification/domain_of_definition.png}}
		\begin{scnindent}
			\scnidtf{SCg-текст. Шаблон для описания области определения}
		\end{scnindent}
	\end{scnindent}

	\scnheader{спецификация определения}
	\begin{scnrelfromset}{обобщенная декомпозиция}
		\scnitem{знак специфицируемого объекта}
		\scnitem{определение}
		\scnitem{дуга, связывающая специфицируемый объект с определением}
		\scnitem{дуга принадлежности отношению \textit{определение*}}
		\scnitem{определение*}
	\end{scnrelfromset}
	\scnrelfrom{шаблон}{шаблон для описания определения}
	\begin{scnindent}
		\scnrelfrom{иллюстрация}{\scnfileimage[20em]{Contents/part_kb/src/images/sd_basic_specification/definition.png}}
		\begin{scnindent}
			\scnidtf{SCg-текст. Шаблон для описания определения}
		\end{scnindent}
	\end{scnindent}

	\scnheader{определение*}
	\scnsubset{отношение}
	\scnsubset{бинарное отношение}
	\scnsubset{неролевое отношение}
	\scnsubset{ориентированное отношение}
	\scnsubset{антисимметричное отношение}
	\scnsubset{антитранзитивное отношение}
	\scnsubset{антирефлексивное отношение}
	\begin{scnrelfrom}{первый домен}
		{сущность}
	\end{scnrelfrom}
	\begin{scnrelfrom}{второй домен}
		{определение}
	\end{scnrelfrom}
	\begin{scnrelfromset}{область определения}
		\scnitem{сущность}
		\scnitem{определение}
	\end{scnrelfromset}
	\scntext{определение}{\textit{определение*} - бинарное ориентированное неролевое отношение, связывающее сущность и её определение. Позволяет дать опеределение конкретной сущности и передать некоторую информацию о ней.}
		\begin{scnindent}
			\begin{scnrelfromlist}{используемые константы}
				\scnitem{отношение}
				\scnitem{ориентированное отношение}
				\scnitem{бинарное отношение}
				\scnitem{неролевое отношение}
				\scnitem{сущность}
				\scnitem{определение}
				\scnitem{информация}
			\end{scnrelfromlist}
		\end{scnindent}
\end{scnsubstruct}

\scnheader{Подпункт 23.1.1.11 Формализация базовой спецификации класса}
\begin{scnsubstruct}
%\\\\\\\\\\\\\\\\ КЛАСС СУЩНОСТЕЙ
	\scnheader{базовая спецификация класса}
	\scntext{примечание}{Базовая спецификация класса сущностей отражает набор свойств, которые должны быть описаны для каждого класса сущностей}
	\begin{scnrelto}{обобщенная базовая спецификация}
		{класс сущностей}
	\end{scnrelto}
	\begin{scnrelfromvector}{обобщенная декомпозиция}
		\scnitem{спецификация подкласс}
		\scnitem{спецификация надкласс}
		\scnitem{спецификация экземпляры класса}
		\scnitem{спецификация определения}
	\end{scnrelfromvector}

	\scnheader{спецификация подкласса}
	\begin{scnrelfromset}{обобщенная декомпозиция}
		\scnitem{знак специфицируемого объекта}
		\scnitem{класс сущностей}
		\scnitem{дуга, связывающая специфицируемый объект с классом сущностей}
		\scnitem{дуга принадлежности отношению \textit{включение*}}
		\scnitem{включение*}
	\end{scnrelfromset}

	\scnheader{спецификация надкласса}
	\begin{scnrelfromset}{обобщенная декомпозиция}
		\scnitem{знак специфицируемого объекта}
		\scnitem{класс сущностей}
		\scnitem{дуга, связывающая класс сущностей со специфицируемым объектом}
		\scnitem{дуга принадлежности отношению \textit{включение*}}
		\scnitem{включение*}
	\end{scnrelfromset}

	\scnheader{спецификация экземпляра класса}
	\begin{scnrelfromset}{обобщенная декомпозиция}
		\scnitem{знак специфицируемого объекта}
		\scnitem{экземпляр класса}
		\scnitem{дуга принадлежности связывающая специфицируемый объект и экземпляр класса}
	\end{scnrelfromset}
\end{scnsubstruct}


\scnheader{Подпункт 23.1.1.12 Формализация базовой спецификации библиографического источника}
%\\\\\\\\\\\\\\\\\\ БИБЛИОГРАФИЧЕСКИЙ ИСТОЧНИК
\begin{scnsubstruct}
	\scnheader{базовая спецификация библиографического источника}
	\scntext{примечание}{Базовая спецификация библиографический источник отражает набор свойств, которые должны быть описаны для каждого библиографического источника}
	\begin{scnrelto}{обобщенная базовая спецификация}
		{библиографический источник}
	\end{scnrelto}
	\begin{scnrelfromvector}{обобщенная декомпозиция}
		\scnitem{спецификация названия} 
		\scnitem{спецификация автора}
		\scnitem{спецификация типа библиографического источника}
	\end{scnrelfromvector}

	\scnheader{спецификация типа библиографического источника}
	\begin{scnrelfromset}{обобщенная декомпозиция}
		\scnitem{знак специфицируемого объекта}
		\scnitem{тип библиографического источника}
		\scnitem{дуга принадлежности связывающая тип библиографического источника и специфицируемый объект}
	\end{scnrelfromset}

\end{scnsubstruct}
\end{SCn}

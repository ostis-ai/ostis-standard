\begin{SCn}
	\scnsectionheader{Предметная область и онтология событий и действий}
	\begin{scnhaselementrolelist}{класс объектов исследования}
		\scnitem{sc-элемент}
		\scnitem{событие в sc-памяти}
		\scnitem{понятие, переходящее из основного в неосновное}
		\scnitem{понятие, переходящее из неосновного в основное}
	\end{scnhaselementrolelist}
	
	\scnheader{результат обработки информации в sc-памяти}
	\begin{scnrelfromlist}{ведёт}
		\begin{scnindent}
			\scnitem{к появлению в sc-памяти новых актуальных sc-узлов и sc-коннекторов}
			\scnitem{к логическому удалению актуальных sc-элементов, то есть к переводу их в неактуальное состояние (это необходимо для хранения протокола изменения состояния базы знаний, в рамках которого могут описываться действия по удалению sc-элементов)}
			\scnitem{к возврату логически удаленных sс-элементов в статус актуальных (необходимость в этом может возникнуть при откате базы знаний к какой-нибудь ее прошлой версии)}
			\scnitem{к физическому удалению sc-элементов}
			\scnitem{к изменению состояния актуальных (логически не удаленных sc-элементов)}
		\end{scnindent}
		\scntext{пояснение}{Подчеркнем, что временный характер самого sc-элемента (так как он может появиться или исчезнуть) никак не связан с возможно временным характером сущности, обозначаемой этим sc-элементом. То есть временный характер самого sc-элемента и временный характер сущности, которую он обозначает — абсолютно разные вещи.}
	\end{scnrelfromlist}
	\scnheader{следует отличать*}
	\begin{scnhaselementset}
		\scnitem{\scnnonamednode}
		\begin{scnindent}
			\begin{scneqtoset}
				\scnitem{динамика внешнего мира}
				\scnitem{динамика бащы знаний (динамика внешнего мира)}
			\end{scneqtoset}
			\scniselement{следует отличать*}
		\end{scnindent}
		\scnitem{\scnnonamednode}
	\end{scnhaselementset}
	
	\scnheader{понятие, используемые для описания динамики базы знаний}
	\begin{scnrelfromlist}{разбиение}
		\begin{scnindent}
			\scnitem{логически удаленный sc-элемент}
			\scnitem{сформированное множество}
			\scnitem{вычисленное число}
			\scnitem{сформированное высказывание}
		\end{scnindent}
	\end{scnrelfromlist}
	
	\scnheader{sc-элемент}
	\begin{scnrelfromlist}{разбиение}
		\begin{scnindent}
			\scnitem{настоящий sc-элемент}
			\scnitem{логически удаленный sc-элемент}
		\end{scnindent}            
	\end{scnrelfromlist}
	\scnheader{настоящий sc-элемент}
	\scniselement{ситуативное множество}
	
	\scnheader{логически удаленный sc-элемент}
	\scniselement{ситуативное множество}
	
	
	\scnheader{основное понятие}
	\scnidtf{основное понятие для данной ostis-системы}
	\scniselement{ситуативное множество}
	\scntext{пояснение}{К основным понятиям относятся те понятия, которые активно используются в системе и могут быть ключевыми элементами sc-агентов. К основным понятиям относятся также все неопределяемые понятия}
	
	\scnheader{неосновное понятие}
	\scnidtf{дополнительное понятие}
	\scnidtf{вспомогательное понятие}
	\scnidtf{неосновное понятие для данной ostis-системы}
	\scniselement{ситуативное множество}
	\scntext{пояснение}{Каждое неосновное понятие должно быть строго определено на основе основных понятий. Такие неосновные понятия используются только для понимания и правильного восприятия вводимой информации, в том числе, для выравнивания онтологий. Ключевым элементом sc-агентов неосновные понятия быть не могут.}
	\scntext{правило идентификации экземпляров}{В случае, когда некоторое понятие полностью перешло из основных понятий в неосновные, то есть стало неосновным понятием, и соответствующее ему основное понятие (через которое оно определяется) в рамках некоторого внешнего языка имеет одинаковый с ним основной идентификатор, то к идентификатору неосновного понятия спереди добавляется знак \#. Если при этом соответствуюшее основное понятие имеет в идентификаторе знак \$, добавленный в процессе перехода, то этот знак удаляется. Если указанные понятия имеют разные основные идентификаторы в рамках этого внешнего языка, то никаких дополнительных средств идентификации не используется.}
	
	\scnheader{понятие, переходящее из основного в неосновное}
	\scniselement{ситуативное множество}
	
	\scnheader{понятие, переходящее из неосновного в основное}
	\scniselement{ситуативное множество}
	\scntext{правило идентификации экземпляров}{В случае, когда текущее основное понятие и соответствующее ему понятие, переходящее из неосновного в основное в рамках некоторого внешнего языка имеют одинаковый основной идентификатор, то к идентификатору понятия, переходящего из неосновного в основное спереди добавляется знак \$. Если указанные понятия имеют разные основные идентификаторы в рамках этого внешнего языка, то никаких дополнительных средств идентификации не используется}
	
	\scnheader{специфицированная сущность}
	\begin{scnrelfromlist}{разбиение}
		
		\scnitem{недостаточно специфицированная сущность}
		\scnitem{достаточно специфицированная сущность}
		\begin{scnrelfromlist}{необходимость для сущности, не являющейся понятием}
			\scnitem{указания различных вариантов обозначающих ее внешних знаков}
			\scnitem{указания классов}
			\scnitem{указания связок}
			\scnitem{указания значения параметров}
			\scnitem{указания разделов баз знаний}
			\scnitem{указания предметных областей}
		\end{scnrelfromlist}
		\begin{scnrelfromlist}{необходимость для сущности, являющейся понятием}
			\scnitem{указания различных вариантов внешних обозначений этого понятия}
			\scnitem{указания предметных областей, в которых это понятие исследуется}
			\scnitem{указания определения понятия}
			\scnitem{указания пояснения}
			\scnitem{указания разделов базы знаний, в которых это понятие является ключевым}
			\scnitem{указания описания примера — пример экземпляра понятия}
		\end{scnrelfromlist}
	\end{scnrelfromlist}
	\scnheader{событие в sc-памяти}
	\scnsuperset{событие}
	
	\scnheader{элементарное событие в sc-памяти}
	\scnsubset{событие}
	\begin{scnrelfromlist}{разбиение}
		\scnitem{событие добавления sc-дуги, выходящей из заданного sc-элемента}
		\scnitem{событие добавления sc-дуги, входящей в заданный sc-элемент}
		\scnitem{событие добавления sc-ребра, инцидентного заданному sc-элементу}
		\scnitem{событие удаления sc-дуги, выходящей из заданного sc-элемента}
		\scnitem{событие удаления sc-дуги, входящей в заданный sc-элемент}
		\scnitem{событие удаления sc-ребра, инцидентного заданному sc-элементу}
		\scnitem{событие удаления sc-элемента}
		\scnitem{событие изменения содержимого файла}
		\scntext{пояснение}{Под элементарным событием в sc-памяти понимается такое событие, в результате выполнения которого изменяется состояние только одного sc-элемента.}
	\end{scnrelfromlist}
	\scnheader{точечный процесс}
	\scnidtf{атомарный процесс}
	\scnidtf{условно мгновенный процесс}
	\scnidtf{процесс, длительность которого в данном контексте считается несущественной (пренебрежимо малой)}
	
	\scnheader{элементарный процесс}
	\scnidtf{процесс, детализация которого на входящие в него подпроцессы в текущем контексте не производится}
	
\end{SCn}
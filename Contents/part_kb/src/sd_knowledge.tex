\begin{SCn}
    \scnsectionheader{Предметная область и онтология знаний и баз знаний ostis-систем}
    \begin{scnsubstruct}
        \begin{scnrelfromlist}{дочерний раздел}
            \scnitem{Предметная область и онтология множеств}
            \begin{scnindent}
                \scnidtf{Предметная область и онтология \textit{знаний о множествах}}
                \begin{scnindent}
                    \scntext{примечание}{\textit{знания о множествах} являются \uline{частным видом} \textit{знаний} и, следовательно, общие свойства сущностей, описываемых знаниями, могут наследоваться \textit{Предметной областью и онтологией множеств}.}
                \end{scnindent}
            \end{scnindent}
            \scnitem{Предметная область и онтология связок и отношений}
            \scnitem{Предметная область и онтология параметров, величин и шкал}
            \scnitem{Предметная область и онтология чисел и числовых структур}
            \scnitem{Предметная область и онтология структур}
            \scnitem{Предметная область и онтология темпоральных сущностей}
            \scnitem{Предметная область и онтология темпоральных сущностей баз знаний ostis-систем}
            \scnitem{Предметная область и онтология семантических окрестностей}
            \scnitem{Предметная область и онтология предметных областей}
            \scnitem{Предметная область и онтология онтологий}
            \scnitem{Предметная область и онтология логических формул, высказываний и формальных теорий}
            \scnitem{Предметная область и онтология внешних информационных конструкций и файлов ostis-систем}
            \scnitem{Глобальная предметная область действий и задач и соответствующая ей онтология методов и технологий}
        \end{scnrelfromlist}
		\begin{scnrelfromlist}{введение}
            \scnfileitem{Развитие информационных технологий привело к расширению многообразия используемой информации, и вследствие этого к необходимости создания \textit{интеллектуальных компьютерных систем}, способных оперировать объемными информационными ресурсами. Важнейшими видами таких ресурсов являются \textit{базы знаний}.}
			\scnfileitem{\textit{база знаний} представляет собой систематизированную совокупность \textit{знаний}, хранимую в памяти \textit{интеллектуальной компьютерной системы} и достаточную для обеспечения целенаправленного (целесообразного, адекватного) функционирования (поведения) этой системы как в своей внешней среде, так и в своей внутренней среде (в собственной \textit{базе знаний}).}
			\scnfileitem{Важным этапом разработки \textit{баз знаний} \textit{интеллектуальных компьютерных систем} является их структуризация. Структуризация \textit{базы знаний}, то есть выделение в ней различных связанных между собой подструктур, необходима по целому ряду причин. В частности, это необходимо для обеспечения их синтаксической совместимости, что подразумевает унификацию формы представления \textit{знаний}.}
            \begin{scnindent}
                \begin{scnrelfromset}{источник}
                    \scnitem{\scncite{Davydenko2016a}}
                \end{scnrelfromset}
            \end{scnindent}
			\scnfileitem{На сегодняшний день существуют десятки моделей представления \textit{знаний}. Каждая из которых адаптирована для представления \textit{знаний} определенного вида, в то время как при создании \textit{интеллектуальных компьютерных систем} часто возникает необходимость представить различные \textit{виды знаний} в рамках одной \textit{базы знаний}. Однако, в настоящее время ни одна из существующих моделей, взятых в отдельности, не может этого обеспечить.} 
			\scnfileitem{В связи с этим возникает необходимость в создании универсальной структурированной модели представления \textit{знаний}, которая позволила бы представлять любые \textit{виды знаний} в унифицированном виде.}
        \end{scnrelfromlist}
        \scniselement{раздел базы знаний}
        \scnhaselementrole{ключевой sc-элемент}{Предметная область знаний и баз знаний ostis-систем}

        \scnheader{Предметная область знаний и баз знаний ostis-систем}
        \scniselement{предметная область}
        \begin{scnhaselementrolelist}{максимальный класс объектов исследования}
            \scnitem{знание}
        \end{scnhaselementrolelist}
        \begin{scnhaselementrolelist}{исследуемый класс классов}
            \scnitem{вид знаний}
            \scnitem{отношение, заданное на множестве знаний}
        \end{scnhaselementrolelist}
        \scntext{аннотация}{\textit{Предметная область знаний и баз знаний ostis-систем} посвящена онтологическому подходу к проектированию баз знаний интеллектуальных компьютерных систем нового поколения. Данный подход основан на представлении базы знаний как иерархической структуры взаимосвязанных предметных областей и их онтологий, построенных на базе онтологий верхнего уровня.}

        \scnheader{знание}
        \scnidtf{синтаксически корректная (для соответствующего языка) и семантически целостная информационная конструкция}
        \scnsubset{информационная конструкция}
        \begin{scnindent}
            \scniselementrole{класс объектов исследования}{\nameref{intro_lang}}
        \end{scnindent}
        \scnrelfrom{покрытие}{вид знаний}
        \begin{scnindent}
            \scnidtf{Множество \uline{всевозможных} видов знаний}
            \scntext{примечание}{Тот факт, что семейство \textit{видов знаний} является \textit{покрытием} \textbf{Множества всевозможных \textit{знаний}}, означает то, что каждое \textit{знание} принадлежит по крайней мере одному выделенному нами \textit{виду знаний}.}
        \end{scnindent}

        \scnheader{вид знаний}
        \scnhaselement{спецификация}
        \begin{scnindent}
            \scnidtf{описание заданной сущности}
            \scnsuperset{спецификация материальной сущности}
            \scnsuperset{спецификация обратной сущности, не являющейся множеством}
            \begin{scnindent}
                \scnsuperset{спецификация геометрической точки}
                \scnsuperset{спецификация числа}
            \end{scnindent}
            \scnsuperset{спецификация множества}
            \begin{scnindent}
                \scnsuperset{спецификация связи}
                \scnsuperset{спецификация структуры}
                \scnsuperset{спецификация класса}
                \begin{scnindent}
                    \scnsuperset{спецификация класса сущностей, не являющихся множествами}
                    \scnsuperset{спецификация отношения}
                    \begin{scnindent}
                        \scnidtf{спецификация класса связей (связок)}
                    \end{scnindent}
                    \scnsuperset{спецификация класса классов}
                    \begin{scnindent}
	                    \scnsuperset{спецификация параметра}
                    \end{scnindent}
                    \scnsuperset{спецификация класса структур}
                    \scnsuperset{спецификация понятий}
                    \begin{scnindent}
                        \scnsuperset{пояснение}
                        \scnsuperset{определение}
                        \scnsuperset{утверждение}
                        \begin{scnindent}
                            \scnidtf{утверждение, описывающее свойства экземпляров (элементов) специфицируемого понятия}
                            \scnidtf{закономерность}
                        \end{scnindent}
                    \end{scnindent}
                \end{scnindent}
            \end{scnindent}
            \scnsuperset{семантическая окрестность}
            \scnsuperset{однозначная спецификация}
            \scnsuperset{сравнительный анализ}
            \scnsuperset{достоинства}
            \scnsuperset{недостатки}
            \scnsuperset{структура специфицируемой сущности}
            \scnsuperset{принципы, лежащие в основе}
            \scnsuperset{обоснование предлагаемого решения}
            \begin{scnindent}
                \scnidtf{аргументация предлагаемого решения}
            \end{scnindent}
        \end{scnindent}
        \scnhaselement{сравнение}
        \scnhaselement{высказывание}
        \begin{scnindent}
            \scnsuperset{фактографическое высказывание}
            \scnsuperset{закономерность}
        \end{scnindent}
        \scnhaselement{формальная теория}
        \scnhaselement{предметная область}
        \scnhaselement{предметная область и онтология}
        \begin{scnindent}
            \scnidtf{предметная область и её онтология}
            \scnidtf{предметная область и соответствующая ей объединенная онтология}
        \end{scnindent}
        \scnhaselement{метазнание}
        \begin{scnindent}
            \scnidtf{спецификация знания}
            \scnsuperset{аннотация}
            \scnsuperset{введение}
            \scnsuperset{предисловие}
            \scnsuperset{заключение}
            \scnsuperset{онтология}
            \begin{scnindent}
                \scnsuperset{онтология предметной области}
                \begin{scnindent}
                    \scnsuperset{структурная онтология предметной области}
                    \scnsuperset{теоретико-множественная онтология предметной области}
                    \scnsuperset{логическая онтология предметной области}
                    \scnsuperset{терминологическая онтология предметной области}
                    \scnsuperset{объединенная онтология предметной области}
                \end{scnindent}
            \end{scnindent}
        \end{scnindent}
        \scnhaselement{задача}
        \begin{scnindent}
            \scnidtf{спецификация действия}
        \end{scnindent}
        \scnhaselement{план}
        \scnhaselement{протокол}
        \scnhaselement{результативная часть протокола}
        \scnhaselement{метод}
        \scnhaselement{технология}
        \scnhaselement{история использования предметной области и её онтологии по решению информационных задач}
        \scnhaselement{история использования предметной области и её онтологии по решению задач во внешней среде}
        \scnhaselement{история эволюции предметной области и её онтологии}
        \scntext{уточнение}{Важным видом знаний являются \textbf{базы знаний}.}
        \scnhaselement{база знаний}
        \begin{scnindent}
            \scnidtf{совокупность знаний, хранимых в памяти интеллектуальной компьютерной системы и \uline{достаточных} для того, чтобы указанная система удовлетворяла соответствующим предъявляемым к ней требованиям (в частности, чтобы она имела соответствующий уровень интеллекта)}
            \scnidtf{систематизированная совокупность знаний, хранимая в памяти интеллектуальной компьютерной системы и достаточная для обеспечения целенаправленного (целесообразного, адекватного) функционирования (поведения) этой системы как в своей внешней среде, так и в своей внутренней среде (в собственной базе знаний)}
            \begin{scnrelfromset}{обобщенная декомпозиция}
                \scnitem{согласованная часть базы знаний}
                \begin{scnindent}
                    \scnidtf{часть базы знаний, признанная коллективом авторов на текущий момент}
                \end{scnindent}
                \scnitem{история эксплуатации базы знаний}
                \scnitem{история эволюции базы знаний}
                \scnitem{план эволюции базы знаний}
                \begin{scnindent}    
                    \scnidtf{система специфицированных и согласованных действий авторов базы знаний, направленных на повышение её качества}
                \end{scnindent}
            \end{scnrelfromset}
            \scntext{примечание}{Основным факторами, определяющими качество интеллектуальной компьютерной системы, являются:
                \begin{scnitemize}
                    \item качественная структуризация (систематизация) и \uline{стратификация} базы знаний интеллектуальной компьютерной системы;
                    \item систематизация и стратификация \uline{деятельности}, которая осуществляется интеллектуальной компьютерной системой и спецификация которой является важнейшей частью базы знаний этой системы (см. \nameref{sd_actions}).
                \end{scnitemize}}
        \end{scnindent}
        \scntext{примечание}{Даже небольшой перечень \textit{видов знаний} свидетельствует об огромном многообразии \textit{видов знаний}.}
        
        \scnheader{знание}
        \begin{scnsubdividing}
            \scnitem{декларативное знание}
            \begin{scnindent}
                \scnidtf{\textit{знание}, имеющее \uline{только} \textit{денотационную семантику}, которая представляется в виде семантической \textit{спецификации} системы \textit{понятий}, используемых в этом \textit{знании}}
            \end{scnindent}
            \scnitem{процедурное знание}
            \begin{scnindent}
                \scnidtf{\textit{знание}, имеющее не только \textit{денотационную семантику}, но и \textit{операционную семантику}, которая представляется в виде семейства \textit{спецификаций агентов}, осуществляющих интерпретацию \textit{процедурного знания}, направленную на решение некоторой инициированной \textit{задачи}}
                \scnidtf{функционально интерпретируемое знание, обеспечивающее решение либо конкретной задачи, либо некоторого множества инициируемых задач}
                \scnsuperset{задача}
                \begin{scnindent}
                    \scnidtf{формулировка конкретной задачи}
                    \scnsuperset{декларативная формулировка задачи}
                    \scnsuperset{процедурная формулировка задачи}
                \end{scnindent}
                \scnsuperset{план}
                \begin{scnindent}
                    \scnidtf{план решения конкретной задачи}
                    \scnidtf{контекст конкретной задачи, предоставляющий всю информацию для решения всех подзадач для указанной конкретной задачи}
                    \scnidtf{описание системы подзадач некоторой задачи}
                \end{scnindent}
                \scnsuperset{метод}
                \begin{scnindent}
                    \scnidtf{обобщенное описание плана решения любой задачи из некоторого заданного класса задач}
                \end{scnindent}
                \scnsuperset{навык}
                \begin{scnindent}
                    \scnidtf{метод, детализированный до уровня элементарных подзадач}
                \end{scnindent}
            \end{scnindent}
        \end{scnsubdividing}

        \scnheader{отношение, заданное на множестве знаний}
        \scnhaselement{дочернее знание*}
        \begin{scnindent}
            \scnidtf{знание, которое от \scnqq{материнского} знания наследует все описанные там свойства объектов исследования}
            \scntext{примечание}{Факт наследования свойств описываемых объектов от \scnqq{материнского} знания подчеркивается использованием прилагательного дочернее в sc-идентификаторе данного отношения, заданного на множестве знаний.}
            \scnsuperset{дочерний раздел*}
            \begin{scnindent}
                \scnidtf{частный раздел*}
            \end{scnindent}
            \scnsuperset{дочерняя предметная область и онтология*}
        \end{scnindent}
        \scnhaselement{спецификация*}
        \begin{scnindent}
            \scnidtf{быть знанием, которое является спецификацией (описанием) заданной сущности}
            \scntext{примечание}{Специфицируемой сущностью может быть сущность любого вида, в том числе, и другое знание.}
        \end{scnindent}
        \scnhaselement{онтология*}
        \begin{scnindent}
            \scnidtf{быть семантической спецификацией заданного знания*}
        \end{scnindent}
        \scnhaselement{семантическая эквивалентность*}
        \scnhaselement{следовательно*}
        \begin{scnindent}
            \scnidtf{логическое следствие*}
        \end{scnindent}
        \scnhaselement{логическая эквивалентность*}
        \scntext{примечание}{В рамках Технологии OSTIS также выделяются отношения, заданные на множестве знаний.}
        \bigskip
    \end{scnsubstruct}
    \scnendcurrentsectioncomment
\end{SCn}

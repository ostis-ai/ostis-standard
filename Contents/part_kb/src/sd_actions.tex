\begin{SCn}
    \scnsectionheader{Глобальная предметная область и онтология, описывающая воздействия, действия, методы, средства и технологии}
    \begin{scnsubstruct}
        \begin{SCn}
    \scniselement{раздел}
    \scniselement{предметная область и онтология}
    \begin{scnreltovector}{конкатенация сегментов}
        \scnitem{Уточнение понятия воздействия и понятия действия. Типология воздействий и действий}
        \scnitem{Уточнение понятия задачи. Типология задач}
        \scnitem{Уточнение семейства параметров и отношений, заданных на множестве воздействий, действий и задач}
        \scnitem{Предметная область и онтология субъектно-объектных спецификаций воздействий}
        \scnitem{Уточнение понятий плана сложного действия, класса задач, метода}
        \scnitem{Уточнение понятия навыка, понятия класса методов и понятия модели решения задач}
        \scnitem{Уточнение понятия деятельности, понятия вида деятельности и понятия технологии}
    \end{scnreltovector}
    \bigskip
    \begin{scnhaselementrolelist}{исследуемый класс первичных объектов исследования}
        \scnitem{воздействие}
        \begin{scnindent}
            \scnidtf{\textit{процесс} воздействия одних \textit{сущностей} на другие}
            \scnsubset{процесс}
        \end{scnindent}
        \scnitem{действие}
        \begin{scnindent}
            \scnidtf{\textit{процесс}, \scnqq{осознанно} и целенаправленно выполняемый (управляемый) некоторой \textit{кибернетической системой}}
            \scnsubset{воздействие}
            \scnsubset{процесс}
        \end{scnindent}
        \bigskip
        \scnitem{неосознанное воздействие}
        \bigskip
        \scnitem{действие, выполняемое в памяти субъекта действия}
        \scnitem{действие, выполняемое во внешней среде субъекта действия}
        \scnitem{рецепторное действие}
        \begin{scnindent}
            \scnidtf{действие, выполняемое рецептором субъекта действия}
        \end{scnindent}
        \scnitem{эффекторное действие}
        \begin{scnindent}
            \scnidtf{действие, выполняемое эффектором субъекта действия}
        \end{scnindent}
        \bigskip
        \scnitem{элементарное действие}
        \begin{scnindent}
            \scnidtf{действие, выполнение которого не требует его декомпозиции на взаимосвязанные поддействия}
        \end{scnindent}
        \scnitem{сложное действие}
        \scnitem{легко выполнимое сложное действие}
        \begin{scnindent}
            \scnidtf{сложное действие, которое известно, как выполнять}
        \end{scnindent}
        \scnitem{интеллектуальное действие}
        \begin{scnindent}
            \scnidtf{сложное действие, для которого априори не известно, как его выполнять}
        \end{scnindent}
        \scnitem{индивидуальное действие}
        \begin{scnindent}
            \scnidtf{действие, выполняемое индивидуальной кибернетической системой}
        \end{scnindent}
        \scnitem{коллективное действие}
        \begin{scnindent}
            \scnidtf{действие, выполняемое коллективом кибернетических систем (многоагентной системой)}
        \end{scnindent}
        \bigskip
        \scnitem{планируемое действие}
        \scnitem{инициированное действие}
        \scnitem{выполняемое действие}
        \begin{scnindent}
            \scnidtf{активное действие}
        \end{scnindent}
        \scnitem{прерванное действие}
        \begin{scnindent}
            \scnidtf{выполняемое действие, находящееся в состоянии прерывания}
        \end{scnindent}
        \scnitem{выполненное действие}
        \scnitem{отмененное действие}
        \scnitem{действие с очень высоким приоритетом}
        \scnitem{действие с высоким приоритетом}
        \scnitem{действие со средним приоритетом}
        \scnitem{действие с низким приоритетом}
        \scnitem{действие с очень низким приоритетом}
    \end{scnhaselementrolelist}
    \bigskip
    \begin{scnhaselementrolelist}{исследуемый класс классов первичных объектов исследования}
        \scnitem{осмысленность воздействия\scnsupergroupsign}
        \scnitem{длительность воздействия\scnsupergroupsign}
        \scnitem{место выполнения действия\scnsupergroupsign}
        \scnitem{функциональная сложность действия\scnsupergroupsign}
        \scnitem{многоагентность действия\scnsupergroupsign}
        \begin{scnindent}
            \scnidtf{коллективность субъекта действия}
        \end{scnindent}
        \scnitem{текущее состояние действия\scnsupergroupsign}
        \scnitem{приоритет действия\scnsupergroupsign}
        \begin{scnindent}
            \scnidtf{важность действия\scnsupergroupsign}
        \end{scnindent}
        \scnitem{срочность действия\scnsupergroupsign}
        \bigskip
        \scnitem{класс действий}
        \scnitem{класс функционально эквивалентных действий\scnsupergroupsign}
        \scnitem{класс логически эквивалентных действий\scnsupergroupsign}
        \scnitem{класс семантических эквивалентных задач\scnsupergroupsign}
        \scnitem{класс логически эквивалентных задач\scnsupergroupsign}
        \scnitem{класс задач, для которого существует общий метод их решения\scnsupergroupsign}
        \scnitem{класс аналогичных семантически элементарных процессов воздействия\scnsupergroupsign}
        \begin{scnindent}
            \scnidtf{класс однотипных семантически элементарных воздействий\scnsupergroupsign}
        \end{scnindent}
    \end{scnhaselementrolelist}
    \bigskip
    \begin{scnhaselementrolelist}{исследуемый класс классов}
        \scnitem{отношение, заданное на множестве* (действие)}
        \begin{scnindent}
            \scnidtf{отношение, заданное на множестве действий}
        \end{scnindent}
        \scnitem{отношение, заданное на множестве* (задача)}
        \scnitem{параметр, заданный на множестве* (действие)}
        \scnitem{параметр, заданный на множестве* (задача)}
    \end{scnhaselementrolelist}
    \scnsourcecomment{Здесь указаны классы классов, которые не являются классами классов \uline{первичных} объектов исследования}
    \bigskip
    \begin{scnhaselementrolelist}{исследуемое отношение, заданное на множестве первичных объектов исследования}
        \scnitem{воздействующая сущность\scnrolesign}
        \scnitem{воздействуемая сущность\scnrolesign}
        \scnitem{посредник\scnrolesign}
        \scnitem{медиатор\scnrolesign}
        \scnitem{субъект\scnrolesign}
    	\begin{scnindent}
            \scnidtf{быть субъектом заданного действия}
        \end{scnindent}
        \scnitem{спецификация воздействия*}
    	\begin{scnindent}
            \scnsuperset{спецификация действия*}
        \end{scnindent}
        \scnitem{спецификация действия*}
    	\begin{scnindent}
            \scnsuperset{задача*}
            \begin{scnindent}
	            \begin{scnsubdividing}
		            \scnitem{декларативная формулировка задачи*}
		            \scnitem{процедурная формулировка задачи*}
	            \end{scnsubdividing}
	        \end{scnindent}
            \scnsuperset{план сложного действия*}
            \scnsuperset{декларативная спецификация выполнения сложного действия*}
            \scnsuperset{протокол*}
            \scnsuperset{результативная часть протокола*}
        \end{scnindent}
        \scnitem{декларативная формулировка задачи*}
        \scnitem{процедурная формулировка задачи*}
        \scnitem{план сложного действия*}
        \scnitem{декларативная спецификация выполнения сложного действия*}
        \scnitem{протокол*}
        \scnitem{результативная часть протокола*}
    \end{scnhaselementrolelist}
    \bigskip
    \begin{scnhaselementrolelist}{исследуемое отношение}
        \scnitem{спецификация класса действий*}
        \scnitem{спецификация метода*}
        \scnitem{спецификация класса методов*}
        \scnitem{спецификация деятельности*}
        \scnitem{спецификация вида деятельности*}
    \end{scnhaselementrolelist}
    \begin{scnindent}
    	\scnsourcecomment{Здесь указаны исследуемые отношения, которые заданы не на множестве первичных объектов исследования}	
    \end{scnindent}
    \bigskip
    \begin{scnhaselementrolelist}{исследуемый класс структур, специфицирующих первичные объекты исследования}
        \scnitem{задача}
    	\begin{scnindent}
            \scnidtf{формулировка задачи}
            \scnidtf{спецификация действия}
            \scnidtf{структура (sc-конструкция), содержащая в идеале достаточную информацию для выполнения соответствующего (специфицируемого) действия}
        \end{scnindent}
        \scnitem{декларативная формулировка задачи}
        \begin{scnindent}
            \scnidtf{семантическая спецификация действия}
        \end{scnindent}
        \scnitem{процедурная формулировка задачи}
        \begin{scnindent}
            \scnidtf{функциональная спецификация действия}
        \end{scnindent}
        \scnitem{план сложного действия}
        \begin{scnindent}
            \scnidtf{план выполнения сложного действия}
        \end{scnindent}
        \scnitem{процедурный план сложного действия}
        \scnitem{непроцедурный план сложного действия}
        \begin{scnindent}
            \scnidtf{декларативный план сложного действия}
            \scnidtf{иерархическая система подзадач заданной сложной задачи}
        \end{scnindent}
    \end{scnhaselementrolelist}
    \bigskip
    \begin{scnhaselementrolelist}{исследуемый класс структур}
        \scnitem{метод}
        \begin{scnindent}
            \scnidtf{спецификация класса сложных действий}
        \end{scnindent}
        \scnitem{денотационная семантика метода}
        \scnitem{операционная семантика метода}
        \scnitem{навык}
        \scnitem{модель решения задач}
        \scnitem{технология}
    \end{scnhaselementrolelist}
    \begin{scnindent}
    	\scnsourcecomment{Здесь указаны классы структур, не являющихся спецификациями первичных объектов исследования}
    \end{scnindent}
    \bigskip
    \begin{scnhaselementrolelist}{вводимое, но не исследуемое понятие}
        \scnitem{действие, выполняемое в памяти ostis-системы}
        \scnitem{действие, выполняемое ostis-системой в своей внешней среде}
        \scnitem{рецептурное действие ostis-системы}
        \scnitem{эффекторное действие ostis-системы}
        \scnitem{sc-агент}
    	\begin{scnindent}
            \scnidtf{внутренний субъект ostis-системы}
            \scnidtf{субъект, реализующий действия, выполняемые в памяти ostis-системы}
        \end{scnindent}
    \end{scnhaselementrolelist}
    \bigskip
    \begin{scnhaselementrolelist}{используемое понятие, исследуемое в другой предметной области и онтологии}
        \scnitem{кибернетическая система}
        \begin{scnindent}
            \scnidtf{сущность, обладающая способностью быть субъектом различного вида действий}
        \end{scnindent}
        \scnitem{компьютерная система}
        \begin{scnindent}
            \scnidtf{искусственная кибернетическая система}
            \scnsubset{кибернетическая система}
        \end{scnindent}
        \scnitem{интеллектуальная компьютерная система}
        \begin{scnindent}
            \scnsubset{компьютерная система}
            \scnsuperset{ostis-система}
        \end{scnindent}
        \scnitem{человек}
        \begin{scnindent}
            \scnsubset{кибернетическая система}
        \end{scnindent}
        \scnitem{ostis-система}
        \scnitem{спецификация*}
        \begin{scnindent}
            \scnidtf{быть спецификацией (описанием, семантической окрестностью заданной сущности*)}
            \scnidtf{семантическая окрестность*}
        \end{scnindent}
    \end{scnhaselementrolelist}
   
    \scnheader{следует отличать*}
    \begin{scnhaselementset}
        \scnitem{\scnnonamednode}
        \begin{scnindent}
            \begin{scneqtovector}
                \scnitem{действие}
                \scnitem{класс действий}
            \end{scneqtovector}
        \end{scnindent}
        \scnitem{\scnnonamednode}
        \begin{scnindent}
            \begin{scneqtovector}
                \scnitem{метод}
                \scnitem{класс методов}
            \end{scneqtovector}
        \end{scnindent}
        \scnitem{\scnnonamednode}
        \begin{scnindent}
            \begin{scneqtovector}
                \scnitem{деятельность}
                \scnitem{вид деятельности}
            \end{scneqtovector}
        \end{scnindent}
    \end{scnhaselementset}
    \begin{scnindent}
	    \scnsubset{семейство подклассов*}
	    \scntext{примечание}{Все сущности, принадлежащие рассмотренным \textit{понятиям}, требуют достаточно детальной \textit{спецификации}. При этом не следует путать сами сущности и их \textit{спецификации}. Так, например, не следует путать \textit{действие} и \textit{задачу}, которая специфицирует (уточняет) это \textit{действие}. Особое место среди указанных понятий занимает понятие \textit{метода}, т.к. каждый конкретный \textit{метод}, с одной стороны, является \textit{спецификацией} соответствующего \textit{класса действий}, а, с другой стороны, сам нуждается в \textit{спецификации}, которая уточняет либо \textit{декларативную семантику} этого \textit{метода} (т.е. обобщенную декларативную формулировку класса задач, решаемых с помощью этого \textit{метода}), либо \textit{операционную семантику} этого \textit{метода}, (т.е. множество \textit{методов}, обеспечивающих \textit{интерпретацию} данного специфицируемого \textit{метода}) и тем самым преобразует специфицируемый \textit{метод} в \textit{навык}.}
    \end{scnindent}
    	
    \scnheader{следует отличать*}
    \begin{scnhaselementvector}
        \scnitem{первый домен*(спецификация*)}
        \begin{scnindent}
            \scnidtf{специфицируемая сущность}
            \scnidtf{сущность, использование которой требует вполне определенной ее спецификации}
            \scnsuperset{действие}
            \scnsuperset{класс действий}
            \scnsuperset{метод}
            \scnsuperset{класс методов}
            \scnsuperset{деятельность}
            \scnsuperset{вид деятельности}
        \end{scnindent}
        \scnitem{второй домен*(спецификация*)}
    	\begin{scnindent}
            \scnidtf{спецификация}
            \scnsuperset{задача}
            \begin{scnindent}
	            \scnsuperset{декларативная формулировка задачи}
	            \begin{scnindent}
	            	\scnidtf{семантическая формулировка задачи}
	            \end{scnindent}
	            \scnsuperset{процедурная формулировка задачи}
	            \begin{scnindent}
	            	\scnidtf{функциональная формулировка задачи}
	            \end{scnindent}
	        \end{scnindent}
            \scnsuperset{план действия}
            \begin{scnindent}
	            \scnidtf{план}
	            \scnidtf{план выполнения действия}
	    	\end{scnindent}
            \scnsuperset{декларативная спецификация выполнения действий}
            \begin{scnindent}
            	\scnidtf{иерархическая система подзадач}
            \end{scnindent}
            \scnsuperset{протокол}
            \scnsuperset{результативная часть протокола}
            \scnsuperset{обобщенная декларативная формулировка класса задач}
            \scnsuperset{метод}
            \scnsuperset{декларативная семантика метода}
            \scnsuperset{операционная семантика метода}
            \scnsuperset{модель решения задач}
        \end{scnindent}
    \end{scnhaselementvector}
	\begin{scnindent}
    	\scntext{примечание}{При этом следует отличать:
	        \begin{scnitemize}
	            \item спецификацию конкретного \textit{действия} (\textit{задачу}, \textit{план}, \textit{декларативную спецификацию выполнения действия}, \textit{протокол}, \textit{результативную часть протокола});
	            \item спецификацию конкретной \textit{деятельности} (\textit{контекст}*, \textit{множество используемых методов}*);
	            \item спецификацию \textit{класса действий} (\textit{обобщенную декларативную формулировку класса задач}, \textit{метод});
	            \item спецификацию \textit{вида деятельности} (\textit{технологию});
	            \item спецификацию \textit{метода} (\textit{декларативную семантику метода}, \textit{операционную семантику метода});
	            \item спецификацию \textit{класса методов} (\textit{модель решения задач}).
	        \end{scnitemize}}
    \end{scnindent}
    
    \scnheader{следует отличать*}
    \begin{scnhaselementset}
        \scnitem{\scnnonamednode}
        \begin{scnindent}
            \begin{scneqtovector}
		      \scnitem{действие}
		      \scnitem{класс действий, метод}
            \end{scneqtovector}
        \end{scnindent}
        \scnitem{\scnnonamednode}
        \begin{scnindent}
            \begin{scneqtovector}
             \scnitem{деятельность}
             \scnitem{вид деятельности, технология}
            \end{scneqtovector}
        \end{scnindent}
    \end{scnhaselementset}
    
    \bigskip
    
    %в стандарте нет этого
    \begin{scnrelfromlist}{библиографический источник}
        \scnitem{\cite{Martynov1984}}
        \begin{scnindent}
            \scnciteannotation{Martynov1984}
    	\end{scnindent}
        \scnitem{\cite{Ikeda1998}}
    	\begin{scnindent}
            \scnciteannotation{Ikeda1998}
            \scnrelfrom{ключевой знак}{онтология классов задач}
        	\begin{scnindent}
	            \scnidtf{задачная онтология}
	            \scnidtf{онтология классов задач, решаемых в данной предметной области}
        	\end{scnindent}
        \end{scnindent}
        \scnitem{\cite{Studer1996}}
        \begin{scnindent}
            \scnciteannotation{Studer1996}
     	\end{scnindent}
        \scnitem{\cite{Benjamins1999}}
        \begin{scnindent}
            \scnciteannotation{Benjamins1999}
        \end{scnindent}
        \scnitem{\cite{Chandrasekaran1999}}
        \begin{scnindent}
            \scnciteannotation{Chandrasekaran1999}
        \end{scnindent} 
        \scnitem{\cite{Chandrasekaran1998}}
        \begin{scnindent}
            \scnciteannotation{Chandrasekaran1998}
        \end{scnindent}
        \scnitem{\cite{Fensel1998Reuse}}
        \begin{scnindent}
            \scnciteannotation{Fensel1998Reuse}
        \end{scnindent}
        \scnitem{\cite{Kemke2001}}
        \begin{scnindent}
            \scnciteannotation{Kemke2001}
    	\end{scnindent}
        \scnitem{\cite{Tu1995}}
        \begin{scnindent}
            \scnciteannotation{Tu1995}
        \end{scnindent}
        \scnitem{\cite{Trypuz2007}}
        \begin{scnindent}
            \scnciteannotation{Trypuz2007}
        \end{scnindent}
        \scnitem{\cite{Fang2019}}
        \begin{scnindent}
            \scnciteannotation{Fang2019}
        \end{scnindent}
        \scnitem{\cite{Fensel1997}}
        \scnitem{\cite{McBride2021}}
        \begin{scnindent}
            \scnciteannotation{McBride2021}
        \end{scnindent}
        \scnitem{\cite{Crowther2020}}
        \begin{scnindent}
            \scnciteannotation{Crowther2020}
        \end{scnindent}
        \scnitem{\cite{McCann1998}}
        \begin{scnindent}
            \scnciteannotation{McCann1998}
        \end{scnindent}
        \scnitem{\cite{Yan2014}}
        \begin{scnindent}
            \scnciteannotation{Yan2014}
        \end{scnindent}
        \scnitem{\cite{Ansari2018}}
        \scnitem{\cite{Crubezy2004}}
        \begin{scnindent}
            \scnciteannotation{Crubezy2004}
        \end{scnindent}
        \scnitem{\cite{Coelho1996}}
    \end{scnrelfromlist}
\end{SCn}

        \begin{SCn}
	
\scniselement{раздел}
\scniselement{предметная область и онтология}
\scnreltovector{конкатенация сегментов}{Уточнение понятия воздействия и понятия действия. Типология воздействий и действий;Уточнение понятия задачи. Типология задач;Уточнение семейства параметров и отношений, заданных на множестве воздействий, действий и задач;Предметная область и онтология субъектно-объектных спецификаций воздействий;Уточнение понятий плана сложного действия, класса задач, метода; Уточнение понятия навыка, понятия класса методов и понятия модели решения задач;Уточнение понятия деятельности, понятия вида деятельности и понятия технологии}

\scnhaselementlist{исследуемый класс первичных объектов исследования}{
воздействие\\
	\scnaddlevel{1}
	\scnidtf{\textit{процесс} воздействия одних \textit{сущностей} на другие}
	\scnsubset{процесс}
	\scnaddlevel{-1}
;действие\\
	\scnaddlevel{1}
	\scnidtf{\textit{процесс}, "осознанно"{} и целенаправленно выполняемый (управляемый) некоторой \textit{кибернетической системой}}
	\scnsubset{воздействие}
	\scnsubset{процесс}
	\scnaddlevel{-1}
\bigskip
;неосознанное воздействие
\bigskip
;действие, выполняемое в памяти субъекта действия
;действие, выполняемое во внешней среде субъекта действия
;рецепторное действие\\
	\scnaddlevel{1}
	\scnidtf{действие, выполняемое рецептором субъекта действия}
	\scnaddlevel{-1}
;эффекторное действие\\
	\scnaddlevel{1}
	\scnidtf{действие, выполняемое эффектором субъекта действия}
	\scnaddlevel{-1}
\bigskip
;элементарное действие \\
	\scnaddlevel{1}
	\scnidtf{действие, выполнение которого не требует его декомпозиции на взаимосвязанные поддействия}
	\scnaddlevel{-1}
;сложное действие
;легко выполнимое сложное действие\\
	\scnaddlevel{1}
	\scnidtf{сложное действие, которое известно, как выполнять}
	\scnaddlevel{-1}
;интеллектуальное действие\\
	\scnaddlevel{1}
	\scnidtf{сложное действие, для которого априори не известно, как его выполнять}
	\scnaddlevel{-1}
\bigskip
;индивидуальное действие\\
	\scnaddlevel{1}
	\scnidtf{действие, выполняемое индивидуальной кибернетической системой}
	\scnaddlevel{-1}
;коллективное действие\\
	\scnaddlevel{1}
	\scnidtf{действие, выполняемое коллективом кибернетических систем (многоагентной системой)}
	\scnaddlevel{-1}
\bigskip
;запланированное, но не инициированное действие
;инициированное, но не выполняемое действие
	\scnaddlevel{1}
	\scnidtf{команда}
	\scnaddlevel{-1}
;непосредственно выполняемое действие
	\scnaddlevel{1}
	\scnidtf{активное действие}
	\scnaddlevel{-1}
;прерванное действие
	\scnaddlevel{1}
	\scnidtf{выполняемое действие, находящееся в состоянии прерывания}
	\scnaddlevel{-1}
;успешно выполненное действие без ошибок
;успешно выполненное действие с ошибкой
;безуспешно выполненное действие
;отмененное действие
\bigskip
;действие с очень высоким приоритетом
;действие с высоким приоритетом
;действие со средним приоритетом
;действие с низким приоритетом
;действие с очень низким приоритетом}
\bigskip

\scnhaselementlist{исследуемый класс классов первичных объектов исследования}{
осмысленность воздействия\scnsupergroupsign
;длительность воздействия\scnsupergroupsign
;место выполнения действия\scnsupergroupsign
;функциональная сложность действия\scnsupergroupsign
;многоагентность действия\scnsupergroupsign\\
	\scnaddlevel{1}
	\scnidtf{коллективность субъекта действия}
	\scnaddlevel{-1}
;текущее состояние действия\scnsupergroupsign
;приоритет действия\scnsupergroupsign\\
	\scnaddlevel{1}
	\scnidtf{важность действия\scnsupergroupsign}
	\scnaddlevel{-1}
;срочность действия\scnsupergroupsign
\bigskip
;класс действий
;класс функционально эквивалентных действий\scnsupergroupsign
;класс логически эквивалентных действий\scnsupergroupsign
;класс семантических эквивалентных задач\scnsupergroupsign
;класс логически эквивалентных задач\scnsupergroupsign
;класс задач, для которого существует общий метод их решения\scnsupergroupsign
;класс аналогичных семантически элементарных процессов воздействия\scnsupergroupsign\\
	\scnaddlevel{1}
	\scnidtf{класс однотипных семантически элементарных воздействий\scnsupergroupsign}
	\scnaddlevel{-1}}

	
\scnhaselementlist{исследуемый класс классов}
{отношение, заданное на множестве* (действие)\\
	\scnaddlevel{1}
	\scnidtf{отношение, заданное на множестве действий}
	\scnaddlevel{-1}
;отношение, заданное на множестве* (задача)
;параметр, заданный на множестве* (действие)
;параметр, заданный на множестве* (задача)}
\scnaddlevel{1}
\scnsourcecommentpar{Здесь указаны классы классов, которые не являются классами классов \uline{первичных} объектов исследования}
\scnaddlevel{-1}
\bigskip

\scnhaselementlist{исследуемое отношение, заданное на множестве первичных объектов исследования}{воздействующая сущность\scnrolesign
;воздействуемая сущность\scnrolesign
;посредник\scnrolesign
;медиатор\scnrolesign
;субъект\scnrolesign\\
	\scnaddlevel{1}
	\scnidtf{быть субъектом заданного действия}
	\scnaddlevel{-1}
;спецификация воздействия*\\
	\scnaddlevel{1}
	\scnsuperset{спецификация действия*}
	\scnaddlevel{-1}
;спецификация действия*\\
	\scnaddlevel{1}
	\scnsuperset{задача*}
		\scnaddlevel{1}
		\scnsubdividing{декларативная формулировка задачи*;процедурная формулировка задачи*}
		\scnaddlevel{-1}
		\scnsuperset{план сложного действия*}
		\scnsuperset{декларативная спецификация выполнения сложного действия*}
		\scnsuperset{протокол*}
		\scnsuperset{результативная часть протокола*}
	\scnaddlevel{-1}
;декларативная формулировка задачи*
;процедурная формулировка задачи*
;план сложного действия*
;декларативная спецификация выполнения сложного действия*
;протокол*
;результативная часть протокола*}
\bigskip

\scnhaselementlist{исследуемое отношение}{спецификация класса действий*
;спецификация метода*
;спецификация класса методов*
;спецификация деятельности*
;спецификация вида деятельности*
}
\scnaddlevel{1}
	\scnsourcecommentpar{Здесь указаны исследуемые отношения, которые заданы не на множестве первичных объектов исследования}
\scnaddlevel{-1}
\bigskip

\scnhaselementlist{исследуемый класс структур, специфицирующих первичные объекты исследования}{
задача\\
	\scnaddlevel{1}
	\scnidtf{формулировка задачи}
	\scnidtf{спецификация действия}
	\scnidtf{структура (sc-конструкция), содержащая в идеале достаточную информацию для выполнения соответствующего (специфицируемого) действия}
	\scnaddlevel{-1}
;декларативная формулировка задачи\\
	\scnaddlevel{1}
	\scnidtf{семантическая спецификация действия}
	\scnaddlevel{-1}
;процедурная формулировка задачи\\
	\scnaddlevel{1}
	\scnidtf{функциональная спецификация действия}
	\scnaddlevel{-1}
;план сложного действия\\
	\scnaddlevel{1}
	\scnidtf{план выполнения сложного действия}
	\scnaddlevel{-1}
;процедурный план сложного действия
;непроцедурный план сложного действия\\
	\scnaddlevel{1}
	\scnidtf{декларативный план сложного действия}
	\scnidtf{иерархическая система подзадач заданной сложной задачи}
	\scnaddlevel{-1}}
\bigskip

\scnhaselementlist{исследуемый класс структур}{метод\\
	\scnaddlevel{1}
	\scnidtf{спецификация класса сложных действий}
	\scnaddlevel{-1}
;денотационная семантика метода
;операционная семантика метода
;навык
;модель решения задач
;технология}
\scnaddlevel{1}
	\scnsourcecommentpar{Здесь указаны классы структур, не являющихся спецификациями первичных объектов исследования}
\scnaddlevel{-1}
\bigskip

\scnhaselementlist{вводимое, но не исследуемое понятие}
{действие, выполняемое в памяти ostis-системы
;действие, выполняемое ostis-системой в своей внешней среде
;рецептурное действие ostis-системы
;эффекторное действие ostis-системы
;sc-агент\\
	\scnaddlevel{1}
	\scnidtf{внутренний субъект ostis-системы}
	\scnidtf{субъект, реализующий действия, выполняемые в памяти ostis-системы}
	\scnaddlevel{-1}}
\scnaddlevel{1}
	\scnsourcecommentpar{Здесь указаны понятия, исследуемые в предметной области (и соответствующей онтологии), которая является \uline{частной} по отношению к заданной и которая наследует все свойства заданной предметной области и онтологии}
\scnaddlevel{-1}
\bigskip

\scnhaselementlist{используемое понятие, исследуемое в другой предметной области и онтологии}{
кибернетическая система\\
	\scnaddlevel{1}
	\scnidtf{сущность, обладающая способностью быть субъектом различного вида действий}
	\scnaddlevel{-1}
;компьютерная система\\
	\scnaddlevel{1}
	\scnidtf{искусственная кибернетическая система}
	\scnsubset{кибернетическая система}
	\scnaddlevel{-1}
;интеллектуальная компьютерная система\\
	\scnaddlevel{1}
	\scnsubset{компьютерная система}
	\scnsuperset{ostis-система}
	\scnaddlevel{-1}
;человек\\
	\scnaddlevel{1}
	\scnsubset{кибернетическая система}
	\scnaddlevel{-1}
;ostis-система
;спецификация*
	\scnaddlevel{1}
	\scnidtf{быть спецификацией (описанием, семантической окрестностью заданной сущности*)}
	\scnidtf{семантическая окрестность*}
	\scnaddlevel{-1}}


\scnheader{следует отличать*}
\scnhaselementset{
	\scnmakevectorlocal{действие;класс действий};
	\scnmakevectorlocal{метод;класс методов};
	\scnmakevectorlocal{деятельность;вид деятельности}
}
\scnaddlevel{1}
\scnsubset{семейство подклассов*}
\scnnote{Все сущности, принадлежащие рассмотренным \textit{понятиям}, требуют достаточно детальной \textit{спецификации}. При этом не следует путать сами сущности и их \textit{спецификации}. Так, например, не следует путать \textit{действие} и \textit{задачу}, которая специфицирует (уточняет) это \textit{действие}. Особое место среди указанных понятий занимает понятие \textit{метода}, т.к. каждый конкретный \textit{метод}, с одной стороны, является \textit{спецификацией} соответствующего \textit{класса действий}, а, с другой стороны, сам нуждается в \textit{спецификации}, которая уточняет либо \textit{декларативную семантику} этого \textit{метода} (т.е. обобщенную декларативную формулировку класса задач, решаемых с помощью этого \textit{метода}), либо \textit{операционную семантику} этого \textit{метода}, (т.е. множество \textit{методов}, обеспечивающих \textit{интерпретацию} данного специфицируемого \textit{метода}) и тем самым "преобразует"{} специфицируемый \textit{метод} в \textit{навык}.}
\scnaddlevel{-1}


\scnheader{следует отличать*}
\scnhaselementvector{первый домен*(спецификация*)\\
\scnaddlevel{1}
\scnidtf{специфицируемая сущность}
\scnidtf{сущность, использование которой требует вполне определенной ее спецификации}
\scnsuperset{действие}
\scnsuperset{класс действий}
\scnsuperset{метод}
\scnsuperset{класс методов}
\scnsuperset{деятельность}
\scnsuperset{вид деятельности}
\scnaddlevel{-1};
второй домен*(спецификация*)\\
\scnaddlevel{1}
\scnidtf{спецификация}
\scnsuperset{задача}
\scnaddlevel{1}
\scnsuperset{декларативная формулировка задачи}
	\scnaddlevel{1}
	\scnidtf{семантическая формулировка задачи}
	\scnaddlevel{-1}
\scnsuperset{процедурная формулировка задачи}
	\scnaddlevel{1}
	\scnidtf{функциональная формулировка задачи}
	\scnaddlevel{-1}
\scnaddlevel{-1}
\scnsuperset{план действия}
	\scnaddlevel{1}
	\scnidtf{план}
	\scnidtf{план выполнения действия}
	\scnaddlevel{-1}
\scnsuperset{декларативная спецификация выполнения действий}
	\scnaddlevel{1}
	\scnidtf{иерархическая система подзадач}
	\scnaddlevel{-1}
\scnsuperset{протокол}
\scnsuperset{результативная часть протокола}
\scnsuperset{обобщенная декларативная формулировка класса задач}
\scnsuperset{метод}
\scnsuperset{декларативная семантика метода}
\scnsuperset{операционная семантика метода}
\scnsuperset{модель решения задач}
\scnaddlevel{-1}
}
\scnaddlevel{1}
\scnnote{
При этом следует отличать:
\begin{scnitemize}
\item спецификацию конкретного \textit{действия} (\textit{задачу}, \textit{план}, \textit{декларативную спецификацию выполнения действия}, \textit{протокол}, \textit{результативную часть протокола});
\item спецификацию конкретной \textit{деятельности} (\textit{контекст}*, \textit{множество используемых методов}*);
\item спецификацию \textit{класса действий} (\textit{обобщенную декларативную формулировку класса задач}, \textit{метод});
\item спецификацию \textit{вида деятельности} (\textit{технологию});
\item спецификацию \textit{метода} (\textit{декларативную семантику метода}, \textit{операционную семантику метода});
\item спецификацию \textit{класса методов} (\textit{модель решения задач}).
\end{scnitemize}
}
\scnaddlevel{-1}


\scnheader{следует отличать*}
\scnhaselementset{
	\scnmakevectorlocal{действие;класс действий, метод};
	\scnmakevectorlocal{деятельность;вид деятельности, технология}
}

\end{SCn}
        \begin{SCn}
    \scnsegmentheader{Уточнение понятия задачи. Типология задач}
    \begin{scnsubstruct}
        \scniselement{сегмент базы знаний}
        
        \scnheader{задача}
        \scnidtf{описание желаемого состояния или события в рамках внешней среды кибернетической системы либо в рамках её базы знаний}
        \scnidtf{формулировка задачи}
        \scnidtf{задание на выполнение некоторого действия}
        \scnidtf{постановка задачи}
        \scnidtf{описание задачной ситуации}
        \scnidtf{спецификация некоторого действия, обладающая достаточной полнотой для выполнения этого действия}
        \scnidtf{цель плюс дополнительные условия (ограничения) накладываемые на результат или процесс получения этого результата}
        \scnidtf{описание того, что требуется сделать}
        \scntext{пояснение}{\textbf{\textit{Задача}}, т.е. формальное описание условия некоторой задачи есть, по сути, формальная \textit{спецификация} некоторого \textit{действия}, направленного на решение данной \textit{задачи}, достаточная для выполнения данного \textit{действия} каким-либо \textit{субъектом}. В зависимости от конкретного \textit{класса задач}, описываться может как внутреннее состояние самой интеллектуальной системы, так и требуемое состояние \textit{внешней среды}. \textit{sc-элемент}, обозначающий \textit{действие} входит в \textit{задачу} под атрибутом \textit{ключевой знак\scnrolesign}.\\
            Каждая \textbf{\textit{задача}} представляет собой спецификацию \textit{действия}, которое либо уже выполнено, либо выполняется в текущий момент (в настоящее время), либо планируется (должно) быть выполненным, либо может быть выполнено (но не обязательно).\\
            Классификация \textit{задач} может осуществляться по дидактическому признаку в рамках каждой предметной области, например, задачи на треугольники, задачи на системы уравнений и т.п.\\
            Каждая \textit{задача} может включать:
            \begin{scnitemize}
                \item факт принадлежности \textit{действия} какому-либо частному классу \textit{действий} (например,\textit{действие. сформировать полную семантическую окрестность указываемой сущности}), в том числе состояние \textit{действия} с точки зрения жизненного цикла (инициированное, выполняемое и т.д.);
                \item описание \textit{цели*} (\textit{результата*}) \textit{действия}, если она точно известна;
                \item указание \textit{заказчика*} действия;
                \item указание \textit{исполнителя* действия} (в том числе, коллективного);
                \item указание \textit{аргумента(ов) действия\scnrolesign};
                \item указание инструмента или посредника \textit{действия};
                \item описание \textit{декомпозиции действия*};
                \item указание \textit{последовательности действий*} в рамках \textit{декомпозиции действия*}, т.е построение плана решения задачи. Другими словами, построение плана решения представляет собой декомпозицию соответствующего \textit{действия} на систему взаимосвязанных между собой поддействий;
                \item указание области \textit{действия};
                \item указание условия инициирования \textit{действия};
                \item момент начала и завершения \textit{действия}, в том числе планируемый и фактический, предполагаемая и/или фактическая длительность выполнения;
            \end{scnitemize}
            Некоторые \textit{задачи} могут быть дополнительно уточнены контекстом --- дополнительной информацией о сущностях, рассматриваемых в формулировке \textit{задачи}, т.е. описанием того, что дано, что известно об указанных сущностях.\\
            Кроме этого, \textit{задача} может включать любую дополнительную информацию о действии, например:
            \begin{scnitemize}
                \item перечень ресурсов и средств, которые предполагается использовать при решении задачи, например список доступных исполнителей, временные сроки, объем имеющихся финансов и т.д.;
                \item ограничение области, в которой выполняется \textit{действие}, например, необходимо заменить одну \textit{\mbox{sc-конструкцию}} на другую по некоторому правилу, но только в пределах некоторого \textit{раздела базы знаний};
                \item ограничение знаний, которые можно использовать для решения той или иной задачи, например, необходимо решить задачу по алгебре используя только те утверждения, которые входят в курс школьной программы до седьмого класса включительно, и не используя утверждения, изучаемые в старших классах;
                \item и прочее
            \end{scnitemize}
            С одной стороны, решаемые системой \textit{задачи}, можно классифицировать на \textit{информационные задачи} и \textit{поведенческие задачи}.\\
            С точки зрения формулировки поставленной задачи можно выделить \textit{декларативные формулировки задачи} и \textit{процедурные формулировки задачи}. Следует отметить, что данные классы задач не противопоставляются, и могут существовать формулировки задач, использующие оба подхода.}
        \scntext{правило идентификации экземпляров}{Экземпляры класса \textbf{\textit{задач}} в рамках \textit{Русского языка} именуются по следующим правилам:
            \begin{scnitemize}
                \item в начале идентификатора пишется слово \scnqqi{\textit{Задача}} и ставится точка;
                \item далее с прописной буквы идет либо содержащее глагол совершенного вида в инфинитиве описание сути того, что требуется получить в результате выполнения действия, либо вопросительное предложение, являющееся спецификацией запрашиваемой (ответной) информации.
            \end{scnitemize}
            Например:\\
            \textit{Задача. Сформировать полную семантическую окрестность понятия треугольник}\\
            \textit{Задача. Верифицировать Раздел. Предметная область sc-элементов}}
        \scnsubset{семантическая окрестность}
        \scnsuperset{вопрос}
        \scnsuperset{команда}
        \scnidtf{спецификация действия, которое выполнилось, выполняется или может быть выполнено соответствующей кибернетической системой}
        \scntext{примечание}{Каждой задаче и, соответственно, каждому специфицируемому действию соответствует определенная кибернетическая система, являющаяся субъектом, выполняющим это действие.}
        \scnsubset{знание}
        \scntext{примечание}{Каждая \textit{задача} --- это \textit{знание}, описывающее то какое действие возможно потребуется выполнить.}
        \scnsuperset{инициированная задача}
        \begin{scnindent}
        	\scnidtf{формулировка задачи, которая подлежит выполнению}
        \end{scnindent}
        \scnidtf{спецификация (описание) соответствующего действия}
        \scnsuperset{декларативная формулировка задачи}
        \begin{scnindent}
        	\scnidtf{задача, в формулировке которой явно указывается (описывается) целевая ситуация, т.е. то, что является результатом выполнения (решения) данной задачи}
        \end{scnindent}
        \scnsuperset{процедурная формулировка задачи}
        \begin{scnindent}
        	\scnidtf{задача, в формулировке которой явно указывается характеристика действия, специфицируемого этой задачей, а именно, например, указывается:
	            \begin{scnitemize}
	                \item субъект или субъекты, выполняющие это действие,
	                \item объекты, над которыми действие выполняется, --- аргументы действия,
	                \item инструменты, с помощью которых выполняется действие,
	                \item момент и, возможно, дополнительные условия начала и завершения выполнения действия
	            \end{scnitemize}}
        \end{scnindent}
        \scnsuperset{декларативно-процедурная формулировка задачи}
        \begin{scnindent}
        	\scnidtf{задача, в формулировке которой присутствуют как декларативные (целевые), так и процедурные аспекты}
        \end{scnindent}
        \scntext{примечание}{От качества (корректности и полноты) формулировки задачи, т.е. спецификации соответствующего действия, во многом зависит качество (эффективность) выполнения этого действия, т.е. качество процесса решения указанной задачи.}
        \scnsuperset{проблема}
        \begin{scnindent}
	        \scnidtf{проблемная задача}
	        \scnidtf{сложная, трудно решаемая задача}
	        \scnsuperset{изобретательская задача}
        \end{scnindent}
        
        \scnheader{процедурная формулировка задачи}
        \scnidtf{спецификация действия, которое планируется быть выполненным}
        \scntext{пояснение}{В случае \textbf{\textit{процедурной формулировки задачи}}, в формулировке задачи явно указываются аргументы соответствующего задаче \textit{действия}, и в частности, вводится семантическая типология \textit{действий}. При этом явно не уточняется, что должно быть результатом выполнения данного действия. Заметим, что, при необходимости, \textit{процедурная формулировка задачи} может быть сведена к \textit{декларативной формулировке задачи} путем трансляции на основе некоторого правила, например определения класса действия через более общий класс.}
        
        \scnheader{декларативная формулировка задачи}
        \scnidtf{описание ситуации (состояния), которое должно быть достигнуто в результате выполнения планируемого действия}
        \scntext{пояснение}{В случае \textit{декларативной формулировки задачи}, при описании условия задачи специфицируется цель \textit{действия}, т.е. результат, который должен быть получен при успешном выполнении \textit{действия}.}
        \scnrelto{второй домен}{декларативная формулировка задачи*}
        \scnidtf{описание исходной (начальной) ситуации, являющейся условием выполнения соответствующего действия и целевой (конечной) ситуации, являющейся результатом выполнения этого действия}
        \scnidtf{семантическая спецификация действия}
        \scntext{примечание}{Формулировка \textit{задачи} может не содержать указания контекста (области решения) \textit{задачи} (в этом случае областью решения \textit{задачи} считается либо вся \textit{база знаний}, либо ее согласованная часть), а также может не содержать либо описания исходной ситуации, либо описания целевой ситуации. Так, например, описания целевой ситуации для явно специфицированного противоречия, обнаруженного в \textit{базе знаний} не требуется.}
        \scnidtf{формулировка (описание) задачной ситуации с явным или неявным описанием контекста (условий) выполнения специфицируемого действия, а также результата выполнения этого действия}
        \scnidtf{явное или неявное описание
            \begin{scnitemize}
                \item того, что \uline{дано} --- исходные данные, условия выполнения специфируемого действия,
                \item того, что \uline{требуется} --- формулировка цели, результата выполнения указанного действия
            \end{scnitemize}}
        \scnhaselementrole{пример}{\scnfileimage[40em]{Contents/part_kb/images/sd_task/declarative_task_statement.png}}
        \begin{scnindent}
	        \scntext{пояснение}{Выполнение данного действия сведется к следующим \uline{событиям}:
	            \begin{scnitemize}
	                \item для числа \textit{с} будет сгенерирован уникальный идентификатор, являющийся его представлением в соответствующей системе счисления
	                \item будет сгенерирована константная настоящая позитивная пара принадлежности, соединяющая узел \textit{вычислено} с узлом \textit{с}
	                \item удалится константная будущая позитивная пара принадлежности, а также константная настоящая нечеткая пара принадлежности, выходящие из узла \textit{вычислено}.
	            \end{scnitemize}
	            Таким образом, после выполнения действия \uline{все} \uline{будущие} сущности, входящие в целевую ситуацию, становятся \uline{настоящими} сущностями, а некоторые \uline{настоящие} сущности, входящие в исходную ситуацию, становятся \uline{прошлыми}.}
        \end{scnindent}
        
        \scnheader{задача}
        \scnsuperset{задача, решаемая в памяти кибернетической системы}
        \begin{scnindent}
	        \scnsuperset{задача, решаемая в памяти индивидуальной кибернетической системы}
	        \scnsuperset{задача, решаемая в общей памяти многоагентной системы}
	        \scnidtf{информационная задача}
	        \scnidtf{задача, направленная либо на \uline{генерацию} или поиск информации, удовлетворяющей заданным требованиям, либо на некоторое \uline{преобразование} заданной информации}
	        \scnsuperset{математическая задача}
	    \end{scnindent}
        \scnsuperset{элементарная информационная задача}
        \scnsuperset{простая информационная задача}
        \scnsuperset{проблемная информационная задача}
        \begin{scnindent}
	        \scnidtf{интеллектуальная информационная задача}
	        \scnsuperset{проблема Гильберта}
        \end{scnindent}
        
        \scnheader{вопрос}
        \scnidtf{запрос}
        \scnsubset{задача, решаемая в памяти кибернетической системы}
        \scnidtf{непроцедурная формулировка задачи на поиск (в текущем состоянии базы знаний) или на генерацию знания, удовлетворяющего заданным требованиям}
        \scnsuperset{вопрос --- что это такое}
        \scnsuperset{вопрос --- почему}
        \scnsuperset{вопрос --- зачем}
        \scnsuperset{вопрос --- как}
        \begin{scnindent}
	        \scnidtf{каким способом}
	        \scnidtf{запрос метода (способа) решения заданного (указываемого) вида задач или класса задач либо, плана решения конкретной указываемой задачи}
	    \end{scnindent}
        \scnidtf{задача, направленная на удовлетворение информационной потребности некоторого субъекта-заказчика}
        
        \scnheader{команда}
        \scnidtf{инициированная задача}
        \scnidtf{спецификация инициированного действия}
        \scntext{пояснение}{Идентификатор экземпляров конкретного класса \textbf{\textit{команд}} в рамках \textit{Русского языка} пишется с прописной буквы и представляет собой либо содержащее глагол совершенного вида в инфинитиве описание сути того, что требуется получить в результате выполнения действия, соответствующего данной \textbf{\textit{команде}}, либо вопросительное предложение, являющееся спецификацией запрашиваемой (ответной) информации.\\
        	Например:\\
            \textit{Сформировать полную семантическую окрестность понятия треугольник}\\
            \textit{Верифицировать Раздел. Предметная область sc-элементов}}
            
        \scnheader{задача}
        \scntext{примечание}{Сужение бинарного ориентированного отношения \textit{спецификация*} (быть спецификацией*), связывающее \textit{действия} с \textit{задачами}, которые решаются в результате выполнения этих \textit{действий}, не является взаимно однозначным.Каждому \textit{действию} может соответствовать несколько формулировок \textit{задач}, которые с разной степенью детализации или с разных аспектов специфицируют указанное \textit{действие}.Кроме того, интерпретация \uline{разных} формулировок семантически одной и той же \textit{задачи} в общем случае приводит к \uline{разным} \textit{действиям}, решающим эту \textit{задачу}.Подчеркнем, что \uline{разные}, но семантически эквивалентные формулировки \textit{задач} считаются формулировками формально \uline{разных} \textit{задач}.}
        
        \scnheader{отношение, заданное на множестве*(задача)}
        \scnhaselement{\scnkeyword{задача}*}
        \begin{scnindent}
	        \scniselement{неосновное понятие}
	        \scnidtf{формулировка задачи*}
	        \scnidtf{спецификация действия, уточняющая то, \uline{что} должно быть сделано*}
	        \begin{scnsubdividing}
	            \scnitem{декларативная формулировка задачи*}
	            \scnitem{процедурная формулировка задачи*}
	        \end{scnsubdividing}
	        \scnrelfrom{второй домен}{\scnkeyword{задача}}
	        \begin{scnindent}
		        \scniselement{основное понятие}
		        \scnsuperset{задача обработки базы знаний}
		        \scnsuperset{задача обработки файлов}
		        \scnsuperset{задача, решаемая кибернетической системой во внешней среде}
		        \scnsuperset{задача, решаемая кибернетической системой в собственной физической оболочке}
		    \end{scnindent}
	    \end{scnindent}
        \scnhaselement{\scnkeyword{декларативная формулировка задачи}*}
        \begin{scnindent}
	        \scniselement{неосновное понятие}
	        \scnidtf{описание исходной ситуации и целевой ситуации специфицируемого действия*}
	        \scntext{пояснение}{декларативная формулировка задачи включает в себя:
	            \begin{scnitemize}
	                \item связку отношения \textit{цель}*, связывающую специфицируемое действие с описанием целевой ситуации;
	                \item само описание целевой ситуации;
	                \item связку отношения \textit{исходная ситуация*}, связывающую специфицируемое действие с описанием исходной ситуации;
	                \item непосредственно описание исходной ситуации;
	                \item указание контекста (области решения) задачи.
	            \end{scnitemize}
	            При этом указание и описание исходной ситуации может отсутствовать.}
	    \end{scnindent}
        \scnhaselement{\scnkeyword{процедурная формулировка задачи}*}
        \begin{scnindent}
	        \scniselement{неосновное понятие}
	        \scnidtftext{пояснение}{указание
	            \begin{scnitemize}
	                \item \textit{класса действий}, которому принадлежит специфицируемое \textit{действие}, а также
	                \item \textit{субъекта} или субъектов, выполняющих это действие (с дополнительным указанием роли каждого участвующего субъекта);
	                \item \textit{объекта} или объектов, над которыми осуществляется действие (с указанием роли каждого такого объекта);
	                \item используемых материалов;
	                \item используемых инструментов (инструментальных средств);
	                \item дополнительных темпоральных характеристик специфицируемого действия (сроки, длительность);
	                \item приоритета (важности) специфицируемого действия;
	                \item если описываемое действие не является элементарным в текущем контексте, то декомпозиции описываемого действия на поддействия и этих поддействий на еще более простые поддействия, вплоть до элементарных действий.
	            \end{scnitemize}}
        \end{scnindent}
        
        \scnheader{задача}
        \scntext{примечание}{Для некоторых \textit{отношений}, заданных на множестве \textit{действий}, вводятся аналогичные \textit{отношения}, заданные на множестве \textit{задач}, соответствующих этим \textit{действиям}. От \textit{отношений}, связывающих некоторые сущности, несложно перейти к \textit{отношениям}, связывающим \textit{спецификации*} этих сущностей.}
        
        \bigskip
    \end{scnsubstruct}
    
    \scnendsegmentcomment{Уточнение понятия задачи. Типология задач}
    \scnsegmentheader{Уточнение семейства параметров и отношений, заданных на множестве воздействий, действий и задач. Типология спецификаций воздействий, действий и задач}
    \begin{scnsubstruct}
        \scniselement{сегмент базы знаний}
        
        \scnheader{отношение, заданное на множестве*(действие)}
        \scnidtf{отношение, заданное на множестве действий*}
        \scnhaselement{поддействие*}
        \begin{scnindent}
	        \scnidtf{частное действие*}
	        \scnsubset{темпоральная часть*}
	        \scntext{пояснение}{Связки отношения \textit{поддействие*} связывают \textit{действие} и некоторое более простое частное \textit{действие}, выполнение которого необходимо для выполнения исходного более общего \textit{действия}.}
	        \scniselement{бинарное отношение}
	        \scniselement{отношение таксономии}
	        \scnrelboth{обратное отношение}{наддействие*}
	        \scnidtf{быть действием, являющимся частью заданного действия более высокого уровня иерархии*}
	        \scnidtf{быть действием, направленным на решение задачи, которая является подзадачей по отношению к задаче, решение которой осуществляется заданным действием*}
	        \scnsuperset{\textit{непосредственное поддействие*}}
	        \begin{scnindent}
	        	\scnidtf{быть таким поддействием заданного действия, для которого не существует наддействия, которое было бы также поддействием заданного действия*}
	        \end{scnindent}
	    \end{scnindent}
        
        \scnheader{спецификация*}
        \scnsuperset{сужение отношения по первому домену*(спецификация*; действие)*}
        \begin{scnindent}
	        \scnidtftext{часто используемый sc-идентификатор}{спецификация действия*}
	        \begin{scnsubdividing}
	            \scnitem{задача*}
	            \begin{scnindent}
	                \begin{scnsubdividing}
	                    \scnitem{декларативная формулировка задачи*}
	                    \begin{scnindent}
	                        \scnrelfrom{второй домен}{декларативная формулировка задачи}
	                    \end{scnindent}
	                    \scnitem{процедурная формулировка задачи*}
	                    \begin{scnindent}
	                        \scnrelfrom{второй домен}{процедурная формулировка задачи}
	                    \end{scnindent}
	                \end{scnsubdividing}
	   			\end{scnindent}
	            \scnitem{исходная ситуация действия*}
	            \scnitem{цель*}
	            \scnitem{план*}
	            \scnitem{декларативная спецификация выполнения действия*}
	            \scnitem{контекст действия*}
	            \begin{scnindent}
	                \scnidtf{информационный ресурс необходимый для выполнения заданного действия*}
	            \end{scnindent}
	            \scnitem{множество используемых методов*}
	            \begin{scnindent}
	                \scnidtf{множество методов, используемых для выполнения заданного действия*}
	                \scnidtf{операционный (функциональный) ресурс, необходимый для выполнения заданного действия*}
	            \end{scnindent}
	            \scnitem{протокол*}
	            \scnitem{результативная часть протокола*}
	        \end{scnsubdividing}
	    \end{scnindent}
        \scntext{примечание}{Таким образом, каждому \textit{действию} может быть поставлен в соответствие целый ряд видов спецификации этого \textit{действия}, которые описывают различные аспекты специфицируемого действия --- и то, что является причиной (условием) инициирования этого действия, и то, что является результатом (сухим остатком) его выполнения, и то, и то, с помощью таких ресурсов оно может быть выполнено, и то, как управлять этими ресурсами в процессе выполнения действия, и то, как на самом деле это действие было выполнено.}
        
        \scnheader{следует отличать*}
        \begin{scnhaselementset}
            \scnitem{\scnnonamednode}
            \begin{scnindent}
                \begin{scneqtovector}
                    \scnitem{действие}
                    \scnitem{задача*}
                \end{scneqtovector}
            \end{scnindent}
            \scnitem{\scnnonamednode}
            \begin{scnindent}
                \begin{scneqtovector}
                    \scnitem{действие}
                    \scnitem{исходная ситуация*}
                \end{scneqtovector}
            \end{scnindent}
            \scnitem{\scnnonamednode}
            \begin{scnindent}
                \begin{scneqtovector}
                    \scnitem{действие}
                    \scnitem{декларативная формулировка задачи*}
                \end{scneqtovector}
            \end{scnindent}
            \scnitem{\scnnonamednode}
            \begin{scnindent}
                \begin{scneqtovector}
                    \scnitem{действие}
                    \scnitem{процедурная формулировка задачи*}
                \end{scneqtovector}
            \end{scnindent}
            \scnitem{\scnnonamednode}
            \begin{scnindent}
                \begin{scneqtovector}
                    \scnitem{действие}
                    \scnitem{план*}
                \end{scneqtovector}
            \end{scnindent}
            \scnitem{\scnnonamednode}
            \begin{scnindent}
                \begin{scneqtovector}
                    \scnitem{действие}
                    \scnitem{декларативная спецификация выполнения действия*}
                \end{scneqtovector}
            \end{scnindent}
            \scnitem{\scnnonamednode}
            \begin{scnindent}
                \begin{scneqtovector}
                    \scnitem{действие}
                    \scnitem{протокол*}
                \end{scneqtovector}
            \end{scnindent}
            \scnitem{\scnnonamednode}
            \begin{scnindent}
                \begin{scneqtovector}
                    \scnitem{действие}
                    \scnitem{результативная часть протокола*}
                \end{scneqtovector}
            \end{scnindent}
        \end{scnhaselementset}
        
        \scnheader{сужение отношения по первому домену*(спецификация*; действие)*}
        \scnidtf{спецификация действия*}
        \scntext{примечание}{\textit{спецификацию действия} (базовое описание действия) условно можно разбить на следующие части:
            \begin{scnitemize}
                \item описание состояния действия в текущий момент времени --- действие может принадлежать:
                \begin{scnitemizeii}
                    \item либо классу \textit{прогнозируемых сущностей} (в случае действий --- это планируемые действия, которые могут быть, но не обязательно выполняться в будущем);
                    \item либо классу \textit{настоящих сущностей}, т.е. сущностей, существующих в настоящий (текущий) момент времени;
                    \item либо классу \textit{прошлых сущностей}, завершивших свое существование (в случае действий --- это действия, выполнение которых уже завершено);
                \end{scnitemizeii}
                \item формулировки \textit{задачи}, которая должна быть решена в результате выполнения специфицируемого действия. Такая формулировка представляет собой логико-семантическое описание \textit{задачной продукции}, включающей в себя:
                \begin{scnitemizeii}
                    \item описание \textit{исходной ситуации} и/или события (исходных условий того, что должно быть дано, исходных данных, исходного контекста). Для \textit{действий во внешней среде} (действий/задач, выполняемых во внешней среде) в описании \textit{исходной ситуации} должно быть включено описание необходимых для решения задачи материальных ресурсов (сырья, комплектации) с указанием их количества;
                    \item описание \textit{целевой ситуации} и/или события (того, что требуется получить в результате решения данной задачи);
                    \item указание дополнительных \textit{инструментальных средств}, используемых для выполнения специфицируемого действия (такие средства могут быть использованы только при выполнении \textit{действий во внешней среде}).
                \end{scnitemizeii}
                \item указание субъектов-исполнителей специфицируемого действия:
                \begin{scnitemizeii}
                    \item множество тех, кто может выполнить это действие;
                    \item тот, кто должен (которому поручено выполнить это действие);
                \end{scnitemizeii}
                \item указание метода, на основании (путем интерпретации) которого специфицируемое действие может быть выполнено --- таких методов в общем случае может быть несколько;
                \item спецификация выполненного действия, т.е. действия, отнесенного к классу \textit{прошлых сущностей}:
                \begin{scnitemizeii}
                    \item указание отрезка времени выполнения действия (момента начала и момента завершения);
                    \item указание числа прерываний (ожиданий) процесса выполнения действия;
                    \item указание чистой длительности процесса выполнения действия;
                    \item указание успешности выполнения процесса (в случае неуспешности --- указание штатных причин и сбоев).
                \end{scnitemizeii}
            \end{scnitemize}}
        
        \scnheader{отношение, связывающее действия с их спецификациями}
        \scnhaselement{\scnkeyword{декларативная спецификация выполнения действия*}}
        \begin{scnindent}
	        \scnsubset{спецификация*}
	        \scnrelfrom{второй домен}{декларативная спецификация выполнения действия}
		    \begin{scnindent}
		        \scntext{пояснение}{В состав такой спецификации действия входят:
		            \begin{scnitemize}
		                \item \scnkeyword{\textit{контекст действия*}}, содержащий информацию, достаточную для его выполнения;
		                \item \scnkeyword{множество используемых методов*} и инструментов, достаточных для выполнения действия.
		            \end{scnitemize}}
		        \scnidtf{непроцедурное описание выполнения сложного (неэлементарного) действия}
		        \scnsuperset{функциональная спецификация выполнения действия}
		        \scnsuperset{логическая спецификация выполнения действия}
		        \scntext{примечание}{\textit{декларативная спецификация выполнения действия} --- это такой выделенный фрагмент \textit{базы знаний} (такой \textit{контекст*} выполнения соответствующего конкретного \textit{действия}), которого \uline{достаточно} для выполнения этого \textit{действия} с помощью заданного множества \textit{методов}, используемых в рамках указанного \textit{контекста*}. При этом важна \uline{минимизация} и самого \textit{контекста*} и \textit{множества используемых методов*}.}
		    \end{scnindent}
	    \end{scnindent}
        \scnhaselement{\scnkeyword{контекст действия*}}
        \begin{scnindent}
	        \scnidtf{задачная ситуация*}
	        \scnidtf{что дано*}
	        \scnidtf{дополнительная информация о тех сущностях, которые входят в описание цели*}
	        \scnidtf{связь между некоторой задачей (формулировкой задачи) и состоянием базы знаний, возможностей и навыков некоторого субъекта, перед которым поставлена указанная задача*}
	        \scnidtf{связь между формулировкой задачи, т.е. описанием того, что требуется, и контекстом этой задачи, т.е. описанием имеющихся ресурсов, описанием того, что дано*}
	        \scntext{пояснение}{Связки отношения \textbf{\textit{контекст действия*}} связывают \textit{sc-элементы}, обозначающие \textit{действие} и \textit{структуры}, обозначающие контекст выполнения данного \textit{действия}, то есть некоторую дополнительную информации о тех сущностях, которые входят в описание \textit{цели*}. Как правило, контекст используется для указания собственно условия некоторой задачи, того, что дано, т.е. тех знаний, которые можно использовать для вывода новых знаний при решении задачи. Таким образом, контекст непосредственно влияет на то, как будет решаться та или иная задача, при этом даже задачи соответствующие одному классу действий, могут решаться по-разному.\\
	            Контекст может быть представлен не только в виде атомарного фактографического высказывания, но и в виде высказывания более сложного вида. Это может быть, например:
	            \begin{scnitemize}
	                \item определение множества, используемого в описании \textit{цели*};
	                \item утверждение, учет которого может быть полезен в решении задач;
	            \end{scnitemize}
	            Первым компонентом связок отношения \textbf{\textit{контекст действия*}} является знак \textit{действия}, вторым --- знак \textit{структуры}, обозначающей контекст.}
	    \end{scnindent}
        \scnhaselement{\scnkeyword{множество используемых методов*}}
        \scnhaselement{\scnkeyword{протокол}*}
        \begin{scnindent}
	        \scniselement{неосновное понятие}
	        \scnidtf{декомпозиция выполненного действия на систему последовательно-параллельно выполненных его \uline{под}действий*}
	        \scnidtf{описание того, как действительно было выполнено соответствующее действие и, в частности, описание последовательности соответствующих ситуаций и событий*}
	        \scnidtf{протокол выполнения сложного действия, включающий в себя протоколы выполнения всех поддействий этого действия*}
	        \scnidtf{протокол решения задачи*}
	        \scnidtf{история решения выполненной задачи*}
	        \scntext{пояснение}{Каждый \textbf{\textit{протокол}} представляет собой \textit{семантическую окрестность, ключевым sc-элементом\scnrolesign} является \textit{действие}, для которого собственно описывается весь процесс его выполнения, то есть все более простые поддействия, в том числе те, выполнение которых, как выяснилось позже, не было целесообразным. Подразумевается, что \textit{sc-элемент}, обозначающий данное действие, входит во множество прошлых сущностей.\\
	        Таким образом, \textbf{\textit{протокол}} представляет собой \textit{sc-текст}, содержащий декомпозицию рассматриваемого \textit{действия} на поддействия с указанием порядка их выполнения, а также необходимой спецификацией каждого такого поддействия.\\
	        В отличие от \textit{плана}, \textbf{\textit{протокол}} всегда формируется по факту выполнения соответствующего \textit{действия}.}
	    \end{scnindent}
        \scnhaselement{\scnkeyword{результативная часть протокола}*}
        \begin{scnindent}
	        \scniselement{неосновное понятие}
	        \scnidtf{часть протокола соответствующего выполненного действия, которая включает в себя только те его поддействия, которые действительно внесли вклад в построение результата (сухого остатка) этого выполненного действия*}
	        \scntext{примечание}{протокол выполненного действия и результативная часть этого протокола могут сильно отличаться. Примером тому является, например, соотношение между протоколом доказательства некоторой конкретной теоремы и результативной частью этого протокола, которая является подтверждением корректности проведенного доказательства и, соответственно, обоснованием истинности доказанной теоремы.}
	        \scntext{пояснение}{Каждая \textbf{\textit{результативная часть протокола}} представляет собой \textit{семантическую окрестность, ключевым sc-элементом\scnrolesign} является \textit{действие}, для которого собственно описывается процесс его выполнения, то есть решение соответствующей задачи. Подразумевается, что \textit{sc-элемент}, обозначающий данное \textit{действие}, входит во множество \textit{успешно выполненных действий}.\\
	        Таким образом, \textbf{\textit{результативная часть протокола}} представляет собой \textit{sc-текст}, содержащий декомпозицию рассматриваемого \textit{действия} на поддействия с указанием порядка их выполнения, а также необходимой спецификацией каждого такого поддействия.}
	    \end{scnindent}
        \scnhaselement{\scnkeyword{декомпозиция действия*}}
        \begin{scnindent}
	        \scniselement{отношение декомпозиции}
	        \scniselement{квазибинарное отношение}
	        \scnidtf{сведение действия ко множеству более простых взаимосвязанных действий*}
	        \scntext{пояснение}{Связки отношения \textit{декомпозиция действия*} связывают \textit{действие}, и множество частных \textit{действий}, на которые декомпозируется данное \textit{действие}. При этом первым компонентом связки является знак указанного множества, вторым компонентом  знак более общего \textit{действия}.\\
	        Таким образом, \textit{декомпозиция действия*} это \textit{квазибинарное отношение}, связывающее действие со множеством действий более низкого уровня, к выполнению которых сводится выполнение исходного декомпозируемого действия.\\
	        Стоит отметить, что каждое \textit{действие} может иметь несколько вариантов декомпозиции в зависимости от конкретного набора элементарных действий, которые способна выполнять та или иная система \textit{субъектов}.\\
	        Принцип, по которому осуществляется такая декомпозиция в различных подходах к решению задач будем называть \textit{стратегией решения задач}.}
	    \end{scnindent}
        \scnhaselement{\scnkeyword{исходная ситуация}*}
        \begin{scnindent}
	        \scnidtf{исходная ситуация, соответствующая заданному действию*}
	        \scnidtf{начальная ситуация действия*}
	        \scnidtf{описание того, что дано (что имеется) перед началом выполнения заданного (специфицируемого) действия*}
	        \scntext{примечание}{\textit{исходная ситуация*} действия содержит описание тех исходных условий, которых оказалось достаточно для инициирования данного действия, в то время как \textit{начальная ситуация*} является более общим понятием и в общем случае не предполагает такого описания.}
	        \scnsubset{спецификация*}
	        \scnrelfrom{второй домен}{ситуация}
	    \end{scnindent}
        \scnhaselement{\scnkeyword{конечная ситуация}*}
        \begin{scnindent}
	        \scnidtf{результирующая ситуация, соответствующая заданному действию*}
	        \scnidtf{конечная ситуация, соответствующая заданному действию*}
	        \scnsubset{спецификация*}
	        \scnrelfrom{второй домен}{ситуация}
	    \end{scnindent}
	    \scnhaselement{\scnkeyword{цель}*}
	    \begin{scnindent}
	        \scnidtf{целевая ситуация*}
	        \scnsubset{спецификация*}
	        \scnrelfrom{второй домен}{ситуация}
	        \scnidtf{описание того, что требуется получить (какая ситуация должна быть достигнута) в результате выполнения заданного (специфицируемого) действия*}
	        \scnidtf{цель выполнения действия*}
	        \scnidtf{интенция, стремление, намерение, замысел, желание, устремление, направленность действия*}
	    \end{scnindent}
        
        \scnheader{следует отличать*}
        \begin{scnhaselementset}
            \scnitem{конечная ситуация*}
            \scnitem{цель*}
        \end{scnhaselementset}
    	\begin{scnindent}
        	\scntext{примечание}{Далеко не всегда конечная ситуация, сформированная в результате выполнения некоторого действия, совпадает с ситуацией, которая изначально (еще до выполнения действия или в процессе его выполнения) рассматривалась как целевая, желаемая ситуация.}
    	\end{scnindent}
        \bigskip
        \begin{scnhaselementset}
            \scnitem{действие}
            \begin{scnindent}
                \scnidtf{процесс решения задачи}
            \end{scnindent}
            \scnitem{результат действия*}
            \begin{scnindent}
                \scnidtf{результат выполнения действия*}
                \scnidtf{результат решения задачи*}
                \scnidtf{достигнутая цель*}
            \end{scnindent}
        \end{scnhaselementset}
        \bigskip
        \begin{scnhaselementset}
            \scnitem{процесс доказательства}
            \begin{scnindent}
                \scnidtf{процесс доказательства утверждения}
            \end{scnindent}
            \scnitem{результат доказательства*}
            \begin{scnindent}
                \scnidtf{текст доказательства*}
                \scnidtf{текст, обосновывающий истинность доказываемого утверждения*}
            \end{scnindent}
        \end{scnhaselementset}
        \bigskip
    \end{scnsubstruct}
    \scnendsegmentcomment{Уточнение семейства параметров и отношений, заданных на множестве воздействий, действий и задач. Типология спецификаций воздействий, действий и задач}
\end{SCn}

        \scnsegmentheader{Предметная область и онтология субъектно-объектных спецификаций воздействий}
\begin{scnsubstruct}
    \begin{scnrelfromlist}{соавтор}
        \scnitem{Гордей А.Н.}
        \scnitem{Никифоров С.А.}
        \scnitem{Бобёр Е.С.}
        \scnitem{Святощик М.И.}
    \end{scnrelfromlist}
    \scniselement{предметная область и онтология}
    
    \scnheader{индивид}
    \scnidtftext{часто используемый sc-идентификатор}{субъект}
    \begin{scnindent}
    	\scnrelfrom{источник}{\cite{Hardzei2005}}
    \end{scnindent}
    
    \scnheader{участник воздействия\scnrolesign}
    \scnidtf{участник акции\scnrolesign}
    \scniselement{ролевое отношение}
    \scnrelfrom{первый домен}{индивид}
    \scnrelfrom{второй домен}{воздействие}
    \scntext{пояснение}{\textit{участник акции\scnrolesign} --- это ролевое отношение, которое связывает акцию с участвующим в ней индивидом.}
    \begin{scnindent}
	    \scnrelfrom{источник}{\cite{Hardzei2021}}
	    \scnrelfrom{источник}{\cite{Fillmore1977}}
	    \scnrelfrom{источник}{\cite{Fillmore1982}}
	\end{scnindent}
    \begin{scnsubdividing}
        \scnitem{субъект\scnrolesign}
        \begin{scnindent}
            \scntext{пояснение}{\textit{субъект\scnrolesign} --- инициатор акции.}
	        \begin{scnsubdividing}
	             \scnitem{инициатор\scnrolesign}
	             \scnitem{вдохновитель\scnrolesign}
	             \scnitem{распространитель\scnrolesign}
	             \scnitem{вершитель\scnrolesign}
	             \begin{scnindent}
	                 \scntext{пояснение}{\textit{вершитель\scnrolesign} завершает акцию производством из объекта продукта.}
	             \end{scnindent}
	        \end{scnsubdividing}
        \end{scnindent}
        \scnitem{инструмент\scnrolesign}
        \begin{scnindent}
            \scntext{пояснение}{\textit{инструмент\scnrolesign} --- исполнитель акции.}
            \begin{scnsubdividing}
                \scnitem{активатор\scnrolesign}
                \begin{scnindent}
                    \scntext{пояснение}{\textit{активатор\scnrolesign} непосредственно воздействует на медиатор.}
               	\end{scnindent}
                \scnitem{супрессор\scnrolesign}
                \begin{scnindent}
                    \scntext{пояснение}{\textit{супрессор\scnrolesign} подавляет сопротивление медиатора.}
                \end{scnindent}
                \scnitem{усилитель\scnrolesign}
               	\begin{scnindent}
                    \scntext{пояснение}{\textit{усилитель\scnrolesign} наращивает воздействие на медиатор.}
               	\end{scnindent}
                \scnitem{преобразователь\scnrolesign}
              	\begin{scnindent}
                    \scntext{пояснение}{\textit{преобразователь\scnrolesign} преобразует медиатор в инструмент.}
               	\end{scnindent}
            \end{scnsubdividing}
        \end{scnindent}
        \scnitem{медиатор\scnrolesign}
        \begin{scnindent}
            \scntext{пояснение}{\textit{медиатор\scnrolesign} --- посредник акции.}
            \begin{scnsubdividing}
                \scnitem{ориентир\scnrolesign}
                \scnitem{локус\scnrolesign}
                \begin{scnindent}
                    \scntext{пояснение}{\textit{локус\scnrolesign} частично или полностью окружает объект и тем самым локализует его в пространстве.}
                \end{scnindent}
                \scnitem{транспортёр\scnrolesign}
                \begin{scnindent}
                    \scntext{пояснение}{\textit{транспортёр\scnrolesign} перемещает объект.}
                \end{scnindent}
                \scnitem{адаптер\scnrolesign}
                \begin{scnindent}
                    \scntext{пояснение}{\textit{адаптер\scnrolesign} приспосабливает инструмент к воздействию на объект.}
                \end{scnindent}
                \scnitem{материал\scnrolesign}
                \begin{scnindent}
                    \scntext{пояснение}{\textit{материал\scnrolesign} используется в качестве объекта-сырья для производства продукта.}
                \end{scnindent}
                \scnitem{макет\scnrolesign}
                \begin{scnindent}
                    \scntext{пояснение}{\textit{макет\scnrolesign} является исходным образцом для производства из объекта продукта.}
                \end{scnindent}
                \scnitem{фиксатор\scnrolesign}
                \begin{scnindent}
                    \scntext{пояснение}{\textit{фиксатор\scnrolesign} превращает переменный локус объекта в постоянный.}
                \end{scnindent}
                \scnitem{ресурс\scnrolesign}
                \begin{scnindent}
                    \scntext{пояснение}{\textit{ресурс\scnrolesign} питает инструмент.}
                \end{scnindent}
                \scnitem{стимул\scnrolesign}
                \begin{scnindent}
                    \scntext{пояснение}{\textit{стимул\scnrolesign} проявляет параметр объекта.}
                \end{scnindent}
                \scnitem{регулятор\scnrolesign}
                \begin{scnindent}
                    \scntext{пояснение}{\textit{регулятор\scnrolesign} служит инструкцией в производстве из объекта продукта.}
                \end{scnindent}
                \scnitem{хронотоп\scnrolesign}
                \begin{scnindent}
                    \scntext{пояснение}{\textit{хронотоп\scnrolesign} локализует объект во времени.}
                \end{scnindent}
                \scnitem{источник\scnrolesign}
                \begin{scnindent}
                    \scntext{пояснение}{\textit{источник\scnrolesign} обеспечивает инструкциями инструмент.}
                \end{scnindent}
                \scnitem{индикатор\scnrolesign}
                \begin{scnindent}
                    \scntext{пояснение}{\textit{индикатор\scnrolesign} отображает параметр воздействия на объект или параметр продукта как результата воздействия на объект.}
             	\end{scnindent}
            \end{scnsubdividing}
        \end{scnindent}
        \scnitem{объект\scnrolesign}
        \begin{scnindent}
            \scntext{пояснение}{\textit{объект\scnrolesign} --- реципиент акции.}
            \begin{scnsubdividing}
                \scnitem{покрытие\scnrolesign}
                \begin{scnindent}
                    \scntext{пояснение}{\textit{покрытие\scnrolesign} --- внешняя изоляция оболочки индивида.}
                \end{scnindent}
                \scnitem{корпус\scnrolesign}
                \begin{scnindent}
                    \scntext{пояснение}{\textit{корпус\scnrolesign} --- оболочка индивида.}
                \end{scnindent}
                \scnitem{прослойка\scnrolesign}
                \begin{scnindent}
                    \scntext{пояснение}{\textit{прослойка\scnrolesign} --- внутренняя изоляция оболочки индивида.}
                \end{scnindent}
                \scnitem{сердцевина\scnrolesign}
                \begin{scnindent}
                    \scntext{пояснение}{\textit{сердцевина\scnrolesign} --- ядро индивида.}
               \end{scnindent} 
            \end{scnsubdividing}
        \end{scnindent}
        \scnitem{продукт\scnrolesign}
        \begin{scnindent}
            \scntext{пояснение}{\textit{продукт\scnrolesign} --- результат воздействия субъекта на объект.}
            \begin{scnsubdividing}
                \scnitem{заготовка\scnrolesign}
                \begin{scnindent}
                    \scntext{пояснение}{\textit{заготовка\scnrolesign} --- превращённый в сырьё объект.}
                \end{scnindent}
                \scnitem{полуфабрикат\scnrolesign}
                \begin{scnindent}
                    \scntext{пояснение}{\textit{полуфабрикат\scnrolesign} --- наполовину изготовленный из сырья продукт.}
                \end{scnindent}
                \scnitem{прототип\scnrolesign}
                \begin{scnindent}
                    \scntext{пояснение}{\textit{прототип\scnrolesign} --- опытный образец продукта.}
                \end{scnindent}
                \scnitem{изделие\scnrolesign}
                \begin{scnindent}
                    \scntext{пояснение}{\textit{изделие\scnrolesign} --- готовый продукт.}
                \end{scnindent}
            \end{scnsubdividing}
        \end{scnindent}
    \end{scnsubdividing}
    
    \scnheader{воздействие}
    \scnidtf{акция}
    \scnrelfrom{источник}{\cite{Hardzei2017}}
    \scnrelfrom{разбиение}{\scnkeyword{Типология по характеру взаимодействия участников\scnsupergroupsign}}
    \begin{scnindent}
	    \begin{scneqtoset}
	        \scnitem{воздействие активизации}
	        \begin{scnindent}
	            \scntext{пояснение}{\textit{воздействие активизации} --- воздействие, в ходе которого взаимодействие осуществляется между субъектом и инструментом.}
	        \end{scnindent}
	        \scnitem{воздействие эксплуатации}
	        \begin{scnindent}
	            \scntext{пояснение}{\textit{воздействие эксплуатации} --- воздействие, в ходе которого взаимодействие осуществляется между инструментом и медиатором.}
	        \end{scnindent}
	        \scnitem{воздействие трансформации}
	        \begin{scnindent}
	            \scntext{пояснение}{\textit{воздействие трансформации} --- воздействие, в ходе которого взаимодействие осуществляется между объектом и продуктом.}
	        \end{scnindent}
	        \scnitem{воздействие нормализации}
	        \begin{scnindent}
	            \scntext{пояснение}{\textit{воздействие нормализации} --- воздействие, в ходе которого взаимодействие осуществляется между объектом и продуктом.}
	        \end{scnindent}
	    \end{scneqtoset}
    \end{scnindent}
    \scnrelfrom{разбиение}{\scnkeyword{Типология воздействий по виду взаимодействующих подсистем\scnsupergroupsign}}
    \begin{scnindent}
	    \begin{scneqtoset}
	        \scnitem{воздействие среда-оболочка}
	        \scnitem{воздействие оболочка-ядро}
	        \scnitem{воздействие ядро-оболочка}
	        \scnitem{воздействие оболочка-среда}
	    \end{scneqtoset}
	\end{scnindent}
	
    \scnheader{воздействие}
    \scnrelfrom{разбиение}{\scnkeyword{Типология воздействий по фазам наращивания воздействия\scnsupergroupsign}}
	\begin{scnindent}
	    \begin{scneqtoset}
	        \scnitem{воздействие инициации}
	        \begin{scnindent}
	            \scntext{пояснение}{\textit{воздействие инициации} --- воздействие, в ходе которого оно начинается в каждой подсистеме.}
	        \end{scnindent} 
	        \scnitem{воздействие аккумуляции}
	        \begin{scnindent}
	            \scntext{пояснение}{\textit{воздействие аккумуляции} --- воздействие, в ходе которого происходит его накапливание в каждой подсистеме.}
	        \end{scnindent}
	        \scnitem{воздействие амплификации}
	        \begin{scnindent}
	            \scntext{пояснение}{\textit{воздействие амплификации} --- воздействие, в ходе которого происходит его усиление.}
	        \end{scnindent}
	        \scnitem{воздействие генерации}
	        \begin{scnindent}
	            \scntext{пояснение}{\textit{воздействие генерации} --- воздействие, которое представляет собой переход в каждой подсистеме с одного уровня, например, среды-оболочки, на другой, например, оболочки-ядра.}
	        \end{scnindent}
	    \end{scneqtoset}
	\end{scnindent}
	
    \scnheader{воздействие}
    \scnrelfrom{разбиение}{\scnkeyword{Типология воздействий по виду инструмента\scnsupergroupsign}}
	\begin{scnindent}
	    \begin{scneqtoset}
	        \scnitem{физическое воздействие}
	        \begin{scnindent}
	            \scntext{пояснение}{\textit{физическое воздействие} --- воздействие, в котором в роли инструмента выступает оболочка субъекта.}
	        \end{scnindent}
	        \scnitem{информационное воздействие}
	        \begin{scnindent} 
	            \scntext{пояснение}{\textit{информационное воздействие} --- воздействие, в котором в роли инструмента выступает среда субъекта.}
	       \end{scnindent}
	    \end{scneqtoset}
	\end{scnindent}

    \scnheader{Рис. Таблица воздействий}
    \scneqfile{\includegraphics{Contents/part_kb/images/sd_actions/macroproc_table.png}}
    \scnrelfrom{источник}{\cite{Hardzei2017}}
    \scntext{пояснение}{На изображении представлена типология \textit{воздействий}. Любое воздействие характеризуется принадлежностью четырём классам, соответствующим признакам классификации. Заштрихованы \textit{физические воздействия}.}
   
    \scnheader{Специфицируемые классы воздействий}
    \begin{scnhassubset}
            \scnitem{формование}
            \begin{scnindent}
	            \begin{scnreltoset}{пересечение}
	                \scnitem{воздействие трансформации}
	                \scnitem{воздействие среда-оболочка}
	                \scnitem{воздействие генерации}
	                \scnitem{физическое воздействие}
	            \end{scnreltoset}
        	\end{scnindent}
            \scnitem{притягивание}
            \begin{scnindent}
	            \begin{scnreltoset}{пересечение}
	                \scnitem{воздействие активизации}
	                \scnitem{воздействие среда-оболочка}
	                \scnitem{воздействие инициации}
	                \scnitem{физическое воздействие}
	            \end{scnreltoset}
        	\end{scnindent}
            \scnitem{выхолащивание}
            \begin{scnindent}
	            \begin{scnreltoset}{пересечение}
	                \scnitem{воздействие трансформации}
	                \scnitem{воздействие ядро-оболочка}
	                \scnitem{воздействие генерации}
	                \scnitem{физическое воздействие}
	            \end{scnreltoset}
        	\end{scnindent}
            \scnitem{аннигилирование}
            \begin{scnindent}
	            \begin{scnreltoset}{пересечение}
	                \scnitem{воздействие трансформации}
	                \scnitem{воздействие оболочка-среда}
	                \scnitem{воздействие генерации}
	                \scnitem{физическое воздействие}
	            \end{scnreltoset}
        	\end{scnindent}
            \scnitem{введение}
            \begin{scnindent}
	            \begin{scnreltoset}{пересечение}
	                \scnitem{воздействие эксплуатации}
	                \scnitem{воздействие оболочка-ядро}
	                \scnitem{воздействие инициации}
	                \scnitem{физическое воздействие}
	            \end{scnreltoset}
        	\end{scnindent}
            \scnitem{распускание}
            \begin{scnindent}
	            \begin{scnreltoset}{пересечение}
	                \scnitem{воздействие трансформации}
	                \scnitem{воздействие оболочка-среда}
	                \scnitem{воздействие амплификации}
	                \scnitem{физическое воздействие}
	            \end{scnreltoset}
        	\end{scnindent}
            \scnitem{разжимание}
            \begin{scnindent}
	            \begin{scnreltoset}{пересечение}
	                \scnitem{воздействие трансформации}
	                \scnitem{воздействие ядро-оболочка}
	                \scnitem{воздействие амплификации}
	                \scnitem{физическое воздействие}
	            \end{scnreltoset}
        	\end{scnindent}
            \scnitem{разъединение}
            \begin{scnindent}
	            \begin{scnreltoset}{пересечение}
	                \scnitem{воздействие эксплуатации}
	                \scnitem{воздействие ядро-оболочка}
	                \scnitem{воздействие генерации}
	                \scnitem{физическое воздействие}
	            \end{scnreltoset}
        	\end{scnindent}
    \end{scnhassubset}
    
    \scnheader{Пример sc.g-текста, описывающего спецификацию воздействия}
    \scneq{\scnfileimage[40em]{Contents/part_kb/images/sd_actions/tapaz_description_example.png}}
    \scniselement{sc.g-текст}
    \scntext{пояснение}{Представленный фрагмент базы знаний содержит декомпозицию воздействия во времени, указание принадлежности данного декомпозируемого воздействия и полученных в результате данной декомпозиции воздействий определенному их классу из приведенной выше классификации, а также указание участников данных акций.}
    \scntext{пояснение}{Представленный фрагмент базы знаний можно протранслировать в следующий текст естественного языка: <<Некто принимает молоко, затем окисляет молоко, а именно: нормализует молоко до 15-процентной жирности, затем очищает молоко, затем пастеризует молоко, затем охлаждает молоко до определённой температуры, затем вносит закваску в молоко, затем сквашивает молоко, затем режет сгусток, затем подогревает сгусток, затем обрабатывает сгусток, затем отделяет сыворотку, затем охлаждает сгусток и, в итоге, производит творог>>.}
    
    \scnheader{субъект}
    \scnidtftext{часто используемый sc-идентификатор}{индивид}
    \scnidtf{активная сущность}
    \scnidtf{сущность, способная самостоятельно выполнять некоторые виды действий}
    \scnidtf{агент деятельности}
    \scnsuperset{Собственное Я}
    \scnsuperset{внутренний субъект ostis-системы}
    \scnsuperset{внешний субъект ostis-системы, с которым осуществляется взаимодействие}
    \scnsuperset{внешний субъект ostis-системы, с которым взаимодействие не происходит}
    \scnheader{внутренний субъект ostis-системы}
    \scnidtf{субъект, входящий в состав той \textit{ostis-системы, в базе знаний} которой он описывается}
    \scnsuperset{sc-агент}
    \scntext{пояснение}{Под \textit{внутренним субъектом ostis-системы} понимается такой \textit{субъект}, который выполняет некоторые \textit{действия} в \uline{той же памяти}, в которой хранится его знак.\\
        К числу \textit{внутренних субъектов ostis-системы} относятся входящие в нее \textit{sc-агенты}, частные sc-машины, целые интеллектуальные подсистемы.}
    
    \scnheader{внешний субъект ostis-системы, с которым осуществляется взаимодействие}
    \scntext{пояснение}{К числу \textit{внешних субъектов ostis-системы, с которыми осуществляется взаимодействие}, относятся конечные пользователи \textit{ostis-системы}, ее разработчики, а также другие компьютерные системы(причем не только интеллектуальные).}
    
    \scnheader{субъект действия\scnrolesign}
    \scnsubset{субъект\scnrolesign}
    \scnidtf{сущность, воздействующая на некоторую другую сущность в процессе заданного действия\scnrolesign}
    \scnidtf{сущность, создающая \textit{причину} изменений другой сущности (объекта действия)\scnrolesign}
    \scnidtf{быть субъектом данного действия\scnrolesign}
    \scnsuperset{субъект неосознанного воздействия\scnrolesign}
    \scnsuperset{субъект осознанного воздействия\scnrolesign}
	\begin{scnindent}
	    \scnidtf{субъект целенаправленного, активного воздействия\scnrolesign}
	\end{scnindent}
    
    \scnheader{исполнитель*}
    \scntext{пояснение}{Связки отношения \textit{исполнитель*} связывают \textit{sc-элементы}, обозначающие \textit{действие} и \textit{sc-элементы}, обозначающие \textit{субъекта}, который предположительно будет осуществлять, осуществляет или осуществлял выполнение указанного \textit{действия}. Данное отношение может быть использовано при назначении конкретного исполнителя для проектной задачи по развитию баз знаний.\\
        В случае, когда заранее неизвестно, какой именно \textit{субъект*} будет исполнителем данного \textit{действия}, отношение \textit{исполнитель*} может отсутствовать в первоначальной формулировке \textit{задачи} и добавляться позже, уже непосредственно при исполнении.\\
        Когда действие выполняется (является \textit{настоящей сущностью}) или уже выполнено (является \textit{прошлой сущностью}), то исполнитель этого действия в каждый момент времени уже определён. Но когда действие только инициировано, тогда важно знать:
        \begin{enumerate}
            \item кто \uline{хочет} выполнить это действие и насколько важно для него стать исполнителем данного действия;
            \item кто \uline{может} выполнить данное действие и каков уровень его квалификации и опыта;
            \item кто и кому поручает выполнить это действие и каков уровень ответственности за невыполнение (приказ, заказ, официальный договор, просьба...)
        \end{enumerate}
        При этом следует помнить, что связь отношения \textit{исполнитель*} в данном случае также является временной прогнозируемой сущностью.\\
        Первым компонентом связок отношений \textit{исполнитель*} является знак \textit{действия}, вторым --- знак \textit{субъекта} исполнителя}
        
    \scnheader{объект воздействия\scnrolesign}
    \scnsubset{объект\scnrolesign}
    \scnidtf{сущность, на которую осуществляется воздействие в рамках заданного действия\scnrolesign}
    \scnidtf{сущность, являющаяся в рамках заданного действия исходным условием (аргументом), необходимым для выполнения этого действия\scnrolesign}
    \scntext{примечание}{Для разных действий количество объектов действий может быть различным.}
    \scntext{примечание}{Поскольку действие является процессом и, соответственно, представляет собой \textit{динамическую структуру}, то и знак \textit{субъекта действия\scnrolesign}, и знак \textit{объекта действия\scnrolesign} являются элементами данной структуры. В связи с этим можно рассматривать отношения \textit{субъект действия\scnrolesign} и \textit{объект действия\scnrolesign} как \textit{ролевые отношения}. Данный факт не  запрещает вводить аналогичные \textit{неролевые отношения}, однако это нецелесообразно.}
    
    \scnheader{продукт\scnrolesign}
    \scnidtf{быть продуктом заданного действия\scnrolesign}
    \scnsubset{продукт*}
    \scnsubset{результат*}
    \scnidtf{сухой остаток\scnrolesign}
    \scnidtf{то, ради чего может быть выполнено, выполняется или будет выполняться заданное действие\scnrolesign}
    \scntext{примечание}{Продуктом действия может быть некоторая материальная сущность, некоторое множество (тираж) одинаковых материальных сущностей, некоторая информационная конструкция}
    
    \scnheader{результат*}
    \scntext{пояснение}{Связки отношения \textit{результат*} связывают \textit{sc-элемент}, обозначающий \textit{действие}, и \textit{sc-конструкцию}, описывающую результат выполнения рассматриваемого действия, другими словами, цель, которая должна быть достигнута при выполнении \textit{действия}.\\
    Результат может специфицироваться как атомарным высказыванием, так и неатомарным, т.е. конъюнктивным, дизъюнктивным, строго дизъюнктивным и т.д.\\
    В случае, когда успешное выполнение \textit{действия} приводит к изменению какой-либо конструкции в \textit{\mbox{sc-памяти}}, которое необходимо занести в историю изменений базы знаний или использовать для демонстрации протокола решении задачи, генерируется соответствующая связка отношения \textit{результат*}, связывающая задачу и \textit{sc-конструкцию}, описывающую данное изменение. Конкретный вид указанной \textit{\mbox{sc-конструкции}} зависит от типа действия.}
    \scnrelboth{следует отличать}{цель*}
	\begin{scnindent}
	    \scnidtf{спецификация планируемого результата*}
	    \scntext{примечание}{Следует также отмечать то, что является непосредственно результатом (продуктом) некоторого действия, и то, что является предварительной (исходной, стартовой) спецификацией этого результата. Далеко не всегда результатом действия является именно то, что планировалось (было целью) изначально.}
	\end{scnindent}
    
    \scnheader{класс выполняемых действий*}
    \scnidtf{класс действий, выполняемых классом субъектов*}
    \scntext{пояснение}{Связки отношения \textit{класс выполняемых действий*} связывают классы \textit{субъектов} и классы действий, при этом предполагается, что каждый субъект указанного класса способен выполнять действия указанного класса действий.}
    
    \scnheader{заказчик*}
    \scntext{пояснение}{Связки отношения \textit{заказчик*} связывают классы \textit{sc-элементы}, обозначающие \textit{действие}, и \textit{sc-элементы}, обозначающие \textit{субъекта}, который заинтересован в выполнении данного действия и, как правило, инициирует его выполнение. Данное отношение может быть использовано при указании того, кто поставил проектную задачу по развитию баз знаний.\\
        Первым компонентов связок отношения \textit{заказчик*} является знак \textit{действия}, вторым --- знак \textit{субъекта}.}
    
    \scnheader{инициатор*}
    \scntext{пояснение}{Связки отношения \textit{инициатор*} связывают \textit{sc-элемент}, обозначающий \textit{инициированное действие}, и знак \textit{субъекта}, который является инициатором данного \textit{действия}, то есть \textit{субъектом}, который инициировал данное \textit{действие} и, как правило, заинтересован в его успешном выполнении.}
    
    \scnheader{объект\scnrolesign}
    \scnidtf{аргумент действия\scnrolesign}
    \scntext{пояснение}{Связки отношения \textit{объект\scnrolesign} связывают \textit{sc-элемент}, обозначающий \textit{действие}, и знак той сущности, над которой (по отношению к которой) осуществляется данное \textit{действие}, и, например, знак \textit{структуры}, подлежащий верификации.}
    \scnsuperset{первый аргумент действия\scnrolesign}
    \scnsuperset{второй аргумент действия\scnrolesign}
    \scnsuperset{третий аргумент действия\scnrolesign}
    
    \scnheader{класс аргументов*}
    \scnidtf{класс аргументов класса команд*}
    \scnidtf{быть классом sc-элементов, экземпляры которого являются аргументами для заданного класса команд*}
    \scnsuperset{класс первых аргументов*}
    \scnsuperset{класс вторых аргументов*}
    \scntext{пояснение}{Связки отношения \textit{класс аргументов*} связывают \textit{классы команд} (подмножества множества \textit{команд}) и классы \textit{sc-элементов}, которые могут быть аргументами действий, соответствующих данному \textit{классу команд}. В случае, когда \textit{команды} данного класса имеют один аргумент, используется собственно отношение  \textit{класс аргументов*}, в случае, когда команды данного класса имеют более одного аргумента, то используются подмножества данного отношения, такие как \textit{класс первых аргументов*}, \textit{класс вторых аргументов*} и т.д.\\
        Если для некоторого \textit{класса команд} не указан тип какого-либо из аргументов, то предполагается, что в качестве данного аргумента может выступать любой \textit{sc-элемент}.\\
        Первым компонентом связок отношения \textit{класс аргументов*} является знак \textit{класса команд}, вторым --- знак класса \textit{sc-элемента}, которые могут быть \textit{аргументами действий\scnrolesign}, соответствующих данному \textit{классу команд}.}
    
        \bigskip
\end{scnsubstruct}
\scnendsegmentcomment{Предметная область и онтология субъектно-объектных спецификаций воздействий}

        \begin{SCn}
    \scnsegmentheader{Уточнение понятий план сложного действия, классы действий, класса задач, метода}
    \begin{scnsubstruct}
        \scniselement{сегмент базы знаний}
        
        \scnheader{план сложного действия}
        \scnidtf{план}
        \scnidtf{план выполнения сложного действия}
        \scnidtf{план решения \textit{сложной задачи}}
        \scnidtf{план выполнения действия}
        \scnidtf{спецификация выполнения действия}
        \scnidtf{декомпозиция выполняемого действия на систему последовательно/параллельно выполняемых поддействий*}
        \scnidtf{описание того, как может быть выполнено соответствующее сложное действие}
        \scnidtf{спецификация соответствующего действия, уточняющая то, \uline{как} предполагается выполнять это действие}
        \scnidtf{план решения задачи (выполнения сложного действия) путем описания последовательности выполнения поддействий с описанием того, как передается управление от одних поддействий другим, как осуществляется распараллеливание, как организуется выполнение циклов}
        \scntext{определение}{вид спецификации \textit{сложного действия}, представляющий собой систему \textit{задач}, \textit{интерпретация} которой (предполагающая решение указанных \textit{задач} в определенной последовательности) обеспечивает выполнение специфицируемого \textit{сложного действия}}
        \scnsubset{знание}
        \scntext{пояснение}{Каждый \textit{план} представляет собой \textit{семантическую окрестность, ключевым sc-элементом\scnrolesign} является \textit{действие}, для которого дополнительно детализируется предполагаемый процесс его выполнения. Основная задача такой детализации --- локализация области базы знаний, в которой предполагается работать, а также набора агентов, необходимого для выполнения  описываемого действия. При этом детализация не обязательно должна быть доведена до уровня элементарных действий, цель составления плана --- уточнение подхода к решению той или иной задачи, не всегда предполагающее составления подробного пошагового решения.\\
            При описании \textit{плана} может быть использован как процедурный, так и декларативный подход. В случае процедурного подхода для соответствующего \textit{действия} указывает его декомпозиция на более частные поддействия, а также необходимая спецификация этих поддействий. В случае декларативного подхода указывается набор подцелей (например, при помощи логических утверждений), достижение которых необходимо для выполнения рассматриваемого \textit{действия}. На практике оба рассмотренных подхода можно комбинировать.\\
            В общем случае \textit{план} может содержать и переменные, например в случае, когда часть плана задается в виде цикла (многократного повторения некоторого набора действий). Также план может содержать константы, значение которых в настоящий момент не установлено и станет известно, например, только после выполнения предшествующих ему \textit{действий}.\\
            Каждый \textit{план} может быть задан заранее как часть спецификации \textit{действия}, т.е. \textit{задачи}, а может формироваться \textit{субъектов} уже собственно в процессе выполнения \textit{действия}, например, в случае использования стратегии разбиения задачи над подзадачи. В первом случае \textit{план} \textit{включается*} в \textit{задачу}, соответствующую тому же действию.}
        \begin{scnsubdividing}
            \scnitem{процедурный план сложного действия}
        	\begin{scnindent}
                \scnidtf{декомпозиция \textit{сложного действия} на множество последовательно и/или параллельно выполняемых \textit{поддействий}}
            \end{scnindent}
            \scnitem{непроцедурный план сложного действия}
        	\begin{scnindent}
                \scnidtf{декомпозиция исходной \textit{задачи}, соответствующей заданному \textit{сложному действию}, на иерархическую систему и/или подзадач}
            \end{scnindent}
        \end{scnsubdividing}
        
        \scnheader{процедурный план сложного действия}
        \scntext{примечание}{В \textit{процедурном плане выполнения сложного действия} соответствующие \textit{поддействия*} декомпозируемого \textit{сложного действия} представляются специфицирующими их \textit{задачами}. Но, кроме такого рода \textit{задач}, в \textit{процедурный план выполнения сложного действия} входят также \textit{задачи}, которые специфицируют \textit{действия}, обеспечивающие:
            \begin{scnitemize}
                \item синхронизацию выполнения \textit{поддействий*} заданного \textit{сложного действия};
                \item передачу управления указанным \textit{поддействиям*} (а точнее, соответствующим им \textit{задачам}), т.е. инициирование указанных \textit{поддействий*} (и соответствующих им \textit{задач}).
            \end{scnitemize}}
        
        \scnheader{действие управления интерпретацией процедурного плана сложного действия}
        \scnrelboth{семантически близкий знак}{задача управления интерпретацией процедурного плана сложного действия}
        \scnsuperset{безусловная передача управления от одного поддействия к другому}
        \scnsuperset{инициирование заданного поддействия при возникновении в базе знаний ситуации или события заданного вида}
        \scnsuperset{инициирование заданного множества поддействий при успешном завершении выполнения \uline{всех} поддействий другого заданного множества}
        \scnsuperset{инициирование заданного множества поддействий при успешном завершении выполнения \uline{по крайней мере одного} поддействия другого заданного множества}%примечания не было
        \scntext{примечание}{Выделенные классы \textit{действий управления интерпретацией процедурного плана сложного действия} дают возможность реализовать различные виды параллелизма, если это позволяет задача.}
        
        \scnheader{класс действий}
        \scnrelto{семейство подклассов}{действие}
        \scnidtftext{пояснение}{\uline{максимальное} множество аналогичных (похожих в определенном смысле) действий, для которого существует (но не обязательно известных в текущий момент) по крайней мере один \textit{метод} (или средство), обеспечивающий выполнение \uline{любого} действия из указанного множества действий}
        \scnidtf{множество однотипных действий}
        \scnsuperset{класс элементарных действий}
        \scnsuperset{класс легковыполнимых сложных действий}
        \scntext{примечание}{Тот факт, что каждому выделяемому \textit{классу действий} соответствует по крайней мере один общий для них \textit{метод} выполнения этих \textit{действий}, означает то, что речь идет о \uline{семантической} кластеризации множества \textit{действий}, т.е. выделении \textit{классов действий} по признаку \uline{семантической близости} (сходства) \textit{действий}, входящих в состав выделяемого \textit{класса действий}. При этом прежде всего учитывается аналогичность (сходство) \textit{исходных ситуаций и целевых ситуаций} рассматриваемых \textit{действий}, т.е. аналогичность \textit{задач}, решаемых в результате выполнения соответствующих \textit{действий}. Поскольку одна и та же \textit{задача} может быть решена в результате выполнения нескольких \uline{разных} \textit{действий}, принадлежащих \uline{разным} \textit{классам действий}, следует говорить не только о \textit{классах действий} (множествах аналогичных действий), но и о \textit{классах задач} (о множествах аналогичных задач), решаемых этими \textit{действиями}. Так, например, на множестве \textit{классом действий} заданы следующие \textit{отношения}:
            \begin{scnitemize}
                \item \textit{отношение}, каждая связка которого связывает два разных (непересекающихся) \textit{класса действий}, осуществляющих решение одного и того же \textit{класса задач};
                \item \textit{отношение}, каждая связка которого связывает два разных \textit{класса действий}, осуществляющих решение разных \textit{классов задач}, один из которых является \textit{надмножеством} другого.
            \end{scnitemize}}
        \scntext{правило идентификации экземпляров}{Конкретные \textit{классы действий} в рамках \textit{Русского языка} именуются по следующим правилам:
            \begin{scnitemize}
                \item в начале идентификатора пишется слово \scnqqi{\textit{действие}} и ставится точка;
                \item далее со строчной буквы идет либо содержащее глагол совершенного вида в инфинитиве описание сути того, что требуется получить в результате выполнения действий данного класса, либо вопросительное предложение, являющееся спецификацией запрашиваемой (ответной) информации.
            \end{scnitemize}
            Например:\\
            \textit{действие, сформировать полную семантическую окрестность указываемой сущности\\
                действие, верифицировать заданную структуру}Допускается использовать менее строгие идентификаторы, которые, однако, обязаны оперировать словом \scnqqi{\textit{действие}} и достаточно четко специфицировать суть действий описываемого класса.\\
                Например:\\
            \textit{действие редактирования базы знаний}\\
            \textit{действие, направленное на установление темпоральных характеристик указываемой сущности}}
            
        \scnheader{класс элементарных действий}
        \scnidtf{множество элементарных действий, указание принадлежности которому является \uline{необходимым} и достаточным условием для выполнения этого действия}
        \scntext{примечание}{Множество всевозможных элементарных действий, выполняемых каждым субъектом, должно быть \uline{разбито} на классы элементарных действий.}
        \scntext{пояснение}{Принадлежность некоторого \textit{класса действий} множеству \textit{классу элементарных действий}, фиксирует факт того, что при указании всех необходимых аргументов принадлежности \textit{действия} данному классу достаточно для того, чтобы некоторый субъект мог приступить к выполнению этого действия.\\
            При этом, даже если \textit{класс действий} принадлежит множеству \textit{класс элементарных действий}, не запрещается вводить более частные \textit{классы действий}, для которых, например, заранее фиксируется один из аргументов.\\
            Если конкретный \textit{класс элементарных действий} является более частным по отношению к \textit{действиям в sc-памяти}, то это говорит о наличии в текущей версии системы как минимум одного \textit{sc-агента}, ориентированного на выполнение действий данного класса.}
        
        \scnheader{класс легковыполнимых сложных действий}
        \scnidtf{множество сложных действий, для которого известен и доступен по крайней мере один \textit{метод}, интерпретация которого позволяет осуществить полную (окончательную, завершающуюся элементарными действиями) декомпозицию на поддействия \uline{каждого} сложного действия из указанного выше множества}
        \scnidtf{множество всех сложных действий, выполнимых с помощью известного \textit{метода}, соответствующего этому множеству}
        \scntext{пояснение}{Принадлежность некоторого \textit{класса действий} множеству \textit{класс легковыполнимых сложных действий} фиксирует факт того, что даже при указании всех необходимых аргументов принадлежности \textit{действия} данному классу недостаточно для того, чтобы некоторый \textit{субъект} приступил к выполнению этого действия, и требуются дополнительные уточнения.}
        
        \scnheader{сужение отношения по первому домену*(спецификация*;класс действий*)}
        \scnidtftext{часто используемый идентификатор}{спецификация класса действий*}
        \begin{scnsubdividing}
            \scnitem{обобщенная формулировка задач соответствующего класса*}
        	\begin{scnindent}
                \begin{scnsubdividing}
                    \scnitem{обобщенная декларативная формулировка задач соответствующего класса*}
                    \scnitem{обобщенная процедурная формулировка задач соответствующего класса*}
                \end{scnsubdividing}
            \end{scnindent}
            \scnitem{метод*}
        	\begin{scnindent}
                \scnidtf{метод решения задач заданного класса*}
                \scnidtf{метод выполнения действий соответствующего (заданного) класса*}
                \begin{scnindent}
	                \begin{scnsubdividing}
	                    \scnitem{процедурный метод выполнения действий соответствующего класса*}
	                	\begin{scnindent}
	                        \scnidtf{обобщенный план выполнения действий заданного класса*}
	                    \end{scnindent}
	                    \scnitem{декларативный метод выполнения действий соответствующего класса*}
	                	\begin{scnindent}
	                        \scnidtf{обобщенная декларативная спецификация выполнения действий заданного класса*}
	                    \end{scnindent}
	                \end{scnsubdividing}
                \end{scnindent}
            \end{scnindent}
        \end{scnsubdividing}
        
        \scnheader{класс задач}
        \scnidtf{множество аналогичных задач}
        \scnidtf{множество задач, для которого можно построить обобщенную формулировку задач, соответствующую всему этому множеству задач}
        \scntext{примечание}{Каждая \textit{обобщенная формулировка задач соответствующего класса} по сути есть не что иное, как строгое логическое определение указанного класса задач.}
        \scnrelto{семейство подмножеств}{задача}
        \scntext{правило идентификации экземпляров}{Конкретные \textit{классы задач} в рамках \textit{Русского языка} именуются по следующим правилам:
            \begin{scnitemize}
                \item в начале идентификатора пишется слово \scnqqi{\textit{задача}} и ставится точка;
                \item далее с прописной буквы идет либо содержащее глагол совершенного вида в инфинитиве описание сути того, что требуется получить в результате решения данного \textit{класса задач}, либо вопросительное предложение, являющееся спецификацией запрашиваемой (ответной) информации.
            \end{scnitemize}
            Например:\\
            \textit{задача. сформировать полную семантическую окрестность указываемой сущности}\\
            \textit{задача. верифицировать заданную структуру}Допускается использовать менее строгие идентификаторы, которые, однако, обязаны оперировать словом \scnqqi{\textit{задача}} и достаточно четко специфицировать суть задач описываемого класса.\\
            Например:\\
            \textit{задача на установление значения величины}\\
            \textit{задача на доказательство}}
            
        \scnheader{класс команд}
        \scnrelto{семейство подмножеств}{задача}
        \scnsuperset{класс интерфейсных пользовательских команд}
        \begin{scnindent}
        	\scnsuperset{класс интерфейсных команд пользователя ostis-системы}
        \end{scnindent}
        \scnsuperset{класс команд без аргументов}
        \scnsuperset{класс команд с одним аргументом}
        \scnsuperset{класс команд с двумя аргументами}
        \scnsuperset{класс команд с произвольным числом аргументов}
        \scntext{пояснение}{Идентификатор конкретного класса \textit{класса команд} в рамках \textit{Русского языка} пишется со строчной буквы и представляет собой либо содержащее глагол совершенного вида в инфинитиве описание сути того, что требуется получить в результате выполнения действий, соответствующих данному \textit{классу команд}, либо вопросительное предложение, являющееся спецификацией запрашиваемой (ответной) информации.\\ Например:\\
            \textit{сформировать полную семантическую окрестность указываемой сущности}\\
            \textit{верифицировать заданную структуру}Допускается использовать менее строгие идентификаторы, которые, однако, обязаны оперировать словом \scnqqi{\textit{команда}} и достаточно четко специфицировать суть задач описываемого класса.\\
            Например:\\
            \textit{команда редактирования базы знаний}\\
            \textit{команда установления темпоральных характеристик указываемой сущности}}
        \begin{scnsubdividing}
            \scnitem{атомарный класс команд}
            \scnitem{неатомарный класс команд}
        \end{scnsubdividing}
        
        \scnheader{атомарный класс команд}
        \scntext{пояснение}{Принадлежность некоторого \textit{класса команд} множеству \textit{атомарных классов команд} фиксирует факт того, что данная спецификация является достаточной для того, чтобы некоторый субъект приступил к выполнению соответствующего действия.\\
            При этом, даже если \textit{класса команд} принадлежит множеству \textit{атомарных классов команд} не запрещается вводить более частные \textit{классы команд}, в состав которых входит информация, дополнительно специфицирующая соответствующее \textit{действие}.\\
            Если соответствующий данному \textit{классу команд класс действий} является более частным по отношению к \textit{действиям в sc-памяти}, то попадание данного класса команд во множество \textit{атомарных классов команд} говорит о наличии в текущей версии системы как минимум одного \textit{sc-агента}, условие инициирования которого соответствует формулировке команд данного класса.}
        
        \scnheader{неатомарный класс команд}
        \scntext{пояснение}{Принадлежность некоторого \textit{класса команд} множеству \textit{неатомарных классов команд} фиксирует факт того, что данная спецификация не является достаточной для того, чтобы некоторый субъект приступил к выполнению соответствующего действия, и требует дополнительных уточнений.}
        
        \scnheader{класс действий}
        \begin{scnsubdividing}
            %TODO: check by human--->
            \scnitem{\textit{класс действий, однозначно задаваемый решаемым классом задач}}
        	\begin{scnindent}
                \scnidtf{\textit{класс действий}, обеспечивающих решение соответствующего \textit{класса задач} и использующих при этом любые, самые разные \textit{методы} решения задач этого класса}
            \end{scnindent}
            \scnitem{\textit{класс действий, однозначно задаваемый используемым методом решения задач}}
        \end{scnsubdividing}
        
        \scnheader{метод}
        \scnrelto{второй домен}{метод*}
        \scnidtf{описание того, \uline{как} может быть выполнено любое или почти любое действие, принадлежащее соответствующему классу действий}
        \scnidtf{метод решения соответствующего класса задач, обеспечивающий решение любой или большинства задач указанного класса}
        \scnidtf{обобщенная спецификация выполнения действий соответствующего класса}
        \scnidtf{обобщенная спецификация решения задач соответствующего класса}
        \scnidtf{программа решения задач соответствующего класса, которая может быть как процедурной, так и декларативной (непроцедурной)}
        \scnidtf{знание о том, как можно решать задачи соответствующего класса}
        \scnsubset{знание}
        \scniselement{вид знаний}
        \scnidtf{способ}
        \scnidtf{знание о том, как надо решать задачи соответствующего класса задач (множества эквивалентных (однотипных, похожих) задач)}
        \scnidtf{метод (способ) решения некоторого (соответствующего) класса задач}
        \scnidtf{информация (знание), достаточная для того, чтобы решить любую \textit{задачу}, принадлежащую соответствующему \textit{классу задач} с помощью соответствующей \textit{модели решения задач}}
        \scnidtf{обобщенный план выполнения некоторого класса сложных действий (или обобщенный план решения соответствующего класса сложных задач), привязка которого к конкретной задаче указанного класса и последующая интерпретация обеспечивает решение любой или почти любой задачи этого класса}
        \scntext{примечание}{Очевидно, что трудоемкость разработки метода определяется не столько мощностью класса задач, решаемых с помощью разрабатываемого метода, сколько их семантической близостью, аналогичностью.}
        \scnidtf{метод решения соответствующего класса задач}
        \scnidtf{метод выполнения соответствующего класса действий}
        \scnidtf{пассивный метод, хранимый в базе знаний и используемый соответствующими коллективами агентов при соответствующем их инициировании}
        \scnidtf{метод выполнения действий некоторого класса или метод решения задач некоторого класса или метод, который может быть использован для выполнения некоторого конкретного действия или для решения некоторой конкретной задачи}
        \scnidtf{обобщенное описание того, как можно выполнить действия из соответствующего класса действий или как можно решить задачу из соответствующего класса задач}
        \scnsuperset{метод сведения задач к подзадачам}
        \begin{scnindent}
        	\scnidtf{класс логически эквивалентных методов, обеспечивающих решение задач путем сведения этих задач к подзадачам и отличающихся только деталями формализации}
        \end{scnindent}
        \scntext{примечание}{В состав спецификации каждого \textit{класса задач} входит описание способа привязки \textit{метода} к исходным данным конкретной \textit{задачи}, решаемой с помощью этого \textit{метода}. Описание такого способа привязки включает в себя:
            \begin{scnitemize}
                \item набор переменных, которые входят как в состав \textit{метода}, так и в состав \textit{обобщенной формулировки задач соответствующего класса} и значениями которых являются соответствующие элементы исходных данных каждой конкретной решаемой задачи;
                \item часть \textit{обобщенной формулировки задач} того класса, которому соответствует рассматриваемый \textit{метод}, являющихся описанием \uline{условия применения} этого \textit{метода}.
            \end{scnitemize}
            \bigskip
            Сама рассматриваемая привязка \textit{метода} к конкретной \textit{задаче}, решаемой с помощью этого \textit{метода} осуществляется путем \uline{поиска} в \textit{базе знаний} такого фрагмента, который удовлетворяет условиям применения указанного \textit{метода}. Одним из результатов такого поиска и является установление соответствия между указанными выше переменными используемого \textit{метода} и значениями этих переменных в рамках конкретной решаемой \textit{задачи}.\\
            Другим вариантом установления рассматриваемого соответствия является явное обращение (вызов, call) соответствующего \textit{метода} (программы) с явной передачей соответствующих параметров. Но такое не всегда возможно, т.к. при выполнении процесса решения конкретной \textit{задачи} на основе декларативной спецификации выполнения этого действия нет возможности установить:
            \begin{scnitemize}
                \item когда необходимо инициировать вызов (использование) требуемого \textit{метода};
                \item какой конкретно \textit{метод} необходимо использовать;
                \item какие параметры, соответствующие конкретной инициируемой \textit{задаче}, необходимо передать для привязки используемого \textit{метода} к этой \textit{задаче}.
            \end{scnitemize}
            Процесс привязки \textit{метода} решения \textit{задач} к конкретной \textit{задаче}, решаемой с помощью этого \textit{метода}, можно также представить как процесс, состоящий из следующих этапов:
            \begin{scnitemize}
                \item построение копии используемого \textit{метода};
                \item склеивание основных (ключевых) переменных используемого \textit{метода} с основными параметрами конкретной решаемой \textit{задачи}.
            \end{scnitemize}
            В результате этого на основе рассматриваемого \textit{метода} используемого в качестве образца (шаблона) строится спецификация процесса решения конкретной задачи --- процедурная спецификация (\textit{план}) или декларативная.}
        \scntext{примечание}{Заметим, что \textit{методы} могут использоваться даже при построении \textit{планов} решения конкретных \textit{задач}, в случае, когда возникает необходимость многократного повторения неких цепочек \textit{действий} при априори неизвестном количестве таких повторений. Речь идет о различного вида \textit{циклах}, которые являются простейшим видом процедурных \textit{методов} решения задач, многократно используемых (повторяемых) при реализации \textit{планов} решения некоторых \textit{задач}.}
        \scnidtf{программа}
        \scnidtf{программа выполнения действий некоторого класса}
        \scntext{примечание}{Одному \textit{классу действий} может соответствовать несколько \textit{методов} (программ).}
        \scnsuperset{программа в sc-памяти}
        \scnsuperset{процедурная программа}
        \begin{scnindent}
	        \scnidtf{обобщенный процедурный план}
	        \scnidtf{обобщенный процедурный план выполнения некоторого класса действий}
	        \scnidtf{обобщенный процедурный план решения некоторого класса задач}
	        \scnidtf{обобщенная спецификация декомпозиции любого действия, принадлежащего заданному классу действий}
	        \scnidtf{знание о некотором классе действий (и соответствующем классе задач), позволяющее для каждого из указанных действий достаточно легко построить процедурный план его выполнения}
	        \scnsubset{алгоритм}
	        \scntext{пояснение}{Каждая \textit{процедурная программа} представляет собой обобщенный процедурный план выполнения \textit{действий}, принадлежащих некоторому классу, то есть \textit{семантическую окрестность, ключевым sc-элементом\scnrolesign} является \textit{класс действий}, для элементов которого дополнительно детализируется процесс их выполнения.\\
                В остальном описание \textit{процедурной программы} аналогично описанию \textit{плана} выполнения конкретного \textit{действия} из рассматриваемого \textit{класса действий}.\\
                Входным параметрам \textit{процедурной программы} в традиционном понимании соответствуют аргументы, соответствующие каждому \textit{действию} из \textit{класса действий}, описываемого \textit{процедурной программой}. При генерации на основе \textit{процедурной программы} \textit{плана} выполнения конкретного \textit{действия} из данного класса эти аргументы принимают конкретные значения.\\
                Каждая \textit{процедурная программа} представляет собой систему описанных действий с дополнительным указанием для действия:
	            \begin{scnitemize}
	                \item либо \textit{последовательности выполнения действий*} (передачи инициирования), когда условием выполнения (инициирования) действий является завершение выполнения одного из указанных или всех указанных действий;
	                \item либо события в базе знаний или внешней среде, являющегося условием его инициирования;
	                \item либо ситуации в базе знаний или внешней среде, являющейся условием его инициирования;
	            \end{scnitemize}}
         \end{scnindent}
	     \scnsuperset{программа sc-агента}
	     \begin{scnindent}
		     \scnidtf{метод, которому однозначно соответствует некоторыйsc-агент, способный выполнять действия, принадлежащие соответствующему данному методу классу действий}
		     \scntext{примечание}{Частным случаем метода является \textit{программа sc-агента}, в этом случае в качестве \textit{операционной семантики метода} выступает коллектив \textit{sc-агентов} более низкого уровня, интерпретирующий соответствующую программу (в предельном случае это будут \textit{sc-агенты}, являющиеся частью \textit{платформы интерпретации sc-моделей компьютерных систем}, в том числе аппаратной).Таким образом:
		         \begin{scnitemize}
		             \item можно говорить об иерархии \textit{методов} и о \textit{методах} интерпретации других \textit{методов};
		             \item можно сказать, что понятие метода обобщает понятие \textit{программы sc-агента} для \textit{неатомарных sc-агентов}.
		         \end{scnitemize}}
	     \end{scnindent}
	     \scntext{примечание}{Отметим, что понятие \textit{метода} фактически позволяет локализовать область решения задач соответствующего класса, то есть ограничить множество знаний, которых достаточно для решения задач данного класса определенным способом. Это, в свою очередь, позволяет повысить эффективность работы системы в целом, исключая число лишних действий.}
	     \scntext{примечание}{Каждый конкретный метод рассматривается нами как важный вид спецификации соответствующего класса задач, но также и как \textit{объект}, который и сам нуждается в спецификации, обеспечивающей непосредственное применение этого метода. Другими словами, метод является не только спецификацией (спецификацией соответствующего класса задач), но и \uline{объектом} спецификации.}
        
        \scnheader{следует отличать*}
        \begin{scnhaselementset}
            \scnitem{план сложного действия}
            \scnitem{метод}
        \end{scnhaselementset}
        \scntext{примечание}{В отличие от \textit{метода}, \textit{план сложного действия} описывает не \textit{класс действий}, а конкретное \textit{действие}, возможно с учетом особенностей его выполнения в текущем контексте.}
         
        \scnheader{эквивалентность задач*}
        \scnidtf{быть эквивалентной задачей*}
        \scniselement{отношение}
        \scntext{определение}{Задачи являются эквивалентными в том и только в том случае, если они могут быть решены путем интерпретации одного и того же \textit{метода} (способа), хранимого в памяти кибернетической системы.}
        \scntext{примечание}{Некоторые \textit{задачи} могут быть решены разными \textit{методами}, один из которых, например, является обобщением другого.}
        
        \scnheader{отношение, заданное на множестве*(метод)}
        \scnhaselement{подметод*}
        \begin{scnindent}
	        \scnidtf{подпрограмма*}
	        \scnidtf{быть методом, использование которого (обращение к которому) предполагается при реализации заданного метода*}
	        \scnrelboth{следует отличать}{частный метод*}
	        \begin{scnindent}
	        	\scnidtf{быть методом, обеспечивающим решение класса задач, который является подклассом задач, решаемых с помощью заданного метода*}
	        \end{scnindent}
        \end{scnindent}
        	
        \scnheader{стратегия решения задач}
        \scnsubset{метод}
        \scnidtf{метаметод решения задач, обеспечивающий либо поиск одного релевантного известного метода, либо синтез целенаправленной последовательности акций применения в общем случае различных известных методов}
        \scntext{примечание}{Можно говорить об универсальном метаметоде (универсальной стратегии) решения задач, объясняющем всевозможные частные стратегии.}
        \scntext{пояснение}{Можно говорить о нескольких глобальных \textit{стратегиях решения информационных задач} в базах знаний. Пусть в базе знаний появился знак инициированного действия с формулировкой соответствующей информационной цели, т.е. цели, направленной только на изменение состояния базы знаний. И пусть текущее состояние базы знаний не содержит контекста (исходных данных), достаточного для достижения указанной выше цели, т.е такого контекста, для которого в доступном пакете (наборе) методов (программ) имеется метод (программа), использование которого позволяет достигнуть указанную выше цель. Для достижения такой цели, контекст (исходные данные) которой недостаточен, существует три подхода (три стратегии):
            \begin{scnitemize}
                \item декомпозиция (сведение изначальной цели к иерархической системе и/или подцелей (и/или подзадач) на основе анализа текущего состояния базы знаний и анализа того, чего в базе знаний не хватает для использования того или иного метода.) При этом наибольшее внимание уделяется методам, для создания условий использования которых требуется меньше усилий. В конечном счете мы должны дойти (на самом нижнем уровне иерархии) до подцелей, контекст которых достаточен для применения одного из имеющихся методов (программ) решения задач;
                \item генерация новых знаний в семантической окрестности формулировки изначальной цели с помощью \uline{любых} доступных методов в надежде получить такое состояние базы знаний, которое будет содержать нужный контекст (достаточные исходные данные) для достижения изначальной цели с помощью какого-либо имеющегося метода решения задач;
                \item комбинация первого и второго подхода.
            \end{scnitemize}
            Аналогичные стратегии существуют и для поиска пути решения задач, решаемых во внешней среде.}
            
        \scnheader{сужение отношения по первому домену(спецификация*, метод)}
        \scnidtf{спецификация метода*}
        \begin{scnsubdividing}
            \scnitem{денотационная семантика метода*}
        	\begin{scnindent}
                \scnidtf{обобщенная формулировка класса задач, решаемых с помощью данного метода*}
                \scnrelboth{семантически близкий знак*}{обобщенная формулировка задач соответствующего класса*}
                \begin{scnindent}
                	\scntext{примечание}{Данное отношение связывает обобщенную формулировку задач не с методом, а с классом задач}
                \end{scnindent}
            \end{scnindent}
            \scnitem{операционная семантика метода*}
        	\begin{scnindent}
                \scnidtf{перечень обобщенных агентов, обеспечивающих интерпретацию метода*}
                \scnidtf{семейство методов интерпретации данного метода*}
                \scnidtf{формальное описание интерпретатора заданного метода*}
            \end{scnindent}
        \end{scnsubdividing}
        \bigskip
    \end{scnsubstruct}
    \scnendsegmentcomment{Уточнение понятий план сложного действия, классы действий, класса задач, метода}
\end{SCn}

        \begin{SCn}
    \scnsegmentheader{Уточнение понятия навыка, понятия класса методов и понятия модели решения задач}
    \begin{scnsubstruct}
        \scniselement{сегмент базы знаний}
        \scnheader{навык}
        \scnidtf{умение}
        \scnidtf{объединение \textit{метода} с его исчерпывающей спецификацией -- \textit{полным представлением операционной семантики метода}}
        \scnidtf{метод, интерпретация (выполнение, использование) которого полностью может быть осуществлено данной кибернетической системой, в памяти которой указанный метод хранится}
        \scnidtf{метод, который данная кибернетическая система умеет (может) применять}
        \scnidtf{метод + метод его интерпретации}
        \scnidtf{умение решать соответствующий класс эквивалентных задач}
        \scnidtf{метод плюс его операционная семантика, описывающая то, как интерпретируется (выполняется, реализуется) этот метод, и являющаяся одновременно операционной семантикой соответствующей модели решения задач}
        \begin{scnsubdividing}
            \scnitem{активный навык
                ~\\\scnidtf{самоинициирующийся навык}}
            \scnitem{пассивный навык}
        \end{scnsubdividing}
        \scntext{explanation}{\textit{Навыки} могут быть \textit{пассивными навыками}, то есть такими \textit{навыками}, применение которых должно явно инициироваться каким-либо агентом, либо \textit{активными навыками}, которые инициируются самостоятельно при возникновении соответствующей ситуации в базе знаний. Для этого в состав \textit{активного навыка} помимо \textit{метода} и его операционной семантики включается также \textit{sc-агент}, который реагирует на появление соответствующей ситуации в базе знаний и инициирует интерпретацию \textit{метода} данного \textit{навыка}.Такое разделение позволяет реализовать и комбинировать различные подходы к решению задач, в частности, \textit{пассивные навыки} можно рассматривать в качестве способа реализации концепции интеллектуального пакета программ.}
        \scnheader{класс методов}
        \scnrelto{семейство подклассов}{метод}
        \scnidtf{множество методов, для которых можно \uline{унифицировать} представление (спецификацию) этих методов}
        \scnidtf{множество всевозможных методов решения задач, имеющих общий язык представления этих методов}
        \scnidtf{множество всевозможных методов, представленных на данном языке}
        \scnidtf{множество методов, для которых задан язык представления этих методов}
        \scnhaselement{процедурный метод решения задач}
        \scnsuperset{алгоритмический метод решения задач}
        \scnhaselement{логический метод решения задач}
        \scnsuperset{продукционный метод решения задач}
        \scnsuperset{функциональный метод решения задач}
        \scnhaselement{искусственная нейронная сеть}
        \scnidtf{класс методов решения задач на основе искусственных нейронных сетей}
        \scnhaselement{генетический алгоритм}
        \scnidtf{множество методов основанных на общей онтологии}
        \scnidtf{множество методов, представленных на одинаковом языке}
        \scnidtf{множество методов решений задач, которому соответсвует специальный язык (например, sc-язык), обеспечивающий представление методов из этого множества}
        \scnidtf{множество методов, которому ставится в соответствие отдельная модель решения задач}
        \scnheader{язык представления методов}
        \scnidtf{язык методов}
        \scnidtf{язык представления методов, соответствующих определенному классу методов}
        \scntext{note}{Таких специализированных языков может быть выделено целое множество, каждому из которых будет соответствовать своя модель решения задач (т.е. свой интерпретатор)}
        \scnidtf{язык (например sc-язык) представлений методов соответствующего класса методов}
        \scnsubset{язык}
        \scnidtf{язык программирования}
        \scnsuperset{язык представления методов обработки информации}
        \scnidtf{язык программирования внутренних действий кибернетической системы, выполняемых в их памяти}
        \scnidtf{язык представления методов решения задач в памяти кибернетических систем}
        \scnsuperset{язык представления методов решения задач во внешней среде кибернетических систем}
        \scnidtf{язык программирования внешних действий кибернетических систем}
        \scnheader{модель решения задач}
        \scnidtf{метаметод интерпретации соответствующего класса методов}
        \scnsubset{метод}
        \scnidtf{метаметод}
        \scnidtf{абстрактная машина интерпретации соответствующего класса методов}
        \scnidtf{иерархическая система микропрограмм, обеспечивающих интерпретацию соответствующего класса методов}
        \scnsuperset{алгоритмическая модель решения задач}
        \scnsuperset{процедурная параллельная синхронная модель решения задач}
        \scnsuperset{процедурная параллельная асинхронная модель решения задач}
        \scnsuperset{продукционная модель решения задач}
        \scnsuperset{функциональная модель решения задач}
        \scnsuperset{логическая модель решения задач}
        \scnsuperset{четкая логическая модель решения задач}
        \scnsuperset{нечеткая логическая модель решения задач}
        \scnsuperset{нейросетевая модель решения задач}
        \scnsuperset{генетическая модель решения задач}
        \scntext{note}{Для интерпретации \uline{всех} моделей решения задач может быть использован агентно-ориентированный подход}
        \scntext{explanation}{Каждая \textit{модель решения задач} задается:
            \begin{scnitemize}
                \item соответствующим классом методов решения задач, т.е. языком представления методов этого класса;
                \item предметной областью этого класса методов;
                \item онтологией этого класса методов (т.е. денотационной семантикой языка представления этих методов);
                \item операционной семантикой указанного класса методов.
            \end{scnitemize}
        }
        \scnheader{модель решения задач*}
        \scneq{сужение отношения по первому домену(спецификация*; класс методов)*}
        \scnidtf{спецификация \textit{класса методов}*}
        \scnidtf{спецификация \textit{языка представления методов}*}
        \begin{scnreltoset}{обобщенное объединение}
            \scnitem{синтаксис языка представления методов соответствующего класса*}
            \scnitem{денотационная семантика языка представления методов соответствующего класса*}
            \scnitem{операционная семантика языка представления методов соответствующего класса*}
        \end{scnreltoset}
        \scntext{note}{Каждому конкретному \textit{классу методов} взаимно однозначно соответствует \textit{язык представления методов}, принадлежащих этому (специфицируемому) \textit{классу методов}. Таким образом, спецификация каждого \textit{класса методов} сводится к спецификации соответствующего \textit{языка представления методов}, т.е. к описанию его синтаксической, денотационной семантики и операционной семантики.Примерами \textit{языков представления методов} являются все \textit{языки программирования}, которые в основном относятся к подклассу \textit{языков представления методов} -- к \textit{языкам представления методов обработки информации}. Но сейчас все большую актуальность приобретает необходимость создания эффективных формальных языков представления методов выполнения действий во внешней среде кибернетических систем. Без этого комплексная автоматизация, в частности, в промышленной сфере невозможна.}
        \scnheader{денотационная семантика языка представления методов соответствующего класса}
        \scnrelto{второй домен}{денотационная семантика языка представления методов соответствующего класса*}
        \scnidtf{онтология соответствующего класса методов}
        \scnidtf{денотационная семантика соответствующего класса методов}
        \scnidtf{денотационная семантика языка (sc-языка), обеспечивающего представление методов соответствующего класса}
        \scnidtf{денотационная семантика соответствующей модели решения задач}
        \scntext{note}{если речь идет о языке, обеспечивающем внутреннее представление методов соответствующего класса в ostis-системе, то синтаксис этого языка совпадает с синтаксисом SC-кода}
        \scnsubset{онтология}
        \scnheader{операционная семантика языка представления методов соответствующего класса}
        \scnrelto{второй домен}{операционная семантика языка представления методов соответствующего класса*}
        \scnidtf{метаметод интерпретации соответствующего класса методов}
        \scnidtf{семейство агентов, обеспечивающих интерпретацию (использования) любого метода, принадлежащего соответствующему классу методов}
        \scnidtf{операционная семантика соответствующей модели решения задач}
        \scnheader{язык представления обобщенных формулировок задач для различных классов задач}
        \scntext{note}{Поскольку каждому \textit{методу} соответствует \textit{обобщенная формулировка задач}, решаемых с помощью этого \textit{метода}, то каждому \textit{классу методов} должен соответствовать не только определенный \textit{язык представления методов}, принадлежащих указанному \textit{классу методов}, но и определенный \textit{язык представления обобщенных формулировок задач для различных классов задач}, решаемых с помощью \textit{методов}, принадлежащих указанному \textit{классу методов}.}
        \bigskip
    \end{scnsubstruct}
    \scnendsegmentcomment{Уточнение понятия навыка, понятия класса методов и понятия модели решения задач}
    \scnsegmentheader{Уточнение понятия деятельности, понятия вида деятельности и понятия технологии}
    \begin{scnsubstruct}
        \scniselement{сегмент базы знаний}
        \scnheader{деятельность}
        \scnidtftext{explanation}{сложный процесс, состоящий из действий, направленных на достижение нескольких \uline{разных} целей (т.е. целей, не связанных отношением цель-подцель). При этом некоторые из указанных максимальных целей могут достигаться с помощью одного и того же метода или одного и того же (фиксированного) семейства методов}
        \scnsuperset{физическая деятельность}
        \scnidtf{деятельность по преобразованию материальных сущностей (физических объектов)}
        \scnsuperset{информационная деятельность}
        \scnidtf{деятельность, направленная на обработку информацию}
        \scnidtf{умственная деятельность}
        ~\\\scntext{note}{Информационная деятельность является необходимым компонентом физической деятельности, обеспечивающим принятие решений, планирование и управление физическим процессом}
        \scnsubset{процесс}
        \scnidtf{целостный, целенаправленный процесс \uline{поведения} (функционирования) одного субъекта или сообщества субъектов, осуществляемый на основе хорошо или не очень хорошо продуманной и согласованной \textit{технологии} в последнем случае качество деятельности определяется уровнем интеллекта единоличного или коллективного субъекта, осуществляющего этот целенаправленный процесс.}
        \scnidtf{система действий, являющаяся некоторым кластером семантически близких действий, обладающих семантической близостью, семантической связностью и семантической целостностью}
        \scnidtf{трудно выполнимая семантически целостная система действий}
        \scnidtf{кластер множества действий, определяемый семантической близостью этих действий}
        \scnidtf{система связанных между собой действий, имеющих общий контекст, общую область выполнения этих действий}
        \scntext{note}{В состав каждой конкретной \textit{деятельности} входят \textit{действия}, являющиеся \textit{поддействиями}* других \textit{действий}, входящих в состав этой же \textit{деятельности}. При этом для каждого \textit{действия}, входящего в состав \textit{деятельности}, все поддействия этого \textit{действия} также входят в состав этой \textit{деятельности}.В состав каждой конкретной \textit{деятельности} входят также \textit{действия}, не являющиеся \textit{поддействиями}* других \textit{действий}, входящих в состав этой же \textit{деятельности}. Такие первичные (независимые, самостоятельные, автономные) \textit{действия} для заданной \textit{деятельности} могут инициироваться \uline{извне} этой \textit{деятельности} с помощью соответствующих инициирующих эти \textit{действия ситуаций} или \textit{событий}. Примерами таких инициирующих ситуаций, порождающих соответствующие действия, являются:
            \begin{scnitemize}
                \item появление в \textit{базе знаний} каких-либо противоречий, информационных дыр, информационного мусора;
                \item появление в \textit{базе знаний} описаний (информационных моделей) каких-либо нештатных ситуаций в сложном объекте управления, на которые необходимо реагировать;
                \item появление в \textit{базе знаний} формулировок различного рода задач с явным указанием инициирования соответствующих действий, направленных на решение этих задач.
            \end{scnitemize}
            К числу указанных первичных (независимых) \textit{действий}, входящих в состав \textit{объединенной деятельности кибернетической системы}, также относятся:
            \begin{scnitemize}
                \item сложное действие, целью которого является перманентное обеспечение комплексной \textit{безопасности кибернетической системы};
                \item сложное действие, целью которого является перманентное  повышение качества информации (базы знаний), хранимой в памяти \textit{кибернетической системы};
                \item сложное действие, целью которого является перманентное повышение \textit{качества решателя задач кибернетической системы};
                \item сложное действие, целью которого является перманентная поддержка высокого уровня \textit{семантической совместимости} кибернетической системы со своими партнерами.
            \end{scnitemize}
        }
        \scnheader{отношение, заданное на множестве*(деятельность)}
        \scnhaselement{субъект*}
        \scnidtf{быть субъектом заданного действия или деятельности*}
        \scnidtf{кибернетическая система, которая в рамках заданного действия или деятельности выполняет ту или иную роль, воздействует на некий объект действия, используя тот или иной инструмент*}
        \scniselement{отношение, заданное на множестве*(действие)}
        \scnhaselement{контекст*}
        \scnidtf{информационный контекст, в рамках которого осуществляется выполнение заданного действия или деятельности*}
        \scnidtf{область исполнения действия или деятельности*}
        \scnidtf{область действия или деятельности*}
        \scnrelfrom{первый домен}{(действие $\cup$ деятельность)}
        \scnidtf{совокупность знаний, достаточных для информационного обеспечения заданного действия или заданной деятельности}
        \scniselement{отношение, заданное на множестве* (действие)}
        \scntext{note}{Локализация (минимизация) \textit{контекста} заданного действия или деятельности является важнейшим подготовленным этапом, обеспечивающим существенное снижение накладных расходов при непосредственном выполнении этого \textit{действия} или \textit{деятельности}.}
        \scntext{note}{Чаще всего \textit{контекстом} заданного \textit{действия} или \textit{деятельности} является некоторая \textit{предметная область} вместе с соответствующей ей интегрированной (объединенной) \textit{онтологией}. Поэтому хорошо продуманная декомпозиция \textit{базы знаний} интеллектуальной компьютерной системы на иерархическую систему \textit{предметных областей} и соответствующих им \textit{онтологий} имеет важное практическое значение, существенно повышающее качество (в частности, быстродействие) \textit{решателя задач} интеллектуальной компьютерной системы благодаря априорному  разбиению множества выполняемых \textit{действий} (решаемых задач) по соответствующих им \textit{контекстам}.}
        \scnheader{следует отличать*}
        \scnsuperset{\begin{scnset}
                \item действие
                ~\\\scnidtf{процесс достижения конкретной цели конкретных обстоятельствах}
                \scnidtf{процесс решения конкретной задачи в конкретных условиях}
                \scnidtf{процесс задуманный, инициированный и осуществленный некоторым (или некоторыми) субъектами (кибернетическими системами)}
                \scntext{note}{\textit{действие} (точнее, соответствующая форма участия в его выполнении) является частью (фрагментом) \textit{деятельности} всех участвующих в этом субъектов (кибернетических систем)}
                \item деятельность
                ~\\\scnidtf{система действий выполняемых соответствующим субъектом (кибернетической системой) скрепленное общим контекстом и определенным набором используемых навыков и инструментов}
                \scntext{note}{В отличие от \textit{действия}, \textit{деятельность} носит чаще всего перманентный характер в рамках времени существования соответствующего субъекта}
            \end{scnset}
        }
        \scnheader{деятельность кибернетической системы}
        \scnidtf{полная система действий, выполняемых соответствующей кибернетической системой}
        \scnidtf{деятельность субъекта}
        \scnidtf{система всех действий соответствующего субъекта}
        \begin{scnsubdividing}
            \scnitem{внутренняя деятельность субъекта
                ~\\\scnidtf{внутренняя деятельность соответствующего субъекта}
                \scnidtf{деятельность некоторого субъекта по обработке информации}
                \scnidtf{информационная деятельность}
            }
            \scnitem{поведение субъекта
                ~\\\scnidtf{внешнее поведение соответствующего субъекта}
                \scnidtf{деятельность субъекта во внешней среде}
            }
        \end{scnsubdividing}
        \scnheader{вид деятельности}
        \scnrelto{семейство подклассов}{деятельность}
        \scnidtf{класс семантически целостных систем действия, для которых можно унифицировать используемые методы, информационные ресурсы и инструменты}
        \scnidtf{класс трудно выполнимых и семантически целостных систем сложных действий}
        \scnidtf{класс кластеров систем действий}
        \scnidtf{множество деятельностей, которые могут быть реализованы с помощью общей технологии}
        \scnhaselement{устранение противоречий в базе знаний}
        \scnhaselement{устранение информационных дыр в базе знаний}
        \scnhaselement{ликвидация информационного мусора в базе знаний}
        \scnhaselement{управление сложным внешним объектом}
        \scnhaselement{поддержка семантической совместимости с партнерами}
        \scnhaselement{проектирование}
        \scnidtf{проектная деятельность}
        \scnidtf{построение такого описания (в частности, описания структуры) некоторого материального объекта, которого достаточно для воспроизводства (реализации, материализации) этого объекта либо при одиночном (уникальном), либо при массовом (промышленном) воспроизводстве указанного объекта}
        \scntext{note}{Примерами проектирования являются:
            \begin{scnitemize}
                \item проектирование здания;
                \item проектирование машиностроительной конструкции;
                \item проектирование микросхемы;
                \item проектирование ostis-системы;
                \item разработка системы шунтирования сердца;
                \item разработка такого описания сложной геометрической фигуры, которого было бы достаточно для построения изображения (рисунка) этой фигуры с помощью, например, циркуля и линейки.
            \end{scnitemize}
        }
        \scnhaselement{разработка плана производства материального объекта по заданному проекту этого объекта}
        \scnsuperset{разработка плана единичной реализации материального объекта по заданному проекту этого объекта}
        \scnsuperset{разработка плана массовой реализации материальных объектов по заданному их типовому проекту}
        \scntext{note}{Примерами данного вида действий являются:
            \begin{scnitemize}
                \item разработка плана-графика строительства конкретного здания;
                \item разработка типового плана строительства зданий по заданному их типовому проекту;
                \item разработка типового плана операций шунтирования сердца;
                \item разработка алгоритма построения \uline{изображения} заданной геометрической фигуры с помощью циркуля и линейки
            \end{scnitemize}
        }
        \scnhaselement{производство}
        \scnidtf{воспроизводство материального объекта по заданному его проекту и плану реализации}
        \scnidtf{производственная деятельность}
        \scntext{note}{Примерами данного вида действий являются:
            \begin{scnitemize}
                \item непосредственно строительство конкретного здания;
                \item проведение конкретной хирургической операции;
                \item процесс построения \uline{изображения} (рисунка) геометрической фигуры с помощью циркуля и линейки.
            \end{scnitemize}
        }
        \scnhaselement{реинжиниринг}
        \scnhaselement{анализ}
        \scnhaselement{интеграция}
        \scnidtf{синтез}
        \scnhaselement{деятельность в области здравоохранения}
        \scnhaselement{образовательная деятельность}
        \scnhaselement{эксплуатация сложного объекта}
        \scnhaselement{научно-исследовательская деятельность}
        \scnhaselement{управление}
        \scnsuperset{целенаправленная координация деятельности нескольких субъектов}
        \scnidtf{управление целенаправленной коллективной деятельностью нескольких субъектов}
        \scnheader{проектирование}
        \scnidtf{действие, направленное на построение (разработку) такой \uline{информационной} модели (проекта) некоторой \uline{материальной} сущности, которой \uline{достаточно}, чтобы соответствующий индивидуальный или коллективный субъект по соответствующей технологии (т.е. с помощью соответствующих методов и средств (инструментов)) смог воспроизвести (изготовить) указанную материальную сущность либо в одном экземпляре, либо в достаточно большом количестве таких экземпляров (копий), т.е. воспроизвести в промышленном масштабе}
        \scnheader{производство}
        \scnidtf{воспроизводство}
        \scnidtf{изготовление}
        \scnidtf{реализация}
        \scnidtf{материализация}
        \scnidtf{построение, синтез материальной сущности (артефакта)}
        \scnidtf{изготовление материальной сущности в одной или во множестве экземпляров (копий)}
        \scnidtf{производство (как действие)}
        \scnheader{реинжиниринг}
        \scnidtf{модификация}
        \scnidtf{внесение изменений в некую сущность}
        \scnidtf{обновление}
        \scnidtf{реинжиниринг}
        \scnidtf{перепроектирование}
        \scnidtf{реконфигурация}
        \scnidtf{трансформирование}
        \scnsuperset{совершенствование}
        \scnidtf{модификация, направленной на повышение качества модифицируемой сущности}
        \scnidtf{повышение качества}
        \scnidtf{улучшение}
        \scnsuperset{самосовершенствование}
        \scnidtf{совершенствование, выполняемое самой совершенствуемой сущностью}
        \scnsuperset{совершенствование, осуществляемое извне}
        \scntext{note}{Самосовершенствоваться и обучаться могут только достаточно развитые кибернетические системы. Но совершенствоваться усилиями внешних субъектов могут любые сущности.}
        \scnheader{анализ}
        \scnidtf{построение (разработка, создание) спецификации (описания) основных связей и/или структуры, свойств, закономерностей, соответствующих (описываемой) сущности}
        \scntext{note}{Объектом анализа может быть не только материальная сущность, но и процесс, ситуация, статическая структура, внешняя информационная конструкция, знание, понятие и другие абстрактные сущности}
        \scnheader{интеграция}
        \scnidtf{синтез}
        \scnidtf{соединение}
        \scnidtf{объединение}
        \scnidtf{сборка}
        \begin{scnsubdividing}
            \scnitem{эклектичная интеграция
                ~\\\scnidtf{интеграция без разрушения целостности интегрируемых сущностей}
                \scnidtf{интеграция без взаимопроникновения}
                \scnidtf{соединение систем по их входам/выходам}
            }
            \scnitem{глубокая интеграция
                ~\\\scnidtf{интеграция, в результате которой получается гибридная сущность}
                \scnidtf{интеграция с разрушением целостности (взаимопроникновением диффузий) интегрируемых сущностей}
                \scnidtf{бесшовная интеграция}
            }
        \end{scnsubdividing}
        \scnheader{вид деятельности}
        \scntext{explanation}{Если классу легко выполнимых сложных действий ставится в соответствие чаще всего \uline{один} \textit{метод} и, возможно, некоторый набор инструментальных средств, используемых в этом методе, то каждому виду деятельности ставится в соответствие своя \textbf{\textit{технология}}, включающая в себя некоторый набор используемых \textit{методов}, а также набор \textit{инструментальных средств}, используемых в этих \textit{методах}. Сложность здесь заключается:
            \begin{scnitemize}
                \item в нетривиальности организации использования всего арсенала имеющейся \textit{технологии} для реализации (выполнения) каждой соответствующей \textit{деятельности};
                \item в трудности, а часто и в принципиальной невозможности \uline{полностью} автоматизировать реализацию соответствующей \textit{деятельности}.
            \end{scnitemize}
        }
        \scnheader{следует отличать*}
        \begin{scnhaselementset}
            \scnitem{действие
                ~\\\scnhaselementrole{пример}{Процесс доказательства Теоремы Пифагора}
                \scniselement{действие направленное на построение доказательства теоремы Геометрии Евклида}
            }
            \scnitem{класс действий
                ~\\\scnhaselementrole{пример}{процесс доказательства теоремы}
                \scnidtftext{имя нарицательное}{действие, направленное на построение доказательства (логического обоснования) теоремы}
                \scnidtftext{имя собственное}{Класс действий, направленных на построение доказательств (логических обоснований) всевозможных теорем в различных формальных теориях}
            }
            \scnitem{деятельность
                ~\\\scnhaselementrole{пример}{Процесс эволюции Геометрии Евклида}
                \scnidtf{Процесс эволюции формальной теории, являющейся формальным представлением Геометрии Евклида}
                \scntext{explanation}{В данный процесс входит и генерация гипотез в рамках Геометрии Евклида, и доказательство теорем, и выявление противоречий между высказываниями, и разрешение этих противоречий, и минимизация числа используемых определяемых понятий, и многое другое}
                \scntext{note}{\textit{деятельность} -- это то, что превращает множество самостоятельных и в определенной степени независимых \textit{действий}, принадлежащих разным \textit{классам действий}, в целостную, целенаправленную, сбалансированную систему \textit{действий}, ориентированную, прежде всего на поддержание качества и эволюцию \textit{кибернетических систем}, а также на обеспечение их адаптации к новым, ранее не предусмотренным обстоятельствам.}}
            \scnitem{вид деятельности
                ~\\\scnhaselementrole{пример}{процесс эволюции формальной теории}
                \scnidtftext{имя собственное}{Класс процессов, направленных на эволюцию всевозможных формальных теорий (логических онтологий), которая также включает в себя возможность коррекции этих теорий.}
            }
        \end{scnhaselementset}
        \scnheader{сужение отношения по первому домену(спецификация*; вид деятельности)*}
        \scnidtftext{часто используемый sc-идентификатор}{спецификация вида деятельности*}
        \scneq{технология*}
        \scnidtf{технология реализации (выполнения) деятельности соответствующего (заданного) вида*}
        \scnrelfrom{второй домен}{\textbf{технология}
            ~\\\scnidtf{технология соответствующего вида деятельности}
            \scnrelboth{аналог}{декларативный метод выполнения действий соответствующего класса}
            \scnrelboth{аналог}{декларативная спецификация выполнения действия}
            \scntext{explanation}{\textit{технология} (как спецификация соответствующего вида деятельности) включает в себя:
            \begin{scnitemize}
                    \item указание \textit{контекста}* специфицируемого \textit{вида деятельности};
                    \item указание \textit{множества используемых методов}*, множества используемых инструментов, а также используемых материалов.
                \end{scnitemize}
            }}
        \scnheader{технология}
        \scntext{explanation}{Каждая \textit{технология} представляет собой комплекс \textit{методов} (методик) и средств, обеспечивающих выполнение некоторого множества \textit{действий}, входящих в состав соответствующего \textit{вида деятельности}. Каждая \textit{технология} задается:
            \begin{scnitemize}
                \item множеством методов (методик), которое разбивается на классы методов, эквивалентных по своей операционной семантике (по набору агентов, осуществляющих интерпретацию соответствующего класса методов);
                \item множеством агентов, являющихся средством интерпретации методов из указанного выше множества.
            \end{scnitemize}
            Указанное множество агентов также разбивается на подмножества, каждое из которых  соответствует своему классу методов и обеспечивает интерпретацию методов только этого класса.}
        \scnidtf{множество (комплекс) навыков, обеспечивающих выполнение такого множества действий (задач), для которых отсутствует общий метод их выполнения}
        \scnidtf{методика, инструментарий и дополнительные ресурсы, которые обеспечивают выполнение каждой конкретной деятельности, принадлежащей соответствующему виду деятельности}
        \scntext{explanation}{с формальной точки зрения каждая технология задается ориентированной связкой, компонентами которой являются
            \begin{scnitemize}
                \item знак множества используемых методов
                \item знак множества используемых инструментов
                \item знак множества дополнительных используемых ресурсов
            \end{scnitemize}
        }
        \scnidtf{комплекс методов и средств (инструментов), с помощью которого некий субъект (который может быть как индивидуальным, так и коллективным) осуществляет некоторую деятельность (некоторое целенаправленное множество действий, входящих в состав этой деятельности)}
        \scnsuperset{технология научно-теоретической деятельности}
        \scnsuperset{технология проектирования}
        \scnidtf{технология проектной деятельности}
        \scnidtf{технология построения такой информационной модели соответствующей сущности (артефакта), которой достаточно для воспроизводства этой сущности}
        \scnsuperset{технология производства}
        \scnidtf{технология производственной деятельности}
        \scnidtf{технология воспроизводства некоторого вида сущностей по заданным проектам этих сущностей}
        \scnsuperset{технология здравоохранения}
        \scnsuperset{технология образования}
        \scnidtf{технология подготовки молодых специалистов}
        \scnidtf{технология образовательной деятельности}
        \scnheader{отношение, заданное на множестве* (технология*)}
        \scnhaselement{методы*}
        \scnidtf{семейство методов, используемых в специфицируемой технологии  с дополнительным указанием их иерархии (т.е. с указанием того, какие методы используются при реализации других методов)}
        \scnhaselement{активные инструменты*}
        \scnidtf{средства, которые сами способны выполнять некоторые действия, но при этом ими надо как-то управлять (например, транспортные средства, компьютеры, )}
        \scnidtf{средства автоматизации*}
        \scnhaselement{пассивные инструменты*}
        \scnidtf{средства, которые сами ничего делать не могут (например, молоток, лопата, ножницы, )}
        \scnhaselement{комплектация*}
        \scnhaselement{расходные средства*}
        \scnhaselement{сырье*}
        \scnhaselement{продукты*}
        \scnhaselement{общий продукт*}
        \scnidtf{объединенный (интегрированный) продукт*}
        \scnhaselement{реализация технологии*}
        \scnidtf{вариант (форма) реализации технологии*}
        \scnhaselement{частная технология*}
        \scnidtf{быть частной технологией по отношению к заданной технологии*}
        \scnheader{продукты*}
        \scnidtf{производимые сущности*}
        \scnidtf{изготавливаемые материальные сущности*}
        \scnidtf{продукция*}
        \scnidtf{результаты выполнения соответствующего множества действий, осуществляемых во внешней среде*}
        \scnidtf{продукты технологии*}
        \scnidtf{множество материальных сущностей, производимых (создаваемых, порождаемых, изготавливаемых) с помощью заданной технологии*}
        \scnidtf{то, что является сухим остатком при использовании данной технологии*}
        \scnheader{технология}
        \scntext{note}{Поскольку разработка каждой конкретной \textit{технологии} требует больших затрат, очень важно, чтобы \textit{технологии} создавались не под конкретные \textit{деятельности}, а для целых классов деятельностей (\textit{видов деятельности}). При этом важно, чтобы разрабатываемые \textit{технологии} охватывали как можно большее количество деятельностей, входящих в состав указанных \textit{видов деятельности}. Из этого следует целесообразность конвергенции и унификации различных сфер \textit{деятельности} для того, чтобы повысить мощность применения (использования) каждой разрабатываемой \textit{технологии}. Кроме того важна \textit{совместимость технологий}, позволяющая решать \textit{задачи}, требующие одновременного использования нескольких \textit{технологий}, причем, в непредсказуемых сочетаниях. Очень важно также, кроме \textit{видов деятельности}, которым соответствуют конкретные \textit{технологии}, ввести \textit{обобщенные виды деятельности} и построить их иерархии явно фиксировать стандарты, которым должны соответствовать все виды соответствующего обобщенного \textit{вида деятельности}. Это необходимо для обеспечения совместимости \textit{технологий}. Все используемые технологии должны пронизывать друг друга и составлять стройную иерархическую систему совместимых технологий (сумму технологий).}
        \scnheader{класс технологий}
        \scnidtf{множество похожих технологий, использующих, например, одинаковые методики и/или одинаковые активные инструменты и/или одинаковые пассивные инструменты и/или похожие множества продуктов}
        \scnhaselement{технология проектирования}
        \scnsuperset{технология проектирования интеллектуальных компьютерных систем}
        \scnsuperset{технология проектирования программных систем}
        \scnsuperset{технология проектирования микросхем}
        \scnsuperset{технология машиностроительного проектирования}
        \scnhaselement{технология рецептурного производства}
        \scnsuperset{технология производства молочных продуктов}
        \scnsuperset{технология производства мясных продуктов}
        \scnsuperset{технология фармацевтического производства}
        \bigskip
    \end{scnsubstruct}
    \scnendsegmentcomment{Уточнение понятия деятельности, понятия вида деятельности и понятия технологии}
\end{SCn}

        \bigskip
    \end{scnsubstruct}
    \scntext{заключение}{Дальнейшее развитие представленных в данной предметной области онтологий предполагает формализацию классификации задач, решаемых интеллектуальными системами, унификацию описания задач и классов задач, описания целей, хода и результата решения задачи, методов решения задач, связей между классами задач и методами решения задач данного класса.
        \\Это позволит обеспечить возможность глубокой интеграции всевозможных моделей решения задач различных классов и возможность облегчить процесс интеграции новых моделей решения задач в интеллектуальную систему, а также положит основу для решения ряда проблем в области разработки гибридных решателей задач, рассмотренных в \textit{\nameref{sd_ps}}.}
    \bigskip
    \scnendcurrentsectioncomment
\end{SCn}

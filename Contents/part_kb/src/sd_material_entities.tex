\begin{SCn}
	\scnsectionheader{Предметная область и онтология материальных сущностей}
	\scntext{вступление}{В данной статье рассмотрена структура онтологии предметной области материальных сущностей, взаимосвязь предметной области с другими предметными областями верхнего уровня.}
	\begin{scnrelfromlist}{максимальный класс исследования}
		\scnitem{пространственная сущность}
		\scnitem{материальная сущность}
		\scnitem{вещество}
		\scnitem{физическое поле}
		\scnitem{персона}
		\scnitem{юридическое лицо}
		\scnitem{предприятие}
		\scnitem{географический объект}
	\end{scnrelfromlist}
	
	\scnheader{материальная сущность}
	\begin{scnrelfromlist}{связь}
		\scnitem{предметная область пространственных сущностей}
		\scnitem{предметная область ситуаций и событий}
		\scnitem{предметная область временных сущностей}
	\end{scnrelfromlist}
	
	\scnheader{предметная область}
	\scnidtf{sc-модель предметной области}
	\scnidtf{sc-текст предметной области}
	\scnidtf{sc-граф предметной области}
	\scnidtf{представление предметной области в \textit{SC-коде}}
	\scnsubset{знание}
	\scnsubset{бесконечное множество}
	\scntext{пояснение}{\textbf{\textit{предметная область}} --- это результат интеграции (объединения) частичных семантических окрестностей, описывающих все исследуемые сущности заданного класса и имеющих одинаковый (общий) предмет исследования (то есть один и тот же набор отношений, которым должны принадлежать связки, входящие в состав интегрируемых семантических окрестностей).
		\begin{scnitemize}
			\item \textit{Предметная область предметных областей}, объектами исследования которой являются всевозможные \textbf{\textit{предметные области}}, а предметом исследования --- всевозможные \textit{ролевые отношения}, связывающие предметные области с их элементами, отношения, связывающие предметные области между собой, отношение, связывающее предметные области с их онтологиями
			\item \textit{Предметная область сущностей}, являющаяся предметной областью самого высокого уровня и задающая базовую семантическую типологию \textit{sc-элементов}(знаков, входящих в тексты \textit{SC-кода})
			\item Семейство \textbf{\textit{предметных областей}}, каждая из которых задает семантику и синтаксис некоторого \textit{sc-языка}, обеспечивающего представление онтологий соответствующего вида (например, \textit{теоретико-множественных онтологий}, \textit{логических онтологий}, \textit{терминологических онтологий}, \textit{онтологий задач и способов их решения} и т.д.)
			\item Семейство \textbf{\textit{предметных областей}} верхнего уровня, в которых классами объектов исследования являются весьма крупные классы сущностей. К таким классам, в частности
			\begin{scnitemizeii}
				\item класс всевозможных \textit{материальных сущностей},
				\item класс всевозможных \textit{множеств},
				\item класс всевозможных \textit{связей},
				\item класс всевозможных \textit{отношений},
				\item класс всевозможных \textit{структур},
				\item класс всевозможных \textit{временных (временно существующих, непостоянных сущностей) сущностей},
				\item класс всевозможных \textit{действий} (акций),
				\item класс всевозможных \textit{параметров} (характеристик),
				\item класс \textit{знаний} всевозможного вида
				\item и т.п.
			\end{scnitemizeii}
		\end{scnitemize}
	}
	\scnheader{материальная сущность}
	\scnidtf{философский термин, используемый для обозначения физического или конкретного объекта в реальном мире}
	\scnidtf{сущность, относящаяся к представлению или моделированию реального объекта в цифровой форме}
	\scnidtf{сущность, относящаяся к представлению или моделированию реального объекта в sc-памяти}
	\begin{scnrelfromlist}{включает}
		\scnitem{материал объекта}
		\scnitem{свойство объекта}
		\scnitem{структура объекта}
		\scnitem{процесс}
	\end{scnrelfromlist}
	
	
	\scnheader{материальная сущность}
	\begin{scnrelfromlist}{пояснение}
		\scnidtf{Каждой материальной сущности можно поставить в соответствие различные \textbf{процессы}, описывающие ее преобразование из одного состояния в другое}
	\end{scnrelfromlist}
	
	\scnheader{процесс}
	\scnidtf{процесс преобразования некоторой временной сущности из одного состояния в другое}
	\scnidtf{процесс перехода от одной ситуации к другой}
	\scnidtf{абстрактный процесс}
	\scnidtf{информационная модель некоторых преобразований}
	\scnidtf{динамическая sc-модель}
	\scnidtf{динамическая структура}
	\scnrelfrom{включение}{воздействие}
	\scntext{пояснение}{Каждый \textbf{\textit{процесс}} определяется (задается) либо возникновением или исчезновением связей, связывающих заданную \textit{временную сущность} с другими сущностями, либо возникновением или исчезновением связей, связывающих части указанной \textit{временной сущности} с другими сущностями.\\
		Многим \textbf{\textit{процессам}} можно поставить в соответствие \textit{ситуацию}, которая является его \textit{начальной ситуацией*} и \textit{ситуацию}, которая является его \textit{конечной ситуацией*}, то есть указать \textit{ситуации}, переход между которыми осуществляется в ходе \textbf{\textit{процесса}}.\\
		При этом знаки тех \textit{временных сущностей}, с которыми связаны изменения, описываемые некоторым \textbf{\textit{процессом}}, входят в данный \textbf{\textit{процесс}} как элементы и при необходимости уточняются дополнительными \textit{ролевыми отношениями}.}
	
	\scntext{примечание}{Каждой \textbf{\textit{материальной сущности}} можно поставить в соответствие различные \textit{процессы}, описывающие ее преобразование из одного состояния в другое.}
	\scntext{примечание}{Поскольку \textit{процесс} представляет собой изменяющуюся во времени динамическую структуру, то полностью представить процесс в базе знаний в общем случае не представляется возможным. Однако, можно ввести sc-элемент, обозначающий конкретный процесс, с необходимой степенью детализации описать его декомпозицию на более частные подпроцессы и/или описать ситуации, соответствующие состояниям процесса в разные моменты времени. В данном случае можно провести некоторую аналогию с \textit{бесконечными множествами}, все элементы которых физически не могут быть представлены в базе знаний одновременно, тем не менее, само множество и некоторые из его элементов могут быть описаны с необходимой степенью детализации.}
	
	\scnheader{воздействие}
	\scnidtf{процесс, осуществляющийся на основе (в результате) воздействия одной сущности на другую}
	\scnrelfrom{включение}{действие}
	\scntext{пояснение}{Каждому \textbf{\textit{воздействию}} может быть поставлена в соответствие (1) некоторая \textit{воздействующая сущность*}, т.е. сущность, осуществляющая \textbf{\textit{воздействие}} (в частности, это может быть некоторое физическое поле), и (2) некоторый \textit{объект воздействия*}, т.е. сущность, на которую воздействие направлено. Если \textbf{\textit{воздействие}} связано с \textit{материальной сущностью}, то его объектом воздействия является либо сама эта \textit{материальная сущность}, либо некоторая ее пространственная часть.}
	
	
	\scnheader{прошлая сущность}
	\scnidtf{сущность, существовавшая в прошлом времени}
	\scnidtf{сущность прошлого времени}
	\scnidtf{сущность, завершившая свое существование}
	
	\scnheader{настоящая сущность}
	\scnidtf{сущность, существующая в текущий момент времени}
	\scnidtf{сущность, существующая сейчас}
	\scnidtf{сущность настоящего времени}
	
	\scnheader{будущая сущность}
	\scnidtf{возможно будущая сущность}
	\scnidtf{прогнозируемая временная сущность}
	\scnidtf{временная сущность, которая может существовать в будущем}
	\scnidtf{сущность, которая может или должна начать свое существование в будущем времени}
	\scnrelfrom{включение}{инициированное действие}
	\scntext{пояснение}{Каждой \textbf{\textit{будущей сущности}} можно поставить в соответствие вероятность ее возникновения.}
	
	
	\scnheader{основные отношения между материальными сущностями}
	\begin{scnrelfromlist}{разбиение}
		\scnitem{является частью*}
		\begin{scnindent}
			\scnidtf{отношение указывает на то, что один объект является частью другого объекта}
		\end{scnindent}
		\scnitem{состоит из*}
		\begin{scnindent}
			\scnidtf{отношение указывает на то, из каких материалов или компонентов состоит объект}
		\end{scnindent}
		\scnitem{находится в*}
		\begin{scnindent}
			\scnidtf{ отношение указывает на местоположение объекта относительно другого объекта}
		\end{scnindent}
		\scnitem{связано с*}
		\begin{scnindent}
			\scnidtf{отношение указывает на любую связь между двумя объектами, которая не покрывается другими отношениями}
		\end{scnindent}
		\scnitem{имеет свойство*}
		\begin{scnindent}
			\scnidtf{отношение указывает на то, что объект обладает определенным свойством}
		\end{scnindent}
		\scnitem{используется в*}
		\begin{scnindent}
			\scnidtf{отношение указывает на то, в каких процессах или приложениях используется объект}
		\end{scnindent}
	\end{scnrelfromlist}
		\scnendcurrentsectioncomment
\end{SCn}
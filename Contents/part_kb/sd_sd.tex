\begin{SCn}
\scnsectionheader{\currentname}
\begin{scnsubstruct}
\begin{scnreltovector}{конкатенация сегментов}

\scnitem{Что такое предметная область}
\scnitem{Роли знаков, входящих в состав предметных областей}
\scnitem{Типология предметных областей и отношения, заданных на множестве предметных областей}
\scnitem{Что такое sc-язык}

\end{scnreltovector}
\scnheader{Предметная область предметных областей}
\scnidtf{Предметная область, объектами исследования которой являются предметные области}
\scntext{explanation}{В состав \textbf{\textit{Предметной области предметных областей}} входят структурные спецификации всех \textit{предметных областей}, входящих в состав базы знаний \textit{ostis-системы}, в том числе, самой \textbf{\textit{Предметной области предметных областей}}. Таким образом, \textbf{\textit{Предметная область предметных областей}} является, во-первых, \textit{рефлексивным множеством}, во-вторых, рефлексивной предметной областью, то есть \textit{предметной областью}, одним из объектов исследования которой является она сама.}\scniselement{рефлексивное множество}
\begin{scnhaselementrole}{класс объектов исследования} предметная область
\end{scnhaselementrole}
\begin{scnhaselementrolelist}{класс объектов исследования}

статическая предметная область;динамическая предметная область;понятие;sc-язык

\end{scnhaselementrolelist}
\begin{scnhaselementrolelist}{исследуемое отношение}

понятие предметной области\scnrolesign ;исследуемое понятие\scnrolesign ;максимальный класс объектов исследования\scnrolesign ;немаксимальный класс объектов исследования\scnrolesign ;исследуемый класс первичных элементов\scnrolesign ;исследуемое отношение\scnrolesign ;класс исследуемых структур\scnrolesign ;понятие, исследуемое в дочерней предметной области\scnrolesign ;понятие, исследуемое в материнской предметной области\scnrolesign ;вспомогательное понятие\scnrolesign ;дочерняя предметная область*;дочерняя предметная область по классу первичных элементов*;дочерняя предметная область по исследуемым отношениям*;предметная область sc-языка*

\end{scnhaselementrolelist}
\scnsegmentheader{Что такое предметная область}
\begin{scnsubstruct}
\scnheader{предметная область}
\scnidtf{sc-модель предметной области}
\scnidtf{sc-текст предметной области}
\scnidtf{sc-граф предметной области}
\scnidtf{представление предметной области в \textit{SC-коде}}
\scnsubset{знание}
\scnsubset{бесконечное множество}
\scntext{explanation}{\textbf{\textit{предметная область}} -- это результат интеграции (объединения) частичных семантических окрестностей, описывающих все исследуемые сущности заданного класса и имеющих одинаковый (общий) предмет исследования (то есть один и тот же набор отношений, которым должны принадлежать связки, входящие в состав интегрируемых семантических окрестностей).\textbf{\textit{предметная область}} -- \textit{структура}, в состав которой входят:\begin{scnitemize}
\item \textnormal{основные исследуемые (описываемые) объекты -- первичные и вторичные;}\item \textnormal{различные классы исследуемых объектов;}\item \textnormal{различные связки, компонентами которых являются исследуемые объекты (как первичные, так и вторичные), а также, возможно, другие такие связки -- то есть связки (как и объекты исследования) могут иметь различный структурный уровень;}\item \textnormal{различные классы указанных выше связок (то есть отношения);}\item \textnormal{различные классы объектов, не являющихся ни объектами исследования, ни указанными выше связками, но являющихся компонентами этих связок.}\end{scnitemize}
При этом все классы, объявленные исследуемыми понятиями, должны быть полностью представлены в рамках данной предметной области вместе со своими элементами, элементами элементов и т.д. вплоть до терминальных элементов.Можно говорить о типологии \textbf{\textit{предметных областей}} по разным структурным признакам:
\begin{scnitemize}
\item наличие метасвязей;\item наличие исследуемых структур, входящих в состав предметной области;\item наличие исследуемых (смежных, дополнительных) объектов, которых исследуются в других предметных областях;\end{scnitemize}
Понятие \textbf{\textit{предметной области}} является важнейшим методологическим приемом, позволяющим выделить из всего многообразия исследуемого Мира только определенный класс исследуемых сущностей и только определенное семейство отношений, заданных на указанном классе. То есть осуществляется локализация, фокусирование внимания только на этом, абстрагируясь от всего остального исследуемого Мира.Во всем многообразии \textbf{\textit{предметных областей}} особое место занимают
\begin{scnitemize}
\item \textit{Предметная область предметных областей}, объектами исследования которой являются всевозможные \textbf{\textit{предметные области}}, а предметом исследования -- всевозможные \textit{ролевые отношения}, связывающие предметные области с их элементами, отношения, связывающие предметные области между собой, отношение, связывающее предметные области с их онтологиями\item \textit{Предметная область сущностей}, являющаяся предметной областью самого высокого уровня и задающая базовую семантическую типологию \textit{sc-элементов}(знаков, входящих в тексты \textit{SC-кода})\item Семейство \textbf{\textit{предметных областей}}, каждая из которых задает семантику и синтаксис некоторого \textit{sc-языка}, обеспечивающего представление онтологий соответствующего вида (например, \textit{теоретико-множественных онтологий}, \textit{логических онтологий}, \textit{терминологических онтологий}, \textit{онтологий задач и способов их решения} и т.д.)\item Семейство \textbf{\textit{предметных областей}} верхнего уровня, в которых классами объектов исследования являются весьма крупные
\end{scnitemize}
\end{scnsubstruct}
\end{scnsubstruct}
\end{SCn}


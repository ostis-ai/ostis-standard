\begin{SCn}
\scnsectionheader{\currentname}
\begin{scnsubstruct}
\begin{scnreltovector}{конкатенация сегментов}
\scnitem{Понятие внешнего идентификатора sc-элемента}
\scnitem{Понятие простого идентификатора sc-элемента}
\scnitem{Понятие сложного идентификатора sc-элемента}
\end{scnreltovector}
\scnheader{Предметная область внешних идентификаторов знаков, входящих в информационные конструкции внутреннего языка ostis-систем}
\scniselement{предметная область}
\begin{scnhaselementrole}{класс объектов исследования}
sc-идентификатор\end{scnhaselementrole}
\begin{scnhaselementrolelist}{класс объектов исследования}
основной sc-идентификатор;строковый sc-идентификатор;системный sc-идентификатор;нетранслируемый sc-идентификатор;простой sc-идентификатор;простой строковый sc-идентификатор;имя нарицательное;имя собственное;sc-выражение;ограничитель sc-выражений
\end{scnhaselementrolelist}
\begin{scnhaselementrolelist}{исследуемое отношение}
sc-идентификатор*
\end{scnhaselementrolelist}
\scnsegmentheader{Понятие внешнего идентификатора sc-элемента}
\begin{scnsubstruct}
\scnheader{sc-идентификатор}
\scnidtf{строка символов или пиктограмма, взаимно однозначно представляющая соответствующий sc-элемент, хранимый в sc-памяти}
\scnidtf{внешний идентификатор sc-элемента}
\scntext{пояснение}{Внешние идентификаторы \textit{sc-элементов} (или, сокращенно \scnkeyword{sc-идентификаторы}
) необходимы \mbox{\textit{ostis-системе}} исключительно для того, чтобы осуществлять обмен информацией с другими \textit{ostis-системами} или со своими пользователями. Для того чтобы представить свою \textit{базу знаний}, решать самые различные \textit{задачи}, связанные с анализом текущего состояния и эволюцией своей \textit{базы знаний}, задачи, связанные с анализом текущего состояния (текущих ситуаций) окружающей среды, принятием соответствующих решений (целей) и организацией соответствующих \textit{действий}, направленных на выполнение принятых решений (на достижение поставленных целей), \textit{ostis-системе} не нужны никакие внешние идентификаторы (в частности, имена) соответствующие \textit{sc-элементам}. Но для \uline{понимания} сообщений, принимаемых от других субъектов (что для \textit{ostis-системы} означает построение \textit{sc-текста},~~ \textit{семантически эквивалентного} принятому сообщению) и для анализа сообщений, передаваемых другим субъектам (что для \textit{ostis-системы} означает синтез \textit{внешнего текста},~~\textit{семантически эквивалентного} заданному \textit{sc-тексту} и удовлетворяющего некоторым дополнительным требованиям, например, эмоционального характера) \textit{ostis-системе} необходимо знать, как в принимаемом или передаваемом сообщении изображаются (представляются) \textit{знаки}, \uline{синонимичные sc-элементам}, которые уже хранятся или могут храниться в составе \textit{базы знаний}~~\textit{ostis-системы}. В качестве внешних идентификаторов \textit{sc-элементов} чаще всего используются имена (термины) соответствующих (обозначаемых) сущностей, представленные отдельными словами или словосочетаниями на различных естественных языках, но также могут использоваться иероглифы, условные обозначения, пиктограммы.В общем случае \textit{sc-элементу} может соответствовать несколько синонимичных ему имен на разных \textit{естественных языках}. Более того, \textit{sc-элементу} может соответствовать несколько синонимичных ему имен на одном и том же \textit{естественном языке}. В этом случае одно из этих имен объявляется как основной внешний идентификатор для соответствующего \textit{sc-элемента} и соответствующего \textit{естественного языка}. Основное требование, предъявляемое к таким внешним идентификаторам это отсутствие как синонимов, так и омонимов в рамках множества основных внешних идентификаторов sc-элементов для каждого естественного языка. Каждый внешний идентификатор \textit{sc-элемента}, используемый ostis-системой, может быть описан (представлен) в её памяти в виде \textit{внутреннего файла ostis-системы}, т.е. в виде электронного образа всевозможных вхождений данного внешнего идентификатора во всевозможные внешние тексты соответствующего внешнего языка. В некоторых случаях явное представление в памяти не требуется, например, в случае \textit{нетранслируемых sc-идентификаторов}.}
\begin{scnsubdividing}
\scnitem{простой sc-идентификатор\\\scnidtf{простой внешний идентификатор sc-элемента}
}
\scnitem{sc-выражение\\\scnidtf{сложный внешний идентификатор sc-элемента, в состав которого входит один или несколько идентификаторов других sc-элементов}
}
\end{scnsubdividing}
\begin{scnsubdividing}
\scnitem{основной sc-идентификатор\\\scnidtf{основной sc-идентификатор для носителей дополнительно указываемого языка общения (например, соответствующего естественного языка)}
\scntext{примечание}{\textit{основной sc-идентификатор} является уникальным (не имеет синонимов и омонимов) в рамках соответствующего естественного языка}
\scnsuperset{основной международный sc-идентификатор}
\scntext{примечание}{В качестве \textit{основных sc-идентификаторов} могут использоваться также общепринятые международные условные обозначения некоторых сущностей, например, обозначения часто используемых функций (sin, cos, tg, log, и т.д.), единиц измерения, денежных единиц и многое другое. Формально каждый основной международный sc-идентификатор считается основным sc-идентификатором также и для каждого естественного языка, несмотря на то, что символы, используемые в основных международных sc-идентификаторах, могут не соответствовать алфавиту некоторых или даже всех естественных языков.}
}
\scnitem{неосновной sc-идентификатор\\\scntext{примечание}{С помощью неосновных sc-идентификаторов указываются возможные \textit{синонимы*} соответствующего \textit{основного sc-идентификатора}, которые в частности, могут пояснять или даже определять обозначаемую сущность, указывает на важные свойства этой сущности.}
\scnsuperset{\normalfont(}
неосновной sc-идентификатор $\cap$ пояснение\normalfont)}
\end{scnsubdividing}

\end{scnsubstruct}

\end{scnsubstruct}

\end{SCn}
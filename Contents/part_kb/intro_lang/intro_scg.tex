\begin{SCn}
\scnsectionheader{\currentname}
\begin{scnsubstruct}
\scnidtf{Описание SCg-кода}
\begin{scnreltovector}{конкатенация сегментов}
\scnitem{Основные положения языка графического представления знаний ostis-систем}
\scnitem{Описание Ядра SCg-кода}
\scnitem{Описание Первого расширения Ядра SCg-кода}
\scnitem{Описание Второго расширения Ядра SCg-кода}
\scnitem{Описание Третьего расширения Ядра SCg-кода}
\scnitem{Описание Четвертого расширения Ядра SCg-кода}
\scnitem{Описание Пятого расширения Ядра SCg-кода}
\scnitem{Описание Шестого расширения Ядра SCg-кода}
\scnitem{Описание Седьмого расширения Ядра SCg-кода}
\end{scnreltovector}
\scnsegmentheader{Основные положения языка графического представления знаний ostis-систем}
\begin{scnsubstruct}
\scnheader{SCg-код}
\scnidtf{Semantic Code graphical}
\scnidtf{Язык визуального (графического) представления баз знаний ostis-систем}
\scniselement{графовый язык}
\scntext{explanation}{\textit{SCg-код} представляет собой способ визуализации \textit{sc-текстов} (информационных конструкций SC-кода) в виде рисунков этих абстрактных конструкций. Подчеркнем, что абстрактная \textit{графовая структура} и её рисунок (графическое изображение) -- это не одно и то же даже если они \textit{изоморфны} друг другу. \mbox{\textit{SCg-код}} рассматривается нами как объединение \textit{Ядра SCg-кода}, обеспечивающего изоморфное графическое изображение любого \textit{sc-текста}, а также нескольких направлений расширения этого ядра, обеспечивающих повышение компактности и читабельности}
\end{scnsubstruct}
\end{scnsubstruct}
\end{SCn}
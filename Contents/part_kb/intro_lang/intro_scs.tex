\begin{SCn}
\scnsectionheader{\currentname}
\begin{scnsubstruct}
\scnidtf{Описание \textit{SCs-кода}}
\begin{scnreltovector}{конкатенация сегментов}
\scnitem{Описание Алфавита SCs-кода}
\scnitem{Описание sc.s-разделителей и sc.s-ограничителей}
\scnitem{Описание sc.s-предложений}
\scnitem{Описание Ядра SCs-кода и различных направлений его расширения}
\end{scnreltovector}
\scnheader{SCs-код}
\scnidtf{Semantic Code string}
\scnidtf{Язык линейного представления знаний ostis-систем}
\scnidtf{Множество всевозможных текстов \textit{SCs-кода}}
\scnidtf{Тексты \textit{SCs-кода}}
\scniselement{имя собственное}
\scnidtf{текст \textit{SCs-кода}}
\scniselement{имя нарицательное}
\scnidtf{sc.s-текст}
\scniselement{линейный язык}
\scnrelfrom{алфавит}{Алфавит SCs-кода}
\scnrelfrom{разделители}{sc.s-разделитель}
\scnrelfrom{ограничители}{sc.s-ограничитель}
\scnrelfrom{предложения}{sc.s-предложение}
\scnrelfrom{неоднозначные обозначения описываемых сущностей}{неоднозначное sc.s-изображение sc-элемента}
\scnidtftext{explanation}{Множество линейных текстов (\textit{sc.s-текстов}), каждый из которых состоит из предложений (\textit{sc.s-предложений}), разделенных друг от друга двойной \textit{точкой с запятой} (разделителем \textit{sc.s-предложений}). При этом \mbox{\textit{sc.s-предложение}} представляет собой последовательность \textit{sc-идентификаторов}, являющихся именами описываемых \textit{сущностей} и разделяемых между собой различными \textit{sc.s-разделителями} и \textit{sc.s-ограничителями}}
\scnheader{неоднозначное sc.s-изображение sc-элемента}
\scnrelboth{пара пересекающихся множеств}{sc-выражение}
\scnidtf{условное обозначение неименуемой (неидентифицируемой) сущности}
\scnsuperset{sc.s-коннектор}
\scnidtf{неоднозначное sc.s-изображение \textit{sc-коннектора}, являющееся также одновременно одним из видов \textit{sc.s-разделителей}}
\scnsubset{sc.s-разделитель}
\scnsuperset{неоднозначное sc.s-изображение sc-узла}
\scnsuperset{условное обозначение неименуемого множества sc-элементов}
\scntext{explanation}{условное обозначение неименуемого множества sc-элементов в \textit{SCs-коде} представляется строкой из двух символов -- \textit{левой фигурной скобки} и \textit{правой фигурной скобки}.}\scnsuperset{условное обозначение неименуемого кортежа sc-элементов}
\scntext{explanation}{В \textit{SCs-коде} такое обозначение представляется двух-символьной \textit{строкой}, состоящей из \textit{левой угловой скобки} и \textit{правой угловой скобки}}\newpage\scnsuperset{условное обозначение неименуемого файла-экземпляра ostis-системы}
\scntext{explanation}{В \textit{SCs-коде} такое обозначение представляется двух-символьной \textit{строкой}, состоящей из \textit{левой квадратной скобки} и \textit{правой квадратной скобки}}\scnsuperset{условное обозначение неименуемого файла-образца ostis-системы}
\scntext{explanation}{В \textit{SCs-коде} такое обозначение представляется \textit{строкой}, состоящей из \textit{восклицательного знака}, \textit{левой квадратной скобки}, \textit{правой квадратной скобки} и еще одного \textit{восклицательного знака}}\scnsegmentheader{Описание Алфавита SCs-кода}
\begin{scnsubstruct}
\scnheader{Алфавит SCs-кода}
\scnidtf{Алфавит символов SCs-кода}
\scnidtf{множество символов SCs-кода}
\scnidtf{символ, используемый в текстах SCs-кода}
\begin{scnreltoset}{объединение}
\scnitem{Алфавит символов, используемых в sc.s-разделителях}
\scnitem{Алфавит символов, используемых в sc.s-ограничителях}
\scnitem{Алфавит символов, используемых в sc-идентификаторах\\\begin{scnreltoset}{объединение}
\scnitem{Алфавит символов, используемых в простых строковых sc-идентификаторах}
\scnitem{Алфавит символов, используемых в sc-выражениях}
\end{scnreltoset}
}
\scnitem{Алфавит символов, используемых в неоднозначных sc.s-изображениях sc-узлов}
\end{scnreltoset}
\begin{scnrelfromlist}{принципы}
\scnitem{\scnfileitem{Алфавит SCs-кода строится на основе современных общепринятых наборов символов, что позволяет упростить разработку средств для работы с sc.s-текстами с использованием современных технологий.}
}
\scnitem{\scnfileitem{В состав sc.s-текстов, как и в состав текстов любых других языков, являющихся вариантами внешнего отображения текстов SC-кода, могут входить различные файлы, в том числе естественно-языковые или даже файлы, содержащие другие sc.s-тексты. В общем случае в таких файлах могут использоваться самые разные символы, в связи с чем будем считать, что в Алфавит SCs-кода эти символы не включаются.}
}
\end{scnrelfromlist}
\scnheader{Алфавит символов, используемых в sc.s-разделителях}
\begin{scnhaselementrolelist}
\scnitem{\textit{пробел}}
\scnitem{ \textit{точка с запятой}}
\scnitem{ \textit{двоеточие}}
\scnitem{ \textit{круглый маркер}}
\scnitem{ \textit{знак равенства}}
\end{scnhaselementrolelist}
\scnsuperset{Алфавит символов, используемых в sc.s-разделителях, изображающих связь инцидентности sc-элементов}
\begin{scnhaselementrolelist}
\scnitem{\scnfileclass{<}
}
\scnitem{~\scnfileclass{>}
}
\scnitem{~\scnfileclass{|}
}
\scnitem{~\scnfileclass{-}
}
\end{scnhaselementrolelist}
\scnsuperset{Алфавит символов, используемых в sc.s-коннекторах}
\scnsuperset{Расширенный алфавит символов, используемых в sc.s-коннекторах}
\scnidtf{Расширенный алфавит sc.s-коннекторов}
\begin{scnhaselementrolelist}
\scnitem{\scnfileclass{$\in$}
}
\scnitem{~\scnfileclass{$\ni$}
}
\scnitem{~\scnfileclass{$\notin$}
}
\scnitem{~\scnfileclass{$\not \ni$}
}
\scnitem{~\scnfileclass{$\subseteq$}
}
\scnitem{~\scnfileclass{$\supseteq$}
}
\scnitem{~\scnfileclass{$\subset$}
}
\scnitem{~\scnfileclass{$\supset$}
}
\scnitem{~\scnfileclass{$\leq$}
}
\scnitem{~\scnfileclass{$\geq$}
}
\scnitem{~\scnfileclass{$\Leftarrow$}
}
\scnitem{~\scnfileclass{$\Rightarrow$}
}
\scnitem{~\scnfileclass{$\Leftrightarrow$}
}
\scnitem{~\scnfileclass{$\leftarrow$}
}
\scnitem{~\scnfileclass{$\rightarrow$}
}
\scnitem{~\scnfileclass{$\leftrightarrow$}
}
\end{scnhaselementrolelist}
\scnsuperset{Базовый алфавит символов, используемых в sc.s-коннекторах}
\scnidtf{Базовый алфавит sc.s-коннекторов}
\begin{scnhaselementrolelist}
\scnitem{\scnfileclass{$\sim$}
}
\scnitem{~\textit{знак подчеркивания}}
\scnitem{~ \textit{знак равенства}}
\scnitem{~\scnfileclass{>}
}
\scnitem{~ \scnfileclass{<}
}
\scnitem{~\textit{двоеточие}}
\scnitem{~\scnfileclass{-}
}
\scnitem{~\scnfileclass{|}
}
\scnitem{~\scnfileclass{/}
}
\end{scnhaselementrolelist}
\scntext{note}{Как в Базовом, так и в Расширенном Алфавитах sc.s-коннекторов используются следующие общие признаки, характеризующие тип изображаемого sc-коннектора:\begin{scnitemize}
\item \textit{знак подчеркивания} как признак изображений переменных sc-коннекторов (один  \textit{знак подчеркивания} для sc-коннекторов, являющихся первичными sc-переменными, два  \textit{знака подчеркивания} для sc-коннекторов, являющихся вторичными sc-переменными (sc-метапеременными));\item \textit{вертикальная черта} (|) как признак изображений негативных sc-дуг принадлежности;\item \textit{косая черта} (/) как признак изображений нечетких sc-дуг принадлежности;\item \textit{тильда} ($\sim$) как признак изображений временных sc-дуг принадлежности\end{scnitemize}
При необходимости комбинации указанных признаков перечисленные символы комбинируются так, как показано в сегменте \textit{Описание sc.s-разделителей и sc.s-ограничителей}.}\bigskip\begin{scnset}
\scnitem{Расширенный алфавит символов, используемых в sc.s-коннекторах}
\scnitem{ Базовый алфавит символов, используемых в sc.s-коннекторах}
\end{scnset}
\scntext{explanation}{Для упрощения процесса разработки исходных текстов баз знаний с использованием SCs-кода и создания соответствующих средств вводятся два алфавита символов. \textit{Базовый алфавит символов, используемых в sc.s-коннекторах} включает только символы, входящие в переносимый набор символов (portable character set) и имеющиеся на стандартной современной клавиатуре. Таким образом, для разработки исходных текстов баз знаний, использующих только \textit{Базовый алфавит символов, используемых в sc.s-коннекторах} достаточно обычного текстового редактора. \textit{Расширенный алфавит символов, используемых в sc.s-коннекторах} включает также дополнительные символы, которые позволяют сделать sc.s-тексты (и sc.n-тексты) более читабельными и наглядными. Для визуализации и разработки sc.s-текстов с использованием расширенного алфавита требуется наличие специализированных средств.}\scnheader{Алфавит символов, используемых в sc.s-ограничителях}
\begin{scnhaselementrolelist}
\scnitem{\scnfileclass{(}
}
\scnitem{~\scnfileclass{)}
}
\scnitem{~\scnfileclass{*}
}
\end{scnhaselementrolelist}
\scnheader{Алфавит символов, используемых в неоднозначных sc.s-изображениях sc-узлов}
\begin{scnhaselementrolelist}
\scnitem{\scnfileclass{\}
\scnitem{~\scnfileclass{\}
}
}
\scnitem{~\scnfileclass{-}
}
\scnitem{~\scnfileclass{!}
}
\scnitem{~\scnfileclass{}
}
\scnitem{~\scnfileclass{}
}
\end{scnhaselementrolelist}
\end{scnsubstruct}
\end{scnsubstruct}
\end{SCn}
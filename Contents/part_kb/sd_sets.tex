\begin{SCn}
\scnsectionheader{\currentname}
\begin{scnsubstruct}
\scnheader{Предметная область множеств}
\scnidtf{Теоретико-множественная предметная область}
\scnidtf{Предметная область теории множеств}
\scnidtf{Предметная область, объектами исследования которой являются множества}
\scniselement{предметная область}
\begin{scnhaselementrole}{класс объектов исследования}
множество
\end{scnhaselementrole}
\begin{scnhaselementrolelist}{класс объектов исследования}
\begin{itemize}
    \item\textit{конечное множество}
    \item\textit{бесконечное множество}
    \item\textit{счетное множество}
    \item\textit{несчетное множество}
    \item\textit{множество без кратных элементов}
    \item\textit{мультимножество}
    \item\textit{кратность принадлежности}
    \item\textit{класс}
    \item\textit{класс первичных sc-элементов}
    \item\textit{класс множеств}
    \item\textit{класс структур}
    \item\textit{класс классов}
    \item\textit{нечеткое множество}
    \item\textit{четкое множество}
    \item\textit{множество первичных сущностей}
    \item\textit{семейство множеств}
    \item\textit{нерефлексивное множество}
    \item\textit{рефлексивное множество}
    \item\textit{множество первичных сущностей и множеств}
    \item\textit{сформированное множество}
    \item\textit{несформированное множество}
    \item\textit{пустое множество}
    \item\textit{синглетон}
    \item\textit{пара}
    \item\textit{пара разных элементов}
    \item\textit{пара-мультимножество}
    \item\textit{тройка}
    \item\textit{кортеж}
    \item\textit{декартово произведение}
    \item\textit{булеан}
    \item\textit{мощность множества}
\end{itemize}
\end{scnhaselementrolelist}
\begin{scnhaselementrolelist}{исследуемое отношение}
\begin{itemize}
    \item\textit{принадлежность*}
    \item\textit{пример\scnrolesign}
    \item\textit{включение*}
    \item\textit{строгое включение*}
    \item\textit{объединение*}
    \item\textit{разбиение*}
    \item\textit{пересечение*}
    \item\textit{пара пересекающихся множеств*}
    \item\textit{попарно пересекающиеся множества*}
    \item\textit{пересекающиеся множества*}
    \item\textit{пара непересекающихся множеств*}\item\textit{попарно непересекающиеся множества*}\item\textit{непересекающиеся множества*}\item\textit{разность множеств*}\item\textit{симметрическая разность множеств*}\item\textit{декартово произведение*}\item\textit{семейство подмножеств*}\item\textit{булеан*}\item\textit{равенство множеств*}    
\end{itemize}
\end{scnhaselementrolelist}
\scnheader{множество}
\scnidtf{множество sc-элементов}
\scnidtf{sc-множество}
\scnidtf{множество знаков}
\scnidtf{множество знаков описываемых сущностей}
\scnidtf{семантически нормализованное множество}
\scnidtf{sc-знак множества sc-элементов}
\scnidtf{sc-знак множества sc-знаков}
\scnidtf{sc-текст}
\scnidtf{текст SC-кода}
\scnidtf{SC-код}
\begin{scnsubdividing}
\scnitem{конечное множество}
\scnitem{бесконечное множество}
\end{scnsubdividing}
\begin{scnsubdividing}
\scnitem{множество без кратных элементов}
\scnitem{мультимножество}
\end{scnsubdividing}
\begin{scnsubdividing}
\scnitem{связка}
\scnitem{класс\\\scnidtf{sc-элемент, обозначающий класс sc-элементов}
\scnidtf{sc-знак множества sc-элементов, эквивалентных в том или ином смысле}
}
\scnitem{структура\\\scnidtf{sc-знак множества sc-элементов, в состав которого входят sc-связки или структуры, связывающие эти sc-элементы}
}
\end{scnsubdividing}
\begin{scnsubdividing}
\scnitem{четкое множество}
\scnitem{нечеткое множество}
\end{scnsubdividing}
\begin{scnsubdividing}
\scnitem{множество первичных сущностей}
\scnitem{множество множеств}
\scnitem{множество первичных сущностей и множеств}
\end{scnsubdividing}
\begin{scnsubdividing}
\scnitem{рефлексивное множество}
\scnitem{нерефлексивное множество}
\end{scnsubdividing}
\begin{scnsubdividing}
\scnitem{сформированное множество}
\scnitem{несформированное множество}
\end{scnsubdividing}
\begin{scnsubdividing}
\scnitem{кортеж}
\scnitem{неориентированное множество}
\end{scnsubdividing}
\scnsuperset{пустое множество}
\scnsuperset{синглетон}
\scnsuperset{пара}
\scnsuperset{тройка}
\scntext{explanation}{Под \textbf{\textit{множеством}} понимается соединение в некое целое M определённых хорошо различимых предметов m нашего созерцания или нашего мышления (которые будут называться элементами множества M). \textbf{\textit{множество}}  мысленная сущность, которая связывает одну или несколько сущностей в целое.Более формально \textbf{\textit{множество}}  это абстрактный математический объект, состоящий из элементов. Связь множеств с их элементами задается бинарным ориентированным отношением \textbf{\textit{принадлежность*}}.\textbf{\textit{множество}} может быть полностью задано следующими тремя способами:\begin{scnitemize}
\item путем перечисления (явного указания) всех его элементов (очевидно, что таким способом можно задать только конечное множество)\item с помощью определяющего высказывания, содержащего описание общего характеристического свойства, которым обладают все те и только те объекты, которые являются элементами (т.е. принадлежат) задаваемого множества.\item с помощью теоретико-множественных операций, позволяющих однозначно задавать новые множества на основе уже заданных (это операции объединения, пересечения, разности множеств и др.)\end{scnitemize}
Для любого семантически ненормализованного \textbf{\textit{множества}} существует единственное семантически нормализованное \textbf{\textit{множество}}, в котором все элементы, не являющиеся знаками множеств, заменены на знаки множеств.}\scnheader{принадлежность*}
\scnidtf{принадлежность элемента множеству*}
\scnidtf{отношение принадлежности элемента множеству*}
\scniselement{бинарное отношение}
\scniselement{ориентированное отношение}
\scntext{explanation}{\textbf{\textit{принадлежность*}}  это бинарное ориентированное отношение, каждая связка которого связывает множество с одним из его элементов. Элементами отношения \textbf{\textit{принадлежность*}} по умолчанию являются \textit{позитивные sc-дуги принадлежности}.}\scnheader{конечное множество}
\scnidtf{множество с конечным числом элементов}
\scntext{explanation}{\textbf{\textit{конечное множество}}  это \textit{множество}, количество элементов которого конечно, т.е. существует неотрицательное целое число \textit{k}, равное количеству элементов этого множества.}\scnheader{бесконечное множество}
\scnidtf{множество с бесконечным числом элементов}
\begin{scnsubdividing}
\scnitem{счетное множество}
\scnitem{несчетное множество}
\end{scnsubdividing}
\scntext{explanation}{\textbf{\textit{бесконечное множество}}  это \textit{множество}, в котором для любого натурального числа \textit{n} найдётся конечное подмножество из \textit{n} элементов.}\scnheader{счетное множество}
\scntext{explanation}{\textbf{\textit{счетное множество}}  это \textit{бесконечное множество}, для которого существует \textit{взаимно-однозначное соответствие} с натуральным рядом чисел.}\scnheader{несчетное множество}
\scnidtf{континуальное множество}
\scntext{explanation}{\textbf{\textit{несчетное множество}} - это \textit{бесконечное множество}, элементы которого невозможно пронумеровать натуральными числами.}\scnheader{множество без кратных элементов}
\scnidtf{классическое множество}
\scnidtf{канторовское множество}
\scnidtf{множество, состоящее из разных элементов}
\scnidtf{множество без кратного вхождения элементов}
\scnidtf{множество, все элементы которого входят в него однократно}
\scnidtf{множество, не имеющее кратного вхождения элементов}
\scntext{explanation}{\textbf{\textit{множество без кратных элементов}} - это \textit{множество}, для каждого элемента которого существует только одна пара принадлежности, выходящая из знака этого множества в указанный элемент.}\scnheader{мультимножество}
\scnidtf{множество, имеющее кратные вхождения хотя бы одного элемента}
\scnidtf{множество, по крайней мере один элемент которого входит в его состав многократно}
\scntext{explanation}{\textbf{\textit{мультимножество}} - это \textit{множество}, для которого существует хотя бы одна кратная пара принадлежности, выходящая из знака этого множества.}\scnheader{кратность принадлежности}
\scnidtf{кратность принадлежности элемента}
\scnidtf{кратность вхождения элемента во множество}
\scniselement{параметр}
\scntext{explanation}{\textbf{\textit{кратность принадлежности}} - \textit{параметр}, значением которого являются числовые величины, показывающие сколько раз входит тот или иной элемент в рассматриваемое множество. Элементами данного параметра являются классы \textit{позитивных sc-дуг принадлежности}, связывающих данное множество с элементом, кратность вхождения которого в данное множество мы хотим задать.Таким образом, кратное вхождение элемента в мультимножество может быть задано как явным указанием \textit{позитивных sc-дуг принадлежности} этого элемента данному \textit{множеству}, так и склеиванием этих дуг в одну и включением ее в некоторый класс \textbf{\textit{кратности принадлежности}}.}\scnrelfrom{описание примера}{\scnfileimage[20em]{figures/sd_sets/multiplicityOfMembership.png}
}
\scnheader{класс}
\scnidtf{класс sc-элементов}
\begin{scnsubdividing}
\scnitem{класс первичных sc-элементов}
\scnitem{класс множеств}
\end{scnsubdividing}
\scntext{explanation}{\textbf{\textit{класс}}  множество элементов, обладающих какими-либо явно указываемыми общими свойствами.}\scnheader{класс первичных sc-элементов}
\scntext{explanation}{\textbf{\textit{класс первичных sc-элементов}}  класс, элементами которого являются только \textit{sc-элементы}, не являющиеся знаками множеств.}\scnheader{класс множеств}
\begin{scnsubdividing}
\scnitem{отношение}
\scnitem{класс структур}
\scnitem{класс классов}
\end{scnsubdividing}
\scntext{explanation}{\textbf{\textit{класс множеств}}  класс, элементами которого являются только \textit{sc-элементы}, являющиеся знаками множеств.}\scnheader{класс структур}
\scntext{explanation}{\textbf{\textit{класс структур}}  класс, элементами которого являются \textit{структуры}.}\scnheader{класс классов}
\scntext{explanation}{\textbf{\textit{класс классов}}  класс, элементами которого являются \textit{классы}.}\scnheader{нечеткое множество}
\scntext{explanation}{\textbf{\textit{нечеткое множество}}  это \textit{множество}, которое представляет собой совокупность элементов произвольной природы, относительно которых нельзя точно утверждать  обладают ли эти элементы некоторым характеристическим свойством, которое используется для задания этого нечеткого множества. Принадлежность элементов такому множеству указывается при помощи \textit{нечетких позитивных sc-дуг принадлежности}.}\scnheader{четкое множество}
\scntext{explanation}{\textbf{\textit{четкое множество}}  это \textit{множество}, принадлежность элементов которому достоверна и указывается при помощи \textit{четких позитивных sc-дуг принадлежности}.}\scnheader{множество первичных сущностей}
\scnsuperset{класс первичных сущностей}
\scnsubset{множество}
\scntext{explanation}{\textbf{\textit{множество первичных сущностей}}  это \textit{множество}, элементы которого не являются знаками множеств.}\scnheader{семейство множеств}
\scnidtf{множество множеств}
\scnsuperset{класс классов}
\scntext{explanation}{\textbf{\textit{семейство множеств}}  это \textit{множество}, элементами которого являются знаки множеств.}\scnheader{нерефлексивное множество}
\scntext{explanation}{\textbf{\textit{нерефлексивное множеств}}  это \textit{множество}, знак которого не является элементом этого множества}\scnheader{рефлексивное множество}
\scntext{explanation}{\textbf{\textit{рефлексивное множеств}}  это \textit{множество}, знак которого является элементом этого множества}\scnheader{множество первичных сущностей и множеств}
\scntext{explanation}{\textbf{\textit{множество первичных сущностей и множеств}}  это \textit{множество}, элементами которого являются как знаки множеств, так и знаки сущностей, не являющихся множествами.}\scnheader{сформированное множество}
\scniselement{ситуативное множество}
\scntext{explanation}{\textbf{\textit{сформированное множество }} - это \textit{множество}, все элементы которого известны и перечислены в данный момент.}\scnheader{несформированное множество}
\scniselement{ситуативное множество}
\scntext{explanation}{\textbf{\textit{несформированное множество}} - это \textit{множество}, не все элементы которого известны и перечислены в данный момент.}\scnheader{пустое множество}
\scniselement{мощность множества}
\scntext{explanation}{\textbf{\textit{пустое множество}}  это \textit{множество}, которому не принадлежит ни один элемент.}\scnheader{синглетон}
\scniselement{мощность множества}
\scnidtf{множество мощности 1}
\scnidtf{одноэлементное множество}
\scnidtf{одномощное множество}
\scnidtf{множество, мощность которого равна 1}
\scnidtf{множество, имеющее мощность равную единице}
\scnidtf{синглетон из sc-элемента}
\scnidtf{sc-синглеон}
\scnsubset{конечное множество}
\scntext{explanation}{\textbf{\textit{синглетон}}  это \textit{множество}, состоящее из одного элемента.Другими словами - любое множество \textit{Si} есть \textbf{\textit{синглетон}} тогда и только тогда, когда существует принадлежность этому множеству, которая совпадает с любой принадлежностью этому множеству.}\scnheader{пара}
\scniselement{мощность множества}
\scnidtf{множество мощности два}
\scnidtf{двухэлементное множество}
\scnidtf{двумощное множество}
\scnidtf{множество, мощность которого равна 2}
\scnidtf{sc-пара}
\scnidtf{пара sc-элементов}
\scnsubset{конечное множество}
\begin{scnsubdividing}
\scnitem{пара разных элементов}
\scnitem{пара-мультимножество}
\end{scnsubdividing}
\scntext{explanation}{\textbf{\textit{пара}}  это \textit{множество}, состоящее из двух элементов.Другими словами  любое множество есть \textbf{\textit{пара}} тогда и только тогда, когда существуют две различные принадлежности этому множеству такие, что любая принадлежность этому множеству совпадает с одной из них.}\scnheader{пара разных элементов}
\scnidtf{канторовская пара}
\scnidtf{канторовская пара sc-элементов}
\scnidtf{канторовское двумощное множество}
\scnheader{пара-мультимножество}
\scnidtf{пара-петля}
\scnidtf{sc-петля}
\scnidtf{двумощное мультимножество}
\scnheader{тройка}
\scniselement{мощность множества}
\scnidtf{тройка}
\scnidtf{sc-тройка}
\scnidtf{множество мощности три}
\scnidtf{множество, мощность которого равна 3}
\scnsubset{конечное множество}
\scntext{explanation}{\textbf{\textit{тройка}}  это \textit{множество}, состоящее из трех элементов.Другими словами  любое множество есть \textbf{\textit{тройка}} тогда и только тогда, когда существуют три различные принадлежности этому множеству такие, что любая принадлежность этому множеству совпадает с одной из них.}\scnheader{кортеж}
\scnidtf{вектор}
\scntext{explanation}{\textbf{\textit{кортеж}}  это множество, представляющее собой упорядоченный набор элементов, т.е. такое множество, порядок элементов в котором имеет значение. Пары принадлежности элементов \textbf{\textit{кортежу}} могут дополнительно принадлежать каким-либо \textit{ролевым отношениям}, при этом, в рамках каждого \textbf{\textit{кортежа}} должен существовать хотя бы один элемент, роль которого дополнительно уточнена \textit{ролевым отношением}.}\scnheader{пример\scnrolesign}
\scnidtf{типичный пример\scnrolesign}
\scnidtf{типичный экземпляр заданного класса\scnrolesign}
\scniselement{ролевое отношение}
\scntext{explanation}{\textbf{\textit{пример\scnrolesign}}  это \textit{ролевое отношение}, связывающее некоторое \textit{множество} с элементом, являющимся примером этого множества.}\scnheader{включение*}
\scnidtf{включение множеств*}
\scnidtf{быть подмножеством*}
\scniselement{бинарное отношение}
\scniselement{ориентированное отношение}
\scniselement{транзитивное отношение}
\scnrelfrom{область определения}{множество}
\scnsuperset{строгое включение*}
\scntext{определение}{\textbf{\textit{включение*}}  это бинарное ориентированное отношение, каждая связка которого связывает два множества. Будем говорить, что \textit{Множество Si} \textbf{\textit{включает*}} в себя \textit{Множество Sj} в том и только том случае, если каждый элемент \textit{Множества Sj} является также и элементом \textit{Множества Si}.}
\scnrelfrom{описание примера}{\scnfileimage[20em]{figures/sd_sets/inclusion.png}
}
\scntext{explanation}{Множество Sj}
\end{scnsubstruct}
\end{SCn}

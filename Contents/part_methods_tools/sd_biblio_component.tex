\begin{SCn}
\scnsectionheader{\currentname}

\scnstartsubstruct

\scnheader{Предметная область многократно используемых компонентов ostis-систем}
\scniselement{предметная область}
\scnsdmainclass{библиотека многократно используемых компонентов ostis-систем;многократно используемый компонент ostis-систем}
\scnsdclass{независимый многократно используемый компонент ostis-систем;зависимый многократно используемый компонент ostis-систем;атомарный многократно используемый компонент ostis-систем;неатомарный многократно используемый компонент ostis-систем; многократно используемый компонент ostis-систем, хранящийся в виде внешних файлов;многократно используемый компонент ostis-систем, хранящийся в виде файлов исходных текстов;многократно используемый компонент ostis-систем, хранящийся в виде бинарных файлов;многократно используемый компонент, хранящийся в базе знаний ostis-системы;хранилище многократно используемого компонента ostis-систем, хранящегося в виде внешних файлов;хранилище многократно используемого компонента ostis-систем, хранящегося в виде файлов исходных текстов;хранилище многократно используемого компонента ostis-систем, хранящегося в виде бинарных файлов; спецификация многократно используемого компонента ostis-систем;отношение, специфицирующее многократно используемый компонент ostis-систем; файл, содержащий url-адрес многократно используемого компонента ostis-систем}
\scnsdrelation{зависимый компонент*;несовместимый компонент*;адрес хранилища*;автор компонента*;установленные компоненты*; доступные к установке компоненты*}
\scnrelfromset{основные положения}{
\scnfileitem{Важнейшим этапом эволюции любой технологии является переход к компонентному проектированию на основе постоянно пополняемый библиотеки многократно используемых компонентов.}
\scnaddlevel{1}
\scnrelfromset{известные проблемы} {
	\scnfileitem{Отсутствие унификации в принципах представления различных видов знаний в рамках одной базы знаний, и, как следствие, отсутствие унификации в принципах выделения и спецификации многократно используемых компонентов приводит к несовместимости компонентов, разработанных в рамках разных проектов.}
	;\scnfileitem{Не ведётся разработка стандартов, обеспечивающих совместимость этих компонентов.}
	;\scnfileitem{Многие компоненты используют для идентификации язык разработчика (как правило, английский), и предполагается, что все пользователи будут использовать этот же язык. Однако для многих приложений это недопустимо – понятные только разработчику идентификаторы должны быть скрыты от конечных пользователей, которые должны быть в состоянии выбрать язык для идентификаторов, которые они видят}
	;\scnfileitem{Отсутствие средств поиска компонентов, удовлетворяющих заданных критериям.}
}
\scnrelfromset{необходимые требования} {
	\scnfileitem{Универсальный язык представления знаний.}
	;\scnfileitem{Универсальная процедура интеграции знаний в рамках указанного языка.}
	;\scnfileitem{Разработка стандарта, обеспечивающего семантическую совместимость интегрируемых знаний (таким стандартом является согласованная система используемых понятий).}
}
\scnaddlevel{-1}
;\scnfileitem{Повторное использование готовых компонентов широко применяется во многих отраслях, связанных с проектирование различного рода систем, поскольку позволяет уменьшить трудоемкость разработки и ее стоимость (путем минимизации количества труда за счет отсутствия необходимости разрабатывать какой-либо компонент), повысить качество создаваемого контента и снизить профессиональные требования к разработчикам компьютерных систем. Использование готовых компонентов предполагает, что распространяемый компонент верифицирован и документирован, а возможные ошибки и ограничения устранены либо специфицированы и известны.}
}

% Использовать \nameref для ПрО (криво отображаются мячики)
\scnheader{семантически смежный раздел*}
\scnhaselementset{Предметная область и онтология комплексной библиотеки многократно используемых семантически совместимых компонентов ostis-систем;Логико-семантическая модель Метасистемы OSTIS\\
}
\scnaddlevel{1}
\scnexplanation{Метасистема IMS.ostis ориентирована на разработку и
	практическое внедрение методов и средств компонентного проектирования семантически совместимых интеллектуальных систем, которая предоставляет возможность быстрого создания интеллектуальных приложений различного назначения. В состав метасистемы OSTIS входит библиотека многократно используемых компонентов OSTIS.\\
	Сферы практического применения технологии компонентного проектирования семантически совместимых интеллектуальных систем ничем не ограничены.}
\scnexplanation{Основу для реализации компонентного подхода в рамках Технологии OSTIS составляет Библиотека многократно используемых компонентов ostis-систем, входящая в состав Метасистемы IMS.ostis.}
\scnaddlevel{-1}

\scnheader{семантически смежный раздел*}
\scnhaselementset{Предметная область и онтология комплексной библиотеки многократно используемых семантически совместимых компонентов ostis-систем;Предметная область и онтология комплексной технологии поддержки жизненного цикла интеллектуальных компьютерных систем нового поколения\\
}
\scnaddlevel{1}
\scnexplanation{Основным требованием, предъявляемым к Технологии OSTIS, является обеспечение возможности совместного использования в рамках ostis-систем различных видов знаний и различных моделей решения задач с возможностью неограниченного расширения перечня используемых в ostis-системе видов знаний и моделей решения задач без существенных трудозатрат. Следствием данного требования является необходимость реализации компонентного подхода на всех уровнях, от простых компонентов баз знаний и решателей задач до целых ostis-систем.
}}
\scnaddlevel{-1}

\scnheader{дочерний раздел*}
\scnhaselementvector{Предметная область и онтология комплексной библиотеки многократно используемых семантически совместимых компонентов ostis-систем;Предметная область и онтология многократно используемых компонентов баз знаний ostis-систем}
\scnaddlevel{1}
\scnexplanation{На сегодняшний день разработано большое число баз знаний по самым различным предметным областям. Однако в большинстве случаев каждая база знаний разрабатывается отдельно и независимо от других, в отсутствие единой унифицированной формальной основы для представления знаний, а также единых принципов формирования систем понятий для описываемой предметной области. В связи с этим разработанные базы оказываются, как правило, несовместимы между собой и не пригодны для повторного использования.	
}}
\scnaddlevel{-1}

\scnheader{дочерний раздел*}
\scnhaselementvector{Предметная область и онтология комплексной библиотеки многократно используемых семантически совместимых компонентов ostis-систем;Предметная область и онтология многократно используемых компонентов решателей задач ostis-систем\\
}
\scnaddlevel{1}
\scnexplanation{В области разработки решателей задач существует большое количество конкретных реализаций, однако вопросы совместимости различных решателей при решении одной задачи практически не рассматриваются.		
}}
\scnaddlevel{-1}

\scnheader{дочерний раздел*}
\scnhaselementvector{Предметная область и онтология комплексной библиотеки многократно используемых семантически совместимых компонентов ostis-систем;Предметная область и онтология многократно используемых компонентов интерфейсов ostis-систем\\
}

\scnheader{библиотека многократно используемых компонентов ostis-систем}
\scntext{сокращение}{библиотека ostis-систем}
\scnidtf{библиотека многократно используемых компонентов OSTIS}
\scnhaselement{\textbf{Библиотека IMS.ostis}}
\scnaddlevel{1}
\scnidtf{Библиотека многократно используемых компонентов ostis-систем в составе Метасистемы IMS.ostis}
\scnidtf{Библиотека OSTIS}
\scnaddlevel{-1}
\scnnote{Разработчики любой ostis-системы могут включить в ее состав библиотеку, которая позволит им накапливать и распространять результаты своей деятельности среди других участников Экосистемы OSTIS в виде многократно используемых компонентов.}
\scnrelfromset{функциональные возможности}{
	\scnfileitem{Хранение многократно используемых компонентов ostis-систем и их спецификаций.}
	\scnaddlevel{1}
	\scnnote{При этом часть компонентов, специфицированных в рамках библиотеки, могут физически храниться в другом месте ввиду особенностей их  технической реализации (например, исходные тексты платформы интерпретации sc-моделей компьютерных систем могут физически храниться в каком-либо отдельном репозитории, но специфицированы как компонент будут в соответствующей библиотеке). В этом случае спецификация компонента в рамках библиотеки должна также включать описание (1) того где располагается компонент и (2) сценария его автоматической или хотя бы ручной установки в дочернюю ostis-систему.}
	%TODO: Перенести к понятию хранилища МИК
	% При этом спецификация компонента хранится как непосредственно рядом с компонентом (в виде исходных текстов или в той же самой базе знаний), так и дублируется в рамках библиотеки. Соответственно, существует процедура публикации спецификации компонента в библиотеке и последующая процедура синхронизации обновленной спецификации компонента с библиотекой.
	\scnaddlevel{-1}
	;\scnfileitem{Просмотр имеющихся компонентов и их спецификаций, а также поиск компонентов по фрагментам их спецификации.}
	;\scnfileitem{Хранение сведений о том, в каких ostis-системах-потребителях какие из компонентов библиотеки и какой версии используются (были скачаны). Это необходимо как минимум для учета востребованности того или иного компонента, оценки его важности и популярности.}
	;\scnfileitem{Обеспечение автоматического обновления компонентов, заимствованных в дочерние ostis-системы. Данная функция может включаться и отключаться по желанию разработчиков дочерней ostis-системы.}
	;\scnfileitem{Хранение сведений о совместимости/несовместимости имеющихся в библиотеке компонентов с учетом версий.}
}
\scnnote{Постоянно расширяемая Библиотека OSTIS существенно сокращает сроки разработки новых интеллектуальных компьютерных систем.}
\scnnote{Библиотека ostis-систем позволяет избавиться от дублирования семантически эквивалентных информационных компонентов.  А также от многообразия форм технической реализации используемых моделей решения задач.}

\scnheader{библиотека многократно используемых компонентов ostis-систем}
\scnreltoset{объединение}{библиотека типовых подсистем ostis-систем;библиотека шаблонов типовых компонентов ostis-систем;библиотека платформ интерпретации sc-моделей компьютерных систем;библиотека многократно используемых компонентов баз знаний; библиотека многократно используемых компонентов решателей задач;библиотека многократно используемых компонентов пользовательских интерфейсов}
\scnrelfromset{обобщенная декомпозиция}{база знаний библиотеки многократно используемых компонентов ostis-систем \\
\scnaddlevel{1}
\scnnote{База знаний библиотеки мнoгократно используемых компонентов ostis-систем представляет собой иерархию многократно используемых компонентов ostis-систем и их спецификаций.}
\scnaddlevel{-1}    
;решатель задач библиотеки многократно используемых компонентов ostis-систем\\
\scnaddlevel{1}
\scnrelfromset{функциональные возможности}{
	\scnfileitem{Систематизация многократно используемых компонентов ostis-систем.}
	;\scnfileitem{Обеспечение версионирования многократно используемых компонентов ostis-систем.}
	;\scnfileitem{Поиск зависимостей и конфликтов между многократно используемыми компонентами в рамках библиотеки компонентов.}
	;\scnfileitem{Формирование отдельных фрагментов многократно используемых компонентов ostis-систем.}}
\scnaddlevel{-1}
;интерфейс библиотеки многократно используемых компонентов ostis-систем\\
\scnaddlevel{1}
\scnnote{Интерфейс обеспечивает доступ к многократно используемым компонентам. Позволяет получить информацию о зависимых, конфликтующих компонентах.}
\scnrelfromset{декомпозиция}{минимальный интерфейс библиотеки многократно используемых компонентов ostis-систем\\
	\scnaddlevel{1}
	\scnnote{Данный вид интерфейса позволяет \textit{менеджеру многократно используемых компонентов ostis-систем}, входящему в состав какой-либо дочерней ostis-системы, подключиться к библиотеке многократно используемых компонентов ostis-систем и использовать ее функциональные возможности, то есть, например, получить доступ к спецификации компонентов и установить выбранные компоненты в дочернюю ostis-систему, получить сведения до доступных версиях компонента, его зависимостях и т.д.}
	\scnaddlevel{-1}
	;расширенный интерфейс библиотеки многократно используемых компонентов ostis-систем
	\scnaddlevel{1}
	\scnidtf{графический интерфейс библиотеки многократно используемых компонентов ostis-систем}
	\scnnote{В частном случае у библиотеки может быть расширенный пользовательский интерфейс, который, в отличие от минимального интерфейса, позволяет не только получить доступ к компонентам для дальнейшей работы с ними, но и просматривать существующую структуру библиотеки,  а также компоненты и их элементы в удобном и интуитивно понятном для пользователя виде.}
	\scnaddlevel{-1}}
\scnaddlevel{-1}}

\scnheader{ostis-система}
\scnsuperset{материнская ostis-система}
\scnaddlevel{1}
\scnexplanation{ostis-система, имеющая в своем составе библиотеку многократно используемых компонентов.}
\scnhaselement{Метасистема IMS.ostis}
\scnnote{Материнская ostis-система в свою очередь может являться дочерней ostis-системой для какой-либо другой ostis-системы, заимствуя компоненты из библиотеки, входящей в состав этой другой ostis-системы.}
\scnaddlevel{-1}
\scnsuperset{дочерняя ostis-система}
\scnaddlevel{1}
\scnexplanation{ostis-система, в составе которой имеется компонент, заимствованный из какой-либо библиотеки многократно используемых компонентов.}
\scnaddlevel{-1}

\scnheader{многократно используемый компонент ostis-систем}
\scnidtf{многократно используемый компонент OSTIS}
\scnrelfrom{аббревиатура}{\scnfilelong{МИК ostis-систем}}
\scnexplanation{Под многократно используемым компонентом ostis-систем понимается компонент некоторой ostis-системы, который может быть использован в рамках другой ostis-системы.}
\scnexplanation{Компонент ostis-системы, который может быть использован в других ostis-системах (дочерних ostis-системах) и содержит все те (и только те) sc-элементы, которые необходимы для функционирования компонента в дочерней ostis-системе.}
\scnexplanation{Компонент некоторой материнской ostis-системы, который может быть использован в некоторой дочерней ostis-системе.}
\scnsubdividing{атомарный многократно используемый компонент ostis-систем;неатомарный многократно используемый компонент ostis-систем}
\scnsubdividing{зависимый многократно используемый компонент ostis-систем;независимый многократно используемый компонент ostis-систем}
\scnsubset{компонент ostis-системы}
\scnaddlevel{1}
\scnexplanation{Целостная часть ostis-системы, которая содержит все те (и только те) sc-элементы, которые необходимы для её функционирования в ostis-системе.}
\scnaddlevel{-1}
\scnrelfrom{разбиение}{\scnkeyword{Типология компонентов ostis-систем по типу хранения\scnsupergroupsign}}
\scnaddlevel{1} 
\scneqtoset{многократно используемый компонент ostis-систем, хранящийся в виде внешних файлов\\
	\scnaddlevel{1}
	\scnsubdividing{многократно используемый компонент ostis-систем, хранящийся в виде файлов исходных текстов;многократно используемый компонент ostis-систем, хранящийся в виде скомпилированных файлов}
	\scnaddlevel{-1}
	;многократно используемый компонент, хранящийся в базе знаний ostis-системы}
\scnaddlevel{1}
\scnnote{На данном этапе развития \textit{Технологии OSTIS} более удобным является хранение компонентов в виде исходных текстов.}
\scnaddlevel{-1}
\scnaddlevel{-1}

\scnheader{следует отличать*}
\scnhaselementset{многократно используемый компонент ostis-систем;компонент ostis-системы}
\scntext{отличие}{\textbf{\textit{многократно используемый компонент ostis-систем}} имеет спецификацию, достаточную для установки этого компонента в дочернюю ostis-систему. Спецификация является частью базы знаний \textbf{\textit{библиотеки многократно используемых компонентов}} соответствующей материнской ostis-системы.}

\scnheader{независимый многократно используемый компонент ostis-систем}
\scnexplanation{Многократно используемый компонент, который не зависит от других компонентов.}

\scnheader{зависимый многократно используемый компонент ostis-систем}
\scnexplanation{Многократно используемый компонент, который зависит от хотя бы одного другого компонента, т.е. не может быть встроен в дочернюю ostis-систему без компонентов, от которых он зависит.}

\scnheader{атомарный многократно используемый компонент ostis-систем}
\scnexplanation{Многократно используемый компонент, который в текущем состоянии библиотеки компонентов рассматривается как неделимый, то есть не содержит в своем составе других компонентов.}

\scnheader{неатомарный многократно используемый компонент ostis-систем}
\scnexplanation{Многократно используемый компонент, который в текущем состоянии библиотеки компонентов содержит в своем составе атомарные компоненты.}

\scnheader{установленные компоненты*}
\scniselement{квазибинарное отношение}
\scniselement{ориентированное отношение}
\scnexplanation{Квазибинарное отношение, связывающее некоторую ostis-систему и компоненты, которые установлены в ней.}
\scnrelfrom{первый домен}{ostis-система}
\scnrelfrom{второй домен}{многократно используемый компонент ostis-систем}
\scnnote{Данное отношение позволяет хранить сведения о системах и компонентах, которые установлены в них, тем самым предоставляя возможность анализировать функциональные возможности системы.}
\scnnote{Данное отношение позволяет оценивать частоту скачивания компонентов, то есть их использования в дочерних ostis-системах.}

\scnheader{доступные к установке компоненты*}
\scniselement{квазибинарное отношение}
\scniselement{ориентированное отношение}
\scnexplanation{Квазибинарное отношение, связывающее некоторую ostis-систему и компоненты, которые доступны для установки в данной ostis-системе.}
\scnrelfrom{первый домен}{ostis-система}
\scnrelfrom{второй домен}{многократно используемый компонент ostis-систем}
% \scnnote{Данное отношение позволяет давать рекомендации по развитию ostis-системы, в которую можно установить многократно используемые компоненты на основе уже установленных компонентов и решаемых задач ostis-системой.}
% \scnnote{Доступные к установке компоненты ostis-системы выбираются исходя из тех библиотек ostis-систем, к которым есть доступ у ostis-системы.}
% \scnnote{Смотрите \textbf{\textit{менеджер многократно используемых компонентов ostis-систем.}}}

\scnheader{хранилище многократно используемого компонента ostis-систем, хранящегося в виде внешних файлов}
% \scnexplanation{Место, предназначенное для хранения многократно используемого компонента ostis-систем.}
\scnsuperset{хранилище многократно используемого компонента ostis-систем, хранящегося в виде файлов исходных текстов}
\scnaddlevel{1}
\scnexplanation{Место хранения файлов исходных текстов многократно используемого компонента.}
\scnsuperset{хранилище на основе системы контроля версий Git}
\scnaddlevel{1}
\scnsuperset{репозиторий GitHub}
\scnnote{На данном этапе в рамках \textit{Технологии OSTIS} (в силу открытости технологии, а также хранения компонентов в виде файлов исходных текстов) для хранения компонентов чаще всего используются хранилища на основе системы контроля версий Git.}
\scnaddlevel{-1}
\scnnote{Помимо исходных текстов компонента в хранилище должна находиться его спецификация, а также набор инструкций, позволяющий интегрировать данный компонент в дочернюю ostis-систему.}
\scnaddlevel{-1}
\scnsuperset{хранилище многократно используемого компонента ostis-систем, хранящегося в виде скомпилированных файлов}
\scnaddlevel{1}
\scnexplanation{Место хранения скомпилированных файлов многократно используемого компонента.}
\scnnote{Помимо скомпилированных файлов компонента в хранилище должна находиться его спецификация, а также набор инструкций, позволяющий интегрировать данный компонент в дочернюю ostis-систему.}
\scnaddlevel{-1}

\scnheader{спецификация многократно используемого компонента ostis-систем}
\scnsubset{спецификация}
\scnidtf{описание многократно используемого компонента ostis-систем}
\scnnote{Каждый \textit{многократно используемый компонент ostis-систем} должен быть специфицирован в рамках библиотеки. Данные спецификации включают в себя основные знания о компоненте, которые позволяют обеспечить построение полной иерархии компонентов и их зависимостей, а также обеспечивают беспрепятственную интеграцию компонентов в дочерние ostis-системы.}

\scnheader{отношение, специфицирующее многократно используемый компонент ostis-систем}
\scnidtf{отношение, которое используется при спецификации многократно используемого компонента ostis-систем}
\scnhaselement{адрес хранилища*}
\scnaddlevel{1}
\scniselement{бинарное отношение}
\scniselement{ориентированное отношение}
\scnexplanation{Связки отношения \textit{адрес хранилища*} связывают многократно используемый компонент, хранящийся в виде внешних файлов и файл, содержащий url-адрес.}
\scnrelfrom{первый домен}{многократно используемый компонент ostis-систем, хранящийся в виде внешних файлов}
\scnrelfrom{второй домен}{файл, содержащий url-адрес многократно используемого компонента ostis-систем}
\scnaddlevel{1}
\scnsuperset{файл}
\scnaddlevel{-1}
\scnaddlevel{-1}
\scnhaselement{автор компонента*}
\scnaddlevel{1}
\scniselement{бинарное отношение}
\scniselement{ориентированное отношение}
\scnexplanation{Связки отношения \textit{автор компонента*} связывают многократно используемый компонент и его разработчика.}
\scnrelfrom{первый домен}{многократно используемый компонент ostis-систем}
\scnrelfrom{второй домен}{субъект}
\scnaddlevel{-1}
\scnhaselement{зависимый компонент*}
\scnaddlevel{1}
\scniselement{бинарное отношение}
\scniselement{ориентированное отношение}
\scnexplanation{Бинарное отношение, связывающее зависимый многократно используемый компонент и компонент, без которого тот не может быть встроен в дочернюю ostis-систему.}
\scnrelfrom{первый домен}{многократно используемый компонент ostis-системы}
\scnrelfrom{второй домен}{зависимый многократно используемый компонент ostis-систем}
\scnaddlevel{-1}
% \scnhaselement{несовместимый компонент*}
% \scnaddlevel{1}
% \scniselement{бинарное отношение}
% \scniselement{неориентированное отношение}
% \scnexplanation{Бинарное отношение, связывающее два компонента, которые не могут одновременно присутствовать в одной ostis-системе.}
% \scnaddlevel{1}
% \scnexplanation{Невозможность существования таких компонентов в одной ostis-системе обуславливается тем, что зачастую эти компоненты принадлежат разных формальным теориям, при объединении которых образуются противоречивые высказывания. Такими компонентами могут быть, например, компонент базы знаний по геометрии Евклида и компонент базы знаний по геометрии Лобачевского.}
% \scnaddlevel{1}
% \scnrelfrom{смотрите}{\nameref{sd_logics}}
% \scnaddlevel{-1}
% \scnaddlevel{-1}
% \scnrelfrom{первый домен}{многократно используемый компонент ostis-систем}
% \scnrelfrom{второй домен}{многократно используемый компонент ostis-систем}
% \scnaddlevel{-1}
\scnnote{Для спецификации многократно используемого компонента также необходимо указывать классы, к которым он принадлежит, дату последнего изменения, описание назначения компонента.}

\scnheader{многократно используемый компонент ostis-системы}
\scnsubdividing{многократно используемый компонент базы знаний; многократно используемый компонент решателя задач; многократно используемый компонент интерфейса}

%\scnheader{Решатель задач библиотеки многократно используемых компонентов баз знаний}
%\scnrelfromset{декомпозиция абстрактного sc-агента}{
%	Неатомарный агент поиска компонента\\
%	\scnaddlevel{1}
%	\scnexplanation{Множество агентов, обеспечивающих поиск компонентов в рамках библиотеки по определенным критериям}
%	\scnnote{Существующие критерии регламентированы спецификацией многократно используемых компонентов}
%	\scnaddlevel{-1}
%	;Неатомарный агент формирования фрагментов компонента\\
%	\scnaddlevel{1}
%	\scnexplanation{Множество агентов, позволяющих формировать фрагменты компонентов по заданным критериям, обеспечивая возможность использование только тех знаний, которые непосредственно нужны для функционирования интеллектуальной системы.}
%	\scnrelfromset{декомпозиция абстрактного sc-агента}{
	%		Агент формирования компонента по семантической окрестности заданного понятия;Агент формирования компонента по неатомарным компонентам}
%	\scnaddlevel{-1}
%	;Агент поиска зависимостей
%	\scnaddlevel{1}
%	\scnidtf{Агент поиска всех зависимостей, без которых использование запрашиваемого компонента невозможно}
%	\scnaddlevel{-1}
%	;Агент поиска конфликтов между компонентами
%	\scnaddlevel{1}
%	\scnidtf{Агент проверки отсутствия/присутствия конфликтов между установленным и устанавливаемым компонентами}
%	\scnaddlevel{-1}
%	;Неатомарный агент версионирования\\
%	\scnaddlevel{1}
%	\scnexplanation{Множество агентов, решающих задачу версионирования фрагментов БЗ. Данные агенты позволяют формировать начальное состояние многократно используемого компонента и интегрировать последующие изменения компонента в существую структуру. Затем по запросу пользователя возвращать состояние данной структуры на определенный промежуток времени, что разрешает использование в разработках интеллектуальных систем различных версий одного и того же компонента базы знаний}
%	\scnrelfromset{декомпозиция абстрактного sc-агента}{
	%		Агент формирования начального состояния в дереве
	%		состояний фрагмента БЗ
	%		;Агент интеграции изменений с текущим состоянием
	%		фрагмента БЗ;Агент воссоздания версии фрагмента БЗ по его заданному состоянию;Агент идентификации состояний в дереве состояний
	%		заданного фрагмента БЗ; Агент получения состояния по его идентификатору}
%	\scnaddlevel{-1}
%	;Агент спецификации компонента
%	\scnaddlevel{1}
%	\scnidtf{Агент, позволяющий сформировать спецификацию разрабатываемого компонента для его дальнейшей публикации}
%	\scnaddlevel{-1}}

\bigskip
\scnendstruct \scnendcurrentsectioncomment

\end{SCn}
\begin{SCn}
    \scnsectionheader{Предметная область и онтология комплексной библиотеки многократно используемых семантически совместимых компонентов ostis-систем}
    \begin{scnsubstruct}
        \scntext{основной тезис}{Всё, что можно сделать одинаково, нужно делать одинаково.}
        \scntext{аннотация}{Идея библиотеки компонентов не нова, но семантическая мощность \scnkeyword{Библиотеки OSTIS} значительно выше аналогов за счет того, что подавляющее большинство компонентов библиотеки --- компоненты базы знаний, представленные на унифицированном языке представления знаний (\textit{SC-коде}). Таким образом, в \scnkeyword{Библиотеке OSTIS} обеспечивается высокий уровень семантической совместимости компонентов.}
        \scnheader{Предметная область многократно используемых компонентов ostis-систем}
        \scniselement{предметная область}
        \scnhaselementrole{максимальный класс объектов исследования}{библиотека многократно используемых компонентов ostis-систем;многократно используемый компонент ostis-систем}
        \begin{scnhaselementrolelist}{класс объектов исследования}
            \scnitem{многократно используемый компонент базы знаний}
            \scnitem{многократно используемый компонент решателя задач}
            \scnitem{многократно используемый компонент пользовательского интерфейса}
            \scnitem{атомарный многократно используемый компонент ostis-систем}
            \scnitem{неатомарный многократно используемый компонент ostis-систем}
            \scnitem{зависимый многократно используемый компонент ostis-систем}
            \scnitem{независимый многократно используемый компонент ostis-систем}
            \scnitem{платформенно независимый многократно используемый компонент ostis-систем}
            \scnitem{платформенно зависимый многократно используемый компонент ostis-систем}
            \scnitem{многократно используемый компонент ostis-систем, хранящийся в виде внешних файлов}
            \scnitem{многократно используемый компонент ostis-систем, хранящийся в виде файлов исходных текстов}
            \scnitem{многократно используемый компонент ostis-систем, хранящийся в виде скомпилированных файлов}
            \scnitem{многократно используемый компонент, хранящийся в виде sc-структуры}
            \scnitem{типовая подсистема ostis-систем;платформа интерпретации sc-моделей компьютерных систем}
            \scnitem{динамически устанавливаемый многократно используемый компонент ostis-систем}
            \scnitem{многократно используемый компонент, при установке которого система требует перезапуска}
            \scnitem{хранилище многократно используемого компонента ostis-систем, хранящегося в виде внешних файлов}
            \scnitem{хранилище многократно используемого компонента ostis-систем, хранящегося в виде файлов исходных текстов}
            \scnitem{хранилище многократно используемого компонента ostis-систем, хранящегося в виде скомпилированных файлов}
            \scnitem{спецификация многократно используемого компонента ostis-систем}
            \scnitem{отношение, специфицирующее многократно используемый компонент ostis-систем}
            \scnitem{параметр, заданный на многократно используемых компонентах ostis-систем\scnsupergroupsign}
            \scnitem{файл, содержащий url-адрес многократно используемого компонента ostis-систем}
            \scnitem{менеджер многократно используемых компонентов ostis-систем}
        \end{scnhaselementrolelist}
        \begin{scnhaselementrolelist}{исследуемое отношение}
            \scnitem{метод установки*}
            \scnitem{адрес хранилища*}
            \scnitem{зависимости компонента*}
            \scnitem{установленные компоненты*}
        \end{scnhaselementrolelist}
        \begin{scnhaselementrolelist}{исследуемый параметр}
            \scnitem{класс многократно используемого компонента ostis-систем\scnsupergroupsign}
        \end{scnhaselementrolelist}
        \begin{scnhaselementrolelist}{отношение, используемое в предметной области}
            \scnitem{автор*}
            \scnitem{ключевой sc-элемент*}
            \scnitem{пояснение*}
            \scnitem{sc-идентификатор*}
            \scnitem{история изменений*}
        \end{scnhaselementrolelist}
        \begin{scnrelfromset}{основные положения}
            \scnfileitem{Важнейшим этапом эволюции любой технологии является переход к компонентному проектированию на основе постоянно пополняемый библиотеки многократно используемых компонентов.}
            \begin{scnindent}
                \begin{scnrelfromset}{известные проблемы}
                    \scnfileitem{Отсутствие унификации в принципах представления различных видов знаний в рамках одной базы знаний, и, как следствие, отсутствие унификации в принципах выделения и спецификации многократно используемых компонентов приводит к несовместимости компонентов, разработанных в рамках разных проектов.}
                    \scnfileitem{Не ведётся разработка стандартов, обеспечивающих совместимость этих компонентов.}
                    \scnfileitem{Многие компоненты используют для идентификации язык разработчика (как правило, английский), и предполагается, что все пользователи будут использовать этот же язык. Однако для многих приложений это недопустимо  понятные только разработчику идентификаторы должны быть скрыты от конечных пользователей, которые должны быть в состоянии выбрать язык для идентификаторов, которые они видят}
                    \scnitem{Отсутствие средств поиска компонентов, удовлетворяющих заданных критериям.}
                \end{scnrelfromset}
                \begin{scnrelfromset}{необходимые требования}
                    \scnfileitem{Универсальный язык представления знаний.}
                    \scnfileitem{Универсальная процедура интеграции знаний в рамках указанного языка.}
                    \scnfileitem{Разработка стандарта, обеспечивающего семантическую совместимость интегрируемых знаний (таким стандартом является согласованная система используемых понятий).}
                \end{scnrelfromset}
            \end{scnindent}
            \scnfileitem{Повторное использование готовых компонентов широко применяется во многих отраслях, связанных с проектирование различного рода систем, поскольку позволяет уменьшить трудоемкость разработки и ее стоимость (путем минимизации количества труда за счет отсутствия необходимости разрабатывать какой-либо компонент), повысить качество создаваемого контента и снизить профессиональные требования к разработчикам компьютерных систем. Использование готовых компонентов предполагает, что распространяемый компонент верифицирован и документирован, а возможные ошибки и ограничения устранены либо специфицированы и известны.}
        \end{scnrelfromset}
        \scnheader{семантически смежный раздел*}
        \begin{scnhaselementset}
            \scnitem{Предметная область и онтология комплексной библиотеки многократно используемых семантически совместимых компонентов ostis-систем}
            \scnitem{Логико-семантическая модель Метасистемы OSTIS}
        \end{scnhaselementset}
        \begin{scnindent}
            \scntext{пояснение}{\textit{Метасистема OSTIS} ориентирована на разработку ипрактическое внедрение методов и средств компонентного проектирования семантически совместимых интеллектуальных систем, которая предоставляет возможность быстрого создания интеллектуальных приложений различного назначения. В состав Метасистемы OSTIS входит \scnkeyword{Библиотека Метасистемы OSTIS}. Сферы практического применения технологии компонентного проектирования семантически совместимых интеллектуальных систем ничем не ограничены.}
            \scntext{пояснение}{Основу для реализации компонентного подхода в рамках \textit{Технологии OSTIS} составляет \scnkeyword{Библиотека OSTIS}.}
        \end{scnindent}
        \scnheader{семантически смежный раздел*}
        \begin{scnhaselementset}
            \scnitem{Предметная область и онтология комплексной библиотеки многократно используемых семантически совместимых компонентов ostis-систем}
            \scnitem{Предметная область и онтология комплексной технологии поддержки жизненного цикла интеллектуальных компьютерных систем нового поколения}
        \end{scnhaselementset}
        \begin{scnindent}
            \scntext{пояснение}{Основным требованием, предъявляемым к \textit{Технологии OSTIS}, является обеспечение возможности совместного использования в рамках ostis-систем различных \textit{видов знаний} и различных \textit{моделей решения задач} с возможностью \uline{неограниченного} расширения перечня используемых в ostis-системе видов знаний и моделей решения задач без существенных трудозатрат. Следствием данного требования является необходимость реализации компонентного подхода на всех уровнях, от простых компонентов баз знаний и решателей задач до целых ostis-систем.}
        \end{scnindent}
        \scnheader{дочерний раздел*}
        \begin{scnhaselementvector}
            \scnitem{Предметная область и онтология комплексной библиотеки многократно используемых семантически совместимых компонентов ostis-систе}
            \scnitem{Предметная область и онтология многократно используемых компонентов баз знаний ostis-систем}
        \end{scnhaselementvector}
        \begin{scnindent}
            \scntext{пояснение}{На сегодняшний день разработано большое число \textit{баз знаний} по самым различным предметным областям. Однако в большинстве случаев каждая база знаний разрабатывается отдельно и независимо от других, в отсутствие единой унифицированной формальной основы для представления знаний, а также единых принципов формирования систем понятий для описываемой предметной области. В связи с этим разработанные базы оказываются, как правило, несовместимы между собой и не пригодны для повторного использования. Компонентный подход к разработке интеллектуальных компьютерных систем, реализуемый в виде \scnkeyword{библиотеки многокртно используемых компонентов ostis-систем}, позволяет решить описанные проблемы.}
        \end{scnindent}
        \scnheader{дочерний раздел*}
        \begin{scnhaselementvector}
            \scnitem{Предметная область и онтология комплексной библиотеки многократно используемых семантически совместимых компонентов ostis-систем}
            \scnitem{Предметная область и онтология многократно используемых компонентов решателей задач ostis-систем}
        \end{scnhaselementvector}
        \begin{scnindent}
            \scntext{пояснение}{В области разработки \textit{решателей задач} существует большое количество конкретных реализаций, однако вопросы совместимости различных решателей при решении одной задачи практически не рассматриваются.}
        \end{scnindent}
        \scnheader{дочерний раздел*}
        \begin{scnhaselementvector}
            \scnitem{Предметная область и онтология комплексной библиотеки многократно используемых семантически совместимых компонентов ostis-систем}
            \scnitem{Предметная область и онтология многократно используемых компонентов интерфейсов ostis-систем}
        \end{scnhaselementvector}
        \scnheader{библиотека многократно используемых компонентов ostis-систем}
        \scntext{сокращение}{библиотека ostis-систем}
        \scnidtf{библиотека многократно используемых компонентов OSTIS}
        \scnhaselementrole{типичный пример}{\scnkeyword{Библиотека OSTIS}}
        \begin{scnindent}
            \scnidtf{Библиотека многократно используемых компонентов ostis-систем в составе Метасистемы OSTIS}
            \scnidtf{Библиотека Метасистемы OSTIS}
        \end{scnindent}
        \scntext{назначение}{Основное назначение Библиотеки OSTIS --- создание условий для эффективного, осмысленного и массового проектирования ostis-систем и их компонентов путём создания среды для накопления и совместного использования компонентов ostis-систем.}
        \begin{scnindent}
            \scntext{примечание}{Такие условия осуществляются путём неограниченного расширения постоянно эволюционируемых семантически совместимых ostis-систем и их компонентов, входящих в \textit{Экосистему OSTIS}.}
        \end{scnindent}
        \scntext{примечание}{Разработчики \uline{любой} \textit{ostis-системы} могут включить в ее состав библиотеку, которая позволит им накапливать и распространять результаты своей деятельности среди других участников \textit{Экосистемы OSTIS} в виде \scnkeyword{многократно используемых компонентов}. Решение о включении компонента в библиотеку принимается экспертным сообществом разработчиков, ответственным за качество этой библиотеки. Верификацию компонентов можно автоматизировать.}
        \scntext{примечание}{В рамках \textit{Экосистемы OSTIS} существует множество библиотек многократно используемых компонентов ostis-систем, являющихся подсистемами соответствующих материнских ostis-систем. Главной библиотекой многократно используемых компонентов ostis-систем является \textit{Библиотека Метасистемы OSTIS}. \textit{Метасистема OSTIS} выступает \scnkeyword{материнской системой} для всех разрабатываемых ostis-систем, поскольку содержит все базовые компоненты.}
        \begin{scnindent}
            \scnrelfrom{описание примера}{\scnfileimage[20em]{Contents/part_methods_tools/images/sd_ostis_library/ecosystem_architecture.png}}
            \begin{scnindent}
                \scnrelfrom{смотрите}{менеджер многократно используемых компонентов ostis-систем}
            \end{scnindent}
            \scntext{примечание}{Материнская система отвечает за какой-то класс компонентов и является САПРом для этого класса, например, содержит методики разработки таких компонентов, их классификацию, подробные пояснения ко всем подклассам компонентов. Таким образом, формируется иерархия \scnkeyword{материнских ostis-систем}.}
        \end{scnindent}
        \begin{scnrelfromset}{функциональные возможности}
            \scnfileitem{Хранение многократно используемых компонентов ostis-систем и их спецификаций.}
            \begin{scnindent}
                \scntext{примечание}{При этом часть компонентов, специфицированных в рамках библиотеки, могут физически храниться в другом месте ввиду особенностей их  технической реализации (например, исходные тексты платформы интерпретации sc-моделей компьютерных систем могут физически храниться в каком-либо отдельном репозитории, но специфицированы как компонент будут в соответствующей библиотеке). В этом случае спецификация компонента в рамках библиотеки должна также включать описание (1) того где располагается компонент и (2) сценария его автоматической или хотя бы ручной установки в дочернюю ostis-систему.}
                \begin{scnindent}
                    \scnrelfrom{смотрите}{менеджер многократно используемых компонентов ostis-систем}
                \end{scnindent}
            \end{scnindent}
            \scnfileitem{Просмотр имеющихся компонентов и их спецификаций, а также поиск компонентов по фрагментам их спецификации.}
            \scnfileitem{Хранение сведений о том, в каких ostis-системах-потребителях какие из компонентов библиотеки и какой версии используются (были скачаны). Это необходимо как минимум для учета востребованности того или иного компонента, оценки его важности и популярности.}
            \scnfileitem{Обеспечение автоматического обновления компонентов, заимствованных в дочерние ostis-системы. Данная функция может включаться и отключаться по желанию разработчиков дочерней ostis-системы.}
            \begin{scnindent}
                \scntext{примечание}{Одновременное обновление одних и тех же компонентов во всех системах, его использующих, не должно ни в каком контексте приводить к несогласованности между этими системами. Это требование может оказаться довольно сложным, но без него работа Экосистемы невозможна.}
            \end{scnindent}
        \end{scnrelfromset}
        \scntext{примечание}{Постоянно расширяемая Библиотека OSTIS существенно сокращает сроки разработки новых интеллектуальных компьютерных систем.}
        \scntext{примечание}{Библиотека ostis-систем позволяет избавиться от дублирования семантически эквивалентных информационных компонентов. А также от многообразия форм технической реализации используемых моделей решения задач.}
        \scntext{примечание}{Интеграция многократно используемых компонентов ostis-систем сводится к склеиванию ключевых узлов по идентификаторам и устранениювозможных дублирований и противоречий исходя из спецификации компонента и его содержания. Интеграция любых компонентов ostis-систем происходит автоматически, без вмешательства разработчика.}
        \begin{scnindent}
            \scntext{примечание}{Это достигается за счёт использования \textit{SC-кода} и его преимуществ, унификации спецификации многократно используемых компонентов и тщательного отбора компонентов в библиотеках экспертным сообществом, ответственным за эту библиотеку.}
        \end{scnindent}
        \scnheader{ostis-система}
        \scnsuperset{материнская ostis-система}
        \begin{scnindent}
            \scntext{пояснение}{ostis-система, имеющая в своем составе библиотеку многократно используемых компонентов.}
            \scnhaselement{Метасистема OSTIS}
            \scntext{примечание}{материнская ostis-система в свою очередь может являться дочерней ostis-системой для какой-либо другой ostis-системы, заимствуя компоненты из библиотеки, входящей в состав этой другой ostis-системы.}
        \end{scnindent}
        \scnsuperset{дочерняя ostis-система}
        \begin{scnindent}
            \scntext{пояснение}{ostis-система, в составе которой имеется компонент, заимствованный из какой-либо библиотеки многократно используемых компонентов.}
        \end{scnindent}
        \scnheader{библиотека многократно используемых компонентов ostis-систем}
        \begin{scnreltoset}{объединение}
            \scnitem{библиотека типовых подсистем ostis-систем}
            \scnitem{библиотека шаблонов типовых компонентов ostis-систем}
            \scnitem{библиотека платформ интерпретации sc-моделей компьютерных систем}
            \scnitem{библиотека многократно используемых компонентов баз знаний}
            \scnitem{библиотека многократно используемых компонентов решателей задач}
            \scnitem{библиотека многократно используемых компонентов пользовательских интерфейсов}
        \end{scnreltoset}
        \begin{scnrelfromset}{обобщенная декомпозиция}
            \scnitem{база знаний библиотеки многократно используемых компонентов ostis-систем}
            \begin{scnindent}
                \scntext{примечание}{База знаний библиотеки мнoгократно используемых компонентов ostis-систем представляет собой иерархию многократно используемых компонентов ostis-систем и их спецификаций.}
            \end{scnindent}
            \scnitem{решатель задач библиотеки многократно используемых компонентов ostis-систем}
            \begin{scnindent}
                \begin{scnrelfromset}{функциональные возможности}
                    \scnfileitem{Систематизация многократно используемых компонентов ostis-систем.}
                    \scnfileitem{Обеспечение версионирования многократно используемых компонентов ostis-систем.}
                    \scnfileitem{Поиск зависимостей и конфликтов между многократно используемыми компонентами в рамках библиотеки компонентов.}
                    \scnfileitem{Формирование отдельных фрагментов многократно используемых компонентов ostis-систем.}
                \end{scnrelfromset}
            \end{scnindent}
            \scnitem{интерфейс библиотеки многократно используемых компонентов ostis-систем}
            \begin{scnindent}
                \scntext{примечание}{Интерфейс обеспечивает доступ к многократно используемым компонентам. Позволяет получить информацию о зависимых, конфликтующих компонентах.}
                \begin{scnrelfromset}{декомпозиция}
                    \scnitem{минимальный интерфейс библиотеки многократно используемых компонентов ostis-систем}
                    \begin{scnindent}
                        \scntext{примечание}{Данный вид интерфейса позволяет \textit{менеджеру многократно используемых компонентов ostis-систем}, входящему в состав какой-либо дочерней ostis-системы, подключиться к библиотеке многократно используемых компонентов ostis-систем и использовать ее функциональные возможности, то есть, например, получить доступ к спецификации компонентов и установить выбранные компоненты в дочернюю ostis-систему, получить сведения до доступных версиях компонента, его зависимостях и т.д.}
                    \end{scnindent}
                    \scnitem{расширенный интерфейс библиотеки многократно используемых компонентов ostis-систем}
                    \begin{scnindent}
                        \scnidtf{графический интерфейс библиотеки многократно используемых компонентов ostis-систем}
                        \scntext{примечание}{В частном случае у библиотеки может быть расширенный пользовательский интерфейс, который, в отличие от минимального интерфейса, позволяет не только получить доступ к компонентам для дальнейшей работы с ними, но и просматривать существующую структуру библиотеки,  а также компоненты и их элементы в удобном и интуитивно понятном для пользователя виде.}
                    \end{scnindent}
                \end{scnrelfromset}
            \end{scnindent}
        \end{scnrelfromset}
        \scnheader{многократно используемый компонент ostis-систем}
        \scnidtf{многократно используемый компонент OSTIS}
        \scnidtftext{часто используемый sc-идентификатор}{многократно используемый компонент}
        \scnrelfrom{аббревиатура}{\scnfilelong{МИК ostis-систем}}
        \scnsubset{компонент ostis-системы}
        \begin{scnindent}
            \scntext{пояснение}{Целостная часть ostis-системы, которая содержит все те (и только те) sc-элементы, которые необходимы для её функционирования в ostis-системе.}
        \end{scnindent}
        \scntext{определение}{многократно используемый компонент ostis-систем --- компонент некоторой ostis-системы, который может быть использован в рамках другой ostis-системы.}
        \scntext{пояснение}{Компонент ostis-системы, который может быть использован в других ostis-системах (\scnkeyword{дочерних ostis-системах}) и содержит все те и только те sc-элементы, которые необходимы для функционирования компонента в дочерней ostis-системе.}
        \scntext{пояснение}{Компонент некоторой \scnkeyword{материнской ostis-системы}, который может быть использован в некоторой \scnkeyword{дочерней ostis-системе}.}
        \scntext{примечание}{Многократно используемые компоненты должны иметь унифицированную спецификацию и иерархию для поддержки \uline{совместимости} с другими компонентами.}
        \scntext{примечание}{Совместимость многократно используемых компонентов приводит систему к новому качеству, к дополнительному расширению множества решаемых задач при интеграции компонентов.}
        \begin{scnrelfromset}{необходимые требования}
            \scnfileitem{Существует техническая возможность встроить многократно используемый компонент в \scnkeyword{дочернюю ostis-систему}.}
            \scnfileitem{Полнота многократно используемого компонента: компонент должен в полной мере выполнять свои функции, соответствовать своему назначению.}
            \scnfileitem{Связность многократно используемого компонента: компонент должен логически выполнять только одну задачу из предметной области, для которой он предназначен. Многократно используемый компонент должен выполнять свои функции наиболее общим образом, чтобы круг возможных систем, в которые он может быть встроен, был наиболее широким.}
            \scnfileitem{Совместимость многократно используемого компонента: компонент должен стремиться повышать уровень \uline{договороспособности} ostis-систем, в которые он встроен и иметь возможность \uline{автоматической} интеграции в другие системы.}
        \end{scnrelfromset}
        \scnheader{следует отличать*}
        \begin{scnhaselementset}
            \scnitem{многократно используемый компонент ostis-систем}
            \scnitem{компонент ostis-системы}
        \end{scnhaselementset}
        \scntext{отличие}{многократно используемый компонент ostis-систем имеет \uline{спецификацию, достаточную для установки} этого компонента в \scnkeyword{дочернюю ostis-систему}. Спецификация является частью базы знаний \scnkeyword{библиотеки многократно используемых компонентов} соответствующей \scnkeyword{материнской ostis-системы}. Есть техническая возможность встроить многократно используемый компонент в дочернюю ostis-систему.}
        \scnheader{параметр, заданный на многократно используемых компонентах ostis-систем\scnsupergroupsign}
        \scnsubset{параметр}
        \scnhaselement{класс многократно используемого компонента ostis-систем\scnsupergroupsign}
        \begin{scnindent}
            \scntext{примечание}{класс многократно используемого компонента ostis-систем является важной частью спецификации компонента, позволяющей лучше понять назначение и область применения данного компонента, а также класс многократно используемого компонента является важнейшим критерием поиска компонентов в библиотеке ostis-систем.}
        \end{scnindent}
        \scnhaselement{начало\scnsupergroupsign}
        \scnhaselement{завершение\scnsupergroupsign}
        \scnheader{многократно используемый компонент ostis-систем}
        \scntext{примечание}{Основной признак классификации многократно используемых компонентов является признак предметной области, к которой относится компонент. Здесь структура может быть довольно богатой в соответствии с иерархией областей человеческой деятельности.}
        \begin{scnrelfromset}{разбиение}
            \scnitem{многократно используемый компонент базы знаний}
            \begin{scnindent}
                \scniselement{класс многократно используемого компонента ostis-систем\scnsupergroupsign}
                \scntext{примечание}{Важнейшим признаком классификации многократно используемых компонентов баз знаний является вид знаний.}
                \begin{scnindent}
                    \scnrelfrom{смотрите}{вид знаний}
                \end{scnindent}
                \scnsuperset{семантическая окрестность}
                \scnsuperset{предметная область и онтология}
                \scnsuperset{база знаний}
                \scnsuperset{шаблон типового компонента ostis-систем}
                \begin{scnindent}
                    \scnhaselement{Шаблон описания sc-модели предметной области}
                    \scnhaselement{Шаблон описания отношения}
                \end{scnindent}
            \end{scnindent}
            \scnitem{многократно используемый компонент решателя задач}
            \begin{scnindent}
                \scniselement{класс многократно используемого компонента ostis-систем\scnsupergroupsign}
                \scnsuperset{атомарный агент обработки знаний}
                \scnsuperset{программа обработки знаний}
            \end{scnindent}
            \scnitem{многократно используемый компонент интерфейса}
            \begin{scnindent}
                \scniselement{класс многократно используемого компонента ostis-систем\scnsupergroupsign}
                \scnsuperset{многократно используемйы компонент пользовательского интерфейса для отображения}
                \scnsuperset{интерактивный многократно используемый компонент пользовательского интерфейса}
            \end{scnindent}
        \end{scnrelfromset}
        \scnrelfrom{разбиение}{\scnkeyword{Типология компонентов ostis-систем по атомарности\scnsupergroupsign}}
        \begin{scnindent}
            \scnsubset{класс многократно используемого компонента ostis-систем\scnsupergroupsign}
            \begin{scneqtoset}
                \scnitem{атомарный многократно используемый компонент ostis-систем}
                \begin{scnindent}
                    \scntext{пояснение}{Многократно используемый компонент, который в текущем состоянии библиотеки ostis-систем рассматривается как неделимый, то есть не содержит в своем составе других компонентов.}
                    \scntext{примечание}{Принадлежность МИК ostis-систем классу атомарных компонентов зависит от того, как специфицирован этот компонент и от существующих на данный момент компонентов в библиотеке.}
                    \begin{scnindent}
                        \scntext{примечание}{В библиотеку ostis-систем нельзя опубликовать многократно используемый компонент как атомарный, в составе которого есть какой-либо другой известный библиотеке ostis-систем компонент.}
                    \end{scnindent}
                \end{scnindent}
                \scnitem{неатомарный многократно используемый компонент ostis-систем}
                \begin{scnindent}
                    \scntext{пояснение}{Многократно используемый компонент, который в текущем состоянии библиотеки ostis-систем содержит в своем составе другие атомарные или неатомарные компоненты.}
                    \scntext{примечание}{Неатомарный многократно используемый компонент не зависит от своих частей. Без какой-либо части неатомарного компонента его назначение сужается.}
                \end{scnindent}
            \end{scneqtoset}
        \end{scnindent}
        \scnrelfrom{разбиение}{\scnkeyword{Типология компонентов ostis-систем по зависимости\scnsupergroupsign}}
        \begin{scnindent}
            \scnsubset{класс многократно используемого компонента ostis-систем\scnsupergroupsign}
            \begin{scneqtoset}
                \scnitem{зависимый многократно используемый компонент ostis-систем}
                \begin{scnindent}
                    \scntext{пояснение}{Многократно используемый компонент, который зависит хотя бы от одного другого компонента библиотеки ostis-систем, т.е. не может быть встроен в дочернюю ostis-систему без компонентов, от которых он зависит.}
                \end{scnindent}
                \scnitem{независимый многократно используемый компонент ostis-систем}
                \begin{scnindent}
                    \scniselement{класс многократно используемого компонента ostis-систем\scnsupergroupsign}
                    \scntext{пояснение}{Многократно используемый компонент, который не зависит ни от одного другого компонента библиотеки ostis-систем.}
                \end{scnindent}
            \end{scneqtoset}
        \end{scnindent}
        \scnrelfrom{разбиение}{\scnkeyword{Типология компонентов ostis-систем по способу их хранения\scnsupergroupsign}}
        \begin{scnindent}
            \scnsubset{класс многократно используемого компонента ostis-систем\scnsupergroupsign}
            \begin{scneqtoset}
                \scnitem{многократно используемый компонент ostis-систем, хранящийся в виде внешних файлов}
                \begin{scnindent}
                    \begin{scnrelfromset}{разбиение}
                        \scnitem{многократно используемый компонент ostis-систем, хранящийся в виде файлов исходных текстов}
                        \scnitem{многократно используемый компонент ostis-систем, хранящийся в виде скомпилированных файлов}
                    \end{scnrelfromset}
                \end{scnindent}
                \scnitem{многократно используемый компонент, хранящийся в виде sc-структуры}
            \end{scneqtoset}
            \scntext{примечание}{На данном этапе развития \textit{Технологии OSTIS} более удобным является хранение компонентов в виде исходных текстов.}
        \end{scnindent}
        \scnrelfrom{разбиение}{\scnkeyword{Типология компонентов ostis-систем по зависимости от платформы\scnsupergroupsign}}
        \begin{scnindent}
            \scnsubset{класс многократно используемого компонента ostis-систем\scnsupergroupsign}
            \begin{scneqtoset}
                \scnitem{платформенно зависимый многократно используемый компонент ostis-систем}
                \begin{scnindent}
                    \scntext{пояснение}{Под платформенно-зависимым многократно используемым компонентом OSTIS понимается компонент, частично или полностью реализованный при помощи каких-либо сторонних с точки зрения \textit{Технологии OSTIS} средств.}
                    \scntext{недостаток}{Интеграция таких компонентов в интеллектуальные системы может сопровождаться дополнительными трудностями, зависящими от конкретных средств реализации компонента.}
                    \scntext{примечание}{С точки зрения \textit{Технологии OSTIS} любая платформа интерпретации sc-моделей компьютерных систем является платформенно зависимым многократно используемым компонентом.}
                    \scnsuperset{платформа интерпретации sc-моделей компьютерных систем}
                    \begin{scnindent}
                        \scnhaselement{Платформа OSTIS}
                    \end{scnindent}
                    \scnsuperset{абстрактный sc-агент, не реализуемый на Языке SCP}
                \end{scnindent}
                \scnitem{платформенно независимый многократно используемый компонент ostis-систем}
                \begin{scnindent}
                    \scntext{пояснение}{Под платформенно независимым многократно используемым компонентом ostis-систем понимается компонент, который целиком и полностью представлен на \textit{SC-коде}.}
                    \scnsuperset{многократно используемый компонент базы знаний}
                    \scnsuperset{SCP-агент}
                    \scnsuperset{SCP-программа}
                \end{scnindent}
            \end{scneqtoset}
            \scntext{примечание}{Наиболее ценными являются платформенно независимые многократно используемые компоненты ostis-систем.}
        \end{scnindent}
        \scnrelfrom{разбиение}{\scnkeyword{Типология компонентов ostis-систем по динамичности их установки\scnsupergroupsign}}
        \begin{scnindent}
            \scnsubset{класс многократно используемого компонента ostis-систем\scnsupergroupsign}
            \begin{scneqtoset}
                \scnitem{динамически устанавливаемый многократно используемый компонент ostis-систем}
                \begin{scnindent}
                    \scnidtf{многократно используемый компонент, при установке которого система не требует перезапуска}
                    \begin{scnrelfromset}{декомпозиция}
                        \scnitem{многократно используемый компонент, хранящийся в виде скомпилированных файлов}
                        \scnitem{многократно используемый компонент базы знаний}
                    \end{scnrelfromset}
                \end{scnindent}
                \scnitem{многократно используемый компонент, при установке которого система требует перезапуска}
            \end{scneqtoset}
            \scntext{примечание}{Процесс интеграции компонентов разных видов на разных этапах жизненного цикла osits-систем бывает разным. Наиболее ценными являются компоненты, которые могут быть интегрированы в работающую систему \uline{без} прекращения её функционирования. Некоторые системы, особенно системы управления, нельзя останавливать, а устанавливать и обновлять компоненты нужно.}
        \end{scnindent}
        \scnsuperset{типовая подсистема ostis-систем}
        \begin{scnindent}
            \scnsubset{класс многократно используемого компонента ostis-систем\scnsupergroupsign}
            \scnhaselement{Среда коллективной разработки баз знаний ostis-систем}
            \scnhaselement{Визуальный web-ориентированный редактор sc.g-текстов}
        \end{scnindent}
        \scnheader{хранилище многократно используемых компонентов ostis-систем, хранящихся в виде внешних файлов}
        \scnsuperset{хранилище многократно используемого компонента ostis-систем, хранящегося в виде файлов исходных текстов}
        \begin{scnindent}
            \scntext{пояснение}{Место хранения файлов исходных текстов многократно используемого компонента.}
            \scnsuperset{хранилище на основе системы контроля версий Git}
            \begin{scnindent}
                \scnsuperset{репозиторий GitHub}
                \scntext{примечание}{На данном этапе в рамках \textit{Технологии OSTIS} (в силу открытости технологии, а также хранения компонентов в виде файлов исходных текстов) для хранения компонентов чаще всего используются хранилища на основе системы контроля версий Git.}
            \end{scnindent}
        \end{scnindent}
        \scnsuperset{хранилище многократно используемого компонента ostis-систем, хранящегося в виде скомпилированных файлов}
        \begin{scnindent}
            \scntext{пояснение}{Место хранения скомпилированных файлов многократно используемого компонента.}
        \end{scnindent}
        \scntext{примечание}{Помимо внешних файлов компонента в хранилище должна находиться его \uline{спецификация}.}
        \scnheader{спецификация многократно используемого компонента ostis-систем}
        \scnsubset{спецификация}
        \scnidtf{описание многократно используемого компонента ostis-систем}
        \scnrelfrom{ключевой sc-элемент}{многократно используемый компонент ostis-систем}
        \scntext{примечание}{Каждый \textit{многократно используемый компонент ostis-систем} должен быть специфицирован в рамках библиотеки. Данные спецификации включают в себя основные знания о компоненте, которые позволяют обеспечить построение полной иерархии компонентов и их зависимостей, а также обеспечивают \uline{беспрепятственную} интеграцию компонентов в \scnkeyword{дочерние ostis-системы}.}
        \scnrelfrom{параметры, специфицирующие многократно используемый компонент ostis-систем}{параметр, заданный на многократно используемых компонентах ostis-систем\scnsupergroupsign}
        \scnrelfrom{классы отношений, специфицирующие многократно используемый компонент ostis-систем}{отношение, специфицирующее многократно используемый компонент ostis-систем\scnsupergroupsign}
        \scnrelfrom{описание примера}{\scnfileimage[20em]{Contents/part_methods_tools/images/sd_ostis_library/component_specification_example.png}}
        \scnheader{отношение, специфицирующее многократно используемый компонент ostis-систем\scnsupergroupsign}
        \scnidtf{отношение, которое используется при спецификации многократно используемого компонента ostis-систем}
        \begin{scnrelfromset}{разбиение}
            \scnitem{необходимое для установки отношение, специфицирующее многократно используемый компонент ostis-систем}
            \begin{scnindent}
                \scntext{примечание}{Чтобы многократно используемый компонент мог быть принят в библиотеку, нужно специфицировать его используя \textit{необходимое для установки отношение, специфицирующее многократно используемый компонент ostis-систем}.}
                \scnhaselement{метод установки*}
                \scnhaselement{адрес хранилища*}
                \scnhaselement{зависимости компонента*}
            \end{scnindent}
            \scnitem{необязательное для установки отношение, специфицирующее многократно используемый компонент ostis-систем}
            \begin{scnindent}
                \scntext{примечание}{\textit{необязательное для установки отношение, специфицирующее многократно используемый компонент ostis-систем} помогает лучше понять суть компонента, упрощает поиск, но не является обязательным для того, чтобы компонент мог быть установлен в ostis-систему.}
                \scnhaselement{история изменений*}
                \scnhaselement{авторы*}
                \scnhaselement{примечание*}
                \scnhaselement{пояснение*}
                \scnhaselement{идентификатор*}
                \scnhaselement{ключевой sc-элемент*}
                \scnhaselement{назначение*}
            \end{scnindent}
        \end{scnrelfromset}
        \scnheader{метод установки*}
        \scniselement{бинарное отношение}
        \scniselement{ориентированное отношение}
        \scntext{пояснение}{Пользователь может установить компонент вручную, а \scnkeyword{менеджер компонентов} - автоматически.}
        \scnrelfrom{первый домен}{многократно используемый компонент ostis-систем}
        \scnrelfrom{второй домен}{метод установки многократно используемого компонента}
        \begin{scnindent}
            \scnsubset{метод}
            \scnsuperset{метод установки динамически устанавливаемого многократно используемого компонента ostis-систем}
            \begin{scnindent}
                \scntext{примечание}{Установка таких компонентов происходит путём скачивания компонента в систему.}
                \scnrelfrom{описание примера}{\scnfileimage[20em]{Contents/part_methods_tools/images/sd_ostis_library/install_dynamic_method.png}}
                \begin{scnindent}
                    \scniselement{sc.g-текст}
                \end{scnindent}
            \end{scnindent}
            \scnsuperset{метод установки многократно используемого компонента, при установке которого система требует перезапуска}
            \begin{scnindent}
                \scntext{примечание}{Установка таких компонентов происходит путём скачивания компонента и его трансляции в память системы.}
                \scnrelfrom{описание примера}{\scnfileimage[20em]{Contents/part_methods_tools/images/sd_ostis_library/install_with_reboot_method.png}}
                \begin{scnindent}
                    \scniselement{sc.g-текст}
                \end{scnindent}
            \end{scnindent}
        \end{scnindent}
        \scnheader{адрес хранилища*}
        \scniselement{бинарное отношение}
        \scniselement{ориентированное отношение}
        \scntext{пояснение}{Связки отношения \textit{адрес хранилища*} связывают многократно используемый компонент, хранящийся в виде внешних файлов и файл, содержащий url-адрес многократно используемого компонента ostis-систем.}
        \scnrelfrom{первый домен}{многократно используемый компонент ostis-систем, хранящийся в виде внешних файлов}
        \scnrelfrom{второй домен}{файл, содержащий url-адрес многократно используемого компонента ostis-систем}
        \begin{scnindent}
            \scnsuperset{файл}
            \scnsubset{файл, содержащий url-адрес на GitHub многократно используемого компонента ostis-систем}
            \scnsubset{файл, содержащий url-адрес на Google Drive многократно используемого компонента ostis-систем}
            \scnsubset{файл, содержащий url-адрес на Docker Hub многократно используемого компонента ostis-систем}
        \end{scnindent}
        \scnheader{зависимости компонента*}
        \scniselement{квазибинарное отношение}
        \scniselement{ориентированное отношение}
        \scntext{пояснение}{Связки отношения \textit{зависимости компонента*} связывают многократно используемый компонент, и множество компонентов, без которых устанавливаемый компонент \uline{не может быть} встроен в \scnkeyword{дочернюю ostis-систему}.}
        \scnrelfrom{первый домен}{многократно используемый компонент ostis-систем}
        \scnrelfrom{второй домен}{множество многократно используемых компонентов ostis-систем}
        \scnheader{менеджер многократно используемых компонентов ostis-систем}
        \scnidtftext{часто используемый sc-идентификатор}{менеджер многократно используемых компонентов}
        \scnidtftext{часто используемый sc-идентификатор}{менеджер компонентов}
        \begin{scnrelfromset}{обобщенная декомпозиция}
            \scnitem{база знаний менеджера многократно используемых компонентов ostis-систем}
            \begin{scnindent}
                \scntext{примечание}{База знаний менеджера компонентов содержит все те знания, которые необходимы для установки многократно используемого компонента в \scnkeyword{дочернюю ostis-систему}. К таким знаниям относятся знания о спецификации многократно используемых компонентов, методы установки компонентов,знание о  библиотеках ostis-систем, с которыми происходит взаимодействие.}
            \end{scnindent}
            \scnitem{решатель задач менеджера многократно используемых компонентов ostis-систем}
            \begin{scnindent}
                \scntext{примечание}{Решатель задач менеджера компонентов взаимодействует с библиотекой ostis-систем и позволяет установить и интегрировать многократно используемые компоненты в \scnkeyword{дочернюю ostis-систему}, также выполнять поиск, обновление, публикацию, удаление компонентов.}
            \end{scnindent}
            \scnitem{интерфейс менеджера многократно используемых компонентов ostis-систем}
            \begin{scnindent}
                \scntext{примечание}{Интерфейс менеджера многократно используемых компонентов обеспечивает удобное для пользователя и других систем использование менеджера компонентов.}
            \end{scnindent}
        \end{scnrelfromset}
        \begin{scnrelfromset}{функциональные возможности}
            \scnitem{поиск многократно используемых компонентов ostis-систем}
            \begin{scnindent}
                \scntext{примечание}{Множество возможных критериев поиска соответствует спецификации многократно используемых компонентов. Такими критериями могут быть классы компонента, его авторы, идентификатор, фрагмент примечания, назначение, принадлежность какой-либо предметной области, вид знаний компонента и другие.}
            \end{scnindent}
            \scnitem{установка многократно используемого компонента ostis-систем}
            \begin{scnindent}
                \scntext{примечание}{Установка многократно используемого компонента происходит вне зависимости от типологии, способа установки и местоположения компонента. Необходимое условие для возможности установки многократно используемого компонента --- наличие \scnkeyword{спецификации многократно используемого компонента ostis-систем}.}
                \scntext{примечание}{Перед установкой многократно используемого компонента в дочернюю систему необходимо разрешить все зависимости путём установки зависимых компонентов.}
                \scntext{примечание}{После успешной установки компонента, в базе знаний дочерней системы генерируется информационная конструкция, обозначающая факт установки компонента в систему с помощью отношения \textit{установленные компоненты*}.}
                \scntext{примечание}{После установки компонента в ostis-систему могут возникнуть противоречия в базе знаний, которые  устраняются с помощью средств обнаружения и анализа ошибок и противоречий в базе знаний ostis-системы.}
                \begin{scnindent}
                    \scnrelfrom{смотрите}{Логико-семантическая модель ostis-системы обнаружения и анализа ошибок и противоречий в базе знаний ostis-системы}
                \end{scnindent}
            \end{scnindent}
            \scnitem{публикация многократно используемого компонента ostis-систем в библиотеку ostis-систем}
            \begin{scnindent}
                \scntext{примечание}{При публикации компонента в библиотеку ostis-систем происходит верификация на основе спецификации компонента.}
                \scntext{примечание}{Также существует возможность обновления версии опубликованного компонента сообществом его разработчиков.}
            \end{scnindent}
            \scnitem{обновление установленного многократно используемого компонента ostis-систем}
            \scnitem{удаление установленного многократно используемого компонента}
            \begin{scnindent}
                \scntext{примечание}{Как и в случае установки после удаления многократно используемого компонента из ostis-системы в базе знаний системы устанавливается факт удаления компонента. Эта информация является важной частью \uline{истории эксплуатации} ostis-системы.}
            \end{scnindent}
            \scnitem{добавление и удаление отслеживаемых ostis-системой библиотек}
            \begin{scnindent}
                \scntext{примечание}{Менеджер компонентов содержит информацию о множестве источников для установки компонентов, перечень которых можно дополнять вручную. По умолчанию менеджер компонентов отслеживает Библиотеку Метасистемы OSTIS, однако можно создавать и добавлять дополнительные библиотеки ostis-систем.}
            \end{scnindent}
        \end{scnrelfromset}
        \scnheader{установленные компоненты*}
        \scniselement{квазибинарное отношение}
        \scniselement{ориентированное отношение}
        \scntext{пояснение}{Квазибинарное отношение, связывающее некоторую ostis-систему и компоненты, которые установлены в ней.}
        \scnrelfrom{первый домен}{ostis-система}
        \scnrelfrom{второй домен}{множество многократно используемых компонентов ostis-систем}
        \begin{scnindent}
            \scntext{пояснение}{множество многократно используемых компонентов ostis-систем --- это множество, все элементы которого являются многократно используемыми компонентами ostis-систем.}
        \end{scnindent}
        \scntext{примечание}{Данное отношение позволяет хранить сведения о системах и компонентах, которые установлены в них, тем самым предоставляя возможность анализировать функциональные возможности системы.}
        \scntext{примечание}{Данное отношение позволяет оценивать частоту скачивания компонентов, то есть их использования в \scnkeyword{дочерних ostis-системах}.}
        \bigskip
    \end{scnsubstruct}
\end{SCn}

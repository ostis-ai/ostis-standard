\begin{SCn}
    \scnsectionheader{Предметная область и онтология многократно используемых компонентов баз знаний ostis-систем}
    \begin{scnsubstruct}
	    \scntext{аннотация}{Для широкого применения интеллектуальных систем, способных повысить качество решения прикладных задач, разработано большое число \textit{баз знаний} по самым различным предметным областям. Однако в большинстве случаев каждая база знаний разрабатывается отдельно и независимо от других, в отсутствие единой унифицированной формальной основы для представления знаний, а также единых принципов формирования систем понятий для описываемой предметной области. В связи с этим разработанные базы оказываются, как правило, несовместимы между собой и не пригодны для повторного использования. Для быстрой разработки достаточного количества баз знаний, кроме наличия средств разработки интеллектуальных систем, обеспечивающих разработку и проектирование различных компонентов интеллектуальной системы, включая базу знаний, требуется наличие соответствующей отлаженной технологии проектирования баз знаний.}
	    \begin{scnindent}
	        \begin{scnrelfromlist}{источник}
				\scnitem{Ivashenko2011}
			\end{scnrelfromlist}
		\end{scnindent}
        \scniselement{раздел базы знаний}
		\scnhaselementrole{ключевой sc-элемент}{Предметная область многократно используемых компонентов баз знаний ostis-систем}
    	
        \scnheader{Предметная область многократно используемых компонентов баз знаний ostis-систем}
        \scniselement{предметная область}
        \scnrelto{частная предметная область}{Предметная область многократно используемых компонентов ostis-систем}
        \begin{scnhaselementrolelist}{класс объектов исследования}
            \scnitem{многократно используемый компонент базы знаний ostis-систем}
        \end{scnhaselementrolelist}
        \scnhaselementrole{класс объектов исследования}{отношение, специфицирующее многократно используемый компонент базы знаний ostis-систем}
        \begin{scnrelfromlist}{ключевое понятие}
        	\scnitem{компонентное проектирование баз знаний интеллектуальных систем}
        \end{scnrelfromlist}
    
        \begin{scnrelfromlist}{библиографическая ссылка}
    		\scnitem{Ivashenko2011}
    		\scnitem{Golenkov2013}
    		\scnitem{Davydenko2013}
    	\end{scnrelfromlist}
        
        \scnheader{многократно используемый компонент базы знаний}
        \scnhaselement{Расширенное ядро базы знаний}
        \scnhaselement{Ядро базы знаний}
        \scntext{пояснение}{\textit{Ядро базы знаний} представляет собой компонент, входящий в состав каждой базы знаний, разрабатываемой по \textit{Технологии OSTIS}, и устанавливаемый в первую очередь.}
        
        \scnheader{отношение, специфицирующее многократно используемый компонент базы знаний ostis-систем}
        \scnsubset{отношение, специфицирующее многократно используемый компонент ostis-систем}
        \scnhaselement{максимальный класс объектов исследования\scnrolesign}
        \scnhaselement{немаксимальный класс объектов исследования\scnrolesign}
        \scnhaselement{исследуемое отношение\scnrolesign}
        
        \scnnote{Компонентный подход к разработке интеллектуальных компьютерных систем, реализуемый в виде \textbf{\textit{библиотеки многократно используемых компонентов ostis-систем}}, позволяет решить описанные проблемы. \textit{Библиотека многократно используемых компонентов баз знаний ostis-систем в составе Метасистемы OSTIS} является важнейшим фрагментом \textit{Метасистемы OSTIS}, который обеспечивает надежность и совместимость проектируемых фрагментов баз знаний, а также повышение скорости разработки \textit{баз знаний} \textit{интеллектуальных компьютерных систем}.
        
        Описываемая библиотека включает в себя множество компонентов баз знаний и их спецификаций.}
        
        \scnheader{многократно используемый компонент базы знаний ostis-систем}
        \scnsuperset{предметная область и онтология}
        \begin{scnindent}
        	\scnsubset{раздел базы знаний}
        \end{scnindent}
        \scnsuperset{семантическая окрестность}
        \begin{scnindent}
        	\scnsuperset{семантическая окрестность по инцидентным коннекторам}
        	\scnsuperset{полная семантическая окрестность}
        	\scnsuperset{базовая семантическая окрестность}
        	\scnsuperset{специализированная семантическая окрестность}
        \end{scnindent}
        \scnsuperset{базовые фрагменты предметных областей и онтологий}
        \begin{scnindent}
        	\scnnote{Базовый фрагмент предметной области и онтологии включает в себя теоретико-множественную, логическую онтологии, а также терминологические фрагменты.}
        	\scnnote{Данный вид многократно используемых компонентов позволяет использовать только те знания, которые непосредственно необходимы для функционирования интеллектуальных систем, исключив то, что никак не влияет на работу конечной системы (пояснения, примеры, дидактический материал и так далее).}
        	\scnhaselement{Базовый фрагмент теории логических формул, высказываний и логических sc-языков}
        	\scnhaselement{Базовый фрагмент теории множеств}
        	\scnhaselement{Базовый фрагмент теории связок и отношений}
        \end{scnindent}
        \scnsuperset{база знаний}
        \begin{scnindent}
        	\scnnote{Целые базы знаний могут быть многократно используемыми компонентами в случае разработки интеллектуальных систем, назначение которых совпадает.}
        \end{scnindent}
		\scntext{примечание}{Список приведенных классов многократно используемых компонентов не является окончательным. В случае, когда разработчик базы знаний интеллектуальной системы считает, что разработанный им компонент сможет стать неотъемлемой частью библиотеки, то компонент будет добавлен в библиотеку, как многократно используемый, в случае, если компонент прошел верификацию и соответствует требованиям разработчиков библиотеки.}
        
        \scnheader{онтология предметной области}
        \scnnote{\textit{онтологии предметных областей}, описывающие виды знаний, которые являются основой для построения \textit{базы знаний} любой \textit{интеллектуальной системы}, входят в \textit{Ядро базы знаний}, поскольку являются \textit{онтологиями верхнего уровня}. Следовательно, \textit{Ядро базы знаний} представляет собой компонент, входящий в состав каждой базы знаний, разрабатываемой по \textit{Технологии OSTIS}, и устанавливающийся в первую очередь.}
        \scnnote{\textit{онтологии предметных областей}, которые используются в большинстве интеллектуальных систем, являются частью \textit{Расширенного ядра базы знаний}.}
        
        \scnheader{многократно используемый компонент базы знаний ostis-систем}
        \scnhaselement{Ядро базы знаний}
        \begin{scnindent}
        	\scnhaselement{Предметная область и онтология множеств}
        	\scnhaselement{Предметная область и онтология связок и отношений}
        	\scnhaselement{Предметная область и онтология структур}
        	\scnhaselement{Предметная область и онтология семантических окрестностей}   
        	\scnhaselement{Предметная область и онтология предметных областей}
        	\scnhaselement{Предметная область и онтология онтологий}
        \end{scnindent}
       
        
        \scnheader{многократно используемый компонент базы знаний ostis-систем}
        \scnhaselement{Расширенное ядро базы знаний}
        \begin{scnindent}
        	\scnrelfrom{включение}{Ядро базы знаний}
        	\scnexplanation{В отличие от \textit{Ядра базы знаний} \textit{Расширенное ядро базы знаний} содержит в себе не только обязательные для установки \textit{онтологии предметных областей}, но и такие \textit{онтологии предметных областей}, которые используются в большинстве \textit{интеллектуальных компьютерных систем}. Следовательно, являются компонентами, которые наиболее часто устанавливаются пользователями \textit{Библиотеки многократно используемых компонентов баз знаний ostis-систем в составе Метасистемы OSTIS}.}
        	\scnhaselement{Предметная область и онтология параметров, величин и шкал}
        	\scnhaselement{Предметная область и онтология чисел и числовых структур}
        	\scnhaselement{Предметная область и онтология темпоральных сущностей}
        	\scnhaselement{Предметная область и онтология пространственных сущностей различных форм}
        	\scnhaselement{Предметная область и онтология материальных сущностей}
	        \scntext{примечание}{Представленный список \textit{многократно используемых компонентов баз знаний} не является окончательным. В случае, когда разработчик \textit{базы знаний} интеллектуальной системы считает, что разработанный им компонент сможет стать неотъемлемой частью библиотеки, то компонент будет добавлен в библиотеку, как многократно используемый, если:
	        \begin{scnitemize}
	        	\item компонент специфицирован;
	        	\item компонент прошел верификацию и соответствует требованиям разработчиков библиотеки
	        \end{scnitemize}
	    	}
	        \scnnote{Чтобы \textit{многократно используемый компонент базы знаний} мог быть принят в библиотеку, он должен быть корректно специфицирован. Для этого используются отношения класса \textit{необходимое для установки отношение, специфицирующее многократно используемый компонент ostis-систем}, а также \textit{необязательное для установки отношение, специфицирующее многократно используемый компонент ostis-систем}. Однако в зависимости от типа компонента, спецификация может расширяться. Рассмотрим \textit{необязательное для установки отношение, специфицирующее многократно используемый компонент базы знаний ostis-систем}, его поиска и установки в \textit{дочернюю ostis-систему}, если таковым компонентом является \textit{предметная область и онтология}.}
    	\end{scnindent}
        \begin{scnrelfromset}{смотрите}
            \scnitem{Предметная область и онтология знаний и баз знаний ostis-систем}
        \end{scnrelfromset}
        
        \scnheader{необязательное для установки отношение, специфицирующее многократно используемый компонент базы знаний ostis-систем}
        \scnsubset{необязательное для установки отношение, специфицирующее многократно используемый компонент ostis-систем}
        \scnhaselement{максимальный класс объектов исследования\scnrolesign}
        \begin{scnindent}
        	\scnidtf{класс объектов исследования, для которого в заданной предметной области отсутствует другой класс объектов исследования, который был бы его надмножеством\scnrolesign}
        	\scnrelfrom{первый домен}{предметная область и онтология}
        	\scnrelfrom{второй домен}{понятие}
        \end{scnindent}
        \scnhaselement{немаксимальный класс объектов исследования\scnrolesign}
        \begin{scnindent}
        	\scnrelfrom{первый домен}{предметная область и онтология}
        	\scnrelfrom{второй домен}{понятие}
        \end{scnindent}
        \scnhaselement{исследуемое отношение\scnrolesign}
        \begin{scnindent}
        	\scnrelfrom{первый домен}{предметная область и онтология}
        	\scnrelfrom{второй домен}{понятие}
        \end{scnindent}
        
        \scnnote{Для компонентов, которые являются частью \textit{Библиотеки многократно используемых компонентов баз знаний ostis-систем в составе Метасистемы OSTIS}, также существуют средства поиска, обновления.} 
        
        \scnnote{Корректно спроектированные спецификации компонентов позволят построить полную иерархию зависимостей компонентов, а также их структуру, что в свою очередь позволит беспрепятственное использование компонентов и их фрагментов в рамках компонентного проектирования баз знаний.}
        
    \end{scnsubstruct}
	\begin{scnrelfromvector}{заключение}
		\scnfileitem{Проектирование и анализ качества \textit{баз знаний} являются важнейшими этапами разработки \textit{интеллектуальных компьютерных систем}, так как они во многом определяют качество всей интеллектуальной системы.}
		\scnfileitem{Предложенная методология коллективной разработки базы знаний на основе \textit{Технологии OSTIS}, которая включает в себя модель верификации и контроля качества \textit{базы знаний}, а также \textit{компонентный подход} к проектированию \textit{баз знаний}, позволяет повысить эффективность проектирования \textit{интеллектуальных компьютерных систем} и средств автоматизации разработки таких систем}
	\end{scnrelfromvector}
\end{SCn}

\begin{SCn}
	\scnsectionheader{Предметная область и онтология действий и методик проектирования решателей задач ostis-систем}
	\scntext{аннотация}{\textit{Предметная область и онтология действий и методик проектирования решателей задач ostis-систем} посвящена методике построения и модификации \textit{решателей задач} \textit{ostis-систем}.}
		
	\begin{scnrelfromvector}{введение}
		\scnfileitem{В области разработки \textit{решателей задач} существует большое количество конкретных реализаций, однако вопросы совместимости различных решателей задачи и их компонентов практически не рассматриваются. Гипотетически возможно существование универсального решателя задач, объединяющего в себе все известные способы и методы решения задач. Однако использование такого решателя в прикладных целях не является целесообразным. Таким образом, наиболее приемлемым вариантом становится создание библиотеки совместимых между собой компонентов, из которых впоследствии может быть собран решатель, удовлетворяющий необходимым требованиям.}
		\begin{scnindent}
			\scnrelfrom{смотрите}{Предметная область и онтология комплексной библиотеки многократно используемых семантически совместимых компонентов ostis-систем}
		\end{scnindent}
		
		\scnfileitem{\scnkeyword{решатель задач ostis-системы} представляет собой иерархическую систему \textit{навыков}, которыми владеет ostis-система на текущий момент. В свою очередь, \textit{навык} представляет собой некоторый \textit{метод} решения задач заданного класса и соответствующую ему систему \textit{sc-агентов} (\textit{машину обработки знаний} частного вида), обеспечивающих интерпретацию данного метода. В связи с этим методика проектирования решателей задач ostis-систем фактически сводится к методике проектирования баз знаний ostis-систем и методике проектирования машин обработки знаний ostis-систем, которая детально рассматривается в данной главе.}
		\begin{scnindent}
			\scnrelfrom{смотрите}{Предметная область и онтология действий и методик проектирования баз знаний ostis-систем}
		\end{scnindent}
			
		\scnfileitem{Вопросам разработки \textit{решателей задач ostis-систем} посвящен ряд работ, таких как \scncite{Shunkevich2013}, \scncite{Shunkevich2015}, \scncite{Shunkevich2015a}, \scncite{Golenkov2015}, \scncite{Shunkevich2018}. Среди других известных работ, посвященных вопросам компонентного проектирования решателей задач в целом, стоит отметить работы \scncite{Borisov2014}, \scncite{Gribova2015a}, в которых, однако, не уделяется внимание разработке комплексной методики проектирования решателей задач, которой посвящена данная глава.}
	\end{scnrelfromvector}

	\scntext{заключение}{В данной предметной области предложена \textit{Методика построения и модификации машины обработки знаний ostis-систем}, которая включает несколько этапов:
	\begin{scnitemize}
		\item формирование требований и спецификация \textit{машины обработки знаний};
		\item формирование коллектива sc-агентов, входящих в состав разрабатываемой машины;
		\item разработка алгоритмов атомарных sc-агентов;
		\item реализация scp-программ;
		\item верификация разработанных компонентов;
		\item отладка разработанных компонентов, исправление ошибок.
	\end{scnitemize}
	
	\textit{Методика построения и модификации машины обработки знаний ostis-систем} основана на онтологии деятельности разработчиков \textit{машины обработки знаний}. Наличие такой онтологии позволяет:
	\begin{scnitemize}
		\item частично автоматизировать процесс построения и модификации \textit{машины обработки знаний};
		\item повысить эффективность информационной поддержки разработчиков, поскольку данная онтология может быть включена в \textit{Базу знаний Метасистемы OSTIS}.
	\end{scnitemize}
	
	Также предложена модель \textit{Системы автоматизации проектирования решателей задач ostis-систем}. Cистема может использоваться тремя способами:
	\begin{scnitemize}
		\item Как подсистема в рамках \textit{Метасистемы OSTIS}. Данный вариант использования предполагает отладку необходимых компонентов в рамках метасистемы с последующим переносом их в \textit{дочернюю ostis-систему}.
		\item Как \textit{самостоятельная ostis-система}, предназначенная исключительно для разработки и отладки компонентов решателей задач. В этом случае проектируемые компоненты отлаживаются в рамках такой системы, а затем должны быть перенесены в дочернюю ostis-систему.
		\item Как подсистема в рамках дочерней ostis-системы. В таком варианте отладка компонентов осуществляется непосредственно в той же системе, в которой предполагается их использование, и дополнительного переноса не требуется.
	\end{scnitemize}
}
	
	\begin{scnrelfromlist}{ключевое понятие}
		\scnitem{решатель задач ostis-системы}
		\scnitem{sc-агент}
		\scnitem{scp-программа}
		\scnitem{точка останова}
		\scnitem{некорректность в scp-программе}
		\scnitem{ошибка в scp-программе}
		\scnitem{многократно используемый компонент решателей задач}
	\end{scnrelfromlist}
	
	\begin{scnrelfromlist}{библиографическая ссылка}
		\scnitem{Shunkevich2013}
		\scnitem{Shunkevich2015}
		\scnitem{Golenkov2015}
		\scnitem{Shunkevich2015a}
		\scnitem{Shunkevich2018}
		\scnitem{Borisov2014}
		\scnitem{Gribova2015a}
		\scnitem{Zalivako2012}
	\end{scnrelfromlist}
	
	\begin{scnsubstruct}
		\begin{scnrelfromlist}{ключевой знак}
			\scnitem{Методика построения и модификации машины обработки знаний ostis-систем}
		\end{scnrelfromlist}
		\scnnote{Основу любого решателя задач составляет система sc-агентов, то есть \textit{машина обработки знаний ostis-системы}, в связи с этим целесообразно рассмотреть более детально процесс проектирования \textit{машин обработки знаний ostis-систем}. Приведенная методика разработана с учетом того, что каждая \textit{машина обработки знаний ostis-систем} представляет собой иерархический коллектив sc-агентов.}
		\scnnote{Построение общей методики проектирования решателей задач ostis-систем требует также уточнения принципов разработки \textit{методов} решения задач, которые, в свою очередь, тесно пересекаются с общими принципами разработки баз знаний ostis-систем.}
		\begin{scnindent}
			\scnrelfrom{смотрите}{Предметная область и онтология действий и методик проектирования баз знаний ostis-систем}
		\end{scnindent}
		
	\scnheader{Методика построения и модификации машин обработки знаний ostis-систем}
	\begin{scnrelfromset}{этапы}
		\scnitem{формирование требований к машине обработки знаний и ее формальная спецификация}
		\begin{scnindent}
		\begin{scnrelfromvector}{этапы}
			\scnfileitem{\textit{Выделение задач}, решение которых должен обеспечивать \textit{решатель задач}.}
			\scnfileitem{\textit{Продумывание предполагаемых способов решения таких задач} и на основе данного анализа определение места будущей машины обработки знаний \textit{решателя задач} в общей иерархии \textit{решателей задач}.}
		\end{scnrelfromvector}	
		\scnnote{Важность данного этапа заключается в том, что при правильной классификации существует вероятность того, что в составе \textit{библиотеки многократно используемых компонентов решателей задач ostis-систем} уже есть реализованный вариант требуемой машины обработки знаний. В противном случае, у разработчика появляется возможность включить разработанную машину в \textit{библиотеку многократно используемых компонентов решателей задач ostis-систем} для последующего использования. Данные факты обусловлены тем, что структура \textit{библиотеки многократно используемых компонентов решателей задач ostis-систем} основана на семантической классификации \textit{решателей задач} и, соответственно, их компонентов.}
		\scnnote{При недостаточно четкой спецификации и классификации разрабатываемого \textit{решателя задач} повышается вероятность того, что подходящая машина обработки знаний не будет найдена в библиотеке компонентов даже в случае, если она там есть, а вновь разработанный \textit{решатель задач} не сможет быть включен в библиотеку. Таким образом, принцип многократного использования уже разработанных компонентов будет нарушен, что существенно повысит затраты на разработку.}
		\end{scnindent}

		\scnitem{формирование коллектива sc-агентов, входящих в состав разрабатываемой машины}
		\begin{scnindent}
			\scnnote{В случае, когда найти в библиотеке готовую машину обработки знаний, удовлетворяющую всем предъявляемым требованиям, не удалось, необходимо выделить (то есть собственно определить множество требуемых компонентов) и специфицировать все ее компоненты.}
			\scntext{результат}{Результатом данного этапа является перечень полностью специфицированных \textit{абстрактных sc-агентов}, которые войдут в состав разрабатываемой машины обработки знаний, с их иерархией вплоть до \textit{атомарных абстрактных sc-агентов}.}
			\scnnote{В рамках данного этапа очень важно проектировать коллектив агентов таким образом, чтобы максимально задействовать уже имеющиеся в библиотеке многократно используемые компоненты ostis-систем, а при отсутствии нужного компонента --- иметь возможность включить его в библиотеку после реализации.}
			\begin{scnrelfromset}{принципы}
				\scnfileitem{Каждый разрабатываемый sc-агент должен быть, по возможности, предметно независим, то есть во множество ключевых узлов данного sc-агента не должны входить понятия, имеющие отношение непосредственно к рассматриваемой предметной области. Исключение составляют понятия из общих предметных областей, которые носят междисциплинарный характер (например, отношение \textit{включение*} или понятие \textit{действие}). Данное правило также может быть нарушено в случае, если sc-агент является вспомогательным и ориентирован на обработку какого-либо конкретного класса объектов (например, sc-агенты, выполняющие арифметические вычисления, могут напрямую работать с конкретными отношениями \textit{сложение*} и \textit{умножение*} и тому подобное). Всю необходимую для решения задачи информацию sc-агент должен извлекать из семантической окрестности соответствующего инициированного действия. Разработанный с учетом указанных требований sc-агент может быть использован при проектировании большего числа ostis-систем, чем в случае, если бы он был реализован с ориентацией на конкретную предметную область. После завершения разработки и отладки такой sc-агент должен быть включен в \textit{библиотеку многократно используемых абстрактных sc-агентов}, которая входит в состав \textit{библиотеки многократно используемых компонентов решателей задач ostis-систем}.}
				\scnfileitem{Не стоит путать понятия sc-агент и \textit{агентная программа} (в том числе \textit{агентная scp-программа}). Взаимодействие sc-агентов осуществляется исключительно посредством спецификации информационных процессов в общей памяти, каждый sc-агент реагирует на некоторый класс событий в sc-памяти. Таким образом, каждому sc-агенту соответствует некоторое условие инициирования и одна агентная программа, которая запускается автоматически при возникновении в sc-памяти соответствующего условия инициирования. При этом в рамках данной программы могут сколько угодно раз вызываться различные подпрограммы. Однако не стоит путать инициирование sc-агента, которое осуществляется при появлении в sc-памяти соответствующей конструкции, и вызов подпрограммы другой программой, который предполагает явное указание вызываемой подпрограммы и перечня ее параметров.}
				\scnfileitem{Каждый sc-агент должен самостоятельно проверять полноту соответствия своего условия инициирования текущему состоянию sc-памяти. В процессе решения какой-либо задачи может возникнуть ситуация, когда на появление одной и той же структуры среагировали несколько sc-агентов. В таком случае выполнение продолжают только те из них, условие инициирования которых полностью соответствует сложившейся ситуации. Остальные sc-агенты в данном случае прекращают выполнение и возвращаются в режим ожидания. Выполнение данного принципа достигается за счет тщательного уточнения спецификаций разрабатываемых sc-агентов. В общем случае условия инициирования у нескольких sc-агентов могут совпадать, например, в случае, когда одна и та же задача может быть решена разными способами и заранее неизвестно, какой из них приведет к желаемому результату.}
				\scnfileitem{Необходимо помнить, что неатомарный sc-агент с точки зрения других sc-агентов, не входящих в его состав, должен функционировать как целостный sc-агент (выполнять логически атомарные действия), что накладывает определенные требования на спецификации атомарных sc-агентов, входящих в его состав: как минимум, необходимо, чтобы в составе неатомарного sc-агента присутствовал хотя бы один атомарный sc-агент, условие инициирования которого полностью совпадает с условием инициирования данного неатомарного sc-агента.}
				\scnitem{принципы выделения атомарных абстрактных \textit{sc-агентов}}
				\begin{scnindent}
					\begin{scneqtoset}
						\scnfileitem{Проектируемый sc-агент должен быть максимально независим от предметной области, что позволит в дальнейшем использовать его при разработке \textit{решателей задач} максимально возможного числа ostis-систем. При этом универсальность предполагает не только минимизацию числа ключевых узлов sc-агента, но и выделение класса действий, выполняемых данным sc-агентом таким образом, чтобы имело смысл включить данный sc-агент в \textit{библиотеку многократно используемых абстрактных sc-агентов} и использовать его при разработке \textit{решателей задач} других ostis-систем. Не следует искусственно увязывать ряд действий в один sc-агент и, наоборот, расчленять одно самодостаточное действие на поддействия: это вызовет сложности восприятия принципов работы sc-агента разработчиками и не позволит использовать sc-агент в ряде систем (например, в обучающих системах, которые должны объяснять ход решения пользователю).}
						\scnfileitem{Выполняемое данным sc-агентом действие должно быть логически целостным и завершенным. Следует помнить, что все sc-агенты взаимодействуют исключительно через общую sc-память, и избегать ситуаций, в которых инициирование одного sc-агента осуществляется путем явной генерации известного условия инициирования другим \textit{sc-агентом} (то есть, по сути, явным непосредственным вызовом одного sc-агента другим).}
						\scnfileitem{Имеет смысл выделять в отдельные sc-агенты относительно крупные фрагменты реализации некоторого общего алгоритма, которые могут выполняться независимо друг от друга.}
					\end{scneqtoset}
				\end{scnindent}
				\scnfileitem{При объединении sc-агентов в коллективы рекомендуется проектировать их таким образом, чтобы они могли быть использованы не только как часть рассматриваемого \textit{неатомарного абстрактного sc-агента}. В случае, если это не представляется возможным и некоторые sc-агенты, будучи отделенными от коллектива, теряют \textit{смысл}, необходимо указать данный факт при документировании рассматриваемых sc-агентов.}
				\scnfileitem{Фактическим инициатором запуска sc-агента посредством общей памяти (автором соответствующей конструкции) может быть как непосредственно пользователь системы, так и другой sc-агент, что никак не должно отражаться в работе самого sc-агента.}
			\end{scnrelfromset}	
		\end{scnindent}

  		\scnitem{разработка алгоритмов атомарных sc-агентов}
	 	\begin{scnindent}
			\begin{scnrelfromset}{предполагает}
				\scnfileitem{\textit{Продумывание алгоритма} работы каждого разрабатываемого \textit{атомарного sc-агента}.}
			\end{scnrelfromset}
			\scnnote{Разработка алгоритма подразумевает выделение в нем логически целостных фрагментов, которые могут быть реализованы как отдельные \textit{scp-программы}, в том числе выполняемые параллельно. Таким образом, появляется необходимость говорить не только о \textit{библиотеке многократно используемых абстрактных sc-агентов}, но и о \textit{библиотеке многократно используемых программ обработки sc-текстов} на различных языках программирования, в том числе \textit{библиотеке многократно используемых scp-программ}. Благодаря этому часть scp-программ, реализующих алгоритм работы некоторого sc-агента, может быть заимствована из соответствующей библиотеки.}
			\scnnote{Важно помнить, что если в процессе работы \textit{sc-агент} генерирует в памяти какие-либо временные структуры, то при завершении работы он обязан удалять всю информацию, использование которой в системе более нецелесообразно (убрать за собой информационный мусор). Исключение составляют ситуации, когда подобная информация необходима нескольким \textit{sc-агентам} для решения одной задачи, однако после решения задачи информация становится бесполезной или избыточной и требует удаления. В данном случае может возникнуть ситуация, когда ни один из \textit{sc-агентов} не в состоянии удалить информационный мусор. В таком случае возникает необходимость говорить о включении в состав \textit{решателя задач} специализированных \textit{sc-агентов}, задачей которых является выявление и уничтожение информационного мусора.}	
		\end{scnindent}
  
  		\scnitem{реализация программ sc-агентов}
		\begin{scnindent}
			\scntext{пояснение}{Конечным этапом непосредственно разработки является реализация специфицированных ранее \textit{scp-программ} или при необходимости программ, реализуемых на уровне \textit{ostis-платформы}.}
		\end{scnindent}

		\scnitem{верификация разработанных компонентов}
		\begin{scnindent}
			\scnnote{Верификация разработанных компонентов может осуществляться как вручную, так и с использованием специфицированных средств, входящих в состав системы автоматизации проектирования \textit{решателей задач} \textit{ostis-систем}.}
		\end{scnindent}
    
		\scnitem{отладка разработанных компонентов, исправление ошибок}
		\begin{scnindent}
				\begin{scnrelfromset}{декомпозиция}
				\scnitem{отладка отдельных scp-программ или программ, реализуемых на уровне платформы}
				\scnitem{отладка отдельных атомарных sc-агентов}
				\scnitem{отладка неатомарных sc-агентов, входящих в состав машины обработки знаний}
				\scnitem{отладка всей машины обработки знаний}
			\end{scnrelfromset}
		\end{scnindent}
  
    
	\end{scnrelfromset}
	\scnnote{\textit{Верификация разработанных компонентов} и \textit{отладка разработанных компонентов и исправление ошибок} могут выполняться параллельно и повторяются до тех пор, пока разработанные компоненты не будут удовлетворять необходимым требованиям.}
	\scnnote{Предлагаемая методика может быть применена как при разработке \textit{машины обработки знаний} объединенных \textit{решателей задач}, так и при разработке \textit{машины обработки знаний} \textit{решателей задач} частного вида, поскольку с формальной точки зрения все такие машины трактуются как \textit{неатомарные абстрактные sc-агенты}.}
	\scnnote{\textit{Методика построения и модификации машин обработки знаний ostis-систем} основана на онтологии деятельности разработчиков машин обработки знаний.}
	\begin{scnindent}
		\scnnote{Наличие такой онтологии позволяет:
		\begin{scnitemize}
			\item частично автоматизировать процесс построения и модификации машины обработки знаний при помощи соответствующей \textit{Системы автоматизации проектирования решателей задач ostis-систем};
			\item повысить эффективность информационной поддержки разработчиков, поскольку данная онтология включена в базу знаний \textit{Метасистемы OSTIS}.
		\end{scnitemize}}
	\end{scnindent}
				
	\scnheader{действие. разработать машину обработки знаний ostis-системы}
	\scntext{пояснение}{Каждый \textit{решатель задач ostis-системы} представляет собой совокупность навыков, а машина обработки знаний --- совокупность интерпретаторов навыков, составляющих некоторый \textit{решатель задач ostis-системы}, то есть его операционную семантику. Таким образом \textit{машина обработки знаний} представляет собой \textit{абстрактный sc-агент}, в связи с чем разработка машины сводится к разработке такого агента.}
	\scnsubset{действие. разработать абстрактный sc-агент}
	\begin{scnrelfromset}{разбиение}
		\scnitem{действие. разработать атомарный абстрактный sc-агент}
		\begin{scnindent}
			\scnsuperset{действие. разработать платформенно-независимый атомарный абстрактный sc-агент}
		\end{scnindent}
		\scnitem{действие. разработать неатомарный абстрактный sc-агент}
	\end{scnrelfromset}
	\begin{scnrelfromlist}{обобщенное поддействие}
		\scnitem{действие. специфицировать абстрактный sc-агент}
		\scnitem{действие. найти в библиотеке абстрактный sc-агент, удовлетворяющий заданной спецификации}
		\scnitem{действие. верифицировать sc-агент}
		\scnitem{действие. отладить sc-агент}
	\end{scnrelfromlist}
		
	\scnheader{действие. разработать платформенно-независимый атомарный абстрактный sc-агент}
	\begin{scnrelfromlist}{обобщенное поддействие}
		\scnitem{действие. декомпозировать платформенно-независимый атомарный абстрактный sc-агент на scp-программы}
		\scnitem{действие. разработать scp-программу}
	\end{scnrelfromlist}
		
	\scnheader{действие. разработать неатомарный абстрактный sc-агент}
	\begin{scnrelfromlist}{обобщенное поддействие}
		\scnitem{действие. декомпозировать неатомарный абстрактный sc-агент}
		\scnitem{действие. разработать абстрактный sc-агент}
	\end{scnrelfromlist}
	
	\scnheader{действие. разработать scp-программу}
	\begin{scnrelfromlist}{обобщенное поддействие}
		\scnitem{действие. специфицировать scp-программу}
		\scnitem{действие. найти в библиотеке scp-программу, удовлетворяющую заданной спецификации}
		\scnitem{действие. реализовать специфицированную scp-программу}
		\scnitem{действие. верифицировать scp-программу}
		\scnitem{действие. отладить scp-программу}
	\end{scnrelfromlist}
	
	\scnheader{действие. верифицировать sc-агент}
	\begin{scnrelfromset}{разбиение}
		\scnitem{действие. верифицировать атомарный sc-агент}
		\scnitem{действие. верифицировать неатомарный sc-агент}
	\end{scnrelfromset}
	
	\scnheader{действие. отладить sc-агент}
	\begin{scnrelfromset}{разбиение}
		\scnitem{действие. отладить атомарный sc-агент}
		\scnitem{действие. отладить неатомарный sc-агент}
	\end{scnrelfromset}

	\end{scnsubstruct}
\end{SCn}

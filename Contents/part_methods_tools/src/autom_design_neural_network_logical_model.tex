\begin{SCn}
    \scnsectionheader{Логико-семантическая модель ostis-системы автоматизации проектирования искусственных нейронных сетей, семантически совместимых с базами знаний ostis-систем}
    \begin{scnsubstruct}
        \scntext{введение}{
            Наличия \textit{Языка представления нейросетевых методов в базах знаний} и его интерпретатора позволяет обеспечить интерпретацию \textit{нейросетевого метода} в памяти \textit{ostis-системы}. Наличие в единой памяти не только экземпляров методов, но и понятий, их описывающих, создает основу для автоматизации процесса построения нейросетевых методов. В памяти \textit{ostis-системы} хранятся знания о том, методы какого класса могут решить задачу заданного класса, но экземпляров класса этого метода может не быть представлено в системе. На этот случай система должна иметь возможность сообщить пользователю о возможности решения, для которого, однако, необходимо погрузить в систему определенный метод. Так как система хранит в единой памяти задачу и требования к методу ее решения, появляется возможность спроектировать необходимый метод. Для этого необходимо наличие среды проектирования методов соответствующих классов. В случае \textit{нейросетевого метода}, речь идет об интеллектуальной среде построения \textit{нейросетевых методов}.\\
            В основе интеллектуальной среды построения \textit{нейросетевых методов} лежат соответствующие другу другу иерархии действий, задач и методов построения \textit{и.н.с.} Наличие такой иерархии позволит описать язык представления методов построения \textit{и.н.с.} и разработать интерпретатор этого языка.\\
            Построение иерархии соответствующих действий построения \textit{и.н.с.} следует начать с изучения этапов проектирования и обучения \textit{и.н.с.}, которые, в общем случае, выполняют все разработчики и.н.с.
        }

        \scnsegmentheader{Логико-семантическая модель ostis-системы автоматизации проектирования искусственных нейронных сетей, семантически совместимых с базами знаний ostis-систем}
        \begin{scnsubstruct}

            \scnheader{Предметная область интеллектуальной среды построения нейросетевых методов}
            \scnidtf{Предметная область фреймворка нейросетей}
            \scniselement{предметная область}
            \begin{scnhaselementrole}{максимальный класс объектов исследования}
            {интеллектуальная среда построения нейросетевых методов}
            \end{scnhaselementrole}
            \begin{scnhaselementrolelist}{класс объектов исследования}
                \scnitem{интеллектуальная среда построения нейросетевых методов}
                \scnitem{постановка задачи}
                \scnitem{предобработка выборки}
                \scnitem{разбиение выборки на обучающую, валидационную и тестовую (контрольную)}
                \scnitem{выбор класса нейросетевых методов в соответствии со сформулированной задачей}
                \scnitem{формирование спецификации на входные и выходные данные}
                \scnitem{выбор метода оптимизации}
                \scnitem{выбор минимизируемой функции ошибки}
                \scnitem{начальная инициализация параметров нейронной сети}
                \scnitem{выбора гиперпараметров и.н.с.}
                \scnitem{обучение модели на обучающей выборке}
                \scnitem{оценка эффективности и.н.с}
                \scnitem{действие трансляции условия задачи}
                \scnitem{действие классификации задачи}
                \scnitem{действие поиска подходящей обучающей выборки}
                \scnitem{действие формирования требований к обучающей выборке}

            \end{scnhaselementrolelist}

            \scnheader{интеллектуальная среда построения нейросетевых методов}
            \begin{scnrelfromset}{этапы построения нейросетевых методов}
                \begin{scnindent}
                    \scnidtf{этапы построения и.н.с}
                    \scnitem{постановка задачи}
                    \scnitem{предобработка выборки}
                    \scnitem{разбиение выборки на обучающую, валидационную и тестовую (контрольную)}
                    \scnitem{выбор класса нейросетевых методов в соответствии со сформулированной задачей}
                    \scnitem{формирование спецификации на входные и выходные данные}
                    \scnitem{выбор метода оптимизации}
                    \scnitem{выбор минимизируемой функции ошибки}
                    \scnitem{начальная инициализация параметров нейронной сети}
                    \scnitem{выбора гиперпараметров и.н.с.}
                    \scnitem{обучение модели на обучающей выборке} %todo: Ask(уже формализована с стандарте sd-ann, что делать) --- ответ(перенести из sd_ann и удалить соответственно в \item предметной области действий соответствующие объйкты)
                    \scnitem{оценка эффективности и.н.с}
                \end{scnindent}
            \end{scnrelfromset}

            %\scnheader{первый пункт(все как у меня в oper_sem), дальше пишешь все, как там}

            \scnheader{постановка задачи}
            \begin{scnrelfromset}{содержимое}
                \scnitem{входные данные}
                \begin{scnindent}
                    \scntext{пример}{изображения/видео, временные ряды, текст}
                \end{scnindent}
                \scnitem{выходные данные}
                \scnitem{требования к методу решения}
                \begin{scnindent}
                    \scntext{пример}{скорость, затраты по памяти и так далее}
                \end{scnindent}
                \scnitem{дополнительная информация}
                \begin{scnindent}
                    \scntext{пояснение}{информация, которая может помочь в построении метода решения задачи}
                \end{scnindent}
            \end{scnrelfromset}
            \begin{scnrelfromset}{декомпозиция}
                \scnitem{\textbf{действие трансляции условия задачи}}
                \scnitem{\textbf{действие классификации задачи}}
                \scnitem{\textbf{действие поиска подходящей обучающей выборки}}
                \scnitem{\textbf{действие формирования требований к обучающей выборке}}
            \end{scnrelfromset}


            \scnheader{действие трансляции условия задачи}
            \scniselement{действие}
            \scntext{пояснение}{Действие транслирует заданное с помощью \textit{интерфейса ostis-системы} (к примеру, естественно-языкового интерфейса) описание задачи в память ostis-системы. Действие необходимо в случае, когда условие задачи задается пользователем. Необходимо понимать, что описание задачи поступает в базу знаний не только от \textit{пользовательского интерфейса}. К примеру, задача может быть сформулирована самой системой в ходе ее жизнедеятельности.}
            \scntext{пояснение}{Данное действие является общим для всех ostis-систем, поэтому его рассмотрение выходит за рамки рассмотрения процесса построения интеллектуальной среды проектирования \textit{и.н.с.}}

            \scnheader{действие классификации задачи}
            \scniselement{действие}
            \scntext{пояснение}{Действие определяет класс задачи (задача регрессии, детекции, кластеризации и так далее), исходя из описания задачи в базе знаний.}

            \scnheader{действие поиска подходящей обучающей выборки}
            \scniselement{действие}
            \scntext{пояснение}{В базе знаний может храниться набор спецификаций выборок, к которым у ostis-системы есть доступ. Действие производит поиск выборок, которые могут быть использованы в качестве обучающей выборки.}

            \scnheader{действие формирования требований к обучающей выборке}
            \scniselement{действие}
            \scntext{пояснение}{Если обучающая выборка не была предоставлена и не была найдена, то необходимо сформировать описание требований к обучающей выборке, которое можно будет транслировать на язык пользовательского интерфейса и запросить необходимую выборку у пользователя.}

            \scnheader{предобработка выборки}
            \begin{scnrelfromset}{этапы}
                \scnitem{очистка} %1
                \begin{scnindent}
                    \begin{scnrelfromset}{содержимое}
                        \scnitem{обнаружение признаков, которые имеют в общем случае некорректные значения}
                        \begin{scnindent}
                            \scntext{пример}{для каких-то образов значение признака может иметь неопределенное значение, либо значение, не совпадающее по типу, либо аномально большое или очень маленькое значение, которое встречается в редком числе случаев}
                        \end{scnindent}
                        \scnitem{методы устранения признаков, имеющие неопределённые значения}
                        \begin{scnindent}
                            \scntext{пояснение}{значения могут быть заменены средним значением этого признака, рассчитанным по всем образам (для непоследовательных данных), либо они могут быть заменены средним значением по соседним образам (в случае временных рядов), либо каким-то фиксированным значением}
                            \scntext{радикальное решение}{удаление образов, имеющих неопределенные значения признаков из выборки}
                            \begin{scnindent}
                                \scntext{примечание}{однако его лучше применять, если образов с отсутствующими значениями признаков немного. Для выбросов и аномалий применяются схожие стратегии (но только в том случае, если задача не состоит в прогнозировании этих аномалий)}
                            \end{scnindent}
                        \end{scnindent}
                    \end{scnrelfromset}
                    \scntext{примечание}{в интеллектуальной среде проектирования данный этап соответствует выполнению \textbf{\textit{действия очистки выборки}}, которое выполняется в случае обработки выборки, которая ранее не была представлена в памяти системы (к примеру, была получена от пользователя). Реализация интерпретатора (агента) данного действия требует описания в памяти классификации стратегий очистки данных и реализации методов применения этих стратегий} %todo действия очистки выборки - непонятно где в стандарте, что формализовывать
                \end{scnindent}
                \scnitem{выявление содержательных признаков} %2
                \begin{scnindent}
                    \scntext{цель}{уменьшение размерности пространства признаков для снижения влияния эффекта переобучения на модель}
                    \begin{scnindent}
                        \scntext{реализация}{использование методов отбора признаков и выделения признаков}
                    \end{scnindent}
                    \begin{scnrelfromset}{содержимое} %todo возможно надо убрать тут scnrelfromset тк всего один элемент и как-то переделать
                        \scnitem{инжиниринг признаков, состоящий в отборе признаков, влияющих на результат работы модели}
                        \begin{scnindent}
                            \scntext{примечание}{несодержательные признаки, которые никак не коррелируют с выходом модели, удаляются}
                            \begin{scnrelfromset}{содержимое}
                                \scnitem{формирование подмножества из исходных признаков (алгоритм последовательного обратного отбора, рекурсивный алгоритм обратного устранения признаков,  алгоритмы с использованием случайных лесов)}
                                \scnitem{извлечение информации для построения нового подпространства признаков (алгоритмы с использованием автоэнкодера)}
%                                \scnitem{}
                            \end{scnrelfromset}
                        \end{scnindent}
                    \end{scnrelfromset}
                    \scntext{примечание}{в интеллектуальной среде проектирования данный этап соответствует выполнению \textbf{\textit{действия выявления содержательных признаков}}. Реализация интерпретатора (агента) данного действия требует описания в памяти классификации стратегий уменьшения размерности признакового пространства и реализации методов применения этих стратегий}
                \end{scnindent}
                \scnitem{трансформация} %3
            \end{scnrelfromset}

            \bigskip
        \end{scnsubstruct}

        \scnendsegmentcomment{Операционная семантика sc-моделей искусственных нейронных сетей, используемых в ostis-системах}

        \bigskip
    \end{scnsubstruct}

    \scnendcurrentsectioncomment


\end{SCn}
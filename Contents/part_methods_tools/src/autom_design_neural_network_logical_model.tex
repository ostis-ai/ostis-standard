\usepackage{scn}\begin{SCn}
                    \scnsectionheader{Логико-семантическая модель ostis-системы автоматизации проектирования искусственных нейронных сетей, семантически совместимых с базами знаний ostis-систем}
                    \begin{scnrelfromvector}{введение}
                        \scnfileitem{Наличия \textit{Языка представления нейросетевых методов в базах знаний} и его интерпретатора позволяет обеспечить интерпретацию \textit{нейросетевого метода} в памяти \textit{ostis-системы}. Наличие в единой памяти не только экземпляров методов, но и понятий, их описывающих, создает основу для автоматизации процесса построения нейросетевых методов. В памяти \textit{ostis-системы} хранятся знания о том, методы какого класса могут решить задачу заданного класса, но экземпляров класса этого метода может не быть представлено в системе. На этот случай система должна иметь возможность сообщить пользователю о возможности решения, для которого, однако, необходимо погрузить в систему определенный метод. Так как система хранит в единой памяти задачу и требования к методу ее решения, появляется возможность спроектировать необходимый метод. Для этого необходимо наличие среды проектирования методов соответствующих классов. В случае \textit{нейросетевого метода}, речь идет об интеллектуальной среде построения \textit{нейросетевых методов}.}
                        \scnfileitem{В основе интеллектуальной среды построения \textit{нейросетевых методов} лежат соответствующие другу другу иерархии действий, задач и методов построения \textit{и.н.с.} Наличие такой иерархии позволит описать язык представления методов построения \textit{и.н.с.} и разработать интерпретатор этого языка.}
                        \scnfileitem{Построение иерархии соответствующих действий построения \textit{и.н.с.} следует начать с изучения этапов проектирования и обучения \textit{и.н.с.}, которые, в общем случае, выполняют все разработчики и.н.с.}
                    \end{scnrelfromvector}

                    \begin{scnrelfromlist}{ключевое понятие}
                        \scnitem{действие трансляции условия задачи}
                        \scnitem{действие классификации задачи}
                        \scnitem{действие поиска подходящей обучающей выборки}
                        \scnitem{действие формирования требований к обучающей выборке}
                        \scnitem{действие очистки выборки}
                        \scnitem{действие выявления содержательных признаков}
                        \scnitem{действие трансформации выборки}
                        \scnitem{действие разбиения выборки}
                        \scnitem{действие выбора класса нейросетевых методов}
                        \scnitem{действие формирования спецификации входов и выходов и.н.с.}
                        \scnitem{действие выбора метода оптимизации}
                        \scnitem{действие выбора минимизируемой функции ошибки}
                        \scnitem{действие начальной инициализации и.н.с.}
                        \scnitem{действие выбора гиперпараметров и.н.с.}
                        \scnitem{метод обучения с учителем}
                        \scnitem{метод обучения без учителя}
                        \scnitem{действие обучения и.н.с.}
                    \end{scnrelfromlist}

                    \begin{scnsubstruct}
                        %todo: Ask (возможно не как предметная область)
                        \scnheader{Предметная область действий и методик проектирования автоматизации проектирования искусственных нейронных сетей, семантически совместимых с базами знаний ostis-систем}
                        \scniselement{предментная область}
                        \scnhaselementrole{максимальный класс объектов исследования}{}  %todo
                        \begin{scnhaselementrolelist}{класс объектов исследования}
                            \scnitem{}  %todo
                        \end{scnhaselementrolelist}
                    \end{scnsubstruct}

                    \scnheader{Проектирование и обучения и.н.с.}
                    \begin{scnrelfromset}{этапы}
                        \scnitem{Постановка задачи}
                        \begin{scnindent}

                        \end{scnindent}
                        \scnitem{}
                    \end{scnrelfromset}
\end{SCn}
\begin{SCn}
    \scnsectionheader{Предметная область и онтология комплексной библиотеки многократно используемых семантически совместимых компонентов ostis-систем}
    \begin{scnsubstruct}
        \scntext{эпиграф}{Всё, что можно сделать одинаково, нужно делать одинаково.}
        \scntext{аннотация}{Важнейшим этапом эволюции любой технологии является переход к компонентному проектированию на основе постоянно пополняемой библиотеки многократно используемых компонентов. Идея библиотеки компонентов не нова, но семантическая мощность \textbf{\textit{Библиотеки Экосистемы OSTIS}} значительно выше аналогов за счет того, что подавляющее большинство компонентов библиотеки --- компоненты \textit{базы знаний}, представленные на унифицированном языке смыслового представления знаний (\textit{SC-коде}). Таким образом, в Библиотеке Экосистемы OSTIS обеспечивается высокий уровень семантической совместимости компонентов, что приводит к высокому уровню семантической совместимости \textit{ostis-систем}, использующих комплексную библиотеку многократно используемых семантически совместимых компонентов ostis-систем.}
       	\begin{scnreltovector}{конкатенация сегментов}
       	\scnitem{Сегмент. Введение в Предметную область и онтологию комплексной библиотеки многократно используемых семантически совместимых компонентов ostis-систем}
       	\begin{scnindent}
       		\scnidtf{Сегмент. Введение в компонентное проектирование интеллектуальных компьютерных систем}
       	\end{scnindent}
       	\scnitem{Сегмент. Анализ библиотек многократно используемых компонентов}
       	\scnitem{Сегмент. Понятие библиотеки многократно используемых компонентов ostis-систем}
       	\scnitem{Сегмент. Понятие многократно используемого компонента ostis-систем}
       	\scnitem{Сегмент. Уточнение спецификации многократно используемого компонента ostis-систем}
       	\scnitem{Сегмент. Понятие менеджера многократно используемых компонентов ostis-систем}
       	\scnitem{Сегмент. Заключение в Предметную область и онтологию комплексной библиотеки многократно используемых семантически совместимых компонентов ostis-систем}
       	\begin{scnindent}
       		\scnidtf{Сегмент. Заключение в компонентное проектирование интеллектуальных компьютерных систем}
       	\end{scnindent}
       \end{scnreltovector}
       \begin{scnrelfromlist}{библиография}
	       	\scnitem{\scncite{Zhitko2011a}}
	       	\scnitem{\scncite{Zalivako2012}}
	       	\scnitem{\scncite{Borisov2014}}
	       	\scnitem{\scncite{Golenkov2015}}
	       	\scnitem{\scncite{Zhitko2011b}}
	       	\scnitem{\scncite{Golenkov2013}}
	       	\scnitem{\scncite{Shunkevich2015a}}
	       	\scnitem{\scncite{Zalivako2011}}
	       	\scnitem{\scncite{Golenkov2014a}}
	       	\scnitem{\scncite{Iyengar2021}}
	       	\scnitem{\scncite{Ford2019}}
	       	\scnitem{\scncite{Wilkes1951}}
	       	\scnitem{\scncite{Wilkes1953}}
	       	\scnitem{\scncite{Volchenskova1984}}
	       	\scnitem{\scncite{Blahser2021}}
	       	\scnitem{\scncite{Hepp2008}}
	       	\scnitem{\scncite{Memduhoglu2018}}
	       	\scnitem{\scncite{Fritzson2014}}
	       	\scnitem{\scncite{Prakash2022}}
	       	\scnitem{\scncite{Moskalenko2016}}
	       	\scnitem{\scncite{Pivovarchik2015}}
	       	\scnitem{\scncite{Koronchik2011}}
	       	\scnitem{\scncite{Davydenko2013}}
	       	\scnitem{\scncite{Ivashenko2013}}
	       	\scnitem{\scncite{Golenkov2014b}}
	       	\scnitem{\scncite{Orlov2022a}}
	       	\scnitem{\scncite{Golenkov2011}}
	       	\scnitem{\scncite{Ivashenko2011}}
	       	\scnitem{\scncite{Lazyrkin2011}}
	       	\scnitem{\scncite{Davydenko2011}}
	       	\scnitem{\scncite{Koronchik2012}}
	       	\scnitem{\scncite{Shunkevich2013}}
	       	\scnitem{\scncite{Koronchik2013}}
	       	\scnitem{\scncite{Eliseeva2013}}
	       	\scnitem{\scncite{Shunkevich2015}}
	       	\scnitem{\scncite{Davydenko2017}}
	       	\scnitem{\scncite{Davydenko2018}}
	       	\scnitem{\scncite{Tu1995}}
	       	\scnitem{\scncite{Studer1996}}
	       	\scnitem{\scncite{Benjamins1999}}
       	\end{scnrelfromlist}
		\scniselement{раздел базы знаний}
		\scnhaselementrole{ключевой sc-элемент}{Предметная область многократно используемых компонентов ostis-систем}
       
        \scnheader{Предметная область многократно используемых компонентов ostis-систем}
        \scniselement{предметная область}
        \begin{scnhaselementrolelist}{максимальный класс объектов исследования}
        	\scnitem{библиотека многократно используемых компонентов ostis-систем}
        	\scnitem{многократно используемый компонент ostis-систем}
        \end{scnhaselementrolelist}
        \begin{scnhaselementrolelist}{класс объектов исследования}
            \scnitem{многократно используемый компонент базы знаний}
            \scnitem{многократно используемый компонент решателя задач}
            \scnitem{многократно используемый компонент пользовательского интерфейса}
            \scnitem{атомарный многократно используемый компонент ostis-систем}
            \scnitem{неатомарный многократно используемый компонент ostis-систем}
            \scnitem{зависимый многократно используемый компонент ostis-систем}
            \scnitem{независимый многократно используемый компонент ostis-систем}
            \scnitem{платформенно-независимый многократно используемый компонент ostis-систем}
            \scnitem{платформенно-зависимый многократно используемый компонент ostis-систем}
            \scnitem{многократно используемый компонент ostis-систем, хранящийся в виде внешних файлов}
            \scnitem{многократно используемый компонент ostis-систем, хранящийся в виде файлов исходных текстов}
            \scnitem{многократно используемый компонент ostis-систем, хранящийся в виде скомпилированных файлов}
            \scnitem{многократно используемый компонент, хранящийся в виде sc-структуры}
            \scnitem{типовая подсистема ostis-систем;платформа интерпретации sc-моделей компьютерных систем}
            \scnitem{динамически устанавливаемый многократно используемый компонент ostis-систем}
            \scnitem{многократно используемый компонент, при установке которого система требует перезапуска}
            \scnitem{хранилище многократно используемого компонента ostis-систем, хранящегося в виде внешних файлов}
            \scnitem{хранилище многократно используемого компонента ostis-систем, хранящегося в виде файлов исходных текстов}
            \scnitem{хранилище многократно используемого компонента ostis-систем, хранящегося в виде скомпилированных файлов}
            \scnitem{спецификация многократно используемого компонента ostis-систем}
            \scnitem{отношение, специфицирующее многократно используемый компонент ostis-систем}
            \scnitem{параметр, заданный на многократно используемых компонентах ostis-систем\scnsupergroupsign}
            \scnitem{файл, содержащий url-адрес многократно используемого компонента ostis-систем}
            \scnitem{менеджер многократно используемых компонентов ostis-систем}
        \end{scnhaselementrolelist}
        \begin{scnhaselementrolelist}{исследуемое отношение}
            \scnitem{метод установки*}
            \scnitem{адрес хранилища*}
            \scnitem{зависимости компонента*}
            \scnitem{установленные компоненты*}
        \end{scnhaselementrolelist}
        \begin{scnhaselementrolelist}{исследуемый параметр}
            \scnitem{класс многократно используемого компонента ostis-систем\scnsupergroupsign}
        \end{scnhaselementrolelist}
        \begin{scnhaselementrolelist}{отношение, используемое в предметной области}
            \scnitem{автор*}
            \scnitem{ключевой sc-элемент\scnrolesign}
            \scnitem{пояснение*}
            \scnitem{sc-идентификатор*}
            \scnitem{история изменений*}
        \end{scnhaselementrolelist}

\scnheader{семантически смежный раздел*}
\begin{scnhaselementset}
	\scnitem{Предметная область и онтология комплексной библиотеки многократно используемых семантически совместимых компонентов ostis-систем}
	\scnitem{Логико-семантическая модель Метасистемы OSTIS}
\end{scnhaselementset}
\begin{scnindent}
	\scntext{пояснение}{\textit{Метасистема OSTIS} ориентирована на разработку и практическое внедрение методов и средств компонентного проектирования семантически совместимых интеллектуальных систем, которая предоставляет возможность быстрого создания интеллектуальных приложений различного назначения. В состав Метасистемы OSTIS входит \scnkeyword{Библиотека Метасистемы OSTIS}. Сферы практического применения технологии компонентного проектирования семантически совместимых интеллектуальных систем ничем не ограничены.}
	\scntext{пояснение}{Основу для реализации компонентного подхода в рамках \textit{Технологии OSTIS} составляет \scnkeyword{Библиотека Метасистемы OSTIS}.}
\end{scnindent}
\scnheader{семантически смежный раздел*}
\begin{scnhaselementset}
	\scnitem{Предметная область и онтология комплексной библиотеки многократно используемых семантически совместимых компонентов ostis-систем}
	\scnitem{Предметная область и онтология комплексной технологии поддержки жизненного цикла интеллектуальных компьютерных систем нового поколения}
\end{scnhaselementset}
\begin{scnindent}
	\scntext{пояснение}{Основным требованием, предъявляемым к \textit{Технологии OSTIS}, является обеспечение возможности совместного использования в рамках ostis-систем различных \textit{видов знаний} и различных \textit{моделей решения задач} с возможностью \uline{неограниченного} расширения перечня используемых в ostis-системе видов знаний и моделей решения задач без существенных трудозатрат. Следствием данного требования является необходимость реализации компонентного подхода на всех уровнях, от простых компонентов баз знаний и решателей задач до целых встраиваемых ostis-систем.}
\end{scnindent}
\scnheader{дочерний раздел*}
\begin{scnhaselementvector}
	\scnitem{Предметная область и онтология комплексной библиотеки многократно используемых семантически совместимых компонентов ostis-систем}
	\scnitem{Предметная область и онтология многократно используемых компонентов баз знаний ostis-систем}
\end{scnhaselementvector}
\begin{scnindent}
	\scntext{пояснение}{На сегодняшний день разработано большое число \textit{баз знаний} по самым различным предметным областям. Однако в большинстве случаев каждая база знаний разрабатывается отдельно и независимо от других, вследствие отсутствия единой унифицированной формальной основы для представления знаний, а также единых принципов формирования систем понятий для описываемой предметной области. В связи с этим разработанные базы оказываются, как правило, несовместимы между собой и непригодны для повторного использования. Компонентный подход к разработке интеллектуальных компьютерных систем, реализуемый в виде \scnkeyword{библиотеки многократно используемых компонентов ostis-систем}, позволяет решить описанные проблемы.}
\end{scnindent}
\scnheader{дочерний раздел*}
\begin{scnhaselementvector}
	\scnitem{Предметная область и онтология комплексной библиотеки многократно используемых семантически совместимых компонентов ostis-систем}
	\scnitem{Предметная область и онтология многократно используемых компонентов решателей задач ostis-систем}
\end{scnhaselementvector}
\begin{scnindent}
	\scntext{пояснение}{В области разработки \textit{решателей задач} существует большое количество конкретных реализаций, однако вопросы совместимости различных решателей при решении одной задачи практически не рассматриваются.}
\end{scnindent}
\scnheader{дочерний раздел*}
\begin{scnhaselementvector}
	\scnitem{Предметная область и онтология комплексной библиотеки многократно используемых семантически совместимых компонентов ostis-систем}
	\scnitem{Предметная область и онтология многократно используемых компонентов интерфейсов ostis-систем}
\end{scnhaselementvector}

\begin{SCn}
	\scnsectionheader{Сегмент. Введение в Предметную область и онтологию комплексной библиотеки многократно используемых семантически совместимых компонентов ostis-систем}

	\begin{scnsubstruct}
	
	\scnheader{компонентное проектирование интеллектуальных компьютерных систем}
	\begin{scnrelfromset}{основные положения}
		\scnfileitem{Важнейшим этапом эволюции любой технологии является переход к компонентному проектированию на основе постоянно пополняемый библиотеки многократно используемых компонентов.}
		\begin{scnindent}
			\begin{scnrelfromset}{необходимые требования}
				\scnfileitem{Универсальный язык представления знаний.}
				\scnfileitem{Универсальная процедура интеграции знаний в рамках указанного языка.}
				\scnfileitem{Разработка стандарта, обеспечивающего семантическую совместимость интегрируемых знаний (таким стандартом является согласованная система используемых понятий).}
			\end{scnrelfromset}
		\end{scnindent}
		\scnfileitem{Повторное использование готовых компонентов широко применяется во многих отраслях, связанных с проектированием различного рода систем, поскольку позволяет уменьшить трудоемкость разработки и ее стоимость (путем минимизации количества труда за счет отсутствия необходимости разрабатывать какой-либо компонент), повысить качество создаваемого контента и снизить профессиональные требования к разработчикам компьютерных систем. Таким образом, осуществляется  переход от программирования компонентов или целых систем к их проектированию (дизайну, сборке) на основе готовых компонентов. \textbf{\textit{компонентное проектирование интеллектуальных компьютерных систем}} предполагает подбор существующих компонентов, способных решить поставленную задачу целиком или декомпозицию задачи на подзадачи с выделением компонентов для каждой из них.}
		\begin{scnindent}
			\begin{scnrelfromlist}{источник}
				\scnitem{\cite{Zhitko2011b}}
				\scnitem{\cite{Zalivako2012}}
				\scnitem{\cite{Borisov2014}}
			\end{scnrelfromlist}
		\end{scnindent}
	\end{scnrelfromset}
	\scntext{назначение}{Позволяет уменьшить трудоемкость создания компьютерных систем и их стоимость (путем минимизации количества труда за счет отсутствия необходимости разрабатывать какой-либо компонент), повысить качество создаваемых компьютерных систем и снизить профессиональные требования к разработчикам этих систем.}      
	\scntext{пояснение}{Компонентное проектирование интеллектуальных компьютерных систем предполагает подбор существующих компонентов, способных решить поставленную задачу целиком или декомпозицию задачи на подзадачи с выделением компонентов для каждой из них.}  
	\scntext{преимущество}{Проектируемые системы по предлагаемой технологии обладают высоким уровнем гибкости, их разработка осуществляется поэтапно, переходя от одной целостной версии системы к другой. При этом стартовая версия системы может быть ядром соответствующего класса систем, входящим в библиотеку многократно используемых компонентов.}
	\scnheader{технология компонентного проектирования интеллектуальных компьютерных систем}		
	\scnhaselementrole{главный ключевой sc-элемент}{библиотека совместимых многократно используемых компонентов}
	\scntext{преимущество}{Позволяет проектировать интеллектуальные системы, комбинируя уже существующие компоненты, выбирая нужные из соответствующих библиотек. Использование готовых компонентов предполагает, что распространяемый компонент верифицирован и документирован, а возможные ошибки и ограничения устранены либо специфицированы и известны. Создание \textit{библиотеки многократно используемых компонентов} не означает пересоздание всех уже существующих современных продуктов информационных технологий. Технология компонентного проектирования интеллектуальных компьютерных систем предполагает использование огромного опыта в разработке современных компьютерных систем, однако обязательным является \uline{спецификация} каждого компонента (как вновь созданного, так и интегрируемого существующего) для обеспечения возможности его установки и совместимости с другими компонентами и системами. Тем не менее эффективная технология компонентного проектирования появится только тогда, когда сформируется \scnqq{критическая масса} разработчиков прикладных систем, участвующих в пополнении \textit{библиотек многократно используемых компонентов} проектируемых систем.}
	\begin{scnindent}
		\begin{scnrelfromlist}{источник}
			\scnitem{\cite{Zhitko2011b}}
			\scnitem{\cite{Golenkov2013}}
		\end{scnrelfromlist}
	\end{scnindent}
	\scnrelfrom{проблемы текущего состояния}{Проблемы в реализации компонентного проектирования интеллектуальных компьютерных систем}
	\begin{scnindent}
		\scntext{примечание}{Проблемы реализации компонентного подхода к проектированию интеллектуальных компьютерных систем наследуют проблемы современных \textit{технологий проектирования интеллектуальных систем}.}
		\begin{scneqtoset}
			\scnfileitem{\uline{Несовместимость} компонентов, разработанных в рамках разных проектов, вследствие отсутствия унификации в принципах представления различных видов знаний в рамках одной \textit{базы знаний}, и, как следствие, отсутствие унификации в принципах выделения и спецификации \textbf{\textit{многократно используемых компонентов}}, которое приводит к несовместимости компонентов, разработанных в рамках разных проектов.}
			\begin{scnindent}
				\scntext{примечание}{Большинство существующих систем создано как автономные программные продукты, которые не могут быть использованы в качестве компонентов других систем. Необходимо использовать либо целую систему, либо ничего. Небольшое число систем поддерживает компонентно-ориентированную архитектуру способную интегрироваться с другими системами. Однако их интеграция возможна при условии использования одинаковых технологий и только при проектировании одной командой разработчиков.}
				\begin{scnindent}
					\begin{scnrelfromlist}{источник}
						\scnitem{\cite{Iyengar2021}}
						\scnitem{\cite{Ford2019}}
					\end{scnrelfromlist}
				\end{scnindent}
				\scntext{примечание}{Многократная повторная разработка уже имеющихся технических решений обусловлена либо тем, что известные технические решения \uline{плохо} интегрируются в разрабатываемую систему, либо тем, что эти технические решения трудно найти. Данная проблема актуальна как в целом в сфере разработки компьютерных систем, так и в сфере разработки систем, основанных на знаниях, поскольку в системах такого рода степень согласованности различных видов знаний влияет на возможность системы решать нетривиальные задачи.}  
			\end{scnindent}
			\scnfileitem{Невозможность автоматической интеграции компонентов в систему \uline{без} ручного вмешательства пользователя.}
			\scnfileitem{Автоматическое обновление компонентов приводит к рассогласованности как отдельных модулей компьютерных систем, так и самих систем между собой.}
			\scnfileitem{Отсутствие классификации компонентов на различных уровнях детализации.}
			\scnfileitem{Не ведётся разработка стандартов, обеспечивающих совместимость этих компонентов.}
			\scnfileitem{Не проводится тестирование, верификация и анализ качества компонентов, не выделяются преимущества, недостатки, ограничения компонентов.}
			\scnfileitem{Многие компоненты используют для идентификации язык разработчика (как правило, английский), и предполагается, что все пользователи будут использовать этот же язык. Однако для многих приложений это недопустимо --- понятные только разработчику идентификаторы должны быть скрыты от конечных пользователей, которые должны быть в состоянии выбрать язык для идентификаторов, которые они видят.}
			\scnfileitem{Отсутствие средств поиска компонентов, удовлетворяющих заданным критериям.}
		\end{scneqtoset}
		\scnrelfrom{источник}{\cite{Shunkevich2015a}}
	\end{scnindent}
	\scntext{примечание}{\scnkeyword{компонентное проектирование интеллектуальных компьютерных систем} возможно только в том случае, если отбор компонентов будет осуществляться на основе тщательного анализа качества этих компонентов. Одним из важнейших критериев такого анализа является уровень семантической совместимости анализируемых компонентов со всеми компонентами, имеющимися в текущей версии библиотеки.}
	\scnrelfrom{предъявляемые требования}{Требования к реализации компонентного проектирования интеллектуальных компьютерных систем}
	\begin{scnindent}
		\begin{scneqtoset}
			\scnfileitem{Обеспечение совместимости (интегрируемости) компонентов интеллектуальных компьютерных систем на основе унификации представления этих компонентов.}
			\scnfileitem{Четкое разделение процесса разработки формальных описаний интеллектуальных компьютерных систем и процесса их реализации по этому описанию.}
			\scnfileitem{Четкое разделение разработки формального описания проектируемой интеллектуальной системы от разработки различных вариантов интерпретации таких формальных описаний систем.}
			\scnfileitem{Наличие онтологии компонентного проектирования интеллектуальных компьютерных систем, включающей (1) описание методов компонентного проектирования, (2) модель \textit{библиотеки многократно используемых компонентов}, (3) модель \textit{спецификации многократно используемых компонентов}, (4) полную \textit{классификацию многократно используемых компонентов}, (5) описание средств взаимодействия разрабатываемой интеллектуальной компьютерной системы с \textit{библиотеками многократно используемых компонентов}.}
			\scnfileitem{Наличие \textit{библиотек многократно используемых компонентов интеллектуальных компьютерных систем}, включающих спецификации компонентов.}
			\scnfileitem{Наличие средств взаимодействия разрабатываемой интеллектуальной компьютерной системы с библиотеками многократно используемых компонентов для установки любых видов компонентов и управления ими в создаваемой системе.}
			\begin{scnindent}
				\scnnote{Под установкой компонента понимается его транспортировка в систему (копирование sc-элементов и/или скачивание файлов компонента), а также выполнение вспомогательных действий для того, чтобы компонент мог функционировать в создаваемой системе.}
			\end{scnindent}
		\end{scneqtoset}
		\begin{scnrelfromlist}{источник}
			\scnitem{\cite{Zalivako2011}}
			\scnitem{\cite{Golenkov2013}}
			\scnitem{\cite{Golenkov2014a}}
		\end{scnrelfromlist}
		\scntext{примечание}{Для того, чтобы решить возникшие проблемы при проектировании интеллектуальных систем и библиотек их многократно используемых компонентов, необходимо придерживаться общих принципов технологии проектирования интеллектуальных компьютерных систем, а также выполнить эти требования.}
	\end{scnindent}
	
		\bigskip
	\end{scnsubstruct}
	\scnsourcecomment{Завершили \scnqqi{Сегмент. Введение в Предметную область и онтологию комплексной библиотеки многократно используемых семантически совместимых компонентов ostis-систем}}
\end{SCn}
\input{Contents/part_methods_tools/src/sd_biblio_component_segments/segment_library_analysis}
\input{Contents/part_methods_tools/src/sd_biblio_component_segments/segment_library_concept}
\input{Contents/part_methods_tools/src/sd_biblio_component_segments/segment_reusable_component_concept}
\begin{SCn}
	\scnsectionheader{Сегмент. Уточнение спецификации многократно используемого компонента ostis-систем}
	
	\begin{scnsubstruct}
		
	\scnheader{спецификация многократно используемого компонента ostis-систем}
	\scnsubset{спецификация}
	\scnidtf{описание многократно используемого компонента ostis-систем}
	\scnhaselementrole{ключевой sc-элемент}{многократно используемый компонент ostis-систем}
	\scntext{примечание}{Каждый \textit{многократно используемый компонент ostis-систем} должен быть специфицирован в рамках библиотеки. Данные спецификации включают в себя основные знания о компоненте, которые позволяют обеспечить построение полной иерархии компонентов и их зависимостей, а также обеспечивают \uline{беспрепятственную} интеграцию компонентов в \scnkeyword{дочерние ostis-системы}. Для спецификации компонента используются как отношения, так и классы компонента.}
	\begin{scnindent}
		\begin{scnrelfromlist}{источник}
			\scnitem{\cite{Orlov2022a}}
			\scnitem{\cite{Davydenko2013}}
		\end{scnrelfromlist}
		\scntext{примечание}{Указание класса \scnkeyword{многократно используемый компонент ostis-систем} является обязательным.}
	\end{scnindent}
	\scntext{примечание}{Сам многократно используемый компонент в рамках спецификации является \textit{ключевым sc-элементом\scnrolesign}, а также может иметь множество своих ключевых sc-элементов.}
	\scnrelfrom{параметры, специфицирующие многократно используемый компонент ostis-систем}{параметр, заданный на многократно используемых компонентах ostis-систем\scnsupergroupsign}
	\scnrelfrom{классы отношений, специфицирующие многократно используемый компонент ostis-систем}{отношение, специфицирующее многократно используемый компонент ostis-систем\scnsupergroupsign}
	\scnrelfrom{описание примера}{\scnfileimage[40em]{Contents/part_methods_tools/src/images/sd_ostis_library/component_specification_example.png}}
	\scnheader{отношение, специфицирующее многократно используемый компонент ostis-систем\scnsupergroupsign}
	\scnidtf{отношение, которое используется при спецификации многократно используемого компонента ostis-систем}
	\begin{scnrelfromset}{разбиение}
		\scnitem{необходимое для установки отношение, специфицирующее многократно используемый компонент ostis-систем}
		\begin{scnindent}
			\scntext{примечание}{Чтобы многократно используемый компонент мог быть принят в библиотеку, нужно специфицировать его, используя каждое отношение из множества \textit{необходимое для установки отношение, специфицирующее многократно используемый компонент ostis-систем}. Здесь описана спецификация, общая для любых типов компонентов, однако в зависимости от типа компонента, спецификация может расширяться.}
			\scnhaselement{метод установки*}
			\scnhaselement{адрес хранилища*}
			\scnhaselement{зависимости компонента*}
		\end{scnindent}
		\scnitem{необязательное для установки отношение, специфицирующее многократно используемый компонент ostis-систем}
		\begin{scnindent}
			\scntext{примечание}{\textit{необязательное для установки отношение, специфицирующее многократно используемый компонент ostis-систем} помогает лучше понять суть компонента, упрощает поиск, но не является обязательным для того, чтобы компонент мог быть установлен в ostis-систему.}
			\scnhaselement{сопутствующие компоненты*}
			\scnhaselement{история изменений*}
			\scnhaselement{модификации компонентов*}
			\scnhaselement{авторы*}
			\scnhaselement{примечание*}
			\scnhaselement{пояснение*}
			\scnhaselement{идентификатор*}
			\scnhaselement{ключевой sc-элемент\scnrolesign}
			\scnhaselement{назначение*}
			\scnhaselement{требования полноты*}
			\scnhaselement{требования безошибочности*}
			\scnhaselement{преимущества*}
			\scnhaselement{недостатки*}
		\end{scnindent}
	\end{scnrelfromset}
	\scnheader{метод установки*}
	\scniselement{бинарное отношение}
	\scniselement{ориентированное отношение}
	\scntext{пояснение}{Пользователь может установить компонент вручную, а \scnkeyword{менеджер компонентов} --- автоматически.}
	\scnrelfrom{первый домен}{многократно используемый компонент ostis-систем}
	\scnrelfrom{второй домен}{метод установки многократно используемого компонента}
	\begin{scnindent}
		\scnsubset{метод}
		\scnsuperset{метод установки динамически устанавливаемого многократно используемого компонента ostis-систем}
		\begin{scnindent}
			\scntext{примечание}{При динамической установке необходимо только скачать компонент и, при необходимости, его зависимые компоненты, и он сразу же функционирует в системе.}
			\scnrelfrom{описание примера}{\scnfileimage[35em]{Contents/part_methods_tools/src/images/sd_ostis_library/install_dynamic_method.png}}
			\begin{scnindent}
				\scniselement{sc.g-текст}
			\end{scnindent}
		\end{scnindent}
		\scnsuperset{метод установки многократно используемого компонента, при установке которого система требует перезапуска}
		\begin{scnindent}
			\scntext{примечание}{Установка таких компонентов происходит путём скачивания компонента и его трансляции в память системы.}
			\scnrelfrom{описание примера}{\scnfileimage[35em]{Contents/part_methods_tools/src/images/sd_ostis_library/install_with_reboot_method.png}}
			\begin{scnindent}
				\scniselement{sc.g-текст}
			\end{scnindent}
		\end{scnindent}
	\end{scnindent}
	\scnheader{адрес хранилища*}
	\scniselement{бинарное отношение}
	\scniselement{ориентированное отношение}
	\scntext{пояснение}{Связки отношения \textit{адрес хранилища*} связывают многократно используемый компонент, хранящийся в виде внешних файлов и файл, содержащий url-адрес многократно используемого компонента ostis-систем.}
	\scnrelfrom{первый домен}{многократно используемый компонент ostis-систем, хранящийся в виде внешних файлов}
	\scnrelfrom{второй домен}{файл, содержащий url-адрес многократно используемого компонента ostis-систем}
	\begin{scnindent}
		\scnsuperset{файл}
		\scnsubset{файл, содержащий url-адрес на GitHub многократно используемого компонента ostis-систем}
		\scnsubset{файл, содержащий url-адрес на Google Drive многократно используемого компонента ostis-систем}
		\scnsubset{файл, содержащий url-адрес на Docker Hub многократно используемого компонента ostis-систем}
	\end{scnindent}
	\scnheader{зависимости компонента*}
	\scniselement{квазибинарное отношение}
	\scniselement{ориентированное отношение}
	\scntext{пояснение}{Связки отношения \textit{зависимости компонента*} связывают многократно используемый компонент, и множество компонентов, без которых устанавливаемый компонент \uline{не может быть} встроен в \scnkeyword{дочернюю ostis-систему}.}
	\scnrelfrom{первый домен}{многократно используемый компонент ostis-систем}
	\scnrelfrom{второй домен}{множество многократно используемых компонентов ostis-систем}
	\scnheader{сопутствующие компоненты*}
	\scniselement{квазибинарное отношение}
	\scniselement{ориентированное отношение}
	\scntext{пояснение}{В некоторых случаях может оказаться, что для использования одного многократно используемого компонента OSTIS целесообразно или даже необходимо дополнительно использовать несколько других \textit{многократно используемых компонентов OSTIS}. Например, может оказаться целесообразным вместе с каким-либо sc-агентом информационного поиска использовать соответствующую команду интерфейса, которая представлена отдельным компонентом и позволит пользователю задавать вопрос для указанного sc-агента через интерфейс системы. В таких случаях для связи компонентов используется отношение \textit{сопутствующие компоненты*}. Наличие таких связей позволяет устранить возможные проблемы неполноты знаний и навыков в дочерней системе, из-за которых какие-либо из компонентов могут не выполнять свои функции. Связки отношения \textit{сопутствующий компонент*} связывают многократно используемые компоненты ostis-систем, которые целесообразно использовать в дочерней системе вместе. Каждая такая связка может дополнительно быть снабжена sc-комментарием или sc-пояснением, отражающим суть указываемой зависимости.}
	\scnrelfrom{первый домен}{многократно используемый компонент ostis-систем}
	\scnrelfrom{второй домен}{множество многократно используемых компонентов ostis-систем}
	\scnheader{история изменений*}
	\scniselement{бинарное отношение}
	\scniselement{ориентированное отношение}
	\scntext{пояснение}{Отношение \textit{история изменений*} позволяет специфицировать различные версии компонента и, при необходимости, устанавливать выбранную пользователем версию. Различные версии, как правило, отражают какие-либо улучшения или исправления ошибок.}
	\scnrelfrom{первый домен}{многократно используемый компонент ostis-систем}
	\scnrelfrom{второй домен}{история изменений}
	\scnheader{модификации компонентов*}
	\scniselement{бинарное отношение}
	\scniselement{ориентированное отношение}
	\scntext{пояснение}{\textit{модификации компонентов*} --- это функционально эквивалентные реализации одного и того же компонента, которые могут быть синтаксически эквивалентны (например, реализация одного и того же sc-агента на платформенно-зависимом и платформенно-независимом уровнях). Развитие \textit{Библиотеки Экосистемы OSTIS} происходит не только за счет ее пополнения новыми компонентами, но и за счет появления новых версий и модификаций уже существующих компонентов.}
	\scnrelfrom{первый домен}{многократно используемый компонент ostis-систем}
	\scnrelfrom{второй домен}{множество многократно используемых компонентов ostis-систем}
	\scnheader{авторы*}
	\scntext{пояснение}{Связки отношения \textit{авторы*} связывают многократно используемый компонент со множеством авторов этого компонента. Спецификация может также содержать дополнительную информацию об авторах при необходимости.}
	\scnheader{назначение*}
	\scntext{пояснение}{Отношение \textit{назначение*} позволяет описать ожидаемый сценарий, выделить рекомендации использования многократно используемого компонента. Требования полноты и безошибочности специфицируют возможные ограничения и ошибки компонента, область его использования.}
	
	\scnheader{параметр, заданный на многократно используемых компонентах ostis-систем\scnsupergroupsign}
	\scntext{примечание}{Для уточнения типа компонента могут использоваться другие классы, например дата публикации первой и последней версии компонента, качественно-количественные характеристики, такие как уровень семантической совместимости компонентов, сложность реализации компонента, уровень производительности компонента (для программ можно использовать O-нотацию), количество sc-элементов, входящих в состав многократно используемого компонента, количество ключевых узлов компонента, рейтинг компонента в рамках \textit{Экосистемы OSTIS}, количество скачиваний компонента и другие. Параметр \textit{лицензия многократно используемого компонента} используется для обозначения условий использования и распространения компонента. По умолчанию лицензия компонента открытая, если не указано иное.}
	\begin{scnindent}
		\scnrelfrom{источник}{\cite{Davydenko2013}}
	\end{scnindent}
		
		\bigskip
	\end{scnsubstruct}
	\scnsourcecomment{Завершили \scnqqi{Сегмент. Уточнение спецификации многократно используемого компонента ostis-систем}}
\end{SCn}
\begin{SCn}
\scnsectionheader{Сегмент. Понятие менеджера многократно используемых компонентов ostis-систем}

\begin{scnsubstruct}
		
\scnheader{хранилище многократно используемых компонентов ostis-систем, хранящихся в виде внешних файлов}
\scntext{примечание}{Для того, чтобы хранить \textit{многократно используемые компоненты ostis-систем}, необходимо некоторое хранилище. Таким хранилищем может выступать как какая-либо ostis-система, так и стороннее хранилище, например, облачное. Помимо исходных файлов компонента в хранилище должна находиться его \uline{спецификация}.}
\scnsuperset{хранилище многократно используемого компонента ostis-систем, хранящегося в виде файлов исходных текстов}
\begin{scnindent}
	\scntext{пояснение}{Место хранения файлов исходных текстов многократно используемого компонента.}
	\scnsuperset{хранилище на основе системы контроля версий Git}
	\begin{scnindent}
		\scnsuperset{репозиторий GitHub}
		\scntext{примечание}{На данном этапе в рамках \textit{Технологии OSTIS} (в силу открытости технологии, а также хранения компонентов в виде файлов исходных текстов) для хранения компонентов чаще всего используются хранилища на основе системы контроля версий Git.}
	\end{scnindent}
\end{scnindent}
\scnsuperset{хранилище многократно используемого компонента ostis-систем, хранящегося в виде скомпилированных файлов}
\begin{scnindent}
	\scntext{пояснение}{Место хранения скомпилированных файлов многократно используемого компонента.}
\end{scnindent}
\scntext{примечание}{Помимо внешних файлов компонента в хранилище должна находиться его \uline{спецификация}.}

\scnheader{менеджер многократно используемых компонентов ostis-систем}
\scnidtftext{часто используемый sc-идентификатор}{менеджер многократно используемых компонентов}
\scnidtftext{часто используемый sc-идентификатор}{менеджер компонентов}
\scnsubset{платформенно-зависимый многократно используемый компонент ostis-систем}
\scntext{пояснение}{менеджер многократно используемых компонентов ostis-систем --- подсистема ostis-системы, с помощью которой происходит взаимодействие с библиотекой компонентов ostis-систем.}
\scnhaselement{Реализация менеджера многократно используемых компонентов ostis-систем}
\begin{scnindent}
	\scntext{адрес компонента}{https://github.com/ostis-ai/sc-component-manager}
\end{scnindent}
\begin{scnrelfromset}{обобщенная декомпозиция}
	\scnitem{база знаний менеджера многократно используемых компонентов ostis-систем}
	\begin{scnindent}
		\scntext{примечание}{база знаний менеджера компонентов содержит все те знания, которые необходимы для установки многократно используемого компонента в \scnkeyword{дочернюю ostis-систему}. К таким знаниям относятся знания о спецификации многократно используемых компонентов, методы установки компонентов, знание о библиотеках ostis-систем, с которыми происходит взаимодействие, \textit{классификация компонентов} и другие.}
	\end{scnindent}
	\scnitem{решатель задач менеджера многократно используемых компонентов ostis-систем}
	\begin{scnindent}
		\scntext{примечание}{решатель задач менеджера компонентов взаимодействует с библиотекой ostis-систем и позволяет установить и интегрировать многократно используемые компоненты в \scnkeyword{дочернюю ostis-систему}, также выполнять поиск, обновление, публикацию, удаление компонентов и другие операции с ними.}
		\begin{scnrelfromset}{декомпозиция абстрактного sc-агента}
			\scnitem{Абстрактный sc-агент поиска многократно используемых компонентов ostis-систем}
			\scnitem{Абстрактный sc-агент установки многократно используемых компонентов ostis-систем}
			\scnitem{Абстрактный sc-агент управления отслеживаемых менеджером компонентов библиотек}
			\begin{scnindent}
				\begin{scnrelfromset}{декомпозиция абстрактного sc-агента}
					\scnitem{Абстрактный sc-агент добавления отслеживаемой менеджером компонентов библиотеки}
					\scnitem{Абстрактный sc-агент удаления отслеживаемой менеджером компонентов библиотеки}
				\end{scnrelfromset}
			\end{scnindent}
		\end{scnrelfromset}
	\end{scnindent}
	\scnitem{интерфейс менеджера многократно используемых компонентов ostis-систем}
	\begin{scnindent}
		\scntext{примечание}{интерфейс менеджера многократно используемых компонентов обеспечивает удобное для пользователя и других систем использование менеджера компонентов.}
	\end{scnindent}
\end{scnrelfromset}
\scnrelfrom{минимальные функциональные возможности}{Минимальные функциональные возможности менеджера компонентов}
\begin{scnindent}
	\scntext{примечание}{Используя минимальные функциональные возможности, менеджер компонентов может установить компоненты, которые будут расширять его же функционал.}
	\begin{scneqtoset}
		\scnfileitem{\textbf{Поиск многократно используемых компонентов ostis-систем.} Множество возможных критериев поиска соответствует спецификации многократно используемых компонентов. Такими критериями могут быть классы компонента, его авторы, идентификатор, фрагмент примечания, назначение, принадлежность какой-либо предметной области, вид знаний компонента и другие.}
		\scnfileitem{\textbf{Установка многократно используемого компонента ostis-систем.} Установка многократно используемого компонента происходит вне зависимости от типологии, способа установки и местоположения компонента. Необходимое условие для возможности установки многократно используемого компонента --- наличие \textbf{\textit{спецификации многократно используемого компонента ostis-систем}}. Перед установкой многократно используемого компонента в дочернюю систему необходимо установить все зависимые компоненты. Также для платформенно-зависимых компонентов может быть необходимо установить иные зависимости, которые не являются компонентами какой-либо библиотеки ostis-систем. После успешной установки компонента в базе знаний дочерней системы генерируется информационная конструкция, обозначающая факт установки компонента в систему с помощью отношения \textit{установленные компоненты*}.}
		\scnfileitem{\textbf{Добавление и удаление отслеживаемых менеджером компонентов библиотек.} Менеджер компонентов содержит информацию о множестве источников для установки компонентов, перечень которых можно дополнять. По умолчанию менеджер компонентов отслеживает \textit{Библиотеку Метасистемы OSTIS}, однако можно создавать и добавлять дополнительные библиотеки ostis-систем.}
	\end{scneqtoset}
\end{scnindent}
\scnrelfrom{расширенные функциональные возможности}{Расширенные функциональные возможности менеджера компонентов}
\begin{scnindent}
	\scntext{примечание}{Компоненты, реализующие расширенный функционал менеджера компонентов являются частью \textit{Библиотеки Метасистемы OSTIS}.}
	\begin{scneqtoset}
		\scnfileitem{\textbf{Спецификация} многократно используемого компонента ostis-систем. Менеджер компонентов позволяет специфицировать компоненты, входящие в состав библиотеки ostis-систем, а также специфицировать новые создаваемые компоненты, которые будут публиковаться в библиотеку ostis-систем. При этом спецификация может происходить как автоматически, так и вручную. Например, менеджер компонентов может обновить спецификацию используемого компонента в соответствии с тем, в какие новые ostis-системы его установили, обновить спецификацию авторства компонента при его редактировании в библиотеке ostis-систем, спецификацию ошибок, выявленных в процессе эксплуатации компонента и так далее.}
		\scnfileitem{\textbf{Формирование} многократно используемого компонента ostis-систем по шаблону с заданными параметрами. При установке шаблона многократно используемого компонента ostis-систем менеджер компонентов позволяет сформировать по нему конкретный компонент. Для этого пользователю предлагается определить значения всех sc-переменных в шаблоне для формирования конкретного компонента из некоторой предметной области. Например, для формирования многократно используемого компонента баз знаний, представляющего собой семантическую окрестность некоторого отношения, нужно определить значения всех переменных, кроме переменной, являющейся ключевым sc-элементом данной структуры.}
		\begin{scnindent}
			\scnrelfrom{описание примера}{\scnfileimage[30em]{Contents/part_methods_tools/src/images/sd_ostis_library/relation_template.png}}
			\begin{scnindent}
				\scntext{пояснение}{Пример шаблона многократно используемого компонента ostis-систем.}
			\end{scnindent}
		\end{scnindent}
		\scnfileitem{\textbf{Публикация} многократно используемого компонента ostis-систем в библиотеку ostis-систем. При публикации компонента в библиотеку ostis-систем происходит верификация на основе спецификации компонента. Публикация компонента может сопровождаться сборкой неатомарного компонента из существующих атомарных. Также существует возможность обновления версии опубликованного компонента сообществом его разработчиков.}
		\scnfileitem{\textbf{Обновление} установленного многократно используемого компонента ostis-систем.}
		\scnfileitem{\textbf{Удаление} установленного многократно используемого компонента. Как и в случае установки после удаления многократно используемого компонента из ostis-системы в базе знаний системы устанавливается факт удаления компонента. Эта информация является важной частью \uline{истории эксплуатации} ostis-системы.}
		\scnfileitem{\textbf{Редактирование} многократно используемого компонента в библиотеке ostis-систем.}
		\scnfileitem{\textbf{Сравнение} многократно используемых компонентов ostis-систем.}
	\end{scneqtoset}
\end{scnindent}
\scntext{примечание}{Для того, чтобы создать новую ostis-систему \scnqq{с нуля}, используя \textit{ostis-платформу}, необходимо установить некоторую \textit{Программную платформу ostis-систем} с помощью сторонних средств. Такими средствами могут быть (1) хранилища исходного кода платформы, например, облачные хранилища, такие как GitHub репозиторий, с соответствующим набором инструкций по установке платформы или (2) средства установки заранее скомпилированной программной реализации платформы, например, средство установки программного обеспечения apt. Далее установка многократно используемых компонентов в ostis-систему (независимо от типа компонентов) осуществляется с помощью менеджера компонентов. При установке платформенно-зависимых компонентов менеджер компонентов должен управлять соответствующими средствами сборки таких компонентов (CMake, Ninja, npm, grunt и другие).}
\scntext{примечание}{Компонент находится в некотором хранилище --- (1) \textit{библиотеке компонентов ostis-систем} или (2) в виде файлов в некотором облачном хранилище. В случае, когда компонент хранится в библиотеке, для его установки менеджер компонентов копирует все sc-элементы, которые представляют собой компонент, в дочернюю ostis-систему. В случае, когда компонент хранится в виде файлов в облачном хранилище, менеджер компонентов скачивает файлы компонента и устанавливает их в соответствии со спецификацией. Адреса хранилищ спецификаций компонентов должны храниться в базе знаний менеджера компонентов, чтобы иметь доступ к спецификациям компонентов для их последующего использования (поиска, установки и так далее).}
\scntext{примечание}{\textit{менеджер многократно используемых компонентов ostis-систем} является \uline{необязательной} подсистемой \textit{ostis-платформы}. Однако система, имеющая менеджер компонентов, может устанавливать компоненты не только сама в себя, но и в другие системы при наличии доступа. Таким образом, одна система может заменить \textit{ostis-платформу} другой системы, оставив при этом \textit{sc-модель кибернетической системы}. Таким же образом некоторая ostis-система может порождать другие ostis-системы, используя компонентный подход.}
\scntext{примечание}{Включение компонента в \textit{дочернюю ostis-систему} в общем случае состоит из следующих этапов:
	\begin{itemize}
		\item поиск подходящего компонента во множестве доступных библиотек;
		\item выделение компонента в виде, удобном для транспортировки в дочернюю ostis-систему с указанием версии и модификации при необходимости (например, выбор доступного хранилища компонента, выбор оптимального варианта реализации компонента с учетом состава дочерней системы);
		\item установка многократно используемого компонента и его зависимостей (с указанием версии и модификации при необходимости);
		\item интеграция компонента в дочернюю систему;
		\item поиск и устранение ошибок и противоречий в дочерней системе.
\end{itemize}}

\scnheader{установленные компоненты*}
\scniselement{квазибинарное отношение}
\scniselement{ориентированное отношение}
\scntext{пояснение}{Квазибинарное отношение, связывающее некоторую ostis-систему и компоненты, которые установлены в ней.}
\scnrelfrom{первый домен}{ostis-система}
\scnrelfrom{второй домен}{множество многократно используемых компонентов ostis-систем}
\begin{scnindent}
	\scntext{пояснение}{множество многократно используемых компонентов ostis-систем --- это множество, все элементы которого являются многократно используемыми компонентами ostis-систем.}
\end{scnindent}
\scntext{примечание}{Данное отношение позволяет хранить сведения о системах и компонентах, которые установлены в них, тем самым предоставляя возможность анализировать функциональные возможности системы.}
\scntext{примечание}{Данное отношение позволяет оценивать частоту скачивания компонентов, то есть их использования в \scnkeyword{дочерних ostis-системах}.}
	
	\bigskip
\end{scnsubstruct}
\scnsourcecomment{Завершили \scnqqi{Сегмент. Понятие менеджера многократно используемых компонентов ostis-систем}}
\end{SCn}   
\input{Contents/part_methods_tools/src/sd_biblio_component_segments/segment_library_conclusion}
        
        \bigskip
    \end{scnsubstruct}
\end{SCn}

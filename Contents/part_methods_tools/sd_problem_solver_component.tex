\begin{SCn}
\scnsectionheader{\currentname}

\scnstartsubstruct

\scnheader{Предметная область многократно используемых компонентов решателей задач ostis-систем}
\scniselement{предметная область}
\scnrelto{частная предметная область}{Предметная область и онтология комплексной библиотеки многократно используемых семантически совместимых компонентов ostis-систем}
\scnsdmainclasssingle{многократно используемый компонент решателей задач ostis-систем}
\scnsdclass{отношение,специфицирующее многократно используемый компонент решателей задач ostis-систем}
\scnhaselementlist{понятие, используемое в предметной области}{агентная scp-программа}
\scnhaselementlist{отношение, используемое в предметной области}{программа sc-агента*}

\scnheader{библиотека многократно используемых компонентов решателей задач}
\scnrelfromset{обобщенная декомпозиция}{база знаний библиотеки многократно используемых компонентов решателей задач\\
\scnaddlevel{1}
\scnnote{База знаний представляет собой иерархию многократно используемых компонентов решателей задач ostis-систем и их спецификацию.}
\scnaddlevel{-1}    
;решатель задач библиотеки многократно используемых компонентов решателей задач\\
\scnaddlevel{1}
\scnnote{Решатель задач позволяет систематизировать, находить зависимости, конфликты.}
\scnaddlevel{-1}
;интерфейс библиотеки многократно используемых компонентов решателей задач\\
\scnaddlevel{1}
\scnnote{Интерфейс библиотеки позволяет подключиться к библиотеке и получить доступ к компонентам хранящимся в ней и к её функционалу}
\scnaddlevel{-1}}

\scnheader{многократно используемый компонент решателей задач}
\scnsuperset{программа}
\scnaddlevel{1}
\scnexplanation{Комбинация компьютерных инструкций и данных, позволяющая аппаратному обеспечению вычислительной системы выполнять вычисления или функции управления.}
\scnaddlevel{-1}
\scnsuperset{пакет программ}
\scnaddlevel{1}
\scnexplanation{Интегрированная система, позволяющая пользователю решать задачу без программирования путем описания задачи и её исходных данных.}
\scnaddlevel{-1}
\scnsuperset{абстрактный sc-агент}
\scnsuperset{решатель задач ostis-системы}
\scnaddlevel{1}
\scnnote{Целые решатели задач могут быть многократно используемыми компонентами в случае разработки интеллектуальных систем, назначение которых совпадает.}
\scnaddlevel{-1}

\scnheader{метод}
\scnidtf{программа}
\scnsuperset{программа на основе нейросетевых моделей}
\scnsuperset{программа на основе генетических алгоритмов}
\scnsuperset{императивная программа}
\scnaddlevel{1}
\scnsuperset{процедурная программа}
\scnsuperset{объектно-ориентированная программа}
\scnaddlevel{-1}
\scnsuperset{декларативная программа}
\scnaddlevel{1}
\scnsuperset{логическая программа}
\scnsuperset{функциональная программа}
\scnaddlevel{-1}
\scnsuperset{программа sc-агента}

\scnheader{абстрактный sc-агент}
\scnnote{Поскольку предполагается, что копии одного и того же \textit{sc-агента} или функционально эквивалентные \textit{sc-агенты} могут работать в разных ostis-системах, будучи при этом физически разными sc-агентами, то целесообразно рассматривать свойства и классификацию не sc-агентов, а классов функционально эквивалентных sc-агентов, которые будем называть \textit{абстрактными sc-агентами}.}
\scnexplanation{Под \textbf{\textit{абстрактным sc-агентом}} понимается некоторый класс функционально эквивалентных \textit{sc-агентов}, разные экземпляры (т.е. представители) которого могут быть реализованы по-разному.
	
Каждый \textbf{\textit{абстрактный sc-агент}} имеет соответствующую ему спецификацию. В спецификацию каждого \textbf{\textit{абстрактного sc-агента}} входит: \vspace{0.2cm}
	
\begin{scnitemize}
	\item указание ключевых \textit{sc-элементов} этого \textit{sc-агента}, т.е. тех \textit{sc-элементов}, хранимых в \textit{sc-памяти}, которые для данного \textit{sc-агента} являются «точками опоры»;
	\item формальное описание условий инициирования данного \textit{sc-агента}, т.е. тех \textit{ситуация} в \textit{sc-памяти}, которые инициируют деятельность данного \textit{sc-агента};
	\item формальное описание первичного условия инициирования данного \textit{sc-агента}, т.е. такой ситуации в \textit{sc-памяти}, которая побуждает \textit{sc-агента} перейти в активное состояние и начать проверку наличия своего полного условия инициирования (для \textit{внутренних абстрактных sc-агентов});
	\item строгое, полное, однозначно понимаемое описание деятельности данного \textit{sc-агента}, оформленное при помощи каких-либо понятных, общепринятых средств, не требующих специального изучения, например на естественном языке.
	\item описание результатов выполнения данного \textit{sc-агента}.
\end{scnitemize}
}
\scnsubdividing{неатомарный абстрактный sc-агент;атомарный абстрактный sc-агент}
\scnsubdividing{внутренний абстрактный sc-агент;эффекторный абстрактный sc-агент;рецепторный абстрактный sc-агент}
\scnsubdividing{абстрактный sc-агент, не реализуемый на Языке SCP;абстрактный sc-агент, реализуемый на Языке SCP}
\scnsubdividing{абстрактный sc-агент интерпретации scp-программ;абстрактный программный sc-агент;абстрактный sc-метаагент}
\scnsubdividing{платформенно-зависимый абстрактный sc-агент\\
\scnaddlevel{1}
\scnsuperset{абстрактный sc-агент, не реализуемый на Языке SCP}
\scnaddlevel{-1}
;платформенно-независимый абстрактный sc-агент}

\scnheader{абстрактный sc-агент, не реализуемый на Языке SCP}
\scnidtf{абстрактный sc-агент, который не может быть реализован на платформенно-независимом уровне}
\scnsubdividing{эффекторный абстрактный sc-агент;рецепторный абстрактный sc-агент
;абстрактный sc-агент интерпретации scp-программ}

\scnheader{абстрактный sc-агент, реализуемый на Языке SCP}
\scnidtf{абстрактный sc-агент, который может быть реализован на платформенно-независимом уровне}
\scnsubdividing{абстрактный sc-метаагент;абстрактный программный sc-агент, реализуемый на Языке SCP}

\scnheader{абстрактный программный sc-агент}
\scnsubdividing{эффекторный абстрактный sc-агент;рецепторный абстрактный sc-агент
;абстрактный программный sc-агент, реализуемый на Языке SCP}

\scnheader{неатомарный абстрактный sc-агент}
\scnexplanation{Под \textbf{\textit{неатомарным абстрактным sc-агентом}} понимается \textit{абстрактный sc-агент}, который декомпозируется на коллектив более простых \textit{абстрактных sc-агентов}, каждый из которых в свою очередь может быть как \textit{атомарным абстрактным sc-агентом}, так и \textbf{\textit{неатомарным абстрактным sc-агентом}}. При этом в каком либо варианте \textit{декомпозиции абстрактного sc-агента*} дочерний \textbf{\textit{неатомарный абстрактный sc-агент}} может стать \textit{атомарным абстрактным sc-агентом}, и реализовываться соответствующим образом.}

\scnheader{атомарный абстрактный sc-агент}
\scnexplanation{Под \textbf{\textit{атомарным абстрактным sc-агентом}} понимается \textit{абстрактный sc-агент}, для которого уточняется платформа его реализации, т.е. существует соответствующая связка отношения \textit{программа sc-агента*}.}
\scnsubdividing{платформенно-независимый абстрактный sc-агент;платформенно-зависимый абстрактный sc-агент}

\scnheader{платформенно-независимый абстрактный sc-агент}
\scnexplanation{К \textbf{\textit{платформенно-независимым абстрактным \mbox{sc-агентам}}} относят \textit{атомарные абстрактные sc-агенты}, реализованные на базовом языке программирования Технологии OSTIS, т.е. на \textit{Языке SCP}.
	
При описании \textbf{\textit{платформенно-независимых абстрактных sc-агентов}} под платформенной независимостью понимается платформенная независимость с точки зрения Технологии OSTIS, т.е реализация на специализированном языке программирования, ориентированном на обработку семантических сетей (\textit{Языке SCP}), поскольку \textit{атомарные sc-агенты}, реализованные на указанном языке могут свободно переноситься с одной платформы интерпретации \textit{sc-моделей} на другую. При этом языки программирования, традиционно считающиеся платформенно-независимыми в данном случае не могут считаться таковыми.
	
Существуют \textit{sc-агенты}, которые принципиально не могут быть реализованы на платформенно-независимом уровне, например, собственно \textit{sc-агенты} интерпретации \textit{sc-моделей} или рецепторные и эффекторные \textit{sc-агенты}, обеспечивающие взаимодействие с внешней средой.}

\scnheader{платформенно-зависимый абстрактный sc-агент}
\scnexplanation{К \textbf{\textit{платформенно-зависимым абстрактным sc-агентам}} относят \textit{атомарные абстрактные sc-агенты}, реализованные ниже уровня sc-моделей, т.е. не на \textit{Языке SCP}, а на каком-либо другом языке описания программ.
	
Существуют \textit{sc-агенты}, которые принципиально должны быть реализованы на платформенно-зависимом уровне, например, собственно \textit{sc-агенты} интерпретации \textit{sc-моделей} или рецепторные и эффекторные \textit{sc-агенты}, обеспечивающие взаимодействие с внешней средой.}

\scnheader{внутренний абстрактный sc-агент}
\scnexplanation{Каждый \textbf{\textit{внутренний абстрактный sc-агент}} обозначает класс \textit{sc-агентов}, которые реагируют на события в \textit{sc-памяти} и осуществляют преобразования исключительно в рамках этой же \textit{sc-памяти}.}

\scnheader{эффекторный абстрактный sc-агент}
\scnexplanation{Каждый \textbf{\textit{эффекторный абстрактный sc-агент}} обозначает класс \textit{sc-агентов}, которые реагируют на события в \textit{sc-памяти} и осуществляют преобразования во внешней относительно данной \textit{ostis-системы} среде.}

\scnheader{рецепторный абстрактный sc-агент}
\scnexplanation{Каждый \textbf{\textit{рецепторный абстрактный sc-агент}} обозначает класс \textit{sc-агентов}, которые реагируют на события во внешней относительно данной \textit{ostis-системы} среде и осуществляют преобразования в памяти данной системы.}

\scnheader{абстрактный sc-агент, не реализуемый на Языке SCP}
\scnexplanation{Каждый \textbf{\textit{абстрактный sc-агент, не реализуемый на Языке SCP}} должен быть реализован на уровне платформы интерпретации sc-моделей, в том числе, аппаратной. К таким \textit{абстрактным sc-агентам} относятся абстрактные sc-агенты интерпретации scp-программ, а также эффекторные и рецепторные абстрактные sc-агенты.}

\scnheader{абстрактный sc-агент, реализуемый на Языке SCP}
\scnexplanation{Каждый \textbf{\textit{абстрактный sc-агент, реализуемый на Языке SCP}} может быть реализован на Языке SCP, то есть платформенно-независимом уровне, но при необходимости, может реализовываться и на уровне платформы, например, с целью повышения производительности.}

\scnheader{абстрактный sc-агент интерпретации scp-программ}
\scnexplanation{К \textbf{\textit{абстрактным sc-агентам интерпретации scp-программ}} относятся не реализуемые на платформенно-независимом уровне \textit{абстрактные sc-агенты}, обеспечивающие интерпретацию \textit{scp-программ} и \textit{\mbox{scp-метапрограмм}}, в том числе создание \textit{scp-процессов}, собственно интерпретацию \textit{scp-операторов}, а также другие вспомогательные действия. По сути, агенты данного класса обеспечивают работу sc-агентов более высоких уровней (программных sc-агентов и sc-метаагентов), реализованных на Языке SCP, в частности, обеспечивают соблюдение указанными агентами общих принципов синхронизации.}

\scnheader{абстрактный программный sc-агент}
\scnexplanation{К \textbf{\textit{абстрактным программным sc-агентам}} относятся все \textit{абстрактные sc-агенты}, обеспечивающие основной функционал системы, то есть ее возможность решать те или иные задачи. Агенты данного класса должны работать в соответствии с общими принципами синхронизации деятельности субъектов в sc-памяти.}

\scnheader{абстрактный sc-метаагент}
\scnexplanation{Задачей \textbf{\textit{абстрактных sc-метаагентов}} является координация деятельности \textit{абстрактных программных sc-агентов}, в частности, решение проблемы взаимоблокировок. Агенты данного класса могут быть реализованы на Языке SCP, однако для синхронизации их деятельности используются другие принципы, соответственно, для реализации таких агентов требуется Язык SCP другого уровня, типология операторов которого полностью аналогична типологии scp-операторов, однако эти операторы имеют другую операционную семантику, учитывающую отличия в принципах синхронизации (работы с \textit{блокировками*}). Программы такого языка будем называть \textit{scp-метапрограммами}, соответствующие им \mbox{\textit{процессы в sc-памяти} – \textit{scp-метапроцессами}}, операторы – \textit{scp-метаоператорами}.}

\scnheader{агентная scp-программа}
\scnidtf{Язык SCP}
\scnexplanation{Агентные scp-программы представляют собой частный случай scp-программ вообще, однако заслуживают отдельного рассмотрения, поскольку используются наиболее часто. scp-программы данного класса представляют собой реализации программ агентов обработки знаний, и имеют жестко фиксированный набор параметров. Каждая такая программа имеет ровно два in-параметра’. Значение первого параметра является знаком бинарной ориентированной пары, являющееся вторым компонентом связки отношения первичное условие инициирования* для абстрактного sc-агента, в множество программ sc-агента* которого входит рассматриваемая агентная scp-программа.
	
Значением второго параметра является sc-элемент, с которым непосредственно связано событие, в результате возникновения которого был инициирован соответствующий sc-агент, т.е., например, сгенерированная либо удаляемая sc-дуга или sc-ребро.}

\scnheader{программа sc-агента*}
\scnexplanation{Связки отношения \textit{программа sc-агента}* связывают между собой sc-узел, обозначающий \textit{атомарный абстрактный sc-агент} и sc-узел, обозначающий множество программ, реализующих указанный \textit{атомарный абстрактный sc-агент}.}

\scnheader{решатель задач ostis-системы}
\scnsuperset{решатель задач с использованием хранимых методов}
\scnaddlevel{1}
\scnidtf{решатель, способный решать задачи тех классов, для которых на данный момент времени известен соответствующий метод решения}
\scnsuperset{решатель задач на основе нейросетевых моделей}
\scnsuperset{решатель задач на основе генетических алгоритмов}
\scnsuperset{решатель задач на основе императивных программ}
\scnaddlevel{1}
\scnsuperset{решатель задач на основе процедурных программ}
\scnsuperset{решатель задач на основе объектно-ориентированных программ}
\scnaddlevel{-1}
\scnsuperset{решатель задач на основе декларативных программ}
\scnaddlevel{1}
\scnsuperset{решатель задач на основе логических программ}
\scnsuperset{решатель задач на основе функциональных программ}
\scnaddlevel{-1}
\scnaddlevel{-1}
\scnsuperset{решатель задач в условиях, когда метод решения задач данного класса в текущий момент времени не известен}
\scnaddlevel{1}
\scnidtf{решатель, реализующий стратегии решения задач, позволяющие породить метод решения задачи, который в текущий момент времени не известен ostis-системе}
\scnidtf{решатель, использующий для решения задач метаметоды, соответствующие более общим классам задач по отношению к заданной}
\scnidtf{решатель задач, позволяющий породить метод, который является частным по отношению какому-либо известному ostis-системе методу и интерпретируется соответствующей машиной обработки знаний}
\scnsuperset{решатель, реализующий стратегию поиска путей решения задачи в глубину}
\scnsuperset{решатель, реализующий стратегию поиска путей решения задачи в ширину}
\scnsuperset{решатель, реализующий стратегию проб и ошибок}
\scnsuperset{решатель, реализующий стратегию разбиения задачи на подзадачи}
\scnsuperset{решатель, реализующий стратегию решения задач по аналогии}
\scnsuperset{решатель, реализующий концепцию интеллектуального пакета программ}
\scnaddlevel{-1}

\scnheader{отношение, специфицирующее многократно используемый компонент решателей задач ostis-систем}
\scnsubset{отношение, специфицирующее многократно используемый компонент ostis-систем}
\scnhaselement{первичное условие инициирования*}
\scnaddlevel{1}
\scnexplanation{Связки отношения первичное условие инициирования* связывают между собой sc-узел, обозначающий абстрактный sc-агент и бинарную ориентированную пару, описывающую первичное условие инициирования данного абстрактного sc-агента, т.е. такой ситуации в sc-памяти, которая побуждает sc-агента перейти в активное состояние и начать проверку наличия своего полного условия инициирования.
	
Первым компонентом данной ориентированной пары является знак некоторого подмножества понятия событие, например событие добавления выходящей sc-дуги, т.е. по сути конкретный тип события в sc-памяти.
	
Вторым компонентом данной ориентированной пары является произвольный в общем случае sc-элемент, с которым непосредственно связан указанный тип события в sc-памяти, т.е., например, sc-элемент, из которого выходит либо в который входит генерируемая либо удаляемая sc-дуга, либо sc-ссылка, содержимое которой было изменено.
	
После того, как в sc-памяти происходит некоторое событие, активизируются все активные sc-агенты, первичное условие инициирования* которых соответствует произошедшему событию.}
\scnrelfrom{первый домен}{абстрактный sc-агент}
\scnrelfrom{второй домен}{бинарная ориентированная пара}
\scnaddlevel{-1}
\scnhaselement{условие инициирования и результат*}
\scnaddlevel{1}
\scnexplanation{Связки отношения условие инициирования и результат* связывают между собой sc-узел, обозначающий абстрактный sc-агент и бинарную ориентированную пару, связывающую условие инициирования данного абстрактного sc-агента и результаты выполнения данного экземпляров данного sc-агента в какой-либо конкретной системе.
	
Указанную ориентированную пару можно рассматривать как логическую связку импликации, при этом на sc-переменные, присутствующие в обеих частях связки, неявно накладывается квантор всеобщности, на sc-переменные, присутствующие либо только в посылке, либо только в заключении неявно накладывается квантор существования.
	
Первым компонентом указанной ориентированной пары является логическая формула, описывающая условие инициирования описываемого абстрактного sc-агента, то есть конструкции, наличие которой в sc-памяти побуждает sc-агент начать работу по изменению состояния sc-памяти. Данная логическая формула может быть как атомарной, так и неатомарной, в которой допускается использование любых связок логического языка.
	
Вторым компонентом указанной ориентированной пары является логическая формула, описывающая возможные результаты выполнения описываемого абстрактного sc-агента, то есть описание произведенных им изменений состояния sc-памяти. Данная логическая формула может быть как атомарной, так и неатомарной, в которой допускается использование любых связок логического языка.}
\scnrelfrom{первый домен}{абстрактный sc-агент}
\scnrelfrom{второй домен}{бинарная ориентированная пара}
\scnaddlevel{-1}
\scnhaselement{эквивалентный компонент*}
\scnaddlevel{1}
\scnhaselement{неориентированное отношение}
\scnexplanation{Бинарное отношение связывающее функционально эквивалентные многократно используемые компоненты решателей задач.}
\scnrelfrom{первый домен}{многократно используемый компонент решателей задач}
\scnrelfrom{второй домен}{многократно используемый компонент решателей задач}
\scnaddlevel{-1}

\scnheader{Решатель задач библиотеки многократно используемых компонентов решателей задач}
\scnrelfromset{декомпозиция абстрактного sc-агента}{
Неатомарный агент поиска компонента\\
\scnaddlevel{1}
\scnexplanation{Множество агентов, обеспечивающих поиск компонентов в рамках библиотеки по определенным критериям}
\scnnote{Существующие критерии регламентированы спецификацией многократно используемых компонентов}
\scnaddlevel{-1}
;Агент поиска зависимостей
\scnaddlevel{1}
\scnidtf{Агент поиска всех зависимостей, без которых использование запрашиваемого компонента невозможно}
\scnaddlevel{-1}
;Агент поиска эквивалентных компонентов
\scnaddlevel{1}
\scnidtf{Агент поиска всех функционально эквивалентных компонентов, которые дают эквивалентный результат}
\scnaddlevel{-1}
;Агент поиска конфликтов между компонентами
\scnaddlevel{1}
\scnidtf{Агент проверки отсутствия/присутствия конфликтов между установленным и устанавливаемым компонентами}
\scnaddlevel{-1}
;Агент спецификации компонента
\scnaddlevel{1}
\scnidtf{Агент, позволяющий сформировать спецификацию разрабатываемого компонента для его дальнейшей публикации}
\scnaddlevel{-1}}


\bigskip
\scnendstruct \scnendcurrentsectioncomment

\end{SCn}
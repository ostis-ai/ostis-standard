\begin{SCn}
 
\scnsectionheader{\currentname}
    
\scnstartsubstruct
    
\scnheader{Предметная область речевых интерфейсов современных интеллектуальных компьютерных систем}
\scniselement{предметная область}
    
\scnsdmainclasssingle{речевой интерфейс}
\scnsdclass{естественно-языковой интерфейс}
\scnsdrelation{реализация}

\scnheader{автоматическое распознавание речевого сообщения}
\scnidtf{Automatic Speech Recognition}
\scnidtf{ASR}
\scnsubset{обработка естественно языкового сообщения}
\scnaddlevel{1}
    \scnidtf{Natural Language Processing}
    \scnidtf{NLP}
\scnaddlevel{-1}
\scnsubset{обработка речевого сигнала}
\scnaddlevel{1}
    \scnidtf{Speech Processing}
    \scnidtf{SP}
\scnaddlevel{-1}
\scnaddlevel{1}
    \scnsubset{цифровая обработка сигналов}
    \scnaddlevel{1}
        \scnidtf{Digital Signal Processing}
        \scnidtf{DSP}
    \scnaddlevel{-1}
\scnaddlevel{-1}
\scnexplanation{\textit{автоматическое распознавание речевого сообщения} -- процесс автоматического анализа речевого сигнала и получения данных о том, что было сказано пользователем, без определения смысловой составляющей \textit{(синтаксический уровень)}. Наиболее часто применяется для преобразования информацию из речевой в текстовую форму.}

\scnheader{понимание речевого сообщения}
\scnidtf{ПРС}
\scnidtf{Spoken Language Understanding}
\scnidtf{SLU}
\scnsubset{понимание естественно языкового сообщения}
\scnaddlevel{1}
    \scnidtf{Natural Language Understanding}
    \scnidtf{NLU}
\scnaddlevel{-1}
\scnexplanation{\textit{понимание речевого сообщения} -- процесс \textit{автоматического распознавания речевого сообщения}, а также выделение из сообщения данных и знаний о смысле высказывания \textit{(семантический уровень)}. Применяется для семантического анализа речевого сообщения.}

\scnheader{cистема понимания речевого сообщения}
\scnexplanation{программно-аппаратная техническая система, обладающая средствами записи речевого сигнала и реализующий процесс \textit{понимания речевого сигнала}}
\scnrelfrom{класс выполняемых действий}{понимание речевого сигнала}
%Выше говорилось про сообщение, почему стало сигнала? это что-то другое?
%насчет отношения не уверен, 

\scnheader{трехуровневая архитектура систем \textit{понимания речевого сообщения}}
\scntext{обоснование актуальности}{Большинство современных \texit{систем понимания речевого сообщения} построены на основе трехуровневой архитектуры, когда речевое сообщение последовательно проходит этапы акустического анализа речевого сигнала, лингвистического анализа, в результате которого получается текстовая форма представления исходного сообщения, 
а уже только потом производится его семантический анализ. Однако, из работ по психолингвистике и когнитивной психологии известно, 
что процессы восприятия и понимания в человеческом сознании протекают непрерывно [21], [27], и в общем случае нет необходимости в 
предварительном приведении речевого сообщения к текстовой форме для выполнения смыслового анали за его содержимого. 
Устная и письменная формы речи с равным успехом могут быть обработаны сенсорной и когнитивной системами человека [22]. 
Поэтому вопрос создания методов и систем в которых осуществляется непосредственный переход от обработки сообщения 
в речевой форме к анализу смыслового его содержимого (семантико-акустического анализа) является весьма актуальным.}
\scnrelfromset{недостатки}{
\scnfileitem{Трехуровневая архитектура, в случае решения задачи \textit{понимания речевых сообщений}, обладает рядом ограничений и недостатков:
\begin{scnitemize}
\item{введение промежуточного этапа преобразования речевого сигнала в текст, влечет дополнительные накладные расходы, связанные с необходимостью лингвистической обработки, увеличивая тем самым общую вычислительную сложность алгоритма}
\item{наличие текстового этапа обработки привносит дополнительные ошибки и искажения в следствие ограничений и неполного соответствия лингвистических моделей описываемому процессу, используемых для перехода к текстовому представлению информации на различных стадиях преобразования (фонема-морфема, морфема-лексема, лексема-словосочетание и т.д.) [14]}
\item{при переводе речевого сигнала в текст теряется часть информации, которая может оказаться важной для понимания смысла сообщения, например, громкость, продолжительность звучания, интонация, паузы между словами, которые в тексте могут не всегда однозначно выражаться знаками препинания и др. Особенно актуальной эта проблема становится при анализе сообщений, которые не являются полными предложениями, но при этом могут быть интерпретированы слушателем. Так, например, в повседневной речи предложение, состоящее из одного только звука [a] в зависимости от громкости, интонации и продолжительности звучания может выражать боль, удивление, вопрос, выступать союзом или частицей («ааа, ну его...», «а кто это?», «а если бы сделали по-другому...») [26]}
\end{scnitemize}}
;\scnfileitem{Перевод звукового сигнала в текст делает невозможным анализ аудиофрагментов, не являющихся речевыми сообщениями, но несущих потенциально важную для системы информацию, например:
\begin{scnitemize}
\item{условных сигналов, издаваемых объектами внешней среды, в частности, оборудованием на производстве, автомобилями на дороге и др}
\item{звуков, которые могут соответствовать нештатным ситуациям или сигнализировать об опасности (грохот, лязг, шипение, взрывы, т.д.)}
\item{других звуков, которые потенциально несут информацию о состоянии окружающей среды автоматизированной системы}
\end{scnitemize}
}}
\scnaddlevel{1}
\scntext{вывод}{Отсутствие средств анализа такого рода сигналов сильно ограничивает возможности автоматизированных систем, ориентированных на работу в постоянно меняющейся среде, в том числе -- трудно предсказуемой.}
\scnaddlevel{-1}

\scnheader{семантико-акустический анализ}
\scnexplanation{процесс автоматического \textit{понимания речевого сообщения} подразумевает первичный разбор речевого сообщения с использованием специальных техник обработки сигнала. В ходе их применения про-
изводится вычленение из потока отдельных "акустических образов" слов, которые в свою очередь будут соответствовать определенным узлам (знакам конкретных сущностей или понятий) в семантической сети.
Предполагается, что результаты этапа акустического анализа будут итерационно корректироваться с учетом информации хранящейся в базе знаний системы, в том числе за счет семантического анализа контекстно-зависимой информации.
}

\scnendstruct

\end{SCn}
    
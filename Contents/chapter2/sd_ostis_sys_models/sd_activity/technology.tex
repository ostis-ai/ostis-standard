\begin{SCn}

\bigskip
\scnfragmentcaptiontext{Понятие вида деятельности и технологии}

\scnheader{вид деятельности}
\scnrelto{семейство подклассов}{деятельность}
\scnidtf{класс семантически целостных систем действия, для которых можно унифицировать используемые методы, информационные ресурсы и инструменты}
\scnidtf{класс трудно выполнимых и семантически целостных систем сложных действий}
\scnidtf{класс кластеров систем действий}
\scnidtf{множество деятельностей, которые могут быть реализованы с помощью общей технологии}
\scnhaselement{устранение противоречий в базе знаний}
\scnhaselement{устранение информационных дыр в базе знаний}
\scnhaselement{ликвидация информационного мусора в базе знаний}
\scnhaselement{управление сложным внешним объектом}
\scnhaselement{поддержка семантической совместимости с партнерами}
\scnhaselement{проектирование}
	\scnaddlevel{1} 
		\scnidtf{проектная деятельность}
		\scnidtf{построение такого описания (в частности, описания структуры) некоторого материального объекта, которого достаточно для воспроизводства (реализации, материализации) этого объекта либо при одиночном (уникальном), либо при массовом (промышленном) воспроизводстве указанного объекта}
		\scnnote{Примерами проектирования являются:
\begin{scnitemize}
\item проектирование здания;
\item проектирование машиностроительной конструкции;
\item проектирование микросхемы;
\item проектирование ostis-системы;
\item разработка системы шунтирования сердца;
\item разработка такого описания сложной геометрической фигуры, которого было бы достаточно для построения изображения (рисунка) этой фигуры с помощью, например, циркуля и линейки.
\end{scnitemize}}
	\scnaddlevel{-1}
\scnhaselement{разработка плана производства материального объекта по заданному проекту этого объекта}
	\scnaddlevel{1}
		\scnsuperset{разработка плана единичной реализации материального объекта по заданному проекту этого объекта}
		\scnsuperset{разработка плана массовой реализации материальных объектов по заданному их типовому проекту}
		\scnnote{Примерами данного вида действий являются:
\begin{scnitemize}
\item разработка плана-графика строительства конкретного здания;
\item разработка типового плана строительства зданий по заданному их типовому проекту;
\item разработка типового плана операций шунтирования сердца;
\item разработка алгоритма построения \uline{изображения} заданной геометрической фигуры с помощью циркуля и линейки
\end{scnitemize}}
	\scnaddlevel{-1}
\scnhaselement{производство}
	\scnaddlevel{1}
		\scnidtf{воспроизводство материального объекта по заданному его проекту и плану реализации}
		\scnidtf{производственная деятельность}
		\scnnote{Примерами данного вида действий являются:
\begin{scnitemize}
\item непосредственно строительство конкретного здания;
\item проведение конкретной хирургической операции;
\item процесс построения \uline{изображения} (рисунка) геометрической фигуры с помощью циркуля и линейки.
\end{scnitemize}}
	\scnaddlevel{-1}
\scnhaselement{реинжиниринг}
\scnhaselement{анализ}
\scnhaselement{интеграция}
	\scnaddlevel{1}
		\scnidtf{синтез}
	\scnaddlevel{-1}
\scnhaselement{деятельность в области здравоохранения}
\scnhaselement{образовательная деятельность}
\scnhaselement{эксплуатация сложного объекта}
\scnhaselement{научно-исследовательская деятельность}
\scnhaselement{управление}
	\scnaddlevel{1}
		\scnsuperset{целенаправленная координация деятельности нескольких субъектов}
		\scnaddlevel{1}
			\scnidtf{управление целенаправленной коллективной деятельностью нескольких субъектов}
		\scnaddlevel{-1}
	\scnaddlevel{-1}

\scnheader{проектирование}
\scnidtf{действие, направленное на построение (разработку) такой \uline{информационной} модели (проекта) некоторой \uline{материальной} сущности, которой \uline{достаточно}, чтобы соответствующий индивидуальный или коллективный субъект по соответствующей технологии (т.е. с помощью соответствующих методов и средств (инструментов)) смог воспроизвести (изготовить) указанную материальную сущность либо в одном экземпляре, либо в достаточно большом количестве таких экземпляров (копий), т.е. воспроизвести в промышленном масштабе}

\scnheader{производство}
\scnidtf{воспроизводство}
\scnidtf{изготовление}
\scnidtf{реализация}
\scnidtf{материализация}
\scnidtf{построение, синтез материальной сущности (артефакта)}
\scnidtf{изготовление материальной сущности в одной или во множестве экземпляров (копий)}
\scnidtf{производство (как действие)}

\scnheader{реинжиниринг}
\scnidtf{модификация}
\scnidtf{внесение изменений в некую сущность}
\scnidtf{обновление}
\scnidtf{реинжиниринг}
\scnidtf{перепроектирование}
\scnidtf{реконфигурация}
\scnidtf{трансформирование}
\scnsuperset{совершенствование}
	\scnaddlevel{1}
		\scnidtf{модификация, направленной на повышение качества модифицируемой сущности}
		\scnidtf{повышение качества}
		\scnidtf{улучшение}
		\scnsuperset{самосовершенствование}
		\scnaddlevel{1}
			\scnidtf{совершенствование, выполняемое самой совершенствуемой сущностью}
		\scnaddlevel{-1}
		\scnsuperset{совершенствование, осуществляемое извне}
		\scnnote{Самосовершенствоваться и обучаться могут только достаточно развитые кибернетические системы. Но совершенствоваться усилиями внешних субъектов могут любые сущности.}
	\scnaddlevel{-1}

\scnheader{анализ}
\scnidtf{построение (разработка, создание) спецификации (описания) основных связей и/или структуры, свойств, закономерностей, соответствующих (описываемой) сущности}
\scnnote{Объектом анализа может быть не только материальная сущность, но и процесс, ситуация, статическая структура, внешняя информационная конструкция, знание, понятие и другие абстрактные сущности}

\scnheader{интеграция}
\scnidtf{синтез}
\scnidtf{соединение}
\scnidtf{объединение}
\scnidtf{сборка}
\scnsubdividing{эклектичная интеграция\\
	\scnaddlevel{1}
		\scnidtf{интеграция без разрушения целостности интегрируемых сущностей}
		\scnidtf{интеграция без взаимопроникновения}
		\scnidtf{соединение систем по их входам/выходам}
	\scnaddlevel{-1}
;глубокая интеграция\\
	\scnaddlevel{1}
		\scnidtf{интеграция, в результате которой получается гибридная сущность}
		\scnidtf{интеграция с разрушением целостности (взаимопроникновением "диффузий"{}) интегрируемых сущностей}
		\scnidtf{"бесшовная"{} интеграция}
	\scnaddlevel{-1}
}

\scnheader{вид деятельности}
\scnexplanation{Если классу легко выполнимых сложных действий ставится в соответствие чаще всего \uline{один} \textit{метод} и, возможно, некоторый набор инструментальных средств, используемых в этом методе, то каждому виду деятельности ставится в соответствие своя \textbf{\textit{технология}}, включающая в себя некоторый набор используемых \textit{методов}, а также набор \textit{инструментальных средств}, используемых в этих \textit{методах}. Сложность здесь заключается:
\begin{scnitemize}
	\item в нетривиальности организации использования всего арсенала имеющейся \textit{технологии} для реализации (выполнения) каждой соответствующей \textit{деятельности};
	\item в трудности, а часто и в принципиальной невозможности \uline{полностью} автоматизировать реализацию соответствующей \textit{деятельности}.
\end{scnitemize}}

\scnheader{следует отличать*}
\scnhaselementset{действие\\
	\scnaddlevel{1}	
		\scnhaselementrole{пример}{Процесс доказательства Теоремы Пифагора}
		\scnaddlevel{1}	
		\scniselement{действие направленное на построение доказательства теоремы Геометрии Евклида}
	\scnaddlevel{-2}
;класс действий\\
	\scnaddlevel{1}
		\scnhaselementrole{пример}{процесс доказательства теоремы}
		\scnaddlevel{1}
		\scnidtftext{имя нарицательное}{действие, направленное на построение доказательства (логического обоснования) теоремы}
		\scnidtftext{имя собственное}{Класс действий, направленных на построение доказательств (логических обоснований) всевозможных теорем в различных формальных теориях}
	\scnaddlevel{-2}
;деятельность\\
	\scnaddlevel{1}
		\scnhaselementrole{пример}{Процесс эволюции Геометрии Евклида}
			\scnaddlevel{1}
			\scnidtf{Процесс эволюции формальной теории, являющейся формальным представлением Геометрии Евклида}
			\scnexplanation{В данный процесс входит и генерация гипотез в рамках Геометрии Евклида, и доказательство теорем, и выявление противоречий между высказываниями, и разрешение этих противоречий, и минимизация числа используемых определяемых понятий, и многое другое}
			\scnaddlevel{-1}
		\scnnote{\textit{деятельность} -- это то, что "превращает"{} множество самостоятельных и в определенной степени независимых \textit{действий}, принадлежащих разным \textit{классам действий}, в целостную, целенаправленную, сбалансированную систему \textit{действий}, ориентированную, прежде всего на поддержание качества и эволюцию \textit{кибернетических систем}, а также на обеспечение их адаптации к новым, ранее не предусмотренным обстоятельствам.}
	\scnaddlevel{-1}
;вид деятельности\\
	\scnaddlevel{1}
		\scnhaselementrole{пример}{процесс эволюции формальной теории}
		\scnaddlevel{1}
			\scnidtftext{имя собственное}{Класс процессов, направленных на эволюцию всевозможных формальных теорий (логических онтологий), которая также включает в себя возможность коррекции этих теорий.}
	\scnaddlevel{-2}
}

\scnheader{спецификация*}
\scnsuperset{сужение отношения по первому домену(спецификация*; вид деятельности)*}
	\scnaddlevel{1}
		\scnidtftext{часто используемый sc-идентификатор}{
спецификация вида деятельности*}
		\scneq{технология*}
		\scnaddlevel{1}
			\scnidtf{технология реализации (выполнения) деятельности соответствующего (заданного) вида*}
			\scnrelfrom{второй домен}{\textbf{технология}\\
			\scnidtf{технология соответствующего вида деятельности}
			\scnrelboth{аналог}{декларативный метод выполнения действий соответствующего класса}
			\scnaddlevel{1}
				\scnrelboth{аналог}{декларативная спецификация выполнения действия}
			\scnaddlevel{-1}
			\scnexplanation{\textit{технология} (как спецификация соответствующего вида деятельности) включает в себя:
			\begin{scnitemize}
				\item указание \textit{контекста}* специфицируемого \textit{вида деятельности};
				\item указание \textit{множества используемых методов}*, множества используемых инструментов, а также используемых материалов.
			\end{scnitemize}}
			}
		\scnaddlevel{-1}
	\scnaddlevel{-1}	

\scnheader{технология}
\scnexplanation{Каждая \textit{технология} представляет собой комплекс \textit{методов} (методик) и средств, обеспечивающих выполнение некоторого множества \textit{действий}, входящих в состав соответствующего \textit{вида деятельности}. Каждая \textit{технология} задается:
\begin{scnitemize}
	\item множеством методов (методик), которое разбивается на классы методов, эквивалентных по своей операционной семантике (по набору агентов, осуществляющих интерпретацию соответствующего класса методов);
	\item множеством агентов, являющихся средством интерпретации методов из указанного выше множества.
\end{scnitemize}
Указанное множество агентов также разбивается на подмножества, каждое из которых  соответствует своему классу методов и обеспечивает интерпретацию методов только этого класса.}

\scnidtf{множество (комплекс) навыков, обеспечивающих выполнение такого множества действий (задач), для которых отсутствует общий метод их выполнения}
\scnidtf{методика, инструментарий и дополнительные ресурсы, которые обеспечивают выполнение каждой конкретной деятельности, принадлежащей соответствующему виду деятельности}
\scnexplanation{с формальной точки зрения каждая технология задается ориентированной связкой, компонентами которой являются
\begin{scnitemize} 
	\item знак множества используемых методов 
	\item знак множества используемых инструментов 
	\item знак множества дополнительных используемых ресурсов
\end{scnitemize}}
\scnidtf{комплекс методов и средств (инструментов), с помощью которого некий субъект (который может быть как индивидуальным, так и коллективным) осуществляет некоторую деятельность (некоторое целенаправленное множество действий, входящих в состав этой деятельности)}
\scnsuperset{технология научно-теоретической деятельности}
\scnsuperset{технология проектирования}
	\scnaddlevel{1}
		\scnidtf{технология проектной деятельности}
		\scnidtf{технология построения такой информационной модели соответствующей сущности (артефакта), которой достаточно для воспроизводства этой сущности}
	\scnaddlevel{-1}
\scnsuperset{технология производства}
	\scnaddlevel{1}
		\scnidtf{технология производственной деятельности}
		\scnidtf{технология воспроизводства некоторого вида сущностей по заданным проектам этих сущностей}
	\scnaddlevel{-1}
\scnsuperset{технология здравоохранения}
\scnsuperset{технология образования}
	\scnaddlevel{1}
		\scnidtf{технология подготовки молодых специалистов}
		\scnidtf{технология образовательной деятельности}
	\scnaddlevel{-1}

\scnheader{отношение, заданное на множестве* (технология*)}
\scnhaselement{методы*}
	\scnaddlevel{1}
		\scnidtf{семейство методов, используемых в специфицируемой технологии  с дополнительным указанием их иерархии (т.е. с указанием того, какие методы используются при реализации других методов)}
	\scnaddlevel{-1}
\scnhaselement{активный инструмент*}
	\scnaddlevel{1}
		\scnidtf{средство, которое само способно выполнять некоторые действия, но при этом им надо как-то управлять (например, транспортные средства, компьютеры, …)}
		\scnidtf{средства автоматизации}
	\scnaddlevel{-1}
\scnhaselement{пассивный инструмент*}
	\scnaddlevel{1}
		\scnidtf{средство, которое само ничего делать не может (например, молоток, лопата, ножницы, …)}
	\scnaddlevel{-1}
\scnhaselement{комплектация*}
\scnhaselement{расходные средства*}
\scnhaselement{сырье*}
\scnhaselement{продукты*}
\scnhaselement{общий продукт*}
	\scnaddlevel{1}
		\scnidtf{объединенный (интегрированный) продукт*}
	\scnaddlevel{-1}
\scnhaselement{реализация технологии*}
	\scnaddlevel{1}
		\scnidtf{вариант (форма) реализации технологии*}
	\scnaddlevel{-1}
\scnhaselement{частная технология*}
	\scnaddlevel{1}
		\scnidtf{быть частной технологией по отношению к заданной технологии*}
	\scnaddlevel{-1}

\scnheader{продукты*}
\scnidtf{производимые сущности*}
\scnidtf{изготавливаемые материальные сущности*}
\scnidtf{продукция*}
\scnidtf{результаты выполнения соответствующего множества действий, осуществляемых во внешней среде*}
\scnidtf{продукты технологии*}
\scnidtf{множество материальных сущностей, производимых (создаваемых, порождаемых, изготавливаемых) с помощью заданной технологии*}
\scnidtf{то, что является "сухим остатком"{} при использовании данной технологии*}

\scnheader{технология}
\scnnote{Поскольку разработка каждой конкретной \textit{технологии} требует больших затрат, очень важно, чтобы \textit{технологии} создавались не под конкретные \textit{деятельности}, а для целых классов деятельностей (\textit{видов деятельности}). При этом важно, чтобы разрабатываемые \textit{технологии} охватывали как можно большее количество деятельностей, входящих в состав указанных \textit{видов деятельности}. Из этого следует целесообразность конвергенции и унификации различных сфер \textit{деятельности} для того, чтобы повысить мощность применения (использования) каждой разрабатываемой \textit{технологии}. Кроме того важна \textit{совместимость технологий}, позволяющая решать \textit{задачи}, требующие одновременного использования нескольких \textit{технологий}, причем, в непредсказуемых сочетаниях. Очень важно также, кроме \textit{видов деятельности}, которым соответствуют конкретные \textit{технологии}, ввести \textit{обобщенные виды деятельности} и построить их иерархии явно фиксировать стандарты, которым должны соответствовать все виды соответствующего обобщенного \textit{вида деятельности}. Это необходимо для обеспечения совместимости \textit{технологий}. Все используемые технологии должны "пронизывать"{} друг друга и составлять стройную иерархическую систему совместимых технологий (сумму технологий).}

\scnheader{класс технологий}
\scnidtf{множество похожих технологий, использующих, например, одинаковые методики и/или одинаковые активные инструменты и/или одинаковые пассивные инструменты и/или похожие множества продуктов}
\scnhaselement{технология проектирования}
	\scnaddlevel{1}
		\scnsuperset{технология проектирования интеллектуальных компьютерных систем}
		\scnsuperset{технология проектирования программных систем}
		\scnsuperset{технология проектирования микросхем}
		\scnsuperset{технология машиностроительного проектирования}
	\scnaddlevel{-1}
\scnhaselement{технология рецептурного производства}
	\scnaddlevel{1}
		\scnsuperset{технология производства молочных продуктов}
		\scnsuperset{технология производства мясных продуктов}
		\scnsuperset{технология фармацевтического производства}
	\scnaddlevel{-1}

\end{SCn}

\begin{SCn}

\bigskip
\scnfragmentcaptiontext{Понятие деятельности}

\scnheader{деятельность} 
\scnidtf{целостный, целенаправленный процесс \uline{поведения} (функционирования) одного субъекта или сообщества субъектов, осуществляемый на основе хорошо или не очень хорошо продуманной и согласованной \textit{технологии} в последнем случае качество деятельности определяется уровнем интеллекта единоличного или коллективного субъекта, осуществляющего этот целенаправленный процесс.}
\scnidtf{система действий, являющаяся некоторым кластером семантически близких действий, обладающих семантической близостью, семантической связностью и семантической целостностью}
\scnidtf{трудно выполнимая семантически целостная система действий}
\scnidtf{кластер множества действий, определяемый семантической близостью этих действий}
\scnidtf{система связанных между собой действий, имеющих общий контекст, общую область выполнения этих действий}
\scnnote{В состав каждой конкретной \textit{деятельности} входят \textit{действия}, являющиеся \textit{поддействиями}* других \textit{действий}, входящих в состав этой же \textit{деятельности}. При этом для каждого \textit{действия}, входящего в состав \textit{деятельности}, все поддействия этого \textit{действия} также входят в состав этой \textit{деятельности}. 

В состав каждой конкретной \textit{деятельности} входят также \textit{действия}, не являющиеся \textit{поддействиями}* других \textit{действий}, входящих в состав этой же \textit{деятельности}. Такие "первичные"{} ("независимые"{}, "самостоятельные"{}, "автономные"{}) \textit{действия} для заданной \textit{деятельности} могут инициироваться \uline{извне} этой \textit{деятельности} с помощью соответствующих инициирующих эти \textit{действия ситуаций} или \textit{событий}. Примерами таких инициирующих ситуаций, "порождающих"{} соответствующие действия, являются:
\begin{scnitemize}
	\item появление в \textit{базе знаний} каких-либо противоречий, информационных дыр, информационного мусора;
	\item появление в \textit{базе знаний} описаний (информационных моделей) каких-либо нештатных ситуаций в сложном объекте управления, на которые необходимо реагировать;
	\item появление в \textit{базе знаний} формулировок различного рода задач с явным указанием инициирования соответствующих действий, направленных на решение этих задач.
\end{scnitemize}
	К числу указанных "первичных"{} ("независимых"{}) \textit{действий}, входящих в состав \textit{объединенной деятельности кибернетической системы}, также относятся:
\begin{scnitemize}
	\item сложное действие, целью которого является перманентное обеспечение комплексной \textit{безопасности кибернетической системы};
	\item сложное действие, целью которого является перманентное  повышение качества информации (базы знаний), хранимой в памяти \textit{кибернетической системы};
	\item сложное действие, целью которого является перманентное повышение \textit{качества решателя задач кибернетической системы};
	\item сложное действие, целью которого является перманентная поддержка высокого уровня \textit{семантической совместимости} кибернетической системы со своими партнерами.
\end{scnitemize}
}

\scnheader{отношение, заданное на множестве*(деятельность)}
\scnhaselement{субъект*}
	\scnaddlevel{1}
		\scnidtf{быть субъектом заданного действия или деятельности*}
 		\scnidtf{кибернетическая система, которая в рамках заданного действия или деятельности выполняет ту или иную роль, воздействует на некий объект действия, используя тот или иной инструмент*}
 		\scniselement{отношение, заданное на множестве*(действие)}
 	\scnaddlevel{-1}
\scnhaselement{контекст*}
	\scnaddlevel{1} 
		\scnidtf{информационный контекст, в рамках которого осуществляется выполнение заданного действия или деятельности*}
		\scnidtf{область исполнения действия или деятельности*}
		\scnidtf{область действия или деятельности*}
		\scnrelfrom{первый домен}{(действие $\cup$ деятельность)}
		\scnidtf{совокупность знаний, достаточных для информационного обеспечения заданного действия или заданной деятельности}
		\scniselement{отношение, заданное на множестве* (действие)}
		\scnnote{Локализация (минимизация) \textit{контекста} заданного действия или деятельности является важнейшим "подготовленным"{} этапом, обеспечивающим существенное снижение "накладных расходов"{} при непосредственном выполнении этого \textit{действия} или \textit{деятельности}.}
	\scnaddlevel{-1}
\scnnote{Чаще всего \textit{контекстом} заданного \textit{действия} или \textit{деятельности} является некоторая \textit{предметная область} вместе с соответствующей ей интегрированной (объединенной) \textit{онтологией}. Поэтому хорошо продуманная декомпозиция \textit{базы знаний} интеллектуальной компьютерной системы на иерархическую систему \textit{предметных областей} и соответствующих им \textit{онтологий} имеет важное "практическое"{} значение, существенно повышающее качество (в частности, быстродействие) \textit{решателя задач} интеллектуальной компьютерной системы благодаря априорному  разбиению множества выполняемых \textit{действий} (решаемых задач) по соответствующих им \textit{контекстам}.}

\scnheader{следует отличать*}
\scnsuperset{\scnmakesetlocal{действие\\
 	\scnaddlevel{1}
		\scnidtf{процесс достижения конкретной цели конкретных обстоятельствах}
		\scnidtf{процесс решения конкретной задачи в конкретных условиях}
		\scnidtf{процесс задуманный, инициированный и осуществленный некоторым (или некоторыми) субъектами (кибернетическими системами)}
		\scnnote{\textit{действие} (точнее, соответствующая форма участия в его выполнении) является частью (фрагментом) \textit{деятельности} всех участвующих в этом субъектов (кибернетических систем)}
	\scnaddlevel{-1}
;деятельность\\
	\scnaddlevel{1}
		\scnidtf{система действий выполняемых соответствующим субъектом (кибернетической системой) "скрепленное"{} общим контекстом и определенным набором используемых навыков и инструментов}
		\scnnote{В отличие от \textit{действия}, \textit{деятельность} носит чаще всего перманентный характер в рамках времени существования соответствующего субъекта}
	\scnaddlevel{-1}
}}

\scnheader{деятельность кибернетической системы}
\scnidtf{полная система действий, выполняемых соответствующей кибернетической системой}
\scnidtf{деятельность субъекта}
\scnidtf{система всех действий соответствующего субъекта}
\scnsubdividing{внутренняя деятельность субъекта\\
	\scnaddlevel{1}
		\scnidtf{внутренняя деятельность соответствующего субъекта}
		\scnidtf{деятельность некоторого субъекта по обработке информации}
		\scnidtf{информационная деятельность}
	\scnaddlevel{-1}
;поведение субъекта\\
	\scnaddlevel{1}
		\scnidtf{внешнее поведение соответствующего субъекта}
		\scnidtf{деятельность субъекта во внешней среде}
	\scnaddlevel{-1}
}

\scnheader{(действие $\cup$ деятельность)}
\scnrelfrom{смотрите}{Теория действий, воздействий, деятельности (В.В. Мартынов), субъект, объект, инструмент, метод, навык, технология!!!}
\end{SCn}

\begin{SCn}
	
\scnsectionheader{\currentname}
	
\scnstartsubstruct
	
\scnheader{Предметная область искусственных нейронных сетей}
\scniselement{предметная область}
\scnsdmainclasssingle{искусственная нейронная сеть}
\scnsdclass{***}

\scnheader{искусственных нейронная сеть}	
\scnreltoset{разбиение}{конечнозначные нейронные сети\\
	\scnaddlevel{1}
	\scnreltoset{разбиение}{двоичные нейронные сети\\
		\scnaddlevel{1}
		\scnreltoset{разбиение}{бинарные нейронные сети;биполярные нейронные сети}
		\scnaddlevel{-1}
	;троичные нейронные сети}
	\scnaddlevel{-1}
;комплексно-численные нейронные сети\\
	\scnaddlevel{1}
	\scnrelto{включение}{вещественно-численные нейронные сети\\
		\scnrelto{включение}{целочисленные нейронные сети}
	}
	\scnaddlevel{-1}
;рациональные нейронные сети
;повышающие размерность нейронные сети
;понижающие размерность нейронные сети
;прерывные нейронные сети
;непрерывные нейронные сети
;дифференцируемые нейронные сети
;недифференцируемые нейронные сети
;гомогенные нейронные сети
;гетерогенные нейронные сети
;нейронные сети без контекстных нейронов
;нейронные сети с контекстными нейронами
;нейронные сети без обратных связей
;нейронные сети с обратными связями
;нейронные сети без скрытых нейронов
;нейронные сети со скрытыми нейронами
;многослойные нейронные сети
;однослойные нейронные сети
;стохастические нейронные сети
;детерминированные нейронные сети
;релаксационные нейронные сети\\
	\scnaddlevel{1}
	\scnreltoset{разбиение}{релаксационные сети с хаотическим поведением;релаксационные сети с устойчивым поведением}
	\scnaddlevel{-1}	
}
}	
\end{SCn}
\begin{SCn}
	
\scnsectionheader{\currentname}
	
\scnstartsubstruct
	
\scnheader{Предметная область искусственных нейронных сетей}
\scniselement{предметная область}
\scnsdmainclasssingle{искусственная нейронная сеть}

\scnrelfromset{частная предметная область}{
Предметная область ИНС с заданным направлением связей\\
    \scnaddlevel{1}
    \scnrelfromset{частная предметная область}{
    Предметная область ИНС с прямым связями\\
        \scnaddlevel{1}
        \scnrelfromset{частная предметная область}{
        Предметная область персептронов\\
            \scnaddlevel{1}
            \scnrelfromset{частная предметная область}{
            Предметная область персептронов Розенблатта
            ;Предметная область персептронов Румельхарта
            ;Предметная область автоэнкодерных ИНС
            }
            \scnaddlevel{-1}
        ;Предметная область ИНС радиально-базисных функций
        ;Предметная область машин опорных векторов
        }
        \scnaddlevel{-1}
    ;Предметная область ИНС с обратными связями\\
        \scnaddlevel{1}
        \scnidtf{Предметная область рекуррентных ИНС}
        \scnrelfromset{частная предметная область}{
        Предметная область ИНС Джордана
        ;Предметная область ИНС Элмана
        ;Предметная область LSTM-элементов
        ;Предметная область GRU-элементов
        }
        \scnaddlevel{-1}
    }
    \scnaddlevel{-1}
;Предметная область обучения ИНС\\
    \scnaddlevel{1}
    \scnrelfromset{частная предметная область}{
    Предметная область ИНС, обучающихся с учителем
    ;Предметная область ИНС, обучающихся без учителя\\
        \scnaddlevel{1}
        \scnrelfromset{частная предметная область}{
        Предметная область обучающихся автоэнкодерных ИНС
        ;Предметная область ИНС глубокого доверия
        ;Предметная область генеративно-состязательных ИНС
        ;Предметная область самоорганизующихся карт Кохонена
        ;Предметная область ИНС Хопфилда
        ;Предметная область подкрепляющего обучения ИНС
        }
        \scnaddlevel{-1}
    }
    \scnaddlevel{-1}
;Предметная область топологий ИНC\\
    \scnaddlevel{1}
    \scnrelfromset{частная предметная область}{
    Предметная область полносвязных ИНC
    ;Предметная область многослойных ИНC
    ;Предметная область слабосвязных ИНC
    }
    \scnaddlevel{-1}
;Предметная область задач, решаемых с помощью ИНС\\
    \scnaddlevel{1}
    \scnrelfromset{частная предметная область}{
    Предметная область ИНС, решающих задачу классификации
    ;Предметная область ИНС, решающих задачу аппроксимации
    ;Предметная область ИНС, решающих задачу управления
    ;Предметная область ИНС, решающих задачу фильтрации
    ;Предметная область ИНС, решающих задачу детекции
    ;Предметная область ИНС, решающих задачу с ассоциативной памятью
    }
    \scnaddlevel{-1}
;Предметная область интеграции ИНС с базой знаний
}
\scnheader{Искусственная нейронная сеть}
\scnaddlevel{1}
	\scnidtf{и.н.с.}
	\scnidtf{нейронная сеть}
	\scnidtf{биологически инспирированная математическая модель, обладающая обобщающей способностью после выполнения процедуры обучения}
	\scniselement{математическая модель}
\scnaddlevel{-1}

\scnheader{Математическая модель}
\scnaddlevel{1}
	\scnidtf{упрощенное описание объекта реального мира, выраженное с помощью математической символики}
\scnaddlevel{-1}

\scnheader{Обобщающая способность}
\scnaddlevel{1}
	\scnidtf{generalization ability}
	\scnidtf{способность модели выдавать корректные результаты для экземпляров, не входящих в обучающую выборку}
\scnaddlevel{-1}

\scnheader{Экземпляр}
\scnaddlevel{1}
	\scnidtf{instance}
	\scnidtf{пример}
	\scnidtf{example}
	\scnidtf{один объект, наблюдение, транзакция или запись, выраженный в виде вектора, компоненты которого представлены численными и/или категориальными значениями}
\scnaddlevel{-1}

\scnheader{Признаки}
\scnaddlevel{1}
	\scnidtf{features}
	\scnidtf{входные атрибуты, используемые для предсказания целевой переменной}
	\scnidtf{компоненты вектора экземпляра}
	\scnnote{могут быть как численными, так и категориальными}
\scnaddlevel{-1}

\scnheader{Обучающая выборка}
\scnaddlevel{1}
	\scnidtf{выборка экземпляров, используемая для изменения параметров н.с. в процессе ее обучения}
	\scnidtf{training set}
\scnaddlevel{-1}

\scnheader{Параметры нейронной сети}
\scnaddlevel{1}
\scnidtf{переменные, значения которых изменяются в ходе процедуры обучения}
\scnidtf{компоненты векторов весовых коэффициентов, ядер свертки и пороги нейронов и.н.с.}
\scnaddlevel{-1}

\scnheader{Вектор весовых коэффициентов}
\scnaddlevel{1}
	\scnidtf{вектор параметров отдельно взятого нейрона, компоненты которого изменяются в процессе обучения и.н.с.}
\scnaddlevel{-1}

\scnheader{Ядро свертки}
\scnaddlevel{1}
	\scnidtf{квадратная матрица произвольной размерности, компоненты которой изменяются в процессе обучения и.н.с.}
\scnaddlevel{-1}

\scnheader{Порог нейрона}
\scnaddlevel{1}
	\scnidtf{скаляр, значение которого изменяется в процессе обучения и.н.с.}
\scnaddlevel{-1}

\scnheader{Нейрон}
\scnaddlevel{1}
	\scnidtf{отдельный обрабатывающий элемент и.н.с., выполняющий функциональное преобразование взвешенной суммы компонент вектора входных значений с помощью функции активации}
	\scnidtf{математическая модель реального биологического нейрона}
\scnaddlevel{-1}

\scnheader{Вектор входных значений}
\scnaddlevel{1}
	\scnidtf{вектор экземпляра, компоненты которого прошли предварительную обработку}
	\scnexplanation{такая предварительная обработка как правило включает в себя трансформацию категориальных признаков в численные, а также нормализацию, проектирование признаков и т.д.}
\scnaddlevel{-1}

\scnheader{Взвешенная сумма входных значений}
\scnaddlevel{1}
	\scnidtf{сумма покомпонентного произведения векторов входных значений и весовых коэффициентов нейрона}
	\scnidtf{взвешенная сумма}
	\scnidtf{в.с.}
	\scnrelfrom{формула}{
		\begin{equation*}
			S = \sum_{i=1}^{n} w_ix_i
		\end{equation*}
		где \textit{n} -- размерность вектора входных значений, $w_i$ -- \textit{i}-тый компонент вектора весовых коэффициентов, $x_i$ -- \textit{i}-тый компонент вектора входных значений
	}
\scnaddlevel{-1}

\scnheader{Функция активации нейрона}
\scnaddlevel{1}
	\scnidtf{Функция, результат применения которой к в.с. нейрона определяет его выходное значение}
	\scnrelfromvector{варианты функций активации}{
		Линейная\\
		\scnaddlevel{1}
		\scnrelfrom{формула}{
			\begin{equation*}
				y = kS
			\end{equation*}
			где \textit{k} -- коэффициент наклона прямой, \textit{S} -- в.с.
		}
		\scnaddlevel{-1}
		;Пороговая\\
		\scnaddlevel{1}
		\scnrelfrom{формула}{
			\begin{equation*}
				y = sign(S) = 				
				\begin{cases}
					1, S > 0,\\
					0, S \leq 0
				\end{cases}
			\end{equation*}
		}
		\scnaddlevel{-1}
		;Сигмоидная\\
		\scnaddlevel{1}
		\scnrelfrom{формула}{
			\begin{equation*}
				y = \frac{1}{1+e^{-cS}}
			\end{equation*}
			где \textit{с} > 0 -- коэффициент, характеризующий ширину сигмоидной функции по оси абсцисс, \textit{S} -- в.с.
		}
		\scnaddlevel{-1}
		;Гиперболический тангенс\\
		\scnaddlevel{1}
		\scnrelfrom{формула}{
			\begin{equation*}
				y = \frac{e^{cS}-e^{-cS}}{e^{cs}+e^{-cS}}
			\end{equation*}
			где \textit{с} > 0 -- коэффициент, характеризующий ширину сигмоидной функции по оси абсцисс, \textit{S} -- в.с.
		}
		\scnaddlevel{-1}
		;Softmax\\
		\scnaddlevel{1}
		\scnrelfrom{формула}{
			\begin{equation*}
				y_j = softmax(S_j) = \frac{e^{S_j}}{\sum_{j} e^{S_j}}
			\end{equation*}
			где $S_j$ -- в.с. \textit{j}-го выходного нейрона
		}
		\scnaddlevel{-1}
		;ReLU\\
		\scnaddlevel{1}
		\scnrelfrom{формула}{
			\begin{equation*}
				y = F(S) =
				\begin{cases}
					S, S > 0,\\
					kS, S \leq 0
				\end{cases}
			\end{equation*}
			где \textit{k} = 0 или принимает небольшое значение, например, 0.01 или 0.001.
		}
		\scnaddlevel{-1}
	}
\scnaddlevel{-1}

\scnheader{Тестовая (контрольная) выборка}
\scnaddlevel{1}
	\scnidtf{выборка экземпляров, используемая для проверки обобщающей способности обученной и.н.с.}
	\scnidtf{test set}
	\scnnote{элементы контрольной выборки не используются в процессе обучения}
\scnaddlevel{-1}

\scnheader{Валидационная выборка}
\scnaddlevel{1}
	\scnidtf{выборка экземпляров, используемая для определения (настройки) гиперпараметров (архитектуры) и.н.с.}
	\scnnote{элементы валидационной выборки не используются в процессе обучения}
\scnaddlevel{-1}

\scnheader{Гиперпараметры и.н.с}
\scnaddlevel{1}
\scnidtf{набор параметров и.н.с., определяющих ее архитектуру (количество слоев и.н.с., количество нейронов в каждом слое и т.д.)}
\scnaddlevel{-1} 

\scnheader{Слой и.н.с.}

\scnheader{Обучение}
\scnaddlevel{1}
	\scnidtf{процесс итеративного изменения параметров и.н.с., минимизирующий некоторую заданную функцию ошибки, для достижения приемлемого уровня обобщающей способности}
	\scnrelfromvector{основные подходы}{
		Обучение с учителем
		\scnaddlevel{1}
			\scnidtf{процесс изменения параметров и.н.с, минимизирующий разницу между выходом и.н.с. и целевой переменной для элементов обучающей выборки, относительно некоторой заданной функции ошибки}
		\scnaddlevel{-1}
		;Обучение без учителя
		\scnaddlevel{1}
			\scnidtf{процесс изменения параметров и.н.с. без использования заданных целевых переменных (в режиме самоорганизации)}
		\scnaddlevel{-1}
	}
\scnaddlevel{-1}

\scnheader{Функция ошибки}

\scnheader{Целевая переменная}
\scnaddlevel{1}
	\scnidtf{цель}
	\scnidtf{target}
	\scnidtf{метка}
	\scnidtf{label}
	\scnidtf{численная или категориальная переменная, которая предсказывается для каждого нового экземпляра}
\scnaddlevel{-1}

\scnendstruct \scnendcurrentsectioncomment

\end{SCn}
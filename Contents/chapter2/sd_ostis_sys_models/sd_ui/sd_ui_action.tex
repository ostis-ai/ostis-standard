\begin{SCn}

\scnsectionheader{\currentname}

\scnstartsubstruct

\scnheader{Предметная область интерфейсных команд пользователя ostis-системы}
\scniselement{предметная область}
\scnsdmainclasssingle{***}
\scnsdclass{***}

\scnheader{интерфейсное действие пользователя ostis-системы}
\scnexplanation{Минимально осмысленный фрагмент интерфейсного языка пользователей ostis-системы}
\scnreltoset{разбиение}{элементарное пользовательское действие;интерфейсная команда пользователя ostis-системы;сообщение пользователя ostis-системы}
\scnheader{класс интерфейсных команд пользователя ostis-системы}
\scnidtf{класс команд}
\scnhaselement{интерфейсная команда пользователя ostis-системы}
	\scnaddlevel{1}
	\scnexplanation{Спецификация инициируемого пользователем действия ostis-системы, оформленная 		на языке интерфейсных действий пользователей ostis-системы и включающая в себя указание класса 	инициируемого действия и указание аргументов этого действия, то есть множества сущностей, над 		которыми это действие должно быть выполнено.
	}
	\scnreltoset{разбиение}{команда, оформленная на языке интерфейсных действий;команда, оформленная на внешнем языке}
	\scnaddlevel{-1}
\scnreltoset{разбиение}{атомарный класс интерфейсных команд\\
	\scnreltoset{разбиение}{класс интерфейсных команд с фиксированым количеством аргументов\\
		\scnreltoset{разбиение}{класс интерфейсных команд без аргументов;класс интерфейсных команд 		с одним аргументом;класс интерфейсных команд с двумя аргументами;класс интерфейсных команд 		с тремя и более аргументами};
	класс интерфейсных команд с произвольным количеством аргументов};
неатомарный класс интерфейсных команд}

\scnheader{команда, оформленная на языке интерфейсных действий}
\scnexplanation{Команда, спецификация которой генерируется автоматически путём инициирования элементарных пользовательских действий при использовании компонентов пользовательского интерфейса.
}

\scnheader{команда, оформленная на внешнем языке}
\scnexplanation{Команда, спецификация которой генерируется либо самим пользователем с использованием средств редакторов базы знаний, либо с помощью инструкций на естественном языке.}

\scnheader{атомарный класс команд}
\scnexplanation{Принадлежность некоторого класса команд множеству атомарных классов команд фиксирует факт того, что данная спецификация является достаточной для того, чтобы некоторый субъект приступил к выполнению соответствующего действия.При этом, даже если класс команд принадлежит множеству атомарных классов команд не запрещается вводить более частные классы команд, в состав которых входит информация, дополнительно специфицирующая соответствующее действие.Если соответствующий данному классу команд класс действий является более частным по отношению к действиям в sc-памяти, то попадание данного класса команд во множество атомарных классов команд говорит о наличии в текущей версии системы как минимум одного sc-агента, условие инициирования которого соответствует формулировке команд данного класса.
}

\scnheader{неатомарный класс команд}
\scnexplanation{Принадлежность некоторого класса команд множеству неатомарных классов команд фиксирует факт того, что данная спецификация не является достаточной для того, чтобы некоторый субъект приступил к выполнению соответствующего действия, и требует дополнительных уточнений.
}

\scnheader{класс команд с фиксированным количеством аргументов}
\scnexplanation{Класс команд, фиксирующий факт того, что спецификация выполняемого действия должна содержать конкретное количество входных аргументов конкретного типа.
}

\scnheader{класс команд с произвольным количеством аргументов}
\scnexplanation{Класс команд, фиксирующий факт того, что спецификация выполняемого действия может содержать неограниченное количество входных аргументов, заданных в любом порядке.
}

\scnheader{класс команд без аргументов}
\scnexplanation{Класс команд, фиксирующий факт того, что спецификация выполняемого действия не должна содержать входных аргументов.
}

\scnheader{класс команд с одним аргументом}
\scnexplanation{Класс команд, фиксирующий факт того, что спецификация выполняемого действия должна содержать один входной аргумент конкретного типа.
}

\scnheader{класс команд с двумя аргументами}
\scnexplanation{Класс команд, фиксирующий факт того, что спецификация выполняемого действия должна содержать два входных аргумента конкретного типа. Если порядок задания аргументов для действия имеет значение, то этот факт должен быть указан в спецификации этого действия.
}

\scnheader{класс команд с тремя и более аргументами}
\scnexplanation{Класс команд, фиксирующий факт того, что спецификация выполняемого действия должна содержать три и более входных аргумента конкретного типа. Если порядок задания аргументов для действия имеет значение, то этот факт должен быть указан в спецификации этого действия.
}

\scnheader{класс команд с произвольным количеством аргументов}
\scnexplanation{Класс команд, фиксирующий факт того, что спецификация выполняемого действия может содержать неограниченное количество входных аргументов, заданных в любом порядке.
}

\scnendstruct

\end{SCn}

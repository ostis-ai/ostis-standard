\begin{SCn}

\scnsectionheader{Предметная область и онтология действий, задач, планов и методов}

\scnstartsubstruct

\scnrelfromvector{ключевые знаки}{действие;класс действий;метод;класс методов;деятельность;вид деятельности}

\bigskip
\scnfragmentcaptiontext{Понятие действия}

\scnheader{действие}
\scnidtf{целенаправленный процесс, выполняемый одним или несколькими субъектами (кибернетическими системами с возможным применением некоторых инструментов}
\scnidtf{воздействие}
\scnidtf{акция}
\scnidtf{акт}
\scnidtf{операция}
\scnidtf{\uline{процесс} воздействия некоторой (возможно, коллективной) сущности (субъекта воздействия) на некоторую одну или несколько сущностей (объектов воздействия -- исходных объектов (аргументов) или целевых (создаваемых или модифицируемых) объектов)}
\scnsubset{процесс}
\scnidtf{целенаправленный ("осознанный"{}) процесс выполняемый (управляемый, реализуемый) неким субъектом}
\scnidtf{акция реализации некоторого замысла}
\scnidtf{преднамеренная акция}

\bigskip
\scnfragmentcaptiontext{Типология действий}

\scnheader{действие}
\scnsuperset{элементарное действие}
	\scnaddlevel{1}
	\scnidtf{действие, выполнение которого не требует его декомпозиции на множество поддействий (частных действий, действий более низкого уровня)}
	\scnexplanation{Элементарное действие выполняется одним индивидуальным субъектом и является либо элементарным действием, выполняемым в памяти этого субъекта (элементарным действием его "процессора"{}), либо элементарным действием одного из его эффекторов.}
	\scnaddlevel{-1}
\scnsuperset{сложное действие}
	\scnaddlevel{1}
	\scnsuperset{легко выполнимое сложное действие}
	\scnsuperset{трудно выполнимое действие}
	\scnaddlevel{-1}
\scnsubdividing{индивидуальное действие\\
	\scnaddlevel{1}
	\scnidtf{действие, выполняемое индивидуальной кибернетической системой}
	\scnsuperset{индивидуальное действие, выполняемое человеком}
	\scnsuperset{индивидуальное действие, выполняемое компьютерной системой}
	\scnaddlevel{-1}
;коллективное действие\\
	\scnaddlevel{1}
	\scnidtf{действие, выполняемое коллективом кибернетических систем (коллективом субъектов)}
	\scnsuperset{действие, выполняемое коллективом людей}
	\scnsuperset{действие, выполняемое коллективом индивидуальных компьютерных систем}
	\scnsuperset{действие, выполняемое коллективом людей и индивидуальных компьютерных систем}
		\scnaddlevel{1}
		\scnsuperset{действие, выполняемое Экосистемой OSTIS}
		\scnaddhind{-1}
		\scnsuperset{действие, выполняемое одним человеком во взаимодействии с одной индивидуальной компьютерной системой}
		\scnaddlevel{-1}
	\scnaddlevel{-1}}
\scnsubdividing{действие, выполняемое кибернетической системой в собственной памяти
;действие, выполняемое кибернетической системой в своей внешней среде
;действие, выполняемое кибернетической системой над своей физической оболочкой}


\scnheader{действие, выполняемое кибернетической системой в собственной памяти}
\scnidtf{действие, выполняемое в памяти}
\scnidtf{действие кибернетической системы, направленное на обработку информации, хранимой в её памяти}
\scnsuperset{действие, выполняемое кибернетической системой в собственной памяти и направленное на организацию её деятельности во внешней среде}
	\scnaddlevel{1}
	\scnidtf{действие, выполняемое кибернетической системой в её памяти и направленное на организацию её деятельности во внешней среде и в конечном счете -- на сенсо-моторную координацию деятельности её эффекторов}
	\scnaddlevel{-1}

\scnheader{действие, выполняемое кибернетической системой в своей внешней среде}  
\scnidtf{действие, выполняемое кибернетической системой в её внешней среде и осуществляемое (на самом низком уровне) эффекторами этой кибернетической системы}
   
\scnheader{действие}
\scnsubset{процесс}
\scnrelfrom{разбиение}{Темпоральный признак классификации действий}
\scnaddlevel{1}
\scneqtoset{настоящее действие\\
    \scnaddlevel{1}
    \scnidtf{активное действие}
    \scnidtf{действие, выполняемое в текущий момент}
    \scnaddlevel{-1}
;прошлое действие\\
    \scnaddlevel{1}
    \scnidtf{выполненное, завершенное действие}
    \scnaddlevel{-1}
;инициированное действие\\
    \scnaddlevel{1}
    \scnidtf{действие, ожидающее начала своего выполнение}
    \scnaddlevel{-1}
;прерванное действие\\
    \scnaddlevel{1}
    \scnidtf{действие, ожидающее продолжения своего выполнения}
    \scnaddlevel{-1}
;планируемое действие\\
    \scnaddlevel{1}
    \scnidtf{будущее действие}
    \scnaddlevel{-1}
;возможное действие}
\scnaddlevel{-1}

\scnrelfrom{разбиение}{Признак классификаций действий по их длительности}
\scnaddlevel{1}
\scneqtoset{краткосрочное действие\\
    \scnaddlevel{1}
    \scnidtf{тактическое действие}
    \scnaddlevel{-1}
;долгосрочное действие\\
    \scnaddlevel{1}
    \scnidtf{стратегическое действие}
    \scnaddlevel{-1}
;перманентное действие\\
    \scnaddlevel{1}
    \scnidtf{действие, постоянно выполняемое соответствующим субъектом, пока этот субъект существует}
    \scnaddlevel{-1}}
\scnaddlevel{-1}

\scnsuperset{действие, у которого цель известна, но задана не совсем точно}
	\scnaddlevel{1}
	\scnsuperset{действие, направленное на выявление противоречий в базе знаний}
		\scnaddlevel{1}
		\scnnote{Это действие декомпозируется на несколько самостоятельных поддействий, каждое из которых выявляет (локализует) противоречия (ошибки) конкретного формализуемого вида, для которого в базе знаний существует точное определение.}
		\scnaddlevel{-1}
	\scnaddlevel{-1}
\scnsuperset{действие, для которого априори не известен метод, обеспечивающий его выполнение}
	\scnaddlevel{1}
	\scnnote{Соответствующий метод либо не найден, либо его вообще нет в памяти.}
	\scnaddlevel{-1}

   
\scnheader{сложное действие}
\scnidtf{неэлементарное действие}
\scnidtf{действие выполнение которого требует декомпозиции этого действия на множество его \uline{поддействий}, т.е. частных действий более низкого уровня}
\scnnote{Декомпозиция сложного действия на поддействия может иметь весьма сложный иерархический вид с большим числом уровней иерархии, т.е. поддействиями \textit{сложного действия} могут также \textit{сложные действия}. Уровень сложности действия можно определять (1) общим числом его поддействий и (2) числом уровней иерархии этих поддействий.}
\scnnote{Декомпозиция \textit{сложного действия} на поддействия в конечном счете должна завершаться элементарными действиями.}
\scnnote{Темпоральные соотношения между поддействиями сложного действия могут быть самые различные, но в простейшем случае сложное действие представляет собой строгую последователность действий более низкого уровня иерархии.}


\scnheader{легко выполнимое сложное действие} 
\scnidtf{сложное действие, для выполнения которого известен соответствующий \textbf{\textit{метод}} и соответствующие этому методу исходные данные, а также (для действий, выполняемых во внешней среде) имеются в наличии все необходимые исходные объекты (расходные материалы и комплектация), а также средства (инструменты)}

\scnheader{трудно выполнимое действие}
\scnidtf{сложное действие, для выполнения которого в текущий момент либо не известен соответствующий метод, либо возможные методы известны, но отсутствуют условия их применения}\end{SCn}

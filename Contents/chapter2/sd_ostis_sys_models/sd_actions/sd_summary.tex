\begin{SCn}

\bigskip
\scnfragmentcaptiontext{Резюме предметной области и онтологии действий, задач, планов и методов}

\scnheaderlocal{следует отличать*}
\scnhaselementset{
	\scnmakevectorlocal{действие;класс действий};
	\scnmakevectorlocal{метод;класс методов};
	\scnmakevectorlocal{деятельность;вид деятельности}
}
\scnaddlevel{1}
\scnsubset{семейство подклассов*}

\scnnote{Все сущности, принадлежащие рассмотренным \textit{понятиям}, требуют достаточно детальной \textit{спецификации}. При этом не следует путать сами сущности и их \textit{спецификации}. Так, например, не следует путать \textit{действие} и \textit{задачу}, которая специфицирует (уточняет) это \textit{действие}. Особое место среди указанных понятий занимает понятие \textit{метода}, т.к. каждый конкретный \textit{метод}, с одной стороны, является \textit{спецификацией} соответствующего \textit{класса действий}, а, с другой стороны, сам нуждается в \textit{спецификации}, которая уточняет \textit{операционную семантику} этого \textit{метода}, (т.е. множество \textit{методов}, обеспечивающих \textit{интерпретацию} данного специфицируемого \textit{метода}) и тем самым "преобразует"{} специфицируемый \textit{метод} в \textit{навык}.}
\scnaddlevel{-1}

\scnheader{следует отличать*}
\scnhaselementvector{первый домен*(спецификация*)\\
\scnaddlevel{1}
\scnidtf{специфицируемая сущность}
\scnidtf{сущность, использование которой требует вполне определенной ее спецификации}
\scnsuperset{действие}
\scnsuperset{класс действий}
\scnsuperset{метод}
\scnsuperset{класс методов}
\scnsuperset{деятельность}
\scnsuperset{вид деятельности}
\scnaddlevel{-1};
второй домен*(спецификация*)\\
\scnaddlevel{1}
\scnidtf{спецификация}
\scnsuperset{задача}
\scnaddlevel{1}
\scnsuperset{декларативная формулировка задачи}
\scnsuperset{процедурная формулировка задачи}
\scnaddlevel{-1}
\scnsuperset{план}
\scnsuperset{декларативная спецификация выполнения действий}
\scnsuperset{протокол}
\scnsuperset{результативная часть протокола}
\scnsuperset{обобщенная формулировка задач соответствующего класса}
\scnsuperset{метод}
\scnsuperset{операционная семантика метода}
\scnsuperset{модель решения задач}
\scnaddlevel{-1}
}

\scnnote{
При этом следует отличать:
\begin{scnitemize}
\item спецификацию конкретного \textit{действия} (задачу, план, декларативную спецификацию выполнения действия, протокол, результативную часть протокола);
\item спецификацию конкретной деятельности (контекст*, множество используемых методов*);
\item спецификацию класса действий (обобщенную формулировку задачи, метод);
\item спецификацию вида деятельности (технологию);
\item спецификацию метода (операционную семантику метода);
\item спецификацию класса методов (модель решения задач).
\end{scnitemize}
}
	
\end{SCn}

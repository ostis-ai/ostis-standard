\begin{SCn}
	
\bigskip
\scnfragmentcaptiontext{Спецификация метода и понятие навыка}

\scnheader{метод}
\scnnote{Каждый конкретный метод рассматривается нами не только как важный вид спецификации соответствующего класса задач, но также и как объект, который и сам нуждается в спецификации, обеспечивающей непосредственное применение этого метода. Другими словами, метод является не только спецификацией (спецификацией соответствующего класса задач), но и \uline{объектом} спецификации.}

\scnheader{спецификация*}
\scnsuperset{\textbf{операционная семантика метода*}}
	\scnaddlevel{1}
		\scnidtf{спецификация метода*}
		\scneq{сужение отношения по первому домену*(спецификация*; метод)*}
		\scnidtf{семейство методов, обеспечивающих интерпретацию заданного метода*}
		\scnidtf{формальное описание интерпретатора заданного метода*}
		\scnrelfrom{второй домен}{\textbf{операционная семантика метода}}
		\scnaddlevel{1}
			\scnsuperset{\textbf{полное представление операционной семантики метода}}
			\scnaddlevel{1}
				\scnidtf{представление \textit{операционной семантики метода}, доведенное (детализированное) до уровня всех \textit{спецификаций элементарных действий}, выполняемых в процессе интерпретации соответствующего \textit{метода}}
			\scnaddlevel{-1}
		\scnaddlevel{-1}
	\scnaddlevel{-1}
\scnheader{навык}
\scnidtf{умение}
\scnidtf{объединение \textit{метода} с его исчерпывающей спецификацией -- \textit{полным представлением операционной семантики метода}}
\scnidtf{метод, интерпретация (выполнение, использование) которого полностью может быть осуществлено данной кибернетической системой, в памяти которой указанный метод хранится}
\scnidtf{метод, который данная кибернетическая система умеет (может) применять}
\scnidtf{метод + метод его интерпретации}
\scnidtf{умение решать соответствующий класс эквивалентных задач}
\scnidtf{метод плюс его операционная семантика, описывающая то, как интерпретируется (выполняется, реализуется) этот метод, и являющаяся одновременно операционной семантикой соответствующей модели решения задач}

\bigskip
\scnfragmentcaptiontext{Понятие класса методов и понятие модели решения задач}

\scnheader{класс методов}
\scnrelto{семейство подклассов}{метод}
\scnidtf{множество методов, для которых можно \uline{унифицировать} представление (спецификацию) этих методов}
\scnidtf{множество всевозможных методов решения задач, имеющих общий язык представления этих методов}
\scnidtf{множество всевозможных методов, представленных на данном языке}
\scnidtf{множество методов, для которых задан язык представления этих методов}

\scnhaselement{процедурный метод решения задач}
	\scnaddlevel{1}
		\scnsuperset{алгоритмический метод решения задач}
	\scnaddlevel{-1}
\scnhaselement{логический метод решения задач}
	\scnaddlevel{1}
		\scnsuperset{продукционный метод решения задач}
		\scnsuperset{функциональный метод решения задач}
	\scnaddlevel{-1}
\scnhaselement{искусственная нейронная сеть}
	\scnaddlevel{1}
		\scnidtf{класс методов решения задач на основе искусственных нейронных сетей}
	\scnaddlevel{-1}
\scnhaselement{генетический "алгоритм"{}}
\scnidtf{множество методов основанных на общей онтологии}
\scnidtf{множество методов, представленных на одинаковом языке}
\scnidtf{язык методов}
	\scnaddlevel{1}
		\scnnote{Таких специализированных языков может быть выделено целое множество, каждому из которых будет соответствовать своя модель решения задач (т.е. свой интерпретатор)}
	\scnaddlevel{-1}

\scnidtf{язык (например sc-язык) представлений методов соответствующего класса методов}
\scnidtf{множество методов решений задач, которому соответствует специальный язык (например sc-язык), обеспечивающий представление методов из этого множества}
\scnidtf{множество методов, которому ставится в соответствие отдельная модель решения задач}

\scnheader{язык представления методов} 
\scnidtf{язык представления методов, соответствующих определенному классу методов}
\scnsubset{язык}
\scnidtf{язык программирования}
\scnsuperset{язык представления методов обработки информации}
	\scnaddlevel{1}
		\scnidtf{язык программирования внутренних действий кибернетической системы, выполняемых в их памяти}
		\scnidtf{язык представления методов решения задач в памяти кибернетических систем}
	\scnaddlevel{-1}
\scnsuperset{язык представления методов решения задач во внешней среде кибернетических систем}
	\scnaddlevel{1}
		\scnidtf{язык программирования внешних действий кибернетических систем}
	\scnaddlevel{-1}

\scnheader{модель решения задач}
\scnidtf{метаметод интерпретации соответствующего класса методов}
\scnsubset{метод}
\scnidtf{метаметод}
\scnidtf{абстрактная машина интерпретации соответствующего класса методов}
\scnidtf{иерархическая система "микропрограмм"{}, обеспечивающих интерпретацию соответствующего класса методов}
\scnsuperset{алгоритмическая модель решения задач}
\scnsuperset{процедурная параллельная синхронная модель решения задач}
\scnsuperset{процедурная параллельная асинхронная модель решения задач}
\scnsuperset{продукционная модель решения задач}
\scnsuperset{функциональная модель решения задач}
\scnsuperset{логическая модель решения задач}
\scnaddlevel{1}
	\scnsuperset{четкая логическая модель решения задач}
	\scnsuperset{нечеткая логическая модель решения задач}
\scnaddlevel{-1}
\scnsuperset{"нейросетевая"{} модель решения задач}
\scnsuperset{"генетическая"{} модель решения задач}
\scnnote{Для интерпретации \uline{всех} моделей решения задач может быть использован агентно-ориентированный подход}
\scnexplanation{Каждая \textit{модель решения задач} задается:
\begin{scnitemize}
	\item соответствующим классом методов решения задач, т.е. языком представления методов этого класса;
	\item предметной областью этого класса методов; 
	\item онтологией этого класса методов (т.е. денотационной семантикой языка представления этих методов);
	\item операционной семантикой указанного класса методов.
\end{scnitemize}
}

\scnheader{спецификация*}
\scnsuperset{\textbf{модель решения задач}*}
	\scnaddlevel{1}
		\scneq{сужение отношения по первому домену(спецификация*; класс методов)*}
		\scnidtf{спецификация \textit{класса методов}*}
		\scnidtf{спецификация \textit{языка представления методов}*}
		\scnsubdividing{\textbf{синтаксис языка представления методов соответствующего класса}*;
\textbf{денотационная семантика языка представления методов соответствующего класса}*;
\textbf{операционная семантика языка представления методов соответствующего класса}*}
		\scnnote{Каждому конкретному \textit{классу методов} взаимно однозначно соответствует \textit{язык представления методов}, принадлежащих этому (специфицируемому) \textit{классу методов}. Таким образом, спецификация каждого \textit{класса методов} сводится к спецификации соответствующего \textit{языка представления методов}, т.е. к описанию его синтаксической, денотационной семантики и операционной семантики.
Примерами \textit{языков представления методов} являются все \textit{языки программирования}, которые в основном относятся к подклассу \textit{языков представления методов} -- к \textit{языкам представления методов обработки информации}. Но сейчас все большую актуальность приобретает необходимость создания эффективных формальных языков представления методов выполнения действий во внешней среде кибернетических систем. Без этого комплексная автоматизация, в частности, в промышленной сфере невозможна.}
	\scnaddlevel{-1}
	
\scnheader{денотационная семантика языка представления методов соответствующего класса}
\scnrelto{второй домен}{денотационная семантика языка представления методов соответствующего класса*}
\scnidtf{онтология соответствующего класса методов}
\scnidtf{денотационная семантика соответствующего класса методов}
\scnidtf{денотационная семантика языка (sc-языка), обеспечивающего представление методов соответствующего класса}
\scnidtf{денотационная семантика соответствующей модели решения задач}
\scnnote{речь идет о языке, обеспечивающем внутреннее представление методов соответствующего класса в ostis-системе, то синтаксис этого языка совпадает с синтаксисом sc-кода}
\scnsubset{онтология}

\scnheader{операционная семантика языка представления методов соответствующего класса}
\scnrelto{второй домен}{операционная семантика языка представления методов соответствующего класса*}
\scnidtf{метаметод интерпретации соответствующего класса методов}
\scnidtf{семейство агентов, обеспечивающих интерпретацию (использования) любого метода, принадлежащего соответствующему классу методов}
\scnidtf{операционная семантика соответствующей модели решения задач}

\scnheader{язык представления обобщенных формулировок задач для различных классов задач}
\scnnote{Поскольку каждому \textit{методу} соответствует \textit{обобщенная формулировка задач}, решаемых с помощью этого \textit{метода}, то каждому \textit{классу методов} должен соответствовать не только определенный \textit{язык представления методов}, принадлежащих указанному \textit{классу методов}, но и определенный \textit{язык представления обобщенных формулировок задач для различных классов задач}, решаемых с помощью \textit{методов}, принадлежащих указанному \textit{классу методов}.}
\end{SCn}

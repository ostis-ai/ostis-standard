\begin{SCn}

\scnheader{задача}
\scnidtf{спецификация действия, которое выполнилось, выполняется или может быть выполнено соответствующей кибернетической системой}
\scnnote{Каждой задаче и, соответственно, каждому специфицируемому действию соответствует определенная кибернетическая система, являющаяся субъектом, выполняющим это действие.}
\scnsubset{знание}
\scnnote{Каждая \textit{задача} - это \textit{знание}, описывающее то какое действие возможно потребуется выполнить.}
\scnsuperset{инициированная задача}
\scnaddlevel{1}
\scnidtf{формулировка задачи, которая подлежит выполнению}
\scnaddlevel{-1}
\scnidtf{спецификация (описание) соответствующего действия}
\scnidtf{задачная ситуация}
\scnidtf{формулировка задачи}
\scnidtf{постановка задачи}

\scnsuperset{декларативная формулировка задачи}
\scnaddlevel{1}
\scnidtf{задача, в формулировке которой явно указывается (описывается) целевая ситуация, т.е. то, что является результатом выполнения (решения) данной задачи}
\scnidtf{декларативаная формулировка задачи}
\scnaddlevel{-1}
\scnsuperset{процедурная формулировка задачи}
\scnaddlevel{1}
\scnidtf{процедурная формулировка задачи}
\scnidtf{задача, в формулировке которой явно указывается характеристика действия, специфицируемого этой задачей, а именно, например, указывается:
\begin{scnitemize}
\item субъект или субъекты, выполняющие это действие,
\item объекты, над которыми действие выполняется, - аргументы действия,
\item инструменты, с помощью которых выполняется действие,
\item момент и, возможно, дополнительные условия начала и завершения выполнения действия
\end{scnitemize}}
\scnaddlevel{-1}
\scnsuperset{декларативно-процедурная формулировка задачи}
\scnaddlevel{1}
\scnidtf{задача, в формулировке которой присутствуют как декларативные (целевые), так и процедурные аспекты}
\scnaddlevel{-1}
\scnnote{От качества (корректности и полноты) формулировки задачи, т.е. спецификации соответствующего действия, во многом зависит качество (эффективность) выполнения этого действия, т.е. качество процесса решения указанной задачи.}
\scnsuperset{проблема}
\scnaddlevel{1}
\scnidtf{проблемная задача}
\scnidtf{сложная, трудно решаемая задача}
\scnsuperset{изобретательская задача}
\scnaddlevel{-1}

\scnheader{декларативная формулировка задачи}
\scnrelto{второй домен}{декларативная формулировка задачи*}
\scnidtf{описание исходной (начальной) ситуации, являющейся условием выполнения соответствующего действия и целевой (конечной) ситуации, являющейся результатом выполнения этого действия}
\scnidtf{семантическая спецификация действия}
\scnnote{Формулировка \textit{задачи} может не содержать указания контекста (области решения) \textit{задачи} (в этом случае областью решения \textit{задачи} считается либо вся \textit{база знаний}, либо ее согласованная часть), а также может не содержать либо описания исходной ситуации, либо описания целевой ситуации. Так, например, описания целевой ситуации для явно специфицированного противоречия, обнаруженного в \textit{базе знаний} не требуется.}
\scnidtf{формулировка (описание) задачной ситуации с явным или неявным описанием контекста (условий) выполнения специфицируемого действия, а также результата выполнения этого действия}
\scnidtf{явное или неявное описание 
\begin{scnitemize}
\item того, что \uline{дано} - исходные данные, условия выполнения специфируемого действия,
\item того, что \uline{требуется} - формулировка цели, результата выполнения указанного действия
\end{scnitemize}}

\scnhaselementrole{пример}{\scnfilescg{figures/sd_task/declarative_task_statement.png}}

\scnnote{Выполнение данного действия сведется к следующим \uline{событиям}:
\begin{scnitemize}
\item для числа \textit{с} будет сгенерирован уникальный идентификатор, являющийся его представлением в соответствующей системе счисления
\item будет сгенерирована константная настоящая позитивная пара принадлежности, соединяющая узел "\textit{вычислено}"{} с узлом "\textit{с}"{}
\item удалится константная будущая позитивная пара принадлежности, а также константная настоящая нечеткая пара принадлежности, выходящие из узла "\textit{вычислено}".
\end{scnitemize}
Таким образом, после выполнения действия \uline{все} \uline{будущие} сущности, входящие в целевую ситуацию, становятся \uline{настоящими} сущностями, а некоторые \uline{настоящие} сущности, входящие в исходную ситуацию, становятся \uline{прошлыми}.}

\scnheader{задача}
\scnsuperset{задача, решаемая в памяти кибернетической системы}
\scnaddlevel{1}
\scnsuperset{задача, решаемая в памяти индивидуальной кибернетической системы}
\scnsuperset{задача, решаемая в общей памяти многоагентной системы}
\scnidtf{информационная задача}
\scnidtf{задача, направленная либо на \uline{генерацию} или поиск информации, удовлетворяющей заданным требованиям, либо на некоторое \uline{преобразование} заданной информации}
\scnsuperset{математическая  задача}
\scnaddlevel{-1}
\scnsuperset{элементарная информационная задача}
\scnsuperset{простая информационная задача}
\scnsuperset{проблемная информационная задача}
\scnaddlevel{1}
\scnidtf{интеллектуальная информационная задача}
\scnsuperset{проблема Гильберта}
\scnaddlevel{-1}

\scnheader{вопрос}
\scnidtf{запрос}
\scnsubset{задача, решаемая в памяти кибернетической системы}
\scnidtf{непроцедурная формулировка задачи на поиск (в текущем состоянии базы знаний) или на генерацию знания, удовлетворяющего заданным требованиям}
\scnsuperset{вопрос - что это такое}
\scnsuperset{вопрос - почему}
\scnsuperset{вопрос - зачем}
\scnsuperset{вопрос - как}
\scnaddlevel{1}
\scnidtf{каким способом}
\scnidtf{запрос метода (способа) решения заданного (указываемого) вида задач или класса задач либо, плана решения конкретной указываемой задачи}
\scnaddlevel{-1}

\scnheader{спецификация*}
\scnsuperset{сужение отношения по первому домену*(спецификация*; действие)*}
\scnaddlevel{1}
\scnidtftext{часто используемый sc-идентификатор}{спецификация действия*}
\scnsubdividing{
задача*\\
\scnaddlevel{1}
\scnsubdividing{
декларативная формулировка задачи*\\
\scnaddlevel{1}
\scnrelfrom{второй домен}{декларативная формулировка задачи}
\scnaddlevel{-1}
;процедурная формулировка задачи*\\
\scnaddlevel{1}
\scnrelfrom{второй домен}{процедурная формулировка задачи}
\scnaddlevel{-1}}
\scnaddlevel{-1}
;исходная ситуация*
;цель*
;план*
;декларативная спецификация выполнения действия*
;контекст действия*
\scnaddlevel{1}
\scnidtf{информационный ресурс необходимый для выполнения заданного действия*}
\scnaddlevel{-1}
;множество используемых методов*
\scnaddlevel{1}
\scnidtf{множество методов, используемых для выполнения заданного действия*}
\scnidtf{операционный (функциональный) ресурс, необходимый для выполнения заданного действия*}
\scnaddlevel{-1}
;протокол*
;результативная часть протокола*}
\scnaddlevel{-1}

\scnnote{Таким образом, каждому действию может быть поставлен в соответствие целый ряд видов спецификации этого действия, которые описывают различные аспекты специфицируемого действия - и то, что является причиной (условием) инициирования этого действия, и то, что является результатом ("сухим остатком") его выполнения, и то, и то, с помощью таких ресурсов оно может быть выполнено, и то, как управлять этими ресурсами в процессе выполнения действия, и то, как на самом деле это действие было выполнено.}

\scnheader{трансформация отношения путем обобщения компонентов его связок*(спецификация*)}
\scnhaselementvector{действие;задача}
\scnhaselementvector{действие;ситуация}
\scnhaselementvector{действие;декларативная формулировка задачи}
\scnhaselementvector{действие;процедурная формулировка задачи}
\scnhaselementvector{действие;план}
\scnhaselementvector{действие;декларативная спецификация выполнения действия}
\scnhaselementvector{действие;протокол}
\scnhaselementvector{действие;результативная часть протокола}

\scnheader{сужение отношения по первому домену*(спецификация*; действие)*}
\scnnote{\textit{спецификацию действия} (базовое описание действия) условно можно разбить на следующие части:
\begin{scnitemize}
\item описание состояния действия в текущий момент времени -- действие может принадлежать:
\begin{scnitemizeii}
\item либо классу \textit{прогнозируемых сущностей} (в случае действий -- это планируемые действия, которые могут быть, но не обязательно выполняться в будущем);
\item либо классу \textit{настоящих сущностей}, т.е. сущностей, существующих в настоящий (текущий) момент времени;
\item либо классу \textit{прошлых сущностей}, завершивших свое существование (в случае действий - это действия, выполнение которых уже завершено);
\end{scnitemizeii}
\item формулировки \textit{задачи}, которая должна быть решена в результате выполнения специфицируемого действия. Такая формулировка представляет собой логико-семантическое описание \textit{задачной продукции}, включающей в себя:
\begin{scnitemizeii}
\item описание \textit{исходной ситуации} и/или события (исходных условий того, что должно быть дано, исходных данных, исходного контекста). Для \textit{действий во внешней среде} (действий/задач, выполняемых во внешней среде) в описании \textit{исходной ситуации} должно быть включено описание необходимых для решения задачи материальных ресурсов (сырья, комплектации) с указанием их количества;
\item описание \textit{целевой ситуации} и/или события (того, что требуется получить в результате решения данной задачи);
\item указание дополнительных \textit{инструментальных средств}, используемых для выполнения специфицируемого действия (такие средства могут быть использованы только при выполнении \textit{действий во внешней среде}).
\end{scnitemizeii}
\item указание субъектов-исполнителей специфицируемого действия:
\begin{scnitemizeii}
\item множество тех, кто может выполнить это действие;
\item тот, кто должен (которому поручено выполнить это действие);
\end{scnitemizeii}
\item указание метода, на основании (путем интерпретации) которого специфицируемое действие может быть выполнено - таких методов в общем случае может быть несколько;
\item спецификация выполненного действия, т.е. действия, отнесенного к классу \textit{прошлых сущностей}:
\begin{scnitemizeii}
\item указание отрезка времени выполнения действия (момента начала и момента завершения);
\item указание числа прерываний (ожиданий) процесса выполнения действия;
\item указание "чистой"{} длительности процесса выполнения действия;
\item указание успешности выполнения процесса (в случае неуспешности - указание "штатных"{} причин и сбоев).
\end{scnitemizeii}
\end{scnitemize}}

\scnheader{следует отличать*}
\scnhaselementset{действие\\
	\scnaddlevel{1}
\scnnote{Каждому действию становится в соответствие кибернетическая система, являющаяся субъектом этого действия. Указанный субъект может быть либо индивидуальной, либо коллективной кибернетической системой.}
	\scnaddlevel{-1}
;воздействие\\
	\scnaddlevel{1}
	\scnsuperset{действие}
	\scnsubset{процесс}
	\scnnote{Сущностью, осуществляющей воздействие на какой-либо объект, может быть не только кибернетическая система, но также, например, и пассивный инструмент, управляемый некоторой кибернетической системой.}
	\scnaddlevel{-1}
}

\end{SCn}
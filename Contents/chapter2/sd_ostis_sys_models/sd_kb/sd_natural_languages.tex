\begin{SCn}

\scnsectionheader{\currentname}

\scnstartsubstruct

\scnrelfromlist{соавтор}{Гордей А.Н.;Никифоров С.А.;Бобёр Е.С.;Святощик М.И.}

\scnheader{Предметная область естественных языков}
\scniselement{предметная область}
\scnsdmainclasssingle{язык}
\scnsdclass{плановый язык;язык общения;лексема;номинативная единица;комбинаторный вариант лексемы;естественный язык;тайген;ёген}
\scnsdrelation{морфологическая парадигма*;член предложения\scnrolesign }

\scnheader{язык}
\scnsubdividing{естественный язык\\
	\scnaddlevel{1}
	\scnexplanation{Естественный язык представляет собой язык, который не был создан целенаправленно}
	\scnaddlevel{-1}
;искусственный язык\\
	\scnaddlevel{1}
	\scnexplanation{Искусственный язык представляет собой язык, специально разработанный для достижения определённых целей}
	\scnhaselement{Эсперанто}
	\scnhaselement{Python}
	\scnsuperset{сконструированный язык}
	\scnaddlevel{1}
	\scnexplanation{Сконструированный язык представляет собой искусственный язык, предназначенный для общения людей}
	\scnhaselement{Эсперанто}
	\scnaddlevel{-1}
	\scnaddlevel{-1}
}
\scnsuperset{международный язык}
	\scnaddlevel{1}
	\scnexplanation{Международный язык представляет собой естественный или искусственный язык, использующийся для общения людей разных из стран}
	\scnhaselement{Английский язык}
	\scnhaselement{Русский язык}
	\scnaddlevel{-1}

\scnheader{плановый язык}
\scnreltoset{пересечение}{сконструированный язык;международный язык}

\scnheader{язык общения}
\scnreltoset{объединение}{естественный язык;сконструированный язык}
\scnhaselement{Английский язык}
\scnhaselement{Русский язык}
\scnhaselement{Эсперанто}
\scnreltoset{объединение}{корневой язык\\
	\scnaddlevel{1}
	\scnexplanation{Корневой язык представляет собой язык, для которого характерно полное отсутствие словоизменения и наличие грамматической значимости порядка слов, состоящих только из корня.}
	\scnhaselement{Английский язык}
	\scnaddlevel{-1}
;агглютинативный язык\\
	\scnaddlevel{1}
	\scnexplanation{Агглютинативный язык характеризуется развитой системой употребления суффиксов, приставок, добавляемых к неизменяемой основе слова, которые используются для выражения категорий числа, падежа, рода и др.}
	\scnhaselement{Английский язык}
	\scnaddlevel{-1}
;флективный язык\\
	\scnaddlevel{1}
	\scnexplanation{Для флективного языка характерно развитое употребление окончаний для выражения категорий рода, числа, падежа, сложная система склонения глаголов, чередование гласных в корне, а также строгое различение частей речи.}
	\scnhaselement{Русский язык}
	\scnaddlevel{-1}
;профлективный язык\\
	\scnaddlevel{1}
	\scnexplanation{Для профлективного языка характерны агглютинация (в случае именного словоизменения), флексия и чередование гласных (аблаут)(в случае глагольного словоизменения).}
	\scnaddlevel{-1}
}

\scnheader{лексема}
\scnsubset{файл}
\scnexplanation{\textit{Лексема} -- тайген или ёген конкретного естественного языка.}
	\scnaddlevel{1}
	\scnrelfrom{источник}{\scncite{Hardzei2005}}
	\scnaddlevel{-1}
	
\scnheader{номинативная единица}
\scnsubset{файл}
\scnexplanation{\textit{Номинативная единица} -- устойчивая последовательность комбинáторных вариантов лексем, в которой один вариант лексемы (модификатор) определяет другой (актуализатор), например: ‘записная книжка’, ‘бежать галопом’.}
	\scnaddlevel{1}
	\scnrelfrom{источник}{\scncite{Hardzei2005}}
	\scnaddlevel{-1}
	
\scnheader{комбинаторный вариант лексемы}
\scnsubset{файл}
\scnexplanation{\textit{Комбинáторный вариант лексемы } -- вариант лексемы в упорядоченном наборе её вариантов (парадигме).}
	\scnaddlevel{1}
	\scnrelfrom{источник}{\scncite{Hardzei2007}}
	\scnaddlevel{-1}


\scnheader{морфологическая парадигма*}
\scniselement{квазибинарное отношение}
\scnexplanation{\textit{Морфологическая парадигма*} -- квазибинарное отношение, связывающее лексему с её комбинáторными вариантами.}
\scnrelfrom{первый домен}{словоформа}
\scnrelfrom{второй домен}{лексема}

\scnheader{естественный язык}
\scnsubdividing{
	часть языка\\
	\scnaddlevel{1}
	\scnsubdividing{тайген;ёген}
	\scnaddlevel{-1}
	;знак алфавита синтаксиса\\
		\scnaddlevel{1}
		\scnexplanation{\textit{Знаки алфавита синтаксиса} -- вспомогательные средства синтаксиса (на макроуровне -- предлоги, послелоги, союзы, частицы и др., на микроуровне -- флексии, префиксы, постфиксы, инфиксы и др.), служащие для соединения составных частей языковых структур и образования морфологических парадигм.}
			\scnaddlevel{1}
			\scnrelfrom{источник}{\scncite{Hardzei2005}}
			\scnaddlevel{-1}
		\scnaddlevel{-1}
}

\scnheader{тайген}
\scnexplanation{\textit{Тайген} -- часть языка, обозначающая индивида.}
	\scnaddlevel{1}
	\scnrelfrom{источник}{\scncite{Hardzei2006}}
	\scnrelfrom{источник}{\scncite{Hardzei2015}}
	\scnaddlevel{-1}
\scnsubdividing{
	развёрнутый тайген\\
	\scnaddlevel{1}
	\scnsubdividing{
		составной тайген\\
		;сложный тайген\\
	}   
	\scnaddlevel{-1}
	;свёрнутый тайген\\
	\scnaddlevel{1}
	\scnsubdividing{
		сокращённый тайген\\
		;сжатый тайген\\
			\scnaddlevel{1}
			\scnsubdividing{
				информационный тайген\\
				\scnaddlevel{1}
				\scnexplanation{\textit{Информационный тайген} -- тайген, обозначающий индивида в информационном фрагменте модели мира.}
					\scnaddlevel{1}
					\scnrelfrom{источник}{\scncite{Hardzei2006}}
					\scnrelfrom{источник}{\scncite{Hardzei2015}}
					\scnaddlevel{-1}
				\scnaddlevel{-1}
				;физический тайген\\
				\scnaddlevel{1}
				\scnexplanation{\textit{Физический тайген} -- тайген, обозначающий индивида в физическом фрагменте модели мира. }
					\scnaddlevel{1}
					\scnrelfrom{источник}{\scncite{Hardzei2006}}
					\scnrelfrom{источник}{\scncite{Hardzei2015}}
					\scnaddlevel{-1}
				\scnsubdividing{
					постоянный тайген\\
					\scnaddlevel{1}
					\scnexplanation{\textit{Постоянный тайген} -- физический тайген, обозначающий постоянного индивида.}
						\scnaddlevel{1}
						\scnrelfrom{источник}{\scncite{Hardzei2006}}
						\scnrelfrom{источник}{\scncite{Hardzei2015}}
						\scnaddlevel{-1}
					\scnaddlevel{-1}
					;переменный тайген\\
					\scnaddlevel{1}
					\scnexplanation{\textit{Переменный тайген} -- физический тайген, обозначающий переменного индивида.}
						\scnaddlevel{1}
						\scnrelfrom{источник}{\scncite{Hardzei2006}}
						\scnrelfrom{источник}{\scncite{Hardzei2015}}
						\scnaddlevel{-1}
					\scnaddlevel{-1}
				}
				\scnsubdividing{
					качественный тайген
					;количественный тайген\\
				}
				\scnsubdividing{
					одноместный тайген
					;многоместный тайген\\
					\scnaddlevel{1}
					\scnsuperset{интенсивный тайген}
					\scnsuperset{экстенсивный тайген}
					\scnaddlevel{-1}	
				}
			}
			\scnaddlevel{-1}
	}
	\scnaddlevel{-1}
}
\scnaddlevel{-1}	

\scnheader{ёген}
\scnexplanation{\textit{Ёген} -- часть языка, обозначающая признак индивида.}
	\scnaddlevel{1}
	\scnrelfrom{источник}{\scncite{Hardzei2006}}
	\scnrelfrom{источник}{\scncite{Hardzei2015}}
	\scnaddlevel{-1}
\scnsubdividing{
	развёрнутый ёген\\
	\scnaddlevel{1}
	\scnsubdividing{
		составной ёген
		;сложный ёген
	}
	\scnaddlevel{-1}
	;свёрнутый ёген\\
	\scnaddlevel{1}
	\scnsubdividing{
		сокращённый ёген\\
			\scnaddlevel{1}
			\scnsubdividing{
				информационный ёген\\
					\scnaddlevel{1}
					\scnexplanation{\textit{Информационный ёген} -- еген, обозначающий признак индивида в информационном фрагменте модели мира.}
						\scnaddlevel{1}
						\scnrelfrom{источник}{\scncite{Hardzei2006}}
						\scnrelfrom{источник}{\scncite{Hardzei2007a}}
						\scnaddlevel{-1}
					\scnaddlevel{-1}
				;физический ёген\\
				\scnaddlevel{1}
				\scnexplanation{\textit{Физический ёген} -- еген, обозначающий признак индивида в физическом фрагменте модели мира.}
					\scnaddlevel{1}
					\scnrelfrom{источник}{\scncite{Hardzei2006}}
					\scnrelfrom{источник}{\scncite{Hardzei2007a}}
					\scnaddlevel{-1}
				\scnsubdividing{
					постоянный ёген\\
						\scnaddlevel{1}
						\scnexplanation{\textit{Постоянный ёген} - физический ёген, обозначающий постоянный признак индивида.}
							\scnaddlevel{1}
							\scnrelfrom{источник}{\scncite{Hardzei2006}}
							\scnrelfrom{источник}{\scncite{Hardzei2007a}}
							\scnaddlevel{-1}
						\scnaddlevel{-1}
					;переменный ёген\\
						\scnaddlevel{1}
						\scnexplanation{\textit{Переменный ёген} -- физический ёген, обозначающий переменный признак индивида.}
							\scnaddlevel{1}
							\scnrelfrom{источник}{\scncite{Hardzei2006}}
							\scnrelfrom{источник}{\scncite{Hardzei2007a}}
							\scnaddlevel{-1}
						\scnaddlevel{-1}
				}
				\scnsubdividing{
					качественный ёген
					;количественный ёген
				}
				\scnsubdividing{
					одноместный ёген
					;многоместный ёген\\
						\scnaddlevel{1}
						\scnsubdividing{
							интенсивный ёген
							;экстенсивный ёген
						}
						\scnaddlevel{-1}
				}
				\scnaddlevel{-1}
			}
			\scnaddlevel{-1}
		;сжатый ёген
	}
	\scnaddlevel{-1}
}

\scnheader{член предложения\scnrolesign}
\scniselement{ролевое отношение}
\scnexplanation{\textit{Член предложения\scnrolesign} -- это отношение, связывающее декомпозицию текста с файлом, содержимое которого (часть языка) играет в декомпозируемом тексте определенную синтаксическую роль.}
	\scnaddlevel{1}
	\scnrelfrom{источник}{\scncite{Hardzei2005}}
	\scnaddlevel{-1}
\scnsubdividing{
	главный член предложения\scnrolesign\\
	\scnaddlevel{1}
	\scnsubdividing{
		подлежащее\scnrolesign\\
			\scnaddlevel{1}
			\scnexplanation{\textit{Подлежащее\scnrolesign} -- это одно из главных ролевых отношений, связывающее декомпозицию текста с файлом, содержимое которого обозначает исходный пункт описания события, выбранный наблюдателем. }
				\scnaddlevel{1}
				\scnrelfrom{источник}{\scncite{Hardzei2020}}
				\scnaddlevel{-1}
			\scnaddlevel{-1}
		;сказуемое\scnrolesign\\
			\scnaddlevel{1}
			\scnexplanation{\textit{Сказуемое\scnrolesign} -- это одно из главных ролевых отношений, связывающее декомпозицию текста с файлом, содержимое которого обозначает отображение наблюдателем исходного пункта описания события в конечный.}
				\scnaddlevel{1}
				\scnrelfrom{источник}{\scncite{Hardzei2020}}
				\scnaddlevel{-1}
			\scnaddlevel{-1}
		;прямое дополнение\scnrolesign\\
			\scnaddlevel{1}
			\scnexplanation{\textit{Прямое дополнение\scnrolesign} -- это одно из главных ролевых отношений, связывающее декомпозицию текста с файлом, содержимое которого обозначает конечный пункт описания события, выбранный наблюдателем.}
				\scnaddlevel{1}
				\scnrelfrom{источник}{\scncite{Hardzei2020}}
				\scnaddlevel{-1}
			\scnaddlevel{-1}
	}
	\scnaddlevel{-1}
	;второстепенный член предложения\scnrolesign\\
	\scnaddlevel{1}
	\scnsubdividing{
		косвенное дополнение\scnrolesign
		;определение\scnrolesign\\
			\scnaddlevel{1}
			\scnexplanation{\textit{Определение\scnrolesign} -- это одно из второстепенных ролевых отношений, связывающее декомпозицию текста с файлом, содержимое которого обозначает модификацию подлежащего, дополнения, обстоятельства места и времени.}
				\scnaddlevel{1}
				\scnrelfrom{источник}{\scncite{Hardzei2007a}}
				\scnrelfrom{источник}{\scncite{Hardzei2017a}}
				\scnrelfrom{источник}{\scncite{Hardzei2007b}}
				\scnaddlevel{-1}
			\scnaddlevel{-1}
		;обстоятельство\scnrolesign\\
			\scnaddlevel{1}
			\scnexplanation{\textit{Обстоятельство\scnrolesign} -- это одно из второстепенных ролевых отношений, связывающее декомпозицию текста с файлом, содержимое которого обозначает либо модификацию, либо локализацию сказуемого.}
				\scnaddlevel{1}
				\scnrelfrom{источник}{\scncite{Hardzei2007a}}
				\scnrelfrom{источник}{\scncite{Hardzei2017a}}
				\scnrelfrom{источник}{\scncite{Hardzei2007b}}
				\scnaddlevel{-1}
			\scnsubdividing{
				обстоятельство степени\scnrolesign\\
					\scnaddlevel{1}
					\scnexplanation{Обстоятельство степени -- обстоятельство, обозначающее модификацию сказуемого.}
					\scnaddlevel{-1}
				;обстоятельство образа действия\scnrolesign\\
					\scnaddlevel{1}
					\scnexplanation{Обстоятельство образа действия -- обстоятельство, обозначающее модификацию сказуемого.}
					\scnaddlevel{-1}
				;обстоятельство места\scnrolesign\\
				\scnaddlevel{1}
				\scnexplanation{Обстоятельства места -- обстоятельство, обозначающее пространственную локализацию сказуемого.}
				\scnsubdividing{
					динамическое обстоятельство места\scnrolesign;
					статическое обстоятельство места\scnrolesign
				}
				\scnaddlevel{-1}
				;обстоятельство времени\scnrolesign\\
				\scnaddlevel{1}
				\scnexplanation{Обстоятельство времени -- обстоятельство, обозначающее временную локализацию сказуемого.}
				\scnsubdividing{
					динамическое обстоятельство времени\scnrolesign;
					статическое обстоятельство времени\scnrolesign
				}
				\scnaddlevel{-1}
			}
			\scnaddlevel{-1}
	}
	\scnaddlevel{-1}
}

\bigskip
\bigskip
\scnheader{Пример sc.g-текста, описывающего лексему}
\scniselement{sc.g-текст}
\scnexplanation{Здесь представлено описание лексемы с указанием ее принадлежности определённой части речи. Также описание содержит морфологическую парадигму данной лексемы, связывающую ее с ее словоформами.}
\scneqscg{figures/sd_natural_languages/lexeme_example.png}

\scnheader{Пример этапов разбора текста естественного языка}

\scnstartsubstruct
	\scnaddlevel{1}
	\scnfilescg{figures/sd_natural_languages/nl_text.png}
	\scnexplanation{с точки зрения ostis-системы, любой естественно-языкой текст является \textit{файлом.}}
	\scnrelfrom{лексическая структура}{\scnfilescg{figures/sd_natural_languages/nl_lexical.png}}
		\scnaddlevel{1}
		\scnexplanation{Данная конструкция описывает декомпозицию исходного текста на фрагменты с указанием их принадлежности определённой \textit{номинативной единице} или \textit{знаку алфавита синтаксиса}.}
		\scnrelfrom{синтаксическая структура}{\scnfilescg{figures/sd_natural_languages/nl_synactical.png}}
%			\scnaddlevel{1}
%			\bigskip
%			\bigskip
%			\scnexplanation{Здесь приведена только частью синтаксической структуры. Оставшаяся часть записывается аналогично.}
%			\scnaddlevel{-1}
		\scnaddlevel{-1}
	\scnaddlevel{-1}
\scnendstruct

%\bigskip
\scnendstruct \scnendcurrentsectioncomment

\end{SCn}
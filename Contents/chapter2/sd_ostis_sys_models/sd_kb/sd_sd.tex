\begin{SCn}

\scnsectionheader{\currentname}

\scnstartsubstruct

\scnreltovector{конкатенация сегментов}{Что такое предметная область;Роли знаков, входящих в состав предметных областей;Типология предметных областей и отношения, заданных на множестве предметных областей;Что такое sc-язык}

\scnheader{Предметная область предметных областей}
\scnidtf{Предметная область, объектами исследования которой являются предметные области}
\scnexplanation{В состав \textbf{\textit{Предметной области предметных областей}} входят структурные спецификации всех \textit{предметных областей}, входящих в состав базы знаний \textit{ostis-системы}, в том числе, самой \textbf{\textit{Предметной области предметных областей}}. Таким образом, \textbf{\textit{Предметная область предметных областей}} является, во-первых, \textit{рефлексивным множеством}, во-вторых, рефлексивной предметной областью, то есть \textit{предметной областью}, одним из объектов исследования которой является она сама.}
\scniselement{рефлексивное множество}
\scnsdmainclasssingle{предметная область}

\scnsdclass{статическая предметная область;динамическая предметная область;понятие;sc-язык}

\scnsdrelation{понятие предметной области\scnrolesign ;исследуемое понятие\scnrolesign ;максимальный класс объектов исследования\scnrolesign ;немаксимальный класс объектов исследования\scnrolesign ;исследуемый класс первичных элементов\scnrolesign ;исследуемое отношение\scnrolesign ;класс исследуемых структур\scnrolesign ;понятие, исследуемое в дочерней предметной области\scnrolesign ;понятие, исследуемое в материнской предметной области\scnrolesign ;вспомогательное понятие\scnrolesign ;дочерняя предметная область*;дочерняя предметная область по классу первичных элементов*;дочерняя предметная область по исследуемым отношениям*;предметная область sc-языка*}

\scnsegmentheader{Что такое предметная область}

\scnstartsubstruct

\scnheader{предметная область}
\scnidtf{sc-модель предметной области}
\scnidtf{sc-текст предметной области}
\scnidtf{sc-граф предметной области}
\scnidtf{представление предметной области в \textit{SC-коде}}
\scnsubset{знание}
\scnsubset{бесконечное множество}
\scnexplanation{\textbf{\textit{предметная область}} – это результат интеграции (объединения) частичных семантических окрестностей, описывающих все исследуемые сущности заданного класса и имеющих одинаковый (общий) предмет исследования (то есть один и тот же набор отношений, которым должны принадлежать связки, входящие в состав интегрируемых семантических окрестностей).


\textbf{\textit{предметная область}} – \textit{структура}, в состав которой входят:
\begin{scnitemize}
\item \textnormal{основные исследуемые (описываемые) объекты – первичные и вторичные;}
\item \textnormal{различные классы исследуемых объектов;}
\item \textnormal{различные связки, компонентами которых являются исследуемые объекты (как первичные, так и вторичные), а также, возможно, другие такие связки – то есть связки (как и объекты исследования) могут иметь различный структурный уровень;}
\item \textnormal{различные классы указанных выше связок (то есть отношения);}
\item \textnormal{различные классы объектов, не являющихся ни объектами исследования, ни указанными выше связками, но являющихся компонентами этих связок.}
\end{scnitemize}


При этом все классы, объявленные исследуемыми понятиями, должны быть полностью представлены в рамках данной предметной области вместе со своими элементами, элементами элементов и т.д. вплоть до терминальных элементов.


Можно говорить о типологии \textbf{\textit{предметных областей}} по разным структурным признакам:
\begin{scnitemize}
    \item наличие метасвязей;
    \item наличие исследуемых структур, входящих в состав предметной области;
    \item наличие исследуемых (смежных, дополнительных) объектов, которых исследуются в других предметных областях;
\end{scnitemize}


Понятие \textbf{\textit{предметной области}} является важнейшим методологическим приемом, позволяющим выделить из всего многообразия исследуемого Мира только определенный класс исследуемых сущностей и только определенное семейство отношений, заданных на указанном классе. То есть осуществляется локализация, фокусирование внимания только на этом, абстрагируясь от всего остального исследуемого Мира.


Во всем многообразии \textbf{\textit{предметных областей}} особое место занимают
\begin{scnitemize}
    \item \textit{Предметная область предметных областей}, объектами исследования которой являются всевозможные \textbf{\textit{предметные области}}, а предметом исследования – всевозможные \textit{ролевые отношения}, связывающие предметные области с их элементами, отношения, связывающие предметные области между собой, отношение, связывающее предметные области с их онтологиями
    \item \textit{Предметная область сущностей}, являющаяся предметной областью самого высокого уровня и задающая базовую семантическую типологию \textit{sc-элементов}(знаков, входящих в тексты \textit{SC-кода})
    \item Семейство \textbf{\textit{предметных областей}}, каждая из которых задает семантику и синтаксис некоторого \textit{sc-языка}, обеспечивающего представление онтологий соответствующего вида (например, \textit{теоретико-множественных онтологий}, \textit{логических онтологий}, \textit{терминологических онтологий}, \textit{онтологий задач и способов их решения} и т.д.)
    \item Семейство \textbf{\textit{предметных областей}} верхнего уровня, в которых классами объектов исследования являются весьма "крупные"{} классы сущностей. К таким классам, в частности
    
    \begin{scnitemizeii}
        \item класс всевозможных \textit{материальных сущностей},
        \item класс всевозможных \textit{множеств},
        \item класс всевозможных \textit{связей},
        \item класс всевозможных \textit{отношений},
        \item класс всевозможных \textit{структур},
        \item класс всевозможных \textit{временных (временно существующих, непостоянных сущностей) сущностей},
        \item класс всевозможных \textit{действий} (акций),
        \item класс всевозможных \textit{параметров} (характеристик),
        \item класс \textit{знаний} всевозможного вида 
        \item и т.п.
    \end{scnitemizeii}
\end{scnitemize}


Каждой \textbf{\textit{предметной области}} можно поставить в соответствие:
\begin{scnitemize}
    \item семейство соответствующих ей \textit{онтологий} разного вида;
    \item некий язык (в нашем случае – язык, построенный на основе \textit{SC-кода}), тексты которого представляют различные фрагменты соответствующей предметной области
\end{scnitemize}


Указанные языки будем называть \textit{sc-языками}. Их синтаксис и семантика полностью задается \textit{SС-кодом} и \textit{онтологией} соответствующей \textbf{\textit{предметной области}}. Очевидно, что в первую очередь нас должны интересовать те \textit{sc-языки}, которые соответствуют \textbf{\textit{предметным областям}}, имеющим общий (условно говоря, предметно независимый) характер. К таким предметным областям, в частности, относятся:
\begin{scnitemize}
    \item \textit{Предметная область множеств}, описывающая множества и различные связи между ними
    \item \textit{Предметная область отношений и соответствий}
    \item \textit{Предметная область} (в частности, графовых) структур
    \item \textit{Предметная область чисел и числовых структур}
    \item и т.д
\end{scnitemize}


Каждому типу знаний можно поставить в соответствие предметную область, которая является результатом интеграции всех знаний данного типа. Эти знания и становятся объектами исследования в рамках указанной предметной области.


Понятие \textbf{\textit{предметной области}} может рассматриваться как обобщение понятия алгебраической системы. При этом семантическая структура базы знаний может рассматриваться как иерархическая система различных \textbf{\textit{предметных областей}}.
}
\scnsubset{знание}
\scnidtf{система связей некоторого множества объектов исследования, \uline{ключевыми} элементами которой являются:
	\begin{scnitemize}
	\item классы (точнее, знаки классов) объектов исследования (объектов, описываемых этой предметной областью);
	\item конкретные объекты исследования, обладающие особыми свойствами;
	\item классы связей, входящих в состав рассматриваемой системы -- отношения, заданные на множестве элементов рассматриваемой системы;
	\item параметры, заданные на множестве элементов рассматриваемой системы;
	\item классы структур, являющихся фрагментами рассматриваемой системы.
	\end{scnitemize}}
\scnidtf{структура, представляющая собой множество связей (точнее, знаков связей) и соответствующее множество компонентов этих связей, к числу которых относится:
	\begin{scnitemize}
	\item элементы (экземпляры) некоторых заданных классов \uline{объектов исследования} (первичных исследуемых сущностей);
	\item сами связи, входящие в состав указанной структуры;
	\item введенные классы объектов исследования;
	\item введенные отношения (классы связей);
	\item введенные параметры (классы классов эквивалентных сущностей);
	\item значения параметров (и, в частности, величины для измеряемых параметров);
	\item введенные структуры, являющиеся фрагментами (подструктурами) рассматриваемой структуры;
	\item введенные классы подструктур рассматтриваемой структуры.
	\end{scnitemize}}
\scnnote{Выделяемые в рамках \textit{базы знаний} интеллектуальной системы \textit{предметные области} и соответствующие им \textit{онтологии} -- это, своего рода, семантические страты, кластеры, позволяющие "разложить"{} все хранимые в памяти \textit{знания} по "семантическим полочкам"{} при наличии четких критериев, позволяющих \uline{однозначно} определить то, на какой "полочке"{} должны находиться те или иные \textit{знания}}
\scnnote{Существуют предметные области, в которых основным исследуемым понятием является множество всевозможных связей между экземплярами понятий, исследуемых в других предметных областях. Так, например, можно ввести Предметную область треугольников, Предметную область окружностей, а также Предметную область связей между треугольниками и окружностями.}
\scnendstruct \scnendsegmentcomment{Что такое предметная область}

\scnsegmentheader{Роли знаков, входящих в состав предметной области}

\scnstartsubstruct

\scnheader{роль элемента предметной области}
\scnidtf{ролевое отношения, связывающее предметные области с их ключевыми знаками}
\scnidtf{роль ключевого элемента (знака ключевой сущностей) предметной области}
\scnidtf{роль ключевого знака предметной области}
\scnhaselement{класс объектов исследования \scnrolesign}
\scnhaselement{максимальный класс объектов исследования\scnrolesign}
\scnhaselement{ключевой объект исследования\scnrolesign}
\scnhaselement{понятие, используемое в предметной области\scnrolesign}
\scnhaselement{первичный исследуемый элемент предметной области\scnrolesign}
\scnhaselement{вторичный исследуемый элемент предметной области\scnrolesign}
\scnhaselement{неисследуемый элемент предметной области\scnrolesign}


\scnheader{класс объектов исследования\scnrolesign}
\scnidtf{быть классом \uline{первичных} (для данной предметной области) объектов исследования\scnrolesign}
\scnnote{Понятие \uline{первичного} объекта исследования для предметной области является понятием \uline{относительным} и абсолютно не зависит от типа и уровня сложности этого объекта. Само исследование (спецификация) таких первичных исследуемых объектов осуществляется:
	\begin{scnitemize}
	\item путем введения различных классов объектов исследования, которым эти объекты принадлежат;
	\item путем введения различных связок из первичных объектов исследования и различных классов таких связок (отношений), которым принадлежат введенные связки;
	\item путем введения таких классов первичных объектов исследования, которые являются значениями вводимых параметров;
	\item путем введения различных структур, состоящих из первичных объектов исследования, из связок таких объектов, из введенных отношений и классов первичных объектов, из введенных параметров и значений этих параметров, и путем введения различных классов таких структур;
	\item путем введения различных связок из вторичных объектов исследования (т.е. из связок и структур) и путем введения различных классов таких связок;
	\item и далее можно переходить к объектам исследования более высокого уровня сложности, к параметрам, элементами значений которых являются такие объекты, а также к структурам, элементами которых являются объекты такого уровня и, соответственно, к классам таких структур.
	\end{scnitemize}}
\scnrelfromlist{второй домен}{
множество
;отношение\\
	\scnaddlevel{1}
	\scnsubset{множество}
	\scnaddlevel{-1}
;параметр\\
	\scnaddlevel{1}
	\scnsubset{класс классов}
	\scnaddlevel{-1}
;значение параметра\\
	\scnaddlevel{1}
	\scnsubset{класс}
	\scnaddlevel{-1}
;структура\\
	\scnaddlevel{1}
	\scnsubset{множество}
	\scnaddlevel{-1}
;темпоральная сущность
;темпоральная сущность базы знаний ostis-системы
;семантическая окрестность
;предметная область
;онтология
;логическая формула
;действие
;задача
;информационная конструкция
;язык
;sc-конструкция
;кибернетическая система
;интеллектуальная компьютерная система
;знание
;база знаний
;решатель задач интеллектуальной компьютерной системы
;интерфейс интеллектуальной компьютерной системы
;компьютерная система, основанная на смысловом представлении информации
;смысловое представление информации
;многоагентная модель решения задач, основанная на смысловом представлении информации
;логико-семантическая модель интерфейсов компьютерных систем, основанных на смысловом представлении информации
;решатель задач ostis-системы
;действие, выполняемое ostis-системой
;задача, решаемая ostis-системой
:план решения задачи, реализуемый ostis-системой
;протокол решения задачи, реализованный ostis-системой
;метод решения класса задач, реализуемый ostis-системой
;sc-агент\\
	\scnaddlevel{1}
	\scnidtf{внутренний агент ostis-системы, осуществляющий выполнение некоторого вида действий в памяти ostis-системы}
	\scnsuperset{sc-агент обработки информации в памяти ostis-системы}
	\scnsuperset{sc-агент управления внешними действиями ostis-системы}
	\scnaddlevel{-1}
;Базовый язык программирования ostis-систем\\
	\scnaddlevel{1}
	\scnidtf{Язык SCP}	
	\scnaddlevel{-1}
;искусственная нейронная сеть
;интерфейс ostis-системы
;интерфейсное действие пользователя ostis-системы
;sc-агент интерфейса ostis-системы
;естественный язык
;базовый интерпретатор логико-семантических моделей ostis-систем
;базовый интерпретатор логико-семантических моделей ostis-систем, реализованный программно на современных компьютерах
;семантический ассоциативный компьютер
;обучение пользователей ostis-систем
;ostis-система персональной адаптивной поддержки всех видов деятельности пользователя
;ostis-система управления рецептурным производством
;ostis-система, реализующая интеллектуальный портал научно-технических знаний}
\scnnote{Здесь приведено семейство тех \textit{классов объектов исследования}, для которых в текущей версии \textit{Стандарта OSTIS} представлены соответствующие \textit{предметные области}. Очевидно, что это семейство должно быть существенно расширено и включить в себя, например, такие \textit{классы} сущностей, как:
	\begin{scnitemize}
	\item материальная сущность
	\item вещество
	\item физическое поле
	\item персона
	\item пространственная сущность
	\item юридическое лицо
	\item предприятие
	\item географический объект
	\item и многие другие
	\end{scnitemize}}
\scnnote{Особого внимания требуют те \textit{классы объектов исследования}, которые носят наиболее общий характер  которым соответствуют \textit{предметные области и онтологии} \uline{высокого уровня}. Здесь важна продуманная система декомпозиции всего множества окружающих нас сущностей на иерархическую систему \textit{классов объектов исследования}, которой соответствует иерархическая система \textit{предметных областей и онтологий}, определяющая направления \uline{наследования свойств} исследуемых объектов.}


\scnheader{максимальный класс объектов исследования\scnrolesign}
\scnidtf{класс объектов исследования, для которого \uline{в заданной} (!) предметной области отсутствует другой класс объектов исследования, который был бы его надмножеством\scnrolesign}
\scnnote{В некоторых предметных областях может быть \uline{несколько} максимальных классов объектов исследования}


\scnheader{ключевой объект исследования\scnrolesign}
\scnidtf{особый объект исследования\scnrolesign}
\scnidtf{быть знаком особого исследуемого объекта в рамках заданной предметной области\scnrolesign}
\scnidtf{объект исследования, обладающий особыми свойствами\scnrolesign}
\scnhaselementrole{пример}{<Предметная область чисел ; Ноль>}
	\scnaddlevel{1}
	\scnnote{Особыми свойствами Числа \textbf{Ноль} являются:
		\begin{scnitemize}
		\item Результатом сложения Числа \textbf{Ноль} с любым числом \textbf{x} является число \textbf{x};
		\item Результатом умножения Числа \textbf{Ноль} на любое число является Число \textbf{Ноль}
		\end{scnitemize}}
	\scnaddlevel{-1}
\scnhaselement{<Предметная область чисел ; 1>}
\scnhaselement{<Предметная область чисел ; Число-пи>}
\scnhaselement{<Педметная область чисел ; Число-e>}


\scnheader{ключевой элемент предметной области\scnrolesign}
\scnidtf{входящий в состав предметной области знак ключевой сущности\scnrolesign}
\scnsubdividing{понятие, используемое в предметной области\scnrolesign
;ключевой объект исследования\scnrolesign \\
	\scnaddlevel{1}
	\scnidtf{знак ключевого объекта исследования\scnrolesign}
	\scnaddlevel{-1}}


\scnheader{понятие, используемое в предметной области\scnrolesign}
\scnidtf{понятие, используемое в заданной предметной области не в качестве одного из объектов исследования, а в качестве \uline{ключевого} понятия\scnrolesign}
\scnsubset{используемое понятие\scnrolesign}
	\scnaddlevel{1}
	\scnidtf{понятие, используемое в sc-знании\scnrolesign}
	\scnsubset{используемое понятие*}
		\scnaddlevel{1}
		\scnidtf{понятие, используемое в знании, которое может быть представлено не только в SC-коде*}
		\scnaddlevel{-1}
	\scnaddlevel{-1}
\scnnote{Уточнение характера использования понятия в предментной области осуществляется по трем признакам:
	\begin{scnitemize}
	\item семантический тип используемого понятия;
	\item полнота вхождения элементов понятия в данную предметную область;
	\item наличие первого упоминания понятия;
	\item наличие определения понятия или объявления его неопределяемостис подробным пояснением и примерами;
	\item наличие исследования понятия.	
	\end{scnitemize}}
\scnrelfrom{разбиение}{семантический тип используемого понятия}
	\scnaddlevel{1}
	\scneqtoset{класс объектов исследования\scnrolesign
;отношение, используемое в предметной области\scnrolesign
;параметр, используемый в предметной области\scnrolesign
;класс структур, используемый в предметной области\scnrolesign}
	\scnaddlevel{-1}
\scnrelfrom{разбиение}{полнота вхождения элементов понятия в данную предметную область}
	\scnaddlevel{1}
	\scneqtoset{используемое понятие, все элементы которого входят в данную предметную область\scnrolesign \\
	\scnaddlevel{1}
	\scnnote{Для каждого используемого отношения в предметную область здесь должны входить не только знаки связок, но и все связки целиком с их компонентами}
	\scnaddlevel{-1}
;используемое понятие, не все элементы которого входят в данную предметную область\scnrolesign}
	\scnaddlevel{-1}
\scnrelfrom{разбиение}{наличие первого упоминания понятия}
	\scnaddlevel{1}
	\scneqtoset{понятие, вводимое в данной предметной области\scnrolesign
;понятие, которое в данной предметной области используется, но не вводится\scnrolesign}
	\scnaddlevel{1}
	\scnnote{Будем считать, что понятие вводится в данной предметной области в том и только в том случае, если ни в одной предметной области более высокого уровня это понятие не используется. Т.е. речь идет о первом упоминании этого понятия в рамках последовательности предметных областей от материнских к дочерним}
	\scnaddlevel{-1}
	\scnaddlevel{-1}
\scnrelfrom{разбиение}{наличие определения понятия или объявления его неопределяемости с подробным пояснением и примерами}
	\scnaddlevel{1}
	\scneqtoset{понятие, которое в данной предметной области определено или объявлено как неопределяемое
;понятие, которое в данной предметной области не имеет ни определения, ни указания факта его неопределяемости}
	\scnaddlevel{-1}
\scnrelfrom{разбиение}{наличие исследования понятия}
	\scnaddlevel{1}
	\scneqtoset{понятие, исследуемое в данной предметной области\scnrolesign
;понятие, которое в данной предметной области испольуется, но не исследуется\scnrolesign}
	\scnaddlevel{-1}
\scnnote{Понятие, используемое в базе знаний, может быть введено (впервые упомянуто) в одной предметной области, определено в другой, а исследоваться -- в третьей}


\scnheader{первичный исследуемый элемент предметной области\scnrolesign}
\scnidtf{знак первичного объекта исследования в рамках заданной предметной области\scnrolesign}


\scnheader{вторичный исследуемый элемент предметной области\scnrolesign}
\scnidtf{знак вторичного объекта исследования в рамках предметной области\scnrolesign}
\scnsuperset{связка элементов предметной области\scnrolesign}
	\scnaddlevel{1}
\scnsuperset{связка первичных элементов предметной области\scnrolesign}
\scnsuperset{метасвязка элементов предметной области\scnrolesign}
	\scnaddlevel{1}
\scnsuperset{метасвязка, в число компонентов которой входят связки элементов предметной области\scnrolesign}
\scnsuperset{метасвязка, в число компонентов которой входят классы элементов предметной области\scnrolesign}
\scnsuperset{метасвязка, в число компонентов которой входят структуры элементов предметной области\scnrolesign}
	\scnaddlevel{-1}
	\scnaddlevel{-1}
\scnsuperset{класс элементов предметной области\scnrolesign}
		\scnaddlevel{1}
\scnsuperset{класс первичных элементов предметной области\scnrolesign}
\scnsuperset{класс связок элементов предметной области\scnrolesign}
\scnsuperset{класс классов элементов предметной области\scnrolesign}
\scnsuperset{класс структур элементов предметной области\scnrolesign}
	\scnaddlevel{-1}
\scnsuperset{структура элементов предметной области\scnrolesign}
		\scnaddlevel{1}
\scnsuperset{тривиальная структура первичных элементов предметной области\scnrolesign}
\scnsuperset{структура, в число подмножеств которой входят связки элементов предметной области вместе со своими компонентами\scnrolesign}
\scnsuperset{структура, в число подмножеств которой входят классы элементов предметной области вместе со своими знаками\scnrolesign}
\scnsuperset{структура, в число подмножеств которой входят другие структуры вместе со своими знаками\scnrolesign}
	\scnaddlevel{-1}


\scnheader{неисследуемый элемент предметной области\scnrolesign}
\scnidtf{вспомогательный элемент предметной области, исследуемый в другой (смежной) предметной области\scnrolesign}
\scnnote{С помощью неисследуемых элементов предметной области описываются и исследуются различные вида связи между элементами, исследуемыми в данной \textit{предметной области} с элементами, исследуемыми в других \textit{предметных областях}. При этом \textit{связки}, компонентами которых являются как исследуемые, так и неисследуемые элементы данной \textit{предметной области} считаются \uline{исследуемыми} связками этой \textit{предметной области}. Примерами неисследуемых элементов, напримр, геометрической \textit{предметной области} являются \textit{числа}, являющиеся \textit{значениями величин} таких \textit{параметров}, как \textit{расстояние}\scnsupergroupsign, \textit{длина}\scnsupergroupsign, \textit{площадь}\scnsupergroupsign, \textit{объем}\scnsupergroupsign, а также различные числовые \textit{отношения} (\textit{сложение}*, \textit{умножение}*, \textit{возведение в степень}*), теоретико-множественные \textit{отношения} (\textit{включение}*, \textit{объединение}*, \textit{пересечение}*, \textit{принадлежность}*)}


\scnheader{понятие}
\scnidtf{концепт}
\scnidtf{класс сущностей, который входит в состав по крайней мере одной предметной области в качестве (в роли) ключевого исследуемого понятия}
\scnnote{Семейство всех введенных понятий -- это, своего рода, семантическая система координат, позволяющая специфицировать всевозможные сущности в смысловом пространстве.}
\scnidtf{класс сущностей, который по крайней мере в одной \textit{предметной области} "объявлен"{} как \textit{понятие} (вводимое, исследуемое или вспомогательное)}
\scnnote{Каждому \textit{понятию} соответствует по крайней мере одна \textit{предметная область}, в которой это понятие является \textit{исследуемым понятием} и в которой рассматриваются основные характеристики этого \textit{понятия}. Если же в какой-либо \textit{предметной области} необходимо рассмотреть дополнительные связи этого \textit{понятия} с другими \textit{понятиями}, то оно объявляется как \textit{вспомогательное понятия}\scnrolesign .}
\scnidtf{Второй домен Отношения \textit{используемое понятие}*}
\scnrelto{второй домен}{используемое понятие*}
\scnidtf{класс сущностей (класс связок (в т.ч. отношение), класс классов (в т.ч. параметр), класс структур), который по крайней мере в одной \textit{предметной области} является \textit{используемым понятием}\scnrolesign}

\scnendstruct \scnendsegmentcomment{Роли знаков, входящих в состав предметной области}


\scnendstruct \scnendcurrentsectioncomment

\end{SCn}
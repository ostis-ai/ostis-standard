\begin{SCn}

\scnsectionheader{\currentname}

\scnstartsubstruct

\scnheader{Предметная область семантических окрестностей}
\scniselement{предметная область}
\scnsdmainclasssingle{семантическая окрестность}
\scnsdclass{семантическая окрестность по инцидентным коннекторам;семантическая окрестность по выходящим дугам;семантическая окрестность по выходящим дугам принадлежности;семантическая окрестность по входящим дугам;семантическая окрестность по входящим дугам принадлежности;полная семантическая окрестность;базовая семантическая окрестность;специализированная семантическая окрестность;пояснение;примечание;правило идентификации экземпляров;терминологическая семантическая окрестность;теоретико-множественная семантическая окрестность;описание декомпозиции;логическая семантическая окрестность;описание типичного экземпляра;сравнительный анализ;иллюстрация}

\scnheader{семантическая окрестность}
\scnidtf{sc-окрестность}
\scnidtf{семантическая окрестность, представленная в виде sc-текста}
\scnidtf{sc-текст, являющийся семантической окрестностью некоторого sc-элемента}
\scnidtf{спецификация заданной сущности, знак которой указывается как ключевой элемент этой спецификации}
\scnidtf{описание заданной сущности, знак которой указывается как ключевой элемент этой спецификации}
\scnsubset{знание}
\scnsuperset{семантическая окрестность по инцидентным коннекторам}
\scnsuperset{полная семантическая окрестность}
\scnsuperset{базовая семантическая окрестность}
\scnsuperset{специализированная семантическая окрестность}
\scnexplanation{\textbf{\textit{семантическая окрестность}} – это знание, являющееся спецификацией (описанием) некоторой сущности, знак которой является ключевым элементом указанного знания. Заметим, что каждая семантическая окрестность в отличие от знаний других видов имеет только один ключевой элемент (ключевой знак, знак описываемой сущности). Заметим также, что многообразие видов семантических окрестностей свидетельствует о многообразии семантических видов описаний различных сущностей.}

\scnheader{семантическая окрестность по инцидентным коннекторам}
\scnsuperset{семантическая окрестность по выходящим дугам}
\scnsuperset{семантическая окрестность по входящим дугам}
\scnexplanation{\textbf{\textit{семантическая окрестность по инцидентным коннекторам}} – это вид семантической окрестности, в которую входят знаки всех коннекторов, инцидентных заданному элементу, а также знаки всех элементов, инцидентных указанным коннекторам.}

\scnheader{семантическая окрестность по выходящим дугам}
\scnsuperset{семантическая окрестность по выходящим дугам принадлежности}
\scnexplanation{\textbf{\textit{семантическая окрестность по выходящим дугам}} – это вид семантической окрестности, в которую входят знаки всех дуг, выходящих из заданного sc-элемента, а также знаки их вторых компонентов, также указывается факт принадлежности этих дуг каким-либо отношениям.}

\scnheader{семантическая окрестность по выходящим дугам принадлежности}
\scnexplanation{\textbf{\textit{семантическая окрестность по выходящим дугам принадлежности}} – это вид семантической окрестности, в которую входят знаки всех дуг принадлежности, выходящих из заданного sc-элемента, а также знаки их вторых компонентов. При необходимости может указывается факт принадлежности этих дуг каким-либо ролевым отношениям.}

\scnheader{семантическая окрестность по входящим дугам}
\scnsuperset{семантическая окрестность по входящим дугам принадлежности}
\scnexplanation{\textbf{\textit{семантическая окрестность по входящим дугам}} – это вид семантической окрестности, в которую входят знаки всех дуг, входящих в заданный sc-элемент, а также знаки их первых компонентов, также указывается факт принадлежности этих дуг каким-либо отношениям.}

\scnheader{семантическая окрестность по входящим дугам принадлежности}
\scnexplanation{\textbf{\textit{семантическая окрестность по входящим дугам принадлежности}} – это вид семантической окрестности, в которую входят знаки всех дуг принадлежности, входящих в заданный sc-элемент, а также знаки их первых компонентов. При необходимости может указывается факт принадлежности этих дуг каким-либо ролевым отношениям.}

\scnheader{полная семантическая окрестность}
\scnidtf{полная спецификация некоторой описываемой сущности}
\scnexplanation{\textbf{\textit{полная семантическая окрестность}} – это вид семантической окрестности, включающий описание всех связей описываемой сущности. 

Структура полной семантической окрестности определяется прежде всего семантической типологией описываемой сущности. 

Так, например, для понятия в полную семантическую окрестность необходимо включить следующую информацию (при наличии):
\begin{scnitemize}
    \item варианты идентификации на различных внешних языках;
    \item принадлежность некоторой предметной области с указанием роли, выполняемой в рамках этой предметной области;
    \item теоретико-множественные связи заданного понятия с другими sc-элементами;
    \item определение или пояснение;
    \item высказывания, описывающие свойства указанного понятия;
    \item задачи и их классы, в которых данное понятие является ключевым
    \item описание типичного примера использования указанного понятия;
    \item экземпляры описываемого понятия.
\end{scnitemize}
Для понятия, являющегося отношением дополнительно указываются:
\begin{scnitemize}
    \item домены;
    \item область определения;
    \item схема отношения;
    \item классы отношений, которым принадлежит описываемое отношение.
\end{scnitemize}
}

\scnheader{базовая семантическая окрестность}
\scnidtf{минимально достаточная семантическая окрестность}
\scnidtf{минимальная спецификация описываемой сущности}
\scnidtf{сокращенная спецификация описываемой сущности}
\scnidtf{основная семантическая окрестность}
\scnexplanation{\textbf{\textit{базовая семантическая окрестность}} – это вид семантической окрестности, содержащий минимальную (краткую) информацию об описываемой сущности

Структура базовой семантической окрестности определяется прежде всего семантической типологией описываемой сущности. 

Так, например, для понятия в базовую семантическую окрестность необходимо включить следующую информацию (при наличии): 
\begin{scnitemize}
    \item варианты идентификации на различных внешних языках;
    \item принадлежность некоторой предметной области с указанием роли, выполняемой в рамках этой предметной области;
    \item определение или пояснение.
\end{scnitemize}
Для понятия, являющегося отношением дополнительно указываются:
\begin{scnitemize}
    \item домены;
    \item область определения;
    \item описание типичного примера использования указанного отношения.
\end{scnitemize}
}

\scnheader{специализированная семантическая окрестность}
\scnsuperset{пояснение}
\scnsuperset{примечание}
\scnsuperset{правило идентификации экземпляров}
\scnsuperset{терминологическая семантическая окрестность}
\scnsuperset{теоретико-множественная семантическая окрестность}
\scnsuperset{логическая семантическая окрестность}
\scnsuperset{описание типичного экземпляра}
\scnsuperset{описание декомпозиции} 
\scnexplanation{\textbf{\textit{специализированная семантическая окрестность}} – это вид семантической окрестности, набор связей для которой уточняется отдельно для каждого класса такой окрестности.}

\scnheader{пояснение}
\scnidtf{sc-пояснение}
\scnexplanation{\textbf{\textit{пояснение}} – знак sc-текста, поясняющего описываемую сущность.}

\scnheader{примечание}
\scnidtf{sc-примечание}
\scnexplanation{\textbf{\textit{примечание}} – знак sc-текста, являющегося примечанием к описываемой сущности. В примечании обычно описываются особые свойства и исключения из правил для описываемой сущности.}

\scnheader{правило идентификации экземпляров}
\scnidtf{правило идентификации экземпляров заданного класса}
\scnexplanation{\textbf{\textit{правило идентификации экземпляров}} – это sc-текст являющийся описанием правил построения идентификаторов элементов заданного класса.}

\scnheader{терминологическая семантическая окрестность}
\scnexplanation{\textbf{\textit{терминологическая семантическая окрестность}}  –  семантическая окрестность, описывающая идентификацию указанной сущности}

\scnheader{теоретико-множественная семантическая окрестность}
\scnexplanation{\textbf{\textit{теоретико-множественная семантическая окрестность}}  –  описание связи описываемого понятия с другими понятиями с помощью теоретико-множественных отношений}

\scnheader{описание декомпозиции}
\scnidtf{семантическая окрестность, описывающая декомпозицию некоторой сущности}
\scnexplanation{\textbf{\textit{описание декомпозиции}}  –  семантическая окрестность, описывающая декомпозицию некоторой сущности на частные сущности}

\scnheader{логическая семантическая окрестность }
\scnexplanation{\textbf{\textit{логическая семантическая окрестность}}  –  семантическая окрестность, описывающая семейство высказываний, описывающих свойства данного понятия}

\scnheader{описание типичного экземпляра}
\scnidtf{описание типичного экземпляра заданного класса}
\scnidtf{типичная семантическая окрестность}
\scnidtf{типичная sc-окрестность}
\scnexplanation{\textbf{\textit{описание типичного экземпляра}} – это sc-текст являющийся описанием типичного примера использования рассматриваемого класса.}

\scnheader{сравнительный анализ}
\scnexplanation{\textbf{\textit{сравнительный анализ}} –  описание сравнительного анализа некоторой сущности с другими сущностями}

\scnheader{иллюстрация}
\scnsubset{специализированная семантическая окрестность}
\scnexplanation{\textbf{\textit{иллюстрация}} –  семантическая окрестность некоторой сущности (сущностей), иллюстрирующая некоторые свойства указанных сущностей, чаще всего, на некотором конкретном примере.}

\scnendstruct

\end{SCn}
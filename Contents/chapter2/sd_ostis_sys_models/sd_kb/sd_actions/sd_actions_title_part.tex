\begin{SCn}
	
\scniselement{раздел}
\scniselement{предметная область и онтология}
\scnreltovector{конкатенация сегментов}{Уточнение понятия воздействия и понятия действия. Типология воздействий и действий;Уточнение понятия задачи. Типология задач;Уточнение семейства параметров и отношений, заданных на множестве воздействий, действий и задач;Предметная область и онтология субъектно-объектных спецификаций воздействий;Уточнение понятий плана сложного действия, класса задач, метода; Уточнение понятия навыка, понятия класса методов и понятия модели решения задач;Уточнение понятия деятельности, понятия вида деятельности и понятия технологии}

\scnhaselementlist{исследуемый класс первичных объектов исследования}{
воздействие\\
	\scnaddlevel{1}
	\scnidtf{\textit{процесс} воздействия одних \textit{сущностей} на другие}
	\scnsubset{процесс}
	\scnaddlevel{-1}
;действие\\
	\scnaddlevel{1}
	\scnidtf{\textit{процесс}, "осознанно"{} и целенаправленно выполняемый (управляемый) некоторой \textit{кибернетической системой}}
	\scnsubset{воздействие}
	\scnsubset{процесс}
	\scnaddlevel{-1}
\bigskip
;неосознанное воздействие
\bigskip
;действие, выполняемое в памяти субъекта действия
;действие, выполняемое во внешней среде субъекта действия
;рецепторное действие\\
	\scnaddlevel{1}
	\scnidtf{действие, выполняемое рецептором субъекта действия}
	\scnaddlevel{-1}
;эффекторное действие\\
	\scnaddlevel{1}
	\scnidtf{действие, выполняемое эффектором субъекта действия}
	\scnaddlevel{-1}
\bigskip
;элементарное действие \\
	\scnaddlevel{1}
	\scnidtf{действие, выполнение которого не требует его декомпозиции на взаимосвязанные поддействия}
	\scnaddlevel{-1}
;сложное действие
;легко выполнимое сложное действие\\
	\scnaddlevel{1}
	\scnidtf{сложное действие, которое известно, как выполнять}
	\scnaddlevel{-1}
;интеллектуальное действие\\
	\scnaddlevel{1}
	\scnidtf{сложное действие, для которого априори не известно, как его выполнять}
	\scnaddlevel{-1}
\bigskip
;индивидуальное действие\\
	\scnaddlevel{1}
	\scnidtf{действие, выполняемое индивидуальной кибернетической системой}
	\scnaddlevel{-1}
;коллективное действие\\
	\scnaddlevel{1}
	\scnidtf{действие, выполняемое коллективом кибернетических систем (многоагентной системой)}
	\scnaddlevel{-1}
\bigskip
;планируемое действие
;инициированное действие
;выполняемое действие
	\scnaddlevel{1}
	\scnidtf{активное действие}
	\scnaddlevel{-1}
;прерванное действие
	\scnaddlevel{1}
	\scnidtf{выполняемое действие, находящееся в состоянии прерывания}
	\scnaddlevel{-1}
;выполненное действие
;отмененное действие
\bigskip
;действие с очень высоким приоритетом
;действие с высоким приоритетом
;действие со средним приоритетом
;действие с низким приоритетом
;действие с очень низким приоритетом}
\bigskip

\scnhaselementlist{исследуемый класс классов первичных объектов исследования}{
осмысленность воздействия\scnsupergroupsign
;длительность воздействия\scnsupergroupsign
;место выполнения действия\scnsupergroupsign
;функциональная сложность действия\scnsupergroupsign
;многоагентность действия\scnsupergroupsign\\
	\scnaddlevel{1}
	\scnidtf{коллективность субъекта действия}
	\scnaddlevel{-1}
;текущее состояние действия\scnsupergroupsign
;приоритет действия\scnsupergroupsign\\
	\scnaddlevel{1}
	\scnidtf{важность действия\scnsupergroupsign}
	\scnaddlevel{-1}
;срочность действия\scnsupergroupsign
\bigskip
;класс действий
;класс функционально эквивалентных действий\scnsupergroupsign
;класс логически эквивалентных действий\scnsupergroupsign
;класс семантических эквивалентных задач\scnsupergroupsign
;класс логически эквивалентных задач\scnsupergroupsign
;класс задач, для которого существует общий метод их решения\scnsupergroupsign
;класс аналогичных семантически элементарных процессов воздействия\scnsupergroupsign\\
	\scnaddlevel{1}
	\scnidtf{класс однотипных семантически элементарных воздействий\scnsupergroupsign}
	\scnaddlevel{-1}}

	
\scnhaselementlist{исследуемый класс классов}
{отношение, заданное на множестве* (действие)\\
	\scnaddlevel{1}
	\scnidtf{отношение, заданное на множестве действий}
	\scnaddlevel{-1}
;отношение, заданное на множестве* (задача)
;параметр, заданный на множестве* (действие)
;параметр, заданный на множестве* (задача)}
\scnaddlevel{1}
\scnsourcecommentpar{Здесь указаны классы классов, которые не являются классами классов \uline{первичных} объектов исследования}
\scnaddlevel{-1}
\bigskip

\scnhaselementlist{исследуемое отношение, заданное на множестве первичных объектов исследования}{воздействующая сущность\scnrolesign
;воздействуемая сущность\scnrolesign
;посредник\scnrolesign
;медиатор\scnrolesign
;субъект\scnrolesign\\
	\scnaddlevel{1}
	\scnidtf{быть субъектом заданного действия}
	\scnaddlevel{-1}
;спецификация воздействия*\\
	\scnaddlevel{1}
	\scnsuperset{спецификация действия*}
	\scnaddlevel{-1}
;спецификация действия*\\
	\scnaddlevel{1}
	\scnsuperset{задача*}
		\scnaddlevel{1}
		\scnsubdividing{декларативная формулировка задачи*;процедурная формулировка задачи*}
		\scnaddlevel{-1}
		\scnsuperset{план сложного действия*}
		\scnsuperset{декларативная спецификация выполнения сложного действия*}
		\scnsuperset{протокол*}
		\scnsuperset{результативная часть протокола*}
	\scnaddlevel{-1}
;декларативная формулировка задачи*
;процедурная формулировка задачи*
;план сложного действия*
;декларативная спецификация выполнения сложного действия*
;протокол*
;результативная часть протокола*}
\bigskip

\scnhaselementlist{исследуемое отношение}{спецификация класса действий*
;спецификация метода*
;спецификация класса методов*
;спецификация деятельности*
;спецификация вида деятельности*
}
\scnaddlevel{1}
	\scnsourcecommentpar{Здесь указаны исследуемые отношения, которые заданы не на множестве первичных объектов исследования}
\scnaddlevel{-1}
\bigskip

\scnhaselementlist{исследуемый класс структур, специфицирующих первичные объекты исследования}{
задача\\
	\scnaddlevel{1}
	\scnidtf{формулировка задачи}
	\scnidtf{спецификация действия}
	\scnidtf{структура (sc-конструкция), содержащая в идеале достаточную информацию для выполнения соответствующего (специфицируемого) действия}
	\scnaddlevel{-1}
;декларативная формулировка задачи\\
	\scnaddlevel{1}
	\scnidtf{семантическая спецификация действия}
	\scnaddlevel{-1}
;процедурная формулировка задачи\\
	\scnaddlevel{1}
	\scnidtf{функциональная спецификация действия}
	\scnaddlevel{-1}
;план сложного действия\\
	\scnaddlevel{1}
	\scnidtf{план выполнения сложного действия}
	\scnaddlevel{-1}
;процедурный план сложного действия
;непроцедурный план сложного действия\\
	\scnaddlevel{1}
	\scnidtf{декларативный план сложного действия}
	\scnidtf{иерархическая система подзадач заданной сложной задачи}
	\scnaddlevel{-1}}
\bigskip

\scnhaselementlist{исследуемый класс структур}{метод\\
	\scnaddlevel{1}
	\scnidtf{спецификация класса сложных действий}
	\scnaddlevel{-1}
;денотационная семантика метода
;операционная семантика метода
;навык
;модель решения задач
;технология}
\scnaddlevel{1}
	\scnsourcecommentpar{Здесь указаны классы структур, не являющихся спецификациями первичных объектов исследования}
\scnaddlevel{-1}
\bigskip

\scnhaselementlist{вводимое, но не исследуемое понятие}
{действие, выполняемое в памяти ostis-системы
;действие, выполняемое ostis-системой в своей внешней среде
;рецептурное действие ostis-системы
;эффекторное действие ostis-системы
;sc-агент\\
	\scnaddlevel{1}
	\scnidtf{внутренний субъект ostis-системы}
	\scnidtf{субъект, реализующий действия, выполняемые в памяти ostis-системы}
	\scnaddlevel{-1}}
\scnaddlevel{1}
	\scnsourcecommentpar{Здесь указаны понятия, исследуемые в предметной области (и соответствующей онтологии), которая является \uline{частной} по отношению к заданной и которая наследует все свойства заданной предметной области и онтологии}
\scnaddlevel{-1}
\bigskip

\scnhaselementlist{используемое понятие, исследуемое в другой предметной области и онтологии}{
кибернетическая система\\
	\scnaddlevel{1}
	\scnidtf{сущность, обладающая способностью быть субъектом различного вида действий}
	\scnaddlevel{-1}
;компьютерная система\\
	\scnaddlevel{1}
	\scnidtf{искусственная кибернетическая система}
	\scnsubset{кибернетическая система}
	\scnaddlevel{-1}
;интеллектуальная компьютерная система\\
	\scnaddlevel{1}
	\scnsubset{компьютерная система}
	\scnsuperset{ostis-система}
	\scnaddlevel{-1}
;человек\\
	\scnaddlevel{1}
	\scnsubset{кибернетическая система}
	\scnaddlevel{-1}
;ostis-система
;спецификация*
	\scnaddlevel{1}
	\scnidtf{быть спецификацией (описанием, семантической окрестностью заданной сущности*)}
	\scnidtf{семантическая окрестность*}
	\scnaddlevel{-1}}


\scnheader{следует отличать*}
\scnhaselementset{
	\scnmakevectorlocal{действие;класс действий};
	\scnmakevectorlocal{метод;класс методов};
	\scnmakevectorlocal{деятельность;вид деятельности}
}
\scnaddlevel{1}
\scnsubset{семейство подклассов*}
\scnnote{Все сущности, принадлежащие рассмотренным \textit{понятиям}, требуют достаточно детальной \textit{спецификации}. При этом не следует путать сами сущности и их \textit{спецификации}. Так, например, не следует путать \textit{действие} и \textit{задачу}, которая специфицирует (уточняет) это \textit{действие}. Особое место среди указанных понятий занимает понятие \textit{метода}, т.к. каждый конкретный \textit{метод}, с одной стороны, является \textit{спецификацией} соответствующего \textit{класса действий}, а, с другой стороны, сам нуждается в \textit{спецификации}, которая уточняет либо \textit{декларативную семантику} этого \textit{метода} (т.е. обобщенную декларативную формулировку класса задач, решаемых с помощью этого \textit{метода}), либо \textit{операционную семантику} этого \textit{метода}, (т.е. множество \textit{методов}, обеспечивающих \textit{интерпретацию} данного специфицируемого \textit{метода}) и тем самым "преобразует"{} специфицируемый \textit{метод} в \textit{навык}.}
\scnaddlevel{-1}


\scnheader{следует отличать*}
\scnhaselementvector{первый домен*(спецификация*)\\
\scnaddlevel{1}
\scnidtf{специфицируемая сущность}
\scnidtf{сущность, использование которой требует вполне определенной ее спецификации}
\scnsuperset{действие}
\scnsuperset{класс действий}
\scnsuperset{метод}
\scnsuperset{класс методов}
\scnsuperset{деятельность}
\scnsuperset{вид деятельности}
\scnaddlevel{-1};
второй домен*(спецификация*)\\
\scnaddlevel{1}
\scnidtf{спецификация}
\scnsuperset{задача}
\scnaddlevel{1}
\scnsuperset{декларативная формулировка задачи}
	\scnaddlevel{1}
	\scnidtf{семантическая формулировка задачи}
	\scnaddlevel{-1}
\scnsuperset{процедурная формулировка задачи}
	\scnaddlevel{1}
	\scnidtf{функциональная формулировка задачи}
	\scnaddlevel{-1}
\scnaddlevel{-1}
\scnsuperset{план действия}
	\scnaddlevel{1}
	\scnidtf{план}
	\scnidtf{план выполнения действия}
	\scnaddlevel{-1}
\scnsuperset{декларативная спецификация выполнения действий}
	\scnaddlevel{1}
	\scnidtf{иерархическая система подзадач}
	\scnaddlevel{-1}
\scnsuperset{протокол}
\scnsuperset{результативная часть протокола}
\scnsuperset{обобщенная декларативная формулировка класса задач}
\scnsuperset{метод}
\scnsuperset{декларативная семантика метода}
\scnsuperset{операционная семантика метода}
\scnsuperset{модель решения задач}
\scnaddlevel{-1}
}
\scnaddlevel{1}
\scnnote{
При этом следует отличать:
\begin{scnitemize}
\item спецификацию конкретного \textit{действия} (\textit{задачу}, \textit{план}, \textit{декларативную спецификацию выполнения действия}, \textit{протокол}, \textit{результативную часть протокола});
\item спецификацию конкретной \textit{деятельности} (\textit{контекст}*, \textit{множество используемых методов}*);
\item спецификацию \textit{класса действий} (\textit{обобщенную декларативную формулировку класса задач}, \textit{метод});
\item спецификацию \textit{вида деятельности} (\textit{технологию});
\item спецификацию \textit{метода} (\textit{декларативную семантику метода}, \textit{операционную семантику метода});
\item спецификацию \textit{класса методов} (\textit{модель решения задач}).
\end{scnitemize}
}
\scnaddlevel{-1}


\scnheader{следует отличать*}
\scnhaselementset{
	\scnmakevectorlocal{действие;класс действий, метод};
	\scnmakevectorlocal{деятельность;вид деятельности, технология}
}

\end{SCn}
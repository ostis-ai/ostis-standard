\scnsegmentheader{Предметная область и онтология субъектно-объектных спецификаций воздействий}

\scnstartsubstruct

\scniselement{предметная область и онтология}

\scnheader{индивид}
\scnidtftext{часто используемый sc-идентификатор}{субъект}
\scnexplanation{\textit{Индивид} является разновидностью стереотипа как отдельной сущности в выделенном фрагменте модели мира.}
	\scnaddlevel{1}
	\scnrelfrom{источник}{\scncite{Hardzei2005}}
	\scnaddlevel{-1}

\scnheader{субъектно-объектное воздействие}
\scnidtf{акция}
\scnsubset{воздействие}
\scnexplanation{Акция представляет собой воздействие одного индивида на другой.}
	\scnaddlevel{1}
	\scnrelfrom{источник}{\scncite{Hardzei2017}}
	\scnaddlevel{-1}
\scnsubdividing{
	субъектно-объектное воздействие активизации\\
	\scnaddlevel{1}
	\scnsubdividing{
		m\_воспринимание
		;m\_запоминание
		;m\_осмысливание
		;m\_понимание
		;m\_притягивание
		;m\_скапливание
		;m\_ужимание
		;m\_присоединение
		;m\_перенимание
		;m\_заучивание
		;m\_обдумывание
		;m\_усваивание
		;m\_вбирание
		;m\_накапливание
		;m\_центрирование
		;m\_ассимилирование
		;m\_прочувствование
		;m\_созерцание
		;m\_переживание
		;m\_изведывание
		;m\_перевбирание
		;m\_концентрирование
		;m\_центрифугирование
		;m\_диссимилирование
		;m\_отвергание
		;m\_изглаживание
		;m\_переосмысливание
		;m\_изживание
		;m\_выделение
		;m\_разуплотнение
		;m\_отжимание
		;m\_отъединение
	}
	\scnaddlevel{-1}
	;субъектно-объектное воздействие эксплуатации\\
	\scnaddlevel{1}
	\scnsubdividing{
		m\_сообщение
		;m\_рекламирование
		;m\_внушение
		;m\_констатирование
		;m\_подведение
		;m\_наращивание
		;m\_прижимание
		;m\_подсоединение
		;m\_объяснение
		;m\_пропагандирование
		;m\_доказывание
		;m\_удостоверивание
		;m\_введение
		;m\_нагнетание
		;m\_вжимание
		;m\_соединение
		;m\_ниспослание
		;m\_вещание
		;m\_просветление
		;m\_явление
		;m\_проведение
		;m\_распространение
		;m\_выжимание
		;m\_разъединение
		;m\_затемнение
		;m\_шифрование
		;m\_дискредитирование
		;m\_дезавуирование
		;m\_выведение
		;m\_осаживание
		;m\_оттеснение
		;m\_отсоединение
	}
	\scnaddlevel{-1}
	;субъектно-объектное воздействие трансформации\\
	\scnaddlevel{1}
	\scnsubdividing{
		m\_информирование
		;m\_заинтересовывание
		;m\_уверение
		;m\_предрасположение
		;m\_затрагивание
		;m\_обволакивание
		;m\_обжимание
		;m\_формование
		;m\_наставление
		;m\_обучение
		;m\_убеждение
		;m\_воспитание
		;m\_вскрывание
		;m\_наполнение
		;m\_сжимание
		;m\_формирование
		;m\_пронимание
		;m\_преисполнение
		;m\_преображение
		;m\_перевоплощение
		;m\_пронизывание
		;m\_переполнение
		;m\_разжимание
		;m\_выхолащивание
		;m\_донимание
		;m\_зомбирование
		;m\_умопомрачение
		;m\_умалишение
		;m\_пробивание
		;m\_вздымание
		;m\_распускание
		;m\_аннигилирование
	}
	\scnaddlevel{-1}
	;субъектно-объектное воздействие нормализации\\
	\scnaddlevel{1}
	\scnsubdividing{
		m\_воспоминание
		;m\_воссоздание
		;m\_возобновление
		;m\_воспроизведение
		;m\_рекристаллизирование
		;m\_реинтегрирование
		;m\_регенерирование
		;m\_реформование
		;m\_репродуцирование
		;m\_рекультивирование
		;m\_возрождение
		;m\_воскрешение
		;m\_рекуперирование
		;m\_реабилитирование
		;m\_реактивирование
		;m\_реанимирование
	}
	\scnaddlevel{-1}
}
\scnsubdividing{
	субъектно-объектное воздействие среда-оболочка\\
	\scnaddlevel{1}
	\scnsubdividing{
		m\_воспринимание
		;m\_запоминание
		;m\_осмысливание
		;m\_понимание
		;m\_притягивание
		;m\_скапливание
		;m\_ужимание
		;m\_присоединение
		;m\_сообщение
		;m\_рекламирование
		;m\_внушение
		;m\_констатирование
		;m\_подведение
		;m\_наращивание
		;m\_прижимание
		;m\_подсоединение
		;m\_информирование
		;m\_заинтересовывание
		;m\_уверение
		;m\_предрасположение
		;m\_затрагивание
		;m\_обволакивание
		;m\_обжимание
		;m\_формование
		;m\_воспоминание
		;m\_воссоздание
		;m\_возобновление
		;m\_воспроизведение
		;m\_рекристаллизирование
		;m\_реинтегрирование
		;m\_регенерирование
		;m\_реформование
	}
	\scnaddlevel{-1}
	;субъектно-объектное воздействие оболочка-ядро\\
	\scnaddlevel{1}
	\scnsubdividing{
		перенимание
		;m\_заучивание
		;m\_обдумывание
		;m\_усваивание
		;m\_вбирание
		;m\_накапливание
		;m\_центрирование
		;m\_ассимилирование
		;m\_объяснение
		;m\_пропагандирование
		;m\_доказывание
		;m\_удостоверивание
		;m\_введение
		;m\_нагнетание
		;m\_вжимание
		;m\_соединение
		;m\_наставление
		;m\_обучение
		;m\_убеждение
		;m\_воспитание
		;m\_вскрывание
		;m\_наполнение
		;m\_сжимание
		;m\_формирование
		;m\_репродуцирование
		;m\_рекультивирование
		;m\_возрождение
		;m\_воскрешение
		;m\_рекуперирование
		;m\_реабилитирование
		;m\_реактивирование
		;m\_реанимирование
	}
	\scnaddlevel{-1}
	;субъектно-объектное воздействие ядро-оболочка\\
	\scnaddlevel{1}
	\scnsubdividing{
		m\_прочувствование
		;m\_созерцание
		;m\_переживание
		;m\_изведывание
		;m\_перевбирание
		;m\_концентрирование
		;m\_центрифугирование
		;m\_диссимилирование
		;m\_ниспослание
		;m\_вещание
		;m\_просветление
		;m\_явление
		;m\_проведение
		;m\_распространение
		;m\_выжимание
		;m\_разъединение
		;m\_пронимание
		;m\_преисполнение
		;m\_преображение
		;m\_перевоплощение
		;m\_пронизывание
		;m\_переполнение
		;m\_разжимание
		;m\_выхолащивание
	}
	\scnaddlevel{-1}
	;субъектно-объектное воздействие оболочка-среда\\
	\scnaddlevel{1}
	\scnsubdividing{
		m\_отвергание
		;m\_изглаживание
		;m\_переосмысливание
		;m\_изживание
		;m\_выделение
		;m\_разуплотнение
		;m\_отжимание
		;m\_отъединение
		;m\_затемнение
		;m\_шифрование
		;m\_дискредитирование
		;m\_дезавуирование
		;m\_выведение
		;m\_осаживание
		;m\_оттеснение
		;m\_отсоединение
		;m\_донимание
		;m\_зомбирование
		;m\_умопомрачение
		;m\_умалишение
		;m\_пробивание
		;m\_вздымание
		;m\_распускание
		;m\_аннигилирование
	}
	\scnaddlevel{-1}
}
\scnsubdividing{
	субъектно-объектное воздействие инициации\\
	\scnaddlevel{1}
	\scnsubdividing{
		m\_воспринимание
		;m\_притягивание
		;m\_перенимание
		;m\_вбирание
		;m\_прочувствование
		;m\_перевбирание
		;m\_отвергание
		;m\_выделение
		;m\_сообщение
		;m\_подведение
		;m\_объяснение
		;m\_введение
		;m\_ниспослание
		;m\_проведение
		;m\_затемнение
		;m\_выведение
		;m\_информирование
		;m\_затрагивание
		;m\_наставление
		;m\_вскрывание
		;m\_пронимание
		;m\_пронизывание
		;m\_донимание
		;m\_пробивание
		;m\_воспоминание
		;m\_рекристаллизирование
		;m\_репродуцирование
		;m\_рекуперирование
	}
	\scnaddlevel{-1}
	;субъектно-объектное воздействие аккумуляции\\
	\scnaddlevel{1}
	\scnsubdividing{
		m\_запоминание
		;m\_скапливание
		;m\_заучивание
		;m\_накапливание
		;m\_созерцание
		;m\_концентрирование
		;m\_изглаживание
		;m\_разуплотнение
		;m\_рекламирование
		;m\_наращивание
		;m\_пропагандирование
		;m\_нагнетание
		;m\_вещание
		;m\_распространение
		;m\_шифрование
		;m\_осаживание
		;m\_заинтересовывание
		;m\_обволакивание
		;m\_обучение
		;m\_наполнение
		;m\_преисполнение
		;m\_переполнение
		;m\_зомбирование
		;m\_вздымание
		;m\_воссоздание
		;m\_реинтегрирование
		;m\_рекультивирование
		;m\_реабилитирование
	}
	\scnaddlevel{-1}
	;субъектно-объектное воздействие амплификации\\
	\scnaddlevel{1}
	\scnsubdividing{
		m\_осмысливание
		;m\_ужимание
		;m\_обдумывание
		;m\_центрирование
		;m\_переживание
		;m\_центрифугирование
		;m\_переосмысливание
		;m\_отжимание
		;m\_внушение
		;m\_прижимание
		;m\_доказывание
		;m\_вжимание
		;m\_просветление
		;m\_выжимание
		;m\_дискредитирование
		;m\_оттеснение
		;m\_уверение
		;m\_обжимание
		;m\_убеждение
		;m\_сжимание
		;m\_преображение
		;m\_разжимание
		;m\_умопомрачение
		;m\_распускание
		;m\_возобновление
		;m\_регенерирование
		;m\_возрождение
		;m\_реактивирование
	}
	\scnaddlevel{-1}
	;субъектно-объектное воздействие генерации\\
	\scnaddlevel{1}
	\scnsubdividing{
		m\_понимание
		;m\_присоединение
		;m\_усваивание
		;m\_ассимилирование
		;m\_изведывание
		;m\_диссимилирование
		;m\_изживание
		;m\_отъединение
		;m\_констатирование
		;m\_подсоединение
		;m\_удостоверивание
		;m\_соединение
		;m\_явление
		;m\_разъединение
		;m\_дезавуирование
		;m\_отсоединение
		;m\_предрасположение
		;m\_формование
		;m\_воспитание
		;m\_формирование
		;m\_перевоплощение
		;m\_выхолащивание
		;m\_умалишение
		;m\_аннигилирование
		;m\_воспроизведение
		;m\_реформование
		;m\_воскрешение
		;m\_реанимирование
	}
	\scnaddlevel{-1}
}
\scnsubdividing{
	субъектно-объектное физическое воздействие\\
	\scnaddlevel{1}
	\scnexplanation{\textit{субъектно-объектное физическое воздействие} -- воздействие, в котором в роли инструмента выступает оболочка субъекта.}
	\scnsubdividing{
		m\_притягивание
		;m\_скапливание
		;m\_ужимание
		;m\_присоединение
		;m\_вбирание
		;m\_накапливание
		;m\_центрирование
		;m\_ассимилирование
		;m\_перевбирание
		;m\_концентрирование
		;m\_центрифугирование
		;m\_диссимилирование
		;m\_выделение
		;m\_разуплотнение
		;m\_отжимание
		;m\_отъединение
		;m\_подведение
		;m\_наращивание
		;m\_прижимание
		;m\_подсоединение
		;m\_введение
		;m\_нагнетание
		;m\_вжимание
		;m\_соединение
		;m\_проведение
		;m\_распространение
		;m\_выжимание
		;m\_разъединение
		;m\_выведение
		;m\_осаживание
		;m\_оттеснение
		;m\_отсоединение
		;m\_затрагивание
		;m\_обволакивание
		;m\_обжимание
		;m\_формование
		;m\_вскрывание
		;m\_наполнение
		;m\_сжимание
		;m\_формирование
		;m\_пронизывание
		;m\_переполнение
		;m\_разжимание
		;m\_выхолащивание
		;m\_пробивание
		;m\_вздымание
		;m\_распускание
		;m\_аннигилирование
		;m\_рекристаллизирование
		;m\_реинтегрирование
		;m\_регенерирование
		;m\_реформование
		;m\_рекуперирование
		;m\_реабилитирование
		;m\_реактивирование
		;m\_реактивирование
	}
	\scnaddlevel{-1}	
	;субъектно-объектное информационное воздействие\\
	\scnaddlevel{1}
	\scnexplanation{\textit{субъектно-объектное информационное воздействие} -- воздействие, в котором в роли инструмента выступает среда субъекта.}
	\scnsubdividing{
		m\_воспринимание
		;m\_запоминание
		;m\_осмысливание
		;m\_понимание
		;m\_перенимание
		;m\_заучивание
		;m\_обдумывание
		;m\_усваивание
		;m\_прочувствование
		;m\_созерцание
		;m\_переживание
		;m\_изведывание
		;m\_отвергание
		;m\_изглаживание
		;m\_переосмысливание
		;m\_изживание
		;m\_сообщение
		;m\_рекламирование
		;m\_внушение
		;m\_констатирование
		;m\_объяснение
		;m\_пропагандирование
		;m\_доказывание
		;m\_удостоверивание
		;m\_ниспослание
		;m\_вещание
		;m\_просветление
		;m\_явление
		;m\_затемнение
		;m\_шифрование
		;m\_дискредитирование
		;m\_дезавуирование
		;m\_информирование
		;m\_заинтересовывание
		;m\_уверение
		;m\_предрасположение
		;m\_наставление
		;m\_обучение
		;m\_убеждение
		;m\_воспитание
		;m\_пронимание
		;m\_преисполнение
		;m\_преображение
		;m\_перевоплощение
		;m\_донимание
		;m\_зомбирование
		;m\_умопомрачение
		;m\_умалишение
		;m\_воспоминание
		;m\_воссоздание
		;m\_возобновление
		;m\_воспроизведение
		;m\_репродуцирование
		;m\_рекультивирование
		;m\_возрождение
		;m\_воскрешение
	}
	\scnaddlevel{-1}
}

\scnheader{участник субъектно-объектного воздействия*}
\scnidtf{участник акции*}
\scniselement{неролевое отношение}
\scnrelfrom{первый домен}{индивид}
\scnrelfrom{второй домен}{субъектно-объектное воздействие}
\scnexplanation{\textit{Участник акции*} -- это неролевое отношение, которое связывает акцию с участвующим в ней индивидом.}
	\scnaddlevel{1}
	\scnrelfrom{источник}{\scncite{Hardzei2017}}
	\scnrelfrom{источник}{\scncite{Fillmore1977}}
	\scnrelfrom{источник}{\scncite{Fillmore1982}}
	\scnaddlevel{-1}
\scnsubdividing{
	субъект*\\
	\scnaddlevel{1}
	\scnexplanation{\textit{Субъект*} -- инициатор акции.}
	\scnsubdividing{
		инициатор*\\
		\scnaddlevel{1}
		\scnexplanation{\textit{Инициатор*} инициирует акцию.}
		\scnaddlevel{-1}
		;вдохновитель*\\
		\scnaddlevel{1}
		\scnexplanation{\textit{Вдохновитель*} вовлекает в акцию.}
		\scnaddlevel{-1}	
		;распространитель*\\
		\scnaddlevel{1}
		\scnexplanation{\textit{Распространитель*} распространяет акцию.}
		\scnaddlevel{-1}
		;вершитель*\\
		\scnaddlevel{1}
		\scnexplanation{\textit{Вершитель*} завершает акцию производством из объекта продукта.}
		\scnaddlevel{-1}
	}
	\scnaddlevel{-1}
	;инструмент*\\
	\scnaddlevel{1}
	\scnexplanation{\textit{Инструмент*} -- исполнитель акции.}
	\scnsubdividing{
		активатор*\\
		\scnaddlevel{1}
		\scnexplanation{\textit{Активатор*} непосредственно воздействует на медиатор.}
		\scnaddlevel{-1}
		;супрессор*\\
		\scnaddlevel{1}
		\scnexplanation{\textit{Супрессор*} подавляет сопротивление медиатора.}
		\scnaddlevel{-1}
		;усилитель*\\
		\scnaddlevel{1}
		\scnexplanation{\textit{Усилитель*} наращивает воздействие на медиатор.}
		\scnaddlevel{-1}
		;преобразователь*\\
		\scnaddlevel{1}
		\scnexplanation{\textit{Преобразователь*} преобразует медиатор.}
		\scnaddlevel{-1}
	}
	\scnaddlevel{-1}
	;медиатор*\\
	\scnaddlevel{1}
	\scnexplanation{\textit{Медиатор*} -- посредник акции.}
	\scnsubdividing{
		ориентир*\\
		\scnaddlevel{1}
		\scnexplanation{\textit{Ориентир*} ориентирует воздействие на объект.}
		\scnaddlevel{-1}
		;локус*\\
		\scnaddlevel{1}
		\scnexplanation{\textit{Локус*} локализует объект в пространстве.}
		\scnaddlevel{-1}
		;транспортёр*\\
		\scnaddlevel{1}
		\scnexplanation{\textit{Транспортёр*} перемещает объект.}
		\scnaddlevel{-1}
		;адаптер*\\
		\scnaddlevel{1}
		\scnexplanation{\textit{Адаптер*} приспосабливает  инструмент к воздействию на объект.}
		\scnaddlevel{-1}
		;материал*\\
		\scnaddlevel{1}
		\scnexplanation{\textit{Материал*} используется в качестве объекта-сырья для производства продукта.}
		\scnaddlevel{-1}
		;макет*\\
		\scnaddlevel{1}
		\scnexplanation{\textit{Макет*} является исходным образцом для производства из объекта продукта.}
		\scnaddlevel{-1}
		;фиксатор*\\
		\scnaddlevel{1}
		\scnexplanation{\textit{Фиксатор*} превращает переменный локус объекта в постоянный.}
		\scnaddlevel{-1}
		;ресурс*\\
		\scnaddlevel{1}
		\scnexplanation{\textit{Ресурс*} питает инструмент.}
		\scnaddlevel{-1}
		;стимул*\\
		\scnaddlevel{1}
		\scnexplanation{\textit{Стимул*} проявляет параметр объекта.}
		\scnaddlevel{-1}
		;регулятор*\\
		\scnaddlevel{1}
		\scnexplanation{\textit{Регулятор*} служит инструкцией в производстве из объекта продукта.}
		\scnaddlevel{-1}
		;хронотоп*\\
		\scnaddlevel{1}
		\scnexplanation{\textit{Хронотоп*} локализует объект во времени.}
		\scnaddlevel{-1}
		;источник*\\
		\scnaddlevel{1}
		\scnexplanation{\textit{Источник*} обеспечивает инструкциями инструмент.}
		\scnaddlevel{-1}
		;индикатор*\\
		\scnaddlevel{1}
		\scnexplanation{\textit{Индикатор*} отображает параметр воздействия на объект или параметр продукта как результата воздействия на объект.}
		\scnaddlevel{-1}
	}
	\scnaddlevel{-1}
	;объект*\\
	\scnaddlevel{1}
	\scnexplanation{\textit{Объект*} -- реципиент акции.}
	\scnsubdividing{
		покрытие*\\
		\scnaddlevel{1}
		\scnexplanation{\textit{Покрытие*} -- внешняя изоляция оболочки индивида.}
		\scnaddlevel{-1}
		;корпус*\\
		\scnaddlevel{1}
		\scnexplanation{\textit{Корпус*} -- оболочка индивида.}
		\scnaddlevel{-1}
		;прослойка*\\
		\scnaddlevel{1}
		\scnexplanation{\textit{Прослойка*} -- внутренняя изоляция оболочки индивида.}
		\scnaddlevel{-1}
		;сердцевина*\\
		\scnaddlevel{1}
		\scnexplanation{\textit{Сердцевина*} -- ядро индивида.}
		\scnaddlevel{-1}
	}
	\scnaddlevel{-1}
	;продукт*\\
	\scnaddlevel{1}
	\scnexplanation{\textit{Продукт*} -- результат воздействия субъекта на объект.}
	\scnsubdividing{
		заготовка*\\
		\scnaddlevel{1}
		\scnexplanation{\textit{Заготовка*} -- превращённый в сырьё объект.}
		\scnaddlevel{-1}
		;полуфабрикат*\\
		\scnaddlevel{1}
		\scnexplanation{\textit{Полуфабрикат*} -- наполовину изготовленный из сырья продукт.}
		\scnaddlevel{-1}
		;прототип*\\
		\scnaddlevel{1}
		\scnexplanation{\textit{Прототип*} -- опытный образец продукта.}
		\scnaddlevel{-1}
		;изделие*\\
		\scnaddlevel{1}
		\scnexplanation{\textit{Изделие*} -- готовый продукт.}
		\scnaddlevel{-1}
	}
	\scnaddlevel{-1}
}

\scnheader{Пример sc.g-текста, описывающего спецификацию действия}
\scneqscg{figures/sd_actions/tapaz_description_example.png}
\scniselement{sc.g-текст}
\scnexplanation{Представленный фрагмент базы знаний содержит декомпозицию действия на поддействия, указание принадлежности данного декомпозируемого и полученных в результате данной декомпозиции действий определенному их классу из приведенной выше классификации, а также указание участников данных действий.}
\scnexplanation{Представленный фрагмент базы знаний можно протранслировать в следующий текст естественного языка: <<Некто принимает молоко, затем окисляет молоко, а именно: нормализует молоко до 15-процентной жирности, затем очищает молоко, затем пастеризует молоко, затем охлаждает молоко до определённой температуры, затем вносит закваску в молоко, затем сквашивает молоко, затем режет сгусток, затем подогревает сгусток, затем обрабатывает сгусток, затем отделяет сыворотку, затем охлаждает сгусток и, в итоге, производит творог>>.}

\scnheader{субъект}
\scnidtftext{часто используемый sc-идентификатор}{индивид}
\scnidtf{активная сущность}
\scnidtf{сущность, способная самостоятельно выполнять некоторые виды действий}
\scnidtf{агент деятельности}
\scnsuperset{Собственное Я}
\scnsuperset{внутренний субъект ostis-системы}
\scnsuperset{внешний субъект ostis-системы, с которым осуществляется взаимодействие}
\scnsuperset{внешний субъект ostis-системы, с которым взаимодействие не происходит}

\scnheader{внутренний субъект ostis-системы}
\scnidtf{субъект, входящий в состав той \textit{ostis-системы, в базе знаний} которой он описывается}
\scnsuperset{sc-агент}
\scnexplanation{Под \textit{внутренним субъектом ostis-системы} понимается такой \textit{субъект}, который выполняет некоторые \textit{действия} в \uline{той же памяти}, в которой хранится его знак.
	\newline
	К числу \textit{внутренних субъектов ostis-системы} относятся входящие в нее \textit{sc-агенты}, частные sc-машины, целые интеллектуальные подсистемы.}

\scnheader{внешний субъект ostis-системы, с которым осуществляется взаимодействие}
\scnexplanation{К числу \textit{внешних субъектов ostis-системы, с которыми осуществляется взаимодействие}, относятся конечные пользователи \textit{ostis-системы}, ее разработчики, а также другие компьютерные системы(причем не только интеллектуальные).}

\scnheader{субъект действия\scnrolesign}
\scnsubset{субъект*}
\scnidtf{сущность, воздействующая на некоторую другую сущность в процессе заданного действия\scnrolesign}
\scnidtf{сущность, создающая \textit{причину} изменений другой сущности (объекта действия)\scnrolesign}
\scnidtf{быть субъектом данного действия\scnrolesign}
\scnsuperset{субъект неосознанного воздействия\scnrolesign}
\scnsuperset{субъект осознанного воздействия\scnrolesign}
\scnaddlevel{1}
\scnidtf{субъект целенаправленного, активного воздействия\scnrolesign}
\scnaddlevel{-1}

\scnheader{исполнитель*}
\scnexplanation{Связки отношения \textit{исполнитель*} связывают \textit{sc-элементы}, обозначающие \textit{действие} и \textit{sc-элементы}, обозначающие \textit{субъекта}, который предположительно будет осуществлять, осуществляет или осуществлял выполнение указанного \textit{действия}. Данное отношение может быть использовано при назначении конкретного исполнителя для проектной задачи по развитию баз знаний.
	
	В случае, когда заранее неизвестно, какой именно \textit{субъект*} будет исполнителем данного \textit{действия}, отношение \textit{исполнитель*} может отсутствовать в первоначальной формулировке \textit{задачи} и добавляться позже, уже непосредственно при исполнении.
	
	Когда действие выполняется (является \textit{настоящей сущностью}) или уже выполнено (является \textit{прошлой сущностью}), то исполнитель этого действия в каждый момент времени уже определён. Но когда действие только инициировано, тогда важно знать:
	\begin{enumerate}
		\item кто \uline{хочет} выполнить это действие и насколько важно для него стать исполнителем данного действия;	
		\item кто \uline{может} выполнить данное действие и каков уровень его квалификации и опыта;
		\item кто и кому поручает выполнить это действие и каков уровень ответственности за невыполнение (приказ, заказ, официальный договор, просьба...)
	\end{enumerate}	
	При этом следует помнить, что связь отношения \textit{исполнитель*} в данном случае также является временной прогнозируемой сущностью.
	
	Первым компонентом связок отношений \textit{исполнитель*} является знак \textit{действия}, вторым -- знак \textit{субъекта} исполнителя}

\scnheader{объект воздействия\scnrolesign}
\scnsubset{объект*}
\scnidtf{сущность, на которую осуществляется воздействие в рамках заданного действия\scnrolesign}
\scnidtf{сущность, являющаяся в рамках заданного действия исходным условием (аргументом), необходимым для выполнения этого действия\scnrolesign}
\scnnote{Для разных действий количество объектов действий может быть различным.}
\scnnote{Поскольку действие является процессом и, соответственно, представляет собой \textit{динамическую структуру}, то и знак \textit{субъекта действия\scnrolesign}, и знак \textit{объекта действия\scnrolesign} являются элементами данной sc-структуры. В связи с этим можно рассматривать отношения \textit{субъект действия\scnrolesign} и \textit{объект действия\scnrolesign} как \textit{ролевые отношения}. Данный факт не  запрещает вводить аналогичные \textit{неролевые отношения}, однако это нецелесообразно.}

\scnheader{продукт*}
\scnidtf{быть продуктом заданного действия*}
\scnsubset{результат*}
\scnidtf{"сухой"{} остаток*}
\scnidtf{то, ради чего может быть выполнено, выполняется или будет выполняться заданное действие*}
\scnnote{Продуктом действия может быть некоторая материальная сущность, некоторое множество (тираж) одинаковых материальных сущностей, некоторая информационная конструкция}

\scnheader{результат*}
\scnexplanation{Связки отношения \textit{результат*} связывают \textit{sc-элемент}, обозначающий \textit{действие}, и \textit{sc-конструкцию}, описывающую результат выполнения рассматриваемого действия, другими словами, цель, которая должна быть достигнута при выполнении \textit{действия}.
	
	Результат может специфицироваться как атомарным высказыванием, так и неатомарным, т.е. конъюнктивным, дизъюнктивным, строго дизъюнктивным и т.д.
	
	В случае, когда успешное выполнение \textit{действия} приводит к изменению какой-либо конструкции в \textit{sc-памяти}, которое необходимо занести в историю изменений базы знаний или использовать для демонстрации протокола решении задачи, генерируется соответствующая связка отношения \textit{результат*}, связывающая задачу и \textit{sc-конструкцию}, описывающую данное изменение. Конкретный вид указанной \textit{sc-конструкции} зависит от типа действия.}
\scnrelboth{следует отличать}{цель*}
\scnaddlevel{1}
\scnidtf{спецификация планируемого результата*}
\scnnote{Следует также отмечать то, что является непосредственно результатом (продуктом) некоторого действия, и то, что является предварительной (исходной, стартовой) спецификацией этого  результата. Далеко не всегда результатом действия является именно то, что планировалось (было целью) изначально.}
\scnaddlevel{-1}

\scnheader{класс выполняемых действий*}
\scnidtf{класс действий, выполняемых классом субъектов*}
\scnexplanation{Связки отношения \textit{класс выполняемых действий*} связывают классы \textit{субъектов} и классы действий, при этом предполагается, что каждый субъект указанного класса способен выполнять действия указанного класса действий.}

\scnheader{заказчик*}
\scnexplanation{Связки отношения \textit{заказчик*} связывают классы \textit{sc-элементы}, обозначающие \textit{действие}, и \textit{sc-элементы}, обозначающие  \textit{субъекта}, который "заинтересован"{} в выполнении данного действия и, как правило, инициирует его выполнение. Данное отношение может быть использовано при указании того, кто поставил проектную задачу по развитию баз знаний.
	
	Первым компонентов связок отношения \textit{заказчик*} является знак \textit{действия}, вторым -- знак \textit{субъекта}.}

\scnheader{инициатор*}
\scnexplanation{Связки отношения \textit{инициатор*} связывают \textit{sc-элемент}, обозначающий \textit{инициированное действие}, и знак \textit{субъекта}, который является инициатором данного \textit{действия}, то есть \textit{субъектом}, который инициировал данное \textit{действие} и, как правило, заинтересован в его успешном выполнении.}

\scnheader{объект\scnrolesign}
\scnsubset{объект*}
\scnidtf{аргумент действия\scnrolesign}
\scnexplanation{Связки отношения \textit{объект\scnrolesign} связывают \textit{sc-элемент}, обозначающий \textit{действие}, и знак той сущности, над которой (по отношению к которой) осуществляется данное \textit{действие}, и, например, знак \textit{структуры}, подлежащий верификации.}
\scnsuperset{первый аргумент действия\scnrolesign}
\scnsuperset{второй аргумент действия\scnrolesign}
\scnsuperset{третий аргумент действия\scnrolesign}

\scnheader{класс аргументов*}
\scnidtf{класс аргументов класса команд*}
\scnidtf{быть классом sc-элементов, экземпляры которого являются аргументами для заданного класса команд*}
\scnsuperset{класс первых аргументов*}
\scnsuperset{класс вторых аргументов*}
\scnexplanation{Связки отношения \textit{класс аргументов*} связывают \textit{классы команд} (подмножества множества \textit{команд}) и классы \textit{sc-элементов}, которые могут быть аргументами действий, соответствующих данному  \textit{классу команд}. В случае, когда  \textit{команды} данного класса имеют один аргумент, используется собственно отношение  \textit{класс аргументов*}, в случае, когда команды данного класса имеют более одного аргумента, то используются подмножества данного отношения, такие как \textit{класс первых аргументов*}, \textit{класс вторых аргументов*} и т.д.
	
	Если для некоторого \textit{класса команд} не указан тип какого-либо из аргументов, то предполагается, что в качестве данного аргумента может выступать любой \textit{sc-элемент}.
	
	Первым компонентом связок отношения \textit{класс аргументов*} является знак \textit{класса команд}, вторым -- знак класса \textit{sc-элемента}, которые могут быть \textit{аргументами действий\scnrolesign}, соответствующих данному \textit{классу команд}.}

\bigskip

\scnendstruct \scnendsegmentcomment{Предметная область и онтология субъектно-объектных спецификаций воздействий}

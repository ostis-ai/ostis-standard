\scnsegmentheader{Уточнение понятия задачи. Типология задач}

\scnstartsubstruct

\scniselement{сегмент базы знаний}

\scnheader{задача}
\scnidtf{описание желаемого состояния или события в рамках внешней среды ктбернетической системы либо в рамках её базы знаний}
\scnidtf{формулировка задачи}
\scnidtf{задание на выполнение некоторого действия}
\scnidtf{постановка задачи}
\scnidtf{описание задачной ситуации}
\scnidtf{спецификация некоторого действия, обладающая достаточной полнотой для выполнения этого действия}
\scnidtf{цель плюс дополнительные условия (ограничения) накладываемые на результат или процесс получения этого результата}
\scnidtf{описание того, что требуется сделать}
\scnexplanation{\textbf{\textit{Задача}}, т.е. формальное описание условия некоторой задачи есть, по сути, формальная \textit{спецификация} некоторого \textit{действия}, направленного на решение данной \textit{задачи}, достаточная для выполнения данного \textit{действия} каким-либо \textit{субъектом}. В зависимости от конкретного \textit{класса задач}, описываться может как внутреннее состояние самой интеллектуальной системы, так и требуемое состояние \textit{внешней среды}. \textit{sc-элемент}, обозначающий \textit{действие} входит в \textit{задачу} под атрибутом \textit{ключевой знак\scnrolesign}.
	
	Каждая \textbf{\textit{задача}} представляет собой спецификацию \textit{действия}, которое либо уже выполнено, либо выполняется в текущий момент (в настоящее время), либо планируется (должно) быть выполненным, либо может быть выполнено (но не обязательно).
	
	Классификация \textit{задач} может осуществляться по дидактическому признаку в рамках каждой предметной области, например, задачи на треугольники, задачи на системы уравнений и т.п.
	
	Каждая \textit{задача} может включать:
	\begin{scnitemize}
		\item факт принадлежности \textit{действия} какому-либо частному классу \textit{действий} (например,\textit{ действие. сформировать полную семантическую окрестность указываемой сущности}), в том числе состояние \textit{действия} с точки зрения жизненного цикла (инициированное, выполняемое и т.д.);
		\item описание \textit{цели*} (\textit{результата*}) \textit{действия}, если она точно известна;
		\item указание \textit{заказчика*} действия;
		\item указание \textit{исполнителя* действия} (в том числе, коллективного);
		\item указание \textit{аргумента(ов) действия\scnrolesign};
		\item указание инструмента или посредника \textit{действия};
		\item описание \textit{декомпозиции действия*};
		\item указание \textit{последовательности действий*} в рамках \textit{декомпозиции действия*}, т.е построение плана решения задачи. Другими словами, построение плана решения представляет собой декомпозицию соответствующего \textit{действия} на систему взаимосвязанных между собой поддействий;
		\item указание области \textit{действия};
		\item указание условия инициирования \textit{действия};
		\item момент начала и завершения \textit{действия}, в том числе планируемый и фактический, предполагаемая и/или фактическая длительность выполнения;
	\end{scnitemize}
	Некоторые \textit{задачи} могут быть дополнительно уточнены контекстом -- дополнительной информацией о сущностях, рассматриваемых в формулировке \textit{задачи}, т.е. описанием того, что дано, что известно об указанных сущностях.
	
	Кроме этого, \textit{задача} может включать любую дополнительную информацию о действии, например:
	\begin{scnitemize}
		\item перечень ресурсов и средств, которые предполагается использовать при решении задачи, например список доступных исполнителей, временные сроки, объем имеющихся финансов и т.д.;
		\item ограничение области, в которой выполняется \textit{действие}, например, необходимо заменить одну \textit{sc-конструкцию} на другую по некоторому правилу, но только в пределах некоторого \textit{раздела базы знаний};
		\item ограничение знаний, которые можно использовать для решения той или иной задачи, например, необходимо решить задачу по алгебре используя только те утверждения, которые входят в курс школьной программы до седьмого класса включительно, и не используя утверждения, изучаемые в старших классах;
		\item и прочее
	\end{scnitemize}
	С одной стороны, решаемые системой \textit{задачи}, можно классифицировать на \textit{информационные задачи} и \textit{поведенческие задачи}.
	
	С точки зрения формулировки поставленной задачи можно выделить \textit{декларативные формулировки задачи} и \textit{процедурные формулировки задачи}. Следует отметить, что данные классы задач не противопоставляются, и могут существовать формулировки задач, использующие оба подхода.}
\scntext{правило идентификации экземпляров}{Экземпляры класса \textbf{\textit{задач}} в рамках \textit{Русского языка} именуются по следующим правилам:
	\begin{scnitemize}
		\item в начале идентификатора пишется слово \textbf{"Задача"} и ставится точка;
		\item далее с прописной буквы идет либо содержащее глагол совершенного вида в инфинитиве описание сути того, что требуется получить в результате выполнения действия, либо вопросительное предложение, являющееся спецификацией запрашиваемой (ответной) информации.
	\end{scnitemize}
	Например:\\
	\textit{Задача. Сформировать полную семантическую окрестность понятия треугольник}\\
	\textit{Задача. Верифицировать Раздел. Предметная область sc-элементов}}
\scnsubset{семантическая окрестность}
\scnsuperset{процедурная формулировка задачи}
\scnsuperset{декларативная формулировка задачи}
\scnsuperset{вопрос}
\scnsuperset{команда}

\scnidtf{спецификация действия, которое выполнилось, выполняется или может быть выполнено соответствующей кибернетической системой}
\scnnote{Каждой задаче и, соответственно, каждому специфицируемому действию соответствует определенная кибернетическая система, являющаяся субъектом, выполняющим это действие.}
\scnsubset{знание}
\scnnote{Каждая \textit{задача} -- это \textit{знание}, описывающее то какое действие возможно потребуется выполнить.}
\scnsuperset{инициированная задача}
\scnaddlevel{1}
\scnidtf{формулировка задачи, которая подлежит выполнению}
\scnaddlevel{-1}
\scnidtf{спецификация (описание) соответствующего действия}
\scnsuperset{декларативная формулировка задачи}
\scnaddlevel{1}
\scnidtf{задача, в формулировке которой явно указывается (описывается) целевая ситуация, т.е. то, что является результатом выполнения (решения) данной задачи}
\scnaddlevel{-1}
\scnsuperset{процедурная формулировка задачи}
\scnaddlevel{1}
\scnidtf{задача, в формулировке которой явно указывается характеристика действия, специфицируемого этой задачей, а именно, например, указывается:
	\begin{scnitemize}
		\item субъект или субъекты, выполняющие это действие,
		\item объекты, над которыми действие выполняется, -- аргументы действия,
		\item инструменты, с помощью которых выполняется действие,
		\item момент и, возможно, дополнительные условия начала и завершения выполнения действия
\end{scnitemize}}
\scnaddlevel{-1}
\scnsuperset{декларативно-процедурная формулировка задачи}
\scnaddlevel{1}
\scnidtf{задача, в формулировке которой присутствуют как декларативные (целевые), так и процедурные аспекты}
\scnaddlevel{-1}
\scnnote{От качества (корректности и полноты) формулировки задачи, т.е. спецификации соответствующего действия, во многом зависит качество (эффективность) выполнения этого действия, т.е. качество процесса решения указанной задачи.}
\scnsuperset{проблема}
\scnaddlevel{1}
\scnidtf{проблемная задача}
\scnidtf{сложная, трудно решаемая задача}
\scnsuperset{изобретательская задача}
\scnaddlevel{-1}

\scnheader{процедурная формулировка задачи}
\scnidtf{спецификация действия, которое планируется быть выполненным}
\scnexplanation{В случае \textbf{\textit{процедурной формулировки задачи}}, в формулировке задачи явно указываются аргументы соответствующего задаче \textit{действия}, и в частности, вводится семантическая типология \textit{действий}. При этом явно не уточняется, что должно быть результатом выполнения данного действия. Заметим, что, при необходимости, \textit{процедурная формулировка задачи} может быть сведена к \textit{декларативной формулировке задачи} путем трансляции на основе некоторого правила, например определения класса действия через более общий класс.}

\scnhaselementrole{пример}{\scnfilescg{figures/sd_task/declarative_task_statement.png}}
\scnaddlevel{1}
\scnnote{Выполнение данного действия сведется к следующим \uline{событиям}:
	\begin{scnitemize}
		\item для числа \textit{с} будет сгенерирован уникальный идентификатор, являющийся его представлением в соответствующей системе счисления
		\item будет сгенерирована константная настоящая позитивная пара принадлежности, соединяющая узел "\textit{вычислено}"{} с узлом "\textit{с}"{}
		\item удалится константная будущая позитивная пара принадлежности, а также константная настоящая нечеткая пара принадлежности, выходящие из узла "\textit{вычислено}".
	\end{scnitemize}
	Таким образом, после выполнения действия \uline{все} \uline{будущие} сущности, входящие в целевую ситуацию, становятся \uline{настоящими} сущностями, а некоторые \uline{настоящие} сущности, входящие в исходную ситуацию, становятся \uline{прошлыми}.}
\scnaddlevel{-1}

\scnheader{задача}
\scnsuperset{задача, решаемая в памяти кибернетической системы}
\scnaddlevel{1}
\scnsuperset{задача, решаемая в памяти индивидуальной кибернетической системы}
\scnsuperset{задача, решаемая в общей памяти многоагентной системы}
\scnidtf{информационная задача}
\scnidtf{задача, направленная либо на \uline{генерацию} или поиск информации, удовлетворяющей заданным требованиям, либо на некоторое \uline{преобразование} заданной информации}
\scnsuperset{математическая  задача}
\scnaddlevel{-1}
\scnsuperset{элементарная информационная задача}
\scnsuperset{простая информационная задача}
\scnsuperset{проблемная информационная задача}
\scnaddlevel{1}
\scnidtf{интеллектуальная информационная задача}
\scnsuperset{проблема Гильберта}
\scnaddlevel{-1}

\scnheader{вопрос}
\scnidtf{запрос}
\scnsubset{задача, решаемая в памяти кибернетической системы}
\scnidtf{непроцедурная формулировка задачи на поиск (в текущем состоянии базы знаний) или на генерацию знания, удовлетворяющего заданным требованиям}
\scnsuperset{вопрос -- что это такое}
\scnsuperset{вопрос -- почему}
\scnsuperset{вопрос -- зачем}
\scnsuperset{вопрос -- как}
\scnaddlevel{1}
\scnidtf{каким способом}
\scnidtf{запрос метода (способа) решения заданного (указываемого) вида задач или класса задач либо, плана решения конкретной указываемой задачи}
\scnaddlevel{-1}
\scnidtf{задача, направленная на удовлетворение информационной потребности некоторого субъекта-заказчика}

\scnheader{команда}
\scnidtf{инициированная задача}
\scnidtf{спецификация инициированного действия}
\scnexplanation{Идентификатор экземпляров конкретного класса \textbf{\textit{команд}} в рамках \textit{Русского языка} пишется с прописной буквы и представляет собой либо содержащее глагол совершенного вида в инфинитиве описание сути того, что требуется получить в результате выполнения действия, соответствующего данной \textbf{\textit{команде}}, либо вопросительное предложение, являющееся спецификацией запрашиваемой (ответной) информации.
	
	Например:\\
	\textit{Сформировать полную семантическую окрестность понятия треугольник}\\
	\textit{Верифицировать Раздел. Предметная область sc-элементов}
}

\scnheader{задача}
\scnnote{Сужение бинарного ориентированного отношения \textit{спецификация*} (быть спецификацией*), связывающее \textit{действия} с \textit{задачами}, которые решаются в результате выполнения этих \textit{действий}, не является взаимно однозначным.
Каждому \textit{действию} может соответствовать несколько формулировок \textit{задач}, которые с разной степенью детализации или с разных аспектов специфицируют указанное \textit{действие}.
Кроме того, интерпретация \uline{разных} формулировок семантически одной и той же \textit{задачи} в общем случае приводит к \uline{разным} \textit{действиям}, решающим эту \textit{задачу}.
Подчеркнем, что \uline{разные}, но семантически эквивалентные формулировки \textit{задач} считаются формулировками формально \uline{разных} \textit{задач}.}

\scnheader{отношение, заданное на множестве*(задача)}
\scnhaselement{\scnkeyword{задача}*}
\scnaddlevel{1}
\scniselement{неосновное понятие}
\scnidtf{формулировка задачи*}
\scnidtf{спецификация действия, уточняющая то, \uline{что} должно быть сделано*}
\scnsubdividing{декларативная формулировка задачи*;процедурная формулировка задачи*}
\scnrelfrom{второй домен}{\scnkeyword{задача}}
\scnaddlevel{1}
\scniselement{основное понятие}
\scnsuperset{задача обработки базы знаний}
\scnsuperset{задача обработки файлов}
\scnsuperset{задача, решаемая кибернетической системой во внешней среде}
\scnsuperset{задача, решаемая кибернетической системой в собственной физической оболочке}
\scnaddlevel{-1}
\scnaddlevel{-1}

\scnhaselement{\scnkeyword{декларативная формулировка задачи}*}
\scnaddlevel{1}
\scniselement{неосновное понятие}

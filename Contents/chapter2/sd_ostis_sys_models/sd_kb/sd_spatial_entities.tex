\begin{SCn}
	
	\scnsectionheader{\currentname}
	
	\scnstartsubstruct
	
	\scniselement{атомарный раздел}
	
	\scnheader{Предметная область пространственных сущностей и их форм}
	\scniselement{предметная область}
	
	\scnheader{пространственная сущность}
	\scnexplanation{\textbf{\textit{Пространственная сущность}} - любой класс \textit{материальных объектов}, временно находящихся в определенном положении в пространстве.}
	
	\scnheader{форма*}
	\scnrelfrom{первый домен}{пространственная сущность}
	
	\scnheader{система координат}
	\scniselement{пространственная сущность}
	\scnexplanation{\textbf{\textit{Система координат}} - \textit{величины}, определяющие положение \textit{точки} на \textit{плоскости} и в пространстве.}
	
	\scnheader{декартова система координат}
	\scnidtf {прямоугольная система координат}
	\scnsubset{система координат} 
	
	\scnheader{двумерная декартова система координат}
	\scnsubset{декартова система координат}

	\scnheader{трехмерная декартова система координат}	
	\scnsubset{декартова система координат}
	
	\scnheader{начало отсчёта*}
	\scnrelfrom{первый домен}{система координат}
	\scnrelfrom{второй домен}{точка}
	
	\scnheader{точка}
	\scniselement{пространственная сущность}
	\scnexplanation{\textbf{\textit{Точка}} - это неделимый элемент соответствующего математического пространства, определяемого в геометрии, математическом анализе и других разделах математики, не имеющий никаких измеримых характеристик, кроме координат.}
	
	\scnheader{прямая}
	\scniselement{пространственная сущность}	
	\scnrelfrom{включение}{точка}
	\scnexplanation{\textbf{\textit{Прямая}} - это линия, не имеющая неровностей, скруглений и углов, а также являющаяся бесконечной, не имеющей ни начала, ни конца.}
	
	\scnheader{отрезок}
	\scniselement{пространственная сущность}	
	\scnrelfrom{включение}{точка}
	\scnexplanation{\textbf{\textit{Отрезок}} - это множество, состоящее из двух различных \textit{точек} данной \textit{прямой} (которые называются концами \textbf{\textit{отрезка}}) и всех \textit{точек}, лежащих между ними.}
	
	\scnheader{плоскость}
	\scniselement{пространственная сущность}
	\scnexplanation{\textbf{\textit{Плоскость}} - это бесконечная поверхность, к которой принадлежат все \textit{прямые}, проходящие через какие-либо две \textit{точки} \textbf{\textit{плоскости}}.}
	
	\scnheader{длина}
	\scniselement{измеряемый параметр}
	
	\scnheader{расстояние*}
	\scniselement{квазибинарное отношение}
	\scnrelfrom{второй домен}{длина}
	
	\scnheader{толщина*}
	\scniselement{бинарное отношение}
	\scnrelfrom{первый домен}{пространственная сущность}
	\scnrelfrom{второй домен}{отрезок}
	
	\scnheader{высота*}
	\scniselement{бинарное отношение}
	\scnrelfrom{первый домен}{пространственная сущность}
	\scnrelfrom{второй домен}{отрезок}
	
	\scnheader{ширина*}
	\scniselement{бинарное отношение}
	\scnrelfrom{первый домен}{пространственная сущность}
	\scnrelfrom{второй домен}{отрезок}
	
	\scnheader{длина*}
	\scniselement{бинарное отношение}
	\scnrelfrom{первый домен}{пространственная сущность}
	\scnrelfrom{второй домен}{отрезок}
	
	\scnheader{диаметр*}
	\scniselement{бинарное отношение}
	\scnrelfrom{первый домен}{пространственная сущность}
	\scnrelfrom{второй домен}{отрезок}
	
	\scnheader{высотная отметка*}
	\scniselement{бинарное отношение}
	\scnrelfrom{первый домен}{пространственная сущность}
	\scnrelfrom{второй домен}{точка}
	
	\scnheader{начальная точка*}
	\scniselement{бинарное отношение}
	\scnrelfrom{первый домен}{пространственная сущность}
	\scnrelfrom{второй домен}{точка}
	
	\scnheader{конечная точка*}
	\scniselement{бинарное отношение}
	\scnrelfrom{первый домен}{пространственная сущность}
	\scnrelfrom{второй домен}{точка}
	
	\scnheader{уровень}
	\scniselement{измеряемый параметр}
	
	\scnheader{абсцисса}
	\scniselement{измеряемый параметр}
	
	\scnheader{ордината}
	\scniselement{измеряемый параметр}
	
	\scnheader{прямоугольник}
	\scnsubset{четырехугольник}
	\scnexplanation{\textbf{\textit{Прямоугольник}} — \textit{четырёхугольник}, у которого все углы прямые (равны 90 градусам). Данная геометрическая фигура состоит из четырех \textit{точек}, которые соединены между собой двумя парами равных \textit{отрезков}, перпендикулярно пересекающихся только в этих \textit{точках}. Прямоугольник обладает следующими свойствами:
		\begin{scnitemize}
			\item прямоугольник является параллелограммом — его противоположные стороны попарно параллельны;
			\item диагонали любого прямоугольника равны; 
			\item стороны прямоугольника являются его \textit{высотами}. Середины сторон прямоугольника образуют ромб;
			\item квадрат диагонали прямоугольника равен сумме квадратов двух его смежных сторон (по теореме Пифагора);
			\item около любого прямоугольника можно описать окружность, причём диагональ прямоугольника равна диаметру описанной окружности (радиус равен полудиагонали).
		\end{scnitemize}
	}
	
	\scnheader{граничная точка*}
	\scniselement{бинарное отношение}
	\scnrelfrom{первый домен}{отрезок}
	\scnrelfrom{второй домен}{точка}
	
	\scnheader{вершина*}
	\scniselement{бинарное отношение}
	\scnrelfrom{второй домен}{точка}
	
	\scnheader{отверстие*}
	\scniselement{бинарное отношение}
	\scnrelfrom{первый домен}{пространственная сущность}
	\scnrelfrom{второй домен}{отверстие}
	
	\scnheader{отверстие}
	\scniselement{пространственная сущность}
	\scnexplanation{\textbf{\textit{Отверстие}} - это полость в каком-либо предмете, обладающая ярковыраженными геометрическими свойствами. Обычно отверстие - это полость в виде цилиндра, сформированная (обычно) характеристиками вращающегося режущего объект инструмента.}
	
	\scnheader{сечение*}
	\scniselement{бинарное отношение}
	\scnrelfrom{первый домен}{пространственная сущность}
	\scnexplanation{Связки \textit{отношения} \textbf{\textit{сечение*}} связывают некоторую \textit{пространственную сущность} и замкнутую геометрическую фигуру, которая (1) лежит в некоторой \textit{плоскости}, имеющей общие точки с \textit{формой} данной \textit{пространственной сущности} и (2) граница которой включается в границу \textit{формы} данной \textit{пространственной сущности}.}
	
	\scnheader{горизонтальное сечение*}
	\scnsubset{сечение*}
	\scnexplanation{Связки \textit{отношения} \textbf{\textit{горизонтальное сечение*}} связывают некоторую \textit{пространственную сущность} и ее \textit{сечение}, которое лежит в плоскости, перпендикулярной отрезку, являющемуся \textit{высотой} для данной пространственной сущности.}
	
\end{SCn}
\begin{SCn}
\scnsectionheader{\currentname}

\scnstartsubstruct

\scnheader{Предметная область библиотек многократно используемых компонентов ostis-систем}
\scniselement{предметная область}
\scnsdmainclasssingle{библиотека многократно используемых компонентов ostis-систем}
\scnhaselementlist{понятие, используемое в предметной области}{материнская ostis-система;дочерняя ostis-система}


\scnheader{библиотека многократно используемых компонентов ostis-систем}
\scnidtf{библиотека компонентов ostis-систем, многократно используемых в разных ostis-системах}
\scnidtf{библиотека многократно используемых компонентов OSTIS}
\scnhaselement{\textbf{Библиотека IMS.ostis}}
\scnaddlevel{1}
\scnidtf{библиотека многократно используемых компонентов ostis-систем в составе Метасистемы IMS.ostis}
\scnaddlevel{-1}
\scnnote{Разработчики любой ostis-системы могут включить в ее состав библиотеку, которая позволит им накапливать и распространять результаты своей деятельности среди других участников Экосистемы OSTIS в виде многократно используемых компонентов.}
\scnrelfromset{функциональные возможности}{
\scnfileitem{Хранение многократно используемых компонентов ostis-систем и их спецификаций.}
	\scnaddlevel{1}
	\scnrelfrom{смотрите}{\nameref{sd_reusable_component}}
	\scnnote{При этом часть компонентов, специфицированных в рамках библиотеки, могут физически храниться в другом месте ввиду особенностей их  технической реализации (например, исходные тексты платформы интерпретации sc-моделей компьютерных систем могут физически храниться в каком-либо отдельном репозитории, но специфицированы как компонент будут в соответствующей библиотеке). В этом случае спецификация компонента в рамках библиотеки должна также включать описание (1) того где располагается компонент и (2) сценария его автоматической или хотя бы ручной установки в ostis-систему-потребителя. При этом спецификация компонента хранится как непосредственно рядом с компонентом (в виде исходных текстов или в той же самой базе знаний), так и дублируется в рамках библиотеки. Соответственно, существует процедура публикации спецификации компонента в библиотеке и последующая процедура синхронизации обновленной спецификации компонента с библиотекой.}
	\scnaddlevel{-1}
;\scnfileitem{Хранение сведений о совместимости/несовместимости имеющихся в библиотеке компонентов с учетом версий.}
;\scnfileitem{Возможность осуществлять просмотр имеющихся компонентов и их спецификаций, а также поиска компонентов по фрагментам их спецификации.}}}
\scnsubdividing{библиотека типовых подсистем ostis-систем;библиотека шаблонов типовых компонентов ostis-систем;библиотека платформ интерпретации sc-моделей компьютерных систем;библиотека многократно используемых компонентов баз знаний; библиотека многократно используемых компонентов решателей задач;библиотека многократно используемых компонентов интерфейсов}
\scnrelfromset{обобщенная декомпозиция}{база знаний библиотеки многократно используемых компонентов ostis-систем \\
	\scnaddlevel{1}
	\scnnote{База знаний библиотеки мнoгократно используемых компонентов ostis-систем представляет собой иерархию многократно используемых компонентов ostis-систем и их спецификацию.}
	\scnaddlevel{-1}    
	;решатель задач библиотеки многократно используемых компонентов ostis-систем\\
	\scnaddlevel{1}
	\scnrelfromset{функциональные возможности}{
		\scnfileitem{Систематизация многократно используемых компонентов ostis-систем.}
	;\scnfileitem{Обеспечение версионирования многократно используемых компонентов ostis-систем.}
	;\scnfileitem{Поиск зависимостей и конфликтов между многократно используемыми компонентами в рамках библиотеки компонентов.}
	;\scnfileitem{Формирование отдельных фрагментов многократно используемых компонентов ostis-систем.}}
	\scnaddlevel{-1}
	;интерфейс библиотеки многократно используемых компонентов ostis-систем\\
	\scnaddlevel{1}
	\scnnote{Интерфейс обеспечивает доступ к многократно используемым компонентам. Позволяет получить информацию о зависимых, конфликтующих компонентах.}
	\scnrelfromset{декомпозиция}{минимальный интерфейс библиотеки многократно используемых компонентов ostis-систем\\
		\scnaddlevel{1}
		\scnnote{Данный вид интерфейса позволят подключиться к библиотеке многократно используемых компонентов баз знаний, получить доступ к хранящимся там компонентам и использовать функционал библиотеки.}
		\scnaddlevel{-1}
		;расширенный интерфейс библиотеки многократно используемых компонентов ostis-систем
		\scnaddlevel{1}
		\scnidtf{графический интерфейс библиотеки многократно используемых компонентов ostis-систем}
		\scnnote{В частном случае у библиотеки может быть расширенный пользовательский интерфейс, который, в отличие от минимального интерфейса, позволяет не только получить доступ к компонентам для дальнейшей работы с ними, но и просматривать существующую структуру библиотеки,  а также компоненты и их элементы в удобном и интуитивно понятном для пользователя виде.}
		\scnaddlevel{-1}}
	\scnaddlevel{-1}}

\scnheader{ostis-система}
\scnsuperset{материнская ostis-система}
\scnaddlevel{1}
\scnexplanation{ostis-система, имеющая в своем составе библиотеку многократно используемых компонентов.}
\scnaddlevel{1}
\scnexplanation{ostis-система, компоненты которой входят в состав библиотеки многократно используемых компонентов.}
\scnaddlevel{-1}
\scnhaselement{Метасистема IMS.ostis}
\scnnote{Материнская ostis-система в свою очередь может являться дочерней ostis-системой для какой-либо другой ostis-системы, заимствуя компоненты из библиотеки, входящей в состав этой другой ostis-системы.}
\scnaddlevel{-1}
\scnsuperset{дочерняя ostis-система}
\scnaddlevel{1}
\scnexplanation{ostis-система, в составе которой имеется компонент, заимствованный из какой-либо библиотеки многократно используемых компонентов.}
\scnaddlevel{-1}

%\scnheader{Решатель задач библиотеки многократно используемых компонентов баз знаний}
%\scnrelfromset{декомпозиция абстрактного sc-агента}{
%	Неатомарный агент поиска компонента\\
%	\scnaddlevel{1}
%	\scnexplanation{Множество агентов, обеспечивающих поиск компонентов в рамках библиотеки по определенным критериям}
%	\scnnote{Существующие критерии регламентированы спецификацией многократно используемых компонентов}
%	\scnaddlevel{-1}
%	;Неатомарный агент формирования фрагментов компонента\\
%	\scnaddlevel{1}
%	\scnexplanation{Множество агентов, позволяющих формировать фрагменты компонентов по заданным критериям, обеспечивая возможность использование только тех знаний, которые непосредственно нужны для функционирования интеллектуальной системы.}
%	\scnrelfromset{декомпозиция абстрактного sc-агента}{
%		Агент формирования компонента по семантической окрестности заданного понятия;Агент формирования компонента по неатомарным компонентам}
%	\scnaddlevel{-1}
%	;Агент поиска зависимостей
%	\scnaddlevel{1}
%	\scnidtf{Агент поиска всех зависимостей, без которых использование запрашиваемого компонента невозможно}
%	\scnaddlevel{-1}
%	;Агент поиска конфликтов между компонентами
%	\scnaddlevel{1}
%	\scnidtf{Агент проверки отсутствия/присутствия конфликтов между установленным и устанавливаемым компонентами}
%	\scnaddlevel{-1}
%	;Неатомарный агент версионирования\\
%	\scnaddlevel{1}
%	\scnexplanation{Множество агентов, решающих задачу версионирования фрагментов БЗ. Данные агенты позволяют формировать начальное состояние многократно используемого компонента и интегрировать последующие изменения компонента в существую структуру. Затем по запросу пользователя возвращать состояние данной структуры на определенный промежуток времени, что разрешает использование в разработках интеллектуальных систем различных версий одного и того же компонента базы знаний}
%	\scnrelfromset{декомпозиция абстрактного sc-агента}{
%		Агент формирования начального состояния в дереве
%		состояний фрагмента БЗ
%		;Агент интеграции изменений с текущим состоянием
%		фрагмента БЗ;Агент воссоздания версии фрагмента БЗ по его заданному состоянию;Агент идентификации состояний в дереве состояний
%		заданного фрагмента БЗ; Агент получения состояния по его идентификатору}
%	\scnaddlevel{-1}
%	;Агент спецификации компонента
%	\scnaddlevel{1}
%	\scnidtf{Агент, позволяющий сформировать спецификацию разрабатываемого компонента для его дальнейшей публикации}
%	\scnaddlevel{-1}}

\bigskip
\scnendstruct \scnendcurrentsectioncomment

\end{SCn}
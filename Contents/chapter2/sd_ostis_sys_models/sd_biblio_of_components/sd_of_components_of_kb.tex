\begin{SCn}
\scnsectionheader{\currentname}

\scnstartsubstruct

\scnheader{Предметная область многократно используемых компонентов баз знаний ostis-систем}
\scniselement{предметная область}
\scnrelto{частная предметная область}{Предметная область многократно используемых компонентов ostis-систем}
\scnsdmainclasssingle{многократно используемый компонент баз знаний ostis-систем}
\scnhaselementrole{класс объектов исследования}{отношение, специфицирующее многократно используемый компонент баз знаний ostis-систем}

\scnheader{многократно используемый компонент баз знаний}
\scnsuperset{предметная область и онтология}
\scnaddlevel{1}
\scnsubset{раздел базы знаний}
\scnaddlevel{-1}
\scnsuperset{семантическая окрестность}
\scnaddlevel{1}
\scnrelfrom{смотрите}{\nameref{sd_sem_neigh}}
\scnaddlevel{-1}
\scnsuperset{базовые фрагменты предметных областей и онтологий}
\scnaddlevel{1}
\scnnote{Базовый фрагмент предметной области и онтологии включает в себя теоретико-множественную, логическую онтологии, а также терминологические фрагменты.}
\scnnote{Данный вид многократно используемых компонентов позволяет использовать только те знания, которые непосредственно необходимы для функционирования интеллектуальных систем, исключив то, что никак не влияет на работу конечной системы (пояснения, примеры, дидактический материал и т.д.).}
\scnhaselement{Базовый фрагмент теории логических формул, высказываний и логических sc-языков}
\scnhaselement{Базовый фрагмент теории множеств}
\scnhaselement{Базовый фрагмент теории связок и отношений}
\scnaddlevel{-1}
\scnsuperset{базы знаний прикладных систем}
\scnaddlevel{1}
\scnnote{Целые базы знаний могут быть многократно используемыми компонентами в случае разработки интеллектуальных систем, назначение которых совпадает.}
\scnaddlevel{-1}

\scnheader{многократно используемый компонент базы знаний}
\scnhaselement{Расширенное ядро базы знаний}
\scnaddlevel{1}
\scnhaselement{Ядро базы знаний}
\scnaddlevel{1}
\scnexplanation{\textit{Ядро базы знаний} представляет собой компонент, входящий в состав каждой базы знаний, разрабатываемой по \textit{Технологии OSTIS}, и скачивающийся в первую очередь.}
\scnaddlevel{-1}
\scnaddlevel{-1}
\scnnote{Список многократно используемых компонентов не является окончательным. В случае, когда разработчик базы знаний интеллектуальной системы считает, что разработанный им компонент сможет стать неотъемлемой частью библиотеки, то компонент будет добавлен в библиотеку, как многократно используемый, в случае, если компонент прошел верификацию и соответствует требованиям разработчиков библиотеки.}
	
\scnheader{отношение, специфицирующее многократно используемый компонент баз знаний ostis-систем}
\scnsubset{отношение, специфицирующее многократно используемый компонент ostis-систем}
\scnhaselement{максимальный класс объектов исследования'}
\scnhaselement{немаксимальный класс объектов исследования'}
\scnhaselement{исследуемое отношение'}

\bigskip
\scnendstruct \scnendcurrentsectioncomment

\end{SCn}
\begin{SCn}
\scnsectionheader{\currentname}

\scnstartsubstruct

\scnheader{Предметная область многократно используемых компонентов ostis-систем}
\scniselement{предметная область}
\scnsdmainclasssingle{многократно используемый компонент ostis-систем}
\scnsdclass{независимый многократно используемый компонент ostis-систем;зависимый многократно используемый компонент ostis-систем;атомарный многократно используемый компонент ostis-систем;неатомарный многократно используемый компонент ostis-систем; многократно используемый компонент ostis-систем, хранящийся в виде внешних файлов;многократно используемый компонент ostis-систем, хранящийся в виде файлов исходных текстов;многократно используемый компонент ostis-систем, хранящийся в виде бинарных файлов;многократно используемый компонент, хранящийся в базе знаний ostis-системы;хранилище многократно используемого компонента ostis-систем, хранящегося в виде внешних файлов;хранилище многократно используемого компонента ostis-систем, хранящегося в виде файлов исходных текстов;хранилище многократно используемого компонента ostis-систем, хранящегося в виде бинарных файлов; спецификация многократно используемого компонента ostis-систем;отношение, специфицирующее многократно используемый компонент ostis-систем; файл, содержащий url-адрес многократно используемого компонента ostis-систем}
\scnsdrelation{зависимый компонент*;несовместимый компонент*;адрес хранилища*;автор компонента*;установленные компоненты*; доступные к установке компоненты*}

\scnheader{многократно используемый компонент ostis-систем}
\scnexplanation{компонент ostis-системы, который содержит все те (и только те) sc-элементы, которые необходимы для функционирования компонента в дочерней ostis-системе}
\scnexplanation{компонент некоторой материнской ostis-системы, который может быть использован в некоторой дочерней ostis-системе}
\scnsubdividing{атомарный многократно используемый компонент ostis-систем;неатомарный многократно используемый компонент ostis-систем}
\scnsubdividing{зависимый многократно используемый компонент ostis-систем;независимый многократно используемый компонент ostis-систем}
\scnsubset{компонент ostis-системы}
\scnaddlevel{1}
\scnidtf{целостная часть ostis-системы, которая содержит все те (и только те) sc-элементы, которые необходимы для её функционирования в ostis-системе}
\scnrelfrom{разбиение}{\scnkeyword{Типология компонентов ostis-систем по типу хранения\scnsupergroupsign}}
\scnaddlevel{1} 
\scneqtoset{многократно используемый компонент ostis-систем, хранящийся в виде внешних файлов\\
	\scnaddlevel{1}
	\scnsubdividing{многократно используемый компонент ostis-систем, хранящийся в виде файлов исходных текстов;многократно используемый компонент ostis-систем, хранящийся в виде бинарных файлов}
	\scnaddlevel{-1}
	;многократно используемый компонент, хранящийся в базе знаний ostis-системы}
\scnaddlevel{1}
\scnnote{На данном этапе развития \textit{Технологии OSTIS} более удобным является хранение компонентов в виде исходных текстов.}
\scnaddlevel{-1}
\scnaddlevel{-1}
\scnaddlevel{-1}

\scnheader{независимый многократно используемый компонент ostis-систем}
\scnexplanation{многократно используемый компонент, который не зависит от других компонентов}

\scnheader{зависимый многократно используемый компонент ostis-систем}
\scnexplanation{многократно используемый компонент, который зависит от хотя бы одного другого компонента, т.е. не может быть встроен в дочернюю ostis-систему без компонентов, от которых он зависит}

\scnheader{атомарный многократно используемый компонент ostis-систем}
\scnexplanation{многократно используемый компонент, который в текущем состоянии библиотеки компонентов рассматривается как неделимый, то есть не содержит в своем составе других компонентов}

\scnheader{неатомарный многократно используемый компонент ostis-систем}
\scnexplanation{многократно используемый компонент, который в текущем состоянии библиотеки компонентов содержит в своем составе атомарные компоненты}

\scnheader{установленные компоненты*}
\scniselement{квазибинарное отношение}
\scniselement{ориентированное отношение}
\scnexplanation{Квазибинарное отношение, связывающее некоторую ostis-систему и компоненты, которые установлены в ней.}
\scnrelfrom{первый домен}{ostis-система}
\scnrelfrom{второй домен}{многократно используемый компонент ostis-систем}
\scnnote{Данное отношение позволяет хранить сведения о системах и компонентах, которые установлены в них, тем самым предоставляя возможность анализировать навыки системы.}
\scnnote{Данное отношение позволяет информировать о наиболее часто скачиваемых компонентах.}

\scnheader{доступные к установке компоненты*}
\scniselement{квазибинарное отношение}
\scniselement{ориентированное отношение}
\scnexplanation{Квазибинарное отношение, связывающее некоторую ostis-систему и компоненты, которые доступны к установке в конкретной ostis-системе.}
\scnrelfrom{первый домен}{ostis-система}
\scnrelfrom{второй домен}{многократно используемый компонент ostis-систем}
\scnnote{Данное отношение позволяет давать рекомендации по развитию ostis-системы, в которую можно скачать многократно используемые компоненты.}

\scnheader{хранилище многократно используемого компонента ostis-систем, хранящегося в виде внешних файлов}
\scnidtf{место, предназначенное для хранения многократно используемого компонента ostis-систем}
\scnsuperset{хранилище многократно используемого компонента ostis-систем, хранящегося в виде файлов исходных текстов}
\scnaddlevel{1}
\scnidtf{место хранения файлов исходных текстов многократно используемого компонента}
\scnsuperset{хранилище на основе систем контроля версий Git}
\scnaddlevel{1}
\scnsuperset{репозиторий GitHub}
\scnnote{На данном этапе в рамках \textit{Технологии OSTIS} (в силу открытости технологии, а также хранения компонентов в виде файлов исходных текстов) для хранения компонентов используются \textit{хранилища на основе систем контроля версий Git}.}
\scnaddlevel{-1}
\scnnote{Помимо исходных текстов компонента в хранилище должна находиться его спецификация, а также набор инструкций, позволяющий интегрировать данный компонент в дочернюю ostis-систему.}
\scnaddlevel{-1}
\scnsuperset{хранилище многократно используемого компонента ostis-систем, хранящегося в виде бинарных файлов}
\scnaddlevel{1}
\scnidtf{место хранения бинарного файла многократно используемого компонента}
\scnnote{Помимо бинарного файла компонента в хранилище должна находиться его спецификация, а также набор инструкций, позволяющий интегрировать данный компонент в дочернюю ostis-систему.}
\scnaddlevel{-1}

\scnheader{спецификация многократно используемого компонента ostis-систем}
\scnsubset{спецификация}
\scnidtf{описание многократно используемого компонента ostis-систем}
\scnnote{Каждый \textit{многократно используемый компонент ostis-систем} должен быть специфицирован в рамках библиотеки. Данные спецификации включают в себя основные знания о компоненте, которые позволяют обеспечить построение полной иерархии компонентов и их зависимостей, а также обеспечивают беспрепятственную интеграцию компонентов в дочерние ostis-системы.}

\scnheader{отношение, специфицирующее многократно используемый компонент ostis-систем}
\scnidtf{отношение, которое используется при спецификации многократно используемого компонента ostis-систем}
\scnhaselement{адрес хранилища*}
\scnaddlevel{1}
\scniselement{бинарное отношение}
\scniselement{ориентированное отношение}
\scnexplanation{Связки отношения \textit{адрес хранилища*} связывают многократно используемый компонент, хранящийся в виде внешних файлов и файл, содержащий url-адрес.}
\scnrelfrom{первый домен}{многократно используемый компонент ostis-систем, хранящийся в виде внешних файлов}
\scnrelfrom{второй домен}{файл, содержащий url-адрес многократно используемого компонента ostis-систем}
\scnaddlevel{1}
\scnsuperset{файл}
\scnaddlevel{-1}
\scnaddlevel{-1}
\scnhaselement{автор компонента*}
\scnaddlevel{1}
\scniselement{бинарное отношение}
\scniselement{ориентированное отношение}
\scnexplanation{Связки отношения \textit{автор компонента*} связывают многократно используемый компонент и его разработчика.}
\scnrelfrom{первый домен}{многократно используемый компонент ostis-систем}
\scnrelfrom{второй домен}{субъект}
\scnaddlevel{-1}
\scnhaselement{зависимый компонент*}
\scnaddlevel{1}
\scniselement{бинарное отношение}
\scniselement{ориентированное отношение}
\scnexplanation{Бинарное отношение, связывающее зависимый многократно используемый компонент и компонент, без которого тот не может быть встроен в дочернюю ostis-систему.}
\scnrelfrom{первый домен}{многократно используемый компонент ostis-системы}
\scnrelfrom{второй домен}{зависимый многократно используемый компонент ostis-систем}
\scnaddlevel{-1}
\scnhaselement{несовместимый компонент*}
\scnaddlevel{1}
\scniselement{бинарное отношение}
\scniselement{неориентированное отношение}
\scnexplanation{Бинарное отношение, связывающее два компонента, которые не могут одновременно присутствовать в одной ostis-системе.}
\scnrelfrom{первый домен}{многократно используемый компонент ostis-систем}
\scnrelfrom{второй домен}{многократно используемый компонент ostis-систем}
\scnaddlevel{-1}
\scnnote{Для спецификации многократно используемого компонента также необходимо указывать классы, к которым он принадлежит, дату последнего изменения, описание назначения компонента.}

\scnheader{многократно используемый компонент ostis-системы}
\scnsubdividing{многократно используемый компонент базы знаний; многократно используемый компонент решателя задач; многократно используемый компонент интерфейса}

\bigskip
\scnendstruct \scnendcurrentsectioncomment

\end{SCn}
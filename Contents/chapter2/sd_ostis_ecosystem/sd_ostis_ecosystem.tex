\begin{SCn}

\scnsectionheader{\currentname}

\scnstartsubstruct

\scnrelfromlist{дочерний раздел}{\nameref{sd_learning};\nameref{sd_assistants};\nameref{sd_portals};\nameref{sd_ecosys_enterprise}}

\scnheader{Предметная область Экосистемы OSTIS}
\scniselement{предметная область}
\scnsdmainclasssingle{Экосистема OSTIS}
\scnsdclass{ostis-система;самостоятельная ostis-система;поддержка совместимости между компьютерными системами и их пользователями в Экосистеме OSTIS;Экосистемa}
\scnrelfromlist{библиографический источник}{\scncite{DeNicola2021};\scncite{Alrehaili2021};\scncite{Alrehaili2017};\scncite{Shahzad2021};\scncite{Masaharu2018a}}
%\scnsdrelation{***}

\scnheader{Экосистема OSTIS}
\scnidtf{Социотехническая экосистема, представляющая собой коллектив взаимодействующих семантических компьютерных систем и осуществляющая перманентную поддержку эволюции и семантической совместимости всех входящих в нее систем, на протяжении всего их жизненного цикла}
\scnidtf{Неограниченно расширяемый коллектив постоянно эволюционируемых семантических компьютерных систем, которые взаимодействуют между собой и с пользователями для корпоративного решения сложных задач и для постоянной поддержки высокого уровня совместимости и взаимопонимания во взаимодействии как между собой, так и с пользователями}

\scnexplanation{Поскольку \textit{Технология OSTIS} ориентирована на разработку \textit{семантических компьютерных систем}, обладающих высоким уровнем \textit{обучаемости} и, в частности, высоким уровнем семантической \textit{совместимости}, и поскольку обучаемость и совместимость есть только \uline{способность} к обучению (т.е. к высоким темпам расширения и совершенствования своих знаний и навыков), а также \uline{способность} к обеспечению высокого уровня взаимопонимания (согласованности), необходима некая среда, социотехническая инфраструктура, в рамках которой были бы созданы максимально комфортные условия для реализации указанных выше способностей. Такая среда названа нами \textit{\textbf{Экосистемой OSTIS}}, которая представляет собой коллектив взаимодействующих (через сеть Интернет):

\begin{scnitemize}
\item самих \textit{ostis-систем};
\item пользователей указанных \textit{ostis-систем} (как конечных пользователей, так и разработчиков);
\item некоторых компьютерных систем, не являющихся \textit{ostis-системами}, но рассматриваемых ими в качестве дополнительных информационных ресурсов или сервисов.
\end{scnitemize}
}
\scntext{основная задача}{Обеспечить постоянную поддержку совместимости компьютерных систем, входящих в \textit{Экосистему OSTIS} как на этапе их разработки, так и в ходе их эксплуатации. Проблема здесь заключается в том, что в ходе эксплуатации систем, входящих в \textit{Экосистему OSTIS}, они могут изменяться из-за чего совместимость может нарушаться.

Задачами \textit{Экосистемы OSTIS} являются:
\begin{scnitemize}
\item оперативное внедрение всех согласованных изменений стандарта \textit{ostis-систем} (в том числе, и изменений систем используемых понятий и соответствующих им терминов);
\item перманентная поддержка высокого уровня взаимопонимания всех систем, входящих в \textit{Экосистему OSTIS}, и всех их пользователей; 
\item корпоративное решение различных сложных задач, требующих координации деятельности нескольких (чаще всего, априори неизвестных) \textit{ostis-систем}, а также, возможно, некоторых пользователей.
\end{scnitemize}
}
\scnnote{\textit{Экосистема OSTIS} -- это переход от самостоятельных (автономных, отдельных, целостных) \textit{ostis-систем} к коллективам самостоятельных \textit{ostis-систе}м, т.е. к распределенным \textit{ostis-системам}}

\scnheader{ostis-система}
\scnsubdividing{атомарная встроенная ostis-система\\
	\scnaddlevel{1}
		\scnidtf{ostis-система, интегрированная в состав самостоятельной ostis-системы, но не в состав другой встроенной ostis-системы}
	\scnaddlevel{-1};
	неатомарная встроенная ostis-система\\
	\scnaddlevel{1}
		\scnidtf{ostis-система, которая интегрирована в состав самостоятельной ostis-системы, и включает в себя некоторые другие встроенные ostis-системы}
		\scnsuperset{интерфейс ostis-системы}
	\scnaddlevel{-1};
	cамостоятельная ostis-система\\
	\scnaddlevel{1}
		\scnidtf{целостная ostis-система, которая должна самостоятельно решать соответствующее множество задач и, в частности, взаимодействовать с внешней средой (вербально -- с пользователями и другими компьютерными системами, так и невербально)}	
	\scnaddlevel{-1};
	коллектив ostis-систем\\
	\scnaddlevel{1}
		\scnidtf{группа общающихся ostis-систем, в состав которой могут входить не только самостоятельные ostis-системы, но и коллективы ostis-систем}
		\scnidtf{распределенная ostis-система}
	\scnaddlevel{-1};
	Экосистема OSTIS\\
	\scnaddlevel{1}
	\scniselement{максимальный коллектив ostis-систем}
	\scniselement{коллектив ostis-систем, не являющийся частью другого коллектива ostis-систем}
	\scnaddlevel{-1}
	}

\scnheader{cамостоятельная ostis-система}
\scnexplanation{Подчеркнем, что к \textit{\textbf{самостоятельным ostis-системам}}, входящим в состав \textit{Экосистемы OSTIS}, предъявляются особые требования:
\begin{scnitemize}
    \item они должны обладать всеми необходимыми знаниями и навыками для обмена сообщениями и целенаправленной организации взаимодействия с другими \textit{ostis-системам}и, входящими в \textit{Экосистему OSTIS};
    \item в условиях постоянного изменения и эволюции \textit{ostis-систем}, входящих в \textit{Экосистему OSTIS}, каждая из них должна \uline{сама следить за состоянием своей совместимости} (согласованности) со всеми остальными \textit{ostis-системами},  т.е. должна самостоятельно поддерживать эту совместимость, согласовывая с другими ostis-системами все требующие согласования изменения, происходящие у себя и в других системах.
    \item каждая система, входящая в состав \textit{Экосистемы OSTIS}, должна:
    \begin{scnitemizeii}
        \item интенсивно, активно и целенаправленно обучаться ( как с помощью  учителей-разработчиков, так и самостоятельно);
        \item сообщать всем другим системам о предлагаемых или окончательно утвержденных изменениях в \textit{онтологиях} и, в частности, в наборе используемых \textit{понятий};
        \item принимать от других \textit{ostis-систем} предложения об изменениях в \textit{онтологиях} ( в том числе в наборе используемых понятий) для согласования или утверждения этих предложений;
        \item реализовывать утвержденные изменения в \textit{онтологиях}, хранимых в ее базе знаний;
        \item способствовать поддержанию высокого уровня семантической совместимости не только с другими \textit{ostis-системами}, входящими в \textit{Экосистему OSTIS}, но и со своими \textit{пользователями} ( т.е. обучать их, информировать их об изменениях в онтологиях).
    \end{scnitemizeii}
\end{scnitemize}}

\scnheader{Экосистема OSTIS}
\scnexplanation{\textit{Экосистема OSTIS} является формой реализации, совершенствования и применения \textit{Технологии OSTIS} и, следовательно, является формой создания, развития, самоорганизации рынка семантически совместимых компьютерных систем  и включает в себя все необходимые для этого ресурсы --  информационные, технологические, кадровые, организационные, инфраструктурные. 

\textit{Экосистеме OSTIS} ставится в соответствие ее \textit{\textbf{объединенная база знаний}}, которая представляет собой \textbf{виртуальное объединение} \textit{баз знаний} всех \textit{ostis-систем}, входящих в состав \textit{Экосистемы OSTIS}. Качество этой \textit{базы знаний} (полнота, непротиворечивость, чистота) является постоянной заботой всех самостоятельных \textit{ostis-систем}, входящих в состав \textit{Экосистемы OSTIS}. Соответственно этому каждой указанной \textit{ostis-системе} ставится в соответствие своя \textit{база знаний} и своя иерархическая система \textit{sc-агентов}.

По назначению \textit{ostis-системы}, входящие в \textit{Экосистему OSTIS}, могут быть:
\begin{scnitemize}
    \item ассистентами конкретных пользователей или конкретных пользовательских коллективов;
    \item типовыми встраиваемыми подсистемами \textit{ostis-систем};
    \item системами информационной и инструментальной поддержки проектирования различных компонентов и различных классов \textit{ostis-систем};
    \item системами информационной и инструментальной поддержки проектирования или производства различных классов технических и других искусственно создаваемых систем;
    \item порталами знаний по самым различным научным дисциплинам; 
    \item системами автоматизации управления различными сложными объектами (производственными предприятиями, учебными заведениями, кафедрами вузов, конкретными обучаемыми);
    \item интеллектуальными справочными и help-системами;
    \item интеллектуальными обучающими системами, семантическими электронными учебными пособиями;
    \item интеллектуальными робототехническими системами.
\end{scnitemize}
}

\scnresetlevel

\scnheader{поддержка совместимости между компьютерными системами и их пользователями в Экосистеме OSTIS}
\scnexplanation{Есть три аспекта поддержки совместимости и взаимопонимания в \textit{Экосистеме OSTIS}

\begin{scnitemize}
\item поддержка совместимости между самими \textit{ostis-системами}, входящими в \textit{Экосистему OSTIS} в процессе их эволюции;
\item поддержка совместимости между каждой ostis-системой и текущим состоянием Технологии OSTIS в процессе эволюции этой технологии;
\item поддержка совместимости и взаимопонимания между \textit{ostis-системами}, входящими в \textit{Экосистему OSTIS}, и их пользователями при активном стимулировании со стороны \textit{Экосистемы OSTIS} того, чтобы каждый пользователь \textit{Экосистемы OSTIS} был одновременно не только активным ее конечным пользователем, но и активным ее разработчиком.
\end{scnitemize}

Таким образом, для обеспечения высокой эффективности эксплуатации и высоких темпов эволюции  \textit{Экосистемы OSTIS}, необходимо постоянно повышать уровень информационной совместимости (уровень взаимопонимания) не только между компьютерными системами, входящими в состав \textit{Экосистемы OSTIS}, но также между этими системами и их пользователями. Одним из направлений обеспечения такой совместимости является стремление к тому, чтобы \textit{база знаний} (картина мира) каждого пользователя стала частью (фрагментом) \textbf{\textit{Объединенной базы знаний Экосистемы OSTIS}}.  Это значит, что каждый пользователь должен знать, как устроена структура каждой научно-технической дисциплины (объекты исследования, предметы исследования, определения, закономерности и т.д.), как могут быть связаны между собой различные дисциплины.

Формирование таких навыков системного построения картины Мира необходимо начинать со средней школы. Для этой цели необходимо создать комплекс совместимых интеллектуальных обучающих систем по всем дисциплинам среднего образования с четко описанными междисциплинарными связями (\scncite{Bashmakov}, \scncite{Taranchuk2015}). Благодаря этому можно предотвратить формирование у пользователей "мозаичной"{} картины Мира как множества слабо связанных между собой дисциплин. А это, в свою очередь, означает существенное повышение качества образования, которое абсолютно необходимо для качественной эксплуатации компьютерных систем следующего поколения -- \textit{семантических компьютерных систем}.

Пользователи и, первую очередь, разработчики \textit{Экосистемы OSTIS}  должны иметь высокий уровень:
\begin{scnitemize}
\item математической культуры (культуры формализации) при построении формальной модели среды, в которой функционирует интеллектуальная система, формальных моделей решаемых ею задач и формальных моделей различных используемых ею способов решения задач;
\item системной культуры, позволяющей адекватно оценивать качество разрабатываемых систем с точки зрения общей теории систем и, в частности, оценивать общий уровень автоматизации, реализуемый с помощью этих систем. Системная культура предполагает стремление и умение избегать эклектики, стремление и умение обеспечить качественную стратифицированность, гибкость, рефлексивность, а также качественное сопровождение, высокий уровень обучаемости и комфортный пользовательский интерфейс разрабатываемых систем;
\item технологической культуры, обеспечивающей совместимость разрабатываемых систем и их компонентов, а также постоянное расширение библиотеки многократно используемых компонентов создаваемых систем и предполагающей высокий уровень проектной дисциплины;
\item умения работать в команде разработчиков наукоемких систем, что предполагает высокий уровень умения работать на междисциплинарных стыках, высокий уровень коммуникабельности и \uline{договороспособности}, т.е. способности не столько отстаивать свою точку зрения, сколько согласовывать ее  с точками зрения других разработчиков в интересах развития \textit{Экосистемы OSTIS};
\item активности и ответственности за общий результат -- высокие темпы эволюции \textit{Экосистемы OSTIS} в целом.
\end{scnitemize}

Таким образом высокие темпы эволюции \textit{Экосистемы OSTIS} обеспечиваются не только профессиональной квалификацией пользователей (знаниями о \textit{Технологии OSTIS}, о текущем состоянии и проблемах \textit{Экосистемы OSTIS} и навыками использования \textit{Технологии OSTIS} и интеллектуальных систем, входящих в \textit{Экосистему OSTIS}), но и соответствующими человеческими качествами. Очевидно, что современный уровень \uline{договороспособности, активности и ответственности} не может быть основой для эволюции таких систем, как \textit{Экосистема OSTIS}.

Поддержка совместимости \textit{Экосистемы OSTIS} с ее пользователями осуществляется следующим образом:


\begin{scnitemize}
\item в каждую \textit{ostis-систему} включаются встроенные ostis-системы, ориентированные
  
  \begin{scnitemizeii}
        \item на перманентный мониторинг деятельности конечных пользователей и разработчиков этой \textit{\mbox{ostis-системы}},
        \item на анализ качества и, в первую очередь, корректности этой деятельности,
        \item на перманентное ненавязчивое персонифицированное обучение, направленное на повышение качества деятельности пользователей, т.е. на повышение их квалификации;
        
  \end{scnitemizeii}
        
\item в состав \textit{Экосистемы OSTIS} включаются \textit{ostis-системы}, специально предназначенные для обучения пользователей \textit{Экосистемы OSTIS} базовым общепризнанным знаниям и навыкам решения соответствующих классов задач. Сюда входят и знания, соответствующие уровню среднего образования, и знания соответствующие базовым дисциплинам высшего образования в области информатики (и, в том числе, в области искусственного интеллекта), и базовые знания по \textit{Технологии OSTIS} и об \textit{Экосистеме OSTIS}.

\end{scnitemize}}

\scnheader{Экосистема OSTIS}
\scntext{обоснование}{Проблема создания рынка совместимых компьютерных систем --  \textbf{вызов современной науке и технике}.  От ученых, работающих в области искусственного интеллекта требуется умение коллективно работать над решением междисциплинарных проблем и доводить эти решения до общей интегрированной теории интеллектуальных систем, предполагающей интеграцию всех направлений искусственного интеллекта, и до технологий, доступных широкому кругу инженеров. От инженеров интеллектуальных систем требуется активное участие в развитии соответствующих технологий и существенное повышение уровня математической, системный, технологической и организационно-психологической культуры.

Но главной задачей здесь является снижение барьера между научными исследованиями в области искусственного интеллекта и инженерией в области разработки интеллектуальных систем. Для этого наука должна стать конструктивной и ориентированной на интеграцию своих результатов в форме комплексной технологии разработки интеллектуальных систем, а инженерия, осознав наукоемкость своей деятельности, должна активно участвовать в разработке технологий.

Особый акцент в \textit{Экосистеме OSTIS}  делается на постоянный процесс согласования \textit{онтологий} (и, в первую очередь, на согласование семейства всех используемых понятий и терминов, соответствующих этим понятиям) между \uline{всеми} (!) активными субъектами \textit{Экосистемы OSTIS} -- между всеми \textit{ostis-системами} и всеми пользователями.

При наличии \textit{ostis-систем}, являющихся персональными ассистентами пользователей во взаимодействии с \textit{Экосистемой OSTIS}, вся эта Экосистема будет восприниматься пользователями как единая интеллектуальная система, объединяющая все имеющиеся в \textit{Экосистеме OSTIS} информационные ресурсы и сервисы.

Принципы организации \textit{Экосистемы OSTIS} создают все необходимые условия для привлечения к разработке и совершенствованию \textit{Технологии OSTIS} научные, организационные и финансовые ресурсы, которые будут направлены на развитие методов и средств искусственного интеллекта и на формирование рынка семантически совместимых интеллектуальных систем.}

\scnheader{Экосистемa}
\scndefinition{Экосистема определяется как «биологическая система, состоящая из всех организмов, обнаруженных в определенной физической среде,
	взаимодействующих с ней и друг с другом.}

\scnexplanation{Существенное значение концепции \textit{экосистемы} заключается в пяти пунктах:
	
	\begin{scnitemize}
		\item Во-первых, экосистемная концепция анализирует органические сети, основываясь не только на их положительных сторонах, 
		но и на их негативных и конкурентных аспектах: конкуренции на уровне экосистемы, хищничества, паразитизма и разрушения всей системы;
		\item Во-вторых, каждое действующее лицо имеет разные атрибуты, принципы принятия решений и цели. Эти различия могут привести к
		непреднамеренным результатам на уровне экосистемы, хотя принятие решений и поведение каждого субъекта является
		рациональным в данный момент времени;
		\item В-третьих, аналитической границей экосистемы является система продукт/услуга; она не ограничивается национальными границами, 
		региональными кластерами, договорными отношениями или взаимодополняющими поставщиками. 
		В экосистему включены не только бизнес-субъекты, но и некоммерческие субъекты;
		\item В-четвертых, экосистемный анализ требует продольного наблюдения за динамической эволюцией системы товар/услуга;
		\item В-пятых, цели экосистемных исследований заключаются в поиске закономерностей принятия решений и поведенческих цепочек,
		которые сильно влияют на рост и упадок экосистемы при определенных граничных условиях.
	\end{scnitemize}
}

\scntext{основная задача}{Целями экосистемных исследований являются поиск принципов принятия решений и поведенческих цепочек, которые сильно влияют на рост и упадок экосистемы при определенных граничных условиях.
}

\scnsubdividing{перспектива промышленной экологии\\
	\scnaddlevel{1}
	\scnidtf{Перспектива отражена в исследованиях, основанных на концепции промышленной экосистемы, которая была введена \textbf{Фрошем и Галлопулосом} в 1989 году. 
		Авторы использовали понятие природной экосистемы в качестве аналогии для понимания и трансформации индустриальной системы. 
		В статье говорилось, что «традиционная модель промышленной деятельности, в которой отдельные производственные процессы берут 
		сырье и генерируют продукты для продажи, а также отходы, подлежащие утилизации, должна быть преобразована в более интегрированную модель: 
		промышленную экосистему»}
	\scnidtf{Исследователи промышленной экологии внесли свой вклад в реализацию устойчивых промышленных систем в реальном мире. Исследователи, которые использовали термин 		       «промышленная экосистема»,
		обычно сосредотачивались на применении концепции или модели \textbf{IE(промышленной экологии)} к обществу. Таким образом, концепция промышленной экосистемы применялась к реальному обществу в течение значительного времени.
		Исследователи \textbf{IE} предприняли оптимизацию энергетических, материальных и денежных сетей. Таким образом, 
		промышленная экосистема — это не просто концепция, модель или имитационный анализ.}
	\scnidtf{Что касается методологии, то в большинстве существующих исследований промышленных экосистем используется анализ энергетических или материальных потоков.
		Некоторые исследователи создали концептуальные модели промышленных экосистем .
		Расширяя концептуальное моделирование, исследователи применили системную динамику или методы химической инженерии для оптимизации симбиоза, 
		стабильности и устойчивости промышленной экосистемы}
	\scnaddlevel{-1};
	Перспектива бизнес-экосистемы\\
	\scnaddlevel{1}
	\scnidtf{Исследователи, использующие перспективу бизнес-экосистемы \textbf{(BEP)}, концентрируются на бизнес-контексте и устанавливают захват ценности или создание ценности в качестве центральных переменных. Целью исследований в этом потоке является выявление динамики и закономерностей экосистем и организационного поведения. Теоретической основой \textbf{BEP} является теория организационных границ в рамках общей теории стратегического управления. \textbf{Сантос и Эйзенхардт} (2005) разработали четыре концепции границ \textbf{(эффективность, власть, компетентность и идентичность)}.}
	\scnidtf{Анализировались пять различных типов экосистем, основное внимание уделялось конкретным концептуальным отношениям.
		Пять различных типов экосистем следующие: \textbf{1)} цифровые экосистемы; \textbf{2)} взаимодополняющие
		(подотраслевые) экосистемы; \textbf{3)} экосистемы поставщиков; \textbf{4)} экосистемы бизнес-групп (M&A); и 
		\textbf{5)} глобальные профессиональные экосистемы человеческих сетей. 
		Диапазон используемых концепций и определений экосистем является широким:
		\begin{scnitemize}
			\item Самая большая группа в рамках \textbf{BEP} состоит из восьми работ, касающихся анализа цифровых экосистем. Исследования
			в этом кластере сосредоточены на ИТ-индустрии. Нет единого мнения по определению цифровой экосистемы. Тем не менее,
			почти во всех исследованиях цифровых экосистем анализировались сложные отношения на основе ИТ между фирмами в ИТ-отраслях,
			таких как сектор программного обеспечения \textbf{(Iyer et al., 2006)}, рынок приложений \textbf{(Selander et al., 2013)} и операторы
			мобильных сетей \textbf{(Aaltonen and Tempini, 2014)}. Исследования в этом кластере могут быть классифицированы под
			следующей перспективой (перспектива управления платформой); однако эти исследования были сосредоточены не на платформе,
			а на сложных отношениях фирм в цифровой экосистеме, например, экосистемах бизнес-процессов \textbf{(Vidgen and Wang, 2006)}
			и стратегических гиперссылках \textbf{(Dellarocas et al., 2013)}.;
			\item Вторая группа, равная по размеру первой группе, сосредоточилась на комплементарной (подотраслевой) экосистеме. 
			\textbf{Аднер и Капур} описали общую схему экосистемы, состоящей из поставщика, фокусной фирмы, комплементара и клиента. Используя данные отрасли полупроводникового литографического оборудования,
			проверили свою гипотезу о динамике экосистем. \textbf{Капур и Ли} (2013) классифицировали отношения между фокусной организацией и 
			комплементаром и проверили значимость корреляций между этими классификациями и технологическими инвестициями. 
			Эти два исследования положили начало анализу бизнес-экосистемы и создали концептуальную основу для этого подхода.;
			\item Третья группа представляет экосистемный подход поставщиков. Исследователи в этой группе исследовали проблему выбора поставщика
			\textbf{(Viswanadham and Samvedi, 2013)} и создание кооперативных и разнообразных сетей поставщиков \textbf{(Hong and Snell, 2013)}.
			\item Исследование в четвертой группе рассматривает экосистему как агломерированную компанию, связанную слияниями и поглощениями.
			Исследователи проанализировали динамические изменения в бизнес-группах и взаимосвязь с экономическим ростом с использованием 
			данных о слияниях и поглощениях и патентах \textbf{(Gomez-Uranga et al., 2014, Li, 2009)}. Уникальная статья посвящена глобальной сети
			талантов человека STEM (наука, технология, инженерия и математика) как источнику глобальной инновационной экосистемы \textbf{(Lewin and Zhong, 2013)}.
		\end{scnitemize}
	}
	\scnidtf{Что касается методологии, то основными используемыми подходами являются качественные тематические исследования
		и анализ нескольких случаев с использованием оригинального обследования или базы данных. Есть предложения, в которых 
		используется качественная методология или процесс исследования. Визуализация и анализ сети, выполненные с помощью
		специального компьютерного программного обеспечения, были применены в трех работах (Basole, 2009, Battistella et al., 2013, Li, 2009).
		\textbf{Battistella et al.} (2013) создали теоретическое предложение по методологии сетевого анализа бизнес-экосистем (MOBENA).
		Методологически MOBENA имеет потенциальную применимость к другим случаям. \textbf{Hung et al.} (2013) объединили методологию исследования Delphi
		с экосистемным анализом). Дельфийский подход эффективен в достижении тщательного анализа на основе признания реальности.}
	\scnaddlevel{-1};
    Перспектива управления платформой\\
	\scnaddlevel{1}
	\scnidtf{Эта перспектива была представлена \textbf{Кусумано и Гавером} (2002) в «Элементах лидерства платформы».
		Первоначально авторы использовали термин «отраслевая экосистема» в качестве ключевого слова. Согласно недавнему обзору исследований
		платформы, существует множество работ, в которых анализируется механизм платформы. \textbf{Thomas et al.} (2014)
		классифицировали исследование платформы по четырем потокам. Среди них четвертый поток касается платформенных экосистем, которые
		состоят из общеотраслевых сетей, основанных на сложных корреляциях между фирмами. Этот поток напоминает \textbf{PMP}. Это означает,
		что исследования \textbf{PMP(управления платформой)} перекрывают значительную часть исследований платформ и пытаются прояснить связанные с ними сложные сети.
		Кроме того, термины «промышленная экосистема» с точки зрения 
		\textbf{IE(промышленной экологии)} и «отраслевая экосистема» с точки зрения \textbf{PMP(управление платформой)} похожи, но представляют
		собой концепции, которые отличаются определенным образом}	
	\scnidtf{Что касается методологии, то преобладающим методом является интенсивное многотематическое исследование.
		В тематических исследованиях большинство ученых применяли статистические или математические тесты с использованием эмпирических данных.
		Также применялся метод сетевого анализа \textbf{(Weiss and Gangadharan, 2010)}.}	
	\scnaddlevel{-1};
	Многопользовательская сеть\\
	\scnaddlevel{1}
	\scnidtf{Перспектива многопользовательской сети \textbf{(MNP)} является четвертой выявленной перспективой.
		Эта точка зрения была расширена, чтобы включить различных участников (предпринимателей и частных инвесторов, новаторов,
		которые находятся за пределами трубопроводов компаний, пользователей / сообществ пользователей, правительственных бюрократов / политиков и консорциумов). }
	\scnidtf{Направления расширения многопользовательской сети разнообразны. Существует пять областей, в которых проводится экосистемный анализ:
		\textbf{1)} предприниматели/частные инвесторы; \textbf{2)} новаторы, которые находятся за пределами конвейеров компании;
		\textbf{3)} пользователи/сообщества пользователей; \textbf{4)} правительственные бюрократы/директивные органы; 
		и \textbf{5)} консорциумы:
		\begin{scnitemize}
			\item Первая группа – это предприниматели и частные инвесторы. \textbf{Autio et al.} (2014) представили структуру, включающую государственных и частных субъектов, и предложили дефицит контекста 
			и предпринимательских инноваций. Два исследования были сосредоточены на венчурном капитале \textbf{(Samila and Sorenson, 2010)}. Еще два исследования проанализировали региональные кластеры 
			Кремниевой долины \textbf{(Bahrami and Evans, 1995)} и Фландрии \textbf{(Clarysse et al., 2014)}, включая различных действующих лиц. Остальные две статьи являются
			тематическими исследованиями начинающих компаний Chez Panisse \textbf{(Chesbrough et al., 2014)} и Acorn-ARM \textbf{(Garnsey et al., 2008)}.;
			\item Вторая группа рассматривает новаторов, которые находятся за пределами конвейеров компании. Есть три исследования,
			касающиеся реконструкции инновационных экосистем созданными компаниями путем открытия своих инновационных процессов для других,
			с использованием таких подходов, как консорциумы или высокотехнологичные кампусы \textbf{(Leten et al., 2013, Rohrbeck et al., 2009,
			van der Borgh et al., 2012)}.;
			\item Третья группа включает пользователей в их экосистемный анализ.\textbf{Hienerth et al.} (2014)
			проанализировали экосистему Lego и обнаружили, что синергия между фирмами, ведущими пользователями и сообществами 
			пользователей влияет на создание прибыльной и устойчивой экосистемы. Остальные два документа
			были посвящены экосистемам, включая взгляд развивающихся стран со стороны спроса, такие как анализ BOP (дно пирамиды)
			\textbf{(Khavul and Bruton, 2013, Ramachandran et al., 2012)}.
			\item Четвертая группа включает правительственных бюрократов/политиков. \textbf{Watanabe} (1999) использовал экосистемную
			концепцию для анализа влияния политики на промышленные технологии Министерства международной торговли и промышленности
			Японии с 1970-х годов. \textbf{Fabrizio and Hawn} (2013) собрали данные об установках солнечной энергии в США
			и проверили влияние политики выделения солнечной энергии и наличие квалифицированных установщиков.
			Авторы обнаружили, что анализ, принимающий концепцию экосистемы в качестве краеугольного камня, может выявить неопределенные 
			механизмы, связанные с их косвенным воздействием. \textbf{Groesser} (2014) проанализировал стандартизацию строительных норм в Швейцарии и обнаружил
			эволюционно динамичный процесс между добровольными строительными нормами, юридическими
			строительными нормами и стандартами.
		\end{scnitemize}
	}
    \scnidtf{Что касается методологии, системная динамика используется в двух работах \textbf{(Daim et al., 2006, Groesser, 2014)}.
	    Основной методологией является тематическое исследование и/или статистическое тестирование с
	    использованием открытых данных и/или данных обследований.}
    \scnaddlevel{-1}
    }


\bigskip
\scnendstruct \scnendcurrentsectioncomment

\end{SCn}

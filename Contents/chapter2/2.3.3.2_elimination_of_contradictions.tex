\begin{SCn}

\bigskip
\scnheader{Механизм устранения противоречий в базе знаний}

\scnrelfromset{рассматриваемые вопросы}{
	\scnfileitem{какие существуют противоречия};
	\scnfileitem{Механизм устранения противоречий в базе знаний};
	\scnfileitem{Спецификации агентов поиска устранения противоречий};
	\scnfileitem{Спецификация агентов для устранения конкретных противоречий}
}
\scnrelfromvector{план изложения}{
	\scnfileitem{Описание проблемы};
	\scnfileitem{Механизм устранения противоречий};
	\scnfileitem{Спецификация агента поиска противоречий};
	\scnfileitem{Устранение дублирования системных идентификаторов};
	\scnfileitem{Спецификация агента поиска дублирования системного идентификаторов};
	\scnfileitem{Спецификация агента устранения дублирования системного идентификатора};
	\scnfileitem{Спецификация aгента слияния элементов};
	\scnfileitem{(optional) Алгоритм работы нового билдера}
}

\bigskip
\scnfragmentcaption

\scnheader{Описание проблемы}
\scnnote{Как при добавлении новых фрагментов базы знаний, так и в процессе работы системы, в ней могут появляться ошибки (противоречия)}
\scnrelfrom{виды противоречий}{
\scnfilescg{figures/sd_agents/struct_contradictions.png}}}

\scnnote{Далее предлагается описание механизма и агентов устранения противоречий на примере устранения дублирования системных идентификаторов. Представленный механизм может дополняться и расширяться другими агентами, которые будут устранять другие виды противоречий.}

\bigskip
\scnfragmentcaption

\scnheader{Агент поиска противоречий}
\scntext{инициирующие событие}{Появление в sc-памяти инициированного действия, принадлежащего классу "действие поиска противоречий". Аргументами могут быть несколько структур.}
\scntext{входные параметры}{фрагмент базы знаний}
\scntext{алгоритм работы}{
	\begin{scnenumerate}
		\item Все классы действий поиска противоречий принадлежат “классу действий поиска противоречий в базе знаний”. Агент поиска генерирует экземпляр каждого такого действия, которое принадлежит “классу действий поиска противоречий в базе знаний”
		\item Ожидание добавления экземпляра действия в множество выполненных действий (событие входящей дуги от question\_finished)
		\item Для каждого найденного противоречия сгенерировать экземпляр “действия устранения противоречий”
		\item Ожидание добавления экземпляра “действия устранения противоречий” в множество выполненных действий(событие входящей дуги от question\_finished)
		\item Если действие не выполнено успешно (действие принадлежит множеству “question\_finished\_unsuccessfully”), то добавить противоречие в множество “требующая внимания разработчика”; Перейти к пункту 7
		\item Если действие выполнено успешно, добавить результат “действия устранения противоречий” в ответ (сгенерировать дугу принадлежности из множества ответа к результату “действия устранения противоречий”)
		\item удалить сгенерированные экземпляры “действия поиска дублирования системных идентификаторов” и “действия устранения противоречий”
		\item Если не осталось незавершенных действий, завершить работу. Иначе перейти к пункту 3.
	\end{scnenumerate}
}
\scntext{результат}{Множество пар для исходного фрагмента БЗ, где первый элемент это структура устраненного противоречия, а второй это множество элементов, которые должны быть удалены при выполнении устранения противоречия.}

\scnnote{//todo пошаговый алгоритм в SCg}

\scnnote{Далее описание алгоритма идет только на примере дублирования системных идентификаторов.}

\bigskip
\scnfragmentcaption

\scnheader{Агент поиска дублирования системного идентификатора}
\scntext{Инициирующие событие}{Появление в sc-памяти инициированного действия, принадлежащего классу "действие поиска дублирования системного идентификатора"}
\scntext{Входные параметры}{фрагмент базы знаний}
\scntext{Алгоритм работы}{
	\begin{scnenumerate}
		\item Поиск различных элементов с одинаковым системным идентификатором (дублирование системных идентификаторов) по всей базе знаний
		\item Проверка принадлежат ли найденные проблемные элементы входным фрагментам
		\item Если да, то генерация структуры, содержащей дублирование системного идентификатора.
	\end{scnenumerate}
}
\scntext{Результат}{множество структур дублирования системного идентификатора}

\scnnote{//todo пошаговый алгоритм в SCg}

\bigskip
\scnfragmentcaption

\scnheader{Агент устранения дублирования системного идентификатора}
\scntext{Инициирующие событие}{Появление в sc-памяти инициированного действия, принадлежащего классу "действие устранения противоречия"}
\scntext{Входные параметры}{структура с противоречием}
\scntext{Алгоритм работы}{
	\begin{scnenumerate}
		\item проверить что структура с противоречием принадлежит классу “структура с дублированием системного идентификатора” Если нет, то агент завершает работу
		\item генерация экземпляра “действие слияния элементов”
		\item ожидание добавления экземпляра “действие слияния элементов” в множество выполненных действий (событие входящей дуги от “question\_finished”)
		\item если действие слияния было завершено успешно (принадлежит множеству “question\_finished\_successfully”), то добавить экземпляр действия устранения в множество “question\_finished\_successfully”, перейти к шагу 6. иначе 5
		\item добавить экземпляр “действия устранения противоречия” в множество “question\_finished\_unsuccessfully” завершить работу
		\item добавить результат “действие слияния элементов” в структуру
		\item добавить все устаревшие элементы, подлежащие удалению в множество
		\item оформить “nrel\_result”, результатом будет являться пара - структура, сформированная в пункте 6 и множество элементов, сформированное в пункте 7
		\item удалить экземпляр действия “действие слияния элементов”
		\item завершить работу.
	\end{scnenumerate}
}
\scntext{Результат}{Пара, которая состоит из структуры решения противоречия и множества элементов, подлежащих удалению при применении решении противоречия.}

\scnnote{//todo пошаговый алгоритм в SCg}

\bigskip
\scnfragmentcaption

\scnheader{Агент слияния элементов}
\scntext{Инициирующие событие}{Появление в sc-памяти инициированного действия, принадлежащего классу "действие слияния элементов"}
\scntext{Входные параметры}{множество элементов для слияния}
\scntext{Алгоритм работы}{
	\begin{scnenumerate}
		\item Создание нового элемента, пересоздание всех связок старых элементов на новый, и их зависимостей.
	\end{scnenumerate}
}
\scntext{Результат}{Новый SC элемент, который будет являться результатом склейки исходных элементов}

\scnnote{//todo пошаговый алгоритм в SCg}

\bigskip
\scnfragmentcaption

\scnheader{Агент слияния структур}
\scntext{Инициирующие событие}{Появление в sc-памяти инициированного действия, принадлежащего классу "действие слияния структур"}
\scntext{Входные параметры}{множество структур для слияния. Наличие отношения rrel\_1 обязательно}
\scntext{Алгоритм работы}{
	\begin{scnenumerate}
		\item перемещение всех элементов в единую структуру.
	\end{scnenumerate}
}
\scntext{Результат}{единая структура, содержащая все элементы}

\scnnote{//todo пошаговый алгоритм в SCg}

\end{SCn}
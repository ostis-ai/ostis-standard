\begin{SCn}
\scnsectionheader{\currentname}

\scnstartsubstruct

\scnheader{компонент ostis-системы}
\scnidtf{независимая часть ostis-системы, которая содержит все те (и только те) sc-элементы, которые необходимы для её функционирования в ostis-системе}

\scnheader{многократно используемый компонент ostis-системы}
\scnidtf{компонент ostis-системы, который содержит все те (и только те) sc-элементы, которые необходимы для функционирования компонента в дочерней ostis-системе}
\scnidtf{компонент некоторой ostis-системы, который может быть использован в другой ostis-системе}
\scnaddlevel{1}
\scnnote{Для этого необходимо выполнение как минимум двух условий:
	\begin{scnitemize}
	\item есть техническая возможность встроить компонент в дочернюю ostis-
систему путем либо физического копирования, переноса и встраивания его
в проектируемую систему, либо использования компонента, размещенного в
исходной системе наподобие сервиса;

	\item использование компонента в каких-либо ostis-системах, кроме метаси-
стемы IMS, является целесообразным, т. е. компонентом не может быть част-
ное решение, ориентированное на узкий круг задач.
	\end{scnitemize}}
\scnaddlevel{-1}
\scnsuperset{компонент ostis-системы}
\scnsubdividing{атомарный многократно используемый компонент ostis-системы;неатомарный многократно используемый компонент ostis-системы}
\scnsubdividing{платформенно-независимый многократно используемый компонент ostis-системы;платформенно-зависимый многократно используемый компонент ostis-системы}
\scnsubdividing{зависимый многократно используемый компонент ostis-системы;независимый многократно используемый компонент ostis-системы}

\scnheader{зависимый многократно используемый компонент ostis-системы}
\scnidtf{компонент, который зависит от хотя бы одного другого компонента, т.е. не может быть встроен в дочернюю ostis-систему без компонентов, от которых он зависит}

\scnheader{независимый многократно используемый компонент ostis-системы}
\scnidtf{компонент, который не зависит от других компонентов}

\scnheader{зависимый компонент*}
\scnidtf{бинарное отношение, связывающее зависимый многократно используемый компонент и компонент, без которого тот не может быть встроен в дочернюю ostis-систему}

\scnheader{несовместимый компонент*}
\scnidtf{бинарное отношение, связывающее два компонента, которые не могут одновременно присутствовать в одной ostis-системе}

\scnheader{атомарный многократно используемый компонент ostis-системы}
\scnidtf{компонент, который в текущем состоянии библиотеки компонентов рассматривается как неделимый, то есть не содержит в своем составе других компонентов}

\scnheader{неатомарный многократно используемый компонент ostis-системы}
\scnidtf{компонент, который в текущем состоянии библиотеки компонентов содержит в своем составе атомарные компоненты}

\scnheader{платформенно-зависимый многократно используемый компонент ostis-системы}
\scnidtf{компонент, частично или полностью реализованный при помощи каких-либо сторонних с точки зрения Технологии OSTIS средств}
\scnrelfromset{недостатки}{\scnfileitem{Интеграция в интеллектуальные системы может сопровождаться дополнительными трудностями, зависящими от конкретных средств реализации компонента.}}
\scnrelfromset{достоинства}{\scnfileitem{Более высокая производительность за счет реализации платформенно-зависимого компонента на более приближенном к платформе уровне.}}
%TODO подумать об этапах интеграции платформенно-зависимого компонента
%\scnnote{В общем случае платформенно-зависимый многократно используемый компонент ostis-системы может поставляться как в виде набора исходных кодов, так и бинарном  виде,  например, в виде скомпилированной библиотеки}
%\scntext{этапы интеграции}{Для того чтобы платформенно-зависимый многократно используемый компонент ostis-системы мог быть успешно встроен в дочернюю систему, необходимо выполнение следующих условий:
%	\begin{scnitemize}
%	\item в состав sc-структуры, соответствующей компоненту, должны входить sc-ссылки, содержащие сведения о местонахождении исходных текстов компонента или уже собранной его версии, то есть ссылки на внешние ресурсы или явно включенные в систему файлы компонента в виде указанных sc-ссылок;
%	\item каждый платформенно-зависимый многократно используемый компонент ostis-системы должен иметь соответствующую подробную, корректную и понятную инструкцию по его установке и внедрению в дочернюю систему
%	\end{scnitemize}}
%\scnaddlevel{1}
%	\scnnote{Процесс интеграции платформенно-зависимого многократно используемого компонента ostis-системы в дочернюю систему, разработанную по Технологии OSTIS, сильно зависит от технологий реализации данного компонента и в каждом конкретном случае может состоять из различных этапов}
%\scnaddlevel{-1}

\scnheader{платформенно-независимый многократно используемый компонент ostis-системы}
\scnidtf{компонент, который целиком и полностью представлен в \textit{sc-коде}}
\scnnote{В случае \textit{неатомарного многократно используемого компонента ostis-системы} это  означает, что все более простые компоненты, входящие в его состав также обязаны быть \textit{платформенно-независимыми многократно используемыми компонентами ostis-системы}.}
% TODO Отношение не очень, не понятно, куда что интегрируем и с чем. учесть следующие вещи: 1) интегрируется не описываемый класс, а его экземпляры 2) интегрируется в любую, но конкретную дочернюю osits-систему из библиотеки
\scnrelfromvector{этапы интеграции}{\scnfileitem{Формирование sc-структуры,  явно содержащей все sc-элементы, входящие в состав компонента, а также все sc-элементы, входящие в спецификацию компонента, необходимую для его функционирования в дочерней системе.}\\
\scnaddlevel{1}
	\scnnote{В случае неатомарного многократно используемого компонента ostis-системы в указанную sc-структуру должны быть полностью включены и все более частные компоненты.}
\scnaddlevel{-1};
\scnfileitem{Проверка совместимости компонента с компонентами дочерней ostis-системы.};
\scnfileitem{При совместимости компонентов, транспортировка компонента в дочернюю систему.};
\scnfileitem{Устранение возникших конфликтов (дублирований понятий, противоречий и т.д.) нового компонента с компонентами дочерней ostis-системы.}}


\scnheader{Библиотека многократно используемых компонентов ostis-систем}
\scnidtf{Библиотека ostis-систем}
\scntext{обоснование}{Проблема разработки новых интеллектуальных систем заключается в отсутствии единой формальной основы, обеспечивающей однозначную интерпретацию представляемых знаний и вводимых
новых понятий. Данный факт приводит к многократной повторной разработке содержательно одних и тех же компонентов для разных интеллектуальных систем. Следовательно приводит к повторной трате ресурсов, усложняя разработку интеллектуальных систем.}
\scnnote{\textit{Библиотека многократно используемых компонентов ostis-систем} включает в себя как сами многократно используемые компоненты ostis-систем, так и средства их спецификации, а также средства поиска компонентов по различным критериям, и, таким образом, представляет собой целую подсистему \textit{Метасистемы IMS.ostis} со своей базой знаний и решателем задач.}
\scnrelfromset{декомпозиция}{Библиотека многократно используемых компонентов баз знаний ostis-систем;Библиотека многократно используемых методов, хранимых в памяти ostis-систем и интерпретируемых их внутренними агентами;Библиотека многократно используемых внутренних агентов ostis-систем;Библиотека многократно используемых компонентов интерфейсов ostis-систем;Библиотека многократно используемых встраиваемых ostis-систем}
\scnrelfromset{достоинства}{\scnfileitem{Накопление разработанных ранее компонентов ostis-систем.};
\scnfileitem{Ускорение разработки новых ostis-систем за счёт разработанных ранее компонентов.};
\scnfileitem{Повышение качества новых ostis-систем за счет использования протестированных и отлаженных компонентов.}}

\bigskip
\scnendstruct \scnendcurrentsectioncomment

\end{SCn}

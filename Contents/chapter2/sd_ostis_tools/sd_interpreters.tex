\begin{SCn}

\scnsectionheader{\currentname}
\scnstartsubstruct

\scnidtf{Предметная область и онтология платформ интерпретации sc-моделей компьютерных систем}
\scniselement{раздел базы знаний}
\scnrelfromlist{дочерний раздел}{\nameref{sd_program_interp};\nameref{sd_sem_comp}}

\scnheader{Предметная область базовых интерпретаторов логико-семантических моделей ostis-систем}
\scniselement{предметная область}
\scnsdmainclasssingle{платформа интерпретации sc-моделей компьютерный систем}
%\scnsdclass{***}
%\scnsdrelation{***}

\scnheader{логико-семантическая модель кибернетической системы}
\scnidtf{формальная модель (формальное описание) функционирования кибернетической системы, состоящая из:
	\begin{scnitemize}
		\item формальной модели информации, хранимой в памяти кибернетической системы;
		\item формальной модели коллектива агентов, осуществляющих обработку указанной информации.
\end{scnitemize}}
\scnsuperset{sc-модель кибернетической системы}
\scnaddlevel{1}
\scnidtf{логико-семантическая модель кибернетической системы, представленная в SC-коде}
\scnidtf{логико-семантическая модель ostis-системы, которая, в частности, может быть функционально эквивалентной моделью какой-либо кибернетической системы, не являющейся ostis-системой}
\scnaddlevel{-1}

\scnheader{кибернетическая система}
\scnsuperset{компьютерная система}
\scnaddlevel{1}
\scnidtf{искусственная кибернетическая система}
\scnsuperset ostis-система}
\scnaddlevel{1}
\scnidtf{компьютерная система, построенная по технологии OSTIS на основе интерпретации спроектированной логико-семантическая sc-модель этой системы}
\scnaddlevel{-2}

\scnheader{платформа интерпретации sc-моделей компьютерных систем}
\scnidtf{базовый интерпретатор логико-семантических моделей ostis-систем}
\scnidtf{Семейство платформ интерпретации sc-моделей компьютерных систем}
\scnidtf{платформа реализации sc-моделей компьютерных систем}
\scnidtftext{часто используемый sc-идентификатор}{универсальный интерпретатор sc-моделей компьютерных систем}
\scnidtf{универсальный интерпретатор унифицированных логико-семантических моделей компьютерных систем}
\scnsubset{встроенная ostis-система}
\scnidtf{встроенная пустая ostis-система}
\scnidtf{универсальный интерпретатор sc-моделей ostis-систем}
\scnidtf{универсальная базовая ostis-система, обеспечивающая имитацию любой ostis-системы путем интерпретации sc-модели имитируемой ostis-системы}
\scnnote{соотношение между имитируемой и универсальной ostis-системой в известной мере аналогично соотношению между машиной Тьюринга и универсальной машиной Тьюринга}
\scnexplanation{Под \textbf{\textit{платформой интерпретации sc-моделей компьютерных систем}} понимается реализация платформы интерпретации sc-моделей, которая в общем случае включает в себя:

Реализация \textit{платформы интерпретации sc-моделей компьютерных систем} (\textit{универсального интерпретатора sc-моделей компьютерных систем}) может иметь большое число вариантов -- как программно, так и аппаратно реализованных. Логическая архитектура \textit{платформы интерпретации sc-моделей компьютерных систем} обеспечивает независимость проектируемых компьютерных систем от многообразия вариантов реализации интерпретатора их моделей и в общем случае включает в себя:

\begin{scnitemize}
    \item хранилище \textit{sc-текстов} (\textit{sc-хранилище}, хранилище знаковых конструкций, представленных SC-коде);
    \item файловую память \textit{sc-машины};
    \item средства, обеспечивающие взаимодействие \textit{sc-агентов} над общей памятью (sc-памятью);
    \item базовые средства интерфейса для взаимодействия системы с внешним миром (пользователем или другими системами). Указанные средства включают в себя, как минимум, редактор, транслятор (в sc-память и из нее) и визуализатор для одного из базовых универсальных вариантов представления \textit{SC-кода} (\textit{SCg-код}, \textit{SCs-код}, \textit{SCn-код}), средства, позволяющие задавать системе вопросы из некоторого универсального класса (например, запрос семантической окрестности некоторого объекта);
    \item реализацию \textit{Абстрактной scp-машины}, то есть интерпретатор \textit{scp-программ} (программ Языка SCP).
\end{scnitemize}
При необходимости, в \textbf{\textit{платформу интерпретации sc-моделей компьютерных систем}} могут быть заранее на платформенно-зависимом уровне включены какие-либо компоненты машин обработки знаний или баз знаний, например, с целью упрощения создания первой версии \textit{прикладной ostis-системы}.

Реализация платформы может осуществляться на основе произвольного набора существующих технологий, включая аппаратную реализацию каких-либо ее частей. С точки зрения компонентного подхода любая \textbf{\textit{платформа интерпретации sc-моделей компьютерных систем}} является \textbf{\textit{платформенно-зависимым многократно используемым компонентом}}.}
\scnsuperset{программный вариант реализации платформы интерпретации sc-моделей компьютерных систем}
\scnsuperset{семантический ассоциативный компьютер}
\scnsubdividing{однопользовательский вариант реализации платформы интерпретации sc-моделей компьютерных систем\\
	\scnaddlevel{1}
		\scnidtf{вариант реализации платформы интерпретации sc-моделей компьютерных систем, рассчитанный на то, что с конкретной ostis-системой взаимодействует только один пользователь (владелец)}
		\scnnote{При таком варианте реализации платформы оказывается невозможным реализовать некоторые важные принципы \textit{Технологии OSTIS}, например, коллективную согласованную разработку базы знаний системы в процессе ее эксплуатации. При этом могут использоваться различные сторонние средства, например для разработки базы знаний на уровне исходных текстов.}
	\scnaddlevel{-1}
;многопользовательский вариант реализации платформы интерпретации sc-моделей компьютерных систем\\
\scnaddlevel{1}
	\scnidtf{вариант реализации платформы интерпретации sc-моделей компьютерных систем, рассчитанный на то, что с конкретной ostis-системой одновременно или в разное время могут взаимодействовать разные пользователи, в общем случае обладающие разными правами, сферами ответственности, уровнем опыта, и имеющие свою конфиденциальную часть хранимой в базе знаний информации}
\scnaddlevel{-1}}

\scnheader{платформа интерпретации sc-моделей компьютерных систем}
\scnsubdividing{программный вариант реализации платформы интерпретации sc-моделей компьютерных систем\\
	\scnaddlevel{1}
	\scnidtf{программная платформа интерпретации sc-моделей ostis-систем}
	\scnidtf{программный базовый интерпретатор sc-моделей ostis-систем}
	\scnaddlevel{-1}
	;семантический ассоциативный компьютер
	\scnaddlevel{1}
	\scnidtf{аппаратная платформа интерпретации sc-моделей ostis-систем}
	\scnidtf{аппаратно реализованный базовый интерпретатор sc-моделей ostis-систем}
	\scnaddlevel{-1}
}

\newpage

\scnfragmentheader{Классификация sc-элементов, на основе которой осуществляется синтаксическая спецификация sc-элементов}
\scnstartsubstruct

\scnheader{синтаксическая классификация sc-элемента}
\scnexplanation{При реализации \textit{sc-памяти} \scnbigspace \textit{синтаксическая спецификация sc-элементов}, хранимых в этой памяти, (т.е. \textit{спецификация sc-элементов}, реализуемая не средствами \textit{SC-кода}, а синтаксическим способом, неявно), осуществляется путем приписывания каждому хранимому \textit{sc-элементу} вектора значений \uline{унифицированного набора признаков}. Каждый из этих признаков характеризует принадлежность или непринадлежность специфицируемого \textit{sc-элемента} некоторому конкретному \textit{классу sc-элементов}. Очень важно задать такое семейство указанных \textit{классов sc-элементов}, которое
\begin{scnitemize}
	\item было бы достаточно компактным;
	\item обеспечивало бы достаточно высокую производительность при обработке sc-конструкций.
\end{scnitemize}

Ниже приведена классификация \textit{sc-элементов}, позволяющая выделить такое семейство \textit{классов sc-элементов}, состоящее из 34-х классов (типов) sc-элементов.

Синтаксически \textit{sc-ребром} можно считать \textit{sc-элемент}, принадлежащий следующему классу:

(\textit{обозначение пары, у которой известны оба ее компонента}
$\cap$ \textit{обозначение пары, которая не является компонентом другой пары} $\cap$ (\textit{обозначение неориентированной пары} $\cup$ \textit{обозначение пары, о которой не известно, является она ориентированной или неориентированной} $\cup$ \textit{обозначение ориентированной пары неизвестной направленности})).

При этом \textit{sc-ребро} формально можно представить либо в виде \textit{sc-дуги}, направленность которой игнорируется и, следовательно, может быть \uline{произвольной}(!), либо в виде \textit{sc-узла}, из которого явно выходят \textit{sc-дуги}, являющиеся \textit{обозначениями пар инцидентности}, которые связывают \textit{sc-ребро} с его компонентами (т.е. с \textit{sc-элементами}, соединяемыми этим \textit{sc-ребром}).

Синтаксически sc-дугой следует считать sc-элемент, который принадлежит классу (\textit{обозначение пары, у которой известны оба ее компонента} $\cap$ \textit{обозначение пары, которая не является компонентом другой пары} $\cap$ \textit{обозначение ориентированной пары известной направленности}).

Все остальные \textit{sc-элементы} с синтаксической точки зрения следует считать \textit{sc-узлами}.}

\scnheader{sc-элемент}
\scnsubdividing{sc-переменная\\
	\scnaddlevel{1}
		\scnsubdividing{sc-переменная, каждое значение которой является sc-константой;sc-переменная, каждое значение которой является sc-переменной;sc-переменная, во множество возможных значений которой входят как sc-константы, так и sc-переменные;sc-переменная, характер множества возможных значений которой не известен}
	\scnaddlevel{-1}
	;sc-константа;sc-элемент, константность или переменность которого не известна}
\scnsubdividing{обозначение временной сущности;обозначение постоянной сущности;обозначение сущности, временность или постоянство которой либо не известны, либо принципиально не могут быть установлены}
\scnsubdividing{логически удаленный sc-элемент\\
	\scnaddlevel{1}
		\scnidtf{прошлый sc-элемент}
	\scnaddlevel{-1};
	sc-элемент, логически присутствующий в текущий момент
	\scnaddlevel{1}
		\scnidtf{настоящий sc-элемент}
	\scnaddlevel{-1}
}
\scnsubdividing{обозначение файла ostis-системы;обозначение терминальной сущности, не являющейся файлом ostis-системы;обозначение множества всех синтаксически эквивалентных файлов заданной структуры\\
	\scnaddlevel{1}
		\scnidtf{обозначение всевозможных копий файла заданной структуры}
	\scnaddlevel{-1}
	;обозначение сущности, принадлежность которой классу множеств/терминальных сущностей/файлов ostis-системы не установлена;обозначение множества, которое не является множеством всех синтаксически эквивалентных файлов заданной структуры, хранимых в памяти ostis-системы\\
	\scnaddlevel{1}
		\scnsubdividing{обозначение множества, которое не является множеством всех синтаксически эквивалентных файлов заданной структуры и не является парой;обозначение множества, о котором не известно, является оно парой или нет;обозначение пары\\
		\scnaddlevel{1}
			\scnidtf{обозначение множества, мощность которого равна двум}
			\scnsubdividing{обозначение пары, у которой известны об ее компонента;обозначение пары, у которой неизвестен один ее компонент;обозначение пары, у которой неизвестны оба ее компонента}
			\scnsubdividing{обозначение пары, которая не является компонентом другой пары;обозначение пары, которая является компонентом другой пары}
			\scnsubdividing{обозначение неориентированной пары;обозначение пары, о которой не известно, является она ориентированной или нет;обозначение ориентированной пары\\
			\scnaddlevel{1}	
				\scnsubdividing{обозначение ориентированной пары известной направленности;обозначение ориентированной пары неизвестной направленности}
				\scnsubdividing{обозначение ориентированной пары, не являющейся парой принадлежности;обозначение ориентированной пары, о которой не известно, является ли она парой принадлежности или нет;обозначение пары принадлежности\\
				\scnaddlevel{1}	
					\scnsubdividing{обозначение позитивной пары принадлежности;обозначение негативной пары принадлежности;обозначение нечеткой пары принадлежности}
					\scnsubdividing{обозначение пары принадлежности, которая не является парой инцидентности, явно связывающей обозначение некоторой другой пары с одним из ее компонентов;обозначение пары инцидентности, явно связывающей обозначение некоторой ориентированной пары со вторым компонентом этой пары;обозначение пары инцидентности, явно связывающей либо обозначение некоторой ориентированной пары с первым компонентом этой пары, либо обозначение неориентированной пары с одним из ее компонентов}
				\scnaddlevel{-1}		
				}
			\scnaddlevel{-1}	
			}
		\scnaddlevel{-1}
		}
	\scnaddlevel{-1}}

\scnendstruct

\bigskip

\scnendstruct \scnendcurrentsectioncomment

\end{SCn}
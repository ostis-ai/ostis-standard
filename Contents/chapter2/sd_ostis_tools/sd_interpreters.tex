\begin{SCn}

\scnsectionheader{\currentname}
\scnrelfromlist{подраздел}{Предметная область и онтология программных вариантов реализации базового интерпретатора семантических моделей ostis-систем на современных компьютерах;Предметная область и онтология семантических ассоциативных компьютеров для ostis-систем}

\scnstartsubstruct

\scnheader{Предметная область базовых интерпретаторов семантических моделей ostis-систем}
\scnsdmainclasssingle{***}
\scnsdclass{***}
\scnsdrelation{***}

\scnheader{универсальный интерпретатор sc-моделей компьютерных систем}
\scnsuperset{встроенная ostis-система}
\scnidtf{встроенная пустая ostis-система}
\scnidtf{универсальный интерпретатор sc-моделей ostis-систем}
\scnidtf{универсальная базовая ostis-система, обеспечивающая имитацию любой ostis-системы путем интерпретации sc-модели имитируемой ostis-системы}
\scnaddlevel{1}
\begin{adjustwidth}{0.4em}{0em}
\scnnote{соотношение между имитируемой и универсальной ostis-системой в известной мере аналогично соотношению между машиной Тьюринга и универсальной машиной Тьюринга}
\end{adjustwidth}
\scnresetlevel
\scnidtf{интерпретатор программ языка SCP -- Semantic Code Programming}

\scnidtf{scp-машина}

\scntext{реализация}{Реализация \textit{универсального интерпретатора sc-моделей компьютерных систем} может иметь большое число вариантов -- как программно, так и аппаратно реализованных. Логическая архитектура \textit{универсального интерпретатора sc-моделей компьютерных систем} обеспечивает независимость проектируемых компьютерных систем от многообразия вариантов реализации интерпретатора их моделей и включает в себя:

\begin{scnitemize}
    \item \textit{смысловую графовую ассоциативную память} (sc-память, sc-хранилище знаковых конструкций, представленных SC-коде);
    \item \textit{интерпретатор языка SCP} - базового процедурного языка программирования, ориентированного на обработку текстов SC-кода, хранимых в смысловой графовой ассоциативной памяти.
\end{scnitemize}}

\scnheader{платформа интерпретации sc-моделей компьютерных систем}
\scnauthorcomment{Интегрировать с предыдущим}
\scnidtf{Библиотека платформ реализации sc-моделей компьютерных систем}
\scnidtf{Семейство платформ интерпретации sc-моделей компьютерных систем}
\scnidtf{Библиотека интерпретаторов унифицированных логико-семантических моделей компьютерных систем}
\scnidtf{интерпретатор унифицированных логико-семантических моделей компьютерных систем}
\scnidtf{платформа реализации sc-моделей компьютерных систем}
\scnexplanation{Под \textbf{\textit{платформой интерпретации sc-моделей компьютерных систем}} понимается реализация платформы интерпретации sc-моделей, которая в общем случае включает в себя:

\begin{scnitemize}
    \item хранилище \textit{sc-текстов} (\textit{sc-хранилище});
    \item файловую память \textit{sc-машины};
    \item средства, обеспечивающие взаимодействие \textit{sc-агентов} над общей памятью;
    \item базовые средства интерфейса для взаимодействия системы с внешним миром (пользователем или другими системами). Указанные средства включают в себя, как минимум, редактор, транслятор (в sc-память и из нее) и визуализатор для одного из базовых универсальных вариантов представления \textit{SC-кода} (\textit{SCg-код}, \textit{SCs-код}, \textit{SCn-код}), средства, позволяющие задавать системе вопросы из некоторого универсального класса (например, запрос семантической окрестности некоторого объекта);
    \item реализацию \textit{абстрактной scp-машины}, то есть интерпретатор \textit{scp-программ}.
\end{scnitemize}
При необходимости, в \textbf{\textit{платформу интерпретации sc-моделей компьютерных систем}} могут быть заранее на платформенно-зависимом уровне включены какие-либо компоненты машин обработки знаний или баз знаний, например, с целью упрощения создания первой версии дочерней системы на основе \textit{Технологии OSTIS}.

Реализация платформы может осуществляться на основе произвольного набора существующих технологий, включая аппаратную реализацию каких-либо ее частей. С точки зрения \textit{Технологии OSTIS} любая \textbf{\textit{платформа интерпретации sc-моделей компьютерных систем}} является \textbf{\textit{платформенно-зависимым многократно используемым компонентом}}.}

\scnendstruct

\end{SCn}
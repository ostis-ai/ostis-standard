\begin{SCn}
	
\scnsectionheader{\currentname}
	
\scnstartsubstruct


	\scnheader{Решатель задач пользовательского интерфейса}
	\scnreltoset{декомпозиция абстрактного sc-агента} {
		Агент интерпретации sc-модели базы знаний пользовательского интерфейса
		;Агент обработки пользовательских действий
	}

\textit{Агент интерпретации sc-модели базы знаний пользовательского интерфейса} в качестве входного параметра принимает экземпляр компонента пользовательского интерфейса для отображения. При этом компонент может быть как атомарным, так и неатомарным (например, компонент главного окна приложения). Результатом работы агента является графическое представление указанного компонента с учетом используемой реализации платформы интерпретации семантических моделей ostis-систем.

Алгоритм работы данного агента следующий:
\begin{itemize}
	\item проверяется тип входного компонента (атомарный или неатомарный);
	\item если компонент является атомарным, то отобразить его графическое представление на основании указанных для него свойств. В случае, если данный компонент не входит в декомпозицию любого другого компонента - завершить выполнение. Иначе определить компонент, в декомпозицию которого входит рассматриваемый компонент, применить его свойства для текущего атомарного компонента и начать обработку найденного неатомарного компонента, перейдя к первому пункту;
	\item если компонент является неатомарным, то проверить, были ли отображены компоненты, на которые он был декомпозирован. Если да, то завершить выполнение, иначе определить еще не отображенный компонент из декомпозиции обрабатываемого неатомарного компонента и начать обработку найденного компонента, перейдя к первому пункту.
\end{itemize}

\textit{Агент обработки пользовательских действий} является неатомарным агентом, который включает в себя множество агентов, каждый из которых обрабатывает действия пользователя определенного класса (например, агент обработки действия нажатия мыши, агент обработки действия отпускания мыши и так далее). Агент реагирует на появление в базе знаний системы экземпляра интерфейсного действия пользователя, находит связанный с ним класс внутреннего действия и генерирует экземпляр данного внутреннего действия для последующей обработки.



\section{Реализация предлагаемого подхода}

Текущий вариант реализации платформы интерпретации sc-моделей является web-ориентированным \cite{web-platform}.

C учетом особенностей платформы и для возможности интеграции предлагаемого подхода с существующими решениями в области создания пользовательских интерфейсов предлагается агент интерпретации sc-модели базы знаний пользовательского интерфейса реализовать как неатомарный агент, который декомпозируется на агент трансляции sc-модели базы знаний пользовательского интерфейса в формат, совместимый с существующими решениями и агент отображения указанного формата в графическое представление пользовательского интерфейса.

В качестве промежуточного формата описания предлагается использовать JSON формат по ряду причин:
\begin{itemize}
	\item это наиболее популярный формат для передачи и хранения данных в современных системах;
	\item компактный и простой синтаксиса;
	\item простота внесения изменений;
	\item простота передачи и обработки.
\end{itemize}

Таким образом, вводится дополнительный \textit{агент трансляции описания компонента пользовательского интерфейса из sc-модели в JSON формат}. В качестве входного параметра данный агент принимает экземпляр компонента пользовательского интерфейса трансляции в JSON формат. Формирование JSON описания осуществляется путем рекурсивной обработки описания компонентов от неатомарных к атомарным.

\textit{Агент отображения указанного формата в графическое представление пользовательского интерфейса} является неатомарным и декомпозируется на набор агентов, выполняющих отображение для различных платформ интерпретации (web-платформы, мобильной платформы, настольного компьютера и так далее). На вход данный агент принимает описание компонента пользовательского интерфейса в JSON формате. Результат работы агента - графическое представление пользовательского интерфейса.

\scnendstruct \scnendcurrentsectioncomment
\end{SCn}
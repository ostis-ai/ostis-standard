\begin{SCn}
	\scnheader{user interface component}
		\scnrelfromlist{включение}{presentation user interface component\\
				\scnaddlevel{1}
			{\scnrelfromlist{включение}{output\\
				\scnaddlevel{1}
				\scnrelfromlist{включение}{image-output;graphical-output\\
				{	\scnaddlevel{1}
					\scnrelfromlist{включение}{chart; map; progress-bar 	\scnaddlevel{-1}}}				
				;video-output\\
				;sound-output\\
				;text-output\\{
				\scnaddlevel{1}
				\scnrelfromlist{включение}{headline; paragraph;message}}}
				\scnaddlevel{-2}
				;decorative user interface component\\
				;label
				;container\\
				\scnaddlevel{1}
				\scnrelfromlist{включение}{single-purpose-container\\
					\scnaddlevel{1}
					\scnrelfromlist{включение}{menu
						;menu-bar
						;tool-bar}
					\scnaddlevel{-1}
					;multi-purpose-container\\
						\scnaddlevel{1}
					\scnrelfromlist{включение}{status-bar;table-row-container;list-container;table-cell-container;tree-container;labeled-group;tab-pane;spin-pane;tree-node-container;scroll-pane
						;window\\
						{
							\scnaddlevel{1}
						\scnrelfromlist{включение}{modal-window;non-modal-window}
						\scnaddlevel{-1}}	}
				\scnaddlevel{-1}
			}	\scnaddlevel{-1}}
			\scnaddlevel{-1}	}
		;interactive user interface component\\
			\scnaddlevel{1}
	\scnrelfromlist{включение}{data-input-component		
		;presentation-manipulation-component \\	
		;operation-trigger-component
	}	\scnaddlevel{-1}
}


\scnheader{presentation user interface component}
\scnexplanation{\textbf{presentation user interface component} -- это класс компонентов пользовательского интерфейса, предназначенный для представления информации.}


\scnheader{output}
\scnexplanation{\textbf{output} -- это класс компонентов пользовательского интерфейса, предназначенный для представления выходной информации.}

\scnheader{image-output}
\scnexplanation{\textbf{image-output} -- это класс компонентов пользовательского интерфейса, предназначенный для отображения изображений.}

\scnheader{graphical-output}
\scnexplanation{\textbf{graphical-output} -- это класс компонентов пользовательского интерфейса, предназначенный для представления графической информации.}

\scnheader{chart}
\scnexplanation{\textbf{chart} -- это компонент пользовательского интерфейса, предназначенный для отображения графиков.}

\scnheader{map}
\scnexplanation{\textbf{map} -- это компонент пользовательского интерфейса, предназначенный для отображения карт.}

\scnheader{progress-bar}
\scnexplanation{\textbf{progress-bar} -- это компонент пользовательского интерфейса, предназначенный для отображения процента выполнения какой-либо задачи.}

\scnheader{video-output}
\scnexplanation{\textbf{video-output} -- это класс компонентов пользовательского интерфейса, предназначенный для представления видеоинформации.}

\scnheader{sound-output}
\scnexplanation{\textbf{sound-output} -- это класс компонентов пользовательского интерфейса, предназначенный для представления аудиоинформации.}

\scnheader{text-output}
\scnexplanation{\textbf{text-output} -- это класс компонентов пользовательского интерфейса, предназначенный для представления текстовой информации.}

\scnheader{headline}
\scnexplanation{\textbf{headline} -- это компонент пользовательского интерфейса, предназначенный для отображения заголовков.}

\scnheader{paragraph}
\scnexplanation{\textbf{paragraph} -- это компонент пользовательского интерфейса, предназначенный для отображения блоков текста. Он отделяется от других блоков пустой строкой или первой строкой с отступом.}

\scnheader{message}
\scnexplanation{\textbf{message} -- это компонент пользовательского интерфейса, предназначенный для отображения сообщения.}

\scnheader{decorative user interface component}
\scnexplanation{\textbf{decorative user interface component} -- это класс компонентов пользовательского интерфейса, предназначенный для стилизации интерфейса.}

\scnheader{label}
\scnexplanation{\textbf{label} -- это класс компонентов пользовательского интерфейса, предназначенный для отображения названий.}

\scnheader{container}
\scnexplanation{\textbf{container} -- это компонент пользовательского интерфейса, задача которого состоит в размещении набора компонентов, включённых в его состав.
}

\scnheader{single-purpose-container}
\scnexplanation{\textbf{single-purpose-container} --это контейнер, предназначенный только для одной цели.}

\scnheader{menu}
\scnexplanation{\textbf{menu} -- это компонент пользовательского интерфейса, с помощью которого пользователь может выбирать из нескольких вариантов.}

\scnheader{menu-bar}
\scnexplanation{\textbf{menu-bar} -- это раскрывающееся меню.}

\scnheader{tool-bar}
\scnexplanation{\textbf{toolbar} -- это компонент пользовательского интерфейса, на котором размещаются экранные кнопки, значки, меню или другие элементы ввода или вывода.}

\scnheader{multi-purpose-container}
\scnexplanation{\textbf{multi-purpose-container} -- это контейнер, предназначенный для нескольких целей.}

\scnheader{status-bar}
\scnexplanation{\textbf{status-bar} -- это компонент пользовательского интерфейса, предназначенный для отображения информации о текущем состоянии окна.}

\scnheader{list-container}
\scnexplanation{\textbf{list-container} -- это контейнер, который располагает элементы интерфейса в виде списка.}

\scnheader{table-cell-container}
\scnexplanation{\textbf{table-cell-container} -- это контейнер, который располагает элементы интерфейса в виде таблицы.}

\scnheader{tree-container}
\scnexplanation{\textbf{tree-container} -- это контейнер, который располагает элементы интерфейса в виде дерева.}

\scnheader{tab-pane}
\scnexplanation{\textbf{tab-pane} -- это контейнер , который может содержать несколько вкладок (секций) внутри, которые могут быть отображены, нажав на вкладке с названием в верхней части tab-pane. Одновременно отображается только одна вкладка.}

\scnheader{scroll-pane}
\scnexplanation{\textbf{scroll-pane} -- это компонент пользовательского интерфейса, который позволяет пользователю прокручивать контент.}

\scnheader{window}
\scnexplanation{\textbf{window} -- это обособленная область экрана, содержащая различные элементы пользовательского интерфейса. Окна могут располагаться поверх друг друга.}

\scnheader{modal-window}
\scnexplanation{\textbf{modal-window} -- это окно, которое блокирует работу пользователя с родительским приложением до тех пор, пока пользователь это окно не закроет.}

\scnheader{non-modal-window}
\scnexplanation{\textbf{non-modal-window} -- это окно, которое позволяет перемещать фокус между окном диалога и другой формой без необходимости закрыть окно диалога.}

\scnheader{interactive user interface component}
\scnexplanation{\textbf{interactive user interface component} -- это класс компонентов пользовательского интерфейса, с помощью которого осуществляется взаимодействие с пользователем.}


\scnheader{data-input-component}
\scnexplanation{\textbf{data-input-component} -- это класс компонентов пользовательского интерфейса, предназначенный для ввода информации.}

\scnheader{presentation-manipulation-component}
\scnexplanation{\textbf{presentation-manipulation-component} -- это класс компонентов пользовательского интерфейса, предназначенный для представления информации и взаимодействия с пользователем.}

\scnheader{operation-trigger-component}
\scnexplanation{\textbf{operation-trigger-component} -- это класс компонентов пользовательского интерфейса, которые провоцируют пользователя совершить действие.}

\end{SCn}
\scseparatedfragment[\scnidtf{Предисловие Стандарта OSTIS-2021}]{Предисловие к первому изданию Стандарта OSTIS}

\begin{SCn}
	
	\scnsectionheader{\currentname}
	
	\scneqfile{В настоящее время уровень требований, предъявляемых к комплексу \textit{технологий Искусственного интеллекта} существенно повысился -- возникла необходимость разработки \textit{компьютерных систем}, которые не только обладают высоким уровнем \textit{интеллекта}, но и обладают \textit{семантической совместимостью}, взаимопониманием, способностью координировать свою деятельность с другими системами при коллективном решении сложных "внештатных"{} задач. Очевидно, что эти требования предполагают существенное развитие  \textit{стандартов интеллектуальных компьютерных систем}. Важнейшая особенность \textit{стандарта интеллектуальных компьютерных систем}, обеспечивающего их \textit{семантическую совместимость} и взаимопонимание, заключается в том, что для этого \textit{интеллектуальные компьютерные системы} должны использовать:
		\begin{scnitemize}
			\item   один и тот же язык внутреннего представления знаний;
			\item   один и тот же язык их коммуникации;
			\item   одну и ту же \textit{систему понятий};
			\item   обладать достаточно большим количеством  общих (одинаковых) знаний.
		\end{scnitemize}
		
		Следовательно, разработка \textit{стандарта интеллектуальных компьютерных систем} в той части этого \textit{стандарта}, которая связана с выделением и формализацией общих (одинаковых) \textit{знаний}, хранимых в \textit{памяти интеллектуальной компьютерной системы} и необходимых для обеспечения их взаимопонимания, фактически осуществляет разработку достаточно большой  одинаковой части всех \textit{интеллектуальных компьютерных систем} (и не только прикладных),  что существенно  сокращает сроки их разработки.
		
		К этому можно добавить возможность и целесообразность  одинаковой  для всех \textit{интеллектуальных компьютерных систем} реализации целого ряда их способностей: 
		\begin{scnitemize}
			\item способности \uline{понимать} информацию, которой они обмениваются между собой,
			\item способности  \uline{договариваться} и \uline{координировать} свои действия при коллективном решении сложных интеллектуальных задач в условиях нештатных (аномальных, нестандартных, непредусмотренных) ситуаций,
			\item способности \uline{принимать решения} на основе их глубокого  обоснования,
			\item способности \uline{обучаться} и многие другие способности, обеспечивающие необходимый уровень интеллекта разрабатываемых интеллектуальных компьютерных систем.
		\end{scnitemize}
		
		Данная монография является первым этапом на пути комплексного решения указанных выше проблем.
		Предназначена она одновременно:
		\begin{scnitemize}
			\item для студентов, магистрантов и аспирантов, обучающихся по специальности ``\textit{Искусственный интеллект}'';
			\item для разработчиков прикладных \textit{интеллектуальных компьютерных систем};
			\item для разработчиков технологий проектирования и производства \textit{интеллектуальных компьютерных систем};
			\item для научных работников, создающих новые \textit{модели} и \textit{методы} для решения \textit{интеллектуальных задач}.
		\end{scnitemize}
		
		Данная монография сочетает:
		\begin{scnitemize}
			\item строгую формализацию представляемой \textit{информации} и её доступность (возможность первичного её понимания без предварительного изучения используемого \textit{\uline{формального} языка});
			\item традиционную ("пассивную"{}) форму представления материала (в "электронном"{} и "бумажном"{} виде) с "активной"{} формой в виде \textit{интеллектуальной справочной системы}, когда \textit{компьютерная система} не только обеспечивает оперативное \textit{редактирование информации}, но и помогает пользователям  быстрее и  качественнее усваивать эту \textit{информацию} (за счет возможности отвечать на широкий спектр вопросов и учитывать индивидуальные особенности, потребности и интересы \textit{пользователей}). 
		\end{scnitemize}
		
		Область \textit{Искусственного интеллекта} сочетает в себе как  научно-исследовательский аспект и   создание \textit{технологий} разработки \textit{интеллектуальных компьютерных систем}, а так и непосредственно разработку самих \textit{интеллектуальных компьютерных систем}. Эта область развивается настолько быстрыми темпами, что за время обучения студентов и магистрантов ситуация в области \textit{Искусственного интеллекта} меняется существенно, поэтому подготовка специалистов в этой области требует особого подхода, учитывающего высокий уровень сложности этой \textit{научно-технической области}, а также быстрые темпы развития теории \textit{интеллектуальных компьютерных систем}, технологий их проектирования, а также непосредственно практики разработки конкретных прикладных \textit{интеллектуальных компьютерных систем}.
		
		Если специалист в области \textit{Искусственного интеллекта} не будет постоянно ориентироваться в тенденциях развития каждого из этих направлений развития работ в этой области, то он быстро перестанет быть конкурентноспособным. Это значит, что специалист в области \textit{Искусственного интеллекта} должен быть в достаточной степени и ученым, и создателем технологий следующего поколения, и разработчиком конкретных приложений.
		
		Таким образом, подготовку специалистов в области \textit{Искусственного интеллекта} необходимо ориентировать не на конкретное состояние науки, технологии и практики в этой области, а на перманентный процесс эволюции всех этих направлений.
		
		Сформировать у студентов и магистрантов реальные навыки в области \textit{Искусственного интеллекта} можно только путем поэтапного и непосредственного их включения в  реальную деятельность в этой области (и в \textit{научно-исследовательскую деятельность}, и в развитие \textit{технологий искусственного интеллекта}, и в разработку \textit{прикладных интеллектуальных компьютерных систем} на основе текущего состояния соответствующих технологий). Но для этого необходимо создать соответствующую научно-исследовательскую и инженерную инфраструктуру.
		
		\textit{Научно-исследовательская деятельность  в области Искусственного интеллекта} предполагает исследование феномена \textit{интеллекта} и создание принципиально новых подходов (моделей и методов) к решению \textit{интеллектуальных задач} и к разработке принципов организации соответствующих \textit{компьютерных систем}.
		
		\uline{\textit{Развитие технологий Искусственного интеллекта}} включает в себя:
		\begin{scnitemize}
			\item разработку стандарта интеллектуальных компьютерных систем, соответствующего текущему состоянию \textit{технологий искусственного интеллекта};
			\item разработку методов, средств проектирования и реализации интеллектуальных компьютерных систем.
		\end{scnitemize}
		
		\textit{Разработка прикладных интеллектуальных компьютерных} систем  предполагает грамотное применение соответствующих \textit{технологий}.
		
		
		\textit{Учебную деятельность в области Искусственного интеллекта} необходимо ориентировать не только на формирование навыков разработки \textit{прикладных интеллектуальных компьютерных систем} по заданной \textit{технологии}, но и на формирование навыков перманентного совершенствования как непосредственно самих \textit{прикладных интеллектуальных компьютерных систем}, так и \textit{технологий} их разработки, а также на изучение принципов (моделей и методов) решения \textit{интеллектуальных задач} и организации \textit{интеллектуальных систем}.
		
		
		Данная монография рассматривается нами как первый выпуск целой серии коллективных монографий, которые будут представлять последующие версии \textbf{\textit{Стандарта Технологии OSTIS}} (Open Semantic Technology for Intelligent Systems -- Стандарта \textit{технологии}, ориентированной на разработку \textit{семантически совместимых интеллектуальных компьютерных систем}). При этом предполагается существенное расширение авторского коллектива и организация всей работы на развитие \textit{Стандарта Технологии OSTIS} как \textit{открытого проекта}, целью которого является коллективное совершенствование \textit{базы знаний}, посвященной детальному описанию этого \textit{стандарта}.
		
		При этом при подготовке даже данного издания текущей версии \textit{Стандарта Технологии OSTIS} мы приобрели хороший опыт организации коллективной деятельности такого рода, привлекая к этой работе целый ряд аспирантов, магистрантов и студентов, а также сотрудников других организаций.
		
		Вклад некоторых из них в ряд разделов монографии позволил включить их в число \textit{соавторов} этих разделов, что отражено непосредственно в тексте монографии.
		
		Авторы выражают благодарность:
		
		\begin{scnitemize}
			\item Cотрудникам кафедры Интеллектуальных информационных технологий Белорусского государственного университета информатики и радиоэлектроники и кафедры Интеллектуальных информационных технологий Брестского государственного технического университета, а также сотрудникам ОАО <<Савушкин продукт>>;
			\item студентам кафедры Интеллектуальных информационных технологий Белорусского государственного университета информатики и радиоэлектроники Банцевич К.А., Бутрину С.В., Василевской А.П., Меньковой Е.А., Жмырко А.В., Григорьевой И.В., Загорскому А.Г., Марковцу В.С., Киневичу Т.О. за оказание технической помощи при подготовке текста к печати;
			\item ООО <<Интелиджент семантик системс>> и его генеральному директору Т. Грюневальду за финансовую поддержку работ по развитию \textit{Технологии OSTIS}, а также финансовую поддержку издания \textit{Стандарта OSTIS};
			\item Рецензентам -- д-ру техн. наук, профессору Александру Николаевичу Курбацкому и д-ру техн. наук, профессору Александру Арсентьевичу Дудкину;
			\item Коллегам из Советской (ныне Российской) Ассоциации Искусственного интеллекта и коллегам из Белорусского объединения специалистов в области искусственного интеллекта;
			\item Членам Программного Комитета ежегодных \textit{конференций OSTIS}, а также всем участникам этих конференций за плодотворное и конструктивное обсуждение направлений развития семантических технологий и \textit{Технологии OSTIS} в частности.
	\end{scnitemize}}
	
	\newpage
	
\end{SCn}
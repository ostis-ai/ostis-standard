\scnsegmentheader{Сегмент. Общие принципы, лежащие в основе ассоциативных семантических компьютеров для ostis-систем}
\begin{scnsubstruct}
	
	\scnheader{ассоциативный семантический компьютер}
	\scntext{примечание}{На данном этапе работы были выявлены несколько основных вариантов вариантов архитектуры \textit{ассоциативных семантических компьютеров}. В основу предлагаемого подхода к разработке \textit{ассоциативного семантического компьютера} положены идеи, предложенные в работах В.В. Голенкова и получившие развитие в работе В.М. Кузьмицкого.}
	\begin{scnindent}
		\begin{scnrelfromlist}{источник}
			\scnitem{\scncite{Golenkov1996}}
			\scnitem{\scncite{Kuzmickij2000}}
		\end{scnrelfromlist}
	\end{scnindent}
	\scntext{предпосылка создания}{При формализации предметных областей, имеющих достаточно сложную семантическую организацию, перерабатываемые данные естественным образом группируются в некоторые сложные структуры. Эффективность решения задач, связанных с переработкой сложноструктурированных данных, на многопроцессорных вычислительных системах значительно возрастает в том случае, когда структура связей между процессорными элементами вычислительной системы, решающей эту задачу, совпадает со структурой данных, перерабатываемых в ходе ее решения (или, в более общем случае --- отображается в структуру перерабатываемых данных простым и естественным образом). При переходе к переработке данных все более сложной структурной и семантической организации (а затем и к переработке знаний) сохранение высокой эффективности вычислительной системы обеспечивается главным образом путем увеличения числа одновременно работающих процессорных элементов и усложнения структуры связей между ними.}
	\begin{scnindent}
		\scnrelfrom{источник}{\scncite{Kuzmickij2000}}
		\scnnote{Такую тенденцию развития технических средств ЭВМ мы и рассмотрим в качестве основной линии эволюции, создающей предпосылки для появления \textit{ассоциативных семантических компьютеров}. К ней относятся параллельные регулярные спецпроцессоры (векторные, матричные), спецвычислители для решения задач на графах и средства аппаратной поддержки семантических и нейронных сетей. К этой линии примыкают также и ассоциативные процессоры (в которых в роли процессорных элементов выступают ячейки ассоциативной памяти), процессоры баз данных и вычислительные системы, эффективно решающие те или иные классы задач за счет совпадения структуры связей между процессорными элементами со структурой информационного графа алгоритма (систолические вычислители, машины потоков данных).}
		\begin{scnindent}
			\scnrelfrom{источник}{\scncite{Kuzmickij2000}}
		\end{scnindent}
	\end{scnindent}
	
	\scnheader{архитектура вычислительной системы}
	\scnnote{Закономерным результатом развития вычислительных систем является переход к системам, изменяющим структуру связей между процессорными элементами в процессе функционирования. Такие системы настраивают свою внутреннюю структуру на структуру перерабатываемых данных и информационные графы алгоритмов решаемых задач и могут решать разные классы задач, сохраняя при этом высокую эффективность.}
	
	\scnheader{ЭВМ, ориентированная на переработку знаний}
	\scnnote{ЭВМ, ориентированная на переработку знаний, должна представлять собой в общем случае коллектив спецпроцессоров, ориентированных на максимально эффективное решение тех или иных классов задач.}
	\begin{scnrelfromset}{свойства}
		\scnfileitem{Спецпроцессоры представляют собой многопроцессорную вычислительную систему.}
		\begin{scnindent}
			\scnnote{В качестве семантического спецпроцессора можно предложить нелинейную (графовую) структурно перестраиваемую (динамическую) процессоро-память, аппаратно реализующую некоторый язык переработки семантических сетей, а саму ЭВМ такого рода можно, таким образом, назвать графодинамическим параллельным ассоциативным компьютером или \textit{ассоциативным семантическим компьютером}.}
		\end{scnindent}
		\scnfileitem{Структура связей между процессорными элементами спецпроцессоров совпадает со структурой данных или (реже) со структурой информационного графа алгоритма решения задачи.}
		\scnfileitem{Связи между процессорными элементами спецпроцессоров имеют перестраиваемую структуру.}
		\scnfileitem{Набор и функции спецпроцессоров определяются для каждой машины переработки знаний конкретно в зависимости от набора предметных областей, на которые эта машина ориентирована, и специфики задач, решаемых в этих областях.}
		\scnfileitem{Выявленный для некоторого семантического процессора набор механизмов переработки знаний должен быть \scnqq{погружен} в язык представления и переработки знаний. При этом наиболее удобными для этой цели представляются языки семантических сетей.}
		\scnfileitem{Процессорные элементы соответствуют вершинам или фрагментам семантической сети.}
		\scnfileitem{Переработка информации сводится к изменению структуры связей между процессорными элементами, соответствующему изменению конфигурации семантической сети.}
	\end{scnrelfromset}
	
	\scnheader{ассоциативный семантический компьютер}
	\begin{scnrelfromset}{принципы, лежащие в основе}
		\scnfileitem{Нелинейная память --- каждый элементарный фрагмент хранимого в памяти текста может быть логически инцидентен неограниченному числу других элементарных фрагментов этого текста. Таким образом, ячейки памяти, в отличие от обычной памяти, связываются не фиксированными условными связями, задающими фиксированную последовательность (порядок) ячеек в памяти, a логически или даже физически (с использованием технических средств коммутации) проводимыми связями произвольной конфигурации. Эти связи соответствуют дугам, ребрам, гиперребрам записанного в памяти графа (sc-текста).}
		\scnfileitem{Структурно-перестраиваемая (реконфигурируемая) память --- процесс отработки хранимой в памяти информации сводится не только к изменению состояния элементов, но и к реконфигурации связей между ними. То есть, в ходе переработки информации в структурно-перестраиваемой памяти меняются на только и даже не столько состояния ячеек памяти, как это имеет место в обычной памяти, сколько конфигурация связей между этими ячейками. Т.е. в структурно-перестраиваемой памяти в ходе переработки информации не только перераспределяются метки на вершинах записанного в памяти графа, но и меняется структура самого этого графа.}
		\scnfileitem{В качестве внутреннего способа кодирования знаний, хранимых в памяти \textit{ассоциативного семантического компьютера}, используется универсальный (!) способ нелинейного (графоподобного) смыслового представления знаний --- SC-код.}
		\scnfileitem{Обработка информации осуществляется коллективом агентов, работающих над общей памятью. Каждый из них реагирует на соответствующую ему ситуацию или событие в памяти (компьютер, управляемый хранимыми знаниями).}
		\scnfileitem{Есть программно реализуемые агенты, поведение которых описывается хранимыми в памяти агентно-ориентированными программами, которые интерпретируются соответствующими коллективами агентов.}
		\scnfileitem{Есть базовые агенты, которые не могут быть реализованы программно (в частности, это агенты интерпретации агентных программ, базовые рецепторные агенты-датчики, базовые эффекторные агенты).}
		\scnfileitem{Все агенты работают над общей памятью одновременно. Более того, если для какого-либо агента в некоторый момент времени в различных частях памяти возникает сразу несколько условий его применения, разные информационные процессы, соответствующие указанному агенту в разных частях памяти могут выполняться одновременно.}
		\scnfileitem{Для того, чтобы информационные процессы агентов, параллельно выполняемые в общей памяти не \scnqq{мешали} друг другу, для каждого информационного процесса в памяти фиксируется и постоянно актуализируется его текущее состояние. То есть каждый информационный процесс сообщает всем остальным о своих намерениях и пожеланиях, которым остальные информационные процессы не должны препятствовать. Реализация такого подхода может выполняться, например, на основе механизма блокировок элементов семантической памяти.}
		\begin{scnindent}
			\scnrelfrom{смотрите}{Предметная область и онтология решателей задач ostis-систем}
		\end{scnindent}
		\scnfileitem{Процессор и память \textit{ассоциативного семантического компьютера} глубоко интегрированы и составляют единую процессоро-память. Процессор \textit{ассоциативного семантического компьютера} равномерно "распределен"{} по его памяти так, что процессорные элементы одновременно являются и элементами памяти компьютера. То есть каждая ячейка дополняется функциональным (процессорным) элементом, a перестраиваемые связи между ячейками становятся коммутируемыми каналами связи между процессорными элементами. Каждый процессорный элемент при этом имеет свою специальную внутреннюю регистровую память, отражающую важные для данного процессорного элемента аспекты текущего состояния процесса выполнения элементарных операций языка микропрограмм, обеспечивающих интерпретацию языка более высокого уровня (Языка SCP).}
		\scnfileitem{Обработка информации в \textit{ассоциативном семантическом компьютере} сводится к реконфигурации каналов связи между процессорными элементами, следовательно память такого компьютера есть не что иное, как \uline{коммутатор} (!) указанных каналов связи. Таким образом, текущее состояние конфигурации этих каналов связи и есть текущее состояние обрабатываемой информации. Этот принцип обеспечивает значительное ускорение переработки информации благодаря исключению этапов передачи информации из памяти в процессор и обратно, но оплачивается ценой большой избыточности процессорных (функциональных) средств, равномерно распределяемых по памяти.}
	\end{scnrelfromset}
	\bigskip
\end{scnsubstruct}
\scnsourcecomment{Завершили \scnqqi{Сегмент. Общие принципы, лежащие в основе ассоциативных семантических компьютеров для ostis-систем}}


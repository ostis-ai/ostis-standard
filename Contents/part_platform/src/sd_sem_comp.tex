\begin{SCn}
    \scnsectionheader{Предметная область и онтология ассоциативных семантических компьютеров для ostis-систем}
    \begin{scnsubstruct}
    	\scntext{аннотация}{В предметной области и онтологии рассмотрены принципы реализации аппаратной платформы для реализации систем, построенных на основе Технологии OSTIS, --- ассоциативного семантического компьютера.}
    	\begin{scnrelfromlist}{ключевое понятие}
    		\scnitem{машина фон-Неймана}	
    		\scnitem{архитектура вычислительной системы}
    		\scnitem{ассоциативный семантический компьютер}
    		\scnitem{scp-компьютер}
    		\scnitem{процессорный модуль}
    		\scnitem{накопительный модуль}
    		\scnitem{терминальный модуль}
    		\scnitem{процессорный элемент}
    		\scnitem{физический канал связи}
    		\scnitem{логический канал связи}
    		\scnitem{волновая микропрограмма}
    		\scnitem{волновой язык программирования}
    	\end{scnrelfromlist}
    	\begin{scnrelfromlist}{библиографическая ссылка}
    		\scnitem{\scncite{Neumann1993}}
    		\scnitem{\scncite{NeumanMachine}}
    		\scnitem{\scncite{Glushkov1974}}
    		\scnitem{\scncite{Ajlif1973}}
    		\scnitem{\scncite{Moldovan1992}}
    		\scnitem{\scncite{Chu1976}}
    		\scnitem{\scncite{Kalynychenko1990}}
    		\scnitem{\scncite{Martin1980}}
    		\scnitem{\scncite{Ozkarahan1989}}
    		\scnitem{\scncite{Kohonen1980}}
    		\scnitem{\scncite{Ignatushhenko1981}}
    		\scnitem{\scncite{Berkovich1975}}
    		\scnitem{\scncite{Ajzerman1977}}
    		\scnitem{\scncite{Marchuk1978}}
    		\scnitem{\scncite{Prangishvili1981}}
    		\scnitem{\scncite{Zatuliver1981}}
    		\scnitem{\scncite{Ackerman1979}}
    		\scnitem{\scncite{Majers1985}}
    		\scnitem{\scncite{Glushkov1980}}
    		\scnitem{\scncite{Glushkov1978}}
    		\scnitem{\scncite{Rabinovich1995}}
    		\scnitem{\scncite{Zadyhajlo1979}}
    		\scnitem{\scncite{Schuster1979}}
    		\scnitem{\scncite{Suvorov1985}}
    		\scnitem{\scncite{Brukle1978}}
    		\scnitem{\scncite{Chu1977}}
    		\scnitem{\scncite{Kohonen1982}}
    		\scnitem{\scncite{Foster1981}}
    		\scnitem{\scncite{Ershov1982}}
    		\scnitem{\scncite{Bershtejn1975}}
    		\scnitem{\scncite{Vasilev1987}}
    		\scnitem{\scncite{Sapatyj1984}}
    		\scnitem{\scncite{Popov2019}}
    		\scnitem{\scncite{Popov2020}}
    		\scnitem{\scncite{Zhang2017}}
    		\scnitem{\scncite{Hu2021}}
    		\scnitem{\scncite{Song2016}}
    		\scnitem{\scncite{Afanasyev2021}}
    		\scnitem{\scncite{Vajncvajg1980}}
    		\scnitem{\scncite{Vajncvajg1987}}
    		\scnitem{\scncite{Somsubhra2006}}
    		\scnitem{\scncite{Rabinovich1979a}}
    		\scnitem{\scncite{Rabinovich1979b}}
    		\scnitem{\scncite{Gladun1977}}
    		\scnitem{\scncite{Gladun1987}}
    		\scnitem{\scncite{Amosov1973}}
    		\scnitem{\scncite{Zolotov1982}}
    		\scnitem{\scncite{Galushkin1997}}
    		\scnitem{\scncite{Heht-Nilsen1998}}
    		\scnitem{\scncite{Komarcova2004}}
    		\scnitem{\scncite{USB_Accelerator}}
    		\scnitem{\scncite{Moussa2013}}
    		\scnitem{\scncite{Altay}}
    		\scnitem{\scncite{Allen1989}}
    		\scnitem{\scncite{CUDA}}
    		\scnitem{\scncite{OpenCL}}
    		\scnitem{\scncite{Tran2018}}
    		\scnitem{\scncite{Shi2018}}
    		\scnitem{\scncite{Lu2021}}
    		\scnitem{\scncite{Golenkov1984}}
    		\scnitem{\scncite{Golenkov1994f}}
    		\scnitem{\scncite{Golenkov1994g}}
    		\scnitem{\scncite{Golenkov1996}}
    		\scnitem{\scncite{Gaponov2000}}
    		\scnitem{\scncite{Kuzmickij2000}}
    		\scnitem{\scncite{Serdiukov2004}}
    		\scnitem{\scncite{Ivashenko2021OSTIS}}
    		\scnitem{\scncite{Ivashenko2016Tatur}}
    		\scnitem{\scncite{Ivashenko2015Tatur}}
    		\scnitem{\scncite{Rasheed2019}}
    		\scnitem{\scncite{Dubrovin2020}}
    		\scnitem{\scncite{Wolfram2002}}
    		\scnitem{\scncite{VonNeuman1971}}
    		\scnitem{\scncite{Moon1987}}
    		\scnitem{\scncite{Smith1984}}
    		\scnitem{\scncite{Steele2011}}
    		\scnitem{\scncite{McJones2015}}
    		\scnitem{\scncite{VanderLeun2017}}
    		\scnitem{\scncite{Ivashenko2020String}}
    		\scnitem{\scncite{Hewitt2009}}
    		\scnitem{\scncite{Ivashenko2020}}
    		\scnitem{\scncite{Ivashenko2016BSUIR}}
    		\scnitem{\scncite{Ivashenko2019InfiniteMemory}}
    		\scnitem{\scncite{Ivashenko2022}}
    		\scnitem{\scncite{Ivashenko2020ReductionScheme}}
    		\scnitem{\scncite{Ivashenko2021PRIP}}
    		\scnitem{\scncite{LegUp}}
    		\scnitem{\scncite{VHDPlus}}
    		\scnitem{\scncite{SystemC}}
    		\scnitem{\scncite{MyHDL}}
    		\scnitem{\scncite{Sapatyj1986}}
    		\scnitem{\scncite{Moldovan1985}}
    		\scnitem{\scncite{Letichevskij2003}}
    		\scnitem{\scncite{Letichevskij2012}}
    		\scnitem{\scncite{Backus1978}}
    		\scnitem{\scncite{Kotov1966}}
    	\end{scnrelfromlist}
    	\scntext{введение}{Применение для разработки \textit{ostis-систем} современных программно-аппаратных платформ, ориентированных на адресный доступ к хранящимся в памяти данным, не всегда оказывается эффективным, поскольку при разработке интеллектуальных систем фактически приходится моделировать нелинейную память на базе линейной. Повышение эффективности решения задач интеллектуальными системами требует разработки специализированных платформ, в том числе аппаратных, ориентированных на унифицированные семантические модели представления и обработки информации. Таким образом, основной целью создания \textit{ассоциативных семантических компьютеров} является повышение производительности ostis-систем.}
    	\begin{scnrelfromset}{заключение}
    		\scnfileitem{В предметной области и онтологии рассмотрены недостатки доминирующей в настоящее время фон-Неймановской архитектуры компьютерных систем в качестве основы для построения интеллектуальных компьютерных систем нового поколения, проведен анализ современных подходов к разработке аппаратных архитектур, устраняющих некоторые из указанных недостатков, обоснована необходимость разработки принципиально новых аппаратных архитектур, представляющих собой аппаратный вариант реализации ostis-платформ --- \textit{ассоциативных семантических компьютеров}.}
    		\scnfileitem{Предложены общие принципы, лежащие в основе \textit{ассоциативных семантических компьютеров}, рассмотрены три возможных варианта архитектуры таких компьютеров, представлены их достоинства и недостатки.}
    		\scnfileitem{Дальнейшее развитие предложенных в главе подходов требует решения ряда задач, как технических, так и организационных:
    			\begin{scnitemize}
    				\item Разработка волнового языка для записи микропрограмм, которыми обмениваются процессорные элементы между собой, и которые исполняются этими процессорными элементами;
    				\item Разработка языка для записи программ управления процессами обмена микропрограммами и управления очередью микропрограмм;
    				\item Организация активного участия специалистов в области микроэлектроники в уточнении принципов реализации процессорных элементов и процессоро-памяти в целом, уточнение элементной базы и более низкоуровневых архитектурных особенностей \textit{ассоциативных семантических компьютеров};
    				\item Разработка алгоритмов оптимизации способов записи sc-конструкций в процессоро-память и переразмещения уже записанной sc-конструкции с целью обеспечения последующей эффективности передачи сообщений между процессорными элементами;
    				\item Уточнение типологии информационных процессов в процессоро-памяти, их свойств и соответствующей типологии меток;
    				\item Уточнение принципов реализации многоагентной обработки знаний в рамках процессоро-памяти, в частности, разработка принципов реализации событийной обработки информации в такой памяти.
    		\end{scnitemize}}
   		\end{scnrelfromset}
		\scniselement{раздел базы знаний}
		\scnhaselementrole{ключевой sc-элемент}{Предметная область семантических ассоциативных компьютеров для ostis-систем}
	
        \scnheader{Предметная область семантических ассоциативных компьютеров для ostis-систем}
        \scniselement{предметная область}
        \begin{scnhaselementrolelist}{класс объектов исследования}
            \scnitem{семантический ассоциативный компьютер}
        \end{scnhaselementrolelist}
        \begin{scnhaselementrolelist}{класс объектов исследования}
            \scnitem{информационно-логическая задача}
            \scnitem{машина, ориентированная на решение информационно-логических задачи}
        \end{scnhaselementrolelist}
        
       	\begin{scnreltovector}{конкатенация сегментов}
        	\scnitem{Сегмент. Современное состояние работ в области разработки компьютеров для интеллектуальных систем}
        	\scnitem{Сегмент. Анализ существующих архитектур вычислительных систем}
        	\scnitem{Сегмент. Общие принципы, лежащие в основе ассоциативных семантических компьютеров для ostis-систем}
        	\scnitem{Сегмент. Архитектура ассоциативных семантических компьютеров для ostis-систем}
        \end{scnreltovector}
        
        \input{Contents/part_platform/src/sd_sem_comp/segment_sem_comp_analysis}
        \input{Contents/part_platform/src/sd_sem_comp/segment_comp_arch_analysis}
        \input{Contents/part_platform/src/sd_sem_comp/segment_sem_comp_principles}
        \input{Contents/part_platform/src/sd_sem_comp/segment_sem_comp_arch}
        
        \bigskip
    \end{scnsubstruct}
    \scnendcurrentsectioncomment
\end{SCn}

 
\scchapter{Вводный раздел Теории и Технологии проектирования интеллектуальных компьютерных систем}

\label{chap_intro} 

\begin{SCn}

\scnsectionheader{\currentname}
\scnstartsubstruct

\scnreltovector{конкатенация подразделов}{\nameref{foreword};\nameref{intro_ostis};\nameref{intro_lang};\nameref{intro_rules}}
\scnrelfrom{следующий раздел}{\nameref{chap_justification}}

\scntext{аннотация}{\textit{Вводная часть Документации Технологии OSTIS} представляет собой предварительный ответ на следующие вопросы:
\begin{scnitemize}
\item Что такое \textit{Технология OSTIS} (Для чего она предназначена, каковы ее конкурентные преимущества)\char59
\item Что такое \textit{Документация Технологии OSTIS} и как она устроена\char59
\item Что необходимо знать для ознакомления с \textit{Документацией Технологии OSTIS} о стиле и формальных языках, используемых для ее представления.
\end{scnitemize}
}

\end{SCn}

%
\scsection{Предисловие к Документации Технологии OSTIS}
\label{foreword}

\begin{SCn}

\scnsectionheader{\currentname}

\scnstartsubstruct

\bigskip
\scnfilelong{Данная публикация посвящена описанию предлагаемой нами Открытой Семантической Технологии Проектирования Интеллектуальных Компьютерных Систем (Технологии OSTIS -- Open Semantic Technology for Intelligent Systems). Интеллектуальные компьютерные системы, разработанные на этой технологии, мы назвали ostis-системами. Особенностью публикации является то, что она оформлена в виде \uline{исходного текста} основной части базы знаний специальной ostis-системы, которая предназначена для комплексной поддержки проектирования семантически совместимых ostis-систем. Эту систему мы назвали \textbf{\textit{Метасистемой IMS.ostis}} (Intelligent MetaSystem for ostis-systems). Последовательность изложения материала в исходном тексте базы знаний не является единственно возможным маршрутом прочтения (просмотра) базы знаний. Каждый читатель, войдя в \textbf{\textit{Метасистему IMS.ostis}}, может выбрать любой другой маршрут навигации по этой базе знаний, задавая указанной метасистеме те вопросы, которые в текущий момент его интересуют. Таким образом, читая данный исходный текст и одновременно работая с \textbf{\textit{Метасистемой IMS.ostis}}, можно значительно быстрее усвоить детали \textbf{\textit{Технологии OSTIS}} и значительно быстрее приступить к непосредственному использованию указанной технологии. Этому также способствует большое количество примеров семантических моделей различных фрагментов интеллектуальных компьютерных систем. 

Основной вид разделов базы знаний ostis-системы -- это формальное представление различных \textbf{\textit{предметных областей}} вместе с соответствующими им \textbf{\textit{онтологиями}}. При этом явно указываются связи между этими \textbf{\textit{предметными областями}} и \textbf{\textit{онтологиями}}. Таким образом, база знаний \textbf{\textit{Метасистемы IMS.ostis}}, как и любых других интеллектуальных компьютерных систем, построенных по \textbf{\textit{Технологии OSTIS}}, представляет собой иерархическую систему связанных между собой формальных моделей предметных областей и соответствующих им онтологий.

В основе Технологии OSTIS лежит предлагаемая нами унификация интеллектуальных компьютерных систем, основанная, в свою очередь, на смысловом представлении знаний в памяти интеллектуальных компьютерных систем. Таким образом, данную публикацию можно рассматривать как проект стандарта семантических моделей интеллектуальных компьютерных систем. Последующие публикации, посвящённые детальному описанию различных компонентов Технологии OSTIS, будут также оформляться как исходные тексты соответствующих разделов базы знаний Метасистемы IMS.ostis и будут отражать следующие этапы развития технологии OSTIS, следующие версии этой технологии, и, соответственно, следующие версии Метасистемы IMS.ostis. 

Надеемся на то, что состав авторов таких публикаций будет расширяться при четкой спецификации вклада каждого из них. 

Все основные положения Технологии OSTIS рассматривались и обсуждались на конференциях OSTIS, которые стали важным стимулирующим фактором становления и развития Технологии OSTIS. Мы благодарим всех активных участников ежегодных конференций OSTIS.
}

\scnendstruct

\end{SCn}

\scsection{Методологические проблемы современного состояния работ в области Искусственного интеллекта}

\label{intro_ostis}

\begin{SCn}

\scnsectionheader{\currentname}

\scnstartsubstruct

\scnsegmentheader{Начало раздела "\currentname"}

\scnstartsubstruct

\scnheader{Методологические проблемы современного состояния работ в области Искусственного интеллекта}

\scnrelfromvector{конкатенация сегментов}
{Структура деятельности в области Искусственного интеллекта;}
\scnauthorcomment{дополнить список}

\scnrelfromset{рассматриваемые вопросы}{
\scnfileitem{Каковы основные стратегические цели (сверхзадачи) научно-технической деятельности в области \textit{Искусственного интеллекта}?};
\scnfileitem{Какие проблемы являются на сегодняшний день актуальными для дальнейшего развития различных направлений \textit{Искусственного интеллекта} и для развития \textit{Искусственного интеллекта} в целом как общей (объединённой) \textit{научно-технической дисциплины}, а также для развития различных форм деятельности в этой области (научно-исследовательской деятельности создания технологий разработки интеллектуальных компьютерных систем, образовательной деятельности, бизнеса)?};
\scnfileitem{Какие проблемы являются на сегодняшний день актуальными для развития других \textit{научно-технических дисциплин} и являются ли эти проблемы аналогичными тем, которые актуальны для развития \textit{Искусственного интеллекта}?};
\scnfileitem{Какие можно предложить подходы к решению указанных выше проблем и как для этого можно использовать создаваемый сейчас новый технологический уклад в области \textit{Искусственного интеллекта} (следующий уровень технологий искусственного интеллекта)?};
\scnfileitem{Как будет выглядеть на основе следующего уровня \textit{технологий Искусственного интеллекта} комплексная автоматизация вех \textit{видов человеческой деятельности}, а также взаимодействие различных \textit{видов человеческой деятельности}, т.е. как будет выглядеть архитектура \textit{smart-общества}?};
\scnfileitem{Устраивает ли нас уровень семантической совместимости взаимопонимания между современными виртуальными компьютерными системами и что необходимо сделать для повышения этого уровня?};
\scnfileitem{Устраивает ли нас уровень семантической совместимости взаимопонимания между современными интеллектуальными компьютерными системами их пользователями и что необходимо сделать для повышения этого уровня?}}
\scntext{аннотация}{Предлагаемое вашему вниманию рассмотрение методологических проблем современного состояния работ в области \textit{Искусственного интеллекта} состоит из следующих частей:
\begin{scnitemize}
\item Анализ актуальных проблем, препятствующих дальнейшему развитию  \textit{Искусственного интеллекта} как \textit{научно-технической дисциплины}:
\begin{scnitemizeii}
\item Проблемы развития научных исследований в области \textit{Искусственного интеллекта} 
\item Проблемы разработки технологий проектирования и реализации \textit{интеллектуальных компьютерных систем};
\item Проблемы формирования рынка \textit{интеллектуальных компьютерных систем}; 
\item Образовательные проблемы в области \textit{Искусственного интеллекта};
\item Проблемы развития бизнеса в области \textit{Искусственного интеллекта}.
\end{scnitemizeii}
\item Анализ проблем автоматизации сложных видов деятельности:
\begin{scnitemizeii}
\item научно-исследовательской деятельности в рамках различных научных дисциплин;
\item создание \textit{технологий проектирования} и производства (реализации) сложных технических систем;
\item \textit{инженерной деятельности} по разработке сложных технических систем;
\item \textit{образовательной деятельности} по наукоёмким техническим специальностям
\end{scnitemizeii}
\item Формулировка принципов, лежащих в основе \textit{Технологии OSTIS}, предназначенной для решения указанных выше проблем;
\item Рассмотрение структуры \textit{Экосистемы OSTIS}, построенной по \textit{Технологии OSTIS} и обеспечивающей комплексную автоматизацию всех видов человеческой деятельности
\end{scnitemize}}

\scnrelfromset{используемые знаки общих понятий и иных сущностей}{деятельность\\
\scnaddlevel{1}
\scnidtf{область деятельности}
\scnsuperset{человеческая деятельность}
\scnaddlevel{-1}
;вид деятельности\\
\scnaddlevel{1}
\scnhaselement{проектирование}
\scnaddlevel{1}
\scnidtf{проектная деятельность}
\scnaddlevel{-1}
\scnhaselement{производство}
\scnaddlevel{1}
\scnidtf{производственная деятельность}
\scnaddlevel{-1}
\scnhaselement{наука}
\scnaddlevel{1}
\scnidtf{научная деятельность}
\scnaddlevel{-2}
;проект\\
\scnaddlevel{1}
\scnsuperset{открытый проект}
\scnaddlevel{-1}
;консорциум
;технология\\
\scnaddlevel{1}
\scnsuperset{информационная технология}
\scnaddlevel{1}
\scnsuperset{технология искусственного интеллекта}
\scnaddlevel{-2}
;кибернетическая система\\
\scnaddlevel{1}
\scnsuperset{интеллектуальная система}
\scnaddlevel{1}
\scnsuperset{интеллектуальная компьютерная система}
\scnaddlevel{1}
\scnidtf{искусственная интеллектуальная система}
\scnaddlevel{-3}
;конвергенция\scnsupergroupsign
\scnaddlevel{1}
\scnidtf{уровень конвергенции (близости)}
\scnsuperset{конвергенция кибернетических систем\scnsupergroupsign}
\scnaddlevel{-1}
;интеграция*\\
\scnaddlevel{1}
\scnsuperset{интеграция кибернетических систем*}
\scnsuperset{эклектичная интеграция*}
\scnsuperset{глубокая интеграция*}
\scnaddlevel{-1}
;интегрированная система\\
\scnaddlevel{1}
\scnsuperset{эклектичная система}
\scnsuperset{гибридная система}
\scnaddlevel{-1}
;экосистема интеллектуальных компьютерных систем
;рынок знаний\\
\scnaddlevel{1}
\scnidtf{рыночная организация порождения эволюции и применения знаний}
\scnaddlevel{-1}
;smart-общество\\
\scnaddlevel{1}
\scnidtf{общество,в основе которого лежит экосистема интеллектуальных компьютерных систем и рынок знаний}
\scnaddlevel{-1}
}
 
\scnrelfromset{ключевые знаки}
{Искусственный интеллект\\
\scnaddlevel{1}
\scniselement{научно-техническая дисциплина}
\scnaddlevel{1}
\scnsubset{научно-техническая деятельность} 
\scnaddlevel{-2};
интеллектуальная система\\
\scnaddlevel{1}
\scnsuperset{интеллектуальная компьютерная система}
\scnaddlevel{-1};
Общая теория интеллектуальных систем;
Базовая комплексная технология проектирования интеллектуальных компьютерных систем;
Технология производства спроектированных интеллектуальных компьютерных систем;
Специализированная инженерия в области Искусственного интеллекта;
Образовательная деятельность в области Искусственного интеллекта;
Бизнес-деятельность в области Искусственного интеллекта\bigskip;
\scnkeyword{Технология OSTIS};
\scnkeyword{ostis-система};
смысловое преставление информации;
агентно-ориентированная модель обработки информации в памяти; стандартизация ostis-систем;
\scnkeyword{SC-код};
абстрактная sc-машина;
конвергенция знаний в памяти;
ostis-систем;
конвергенция моделей решения задач в  ostis-системе;
интеграция знаний в памяти  ostis-системы;
интеграция моделей решения задач в  ostis-системе;
ostis-сообщество;
ostis-технология\\
\scnaddlevel{1}
\scnsuperset{ostis-технология проектирования}
\scnsuperset{ostis-технология производства}
\scnsuperset{технология эксплуатации ostis-систем}
\scnsuperset{технология реинжиниринга ostis-систем}
\scnaddlevel{-1};
\scnkeyword{Ядро Технологии OSTIS}\bigskip;
OSTIS-портал научных знаний в области Искусственного интеллекта;
Проект IMS.ostis;
\scnkeyword{Метасистема IMS.ostis};
Проект Программной реализации универсальной абстрактной sc-машины;
Проект разработки Универсального sc-компьютера;
Специализированная инженерия, осуществляемая на основе Технологии OSTIS;
Образовательная деятельность в области Искусственного интеллекта, осуществляемая на основе технологии OSTIS;
\scnkeyword{Консорциум OSTIS}\bigskip;
\scnkeyword{Экосистема OSTIS};
человеческая деятельность;
вид человеческой деятельности;
автоматизация человеческой деятельности;
качество человеческой деятельности;
субъект Экосистемы OSTIS;
Рынок знаний, реализованный в рамках Экосистемы OSTIS;
smart-общество}

\scnheader{конвергенция в области Искусственного интеллекта}

\scnrelfrom{разбиение}{Направления конвергенции в области Искусственного интеллекта}
     \scnaddlevel{1}
\scnhaselement{конвергенция Искусственного интеллекта со смежными научными дисциплинами}
    \scnaddlevel{1}
\scntext{примечание}{Искусственный интеллект

    \scnaddlevel{-3}
\scnrelboth{смежная дисциплина}{
    $\bullet$ Логика\\
    $\bullet$ Психология человека\\
    $\bullet$ Зоопсихология\\
    $\bullet$ Нейропсихология\\
    $\bullet$ Этология\\
    $\bullet$ Кибернетика\\
    $\bullet$ Общая теория систем\\
    $\bullet$ Семиотика\\
    $\bullet$ Лингвистика\\
    $\bullet$ и др.}
}
\scnaddlevel{2}
\scnhaselement{конвергенция различных направлений Искусственного интеллекта}
\scnaddlevel{1}
\scnidtf{Конвергенция различных направлений исследований в области Искусственного интеллекта, результатом которой должна быть формализованная практически ориентированная общая теория интеллектуальных систем и, в частности, интеллектуальных компьютерных систем. Разобщенность различных направлений исследований в области искусственного интеллекта является главным препятствием создания общей комплексной технологии проектирований интеллектуальных компьютерных систем}
\scnidtf{Конвергенция между различными направлениями и продуктами научных исследований в области искусственного интеллекта. Результатом (целевым продуктом) такой конвергенции должна стать общая формальная теория интеллектуальных компьютерных систем}
\scnaddlevel{-1}
\scnhaselement{конвергенция различного вида знаний в памяти интеллектуальной компьютерной системы}
\scnaddlevel{1}
\scnidtf{Конвергенция и интеграция внутреннего представления в памяти интеллектуальной компьютерной системы различного вида знаний}
\scnaddlevel{-1}
\scnhaselement{конвергенция различных моделей решения задач в памяти интеллектуальной компьютерной системы}
\scnaddlevel{1}
\scnidtf{Конвергенция и интеграция различных моделей решения задач
\begin{itemize}
    \item логико-семантическая типология задач
    \item типология моделей решения задач (задача, класс задач,метод, класс методов, модель решения задач=иерархический метод интерпретации класса методов)
\end{itemize}}
\scnaddlevel{-1}
\scnhaselement{конвергенция интеллектуальных компьютерных систем}
\scnaddlevel{1}
\scnidtf{Обеспечение семантической совместимости (взаимпопонимания) интеллектуальных систем, согласование используемых онтологий}
\scnidtf{Конвергенция между различными прикладными компьютерными системами. Результатом (целевым продуктом) такой конвергенции должна стать экосистема, состоящая из перманентно эволюционируемых, семантически совместимых и взаимодействующих интеллектуальных компьютерных систем, а также их пользователей}
\scnexplanation{Конвергенция (семантическая совместимость) всех разрабатываемых интеллектуальных компьютерных систем(в том числе прикладных), преобразующая набор индивидуальных (самостоятельных) интеллектуальных компьютерных систем различного назначения в коллектив активно взаимодействущих интеллектуальных компьютерных систем для совместного (коллективного) решения сложных (комплексных) задач и для перманентной поддержки семантической совместимости в ходе индивидуальной эволюции каждой интеллектуальной компьютерной системы}
\scnaddlevel{-1}
\scnhaselement{конвергенция средств автоматизации проектирования различного вида компонентов интеллектуальных компьютерных систем}
\scnaddlevel{1}
\scnidtf{Конвергенция (семантическая совместимость) средств автоматизации проектирования различного вида компонентов интеллектуальных компьютерных систем, результатом которой должен быть общий комплекс средств автоматизации проектирования всех компонентов интеллектуальных компьютерных систем}
\scnidtf{Конвергенция между инструментальными средствами, обеспечивающими автоматизацию проектирования различных компонентов или различных классов интеллектуальных компьютерных систем. Результатом (целевым продуктом) такой конвергенции должен стать единый комплекс методологических и инструментальных средств, ориентированный на поддержку комплексного проектирования любых интеллектуальных компьютерных систем}
\scnaddlevel{-1}
\scnhaselement{конвергенция логико-семантических моделей интеллектуальных компьютерных систем}
\scnaddlevel{1}
\scnnote{\textit{логико-семантические модели интеллектуальных компьютерных систем} являются результатом ("сухим" остатком) \textit{проектирования} этих систем и представляют собой формальное представления исходного (начального) состояния \textit{баз знаний} разрабатываемых \textit{интеллектуальных компьютерных систем}}
\scnaddlevel{-1}
\scnhaselement{конвергенция средств интерпретации логико-семантических моделей разрабатываемых интеллектуальных компьютерных систем}
\scnaddlevel{1}
\scnhaselement{конвергенция средств интерпретации логико-семантических моделей разрабатываемых интеллектуальных компьютерных систем}
\scnaddlevel{1}
\scnidtf{Конвергенция (совместимость) средств реализации (производства) интеллектуальных компьютерных систем на основе спроектированных формальных моделей создаваемых интеллектуальной компьютерной системой (средств интерпретации спроектированных моделей интеллектуальных компьютерных систем). Такая интерпретация может осуществляться либо программным путем на современных компьютерах, либо путем создания принципиально новых компьютеров, специально ориентированных на интерпретацию формальных моделей интеллектуальных компьютерных систем, помещаемых в память указанных компьютеров}
\scnaddlevel{-1}
\scnhaselement{конвергенция между информационно-программных и аппаратным обеспечением интеллектуальных компьютерных систем}
\scnaddlevel{1}
\scnidtf{Конвергенция между Software и Hardware интеллектуальных компьютерных систем}
\scnaddlevel{-1}
\scnhaselement{Конвергенция различных форм деятельности в области Искусственного интеллекта}
\scnaddlevel{1}
\scnidtf{Конвергенция между
\begin{scnitemize}
    \item научными исследованиями по созданию общей теории интеллектуальных компьютерных систем;
    \item разработкой средств автоматизации проектирования интеллектуальных компьютерных систем;
    \item разработкой средств интерпретации спроектированныз формальных моделей интеллектуальных компьютерных систем;
    \item разработкой прикладных интеллектуальных компьютерных систем различного назначения;
    \item подготовкой и перманентным повышением квалификации кадров, способных эффективно участвовать во всех перечисленных направлениях деятельности.
    \end{scnitemize}

Глубокая конвергенция между всеми этими формами деятельности возможна только тогда, когда \uline{каждый} участник создания комплексной технологии искусственного интеллекта является участником \uline{каждой} из перечисленныз форм деятельности.    
}
\scnidtf{Конвергенция между (1) научно-исследовательской деятельностью в области искусственного интеллекта; (2) инженерно-технологической деятельностью, которая направлена на разработку комплексной технологии проектирования интеллектуальных компьютерных систем и которая имеет высокий уровень наукоемкости; (3) инженерно-прикладной деятельностью, которая направлена на разработку прикладных интеллектуальных систем и которая также имеет высокий уровень наукоемкости, обусловоенной необходимостью качественной формализации соответствующих предметных областей и, в частности, методов решения задач в этих областях; (4) образованием (образовательной деятельностью) в области искусственного интеллекта, повышение эффективности которого настоятельно требует раннего и поэтапного вовлечения студентов в реальные, а не учебные проекты - сначала в инженерно-прикладные, потом в инженерно- исследовательские проекты; (5) деятельностью, направленной на создание инфраструктуры, обеспечивающей поддержку открытого массового активного международного сотрудничества по консолидации усилий, направленных на решение современных проблем в области искусственного интеллекта; (6) бизнесом в области искусственного интеллекта, который не просто должен обеспечить финансовую поддержку перечисленных видов деятельности, но и обеспечить грамотный баланс между ними, грамотное сочетания тактических и стратегических целей}

\scnheader{Искусственный интеллект}
\scnrelfromset{методологические проблемы текущего состояния}{
\scnfileitem{Далеко не всеми учеными, работающими в области искусственного интеллекта принимается прагматичность практической направленности этой науки;}
;\scnfileitem{Не всеми принимается необходимость конвергенции различных направлений искуссвенного интеллекта и необходимость их интеграции в целях построения общей теории интеллектуальных систем;}
;\scnfileitem{Нет движения к построению общей компьютерной технологии интеллектуальных компьютерных систем;}
;\scnfileitem{Нет движения к построению экосистем интеллектуальных компьютерных систем;}
;\scnfileitem{Не всеми принимается необходимость конвергенции различных форм деятельности в области искуственного интеллекта}}
\scnnote{Современная трактовка целей и задач Искусственного интеллекта как научно-технической дисциплины требует переосмысления, так как, к сожалению, носит несогласованный, а часто и значительно более узкий характер, чем этого требует текущее положение}


\scnheader{Бизнес-деятельность в области Искусственного интеллекта}
\scntext{текущее состояние}{Острая потребность в существенном повышении уровня автоматизации в самых различных областях человеческой деятельности (в промышленности, медицине, транспорте, образовании, строительстве и во многих других), а также современные результаты в развитии \textit{технологий Искусственного интеллекта} привели к существенному расширению работ по созданию \textit{прикладных интеллектуальных компьютерных систем} и к появлению большого количества коммерческих организаций, ориентированных на разработку таких приложений.}
\scnrelfromset{проблемы текущего состояния}{
\scnfileitem{Не так просто обеспечить баланс тактических и стратегических направлений развития всех форм деятельности в области \textit{Искусственного интеллекта} (научно-исследовательской деятельности, разработки технологии проектирования и производства интеллектуальных компьютерных систем, разработки прикладных систем, образовательной деятельности), а также баланс между всеми перечисленными формами деятельности.}
;\scnfileitem{В настоящее время отсутствует глубокая конвергенция различных форм деятельности в области \textit{Искусственного интеллекта} (в первую очередь, конвергенция развития технологий \textit{Искусственного интеллекта} и разработки различных прикладных интеллектуальных компьютерных систем), что существенно затрудняет развитие каждой из этих форм.}
;\scnfileitem{Высокий уровень наукоемкости работ в области \textit{Искусственного интеллекта} предъявляет особые требования к квалификации сотрудников и к их способности работать в составе творческих коллективов.}
;\scnfileitem{Для повышения квалификации своих сотрудников и для обеспечения высокого уровня своих разработок необходимо активное сотрудничество с различными научными школами, с кафедрами, осуществляющими подготовку молодых специалистов в области \textbf{\textit{Искусственного интеллекта}}, активное участие в подготовке и проведении соответствующих конференций, семинаров, выставок.}}

\scnheader{Искусственный интеллект}
\scnrelfromset{сверхзадачи текущего состояния}{
\scnfileitem{Построение и перманентное развитие \textit{общей формальной теории интеллектуальных систем}}
\scnaddlevel{1}
\scnrelfromset{подзадачи}{
\scnfileitem{Уточнение требований, предъявляемых к интеллектуальным компьютерным системам – уточнение свойств интеллектуальных компьютерных систем, определяющих высокий уровень их интеллекта.}
;\scnfileitem{Конвергенция и интеграция всевозможных видов знаний и всевозможных моделей решения задач в рамках каждой интеллектуальной компьютерной системы.}
;\scnfileitem{Ориентация на последующую разработку унифицированных семантически совместимых формальных моделей интеллектуальных систем.}
;\scnfileitem{Ориентация на разработку различного вида универсальных интерпретаторов формальных моделей интеллектуальных систем (и в том числе компьютеров нового поколения ) и обеспечение четкой стратификации между формальными моделями интеллектуальных систем и различными вариантами построения их интерпретаторов, обеспечивающей высокую степень независимости эволюции формальных моделей интеллектуальных систем и эволюции их интерпретаторов. Это требует особой детализации формальных моделей интеллектуальных систем.}
;\scnfileitem{Обеспечение коммуникационной ("социальной"{}) совместимости (договороспособности) интеллектуальных компьютерных систем, позволяющей им самостоятельно формировать коллективы интеллектуальных компьютерных систем и их пользователей, а также самостоятельно согласовывать (координировать) деятельность в рамках этих коллективов при решении сложных задач в непредсказуемых условиях. Без этого невозможна реализация таких проектов, как "умный"{} дом, "умный"{} город, "умное"{} предприятие, "умная"{} больница и т.д.}}
\scnaddlevel{-1}
;\scnfileitem{Создание и перманентное развитие \textit{общей комплексной технологии} проектирования и производства \textit{семантически совместимых} \scnbigspace \textit{интеллектуальных компьютерных систем}, способных координировать свою деятельность с себе подобными}
\scnaddlevel{1}
\scnrelfromset{подзадачи}{
\scnfileitem{Четкое описание стандарта интеллектуальных компьютерных систем, обеспечивающего семантическую совместимость разрабатываемых систем}
;\scnfileitem{Разработка мощных библиотек семантически совместимых и многократно (повторно) используемых компонентов разрабатываемых интеллектуальных компьютерных систем}
;\scnfileitem{Обеспечение низкого порога вхождения в технологию проектирования интеллектуальных компьютерных систем как для пользователей технологии (т.е. разработчиков прикладных или специализированных интеллектуальных компьютерных систем), так и для разработчиков самой технологии}
;\scnfileitem{Обеспечение высоких темпов развития технологии за счет учета опыта разработки различных приложений путем активного привлечения авторов приложений к участию в развитии (совершенствовании) технологии}}
\scnaddlevel{-1}
;\scnfileitem{Разработка компьютеров нового поколения, ориентированных на производство высокопроизводительных \textit{интеллектуальных компьютерных систем} самого различного назначения и высокого качества}
;\scnfileitem{Создание глобальной \textit{экосистемы} взаимодействующих между собой \textit{интеллектуальных компьютерных систем}, обеспечивающих комплексную автоматизацию всех \textit{видов человеческой деятельности}}
\scnaddlevel{1}
\scntext{подзадача}{Построение формальной модели человеческой деятельности в контексте теории smart-общества}
\scnaddlevel{-1}
;\scnfileitem{Создание и перманентное развитие глобальной \textit{социотехнической экосистемы}, которая состоит из \textit{интеллектуальных компьютерных систем}, а также всех пользователей этих систем, которая обеспечивает комплексную автоматизацию всех \textit{видов человеческой деятельности}}
;\scnfileitem{Необходим переход от эклектичного построения сложных \textit{интеллектуальных компьютерных систем}, использующих различные виды \textit{знаний} и различные виды \textit{моделей решения задач}, к их глубокой \textbf{интеграции} и унификации, когда одинаковые модели представления и модели обработки знаний реализуется в разных системах и подсистемах одинаково}
;\scnfileitem{Необходимо сократить дистанцию между современным уровнем \textbf{\textit{теории интеллектуальных компьютерных систем}} и практики их разработки.}}

\scnheader{Искусственный интеллект}
\scnidtf{Деятельность в области Искусственного интеллекта (как совокупность всех форм и направлений этой деятельности)}
\scntext{проблема текущего состояния}{Эпицентром современных проблем развития деятельности в области \textit{Искусственного интеллекта} является \textit{конвергенция} и \textit{глубокая интеграция} всех форм, направлений и результатов этой деятельности. Уровень взаимосвязи, взаимодействия и \textit{конвергенции} между различными формами и направлениями деятельности в области \textit{Искусственного интеллекта} явно недостаточен. Это приводит к тому, что каждая из них развивается обособленно, независимо от других.  Речь идет о \textit{конвергенции} между такими направлениями \textit{Искусственного интеллекта}, как представление знаний, решение интеллектуальных задач, интеллектуальное поведение, понимание и др., а также между такими формами \textit{человеческой деятельности в области Искусственного интеллекта}, как научные исследования, разработка технологий, разработка приложений, образование, бизнес. 
Почему на фоне уже достаточно длительного интенсивного развития научных исследований в области \textit{Искусственного интеллекта} до сих пор не создан рынок интеллектуальных компьютерных систем и комплексная технология \textit{Искусственного интеллекта}, обеспечивающая разработку широкого спектра \textit{интеллектуальных компьютерных систем} самого различного назначения и доступной широкому контингенту инженеров. 
Потому что сочетание высокого уровня наукоемкости и прагматизма этой проблемы требует для ее решения принципиально нового подхода к организации взаимодействия \textit{\uline{ученых}}, работающих в области \textit{Искусственного интеллекта}, \textit{\uline{разработчиков}} средств автоматизации проектирования \textit{интеллектуальных компьютерных систем}, \uline{\textit{разработчиков}} средств реализации интеллектуальных компьютерных систем, включая средства аппаратной поддержки интеллектуальных компьютерных систем, \uline{\textit{разработчиков}} прикладных интеллектуальных компьютерных систем. Такое \uline{целенаправленное} взаимодействие должно осуществляться как в рамках каждой из этих форм деятельности в области \textit{Искусственного интеллекта}, так и между ними. Таким образом, основной тенденцией дальнейшего развития теоретических и практических работ в области \textit{Искусственного интеллекта} является конвергенция как самых разных видов (форм и направлений) человеческой деятельности в области \textit{Искусственного интеллекта}, так и самых разных продуктов (результатов) этой деятельности. Необходимо ликвидировать барьеры между различными видами и продуктами деятельности в области \textit{Искусственного интеллекта} в целях обеспечения их совместимости и интегрируемости.
Проблема создания быстро развивающегося рынка семантически совместимых интеллектуальных систем – это вызов, адресованный специалистам в области \textit{Искусственного интеллекта}, требующий преодоления "вавилонского столпотворения"{} во всех его проявлениях, формирование высокой культуры договороспособности и унифицированной, согласованной формы представления коллективно накапливаемых, совершенствуемых и используемых знаний.
Ученые, работающие в области \textit{Искусственного интеллекта}, должны обеспечить конвергенцию результатов различных направлений \textit{Искусственного интеллекта} и построить: (1) общую теорию интеллектуальных компьютерных систем; (2) общую технологию проектирования семантически совместимых интеллектуальных компьютерных систем, включающую соответствующие стандарты интеллектуальных компьютерных систем и их компонентов. Инженеры, разрабатывающие интеллектуальные компьютерные системы, должны сотрудничать с учеными и участвовать в развитии технологии проектирования интеллектуальных компьютерных систем.}

\newpage
\scnsegmentheader{Понятие технологии OSTIS}

\scnstartsubstruct

\scnheader{Технология OSTIS}
\scnidtf{Комплекс (семейство) технологий, обеспечивающих проектирование, производство, эксплуатацию и реинжиниринг интеллектуальных \textit{компьютерных систем (ostis-систем)}, предназначенных для автоматизации самых различных видов человеческой деятельности и в основе которых лежит смысловое представление и онтологическая систематизация знаний, а также агентно-ориентированная обработка знаний}
\scnidtf{Open Semantic Technology for Intelligent Systems}
\scnaddlevel{1}
\scntext{сокращение}{OSTIS}
\scnaddlevel{-1}
\scnidtf{Семейство (комплекс) \textit{ostis-технологий}}
\scnidtf{Комплексная открытая семантическая технология проектирования, производства, эксплуатации и реинжиниринга гибридных, семантически совместимых, активных и договороспособных \textit{интеллектуальных компьютерных систем}}
\scnheader{Технология OSTIS}
\scnrelfromset{принципы, лежащие в основе}{
\scnfileitem{Ориентация на разработку \textit{интеллектуальных компьютерных систем}, имеющих высокий уровень \textit{интеллекта} и, в частности, высокий уровень \textit{социализации}. Указанные системы, разработанные по \textit{Технологии OSTIS}, будем называть \textbf{ostis-системами}}
;\scnfileitem{Ориентация на \uline{комплексную} автоматизацию всех видов и областей \textit{человеческой деятельности} путем создания сети взаимодействующих и координирующих свою деятельность \textit{ostis-систем}. Указанную сеть \textit{ostis-систем} вместе с их пользователями будем называть \textbf{\textit{Экосистемой OSTIS}}}
;\scnfileitem{Поддержка перманентной эволюции \textit{ostis-систем} в ходе их эксплуатации.}
;\scnfileitem{\textit{Технология OSTIS} реализуется в виде сети \textit{ostis-систем}, которая является частью \textit{Экосистемы OSTIS}.
Ключевой \textit{ostis-системой} указанной сети является \textbf{Метасистема IMS.ostis} (Intelligent MetaSystem), реализующая \textbf{Ядро Технологии OSTIS}, которое включает в себя базовые (предметно независимые) методы и средства проектирования и производства \textit{ostis-систем} с интеграцией в их состав типовых встроенных подсистем поддержки эксплуатации и реинжиниринга \textit{ostis-систем}. Остальные \textit{ostis-системы}, входящие в состав рассматриваемой сети, реализуют различные специализированные \textit{ostis-технологии} проектирования различных классов \textit{ostis-систем}, обеспечивающих автоматизацию любых областей и \textit{видов человеческой деятельности}, кроме \textit{проектирования ostis-систем}.}
;\scnfileitem{Конвергенция и интеграция на основе смыслового представления знаний всевозможных научных направлений \textit{Искусственного интеллекта} (в частности, всевозможных базовых знаний и навыков решения интеллектуальных задач) в рамках \textit{Общей формальной семантической теории ostis-систем}.}
;\scnfileitem{Ориентация на разработку компьютеров нового поколения, обеспечивающих эффективную (в том числе, производительную) интерпретацию логико-семантических моделей \textit{ostis-систем}, представленных базами знаний этих систем и имеющих смысловое представление.}}

\scnsegmentheader{Понятие ostis-системы}

\scnstartsubstruct

\scnheader{ostis-система}
\scnidtf{\textit{интеллектуальная компьютерная система}, спроектированная и реализованная по требованиям и стандартам \textit{Технологии OSTIS}, которые задокументированы в \textit{Общей теории ostis-систем}}

\scnheader{ostis-система}
\scnidtf{интеллектуальная компьютерная система, построенная в соответствии с принципами и требованиями Технологии OSTIS}
\scnidtf{Множество ostis-систем различного назначения}
\scnaddlevel{1}
\scniselement{имя собственное}
\scnaddlevel{-1}
\scnidtf{Множество всевозможных интеллектуальных компьютерных систем, построенных по Технологии OSTIS}

\scnheader{ostis-система}
\scnsubset{интеллектуальная компьютерная система}
\scnidtf{\textit{интеллектуальная компьютерная система}, которая построена в соответствии с требованиями и стандартами \textit{Технологии OSTIS}, что обеспечивает существенное развитие целого ряда \textit{свойств} (способностей) этой \textit{компьютерной системы}, позволяющих значительно повысить \textit{уровень интеллекта} этой системы (и, прежде всего, ее \textit{уровень обучаемости} и \textit{уровень социализации})} 

\scnauthorcomment{Добавить классификацию из пояснения}

\scnsubdividing{индивидуальная ostis-система;коллективная ostis-система\\
\scnaddlevel{1}
    \scnsubdividing{простой коллектив ostis-систем;иерархический коллектив ostis-систем}   
\scnaddlevel{-1}
}

\scnheader{ostis-система}
\scnexplanation{интеллектуальная компьютерная система, разработанная, разрабатываемая или совершенствуемая по технологии OSTIS}
\scnnote{Когда речь идет о таком компоненте технологии OSTIS, как модели ostis-систем, фактически имеется в виду теория ostis-систем, включающая в себя строгое формальное уточнение того, как устроена ostis-система, какова ее архитектура, принципы организации памяти, принципы организации представления информации, принципы организации интерфейса с внешней средой (в том числе, с пользователями)}

\scnheader{ostis-система}
\scnrelfromset{принципы, лежащие в основе}{
\scnfileitem{Информация, хранимая в памяти \textit{ostis-системы}, имеет смысловое представление.}
;\scnfileitem{В основе организации решения задач в памяти \textit{ostis-системы} лежит \textit{агентно-ориентированная модель обработки информации}, управляемая ситуациями и событиями, возникающими в обрабатываемой информации (точнее, в обрабатываемой базе знаний).}
;\scnfileitem{Унификация базового набора (базовой системы) используемых понятий, что является основой обеспечения \textit{семантической совместимости} всех \textit{ostis-систем}.}
;\scnfileitem{В основе структуризации информации (базы знаний), хранимой в памяти \textit{ostis-системы}, лежит иерархическая система \textit{предметных областей} и соответствующих им \textit{формальных онтологий}.}
;\scnfileitem{Способность к пониманию (к семантическому погружению, к семантической интеграции) новых приобретаемых знаний (и, в том числе, новых навыков) в состав текущего состояния \textit{базы знаний}.}
;\scnfileitem{Способность к \textit{семантической конвергенции} (к обнаружению сходств) новых приобретаемых знаний (и, в частности, навыков) со знаниями, входящими в состав текущего состояния базы знаний \textit{ostis-системы}.}
;\scnfileitem{Способность \textit{ostis-системы} поддерживать высокий уровень своей \textit{семантической совместимости} (высокий уровень взаимопонимания) с другими \textit{ostis-системами}.}
;\scnfileitem{Способность ostis-системы согласовывать, координировать свою деятельность с другими \textit{ostis-системами}.}
;\scnfileitem{\scnauthorcomment{статья на OSTIS-2020}}
;\scnfileitem{\scnauthorcomment{статья на OSTIS-2020}}
;\scnfileitem{\scnauthorcomment{статья на OSTIS-2020}}
;\scnfileitem{\scnauthorcomment{статья на OSTIS-2020}}}
\scntext{следовательно}{Перечисленные свойства \textit{ostis-систем} свидетельствуют о том, что они имеют существенно более высокий \textit{уровень интеллекта} и, в частности, более высокий \textit{уровень социализации} по сравнению с современными \textit{интеллектуальными компьютерными системами}. \scnauthorcomment{См. начало Раздела 1.1}}

\scnheader{ostis-система}
\scnrelfromset{принципы, лежащие в основе}{
\scnfileitem{смысловое представление информации в памяти компьютерных систем, направленное на устранение недостатков современных компьютерных систем и технологий путем повышения уровня интеллектуальности компьютерных систем}
;\scnfileitem{децентрализация управления решателем задач
\begin{itemize}
	\item внутренняя МАС
	\item внешняя МАС
\end{itemize}}
;\scnfileitem{интеграция различных видов знаний}
;\scnfileitem{интеграция различных моделей решателей задач}
;\scnfileitem{ориентация на компьютеры нового поколения}
;\scnfileitem{обеспечение семантической совместимости компьютерных систем}
;\scnfileitem{обеспечение поддержания семантической совместимости компьютерных систем в ходе эволюции}
;\scnfileitem{способность к координации деятельности}}

\scnheader{ostis-система}
\scnrelfromset{принципы, лежащие в основе}{
\scnfileitem{Память ostis-системы является графодинамической (т.е. нелинейной (графовой) и структурно перестраиваемой). Переработка информации в памяти ostis-системы сводится не столько к изменению состояния элементов памяти (это происходит только при изменении синтаксического типа элементов и при изменении содержимого тех элементов, которые обозначают файлы), сколько к изменению \uline{конфигурации связей} между ними.}
;\scnfileitem{Хранение информации в памяти ostis-системы ориентируется на \uline{смысловое} представление информации – без синонимов, омонимов знаков и без семантической эквивалентности информационных конструкций.}
;\scnfileitem{С точки зрения архитектуры ostis-система представляет собой \uline{иерархическую} многоагентную систему с общедоступной памятью (т.е. с памятью, общедоступной \uline{всем} агентам ostis-системы). 
Заметим при этом, что общая память большинства исследуемых в настоящее время многоагентных систем является не общедоступной, а распределенной, т.е. представляет собой абстрактное (виртуальное) объединение, в состав которого входит память каждого агента многоагентной системы. Координация деятельности агентов ostis-системы при выполнении сложных \textit{действий в памяти} ostis-системы реализуется также через \textit{память ostis-системы} с помощью хранимых в памяти \textit{методов} решения различных классов задач, а также с помощью хранимых в памяти \textit{планов} решения конкретных задач.
На основании этого можно строить неограниченную иерархическую систему агентов ostis-системы – от элементарных агентов, обеспечивающих выполнение базовых действий в памяти ostis-системы, до неэлементарных агентов, представляющих собой коллективы (группы) элементарных и/или неэлементарных агентов, обеспечивающих решение различных типовых задач с помощью соответствующих методов и планов.}
;\scnfileitem{Организация выполнения \textit{ostis-системой действий во внешней среде} осуществляется следующим образом:
\begin{scnitemize}
	\item Выделяются классы \textit{элементарных действий во внешней среде}, для реализации каждого из которых вводятся \textit{эффекторные агенты} ostis-системы.
	\item Координация деятельности \textit{эффекторных агентов} ostis-системы при выполнении \textit{сложных действий во внешней среде} осуществляется через \textit{память ostis-системы} с помощью хранимых в памяти \textit{методов} и \textit{планов} решения различных задач во \textit{внешней среде}, а также с помощью \textit{рецепторных агентов} ostis-системы, обеспечивающих обратную связь и, соответственно, сенсомоторную координацию.
\end{scnitemize}}}

\scnheader{ostis-система}
\scnrelfromset{принципы, лежащие в основе}{
\scnfileitem{Способность понимать друг друга, а также любого своего пользователя
\scnaddlevel{-2}
\scnidtf {Совместимость используемых понятий (по терминам и по денотационной семантике)}
\scnidtf {Семантическая совместимость}
\scnaddlevel{2}}
;\scnfileitem{Способность поддерживать взаимопонимание в процессе индивидуальной эволюции, приводящей к расширению и/или корректировке системы используемых понятий}
;\scnfileitem{Способность координировать свою деятельность с другими системами при решении задач, которые усилиями одной (индивидуальной) интеллектуальной компьютерной системы не могут быть решены либо принципиально, либо за разумное время}}

\scnheader{ostis-система}
\scnrelfromset{принципы, лежащие в основе}{
\scnfileitem{Высокая степень индивидуальной обучаемости интеллектуальных компьютерных систем
\begin{itemize}
	\item гибкости
	\item стратифицированности
	\item рефлексивности
	\item универсальность средств представления и образования знаний
\end{itemize}}
;\scnfileitem{Высокая степень семантической совместимости и, как следствие, коллективной обучаемости интеллектуальных компьютерных систем
\begin{itemize}
	\item семантической совместимости
\end{itemize}}
;\scnfileitem{Основа для автоматизации рынка знаний}}

\scnmakeset{память*;ostis-система}
\scnrelfrom{сужение второго домена заданного отношения для заданного первого домена}{память ostis-системы}
\scnaddlevel{1}
\scnsubset{смысловая память}
\scnaddlevel{-1}


\scnmakeset{информация, хранимая в памяти кибернетической системы*;  ostis-система}
\scnrelfrom{сужение второго домена заданного отношения для заданного первого домена}{база знаний ostis-системы}
\scnaddlevel{1}
\scnsubset{смысловое представление информации}
\scnaddlevel{-1}


\scnheader{память ostis-системы}
\scnsubset{смысловая память}

\scnheader{информация, хранимая в памяти ostis-системы}
\scnsubset{смысловое представление информации}

\scnheader{решатель задач ostis-системы }
\scnsubset{агентно-ориентированная модель обработки информации в памяти}

\newpage
\scnsegmentheader{Текущее состояние и проблемы дальнейшего развития деятельности в области Искусственного интеллекта}
\scnstartsubstruct

\scntext{аннотация}{Рассмотрим в каких направлениях должна происходить эволюция повышенного качества деятельности в области \textit{Искусственного интеллекта}, а также эволюция продуктов этой деятельности}

\bigskip
\scnfragmentcaption

\scnheader{Научно-исследовательская деятельность в области Искусственного интеллекта}
\scntext{текущее состояние}{}
\scnauthorcomment{добавить из статей}

\scnheader{Научно-исследовательская деятельность в области Искусственного интеллекта}
\scnrelfromset{проблемы текущего состояния}{
\scnfileitem{Отсутствует согласованность систем \textit{понятий} в разных направлениях \textit{Искусственного интеллекта} и, как следствие, отсутствует \textit{семантическая совместимость} и \textit{конвергенция} этих направлений, в результате чего ни о каком движении в направлении построения \textit{общей теории интеллектуальных систем} с высоким уровнем формализации и речи быть не может. Существование и продолжающееся увеличение "высоты барьеров"{} между различными направлениями исследований в области \textit{Искусственного интеллекта} проявляется в том, что специалист, работающий в рамках какого-либо направления \textit{Искусственного интеллекта}, посещая заседания "не своей"{} секции на конференции по \textit{Искусственному интеллекту}, мало что там может понять и, соответственно, извлечь полезного для себя.};
\scnfileitem{Отсутствует мотивация и осознание острой необходимости в указанной \textit{конвергенции} между различными направлениями \textit{Искусственного интеллекта}.};
\scnfileitem{Отсутствует реальное движение в направлении построения \textit{Общей теории интеллектуальных систем}, поскольку отсутствует соответствующая мотивация и осознание острой практической необходимости в этом.}
}

\bigskip
\scnfragmentcaption

\scnheader{Разработка базовой комплексной технологии проектирования интеллектуальных компьютерных систем}
\scntext{текущее состояние}{Современная технология \textit{Искусственного интеллекта} представляет собой целое семейство всевозможных частных технологий, ориентированных на разработку и сопровождение различного вида компонентов \textit{интеллектуальных компьютерных систем}, реализующих самые различные модели представления и обработки информации, различные модели решения задач, ориентированных на разработку различных классов \textit{интеллектуальных компьютерных систем}.}
\scnrelfromset{проблемы текущего состояния}{
\scnfileitem{высокая трудоемкость разработки интеллектуальных компьютерных систем};
\scnfileitem{необходимая высокая квалификация разработчиков};
\scnfileitem{современные технологии \textit{Искусственного интеллекта} принципиально не обеспечивают разработки таких \textit{интеллектуальных компьютерных систем}, в которых устраняются недостатки современных \textit{интеллектуальных компьютерных систем}};
\scnfileitem{совместимость частных технологий \textit{Искусственного интеллекта} практически отсутствует и, как следствие, отсутствует \textit{семантическая совместимость} разрабатываемых \textit{интеллектуальных компьютерных систем}, поэтому их системная интеграция осуществляется \uline{вручную}.};
\scnfileitem{Разрабатываемые \textit{интеллектуальные компьютерные системы} не способны \uline{самостоятельно} координировать свою деятельность друг с другом следовательно
\begin{scnitemize}
\item{нет общей комплексной технологии проектирования интеллектуальных компьютерных систем};
\item{не обеспечивается совместимость и взаимодействие разрабатываемых систем (синтаксическая и семантическая совместимость)};
\item{нет совместимости между существующими частными технологиями проектирования различных компонентов интеллектуальных компьютерных систем (базы знаний, нейросетевые модели, интеллектуальные интерфейсы и т.д.)};
\item{есть инструментальные средства по компонентам, но "склеивать"{} (соединять, интегрировать) это надо вручную};
\item{нет системы инструментальных средств}
\end{scnitemize}
}
}

\bigskip
\scnfragmentcaption

\scnheader{Разработка технологии производства спроектированных интеллектуальных компьютерных систем}
\scntext{текущее состояние}{Был сделан целый ряд попыток разработки \textit{компьютеров} нового поколения, ориентированных на использование в \textit{интеллектуальных компьютерных системах}. Но все они оказались неудачными, так как не были ориентированы на всё многообразие моделей решения задач в \textit{интеллектуальных компьютерных системах}. В этом смысле они не были \textit{\uline{универсальными} компьютерами} для \textit{интеллектуальных компьютерных систем}.}
\scnrelfromset{проблемы текущего состояния}{
\scnfileitem{Разрабатываемые \textit{интеллектуальные компьютерные системы} могут использовать самые различные комбинации \textit{моделей решения интеллектуальных задач} (логических моделей, соответствующих различного вида логикам, нейросетевых моделей различного вида, моделей целеполагания, синтеза планов, моделей управления сложными объектами, моделей понимания и синтеза текстов естественного языка и т.д.). Современные (традиционные, фон-неймановские) \textit{компьютеры} не в состоянии достаточно производительно интерпретировать всё многообразие указанных моделей решения задач. При этом разработка специализированных \textit{компьютеров}, ориентированных на интерпретацию какой-либо одной модели решения задач (нейросетевой модели или какой-либо логической модели) проблему не решает, так как в \textit{интеллектуальной компьютерной системе} необходимо использовать сразу несколько разных моделей решения задач, причём в различных сочетаниях.}
}

\bigskip
\scnfragmentcaption

\scnheader{Специализированная инженерия в области Искусственного интеллекта}
\scnidtf{Деятельность, направленная на разработку \textit{интеллектуальных компьютерных систем} различного назначения с использованием имеющихся для этого моделей, методов и средств}
\scnidtf{Деятельность по проектированию и производству \textit{интеллектуальных компьютерных систем}}
\scnidtf{Деятельность, направленная на формирование рынка \textit{интеллектуальных компьютерных систем}}
\scnrelfrom{в перспективе}{Специализированная инженерия в области \textit{Искусственного интеллекта}, осуществляемая специальной частью Экосистемы OSTIS}
	\scnaddlevel{1}
	\scnrelfrom{продукт}{Экосистема OSTIS}
	\scnrelfrom{субъект действия}{часть Экосистемы OSTIS, осуществляющая специализированную инженерию в области \textit{Искусственного интеллекта}}
	\scnaddlevel{-1}

\scntext{текущее состояние}{}
\scnauthorcomment{добавить из статей}

\scnrelfromset{проблемы текущего состояния}{
\scnfileitem{Отсутствует четкая систематизация многообразия \textit{интеллектуальных компьютерных систем}, соответствующая систематизации автоматизируемых \textit{видов человеческой деятельности}.};
\scnfileitem{Отсутствует \textit{конвергенция} \scnbigspace \textit{интеллектуальных компьютерных систем}, обеспечивающих автоматизацию \textit{областей человеческой деятельности}, принадлежащих одному и тому же \textit{виду человеческой деятельности}.};
\scnfileitem{Отсутствует \textit{семантическая совместимость}(семантическая унификация, взаимопонимание) между \textit{интеллектуальными компьютерными системами}, основной причиной чего является отсутствие согласованной системы общих используемых \textit{понятий}.};
\scnfileitem{Семантическая недружественность \textit{пользовательского интерфейса} и отсутствие встроенной справочной системы, позволяющей запрашивать информацию об элементах интерфейса и возможностях системы, приводят к низкой эффективности эксплуатации всех возможностей \textit{интеллектуальной компьютерной системы}.};
\scnfileitem{Анализ проблем автоматизации всех \textit{видов человеческой деятельности} убеждает в том, что дальнейшая автоматизация \textit{человеческой деятельности} требует не только повышения уровня \textit{интеллекта} соответствующих \textit{интеллектуальных компьютерных систем}, но и реализации их способности
\begin{scnitemize}
\item устанавливать свою \textit{семантическую совместимость} (взаимопонимание) как с другими \textit{компьютерными системами}, так и со своими пользователями\char59
\item поддерживать эту \textit{семантическую совместимость} в процессе собственной эволюции, а также эволюции пользователей и других \textit{компьютерных систем}\char59
\item координировать свою деятельность с пользователями и другими \textit{компьютерными системами} при коллективно решении различных задач\char59
\item участвовать в распределении работ (подзадач) при коллективном решении различных задач.
\end{scnitemize}
Важно подчеркнуть то, что реализация вышеперечисленных способностей создаст возможность для существенной и даже полной автоматизации \textit{системной интеграции} \scnbigspace \textit{компьютерных систем} в комплексы взаимодействующих систем и автоматизации реинжиниринга таких комплексов. Такая автоматизация системной интеграции и её реинжиниринга:
\begin{scnitemize}
\item даст возможность комплексам кибернетических систем \uline{самостоятельно} адаптироваться к решению новых задач\char59
\item существенно повысит эффективность эксплуатации таких комплексов компьютерных систем, так как реинжиниринг системной интеграции компьютерных систем, входящих в такой комплекс, часто востребован (например, при реконструкции предприятия)\char59
\item существенно сокращает число ошибок по сравнению с "ручным"{} (неавтоматизированным) выполнением \textit{системной интеграции} и её \textit{реинжиниринга}, которые, к тому же, требует высокой квалификации.
\end{scnitemize}
Таким образом следующий этап повышения уровня автоматизации \textit{человеческой деятельности} настоятельно требует создания таких \textit{интеллектуальных компьютерных систем}, которые могли бы легко сами (без системного интегратора) объединяться для совместного решения сложных задач. 
}
}

\bigskip
\scnfragmentcaption

\scnheader{Образовательная деятельность в области искусственного интеллекта}
\scntext{текущее состояние}{Целенаправленная подготовка специалистов в области Искусственного интеллекта имеет богатую историю и осуществляется во многих ведущих университетах (Stanford University, MIT, МГУ (Москва), НИУ МЭИ (Москва), РГГУ (Москва), СПбГУ (Санкт-Петербург), ДВФУ (Владивосток), НГТУ (Новосибирск), НТУУ КПИ (Киев), БГУИР (Минск), БГУ (Минск), БрГТУ (Брест) и других).}
\scnrelfromset{проблемы текущего состояния}{
\scnfileitem{Поскольку деятельность в области \textit{Искусственного интеллекта} сочетает в себе и высокую степень наукоемкости и высокую степень сложности инженерных работ, подготовка специалистов в этой области требует одновременного формирования у них как научно-исследовательских навыков, культуры и стиля мышления, так и инженерно-практических навыков, культуры и стиля мышления. С точки зрения методики и психологии обучения сочетание фундаментальной научной и инженерно-практической подготовки специалистов является весьма сложный образовательной педагогической задачей.};
\scnfileitem{Отсутствует \textit{семантическая совместимость} между различными учебными дисциплинами, что приводит к "мозаичности"{} восприятия информации};
\scnfileitem{Отсутствует системный подход к подготовке молодых специалистов в области \textit{Искусственного интеллекта}};
\scnfileitem{Нет персонификации обучения};
\scnfileitem{Нет установки на выявление, раскрытие и развитие таланта творческого проектирования};
\scnfileitem{Отсутствует целенаправленное формирование мотивации к творчеству};
\scnfileitem{Нет формирования навыков работы в реальных коллективах разработчиков};
\scnfileitem{Отсутствует адаптация к реальной практической деятельности};
\scnfileitem{Любая современная технология (в том числе и Технология OSTIS) должна иметь высокие темпы своего развития, поскольку без этого невозможно поддерживать высокий уровень её конкурентоспособности. Но для быстро развиваемой технологии требуется:
\begin{scnitemize}
\item не просто высокая квалификация кадров, использующих и развивающих технологию,
\item но и высокие \uline{темпы} повышения уровня этой квалификации, так как без этого невозможно эффективно использовать и развивать \uline{быстро меняющуюся} технологию.
\end{scnitemize}
\bigskip
Из этого следует, что образовательная деятельность в области \textit{Искусственного интеллекта} и соответствующая ей технология должна быть не просто важной частью деятельности в области \textit{Искусственного интеллекта}, а частью, глубоко интегрированной во все остальные виды деятельности в области \textit{Искусственного интеллекта}. Так, например, каждая \textit{интеллектуальная компьютерная система} должная быть ориентирована не только на обслуживание своих конечных пользователей, не только на организацию целенаправленного взаимодействия со своими разработчиками, которые постоянно совершенствуют эту систему, и не только на обеспечение минимального "порога вхождения"{} для новых конечных пользователей и разработчиков, но и на организацию постоянного и персонифицированного повышения квалификации каждого своего конечного пользователя и разработчика в условиях постоянных изменений, вносимых в указанную \textit{интеллектуальную компьютерную систему}. Для этого эксплуатируемая \textit{интеллектуальная компьютерная система} должна "знать"{}, что в ней изменилось, на что она способна и как эти способности инициировать (содержание и форма, соответствующих пользовательских команд)
}
}

\scnendstruct

\newpage
\scnheader{смысловое представление информации}
\scnrelfromset{принципы, лежащие в основе}{
\scnfileitem{Каждый синтаксически элементарный (атомарный) фрагмент представленной информации является обозначением некоторой сущности, которая может быть реальной или абстрактной, конкретной (фиксированной, константной) или произвольной (переменной), постоянной или временной, четкой (достоверной) или нечеткой (недостоверной с возможным дополнительным уточнением степени правдоподобности).}
	\scnaddlevel{1}
\scntext{следовательно}{В состав смыслового представления информации не могут входить буквы (не являются обозначениями сущностей), слова, словосочетания (не являются элементарными фрагментами), разделители, ограничители (не являются обозначениями сущностей)}
	\scnaddlevel{-1}
;\scnfileitem{В рамках смыслового представления информации отсутствует синонимия (пары синонимичных знаков), омонимия  (омонимичные знаки), семантическая эквивалентность (пары семантически эквивалентных информационных конструкций), т.е. отсутствует любая форма дублирования информации, а также отсутствует неоднозначность соотношения между знаками и их денотатами.}}
	\scnaddlevel{1}
\scntext{следовательно}{Смысловое представление информации не может выглядеть как цепочка (строка, последовательность) синтаксически элементарных фрагментов, поскольку каждая описываемая сущность и взаимно однозначно соответствующий ей ее знак может быть связана не с двумя, а с любым количеством описываемых сущностей. Другими словами, смысловое представление информации является нелинейной (графовой) информационной конструкцией.}
		\scnaddlevel{1}
\scntext{следовательно}{Если внутреннее представление информации в памяти компьютерной системы является смысловым представлением, то обработка информации в такой памяти носит графодинамический характер и сводится не к изменению состояния элементов памяти, а к изменению конфигурации связей между ними.}
	\scnaddlevel{-2}
\scnnote{Ключевая проблема современного этапа развития общей теории интеллектуальных компьютерных систем и технологии их разработки – это проблема обеспечения \textbf{\textit{семантической совместимости}} 
\begin{scnenumerate}
	\item различных видов знаний, входящих в состав баз знаний интеллектуальных компьютерных систем;
	\item различных видов моделей решателей задач;
	\item различных интеллектуальных компьютерных систем в целом;
\end{scnenumerate}
Для решения этой проблемы очевидно необходима унификация (стандартизация) формы представления знаний в памяти интеллектуальных компьютерных систем. Предлагаемым нами подходом для такой унификации и является ориентация на \textbf{\textit{смысловое представление информации}} (знаний) в памяти интеллектуальных компьютерных систем. Основой предполагаемого нами подхода к обеспечению высокого уровня обучаемости т семантической совместимости интеллектуальных компьютерных систем, а также к разработке стандарта интеллектуальных компьютерных систем является унификация \textbf{\textit{смыслового представления информации}} (знаний) в памяти интеллектуальных компьютерных систем и построение глобального \textbf{\textit{смыслового пространства}} знаний.}
\scnaddlevel{1}
\scnnote{Информация в знаковой конструкции в основном содержится не в самих знаках (в их структуре), а в связях между знаками. При этом существенно, чтобы эти связи (синтаксические связи) имели четкую смысловую (семантическую) интерпретацию. 
Если структура знаков содержит информацию об обозначаемой сущности всегда можно заменить на "бесструктурные"{} знаки, которые имеют семантическую окрестность} 

\scnheader{семантическая сеть}
\scnsubset{смысловое представление информации}
\scnexplanation{Семантическая сеть нами рассматривается не как красивая метафора сложноструктурированных знаковых конструкций, а как формальное уточнение понятия смыслового представления информации, как принцип представления информации, лежащей в основе нового поколения компьютерных языков и самих компьютерных систем -- графовых языков и графовых компьютеров.}
\scnsubset{знаковая конструкция}
\scnexplanation{Семантическая сеть -- это знаковая конструкция, обладающая следующими свойствами:
	\begin{scnitemize}
		\item "внутренюю"{} структуру (строение) знаков, входящих в семантическую сеть не требуется учитывать при ее семантическом анализе (понимании)
		\item Смысл семантической сети определяется денотационной семантикой всех входящих в нее знаков и конфигурацией связей инцидентности этих знаков
		\item Из двух инцидентных знаков, входящих в семантическую сеть, один является знаком связи
		\item Отсутствие синонимии, омонимии
	\end{scnitemize}
}
\scnrelfrom{предлагаемый подход}{\scnkeyword{SC-код}}
	\scnaddlevel{1}
	\scnidtf{Предлагаемое в рамках \textit{Технологии OSTIS} уточнение понятия \textit{семантической сети}}
	\scnaddlevel{-1}
\scnsuperset{SC-код}
	\scnaddlevel{1}
	\scnidtf{Semantic Computer Code}
	 \scnrelfromlist{смотрите}{Раздел \ref{intro_sc_code}; Раздел \ref{sd_sc_code}}
	 
\scnheader{многоагентная система}
\scnsubset{кибернетическая система}
\scnexplanation{Кибернетическая система, представляющая собой множество кибернетических систем, способных коммуницировать, т.е. обмениваться информацией друг с другом (причем не обязательно каждый с каждым)}

\scnheader{агент*}
\scnidtf{агент многоагентной системы*}

\scnheader{внешняя среда*}
\scnidtf{внешняя среда кибернетической системы}

\scnheader{память*}
\scnidtf{внутренняя (информационная) среда кибернетической системы}
\scnnote{Не каждая кибернетическая система (в том числе многоагентная система) имеет явно выделенную память, являющуюся хранилищем накапливаемой информации, накапливаемого опыта.}

\scnheader{многоагентная система}
\scnsubdividing{многоагентная система без общей памяти;многоагентная система с общей памятью}
\scnsubdividing{многоагентная система, в которой управление агентами осуществляется только путем обмена сообщениями между ними;многоагентная система, в которой управление агентами осуществляется через общую для них память}
\scnsubdividing{многоагентная система с централизованным управлением агентами;многоагентная система с децентрализованным управлением агентами}
\scnsubdividing{многоагентная система, в которой областью деятельности всех ее агентов является только внешняя среда этой системы;многоагентная система, в которой областью деятельности ее агентов является как внешняя среда, так и память этой системы\\
	\scnaddlevel{1}	
\scnnote{некоторые агенты такой системы могут работать только в памяти}\scnaddlevel{-1}}

	
\scnheader{агентно-ориентированная модель обработки информации в памяти}
\scnidtf{агентно-ориентированная модель решения задач}	
\scnidtf{агентно-ориентированная архитектура решателя задач, представляющая собой многоагентную систему, в которой управление ее агентами осуществляется общей для них памятью и областью деятельности агентов является та же самая общая для них память}
	\scnaddlevel{1}
	\scntext{следовательно}{условием инициирования каждого указанного агента является возникновение в указанной памяти соответствующего вида ситуации или события}
	\scnaddlevel{-1}
\scnreltoset{пересечение}{многоагентная система, в которой управление агентами осуществляется через общую для них память;
многоагентная система с децентрализованным управлением агентами;
многоагентная система, в которой областью деятельности ее агентов является как внешняя среда, так и память этой системы}

\scnheader{агентно-ориентированная модель обработки информации в памяти}
\scnrelfromset{принципы, лежащие в основе}{
\scnfileitem{Распределение целенаправленной деятельности между агентами, выполняющими различные действия в памяти, осуществляется на основе генерируемой в базе знаний иерархической системы, описывающей связь (сведение) инициированных целей (задач) с подцелями (подзадачами).}
;\scnfileitem{Условием инициирования агента является появление в базе знаний формулировки той цели (задачи), которая, во-первых, инициирована, а, во-вторых, либо может быть полностью достигнута (решена) этим агентом, либо может быть этим агентом достигнута (решена) частично.}
;\scnfileitem{В результате частичного достижения (решения) некоторой цели (задачи) агент может сгенерировать новые подцели (подзадачи).}
;\scnfileitem{Таким образом, условием инициирования агента обработки информации (базы знаний) является появление соответствующей этому агенту ситуации или соответствия.}}
\scnrelfrom{предлагаемый подход}{\scnkeyword{абстрактная sc-машина}}
	\scnaddlevel{1}
	\scnidtf{Предлагаемое в рамках технологии OSTIS уточнение понятия агентно-ориентированной модели обработки информации в памяти}
	\scnaddlevel{-1}
\scnsuperset{абстрактная sc-машина}

\scnheader{агентно-ориентированная модель обработки информации в памяти}
\scnnote{Децентрализованное (агентно-ориентированное) управление процессом решения задач в ostis-системах реализуется как на внутреннем уровне (на уровне решателя задач ostis-системы), так и на внешнем уровне (на уровне взаимодействия между ostis-системами)}

\scnheader{стандартизация ostis-систем}
\scnidtf{унификация ostis-систем}
\scnexplanation{Стандартизация ostis-систем включает в себя:
	\begin{scnitemize}
		\item cтандартизацию языка внутреннего представления информации в памяти ostis-систем;
		\item cтандартизацию принципов децентрализованного управления обработкой информации в памяти ostis-систем;
		\item cтандартизацию языка описания ситуаций и событий (в памяти ostis-систем), которые являются условиями инициирования различных информационных процессов в памяти ostis-систем;
		\item стандартизацию базового языка спецификации (описания, программирования) агентов, выполняющих соответствующие информационные процессы в памяти ostis-систем;
		\item стандартизацию базовых языков ввода/вывода информации в/из памяти ostis-систем.
	\end{scnitemize}}

\scnheader{SC-код}
\scnidtf{Стандарт \textit{смыслового представления} информации в памяти \textit{ostis-системы}, а, точнее, \textit{стандарт семантических сетей}}

\scnheader{абстрактная sc-машина}
\scnidtf{Стандарт \textit{агентно-ориентированной модели обработки информации в памяти ostis-системы}}

\scnheader{стандартизация}
\scnidtf{унификация}
\scnrelfromset{проблемы текущего состояния}{
\scnfileitem{Разработка и совершенствование стандартов происходит очень медленно}
;\scnfileitem{В разработке и совершенствовании стандартов принимает участие явно недостаточное число профессионалов -- не учитываются все мнения}
;\scnfileitem{В разработке и совершенствовании стандарта отсутствует четкая методика формирования консенсуса}
;\scnfileitem{При введении новой версии стандарта отсутствует четкая методика перевода на новую версию стандарта всех систем, разработанных по предыдущей версии}}
\scntext{предлагаемый подход}{Стандарт -- это перманентно совершенствуемая база знаний, поддержку эволюции которой осуществляет соответствующий портал}

\scnheader{конвергенция знаний в памяти ostis-системы}
\scnrelfromset{принципы, лежащие в основе}{
\scnfileitem{Вводится \uline{универсальный}  базовый язык внутреннего \uline{смыслового} представления знаний в памяти ostis-систем (\textit{SC-код}), по строению к которому все внутренние языки, ориентированные на представление знаний различного вида (логические языки, языки представления методов решения задач (в частности, программ), язык формулировки задач, онтологические языки и многие другие) являются подъязыками \textit{SC-кода}, синтаксис которых полностью совпадает с синтаксисом \textit{SC-кода}.}
;\scnfileitem{Конвергенция различных знаний сводится к согласованию систем понятий, используемых для представления знаний различного вида. Такое согласование направлено на увеличение числа общих понятий, используемых при представлении различных знаний.}}

\scnheader{конвергенция моделей решения задач в ostis-системе}
\scnrelfromset{принципы, лежащие в основе}{
\scnfileitem{Синтаксис языка представления соответствующего класса методов решения задач в памяти -- синтаксис SC-кода}
;\scnfileitem{Денотационная семантика описывается в виде соответствующей онтологии и представляется в виде текста SC-кода}
;\scnfileitem{Операционная семантика каждой модели решения задач -- коллектив \uline{агентов}. Он может быть иерархическим на основе различных моделей решателей, но есть базовая модель интерпретации \uline{любых} методов -- 
	\begin{scnitemize}
	\item Язык SCP
		\begin{scnitemizeii}
		\item cинтаксис совпадает с синтаксисом SC-кода
		\item денотационная семантика -- процедурный язык программирования в графодинамической памяти
		\item операционная семантика реализуется на уровне программной или аппаратной платформы
		\end{scnitemizeii}
	\item sc-агенты работают в общей среде -- (sc-памяти) параллельно, асинхронно на основе ряда правил, позволяющих им не "мешать"{} друг другу
	\end{scnitemize}}}
	
\scnheader{интеграция знаний в памяти ostis-системы*}
\scnexplanation{Интеграция знаний в памяти ostis-систем сводится к склеиванию (отождествлению) синонимичных знаков}

\scnheader{интеграция моделей решения задач в ostis-системе*}
\scnexplanation{Поскольку модель решения задач, используемая ostis-системой, представлена в памяти ostis-системы как соответствующий вид знаний, интеграция различных моделей решения задач может происходить в ostis-системе точно так же, как и интеграция любых других видов знаний. Кроме того, когда речь идет об интеграции различных моделей решения задач, имеется в виду возможность одновременного использования различных моделей решения задач при обработке одних и тех же знаний и, в частности, при решении одной и той же задачи. Такая возможность в ostis-системе обеспечивается \textit{агентно-ориентированной моделью обработки информации} в памяти ostis-системы. Таким образом, такого рода интеграция различных моделей решения задач для ostis-систем является тривиальной.}

\scnheader{ostis-система}
\scnrelfromset{достоинства}{
\scnfileitem{Высокий уровень способности \textit{ostis-системы} осуществлять семантическую интеграцию знаний в своей памяти (в частности, при погружении новых знаний в текущее состояние базы знаний) \uline{обеспечивается} смысловым характером внутреннего кодирования информации,  хранимой в памяти ostis-системы и, в частности, тем, что во внутреннем коде базы знаний \textit{ostis-системы} запрещены омонимичные знаки и пары синонимичных знаков.}
;\scnfileitem{Высокий уровень способности интегрировать различные виды знаний в \textit{ostis-системах} \uline{обеспечивается} тем, что каждый язык, ориентированный на представление знаний соответствующего вида является \uline{подъязыком} одного и того же базового языка \textit{SC-кода}.}\\
\scnaddlevel{1}
\scnnote{Кроме того можно говорить об иерархии sc-языков}
\scnaddlevel{-1}
;\scnfileitem{Высокий уровень способности интегрировать различные модели решения задач в \textit{ostis-системах} \uline{обеспечивается}:
	\begin{scnitemize}
	\item тем, что все эти модели ориентированы на обработку информации, представленной в \textit{SC-коде}
	\item один и тот же фрагмент базы знаний ostis-системы (т.е. одна и та же конструкция SC-кода) может одновременно обрабатываться несколькими \uline{разными} моделями решения задач
	\item все модели решения задач в ostis-системах интегрируются с помощью одной и той же базовой модели решения задач -- \textit{scp-модели решения задач} \scnauthorcomment{(пояснить)}
	\end{scnitemize}}
;\scnfileitem{Высокий уровень обучаемости \textit{ostis-систем} \uline{обеспечивается}:
	\begin{scnitemize}
	\item высоким уровнем семантической гибкости информации, хранимой в памяти ostis-системы, поскольку каждое удаление или добавление синтаксически элементарного фрагмента хранимой информации, а также удаление или добавление каждой связи инцидентности между такими элементами имеет четкую семантическую интерпретацию;
	\item высоким уровнем стратифицированности хранимой информации, что обеспечивается онтологически ориентированной структуризацией базы знаний ostis-системы; 
	\item высоким уровнем рефлексии ostis-системы, что обеспечивается мощными метаязыковыми возможностями языка внутреннего представления информации (знаний) в памяти \textit{ostis-систем}.
	\end{scnitemize}}
;\scnfileitem{Каждая \textit{ostis-система} имеет высокий \textit{уровень обучаемости} (способности к быстрому расширению своих \textit{знаний} и \textit{навыков}) и высокий \textit{уровень социализации} (способности к эффективному участию в деятельности различных коллективов – коллективов, состоящих из \textit{ostis-систем}, и сообществ, состоящих из \textit{ostis-систем} и людей}
	\scnaddlevel{1}
\scnrelfromset{детализация достоинства}{
\scnfileitem{Существуют четкие формальные критерии, определяющие \textit{уровень семантической совместимости} (уровень семантической конвергенции) различных знаний, навыков, целых \textit{ostis-систем} (точнее, баз знаний этих систем). Очевидно, что \textit{уровень семантической совместимости} прежде всего определяется количеством "точек соприкосновения"{} в сравниваемых \textit{знаниях}, \textit{навыках} и \textit{базах знаний} – это \textit{знаки}, присутствующие \uline{в разных} сравниваемых объектах, но имеющие одинаковые денотаты (т.е. обозначающие одинаковые сущности). При этом среди таких знаков, обозначающих одинаковые сущности и присутствующих в разных сравниваемых объектах особенно важны знаки, обозначающие \textit{понятия}.
Количество таких общих понятий в сравниваемых знаниях, навыках, базах знаний определяет уровень семантической совместимости (уровень согласованности) систем используемых понятий в сравниваемых указанных объектах. Увеличение количества знаков, обозначающих одинаковые сущности и присутствующих в разных сравниваемых объектах, может привести к тому, что в разных указанных сравниваемых объектах будут присутствовать не только семантически эквивалентные знаки, но и семантически эквивалентные целые фрагменты (целые информационные конструкции).
Существенно при этом подчеркнуть, что семантически эквивалентные знаковые конструкции, представленные на внутреннем языке ostis-систем (в SC-коде), в памяти разных ostis-систем всегда являются синтаксически изоморфными графовыми конструкциями, в которых соответствие изоморфизма связывает знаки, хранимые в памяти разных ostis-систем, но обозначающие одинаковые сущности (точнее, одну и ту же сущность). Заметим также, что в рамках памяти каждой индивидуальной \textit{ostis-системы} синонимия знаков и, соответственно, семантическая эквивалентность знаковых конструкций запрещены.}
;\scnfileitem{Благодаря постоянно развиваемым семантическим стандартам \textit{Технологии OSTIS} , которые представлены системой формальных онтологий для самых различных предметных областей, разрабатываемые \textit{ostis-системы} \uline{изначально} имеют достаточно высокий \textit{уровень семантической совместимости} со всеми остальными \textit{ostis-системами}. Более того, в \textit{Технологии OSTIS} выделяется целое ядро всех ostis-систем, содержащее фундаментальные базовые знания и базовые навыки, одинаковые для всех ostis-систем и позволяющее каждой копии этого ядра развиваться (общаться, специализироваться) в любом направлении.}
;\scnfileitem{Каждая ostis-система, взаимодействуя с людьми (пользователями) или с другими ostis-системами, обладает способностью повышать уровень семантической совместимости (взаимопонимания) с ними, а также поддерживать (сохранять) высокий уровень такой совместимости в условиях (1) собственной эволюции, (2) эволюции других ostis-систем и пользователей, (3) эволюции семантических стандартов Технологии OSTIS. Указанное взаимодействие, в основном, направлено на согласование изменений в системе используемых понятий, т.е. корректировки соответствующих фрагментов онтологий.}
;\scnfileitem{Благодаря высокому уровню семантической совместимости ostis-систем и смысловому представлению знаний в памяти ostis-систем существенно снижается сложность и повышается качество семантического анализа и понимания информации, поступающей (сообщаемой, передаваемой) ostis-системе от других ostis-систем или пользователей.}
;\scnfileitem{Каждая ostis-система способна:
	\begin{scnitemize}
	\item самостоятельно или по приглашению войти в состав ostis-коллектива (коллектива ostis-систем) или в состав ostis-сообщества, состоящего из ostis-систем и людей. Такие коллективы и сообщества создаются на временной (разовой) или постоянной основе для коллективного решения сложных задач;
	\item участвовать в распределении (в т.ч. в согласовании распределения) задач -- как "разовых"{} задач, так и долгосрочных задач (обязанностей);
	\item мониторить состояние всего процесса коллективной деятельности и координировать свою деятельность с деятельностью других членов коллектива при возможных непредсказуемых изменениях условий (состояния) соответствующей среды.
	\end{scnitemize}}}
	\scnaddlevel{-1}
;\scnfileitem{Высокий уровень интеллекта ostis-систем и, соответственно, высокий уровень их самостоятельности и целенаправленности позволяет ostis-системам быть полноправными членами самых различных сообществ, в рамках которых ostis-системы получают права самостоятельно инициировать (на основе детального анализа текущего положения дел и, в том числе, текущего состояния плана действий сообщества) широкий спектр действий (задач), выполняемых другими членами сообщества, и тем самым участвовать в согласовании и координации деятельности членов сообщества.}
;\scnfileitem{Способность ostis­системы согласовывать свою деятельность с другими ostis-системами, а также корректировать деятельность всего коллектива ostis-систем, адаптируясь к различного вида изменениям среды (условий), в которой эта деятельность осуществляется, позволяет существенно автоматизировать деятельность системного интегратора как на этапе сборки коллектива ostis-систем, так и на этапе его обновления (реинжиниринга).}}
\scnnote{Достоинства ostis-систем обеспечиваются:
	\begin{scnitemize}
	\item достоинствами SC-кода -- языка внутреннего кодирования информации, хранимой в памяти ostis-систем;
	\item достоинствами организации sc-памяти -- памяти ostis-систем;
	\item достоинствами sc-моделей баз знаний ostis\textit{–}систем – средствами структуризации таких баз знаний;
	\item достоинствами sc-моделей решения задач -- агентно-ориентированных моделей решения задач, используемых в ostis-системах.
	\end{scnitemize}}
	
\scnendstruct \scninlinesourcecommentpar{Завершили рассмотрение понятия ostis-системы}

\scnsegmentheader{Понятие ostis-технологии}
\scnstartsubstruct

\scnheader{ostis-технология}
\scnreltoset{объединение}{
ostis-технология проектирования\\
\scnaddlevel{1}
	\scnrelfromset{разбиение}{
		ostis-технология проектирования ostis-систем соответствующего класса\\
		\scnaddlevel{1}
			\scnhaselement{Базовая ostis-технология проектирования ostis-систем}
		\scnaddlevel{-1}
		;ostis-технология проектирования соответствующего класса компонентов ostis-систем\\
		\scnaddlevel{1}
			\scnhaselement{Базовая ostis-технология проектирования баз знаний ostis-систем}
			\scnhaselement{Базовая ostis-технология проектирования решателей задач ostis-систем}
			\scnhaselement{Базовая ostis-технология проектирования интерфейсов ostis-систем}
		\scnaddlevel{-1}
		;ostis-технология проектирования объектов заданного класса, не являющихся ostis-системами\\
	}
\scnaddlevel{-1}
;ostis-технология производства\\
\scnaddlevel{1}
	\scnsuperset{технология производства спроектированных ostis-систем}
	\scnsuperset{ostis-технология управления производством спроектированных продуктов заданного класса, не являющихся ostis-системами}
\scnaddlevel{-1}
;технология эксплуатации ostis-систем\\
\scnaddlevel{1}
	\scnhaselement{Базовая технология эксплуатации ostis-систем}
	\scnsuperset{технология эксплуатации ostis-систем соответствующего класса}
	\scnaddlevel{1}
		\scnsuperset{ostis-технология управления производством спроектированных продуктов заданного класса, не являющихся ostis-системами}
		\scnaddlevel{1}
			\scnidtf{технология эксплуатации ostis-систем управления производством спроектированных продуктов заданного класса, не являющихся ostis-системами}
			\scnaddlevel{-1}
	\scnaddlevel{-1}
\scnaddlevel{-1}
;технология реинжиниринга ostis-систем\\
\scnaddlevel{1}
	\scnhaselement{Базовая технология реинжиниринга ostis-систем}
	\scnsuperset{технология реинжиниринга ostis-систем соответствующего класса}
\scnaddlevel{-1}
}

\scnheader{ostis-технология}
\scnidtf{компонент Технологии OSTIS}
\scnhaselement{Ядро Технологии OSTIS}
\scnaddlevel{1}
	\scnidtf{Базовая ostis-технология}
\scnaddlevel{-1}
\scnsuperset{частная ostis-технология}
\scnaddlevel{1}
	\scnsuperset{ostis-технология проектирования соответствующего класса компонентов ostis-систем}
	\scnaddlevel{1}
		\scnhaselement{Технология проектирования баз знаний ostis-систем}
		\scnhaselement{Технология проектирования решателей задач ostis-систем}
		\scnhaselement{Технология проектирования невербальных интерфейсов ostis-систем с внешней средой}
		\scnhaselement{Технология проектирования интерфейсов ostis-систем с другими техническими системами}
		\scnhaselement{Технология проектирования пользовательских интерфейсов ostis-систем}
	\scnaddlevel{-1}
\scnaddlevel{-1}
\scnsuperset{специализированная ostis-технология проектирования \scnkeyword{ostis-систем соответствующего класса}}
\scnaddlevel{1}
	\scnhaselement{Технология проектирования ostis-систем управления предприятиями рецептурного производства}
	\scnhaselement{Технология проектирования ostis-систем управления предприятиями производства молочной продукции}
	\scnhaselement{Технология проектирования интеллектуальных обучающих ostis-систем}
	\scnhaselement{Технология проектирования интеллектуальных обучающих ostis-систем для школьников}
	\scnhaselement{Технология проектирования интеллектуальных обучающих ostis-систем для подготовки специалистов в области Математики}
	\scnhaselement{Технология проектирования интеллектуальных обучающих ostis-систем для подготовки специалистов в области Искуственного интеллекта}
\scnaddlevel{-1}

\scnheader{ostis-технология проектирования}
\scnnote{Каждой ostis-технологии проектирования соответсвует своя ostis-система автоматизации проектирования соответствующего класса объектов}
\scnrelfrom{соответствующее семейство средств автоматизации}{ostis-система автоматизации проектирования}
\scnrelfrom{соответствующее семейство классов проектируемых объектов}{
	{\normalfont ( } ostis-система автоматизации проектирования ostis-систем	$\cup$ ostis-система автоматизации проектирования объектов, не являющихся ostis-системами {\normalfont ) }
}
\scnsuperset{ostis-технология проектирования ostis-систем соответствующего класса}

\scnheader{ostis-технология проектирования ostis-систем соответствующего класса}
\scnidtf{технология проектирования \textit{ostis-систем} соответствующего ( заданного) класса, который, в свою очередь, соответствует определенному \textit{виду человеческой деятельности}, подвиды которого автоматизируются с помощью указанных выше проектируемых \textit{ostis-систем}}

\scnheader{ostis-технология}
\scnrelfromlist{отношение, заданное на данном множестве}{частная технология*; специализированная технология*; комплекс специализированных технологий*}
\scnidtf{Базовая частная или специализированная технология, входящая в состав комплексной \textit{Технологии OSTIS}, которая:
\begin{scnitemize}
	\item направлена на автоматизацию конкретного вида человеческой деятельности;
	\item ориентирована на использование ostis-систем (как индивидуальных, так и коллективных) в качестве самостоятельных субъектов или активных интеллектуальных инструментов, либо на использование человеко-машинных ostis-сообществ при решении:
	\begin{scnitemizeii}
		\item как задач, выполняемых в памяти ostis-систем (в т.ч. в памяти коллективов ostis-систем);
		\item так и задач, выполняемых во внешней среде ostis-систем, в процессе решения которых субъектами соответствующих действий либо ostis-системы (индивидуальные или коллективные), либо конкретные персоны, либо ostis-сообщества.
	\end{scnitemizeii}
\end{scnitemize}
}
\scnidtf{Множество всевозможных технологий, соответствующих стандартам технологии OSTIS и направленных на автоматизацию различных конкретных видов человеческой деятельности}
\scnrelboth{следует отличать}{Технология OSTIS}
\scnaddlevel{1}
	\scnnote{\textit{Технология OSTIS} в отличие от понятия \textit{ostis-технологии} представляет собой не множество технологий, а комплекс взаимосвязанных между собой самых различных технологий, превращающий указанное множество технологий в единую объединенную технологию, в сумму взаимосвязанных глубоко интегрированных технологий. В этом смысле Технология OSTIS является максимальной ostis-технологией, в состав которой входят все ostis-технологии.}
\scnaddlevel{-1}
\scnsuperset{пример*:}
\scnaddlevel{1}
	\scnlistitem{ostis-технология проектирования и перепроектирования} 	\scnlistitem{ostis-технология производства}
	\scnlistitem{ostis-технология публикации и согласования результатов научно-технической деятельности (в широком смысле)}
	\scnlistitem{ostis-технология образования}
\scnaddlevel{-1}
\scnhaselements{пример':}
\scnaddlevel{1}
	\scnlistitem{Технолония OSTIS}
	\scnlistitem{OSTIS-технология проектирования, реализации и реинжиниринга ostis-систем}
	\scnlistitem{OSTIS-технология разработки стандартов Технологии OSTIS}
\scnaddlevel{-1}

\scnheader{ostis-технология коллективной разработки информационных ресурсов (различного вида)}
\scnsuperset{ostis-технология коллективного проектирования}
\scnsuperset{ostis-технология коллективной разработки планов}
\scnsuperset{ostis-технология публикации и согласования результатов научно-технической деятельности}
\scnsubset{ostis-технология}

\scnheader{ostis-технология эксплуатации ostis-систем}
\scnidtf{Общие методы и средства (языковые и интерфейсные) организации взаимодействия ostis-систем со своими конечными пользователями}
\scnsubset{ostis-технология}
\scnnote{Поскольку в рамках Экосистемы OSTIS каждому человеку придется взаимодействовать с больщим числом ostis-систем разного назначения, принципы организации взаимодействия всех ostis-систем со своими пользователями должны быть абсолютно одинаковыми. Удобство (usability) пользовательских интерфейсов должно быть направлено не только на синтаксическую красоту, но и на простую семантическую интерпретацию (понятность).}

\scnheader{ostis-технология проектирования ostis-систем}
\scnidtf{Технология построения (разработки) логико-семантических моделей (sc-моделей) ostis-систем}
\scniselement{ostis-технология}
\scnnote{Продуктом каждого завершенного (целостного) коллективного проекта, реализованного в рамках этой технологии, является полная \textit{логико-семантическая модель ostis-системы}.}
\scnrelfrom{класс продуктов}{логико-семантическая модель ostis-системы}
\scnrelfrom{средство}{Метасистема IMS OSTIS}
\scnrelfrom{класс субъектов}{коллектив разработчиков ostis-системы}
\scnrelfrom{класс исходных данных}{исходная спецификация ostis-системы}

\scnheader{ostis-технология реализации ostis-систем}
\scnidtf{Технология сборки и установки ostis-систем}
\scniselement{ostis-технология}
\scnrelfrom{исходная информация}{логико-семантическая модель ostis-системы}
\scnrelfrom{комплектация}{универсальный интерпретатор логико-семантических моделей ostis-систем}
\scnaddlevel{1}
	\scnnote{Это, своего рода, "мотор"{}, "движок"{} ostis-систем}
\scnaddlevel{-1}
\scnrelfrom{методы}{Методика реализации ostis-систем}
\scnrelfrom{активный инструмент}{Метасистема IMS OSTIS}
\scnrelfrom{продукты}{ostis-система}

\scnheader{ostis-технология обновления ostis-систем}
\scnidtf{Технология реинжирования (перепроектирования) ostis-систем в ходе их эксплуатации}
\scniselement{ostis-технология}
\scnheader{следует отличать*}
\scnhaselementset{Технология обновления ostis-систем; Технология проектирования ostis-систем}
\scnaddlevel{1}
	\scnnote{Эти технологии сходны. Их методы и средства совпадают. Не совпадают только исходные данные и результаты, которыми в Технологии обновления ostis-систем являются предшествующие и последующие состояния ostis-систем. В Технологии проектирования ostis-систем исходными данными являются исходные спецификации (замыслы) проектируемых ostis-систем, и результатами -- полные логико-семантические модели этих систем}
\scnaddlevel{-1}

\scnheader{Технология OSTIS}
\scnidtf{Совокупность (интеграция, объединение) всех ostis-технологий}
\scnrelto{интеграция}{ostis-технология}
\scnidtf{Комплекс (множество) семантически совместимых \textit{технологий}, в состав которого входит \textit{Ядро Технологии OSTIS} и иерархическая система \textit{ostis-технологий}, каждая из которых ориентирована на \textit{проектирование}, \textit{производство}, \textit{эксплуатацию} или \textit{реинжиринг} соответствующего \textit{класса ostis-систем}, обеспечивающих автоматизацию соответствующего \textit{вида человеческой деятельности}. При этом каждая такая проектируемая \textit{ostis-система} автоматизирует либо область, либо \textit{вид человеческой деятельности}, которая (который) является соответственно либо экземпляром (элементом), либо подвидом (подклассом) указанного выше \textit{вида человеческой деятельности}, соответствующего используемой \textit{специализированной ostis-технологии}.}

%стр 197 уточнить ключевой элемент (неролевое или ролевое?)
\scnheader{Ядро Технологии OSTIS}
\scnidtf{Универсальная базовая \textit{ostis-технология}}
\scnidtf{Универсальный компонент Технологии OSTIS}

\scniselementrole{ключевой элемент}{\itshape ostis-технология}
\scnrelto{ядро}{Технология OSTIS}
\scnhaselement{технология}
\scnrelfrom{вид деятельности, выполняемой с помощью технологии}{проектирование, производство, эксплуатация и реинжиринг ostis-системы}
\scnaddlevel{1}
	\scnreltoset{объединение}{
		проектирование ostis-системы\\
		\scnaddlevel{1}
			\scnidtf{построение логико-семантической модели ostis-системы}
		\scnaddlevel{-1}
		;производство ostis-системы\\
		\scnaddlevel{1}
			\scnidtf{сборка логико-семантической модели ostis-системы и загрузка этой модели в память универсального интерпретатора таких моделей}
		\scnaddlevel{-1}
		;эксплуатация ostis-системы\\
		\scnaddlevel{1}
			\scnidtf{базовый (предметно-независимый) уровень организации деятельности конечного пользователя ostis-системы с помощью соответствующих методов	и средств}
		\scnaddlevel{-1}
		;реинжиринг ostis-системы\\
		\scnaddlevel{1}
			\scnidtf{совершенствование ostis-системы в процессе её эксплуатации}
		\scnaddlevel{-1}
	}
	\scnrelfrom{создаваемые продукты}{
		ostis-система\\
		\scnidtf{\textit{интеллектуальная компьютерная система}, построенная в соответствии со стандартом \textit{Технологии OSTIS}, предъявляемым к продуктам, создаваемым с помощью этой технологии}
		\scnaddlevel{1}
			\scnnote{Указанный стандарт продуктов, создаваемых с помощью технологии OSTIS есть не что иное, как \textit{общая формальная семантическая теория интеллектуальных компьютерных систем}}
		\scnaddlevel{-1}
	}
\scnaddlevel{-1}
\scnrelfromlist{частная технология}{
	Базовая Технология Проектирования ostis-систем\\
	\scnaddlevel{1}
		\scnrelfromlist{частная технология}{
			Технология проектирования баз знаний ostis-систем\\
			;Технология проектирования решателей задач ostis-систем\\
			;Технология проектирования интерфейсов ostis-систем\\
			\scnrelfromlist{частная технология}{
				Технология проектирования невербальных интерфейсов ostis-систем с внешней средой\\
				;Технология проектирования интерфейсов ostis-систем с другими техническими системами\\
				;Технология проектирования пользовательских интерфейсов ostis-систем
			}
		}
		\scnrelfrom{реализация}{Метасистема IMS.ostis}
		\scnaddlevel{1}
			\scnidtf{Intelligent MetaSystem for ostis-systems design}
			\scnidtf{OSTIS-система автоматизации проектирования ostis-систем}
		\scnaddlevel{-1}
	\scnaddlevel{-1}
	;Технология производства ostis-систем\\
	\scnaddlevel{1}
		\scnexplanation{Основным компонентом, точнее, инструментальным средством \textit{технологии производства ostis-систем} является \textit{универсальный интерпретатор логико-семантических моделей ostis-систем}. Указанные \textit{логико-семантические модели ostis-систем} являются результатом \textit{проектирования ostis-систем} и представляют собой начальные (исходные) состояния \textit{баз знаний} разрабатываемых \textit{ostis-систем}. В отличие от \textit{инструмента производства ostis-систем}, методика их производства весьма проста и сводится к сборке разработанных логико-семантических моделей (начального состояния \textit{баз знаний}) разрабатываемых \textit{ostis-систем} и загрузке этих моделей в память \textit{универсального интерпретатора логико-семантических моделей ostis-систем}.}
		\scnrelfrom{реализация}{универсальный интерпретатор логико-семантических моделей ostis-систем}
		\scnaddlevel{1}
			\scnexplanation{Такой интерпретатор логико-семантических моделей ostis-систем может быть реализован либо программно на \textit{современных компьютерах}, либо аппаратно в виде компьютеров нового поколения, ориентированных на реализацию интеллектуальных компьютерных систем.}
			\scnexplanation{С формальной точки зрения универсальный интерпретатор логико-семантических моделей ostis-систем является "пустой"{} остис системой, способной только на приём формализованной информации и её запись в свою память.}
		\scnaddlevel{-1}
	\scnaddlevel{-1}
	;Базовая технология эксплуатации ostis-систем\\
	\scnaddlevel{1}
		\scnidtf{Общая технология эксплуатации ostis-систем, включающая в себя общие методы и средства, используемые в процессе эксплуатации любых ostis-систем}
		\scnrelfrom{реализация}{встраиваемая ostis-система поддержки эксплуатации ostis-систем}
		\scnaddlevel{1}
			\scnexplanation{Данная ostis-система входит (интегрирована) в состав каждой ostis-системы.}
		\scnaddlevel{-1}
	\scnaddlevel{-1}
	;Базовая технология реинжиниринга ostis-систем\\
	\scnaddlevel{1}
		\scnrelfrom{реализация}{встраиваемая ostis-система поддержки реинжиниринга ostis-систем}
		\scnaddlevel{1}
			\scnexplanation{Данная ostis-система входит (интегрирована) в состав каждой ostis-системы и обеспечивает внесение изменений "руками"{} инженеров, сопровождающих эксплуатацию ostis-системы, или авторов базы знаний этой ostis-системы в текущее состояние базы знаний ostis-системы в ходе её экспуатации}
		\scnaddlevel{-1}
	\scnaddlevel{-1}
}
\scnrelfromlist{специализированная технология}{
	Общая технология проектирования ostis-систем автоматизации проектирования\\
	\scnaddlevel{1}
		\scnrelfromlist{специализированная технология}{
		Технология проектирования ostis-систем автоматизации проектирования строительных объектов\\
		;Технология проектирования ostis-систем автоматизации проектирования автомобилей\\
		;Технология проектирования ostis-систем автоматизации проектирования интегральных микросхем
		}
	\scnaddlevel{-1}
	;Технология проектирования ostis-систем управления производством\\
	\scnaddlevel{1}
		\scnrelfromlist{специализированная технология}{
		Технология проектирования ostis-систем управления строительством различных объектов\\
		;Технология проектирования ostis-систем управления производством автомобилей\\
		;Технология проектирования ostis-систем управления производством микросхем\\
		;Технология проектирования ostis-систем управления предприятиями рецептурного производства\\
		\scnaddlevel{1}
			\scnrelfrom{специализированная технология}{				Технология проектирования ostis-систем управления предприятиями производства молочной продукции}
		\scnaddlevel{-1}
		}
	\scnaddlevel{-1}
	;Технология проектирования интеллектуальных обучающих ostis-систем\\
	\scnaddlevel{1}
		\scnrelfromset{комплекс специализированных технологий}{
		Технология проектирования интеллектуальных обучающих ostis-систем для школьников\\
		;Технология проектирования интеллектуальных обучающих ostis-систем для студентов по общеобразовательным дисциплинам\\
		;Технология проектирования интеллектуальных обучающих ostis-систем для студентов по профильным дисциплинам\\
		;Технология проектирования интеллектуальных обучающих ostis-систем для магистрантов
		}
		\scnrelfromset{комплекс специализированных технологий}{
		Технология проектирования интеллектуальных обучающих ostis-систем по Математике\\
		;Технология проектирования интеллектуальных обучающих ostis-систем по Искусственному интеллекту
		}
		%стр 205 уточнить многоточие
		\scnnote{При проектировании конкретной обучающей ostis-системы необходимо использовать сразу две ostis-технологии...}
	\scnaddlevel{-1}
}

\scnheader{специализированная ostis-технология}
\scnnote{Приведённый нами перечень \textit{специализированных ostis-технологий} охватывает только некоторые области (фрагменты) \textit{человеческой деятельности}, подлежащие автоматизации с помощью \textit{ostis-технологий} в рамках \textit{Экосистемы OSTIS}.}

\scnheader{Ядро Технологии OSTIS}
\scnnote{Форма реализации \textit{Ядра Технологии OSTIS} (в виде ostis-системы \textit{IMS.ostis}) позволяет:
\begin{scnitemize}
	\item использовать достоинства \textit{Технологии OSTIS} для повышения уровня автоматизации развития самой \textit{Технологии OSTIS} и для существенного повышения темпов такого развития;
	\item приобрести очень важный опыт применения \textit{Технологии OSTIS};
	\item создать центрально ядро \textit{Экосистемы OSTIS}, обеспечивающее поддержку семантической совместимости всех \textit{ostis-систем} и \textit{ostis-сообществ}, входящих в состав \textit{Экосистемы OSTIS}.
\end{scnitemize}
}

\scnendstruct
\scninlinesourcecommentpar{Завершили рассмотрение \textit{понятия ostis-технологии}}

\newpage
\begin{SCn}

\scnfragmentcaption

\scnheader{Специализированная инженерия в области Искусственного интеллекта}
\scnrelfrom{предлагаемый подход}{Специализированная инженерия, осуществляемая на основе Технологии OSTIS}
\scnaddlevel{1}
\scnrelfromset{декомпозиция}{Разработка ostis-систем автоматизации проектирования различных классов ostis-систем\\
    \scnaddlevel{1}
    \scnidtf{Разработка специализированных ostis-технологий}
    \scnrelfromlist{часть}{Разработка ostis-систем автоматизации проектирования ostis-систем автоматизации проектирования\\
        \scnaddlevel{1}
        \scnidtf{Разработка ostis-технологий проектирования}
        \scnaddlevel{-1}
    ;Разработка ostis-систем автоматизации проектирования ostis-систем автоматизации производства\\
        \scnaddlevel{1}
        \scnidtf{Разработка ostis-технологий управления производством}
        \scnaddlevel{-1}
    ;Разработка ostis-систем автоматизации проектирования ostis-систем управления транспортными системами\\
        \scnaddlevel{1}
        \scnidtf{Разработка ostis-технологий управления транспортными системами}
        \scnaddlevel{-1}
    ;Разработка ostis-систем автоматизации проектирования диагностических ostis-систем\\
        \scnaddlevel{1}
        \scnidtf{Разработка ostis-технологий диагностики (технической, медицинской)}
        \scnaddlevel{-1}
    ;Разработка ostis-систем автоматизации проектирования обучающих ostis-систем\\
        \scnaddlevel{1}
        \scnidtf{Разработка ostis-технологий обучения людей}
        \scnaddlevel{-1}
    ;Разработка ostis-систем автоматизации проектирования ostis-систем управления "умными"{} домами\\
        \scnaddlevel{1}
        \scnidtf{Разработка ostis-технологий управления "умными"{} домами}
        \scnaddlevel{-1}
    ;Разработка ostis-систем автоматизации проектирования ostis-систем управления "умными"{} больницами\\
        \scnaddlevel{1}
        \scnidtf{Разработка ostis-технологий управления "умными"{} больницами}
        \scnaddlevel{-1}
    ;Разработка ostis-систем автоматизации проектирования ostis-систем управления "умными"{} поликлиниками\\
        \scnaddlevel{1}
        \scnidtf{Разработка ostis-технологий управления "умными"{} поликлиниками}
        \scnaddlevel{-1}
    ;Разработка ostis-систем автоматизации проектирования ostis-систем управления "умными"{} городскими районами\\
        \scnaddlevel{1}
        \scnidtf{Разработка ostis-технологий управления "умными"{} городскими районами}
        \scnaddlevel{-1}
    ;Разработка ostis-систем автоматизации проектирования ostis-систем управления "умными"{} городами\\
        \scnaddlevel{1}
        \scnidtf{Разработка ostis-технологий управления "умными"{} городами}
        \scnaddlevel{-1}
    ;и т. д.}
    \scnaddlevel{-1}
;Разработка (на основе соответствующих ostis-технологий проектирования) ostis-систем автоматизации проектирования различных классов объектов, не являющихся ostis-системами\\
    \scnaddlevel{1}
    \scnrelfromlist{часть}{Разработка семейства ostis-систем автоматизации проектирования различных видов интегральных микросхем;Разработка семейства ostis-систем автоматизации проектирования различных видов автомобилей;Разработка семейства ostis-систем автоматизации проектирования различных видов строительных объектов;и т.д.}
    \scnaddlevel{-1}
;Разработка ostis-систем автоматизации производства\\
    \scnaddlevel{1}
    \scnidtf{Разработка интеллектуальных систем управления производственными предприятиями}
    \scnaddlevel{-1}
;Разработка ostis-систем управления транспортными средствами;Разработка диагностических ostis-систем;Разработка обучающих ostis-систем;Разработка ostis-систем управления "умными"{} домами;Разработка ostis-систем управления "умными"{} больницами;Разработка ostis-систем управления "умными"{} поликлиниками;Разработка ostis-систем управления "умными"{} городскими районами;Разработка ostis-систем управления "умными"{} городами;и т.д.}
\scnaddlevel{-1}

\bigskip
\scnfragmentcaption

\scnheader{Образовательная деятельность в области Искусственного интеллекта}
\scnrelfrom{предлагаемый подход}{Образовательная деятельность в области Искусственного интеллекта, осуществляемая на основе Технологии OSTIS}
    \scnaddlevel{1}
    \scniselement{образовательная деятельность}
    \scniselement{человеческая деятельность, осуществляемая на основе Технологии OSTIS}
        \scnaddlevel{1}
        \scnidtf{человеческая деятельность, комплексная автоматизация которой осуществляется либо индивидуальной \textit{ostis-системой}, либо \textit{коллективом (сетью) ostis-систем} (сетью ostis-систем)}
        \scnaddlevel{-1}
    \scnrelfrom{субъект}{OSTIS-сообщество Образовательной деятельности в области Искусственного интеллекта, осуществляемой на основе Технологии OSTIS}
		\scnaddlevel{1}    	
    	\scnidtf{глобальное (максимальное) OSTIS-сообщество, осуществляющее Образовательную деятельность в области Искусственного интеллекта и обеспечивающее активное и взаимовыгодное сотрудничество между всеми заинтересованными в этом субъектами и, в первую очередь, с соответствующими кафедрами различных вузов}
    	
    	\scnrelto{часть}{Экосистема OSTIS}
        	\scnaddlevel{1}
	        \scnidtf{глобальная сеть ostis-систем вместе с их пользователями}
    	    \scnidtf{глобальное ostis-сообщество}
	        \scnaddlevel{-1}
    	\scniselement{ostis-сообщество}
        	\scnaddlevel{1}
	        \scnidtf{локальная сеть ostis-систем вместе с их пользователями}
	        \scnaddlevel{-1}
	    \scnexplanation{Данное \textit{ostis-сообщество} включает в себя:
\begin{scnitemize}
    \item все кафедры, которые готовят молодых специалистов в области Искусственного интеллекта и которые могут входить в состав самых различных вузов;
    \item все те организации, которые разрабатывают или эксплуатируют интеллектуальные компьютерные системы и которые готовы сотрудничать с вузами для повышения квалификации поступающих к ним молодых специалистов в области Искусственного интеллекта;
    \item студентов, магистрантов и аспирантов, обучающихся в области Искусственного интеллекта в разных вузах;
    \item их преподавателей;
    \item семейство интеллектуальных обучающих ostis-систем по различным дисциплинам (направлениям) Искусственного интеллекта, которые семантически совместимы и тесно связаны с \textit{OSTIS-порталом научных знаний по Искусственному интеллекту} и с \textit{Метасистемой IMS.ostis};
    \item \textit{OSTIS-портал научных знаний по Искусственному интеллекту}, осуществляющий поддержку развития Общей теории интеллектуальных систем как естественного, так и искусственного происхождения;
    \item \textit{Метасистема IMS.ostis}, осуществляющая поддержку развития Общей теории интеллектуальных компьютерных систем (искусственных интеллектуальных систем) и поддержку развития Базовой универсальной комплексной технологии проектирования интеллектуальных компьютерных систем;
    \item семейство персональных ostis-ассистентов студентов, магистрантов и аспирантов, обучающихся в области Искусственного интеллекта;
    \item семейство персональных ostis-ассистентов преподавателей, осуществляющих подготовку молодых специалистов в области Искусственного интеллекта;
    \item семейство кафедральных корпоративных ostis-систем, осуществляющих управление учебным процессом на уровне кафедр, обеспечивающих подготовку молодых специалистов в области Искусственного интеллекта. В рамках таких корпоративных ostis-систем осуществляется:
    \begin{scnitemizeii}
        \item составление кафедрального расписания занятий на следующий семестр и его согласование с расписанием других кафедр этого же вуза;
        \item распределение учебной нагрузки на очередной семестр и учебный год;
        \item мониторинг проведения различного вида занятий (лекций, консультаций, семинаров, практических занятий, зачетов/экзаменов);
        \item мониторинг самостоятельной деятельности обучаемых (курсовых и дипломных проектов, рефератов, диссертаций, тестов и др.);
        \item фиксация текущего соответствия между учебными дисциплинами и разделами Общей теории интеллектуальных систем и Базовой универсальной комплексной технологии проектирования интеллектуальных компьютерных систем (речь идет не только о дисциплинах, непосредственно относящихся к Искусственному интеллекту, но и о различных общеобразовательных и смежных дисциплинах, таких, как теория познания, методология, иностранные языки, современные компьютерные системы и сети, компьютеры нового поколения, теория алгоритмов и программ, ориентированных на современные компьютеры, семантическая теория алгоритмов и программ, ориентированных на обработку баз знаний и др.). Принципиально важно сформировать у студентов, магистрантов и аспирантов целостную картину проблематики Искусственного интеллекта и место Искусственного интеллекта в общей Картине Мира. Барьеров между учебными дисциплинами быть не должно.
    \end{scnitemizeii}
    \item Корпоративная ostis-система OSTIS-сообщества, являющегося субъектом Образовательной деятельности в области Искусственного интеллекта. Через эту корпоративную ostis-систему осуществляется взаимодействие между всеми членами указанного ostis-сообщества и, прежде всего между кафедрами, осуществляющими подготовку молодых специалистов в области Искусственного интеллекта.
\end{scnitemize}}
        \scnaddlevel{-1}        
    \scnaddlevel{-1}

\scnrelfromvector{принципы, лежащие в основе}{\scnfileitem{Подготовка молодых специалистов в области Искусственного интеллекта должна осуществляться путем поэтапного и непосредственного их подключения к реальным коллективным проектам:
\begin{scnitemize}
    \item к развитию базы знаний по Общей теории интеллектуальных систем, хранимой в памяти соответствующего интеллектуального портала знаний{;}
    \item к развитию базы знаний по \textit{Общей теории интеллектуальных компьютерных систем}, хранимой в памяти соответствующего интеллектуального портала знаний (в памяти \textit{Метасистемы IMS.ostis}){;}
    \item к развитию базы знаний по Базовой комплексной технологии проектирования интеллектуальных компьютерных систем, хранимой в памяти интеллектуальной компьютерной системы автоматизации проектирования интеллектуальных компьютерных систем (в памяти Метасистемы IMS.ostis){;}
    \item к развитию различных методов и средств проектирования различных компонентов интеллектуальных компьютерных систем{;}
    \item к развитию различных специализированных технологий проектирования различных классов интеллектуальных компьютерных систем{;}
    \item к разработке различных прикладных интеллектуальных компьютерных систем на основе развиваемой Базовой (универсальной) комплексной технологии проектирования интеллектуальных компьютерных систем.
\end{scnitemize}}
;\scnfileitem{Каждый студент и магистрант в процессе обучения привлекается к нескольким разным формам деятельности в области Искусственного интеллекта и, в частности, обязательно и к разработке приложений, и к развитию технологий. Специалист, занимающийся автоматизацией какой-либо деятельности должен на себе прочувствовать проблемы и трудности этой автоматизируемой деятельности}
;\scnfileitem{Все студенты, магистранты и преподаватели должны активно участвовать в анализе эффективности своей образовательной деятельности и активно способствовать повышению эффективности и повышению уровня автоматизации этой деятельности с помощью развиваемой технологии проектирования и производства интеллектуальных компьютерных систем. Данный принцип можно условно назвать устранением синдрома "сапожника без сапог"{}.}
;\scnfileitem{Результаты самостоятельной работы студентов и магистрантов (лабораторных работ, практических занятий, рефератов, курсовых работ и проектов, дипломных работ и проектов, магистерских диссертаций) должны быть востребованы в тех проектах, к которым они подключены и должны быть доведены до уровня внедрения в эти проекты, т. е. должны быть по соответствующей процедуре согласованы и одобрены. При этом приветствуется и соответствующим образом поощряется любая такого рода инициатива студентов и магистрантов. Указанная востребованность (полезность) результатов самостоятельной работы студентов и магистрантов предполагает то, что отчеты по этим результатам оформляются в формализованном виде -- в виде исходных текстов соответствующих фрагментов баз знаний. При этом указанные результаты могут требовать как весьма высокой квалификации, так и не очень высокой (например, квалификации первокурсника). К таким несложным, но весьма полезным работам относятся:
\begin{scnitemize}
    \item введение в базы знаний полезных библиографических ссылок и цитат;
    \item сравнительный анализ различных положений, представленных в некоторой разрабатываемой базе знаний;
    \item различные пояснения, примечания и комментарии, вводимые в базу знаний;
    \item спецификация выявленных в разрабатываемой базе знаний ошибок,  противоречий, информационных дыр и информационного мусора;
    \item примеры, иллюстрирующие различные понятия;
    \item упражнения к различным разделам разрабатываемых баз знаний, которые особенно актуальны для интеллектуальных компьютерных систем, используемых в учебном процессе (это не только интеллектуальные обучающие системы).
\end{scnitemize}}
;\scnfileitem{Вклад каждого студента и магистранта в развитие всех проектов, в которых он принимает участие, фиксируется и при подведении итогов по каждому семестру соответствующим образом оценивается. Это своего рода предтеча будущего рынка знаний.}
;\scnfileitem{Учебным пособием по каждой учебной дисциплине должна быть база знаний или некоторый раздел базы знаний некоторой интеллектуальной компьютерной системы. Такой может быть либо интеллектуальная обучающая система, либо, например, Метасистема IMS.ostis. Условием максимально эффективного проведения лекционного занятия является предварительное прочтение студентами или магистрантами материала предстоящей лекции (соответствующего раздела базы знаний). Тогда на лекции можно акцентировать внимание не на изложение материала, опубликованного в виде базы знаний, а на обсуждение непонятных фрагментов этого материала, на обсуждение проблем, касающихся содержания (принципиальных положений) этого материала. Все это формирует культуру взаимопонимания и согласования различных точек зрения, а также способствует повышению качества базы знаний, представляющей материал соответствующей учебной дисциплины.}
;\scnfileitem{Важнейшей задачей подготовки молодых специалистов является формирование у них:
\begin{scnitemize}
	\item высокой математической культуры (культуры формализации);
	\item высокой системной культуры (понимания того, что количество далеко не всегда переходит в ожидаемое качество);
	\item высокого уровня технологической культуры, технологической дисциплины, четкого соблюдения текущих стандартов и способности участвовать в эволюции стандартов;
	\item способности работать в наукоемких проектах в составе творческих коллективов с децентрализованным управлением;
	\item способности к достижению семантической совместимости (взаимопонимания) со своими коллегами;
	\item договороспособности (способности к согласованию различных точек зрения).
\end{scnitemize}}
;\scnfileitem{Подготовку молодых специалистов в области Искусственного интеллекта можно осуществлять с ориентацией на следующие условно выделенные уровни их квалификации:
\begin{scnitemize}
	\item инженерия прикладных интеллектуальных компьютерных систем по заданной технологии{;}
	\item инженерия специализированных технологий проектирования различных классов прикладных интеллектуальных компьютерных систем (на основе базовой универсальной комплексной технологии проектирования интеллектуальных компьютерных систем){;}
	\item инженерия базовой универсальной комплексной технологии проектирования интеллектуальных компьютерных систем{;}
	\item инженерия программных и аппаратных средст, интерпретации логико-семантических моделей интеллектуальных компьютерных систем{;}
	\item инженерия комплексов интеллектуальных компьютерных систем{;}
	\item научно-исследовательская деятельность по развитию \textit{Общей формальной теории интеллектуальных компьютерных систем}.
\end{scnitemize}}}

\end{SCn}

\newpage
\scnsegmentheader{Понятие Экосистемы OSTIS}

\scnstartsubstruct

\scnidtf{Использование \textit{Технологии OSTIS} для повышения качества и, в частности, уровня автоматизации всех \textit{областей человеческой деятельности}}
\scnidtf{Понятие \textbf{Экосистемы OSTIS} как формы реализации \textit{smart-общества}, представляющего собой сеть взаимодействующих людей, интеллектуальных компьютерных систем, "умных"{} домов, "умных"{} предприятий, "умных"{} больниц, "умных"{} учебных заведений, "умных"{} городов, "умных"{} транспортных систем и т.п.}

\scnrelfromset{рассматриваемые вопросы}{
\scnfileitem{Какова архитектура \textit{Экосистемы OSTIS}};
\scnfileitem{Какова архитектура \textit{ostis-сообщества}, входящего в состав \textit{Экосистемы OSTIS}};
\scnfileitem{Как взаимодействуют между собой различные \textit{ostis-сообщества} в рамках \textit{Экосистемы OSTIS}};
\scnfileitem{Как интегрируется \textit{деятельность} различных \textit{ostis-сообществ} и результаты этой \textit{деятельности}};
\scnfileitem{Какова типология \textit{ostis-сообществ} и по каким признакам классификации можно эту типологию проводить};
\scnfileitem{Можно ли опыт автоматизации деятельности в области Искусственного интеллекта с помощью \textit{Технологии OSTIS} расширить на все многообразие областей и видов человеческой деятельности};
\scnfileitem{Как выглядит систематизация областей и видов человеческой деятельности};
\scnfileitem{Как осуществляется конвергенция и интеграция различных областей и видов человеческой деятельности};
\scnfileitem{Как взаимодействуют ostis-системы, осуществляющие автоматизацию различных областей видов человеческой деятельности};
\scnfileitem{Как может выглядеть \uline{комплексная} автоматизация всех областей и видов \textit{человеческой деятельности} с помощью \textit{Технологии OSTIS}}
}
\scnrelfromvector{план изложения}{
\scnfileitem{Что такое Экосистема OSTIS};
\scnfileitem{Структура Экосистемы OSTIS};
\scnfileitem{Что такое ostis-система, являющаяся агентом Экосистемы OSTIS};
\scnfileitem{Что такое ostis-сообщетво, являющееся агентом Экосистемы OSTIS};
\scnfileitem{Что такое Проект создания Экосистемы OSTIS};
\scnfileitem{Цель создания и основные свойства Экосистемы OSTIS};
\scnfileitem{Как структурируется человеческая деятельность};
\scnfileitem{Как выглядит рынок знаний, реализуемый в рамках Экосистемы OSTIS};
\scnfileitem{Чем определяется качество человеческой деятельности};
\scnfileitem{Что такое эффективная автоматизация человеческой деятельности};
\scnfileitem{Почему повышение эффективности человеческой деятельности невозможно без интеллектуальных компьютерных систем};
\scnfileitem{Какие достоинства имеет Экосистема OSTIS}
}

\bigskip
\scnfragmentcaption

\scnheader{Экосистема OSTIS}
\scntext{вопрос}{Какова структура Экосистемы OSTIS}
\scnheader{Экосистема OSTIS}
\scnidtf{Популяция
\begin{scnitemize}
\item семантически совместимых
\item эволюционируемых
\item активно взаимодействующих  
\item способных координировать(согласовывать) свою деятельность с другими субъектами
\end{scnitemize}
интеллектуальных компьютерных систем(\textit{ostis-систем}). При этом указанная популяция \textit{ostis-систем} поддерживает децентрализованное управление собственной деятельностью, а также деятельностью людей(пользователей \textit{ostis-систем}) и человеко-машинных сообществ(\textit{ostis-сообществ}), обеспечивая тем самым автоматизацию системной интеграции любых новых субъектов(\textit{ostis-систем}, людей, \textit{ostis-сообществ}) в состав \textit{Экосистемы OSTIS}.
}
\scnheader{Экосистема OSTIS}
\scnrelfromvector{принципы, лежащие в основе}
{
\scnfileitem{\textit{Экосистема OSTIS} представляет собой сеть \textit{ostis-сообществ}};
\scnfileitem{Каждому \textit{ostis-сообществу} взаимно однозначно соответствует \textit{корпоративная ostis-система} этого \textit{ostis-сообщества}, которая:
\begin{scnitemize}
\item обеспечивает координацию деятельности членов соответствующего \textit{ostis-сообщества};
\item является "представителем"{} этого \textit{ostis-сообщества} в других \textit{ostis-сообществах}, членом которых указанное \textit{ostis-сообщество} является.
\end{scnitemize}
};
\scnfileitem{Каждое \textit{ostis-сообщество} может входить в состав любого другого \textit{ostis-сообщества} по своей инициативе. Формально это означает, что \textit{корпоративная ostis-система} первого \textit{ostis-сообщества} является членом другого \textit{ostis-сообщества}.};
\scnfileitem{Каждому специалисту, входящему в состав Экосистемы OSTIS ставится во взаимнооднозначное соответствие его \textit{персональный ostis-ассистент}, который трактуется как \textit{корпоративная ostis-система} вырожденного \textit{ostis-сообщества}, состоящего из одного человека.}
}
\scnheader{следует отличать*}
\scnhaselementset{корпоративная ostis-система*
\scnaddlevel{1}
\scnidtf{корпоративная ostis-система данного ostis-сообщества*}
\scnaddlevel{-1};
корпоративная ostis-система
\scnaddlevel{1}\\
\scnrelto{второй домен}{корпоративная ostis-система*}
\scnaddlevel{-1};
член ostis-сообщества*;
персональный ostis-ассистент*
\scnaddlevel{1}
\scnidtf{персональный ostis-ассистент данного специалиста*}
\scnsubset{корпоративная ostis-система*}
\scnaddlevel{-1};
персональный ostis-ассистент
\scnaddlevel{1}\\
\scnrelto{второй домен}{персональный ostis-ассистент*}
\scnsubset{корпоративная ostis-система}
\scnaddlevel{-1}
}
\scnheader{есть сходства*}
\scnhaselementset{Экосистема OSTIS;
ostis-сообщество\\
\scnaddlevel{1}
\scnhaselement{Экосистема OSTIS}
\scnaddlevel{-1}
}
\scnaddlevel{1}
\scnexplanation{Экосистема OSTIS является максимальным ostis-сообществом, включающим в себя все существующее ostis-сообщества}
\scnaddlevel{-1}
\scnheader{Экосистема OSTIS}
\scnidtf{Максимальное \textit{иерархическое ostis-сообщество}, обеспечивающее комплексную автоматизацию \uline{всех} видов \textit{человеческой деятельности}}
\scnidtf{Максимальное ostis-сообщество такое ostis-сообщество, для которого не существует другого ostis-сообщества, содержащее указанное выше ostis-сообщество в качестве своего члена}
\scnidtf{Симбиоз людей и \textit{компьютерных систем}(точнее, \textit{ostis-систем}) являющийся вариантом реализации \textit{smart-общества}}
\scniselement{иерархическое ostis-сообщество}
\scnaddlevel{1}
\scnidtf{такое \textit{ostis-сообщество}, по крайней мере одним из членов которого является некоторое другое \textit{ostis-сообщество}}
\scnsubset{ostis-сообщество}
\scnaddlevel{1}
\scnrelboth{следует отличать}{коллектив ostis-систем}
\scnaddlevel{-1}
\scnaddlevel{-1}{
\scnrelto{основной продукт}{Технология OSTIS}
\scnrelto{вариант реализации}{smart-общество}
\scnheader{агент Экосистемы OSTIS}
\scnidtf{субъект Экосистемы OSTIS}
\scnidtf{субъект, входящий в состав Экосистемы OSTIS}
\scnsuperset{когнитивный агент Экосистемы OSTIS}
\scnsubdividing{индивидуальная ostis-система Экосистемы OSTIS
\scnaddlevel{1}
\scnidtf{индивидуальная ostis-система, входящая в состав Экосистемы OSTIS}
\scnaddlevel{-1};
ostis-сообщество Экосистемы OSTIS\\
\scnaddlevel{1}
\scnsubdividing{
простое ostis-сообщество Экосистемы OSTIS;
иерархическое ostis-сообщество Экосистемы OSTIS
}
\scnaddlevel{-1};
пользователь Экосистемы OSTIS
}
\scnheader{агент Экосистемы OSTIS}
\scnrelfrom{правила поведения}{ 
Правила поведения агентов Экосистемы OSTIS\\
\scneqtoset{
\scnfileitem{Согласовывать денотационную семантику всех используемых знаков(в первую очередь \uline{понятий})};
\scnfileitem{Согласовывать терминологию, соответствующую введенным знакам устранять противоречия и информационные дыры};
\scnfileitem{Постоянно бороться с синонимией и омонимией как на уровне sc-элементов(внутренних знаков), так и на уровне соответствующих им терминов и прочих внешних идентификаторов(внешних обозначений)};
\scnfileitem{Каждый агент Экосистемы OSTIS по своей инициативе может стать членом любого ostis-сообщества Экосистемы OSTIS после соответствующей регистрации}
}
}
\scnheader{Правила поведения агентов Экосистемы OSTIS}
\scnnote{
Существенно подчеркнуть, что все правила функционирования(поведения) в рамках агентов Экосистемы OSTIS должны соблюдать не только ostis-системы, являющиеся агентами(субъектами) этой Экосистемы, но и люди, которые являются её агентами. И здесь возникают очень важные проблемы, обусловленные человеческим фактором. Дело в том, что убедить человека соблюдать правила, пусть даже те которые направлены на максимальную его самореализацию и в совершенствовании которых он может реально участвовать, очень непросто, поскольку любые  подобные правила многими воспринимаются как ограничение их творческой свободы. Другими словами корректное поведение ostis-системы в роли агентов Экосистемы OSTIS значительно проще, чем корректное поведение людей в качестве таких агентов. Поведение пользователей (естественных агентов) Экосистемы OSTIS необходимо внимательно мониторить и контролировать, постоянно способствуя повышению уровня их квалификации как агентов Экосистемы OSTIS, а также повышению уровня их мотивации, целенаправленности, самореализации.}
\scnheader{следует отличать*}
\scnhaselementset{
агент Экосистемы OSTIS;
член ostis-сообщества*
}
\scnheader{Экосистема OSTIS}
\scntext{архитектура}{
В Экосистеме OSTIS можно выделить следующие уровни иерархии:
\begin{scnitemize}
\item индивидуальные компьютерные системы(\textit{индивидуальные ostis-системы} и \textit{люди}, являющиеся конечными пользователями ostis-систем);
\item иерархическая система ostis-сообществ, членами каждого из которых могут быть \textit{индивидуальные ostis-системы}, люди, а также другие \textit{ostis-сообщества};
\item \textit{Максимальное ostis-сообщество} \scnbigspace \textit{Экосистемы OSTIS}, не являющееся членом никакого другого \textit{ostis-сообщества}, входящего в состав \textit{Экосистемы OSTIS}.
\end{scnitemize}
Подчеркнем, что качество \textit{Экосистемы OSTIS} во многом определяется эффективностью взаимодействия каждой \textit{ostis-системы}(в том числе и каждого \textit{ostis-сообщества}), а также каждого \textit{человека} со своей \textit{внешней средой*}, а также качеством(чистотой), самой \textit{внешней среды*}.
Но внешняя среда каждого \textit{субъекта} каждой ostis-системы и каждого человека, входящего в \textit{Экосистему OSTIS} - это не только \textit{материальная внешняя среда*}, но и \textit{информационная внешняя среда*}, представляющая собой виртуальный распределенный информационный ресурс, являющийся интеграцией(объединением) информации, хранящейся в текущий момент в памяти всех других(остальных) \textit{субъектов}, входящих в \textit{Экосистему OSTIS}. Основной целью \textit{Экосистемы OSTIS} является повышение качества(в том числе чистоты) \textit{информационной внешней среды*} для \uline{всех} \textit{субъектов}, входящих в \textit{Экосистему OSTIS}. Фактически речь идет об \textbf{Информационной экологии человеческого общества}.
}
\scnheader{Информационная экология человеческого общества}
\scnnote{
Говоря об \textit{Информационной экологии человеческого общества} необходимо заметить следующее. Современные подходы к развитию взаимодействия с информационной средой человеческого общества можно разбить на два направления:
\begin{scnitemize} 
\item на разработку средств приспособления к недостаткам текущего состояния этой среды
\item на устранение этих недостатков путем наведения порядка в устной информационной среде и её систематизации.
\end{scnitemize}
Технология OSTIS и реализация Экосистемы OSTIS целенаправленно и в известной степени радикально ориентирована на второе направление, памятуя искусственный(рукотворный) характер происхождения этой информационной среды.}
%памятуя?
\scnendstruct



\scnendstruct
\newpage
\scnaddlevel{1}
\scnidtf{Человеко-машинная деятельность, осуществляемая в рамках \textit{Экосистемы OSTIS} и направленная на разработку и перманентное совершенствование \textit{Метасистемы 
IMS.ostis}, которая является формой представления (отображения) (1) текущего состояния \textit{Технологии OSTIS}, как комплекса методов и средств автоматизации (поддержки) разработки\textit{ostis-систем} и (2) текущего состояния самого \textit{Проекта IMS.ostis}.}
\scntext{примечание}{Принципы (правила) организации деятельности в рамках \textit{Проекта IMS.ostis} полностью совпадают с принципами (правилами) организации деятельности в рамках любого другого проекта, направленного на разработку и совершенствование любой другой ostis-системы.}
\scnrelto{ключевой подпроект}{Проект Экосистемы OSTIS}
\scnaddlevel{1}
\scnidtf{Совместная деятельность ученых, инженеров и ostis-систем, входящих в \textit{Экосистему OSTIS}, направленная на перманентное совершенствование \textit{Экосистемы OSTIS} -- на совершенствование (реинжиниринг) входящих в неё  \textit{ostis-систем} и на создание новых ostis-систем и их включение в состав \textit{Экосистемы OSTIS.}}
\scnaddlevel{-1}
\scntext{пояснение}{
ostis-система, являющаяся:
\begin{scnitemize}
	\item ostis-порталом научно-технических знаний по Технологии OSTIS, база знаний которого включает в себя:
	\begin{scnitemizeii}
		\item формальную теорию ostis-систем
		\item формальную теорию (методику) проектирования 󠇦 ostis-систем
		\item формальную спецификацию средств автоматизации проектирования ostis-систем
		\item библиотеку проектирования ostis-систем
		\item формальную спецификацию средств производства спроектированных ostis-систем
	\end{scnitemizeii}
	\item ostis-системой автоматизации (поддержки) проектирования ostis-систем
	\item ostis-системой поддержки производства (сборки, синтеза, генерации) спроектированных ostis-систем
	\item ostis-системой поддержки реинжиниринга ostis-систем в ходе их эксплуатации
\end{scnitemize}
}
\scnaddlevel{-1}

\scnheader{Метасистема IMS.ostis}
\scnidtf{Универсальная базовая (предметно-независимая) ostis-система автоматизации проектирования ostis-систем (любых ostis-систем)}
\scnrelboth{следует отличать}{специализированная ostis-система автоматизации проектирования ostis-систем}
\scniselement{ostis-система}
\scnrelto{корпоративная ostis-система}
{Консорциум OSTIS}
\scnidtf{IMS.ostis}
\scnidtf{Интеллектуальная метасистема, построенная по стандартам \textit{технологии OSTIS} и предназначенная (1) для инженеров \textit{ostis-систем} -- для поддержки проектирования. Реализации и обновления (реинжиниринга) \textit{ostis-систем} и для разработчиков \textit{Технологии OSTIS} -- для поддержки коллективной деятельности по развитию стандартов и библиотек \textit{Технологии OSTIS.}}
\scnrelto{форма реализации}{Технология OSTIS}
\scnrelto{продукт}{Проект IMS.ostis}
\scnidtf{Интеллектуальная Метасистема, являющаяся формой (вариантом) реализации (представления, оформления) \textit{Технологии OSTIS} в виде \textit{ostis-системы}}
\scntext{примечание}{Тот факт, что Технология OSTIS реализуется в виде ostis-системы, является весьма важным для эволюции Технологии OSTIS, поскольку методы и средства эволюции (перманентного совершенствования) Технологии OSTIS становятся фактически совпадающими с методами и средствами разработки любой (!) ostis-системы на всех этапах их жизненного цикла.\\
Другими словами, эволюция Технологии OSTIS осуществляется методами и средствами самой этой технологии.}
\scnidtf{Система комплексной автоматизации (информационной и инструментальной поддержки) проектирования и реализации ostis-систем, которая сама реализована также в виде ostis-системы.}
\scnidtf{Портал знаний по Технологии OSTIS, интегрированный с САПРом ostis-систем и реализованный в виде ostis-системы.}
\scniselement{портал научно-технических знаний}

\bigskip
\bigskip
\scnstartset
\scnheader{Метасистема IMS.ostis}
\scniselement{система автоматизации проектирования}
\scnaddlevel{1}
\scnidtf{CAD-система}
\scnaddlevel{1}
\scnrelto{аббревиатура}{\scnfilelong{Computer Aided Design system}}
\scnaddlevel{-2}
\scniselement{интеллектуальная обучающая система}
\scnendstruct

\scnrelboth{семантическая эквивалентность}{\scnfilelong{Метасистема IMS.ostis является одновременно и системой автоматизации проектирования ostis-систем, и интеллектуальной системой, обучающей методам  и средствам проектирования ostis-систем.}}
\scnaddlevel{1}
\scntext{следовательно}{этот факт существенно повышает качество проектирования прикладных ostis-систем, расширяет контингент разработчиков ostis-систем и интегрирует проектную (инженерную) деятельность в области искусственного интеллекта с образовательной деятельностью в этой области.}
\scnaddlevel{-1}

\scnheader{следует отличать*}
\scnhaselementset{
конвергенция
\scnaddlevel{1}
\scnidtf{Процесс сближения структурных и/или функциональных характеристик нескольких (как минимум двух) заданных сущностей}
\scnidtf{Процесс конвергенции заданных сущностей в ходе их изменения, совершенствование, эволюции}
\scnsubset{процесс}
\scnaddlevel{-1};
конвергенция\scnsupergroupsign 
\scnaddlevel{1}
\scnidtf{Степень близости (сходство) заданных сущностей}
\scniselement{свойство}
\scnaddlevel{-1}
}
\scnheader{конвергенция} 
\scnnote{ 
\textit{Конвергенция} пар конкретных искусственных сущностей (например, технических систем) есть стремление их унификацию (в частности, к стандартизации), т.е. стремление к минимизации многообразия форм решения аналогичных практических задач -- стремление к тому, чтобы все, что можно сделать одинаково, сделалось одинаково, но без ущерба требуемого качества. Последнее очень важно, так как безграмотная стандартизация может привести к существенному торможению прогресса. Ограничение многообразия форм не должно приводить к ограничению содержания, возможностей. Образно говоря, "словам должно быть тесно, а мыслям -- свободно".}
\scnnote{Методологически конвергенция искусственно создаваемых сущностей (артефактов) сводится (1) к выявлению (обнаружению) принципиальных сходств между этими сущностями, которые часто весьма закамуфлированы и их трудно "увидеть", и (2) к реализации обнаруженных сходств одинаковым образом (в одинаковой форме, в одинаковом "синтаксисе"). Образно говоря, от "семантической"{} (смысловой) эквивалентности требуется перейти и к "синтаксической" эквивалентности. Кстати, в этом как раз и заключается суть (идея) смыслового представления информации (знаний), целью которого является создание такой языковой среды (\textit{смыслового пространства}), в рамках которого (1) семантически эквивалентные информационные конструкции полностью совпадали, а (2) конвергенция информационных конструкций сводилась бы к выявлению изоморфных фрагментов этих конструкций.}
\scnnote{Очень важно уточнить, формализовать понятие конвергенции (конвергенции знаний, методов, модели решения задач, конвергенции интеллектуальных компьютерных систем в целом)}
\scnsuperset{конвергенция информационных конструкций}
\scnaddlevel{1}
\scnidtf{конвергенция синтаксических и семантических свойств информационных конструкций }
\scnaddlevel{-1}
\scnsuperset{конвергенция языков}
\scnsuperset{конвергенция научных дисциплин}
\scnaddlevel{1}
\scnidtf{конвергенция различных научных дисциплин или различных направлений одной и той же и дисциплины}
\scnaddlevel{-1}
\scnsuperset{конвергенция баз знаний}
\scnsuperset{конвергенция моделей решения задач}
\scnsuperset{конвергенция гибридных решателей задач}
\scnsuperset{конвергенция кибернетических систем}
\scnsuperset{конвергенция интеллектуальных систем}
\scnaddlevel{1}
\scnsuperset{конвергенция интеллектуальных систем, направленная на обеспечение их \uline{семантической совместимости}}
\scnaddlevel{-1}

\scnheader{конвергенция результатов научно-технической деятельности}
\scnnote{Важным препятствием для конвергенции результатов научно-технической деятельности является сформировавшийся в науке и технике акцент на выявлении не сходств, а отличий. Чтобы убедиться в этом достаточно обратить внимание на то, что уровень научных результатов оценивается научной \uline{новизной}, которая может имитироваться новизной не по существу, а по форме представления (например, с помощью новых понятий или даже новых терминов). Результаты в технике, например, в патентах также оцениваются \uline{отличиями} от предшествующих технических решений. Но для конвергенции нужны другие акценты -- ни поиск отличий, а выявление неочевидных сходств и превращения их в очевидные сходства, представленные в одинаковой \uline{форме}.}

\scnheader{совместимость\scnsupergroupsign}
\scnidtf{совместимость заданных двух или более сущностей\scnsupergroupsign}
\scnidtf{простота интеграции заданной группы сущностей\scnsupergroupsign}
\scnidtf{интегрируемость\scnsupergroupsign}
\scnnote{Степень (уровень) совместимости заданных сущностей может рассматриваться как оценка результата их конвергенции. Чем качественнее (основательнее, глубже) проведена конвергенция заданных сущностей, тем выше уровень их совместимости и, собственно, тем легче их интегрировать.}

\scnsuperset{cовместимость информационных конструкций\scnsupergroupsign}
\scnaddlevel{1}
\scnsuperset{семантическая совместимость информационных конструкций\scnsupergroupsign}
\scnaddlevel{-1}
\scnsuperset{совместимость языков\scnsupergroupsign}
\scnaddlevel{1}
\scnsuperset{семантическая совместимость языков\scnsupergroupsign}
\scnaddlevel{-1}
\scnsuperset{семантическая совместимость научных дисциплин\scnsupergroupsign}
\scnsuperset{совместимость баз знаний\scnsupergroupsign}
\scnsuperset{совместимость моделей решения задач\scnsupergroupsign}
\scnsuperset{совместимость кибернетических систем\scnsupergroupsign}
\scnaddlevel{1}
\scnsuperset{семантическая совместимость кибернетических систем\scnsupergroupsign}
\scnaddlevel{-1}
\scnsuperset{семантическая совместимость\scnsupergroupsign}

\scnheader{интеграция*}
\scnidtf{объединение нескольких разных сущностей, в результате чего возникает некоторая объединённая целостная сущность*}
\scnsuperset{эклектичная интеграция*}
\scnaddlevel{1}
\scnidtf{Интеграция разнородных (гетерогенных) сущностей, которой не предшествует конвергенция (сближение) этих сущностей*}
\scnaddlevel{-1}
\scnsuperset{глубокая интеграция*}
\scnnote{Понятие \textit{интеграции*} и особенно понятие \textit{глубокой интеграции*} имеет тесную связь с понятием \textit{конвергенции\scnsupergroupsign}. Чем выше степень конвергенции (степень сближения) интегрируемых объектов, тем выше качество результата интеграции. Особенно, если речь идёт о глубокой интеграции.}

\scnheader{глубокая интеграция*}
\scnidtf{"бесшовная"{} интеграция*}
%TODO ссылка на Грибову
\scnidtf{интеграция однородных сущностей, предполагающая глубокую взаимную "диффузию"{} (сращивание) соединяемых сущностей, которая не обязательно должна осуществляться физически}
\scnnote{Примером виртуальной глубокой интеграции является формирование коллектива \uline{семантический совместимых} индивидуальный кибернетических систем}
\scnidtf{бесшовная интеграция*}
\scnidtf{гибридизация*}
\scnidtf{интеграция, результатом которой являются гибридные объекты*}
\scnidtf{интеграция, которой предшествует высокий уровень конвергенции интегрируемых объектов*}
\scnidtf{(конвергенция + интеграция)*}
\scnidtf{"бесшовная"{} интеграция}
\scnidtf{интеграция, в результате которой возникает гибридная система*}
\scnidtf{интеграция, которой предшествует конвергенция (в частности, унификация) интегрируемых систем, приведение этих систем к максимально похожему виду (общему знаменателю)*}
%TODO сложно при чтении воспринимать, конвергенция и приведение как-то сливаются, становится не совсем понятно, к чему относится приведение к конвергенции или к интеграции, может как-то более явно указать, что конвергенция это то приведение?
\scnidtf{интеграция с "диффузией"{} , взаимопроникновением на основе унификации того, что можно сделать одинаковым*}

\scnheader{интеграция*}
\scnsuperset{интеграция информационных конструкций}
\scnsuperset{интеграция языков}
\scnsuperset{интеграция научных дисциплин}
\scnsuperset{интеграция баз знаний}
\scnsuperset{интеграция моделей решения задач}
\scnsuperset{интеграции индивидуальных кибернетических систем}
\scnaddlevel{1}
\scnsuperset{слияние индивидуальных кибернетических систем}
\scnaddlevel{1}
\scnidtf{преобразование нескольких \uline{искусственных} индивидуальных кибернетических систем в интегрированную индивидуальную кибернетическую систему, которая способна решать все задачи, каждая из которых могла бы быть решена в рамках какой-либо из интегрируемых систем}
\scnaddlevel{-1}
\scnsuperset{формирование коллектива индивидуальных кибернетических систем}
\scnaddlevel{1}
\scnidtf{формирования многоагентной системы, состоящей из индивидуальных кибернетических систем}
\scnaddlevel{-1}
\scnnote{Эффективность интеграции индивидуальных кибернетических систем определяется тем, насколько объем задач, решаемых коллективом индивидуальных кибернетических систем, превысит объединение объёмов задач, решаемых членами коллектива в отдельности.}
\scnaddlevel{-1}

\bigskip
\scnendstruct \scninlinesourcecommentpar{Завершили Сегмент "\textit{Текущее состояние и проблемы дальнейшего развития деятельности в области Искусственного интеллекта}"}

\scsection{Понятие ostis-сообщества}
\label{intro_ostis}

\begin{SCn}
\scnsegmentheader{\currentname}

\scnstartsubstruct

\scnheader{ostis-сообщество}
\scnidtf{Человеко-машинный симбиоз, представляющий собой коллектив, состоящий из людей и ostis-систем и обеспечивающий высокий уровень автоматизации определённого (соответствующего) вида человеческой деятельности.}
\filemodefalse
\scnaddlevel{1}
\scnnote{В состав каждого ostis-сообщества входит корпоративная ostis-система, которая в рамках этого ostis-сообщества выполняет: 
\begin{scnitemize}
\item роль координатора деятельности членов данного ostis-сообщества;
\item роль памяти ostis-сообщества, т.е. хранителя общих (обобществляемых, общедоступных) знаний для всех членов данного ostis-сообщества, которое несет ответственность за совершенствование этих знаний, а также для всех членов всех тех ostis-сообществ, в состав которых данное ostis-сообщество входит (указанные субъекты являются пользователями рассматриваемых общих знаний). Таким образом, корпоративная ostis-система некоторого ostis-сообщества является "официальным" представителем этого ostis-сообщества во всех ostis-сообществах, в состав которых входит, и, следовательно, является координатором деятельности даного ostis-сообщества (как единого целого) в рамках всех ostis-сообществ, в состав которых оно входит;
\end{scnitemize}
\scnaddlevel{-1}
}
\scnrelfromset{есть сходства}{ostis-сообщество; решатель задач ostis-системы}

\scnheader{есть сходства*}
\scnhaselementset{ostis-сообщество; решатель задач ostis-системы}
	\scnaddlevel{1}
	\scnexplanation{ ostis-сообщество\\
	\scnsuperset{многоагентная система,в которой управление агентами осуществляется через общую для них память}
	\scnsuperset{многоагентная система, с децентрализованным управлением агентами}
	\scnsuperset{многоагентная система, в которой областью деятельности её агентов является как внешняя среда, так и память этой системы}
	решатель задач ostis-системы\\
	\scnsuperset{многоагентная система,в которой управление агентами осуществляется через общую для них память}
	\scnsuperset{многоагентная система, с децентрализованным управлением агентами}
	\scnsuperset{многоагентная система, в которой областью деятельности её агентов является как внешняя среда, так и память этой системы}
	\scnsuperset{агентно-ориентированная модель обработки информации в памяти}}   
    \scnaddlevel{-1}

\scnrelfromlist{пример}{оркестр, играющий без дирижера или даже без композитора\\
	  \scnaddlevel{1}
	  \scntext{необходимое требование}{каждый участник оркестра должен иметь квалификацию дирижера или композитора};
	  \scnaddlevel{-1}
комплексная строительная бригада, работающая без прораба\\
	  \scnaddlevel{1}
	  \scntext{необходимое требование}{каждый участник строительной бригады должен иметь квалификацию прораба};
	  \scnaddlevel{-1}
научно-исследовательская лаборатория, работающая без заведующего и научного руководителя\\
	  \scnaddlevel{1}
	  \scntext{необходимое требование}{каждый участник научно-исследовательской лаборатории должен иметь квалификацию заведующего или научного руководителя};
	  \scnaddlevel{-1}
кафедра, работающая без заведующего и ученого секретаря\\
	  \scnaddlevel{1}
	  \scntext{необходимое требование}{каждый участник кафедры должен иметь квалификацию заведующего и ученого секретаря}
	  \scnaddlevel{-1}
}

\scnrelfromset{Специфика реализации наукоемких (непредсказуемых) проектов}{цель; мотивация общего результата; квалификция на уровне дирижера-композитора(прораба-архитектора); коммуникабельность; моральные принципы не как я красиво играю, а какая красивая музыка, которую мы вместе делаем}

\scnnote{Новый подход к наукоемкому project-менеджменту "каждый строитель" должен иметь квалификацию прораба (орекстр без дирижера, нот и указаний). Распределение ответственности, а не задач! Почему технология разработки, эксплуатации и совершенствования (реинжиниринг) нового поколения должна быть устроена на основе децентрализованного, не целенаправленного управления.}

\textbf{\textit{Необходимые факторы}}
\begin{scnenumerate}
\item \textnormal{высокая квалификация и человеческие качества (моральные) участников;}
\item \textnormal{согласованное формирование целей и подцелей;}
\item \textnormal{персонификация вкладов (самоконтроль);}
\item \textnormal{открытость.}
\end{scnenumerate}

/* Завершить рассмотрение понятия ostis-сообщества */

\end{SCn}


\scsection{Введение в описание внутреннего языка ostis-систем и близких ему внешних языков, используемых для представления исходных текстов баз знаний}

\begin{SCn}

\scnsectionheader{\currentname}

\end{SCn}


\scsubsection{Введение в описание внутреннего языка ostis-систем}
\label{intro_sc_code}

\begin{SCn}

\scnsectionheader{\currentname}

\scnstartsubstruct

\scnsegmentheader{Первый сегмент Введения описание внутреннего языка ostis-систем}
\scnstartsubstruct

\scnheader{\currentname}
\scnreltovector{конкатенация сегментов}{Первый сегмент Введения описание внутреннего языка ostis-систем;Описание Ядра SC-кода;Описание Расширения Ядра SC-кода}

\scnheader{SC-код}
\scnidtf{Внутренний язык ostis-систем}
\scnidtf{Множество sc-текстов}
\scnidtf{sc-текст}
\scnidtf{Множество sc-конструкций}
\scnidtf{Язык унифицированного смыслового представления знаний в памяти интеллектуальных компьютерных систем}
\filemodetrue
\scnrelfromvector{принципы, лежащие в основе}{Знаки (обозначения) всех сущностей, описываемых в \textit{sc-текстах} (текстах SC-кода) представляются в виде синтаксически элементарных (атомарных) фрагментов \textit{sc-текстов} и, следовательно, не имеющих внутренней структуры, не состоящих из более простых фрагментов текста, как, например, имена (термины), которые представляют знаки описываемых сущностей в привычных языках и состоят из букв.;Имена (термины), естественно-языковые тексты и другие информационные конструкции, не являющиеся \textit{sc-текстами}, могут входить в состав \textit{sc-текста}, но только как файлы, описываемые (специфицируемые) \textit{sc-текстами}. Таким образом, в состав базы знаний \textit{интеллектуальной компьютерной системы}, построенной на основе \textit{SC-кода}, могут входить имена (термины), обозначающие некоторые описываемые сущности и представленные соответствующими файлами. Каждый sc-элемент будем называть внутренним обозначением некоторой сущности, а имя этой сущности, представленное соответствующим файлом, будем называть внешним идентификатором (внешним обозначением) этой сущности. При этом каждый именуемый (идентифицируемый) \textit{sc-элемент} связывается дугой, принадлежащей отношению "\textit{\textbf{быть внешним идентификатором*}}, с узлом, содержимым которого является файл идентификатора (в частности, имени), обозначающего ту же сущность, что и указанный выше \textit{sc-элемент}. Внешним обозначением может быть не только имя (термин), но и иероглиф, пиктограмма, озвученное имя, жест. Особо отметим, что внешние обозначения описываемых сущностей в интеллектуальной компьютерной системе, построенной на основе \textit{SC-кода}, используются только (1) для анализа информации, поступающей в эту систему из вне из различных источников, и ввода (понимания и погружения) этой информации в базу знаний, а также (2) для синтеза различных сообщений, адресуемых различным субъектам (в т.ч. пользователям).;Тексты \textit{SC-кода} (sc-тексты) имеют в общем случае нелинейную (графовую) структуру, поскольку (1) знак каждой описываемой сущности в ходит в состав sc-текста однократно и (2) каждый такой знак может быть инцидентен неограниченному числу других знаков, поскольку каждая описываемая сущность может быть связана неограниченным числом связей с другими описываемыми сущностями.;
База знаний, представленная текстом \textit{SC-кода}, является графовой структурой специального вида, алфавит элементов которой включает в себя множество узлов, множество ребер, множество дуг, множество базовых дуг -- дуг специально выделенного типа, обеспечивающих структуризацию баз знаний, а также множество специальных узлов, каждый из которых имеет содержимое, являющееся файлом, хранящимся в памяти интеллектуальной компьютерной системы. Структурная особенность данной графовой структуры заключается в том, что ее дуги и ребра могут связывать не только узел с узлом, но и узел с ребром или дугой, ребро или дугу с другим ребром или дугой.;
\uline{Все элементы} указанной выше графовой структуры (текста SC-кода), т.е. все ее узлы, ребра и дуги являются обозначениями различных сущностей. При этом ребро является обозначением бинарной неориентированной связки между двумя сущностями, каждая из которых либо представлена в рассматриваемой графовой структуре соответствующим знаком, либо является самим этим знаком. Дуга является обозначением бинарной ориентированной связки между двумя сущностями. Дуга специального вида (\textit{\textbf{базовая дуга}}) является знаком связи между узлом, обозначающим некоторое множество элементов рассматриваемой графовой структуры, и одним из элементов этой графовой структуры, который принадлежит указанному множеству. Узел, имеющий содержимое (узел, для которого содержимое существует, но может в текущий момент быть неизвестным) является знаком файла, который является содержимым этого узла. Узел, не являющийся знаком файла, может обозначать какой-либо материальный объект, первичный абстрактный объект(например, число, точку в некотором абстрактном пространстве), какую-либо бинарную связь, какое-либо множество (в частности, понятие, структуру, ситуацию, событие, процесс). При этом сущности, обозначаемые элементами рассматриваемой графовой структуры, могут быть постоянными (существующими всегда) и временными (сущностями, которым соответствует отрезок времени их существования). Кроме того, сущности, обозначаемые элементами рассматриваемой графовой структуры, могут быть константными (конкретными) сущностями и переменными (произвольными) сущностями. Каждому элементу рассматриваемой графовой структуры, являющемуся обозначением переменной сущности, ставится в соответствие область возможных значений этого обозначения. Область возможных значений каждого переменного ребра является подмножеством множества всевозможных константных ребер, область возможных значений каждой переменной дуги является подмножеством множества всевозможных константных дуг, область возможных значений каждого переменного узла является подмножеством множества всевозможных константных узлов.;
В рассматриваемой графовой структуре, являющейся представлением базы знаний в SC-коде, могут, но не должны существовать разные элементы графовой структуры, обозначающие одну и ту же сущность. Если пара таких элементов обнаруживается, то эти элементы склеиваются (отождествляются). Таким образом, синонимия внутренних обозначений в базе знаний интеллектуальной компьютерной системы, построенной на основе \textit{SC-кода,} запрещена. При этом синонимия внешних обозначений считается нормальным явлением. Формально это означает, что из некоторых элементов рассматриваемой графовой структуры выходит несколько дуг, принадлежащих отношению "\textit{\textbf{быть внешним идентификатором*}}". Из всех указанных дуг, принадлежащих отношению "\textit{\textbf{быть внешним идентификатором*}}" и выходящих из одного элемента рассматриваемой графовой структуры, обязательно выделяется одна (очень редко две) путем включения их в число дуг, принадлежащих отношению "\textit{\textbf{быть основным внешним идентификатором*}}". Это означает, что указываемый таким образом внешний идентификатор не является омонимичным, т.е. не может быть использован как внешний идентификатор, соответствующий другомуэлементу рассматриваемой графовой структуры.;
Кроме файлов, представляющих различные внешние обозначения (имена, иероглифы, пиктограммы), в памяти интеллектуальной компьютерной системе, построенной на основе \textit{SC-кода,} могут хранится файлы различных текстов (книг, статей, документов, примечаний, комментариев, пояснений, чертежей, рисунков, схем, фотографий, видео-материалов, аудио-материалов).;
\uline{Любую сущность}, требующую описания, можно обозначить в виде sc-элемента. Особо подчеркнем, что sc-элементы являются не просто обозначениями различных описываемых сущностей, а обозначениями, которые являются элементарными (атомарными) фрагментами знаковой конструкции, т.е. фрагментами, детализация структуры которых не требуется для "прочтения" и понимания этой знаковой конструкции.;
Текст \textit{\textbf{SC-кода}}, как и любая другая графовой структура, является абстрактным математическим объектом, не требующим детализации (уточнения) его кодирования в памяти компьютерной системы (например, в виде матрицы смежности, матрицы инцидентности, списковой структуры). Но такая детализация потребуется для технической реализации памяти, в которой хранятся и обрабатываются sc-тексты.;
Важнейшим дополнительным свойством \textit{\textbf{SC-кода}} является то,что он удобен не просто для внутреннего представления знаний в памяти интеллектуальной компьютерной системы, но также и для внутреннего представления информации в памяти компьютеров, специально предназначенных для интерпретации семантических моделей интеллектуальных компьютерных систем. Т.е., SC-код определяет синтаксические, семантические и функциональные принципы организации памяти компьютеров нового поколения, ориентированных на реализацию интеллектуальных компьютерных систем, -- принципы организации графодинамической ассоциативной семантической памяти.;
SC-код рассматривается нами как объединение нескольких его подъязыков, в число которых входит ядро SC-кода и его расширение, обеспечивающее ввод и вывод информации для ostis-системы на всевозможных внешних языках.
}
\filemodefalse
\scnaddlevel{1}
\scnsourcecomment{Завершили описание принципов SC-кода}
\scnaddlevel{-1}

\scnendstruct

\scnsegmentheader{Описание Ядра SC-кода}
\scnstartsubstruct

\scnheader{Ядро SC-кода}
\scnrelfrom{алфавит}{Алфавит Ядра SC-кода}
\scnaddlevel{1}
\scnhaselement{sc-узел}
\scnhaselement{sc-ребро}
 \scnaddlevel{1}
 \scnidtf{обозначение бинарной неориентированной связи между sc-элементами}
 \scnaddlevel{-1}
\scnhaselement{sc-дуга}
\scnaddlevel{1}
 \scnidtf{обозначение бинарной ориентированной связи между sc-элементами}
 \scnaddlevel{-1}
\scnhaselement{базовая sc-дуга}
 \scnaddlevel{1}
 \scnidtf{sc-дуга константной позитивной стационарной принадлежности}
 \scnidtf{знак константной позитивной стационарной пары принадлежности}
 \scnaddlevel{-1}
\scnnote{Подчеркнем, что с помощью указанных типов sc-элементов можно описать любые связи между sc-элементами, трактуя эти связи как множества связываемых sc-элементов и используя некоторые sc-узлы как знаки этих множеств.}
\scnaddlevel{-1}

\scnendstruct

\scnsegmentheader{Описание Расширения Ядра SC-кода}
\scnstartsubstruct

\scnheader{SC-код}
\scnidtf{Расширение Ядра SC-кода}
\scnidtf{Результат введения в Ядро SC-кода sc-узлов, имеющих содержимое и обозначающих файлы, хранимые в памяти ostis-системы}
\scnnote{Все файлы, представляющие собой электронные образы инородных для SC-кода информационных конструкций, можно представить в SC-кода с помощью графовых структур, в которых sc-элементы обозначают буквы текстов или пиксели изображений. Но такой вариант кодирования внешних для ostis-системы информационных конструкций не дает возможности непосредственно использовать накопленный человечеством арсенал электронных информационных ресурсов.}
\scnnote{Важнейшим видом файлов ostis-систем являются внешние идентификаторы sc-элементов (в частности, имена sc-элементов), представляющие sc-элементы в текстах внешних языков (в том числе, текстах SCs-кода и SCn-кода)} 
\scnnote{Результатом просмотренного расширения \textit{Ядра SC-кода} является расширение \textit{Алфавита Ядра SC-кода}}

\scnheader{SC-код}
\scnrelfrom{алфавит}{Алфавит SC-кода}
\scnaddlevel{1}
\scnhaselement{sc-узел}
\scnhaselement{sc-ребро}
\scnhaselement{sc-дуга}
\scnhaselement{базовая sc-дуга}
\scnhaselement{файл ostis-системы}
\scnaddlevel{-1}

\scnheader{файл ostis-системы}
\scnidtf{sc-узел с содержимым}
\scnidtf{sc-узел, имеющий содержимое}
\scnidtf{sc-узел, обозначающий файл, хранимый в памяти ostis-системы}
\scnidtf{знак файла ostis-системы}
\scnreltoset{разбиение}{ея-файл ostis-системы\\
\scnaddlevel{1}
\scnidtf{естественно-языковой файл ostis-системы}
\scnaddlevel{-1};файл ostis-системы, являющийся текстом формального языка\\
\scnaddlevel{1}
\scnsuperset{sc.g-файл ostis-системы}
\scnsuperset{sc.s-файл ostis-системы}
\scnsuperset{sc.n-файл ostis-системы}
\scnaddlevel{-1};файл ostis-системы, отражающий процесс изменения sc.g-текста;графический файл ostis-системы;файл ostis-системы, являющийся изображением;видео-файл ostis-системы;аудио-файл ostis-системы}
\scnreltoset{разбиение}{файл-экземпляр ostis-системы
\scnaddlevel{1}
\scnidtf{файл, являющийся конкретным электронным документом или электронным образом конкретной внешней информационной конструкции}
\scnaddlevel{-1};файл-класс ostis-системы
\scnaddlevel{1}
\scnidtf{файл, являющийся знаком множества всевозможных экземпляров (копий) этого файла}
\scnaddlevel{-1}
}

\scnheader{SC-код}
\scnrelfrom{синтаксис}{Cинтаксис SC-кода} 
\scnaddlevel{1}
\scnexplanation{\textit{\textbf{Синтаксис}} \textit{\textbf{SC-кода}} задается
\begin{scnitemize}
\item типологией (алфавитом) sc-элементов (атомарных фрагментов текстов sc-кода);
\item правилами соединения (инцидентности) sc-элементов (например, sc-элементы каких типов не могут быть инцидентными друг другу);
\item типологией конфигураций sc-элементов (связки, классы, структуры), связями между конфигурациями sc-элементов (в частности, теоретико-множественными)
\end{scnitemize}
}
\scnaddlevel{-1}
\scnrelfrom{денотационная семантика}{Денотационная семантика SC-кода} 
\scnaddlevel{1}
\scnexplanation{\textit{\textbf{Денотационная семантика}} \textit{\textbf{SC-кода}} задается
\begin{scnitemize}
\item
 семантической интерпретацией sc-элементов и их конфигураций;
\item
 семантической интерпретацией инцидентности sc-элементов;
\item
 иерархической системой предметных областей;
\item
 структурой используемых понятий в каждой предметной области (исследуемые классы объектов, исследуемые отношения, исследуемые классы объектов отношений из смежных предметных областей, ключевые экземпляры исследуемых классов объектов);
\item
 онтологиями предметных областей.
\end{scnitemize}
}
\scnaddlevel{-1}
\scnnote{Следует особо подчеркнуть, что  унификация и максимально возможное упрощение  \textbf{\textit{синтаксиса}} и \textbf{\textit{денотационной семантики}} внутреннего языка интеллектуальных компьютерных систем необходимы потому, что подавляющий объем \textbf{\textit{знаний}}, хранимых в составе  базы знаний интеллектуальной компьютерной системы, представляют собой \textbf{\textit{метазнания}}, описывающими свойства других знаний. Более того, по указанной причине конструктивное (формальное) развитие теории интеллектуальных компьютерных систем невозможно без уточнения (унификации, стандартизации) и обеспечения семантической совместимости различных видов знаний, хранимых в базе знаний интеллектуальной компьютерной  системы.  Очевидно, что многообразие форм представления семантически эквивалентных знаний делает разработку общей теории  интеллектуальных компьютерных систем практически невозможной. К \textit{метазнаниям}, в частности, следует отнести и различного вида логические высказывания и всевозможного вида программы, описания методов (навыков). Обеспечивающих решение различных классов информационных задач.}

\scnendstruct~
\scnsourcecomment{Завершили сегмент "Описание расширения Ядра SC-кода"}

\scnendstruct~
\scnsourcecomment{Завершили раздел "\currentname"}

\end{SCn}


\scsubsection{Неформальное введение в язык визуального представления баз знаний ostis-систем}

\begin{SCn}

\scnsectionheader{\currentname}

\end{SCn}


\scsubsection{Неформальное введение в язык гипертекстового представления баз знаний ostis-систем}

\begin{SCn}

\scnsectionheader{\currentname}

\end{SCn}

%\scchapter{Общие принципы оформления внутреннего и внешнего представления информационных конструкций в ostis-системах}
\label{intro_rules}

\scnsectionheader{\currentname}

\scnstartsubstruct
\begin{SCn}

\scnsegmentheader{Формальная типология информационных конструкций, хранимых в памяти ostis-системы}

\scnstartsubstruct
\scnheader{информационная конструкция}
\scnsubdividing{знак\\
\scnaddlevel{1}
\scnidtf{семантически и структурно элементарный фрагмент информационной конструкции}
\scnidtf{элементарные атомарная информационная конструкция}
\scnidtf{информационная конструкция, состоящая из одного знака}
\scnaddlevel{-1}
;знаковая конструкция\\
\scnaddlevel{1}
\scnidtf{информационная конструкция, состоящая из \uline{нескольких} знаков} 
\scnaddlevel{-1}
}
\scnrelfrom{покрытие}{язык}
\scnaddlevel{1}
\scnidtf{множество (класс) \textit{информационных конструкций},
\begin{scnitemize}
		\item которому ставится в соответствие
		\begin{scnitemizeii}
			\item семейство \uline{синтаксических} правил построения \textit{информационных конструкций}, принадлежащих этому \textit{множеству}, а также
			\item семейство \uline{семантических} правил построения \textit{информационных конструкций} этого \textit{множества}
		\end{scnitemizeii}
		\item и которому \textit{принадлежат} не только синтаксически и семантически правильно построенные (корректные) \textit{информационные конструкции}, но и также неправильно построенные (некорректные, ошибочные) \textit{информационные конструкции}, которые, тем не менее, дают возможность их семантически интерпретировать (понимать) и, по крайней мере, локализовать допущенные в них ошибки.
\end{scnitemize}}
\scnaddlevel{1}
\scnnote{Очень важно обеспечить эффективный обмен \textit{информацией} не только с \uline{грамотными} \textit{носителями языков}, но также и с не очень грамотными, которые допускают большое количество языковых ошибок.}
\scnaddlevel{-1}
\scnnote{Поскольку теоретически некоторые \textit{информационные конструкции} могут принадлежать одновременно нескольким \textit{языкам} и поскольку понятие синтаксической и семантической корректности \textit{информационных конструкций} определяются правилами соответствующих \textit{языков} (их \textit{синтаксисом} и \textit{денотационной семантикой}), для уточнения понятий структурного уровня и корректности (правильности) \textit{информационных конструкций} вводится ряд \textit{ролевых отношений}, связывающих различные \textit{языки} с принадлежащими им \textit{информационными конструкциями}.}
\scnaddlevel{-1}

\scnheader{знак\scnrolesign}
\scnidtf{быть \textit{знаком}, входящим в состав \textit{информационных конструкций} \uline{заданного} \textit{языка}\scnrolesign}

\scnheader{синтаксически выделяемый класс знаков\scnrolesign}
\scnidtf{быть синтаксические выделяемым \textit{классом знаков} в рамках заданного \textit{языка}\scnrolesign}
\scnhaselementlist{пример}{{\normalfont(}SC-код $\in$ sc-узел{\normalfont)};{\normalfont(}SC-код $\in$ sc-коннектор{\normalfont)}}
\scnsuperset{класс синтаксически эквивалентных знаков\scnrolesign}

\scnheader{класс синонимичных знаков\scnrolesign}
\scnidtf{быть множеством \textit{знаков} заданного \textit{языка}, обозначающих одно это уже описываемую \textit{сущность}\scnrolesign}

\scnheader{знаковая конструкция\scnrolesign}
\scnidtf{быть \textit{знаковой конструкцией} заданного \textit{языка}\scnrolesign}

\scnheader{текст\scnrolesign}
\scnidtf{быть синтаксически корректной \textit{информационной конструкцией} заданного \textit{языка}\scnrolesign}

\scnheader{синтаксически некорректная информационная конструкция\scnrolesign}
\scnidtf{быть синтаксически ошибочной \textit{информационной конструкцией} заданного \textit{языка}\scnrolesign}

\scnheader{семантически корректная информационная конструкция\scnrolesign}
\scnidtf{\textit{информационная конструкция} заданного \textit{языка}, не имеющая семантических ошибок\scnrolesign}

\scnheader{семантически некорректная информационная конструкция\scnrolesign}
\scnidtf{\textit{информационная конструкция} заданного \textit{языка}, имеющая семантические ошибки\scnrolesign}

\scnheader{знание\scnrolesign}
\scnidtf{\textit{текст} заданного \textit{языка}, не имеющий семантических ошибок и обладающий целостностью и ценностью\scnrolesign}

\scnheader{следует отличать*}
\scnhaselementset{текст\scnrolesign\\
\scnaddlevel{1}
\scniselement{ролевое отношение}
\scnaddlevel{-1}
;текст\\
\scnaddlevel{1}
\scnidtf{второй домен ролевого отношения \textit{текст}\scnrolesign}
\scnidtf{текст некоторого языка}
\scnaddlevel{-1}
;знание'\\
\scnaddlevel{1}
\scniselement{ролевое отношение}
\scnaddlevel{-1}
;знание\\
\scnaddlevel{1}
\scnidtf{второй домен ролевого отношения \textit{знание}\scnrolesign}
\scnaddlevel{-1}
;информационная конструкция\\
\scnaddlevel{1}
\scnsuperset{текст}
\scnsuperset{знание}
\scnaddlevel{-1}
}

\scnheader{sc-структура}
\scnidtf{sc-конструкция}
\scnidtf{информационная конструкция SC-кода}
\scnidtftext{часто используемые sc-идентификатор}{SC-код}
\scnidtf{Множество всевозможных sc-структур}
\scnsubset{информационная конструкция}
\scnsuperset{\scnkeyword{sc-текст}} 
\scnaddlevel{1}
\scnidtf{текст \textit{SC-кода}}
\scnidtf{\textit{информационная конструкция}, удовлетворяющая синтаксическим правилам \textit{SC-кода}}
\scnidtf{синтаксически корректная для \textit{SC-кода} \textit{информационная конструкция}}
\scnidtf{связное множество \textit{sc-элементов}, удовлетворяющее синтаксическим правилам \textit{SC-кода}}
\scnidtf{синтаксически правильно построеная \textit{информационная конструкция} \textit{SC-кода}}
\scnsubset{текст}
\scnsuperset{\scnkeyword{sc-знание}}
\scnaddlevel{1}
\scnidtf{знание, представленное в \textit{SC-коде}}
\scnidtf{\textit{sc-текст}, удовлетворяющий определенным семантическим правилам \textit{SC-кода} и, в частности, определённым правилам семантической полноты}
\scnidtf{семантически корректный и целостный \textit{sc-текст} }
\scnsubset{знание}
\scnaddlevel{-2}

\scnheader{sc-знание}
\scnidtf{формализованное знание, хранимое в \textit{памяти ostis-системы} и непосредственно используемое (понимаемое) её \textit{решателем задач}}
\scnidtf{\textit{sc-текст}, имеющий \uline{однозначную} семантическую интерпретацию в рамках \textit{SC-пространства} (формализованного смыслового пространства)}

\scnnote{Не каждый \textit{sc-текст} является \textit{знанием}. В отличие от \textit{sc-текстов} в знаниях важна \uline{семантическая} целостность (полнота) \textit{информационных конструкций}. Так, например, \textit{логическая формула} со свободными переменными не является \textit{знанием}. Но полный текст \textit{высказывания}, включающий в себя все компоненты всех логических связок (вплоть до \textit{атомарных логических формул}) \textit{знанием} является. Второй пример: \textit{sc-текст}, в состав которого входит \textit{sc-коннектор} или \textit{sc-связка}, но не входят все компоненты этого \textit{sc-коннектора} или \textit{sc-связки},\scnbigspace \textit{знанием} не является.}

\scnnote{Семантическая целостность \textit{знания ostis-системы} означает, во-первых, то, что такое \textit{знание} представляет собой \textit{информационную конструкцию}, являющуюся высказыванием, то есть информационную конструкцию, имеющую истинностное значение, которое может быть подтверждено или опровергнуто, например, экспертом (рецензентом) \textit{базы знаний ostis-системы}. Во-вторых, \textit{знание ostis-системы} должно содержать достаточно полную и \uline{однозначную} спецификацию по возможности всех входящих в него неидентифицированных (неименованных) \textit{sc-элементов}. Это необходимо для того, чтобы внешнее представление знания ostis-системы можно было по возможности однозначно погрузить ("вставить") в \textit{базу знаний} \scnbigspace \textit{ostis-системы}.}

\scnheader{информация, представленная в памяти ostis-системы}
\scnidtf{\textit{информационная конструкция}, хранимая в памяти \textit{ostis-системы}}
\scnsubdividing{sc-структура\\
\scnaddlevel{1}
\scnidtf{информация, представленная в виде конструкции \textit{SC-кода}}
\scnaddlevel{-1}
;файл ostis-системы}
\scnexplanation{В \textit{ostis-системе} используются две формы представления информации в памяти системы --
\begin{scnitemize}
	\item полностью формализованное представление, понятное для решателя задач \textit{ostis-системы} (конструкции \textit{SC-кода});
	\item инородное для \textit{SC-кода} представление, используемое исключительно для коммуникации \textit{ostis-систем} с внешними субъектами (файлы \textit{ostis-систем}).
\end{scnitemize}}

\scnheader{файл ostis-системы}
\scnidtf{файл, хранимый в sc-памяти}
\scnidtf{инородная для \textit{SC-кода} \textit{информационная конструкция}, хранимая в памяти \textit{ostis-системы} в виде содержимого sc-узла, обозначающего эту \textit{информационную конструкцию}}
\scnidtf{хранимая в sc-памяти "электронная"{} форма представления \textit{информационной конструкции}}

\bigskip

\scnendstruct \scnendsegmentcomment{Формальная типология информационных конструкций, хранимых в памяти ostis-системы}

\end{SCn}
\begin{SCn}
\scnsuperset{фрагмент знака, представленный файлом ostis-системы}
\scnaddlevel{1}
\scnsuperset{синтаксически элементарный фрагмент информационной конструкции, представленный файлом ostis-системы}
\scnaddlevel{-1}

\scnsuperset{знак, представленный файлом ostis-системы}
\scnaddlevel{1}
\scnsuperset{sc-идентификатор, представленный файлом ostis-системы}
\scnaddlevel{-1}

\scnsuperset{\textbf{знаковая конструкция, представленная файлом ostis-системы}}
\scnaddlevel{1}
\scnsuperset{текст, представленный файлом ostis-системы}
    \scnaddlevel{1}
    \scnsuperset{знание, представленное файлом ostis-системы}
    \scnaddlevel{-1}
\scnaddlevel{-1}

\scnheader{знаковая конструкция, представленная файлом ostis-системы}
\scnsuperset{sc.s-файл ostis-системы}
\scnaddlevel{1}
\scnidtf{конструкция SCs-кода, хранимая в памяти ostis-системы в виде содержимого некоторого SC-узла}
\scnidtf{файл ostis-системы, являющийся sc.s-конструкцией}
\scnaddlevel{-1}
\scnsuperset{sc-идентификатор, представленный файлом ostis-системы}
\scnaddlevel{1}
\scnidtf{файл ostis-системы, являющийся sc-идентификатором}
\scnidtf{(sc-идентификатор $\cap$ файл ostis-системы)}
\scnreltoset{пересечение}{sc-идентификатор;файл ostis-системы}
\scnaddlevel{-1}
\scnsuperset{sc.g-файл ostis-системы}
\scnaddlevel{1}
\scnreltoset{пересечение}{sc.g-конструкция;файл ostis-системы}
\scnaddlevel{-1}
\scnsuperset{sc.n-файл ostis-системы}
\scnaddlevel{1}
\scnreltoset{пересечение}{sc.n-конструкция;файл ostis-системы}
\scnaddlevel{-1}

\scnsuperset{ея-файл ostis-системы}
\scnaddlevel{1}
\scnreltoset{пересечение}{ея-конструкция\\
    \scnaddlevel{1}
    \scnidtf{естественно-языковая конструкция}
    \scnidtf{конструкция естественного языка}
    \scnsuperset{ея-текст}
    \scnaddlevel{-1}
;файл ostis-системы}
\scnidtf{конструкция естественного языка, представленная в виде файла ostis-системы}
\scnsubset{Русский язык}
    \scnaddlevel{1}
    \scnidtf{конструкция Русского языка}
    \scnaddlevel{-1}
\scnsubset{Английский язык}
\scnidtf{естественно-языковая конструкция, являющаяся содержимм sc-узла, обозначающего эту конструкцию}
\scnaddlevel{-1}

\scnheader{файл ostis-системы}
\scnsubdividing{файл ostis-системы, предполагающий одномерную визуализацию хранимой информации
;файл ostis-системы, предполагающий двухмерную визуализацию хранимой информации
;файл ostis-системы, предполагающий трехмерную визуализацию хранимой информации}
\scnsubdividing{файл ostis-системы, представляющий статистическую информацию
;файл ostis-системы, представляющий динамическую информацию\\
\scnaddlevel{1}
\scnsuperset{видео-файл ostis-системы}
\scnsuperset{аудио-файл ostis-системы}
\scnaddlevel{-1}}
\scnsubdividing{файл-экземпляр ostis-системы\\
\scnaddlevel{1}
\scnidtf{\textit{sc-узел}, обозначающий конкретное вхождение \textit{информационной конструкции}, структура которой представлена содержимым этого \textit{sc-узла}}
\scnaddlevel{-1}
;файл-образец ostis-системы\\
\scnaddlevel{1}
\scnidtf{класс \textit{синтаксически эквивалентных} \textit{файлов-экземпляров} \textit{ostis-системы}}
\scnidtf{множество всевозможных \textit{файлов-экземпляров ostis-системы}, которые являются \textit{синтаксически эквивалентными} копиями содержимого заданного \textit{sc-узла}}
\scnaddlevel{-1}
}
\scnsubdividing{сформированный файл ostis-системы\\
\scnaddlevel{1}
\scnidtf{\textit{файл ostis-системы}, представленный \textit{sc-узлом}, имеющим сформированное (полностью построенное) содержимое}
\scnaddlevel{-1}
;несформированный файл ostis-системы\\
\scnaddlevel{1}
\scnidtf{\textit{файл ostis-системы}, представленный \textit{sc-узлом}, содержимое которого либо полностью отсутствует, либо сформировано \uline{частично}}
\scnaddlevel{-1}}
\scnnote{\textit{файл ostis-системы} можеть быть \uline{электронной копией} и знаком (!) внешней \textit{информационной конструкции}, которая может быть:
\begin{scnitemize}
\item общедоступный в сети Internet информационным ресурсом;
\item документом, опублиеованным на бумажном носители в виде какой-либо книги или статьи.
\end{scnitemize}
Кроме того , \textit{файл ostis-системы} может быть просто \uline{обозначением} указанной внешней \textit{информационной конструкции} (т.е. может быть \textit{sc-узлом}, обозначающим \textit{внешнюю информационную конструкцию}, но не имеющим содержимого). Такой \textit{sc-узел} используется формальной спецификации (средствами \textit{SC-кода}) соответствующего обозначаемого им информационного ресурса.
}
\scnheader{ея-файл ostis-системы}
\scnidtf{естественно-языковой \textit{файл ostis-системы}}
\scnexplanation{\textit{файл ostis-системы}, представляющий собой \textit{sc-узел}, содержимым которого является "электронная"{} форма \textit{информационной конструкции} (чаще всего, \textit{текста}) одного из \textit{естественных языков}}
\scnidtf{структурно выделенный ея-текст, хранимый в памяти ostis-системы (в sc-памяти)}
\scnrelfromvector{правила оформления}{
\scnfileitem{При оформлении текстов в естественно-языковых файлах (ея-файлах) \textit{ostis-систем} используются обычные разделители (точки в аббревиатурах и между предложениями\char59 круглые скобки, запятые, пробелы\char59), а также специальный символ $\square$, используемый \textit{в ея-файлах ostis-систем} как разделитель и целый ряд  следующих ограничителей, позволяющих выделять некоторые фрагменты ея-текстов:
\begin{scnitemize}
\item подчеркивание выделяет логически важные фрагменты в предложениях\char59
\item цитатные кавычки являются ограничителем цитат\char59
\item прямые кавычки ограничивают иносказательные термины, метафоры\char59
\item жирным курсивом стандартного размера выделяются идентификаторы \textit{sc-элементов} базы знаний, являющиеся ключевыми для заданного контекста\char59
\item жирным курсивом стандартного размера выделяются также условные внешние идентификаторы (обозначения) условно вводимых \textit{sc-элементов}, например, условные обозначения \textit{sc-элементов} произвольного структурного типа ($\bm{e_i}$, $\bm{e_j}$, ...), условные обозначения \textit{sc-узлов} ($\bm{v_i}$, $\bm{v_j}$, ...), условные обозначения \textit{sc-коннекторов} ($\bm{c_i}$, $\bm{c_j}$, ...), \textit{sc-дуг}, \textit{sc-связок}, не являющихся \textit{sc-коннекторами}, \textit{sc-структур} и так далее\char59
\item нежирным курсивом стандартного размера выделяются \textit{sc-идентификаторы} \textit{sc-элементов} \textit{базы знаний}, не являющиеся ключевыми для данного ея-текста.
\end{scnitemize}}
;\scnfileitem{Если в ея-тексте идентификаторы sc-элементов базы знаний (чаще всего -- \textit{простые sc-идентификаторы}), выделенные курсивом одинаковой жирности, следуют друг за другом, то число пробелов между разными идентификаторами должно быть увеличено. Если при этом выделенный sc-идентификатор включает в себя другие sc-идентификаторы, на которые желательно отдельно сослаться, то такая ссылка оформляется в виде дполнительной фразы типа "Смотрите также ..."{}}
;\scnfileitem{Текст \textit{ея-файла ostis-системы} может иметь \uline{любые} вставки не являющиеся естественно-языковыми текстами, в том числе, и фрагменты, являющиеся формальными внешними текстами представления знаний для ostis-систем (текстами SCg-кода, SCs-кода, SCn-кода). При этом указанные формальные фрагменты (вставки) могут быть как транслируемыми на внутренний язык ostis-системы (SC-код) и погружаемыми в состав ее базы знаний (т.е. фактически являться sc-текстами), так и нетранслируемыми формальными фрагментами, которые входят в состав базы знаний ostis-системы в виде содержимого соответствующих файлов. Все указанные выше "вставки" в ея-файл ostis-системы оформляются как ссылки на соответствующие sc-тексты или файлы ostis-системы. Каждая такая ссылка представляет собой \textit{sc-идентификатор} соответствующего sc-текста или файла ostis-системы и выделяется в ея-файле жирным курсивом стандартного размера со стандартным расстоянием между символами.
Таким образом, если в естественно-языковой \textit{файл ostis-системы}, необходимо "вставить" информационную конструкцию иного рода (\textit{sc.g-текст}, рисунок, таблицу, изображение), то (1) указанная конструкция оформляется как отдельный файл (2) которому приписывается имя (название), построенное по установленным правилам, и (3) на который в указанном естественно-языковом файле делается ссылка.
В естественно-языковых \textit{файлах ostis-систем} можно делать ссылки не только на другие \textit{файлы ostis-системы}, но и на \uline{именуемые} (!) фрагменты базы знаний, которые во внешнем представлении базы знаний оформляются в виде именуемых (идентифицируемых) sc.n-контуров.
Файлы и sc-тексты, на которые делаются ссылки из ея-файла, во внешнем представлении (при визуализации) базы знаний размещаются после указанного ея-файла в порядке первого их упоминания в этом ея-файле, если, конечно, на эти файлы или sc-тексты не было ссылок из ранее представленных ея-файлов.}
;\scnfileitem{В состав \textit{ея-файла ostis-системы} могут входить ссылки на любые идентифицированные (именованные) \textit{информационные конструкции} (Internet-ресурсы, библиографические источники, различные документы). Для этого достаточно указывать соответствующие \textit{sc-идентификаторы}.}
;\scnfileitem{В ея-текстах \uline{все} sc-основные идентификаторы описываемых в базе знаний сущностей должны быть выделены жирным или нежирным курсивом и могут быть представлены в любом склонении и спряжении.}
;\scnfileitem{В ея-текстах используются только основные идентификаторы (термины). Используемые синонимы явно указываются как неосновные идентификаторы.}
;\scnfileitem{Основная часть (содержимого текста) \textit{ея-файла ostis-системы} оформляется стандартным печатным шрифтом.}
}

\scnsourcecomment{Завершили перечень правил оформления содержимого \textit{ея-файлов ostis-систем}}

\scnheader{ея-файл ostis-системы}
\scnnote{Выделенные курсивом в \textit{ея-файле ostis-системы} \textit{sc-идентификаторы sc-элементов} могут являться \uline{ар\-гу\-мен\-та\-ми} различного вида \uline{запросов} к \textit{базе знаний ostis-системы} и, в первую очередь запросов типа "Что это такое"{}, предполагающих выделение из \uline{текущего} состояния \textit{базы знаний ostis-системы} семантической окрестности (спецификации) указываемого \textit{sc-элемента}, содержащей основную информацию о сущности, обозначаемой \textit{этим sc-элементом}.}


\end{SCn}
\begin{SCn}
	
\scnsegmentheader{Структуризация баз знаний ostis-систем}

\scnstartsubstruct

\scnsubset{сегмент базы знаний}
\scnidtf{Структурная типология знаний ostis-системы}
\scntext{введение}{\textit{База знаний ostis-системы} имеет достаточно развитую иерархическую структуру. База знаний делится на разделы. Разделы бывают атомарными и неатомарными. Неатомарный раздел состоит из сегментов. Атомарные разделы не имеют сегментов. Разделы \textit{базы знаний ostis-системы} могут иметь самое различное назначение. Так, например, \textit{База знаний Метасистемы IMS.ostis} включает в себя:
\begin{scnitemize}
\item \textit{раздел}, содержащий текущее состояние постоянно пополняемого и совершенствуемого \textit{Стандарта} Технологии OSTIS;
\item \textit{раздел}, посвящённый описанию \textit{конечных пользователей и разработчиков} \textit{Метасистемы IMS.ostis}; 
\item \textit{раздел}, посвящённый описанию \textit{история эксплуатации  Метасистемы IMS.ostis};
\item \textit{раздел}, посвящённый описанию \textit{истории эволюции  Метасистемы IMS.ostis} (в т.ч. истории эволюции и её \textit{база знаний}); 
\item \textit{раздел} посвящённый описанию \textit{интеллектуальных компьютерных систем}, разработанных(порождённых) с помощью \textit{Метасистемы IMS.ostis}.
\end{scnitemize}}

\scnheader{выделенный фрагмент базы знаний}
\scnidtf{фрагмент базы знаний, для которого в \textit{базе знаний} вводится знак, обозначающий этот фрагмент, т.е. являющийся знаком множества \uline{всех} знаков, входящих в состав этого фрагмента. При представлении фрагмента базы знаний на внешних языках (SCg-коде, SCn-коде) указанный знак выделенного фрагмента базы знаний представляется либо в виде sc.g-контура, либо в виде пары фигурных скобок, ограничивающих текст обозначаемого фрагмента базы знаний}
\scnidtf{выделенный фрагмент базы знаний}
\scnaddlevel{1}
\scniselement{сокращённые sc-идентификатор} 
\scnaddlevel{-1}
\scnidtf{явно структурно выделенный фрагмент базы знаний}
\scnnote{явное выделение фрагмента базы знаний осуществляется:
\begin{scnitemize}
\item в \textit{SC-коде} путем введения \textit{знака}, обозначающего \textit{множество} \uline{всех} знаков, входящих в состав \textit{выделенного фрагмента базы знаний};
\item в \textit{SCg-коде} с помощью \textit{sc.g-контура}, ограничивающего \textit{sc.g-представление} \scnbigspace \textit{выделенного фрагмента базы знаний};
\item в \textit{SCn-коде} с помощью пары фигурных скобок, ограничивающих \textit{sc.n-представление} \scnbigspace \textit{выделенного фрагмента базы знаний}.
\end{scnitemize}}
\scnsubdividing{семейство разделов базы знаний
\scnaddlevel{1}
\scnidtf{семантически целостное множество разделов базы знаний, имеющих достаточно сильные семантические связи между собой}
\scnaddlevel{-1}
;раздел базы знаний
\scnaddlevel{1}
\scnidtf{Основной (по семантической значимости) вид выделенных фрагментов баз знаний}\\
\scnidtf{раздел базы знаний ostis-системы}
\scnidtf{модуль (блок) базы знаний}
\newpage
\scnnote{В общем случае многим \textit{разделам базы знаний} ставятся в соответствие такие тексты, как \uline{предисловие},  \uline{введение}, \uline{заключение}, \uline{аннотация}, \uline{оглавление}, упражнения.\\ Некоторые из этих текстов могут иметь статус разделов.}
\scnsubdividing{неатомарный раздел базы знаний\\
\scnaddlevel{1}
\scnidtf{раздел базы знаний, состоящий из сегментов, декомпозируемый на сегменты} 
\scnnote{В общем случае \textit{неатомарный раздел базы знаний} может иметь неограниченное число \textit{сегментов}. \textit{Сегменты базы знаний} не могут состоять из других сегментов (подсегментов). В этом смысле \textit{сегменты базы знаний} имеют атомарный характер.}
\scnaddlevel{-1}
;атомарный раздел базы знаний 
\scnaddlevel{1}
\scnidtf{раздел базы знаний, не содержащий сегментов}
\scnaddlevel{-1}}
\scnaddlevel{-1}
;сегмент базы знаний
\scnaddlevel{1}
\scnidtf{сегмент раздела базы знаний}
\scnidtf{сегмент базы знаний ostis-системы}
\scnidtf{структурно выделяемое sc-знание ostis-системы, структурный уровень которого ниже уровня разделов базы знаний}
\scnnote{Сегменты базы знаний не могут иметь иерархической структуры, т.е. не могут состоять из сегментов более низкого структурного уровня.}
\scnnote{Сегменты базы знаний входят в состав неатомарных разделов базы знаний.}
\scnaddlevel{-1}
;выделенный фрагмент сегмента или атомарного раздела базы знаний\\
\scnaddlevel{1}
\scnsubdividing{выделенный фрагмент атомарного раздела базы знаний;выделенный фрагмент сегмента базы знаний}
\scnsubset{sc-знание}
\scnnote{Каждый \textit{выделенный фрагмент базы знаний} представляет собой структурно оформленное (структурно выделенное) \textit{знание}, хранимое в \textit{базе знаний} \textit{интеллектуальной компьютерной системы}, (точнее, в \textit{базе знаний ostis-системы}) и представленное в формализованном виде на \textit{внутреннем языке представления знаний} (в \textit{SC-коде}) Представленное таким образом \textit{знание} будем называть \textit{sc-знанием}.}
\scnaddlevel{1}
\scnidtfexp{
	\textit{информационная конструкция}, которая:
	\begin{scnitemize}
		\item принадлежит \textit{SC-коду};
		\item является синтактически корректный;
		\item обладает семантической целостностью -- отсутствием \textit{информационных дыр} ("недомолвок"{}), препятствующих её пониманию;
		\item имеет нетривиальный \textit{объём информации} -- количество описываемых сущностей (в том числе, \textit{связей} и \textit{классов});
		\item имеет достаточно высокое качество по другим характеристикам, в частности, достаточно высокую ценность.
\end{scnitemize}}
\scnnote{Типология \textit{sc-знаний} по объему представленной информации определённым образом коррелирует с типологией \textit{выделенных фрагментов баз знаний} -- объём информации, содержащейся в \textit{разделах баз знаний}, должен быть приблизительно одинаковым, объём, содержащейся в разделе базы знаний, должен быть ниже объёма информации, содержащегося в любом семействе разделов базы знаний, и должен быть выше объёма информации, содержащегося в любом \textit{сегменте базы знаний}. При этом в процессе эволюции базы знаний \textit{раздел базы знаний} может преобразоваться в семейства разделов, а \textit{сегмент базы знаний} может преобразоваться в \textit{раздел базы знаний}.}
\scnaddlevel{-2}}

\scnheader{раздел базы знаний}
\scnnote{Различные \textit{предметные области}, различные \textit{онтологии}, а также предметные области, объединённые с соответствующими им онтологиями, являются важнейшими видами \textit{знаний ostis-систем}, обеспечивающими логически стройную систематизацию \textit{sc-знаний ostis-систем} и, соответственно семантическую структуризацию \textit{баз знаний}. При этом указанные типы знаний обычно представляются в виде \textit{разделов базы знаний}, иерархия которых, задаваемая Отношением \textit{частная предметная область*} или Отношением \textit{частная предметная область и онтология*}, соответствует логико-семантической иерархии \textit{предметных областей}. Но, кроме \textit{разделов базы знаний}, являющихся \textit{предметными областями}, \textit{онтологиями}, \textit{предметными областями}, объединенными с соответствующими им онтологиями, вводится и целый ряд других семантических типов \textit{разделов базы знаний}, определяемых характером соотношения разделов базы знаний с используемыми (рассматриваемыми) предметными областями и онтологиями.}
\scnsuperset{предметная область}
\scnsuperset{фрагмент предметной области}
\scnsuperset{интегрированная онтология}
\scnsuperset{частная онтология}
\scnsuperset{предметная область и онтология}
\scnaddlevel{1}
\scnidtf{предметная область, объединенная (интегрированная) с соответствующей ей онтологией}
\scnidtf{предметная область вместе с онтологией, которая её специфицирует}
\scnaddlevel{-1}
\scnnote{Если каждому \textit{разделу базы знаний} ostis-системы будет четко соответствовать его семантический тип, то к "синтаксическим"{} связям между \textit{разделами базы знаний} добавится большое количество "осмысленных"{} (семантические интерпретируемых) связей, определяющих "семантическое местоположение"{} ("семантические координаты"{}) каждого \textit{раздела базы знаний} во множестве всех разделов, входящих в состав базы знаний ostis-системы.}
\scnnote{В основе представления \textit{базы знаний ostis-системы} лежат развитые средства семантической структуризации баз знаний и семантической систематизации баз знаний \textit{ostis-систем}. Можно выделить следующие уровни систематизации элементов и фрагментов смыслового пространства, построенного на основе \textit{SC-кода}:
\begin{scnitemize}
\item уровень знаков всевозможных сущностей (уровень \textit{sc-элементов});
\item уровень вводимых \textit{понятий}, обозначающих ключевые (исследуемые) в предметных областях классы сущностей;
\item уровень \textit{высказываний}, описывающих закономерности (свойства) экземпляров исследуемых классов сущностей (исследуемых понятий);
\item уровень \textit{предметных областей}, \textit{онтологий} и разделов, семантический тип которых известен.
\end{scnitemize}}

\scnidtf{раздел б.з.}
\scnaddlevel{1}
\scniselement{сокращенный sc-идентификатор}
\scnaddlevel{-1}
\scnexplanation{Множество \textit{разделов баз знаний} имеет:
	\begin{scnitemize}
		\item богатую семантическую типологию;
		\item богатый набор отношений, описывающих семантические связи между разделами.
	\end{scnitemize}
	Синтаксически \textit{разделы баз знаний} могут пересекаться (иметь общие элементы), но никогда один раздел не может включаться (полностью входить в состав) другого раздела. В этом смысле понятие подраздела (точнее, \textit{частного раздела}*) имеет не "синтаксический"{} смысл, а семантический -- глубокое наследование свойств при достаточно большой степени "независимости"{} друг от друга}
\scnnote{Важным свойством \textit{раздела базы знаний} \scnbigspace \textit{ostis-системы} является его семантическая целостность -- наличие достаточно стабильного набора классов исследуемых сущностей (\textit{объектов исследования}) и достаточно стабильного семейства \textit{отношений} и семейства \textit{параметров}, заданных на различных классах объектов исследования, а также семейства \textit{классов структур} использующих указанные выше понятия (введенные \textit{классы объектов исследования}, введенные \textit{отношения} и \textit{классы структур}). Такая целостность дает возможность развивать \textit{раздел базы знаний}, не выходя "за рамки"{} используемой системы \textit{понятий}. Это позволяет развивать каждый \textit{раздел базы знаний} в известной мере \uline{независимо} от других разделов, что существенно повышает \uline{гибкость} и \uline{стратифицированность} \textit{базы знаний}.}
\newpage
\scnexplanation{Основными свойствами \textit{раздела базы знаний} как структурно \textit{выделяемого фрагмента базы знаний} являются следующие:
	\begin{scnitemize}
		\item семантическая целостность -- наличие четкого критерия, позволяющего установить для каждого конкретного знания то, включать или не включать это знание в состав данного раздела;
		\item потенциальная возможность эволюционировать в достаточной степени независимо от других разделов, но при условии соблюдения всех требований, обеспечивающих постоянную поддержку \uline{семантической совместимости} данного раздела со всеми остальными семантически смежными разделами.
\end{scnitemize}}
\scnaddlevel{1}
\scntext{следовательно}{Грамотная декомпозиция \textit{базы знаний} на разделы, основанная на четкой стратификации процесса эволюции накапливаемой человечеством общечеловеческой объединенной \textit{базы знаний}, сутью которой является \uline{минимизация} трудоемкости усилий по согласованию и обеспечению с авторами других разделов \textit{семантической совместимости} со смежными разделами, создает предпосылки высоких темпов эволюции \textit{базы знаний} в целом.}
\scnaddlevel{-1}
\scnnote{Любой \textit{раздел базы знаний} не является структурной частью другого раздела. Каждый раздел самодостаточен и целостен. Это обеспечивается тем, что в состав \textit{титульной спецификации} каждого раздела входит \textit{семантическая окрестность},описывающая связи специфицируемого раздела \uline{со всеми} семантически близкими ему разделами.\\
	При этом разные разделы могут иметь разный семантический тип:
	\begin{scnitemize}
		\item раздел может быть предметной областью, интегрированной со всеми ее онтологиями;
		\item раздел может быть просто предметной областью;
		\item раздел может быть какой-либо онтологией чего угодно (не обязательно предметной области);
		\item и т.д.
\end{scnitemize}}
\scnnote{Если в \textit{титульную спецификацию} каждого раздела будет входить семантическая спецификация каждого раздела, включающая его семантические связи со всеми семантически близкими разделами, то последовательность (порядок) разделов в "линейном"{} исходном тексте, публикуемом в качестве очередной версии Стандарта OSTIS, может быть в достаточной степени \uline{произвольной}.}


\scnheader{семейство разделов базы знаний}
\scnidtf{множество семантически связанных друг с другом разделов базы знаний}
\scnidtf{кластер разделов базы знаний}
\scnnote{Семантические связи между разделами, входящими в состав семейства разделов, представляются в рамках \textit{титульных спецификаций разделов}, каждая из которых является специальной частью соответствующего (специфицируемого) раздела, входящего в состав семейства разделов.
	
	Для каждого \textit{раздела базы знаний} в рамках его титульной спецификации формируется \textit{семантическая окрестность} его связей со всеми семантически близкими ему разделами (и, прежде всего, с теми разделами, которые входят в состав тех семейств разделов, в которые входит заданный раздел). При этом акцентируется внимание именно на семантических связях между разделами. Так, например, вместо структурной ("синтаксической"{}) связи ``быть подразделом*''{} (т.е. быть частью заданного раздела) вводится связь ``быть \textit{дочерним разделом}*''{}. \\
	Данная связь указывает направление наследования свойств исследуемых объектов заданного раздела от разделов, исследующих более общие классы объектов.
	Порядок (последовательность) разделов в рамках \textit{семейства разделов базы знаний} при наличии \uline{явно представленных} семантических связей между разделами, входящими в семейство разделов, может быть достаточно \uline{произвольным}, что очень важно, например, при формировании оглавления очередной издаваемой версии \textit{Стандарта OSTIS}. Таким образом, трактовка \textit{Стандарта OSTIS}, а также всех издаваемых версий этого Стандарта как \textit{семейства разделов базы знаний} \scnbigskip \textit{Метасистемы IMS.ostis} обеспечивает высокий уровень гибкости \textit{Стандарта OSTIS}, а также легкость "переиздаваемости"{} его версий.}


\scnheader{сегмент или атомарный раздел базы знаний}
\scnnote{Простейшей формой \textit{сегмента} или \textit{атомарного раздела базы знаний} является просто последовательность \textit{файлов ostis-системы}. Некоторые из этих файлов могут быть идентифицированными (именованными), если на них ссылаются другие файлы, а некоторые из них могут быть связаны с другими файлами различными отношениями (в частности, один файл может быть пояснением другого). Кроме того, некоторые из этих файлов могут быть формально специфицированы (например, указаны соответствующие им ключевые \textit{sc-элементы}).\\
	В самом простом случае \textit{сегмент} или \textit{атомарный раздел базы знаний} может быть \textit{структурой}, состоящей из \uline{одного} (!) \textit{sc-узла}, обозначающего \textit{файл ostis-системы} (чаще всего, \textit{ея-файл ostis-системы}). Т.е. сам \textit{файл ostis-системы} может быть \textit{знанием ostis-системы}, но не может быть структурно \textit{выделяемым} \textit{фрагментом базы знаний} ostis-системы. При этом \textit{sc-узел}, обозначающий \textit{файл ostis-системы}, являющийся \textit{знанием}, может быть единственным \textit{sc-элементом} структурно выделяемого \textit{знания ostis-системы}.}
\scnnote{Для наглядного отображения (визуализации) \textit{сегмента} или \textit{атомарного раздела базы знаний ostis-системы} целесообразно представить указанное \textit{sc-знание} в виде конкатенации (последовательности) таких sc-знаний, которые, во-первых, были бы достаточно крупными и логико-семантически значимыми для соответствующего \textit{сегмента} или \textit{атомарного раздела базы знаний ostis-системы} и, во-вторых, для которых существовал бы алгоритм \uline{однозначного} (!) размещения (на экране) внешнего представления этих \textit{sc-знаний} (в \textit{SCg-коде} или в \textit{SCn-коде}).\\
	Однозначность здесь означает наличие легко усваиваемого пользователями стандартного \uline{стиля визуализации} \textit{sc-знаний} и заключается в том, что многократная визуализация одного и того же \textit{sc-знания} с помощью указанного алгоритма должна приводить к синтаксически эквивалентным, а в случае \textit{SCg-кода} и к геометрически конгруэнтным текстам. Очевидно, что для произвольных \textit{sc-знаний} большого объёма такого алгоритма не существует, но для \textit{sc-знаний}, содержащих описание собственной структуры и семантической типологии собственных фрагментов, разработка такого алгоритма вполне реальна при наличии достаточного количества указанных \textit{метазнаний} о структуре отображаемых (визуализируемых) \textit{sc-знаний}.}


\scnheader{sc-идентификатор выделенного фрагмента базы знаний}
\scnidtf{название (имя) выделенного фрагмента базы знаний}
\scnexplanation{Не следует путать объект описания (спецификации) и само описание. Поэтому в \textit{sc-идентфикаторе} фрагмента базы знаний должны присутствовать слова, указывающие на семантический или структурный тип именуемого фрагмента базы знаний (описание, спецификация, анализ, сравнительный анализ, сравнение, определение, раздел, предметная область, онтология и т.п.).
	
	Таким образом, \textit{sc-идентификатор выделенного фрагмента базы знаний} ostis-системы должен иметь \uline{явное} (!) указание на то, что он является обозначением именно фрагмента базы знаний, а не того, что описывается в этом фрагменте.}
\scnnote{Мы не будем использовать такой изменчивый для нас способ идентификации разделов \textit{Стандарта OSTIS}, как нумерацию этих разделов, поскольку, например, в разных издаваемых официальных версиях \textit{Стандарта OSTIS} одному и тому же разделу \textit{Стандарта OSTIS} могут соответствовать разные номера.}


\scnheader{выделенный фрагмент базы знаний}
\scntext{основной sc-идентификатор}{выделенный фрагмент базы знаний}
\scnaddlevel{1}
\scntext{используемая аббревиатура}{выделенный фр-нт б.з.}
\scnaddlevel{-1}
\scnsubdividing{именованный фрагмент базы знаний\\
	\scnaddlevel{1}
	\scnidtf{\textit{выделенный фрагмент базы знаний}, имеющий \textit{sc-идентификатор} (имя, название)}
	\scnnote{\textit{Именованными фрагментами баз знаний} могут быть только структурно \textit{выделенные фрагменты баз знаний}}
	\scnnote{Все \textit{семейства разделов баз знаний}, все \textit{разделы баз знаний} и все \textit{сегменты баз знаний} должны быть именованными}
	\scnsuperset{семейство разделов базы знаний}
	\scnsuperset{раздел базы знаний}
	\scnsuperset{сегмент базы знаний}
	\scnaddlevel{-1}
	;неименованный фрагмент базы знаний\\
	\scnaddlevel{1}
	\scnidtf{\textit{выделенный фрагмент базы знаний}, \uline{не} имеющий \textit{sc-идентификатора} (имени, названия)}
	\newpage
	\scnnote{\textit{неименованными фрагментами баз знаний} могут быть только \textit{выделенные фрагменты сегментов баз знаний} либо выделенные фрагменты таких \textit{разделов баз знаний}, которые не состоят из \textit{сегментов}}
	\scnaddlevel{-1}}

\scnheader{титульная спецификация выделенного фрагмента базы знаний}
\scnexplanation{\textit{Титульная спецификация выделенного фрагмента базы знаний} ostis-системы представляет собой \textit{структуру}, описывающую свойства специфицируемого знания и включающую в себя: 
	\begin{scnitemize}
		\item связи принадлежности специфицируемого знания соответствующим классам \textit{знаний ostis-систем};
		\item связи, указывающие логически предшествующее и логически следующее \textit{знание ostis-системы};
		\item связь, описывающую декомпозицию специфицируемого знания на последовательность знаний более низкого структурного уровня (декомпозицию разделов на сегменты);
		\item различного вида связи с другими \textit{знаниями ostis-систем}, которые сами "целиком"{} входят в состав спецификации специфицируемого знания (такими знаниями могут быть аннотации, предисловия, введения, оглавления, заключения);
		\item различного вида связи с другими \textit{знаниями ostis-систем}, которые сами не входят в состав спецификации специфицируемого знания (такого рода связями могут быть связи \textit{семантической близости} специфицируемого знания с другими знаниями, связи \textit{семантической эквивалентности}, связи\textit{семантического включения}, связи \textit{противоречивости знаний});
		\item связи, указывающие различного вида \textit{ключевые sc-элементы} (ключевые знаки), соответствующие специфицируемому знанию;
		\item связи специфицируемого знания с авторским коллективом, коллективом рецензентов, с датой его последнего обновления;
		\item для каждого нового целостного фрагмента, вводимого в состав \textit{базы знаний}, в истории эволюции этой \textit{базы знаний} указываются:
		\begin{scnitemizeii}
			\item \textit{автор*} или \textit{авторы*} первой версии этого фрагмента;
			\item отметка времени появления (дата-час-минута) всех версий этого фрагмента (в том числе и окончательно утверждённой, согласованной версии, которая, собственно, и становится фрагментом, включенным в согласованную часть базы знаний);
			\item \textit{рецензии*} (замечания к доработке) всех предварительных версий разрабатываемого \textit{фрагмента базы знаний};
			\item \textit{авторы*} всех указанных рецензий;
			\item отметка времени появления всех указанных рецензий;
			\item события по одобрению, утверждению различных предварительных версий разрабатываемого \textit{фрагмента базы знаний} различными рецензентами и экспертами с указанием отметки времени появления этих событий;
			\item темпоральная последовательность предварительных версий.
		\end{scnitemizeii}
	\end{scnitemize}
}
\scnnote{Знак такой спецификации явно не вводится, а сама эта спецификация непосредственно входит в состав специфицируемого фрагмента и включает в себя аннотацию, предисловие, авторов, ключевые знаки, декомпозицию специфицируемого фрагмента базы знаний и прочее}
\scnsubdividing{титульная спецификация раздела базы знаний;
	титульная спецификация семейства разделов базы знаний;
	титульная спецификация сегмента базы знаний;
	титульная спецификация выделенного фрагмента сегмента или атомарного раздела базы знаний}
\scnexplanation{\textit{Титульная спецификация выделенного фрагмента базы знаний} содержит общую информацию об этом фрагменте, является непосредственно \uline{частью} специфицируемого фрагмента \textit{базы знаний} и при этом сама \uline{не является} явно \textit{выделенным фрагментом базы знаний}}
\scniselement{спецификация}
\scnidtf{основная \textit{метаинформация} (основное \textit{метазнание}) о \textit{выделенном фрагменте базы знаний} -- о его структуре, \textit{авторах\scnrolesign}, \textit{ключевых знаках\scnrolesign} и т.д.}
\scnexplanation{\uline{неявно} \textit{выделяемый фрагмент базы знаний}, который:
	\begin{scnitemize}
		\item не имеет "собственного"{} ограничителя ("собственного"{} контура или "собственных"{} ограничивающих фигурных скобок);
		\item является семантической спецификацией соответствующего \uline{явно} выделяемого фрагмента базы знаний;
		\item является непосредственной \uline{частью} специфицируемого фрагмента базы знаний
\end{scnitemize}}
\scnnote{В \textit{sc.n-тексте} титульная спецификация фрагмента базы знаний размещается сразу после фигурной скобки, открывающей этот фрагмент}

\scnheader{титульная спецификация раздела базы знаний}
\scnnote{\textit{титульная спецификация раздела базы знаний} должна включать в себя достаточно подробное описание семантических свойств этого раздела и, в частности, подробное описание его связей с другими семантически близкими разделами. Это необходимо для обеспечения автономности разделов баз знаний.}

\scnheader{титульная спецификация семейства разделов базы знаний}
\scnnote{Если \textit{разделы базы знаний} являются семантически \uline{ключевыми} \textit{выделенными фрагментами баз знаний}, определяющими спецификацию систем используемых понятий и направления наследования свойств, то \textit{семейства разделов баз знаний} являются \uline{ключевыми} для структуризации виртуальной \textit{Базы знаний Экосистемы OSTIS}, для обмена \textit{знаниями} между различными субъектами \textit{Экосистемы OSTIS}.\\
	Поэтому типология \textit{семейств разделов баз знаний} и качество \textit{титульной спецификации семейств разделов баз знаний} имеют большое значение.}

\scnheader{титульная спецификация выделенного фрагмента базы знаний}
\scnrelfrom{множество используемых понятий}{Множество понятий используемых в титульных спецификациях выделенных фрагментов баз знаний}
\scnaddlevel{1}
\scnsuperset{класс выделенных фрагментов}
\scnaddlevel{1}
\scnhaselement{семейство разделов базы знаний}
\scnhaselement{раздел базы знаний}
\scnhaselement{неатомарный раздел базы знаний}
\scnhaselement{атомарный раздел базы знаний}
\scnhaselement{сегмент базы знаний}
\scnhaselement{выделенный фрагмент сегмента или атомарного раздела базы знаний}
\scnhaselement{выделенный фрагмент атомарного раздела базы знаний}
\scnhaselement{выделенный фрагмент сегманта базы знаний}
\scnaddlevel{-1}
\scnsuperset{отношение, связывающее выделенные фрагменты баз знаний с персонами}
\scnaddlevel{1}
\scnhaselement{автор*}
\scnhaselement{рецензент*}
\scnhaselement{эксперт*}
\scnhaselement{технический редактор*}
\scnhaselement{консультант*}
\scnaddlevel{1}
\scnidtf{активный участник обсуждения вопросов, рассматриваемых в специфицируемом фрагменте базы знаний*}
\scnaddlevel{-2}
\scnsuperset{отношение, связывающее выделенные фрагменты баз знаний с ея-файлами}
\scnaddlevel{1}
\scnhaselement{аннотация*}
\scnhaselement{предисловие*}
\scnaddlevel{1}
\scnidtf{Бинарное ориентированное отношение, каждая пара которого связывает:
	\begin{scnitemize}
		\item знак некоторого информационного ресурса (в частности, раздела базы знаний или раздела опубликованного документа);
		\item знак информационной конструкции, описывающей цели создания указанного информационного ресурса, предысторию его создания, планируемые направления дальнейшего развития, состав авторов и др.
\end{scnitemize}}
\scnaddlevel{-1}
\scnhaselement{введение*}
\scnhaselement{эпиграф*}
\scnhaselement{заключение*}
\scnhaselement{рассматриваемый вопрос*}
\scnhaselement{основные положения*}
\scnhaselement{вопрос для самопроверки*}
\scnhaselement{упражнение*}
\scnaddlevel{1}
\scnidtf{задача*}
\scnidtf{самостоятельная (индивидуальная) работа*}
\scnaddlevel{-1}
\scnhaselement{коллективный проект*}
\scnhaselement{неосновной sc-идентификатор*}
\scnaddlevel{1}
\scnnote{неосновным sc-идентификатором, в частности, может быть альтернативное название выделенного (специфицируемого) фрагмента базы знаний}
\scnaddlevel{-1}
\scnhaselement{часто используемый sc-идентификатор*}
\scnhaselement{сокращенный sc-идентификатор*}
\scnhaselement{библиографический источник, отражающий аналогичную точку зрения*}
\scnhaselement{библиографический источник, отражающий альтернативную точку зрения*}
\scnhaselement{библиографический источник, дополняющий данную точку зрения*}
\scnhaselement{сокращение*}
\scnaddlevel{1}
\scnidtf{используемое сокращение*}
\scnidtf{сокращение, используемое в специфицируемом фрагменте базы знаний при построении sc-идентификаторов, а также при оформлении ея-файлов*}
\scnidtf{Бинарное ориентированное отношение, каждая пара которого связывает естественно-языковую фразу с ее сокращенной записью*}
\scnaddlevel{-2}
\scnsuperset{отношение, описывающее структурные или семантические связи и между выделенными фрагментами баз знаний}
\scnaddlevel{1}
\scnhaselement{конкатенация сегментов*}
\scnhaselement{предыдущий сегмент*}
\scnaddlevel{1}
\scnidtf{предыдущий сегмент в рамках соответствующего раздела*}
\scnaddlevel{-1}
\scnhaselement{следующий сегмент*}

\scnhaselement{дочерний фрагмент базы знаний*}
\scnaddlevel{1}
\scnsuperset{дочерний раздел базы знаний*}
\scnsuperset{дочерняя предметная область*}
\scnsuperset{дочерняя предметная область и онтология*}
\scnnote{Для фрагмента базы знаний важно указать не только дочерние по отношению к нему фрагменты базы знаний, но и те фрагменты базы знаний, по отношению к которым данный фрагмент базы знаний является дочерним}
\scnaddlevel{-2}

\scnsuperset{отношение, описывающее ролевой статус знаков, входящих в состав выделенных фрагментов баз знаний}
\scnaddlevel{1}
\scnhaselement{ключевой знак*}
\scnhaselement{ключевой знак первого плана*}
\scnhaselement{ключевой знак второго плана*}
\scnhaselement{ключевой объект исследования*}
\scnhaselement{ключевое понятие*}
\scnhaselement{ключевой класс объектов исследования*}
\scnhaselement{исследуемое отношение*}
\scnhaselement{исследуемый параметр*}
\scnhaselement{исследуемый класс структур*}
\scnaddlevel{-2}

\bigskip

\scnendstruct \scnendsegmentcomment{Структуризация баз знаний ostis-систем}

\end{SCn}


\begin{SCn}
	
\scnheader{Структуризация баз знаний ostis-сиситем}
\scnsubset{сегмент базы знаний}
\scnidtf{Структурная типология знаний ostis-системы}
\scntext{введение}{\textit{База знаний ostis-системы} имеет достаточно развитую иерархическую структуру. База знаний делится на разделы. Разделы бывают атомарными и неатомарными. Неатомарный раздел состоит из сегментов. Атомарные разделы не имеют сегментов. Разделы \textit{базы знаний ostis-системы} могут иметь самое различное назначение. Так, например, \textit{База знаний Метасистемы IMS.ostis} включает в себя:
\begin{scnitemize}
\item \textit{раздел}, содержащий текущее состояние постоянно пополняемого и совершенствуемого \textit{Стандарта} Технологии OSTIS;
\item \textit{раздел}, посвящённый описанию \textit{конечных пользователей и разработчиков} \textit{Метасистемы IMS.ostis}; 
\item \textit{раздел}, посвящённый описанию \textit{история эксплуатации  Метасистемы IMS.ostis};
\item \textit{раздел}, посвящённый описанию \textit{истории эволюции  Метасистемы IMS.ostis} (в т.ч. истории эволюции и её \textit{база знаний}); 
\item \textit{раздел} посвящённый описанию \textit{интеллектуальных компьютерных систем}, разработанных(порождённых) с помощью \textit{Метасистемы IMS.ostis}.
\end{scnitemize}}
\scnheader{выделенный фрагмент базы знаний}
\scnidtf{фрагмент базы знаний, для которого в \textit{базе знаний} вводится знак, обозначающий этот фрагмент, т.е. являющийся знаком множества \uline{всех} знаков, входящих в состав этого фрагмента. При представлении фрагмента базы знаний на внешних языках (SCg-коде, SCn-коде) указанный знак выделенного фрагмента базы знаний представляется либо в виде sc.g-контура, либо в виде пары фигурных скобок, ограничивающих текст обозначаемого фрагмента базы знаний}
\scnidtf{выделенный фрагмент базы знаний}
\scnaddlevel{1}
\scniselement{сокращённые sc-идентификатор} 
\scnaddlevel{-1}
\scnidtf{явно структурно выделенный фрагмент базы знаний}
\scnnote{явное выделение фрагмента базы знаний осуществляется:
\begin{scnitemize}
\item в \textit{SC-коде} путем введения \textit{знака}, обозначающего \textit{множество} \uline{всех} знаков, входящих в состав \textit{выделенного фрагмента базы знаний};
\item в \textit{SCg-коде} с помощью \textit{sc.g-контура}, ограничивающего sc.g-представление \textit{выделенного фрагмента базы знаний};
\item в \textit{SCn-коде} с помощью пары фигурных скобок, ограничивающих \textit{sc.n-представление} \textit{выделенного фрагмента базы знаний}.
\end{scnitemize}}
\scnsubdividing{семейство разделов базы знаний
\scnaddlevel{1}
\scnidtf{семантически целостное множество разделов базы знаний, имеющих достаточно сильные семантические связи между собой}
\scnaddlevel{-1}
;раздел базы знаний
\scnaddlevel{1}
\scnidtf{Основной (по семантической значимости) вид выделенных фрагментов баз знаний}\\
\scnaddlevel{-1}
\scnnote{В общем случае \textit{неатомарный раздел базы знаний} может иметь неограниченное число \textit{сегментов}. \textit{Сегменты базы знаний} не могут состоять из других сегментов (подсегментов). В этом смысле \textit{сегменты базы знаний} имеют атомарный характер.}
\scnidtf{раздел базы знаний ostis-системы}
\scnidtf{модуль (блок) база знаний}
\scnnote{В общем случае многим \textit{разделам базы знаний} ставятся в соответствие такие тексты, как \uline{предисловие},  \uline{введение}, \uline{заключение}, \uline{аннотация}, \uline{оглавление}, упражнения.\\ Некоторые из этих текстов могут иметь статус разделов.}
\scnaddlevel{1}
\scnsubdividing{неатомарный раздел базы знаний\\
\scnaddlevel{1}
\scnidtf{раздел базы знаний, состоящий из сегментов, декомпозируемый на сегменты} 
\scnaddlevel{-1}
;атомарный раздел базы знаний 
\scnaddlevel{1}
\scnidtf{раздел базы знаний, не содержащий сегментов}
\scnaddlevel{-1}}
\scnaddlevel{-1}
;сегмент базы знаний
\scnaddlevel{1}
\scnidtf{сегмент раздела базы знаний}
\scnidtf{сегмент базы знаний ostis-системы}
\scnidtf{структурно выделяемое sc-знание ostis-системы, структурный уровень которого ниже уровня разделов базы знаний}
\scnnote{Сегменты базы знаний не могут иметь иерархической структуры, т.е. не могут состоять из сегментов более низкого структурного уровня.}
\scnnote{Сегменты базы знаний входят в состав неатомарных разделов базы знаний.}
\scnaddlevel{-1}
;выделенный фрагмент сегмента или атомарного раздела базы знаний\\
\scnsubdividing{
выделенный фрагмент атомарного раздела базы знаний  
;выделенный фрагмент сегмента базы знаний}}
\scnsubset{sc-знание}
\scnnote{Каждый \textit{выделенный фрагмент базы знаний} представляет собой структурно оформленное (структурно выделенное) \textit{знание}, хранимое в \textit{базе знаний} \textit{интеллектуальной компьютерной системы}, (точнее, в \textit{базе знаний ostis-системы}) и представленное в формализованном виде на \textit{внутреннем языке представления знаний} (в \textit{SC-коде}) Представленное таким образом \textit{знание} будем называть \textit{sc-знанием}.}
\scnidtfexp{
\textit{информационная конструкция}, которая:
\begin{scnitemize}
\item принадлежит \textit{SC-коду};
\item является синтактически корректный;
\item обладает семантической целостностью -- отсутствием \textit{информационных дыр} ("недомолвок"{}), препятствующих её пониманию;
\item имеет нетривиальный \textit{объём информации} -- количество описываемых сущностей (в том числе, \textit{связей} и \textit{классов});
\item имеет достаточно высокое качество по другим характеристикам, в частности, достаточно высокую ценность.
\end{scnitemize}}
\scnnote{Типология \textit{sc-знаний}, в частности, по объему представленной информации определённым образом коррелирует с типологией \textit{выделенных фрагментов баз знаний} -- объём информации, содержащейся в \textit{разделах баз знаний}, должен быть приблизительно одинаковым, объём, содержащейся в разделе базы знаний, должен быть ниже объёма информации, содержащегося в любом семействе разделов базы знаний, и должен быть выше объёма информации, содержащегося в любом \textit{сегменте базы знаний}.\\
При этом в процессе эволюции базы знаний \textit{раздел базы знаний} может преобразоваться в семейства разделов, а \textit{сегмент базы знаний} может преобразоваться в \textit{раздел база знаний}.}
\scnheader{раздел база знаний}
\scnnote{Различные \textit{предметные области}, различные \textit{онтологии}, а также предметные области, объединённые с соответствующими им онтологиями, являются важнейшими видами \textit{sc-знаний ostis-систем}, обеспечивающими логически стройную систематизацию \textit{знаний ostis-систем} и, соответственно семантическую структуризацию \textit{баз знаний}. При этом указанные типы знаний обычно представляются в виде \textit{разделов базы знаний}, иерархия которых, задаваемая Отношением \textit{частная предметная область*} или Отношением \textit{частная предметная область и онтология*}, соответствует логико-семантической иерархии \textit{предметных областей}. Но, кроме \textit{разделов базы знаний}, являющихся \textit{предметными областями}, \textit{онтологиями}, \textit{предметными областями}, объединенными с соответствующими им онтологиями, вводится и целый ряд других семантических типов \textit{разделов базы знаний}, определяемых характером соотношения разделов базы знаний с используемыми (рассматриваемыми) предметными областями и онтологиями.}
\scnsuperset{предметная область}\\
фрагмент предметной области\\
\scnsuperset{интегрированная онтология}
\scnsuperset{частная онтология}
\scnsuperset{предметная область и онтология}
\scnaddlevel{1}
\scnidtf{предметная область, объединенная (интегрированная) с соответствующей ей онтологией}
\scnidtf{предметная область вместе с онтологией, которая её специфицирует}
\scnaddlevel{-1}
\scnnote{Если каждому \textit{разделу базы знаний} ostis-системы будет четко соответствовать его семантический тип, то к "синтаксическим"{} связям между \textit{разделами базы знаний} добавится большое количество "осмысленных"{} (семантические интерпретируемых) связей, определяющих "семантическое местоположение"{} ("семантические координаты"{}) каждого \textit{раздела базы знаний} во множестве всех разделов, входящих в состав базы знаний ostis-системы.}
\scnnote{В основе представления \textit{базы знаний ostis-системы} лежат развитые средства семантической структуризации баз знаний и семантической систематизации баз знаний \textit{ostis-систем}. Можно выделить следующие уровни систематизации элементов и фрагментов смыслового пространства, построенного на основе \textit{SC-кода}:
\begin{scnitemize}
\item уровень знаков всевозможных сущностей (уровень \textit{sc-элементов});
\item уровень вводимых \textit{понятий}, обозначающих ключевые (исследуемые) в предметных областях классы сущностей;
\item уровень \textit{высказываний}, описывающих закономерности (свойства) экземпляров исследуемых классов сущностей (исследуемых понятий);
\item уровень \textit{предметных областей}, \textit{онтологий} и разделов, семантический тип которых известен.
\end{scnitemize}}
\end{SCn}

\begin{SCn}

\scnheader{раздел базы знаний}
\scnidtf{раздел б.з.}
	\scnaddlevel{1}
	\scniselement{сокращенный sc-идентификатор}
	\scnaddlevel{-1}
\scnexplanation{Множество \textit{разделов баз знаний} имеет:
	\begin{scnitemize}
	\item богатую семантическую типологию;
	\item богатый набор отношений, описывающих семантические связи между разделами.
	\end{scnitemize}
	Синтаксически \textit{разделы баз знаний} могут пересекаться (иметь общие элементы), но никогда один раздел не может включаться (полностью входить в состав) другого раздела. В этом смысле понятие подраздела (точнее, \textit{частного раздела}*) имеет не "синтаксический"{} смысл, а семантический -- глубокое наследование свойств при достаточно большой степени "независимости"{} друг от друга}
\scnnote{Важным свойством \textit{раздела базы знаний} \scnbigspace \textit{ostis-системы} является его семантическая целостность -- наличие достаточно стабильного набора классов исследуемых сущностей (\textit{объектов исследования}) и достаточно стабильного семейства \textit{отношений} и семейства \textit{параметров}, заданных на различных классах объектов исследования, а также семейства \textit{классов структур} использующих указанные выше понятия (введенные \textit{классы объектов исследования}, введенные \textit{отношения} и \textit{классы структур}). Такая целостность дает возможность развивать \textit{раздел базы знаний}, не выходя "за рамки"{} используемой системы \textit{понятий}. Это позволяет развивать каждый \textit{раздел базы знаний} в известной мере \uline{независимо} от других разделов, что существенно повышает \uline{гибкость} и \uline{стратифицированность} \textit{базы знаний}.}	
\scnexplanation{Основными свойствами \textit{раздела базы знаний} как структурно \textit{выделяемого фрагмента базы знаний} являются:
	\begin{scnitemize}
	\item семантическая целостность -- наличие четкого критерия, позволяющего установить для каждого конкретного знания то, включать или не включать это знание в состав данного раздела;
	\item потенциальная возможность эволюционировать в достаточной степени независимо от других разделов, но при условии соблюдения всех требований, обеспечивающих постоянную поддержку \uline{семантической совместимости} данного раздела со всеми остальными семантически смежными разделами.
	\end{scnitemize}}
	\scnaddlevel{1}
	\scntext{следовательно}{Грамотная декомпозиция \textit{базы знаний} на разделы, основанная на четкой стратификации процесса эволюции накапливаемой человечеством общечеловеческой объединенной \textit{базы знаний}, сутью которой является \uline{минимизация} трудоемкости усилий по согласованию и обеспечению с авторами других разделов \textit{семантической совместимости} со смежными разделами, создает предпосылки высоких темпов эволюции \textit{базы знаний} в целом.}
	\scnaddlevel{-1}
\scnnote{Любой \textit{раздел базы знаний} не является структурной частью другого раздела. Каждый раздел самодостаточен и целостен. Это обеспечивается тем, что в состав \textit{титульной спецификации} каждого раздела входит \textit{семантическая окрестность},описывающая связи специфицируемого раздела \uline{со всеми} семантически близкими ему разделами.\\
При этом разные разделы могут иметь разный семантический тип:
	\begin{scnitemize}
	\item раздел может быть предметной областью, интегрированной со всеми ее онтологиями;
	\item раздел может быть просто предметной областью;
	\item раздел может быть какой-либо онтологией чего угодно (не обязательно предметной области);
	\item и т.д.
	\end{scnitemize}}
\scnnote{Если в \textit{титульную спецификацию} каждого раздела будет входить семантическая спецификация каждого раздела, включающая его семантические связи со всеми семантически близкими разделами, то последовательность (порядок) разделов в "линейном"{} исходном тексте, публикуемом в качестве очередной версии Стандарта OSTIS, может быть в достаточной степени \uline{произвольной}.}


\scnheader{семейство разделов базы знаний}
\scnidtf{множество семантически связанных друг с другом разделов базы знаний}
\scnidtf{кластер разделов базы знаний}
\scnnote{Семантические связи между разделами, входящими в состав семейства разделов, представляются в рамках \textit{титульных спецификаций разделов}, каждая из которых является специальной частью соответствующего (специфицируемого) раздела, входящего в состав семейства разделов.

Для каждого \textit{раздела базы знаний} в рамках его титульной спецификации формируется \textit{семантическая окрестность} его связей со всеми семантически близкими ему разделами (и, прежде всего, с теми разделами, которые входят в состав тех семейств разделов, в которые входит заданный раздел). При этом акцентируется внимание именно на семантических связях между разделами. Так, например, вместо структурной ("синтаксической"{}) связи "быть подразделом*"{} (т.е. быть частью заданного раздела) вводится связь "быть \textit{частным разделом}*"{}. \\
Данная связь указывает направление наследования свойств исследуемых объектов заданного раздела от разделов, исследующих более общие классы объектов.
Порядок (последовательность) разделов в рамках \textit{семейства разделов базы знаний} при наличии \uline{явно представленных} семантических связей между разделами, входящими в семейство разделов, может быть достаточно \uline{произвольным}, что очень важно, например, при формировании оглавления очередной издаваемой версии \textit{Стандарта OSTIS}. Таким образом, трактовка \textit{Стандарта OSTIS}, а также всех издаваемых версий этого Стандарта как \textit{семейства разделов базы знаний} \scnbigskip \textit{Метасистемы IMS.ostis} обеспечивает высокий уровень гибкости \textit{Стандарта OSTIS}, а также легкость "переиздаваемости"{} его версий.}


\scnheader{сегмент или атомарный раздел базы знаний}
\scnnote{Простейшей формой \textit{сегмента} или \textit{атомарного раздела базы знаний} является просто последовательность \textit{файлов ostis-системы}. Некоторые из этих файлов могут быть идентифицированными (именованными), если на них ссылаются другие файлы, а некоторые из них могут быть связаны с другими файлами различными отношениями (в частности, один файл может быть пояснением другого). Кроме того, некоторые из этих файлов могут быть формально специфицированы (например, указаны соответствующие им ключевые \textit{sc-элементы}).\\
В самом простом случае \textit{сегмент} или \textit{атомарный раздел базы знаний} может быть \textit{sc-структурой}, состоящей из \uline{одного} (!) \textit{sc-узла}, обозначающего \textit{файл ostis-системы} (чаще всего, \textit{ея-файл ostis-системы}). Т.е. сам \textit{файл ostis-системы} может быть \textit{знанием ostis-системы}, но не может быть структурно \textit{выделяемым} \textit{фрагментом базы знаний} ostis-системы. При этом \textit{sc-узел}, обозначающий \textit{файл ostis-системы}, являющийся \textit{знанием}, может быть единственным \textit{sc-элементом} структурно выделяемого \textit{знания ostis-системы}.}
\scnnote{Для наглядного отображения (визуализации) \textit{сегмента} или \textit{атомарного раздела базы знаний ostis-систем\textit{ы} целесообразно представить указанное \textit{sc-знание} в виде конкатенации (последовательности) таких }sc-знаний, которые, во-первых, были бы достаточно крупными и логико-семантически значимыми для соответствующего \textit{сегмента} или \textit{атомарного раздела базы знаний ostis-системы} и, во-вторых, для которых существовал бы алгоритм \uline{однозначного} (!) размещения (на экране) внешнего представления этих \textit{sc-знаний} (в \textit{SCg-коде} или в \textit{SCn-коде}).\\
Однозначность здесь означает наличие легко усваиваемого пользователями стандартного \uline{стиля визуализации} \textit{sc-знаний} и заключается в том, что многократная визуализация одного и того же \textit{sc-знания} с помощью указанного алгоритма должна приводить к синтаксически эквивалентным, а в случае \textit{SCg-кода} и к геометрически конгруэнтным текстам. Очевидно, что для произвольных \textit{sc-знаний} большого объёма такого алгоритма не существует, но для \textit{sc-знаний}, содержащих описание собственной структуры и семантической типологии собственных фрагментов, разработка такого алгоритма вполне реальна при наличии достаточного количества указанных \textit{метазнаний} о структуре отображаемых (визуализируемых) \textit{sc-знаний}.}


\scnheader{sc-идентификатор выделенного фрагмента базы знаний}
\scnidtf{название (имя) выделенного фрагмента базы знаний}
\scnexplanation{Не следует путать объект описания (спецификации) и само описание. Поэтому в \textit{sc-идентфикаторе} фрагмента базы знаний должны присутствовать слова, указывающие на семантический или структурный тип именуемого фрагмента базы знаний (описание, спецификация, анализ, сравнительный анализ, сравнение, определение, раздел, предметная область, онтология и т.п.).

Таким образом, \textit{sc-идентификатор выделенного фрагмента базы знаний} ostis-системы должен иметь \uline{явное} (!) указание на то, что он является обозначением именно фрагмента базы знаний, а не того, что описывается в этом фрагменте.}
\scnnote{Мы не будем использовать такой изменчивый для нас способ идентификации разделов \textit{Стандарта OSTIS}, как нумерацию этих разделов, поскольку, например, в разных издаваемых официальных версиях \textit{Стандарта OSTIS} одному и тому же разделу \textit{Стандарта OSTIS} могут соответствовать разные номера.}


\scnheader{выделенный фрагмент базы знаний}
\scnrelfrom{основной sc-идентификатор}{\scnfilelong{выделенный фрагмент базы знаний}}
	\scnaddlevel{1}
	\scnrelfrom{используемая аббревиатура}{\scnfilelong{выделенный фр-нт б.з.}}
	\scnaddlevel{-1}
\scnsubdividing{именованный фрагмент базы знаний\\
	\scnaddlevel{1}
	\scnidtf{\textit{выделенный фрагмент базы знаний}, имеющий \textit{sc-идентификатор} (имя, название)}
	\scnnote{\textit{Именованными фрагментами баз знаний} могут быть только структурно \textit{выделенные фрагменты баз знаний}}
	\scnnote{Все \textit{семейства разделов баз знаний}, все \textit{разделы баз знаний} и все \textit{сегменты баз знаний} должны быть именованными}
	\scnsuperset{семейство разделов базы знаний}
	\scnsuperset{раздел базы знаний}
	\scnsuperset{сегмент базы знаний}
	\scnaddlevel{-1}
;неименованный фрагмент базы знаний\\
	\scnaddlevel{1}
	\scnidtf{\textit{выделенный фрагмент базы знаний}, \uline{не} имеющий \textit{sc-идентификатор} (имени, названия)}
	\scnnote{\textit{неименованными фрагментами баз знаний} могут быть только \textit{выделенные фрагменты сегментов баз знаний} либовыделенные фрагменты таких \textit{разделов баз знаний}, которые не состоят из \textit{сегментов}}
	\scnaddlevel{-1}}

\end{SCn}
\begin{SCn}
	
\scnheader{титульная спецификация выделенного фрагмента базы знаний}
\scnexplanation{\textit{Титульная спецификация выделенного фрагмента базы знаний} ostis-системы представляет собой \textit{sc-структуру}, описывающую свойства специфицируемого знания и включающую в себя: 
	\begin{scnitemize}
		\item связи принадлежности специфицируемого знания соответствующим классам \textit{знаний ostis-систем};
		\item связи, указывающие логически предшествующее и логически следующее \textit{знание ostis-системы};
		\item связь, описывающую декомпозицию специфицируемого знания на последовательность знаний более низкого структурного уровня (декомпозицию разделов на сегменты);
		\item различного вида связи с другими \textit{знаниями ostis-систем}, которые сами "целиком"\ входят в состав спецификации специфицируемого знания (такими знаниями могут быть аннотации, предисловия, введения, оглавления, заключения);
		\item различного вида связи с другими \textit{знаниями ostis-систем}, которые сами не входят в состав спецификации специфицируемого знания (такого рода связями могут быть связи \textit{семантической близости} специфицируемого знания с другими знаниями, связи \textit{семантической эквивалентности}, связи\textit{семантического включения}, связи \textit{противоречивости знаний});
		\item связи, указывающие различного вида \textit{ключевые sc-элементы} (ключевые знаки), соответствующие специфицируемому знанию;
		\item связи специфицируемого знания с авторским коллективом, коллективом рецензентов, с датой его последнего обновления;
		\item для каждого нового целостного фрагмента, вводимого в состав \textit{базы знаний}, в истории эволюции этой \textit{базы знаний} указываются:
		\begin{scnitemizeii}
			\item \textit{автор*} или \textit{авторы*} первой версии этого фрагмента;
			\item отметка времени появления (дата-час-минута) всех версий этого фрагмента (в том числе и окончательно утверждённой, согласованной версии, которая, собственно, и становится фрагментом, включенным в согласованную часть базы знаний);
			\item \textit{рецензии*} (замечания к доработке) всех предварительных версий разрабатываемого \textit{фрагмента базы знаний};
			\item \textit{авторы*} всех указанных рецензий;
			\item отметка времени появления всех указанных рецензий;
			\item события по одобрению, утверждению различных предварительных версий разрабатываемого \textit{фрагмента базы знаний} различными рецензентами и экспертами с указанием отметки времени появления этих событий;
			\item темпоральная последовательность предварительных версий.
		\end{scnitemizeii}
	\end{scnitemize}
}

\scnheader{титульная спецификация выделенного фрагмента базы знаний}
\scnnote{Знак такой спецификации явно не вводится, а сама эта спецификация непосредственно входит в состав специфицируемого фрагмента и включает в себя аннотацию, предисловие, авторов, ключевые знаки, декомпозицию специфицируемого фрагмента базы знаний и прочее}

\scnsubdividing{титульная спецификация раздела базы знаний;
титульная спецификация семейства разделов базы знаний;
титульная спецификация сегмента базы знаний;
титульная спецификация выделенного фрашмента сегмента или атомарного раздела базы знаний}

\scnheader{титульная спецификация выделенного фрагмента базы знаний}
\scnexplanation{\textit{Титульная спецификация выделенного фрагмента базы знаний} содержит общую информацию об этом фрагменте, является непосредственно \uline{частью} специфицируемого фрагмента \textit{базы знаний} и при этом сама \uline{не является} явно \textit{выделенным фрагментом базы знаний}}
\scnhaselement{спецификация}
\scnidtf{основная \textit{метаинформация} (основное \textit{метазнание}) о \textit{выделенном фрагменте базы знаний} -- о его структуре, \textit{авторах\scnrolesign}, \textit{ключевых знаках\scnrolesign} и т.д.}

\scnheader{титульная спецификация выделенного фрагмента базы знаний}
\scnexplanation{\uline{неявно} \textit{выделяемый фрагмент базы знаний}, который:
\begin{scnitemize}
	\item не имеет "собственного" ограничителя ("собственного" контура или "собственных" ограничивающих фигурных скобок);
	\item является семантической спецификацией соответствующего \uline{явно} выделяемого фрагмента базы знаний;
	\item является непосредственной \uline{частью} специфицируемого фрагмента базы знаний
\end{scnitemize}}
\scnnote{В \textit{sc.n-тексте} титульная спецификация фрагмента базы знаний размещается сразу после фигурной скобки, открывающей этот фрагмент}

\scnheader{титульная спецификация раздела базы знаний}
\scnnote{\textit{тиутульная спецификация раздела базы знаний} должна включать в себя достаточно подробное описание семантических свойств этого раздела и, в частности, подробное описание его связей с другими семантически близкими разделами. Это необходимо для обеспечения автономности разделов баз знаний.}

\scnheader{титульная спецификация семейства разделов базы знаний}
\scnnote{Если \textit{разделы базы знаний} являются семантически \uline{ключевыми} \textit{выделенными фрагментами баз знаний}, определяющими спецификацию систем используемых понятий и направления наследования свойств, то \textit{семейства разделов баз знаний} являются \uline{ключевыми} для структуризации виртуальной \textit{Базы знаний Экосистемы OSTIS}, для обмена \textit{знаниями} между различными субъектами \textit{Экосистемы OSTIS}.\\
Поэтому типология \textit{семейств разделов баз знаний} и качество \textit{титульной спецификации семейств разделов баз знаний} имеют большое значение.}

\scnheader{титульная спецификация выделенного фрагмента базы знаний}
\scnrelfrom{множество используемых понятий}{Множество понятий используемых в титльных спецификациях выделенных фрагментов баз знаний}
\scnaddlevel{1}
\scnsuperset{класс выделенных фрагментов}
\scnaddlevel{1}
\scnhaselement{семейство разделов базы знаний}
\scnhaselement{раздел базы знаний}
\scnhaselement{неатомарный раздел базы знаний}
\scnhaselement{атомарный раздел базы знаний}
\scnhaselement{сегмент базы знаний}
\scnhaselement{выделенный фрагмент сегмента или атомарного раздела базы знаний}
\scnhaselement{выделенный фрагмент атомарного раздела базы знаний}
\scnhaselement{выделенный фрагмент сегманта базы знаний}
\scnaddlevel{-1}
\scnsuperset{отношение, связывающее выделенные фрагменты баз знаний с персонами}
\scnaddlevel{1}
\scnhaselement{автор*}
\scnhaselement{рецензент*}
\scnhaselement{эксперт*}
\scnhaselement{технический редактор*}
\scnhaselement{консультант*}
\scnaddlevel{1}
\scnidtf{активный участник обсуждения вопросов, рассматриваемых в специфицируемом фрагменте базы знаний*}
\scnaddlevel{-1}
\scnsuperset{отношение, связывающее выделенные фрагменты баз знаний с ея-файлами}
\scnhaselement{аннотация*}
\scnhaselement{предисловие*}
\scnaddlevel{1}
\scnidtf{Бинарное ориентированное отношение, каждая пара которого связывает:
\begin{scnitemize}
	\item знак некоторого информационного ресурса (в частности, раздела базы знаний или раздела опубликованного документа)
	\item знак информационной конструкции, описывающей цели создания указанного информационного ресурса, предысторию его создания, планируемые направления дальнейшего развития, состав авторов и др.
\end{scnitemize}}
\scnaddlevel{-1}
\scnhaselement{введение*}
\scnhaselement{эпиграф*}
\scnhaselement{заключение*}
\scnhaselement{рассматриваемый вопрос*}
\scnhaselement{основные положения*}
\scnhaselement{вопрос для самопроверки*}
\scnhaselement{упражнение*}
\scnaddlevel{1}
\scnidtf{задача*}
\scnidtf{самостоятельная (индивидуальная) работа*}
\scnaddlevel{-1}
\scnhaselement{коллективный проект}

\scnhaselement{неосновной sc-идентификатор*}
\scnaddlevel{1}
\scnnote{неосновным sc-идентификатором, в частности, может быть альтернативное название выделенного (специфицируемого) фрагмента базы знаний}
\scnaddlevel{-1}
\scnhaselement{часто используемый sc-идентификатор*}
\scnhaselement{сокращенный sc-идентификатор}
\scnhaselement{используемое сокращение*}
\scnaddlevel{1}
\scnidtf{сокращение, используемое в специфицируемом фрагменте базы знаний при построении sc-идентификаторов, а также при оформлении ея-файлов*}
\scnaddlevel{-1}
\scnhaselement{библиографический источник, отражающий аналогичную точку зрения*}
\scnhaselement{библиографический источник, отражающий альтернативную точку зрения*}
\scnhaselement{библиографический источник, дополняющий данную точку зрения*}
\scnaddlevel{-1}
\scnhaselement{сокращение*}
\scnaddlevel{1}
\scnidtf{Бинарное ориентированное отношение, каждая пара которого связывает естественно-языковую фразу с ее сокращенной записью*}
\scnaddlevel{-1}
\scnsuperset{отношение, описывающее структурные или семантические связи и между выделенными фрагментами баз знаний}
\scnaddlevel{1}
\scnhaselement{конкатенация сегментов*}
\scnhaselement{предыдущий сегмент*}
\scnaddlevel{1}
\scnidtf{предыдущий сегмент в рамках соответствующего раздела*}
\scnaddlevel{-1}
\scnhaselement{следующий сегмент*}

\scnhaselement{частный фрагмент базы знаний*}
\scnaddlevel{1}
\scnsuperset{частный раздел базы знаний*}
\scnsuperset{частная предметная область*}
\scnsuperset{частная предметная область и онтология*}
\scnnote{Для фрагмента базы знаний важно указать не только частные по отношению к нему фрагменту базы знаний, но и те фрагменты базы знаний, по отношению к которым данный фрагмент базы знаний является частным}

\scnsuperset{отношение, описывающее ролевой статус знаков, входящих в состав выделенных фрагментов баз знаний}
\scnaddlevel{1}
\scnhaselement{ключевой знак*}
\scnhaselement{ключевой знак первого плана*}
\scnhaselement{ключевой знак второго плана*}
\scnhaselement{ключевой объект исследования*}
\scnhaselement{ключевое понятие*}
\scnhaselement{ключевой класс объектов исследования*}
\scnhaselement{исследуемое отношение*}
\scnhaselement{исследуемый параметр*}
\scnhaselement{исследуемый класс структур*}
\scnaddlevel{-4}

\scnendstruct \scninlinesourcecommentpar{Завершили Сегмент ``Структуризация баз знаний ostis-систем''}


\end{SCn}


\scnsegmentheader{Онтологическая формализация Базовой денотационной семантики SC-кода}
\begin{scnsubstruct}
	\scntext{примечание}{Суть онтологической формализации различных областей знаний, различных фрагментов \textit{баз знаний} интеллектуальных компьютерных систем заключается в следующем.}
	\begin{scnindent}
		\begin{scnsubdividing}
			\scnfileitem{Выделяется достаточно большой \textit{семантически целостный} фрагмент \textit{баз знаний}, включающий в себя:}
			\begin{scnindent}
				\begin{scnsubdividing}
					\scnfileitem{Все элементы некоторого одного ключевого класса рассматриваемых объектов (объектов исследования) или \textit{конечного} числа таких ключевых классов объектов исследования.}
					\scnfileitem{\textit{Все связи} между выделенными объектами исследования, соответствующие заданному \textit{семейству} отношений, параметров и классов структур, которое условно будем называть предметом исследования.}
				\end{scnsubdividing}
			\end{scnindent}
			\scnfileitem{Указанный семантически целостный фрагмент 	\textit{базы знаний}, являющийся чаще всего \textit{бесконечной} структурой, будем называть \textbf{\textit{предметной областью}}.}
			\scnfileitem{Сама формальная \textbf{\textit{онтология}} представляет собой формальную спецификацию выделенной \textit{предметной области} и включает в себя следующие \textbf{\textit{частные онтологии}}.}
			\begin{scnindent}
				\begin{scnsubdividing}
					\scnfileitem{\textbf{\textit{структурная спецификация предметной области}}, в которой указываются роли всех ключевых элементов (ключевых знаков), входящих в состав \textit{предметной области}. К числу таких ролей относятся}
					\begin{scnindent}
						\begin{scnhaselementset}
							\scnitem{\textit{максимальный класс объектов исследования\scnrolesign}}
							\scnitem{\textit{немаксимальный класс объектов исследования\scnrolesign}}
							\scnitem{\textit{ключевой объект исследования\scnrolesign}}
							\scnitem{\textit{исследуемый класс связок\scnrolesign}}
							\scnitem{\textit{исследуемый класс классов\scnrolesign}}
							\scnitem{\textit{исследуемый класс структур\scnrolesign}}
							\scnitem{\textit{неисследуемый класс\scnrolesign}}
							\begin{scnindent}
								\scnidtf{\textit{sc-класс}, исследуемый в другой (смежной) \textit{предметной области}}
							\end{scnindent}
						\end{scnhaselementset}
					\end{scnindent}
					\scnfileitem{\textbf{\textit{теоретико-множественная онтология}}, в которой описываются теоретико-множественные связи между всеми классами (\textit{sc-классами}), исследуемыми в рамках заданной (специфицируемой) \textit{предметной области}}
					\scnfileitem{\textbf{\textit{логическая онтология}}}
					\begin{scnindent}
						\begin{scnhaselementset}
							\scnfileitem{определения исследуемых классов (исследуемых понятий)}
							\scnfileitem{логическую иерархию исследуемых понятий, которая связывает каждое понятие со множеством тех понятий, которые явно используются в определении этого понятия}
							\scnfileitem{аксиомы и теоремы, описывающие свойства специфицируемой предметной области}
							\scnfileitem{тексты доказательств теорем}
							\scnfileitem{логическую иерархию теорем, которая связывает каждую теорему со множеством теорем, на основе которых она доказывается}
						\end{scnhaselementset}	
					\end{scnindent}	
					\scnfileitem{\textbf{\textit{терминологическая спецификация предметной области}}, в которой указывается \textit{sc-идентификаторы} всех ключевых \textit{sc-элементов} специфицируемой \textit{предметной области}, а также приводятся правила построения \textit{основных sc-идентификаторов} для элементов всех \textit{sc-классов} (понятий), исследуемых в рамках специфицируемой \textit{предметной области}.}
					\scnfileitem{\textbf{\textit{дидактическая спецификация предметной области}}, в которой приводится дополнительная информация, предназначенная для того, чтобы пользователи и разработчики (инженеры знаний), которые используют или совершенствуют специфицируемую \textit{предметную область} и ее \textit{онтологию}, могли быстрее усвоить их особенности.}
					\begin{scnindent}
						\scnrelfrom{смотрите}{Предметная область и онтология предметных областей}
					\end{scnindent}
					\scnfileitem{\textbf{\textit{проектная спецификация предметной области и соответствующей ей онтологии}}, в которой приводится информация об истории эволюции этой \textit{предметной области и онтологии}, а также о направлениях и планах организации дальнейшего их развития.}
				\end{scnsubdividing}
			\end{scnindent}	
		\end{scnsubdividing}
	\end{scnindent}
	\begin{scnrelfromset}{смотрите}
		\scnitem{Предметная область и онтология онтологий}
		\scnitem{Предметная область и онтология предметных областей}
	\end{scnrelfromset}
	\scntext{примечание}{Онтологическая формализация \textit{базовой денотационной семантики SC-кода} трактуется нами как \textit{формальная онтология}, представленная в \textit{SC-коде} и описывающая детонационную семантику \textit{семантически корректных sc-конструкций}. Указанную \textit{формальную онтологию} будем называть \textbf{\textit{Базовой денотационной семантикой SC-кода}}. Для того, чтобы уточнить \textit{предметную область}, специфицируемую этой \textit{онтологией}, введем следующие понятия:
		\begin{scnitemize}
			\item синонимия sc-элементов,
			\item отношение эквивалентности,
			\item sc-память,
			\item база знаний ostis-системы,
			\item ostis-система,
			\item \textup{[}sc-конструкция\textup{]},
			\item sc-знание,
			\item интеграция sc-конструкций*,
			\item sc-пространство.
		\end{scnitemize}}

	\scnheader{синонимия sc-элементов}
	\scnidtf{бинарное ориентированное \textit{отношение эквивалентности}, каждая пара которого связывает два \textit{sc-элемента}, обозначающие одну и ту же сущность*}
	\scnnote{Синонимия двух \textit{sc-элементов} возможна только в том случае, если эти \textit{sc-элементы} хранятся в \textit{sc-памяти} (входят в состав \textit{баз знаний}) \textit{разных} \textit{ostis-систем}. В рамках каждой \textit{ostis-системы} синонимичные \textit{sc-элементы} совпадают (отождействляются, склеиваются, считаются одним и тем же \textit{sc-элементом}).}

	\scnheader{отношение эквивалентности}
	\scnrelto{ключевое понятие}{\textsection~2.4.2. Формальная онтология связок и отношений}

	\scnheader{sc-память}

	\scnheader{база знаний ostis-системы}

	\scnheader{ostis-система}
	\begin{scnsubdividing}
		\scnitem{индивидуальная ostis-система}
		\scnitem{коллективная ostis-система}
	\end{scnsubdividing}

	\scnheader{\textup{[}sc-конструкция\textup{]}}
	\scnrelto{часто используемый sc-идентификатор}{\textbf{sc-множество}}
	\begin{scnindent}
	\scnidtf{информационная конструкция, представляющая собой множество \textit{sc-элементов}}
	\scnsuperset{sc-текст}
	\begin{scnindent}
		\scnidtf{текст SC-кода}
		\scnidtf{\textit{sc-конструкция}, являющаяся семантически корректной по отношению к \textit{Базовой денотационной семантике SC-кода}}
		\scnidtf{\textit{sc-конструкция}, удовлетворяющая (соответствующая) правилам \textit{Базовой денотационной семантики SC-кода}}
		\scnidtftext{часто используемый sc-идентификатор}{\textit{SC-код}}
		\begin{scnindent}
			\scniselement{имя собственное}
			\scnidtf{Класс (Множество всевозможных) sc-текстов}
		\end{scnindent}
		\scnsuperset{sc-знание}
	\end{scnindent} 
	\end{scnindent}

	\scnheader{sc-знание}
	\scnidtf{\textit{sc-текст}, являющийся либо фрагментом (подструктурой) соответствующей \textit{предметной области}, либо \textit{высказыванием}, описывающим некоторое свойство (в частности, некоторую закономерность) этой \textit{предметной области}}
	\scnidtf{знание, представленное в \textit{SC-коде}}
	\scnidtf{\textit{sc-текст}, обладающий истинным значением по отношению к соответствующей \textit{предметной области}}
	\scnsubset{связная sc-конструкция}
	\scnnote{Разные \textit{sc-знания} могут противоречить друг другу, то есть отражать разные точки зрения на некоторую \textit{предметную область}, но любое \textit{sc-знание} должно быть \textit{sc-текстом}, то есть не должно противоречить правилам \textit{Базовой денотационной семантики SC-кода}.}

	\scnheader{интеграция sc-конструкций*}
	\scnidtf{объединение sc-конструкций*}
	\scnidtf{объединение sc-множеств*}
	\scnnote{При интеграции sc-конструкций sc-элементы, обозначающие одну и ту же сущность, то есть синонимичные sc-элементы, считаются одинаковыми (совпадающими, тождественными) и, следовательно, должны склеиваться (отождествляться).}

	\scnheader{SC-пространство}
	\scnidtf{Результат интеграции \textit{всевозможных} sc-конструкций, \textit{семантически корректных} по отношению к \textit{Базовой денотационной семантики SC-кода}}
	\scnidtf{Предметная область, специфицируемая (описываемая) \textit{Базовой денотационной семантикой SC-кода}, которая является формальной онтологией, представленной средствами SC-кода}
	\scnidtf{Результат интеграции всевозможных sc-текстов (текстов SC-кода)}
	\scnidtf{Максимальный sc-текст}
	\scnidtf{Текст SC-кода, включающий в себя всевозможные sc-тексты}
	\scnidtf{Пространство sc-конструкций, семантически корректных по отношению к \textit{Базовой денотационной семантике SC-кода}}
	\begin{scnrelfromlist}{примечание}
		\scnfileitem{Особенностью \textit{SC-пространство} является то, что оно включает в себя и формальную онтологию, описывающую его свойства.}
		\scnfileitem{очевидно, что \textit{SC-пространство} является \textit{бесконечным} \textit{sc-текстом}, то есть текстом, содержащим бесконечное количество \textit{sc-элементов}. В частности, в состав \textit{SC-пространства} входят \textit{все} \textit{sc-элементы}, являющиеся элементами \textit{всех} \textit{sc-множеств}, знаки которых входят в состав \textit{SC-пространства}.}
		\scnfileitem{\textit{SC-пространство} является "вместилищем"{} семантически корректных (по отношению к \textit{Базовой денотационной семантике SC-кода}) частей баз знаний всевозможных ostis-систем и, в том числе, глобальной (объединенной) \textit{Базы знаний Экосистемы OSTIS}. Подчеркнем при этом, что \textit{Экосистема OSTIS} является примером распределенных иерархических \textit{ostis-систем}.}
		\scnfileitem{Тот факт, что корректная (с точки зрения \textit{Базовой денотационной семантики SC-кода}) часть базы знаний \textit{каждой} \textit{ostis-системы} входит в состав \textit{SC-пространства}, позволяет трактовать описание соотношения между текущим состоянием \textit{базы знаний ostis-системы} и \textit{Sc-пространством} как описание того, что указанная \textit{ostis-система} в текущий момент времени не знает. Например, \textit{ostis-система} в некоторый момент времени может не знать (1) всех элементов некоторого конкретного \textit{конечного} \textit{sc-множества} (конечно sc-конструкции), (2) количества элементов указанного конечного \textit{sc-множества}, (3) какому подклассу заданного \textit{sc-класса} принадлежит указанный элемент этого \textit{sc-класса}.}
		\scnfileitem{В \textit{памяти ostis-системы} каждый \textit{sc-элемент} считается в рамках этой памяти \textit{временной} сущностью (имеется в виду сам \textit{sc-элемент}, а не обозначаемая им сущность), поскольку он появляется в \textit{памяти ostis-системы} и удаляется из нее независимо от того, что он обозначает. В отличие от этого в \textit{SC-пространстве} все sc-элементы считаются постоянными (\textit{постоянно} присутствующими) в рамках этого пространства.}
	\end{scnrelfromlist}

	\scnheader{Базовая денотационная семантика SC-кода}
	\scnidtf{Онтология Базовой денотационной семантики SC-кода}
	\scnidtf{Формальная \textit{онтология}, представленная в \textit{SC-коде} и являющаяся материнской \textit{онтологией} (онтологией самого высокого уровня) для всех остальных \textit{формальных онтологий}, представленных в \textit{SC-коде}}
	\scnidtf{Онтология SC-пространства}
	\scnidtf{Описание (представление) системы \textit{правил построения семантически корректируемых sc-конструкций}, удовлетворяющих требованиям Базовой денотационной семантики SC-кода}
	\scniselement{sc-онтология}
	\begin{scnindent}
		\scnidtf{формальная онтология, представленная в SC-коде}
	\end{scnindent} 
	\scnsubset{\textbf{Семантическая классификация sc-элементов по базовым признакам}}
	\scnsubset{\textbf{Уточнение смысла выделенных классов sc-элементов и связей между этими классами}}
	\scnsubset{\textbf{Структура базовой семантической спецификации sc-элемента}}

	\scnheader{Логическая онтология SC-пространства}
	\scnrelto{логическая онтология}{Базовая денотационная семантика SC-пространства}
	\begin{scnrelfromset}{Правила, входящие в состав Логической онтологии SC-пространства}
		\scnfileitem{Вторыми компонентами \textit{sc-пар} константной парой принадлежности могут быть sc-элементы \textit{любого} типа(в том числе, и \textit{sc-переменные}), но первыми компонентами таких \textit{sc-пар} могут быть только \textit{константные} \textit{sc-множества}.}
		\scnfileitem{Знак \textit{sc-ситуации} связан с элементами этой ситуации \textit{sc-парами} константной \textit{постоянной} позитивной принадлежности. То есть позитивная принадлежность считается постоянной в рамках интервала времени существования соответствующей ситуации. В этом смысле ситуацию можно считать квазистатической.}
		\scnfileitem{Знак атомарной логической формулы связан со всеми элементами этой формулы \textit{sc-парами} \textit{константной} постоянной позитивной принадлежности, в том числе, и с теми элементами атомарной формулы, которые являются \textit{sc-переменными}.}
		\scnfileitem{Из переменного \textit{sc-множества} могут выходить только переменные \textit{sc-пары принадлежности}
		\item Не существует sc-пар принадлежности выходящих из обозначений внешних сущностей и \textit{sc-пар}.}
		\end{scnrelfromset}
	

\end{scnsubstruct}

\scnendstruct \scnendcurrentsectioncomment


\scsectionfamily{Часть 2 Стандарта OSTIS. Смысловое представление и онтологическая систематизация знаний в интеллектуальных компьютерных системах нового поколения}
\label{part_representation}


\scsection{Введение в описание внутреннего языка ostis-систем и близких ему внешних языков, используемых для представления исходных текстов баз знаний}

\begin{SCn}

\scnsectionheader{\currentname}

\end{SCn}


\scsubsection{Введение в описание внутреннего языка ostis-систем}
\label{intro_sc_code}

\begin{SCn}

\scnsectionheader{\currentname}

\scnstartsubstruct

\scnsegmentheader{Первый сегмент Введения описание внутреннего языка ostis-систем}
\scnstartsubstruct

\scnheader{\currentname}
\scnreltovector{конкатенация сегментов}{Первый сегмент Введения описание внутреннего языка ostis-систем;Описание Ядра SC-кода;Описание Расширения Ядра SC-кода}

\scnheader{SC-код}
\scnidtf{Внутренний язык ostis-систем}
\scnidtf{Множество sc-текстов}
\scnidtf{sc-текст}
\scnidtf{Множество sc-конструкций}
\scnidtf{Язык унифицированного смыслового представления знаний в памяти интеллектуальных компьютерных систем}
\filemodetrue
\scnrelfromvector{принципы, лежащие в основе}{Знаки (обозначения) всех сущностей, описываемых в \textit{sc-текстах} (текстах SC-кода) представляются в виде синтаксически элементарных (атомарных) фрагментов \textit{sc-текстов} и, следовательно, не имеющих внутренней структуры, не состоящих из более простых фрагментов текста, как, например, имена (термины), которые представляют знаки описываемых сущностей в привычных языках и состоят из букв.;Имена (термины), естественно-языковые тексты и другие информационные конструкции, не являющиеся \textit{sc-текстами}, могут входить в состав \textit{sc-текста}, но только как файлы, описываемые (специфицируемые) \textit{sc-текстами}. Таким образом, в состав базы знаний \textit{интеллектуальной компьютерной системы}, построенной на основе \textit{SC-кода}, могут входить имена (термины), обозначающие некоторые описываемые сущности и представленные соответствующими файлами. Каждый sc-элемент будем называть внутренним обозначением некоторой сущности, а имя этой сущности, представленное соответствующим файлом, будем называть внешним идентификатором (внешним обозначением) этой сущности. При этом каждый именуемый (идентифицируемый) \textit{sc-элемент} связывается дугой, принадлежащей отношению "\textit{\textbf{быть внешним идентификатором*}}, с узлом, содержимым которого является файл идентификатора (в частности, имени), обозначающего ту же сущность, что и указанный выше \textit{sc-элемент}. Внешним обозначением может быть не только имя (термин), но и иероглиф, пиктограмма, озвученное имя, жест. Особо отметим, что внешние обозначения описываемых сущностей в интеллектуальной компьютерной системе, построенной на основе \textit{SC-кода}, используются только (1) для анализа информации, поступающей в эту систему из вне из различных источников, и ввода (понимания и погружения) этой информации в базу знаний, а также (2) для синтеза различных сообщений, адресуемых различным субъектам (в т.ч. пользователям).;Тексты \textit{SC-кода} (sc-тексты) имеют в общем случае нелинейную (графовую) структуру, поскольку (1) знак каждой описываемой сущности в ходит в состав sc-текста однократно и (2) каждый такой знак может быть инцидентен неограниченному числу других знаков, поскольку каждая описываемая сущность может быть связана неограниченным числом связей с другими описываемыми сущностями.;
База знаний, представленная текстом \textit{SC-кода}, является графовой структурой специального вида, алфавит элементов которой включает в себя множество узлов, множество ребер, множество дуг, множество базовых дуг -- дуг специально выделенного типа, обеспечивающих структуризацию баз знаний, а также множество специальных узлов, каждый из которых имеет содержимое, являющееся файлом, хранящимся в памяти интеллектуальной компьютерной системы. Структурная особенность данной графовой структуры заключается в том, что ее дуги и ребра могут связывать не только узел с узлом, но и узел с ребром или дугой, ребро или дугу с другим ребром или дугой.;
\uline{Все элементы} указанной выше графовой структуры (текста SC-кода), т.е. все ее узлы, ребра и дуги являются обозначениями различных сущностей. При этом ребро является обозначением бинарной неориентированной связки между двумя сущностями, каждая из которых либо представлена в рассматриваемой графовой структуре соответствующим знаком, либо является самим этим знаком. Дуга является обозначением бинарной ориентированной связки между двумя сущностями. Дуга специального вида (\textit{\textbf{базовая дуга}}) является знаком связи между узлом, обозначающим некоторое множество элементов рассматриваемой графовой структуры, и одним из элементов этой графовой структуры, который принадлежит указанному множеству. Узел, имеющий содержимое (узел, для которого содержимое существует, но может в текущий момент быть неизвестным) является знаком файла, который является содержимым этого узла. Узел, не являющийся знаком файла, может обозначать какой-либо материальный объект, первичный абстрактный объект(например, число, точку в некотором абстрактном пространстве), какую-либо бинарную связь, какое-либо множество (в частности, понятие, структуру, ситуацию, событие, процесс). При этом сущности, обозначаемые элементами рассматриваемой графовой структуры, могут быть постоянными (существующими всегда) и временными (сущностями, которым соответствует отрезок времени их существования). Кроме того, сущности, обозначаемые элементами рассматриваемой графовой структуры, могут быть константными (конкретными) сущностями и переменными (произвольными) сущностями. Каждому элементу рассматриваемой графовой структуры, являющемуся обозначением переменной сущности, ставится в соответствие область возможных значений этого обозначения. Область возможных значений каждого переменного ребра является подмножеством множества всевозможных константных ребер, область возможных значений каждой переменной дуги является подмножеством множества всевозможных константных дуг, область возможных значений каждого переменного узла является подмножеством множества всевозможных константных узлов.;
В рассматриваемой графовой структуре, являющейся представлением базы знаний в SC-коде, могут, но не должны существовать разные элементы графовой структуры, обозначающие одну и ту же сущность. Если пара таких элементов обнаруживается, то эти элементы склеиваются (отождествляются). Таким образом, синонимия внутренних обозначений в базе знаний интеллектуальной компьютерной системы, построенной на основе \textit{SC-кода,} запрещена. При этом синонимия внешних обозначений считается нормальным явлением. Формально это означает, что из некоторых элементов рассматриваемой графовой структуры выходит несколько дуг, принадлежащих отношению "\textit{\textbf{быть внешним идентификатором*}}". Из всех указанных дуг, принадлежащих отношению "\textit{\textbf{быть внешним идентификатором*}}" и выходящих из одного элемента рассматриваемой графовой структуры, обязательно выделяется одна (очень редко две) путем включения их в число дуг, принадлежащих отношению "\textit{\textbf{быть основным внешним идентификатором*}}". Это означает, что указываемый таким образом внешний идентификатор не является омонимичным, т.е. не может быть использован как внешний идентификатор, соответствующий другомуэлементу рассматриваемой графовой структуры.;
Кроме файлов, представляющих различные внешние обозначения (имена, иероглифы, пиктограммы), в памяти интеллектуальной компьютерной системе, построенной на основе \textit{SC-кода,} могут хранится файлы различных текстов (книг, статей, документов, примечаний, комментариев, пояснений, чертежей, рисунков, схем, фотографий, видео-материалов, аудио-материалов).;
\uline{Любую сущность}, требующую описания, можно обозначить в виде sc-элемента. Особо подчеркнем, что sc-элементы являются не просто обозначениями различных описываемых сущностей, а обозначениями, которые являются элементарными (атомарными) фрагментами знаковой конструкции, т.е. фрагментами, детализация структуры которых не требуется для "прочтения" и понимания этой знаковой конструкции.;
Текст \textit{\textbf{SC-кода}}, как и любая другая графовой структура, является абстрактным математическим объектом, не требующим детализации (уточнения) его кодирования в памяти компьютерной системы (например, в виде матрицы смежности, матрицы инцидентности, списковой структуры). Но такая детализация потребуется для технической реализации памяти, в которой хранятся и обрабатываются sc-тексты.;
Важнейшим дополнительным свойством \textit{\textbf{SC-кода}} является то,что он удобен не просто для внутреннего представления знаний в памяти интеллектуальной компьютерной системы, но также и для внутреннего представления информации в памяти компьютеров, специально предназначенных для интерпретации семантических моделей интеллектуальных компьютерных систем. Т.е., SC-код определяет синтаксические, семантические и функциональные принципы организации памяти компьютеров нового поколения, ориентированных на реализацию интеллектуальных компьютерных систем, -- принципы организации графодинамической ассоциативной семантической памяти.;
SC-код рассматривается нами как объединение нескольких его подъязыков, в число которых входит ядро SC-кода и его расширение, обеспечивающее ввод и вывод информации для ostis-системы на всевозможных внешних языках.
}
\filemodefalse
\scnaddlevel{1}
\scnsourcecomment{Завершили описание принципов SC-кода}
\scnaddlevel{-1}

\scnendstruct

\scnsegmentheader{Описание Ядра SC-кода}
\scnstartsubstruct

\scnheader{Ядро SC-кода}
\scnrelfrom{алфавит}{Алфавит Ядра SC-кода}
\scnaddlevel{1}
\scnhaselement{sc-узел}
\scnhaselement{sc-ребро}
 \scnaddlevel{1}
 \scnidtf{обозначение бинарной неориентированной связи между sc-элементами}
 \scnaddlevel{-1}
\scnhaselement{sc-дуга}
\scnaddlevel{1}
 \scnidtf{обозначение бинарной ориентированной связи между sc-элементами}
 \scnaddlevel{-1}
\scnhaselement{базовая sc-дуга}
 \scnaddlevel{1}
 \scnidtf{sc-дуга константной позитивной стационарной принадлежности}
 \scnidtf{знак константной позитивной стационарной пары принадлежности}
 \scnaddlevel{-1}
\scnnote{Подчеркнем, что с помощью указанных типов sc-элементов можно описать любые связи между sc-элементами, трактуя эти связи как множества связываемых sc-элементов и используя некоторые sc-узлы как знаки этих множеств.}
\scnaddlevel{-1}

\scnendstruct

\scnsegmentheader{Описание Расширения Ядра SC-кода}
\scnstartsubstruct

\scnheader{SC-код}
\scnidtf{Расширение Ядра SC-кода}
\scnidtf{Результат введения в Ядро SC-кода sc-узлов, имеющих содержимое и обозначающих файлы, хранимые в памяти ostis-системы}
\scnnote{Все файлы, представляющие собой электронные образы инородных для SC-кода информационных конструкций, можно представить в SC-кода с помощью графовых структур, в которых sc-элементы обозначают буквы текстов или пиксели изображений. Но такой вариант кодирования внешних для ostis-системы информационных конструкций не дает возможности непосредственно использовать накопленный человечеством арсенал электронных информационных ресурсов.}
\scnnote{Важнейшим видом файлов ostis-систем являются внешние идентификаторы sc-элементов (в частности, имена sc-элементов), представляющие sc-элементы в текстах внешних языков (в том числе, текстах SCs-кода и SCn-кода)} 
\scnnote{Результатом просмотренного расширения \textit{Ядра SC-кода} является расширение \textit{Алфавита Ядра SC-кода}}

\scnheader{SC-код}
\scnrelfrom{алфавит}{Алфавит SC-кода}
\scnaddlevel{1}
\scnhaselement{sc-узел}
\scnhaselement{sc-ребро}
\scnhaselement{sc-дуга}
\scnhaselement{базовая sc-дуга}
\scnhaselement{файл ostis-системы}
\scnaddlevel{-1}

\scnheader{файл ostis-системы}
\scnidtf{sc-узел с содержимым}
\scnidtf{sc-узел, имеющий содержимое}
\scnidtf{sc-узел, обозначающий файл, хранимый в памяти ostis-системы}
\scnidtf{знак файла ostis-системы}
\scnreltoset{разбиение}{ея-файл ostis-системы\\
\scnaddlevel{1}
\scnidtf{естественно-языковой файл ostis-системы}
\scnaddlevel{-1};файл ostis-системы, являющийся текстом формального языка\\
\scnaddlevel{1}
\scnsuperset{sc.g-файл ostis-системы}
\scnsuperset{sc.s-файл ostis-системы}
\scnsuperset{sc.n-файл ostis-системы}
\scnaddlevel{-1};файл ostis-системы, отражающий процесс изменения sc.g-текста;графический файл ostis-системы;файл ostis-системы, являющийся изображением;видео-файл ostis-системы;аудио-файл ostis-системы}
\scnreltoset{разбиение}{файл-экземпляр ostis-системы
\scnaddlevel{1}
\scnidtf{файл, являющийся конкретным электронным документом или электронным образом конкретной внешней информационной конструкции}
\scnaddlevel{-1};файл-класс ostis-системы
\scnaddlevel{1}
\scnidtf{файл, являющийся знаком множества всевозможных экземпляров (копий) этого файла}
\scnaddlevel{-1}
}

\scnheader{SC-код}
\scnrelfrom{синтаксис}{Cинтаксис SC-кода} 
\scnaddlevel{1}
\scnexplanation{\textit{\textbf{Синтаксис}} \textit{\textbf{SC-кода}} задается
\begin{scnitemize}
\item типологией (алфавитом) sc-элементов (атомарных фрагментов текстов sc-кода);
\item правилами соединения (инцидентности) sc-элементов (например, sc-элементы каких типов не могут быть инцидентными друг другу);
\item типологией конфигураций sc-элементов (связки, классы, структуры), связями между конфигурациями sc-элементов (в частности, теоретико-множественными)
\end{scnitemize}
}
\scnaddlevel{-1}
\scnrelfrom{денотационная семантика}{Денотационная семантика SC-кода} 
\scnaddlevel{1}
\scnexplanation{\textit{\textbf{Денотационная семантика}} \textit{\textbf{SC-кода}} задается
\begin{scnitemize}
\item
 семантической интерпретацией sc-элементов и их конфигураций;
\item
 семантической интерпретацией инцидентности sc-элементов;
\item
 иерархической системой предметных областей;
\item
 структурой используемых понятий в каждой предметной области (исследуемые классы объектов, исследуемые отношения, исследуемые классы объектов отношений из смежных предметных областей, ключевые экземпляры исследуемых классов объектов);
\item
 онтологиями предметных областей.
\end{scnitemize}
}
\scnaddlevel{-1}
\scnnote{Следует особо подчеркнуть, что  унификация и максимально возможное упрощение  \textbf{\textit{синтаксиса}} и \textbf{\textit{денотационной семантики}} внутреннего языка интеллектуальных компьютерных систем необходимы потому, что подавляющий объем \textbf{\textit{знаний}}, хранимых в составе  базы знаний интеллектуальной компьютерной системы, представляют собой \textbf{\textit{метазнания}}, описывающими свойства других знаний. Более того, по указанной причине конструктивное (формальное) развитие теории интеллектуальных компьютерных систем невозможно без уточнения (унификации, стандартизации) и обеспечения семантической совместимости различных видов знаний, хранимых в базе знаний интеллектуальной компьютерной  системы.  Очевидно, что многообразие форм представления семантически эквивалентных знаний делает разработку общей теории  интеллектуальных компьютерных систем практически невозможной. К \textit{метазнаниям}, в частности, следует отнести и различного вида логические высказывания и всевозможного вида программы, описания методов (навыков). Обеспечивающих решение различных классов информационных задач.}

\scnendstruct~
\scnsourcecomment{Завершили сегмент "Описание расширения Ядра SC-кода"}

\scnendstruct~
\scnsourcecomment{Завершили раздел "\currentname"}

\end{SCn}


\scsubsection{Неформальное введение в язык визуального представления баз знаний ostis-систем}

\begin{SCn}

\scnsectionheader{\currentname}

\end{SCn}


\scsubsection{Неформальное введение в язык гипертекстового представления баз знаний ostis-систем}

\begin{SCn}

\scnsectionheader{\currentname}

\end{SCn}


\scsection[\scneditor{Банцевич К.А.}\protect\scnmonographychapter{Глава 2.3. Структура баз знаний интеллектуальных компьютерных систем нового поколения: иерархическая система предметных областей и онтологий. Онтологии верхнего уровня. Формализация понятий семантической окрестности, предметной области и онтологии в интеллектуальных компьютерных системах нового поколения}]{Предметная область и онтология знаний и баз знаний ostis-систем}
\label{sd_knowledge}
\begin{SCn}

\scnsectionheader{\currentname}

\scnstartsubstruct

\scnrelfromlist{дочерний раздел}{Предметная область и онтология множеств
    \scnaddlevel{1}
    \scnidtf{Предметная область и онтология \textit{знаний о множествах}}
        \scnaddlevel{1}
        \scnnote{\textit{знания о множествах} являются \uline{частным видом} \textit{знаний} и, следовательно, общие свойства сущностей, описываемых знаниями, могут наследоваться \textit{Предметной областью и онтологией множеств}}
        \scnaddlevel{-1}
    \scnaddlevel{-1}
;Предметная область и онтология связок и отношений
;Предметная область и онтология параметров, величин и шкал
;Предметная область и онтология чисел и числовых структур
;Предметная область и онтология структур
;Предметная область и онтология темпоральных сущностей
;Предметная область и онтология темпоральных сущностей баз знаний ostis-систем
;Предметная область и онтология семантических окрестностей
;Предметная область и онтология предметных областей
;Предметная область и онтология онтологий
;Предметная область и онтология логических формул, высказываний и формальных теорий
;Предметная область и онтология внешних информационных конструкций и файлов ostis-систем
;Глобальная предметная область действий и задач и соответствующая ей онтология методов и технологий}

\scnheader{Предметная область знаний и баз знаний ostis-систем}
\scniselement{предметная область}
\scnsdmainclasssingle{знание}
\scnhaselementlist{исследуемый класс классов}{вид знаний;отношение, заданное на множестве знаний}

\scnheader{знание}
\scnidtf{синтаксически корректная (для соответствующего языка) и семантически целостная информационная конструкция}
\scnsubset{информационная конструкция}
    \scnaddlevel{1}
    \scniselementrole{класс объектов исследования}{\nameref{intro_lang}}
    \scnaddlevel{-1}
\scnaddlevel{1}
\scnrelboth{следует отличать}{данные}
\scnaddlevel{1}
\scnexplanation{
	Принципиальные различия знаний и данных:
	
	\begin{scnitemize}
		\item \textit{Интерпретация}. Хранимые данные могут быть интерпретированы только человеком или программой. Данные не несут информации. Знания содержат как данные, так и их описание (правила интерпретации)
		\item \textit{Наличие связей классификации}. Данные не имеют эффективного описания связей между различными типами данных. Знания структурированы, так как можно установить соответствие между единицами знаний.
		\item \textit{Наличие ситуационных связей}. Связи описывают множество текущих ситуаций объекта. Данные трудно поддаются анализу. Из структуры и состава знаний по ситуации возможно построение процедур анализа знаний.
	\end{scnitemize}
}
\scnaddlevel{1}
\scnrelfrom{цитата}{Helpiks2015}
\scnaddlevel{-1}
\scnaddlevel{-1}
\scnaddlevel{-1}
\scnrelfrom{покрытие}{вид знаний
    \scnidtf{Множество \uline{всевозможных} видов знаний}
    \scnnote{Тот факт, что семейство \textit{видов знаний} является \textit{покрытием} Множества всевозможных \textit{знаний}, означает то, что каждое \textit{знание} принадлежит по крайней мере одному выделенному нами \textit{виду знаний}}}

    
    
   \scnheader{вид знаний}
\scnhaselement{спецификация}
    \scnaddlevel{1}
    \scnidtf{описание заданной сущности}
    \scnsuperset{спецификация материальной сущности}
    \scnsuperset{спецификация обратной сущности, не являющейся множеством}
        \scnaddlevel{1}
        \scnsuperset{спецификация геометрической точки}
        \scnsuperset{спецификация числа}
        \scnaddlevel{-1}
    \scnsuperset{спецификация множества}
        \scnaddlevel{1}
        \scnsuperset{спецификация связи}
        \scnsuperset{спецификация структуры}
        \scnsuperset{спецификация класса}
            \scnaddlevel{1}
            \scnsuperset{спецификация класса сущностей, не являющихся множествами}
            \scnsuperset{спецификация отношения}
                \scnaddlevel{1}
                \scnidtf{спецификация класса связей (связок)}
                \scnaddlevel{-1}
            \scnsuperset{спецификация класса классов}
                \scnaddlevel{1}
                \scnsuperset{спецификация параметра}
                \scnaddlevel{-1}
            \scnsuperset{спецификация класса структур}
            \scnsuperset{спецификация понятий}
                \scnaddlevel{1}
                \scnsuperset{пояснение}
                \scnsuperset{определение}
                \scnsuperset{утверждение}
                    \scnaddlevel{1}
                    \scnidtf{утверждение, описывающее свойства экземпляров (элементов) специфицируемого понятия}
                    \scnidtf{закономерность}
                    \scnaddlevel{-1}
                \scnaddlevel{-1}
            \scnaddlevel{-1}
        \scnaddlevel{-1}
    \scnsuperset{семантическая окрестность}
    \scnsuperset{однозначная спецификация}
    \scnsuperset{сравнительный анализ}
    \scnsuperset{достоинства}
    \scnsuperset{недостатки}
    \scnsuperset{структура специфицируемой сущности}
    \scnsuperset{принципы, лежащие в основе}
    \scnsuperset{обоснование предлагаемого решения}
        \scnaddlevel{1}
        \scnidtf{аргументация предлагаемого решения}
        \scnaddlevel{-1}
    \scnaddlevel{-1}

\scnhaselement{сравнение}

\scnhaselement{высказывание}
    \scnaddlevel{1}
    \scnsuperset{фактографическое высказывание}
    \scnsuperset{закономерность}
    \scnaddlevel{-1}

\scnhaselement{формальная теория}

\scnhaselement{предметная область}

\scnhaselement{предметная область и онтология
    \scnaddlevel{1}
    \scnidtf{предметная область и её онтология}
    \scnidtf{предметная область и соответствующая ей объединенная онтология}
    \scnaddlevel{-1}} 
 
\scnhaselement{метазнание}
    \scnaddlevel{1}
    \scnidtf{спецификация знания}
    \scnsuperset{аннотация}
    \scnsuperset{введение}
    \scnsuperset{предисловие}
    \scnsuperset{заключение}
    \scnsuperset{онтология}
        \scnaddlevel{1}
        \scnsuperset{онтология предметной области}
            \scnaddlevel{1}
            \scnsuperset{структурная онтология предметной области}
            \scnsuperset{теоретико-множественная онтология предметной области}
            \scnsuperset{логическая онтология предметной области}
            \scnsuperset{терминологическая онтология предметной области}
            \scnsuperset{объединенная онтология предметной области}
            \scnaddlevel{-1}
        \scnaddlevel{-1}
    \scnaddlevel{-1}

\scnhaselement{задача}
    \scnaddlevel{1}
    \scnidtf{спецификация действия}
    \scnaddlevel{-1}
\scnhaselement{план}

\scnhaselement{протокол}

\scnhaselement{результативная часть протокола}

\scnhaselement{метод}

\scnhaselement{технология}

\scnhaselement{история использования предметной области и её онтологии по решению информационных задач}
\scnhaselement{история использования предметной области и её онтологии по решению задач во внешней среде}
\scnhaselement{история эволюции предметной области и её онтологии}

\scnhaselement{база знаний}
    \scnaddlevel{1}
    \scnidtf{совокупность знаний, хранимых в памяти интеллектуальной компьютерной системы и \uline{достаточных} для того, чтобы указанная система удовлетворяла соответствующим предъявляемым к ней требованиям (в частности, чтобы она имела соответствующий уровень интеллекта)}
    \scnidtf{систематизированная совокупность знаний, хранимая в памяти интеллектуальной компьютерной системы и достаточная для обеспечения целенаправленного (целесообразного, адекватного) функционирования (поведения) этой системы как в своей внешней среде, так и в своей внутренней среде (в собственной базе знаний)}
    \scnrelfromset{обобщенная декомпозиция}{согласованная часть базы знаний
        \scnaddlevel{1}
        \scnidtf{часть базы знаний, признанная коллективом авторов на текущий момент}
        \scnaddlevel{-1}
    ;история эксплуатации базы знаний;история эволюции базы знаний;план эволюции базы знаний
        \scnaddlevel{1}
        \scnidtf{система специфицированных и согласованных действий авторов базы знаний, направленных на повышение её качества}
        \scnaddlevel{-1}}
    \scnnote{Основным факторами, определяющими качество интеллектуальной компьютерной системы, являются:
    \begin{scnitemize}
        \item качественная структуризация (систематизация) и \uline{стратификация} базы знаний интеллектуальной компьютерной системы, а также
        \item систематизация и стратификация \uline{деятельности}, которая осуществляется интеллектуальной компьютерной системой и спецификация которой является важнейшей частью базы знаний этой системы (Смотрите Раздел \textit{Глобальная предметная область действий и задач и соответствующая ей онтология методов и технологий}).
    \end{scnitemize}}
    \scnaddlevel{-1}
\scnnote{Даже небольшой перечень \textit{видов знаний} свидетельствует об огромном многообразии \textit{видов знаний}}

\scnheader{знание}
\scnsubdividing{декларативное знание
    \scnaddlevel{1}
    \scnidtf{\textit{знание}, имеющее \uline{только} \textit{денотационную семантику}, которая представляется в виде семантической \textit{спецификации} системы \textit{понятий}, используемых в этом \textit{знании}}
    \scnaddlevel{-1}
;процедурное знание
    \scnaddlevel{1}
    \scnidtf{\textit{знание}, имеющее не только \textit{денотационную семантику}, но и \textit{операционную семантику}, которая представляется в виде семейства \textit{спецификаций агентов}, осуществляющих интерпретацию \textit{процедурного знания}, направленную на решение некоторой инициированной \textit{задачи}}
    \scnidtf{функционально интерпретируемое знание, обеспечивающее решение либо конкретной задачи, либо некоторого множества инициируемых задач}
    \scnsuperset{задача}
        \scnaddlevel{1}
        \scnidtf{формулировка конкретной задачи}
        \scnsuperset{декларативная формулировка задачи}
        \scnsuperset{процедурная формулировка задачи}
        \scnaddlevel{-1}
    \scnsuperset{план}
        \scnaddlevel{1}
        \scnidtf{план решения конкретной задачи}
        \scnidtf{контекст конкретной задачи, предоставляющий всю информацию для решения всех подзадач для указанной конкретной задачи}
        \scnidtf{описание системы подзадач некоторой задачи}
        \scnaddlevel{-1}
    \scnsuperset{метод}
        \scnaddlevel{1}
        \scnidtf{обобщенное описание плана решения любой задачи из некоторого заданного класса задач}
        \scnaddlevel{-1}
    \scnsuperset{навык}
        \scnaddlevel{1}
        \scnidtf{метод, детализированный до уровня элементарных подзадач}
        \scnaddlevel{-1}
    \scnaddlevel{-1}}
    
\scnheader{отношение, заданное на множестве знаний}
\scnhaselement{дочернее знание*}
    \scnaddlevel{1}
    \scnidtf{знание, которое от "материнского"{} знания наследует все описанные там свойства объектов исследования}
    \scnnote{Факт наследования свойств описываемых объектов от "материнского"{} знания подчеркивается использованием прилагательного "дочернее"{} в sc-идентификаторе данного отношения, заданного на множестве знаний}
    \scnsuperset{дочерний раздел*}
        \scnaddlevel{1}
        \scnidtf{частный раздел*}
        \scnaddlevel{-1}
    \scnsuperset{дочерняя предметная область и онтология*}
    \scnaddlevel{-1}
\scnhaselement{спецификация*}
    \scnaddlevel{1}
    \scnidtf{быть знанием, которое является спецификацией (описанием) заданной сущности}
    \scnnote{специфицируемой сущностью может быть сущность любого вида, в том числе, и другое знание}
    \scnaddlevel{-1}
\scnhaselement{онтология*}
    \scnaddlevel{1}
    \scnidtf{быть семантической спецификацией заданного знания*}
    \scnaddlevel{-1}
\scnhaselement{семантическая эквивалентность*}
\scnhaselement{следовательно*}
    \scnaddlevel{1}
    \scnidtf{логическое следствие*}
    \scnaddlevel{-1}
\scnhaselement{логическая эквивалентность*}   
    
\bigskip    
\scnendstruct \scnendcurrentsectioncomment

\end{SCn}

\scsubsection[\scnidtf{Предметная область и онтология знаний о множествах}\protect\scnmonographychapter{Глава 2.4. Формальные онтологии базовых классов сущностей - множеств, связей, отношений, параметров, величин, чисел, структур, темпоральных сущностей}]{Предметная область и онтология множеств}
\label{sd_sets}
\begin{SCn}

\scnsectionheader{\currentname}

\scnstartsubstruct

\scnheader{Предметная область множеств}
\scnidtf{Теоретико-множественная предметная область}
\scnidtf{Предметная область теории множеств}
\scnidtf{Предметная область, объектами исследования которой являются множества}
\scniselement{предметная область}
\scnsdmainclasssingle{множество}
\scnsdclass{конечное множество;бесконечное множество;счетное множество;несчетное множество;множество без кратных элементов;мультимножество;кратность принадлежности;класс;класс первичных sc-элементов;класс множеств;класс структур;класс классов;нечеткое множество;четкое множество;множество первичных сущностей;семейство множеств;нерефлексивное множество;рефлексивное множество;множество первичных сущностей и множеств;сформированное множество;несформированное множество;пустое множество;синглетон;пара;пара разных элементов;пара-мультимножество;тройка;кортеж;декартово произведение;булеан;мощность}
\scnsdrelation{принадлежность*;пример\scnrolesign;включение*;строгое включение*;объединение*;разбиение*;пересечение*;пара пересекающихся множеств*;попарно пересекающиеся множества*;пересекающиеся множества*;пара непересекающихся множеств*;попарно непересекающиеся множества*;непересекающиеся множества*;разность множеств*;симметрическая разность множеств*;декартово произведение*;семейство подмножеств*;булеан*;равенство множеств*}

\scnheader{множество}
\scnidtf{множество sc-элементов}
\scnidtf{sc-множество}
\scnidtf{множество знаков}
\scnidtf{множество знаков описываемых сущностей}
\scnidtf{семантически нормализованное множество}
\scnidtf{sc-знак множества sc-элементов}
\scnidtf{sc-знак множества sc-знаков}
\scnidtf{sc-текст}
\scnidtf{текст SC-кода}
\scnidtf{SC-код}
\scnsubdividing{конечное множество;бесконечное множество}
\scnsubdividing{множество без кратных элементов;мультимножество}
\scnsubdividing{связка;класс\\
    \scnaddlevel{1}
    \scnidtf{sc-элемент, обозначающий класс sc-элементов}
    \scnidtf{sc-знак множества sc-элементов, эквивалентных в том или ином смысле}
    \scnaddlevel{-1}
    ;структура\\
    \scnaddlevel{1}
    \scnidtf{sc-знак множества sc-элементов, в состав которого входят sc-связки или sc-структуры, связывающие эти sc-элементы}
    \scnaddlevel{-1}}
\scnsubdividing{четкое множество;нечеткое множество}
\scnsubdividing{множество первичных сущностей;множество множеств;множество первичных сущностей и множеств}
\scnsubdividing{рефлексивное множество;нерефлексивное множество}
\scnsubdividing{сформированное множество;несформированное множество}
\scnsubdividing{кортеж;неориентированное множество}
\scnsuperset{пустое множество}
\scnsuperset{синглетон}
\scnsuperset{пара}
\scnsuperset{тройка}
\scnexplanation{Под \textbf{\textit{множеством}} понимается соединение в некое целое M определённых хорошо различимых предметов m нашего созерцания или нашего мышления (которые будут называться «элементами» множества M). 

\textbf{\textit{множество}} – мысленная сущность, которая связывает одну или несколько сущностей в целое.

Более формально \textbf{\textit{множество}} – это абстрактный математический объект, состоящий из элементов. Связь множеств с их элементами задается бинарным ориентированным отношением \textbf{\textit{принадлежность*}}.

\textbf{\textit{множество}} может быть полностью задано следующими тремя способами:
\begin{scnitemize}
    \item путем перечисления (явного указания) всех его элементов (очевидно, что таким способом можно задать только конечное множество)
    \item с помощью определяющего высказывания, содержащего описание общего характеристического свойства, которым обладают все те и только те объекты, которые являются элементами (т.е. принадлежат) задаваемого множества.
    \item с помощью теоретико-множественных операций, позволяющих однозначно задавать новые множества на основе уже заданных (это операции объединения, пересечения, разности множеств и др.)
\end{scnitemize}
Для любого семантически ненормализованного \textbf{\textit{множества}} существует единственное семантически нормализованное \textbf{\textit{множество}}, в котором все элементы, не являющиеся знаками множеств, заменены на знаки множеств.}

\scnheader{принадлежность*}
\scnidtf{принадлежность элемента множеству*}
\scnidtf{отношение принадлежности элемента множеству*}
\scniselement{бинарное отношение}
\scniselement{ориентированное отношение}
\scnexplanation{\textbf{\textit{принадлежность*}} – это бинарное ориентированное отношение, каждая связка которого связывает множество с одним из его элементов. Элементами отношения \textbf{\textit{принадлежность*}} по умолчанию являются \textit{позитивные sc-дуги принадлежности}.}

\scnheader{конечное множество}
\scnidtf{множество с конечным числом элементов}
\scnexplanation{\textbf{\textit{конечное множество}} – это \textit{множество}, количество элементов которого конечно, т.е. существует неотрицательное целое число \textit{k}, равное количеству элементов этого множества.}

\scnheader{бесконечное множество}
\scnidtf{множество с бесконечным числом элементов}
\scnsubdividing{счетное множество;несчетное множество}
\scnexplanation{\textbf{\textit{бесконечное множество}} – это \textit{множество}, в котором для любого натурального числа \textit{n} найдётся конечное подмножество из \textit{n} элементов.}

\scnheader{счетное множество}
\scnexplanation{\textbf{\textit{счетное множество}} – это \textit{бесконечное множество}, для которого существует \textit{взаимно-однозначное соответствие} с натуральным рядом чисел.}

\scnheader{несчетное множество}
\scnidtf{континуальное множество}
\scnexplanation{\textbf{\textit{несчетное множество}} - это \textit{бесконечное множество}, элементы которого невозможно пронумеровать натуральными числами.}

\scnheader{множество без кратных элементов}
\scnidtf{классическое множество}
\scnidtf{канторовское множество}
\scnidtf{множество, состоящее из разных элементов}
\scnidtf{множество без кратного вхождения элементов}
\scnidtf{множество, все элементы которого входят в него однократно}
\scnidtf{множество, не имеющее кратного вхождения элементов}
\scnexplanation{\textbf{\textit{множество без кратных элементов}} - это \textit{множество}, для каждого элемента которого существует только одна пара принадлежности, выходящая из знака этого множества в указанный элемент.}

\scnheader{мультимножество}
\scnidtf{множество, имеющее кратные вхождения хотя бы одного элемента}
\scnidtf{множество, по крайней мере один элемент которого входит в его состав многократно}
\scnexplanation{\textbf{\textit{мультимножество}} - это \textit{множество}, для которого существует хотя бы одна кратная пара принадлежности, выходящая из знака этого множества.}

\scnheader{кратность принадлежности}
\scnidtf{кратность принадлежности элемента}
\scnidtf{кратность вхождения элемента во множество}
\scniselement{параметр}
\scnexplanation{\textbf{\textit{кратность принадлежности}} - \textit{параметр}, значением которого являются числовые величины, показывающие сколько раз входит тот или иной элемент в рассматриваемое множество. Элементами данного параметра являются классы \textit{позитивных sc-дуг принадлежности}, связывающих данное множество с элементом, кратность вхождения которого в данное множество мы хотим задать.

Таким образом, кратное вхождение элемента в мультимножество может быть задано как явным указанием \textit{позитивных sc-дуг принадлежности} этого элемента данному \textit{множеству}, так и «склеиванием» этих дуг в одну и включением ее в некоторый класс \textbf{\textit{кратности принадлежности}}.}
\scnrelfrom{описание примера}{
\scnfilescg{figures/sd_sets/multiplicityOfMembership.png}
}

\scnheader{класс}
\scnidtf{класс sc-элементов}
\scnsubdividing{класс первичных sc-элементов;класс множеств}
\scnexplanation{\textbf{\textit{класс}} – множество элементов, обладающих какими-либо явно указываемыми общими свойствами.}

\scnheader{класс первичных sc-элементов}
\scnexplanation{\textbf{\textit{класс первичных sc-элементов}} – класс, элементами которого являются только \textit{sc-элементы}, не являющиеся знаками множеств.}

\scnheader{класс множеств}
\scnsubdividing{отношение;класс структур;класс классов}
\scnexplanation{\textbf{\textit{класс множеств}} – класс, элементами которого являются только \textit{sc-элементы}, являющиеся знаками множеств.}

\scnheader{класс структур}
\scnexplanation{\textbf{\textit{класс структур}} – класс, элементами которого являются \textit{структуры}.}

\scnheader{класс классов}
\scnexplanation{\textbf{\textit{класс классов}} – класс, элементами которого являются \textit{классы}.}

\scnheader{нечеткое множество}
\scnexplanation{\textbf{\textit{нечеткое множество}} – это \textit{множество}, которое представляет собой совокупность элементов произвольной природы, относительно которых нельзя точно утверждать – обладают ли эти элементы некоторым характеристическим свойством, которое используется для задания этого нечеткого множества. Принадлежность элементов такому множеству указывается при помощи \textit{нечетких позитивных sc-дуг принадлежности}.}

\scnheader{четкое множество}
\scnexplanation{\textbf{\textit{четкое множество}} – это \textit{множество}, принадлежность элементов которому достоверна и указывается при помощи \textit{четких позитивных sc-дуг принадлежности}.}

\scnheader{множество первичных сущностей}
\scnsuperset{класс первичных сущностей}
\scnsubset{множество}
\scnexplanation{\textbf{\textit{множество первичных сущностей}} – это \textit{множество}, элементы которого не являются знаками множеств.}

\scnheader{семейство множеств}
\scnidtf{множество множеств}
\scnsuperset{класс классов}
\scnexplanation{\textbf{\textit{семейство множеств}} – это \textit{множество}, элементами которого являются знаки множеств.}

\scnheader{нерефлексивное множество}
\scnexplanation{\textbf{\textit{нерефлексивное множеств}} – это \textit{множество}, знак которого не является элементом этого множества}

\scnheader{рефлексивное множество}
\scnexplanation{\textbf{\textit{рефлексивное множеств}} – это \textit{множество}, знак которого является элементом этого множества}

\scnheader{множество первичных сущностей и множеств}
\scnexplanation{\textbf{\textit{множество первичных сущностей и множеств}} – это \textit{множество}, элементами которого являются как знаки множеств, так и знаки сущностей, не являющихся множествами.}

\scnheader{сформированное множество}
\scniselement{ситуативное множество}
\scnexplanation{\textbf{\textit{сформированное множество }} - это \textit{множество}, все элементы которого известны и перечислены в данный момент.}

\scnheader{несформированное множество}
\scniselement{ситуативное множество}
\scnexplanation{\textbf{\textit{несформированное множество}} - это \textit{множество}, не все элементы которого известны и перечислены в данный момент.}

\scnheader{пустое множество}
\scniselement{мощность}
\scnexplanation{\textbf{\textit{пустое множество}} – это \textit{множество}, которому не принадлежит ни один элемент.}

\scnheader{синглетон}
\scniselement{мощность}
\scnidtf{множество мощности 1}
\scnidtf{одноэлементное множество}
\scnidtf{одномощное множество}
\scnidtf{множество, мощность которого равна 1}
\scnidtf{множество, имеющее мощность равную единице}
\scnidtf{синглетон из sc-элемента} 
\scnidtf{sc-синглеон}
\scnsubset{конечное множество}
\scnexplanation{\textbf{\textit{синглетон}} – это \textit{множество}, состоящее из одного элемента.

Другими словами - любое множество \textit{Si} есть \textbf{\textit{синглетон}} тогда и только тогда, когда существует принадлежность этому множеству, которая совпадает с любой принадлежностью этому множеству.}

\scnheader{пара}
\scniselement{мощность}
\scnidtf{множество мощности два}
\scnidtf{двухэлементное множество}
\scnidtf{двумощное множество}
\scnidtf{множество, мощность которого равна 2}
\scnidtf{sc-пара}
\scnidtf{пара sc-элементов}
\scnsubset{конечное множество}
\scnsubdividing{пара разных элементов;пара-мультимножество}
\scnexplanation{\textbf{\textit{пара}} – это \textit{множество}, состоящее из двух элементов.

Другими словами – любое множество есть \textbf{\textit{пара}} тогда и только тогда, когда существуют две различные принадлежности этому множеству такие, что любая принадлежность этому множеству совпадает с одной из них.}

\scnheader{пара разных элементов}
\scnidtf{канторовская пара}
\scnidtf{канторовская пара sc-элементов}
\scnidtf{канторовское двумощное множество}

\scnheader{пара-мультимножество}
\scnidtf{пара-петля}
\scnidtf{sc-петля}
\scnidtf{двумощное мультимножество}

\scnheader{тройка}
\scniselement{мощность}
\scnidtf{тройка}
\scnidtf{sc-тройка}
\scnidtf{множество мощности три}
\scnidtf{множество, мощность которого равна 3}
\scnsubset{конечное множество}
\scnexplanation{\textbf{\textit{тройка}} – это \textit{множество}, состоящее из трех элементов.

Другими словами – любое множество есть \textbf{\textit{тройка}} тогда и только тогда, когда существуют три различные принадлежности этому множеству такие, что любая принадлежность этому множеству совпадает с одной из них.}

\scnheader{кортеж}
%\scnidtf{кортеж}
\scnidtf{вектор}
\scnexplanation{\textbf{\textit{кортеж}} – это множество, представляющее собой упорядоченный набор элементов, т.е. такое множество, порядок элементов в котором имеет значение. Пары принадлежности элементов \textbf{\textit{кортежу}} могут дополнительно принадлежать каким-либо \textit{ролевым отношениям}, при этом, в рамках каждого \textbf{\textit{кортежа}} должен существовать хотя бы один элемент, роль которого дополнительно уточнена \textit{ролевым отношением}.}

\scnheader{пример\scnrolesign}
\scnidtf{типичный пример\scnrolesign}
\scnidtf{типичный экземпляр заданного класса\scnrolesign}
\scniselement{ролевое отношение}
\scnexplanation{\textbf{\textit{пример\scnrolesign}} – это \textit{ролевое отношение}, связывающее некоторое \textit{множество} с элементом, являющимся примером этого множества.}

\scnheader{включение*}
\scnidtf{включение множеств*}
\scnidtf{быть подмножеством*}
\scniselement{бинарное отношение}
\scniselement{ориентированное отношение}
\scniselement{транзитивное отношение}
\scnrelfrom{область определения}{множество}
\scnsuperset{строгое включение*}
\scntext{определение}{\textbf{\textit{включение*}} – это бинарное ориентированное отношение, каждая связка которого связывает два множества. Будем говорить, что \textit{Множество Si} \textbf{\textit{включает*}} в себя \textit{Множество Sj} в том и только том случае, если каждый элемент \textit{Множества Sj} является также и элементом \textit{Множества Si}.}
\scnrelfrom{описание примера}{
\scnfilescg{figures/sd_sets/inclusion.png}}
\scnaddlevel{1}
\scnexplanation{Множество {Sj} включается во множество \textit{Si}.}
\scnaddlevel{-1}

\scnheader{строгое включение*}
\scnidtf{строгое включение множеств*}
\scnsubset{включение*}
\scniselement{бинарное отношение}
\scniselement{ориентированное отношение}
\scnrelfrom{область определения}{множество}
\scntext{определение}{\textbf{\textit{строгое включение*}} – это \textit{бинарное ориентированное отношение}, областью определения которого является семейство всевозможных множеств. Будем говорить, что \textit{Множество Si} \textbf{\textit{строго включает*}} в себя \textit{Множество Sj} в том и только том случае, если каждый элемент \textit{Множество Sj} является также и элементом \textit{Множество Si}, при этом существует хотя бы один элемент \textit{Множество Si}, не являющийся элементом \textit{Множество Sj}.}
\scnrelfrom{описание примера}{
\scnfilescg{figures/sd_sets/strictInclusion.png}}
\scnaddlevel{1}
\scnexplanation{Множество \textit{Sj} строго включается во множество \textit{Si}.}
\scnaddlevel{-1}
\scnrelfrom{изображение}{
\scnfileimage{\includegraphics[width=0.4\linewidth]{figures/sd_sets/inclusion2.png}}}

\scnheader{объединение*}
\scnidtf{объединение множеств*}
\scniselement{квазибинарное отношение}
\scniselement{ориентированное отношение}
\scntext{определение}{\textbf{\textit{объединение*}} – это \textit{квазибинарное ориентированное отношение}, областью определения которого является семейство всевозможных множеств. Будем говорить, что \textit{Множество Si} является объединением \textit{Множество Sj} и \textit{Множество Sk} тогда и только тогда, когда любой элемент \textit{Множество Si} является элементом или \textit{Множество Sj} или \textit{Множество Sk}.}
\scnrelfrom{описание примера}{
\scnfilescg {figures/sd_sets/union.png}}
\scnaddlevel{1}
\scnexplanation{Множество \textit{Si} является объединением множеств \textit{Sj}, \textit{Sk} и \textit{Sm}.}
\scnaddlevel{-1}
\scnrelfrom{изображение}{
\scnfileimage{\includegraphics[width=0.6\linewidth]{figures/sd_sets/union2.png}}}

\scnheader{разбиение*}
\scnidtf{разбиение  множества*}
\scnidtf{объединение попарно непересекающихся множеств*}
\scnidtf{декомпозиция множества*}
\scniselement{квазибинарное отношение}
\scniselement{ориентированное отношение}
\scniselement{отношение декомпозиции}
\scntext{определение}{\textbf{\textit{разбиение*}} – это \textit{квазибинарное ориентированное отношение}, областью определения которого является семейство всевозможных множеств. В результате разбиения множества получается множество попарно непересекающихся множеств, объединение которых есть исходное множество.\\
Семейство множеств \{\textit{S1…, Sn}\} является разбиением множества \textit{Si} в том и только том случае, если:
\begin{scnitemize}
    \item семейство \{\textit{S1…, Sn}\} является семейством \textit{попарно непересекающихся множеств};
    \item семейство \{\textit{S1…, Sn}\} является покрытием множества \textit{Si} (или другими словами, множество \textit{Si} является \textit{объединением} множеств, входящих в указанное выше семейство)
\end{scnitemize}
}
\scnrelfrom{описание примера}{
\scnfilescg{figures/sd_sets/split.png}}
\scnaddlevel{1}
\scnexplanation{Множество \textit{Si} разбивается на множества \textit{Sj}, \textit{Sk} и \textit{Sm}.}
\scnaddlevel{-1}
\scnrelfrom{изображение}{
\scnfileimage{\includegraphics[width=0.5\linewidth]{figures/sd_sets/split2.png}}}

\scnheader{пересечение*}
\scnidtf{пересечение множеств*}
\scniselement{квазибинарное отношение}
\scniselement{ориентированное отношение}
\scntext{определение}{\textbf{\textit{пересечение*}} – это операция над множествами, аргументами которой являются два или большее число множеств, а результатом является множество, элементами которого являются все те и только те сущности, которые одновременно принадлежат каждому множеству, которое входит в семейство аргументов этой операции.\\
Будем говорить, что \textit{Множество Si} является пересечением \textit{Множество Sj} и \textit{Множество Sk} тогда и только тогда, когда любой элемент \textit{Множество Si} является элементом \textit{Множество Sj} и элементом \textit{Множество Sk}.}
\scnrelfrom{описание примера}{
\scnfilescg{figures/sd_sets/intersection.png}}
\scnaddlevel{1}
\scnexplanation{Множество \textit{Si} является результатом пересечения множеств \textit{Sj}, \textit{Sk} и \textit{Sm}.}
\scnaddlevel{-1}
\scnrelfrom{изображение}{
\scnfileimage{\includegraphics[width=0.5\linewidth]{figures/sd_sets/intersection2.png}}}

\scnheader{пара пересекающихся множеств*}
\scniselement{бинарное отношение}
\scniselement{неориентированное отношение}
\scnexplanation{\textbf{\textit{пара пересекающихся множеств*}} – \textit{бинарное неориентированное отношение} между двумя \textit{множествами}, имеющими непустое \textit{пересечение*}.}
\scntext{определение}{\textbf{\textit{пара пересекающихся множеств*}} – \textit{бинарное неориентированное отношение} между двумя \textit{множествами}, имеющими, по крайней мере, один общий для этих двух множеств элемент.}
\scnrelfrom{описание примера}{
\scnfilescg{figures/sd_sets/pairOfIntersectingSets.png}}
\scnaddlevel{1}
\scnexplanation{Множество \textit{Si} и множество \textit{Sj} являются парой пересекающихся множеств.}
\scnaddlevel{-1}
\scnrelfrom{изображение}{
\scnfileimage{\includegraphics[width=0.5\linewidth]{figures/sd_sets/pairOfIntersectingSets2.png}}}

\scnheader{попарно пересекающиеся множества*}
\scnidtf{семейство попарно пересекающихся множеств*}
\scnsuperset{пересекающиеся множества*}
\scniselement{отношение}
\scntext{определение}{\textbf{\textit{попарно пересекающиеся множества*}} – семейство множеств, каждая пара которых является парой пересекающихся множеств, т.е. каждая пара которых имеет хотя бы один общий элемент}
\scntext{примечание}{Не каждое \textit{семейство попарно пересекающихся множеств*} является \textit{семейством пересекающихся множеств*}, хотя обратное верно.}
\scnrelfrom{изображение}{
\scnfilescg{figures/sd_sets/pairwiseIntersectingSets.png}}
\scnaddlevel{1}
\scnexplanation{Множества \textit{Si}, \textit{Sj}, \textit{Sk} и \textit{Sl} являются попарно пересекающимися множествами.}
\scnaddlevel{-1}
\scnrelfrom{изображение}{
\scnfileimage{\includegraphics[width=0.7\linewidth]{figures/sd_sets/pairwiseIntersectingSets2.png}}}

\scnheader{пересекающиеся множества*}
\scnidtf{семейство пересекающихся множеств*}
\scnidtf{быть семейством пересекающихся множеств*}
\scnidtf{семейство множеств, имеющих по крайней мере один элемент, являющийся общим для всех этих множеств*}
\scnsuperset{попарно пересекающиеся множества*}
\scntext{определение}{\textbf{\textit{пересекающиеся множества*}} – это семейство множеств, имеющих по крайней мере один общий для всех этих множеств элемент}
\scnrelfrom{описание примера}{
\scnfilescg{figures/sd_sets/intersectingSets.png}}
\scnaddlevel{1}
\scnexplanation{Множества \textit{Si}, \textit{Sj}, \textit{Sk} и \textit{Sl} являются пересекающимися множествами.}
\scnaddlevel{-1}

\scnheader{пара непересекающихся множеств*}
\scniselement{бинарное отношение}
\scniselement{неориентированное отношение}
\scntext{определение}{\textbf{\textit{пара непересекающихся множеств*}} – это \textit{бинарное неориентированное отношение} между \textit{множествами}, результатом \textit{пересечения*} которых есть пустое множество.}
\scnrelfrom{описание примера}{
\scnfilescg{figures/sd_sets/pairOfNonIntersectingSets.png}}
\scnaddlevel{1}
\scnexplanation{Множества \textit{Si} и \textit{Sj} являются парой непересекающихся множеств.}
\scnaddlevel{-1}
\scnrelfrom{изображение}{
\scnfileimage{\includegraphics[width=0.5\linewidth]{figures/sd_sets/pairOfNonIntersectingSets2.png}}}

\scnheader{попарно непересекающиеся множества*}
\scnidtf{семейство попарно непересекающихся множеств*}
\scnsubset{непересекающиеся множества*}
\scntext{определение}{\textbf{\textit{попарно непересекающиеся множества*}} – семейство множеств, каждая пара которых является парой непересекающихся множеств, т.е. каждая пара которых не имеет ни одного общего элемента}
\scnrelfrom{изображение}{
\scnfilescg{figures/sd_sets/pairwiseNonIntersectingSets.png}}
\scnaddlevel{1}
\scnexplanation{Множества \textit{Si}, \textit{Sj}, \textit{Sk} и \textit{Sl} являются попарно непересекающимися множествами.}
\scnaddlevel{-1}

\scnheader{непересекающиеся множества*}
\scnidtf{семейство непересекающихся множеств*}
\scnidtf{быть семейством непересекающихся множеств*}
\scntext{определение}{\textbf{\textit{непересекающиеся множества*}} – это семейство множеств, не имеющих ни одного общего элемента для всех этих множеств}
\scnrelfrom{изображение}{
\scnfilescg{figures/sd_sets/nonIntersectingSets.png}
\scnexplanation{Множества \textit{Si}, \textit{Sj}, \textit{Sk} и \textit{Sl} являются непересекающимися множествами.}}

\scnheader{разность множеств*}
\scniselement{бинарное отношение}
\scniselement{ориентированное отношение}
\scntext{определение}{\textbf{\textit{разность множеств*}} – это \textit{бинарное ориентированное отношение}, связывающее между собой \textit{ориентированную пару}, первым элементом которой является уменьшаемое множество, вторым - вычитаемое множество, и множество, являющееся результатом вычитания вычитаемого из уменьшаемого, в которое входят все элементы первого множества, не входящие во второе множество.}
\scnrelfrom{описание примера}{
\scnfilescg{figures/sd_sets/setDifference.png}}
\scnaddlevel{1}
\scnexplanation{Множество \textit{Si} является результатом разности множеств \textit{Sj} и \textit{Sk}.}
\scnaddlevel{-1}
\scnrelfrom{изображение}{
\scnfileimage{
\includegraphics[width=0.5\linewidth]{figures/sd_sets/setDifference2.png}}}

\scnheader{симметрическая разность множеств*}
\scniselement{бинарное отношение}
\scniselement{ориентированное отношение}
\scntext{определение}{\textbf{\textit{симметрическая разность множеств*}} – это \textit{бинарное ориентированное отношение}, связывающее между собой \textit{пару} множеств и множество, являющееся результатом симметрической разности элементов указанной пары. Будем называть \textit{Множество Si} результатом симметрической разности \textit{Множества Sj} и \textit{Множества Sk} тогда и только тогда, когда любой элемент \textit{Множества Si} является или элементом \textit{Множества Sj} или \textit{Множества Sk}, но не является элементом обоих множеств.}
\scnrelfrom{описание примера}{
\scnfilescg{figures/sd_sets/symmetricDifferenceOfSets.png}
\scnexplanation{Множество \textit{Si} является результатом симметрической разности множеств \textit{Sj} и \textit{Sk}.}}
\scnrelfrom{изображение}{
\scnfileimage{\includegraphics[width=0.5\linewidth]{figures/sd_sets/symmetricDifferenceOfSets2.png}}}

\scnheader{декартово произведение*}
\scnidtf{декартово произведение множеств*}
\scnidtf{прямое произведение множеств*}
\scniselement{бинарное отношение}
\scniselement{ориентированное отношение}
\scntext{определение}{\textbf{\textit{декартово произведение*}} – это \textit{бинарное ориентированное отношение} между \textit{ориентированной парой} множеств и \textit{множеством}, элементами которого являются всевозможные упорядоченные пары, первыми элементами которых являются элементы первого из указанных множеств, вторыми – элементы второго из указанных множеств.}
\scnrelfrom{описание примера}{
\scnfilescg{figures/sd_sets/cartesianMultiplication.png}}
\scnaddlevel{1}
\scnexplanation{Множество \textit{Si} является результатом декартова произведения множеств \textit{Sj} и \textit{Sk}.}
\scnaddlevel{-1}

\scnheader{декартово произведение}
\scnidtf{второй домен отношения быть декартовым произведением}
\scnrelfrom{второй домен}{декартово произведение*}

\scnheader{семейство подмножеств*}
\scnidtf{семейство подмножеств заданного множества*}
\scniselement{бинарное отношение}
\scniselement{ориентированное отношение}
\scnsuperset{булеан*}
\scntext{определение}{\textbf{\textit{семейство подмножеств*}} – это \textit{бинарное ориентированное отношение} между множеством и некоторым семейством множеств, каждое из которых является подмножеством первого множества.}
\scnrelfrom{описание примера}{
\scnfilescg{figures/sd_sets/familyOfSubsets.png}
}

\scnheader{булеан*}
\scnidtf{булеан множества*}
\scnidtf{семейство всевозможных подмножеств заданного множества*}
\scniselement{бинарное отношение}
\scniselement{ориентированное отношение}
\scntext{определение}{\textbf{\textit{булеан*}} – это \textit{бинарное ориентированное отношение} между множеством и некоторым семейством множеств, каждое из которых является подмножеством первого множества.}
\scnrelfrom{описание примера}{
\scnfilescg{figures/sd_sets/boulean.png}
}

\scnheader{булеан}
\scnidtf{второй домен отношения быть булеаном}
\scnrelfrom{второй домен}{булеан*}

\scnheader{мощность}
\scnidtf{мощность множеств}
\scnidtf{кардинальное число}
\scnidtf{общее число вхождений элементов в заданное множество}
\scnidtf{класс эквивалентности, элементами которого являются знаки всех тех и только тех множеств, которые имеют одинаковую мощность}
\scnidtf{класс эквивалентности, соответствующий отношению быть парой множеств, имеющих одинаковую мощность (равномощность множеств)}
\scnidtf{величина мощности множеств}
\scnidtf{трансфинитное число}
\scnidtf{мощность по Кантору}
\scniselement{параметр}
\scnexplanation{\textbf{\textit{мощность}} – это \textit{параметр}, элементами которых являются \textit{множества}, имеющие одинаковое количество элементов. Значением данного параметра является числовая величина, задающая количество элементов, входящих в данный класс множеств, т.е. по сути, количество \textit{позитивных sc-дуг принадлежности}, выходящих из данного \textit{множества}.

Для двух множеств, имеющих одинаковую мощность, существует взаимно-однозначное соответствие между ними (между множествами вхождений элементов в эти множества – на случай мультимножеств).}
\scnrelfrom{описание примера}{
\scnfilescg{figures/sd_sets/power.png}
}

\scnheader{равенство множеств*}
\scniselement{бинарное отношение}
\scniselement{неориентированное отношение}
\scnidtf{быть равными множествами*}
\scntext{определение}{\textbf{\textit{равенство множеств}}* - бинарное неориентированное отношение, выражающее отношение равенства множеств.

Любые два множества являются равными множествами тогда и только тогда, когда первое является включением второго и второе является включением первого.}
\scnrelfrom{описание примера}{
\scnfilescg{figures/sd_sets/equalityOfSets.png}}
\scnaddlevel{1}
\scnexplanation{Множество \textit{Si} равно множеству \textit{Sj}.}
\scnaddlevel{-1}

\scnendstruct \scnendcurrentsectioncomment

\end{SCn}

\scsubsection[\scnmonographychapter{Глава 2.4. Формальные онтологии базовых классов сущностей - множеств, связей, отношений, параметров, величин, чисел, структур, темпоральных сущностей}]{Предметная область и онтология связок и отношений}
\label{sd_rels}
\begin{SCn}

\scnsectionheader{\currentname}

\scnstartsubstruct

\scnheader{Предметная область связок и отношений}
\scniselement{предметная область}
\scnsdmainclasssingle{связь}
\scnsdclass{бинарная связь;sc-коннектор;неатомарная бинарная связь;небинарная связь;неориентированная связь;ориентированная связь;отношение;класс равномощных связок;класс связок разной мощности;унарное отношение;бинарное отношение;квазибинарное отношение;тернарное отношение;небинарное отношение;ориентированное отношение;неориентированное отношение;рефлексивное отношение;антирефлексивное отношение;частично рефлексивное отношение;симметричное отношение;антисимметричное отношение;частично симметричное отношение;транзитивное отношение;антитранзитивное отношение;частично транзитивное отношение;связанное отношение;отношение порядка;отношение строгого порядка;отношение нестрогого порядка;отношение толерантности;отношение эквивалентности;ролевое отношение;числовой атрибут;неролевое отношение;неролевое бинарное отношение;арность;метаотношение;отношение декомпозиции;отношение интеграции}
\scnsdrelation{область определения*;атрибут отношения*;домен*;первый домен*;второй домен*;композиция отношений*;фактор-множество*;соответствие*;отношение соответствия*;область отправления';область прибытия’;образ';прообраз';всюду определенное соответствие*;частично определенное соответствие*;сюръективное соответствие*;несюръективное соответствие*;однозначное соответствие*;обратное соответствие*;обратимое соответствие*;неоднозначное соответствие*;инъективное соответствие*;взаимно однозначное соответствие*;множество сочетаний*;множество размещений*;множество перестановок*}

\scnheader{связь}
\scnidtf{связка sc-элементов}
\scnidtf{sc-связка}
\scnexplanation{\textbf{\textit{связь}} – множество, являющееся абстрактной моделью связи между описываемыми сущностями, которые или знаки которых являются элементами этого множества.}
\scntext{примечание}{Напомним, что все элементы множества, представленного в SC-коде, являются знаками, но описываемыми сущностями могут быть не только сущности, обозначаемые sc-элементами, но и сами эти sc-элементы.}
\scnsubdividing{бинарная связь;небинарная связь}
\scnsubdividing{неориентированная связь;ориентированная связь}

\scnheader{бинарная связь}
\scnsubdividing{sc-коннектор;неатомарная бинарная связь}
\scntext{примечание}{Данное разбиение осуществляется на основе синтаксического признака, а не семантического, поскольку каждый \textit{sc-коннектор} может быть записан в памяти при помощи семантически эквивалентной конструкции, содержащей знак самой связи и пары принадлежности, ведущие к ее элементам, уточненные, при необходимости ролевыми отношениями.}

\scnheader{sc-коннектор}
\scnidtf{атомарная бинарная связь}
\scnexplanation{Каждый \textbf{\textit{sc-коннектор}} представлен в \textit{sc-памяти} одним \textit{sc-элементом} и семантически эквивалентен конструкции, содержащей знак некоторой \textit{бинарной связи} и пары принадлежности, ведущие к элементам этой связи, уточненные, при необходимости ролевыми отношениями.

Такая конструкция может быть обозначена \textbf{\textit{sc-коннектором}} только в случае, когда роли компонентов соответствующей бинарной связи указываются только при помощи \textit{числовых атрибутов 1’} и \textit{2’} или не уточняются вообще.}

\scnheader{неатомарная бинарная связь}
\scnexplanation{\textbf{\textit{неатомарная бинарная связь}} – \textit{бинарная связь}, роли компонентов которой не могут быть заданы только при помощи \textit{ролевых отношений 1'} и \textit{2'}, или не заданы совсем, а требуют дополнительного уточнения при помощи более частных ролевых отношений.}

\scnheader{небинарная связь}
\scnexplanation{\textbf{\textit{небинарная связь}} – связь, имеющая больше двух элементов.}

\scnheader{неориентированная связь}
\scnsuperset{неориентированное множество}
\scnexplanation{\textbf{\textit{неориентированная связь}} – связь, все элементы которой имеют одинаковые роли (при этом соответствующее ролевое отношение, как правило, явно не указывается).}

\scnheader{ориентированная связь}
\scnsuperset{ориентированное множество}
\scnexplanation{\textbf{\textit{ориентированная связь}} – связь, в которой с помощью ролевых отношений, указываются роли компонентов этой связи.}

\scnheader{отношение}
\scnidtf{класс связей}
\scnidtf{класс sc-связок}
\scnidtf{множество отношений}
\scnidtf{Множество всевозможных отношений}
\scntext{определение}{\textbf{\textit{отношение}}, \textit{заданное на множестве M} – это подмножество \textit{декартового произведения} этого множества самого на себя некоторое количество раз.

В более широком смысле \textbf{\textit{отношение}} – это математическая структура, которая формально определяет свойства различных объектов и их взаимосвязи.}
\scnsubdividing{класс равномощных связок;класс связок разной мощности}
\scnsubdividing{бинарное отношение;небинарное отношение}
\scnsubdividing{ориентированное отношение;неориентированное отношение}
\scnsubdividing{ролевое отношение;неролевое отношение}

\scnheader{класс равномощных связок}
\scnidtf{класс связок фиксированной арности}
\scnidtf{отношение, обладающее свойством арности}
\scnsuperset{унарное отношение}
\scnsuperset{бинарное отношение}
\scnsuperset{тернарное отношение}
\scntext{определение}{\textbf{\textit{класс равномощных связок}} – класс связок, имеющих одинаковую мощность.}

\scnheader{класс связок разной мощности}
\scnidtf{отношение нефиксированной арности}
\scnsubset{небинарное отношение}
\scntext{определение}{\textbf{\textit{класс связок разной мощности}} – класс связок, имеющих разную мощность.}

\scnheader{унарное отношение}
\scnidtf{отношение арности один}
\scnidtf{одноместное отношение}
\scnidtf{множество синглетонов}
\scntext{определение}{\textbf{\textit{унарное отношение}} – это множество таких отношений на множестве M, являющихся любым подмножеством множества M.}

\scnheader{бинарное отношение}
\scnidtf{отношение арности два}
\scnidtf{двухместное отношение}
\scnsuperset{квазибинарное отношение}
\scnsuperset{отношение порядка}
\scnsuperset{отношение толерантности}
\scnsubdividing{рефлексивное отношение;антирефлексивное отношение;частично рефлексивное отношение}
\scnsubdividing{симметричное отношение;антисимметричное отношение;частично симметричное отношение}
\scnsubdividing{транзитивное отношение;антитранзитивное отношение;частично транзитивное отношение}
\scnsubdividing{ролевое отношение;неролевое бинарное отношение}
\scntext{определение}{\textbf{\textit{бинарное отношение}} – это множество таких отношений на множестве \textbf{\textit{M}}, являющихся подмножеством \textit{декартова произведения} множества \textbf{\textit{M}}.\\
Если \textbf{\textit{бинарное отношение R}} задано на \textit{множестве} \textbf{\textit{М}} и два элемента этого множества \textbf{\textit{a}} и \textbf{\textit{b}} связаны данным отношением, то будем обозначать такую связь как \textbf{\textit{aRb}}.}

\scnheader{квазибинарное отношение}
\scnexplanation{\textbf{\textit{квазибинарное отношение}} – множество ориентированных пар, первые компоненты которых являются связками.\\
Таким образом, \textit{sc-дуги}, принадлежащие \textbf{\textit{квазибинарным отношениям}}, всегда выходят из связок.}
\scntext{sc-утверждение}{В область определения квазибинарного отношения будем включать:
\begin{scnitemize}
    \item вторые компоненты ориентированных пар, принадлежащих этому отношению;
    \item элементы первых компонентов ориентированных пар, принадлежащих этому отношению;
    \item других элементов область определения квазибинарного отношения не содержит.
\end{scnitemize}
}

\scnheader{тернарное отношение}
\scnidtf{отношение арности три}
\scnidtf{трехместное отношение}
\scnexplanation{\textbf{\textit{тернарное отношение}} – это множество отношений, связывающих между собой три элемента.}

\scnheader{небинарное отношение}
\scnexplanation{\textbf{\textit{небинарное отношение}} – это множество отношений, хотя бы одна из связок каждого из которых имеет значение мощности больше двух.}

\scnheader{ориентированное отношение}
\scntext{определение}{\textbf{\textit{ориентированное отношение}} – это множество таких отношений, каждая связка которых является ориентированным множеством.}

\scnheader{неориентированное отношение}
\scntext{определение}{\textbf{\textit{неориентированное отношение}} – это множество таких отношений, каждая связка которых является неориентированным множеством.}

\scnheader{рефлексивное отношение}
\scntext{определение}{\textbf{\textit{рефлексивное отношение}} – это \textit{бинарное отношение}, любая пара которого есть канторовское множество.}

\scnheader{антирефлексивное отношение}
\scntext{определение}{\textbf{\textit{антирефлексивное отношение R}} на \textit{множестве} \textbf{\textit{A}} – это \textit{бинарное отношение}, в котором все элементы множества \textbf{\textit{A}} не находятся в отношении \textbf{\textit{R}} к самому себе.}

\scnheader{частично рефлексивное отношение}
\scntext{определение}{\textbf{\textit{частично рефлексивное отношение R}} на \textit{множестве} \textbf{\textit{A}} – это \textit{бинарное отношение},  в котором хотя бы один (но не все) элемент множества \textbf{\textit{A}} находится в отношении \textbf{\textit{R}} к самому себе.}

\scnheader{симметричное отношение}
\scntext{определение}{\textbf{\textit{симметричное отношение R}} на \textit{множестве} \textbf{\textit{A}} – это \textit{бинарное отношение}, в котором для каждой пары элементов \textbf{\textit{а}} и \textbf{\textit{b}} этого множества выполнение отношения \textbf{\textit{aRb}} влечёт выполнение \textbf{\textit{bRa}}.}

\scnheader{антисимметричное отношение}
\scntext{определение}{\textbf{\textit{антисимметричное отношение R}} на \textit{множестве} \textbf{\textit{A}} – это \textit{бинарное отношение}, в котором для каждой пары элементов \textbf{\textit{а}} и \textbf{\textit{b}} этого множества выполнение отношений \textbf{\textit{aRb}} и \textbf{\textit{bRa}} влечёт равенство \textbf{\textit{a}} и \textbf{\textit{b}}.}

\scnheader{частично симметричное отношение}
\scntext{определение}{\textbf{\textit{частично симметричное отношение R}} на \textit{множестве} \textbf{\textit{A}} – это \textit{бинарное отношение}, в котором для каждой пары элементов \textbf{\textit{а}} и \textbf{\textit{b}} (но не для всех таких пар) этого множества выполнение отношения \textbf{\textit{aRb}} влечёт выполнение \textbf{\textit{bRa}}.}

\scnheader{транзитивное отношение}
\scntext{определение}{\textbf{\textit{транзитивное отношение R}} на \textit{множестве} \textbf{\textit{A}} – это \textit{бинарное отношение}, в котором для любых трёх элементов этого множества \textbf{\textit{a, b, c}} выполнение отношений \textbf{\textit{aRb}} и \textbf{\textit{bRc}} влечёт выполнение отношения \textbf{\textit{aRc}}.}

\scnheader{антитранзитивное отношение}
\scntext{определение}{\textbf{\textit{антитранзитивное отношение R}} на \textit{множестве} \textbf{\textit{A}} – это \textit{бинарное отношение}, в котором для любых трёх элементов этого множества \textbf{\textit{a, b, c}} выполнение отношений \textbf{\textit{aRb}} и \textbf{\textit{bRc}} не влечёт выполнение отношения \textbf{\textit{aRc}}.}

\scnheader{частично транзитивное отношение}
\scntext{определение}{\textbf{\textit{частично транзитивное отношение R}} на \textit{множестве} \textbf{\textit{A}} – это \textit{бинарное отношение}, в котором для каждых трёх элементов этого множества \textbf{\textit{a, b, c}} (но не для всех таких троек) выполнение отношений \textbf{\textit{aRb}} и \textbf{\textit{bRc}} влечёт выполнение отношения \textbf{\textit{aRc}}.}

\scnheader{связанное отношение}
\scnsubset{бинарное отношение}
\scntext{определение}{\textbf{\textit{связанное отношение R}} на \textit{множестве} \textbf{\textit{A}} – это \textit{бинарное отношение}, в котором для каждой пары элементов \textbf{\textit{а}} и \textbf{\textit{b}} этого множества выполняется одно из двух отношений: \textbf{\textit{aRb}} или \textbf{\textit{bRa}}.}

\scnheader{отношение порядка}
\scnsubdividing{отношение строгого порядка;отношение нестрогого порядка}
\scntext{определение}{\textbf{\textit{отношение порядка}} – это \textit{бинарное отношение}, обладающее свойством транзитивности и антисимметричности.}

\scnheader{отношение строгого порядка}
\scntext{определение}{\textbf{\textit{отношение строгого порядка}} – это \textit{отношение порядка}, обладающее свойством антирефлексивности.}

\scnheader{отношение нестрогого порядка}
\scntext{определение}{\textbf{\textit{отношение нестрогого порядка}} – это \textit{отношение порядка}, обладающее свойством рефлексивности.}

\scnheader{отношение толерантности}
\scntext{определение}{\textbf{\textit{отношение толерантности}} – это \textit{бинарное отношение}, принадлежащее классам \textit{симметричное отношение} и \textit{рефлексивное отношение}.}

\scnheader{отношение эквивалентности}
\scnidtf{максимальное семейство отношений эквивалентности}
\scnsubset{отношение толерантности}
\scntext{определение}{\textbf{\textit{отношение эквивалентности}} – это \textit{отношение толерантности}, принадлежащее классу \textit{транзитивных отношений}}
\scntext{примечание}{Каждое отношение эквивалентности уточняет то, что мы считаем эквивалентными сущностями, т.е. то, на какие сходства этих сущностей мы обращаем внимание и какие их отличия мы игнорируем (не учитываем).}

\scnheader{ролевое отношение}
\scnidtf{атрибут}
\scnidtf{атрибутивное отношение}
\scnidtf{отношение, которое задает роль элементов в рамках некоторого множества}
\scnidtf{отношение, являющееся подмножеством отношения принадлежности}
\scnrelto{семейство подмножеств}{принадлежность*}
\scnsubset{бинарное отношение}
\scnsuperset{числовой атрибут}
\scnexplanation{\textbf{\textit{ролевое отношение}} – это отношение, являющееся подмножеством отношения принадлежности.}
\scntext{правило идентификации экземпляров}{В конце каждого \textit{идентификатора}, соответствующего экземплярам класса \textbf{\textit{ролевое отношение}}, не являющегося системным, ставится знак «'».

Например:\\
\textit{ключевой экземпляр’}

Из-за ограничений в разрешенном алфавите символов, в системном идентификаторе не может быть использовать знак «'», поэтому в начале каждого \textit{системного идентификатора}, соответствующего экземплярам класса \textbf{\textit{ролевое отношение}} ставится префикс «rrel\_».

Например:\\
\textit{rrel\_key\_sc\_element}}

\scnheader{числовой атрибут}
\scnidtf{порядковый номер}
\scnidtf{номер компонента ориентированной связки}
\scnhaselement{1’; 2’; 3’; 4’; 5’; 6’; 7’; 8’; 9’; 10’}
\scnexplanation{\textbf{\textit{числовой атрибут}} – \textit{ролевое отношение}, задающее порядковый номер элемента некоторой ориентированной связки, не уточняя при этом семантику такой принадлежности. Во многих случаях бывает достаточно использовать числовые атрибуты, чтобы различать компоненты связки, семантика каждого из которых дополнительно оговаривается, например, при определении отношения, которому данная связка принадлежит.}

\scnheader{неролевое отношение}
\scnsubdividing{небинарное отношение;неролевое бинарное отношение}
\scnexplanation{\textbf{\textit{неролевое отношение}} – отношение, не являющееся подмножеством отношения принадлежности.}
\scntext{правило идентификации экземпляров}{В конце каждого \textit{идентификатора}, соответствующего экземплярам класса \textbf{\textit{неролевое отношение}}, не являющегося системным, ставится знак «*».

Например:\\
\textit{включение*}

Из-за ограничений в разрешенном алфавите символов, в системном идентификаторе не может быть использовать знак «*», поэтому в начале каждого \textit{системного идентификатора}, соответствующего экземплярам класса \textbf{\textit{неролевое отношение}} ставится префикс «nrel\_».

Например:\\
\textit{nrel\_inclusion}}

\scnheader{неролевое бинарное отношение}
\scnexplanation{\textbf{\textit{неролевое бинарное отношение}} – \textit{бинарное отношение}, не являющееся \textit{ролевым отношением}.}

\scnheader{арность}
\scnidtf{арность отношения}
\scniselement{параметр}
\scnexplanation{\textbf{\textit{арность}} – это параметр, каждый элемент которого представляет собой класс \textit{отношений}, каждая связка которых имеет одинаковую \textit{мощность}. Значение данного \textit{параметра} совпадает со значением \textit{мощности} каждой из таких связок.}
\scnrelfrom{описание примера}{
\scnfilescg{figures/sd_relations/arity.png}}


\scnheader{область определения*}
\scnidtf{область определения отношения*}
\scniselement{бинарное отношение}
\scnexplanation{\textbf{\textit{область определения*}} – это \textit{бинарное отношение}, связывающее отношение со множеством, являющимся его областью определения.

Областью определения отношения будем называть результат теоретико-множественного объединения всех связок этого отношения, или, другими словами, результат теоретико-множественного объединения всех множеств, являющихся доменами данного отношения.}
\scnrelfrom{описание примера}{
\scnfilescg{figures/sd_relations/domain.png}}

\scnheader{атрибут отношения*}
\scnidtf{ролевой атрибут, используемый в связках заданного отношения*}
\scniselement{бинарное отношение}
\scnexplanation{\textbf{\textit{атрибут отношения*}} – это \textit{бинарное отношение}, связывающее заданное отношение с \textit{ролевым отношением}, используемым в данном отношении для уточнения роли того или иного элемента связок данного отношения.}
\scnrelfrom{описание примера}{
\scnfilescg{figures/sd_relations/relationshipAttribute.png}}


\scnheader{домен*}
\scnidtf{домен отношения по заданному атрибуту*}
\scniselement{бинарное отношение}
\scnexplanation{\textbf{\textit{домен*}} – это \textit{бинарное отношение}, связывающее связку отношения \textit{атрибут отношения*} со множеством, являющимся доменом заданного отношения по заданному атрибуту. Множество \textbf{\textit{di}} является доменом отношения \textbf{\textit{ri}} по атрибуту \textbf{\textit{ai}} в том и только том случае, если элементами этого множества являются все те и только те элементы связок отношения \textbf{\textit{ri}}, которые имеют в рамках этих связок атрибут \textbf{\textit{ai}}.}
\scnrelfrom{описание примера}{
\scnfilescg{figures/sd_relations/domen.png}}


\scnheader{первый домен*}
\scniselement{бинарное отношение}
\scntext{определение}{\textbf{\textit{первый домен*}} – это \textit{бинарное отношение}, связывающее отношение с множеством, являющимся доменом по атрибуту \textbf{\textit {1'}} данного отношения.}
\scnrelfrom{описание примера}{
\scnfilescg{figures/sd_relations/firstDomen.png}}

\scnheader{второй домен*}
\scniselement{бинарное отношение}
\scntext{определение}{\textbf{\textit{второй домен*}} – это \textit{бинарное отношение}, связывающее отношение с множеством, являющимся доменом по атрибуту \textbf{\textit{2'}} данного отношения.}
\scnrelfrom{описание примера}{
\scnfilescg{figures/sd_relations/secondDomen.png}}

\scnheader{композиция отношений*}
\scniselement{квазибинарное отношение}
\scntext{определение}{\textbf{\textit{композиция отношений*}} – это \textit{квазибинарное отношение}, связывающее два бинарных отношения с отношением, являющимся их композицией. Под композицией бинарных отношений \textbf{\textit{R}} и \textbf{\textit{S}} будем понимать множество $\{(x, y) | \exists z(xSz \wedge zRy)\}$}
\scnrelfrom{описание примера}{
\scnfilescg{figures/sd_relations/relationshipComposition.png}}

\scnheader{фактор-множество*}
\scnidtf{быть фактор-множеством*}
\scnidtf{множество всевозможных максимальных множеств из попарно эквивалентных элементов*}
\scnidtf{множество всевозможных классов эквивалентности для заданного отношения эквивалентности*}
\scniselement{бинарное отношение}
\scntext{определение}{\textbf{\textit{фактор множество*}} - это бинарное ориентированное отношение, каждая связка которого связывает некоторое отношение эквивалентности со множеством всех соответствующих этому отношению классов эквивалентности. Каждый такой класс представляет собой максимальное множество сущностей, каждая пара которых принадлежит указанному выше отношению эквивалентности.}
\scnrelfrom{описание примера}{
\scnfilescg{figures/sd_relations/factor_set.png}}

\scnheader{метаотношение}
\scntext{определение}{метаотношение - это \textit{отношение}, в каждой связке которого есть по крайней мере один компонент, являющийся знаком некоторого \textit{отношения}.}

\scnheader{отношение декомпозиции}
\scnhaselement{разбиение*}
\scnhaselement{декомпозиция раздела*}
\scnhaselement{декомпозиция абстрактного объекта*}

\scnheader{отношение интеграции}
\scnhaselement{объединение*}

\scnheader{соответствие*}
\scnidtf{наличие соответствия*}
\scniselement{бинарное отношение}
\scnsubdividing{соответствие между непересекающимися множествами*;соответствие между строго пересекающимися множествами*;соответствие, область отправления и область прибытия которого совпадают*}
\scnsubdividing{всюду определенное соответствие*;частично определенное соответствие*}
\scnsubdividing{сюръекция*;несюръективное соответствие*}
\scnsubdividing{однозначное соответствие*;неоднозначное соответствие*}
\scntext{определение}{\textbf{\textit{соответствие*}} – \textit{бинарное отношение}, заданное на множествах и задающее наличие отношения, в котором участвуют только элементы этих множеств.}
\scnrelfrom{описание примера}{
\scnfilescg{figures/sd_relations/conformity.png}}

\scnheader{отношение соответствия*}
\scniselement{бинарное отношение}
\scntext{определение}{\textbf{\textit{отношение соответствия*}} – \textit{бинарное отношение}, связывающее ориентированную пару множеств, на которых задано \textit{соответствие*} и некоторое подмножество \textit{декартова произведения*} этих \textit{множеств}.}
\scnrelfrom{описание примера}{
\scnfilescg{figures/sd_relations/relationshipConformity.png}}

\scnheader{область отправления'}
\scnidtf{область отправления соответствия’}
\scnidtf{область определения соответствия’}
\scnidtf{первый компонент пары в отношении соответствия’}
\scniselement{ролевое отношение}
\scntext{определение}{\textbf{\textit{область отправления'}} – \textit{ролевое отношение}, указывающее на первый компонент пары в рамках отношения \textit{соответствие*}.}
\scnrelfrom{описание примера}{
\scnfilescg{figures/sd_relations/departureArea.png}}

\scnheader{область прибытия’}
\scnidtf{область прибытия соответствия'}
\scnidtf{область значений соответствия'}
\scniselement{ролевое отношение}
\scntext{определение}{\textbf{\textit{область прибытия’}} – \textit{ролевое отношение}, указывающее на второй компонент пары в рамках отношения \textit{соответствие*}.}
\scnrelfrom{описание примера}{
\scnfilescg{figures/sd_relations/arrivalArea.png}}

\scnheader{образ'}
\scnidtf{образ соответствия’}
\scniselement{ролевое отношение}
\scntext{определение}{\textbf{\textit{образ'}} – \textit{ролевое отношение}, указывающее на второй компонент каждой пары в рамках множества пар, которое является вторым компонентом \textit{отношения соответствия*}.}
\scnrelfrom{описание примера}{
\scnfilescg{figures/sd_relations/form.png}}

\scnheader{прообраз'}
\scnidtf{прообраз соответствия’}
\scniselement{ролевое отношение}
\scntext{определение}{\textbf{\textit{прообраз'}} – \textit{ролевое отношение}, указывающее на первый компонент каждой пары в рамках множества пар, которое является первым компонентом \textit{отношения соответствия*}.}
\scnrelfrom{описание примера}{
\scnfilescg{figures/sd_relations/prototype.png}}

\scnheader{всюду определенное соответствие*}
\scnidtf{полное соответствие*}
\scnidtf{наличие всюду определенного соответствия*}
\scntext{определение}{\textbf{\textit{всюду определенное соответствие*}} – это \textit{соответствие*}, при котором существует \textit{образ’} для каждого элемента \textit{области отправления'} данного \textit{соответствия*}.}
\scnrelfrom{описание примера}{
\scnfilescg{figures/sd_relations/surjection.png}}
\scnrelfrom{изображение}{
\scnfileimage{\includegraphics[width=0.5\linewidth]{figures/sd_relations/surjection2.png}}}


\scnheader{частично определенное соответствие*}
\scnidtf{наличие частично определенного соответствия*}
\scntext{определение}{\textbf{\textit{частично определенное соответствие*}} – это \textit{соответствие*}, при котором существует \textit{образ’} для некоторых, но не всех элементов \textit{области отправления'} данного \textit{соответствия*}.}
\scnrelfrom{описание примера}{
\scnfilescg{figures/sd_relations/partiallyDefinedConformity.png}}
\scnrelfrom{изображение}{
\scnfileimage{\includegraphics[width=0.5\linewidth]{figures/sd_relations/partiallySurjection.png}}}


\scnheader{сюръективное соответствие*}
\scnidtf{наличие сюръективного соответствия*}
\scnidtf{сюръекция*}
\scntext{определение}{\textbf{\textit{сюръективное соответствие*}} – это \textit{соответствие*}, при котором существует \textit{прообраз’} для каждого элемента \textit{области прибытия'} данного \textit{соответствия*}.}
\scnrelfrom{описание примера}{
\scnfilescg{figures/sd_relations/surjectiveConformity.png}}
\scnrelfrom{изображение}{
\scnfileimage{\includegraphics[width=0.5\linewidth]{figures/sd_relations/surjectiveConformity2.png}}}

\scnheader{несюръективное соответствие*}
\scnidtf{наличие несюръективного соответствия*}
\scntext{определение}{\textbf{\textit{несюръективное соответствие*}} – это \textit{соответствие*}, при котором не для каждого элемента \textit{области прибытия'} данного \textit{соответствия*} существует \textit{прообраз’}.}
\scnrelfrom{описание примера}{
\scnfilescg{figures/sd_relations/nonSurjectiveConformity.png}}
\scnrelfrom{изображение}{
\scnfileimage{\includegraphics[width=0.5\linewidth]{figures/sd_relations/nonSurjectiveConformity2.png}}}

\scnheader{однозначное соответствие*}
\scnidtf{наличие однозначного соответствия*}
\scnidtf{функциональное соответветствие*}
\scnidtf{функция*}
\scntext{определение}{\textbf{\textit{однозначное соответствие*}} – это \textit{соответствие*}, при котором каждому элементу из \textit{области отправления'} соответствия ставится не более, чем один элемент из \textit{области прибытия’} соответствия.}
\scnrelfrom{описание примера}{
\scnfilescg{figures/sd_relations/singleConformity.png}}
\scnrelfrom{изображение}{
\scnfileimage{\includegraphics[width=0.5\linewidth]{figures/sd_relations/singleConformity2.png}}}

\scnheader{обратное соответствие*}
\scniselement{бинарное отношение}
\scnrelfrom{область определения}{соответствие*}
\scntext{определение}{\textbf{\textit{обратное соответствие*}} – \textit{бинарное отношение}, связывающее два \textit{соответствия*}, при этом выполняются следующие условия:
\begin{scnitemize}
    \item \textit{область отправления’} первого соответствия является \textit{областью прибытия'} второго;
    \item \textit{область прибытия’} первого соответствия является \textit{областью отправления'} второго;
    \item для каждой пары, входящей в состав отношения первого соответствия, существует пара, входящая в состав отношения второго соответствия, при этом \textit{образ’} и \textit{прообраз'} в рамках первой указанной пары являются соответственно \textit{прообразом'} и \textit{образом’} в рамках второй.
\end{scnitemize}
}

\scnheader{обратимое соответствие*}
\scnsubset{однозначное соответствие*}
\scntext{определение}{\textbf{\textit{обратимое соответствие*}} – такое \textit{однозначное соответствие*}, для которого \textit{обратное соответствие*} также является \textit{однозначным соответствием*}.}

\scnheader{неоднозначное соответствие*}
\scntext{определение}{\textbf{\textit{неоднозначное соответствие*}} – это \textit{соответствие*}, при котором хотя бы одному элементу из \textit{области отправления’} соответствия ставится более, чем один элемент из \textit{области прибытия'} соответствия.}
\scnrelfrom{описание примера}{
\scnfilescg{figures/sd_relations/nonSingleConformity.png}}
\scnrelfrom{изображение}{
\scnfileimage{\includegraphics[width=0.5\linewidth]{figures/sd_relations/nonSingleConformity2.png}}}

\scnheader{инъективное соответствие*}
\scnidtf{инъекция*}
\scnsubset{однозначное соответствие*}
\scntext{определение}{\textbf{\textit{инъективное соответствие*}} – это \textit{соответствие*}, при котором разным элементам из \textit{области отправления’} соответствия всегда соответствуют разные элементы из \textit{области прибытия'} соответствия и наоборот.}
\scnrelfrom{описание примера}{
\scnfilescg{figures/sd_relations/injectiveConformity.png}}
\scnrelfrom{изображение}{
\scnfileimage{\includegraphics[width=0.5\linewidth]{figures/sd_relations/injectiveConformity2.png}}}

\scnheader{взаимно однозначное соответствие*}
\scnidtf{биекция*}
\scnsubset{всюду определенное соответствие*}
\scnsubset{сюръективное соответствие*}
\scnsubset{инъективное соответствие*}
\scntext{определение}{\textbf{\textit{взаимно однозначное соответствие*}} – это \textit{инъективное соответствие*}, являющееся всюду определенным и сюръективным.}
\scnrelfrom{описание примера}{
\scnfilescg{figures/sd_relations/bijectiveConformity.png}}
\scnrelfrom{изображение}{
\scnfileimage{\includegraphics[width=0.5\linewidth]{figures/sd_relations/bijectiveConformity2.png}}}


\scnheader{множество сочетаний*}
\scnidtf{множество всевозможных сочетаний*}
\scnidtf{множество всевозможных сочетаний заданной арности из элементов заданного множества*}
\scnidtf{множество всех неориентированных связок заданной арности*}
\scnidtf{множество всех подмножеств заданной мощности*}
\scnidtf{семейство всевозможных сочетаний*}
\scntext{определение}{\textbf{\textit{множество сочетаний*}} - \textit{отношение}, связывающее некоторое множество и семейство всевозможных множеств, имеющих значение мощности, меньше либо равное мощности исходного множества и состоящих из тех же элементов, что и это множество.}
\scntext{утверждение}{Мощность \textbf{\textit{множества сочетаний*}} может быть вычислена как n!/(k!(n-k)!), где \textbf{\textit{n}} – мощность исходного множества, \textbf{\textit{k}} – мощность элементов множества сочетаний.}
\scnrelfrom{описание примера}{
\scnfilescg{figures/sd_relations/setsOfCombinations.png}
\scnexplanation{Для Множества \textbf{\textit{Si}} представлено множество сочетаний по 2 элемента.}}

\scnheader{множество размещений*}
\scntext{определение}{\textbf{\textit{множество размещений*}} - \textit{отношение}, связывающее некоторое множество и семейство всевозможных кортежей, имеющих значение мощности, меньше либо равное мощности исходного множества и состоящих из тех же элементов, что и это множество.}
\scntext{утверждение}{Мощность \textbf{\textit{множества размещений*}} может быть вычислена как n!/(n-k)!, где \textbf{\textit{n}} – мощность исходного множества, \textbf{\textit{k}} – мощность элементов множества сочетаний.}
\scnrelfrom{описание примера}{
\scnfilescg{figures/sd_relations/setsOfPlacements.png}
\scnexplanation{Для Множества \textbf{\textit{Si}} представлено множество размещений по 2 элемента.}}

\scnheader{множество перестановок*}
\scnsubset{множество размещений*}
\scntext{определение}{\textbf{\textit{множество перестановок*}} - \textit{отношение}, связывающее некоторое множество и семейство всевозможных кортежей, равномощных исходному множеству и состоящих из тех же элементов, что и это множество.}
\scntext{утверждение}{Мощность \textbf{\textit{множества перестановок*}} может быть вычислена как n!, где \textbf{\textit{n}} – мощность исходного множества.}
\scnrelfrom{описание примера}{
\scnfilescg{figures/sd_relations/setsOfPermutations.png}
\scnexplanation{Для Множества \textbf{\textit{Si}} представлено его множество перестановок.}}

\bigskip
\scnendstruct \scnendcurrentsectioncomment

\end{SCn}

\scsubsection[\scnmonographychapter{Глава 2.4. Формальные онтологии базовых классов сущностей - множеств, связей, отношений, параметров, величин, чисел, структур, темпоральных сущностей}]{Предметная область и онтология параметров, величин и шкал}
\label{sd_params}
\begin{SCn}

\scnsectionheader{Предметная область и онтология параметров, величин и измерений}

\scnstartsubstruct

\scnheader{Предметная область параметров, величин и измерений}
\scnidtf{Предметная область параметров и классов эквивалентности, являющихся их элементами (значениями, величинами)}
\scniselement{предметная область}
\scnsdmainclasssingle{параметр}
\scnsdclass{измеряемый параметр;неизмеряемый параметр;уровень класса эквивалентности;величина;точная величина;неточная величина;интервальная величина;параметрическая модель;измерение с фиксированной единицей измерения ;измерение по шкале;арифметическое выражение на величинах;арифметическая операция на величинах;действие. измерение;задача. измерение}
\scnsdrelation{область определения параметра*;эталон';измерение*;точность*;единица измерения*;нулевая отметка*;сумма величин*;произведение величин*;возведение величин в степень*;большая величина*;равенство величин*;большая или равная величина*}

\scnauthorcomment{ввести отношение, показывающее единичную отметку для измерений по шкале}

\scnheader{параметр}
\scnidtf{характеристика}
\scnidtf{свойство}
\scnidtf{признак}
\scnidtf{класс классов}
\scnidtf{измеряемое свойство}
\scnidtf{признак классификации или покрытия некоторого класса сущностей}
\scnidtf{признак разбиения или покрытия некоторого класса сущностей}
\scnidtf{семейство множеств, разбивающих или покрывающих некоторый класс сущностей}
\scnidtf{семейство классов сущностей, обладающих одинаковым соответствующим свойством}
\scnidtf{фактор-множество, соответствующее некоторому отношению эквивалентности, или аналог фактор-множества, соответствующий некоторому отношению толерантности}
\scnreltoset{разбиение}{измеряемый параметр;неизмеряемый параметр}
\scnexplanation{Каждый \textbf{\textit{параметр}} представляет собой класс, являющийся семейством всевозможных классов эквивалентности или толерантности, задаваемых либо \textit{отношением эквивалентности}, либо \textit{отношением толерантности} (симметричным, рефлексивным, но частично транзитивным). Так, например, элементами (значениями, величинами) \textbf{\textit{параметра}} \textit{длина} являются либо классы эквивалентности, задаваемые отношением эквивалентности «иметь точно одинаковую длину*», либо классы толерантности, задаваемые отношением вида «иметь приблизительно одинаковую длину с указываемой точностью*», либо интервальные классы, задаваемые бинарными отношениями вида «иметь длину, находящуюся в одном и том же указываемом интервале*» (например, от 1 метра до 2 метров).\\
Примерами параметров как отношений эквивалентности являются:
\begin{scnitemize}
    \item равновеликость геометрических фигур (по длине, площади, объему – в зависимости от размерности этих фигур);
    \item иметь одинаковый цвет (быть эквивалентными по цвету);
    \item эквивалентность, по вкусу, запаху, твердости и т.д.
\end{scnitemize}

Заметим, что среди элементов (значений, величин) параметра могут встречаться пересекающиеся множества (классы), но объединение всех элементов каждого параметра есть не что иное, как класс всевозможных сущностей, обладающих этим параметром (свойством, характеристикой). Например, класс всех сущностей, имеющих длину, класс всех сущностей, обладающих цветом.

Каждый конкретный параметр (характеристика), т.е. каждый элемент класса всевозможных параметров (характеристик) есть, по сути, признак классификации сущностей, обладающих это характеристикой, по принципу эквивалентности (одинаковости значения) этой характеристики. Например, параметр \textit{цвет} разбивает множество сущностей имеющих цвет на классы, каждый из которых включает в себя сущности, имеющие одинаковый цвет. Параметр может разбиваться на классы для уточнения некоторого свойства, например элементами параметра цвет будут классы, соответствующие конкретным цветам (синий, красный и т.д.), в свою очередь каждый конкретный цвет может включать более частные классы, уточняющие данное свойство, например, темно-синий, светло-красный и т.д.

Другими словами, каждому множеству сущностей может ставиться в соответствие набор (семейство) параметров (параметрическое пространство), которыми обладают сущности этого множества – аналог семейства отношений, определенных (заданных) на этом множестве. Часто бывает важно построить такое параметрическое пространство, «точки» которого взаимно-однозначно соответствуют параметризуемым сущностям (например, набор параметров, позволяющих однозначно идентифицировать, установить личность каждого человека). 

Таким образом, для каждого используемого элемента (значения) какого-либо параметра, необходимо явно указывать спецификацию этого значения (точное значение, неточное значение, интервальное значение, точность, интервал).}
\scnrelfrom{типичная семантическая окрестность}{
\scnfilelong{
\begin{figure}[H]
\centering
\includegraphics[width=0.8\linewidth]{figures/sd_parameters_and_quantities/parameterDescription.png}
\end{figure}
}}

\scnheader{область определения параметра*}
\scnidtf{множество всех тех и только тех сущностей, которые являются компонентами значений заданного параметра*}
\scnidtf{множество всех тех и только тех сущностей, которые обладают заданным параметром*}
\scnrelto{включение}{объединение*}

\scnheader{измеряемый параметр}
\scnidtf{количественный параметр}
\scnidtf{семейство измеряемых величин}
\scnidtf{семейство классов эквивалентности, каждому из которых может быть поставлено в соответствие числовое значение}
\scnexplanation{Каждый \textbf{\textit{измеряемый параметр}} представляет собой \textit{параметр}, значение (элемент, экземпляр) которого трактуется как \textit{величина}, которой можно поставить в соответствие ее числовое значение на основании выбранной единицы измерения и точки отсчета (нулевой отметки выбранной шкалы).}

\scnheader{неизмеряемый параметр}
\scnidtf{качественный параметр}

\scnheader{уровень класса эквивалентности}
\scnidtf{уровень параметра}
\scniselement{параметр}
\scnexplanation{Параметр \textbf{\textit{уровень класса эквивалентности}} задает уровень некоторого значения некоторого \textit{параметра} в иерархии значений этого параметра. Уровень класса эквивалентности равен 1, если значение параметра не является частным по отношению к другому значению этого параметра, равен 2, если значение параметра является частным по отношению к значению этого параметра с уровнем 1 и т.д.}
\scnrelfrom{типичная семантическая окрестность}{
\scnfilelong{
\begin{figure}[H]
\centering
\includegraphics[width=0.8\linewidth]{figures/sd_parameters_and_quantities/color.png}
\end{figure}
}}

\scnheader{величина}
\scnidtf{значение количественного параметра}
\scnidtf{значение измеряемого параметра}
\scnidtf{класс сущностей, имеющих одинаковое значение соответствующего параметра}
\scnrelfromlist{включение}{точная величина;неточная величина;интервальная величина}
\scnexplanation{Каждая \textbf{\textit{величина}} представляет собой однозначный и независящий от шкалы измерения результат измерения некоторой характеристики у некоторой сущности.

Каждой \textbf{\textit{величине}} можно поставить в соответствие ее числовое значение на основании выбранной единицы измерения и точки отсчета (нулевой отметки выбранной шкалы).

Нельзя путать значение параметра (\textbf{\textit{величину}}) и значение величины по некоторой шкале, которое может быть скалярным и векторным.}

\scnheader{точная величина}
\scnidtf{точное значение параметра}
\scnidtf{множество всех точных значений параметра}
\scnidtf{значение параметра, являющееся семейством классов эквивалентности, соответствующим некоторому отношению эквивалентности}
\scnidtf{класс эквивалентности}
\scnexplanation{Каждая \textbf{\textit{точная величина}} имеет одно фиксированное значение в некоторой единице измерения или по какой-либо шкале. При этом считается, что все элементы такого класса имеют одинаковое значение данного параметра и отклонениями можно пренебречь.

Каждой \textbf{\textit{точной величине}} можно поставить в соответствие группу \textit{неточных величин}, являющихся не разбиениями, а покрытиями того же множества, но с разной степенью точности.}
\scnrelfrom{типичная семантическая окрестность}{
\scnfilelong{
\begin{figure}[H]
\centering
\includegraphics[width=0.8\linewidth]{figures/sd_parameters_and_quantities/exactLength.png}
\end{figure}}
\scntext{комментарий}{В данном примере \textit{ki} обозначает класс сущностей, имеющих длину ровно 5 метров. Конкретный пример такой сущности - \textit{bi}.}}

\scnheader{неточная величина}
\scnidtf{множество неточных значений параметра}
\scnidtf{приблизительная величина}
\scnidtf{приблизительное значение параметра}
\scnidtf{значение параметра в интервале с нефиксированными границами}
\scnexplanation{Каждой \textbf{\textit{неточной величине}} ставится в соответствие ее значение в некоторой единице измерения или по какой-либо шкале, а также дополнительно указывается \textit{точность*}, т.е. возможное отклонение от данного значения.}
\scnrelfrom{типичная семантическая окрестность}{
\scnfilelong{
\begin{figure}[H]
\centering
\includegraphics[width=0.8\linewidth]{figures/sd_parameters_and_quantities/approximateLength.png}
\end{figure}}
\scntext{комментарий}{В данном примере \textit{ki} обозначает класс сущностей, имеющих длину примерно 25 метров, при этом максимально возможное отклонение от этого значения составляет \textit{kj}, то есть 2 метра. Конкретный пример такой сущности - \textit{bi}.}}

\scnheader{интервальная  величина}
\scnidtf{интервальное значение параметра}
\scnidtf{значение параметра в интервале с фиксированными границами}
\scnidtf{интервал значения параметра из множества пересекающихся интервалов разной длины, имеющих нефиксированные границы}
\scnexplanation{Каждая \textbf{\textit{интервальная величина}} представляет собой класс сущностей, находящихся в рамках точно заданного интервала, минимальная и максимальная точка которого являются \textit{точными величинами}. Результатом \textit{измерения*} такой величины является ориентированная пара, первым компонентом которой является левая (меньшая) граница интервала, вторым компонентом – правая (большая) граница интервала.}
\scnrelfrom{типичная семантическая окрестность}{
\scnfilelong{
\begin{figure}[H]
\centering
\includegraphics[width=0.8\linewidth]{figures/sd_parameters_and_quantities/intervalLength.png}
\end{figure}}
\scntext{комментарий}{В данном примере \textit{ki} обозначает класс сущностей, имеющих длину, которая лежит в интервале от \textit{kj} до \textit{kl}, то есть в интервале от 4 до 5 метров, а \textit{bi} – конкретный пример такой сущности.}}

\scnheader{эталон'}
\scnidtf{образец'}
\scniselement{ролевое отношение}
\scnexplanation{Ролевое отношение \textit{эталон'} указывает на тот элемент значения некоторого параметра, который в рамках данного класса эквивалентности считается эталонным, то есть он используется как образец при определении данного параметра.

\textit{эталон'} может задаваться как для измеряемых, так и для неизмеряемых параметров, например, эталон метра или эталон красоты.}

\scnheader{измерение*}
\scnidtf{значение параметра*}
\scnidtf{значение величины*}
\scnidtf{измерение как соответствие*}
\scnidtf{результат измерения заданной величины в заданной единице измерения и по заданной шкале*}
\scnidtf{бинарное ориентированное отношение, связывающее различные величины с результатами их измерения в различных единицах измерения и по различным шкалам*}
\scnexplanation{Связки отношения \textit{измерение*} связывают величину и ее значение в некоторой единице измерения (в том числе, в интервале) или по некоторой шкале. Конкретная единица измерения или шкала указывается дополнительно при помощи соответствующего отношения. Одной величине может соответствовать только одно значение в каждой возможной единице измерения или одна точка на некоторой шкале.}

\scnheader{точность*}
\scnidtf{отклонение*}
\scnidtf{степень точности неточного значения параметра*}
\scniselement{бинарное отношение}
\scnexplanation{Связки отношения \textbf{\textit{точность*}} связывают \textit{неточную величину} и \textit{точную величину} того же класса, задающую максимальное возможное отклонение указанной \textit{неточной величины} от своего значения.}

\scnheader{параметрическая модель}
\scnidtf{параметрическая спецификация}
\scnidtf{параметрическое sc-описание заданной сущности}
\scnidtf{описание сущности как точки в некотором параметрическом (признаковом) пространстве}
\scnrelto{включение}{семантическая окрестность}
\scnexplanation{Каждая \textbf{\textit{параметрическая модель}} представляет собой описание заданной сущности в некотором параметрическом пространстве количественных и качественных \textit{параметров}, т.е. указание того, какие значения заданных параметров (характеристик) соответствуют описываемой (заданной) сущности.}

\scnheader{единица измерения*}
\scniselement{бинарное отношение}
\scnexplanation{Связки отношения \textbf{\textit{единица измерения*}} связывают знак конкретного \textbf{\textit{измерения с фиксированной единицей измерения}} и некоторую \textit{точную величину}, входящую в тот же конкретный \textit{параметр}, что и первый компонент связок этого конкретного измерения, и которая используется в данном случае в качестве единицы измерения.}

\scnheader{измерение с фиксированной единицей измерения }
\scnrelto{семейство подмножеств}{измерение*}
\scnexplanation{Каждая \textbf{\textit{измерение с фиксированной единицей измерения}} представляет собой подмножество отношения \textit{измерение*} и характеризуется некоторой \textit{единицей измерения*}, которая является элементом того же параметра (семейством сущностей, имеющих значение данного параметра, совпадающее с этой единицей измерения).}

\scnheader{измерение по шкале }
\scnidtf{шкала}
\scnrelto{семейство подмножеств}{измерение*}
\scnexplanation{Каждая \textbf{\textit{измерение по шкале}} представляет собой подмножество отношения \textit{измерение*} и характеризуется не единицей измерения, а некоторой точкой отсчета для данной \textbf{\textit{шкалы}}. Результатом \textbf{\textit{измерения по шкале}} будет некоторая точка шкалы, отстоящая от точки отсчета на определенное расстояние в нужную сторону (меньшую или большую). Понятно, что это расстояние может быть измерено любыми единицами измерения, но его величина при этом останется неизменной.

Не стоит путать измерение по \textbf{\textit{измерение по шкале}}, которое зависит от \textit{нулевой отметки*}, с измерением изменения того же \textit{параметра}, которое характеризуется единицей измерения и не зависит от точки отсчета. Например, не стоит путать дату по некоторому календарю, соответствующую \textit{началу} какого-либо процесса, и \textit{длительность} этого процесса, которая не зависит от выбранного календаря.}
\scnrelfrom{типичная семантическая окрестность}{
\scnfilelong{
\begin{figure}[H]
\centering
\includegraphics[width=0.8\linewidth]{figures/sd_parameters_and_quantities/scale.png}
\end{figure}}
\scntext{комментарий}{В данном примере \textit{ki} обозначает класс сущностей, имеющих точную температуру в 330 К, а \textit{bi} – конкретный пример такой сущности.}}

\scnheader{нулевая отметка*}
\scnidtf{нуль по шкале*}
\scnidtf{начало отсчета*}
\scniselement{бинарное отношение}
\scnexplanation{Связки отношения \textbf{\textit{нулевая отметка*}} связывают знак некоторого \textit{измерения по шкале} со знаком \textit{точной величины} того же \textit{параметра}, которая в рамках данной шкалы принимается за точку отсчета.}

\scnheader{арифметическое выражение на величинах}
\scnexplanation{Каждое \textbf{\textit{арифметическое выражение на величинах}} представляет собой \textit{связку}, компонентами которой являются элементы или подмножества некоторого \textit{количественного параметра}.}

\scnheader{арифметическая операция на величинах}
\scnrelto{семейство подмножеств}{арифметическое выражение на величинах}
\scnexplanation{Каждая \textbf{\textit{арифметическая операция на величинах}} представляет собой \textit{отношение}, элементами которого являются \textit{арифметические выражения на величинах}, то есть множество \textit{арифметических выражений на величинах} какого-либо одного вида.}

\scnheader{сумма величин*}
\scnidtf{сложение величин*}
\scniselement{арифметическая операция на величинах}
\scniselement{квазибинарное отношение}
\scnexplanation{\textbf{\textit{сумма величин*}} – это \textit{арифметическая операция на величинах}, аналогичная \textit{арифметической операции сумма*} для \textit{чисел}.

Первым компонентом связки отношения \textbf{\textit{сумма величин*}} является подмножество некоторого \textit{количественного параметра} (слагаемые \textit{величины}), содержащее два или более элемента, вторым компонентом – элемент этого же \textit{количественного параметра}, значение которого в любой \textit{единице измерения*} является результатом сложения значений всех слагаемых \textit{величин} в той же \textit{единице измерения*}. При несовпадении \textit{единиц измерения} слагаемых величин необходимо воспользоваться соотношениями между \textit{единицами измерения}, которые задаются при помощи операций \textit{произведение величин*} и \textit{возведение величин в степень*}.}


\scnheader{произведение величин*}
\scnidtf{умножение величин*}
\scniselement{арифметическая операция на величинах}
\scniselement{квазибинарное отношение}
\scnexplanation{\textbf{\textit{произведение величин*}} – это \textit{арифметическая операция на величинах}, аналогичная \textit{арифметической операции произведение*} для \textit{чисел}.

Первым компонентом связки отношения \textbf{\textit{произведение величин*}} является \textit{связка}, элементами которой являются либо \textit{величины количественных параметров}, либо \textit{числа}, но при этом хотя бы один элемент должен быть \textit{величиной}. Вторым компонентов является \textit{величина количественного параметра}.

Операция \textbf{\textit{произведение величин*}} может быть использована для задания соотношения между \textit{единицами измерения*} в рамках одного \textit{количественного параметра}.}
\scnrelfrom{описание типичного экземпляра}{
\scnfilelong{
\begin{figure}[H]
\centering
\includegraphics[width=0.5\linewidth]{figures/sd_parameters_and_quantities/multiplicationOfQuantities.png}
\end{figure}}
}
\scnrelfrom{описание типичного экземпляра}{
\scnfilelong{
\begin{figure}[H]
\centering
\includegraphics[width=0.8\linewidth]{figures/sd_parameters_and_quantities/multiplicationOfQuantities2.png}
\end{figure}}
}

\scnheader{возведение величин в степень*}
\scniselement{арифметическая операция на величинах}
\scniselement{бинарное отношение}
\scnexplanation{\textbf{\textit{возведение величин в степень*}} – это \textit{арифметическая операция на величинах}, аналогичная \textit{арифметической операции возведение в степень*} для \textit{чисел}.

Первым компонентом связки отношения \textbf{\textit{возведение величин в степень*}} является ориентированная пара, первым компонентом которой является \textit{величина количественного параметра} (основание степени), вторым – \textit{число} (показатель степени); Вторым компонентом связки отношения \textbf{\textit{возведение величин в степень*}} является \textit{величина количественного параметра} (результат возведения в степень).}
\scnrelfrom{описание типичного экземпляра}{
\scnfilelong{
\begin{figure}[H]
\centering
\includegraphics[width=0.5\linewidth]{figures/sd_parameters_and_quantities/exponentiation.png}
\end{figure}}
}
\scnrelfrom{описание типичного экземпляра}{
\scnfilelong{
\begin{figure}[H]
\centering
\includegraphics[width=0.8\linewidth]{figures/sd_parameters_and_quantities/exponentiationTo2.png}
\end{figure}}
}

\scnheader{большая величина*}
\scniselement{арифметическая операция на величинах}
\scniselement{бинарное отношение}
\scniselement{отношение строгого порядка}
\scnexplanation{\textbf{\textit{большая величина*}} – это \textit{арифметическая операция на величинах}, аналогичная \textit{арифметической операции больше*} для \textit{чисел}.\\
Из двух величин большей является та, \textit{значение} которой в любой \textit{единице измерения*} \textit{больше*} значения другой \textit{величины} в той же \textit{единице измерения}.}

\scnheader{равенство величин*}
\scniselement{арифметическая операция на величинах}
\scniselement{бинарное отношение}
\scniselement{симметричное отношение}
\scniselement{рефлексивное отношение}
\scniselement{транзитивное отношение}
\scnexplanation{\textbf{\textit{равенство величин*}} – это \textit{арифметическая операция на величинах}, аналогичная \textit{арифметической операции равенство*} для \textit{чисел}.

Отношение \textbf{\textit{равенство величин*}} носит исключительно дидактический характер, и явно не указывается, поскольку связывает попарно все элементы одной и той же \textit{величины} каждого \textit{количественного параметра}.}

\scnheader{большая или равная величина*}
\scniselement{арифметическая операция на величинах}
\scniselement{бинарное отношение}
\scniselement{отношение нестрогого порядка}
\scnexplanation{\textbf{\textit{большая или равная величина*}} – это \textit{арифметическая операция на величинах}, аналогичная \textit{арифметической операции больше или равно*} для \textit{чисел}.

В рамках каждой связки данного отношения первая \textit{величина} (первый компонент связки) может быть \textit{большей величиной*} или быть для второй \textit{равной величиной*}.}

\scnheader{действие. измерение}
\scnidtf{измерение как действие}
\scnidtf{действие, направленное на установление связи, принадлежащей отношению измерение* и связывающей величину, которая принадлежит заданному параметру, и которой принадлежит заданная сущность, и соответствующее значение этой величины на некоторой шкале}
\scnidtf{действие, направленное на решение задачи измерения заданного параметра у заданной сущности}
\scnrelto{включение}{действие}

\scnheader{задача. измерение}
\scnidtf{спецификация действия измерения}
\scnidtf{спецификация действия, целью которого является измерение заданного параметра у заданной сущности}
\scnrelto{включение}{задача}

\scnendstruct

\end{SCn}

\scsubsection[\scnmonographychapter{Глава 2.4. Формальные онтологии базовых классов сущностей - множеств, связей, отношений, параметров, величин, чисел, структур, темпоральных сущностей}]{Предметная область и онтология чисел и числовых структур}
\begin{SCn}

\scnsectionheader{\currentname}

\scnstartsubstruct

\scnheader{Предметная область чисел и числовых структур}
\scniselement{предметная область}
\scnsdmainclasssingle{число}
\scnsdclass{натуральное число;целое число;рациональное число;иррациональное число;действительное число;комплексное число;отрицательное число;положительное число;арифметическое выражение;арифметическая операция;Число Пи;Число нуль;Число один;Мнимая единица;числовая структура;система счисления;десятичная система счисления;двоичная система счисления;шестнадцатеричная система счисления; дробь; обыкновенная дробь; десятичная дробь; цифра; арабская цифра; римская цифра}
\scnsdrelation{противоположные числа*;модуль*;сумма*;произведение*;возведение в степень*;больше*;равенство*;больше или равно*}

\scnheader{число}
\scnidtf{множество чисел}
\scnsubset{абстрактная терминальная сущность}
\scnexplanation{\textbf{\textit{число}} – это основное понятие математики, используемое для количественной характеристики, сравнения, нумерации объектов и их частей. Письменными знаками для обозначения чисел служат \textit{цифры}.}

\scnheader{цифра}
\scnidtf{множество цифр}
\scnsubset{внутренний файл ostis-системы}
\scnrelfromlist{включение}{арабская цифра;римская цифра}
\scnexplanation{\textbf{\textit{цифра}} -– это множество файлов, обозначающих вхождение этой цифры во всевозможные записи чисел с помощью этой цифры.}

\scnheader{натуральное число}
\scnidtf{множество натуральных чисел}
\scnexplanation{\textbf{\textit{натуральное число}} – это подмножество множества \textit{целых чисел}, которые используются при счете предметов.}
\scnsubset{целое число}

\scnheader{целое число}
\scnidtf{множество целых чисел}
\scnexplanation{\textbf{\textit{целое число}} – это подмножество множества \textit{рациональных чисел}, получаемых объединением \textit{натуральных чисел} с множеством чисел, \textit{противоположных* натуральным} и \textit{нулём}.}
\scnsubset{рациональное число}

\scnheader{рациональное число}
\scnidtf{множество рациональных чисел}
\scnexplanation{\textbf{\textit{рациональное число}} – это число, представляемое \textit{обыкновенной дробью}, где числитель — \textit{целое число}, а знаменатель — \textit{натуральное число}.}
\scnsubset{действительное число}

\scnheader{дробь}
\scnidtf{множество дробей}
\scnrelfromlist{включение}{обыкновенная дробь; десятичная дробь}
\scnexplanation{\textbf{\textit{дробь}} — это число, состоящее из одной или нескольких равных частей (долей) единицы}

\scnheader{обыкновення дробь}
\scnidtf{множество обыкновенных дробей}
\scnidtf{множество простых дробей}
\scnexplanation{\textbf{\textit{обыкновенная дробь}} - запись \textit{рационального числа} в виде ${\displaystyle \pm {\frac {m}{n}}}$ или ${\pm m/n}$, где ${n\neq 0}$.Горизонтальная или косая черта обозначает знак деления, в результате которого получается частное. Делимое называется числителем дроби, а делитель — знаменателем.}

\scnheader{десятичная дробь}
\scnidtf{множество десятичных дробей}
\scnexplanation{\textbf{\textit{десятичная дробь}} - Десятичная дробь — разновидность дроби, которая представляет собой способ представления действительных чисел в виде ${\pm d_m \ldots d_1 d_0{,} d_{-1} d_{-2} \ldots}$, где , — десятичная запятая, служащая разделителем между целой и дробной частью числа, ${d_{k}}$m — десятичные цифры.}

\scnheader{иррациональное число}
\scnidtf{множество иррациональных чисел}
\scnexplanation{\textbf{\textit{иррациональное число}} – это \textit{вещественное число}, которое не является рациональным, то есть не может быть представлено в виде дроби, где числитель — \textit{целое число}, знаменатель — \textit{натуральное число}. Любое \textbf{\textit{иррациональное число}} может быть представлено в виде бесконечной непериодической десятичной дроби.}
\scnsubset{действительное число}

\scnheader{действительное число}
\scnidtf{вещественное число}
\scnidtf{множество вещественных чисел}
\scnreltoset{объединение}{рациональное число;иррациональное число}
\scnreltoset{разбиение}{положительное число;отрицательное число;$\{$Число нуль$\}$}
\scnexplanation{\textbf{\textit{действительное число}} – это множество чисел, получаемое в результате объединения иррациональных и \textit{рациональных чисел}.}
\scnsubset{комплексное число}

\scnheader{комплексное число}
\scnidtf{множество комплексных чисел}
\scnexplanation{\textbf{\textit{комплексное число}} – число вида \textit{z=a+b*i}, где \textit{a} и \textit{b} – \textit{вещественные числа}, \textit{i} – \textit{Мнимая единица}.}

\scnheader{отрицательное число}
\scnidtf{множество отрицательных чисел}
\scnexplanation{\textbf{\textit{отрицательное число}} – число \textit{меньше*} нуля.}

\scnheader{положительное число}
\scnidtf{множество положительных чисел}
\scnexplanation{\textbf{\textit{положительное число}} – число \textit{больше*} нуля.}

\scnheader{противоположные числа*}
\scniselement{бинарное неориентированное отношение}
\scnexplanation{\textbf{\textit{противоположные числа*}} – \textit{отношение}, связывающее два числа, одно из которых является \textit{отрицательным числом}, второе – \textit{положительным}, при этом \textit{модули*} этих чисел \textit{равны*}.}

\scnheader{модуль*}
\scnidtf{модуль числа*}
\scniselement{бинарное отношение}
\scnexplanation{Связки отношения \textbf{\textit{модуль*}} связывают некоторое \textit{число} (которое может быть как \textit{отрицательным}, так и \textit{положительным}) и другое \textit{число} (всегда \textit{положительное}), которое выражает расстояние от указанного числа до \textit{Числа нуль} в единицах.}

\scnheader{арифметическое выражение}
\scnidtf{множество арифметических выражений}
\scnexplanation{Каждое \textbf{\textit{арифметическое выражение}} представляет собой \textit{связку}, компонентами которой являются \textit{числа} или множества \textit{чисел}.}

\scnheader{арифметическая операция}
\scnidtf{множество арифметических операций}
\scnrelto{семейство подмножеств}{арифметическое выражение}
\scnexplanation{Каждая \textbf{\textit{арифметическая операция}} представляет собой \textit{отношение}, элементами которого являются \textit{арифметические выражения}, то есть множество \textit{арифметических выражений} какого-либо одного вида.}

\scnheader{сумма*}
\scnidtf{сложение*}
\scniselement{арифметическая операция}
\scniselement{квазибинарное отношение}
\scnexplanation{\textbf{\textit{сумма*}} – это арифметическая операция, в результате которой по данным числам (слагаемым) находится новое число (сумма), обозначающее столько единиц, сколько их содержится во всех слагаемых.

Первым компонентом связки отношения \textbf{\textit{сумма*}} является \textit{множество чисел} (слагаемых), содержащее два или более элемента, вторым компонентом – \textit{число}, являющееся результатом сложения.

Отдельно отметим, что каждая связка отношения \textbf{\textit{сумма*}} вида a = b+c может также трактоваться и как запись о вычитании чисел, например b = a-c, в связи с чем \textit{арифметическая операция} разности чисел отдельно не вводится.}
\scnrelfrom{описание типичного экземпляра}{
\scnfilescg{figures/sd_numbers/sum.png}}

\scnheader{произведение*}
\scnidtf{умножение*}
\scniselement{арифметическая операция}
\scniselement{квазибинарное отношение}
\scnexplanation{\textbf{\textit{произведение*}} – это \textit{арифметическая операция}, в результате которой один аргумент складывается столько раз, сколько показывает другой, затем результат складывается столько раз, сколько показывает третий и т.д.

Первым компонентом связки отношения \textbf{\textit{произведение*}} является \textit{множество чисел} (множителей), содержащее два или более элемента, вторым компонентом – \textit{число}, являющееся результатом произведения.

Отдельно отметим, что каждая связка отношения \textbf{\textit{произведение*}} вида a = b*c может также трактоваться и как запись о делении чисел, например b = a/c, в связи с чем \textit{арифметическая операция} деления чисел отдельно не вводится.}
\scnrelfrom{описание типичного экземпляра}{
\scnfilescg{figures/sd_numbers/multiplication.png}}

\scnheader{возведение в степень*}
\scniselement{арифметическая операция}
\scniselement{бинарное отношение}
\scnexplanation{\textbf{\textit{возведение в степень*}} – это \textit{арифметическая операция}, в результате которой число, называемое основанием степени, умножается само на себя столько раз, каков показатель степени.

Первым компонентом связки отношения \textbf{\textit{возведение в степень*}} является ориентированная пара, первым компонентом которой является \textit{число}, которое является основанием степени, вторым – \textit{число}, которое является показателем степени; Вторым компонентом связки отношения \textbf{\textit{возведение в степень*}} является \textit{число}, которое является результатом возведения в степень.

Отдельно отметим, что каждая связка отношения \textbf{\textit{возведение в степень*}} вида a = $b^c$ может также трактоваться и как запись об извлечении корня или взятии логарифма, в связи с чем \textit{арифметические операции} извлечения корня и взятия логарифма отдельно не вводится.}
\scnrelfrom{описание типичного экземпляра}{
\scnfilescg{figures/sd_numbers/pow.png}}

\scnheader{больше*}
\scniselement{арифметическая операция}
\scniselement{бинарное отношение}
\scniselement{отношение строгого порядка}
\scnexplanation{\textbf{\textit{больше*}} – это \textit{арифметическая операция} сравнения чисел. Из двух чисел на координатной прямой больше то, которое расположено правее. Соответственно, первым компонентом связки \textit{отношения} \textbf{\textit{больше*}} является большее из двух \textit{чисел}.}
\scnrelfrom{описание типичного экземпляра}{
\scnfilescg{figures/sd_numbers/more.png}}

\scnheader{равенство*}
\scnidtf{равенство чисел*}
\scniselement{арифметическая операция}
\scniselement{бинарное отношение}
\scniselement{симметричное отношение}
\scniselement{рефлексивное отношение}
\scniselement{транзитивное отношение}
\scnexplanation{\textbf{\textit{равенство*}} – отношение взаимной заменяемости \textit{чисел}, которые именно в силу этой заменяемости и считаются равными. Равные \textit{числа} на числовой прямой совпадают.}
\scnrelfrom{описание типичного экземпляра}{
\scnfilescg{figures/sd_numbers/equality.png}}
\scnheader{больше или равно*}
\scniselement{арифметическая операция}
\scniselement{бинарное отношение}
\scniselement{отношение нестрогого порядка}
\scnexplanation{\textbf{\textit{больше или равно*}} – это \textit{арифметическая операция} сравнения чисел, при которой первое \textit{число} (первый компонент связки) может быть \textit{больше*} второго или \textit{равняться*} ему.}
\scnrelfrom{описание типичного экземпляра}{
\scnfilescg{figures/sd_numbers/more.png}}

\scnheader{Число Пи}
\scniselement{иррациональное число}
\scnexplanation{\textbf{\textit{Число Пи}} – это  математическая константа, равная отношению длины окружности к длине её диаметра.}

\scnheader{Число нуль}
\scnidtf{0}
\scniselement{целое число}
\scnexplanation{\textbf{\textit{Число нуль}} – это \textit{целое число}, разделяющее на числовой прямой \textit{положительные числа} и \textit{отрицательные числа}.}

\scnheader{Число один}
\scnidtf{1}
\scniselement{целое число}
\scniselement{натуральное число}
\scnexplanation{\textbf{\textit{Число один}} – это наименьшее \textit{натуральное число}.}

\scnheader{Мнимая единица}
\scnidtf{i}
\scniselement{комплексное число}
\scnexplanation{\textbf{\textit{Мнимая единица}} – это \textit{число}, при возведении которого в степень 2 результатом будет число, противоположное \textit{Числу один}.}

\scnheader{числовая структура}
\scnsubset{структура}
\scnexplanation{\textbf{\textit{числовая структура}} – \textit{структура}, в состав которой входят знаки \textit{арифметических выражений}, а также знаки их элементов и связи между выражениями и их элементами.}

\scnheader{система счисления}
\scniselement{параметр}
\scnexplanation{Каждая \textbf{\textit{система счисления}} представляет собой класс синтаксически эквивалентных файлов, хранимых в sc-памяти, каждый из которых может являться идентификатором какого-либо \textit{числа}.

Каждая \textbf{\textit{система счисления}} характеризуется алфавитом, т.е. конечным множеством символов (цифр), которые допускается использовать при построении файлов принадлежащих данной \textbf{\textit{системе счисления}}.}

\scnheader{десятичная система счисления}
\scniselement{система счисления}

\scnheader{двоичная система счисления}
\scniselement{система счисления}

\scnheader{шестнадцатеричная система счисления}
\scniselement{система счисления}

\scnendstruct

\end{SCn}

\scsubsection[\scnmonographychapter{Глава 2.3. Структура баз знаний интеллектуальных компьютерных систем нового поколения: иерархическая система предметных областей и онтологий. Онтологии верхнего уровня. Формализация понятий семантической окрестности, предметной области и онтологии в интеллектуальных компьютерных системах нового поколения}]{Предметная область и онтология структур}
\label{sd_structures}
\begin{SCn}

\scnsectionheader{\currentname}

\scnstartsubstruct

\scnheader{Предметная область структур}
\scnsdmainclasssingle{структура}
\scnsdclass{связная структура;несвязная структура;тривиальная структура;нетривиальная структура;структура второго уровня;семантический уровень структурного элемента;количество семантических уровней элементов структуры}

\scnsdrelation{элемент структуры’;непредставленное множество’;полностью представленное множество’;частично представленное множество’;элемент структуры, не являющийся множеством';максимальное множество’;немаксимальное множество’;первичный элемент’;вторичный элемент’;элемент второго уровня’;метасвязь’;полиморфность*;полиморфизм*;гомоморфность*;гомоморфизм*;изоморфность*;изоморфизм*;автоморфность*;автоморфизм*;аналогичность структур*;сходство*;различие*;первичная синтаксическая структура sc-текста*}

\scnheader{структура}
\scnidtf{sc-структура}
\scnidtf{структура, представленная в виде текста SC-кода}
\scnsubdividing{связная структура;несвязная структура}
\scnsubdividing{тривиальная структура;нетривиальная структура}
\scnexplanation{\textbf{\textit{структура}} — множество \textit{sc-элементов}, удаление одного из которых может привести к нарушению целостности этого множества.}

\scnheader{связная структура}
\scnexplanation{\textit{Структуре}, представленной в \textit{SC-коде}, поставим в соответствие орграф, вершинами которого являются \textit{sc-элементы}, а дугами – пары инцидентности, связывающие \textit{sc-коннекторы} с инцидентными им \textit{sc-элементами}, которые являются компонентами указанных \textit{sc-коннекторов}.

Если полученный таким способом орграф является связным орграфом, то исходную структуру будем считать \textbf{\textit{связной структурой}}.}

\scnheader{несвязная структура}
\scnexplanation{\textit{Структуре}, представленной в \textit{SC-коде}, поставим в соответствие орграф, вершинами которого являются \textit{sc-элементы}, а дугами – пары инцидентности, связывающие \textit{sc-коннекторы} с инцидентными им \textit{sc-элементами}, которые являются компонентами указанных \textit{sc-коннекторов}.

Если полученный таким способом орграф не является связным орграфом, то исходную структуру будем считать \textbf{\textit{несвязной структурой}}.}

\scnheader{тривиальная структура}
\scnidtf{структура первого уровня}
\scnexplanation{\textbf{\textit{тривиальная структура}} – \textit{структура}, не содержащая в качестве элементов связок.}

\scnheader{нетривиальная структура}
\scnsuperset{структура второго уровня}
\scnexplanation{\textbf{\textit{нетривиальная структура}} – \textit{структура}, среди элементов которой есть хотя бы одна связка.}

\scnheader{элемент структуры’}
\scniselement{неосновное понятие}
\scnsubdividing{непредставленное множество';полностью представленное множество’;частично представленное множество’;элемент структуры, не являющийся множеством'}
\scnsubdividing{максимальное множество’;немаксимальное множество'}
\scnexplanation{\textbf{\textit{элемент структуры'}} — \textit{неосновное понятие}, \textit{ролевое отношение}, указывающее на все элементы каждой структуры.

В рамках заданной структуры ее элементы можно классифицировать по заданным признакам:
\begin{scnitemize}
\item насколько полно в рамках \underline{заданной \textit{структуры}} представлено множество, обозначаемое \textit{заданным sc-элементом} вместе с соответствующими дугами принадлежности;
\item существуют ли в рамках \underline{заданной \textit{структуры}} \textit{sc-элементы}, обозначающие множества, являющиеся надмножествами того множества, которое обозначается \underline{заданным \textit{sc-элементом}};
\item уровень («этаж») иерархии перехода от знаков к метазнакам для \underline{заданного \textit{sc-элемента}} в рамках заданной \textit{структуры}.
\end{scnitemize}
}

\scnheader{непредставленное множество’}
\scnidtf{множество, не представленное в рамках данной структуры’}
\scnidtf{быть знаком множества, элементы которого не являются элементами данной структуры'}
\scniselement{ролевое отношение}
\scnexplanation{\textbf{\textit{непредставленное множество’}} – \textit{ролевое отношение}, связывающее структуру со знаком множества, все элементы которого не являются элементами данной структуры.}

\scnheader{полностью представленное множество’}
\scnidtf{множество, полностью представленное в рамках данной структуры'}
\scnidtf{множество, все элементы которого являются элементами данной структуры'}
\scnidtf{полностью представленный класс'}
\scniselement{ролевое отношение}
\scnexplanation{\textbf{\textit{полностью представленное множество’}} – \textit{ролевое отношение}, связывающее \textit{структуру} со знаком множества (любого семантического типа – класса, связки или структуры), все элементы которого являются элементами данной \textit{структуры}.}

\scnheader{частично представленное множество’}
\scnidtf{множество, частично представленное в рамках данной структуры'}
\scnidtf{множество, некоторые элементы которого являются элементами данной структуры'}
\scnidtf{быть знаком множества, некоторые элементы которого являются элементами данной структуры'}
\scniselement{ролевое отношение}
\scnexplanation{\textbf{\textit{частично представленное множество’}} – ролевое отношение, связывающее структуру со знаком множества, не все элементы которого являются элементами данной структуры.}

\scnheader{элемент структуры, не являющийся множеством’}
\scniselement{ролевое отношение}

\scnheader{максимальное множество’}
\scnexplanation{\textbf{\textit{максимальное множество’}} – \textit{ролевое отношение}, связывающее \textit{структуру} со знаком множества, для которого не существует множества, которое было бы надмножеством указанного множества и знак которого был бы элементом этой же структуры.}

\scnheader{немаксимальное множество’}
\scnexplanation{\textbf{\textit{немаксимальное множество’}} – \textit{ролевое отношение}, связывающее \textit{структуру} со знаком множества, для которого в рамках данной \textit{структуры} существует множество, являющееся надмножеством указанного множества.}

\scnheader{первичный элемент’}
\scnidtf{первичный элемент данной структуры'}
\scnidtf{sc-элемент первого уровня в рамках данной структуры'}
\scniselement{ролевое отношение}
\scniselement{семантический уровень структурного элемента}
\scnsubset{элемент структуры’}
\scnexplanation{\textbf{\textit{первичный элемент’}} – ролевое отношение, указывающее на элемент \textit{структуры}, являющийся либо терминальным элементом, либо знаком множества, такого что не существует другого элемента этой же структуры, который был бы элементом множества, обозначаемого первым из указанных элементов структуры. При этом соответствующая пара принадлежности может существовать, но в состав данной структуры не входить.}

\scnheader{вторичный элемент’}
\scnidtf{вторичный элемент данной структуры’}
\scnidtf{элемент данной структуры имеющий семантический уровень более 2'}
\scnidtf{непервичный элемент'}
\scniselement{ролевое отношение}
\scnsubset{элемент структуры’}
\scnexplanation{\textbf{\textit{вторичный элемент’}} – ролевое отношение, указывающее на элемент структуры, обозначающий множество, все или некоторые элементы которого являются элементами указанной структуры.}
\scnsuperset{элемент второго уровня’}

\scnheader{элемент второго уровня’}
\scniselement{ролевое отношение}
\scniselement{семантический уровень структурного элемента}
\scnexplanation{\textbf{\textit{элементом второго уровня’}} в рамках заданной \textit{структуры} может быть связка первичных элементов, тривиальная структура из первичных элементов или класс первичных элементов.}

\scnheader{структура второго уровня’}
\scnexplanation{\textbf{\textit{структура второго уровня}} - \textit{структура}, среди элементов которой есть хотя бы один \textit{элемент второго уровня’}.}

\scnheader{семантический уровень структурного элемента}
\scniselement{параметр}
\scnexplanation{\textbf{\textit{семантический уровень структурного элемента}} представляет собой \textit{параметр}, каждый элемент которого является классом 
\textit{sc-дуг принадлежности}, связывающих некоторую \textit{структуру} с теми ее элементами, который имеют одинаковый семантический уровень в рамках данной структуры. Значением данного параметра является число, обозначающее указанный семантический уровень.

\textbf{\textit{семантический уровень структурного элемента}} вычисляется следующим образом:

\begin{scnitemize}
\item элементы структуры, входящие в нее с атрибутом \textit{первичный элемент'} имеют семантический уровень 1;
\item уровень элемента, не являющегося \textit{первичным элементом'} структуры вычисляется путем прибавления 1 к максимальному из уровней элементов этого элемента (множества), входящих в эту же структуру. Например, \textit{sc-дуга}, соединяющая два \textit{первичных элемента' структуры} будет иметь семантический уровень 2, а \textit{sc-элемент}, обозначающий отношение, которому принадлежит указанная \textit{sc-дуга} – семантический уровень 3.
\end{scnitemize}
}
\scnrelfrom{типичная семантическая окрестность}{
\scnfilescg{figures/sd_structures/sem_level_struct_elem.png}
}

\scnheader{количество семантических уровней элементов структуры}
\scniselement{параметр}
\scnexplanation{\textbf{\textit{количество семантических уровней элементов структуры}} – параметр, каждый элемент которого представляет собой класс структур, у которых совпадает максимальный среди семантических уровней элементов этих структур.


Значением данного параметра является число, совпадающее с указанным максимальным семантическим уровнем элементов.}

\scnheader{метасвязь’}
\scniselement{ролевое отношение}
\scnsubset{вторичный элемент’}
\scnexplanation{
\begin{scnenumerate}
    \item Каждая входящая в структуру связь, хотя бы одним компонентом которой является связь, входящая в эту же структуру, элементами которой являются \textit{первичные элементы’} этой структуры, является \textbf{\textit{метасвязью’}} указанной структуры;
    \item Каждая входящая в структуру связь, хотя бы одним компонентом которой является \textbf{\textit{метасвязь’}} этой структуры также является \textbf{\textit{метасвязью’}} указанной структуры;
\end{scnenumerate}
}

\scnheader{полиморфность*}
\scnsubset{соответствие*}
\scniselement{бинарное отношение}
\scnexplanation{\textbf{\textit{полиморфность*}} - это \textit{соответствие}, заданное на \textit{структурах}, при котором каждому элементу из области определения соответствия (первой \textit{структуры}) ставится в соответствие один или более элемент из области значения соответствия (второй \textit{структуры}), при этом существует хотя бы один элемент области определения соответствия, которому соответствуют два или более элемента из области значения соответствия.}

\scnheader{полиморфизм*}
\scniselement{бинарное отношение}

\scnheader{гомоморфность*}
\scnidtf{гомоморфность структур*}
\scnsubset{соответствие*}
\scniselement{бинарное отношение}
\scnexplanation{\textbf{\textit{гомоморфность*}} - это \textit{соответствие}, заданное на \textit{структурах}, при котором каждому элементу из области определения соответствия (первой \textit{структуры}) ставится в соответствие только один элемент из области значения соответствия (второй \textit{структуры}).}
\scnrelfrom{типичная семантическая окрестность}{
\scnfilescg{figures/sd_structures/homomorphism.png}
}

\scnheader{гомоморфизм*}
\scniselement{бинарное отношение}

\scnheader{изоморфность*}
\scnidtf{изоморфное соответствие*}
\scnidtf{изоморфность структур*}
\scnsubset{гомоморфность*}
\scniselement{бинарное отношение}
\scnexplanation{\textbf{\textit{изоморфность*}} - это \textit{гомоморфность*}, при которой для каждого элемента из области значения существует ровно один соответствующий элемент из области определения.}
\scnrelfrom{типичная семантическая окрестность}{
\scnfilescg{figures/sd_structures/isomorphism.png}
}

\scnheader{изоморфизм*}
\scniselement{бинарное отношение}

\scnheader{автомоморфность*}
\scnsubset{гомоморфность*}
\scniselement{бинарное отношение}
\scnexplanation{\textbf{\textit{автоморфность*}} - это \textit{изоморфность*}, у которой область определения соответствия и область значения соответствия совпадают.}
\scnrelfrom{типичная семантическая окрестность}{
\scnfilescg{figures/sd_structures/automorphism.png}}

\scnheader{автоморфизм*}
\scniselement{бинарное отношение}

\scnheader{аналогичность структур*}
\scnsubset{соответствие*}
\scniselement{бинарное отношение}
\scnexplanation{\textbf{\textit{аналогичность структур*}} - \textit{соответствие*}, задаваемое на структурах, и фиксирующее факт наличия некоторой аналогии на подструктурах (подмножествах) указанных структур. Каждой ориентированной паре, принадлежащей \textbf{\textit{аналогичности структур*}} может быть поставлено в соответствие множество пар, задающих \textit{сходства*} некоторых подструктур и \textit{различия*} некоторых подструктур исходных структур.}
\scnrelfrom{типичная семантическая окрестность}{
\scnfilescg{figures/sd_structures/analogy.png}}

\scnheader{сходство*}
\scniselement{бинарное отношение}

\scnheader{различие*}
\scniselement{бинарное отношение}

\scnheader{первичная синтаксическая структура sc-текста*}
\scniselement{бинарное отношение}
\scnexplanation{\textbf{\textit{первичная синтаксическая структура sc-текста*}} - это бинарное отношение, связывающее некоторый \textit{sc-текст} с другим \textit{sc-текстом}, формируемым по следующим правилам:
\begin{scnitemize}
    \item каждому \textit{sc-узлу} первого \textit{sc-текста} соответствует \textit{синглетон} (\textit{знак sc-узла}) в рамках второго \textit{sc-текста};
    \item каждому \textit{sc-коннектору} из первого \textit{sc-текста} в рамках второго \textit{sc-текста} соответствует \textit{синглетон}, обозначающий данный \textit{sc-коннектор} и соединенный с другими \textit{синглетонами} второго \textit{sc-текста} парами инцидентности двух типов, в зависимости от того, началом или концом данного \textit{sc-коннектора} являются обозначаемые этими \textit{синглетонами sc-элементы}. В случае, когда \textit{sc-коннектор} является \textit{sc-ребром}, то достаточно пар инцидентности первого типа.
    \item для каждого \textit{синглетона} в рамках второго \textit{sc-текста} явно указывается синтаксический тип, определяемый типом соответствующего ему элемента из первого \textit{sc-текста} (\textit{знак sc-константы}, \textit{знак sc-узла} и т.п.).
\end{scnitemize}


Стоит отметить, что подобным образом может быть задана синтаксическая структура любого текста, а не только sc-текста. В этом случае понадобятся другие отношения инцидентности другие классы синтаксических типов.}
\scnrelfrom{типичная семантическая окрестность}{
\scnfilescg{figures/sd_structures/primary_sc_syntax.png}}

\scnendstruct \scnendcurrentsectioncomment

\end{SCn}

\scsubsection[\scnmonographychapter{Глава 2.4. Формальные онтологии базовых классов сущностей - множеств, связей, отношений, параметров, величин, чисел, структур, темпоральных сущностей}]{Предметная область и онтология темпоральных сущностей}
\label{sd_temp_entities}
\begin{SCn}

\scnsectionheader{Предметная область и онтология темпоральных сущностей}

\scnstartsubstruct

\scnheader{Предметная область темпоральных сущностей}
\scnidtf{Предметная область темпоральных связей и отношений}
\scnidtf{Предметная область временных сущностей}
\scniselement{предметная область}
\scnsdmainclasssingle{временная сущность}
\scnsdclass{прошлая сущность;настоящая сущность;будущая сущность;временная связь;ситуация;процесс;процесс в sc-памяти;процесс во внешней среде ostis-системы;материальная сущность;воздействие;отношение;класс временных связей;класс временных и постоянных связей;множество;ситуативное множество;неситуативное множество;частично ситуативное множество;темпоральная связь;темпоральное отношение;начало;завершение;длительность;тысячелетие;век;год;месяц;сутки;час;минута;секунда}
\scnsdrelation{воздействующая сущность*;объект воздействия*;начальная ситуация*;причинная ситуация*;конечная ситуация*;событие*;последний добавленный sc-элемент’;темпоральное включение*;темпоральная часть*;начальный этап*;конечный этап*;промежуточный этап*;темпоральное включение без совпадения начальных и конечных моментов*;темпоральное включение с совпадением начальных моментов*;темпоральное включение с совпадением конечных моментов*;темпоральное совпадение*;темпоральное объединение*;темпоральная декомпозиция*;темпоральная смежность*;темпоральная последовательность с промежутком*;темпоральная последовательность с пересечением*;номер тысячелетия';номер века';номер года';номер месяца в году';номер суток в месяце';номер часа в дне';номер минуты в часе';номер секунды в минуте'}

\scnheader{временная сущность}
\scnidtf{временно существующая сущность}
\scnidtf{нестационарная сущность}
\scnidtf{сущность, имеющая и/или начало, и/или конец своего существования}
\scnidtf{sc-элемент, являющийся знаком некоторой временно существующей сущности}
\scnidtf{сущность, обладающая темпоральными характеристиками (длительностью, начальным моментом, конечным моментом и т.д.)}
\scnreltoset{разбиение}{прошлая сущность;настоящая сущность;будущая сущность}
\scnreltoset{разбиение}{временная связь;ситуация;процесс;материальная сущность}
\scnexplanation{Следует отличать:
\begin{scnitemize}
    \item временный характер сущности, обозначаемой \textit{sc-элементом};
    \item временный характер существования самого \textit{sc-элемента} в рамках \textit{sc-памяти};
\end{scnitemize}
В ходе обработки информации каждый \textit{sc-элемент} может быть удален из \textit{sc-памяти}.}

\scnheader{прошлая сущность}
\scnidtf{сущность, существовавшая в прошлом времени}
\scnidtf{сущность прошлого времени}
\scnidtf{сущность, завершившая свое существование}

\scnheader{настоящая сущность}
\scnidtf{сущность, существующая в текущий момент времени}
\scnidtf{сущность, существующая сейчас}
\scnidtf{сущность настоящего времени}

\scnheader{будущая сущность}
\scnidtf{возможно будущая сущность}
\scnidtf{прогнозируемая временная сущность}
\scnidtf{временная сущность, которая может существовать в будущем}
\scnidtf{сущность, которая может или должна начать свое существование в будущем времени}
\scnrelfrom{включение}{инициированное действие}
\scnexplanation{Каждой \textbf{\textit{будущей сущности}} можно поставить в соответствие вероятность ее возникновения.}

\scnheader{временная связь}
\scnidtf{нестационарная связь}
\scnidtf{временно существующая связь}
\scnexplanation{Каждая \textbf{\textit{временная связь}} представляет собой \textit{связку}, принадлежащую множеству \textit{временных сущностей}.

Понятие \textbf{\textit{временной связи}} не следует путать с понятием \textit{темпоральной связи}, которая сама является \textit{постоянной сущностью}, описывающей то, как связаны во времени некоторые \textit{временные сущности}.
}

\scnheader{ситуация}
\scnidtf{состояние}
\scnidtf{временная структура}
\scnidtf{временно существующая структура}
\scnidtf{квазистационарная структура}
\scnidtf{состояние некоторой динамической системы, описываемое с некоторой степенью детализации (подробности)}
\scnidtf{квазистационарная структура, существующая временно (в течение некоторого отрезка времени)}
\scnrelto{включение}{структура}
\scnexplanation{Под ситуацией понимается \textit{структура}, содержащая, по крайней мере, один элемент, который является \textit{временной сущностью}. Наличие в рамках ситуации нескольких \textit{временных сущностей} говорит о том, что существует момент времени (в прошлом, настоящем или будущем), в который все они существуют одновременно.}

\scnheader{процесс}
\scnidtf{процесс преобразования некоторой временной сущности из одного состояния в другое}
\scnidtf{процесс перехода от одной ситуации к другой}
\scnidtf{переходный процесс}
\scnidtf{абстрактный процесс}
\scnidtf{информационная модель некоторых преобразований}
\scnidtf{динамическая sc-модель}
\scnidtf{динамическая структура}
\scnrelfrom{включение}{воздействие}
\scnrelto{включение}{структура}
\scnexplanation{Каждый \textbf{\textit{процесс}} определяется (задается) либо возникновением или исчезновением связей, связывающих заданную \textit{временную сущность} с другими сущностями, либо возникновением или исчезновением связей, связывающих части указанной \textit{временной сущности} с другими сущностями. 

Многим \textbf{\textit{процессам}} можно поставить в соответствие \textit{ситуацию}, которая является его \textit{начальной ситуацией*} и \textit{ситуацию}, которая является его \textit{конечной ситуацией*}, то есть указать \textit{ситуации}, переход между которыми осуществляется в ходе \textbf{\textit{процесса}}.

При этом знаки тех \textit{временных сущностей}, с которыми связаны изменения, описываемые некоторым \textbf{\textit{процессом}}, входят в данный \textbf{\textit{процесс}} как элементы и, при необходимости уточняются дополнительными \textit{ролевыми отношениями}.}
\scnreltoset{разбиение}{процесс в sc-памяти;процесс во внешней среде ostis-системы}

\scnheader{процесс в sc-памяти}

\scnheader{процесс во внешней среде ostis-системы}

\scnheader{материальная сущность}
\scnexplanation{Каждой \textbf{\textit{материальной сущности}} можно поставить в соответствие различные \textit{процессы}, описывающие ее преобразование из одного состояния в другое.}

\scnheader{воздействие}
\scnidtf{процесс, осуществляющийся на основе (в результате) воздействия одной сущности на другую}
\scnrelfrom{включение}{действие}
\scnexplanation{Каждому \textbf{\textit{воздействию}} может быть поставлена в соответствие (1) некоторая \textit{воздействующая сущность*}, т.е. сущность, осуществляющая \textbf{\textit{воздействие}} (в частности, это может быть некоторое физическое поле), и (2) некоторый \textit{объект воздействия*}, т.е. сущность, на которую воздействие направлено. Если \textbf{\textit{воздействие}} связано с \textit{материальной сущностью}, то его объектом воздействия является либо сама эта \textit{материальная сущность}, либо некоторая ее пространственная часть.}

\scnheader{воздействующая сущность*}

\scnheader{объект воздействия*}

\scnheader{начальная ситуация*}
\scnidtf{начальная ситуация процесса*}
\scnidtf{исходная ситуация*}
\scniselement{бинарное отношение}
\scnexplanation{Связки отношения \textbf{\textit{начальная ситуация*}} связывают некоторый \textit{процесс} и некоторую ситуацию, являющуюся начальной для этого \textit{процесса}, и, как правило, изменяемой в течение выполнения этого \textit{процесса}.

Первым компонентом каждой связки отношения \textbf{\textit{начальная ситуация*}} является знак \textit{процесса}, вторым – знак начальной \textit{ситуации}.}

\scnheader{причинная ситуация*}
\scniselement{бинарное отношение}
\scnrelto{включение}{начальная ситуация*}
\scnexplanation{Под причинной ситуацией понимается такая \textit{начальная ситуация*}, которая обладает достаточной полнотой для однозначного задания инициируемого \textit{процесса}.}

\scnheader{конечная ситуация*}
\scnidtf{конечная ситуация процесса*}
\scnidtf{результирующая ситуация*}
\scniselement{бинарное отношение}
\scnexplanation{Связки отношения \textbf{\textit{конечная ситуация*}} связывают некоторый \textit{процесс} и некоторую \textit{ситуацию}, ставшую результатом выполнения этого \textit{процесса}, то есть его следствием.

Первым компонентом каждой связки отношения \textbf{\textit{конечная ситуация*}} является знак \textit{процесса}, вторым – знак конечной \textit{ситуации}.}

\scnheader{событие*}
\scniselement{бинарное отношение}
\scnexplanation{Связки отношения \textbf{\textit{событие*}} связывают знак процесса и ориентированную пару, первым компонентом которой является знак \textit{начальной ситуации*} данного процесса, вторым компонентом – знак \textit{конечной ситуации*} данного процесса.}
\scnrelfrom{типичная семантическая окрестность}{
\scnfilelong{
\begin{figure}[H]
\centering
\includegraphics[width=1\linewidth]{figures/sd_temp_entities/event.png}
\end{figure}
}}

\scnheader{отношение}
\scnreltoset{разбиение}{класс временных связей;класс постоянных связей;класс временных и постоянных связей}

\scnheader{класс временных связей}
\scnidtf{отношение, все связки которого являются нестационарными}
\scnexplanation{В общем случае \textbf{\textit{класс временных связей}} не является \textit{ситуативным множеством}, поскольку факт принадлежности некоторой \textit{временной связи} такому классу следует считать постоянным, а не временным, поскольку временность/постоянство связи и ее семантический тип, задаваемый классом (отношением), это принципиально разные параметры (характеристики, признаки) любой связи.}

\scnheader{класс постоянных связей}
\scnidtf{отношение, все связки которого являются стационарными}

\scnheader{класс временных и постоянных связей}
\scnidtf{отношение, некоторые (но не все) связки которого являются нестационарными}

\scnheader{множество}
\scnreltoset{разбиение}{ситуативное множество;неситуативное множество;частично ситуативное множество}

\scnheader{ситуативное множество}
\scnidtf{полностью ситуативное множество}
\scnexplanation{Под \textbf{\textit{ситуативным множеством}} понимается постоянное множество, у которого все выходящие из него связи принадлежности являются \textit{временными сущностями}.

В частности, ситуативное множество может использоваться как вспомогательная динамическая структура, которая содержит элементы некоторых структур, обрабатываемые в данный момент, например, это может быть копия некоторого множества, из которой постепенно удаляются элементы по мере их просмотра и обработки. В случае, когда такая структура содержит всего один элемент, ее можно считать \underline{указателем} на данный элемент, при этом в разные моменты времени это могут быть разные элементы.}

\scnheader{последний добавленный sc-элемент’}
\scniselement{ролевое отношение}

\scnheader{неситуативное множество}
\scnexplanation{Под \textbf{\textit{неситуативным множеством}} понимается постоянное множество, у которого все выходящие из него связи принадлежности являются \textit{постоянными сущностями}.}

\scnheader{частично ситуативное множество}
\scnexplanation{Под \textbf{\textit{частично ситуативным множеством}} понимается постоянное множество, у которого некоторые (но не все) выходящие из него связи принадлежности являются \textit{временными сущностями}.}

\scnheader{темпоральная связь}
\scnidtf{постоянная связь, описывающая связь во времени между временными сущностями}

\scnheader{темпоральное отношение}
\scnrelto{семейство подмножеств}{темпоральная связь}
\scnidtf{класс темпоральных связей}
\scnidtf{отношение, задающее темпоральные связи между временными сущностями}
\scnhaselement{темпоральное включение*}
\scnhaselement{темпоральное объединение*}
\scnhaselement{темпоральная декомпозиция*}
\scnhaselement{темпоральная смежность*}
\scnhaselement{темпоральная последовательность с промежутком*}
\scnhaselement{темпоральная последовательность с пересечением*}

\scnheader{темпоральное включение*}
\scnexplanation{Связки отношения \textbf{\textit{темпоральное включение*}} связывают две \textit{временные сущности}, период существования одной из которых полностью включается в период существования второй.\\
Первым компонентом каждой связки отношения \textbf{\textit{темпоральное включение*}} является знак \textit{временной сущности}, \textit{длительность} существования которой больше.}
\scnrelfromlist{включение}{темпоральная часть*;темпоральное включение без совпадения начальных и конечных моментов*;темпоральное совпадение*;темпоральное включение с совпадением начальных моментов*;темпоральное включение с совпадением конечных моментов*}

\scnheader{темпоральная часть*}
\scnidtf{этап (период) заданной временной сущности*}
\scnidtf{этап процесса существования временной сущности*}
\scnrelfromlist{включение}{начальный этап*;конечный этап*;промежуточный этап*}
\scnrelfrom{типичная семантическая окрестность}{
\scnfilelong{
\begin{figure}[H]
\centering
\includegraphics[width=1\linewidth]{figures/sd_temp_entities/temporal_part.png}
\end{figure}
}}
\scnrelfrom{иллюстрация}{
\scnfilelong{
\begin{figure}[H]
\centering
\includegraphics[width=1\linewidth]{figures/sd_temp_entities/img_temporal_part.png}
\end{figure}
}}
\scntext{примечание}{Связки отношения \textbf{\textit{темпоральная часть*}} связывают две \textit{временные сущности}, одна из которых является частью другой, например, действие и одно из его поддействий. Соответственно, период существования одной из этих сущностей всегда будет включаться в период существования другой (большей).

В отличие от более общего отношения \textit{темпоральное включение*}, связки которого могут связывать любые \textit{временные сущности}, связки отношения \textbf{\textit{темпоральное включение*}} связывают только \textit{временные сущности}, одна из которых является частью другой.}

\scnheader{начальный этап*}

\scnheader{конечный этап*}

\scnheader{промежуточный этап*}

\scnheader{темпоральное включение без совпадения начальных и конечных моментов*}
\scnidtf{строгое темпоральное включение*}
\scnrelfrom{типичная семантическая окрестность}{
\scnfilelong{
\begin{figure}[H]
\centering
\includegraphics[width=1\linewidth]{figures/sd_temp_entities/strict_temporal_inclusion.png}
\end{figure}
}}
\scnrelfrom{иллюстрация}{
\scnfilelong{
\begin{figure}[H]
\centering
\includegraphics[width=1\linewidth]{figures/sd_temp_entities/img_strict_temporal_inclusion.png}
\end{figure}
}}
%
% темпоральное включение без совпадения начальных и конечных %моментов
%

\scnheader{темпоральное включение с совпадением начальных моментов*}
\scnrelfrom{типичная семантическая окрестность}{
\scnfilelong{
\begin{figure}[H]
\centering
\includegraphics[width=1\linewidth]{figures/sd_temp_entities/temporal_include_with_match_start_points.png}
\end{figure}
}}
\scnrelfrom{иллюстрация}{
\scnfilelong{
\begin{figure}[H]
\centering
\includegraphics[width=1\linewidth]{figures/sd_temp_entities/img_temporal_include_with_match_start_points.png}
\end{figure}
}}

\scnheader{темпоральное включение с совпадением конечных моментов*}
\scnrelfrom{типичная семантическая окрестность}{
\scnfilelong{
\begin{figure}[H]
\centering
\includegraphics[width=1\linewidth]{figures/sd_temp_entities/temporal_include_with_terminal_point_match.png}
\end{figure}
}}
\scnrelfrom{иллюстрация}{
\scnfilelong{
\begin{figure}[H]
\centering
\includegraphics[width=1\linewidth]{figures/sd_temp_entities/img_temporal_include_with_terminal_point_match.png}
\end{figure}
}}

\scnheader{темпоральное совпадение*}
\scnidtf{совпадение начала и завершения*}

\scnheader{темпоральное объединение*}
\scnrelfrom{типичная семантическая окрестность}{
\scnfilelong{
\begin{figure}[H]
\centering
\includegraphics[width=1\linewidth]{figures/sd_temp_entities/temporal_union.png}
\end{figure}
}}
\scnrelfrom{иллюстрация}{
\scnfilelong{
\begin{figure}[H]
\centering
\includegraphics[width=1\linewidth]{figures/sd_temp_entities/img_temporal_union.png}
\end{figure}
}}

\scnheader{темпоральная декомпозиция*}
\scnrelfrom{типичная семантическая окрестность}{
\scnfilelong{
\begin{figure}[H]
\centering
\includegraphics[width=1\linewidth]{figures/sd_temp_entities/temporal_decomposition.png}
\end{figure}
}}
\scnrelfrom{иллюстрация}{
\scnfilelong{
\begin{figure}[H]
\centering
\includegraphics[width=1\linewidth]{figures/sd_temp_entities/img_temporal_decomposition.png}
\end{figure}
}}

\scnheader{темпоральная смежность*}
\scnidtf{строгая темпоральная последовательность (без темпорального промежутка)*}
\scnidtf{темпоральная последовательность без промежутка*}
\scnrelfrom{типичная семантическая окрестность}{
\scnfilelong{
\begin{figure}[H]
\centering
\includegraphics[width=1\linewidth]{figures/sd_temp_entities/temporal_adjacency.png}
\end{figure}
}}
\scnrelfrom{иллюстрация}{
\scnfilelong{
\begin{figure}[H]
\centering
\includegraphics[width=1\linewidth]{figures/sd_temp_entities/img_temporal_adjacency.png}
\end{figure}
}}

\scnheader{темпоральная последовательность с промежутком*}
\scnrelfrom{типичная семантическая окрестность}{
\scnfilelong{
\begin{figure}[H]
\centering
\includegraphics[width=1\linewidth]{figures/sd_temp_entities/temporal_sequence_with_intermediate.png}
\end{figure}
}}
\scnrelfrom{иллюстрация}{
\scnfilelong{
\begin{figure}[H]
\centering
\includegraphics[width=1\linewidth]{figures/sd_temp_entities/img_temporal_sequence_with_intermediate.png}
\end{figure}
}}

\scnheader{темпоральная последовательность с пересечением*}
\scnrelfrom{типичная семантическая окрестность}{
\scnfilelong{
\begin{figure}[H]
\centering
\includegraphics[width=1\linewidth]{figures/sd_temp_entities/temporal_sequence_with_intersection.png}
\end{figure}
}}
\scnrelfrom{иллюстрация}{
\scnfilelong{
\begin{figure}[H]
\centering
\includegraphics[width=1\linewidth]{figures/sd_temp_entities/img_temporal_cross_sequence.png}
\end{figure}
}}

\scnheader{начало}
\scnidtf{класс одновременно начавшихся сущностей}
\scniselement{параметр}
\scnexplanation{Каждый элемент множества \textbf{начало} представляет собой класс \textit{временных сущностей}, у которых совпадает момент начала их существования. Конкретное значение данного \textit{параметра} может быть как \textit{точной величиной}, так и \textit{неточной величиной} или \textit{интервальной величиной}.}
\scnrelfrom{типичная семантическая окрестность}{
\scnfilelong{
\begin{figure}[H]
\centering
\includegraphics[width=1\linewidth]{figures/sd_temp_entities/start.png}
\end{figure}
}}
\scncomment{В данном примере \textit{ki} обозначает класс сущностей, начавших свое существование 19 февраля 2015 года по григорианскому календарю. Конкретные примеры таких сущностей – \textit{bi} и \textit{bj}. \textit{ti} обозначает временную точку григорианского календаря, соответствующую 19 февраля 2015 года.}

\scnheader{завершение}
\scnidtf{конец}
\scnidtf{класс одновременно завершившихся сущностей}
\scniselement{параметр}
\scnexplanation{Каждый элемент множества \textbf{\textit{завершение}} представляет собой класс \textit{временных сущностей}, у которых совпадает конечный момент их существования (момент завершения существования). Конкретное значение данного \textit{параметра} может быть как \textit{точной величиной}, так и \textit{неточной величиной} или \textit{интервальной величиной}.}
\scnrelfrom{типичная семантическая окрестность}{
\scnfilelong{
\begin{figure}[H]
\centering
\includegraphics[width=1\linewidth]{figures/sd_temp_entities/completion.png}
\end{figure}
}}
\scncomment{В данном примере \textit{ki} обозначает класс сущностей, завершивших свое существование 21 февраля 2015 года по григорианскому календарю. Конкретные примеры таких сущностей – \textit{bi} и \textit{bj}. \textit{ti} обозначает временную точку григорианского календаря, соответствующую 21 февраля 2015 года.}

\scnheader{длительность}
\scnidtf{класс временных сущностей, имеющих одинаковую длительность}
\scniselement{параметр}
\scnhaselement{тысячелетие}
\scnhaselement{век}
\scnhaselement{год}
\scnhaselement{месяц}
\scnhaselement{день}
\scnhaselement{час}
\scnhaselement{минута}
\scnhaselement{секунда}
\scnexplanation{Каждый элемент множества \textbf{\textit{длительность}} представляет собой класс \textit{временных сущностей}, у которых совпадает длительность их существования. Конкретное значение данного \textit{параметра} может быть как \textit{точной величиной}, так и \textit{неточной величиной} или \textit{интервальной величиной}.}
\scnrelfrom{типичная семантическая окрестность}{
\scnfilelong{
\begin{figure}[H]
\centering
\includegraphics[width=1\linewidth]{figures/sd_temp_entities/duration.png}
\end{figure}
}}
\scncomment{В данном примере \textit{ki} обозначает класс сущностей, существовавших в течение 2 месяцев. Конкретный пример такой сущности – \textit{bi}.}

\scnheader{тысячелетие}

\scnheader{век}

\scnheader{год}

\scnheader{месяц}

\scnheader{сутки}

\scnheader{час}

\scnheader{минута}

\scnheader{секунда}

\scnheader{номер тысячелетия'}
\scnheader{номер века'}
\scnheader{номер года'}
\scnheader{номер месяца в году'}
\scnheader{номер суток в месяце'}
\scnheader{номер часа в дне'}
\scnheader{номер минуты в часе'}
\scnheader{номер секунды в минуте'}

\scnendstruct

\end{SCn}

\scsubsubsection[\scnmonographychapter{Глава 2.4. Формальные онтологии базовых классов сущностей - множеств, связей, отношений, параметров, величин, чисел, структур, темпоральных сущностей}]{Предметная область и онтология ситуаций и событий, описывающих динамику баз знаний ostis-систем}
\label{sd_temp_know_base}
\begin{SCn}

\scnsectionheader{\currentname}

\scnstartsubstruct

\scntext{введение}{Обработка информации в \textit{sc-памяти} (т.е. динамика базы знаний, хранимой в \textit{sc-памяти}) в конечном счете сводится:
	\begin{scnitemize}
		\item к появлению в \textit{sc-памяти} новых актуальных \textit{sc-узлов} и \textit{sc-коннекторов};
		\item к логическому удалению актуальных \textit{sc-элементов}, т.е. к переводу их в неактуальное состояние (это необходимо для хранения протокола изменения состояния базы знаний, в рамках которого могут описываться действия по удалению \textit{sc-элементов});
		\item к возврату логически удаленных \textit{sс-элементов} в статус актуальных (необходимость в этом может возникнуть при откате базы знаний к какой-нибудь ее прошлой версии);
		\item к физическому удалению \textit{sc-элементов};
		\item к изменению состояния актуальных (логически не удаленных \textit{sc-элементов}) -- \textit{sc-узел} может превратиться в \textit{sc-ребро}, \textit{sc-ребро} может превратиться в \textit{sc-дугу}, \textit{sc-дуга} может поменять направленность, \textit{sc-дуга} общего вида может превратиться в \textit{константную стационарную sc-дугу принадлежности}, и т.д.;
	\end{scnitemize}
	Подчеркнем, что временный характер самого \textit{sc-элемента} (т.к. он может появиться или исчезнуть) никак не связан с возможно временным характером сущности, обозначаемой этим \textit{sc-элементом}. Т.е. временный характер самого sc-элемента и временный характер сущности, которую он обозначает -- абсолютно разные вещи.
	
	Таким образом, следует четко отличать динамику внешнего мира, описываемого базой знаний, а динамику самой базы знаний (динамику внутреннего мира). При этом очень важно, чтобы описание динамики базы знаний также входило в состав каждой базы знаний.
	
	К числу понятий, используемых для описания динамики базы знаний относятся:
	\begin{scnitemize}
		\item логически удаленный sc-элемент;
		\item сформированное множество;
		\item вычисленное число;
		\item сформированное высказывание;
\end{scnitemize}}

\scnheader{Предметная область темпоральных сущностей базы знаний ostis-системы}
\scnidtf{Предметная область, описывающая динамику базы знаний, хранимой в sc-памяти}
\scniselement{предметная область}
\scnsdmainclasssingle{ситуация}
\scnsdclass{sc-элемент;наcтоящий sc-элемент;логически удаленный sc-элемент;число;невычисленное число;вычисленное число;понятие;основное понятие;неосновное понятие;понятие, переходящее из основного в неосновное;понятие, переходящее из неосновного в основное;специфицированная сущность;недостаточно специфицированная сущность;достаточно специфицированная сущность;средне специфицированная сущность;структура;файл;событие в sc-памяти*;элементарное событие в sc-памяти*;событие добавления sc-дуги, выходящей из заданного sc-элемента*;событие добавления sc-дуги, входящей в заданный sc-элемент*;событие добавления sc-ребра, инцидентного заданному sc-элементу*;событие удаления sc-дуги, выходящей из заданного sc-элемента*;событие удаления sc-дуги, входящей в заданный sc-элемент*;событие удаления sc-ребра, инцидентного заданному sc-элементу*;событие удаления sc-элемента*;событие изменения содержимого файла*}

\scnheader{sc-элемент}
\scnreltoset{разбиение}{наcтоящий sc-элемент;логически удаленный sc-элемент}

\scnheader{наcтоящий sc-элемент}
\scniselement{ситуативное множество}

\scnheader{логически удаленный sc-элемент}
\scniselement{ситуативное множество}

\scnheader{число}
\scnsubdividing{невычисленное число;вычисленное число}

\scnheader{невычисленное число}
\scniselement{ситуативное множество}

\scnheader{вычисленное число}

\scnheader{понятие}
\scnsubdividing{основное понятие;неосновное понятие;понятие, переходящее из основного в неосновное;понятие, переходящее из неосновного в основное}

\scnheader{основное понятие}
\scnidtf{основное понятие для данной ostis-системы}
\scniselement{ситуативное множество}
\scnexplanation{К \textbf{\textit{основным понятиям}} относятся те понятия, которые активно используются в системе и могут быть ключевыми элементами sc-агентов. К \textbf{\textit{основным понятиям}} относятся также все неопределяемые понятия.}

\scnheader{неосновное понятие}
\scnidtf{дополнительное понятие}
\scnidtf{вспомогательное понятие}
\scnidtf{неосновное понятие для данной ostis-системы}
\scniselement{ситуативное множество}
\scnexplanation{Каждое \textbf{\textit{неосновное понятие}} должно быть строго определено на основе \textit{основных понятий}. Такие \textbf{\textit{неосновные понятия}} используются только для понимания и правильного восприятия вводимой информации, в том числе, для выравнивания онтологий. Ключевым элементом \textit{sc-агентов} \textbf{\textit{неосновные понятия}} быть не могут.}
\scntext{правило идентификации экземпляров}{В случае, когда некоторое понятие полностью перешло из \textit{основных понятий} в неосновные, то есть стало \textbf{\textit{неосновным понятием}}, и соответствующее ему \textit{основное понятие} (через которое оно определяется) в рамках некоторого внешнего языка имеет одинаковый с ним основной идентификатор, то к идентификатору \textbf{\textit{неосновного понятия}} спереди добавляется знак \#. Если при этом соответствуюшее \textit{основное понятие} имеет в идентификаторе знак \$, добавленный в процессе перехода, то этот знак удаляется. Если указанные понятия имеют разные основные идентификаторы в рамках этого внешнего языка, то никаких дополнительных средств идентификации не используется.

Например:\\
\textit{\#трансляция sc-текста}\\
\textit{\#scp-программа}}

\scnheader{понятие, переходящее из основного в неосновное}
\scniselement{ситуативное множество}

\scnheader{понятие, переходящее из неосновного в основное}
\scniselement{ситуативное множество}
\scntext{правило идентификации экземпляров}{В случае, когда текущее \textit{основное понятие} и соответствующее ему \textbf{\textit{понятие, переходящее из неосновного в основное}} в рамках некоторого внешнего языка имеют одинаковый основной идентификатор, то к идентификатору понятия, переходящего из неосновного в основное спереди добавляется знак \$. Если указанные понятия имеют разные основные идентификаторы в рамках этого внешнего языка, то никаких дополнительных средств идентификации не используется.

Например:\\
\textit{\$трансляция sc-текста}\\
\textit{\$scp-программа}}

\scnheader{специфицированная сущность}
\scnsubdividing{недостаточно специфицированная сущность;достаточно специфицированная сущность;средне специфицированная сущность}

\scnheader{достаточно специфицированная сущность}
\scnexplanation{К \textbf{\textit{достаточно специфицированным сущностям}} предъявляются следующие требования:
\begin{scnitemize}
    \item если сущность не является понятием, то для нее должны быть указаны
    \begin{scnitemizeii}
    \item различные варианты обозначающих ее внешних знаков;
    \item классы, которым она принадлежит;
    \item связки, которыми она связана с другими сущностями (с указанием соответствующего отношения);
    \item значения параметров, которыми она обладает;
    \item те разделы базы знаний, в которых указанная сущность является ключевой;
    \item предметные области, в которые данная сущность входит.
    \end{scnitemizeii}
    \item если специфицированная сущность является понятием, то для нее должны быть указаны:
    \begin{scnitemizeii}
    \item различные варианты внешних обозначений этого понятия;
    \item предметные области, в которых это понятие исследуется;
    \item определение понятия;
    \item пояснения
    \item разделы базы знаний, в которых это понятие является ключевым;
    \item описание примера -- пример экземпляра понятия.
    \end{scnitemizeii}
\end{scnitemize}}

\scnheader{структура}
\scnsubdividing{сформированная структура;несформированная структура}
\scnsubdividing{недостаточно сформированная структура;достаточно сформированная структура;структура, имеющая средний уровень сформированности}

\scnheader{файл}
\scnsubdividing{недостаточно сформированный внутренний файл;достаточно сформированный внутренний файл;внутренний файл, имеющий средний уровень сформированности}

\scnheader{событие в sc-памяти}
\scnsuperset{событие}

\scnheader{элементарное событие в sc-памяти}
\scnsubset{событие в sc-памяти}
\scnexplanation{Под \textbf{\textit{элементарным событием в sc-памяти}} понимается такое \textit{событие}, в результате выполнения которого изменяется состояние только одного \textit{sc-элемента}.}
\scnsubdividing{событие добавления sc-дуги, выходящей из заданного sc-элемента
;событие добавления sc-дуги, входящей в заданный sc-элемент;событие добавления sc-ребра, инцидентного заданному sc-элементу;событие удаления sc-дуги, выходящей из заданного sc-элемента;событие удаления sc-дуги, входящей в заданный sc-элемент;событие удаления sc-ребра, инцидентного заданному sc-элементу;событие удаления sc-элемента;событие изменения содержимого файла}

\scnheader{точечный процесс}
\scnidtf{атомарный процесс}
\scnidtf{условно мгновенный процесс}
\scnidtf{процесс, длительность которого в данном контексте считается несущественной (пренебрежимо малой)}

\scnheader{элементарный процесс}
\scnidtf{процесс, детализация которого на входящие в него подпроцессы в текущем контексте не производится}

\bigskip
\scnendstruct \scnendcurrentsectioncomment

\end{SCn}

\scsubsection[\scnmonographychapter{Глава 2.4. Формальные онтологии базовых классов сущностей - множеств, связей, отношений, параметров, величин, чисел, структур, темпоральных сущностей}]{Предметная область и онтология пространственных сущностей различных форм}
\label{sd_spatial_entities}

\scsubsection[\scnmonographychapter{Глава 2.4. Формальные онтологии базовых классов сущностей - множеств, связей, отношений, параметров, величин, чисел, структур, темпоральных сущностей}]{Предметная область и онтология материальных сущностей}
\label{sd_material_entities}

\scsubsection[\scnmonographychapter{Глава 2.3. Структура баз знаний интеллектуальных компьютерных систем нового поколения: иерархическая система предметных областей и онтологий. Онтологии верхнего уровня. Формализация понятий семантической окрестности, предметной области и онтологии в интеллектуальных компьютерных системах нового поколения}]{Предметная область и онтология семантических окрестностей}
\label{sd_sem_neigh}
\begin{SCn}

\scnsectionheader{\currentname}

\scnstartsubstruct

\scnheader{Предметная область семантических окрестностей}
\scniselement{предметная область}
\scnsdmainclasssingle{семантическая окрестность}
\scnsdclass{семантическая окрестность по инцидентным коннекторам;семантическая окрестность по выходящим дугам;семантическая окрестность по выходящим дугам принадлежности;семантическая окрестность по входящим дугам;семантическая окрестность по входящим дугам принадлежности;полная семантическая окрестность;базовая семантическая окрестность;специализированная семантическая окрестность;пояснение;примечание;правило идентификации экземпляров;терминологическая семантическая окрестность;теоретико-множественная семантическая окрестность;описание декомпозиции;логическая семантическая окрестность;описание типичного экземпляра;сравнительный анализ;иллюстрация}

\scnheader{семантическая окрестность}
\scnidtf{sc-окрестность}
\scnidtf{семантическая окрестность, представленная в виде sc-текста}
\scnidtf{sc-текст, являющийся семантической окрестностью некоторого sc-элемента}
\scnidtf{спецификация заданной сущности, знак которой указывается как ключевой элемент этой спецификации}
\scnidtf{описание заданной сущности, знак которой указывается как ключевой элемент этой спецификации}
\scnsubset{знание}
\scnsuperset{семантическая окрестность по инцидентным коннекторам}
\scnsuperset{полная семантическая окрестность}
\scnsuperset{базовая семантическая окрестность}
\scnsuperset{специализированная семантическая окрестность}
\scnexplanation{\textbf{\textit{семантическая окрестность}} – это знание, являющееся спецификацией (описанием) некоторой сущности, знак которой является ключевым элементом указанного знания. Заметим, что каждая семантическая окрестность в отличие от знаний других видов имеет только один ключевой элемент (ключевой знак, знак описываемой сущности). Заметим также, что многообразие видов семантических окрестностей свидетельствует о многообразии семантических видов описаний различных сущностей.}

\scnheader{семантическая окрестность по инцидентным коннекторам}
\scnsuperset{семантическая окрестность по выходящим дугам}
\scnsuperset{семантическая окрестность по входящим дугам}
\scnexplanation{\textbf{\textit{семантическая окрестность по инцидентным коннекторам}} – это вид семантической окрестности, в которую входят знаки всех коннекторов, инцидентных заданному элементу, а также знаки всех элементов, инцидентных указанным коннекторам.}

\scnheader{семантическая окрестность по выходящим дугам}
\scnsuperset{семантическая окрестность по выходящим дугам принадлежности}
\scnexplanation{\textbf{\textit{семантическая окрестность по выходящим дугам}} – это вид семантической окрестности, в которую входят знаки всех дуг, выходящих из заданного sc-элемента, а также знаки их вторых компонентов, также указывается факт принадлежности этих дуг каким-либо отношениям.}

\scnheader{семантическая окрестность по выходящим дугам принадлежности}
\scnexplanation{\textbf{\textit{семантическая окрестность по выходящим дугам принадлежности}} – это вид семантической окрестности, в которую входят знаки всех дуг принадлежности, выходящих из заданного sc-элемента, а также знаки их вторых компонентов. При необходимости может указывается факт принадлежности этих дуг каким-либо ролевым отношениям.}

\scnheader{семантическая окрестность по входящим дугам}
\scnsuperset{семантическая окрестность по входящим дугам принадлежности}
\scnexplanation{\textbf{\textit{семантическая окрестность по входящим дугам}} – это вид семантической окрестности, в которую входят знаки всех дуг, входящих в заданный sc-элемент, а также знаки их первых компонентов, также указывается факт принадлежности этих дуг каким-либо отношениям.}

\scnheader{семантическая окрестность по входящим дугам принадлежности}
\scnexplanation{\textbf{\textit{семантическая окрестность по входящим дугам принадлежности}} – это вид семантической окрестности, в которую входят знаки всех дуг принадлежности, входящих в заданный sc-элемент, а также знаки их первых компонентов. При необходимости может указывается факт принадлежности этих дуг каким-либо ролевым отношениям.}

\scnheader{полная семантическая окрестность}
\scnidtf{полная спецификация некоторой описываемой сущности}
\scnexplanation{\textbf{\textit{полная семантическая окрестность}} – это вид семантической окрестности, включающий описание всех связей описываемой сущности. 

Структура полной семантической окрестности определяется прежде всего семантической типологией описываемой сущности. 

Так, например, для понятия в полную семантическую окрестность необходимо включить следующую информацию (при наличии):
\begin{scnitemize}
    \item варианты идентификации на различных внешних языках;
    \item принадлежность некоторой предметной области с указанием роли, выполняемой в рамках этой предметной области;
    \item теоретико-множественные связи заданного понятия с другими sc-элементами;
    \item определение или пояснение;
    \item высказывания, описывающие свойства указанного понятия;
    \item задачи и их классы, в которых данное понятие является ключевым
    \item описание типичного примера использования указанного понятия;
    \item экземпляры описываемого понятия.
\end{scnitemize}
Для понятия, являющегося отношением дополнительно указываются:
\begin{scnitemize}
    \item домены;
    \item область определения;
    \item схема отношения;
    \item классы отношений, которым принадлежит описываемое отношение.
\end{scnitemize}
}

\scnheader{базовая семантическая окрестность}
\scnidtf{минимально достаточная семантическая окрестность}
\scnidtf{минимальная спецификация описываемой сущности}
\scnidtf{сокращенная спецификация описываемой сущности}
\scnidtf{основная семантическая окрестность}
\scnexplanation{\textbf{\textit{базовая семантическая окрестность}} – это вид семантической окрестности, содержащий минимальную (краткую) информацию об описываемой сущности

Структура базовой семантической окрестности определяется прежде всего семантической типологией описываемой сущности. 

Так, например, для понятия в базовую семантическую окрестность необходимо включить следующую информацию (при наличии): 
\begin{scnitemize}
    \item варианты идентификации на различных внешних языках;
    \item принадлежность некоторой предметной области с указанием роли, выполняемой в рамках этой предметной области;
    \item определение или пояснение.
\end{scnitemize}
Для понятия, являющегося отношением дополнительно указываются:
\begin{scnitemize}
    \item домены;
    \item область определения;
    \item описание типичного примера использования указанного отношения.
\end{scnitemize}
}

\scnheader{специализированная семантическая окрестность}
\scnsuperset{пояснение}
\scnsuperset{примечание}
\scnsuperset{правило идентификации экземпляров}
\scnsuperset{терминологическая семантическая окрестность}
\scnsuperset{теоретико-множественная семантическая окрестность}
\scnsuperset{логическая семантическая окрестность}
\scnsuperset{описание типичного экземпляра}
\scnsuperset{описание декомпозиции} 
\scnexplanation{\textbf{\textit{специализированная семантическая окрестность}} – это вид семантической окрестности, набор связей для которой уточняется отдельно для каждого класса такой окрестности.}

\scnheader{пояснение}
\scnidtf{sc-пояснение}
\scnexplanation{\textbf{\textit{пояснение}} – знак sc-текста, поясняющего описываемую сущность.}

\scnheader{примечание}
\scnidtf{sc-примечание}
\scnexplanation{\textbf{\textit{примечание}} – знак sc-текста, являющегося примечанием к описываемой сущности. В примечании обычно описываются особые свойства и исключения из правил для описываемой сущности.}

\scnheader{правило идентификации экземпляров}
\scnidtf{правило идентификации экземпляров заданного класса}
\scnexplanation{\textbf{\textit{правило идентификации экземпляров}} – это sc-текст являющийся описанием правил построения идентификаторов элементов заданного класса.}

\scnheader{терминологическая семантическая окрестность}
\scnexplanation{\textbf{\textit{терминологическая семантическая окрестность}}  –  семантическая окрестность, описывающая идентификацию указанной сущности}

\scnheader{теоретико-множественная семантическая окрестность}
\scnexplanation{\textbf{\textit{теоретико-множественная семантическая окрестность}}  –  описание связи описываемого понятия с другими понятиями с помощью теоретико-множественных отношений}

\scnheader{описание декомпозиции}
\scnidtf{семантическая окрестность, описывающая декомпозицию некоторой сущности}
\scnexplanation{\textbf{\textit{описание декомпозиции}}  –  семантическая окрестность, описывающая декомпозицию некоторой сущности на частные сущности}

\scnheader{логическая семантическая окрестность }
\scnexplanation{\textbf{\textit{логическая семантическая окрестность}}  –  семантическая окрестность, описывающая семейство высказываний, описывающих свойства данного понятия}

\scnheader{описание типичного экземпляра}
\scnidtf{описание типичного экземпляра заданного класса}
\scnidtf{типичная семантическая окрестность}
\scnidtf{типичная sc-окрестность}
\scnexplanation{\textbf{\textit{описание типичного экземпляра}} – это sc-текст являющийся описанием типичного примера использования рассматриваемого класса.}

\scnheader{сравнительный анализ}
\scnexplanation{\textbf{\textit{сравнительный анализ}} –  описание сравнительного анализа некоторой сущности с другими сущностями}

\scnheader{иллюстрация}
\scnsubset{специализированная семантическая окрестность}
\scnexplanation{\textbf{\textit{иллюстрация}} –  семантическая окрестность некоторой сущности (сущностей), иллюстрирующая некоторые свойства указанных сущностей, чаще всего, на некотором конкретном примере.}

\scnendstruct

\end{SCn}

\scsubsection[\scnmonographychapter{Глава 2.3. Структура баз знаний интеллектуальных компьютерных систем нового поколения: иерархическая система предметных областей и онтологий. Онтологии верхнего уровня. Формализация понятий семантической окрестности, предметной области и онтологии в интеллектуальных компьютерных системах нового поколения}]{Предметная область и онтология предметных областей}
\label{sd_sd}
\begin{SCn}
\scnsectionheader{\currentname}
\begin{scnsubstruct}
\begin{scnreltovector}{конкатенация сегментов}
\scnitem{Что такое предметная область}
\scnitem{Роли знаков, входящих в состав предметных областей}
\scnitem{Типология предметных областей и отношения, заданных на множестве предметных областей}
\scnitem{Что такое sc-язык}
\end{scnreltovector}
\scnheader{Предметная область предметных областей}
\scnidtf{Предметная область, объектами исследования которой являются предметные области}
\scntext{explanation}{В состав \textbf{\textit{Предметной области предметных областей}} входят структурные спецификации всех \textit{предметных областей}, входящих в состав базы знаний \textit{ostis-системы}, в том числе, самой \textbf{\textit{Предметной области предметных областей}}. Таким образом, \textbf{\textit{Предметная область предметных областей}} является, во-первых, \textit{рефлексивным множеством}, во-вторых, рефлексивной предметной областью, то есть \textit{предметной областью}, одним из объектов исследования которой является она сама.}\scniselement{рефлексивное множество}
\begin{scnhaselementrole}{класс объектов исследования}
предметная область\end{scnhaselementrole}
\begin{scnhaselementrolelist}{класс объектов исследования}
статическая предметная область;динамическая предметная область;понятие;sc-язык
\end{scnhaselementrolelist}
\begin{scnhaselementrolelist}{исследуемое отношение}
понятие предметной области\scnrolesign ;исследуемое понятие\scnrolesign ;максимальный класс объектов исследования\scnrolesign ;немаксимальный класс объектов исследования\scnrolesign ;исследуемый класс первичных элементов\scnrolesign ;исследуемое отношение\scnrolesign ;класс исследуемых структур\scnrolesign ;понятие, исследуемое в дочерней предметной области\scnrolesign ;понятие, исследуемое в материнской предметной области\scnrolesign ;вспомогательное понятие\scnrolesign ;дочерняя предметная область*;дочерняя предметная область по классу первичных элементов*;дочерняя предметная область по исследуемым отношениям*;предметная область sc-языка*
\end{scnhaselementrolelist}
\end{scnsubstruct}
\scnsegmentheader{Что такое предметная область}
\begin{scnsubstruct}
\scnheader{предметная область}
\scnidtf{sc-модель предметной области}
\scnidtf{sc-текст предметной области}
\scnidtf{sc-граф предметной области}
\scnidtf{представление предметной области в \textit{SC-коде}}
\scnsubset{знание}
\scnsubset{бесконечное множество}
\scntext{explanation}{\textbf{\textit{предметная область}} -- это результат интеграции (объединения) частичных семантических окрестностей, описывающих все исследуемые сущности заданного класса и имеющих одинаковый (общий) предмет исследования (то есть один и тот же набор отношений, которым должны принадлежать связки, входящие в состав интегрируемых семантических окрестностей).\textbf{\textit{предметная область}} -- \textit{структура}, в состав которой входят:\begin{scnitemize}
\item \textnormal{основные исследуемые (описываемые) объекты -- первичные и вторичные;}\item \textnormal{различные классы исследуемых объектов;}\item \textnormal{различные связки, компонентами которых являются исследуемые объекты (как первичные, так и вторичные), а также, возможно, другие такие связки -- то есть связки (как и объекты исследования) могут иметь различный структурный уровень;}\item \textnormal{различные классы указанных выше связок (то есть отношения);}\item \textnormal{различные классы объектов, не являющихся ни объектами исследования, ни указанными выше связками, но являющихся компонентами этих связок.}\end{scnitemize}
При этом все классы, объявленные исследуемыми понятиями, должны быть полностью представлены в рамках данной предметной области вместе со своими элементами, элементами элементов и т.д. вплоть до терминальных элементов.Можно говорить о типологии \textbf{\textit{предметных областей}} по разным структурным признакам:\begin{scnitemize}
\item наличие метасвязей;\item наличие исследуемых структур, входящих в состав предметной области;\item наличие исследуемых (смежных, дополнительных) объектов, которых исследуются в других предметных областях;\end{scnitemize}
Понятие \textbf{\textit{предметной области}} является важнейшим методологическим приемом, позволяющим выделить из всего многообразия исследуемого Мира только определенный класс исследуемых сущностей и только определенное семейство отношений, заданных на указанном классе. То есть осуществляется локализация, фокусирование внимания только на этом, абстрагируясь от всего остального исследуемого Мира.Во всем многообразии \textbf{\textit{предметных областей}} особое место занимают\begin{scnitemize}
\item \textit{Предметная область предметных областей}, объектами исследования которой являются всевозможные \textbf{\textit{предметные области}}, а предметом исследования -- всевозможные \textit{ролевые отношения}, связывающие предметные области с их элементами, отношения, связывающие предметные области между собой, отношение, связывающее предметные области с их онтологиями\item \textit{Предметная область сущностей}, являющаяся предметной областью самого высокого уровня и задающая базовую семантическую типологию \textit{sc-элементов}(знаков, входящих в тексты \textit{SC-кода})\item Семейство \textbf{\textit{предметных областей}}, каждая из которых задает семантику и синтаксис некоторого \textit{sc-языка}, обеспечивающего представление онтологий соответствующего вида (например, \textit{теоретико-множественных онтологий}, \textit{логических онтологий}, \textit{терминологических онтологий}, \textit{онтологий задач и способов их решения} и т.д.)\item Семейство \textbf{\textit{предметных областей}} верхнего уровня, в которых классами объектов исследования являются весьма крупные классы сущностей. К таким классам, в частности\begin{scnitemizeii}
\item класс всевозможных \textit{материальных сущностей},\item класс всевозможных \textit{множеств},\item класс всевозможных \textit{связей},\item класс всевозможных \textit{отношений},\item класс всевозможных \textit{структур},\item класс всевозможных \textit{временных (временно существующих, непостоянных сущностей) сущностей},\item класс всевозможных \textit{действий} (акций),\item класс всевозможных \textit{параметров} (характеристик),\item класс \textit{знаний} всевозможного вида \item и т.п.\end{scnitemizeii}
\end{scnitemize}
Каждой \textbf{\textit{предметной области}} можно поставить в соответствие:\begin{scnitemize}
\item семейство соответствующих ей \textit{онтологий} разного вида;\item некий язык (в нашем случае -- язык, построенный на основе \textit{SC-кода}), тексты которого представляют различные фрагменты соответствующей предметной области\end{scnitemize}
Указанные языки будем называть \textit{sc-языками}. Их синтаксис и семантика полностью задается \textit{SС-кодом} и \textit{онтологией} соответствующей \textbf{\textit{предметной области}}. Очевидно, что в первую очередь нас должны интересовать те \textit{sc-языки}, которые соответствуют \textbf{\textit{предметным областям}}, имеющим общий (условно говоря, предметно независимый) характер. К таким предметным областям, в частности, относятся:\begin{scnitemize}
\item \textit{Предметная область множеств}, описывающая множества и различные связи между ними\item \textit{Предметная область отношений и соответствий}\item \textit{Предметная область структур} (в частности, графовых)\item \textit{Предметная область чисел и числовых структур}\item и т.д\end{scnitemize}
Каждому типу знаний можно поставить в соответствие предметную область, которая является результатом интеграции всех знаний данного типа. Эти знания и становятся объектами исследования в рамках указанной предметной области.Понятие \textbf{\textit{предметной области}} может рассматриваться как обобщение понятия алгебраической системы. При этом семантическая структура базы знаний может рассматриваться как иерархическая система различных \textbf{\textit{предметных областей}}.}\scnidtf{система связей некоторого множества объектов исследования, \uline{ключевыми} элементами которой являются:\begin{scnitemize}
\item классы (точнее, знаки классов) объектов исследования (объектов, описываемых этой предметной областью);\item конкретные объекты исследования, обладающие особыми свойствами;\item классы связей, входящих в состав рассматриваемой системы -- отношения, заданные на множестве элементов рассматриваемой системы;\item параметры, заданные на множестве элементов рассматриваемой системы;\item классы структур, являющихся фрагментами рассматриваемой системы.\end{scnitemize}
}
\scnidtf{структура, представляющая собой множество связей (точнее, знаков связей) и соответствующее множество компонентов этих связей, к числу которых относится:\begin{scnitemize}
\item элементы (экземпляры) некоторых заданных классов \uline{объектов исследования} (первичных исследуемых сущностей);\item сами связи, входящие в состав указанной структуры;\item введенные классы объектов исследования;\item введенные отношения (классы связей);\item введенные параметры (классы классов эквивалентных сущностей);\item значения параметров (и, в частности, величины для измеряемых параметров);\item введенные структуры, являющиеся фрагментами (подструктурами) рассматриваемой структуры;\item введенные классы подструктур рассматтриваемой структуры.\end{scnitemize}
}
\scntext{note}{Выделяемые в рамках \textit{базы знаний} интеллектуальной системы \textit{предметные области} и соответствующие им \textit{онтологии} -- это, своего рода, семантические страты, кластеры, позволяющие разложить все хранимые в памяти \textit{знания} по семантическим полочкам при наличии четких критериев, позволяющих \uline{однозначно} определить то, на какой полочке должны находиться те или иные \textit{знания}}\scntext{note}{Существуют предметные области, в которых основным исследуемым понятием является множество всевозможных связей между экземплярами понятий, исследуемых в других предметных областях. Так, например, можно ввести Предметную область треугольников, Предметную область окружностей, а также Предметную область связей между треугольниками и окружностями.}
\end{scnsubstruct}
\scnsegmentheader{Роли знаков, входящих в состав предметной области}
\begin{scnsubstruct}
\scnheader{роль элемента предметной области}
\scnidtf{ролевое отношения, связывающее предметные области с их ключевыми знаками}
\scnidtf{роль ключевого элемента (знака ключевой сущностей) предметной области}
\scnidtf{роль ключевого знака предметной области}
\scnhaselement{класс объектов исследования\scnrolesign}
\scnhaselement{максимальный класс объектов исследования\scnrolesign}
\scnhaselement{ключевой объект исследования\scnrolesign}
\scnhaselement{понятие, используемое в предметной области\scnrolesign}
\scnhaselement{первичный исследуемый элемент предметной области\scnrolesign}
\scnhaselement{вторичный исследуемый элемент предметной области\scnrolesign}
\scnhaselement{неисследуемый элемент предметной области\scnrolesign}
\scnheader{класс объектов исследования\scnrolesign}
\scnidtf{быть классом \uline{первичных} (для данной предметной области) объектов исследования\scnrolesign}
\scntext{note}{Понятие \uline{первичного} объекта исследования для предметной области является понятием \uline{относительным} и абсолютно не зависит от типа и уровня сложности этого объекта. Само исследование (спецификация) таких первичных исследуемых объектов осуществляется:\begin{scnitemize}
\item путем введения различных классов объектов исследования, которым эти объекты принадлежат;\item путем введения различных связок из первичных объектов исследования и различных классов таких связок (отношений), которым принадлежат введенные связки;\item путем введения таких классов первичных объектов исследования, которые являются значениями вводимых параметров;\item путем введения различных структур, состоящих из первичных объектов исследования, из связок таких объектов, из введенных отношений и классов первичных объектов, из введенных параметров и значений этих параметров, и путем введения различных классов таких структур;\item путем введения различных связок из вторичных объектов исследования (т.е. из связок и структур) и путем введения различных классов таких связок;\item и далее можно переходить к объектам исследования более высокого уровня сложности, к параметрам, элементами значений которых являются такие объекты, а также к структурам, элементами которых являются объекты такого уровня и, соответственно, к классам таких структур.\end{scnitemize}
}\scnrelfrom{второй домен}{класс}
\scnsuperset{\begin{scnset}
множество;отношение\\\scnsubset{множество}
;параметр\\\scnsubset{класс классов}
;значение параметра\\\scnsubset{класс}
;структура\\\scnsubset{множество}
;темпоральная сущность;темпоральная сущность базы знаний ostis-системы;семантическая окрестность;предметная область;онтология;логическая формула;действие;задача;информационная конструкция;язык;sc-конструкция;кибернетическая система;интеллектуальная компьютерная система;знание;база знаний;решатель задач интеллектуальной компьютерной системы;интерфейс интеллектуальной компьютерной системы;компьютерная система, основанная на смысловом представлении информации;смысловое представление информации;многоагентная модель решения задач, основанная на смысловом представлении информации;логико-семантическая модель интерфейсов компьютерных систем, основанных на смысловом представлении информации;решатель задач ostis-системы;действие, выполняемое ostis-системой;задача, решаемая ostis-системой:план решения задачи, реализуемый ostis-системой;протокол решения задачи, реализованный ostis-системой;метод решения класса задач, реализуемый ostis-системой;sc-агент\\\scnidtf{внутренний агент ostis-системы, осуществляющий выполнение некоторого вида действий в памяти ostis-системы}
\scnsuperset{sc-агент обработки информации в памяти ostis-системы}
\scnsuperset{sc-агент управления внешними действиями ostis-системы}
;Базовый язык программирования ostis-систем\\\scnidtf{Язык SCP}
;искусственная нейронная сеть;интерфейс ostis-системы;интерфейсное действие пользователя ostis-системы;sc-агент интерфейса ostis-системы;естественный язык;базовый интерпретатор логико-семантических моделей ostis-систем;базовый интерпретатор логико-семантических моделей ostis-систем, реализованный программно на современных компьютерах;семантический ассоциативный компьютер;обучение пользователей ostis-систем;ostis-система персональной адаптивной поддержки всех видов деятельности пользователя;ostis-система управления рецептурным производством;ostis-система, реализующая интеллектуальный портал научно-технических знаний
\end{scnset}
}
\scntext{note}{Здесь приведено семейство тех \textit{классов объектов исследования}, для которых в текущей версии \textit{Стандарта OSTIS} представлены соответствующие \textit{предметные области}. Очевидно, что это семейство должно быть существенно расширено и включить в себя, например, такие \textit{классы} сущностей, как:\begin{scnitemize}
\item материальная сущность\item вещество\item физическое поле\item персона\item пространственная сущность\item юридическое лицо\item предприятие\item географический объект\item и многие другие\end{scnitemize}
}\scntext{note}{Особого внимания требуют те \textit{классы объектов исследования}, которые носят наиболее общий характер  которым соответствуют \textit{предметные области и онтологии} \uline{высокого уровня}. Здесь важна продуманная система декомпозиции всего множества окружающих нас сущностей на иерархическую систему \textit{классов объектов исследования}, которой соответствует иерархическая система \textit{предметных областей и онтологий}, определяющая направления \uline{наследования свойств} исследуемых объектов.}\scnheader{максимальный класс объектов исследования\scnrolesign}
\scnidtf{класс объектов исследования, для которого \uline{в заданной} (!) предметной области отсутствует другой класс объектов исследования, который был бы его надмножеством\scnrolesign}
\scntext{note}{В некоторых предметных областях может быть \uline{несколько} максимальных классов объектов исследования}\scnheader{ключевой объект исследования\scnrolesign}
\scnidtf{особый объект исследования\scnrolesign}
\scnidtf{быть знаком особого исследуемого объекта в рамках заданной предметной области\scnrolesign}
\scnidtf{объект исследования, обладающий особыми свойствами\scnrolesign}
\scnhaselementrole{пример}{$\langle$Предметная область чисел; Нуль$\rangle$}
\scntext{note}{Особыми свойствами Числа \textit{Нуль} являются:\begin{scnitemize}
\item Результатом сложения Числа \textbf{\textit{Нуль}} с любым числом \textbf{\textit{x}} является число \textbf{\textit{x}};\item Результатом умножения Числа \textbf{\textit{Нуль}} на любое число является Число \textbf{\textit{Нуль}}\end{scnitemize}
}\scnhaselement{$\langle$Предметная область чисел; Единица$\rangle$}
\scnhaselement{$\langle$Предметная область чисел; Число Пи$\rangle$}
\scnhaselement{$\langle$Предметная область чисел; Число Е$\rangle$}
\scnheader{ключевой элемент предметной области\scnrolesign}
\scnidtf{входящий в состав предметной области знак ключевой сущности\scnrolesign}
\begin{scnsubdividing}
\scnitem{понятие, используемое в предметной области\scnrolesign}
\scnitem{ключевой объект исследования\scnrolesign \\\scnidtf{знак ключевого объекта исследования\scnrolesign}
}
\end{scnsubdividing}
\scnheader{понятие, используемое в предметной области\scnrolesign}
\scnidtf{понятие, используемое в заданной предметной области не в качестве одного из объектов исследования, а в качестве \uline{ключевого} понятия\scnrolesign}
\scnsubset{используемое понятие\scnrolesign}
\scnidtf{понятие, используемое в sc-знании\scnrolesign}
\scnsubset{используемое понятие*}
\scnidtf{понятие, используемое в знании, которое может быть представлено не только в SC-коде*}
\scntext{note}{Уточнение характера использования понятия в предментной области осуществляется по трем признакам:\begin{scnitemize}
\item семантический тип используемого понятия;\item полнота вхождения элементов понятия в данную предметную область;\item наличие первого упоминания понятия;\item наличие определения понятия или объявления его неопределяемостис подробным пояснением и примерами;\item наличие исследования понятия.\end{scnitemize}
}\scnrelfrom{разбиение}{семантический тип используемого понятия}
\begin{scneqtoset}
\scnitem{класс объектов исследования\scnrolesign}
\scnitem{отношение, используемое в предметной области\scnrolesign}
\scnitem{параметр, используемый в предметной области\scnrolesign}
\scnitem{класс структур, используемый в предметной области\scnrolesign}
\end{scneqtoset}
\scnrelfrom{разбиение}{полнота вхождения элементов понятия в данную предметную область}
\begin{scneqtoset}
\scnitem{используемое понятие, все элементы которого входят в данную предметную область\scnrolesign \\\scntext{note}{Для каждого используемого отношения в предметную область здесь должны входить не только знаки связок, но и все связки целиком с их компонентами}}
\scnitem{используемое понятие, не все элементы которого входят в данную предметную область\scnrolesign}
\end{scneqtoset}
\scnrelfrom{разбиение}{наличие первого упоминания понятия}
\begin{scneqtoset}
\scnitem{понятие, вводимое в данной предметной области\scnrolesign}
\scnitem{понятие, которое в данной предметной области используется, но не вводится\scnrolesign}
\end{scneqtoset}
\scntext{note}{Будем считать, что понятие вводится в данной предметной области в том и только в том случае, если ни в одной предметной области более высокого уровня это понятие не используется. Т.е. речь идет о первом упоминании этого понятия в рамках последовательности предметных областей от родительских к дочерним}\scnrelfrom{разбиение}{наличие определения понятия или объявления его неопределяемости с подробным пояснением и примерами}
\begin{scneqtoset}
\scnitem{понятие, которое в данной предметной области определено или объявлено как неопределяемое}
\scnitem{понятие, которое в данной предметной области не имеет ни определения, ни указания факта его неопределяемости}
\end{scneqtoset}
\scnrelfrom{разбиение}{наличие исследования понятия}
\begin{scneqtoset}
\scnitem{понятие, исследуемое в данной предметной области\scnrolesign}
\scnitem{понятие, которое в данной предметной области испольуется, но не исследуется\scnrolesign}
\end{scneqtoset}
\scntext{note}{Понятие, используемое в базе знаний, может быть введено (впервые упомянуто) в одной предметной области, определено в другой, а исследоваться -- в третьей}\scnheader{первичный исследуемый элемент предметной области\scnrolesign}
\scnidtf{знак первичного объекта исследования в рамках заданной предметной области\scnrolesign}
\scnheader{вторичный исследуемый элемент предметной области\scnrolesign}
\scnidtf{знак вторичного объекта исследования в рамках предметной области\scnrolesign}
\scnsuperset{связка элементов предметной области\scnrolesign}
\scnsuperset{связка первичных элементов предметной области\scnrolesign}
\scnsuperset{метасвязка элементов предметной области\scnrolesign}
\scnsuperset{метасвязка, в число компонентов которой входят связки элементов предметной области\scnrolesign}
\scnsuperset{метасвязка, в число компонентов которой входят классы элементов предметной области\scnrolesign}
\scnsuperset{метасвязка, в число компонентов которой входят структуры элементов предметной области\scnrolesign}
\scnsuperset{класс элементов предметной области\scnrolesign}
\scnsuperset{класс первичных элементов предметной области\scnrolesign}
\scnsuperset{класс связок элементов предметной области\scnrolesign}
\scnsuperset{класс классов элементов предметной области\scnrolesign}
\scnsuperset{класс структур элементов предметной области\scnrolesign}
\scnsuperset{структура элементов предметной области\scnrolesign}
\scnsuperset{тривиальная структура первичных элементов предметной области\scnrolesign}
\scnsuperset{структура, в число подмножеств которой входят связки элементов предметной области вместе со своими компонентами\scnrolesign}
\scnsuperset{структура, в число подмножеств которой входят классы элементов предметной области вместе со своими знаками\scnrolesign}
\scnsuperset{структура, в число подмножеств которой входят другие структуры вместе со своими знаками\scnrolesign}
\scnheader{неисследуемый элемент предметной области\scnrolesign}
\scnidtf{вспомогательный элемент предметной области, исследуемый в другой (смежной) предметной области\scnrolesign}
\scntext{note}{С помощью неисследуемых элементов предметной области описываются и исследуются различные вида связи между элементами, исследуемыми в данной \textit{предметной области} с элементами, исследуемыми в других \textit{предметных областях}. При этом \textit{связки}, компонентами которых являются как исследуемые, так и неисследуемые элементы данной \textit{предметной области} считаются \uline{исследуемыми} связками этой \textit{предметной области}. Примерами неисследуемых элементов, напримр, геометрической \textit{предметной области} являются \textit{числа}, являющиеся \textit{значениями величин} таких \textit{параметров}, как \textit{расстояние}\scnsupergroupsign, \textit{длина}\scnsupergroupsign, \textit{площадь}\scnsupergroupsign, \textit{объем}\scnsupergroupsign, а также различные числовые \textit{отношения} (\textit{сложение}*, \textit{умножение}*, \textit{возведение в степень}*), теоретико-множественные \textit{отношения} (\textit{включение}*, \textit{объединение}*, \textit{пересечение}*, \textit{принадлежность}*)}\newpage\scnheader{понятие}
\scnidtf{концепт}
\scnidtf{класс сущностей, который входит в состав по крайней мере одной предметной области в качестве (в роли) ключевого исследуемого понятия}
\scntext{note}{Семейство всех введенных понятий -- это, своего рода, семантическая система координат, позволяющая специфицировать всевозможные сущности в смысловом пространстве.}\scnidtf{класс сущностей, который по крайней мере в одной \textit{предметной области} объявлен как \textit{понятие} (вводимое, исследуемое или вспомогательное)}
\scntext{note}{Каждому \textit{понятию} соответствует по крайней мере одна \textit{предметная область}, в которой это понятие является \textit{исследуемым понятием} и в которой рассматриваются основные характеристики этого \textit{понятия}. Если же в какой-либо \textit{предметной области} необходимо рассмотреть дополнительные связи этого \textit{понятия} с другими \textit{понятиями}, то оно объявляется как \textit{вспомогательное понятия}\scnrolesign .}\scnidtf{Второй домен Отношения \textit{используемое понятие}*}
\scnrelto{второй домен}{используемое понятие*}
\scnidtf{класс сущностей (класс связок (в т.ч. отношение), класс классов (в т.ч. параметр), класс структур), который по крайней мере в одной \textit{предметной области} является \textit{используемым понятием}\scnrolesign}
\end{scnsubstruct}
\scnsegmentheader{Типология предметных областей и отношения, заданные на множестве предметных областей}
\begin{scnsubstruct}
\scnheader{предметная область}
\begin{scnsubdividing}
\scnitem{статическая предметная область\\\scnidtf{стационарная предметная область}
\scnidtf{\textit{предметная область}, в которой связи между сущностями, входящими в ее состав, не зависят от времени (не меняются во времени), элементами \textbf{\textit{статической предметной области}} не могут быть \textit{временные сущности}}
}
\scnitem{квазистатическая предметная область\\\scnidtf{\textit{предметная область}, решение задач в которой не требует учета темпоральных свойств объектов исследования}
}
\scnitem{динамическая предметная область\\\scnidtf{нестационарная предметная область}
\scnidtf{\textit{предметная область}, которая описывает изменение состояния (в том числе внутренней структуры) объектов исследования и/или изменение конфигурации связей между объектами исследования}
\scnidtf{\textit{предметная область}, в которой некоторые связи между сущностями, входящими в ее состав, меняются со временем (то есть носят ситуационный, нестационарный характер, другими словами, являются \textit{временными сущностями})}
}
\end{scnsubdividing}
\begin{scnsubdividing}
\scnitem{первичная предметная область\\\scnidtf{\textit{предметная область}, объектами исследования которой являются \uline{внешние} сущности (обозначаемые первичными \textit{sc-элементами})}
}
\scnitem{вторичная предметная область\\\scnidtf{метапредметная область}
\scnidtf{\textit{предметная область}, объектами исследования которой являются \textit{sc-множества} (отношения, параметры, структуры, классы структур, знания, языки и др.)}
}
\end{scnsubdividing}
\scntext{note}{Во всем многообразии предметных областей \uline{особое} местро занимают:\begin{scnitemize}
\item \textbf{\textit{Предметная область предметных областей}}, объектами исследования которой являются всевозможные предметные области, а предметом исследования являются -- всевозможные ролевые отношения, связывающие предметные области с их элементами, отношения, связывающие предметные области между собой, отношение, связывающее предметные области с их онтологиями;\item \textbf{\textit{Предметная область сущностей}}, являющаяся предметной областью самого высокого уровня и задающая базовую семантическую типологию sc-элементов (знаков, входящих в тексты SC-кода);\item Семейство \textit{предметных областей}, каждая из которых задает семантику и синтаксис некоторого \textit{sc-языка}, обеспечивающего представление \textit{\uline{онтологий}} соответствующего вида (например, теоретико множественных онтологий терминологических онтологий);\item Семейство \textit{предметных областей} \uline{верхнего уровня}, в которых классами объектов исследования являются весьма крупные классы сущностей. К таким классам, в частности, относятся: \begin{scnitemizeii}
\item класс всевозможных материальных сущностей,\item класс всевозможных множеств,\item класс всевозможных связей,\item класс всевозможных отношений,\item класс всевозможных структур,\item класс всевозможных темпоральных (нестационарных) сущностей,\item класс всевозможных действий (воздествий, акций),\item класс всевозможных параметров (характеристик),\item класс знаний всевозможного вида и т.п.;\end{scnitemizeii}
\item Предметные области абстрактных пространств (в том числе предметные области метрических пространств). Примерами абстрактного пространства являются Евклидово пространство геометрических точек и фигур, пространство всевозможных множеств, числовое пространство, SC-пространство (унифицированное смысловое пространство знаков всевозможных сущностей).\end{scnitemize}
}\scnheader{отношение, заданное на множестве предметных областей}
\scnhaselement{\scnkeyword{дочерняя предметная область*}
}
\scnidtf{частная предметная область*}
\scnidtf{быть частной предметной областью*}
\scnidtf{близлежащий потомок предметной области*}
\scnidtf{сужение предметной области по классу объектов исследования*}
\scnidtf{предметная область, детализирующая описание одного из классов объектов исследования другой (более общей) предметной области*}
\scnidtf{предметная область, объединение классов объектов исследования которой является подмножеством объединения классов объектов исследования заданной предметной области*}
\scniselement{бинарное отношение}
\scniselement{ориентированное отношение}
\scniselement{неролевое отношение}
\scnsuperset{частная предметная область по классу первичных элементов*}
\scnsuperset{частная предметная область по исследуемым отношениям*}
\scntext{explanation}{\textit{дочерняя предметная область*} -- бинарное ориентированное отношение, с помощью которого задается иерархия предметных областей путем перехода от менее детального к более детальному рассмотрению соответствующих исследуемых понятий.}\scntext{note}{Для любой \textit{предметной области} все свойства ее \textit{объектов исследования} \uline{наследуются} всеми ее \textit{дочерними предметными областями*}.}\scnhaselement{\scnkeyword{интеграция предметных областей*}
}
\scnidtf{Отношение, связывающее заданное семейство предметных областей с предметной областью, которая является результатом их интеграции (это не только теоретико-множественное объединение заданных предметных областей, но и уточнение ролей ключевых понятий в интегрированной предметной области, поскольку одно и то же понятие в интегрируемых предметных областях может иметь разные роли).}
\scnhaselement{\scnkeyword{изоморфность предметных областей*}
}
\scnhaselement{\scnkeyword{гомоморфность предметных областей*}
}
\scnheader{расширение семейства исследуемых отношений*}
\scntext{explanation}{Переход от одной предметной области к предметной области с тем же максимальным классомобъектов исследования, но с расширенным семейством отношений и, возможно, с расширенным семейством явно выделенных классов объектов исследования (подклассов максимального класса).}\scnheader{переход к рассмотрению внутренней структуры объектов исследования*}
\scntext{explanation}{Переход от рассмотрения внешних связей объектов исследования к рассмотрению их внутренней структуры путем декомпозиции исследуемых объектов на части и путем включения в число исследуемых объектов тех, которые являются указанными частями.}\scnheader{переход к рассмотрению структур из объектов исследования*}
\scntext{explanation}{Переход от описаниязаданного класса исследуемых объектов к описанию класса всевозможных множеств, элементами которых являются указанные объекты (например, переход от предметной области геометрических точек к предметной области геометрических фигур).}
\end{scnsubstruct}
\end{SCn}


\scsubsection[\scnmonographychapter{Глава 2.3. Структура баз знаний интеллектуальных компьютерных систем нового поколения: иерархическая система предметных областей и онтологий. Онтологии верхнего уровня. Формализация понятий семантической окрестности, предметной области и онтологии в интеллектуальных компьютерных системах нового поколения}]{Предметная область и онтология онтологий}
\label{sd_ontologies}
\begin{SCn}

\scnsectionheader{\currentname}

\scnstartsubstruct

\scnheader{Предметная область онтологий}
\scnidtf{Предметная область теории онтологий}
\scnidtf{Предметная область, объектами исследования которой являются онтологии}
\scniselement{предметная область}
\scnsdmainclasssingle{онтология}
\scnsdclass{интегрированная онтология;структурная спецификация;теоретико-множественная онтология;логическая онтология;логическая иерархия понятий;логическая иерархия высказываний;терминологическая онтология;онтология задач и решений задач;онтология классов задач и способов решения задач}
\scnsdrelation{онтология*;используемые константы*;используемые утверждения*}

\scnheader{онтология}
\scnidtf{система понятий соответствующей предметной области}
\scnidtf{концептуальный каркас (скелет) описания некоторой предметной области}
\scnidtf{концептуальная (семантическая) основа различных языков, обеспечивающих описание объектов исследования, принадлежащих заданной предметной области}
\scnidtf{семантический интерфейс для интеграции знаний по заданной предметной области и для согласованного понимания различными субъектами этих знаний}
\scnidtf{онтология соответствующей предметной области}
\scnidtf{описание концептов и отношений заданной предметной области}
\scnrelto{включение}{знание}
\scnsubdividing{интегрированная онтология;структурная спецификация;теоретико-множественная онтология;логическая иерархия понятий;логическая онтология;логическая иерархия высказываний;терминологическая онтология;онтология задач и решений задач;онтология классов задач и способов решения задач}
\scnexplanation{\textbf{\textit{онтология}} — это вид знаний, каждое из которых является спецификацией (описанием свойств) соответствующей \textit{предметной области}, ориентированной на описание свойств и взаимосвязей понятий, входящих в состав указанной \textit{предметной области}.}

\scnheader{онтология*}
\scnidtf{sc-онтология*}
\scnidtf{быть онтологией предметной области*}
\scnidtf{sc-онтология, специфицирующая заданную предметную область*}
\scnrelfrom{первый домен}{предметная область}
\scnrelfrom{второй домен}{онтология}
\scnexplanation{\textbf{\textit{онтология*}} — это бинарное отношение, связывающее некоторую предметную область с ее онтологией (спецификацией).}

\scnheader{интегрированная онтология}
\scnexplanation{\textbf{\textit{интегрированная онтология}} — это \textit{онтология}, объединяющая все \textit{онтологии} различного вида некоторой \textit{предметной области}.}

\scnheader{структурная спецификация}
\scnexplanation{\textbf{\textit{структурная спецификация}} — это \textit{онтология}, в которой описываются роли понятий, входящих в состав \textit{предметной области}, а также связи специфицируемых \textit{предметных областей} с другими \textit{предметными областями}.}

\scnheader{теоретико-множественная онтология}
\scnexplanation{\textbf{\textit{теоретико-множественная онтология}} — это \textit{онтология}, описывающая теоретико-множественные связи между понятиями заданной \textit{предметной области} (включение, разбиение, объединение, пересечение, разность множеств, область определения, домен, функция).}

\scnheader{логическая онтология}
\scnexplanation{\textbf{\textit{логическая онтология}} — это \textit{онтология}, описание системы высказываний заданной \textit{предметной области}.}

\scnheader{логическая иерархия понятий}
\scnidtf{логическая иерархия понятий, основанная на их определениях}
\scnexplanation{\textbf{\textit{логическая иерархия понятий}} — это \textit{онтология}, являющаяся надстройкой над \textit{логической онтологией}, включающая описание системы определений понятий заданной \textit{предметной области} с указанием набора понятий, через которые определяется каждое определяемое понятие рассматриваемой \textit{предметной области}.}

\scnheader{используемые константы*}
\scniselement{квазибинарное отношение}
\scnrelfrom{второй домен}{понятие}
\scnexplanation{\textbf{\textit{используемые константы*}} — это \textit{отношение}, связывающее некоторое \textit{определение} со множеством понятий, на основании которых определяется соответствующее данному \textit{определению} понятие в рамках рассматриваемой \textit{предметной области}.}

\scnheader{логическая иерархия высказываний}
\scnidtf{логическая система доказательств}
\scnidtf{логическая иерархия утверждений}
\scnidtf{логическая иерархия высказываний, основанная их на базовых доказательствах}
\scnexplanation{\textbf{\textit{логическая иерархия высказываний}} — это \textit{онтология}, являющаяся надстройкой над \textit{логической онтологией} и включающая описание системы утверждений рассматриваемой \textit{предметной области} с указанием набора \textit{утверждений}, через которые доказывается каждое \textit{утверждение}.}

\scnheader{используемые утверждения*}
\scniselement{квазибинарное отношение}
\scnrelfrom{второй домен}{утверждение}
\scnexplanation{\textbf{\textit{используемые утверждения*}} — это \textit{отношение}, связывающее утверждение со множеством утверждений, на основании которых оно доказывается в рамках рассматриваемой \textit{предметной области}.}

\scnheader{терминологическая онтология}
\scnexplanation{\textbf{\textit{терминологическая онтология}} — это \textit{онтология}, описывающая систему основных и неосновных терминов (имен, внешних обозначений), соответствующих концептам и отношениям заданной \textit{предметной области}, а также описание правил построения терминов для сущностей, являющихся элементами (экземплярами) указанных концептов и \textit{отношений}.}

\scnheader{онтология задач и решений задач}
\scnexplanation{\textbf{\textit{онтология задач и решений задач}} — это \textit{онтология}, описывающая задачи и их классы, решаемые в рассматриваемой \textit{предметной области}.}

\scnheader{онтология классов задач и способов решения задач}
\scnexplanation{\textbf{\textit{онтология классов задач и способов решения задач}} — это \textit{онтология}, описывающая способы решения задач и их классов в рамках \textit{предметной области}. Является \textit{метазнанием*} по отношению к \textit{онтологии задач и классов задач}.}

\scnendstruct \scnendcurrentsectioncomment

\end{SCn}

\scsubsection[\scneditors{Василевская А.П.;Зотов Н.В.;Орлов М.К.}\protect\scnmonographychapter{Глава 2.5. Смысловое представление логических формул и высказываний в различного вида логиках}]{Предметная область и онтология логических формул, высказываний и логических sc-языков}
\label{sd_logics}
\begin{SCn}

\scnsectionheader{\currentname}

\scnstartsubstruct

\scnheader{Предметная область логических формул, высказываний и формальных теорий}
\scniselement{предметная область}
\scnsdmainclasssingle{формальная теория}
\scnsdclass{высказывание;атомарное высказывание;неатомарное высказывание;фактографическое высказывание;логическая формула;атомарная логическая формула;неатомарная логическая формула;утверждение;определение;общезначимая логическая формула;противоречивая логическая формула;нейтральная логическая формула;выполнимая логическая формула;невыполнимая логическая формула;тавтология;квантор;формула существования;число значений переменной;кратность существования;единственное существование;логическая формула и единственность;открытая логическая формула;замкнутая логическая формула}
\scnsdrelation{предметная область\scnrolesign;аксиома\scnrolesign;теорема\scnrolesign;подформула*;логическая связка*;импликация*;если\scnrolesign;то\scnrolesign;эквиваленция*;конъюнкция*;дизъюнкция*;строгая дизъюнкция*;отрицание*;всеобщность*;неатомарное существование*;связываемые переменные\scnrolesign}

\scnheader{формальная теория}
\scnexplanation{\textbf{\textit{формальная теория}} — это множество высказываний, которые считаются истинными в рамках данной \textbf{\textit{формальной теории}}. Высказывания могут быть как фактографическими, так и логическими формулами. Некоторые высказывания считаются аксиомами, а другие доказываются на основе других высказываний в рамках этой же \textbf{\textit{формальной теории}}.

Каждая формальная теория интерпретируется (т.е. ее высказывания являются истинными) на какой-либо \textit{предметной области}, которая является максимальным из \textit{фактографических высказываний} (их \textit{объединением*}),  входящих в состав этой \textbf{\textit{формальной теории}}. Каждой \textbf{\textit{формальной теории}} соответствует одна \textit{предметная область}, которая входит в нее под атрибутом \textit{предметная область\scnrolesign}.

Каждая \textbf{\textit{формальная теория}} может рассматриваться как \textit{конъюнктивное высказывание}, априори истинное (с чьей-то точки зрения) при интерпретации на соответствующей \textit{предметной области}.}

\scnheader{предметная область\scnrolesign}
\scniselement{ролевое отношение}
\scnexplanation{\textbf{\textit{предметная область\scnrolesign}} -- это \textit{ролевое отношение}, связывающее \textit{формальную теорию} с \textit{предметной областью}, на которой данная \textit{формальная теория} интерпретируется (в рамках которой истинны \textit{высказывания}, входящие в состав этой \textit{формальной теории}). Другими словами, эта \textit{предметная область} является максимальным фактографическим высказыванием этой \textit{формальной теории}.}

\scnheader{аксиома\scnrolesign}
\scniselement{ролевое отношение}
\scnexplanation{\textbf{\textit{аксиома\scnrolesign}} -- это \textit{ролевое отношение}, связывающее \textit{формальную теорию} с \textit{высказыванием}, истинность которого не  требует доказательства в рамках этой \textit{формальной теории}.}

\scnheader{теорема\scnrolesign}
\scniselement{ролевое отношение}
\scnexplanation{\textbf{\textit{теорема\scnrolesign}} -- это \textit{ролевое отношение}, связывающее \textit{формальную теорию} с \textit{высказыванием}, истинность которого доказывается в рамках этой \textit{формальной теории}.}

\scnheader{высказывание}
\scnsubdividing{атомарное высказывание;неатомарное высказывание}
\scnsubdividing{фактографическое высказывание;логическая формула}
\scnexplanation{Под \textbf{\textit{высказыванием}} понимается некоторая \textit{структура} (в которую входят \textit{sc-константы} из некоторой предметной области и/или \textit{sc-переменные}) или \textit{логическая связка}, которая может трактоваться как истинная или ложная в рамках какой-либо \textit{предметной области}.}
\scnnote{Истинность \textbf{\textit{высказывания}} задается путем указания принадлежности знака этого высказывания \textit{формальной теории}, соответствующей данной \textit{предметной области}. Ложность высказывания задается путем указания принадлежности знака \textit{отрицания*} этого высказывания данной \textit{формальной теории}. Явно указанная непринадлежность \textbf{\textit{высказывания}} \textit{формальной теории} может говорить как о его ложности в рамках данной теории (если это указано рассмотренным выше образом), так и о том, что данное \textbf{\textit{высказывание}} вообще не рассматривается в данной \textit{формальной теории} (например, использует понятия, не принадлежащие данной \textit{предметной области}). 
Одно и то же \textbf{\textit{высказывание}} может быть истинно в рамках одной \textit{формальной теории} и ложно в рамках другой.}
\scnnote{Каждое высказывание может либо содержать только \textit{sc-элементы}, которые не являются знаками других \textbf{\textit{высказываний}} (быть атомарным), либо содержать знаки других \textbf{\textit{высказываний}} (быть неатомарным).}

\scnheader{высказывание формальной теории\scnrolesign}
\scniselement{неосновное понятие}
\scnsubdividing{истинное высказывание\scnrolesign\\
	\scnaddlevel{1}
		\scnidtf{высказывание, истинное в рамках данной формальной теории\scnrolesign}
		\scnidtf{высказывание, знак которого принадлежит данной формальной теории\scnrolesign}
	\scnaddlevel{-1}
	;ложное высказывание\scnrolesign\\
	\scnaddlevel{1}
		\scnidtf{высказывание, ложное в рамках данной формальной теории\scnrolesign}
		\scnidtf{высказывание, знак отрицания которого принадлежит данной формальной теории\scnrolesign}
	\scnaddlevel{-1}
	;нечеткое высказывание\scnrolesign\\
	\scnaddlevel{1}
		\scnidtf{гипотетическое высказывание\scnrolesign}
		\scnidtf{высказывание, возможно истинное или ложное в рамках данной формальной теории\scnrolesign}
		\scnidtf{высказывание, истинное или ложное в рамках данной формальной теории с некоторой вероятностью\scnrolesign}
	\scnaddlevel{-1}
	;бессмысленное высказывание\scnrolesign\\
	\scnaddlevel{1}
		\scnidtf{высказывание, бессмысленное в рамках данной формальной теории\scnrolesign}
		\scnidtf{высказывание, не рассматриваемое в рамках данной формальной теории\scnrolesign}
		\scnexplanation{Высказывание является бессмысленным в рамках заданной формальной теории, если в какое-либо \textit{атомарное высказывание} в его составе (или в само это высказывание, если оно является атомарным) входит какая-либо \textit{sc-константа}, не являющаяся элементом предметной области, описываемой указанной \textit{формальной теорией}.}
	\scnaddlevel{-1}}

\scnheader{атомарное высказывание}
\scnsubset{структура}
\scnsubdividing{атомарное фактографическое высказывание;атомарная логическая формула}
\scndefinition{\textbf{\textit{атомарное высказывание}} -- это \textit{высказывание}, которое содержит хотя бы один \textit{sc-элемент}, не являющийся знаком другого \textit{высказывания}.}
\scnheader{неатомарное высказывание}
\scndefinition{\textbf{\textit{неатомарное высказывание}} -- это \textit{высказывание}, в состав которого входят только знаки других \textit{высказываний}.}
\scnnote{Следует отметить, что мы не можем говорить об истинности либо ложности \textbf{\textit{неатомарного высказывания}} в рамках какой-либо \textit{формальной теории}, в случае, когда невозможно установить истинность либо ложность любого из его элементов в рамках этой же \textit{формальной теории}.}

\scnheader{фактографическое высказывание}
\scnsuperset{атомарное фактографическое высказывание}
\scnexplanation{Под \textit{фактографическим высказыванием} понимается:
\begin{scnitemize}
    \item \textit{атомарное высказывание}, в состав которого не входит ни одна \textit{sc-переменная};
    \item \textit{неатомарное высказывание}, все элементы которого также являются \textbf{\textit{фактографическими высказываниями}}.
\end{scnitemize}
}

\scnheader{логическая формула}
\scnexplanation{Под \textit{логической формулой} понимается:
\begin{scnitemize}
    \item \textit{атомарное высказывание}, в состав которого входит хотя бы одна \textit{sc-переменная};
    \item \textit{неатомарное высказывание}, хотя бы один элемент которого является \textbf{\textit{логической формулой}}.
\end{scnitemize}}
\scnsubdividing{атомарная логическая формула;неатомарная логическая формула}
\scnsubdividing{открытая логическая формула;замкнутая логическая формула}

\scnheader{атомарная логическая формула}
\scnidtf{обобщенная структура}
\scnidtf{атомарная формула существования}
\scnexplanation{Под \textbf{\textit{атомарной логической формулой}} понимается \textit{атомарное высказывание}, которое является \textit{логической формулой}.

По умолчанию \textbf{\textit{атомарная логическая формула}} трактуется как \textit{высказывание} о существовании, то есть наличия в памяти значений, соответствующих всем \textit{sc-переменным}, входящим в состав данной формулы и не попадающих под действие какого-либо другого \textit{квантора} (указанного явно или по умолчанию). Таким образом, на все \textit{sc-переменные}, входящие в состав \textbf{\textit{атомарной логической формулы}} и не попадающие под действие какого-либо другого \textit{квантора}, неявно накладывается квантор \textit{существования*}.}

\scnheader{неатомарная логическая формула}
\scnsubdividing{общезначимая логическая формула;противоречивая логическая формула;нейтральная логическая формула}
\scnsubdividing{выполнимая логическая формула;невыполнимая логическая формула}
\scnsuperset{тавтология}
\scnexplanation{Под \textbf{\textit{неатомарной логической формулой}} понимается \textit{неатомарное высказывание}, которое является \textit{логической формулой}.

Для того, чтобы рассмотреть типологию \textbf{\textit{неатомарных логических формул}}, будем говорить, что исследуется истинность самой \textbf{\textit{неатомарной логической формулы}} и всех ее \textit{подформул*} в рамках одной и той же \textit{формальной теории}, при этом не важно, какой именно. Также считается, что в рассматриваемой \textit{формальной теории} каждая \textit{подформула*} рассматриваемой \textbf{\textit{неатомарной логической формулы}} в рамках этой \textit{формальной теории} может однозначно трактоваться как либо истинная, либо ложная. В противном случае мы не можем говорить об истинности либо ложности исходной \textbf{\textit{неатомарной логической формулы}} в рамках этой \textit{формальной теории}.}

\scnheader{подформула*}
\scnidtf{частная формула*}
\scniselement{бинарное отношение}
\scniselement{ориентированное отношение}
\scniselement{транзитивное отношение}
\scndefinition{Будем называть \textbf{\textit{подформулой*}} \textit{неатомарной логической формулы} \textbf{\textit{fi}} любую \textit{логическую формулу} \textbf{\textit{fj}}, являющуюся элементом исходной формулы \textbf{\textit{fi}}, а также любую \textbf{\textit{подформулу*}} формулы \textbf{\textit{fj}}.}
\scnrelfrom{описание примера}{
\scnfilescg{figures/sd_logical_formulas/subformula.png}}

\scnheader{утверждение}
\scnidtf{текст логической формулы}
\scndefinition{\textbf{\textit{утверждение}} -- это \textit{семантическая окрестность} некоторой \textit{логической формулы}, в которую входит полный текст этой \textit{логической формулы}, а также факт принадлежности этой \textit{логической формулы} некоторой \textit{формальной теории}.}
\scnexplanation{Знак \textit{логической формулы}, семантическая окрестность которой представляет собой утверждение, является \textit{главным ключевым sc-элементом\scnrolesign} в рамках этого \textbf{\textit{утверждения}}. Знаки понятий соответствующей \textit{предметной области}, которые входят в состав какой-либо \textit{подформулы*} указанной \textit{логической формулы}, будут \textit{ключевыми sc-элементами\scnrolesign} в рамках этого \textbf{\textit{утверждения}}.

Полный текст некоторой \textit{логической формулы} включает в себя:
\begin{scnitemize}
    \item знак самой этой \textit{логической формулы};
    \item знаки всех ее \textit{подформул*};
    \item элементы всех \textit{логических формул}, знаки которых попали в данную структуру;
    \item все пары принадлежности, связывающие \textit{логические формулы}, знаки которых попали в данную структуру, с их компонентами.
\end{scnitemize}
Таким образом, факт принадлежности (истинности) логической формулы нескольким \textit{формальным теориям} будет порождать новое утверждение для каждой такой \textit{формальной теории}. Текст \textbf{\textit{утверждения}} входит в состав \textit{логической онтологии}, соответствующей \textit{предметной области}, на которой интерпретируется \textit{главный ключевой sc-элемент\scnrolesign} данного утверждения.}
\scntext{правило идентификации экземпляров}{\textbf{\textit{утверждения}} в рамках \textit{Русского языка} именуются по следующим правилам:
\begin{scnitemize}
    \item в начале идентификатора пишется сокращение \textbf{Утв.};
    \item далее в круглых скобках через точку с запятой перечисляются основные идентификаторы \textit{ключевых \mbox{sc-элементов}\scnrolesign} данного \textbf{\textit{утверждения}}. Порядок определяется в каждом конкретном случае в зависимости от того, свойства каких из этих \textit{понятий} описывает данное \textbf{\textit{утверждение}} в большей или меньшей степени.
\end{scnitemize}
}
\scnaddlevel{1}
\scntext{описание примера}{\textit{Утв. (треугольник; сторона*)}}
\scnnote{Могут быть исключения для \textbf{\textit{утверждений}}, названия которых закрепились исторически, например, \textit{Теорема Пифагора}, \textit{Аксиома о прямой и точке}.}
\scnaddlevel{-1}

\scnheader{определение}
\scnidtf{текст определения}
\scnsubset{утверждение}
\scndefinition{\textbf{\textit{определение}} -- это \textit{утверждение}, \textit{главным ключевым sc-элементом\scnrolesign} которого является связка \textit{эквиваленции*}, однозначно определяющая некоторое понятие на основе других понятий.}
\scnnote{Каждое определение имеет ровно один \textit{ключевой sc-элемент\scnrolesign} (не считая \textit{главного ключевого sc-элемента\scnrolesign}).}
\scnnote{Для одного и того же понятия в рамках одной \textit{формальной теории} может существовать несколько \textit{утверждений об эквиваленции*}, однозначно задающих некоторое понятие на основе других, однако только одно такое \textit{утверждение} в рамках этой \textit{формальной теории} может быть отмечено как \textbf{\textit{определение}}. Остальные \textit{утверждения об эквиваленции*} могут трактоваться как \textit{пояснения} данного понятия.}
\scntext{правило идентификации экземпляров}{\textbf{\textit{определения}} в рамках \textit{Русского языка} именуются по следующим правилам:
\begin{scnitemize}
    \item в начале идентификатора пишется сокращение \textbf{Опр.};
    \item далее в круглых скобках через точку с запятой записывается основной идентификатор  \textit{ключевого sc-элемента\scnrolesign} данного \textbf{\textit{определения}}.
\end{scnitemize}
}
\scnaddlevel{1}
\scntext{описание примера}{\textit{Опр. (связный граф)}}
\scnaddlevel{-1}

\scnheader{общезначимая логическая формула}
\scnsubset{выполнимая логическая формула}
\scnsubset{тавтология}
\scndefinition{\textbf{\textit{общезначимая логическая формула}} -- это \textit{логическая формула}, для которой не существует \textit{формальной теории}, в рамках которой она была бы ложной с учетом истинности и ложности всех ее \textit{подформул*} в рамках этой же \textit{формальной теории}.}

\scnheader{противоречивая логическая формула}
\scnsubset{невыполнимая логическая формула}
\scnsubset{тавтология}
\scndefinition{\textbf{\textit{противоречивая логическая формула}} -- это \textit{логическая формула}, для которой не существует \textit{формальной теории}, в рамках которой она была бы истинной с учетом истинности и ложности всех ее \textit{подформул*} в рамках этой же \textit{формальной теории}.}

\scnheader{нейтральная логическая формула}
\scnsubset{выполнимая логическая формула}
\scndefinition{\textbf{\textit{нейтральная логическая формула}} -- это \textit{логическая формула}, для которой существует хотя бы одна \textit{формальная теория}, в рамках которой эта формула ложна, и хотя бы одна \textit{формальная теория}, в рамках которой эта формула истинна.}

\scnheader{выполнимая логическая формула}
\scndefinition{\textbf{\textit{выполнимая логическая формула}} -- это \textit{логическая формула}, для которой существует хотя бы одна \textit{формальная теория}, в рамках которой эта формула истинна.}

\scnheader{невыполнимая логическая формула}
\scndefinition{\textbf{\textit{невыполнимая логическая формула}} -- это \textit{логическая формула}, для которой существует хотя бы одна \textit{формальная теория}, в рамках которой эта формула ложна.}

\scnheader{тавтология}
\scnidtf{тождественно истинная формула}
\scndefinition{\textbf{\textit{тавтология}} -- это \textit{логическая формула}, которая является либо только истинной, либо только ложной в рамках всех \textit{формальных теорий}, в которых можно установить ее истинность или ложность.}
\scnexplanation{\textbf{\textit{тавтология}} -- это такая \textit{логическая формула}, которая является либо \textit{общезначимой логической формулой}, либо \textit{противоречивой логической формулой}.}

\scnheader{логическая связка*}
\scnidtf{неатомарная логическая формула}
\scnidtf{логический оператор*}
\scnidtf{пропозициональная связка*}
\scniselement{класс связок разной мощности}
\scnrelto{семейство подмножеств}{неатомарное высказывание}
\scndefinition{\textbf{\textit{логическая связка*}} -- это отношение (класс связок), связками которого являются \textit{высказывания}.}
\scnexplanation{\textbf{\textit{логическая связка*}} -- это \textit{отношение}, областью определения которого является множество \textit{высказываний}, при этом само это отношение и некоторые его подмножества могут быть \textit{классами связок разной мощности}.}

\scnheader{импликация*}
\scnidtf{логическое следование*}
\scnsubset{логическая связка*}
\scniselement{бинарное отношение}
\scniselement{ориентированное отношение}
\scndefinition{\textbf{\textit{импликация*}} -- это множество импликативных \textit{неатомарных высказываний}, каждое из которых состоит из посылки (первый компонент \textit{высказывания}) и следствия (второй компонент \textit{высказывания}). Каждое импликативное \textit{высказывание} ложно в рамках некоторой \textit{формальной теории} в том случае, когда его посылка истинна, а заключение ложно в рамках этой же \textit{формальной теории}. В других случаях такое \textit{высказывание} истинно.}
\scnnote{По умолчанию на все переменные, входящие в обе части высказывания об \textbf{\textit{имликации*}} (или хотя бы одну из \textit{подформул*} каждой части) неявно накладывается квантор \textit{всеобщности*}, при условии, что эти переменные не связаны другим \textit{квантором}, указанным явно.}
\scnrelfrom{описание примера}{
\scnfilescg{figures/sd_logical_formulas/implication.png}}

\scnheader{если\scnrolesign}
\scnsubset{1\scnrolesign}
\scniselement{ролевое отношение}
\scndefinition{\textbf{\textit{если\scnrolesign}} -- это \textit{ролевое отношение}, используемое в связках \textit{импликации*} для указания посылки.}

\scnheader{то\scnrolesign}
\scnsubset{2\scnrolesign}
\scniselement{ролевое отношение}
\scndefinition{\textbf{\textit{то\scnrolesign}} -- это \textit{ролевое отношение}, используемое в связках \textit{импликации*} для указания следствия.}

\scnheader{эквиваленция*}
\scnidtf{эквивалентность*}
\scnsubset{логическая связка*}
\scniselement{бинарное отношение}
\scniselement{неориентированное отношение}
\scndefinition{\textbf{\textit{эквиваленция*}} -- это множество \textit{высказываний} об эквивалентности, каждое из которых истинно в рамках некоторой \textit{формальной теории} только в тех случаях, когда оба его компонента одновременно либо истинны в рамках этой же \textit{формальной теории}, либо ложны.}
\scnnote{По умолчанию на все переменные, входящие в обе части высказывания об \textbf{\textit{эквиваленции*}} (или хотя бы одну из \textit{подформул*} каждой части) неявно накладывается квантор \textit{всеобщности*}, при условии, что эти переменные не связаны другим \textit{квантором}, указанным явно.}
\scnrelfrom{описание примера}{
\scnfilescg{figures/sd_logical_formulas/equivalent.png}}

\scnheader{конъюнкция*}
\scnidtf{логическое и*}
\scnidtf{логическое умножение*}
\scnsubset{логическая связка*}
\scniselement{неориентированное отношение}
\scniselement{класс связок разной мощности}
\scndefinition{\textbf{\textit{конъюнкция*}} -- это множество конъюнктивных \textit{высказываний}, каждое из которых истинно в рамках некоторой \textit{формальной теории} только в том случае, когда все его компоненты истинны в рамках этой же \textit{формальной теории}.}
\scnrelfrom{описание примера}{
\scnfilescg{figures/sd_logical_formulas/conjunction.png}}

\scnheader{дизъюнкция*}
\scnidtf{логическое или*}
\scnidtf{логическое сложение*}
\scnidtf{включающее или*}
\scnsubset{логическая связка*}
\scniselement{неориентированное отношение}
\scniselement{класс связок разной мощности}
\scndefinition{\textbf{\textit{дизъюнкция*}} -- это множество дизъюнктивных \textit{высказываний}, каждое из которых истинно в рамках некоторой \textit{формальной теории} только в том случае, когда хотя бы один его компонент является истинным в рамках этой же \textit{формальной теории}.}
\scnrelfrom{описание примера}{
\scnfilescg{figures/sd_logical_formulas/disjunction.png}}

\scnheader{строгая дизъюнкция*}
\scnidtf{сложение по модулю 2*}
\scnidtf{исключающее или*}
\scnidtf{альтернатива*}
\scnsubset{логическая связка*}
\scniselement{неориентированное отношение}
\scniselement{класс связок разной мощности}
\scndefinition{\textbf{\textit{строгая дизъюнкция*}} -- это множество строго дизъюнктивных \textit{высказываний}, каждое из которых истинно в рамках некоторой \textit{формальной теории} только в том случае, когда ровно один его компонент является истинным в рамках этой же \textit{формальной теории}.}
\scnrelfrom{описание примера}{
\scnfilescg{figures/sd_logical_formulas/strictDisjunction.png}}

\scnheader{отрицание*}
\scnsubset{логическая связка*}
\scnsubset{синглетон}
\scndefinition{\textbf{\textit{отрицание*}} -- это множество \textit{высказываний} об отрицании, каждое из которых истинно в рамках некоторой \textit{формальной теории} только в том случае, когда его единственный элемент является ложным в рамках этой же \textit{формальной теории}.}
\scnrelfrom{описание примера}{
\scnfilescg{figures/sd_logical_formulas/negation.png}}

\scnheader{квантор}
\scnsubset{логическая связка*}
\scndefinition{\textbf{\textit{квантор}} -- это \textit{отношение}, каждая связка которой задает истинность множества \textit{логических формул}, входящих в ее состав, при выполнении дополнительных условий, связанных с некоторыми из переменных, входящих в состав этих \textit{логических формул}.}
\scnnote{Будем говорить, что переменные связаны \textbf{\textit{квантором}} или попадают под область действия данного \textbf{\textit{квантора}} (имея в виду конкретную связку конкретного \textbf{\textit{квантора}}). Таким образом, в состав каждой связки каждого \textbf{\textit{квантора}} входит \textit{атомарная формула}, являющаяся \textit{тривиальной структурой}, в которой перечислены переменные, связанные данным \textbf{\textit{квантором}}.}

\scnheader{всеобщность*}
\scnidtf{квантор всеобщности*}
\scnidtf{квантор общности*}
\scniselement{квантор}
\scniselement{ориентированное отношение}
\scniselement{класс связок разной мощности}
\scndefinition{\textbf{\textit{всеобщность}} -- это \textit{квантор}, для каждой связки которого, истинной в рамках некоторой \textit{формальной теории}, выполняется следующее утверждение: все формулы, входящие в состав этой связки истинны в рамках этой же \textit{формальной теории} при всех (любых) возможных значениях всех элементов множества \textit{связываемых переменных\scnrolesign} входящего в эту связку.}
\scnnote{Каждая связка \textit{квантора} \textbf{\textit{всеобщность*}} может быть представлена как \textit{конъюнкция*} (потенциально бесконечная) исходных \textit{логических формул}, входящих в состав этой связки, в каждой из которых все \textit{связанные переменные\scnrolesign} заменены на их возможные значения.}
\scnnote{Квантор \textbf{\textit{всеобщности*}} зачастую обозначается "$\forall$" \ и читается как "для всех"{}, "для каждого"{}, "для любого"{} или "все"{}, "каждый"{}, "любой".}
\scnrelfrom{описание примера}{
\scnfilescg{figures/sd_logical_formulas/universality.png}}

\scnheader{формула существования}
\scnidtf{существование*}
\scnsubdividing{атомарная логическая формула;неатомарное существование*}

\scnheader{неатомарное существование*}
\scnidtf{квантор неатомарного существования*}
\scniselement{квантор}
\scniselement{ориентированное отношение}
\scniselement{класс связок разной мощности}
\scndefinition{\textbf{\textit{неатомарное существование*}} -- это \textit{квантор}, для каждой связки которого, истинной в рамках некоторой \textit{формальной теории}, выполняется следующее утверждение: существуют значения всех элементов множества \textit{связываемых переменных\scnrolesign} входящего в эту связку, такие, что все формулы, входящие в состав этой связки истинны в рамках этой же \textit{формальной теории}.}
\scnnote{Каждая связка \textit{квантора} \textbf{\textit{неатомарное существование*}} может быть представлена как \textit{дизъюнкция*} (потенциально бесконечная) исходных \textit{логических формул}, входящих в состав этой связки, в каждой из которых все \textit{связанные переменные\scnrolesign} заменены на их возможные значения.}
\scnnote{квантор \textbf{\textit{существования*}} зачастую обозначается "$\exists$" \ и читается как "существует"{}, "для некоторого"{}, "найдется".}
\scnrelfrom{описание примера}{
\scnfilescg{figures/sd_logical_formulas/non_atomicExistence.png}}

\scnheader{число значений переменной}
\scniselement{параметр}
\scnexplanation{Каждый элемент \textit{параметра} \textbf{\textit{число значений переменной}} представляет собой класс ориентированных пар, первым компонентом которых является знак \textit{логической формулы}, вторым -- \textit{sc-переменная}, имеющая в рамках данной \textit{логической формулы} ограниченное известное число значений, при которых данная формула является истинной в рамках соответствующей \textit{формальной теории}.\\
Отметим, что в случае \textit{атомарной логической формулы} каждая такая связка связывает знак формулы и знак принадлежащей ей \textit{sc-переменной}, т.е. является, по сути, частным случаем пары принадлежности. В случае \textit{неатомарной логической формулы} указанная \textit{sc-переменная} может принадлежать любой из \textit{подформул*} исходной формулы.

Таким образом, \textit{измерением*} каждого значения параметра \textbf{\textit{число значений переменной}} является некоторое \textit{число}, задающее количество значений \textit{sc-переменных} в рамках \textit{логической формулы}.}

\scnheader{кратность существования}
\scniselement{параметр}
\scnrelfrom{область определения параметра}{формула существования}
\scnhaselement{единственное существование}
\scnexplanation{Каждый элемент \textit{параметра} \textbf{\textit{кратность существования}} представляет собой класс логических \textit{формул существования}, для которых  при интерпретации на соответствующей \textit{предметной области} существует ограниченное общее для всех таких формул число комбинаций значений переменных, при которых указанные формулы являются истинными в рамках соответствующей \textit{формальной теории}.

Таким образом, \textit{измерением*} каждого значения \textbf{\textit{кратности существования}} является некоторое \textit{число}, задающее количество таких комбинаций.}

\scnheader{единственное существование}
\scnidtf{однократное существование}
\scnidtf{формула существования и единственности}
\scnnote{\textbf{\textit{единственное существование}} зачастую обозначается "$\exists!$" \ и читается как "существует и единственный".}

\scnheader{логическая формула и единственность}
\scnsubset{логическая формула}
\scnsubset{единственное существование}
\scnexplanation{Каждый элемент множества \textbf{\textit{логическая формула и единственность}} представляет собой \textit{логическую формулу} (\textit{атомарную} или \textit{неатомарную}), для которой дополнительно уточняется, что при ее интерпретации на некоторой предметной области существует только один набор значений переменных, входящих в эту формулу (или ее \textit{подформулы*}), при котором указанная логическая формула истинна в рамках \textit{формальной теории}, в которую входит данная \textit{предметная область}.}

\scnheader{связываемые переменные\scnrolesign}
\scniselement{ролевое отношение}
\scndefinition{\textbf{\textit{связываемые переменные\scnrolesign}} -- это \textit{ролевое отношение}, которое связывает связку конкретного \textit{квантора} с множеством переменных, которые связаны этим квантором.}
%\scnrelfrom{описание примера}{
%\scnfilescg{figures/sd_logical_formulas/bindVariables.png}}

\scnheader{открытая логическая формула}
\scndefinition{\textbf{\textit{открытая логическая формула}} -- это \textit{логическая формула}, в рамках которой (и всех ее \textit{подформул*}) существует хотя бы одна переменная, не связанная никаким \textit{квантором}.}

\scnheader{замкнутая логическая формула}
\scndefinition{\textbf{\textit{замкнутая логическая формула}} -- это \textit{логическая формула}, в рамках которой (и всех ее \textit{подформул*}) не существует переменных, не связанных каким-либо \textit{квантором}.}

\scnheader{Примеры логических утверждений}
\scneqtoset{\scgfileitem{figures/sd_logical_formulas/ex_set_union.png};\scgfileitem{figures/sd_logical_formulas/ex_set_diff.png}}

\bigskip
\scnendstruct \scnendcurrentsectioncomment

\end{SCn}

\scsubsection[\scneditors{Садовский М.Е.;Никифоров С.А.}\protect\scnmonographychapter{Глава 2.6. Языковые средства формального описания синтаксиса и денотационной семантики различных языков в интеллектуальных компьютерных системах нового поколения}\protect\scnmonographychapter{Глава 4.1. Структура интерфейсов интеллектуальных компьютерных систем нового поколения}]{Предметная область и онтология файлов, внешних информационных конструкций и внешних языков ostis-систем}
\label{sd_file_internal_inform_struct}

\scsubsubsection[\scneditors{Никифоров С.А.;Бобёр  Е.С.}\protect\scnmonographychapter{Глава 2.6. Языковые средства формального описания синтаксиса и денотационной семантики различных языков в интеллектуальных компьютерных системах нового поколения}]{Предметная область и онтология естественных языков}
\label{sd_natural_languages}
\begin{SCn}

\scnsectionheader{\currentname}

\scnstartsubstruct

\scnheader{язык}
\scnsubdividing{естественный язык\\
	\scnaddlevel{1}
	\scnexplanation{Естественный язык представляет собой язык, который не был создан целенаправленно}
	\scnaddlevel{-1}
;искусственный язык\\
	\scnaddlevel{1}
	\scnexplanation{Искусственный язык представляет собой язык, специально разработанный для воплощения определённых целей}
	\scnhaselement{Эсперанто}
	\scnhaselement{Python}
	\scnsuperset{сконструированный язык}
	\scnaddlevel{1}
	\scnexplanation{Сконструированный язык представляет собой искусственный язык, предназначенный для общения людей}
	\scnhaselement{Эсперанто}
	\scnaddlevel{-1}
	\scnaddlevel{-1}
}
\scnsuperset{международный язык}
	\scnaddlevel{1}
	\scnexplanation{Международный язык представляет собой естественный или искусственный язык, использующийся для общения людей разных стран}
	\scnhaselement{Английски язык}
	\scnhaselement{Русский язык}
	\scnaddlevel{-1}

\scnheader{плановый язык}
\scnreltoset{пересечение}{сконструированный язык;международный язык}

\scnheader{язык общения}
\scnreltoset{объединение}{естественный язык;сконструированный язык}
\scnhaselement{Английски язык}
\scnhaselement{Русский язык}
\scnhaselement{Эсперанто}
\scnreltoset{объединение}{корневой язык\\
	\scnaddlevel{1}
	\scnexplanation{Корневой язык представляет собой язык, для которого характерно полное отсутствие словоизменения и наличие грамматической значимости порядка слов, представленных только корнями.}
	\scnhaselement{Английски язык}
	\scnaddlevel{-1}
;агглютинативный язык\\
	\scnaddlevel{1}
	\scnexplanation{Агглютинативный язык характеризуется развитой системой употребления суффиксов, приставок, добавляемых к неизменяемой основе слова, которые используются для выражения числа, падежа, рода и др.}
	\scnhaselement{Английски язык}
	\scnaddlevel{-1}
;флективный язык\\
	\scnaddlevel{1}
	\scnexplanation{Для флективного языка характерно развитое употребление окончаний для выражения рода, числа, падежа, сложная система склонения глаголов, чередование гласных в корне. Строгое различение частей речи.}
	\scnhaselement{Русский язык}
	\scnaddlevel{-1}
;профлективный язык\\
	\scnaddlevel{1}
	\scnexplanation{Для профлективного языка для именного словоизменения характерна агглютинация, а для глагольного – флексия и чередование гласных (аблаут).}
	\scnaddlevel{-1}
}

\scnheader{словоформа}
\scnsubset{файл}
\scnexplanation{Словоформа - это слово, представленное в определенной грамматической форме.}

\scnheader{лексема}
\scnsubset{файл}
\scnexplanation{Лексема -  слово, рассматриваемое как единица словарного состава языка в совокупности всех его конкретных грамматических форм.}

\scnheader{грамматическая категория*}
\scniselement{бинарное отношение}
\scniselement{ориентированное отношение}
\scnrelfrom{первый домен}{язык}
\scnrelfrom{второй домен}{грамматическая категория}
\scnexplanation{Грамматическая категория* это бинарное ориентированное отношение, связывающее язык со  множеством взаимоисключающих и противопоставленных друг другу грамматических значений, задающих разбиение множества словоформ.}

\scnheader{парадигма*}
\scniselement{квазибинарное отношение}
\scniselement{ориентированное отношение}
\scnsubset{покрытие*}
\scnrelfrom{первый домен}{лексема}
\scnrelfrom{второй домен}{словоформа}
\scnexplanation{Парадигма* это квазибинарное отношение, связывающее лексему со множеством словоформ, принадлежащих данной лексеме и имеющих разные грамматические значения.}

\scnheader{морфема}
\scnsubset{файл}
\scnexplanation{Морфема — наименьшая единица языка, имеющая некоторый смысл.}

\scnheader{суффикс}
\scnsubset{морфема}
\scnexplanation{Суффикс – это морфема, которая стоит после корня и обычно служит для образования новых слов, хотя может использоваться и при образовании формы одного слова.}

\scnheader{корень}
\scnsubset{морфема}
\scnexplanation{Корень — морфема, несущая лексическое значение слова (или основную часть этого значения).}

\scnheader{Английский язык}
\scnrelfromset{грамматическая категория}{
	продолжительная форма';совершенная форма';простая форма'
}
\scnrelfromset{грамматическая категория}{сравнительная степень'\\
	\scnaddlevel{1}
	\scnexplanation{Сравнительную степень используется в случае, когда необходимо отметить, что предмет или человек обладает каким-то качеством в большей степени, чем другие.}
	\scnaddlevel{-1}
;положительная степень сравнения'\\
	\scnaddlevel{1}
	\scnexplanation{Положительная степень используется, чтобы указать, что предмет или человек обладает каким-то признаком или качеством.}
	\scnaddlevel{-1}
;превосходная степень сравнения'\\
	\scnaddlevel{1}
	\scnexplanation{Превосходная степень используется для указания того факта, что предмет или человек обладает каким-то качеством в наибольшей степени.}
	\scnaddlevel{-1}
}
\scnrelfromset{грамматическая категория}{общий падеж';косвенный падеж';притяжательный падеж';именительный падеж'}
\scnrelfromset{грамматическая категория}{женский род';мужской род';средний род'}
\scnrelfromset{грамматическая категория}{множественное число';единственное число'}
\scnrelfromset{грамматическая категория}{первое лицо';второе лицо';третье лицо'}
\scnrelfromset{грамматическая категория}{будущее время';прошедшее время';настоящее время'}

\scnheader{часть речи\scnsupergroupsign}
\scnrelto{семейство подмножеств}{лексема}
\scnexplanation{Часть речи представляют определенные лексико-грамматические разряды, на которые в зависимости от лексического значения от характера морфологических признаков и синтаксической функции делятся все слова языка.}

\scnhaselement{наречие}
	\scnaddlevel{1}
	\scnreltoset{объединение}{наречие места;наречие времени;наречие меры и степени;наречие образа действия}
	\scnsubdividing{простое наречие\\
		\scnaddlevel{1}
		\scnexplanation{Простые наречия не делятся на составные части.}
		\scnaddlevel{-1}
	;производное наречие\\
		\scnaddlevel{1}
		\scnexplanation{Производные наречия образованы при помощи суффиксов.}
		\scnaddlevel{-1}
	;сложное наречие\\
		\scnaddlevel{1}
		\scnexplanation{Сложные наречия образуются из нескольких корней.}
		\scnaddlevel{-1}
	;составное наречие\\
		\scnaddlevel{1}
		\scnexplanation{Составные наречия представляют собой сочетание служебного и знаменательного слова.}
		\scnaddlevel{-1}
	}
	\scnaddlevel{-1}
\scnhaselement{существительное}
	\scnaddlevel{1}
	\scnsubdividing{простое существительное\\
		\scnaddlevel{1}
		\scnexplanation{Простые существительные состоят из одного корня.}
		\scnaddlevel{-1}	
	;производное существительное\\
		\scnaddlevel{1}
		\scnexplanation{Производные существительные состоят из корня и одной или нескольких морфем (приставок или суффиксов).}
		\scnaddlevel{-1}
	;составное существительное\\
		\scnaddlevel{1}
		\scnexplanation{Составные существительные состоят по крайней мере из двух корней.}
		\scnaddlevel{-1}
	}
	\scnsubdividing{имя собственное\\
		\scnaddlevel{1}
		\scnexplanation{Имена собственные обозначают единственные в своем роде предметы или предметы, выделяемые из общего класса.}
		\scnaddlevel{-1}
	;имя нарицательное\\
		\scnaddlevel{1}
		\scnreltoset{объединение}{исчисляемое существительное\\
			\scnaddlevel{1}
			\scnsubdividing{конкретное исчисляемое существительное\\
				\scnaddlevel{1}
				\scnexplanation{Конкретные исчисляемые существительные - названия отдельных предметов и живых существ.}
				\scnaddlevel{-1}	
			;абстрактное исчисляемое существительное существительное\\
				\scnaddlevel{1}
				\scnidtf{отвлеченное исчисляемое существительное}
				\scnexplanation{Абстрактные исчисляемые существительные - названия исчисляемых понятий.}
				\scnaddlevel{-1}
			}
			\scnexplanation{Исчисляемые существительные могут быть посчитаны и имеют форму множественного числа.}
			\scnaddlevel{-1}
		;неисчисляемое существительное\\
			\scnaddlevel{1}
			\scnsubdividing{абстрактное неисчисляемое существительное\\
				\scnaddlevel{1}
				\scnidtf{отвлеченное неисчисляемое существительное}
				\scnexplanation{Абстрактные неисчисляемые существительные - названия неисчисляемых понятий.}
				\scnaddlevel{-1}	
			;вещественное неисчисляемое существительное\\
				\scnaddlevel{1}
				\scnexplanation{Вещественные неисчисляемые существительные - названия различных веществ и материалов.}
				\scnaddlevel{-1}
			}
			\scnexplanation{Неисчисляемые существительные не могут быть посчитаны и не имеют формы множественного числа.}
			\scnaddlevel{-1}
		;собирательное существительное\\
			\scnaddlevel{1}
			\scnexplanation{Собирательные существительные имеют форму единственного числа, но обозначают при этом группы лиц или понятий, рассматриваемые как одно целое.}
			\scnaddlevel{-1}
		}
		\scnexplanation{Имена нарицательные – это общие названия для всех однородных предметов.}
		\scnaddlevel{-1}
	}
	\scnaddlevel{-1}
\scnhaselement{глагол}
	\scnaddlevel{1}
	\scnsubdividing{простой глагол\\
		\scnaddlevel{1}
		\scnexplanation{Простые глаголы состоят только из одного корня.}
		\scnaddlevel{-1}
	;производный глагол\\
		\scnaddlevel{1}
		\scnexplanation{В производных глаголах, кроме корня, есть приставка и/или суффикс.}
		\scnaddlevel{-1}
	;сложный глагол\\
		\scnaddlevel{1}
		\scnexplanation{Сложные глаголы состоят из двух основ.}
		\scnaddlevel{-1}
	;составной глагол\\
		\scnaddlevel{1}
		\scnexplanation{Составные глаголы состоят из глагола и наречия или предлога.}
		\scnaddlevel{-1}
	}
	\scnsubdividing{смысловой глагол\\
		\scnaddlevel{1}
		\scnidtf{самостоятельный глагол}
		\scnsubset{простой глагол}
		\scnexplanation{Смысловые глаголы обладают собственным лексическим значением, они обозначают определенное действие или состояние.}
		\scnaddlevel{-1}
	;служебный глагол\\
		\scnaddlevel{1}
		\scnexplanation{Служебные глаголы не имеют самостоятельного значения. Они используются только для построения сложных форм глагола или составных сказуемых.}
		\scnsubdividing{глагол-связка\\
			\scnaddlevel{1}
			\scnexplanation{Глаголы-связки служат для соединения в предложении подлежащего с определенным состоянием.}
			\scnaddlevel{-1}
		;вспомогательный глагол\\
			\scnaddlevel{1}
			\scnexplanation{Вспомогательные глаголы служат для образования сложных глагольных форм.}
			\scnaddlevel{-1}
		;модальный глагол\\
			\scnaddlevel{1}
			\scnexplanation{Модальные глаголы отражают отношение говорящего к данному действию.}
			\scnaddlevel{-1}
		}
		\scnaddlevel{-1}
	}
\scnaddlevel{-1}
\scnhaselement{прилагательное}
\scnaddlevel{1}
\scnsubdividing{простое прилагательное\\
	\scnaddlevel{1}
	\scnexplanation{Простые прилагательные не имеют в своем составе суффиксов и приставок.}
	\scnaddlevel{-1}
;производное прилагательное\\
	\scnaddlevel{1}
	\scnexplanation{В составе производных прилагательных есть суффикс и/или приставка.}
	\scnaddlevel{-1}
;сложное прилагательное английского языка\\
	\scnaddlevel{1}
	\scnexplanation{Сложные прилагательные состоят из двух или более основ.}
	\scnaddlevel{-1}
}
\scnsubdividing{качественное прилагательное\\
	\scnaddlevel{1}
	\scnexplanation{Качественные прилагательные обозначают качества предмета прямо.}
	\scnaddlevel{-1}
;относительное прилагательное\\
	\scnaddlevel{1}
	\scnexplanation{Относительные прилагательные описывают качества предмета через его отношение к материалам, месту, времени.}
	\scnaddlevel{-1}
}
\scnaddlevel{-1}
\scnhaselement{местоимение}
\scnaddlevel{1}
\scnsubdividing{личное местоимение;притяжательное местоимение;указательное местоимение;возвратное местоимение;взаимное местоимение;вопросительное местоимение;относительное местоимение;неопределенное местоимение;отрицательное местоимение;разделительное местоимение;универсальное местоимение}
\scnaddlevel{-1}
\scnhaselement{предлог}
\scnaddlevel{1}
\scnsubdividing{производный предлог\\
	\scnaddlevel{1}
	\scnexplanation{Производный предлог - предлог, связанный происхождением с другими частями речи.}
	\scnaddlevel{-1}
;непроизводный предлог\\
	\scnaddlevel{1}
	\scnexplanation{Непроизводный предлог - так называемый первообразный предлог, который не может быть соотнесен по образованию с какой-либо частью речи.}
	\scnaddlevel{-1}
;сложный предлог\\
	\scnaddlevel{1}
	\scnexplanation{Сложный предлог - предлог, включающий в себя несколько компонентов.}
	\scnaddlevel{-1}
;составной предлог\\
	\scnaddlevel{1}
	\scnexplanation{Составной предлог представляет собой словосочетание. Он включают в себя слово из другой части речи и один или два предлога.}
	\scnaddlevel{-1}
}
\scnaddlevel{-1}
\scnhaselement{союз}
\scnaddlevel{1}
\scnsubdividing{сочинительный союз\\
	\scnaddlevel{1}
	\scnexplanation{Сочинительные союзы соединяют одинаковые по значимости слова, фразы, однородные члены предложения или независимые предложения в одно сложносочиненное предложение.}
	\scnaddlevel{-1}
;подчинительный союз\\
	\scnaddlevel{1}
	\scnexplanation{Подчинительные союзы соединяют придаточное предложение с основным, от которого оно зависит по смыслу, образуя сложноподчиненное предложение.}
	\scnaddlevel{-1}
;парный союз\\
	\scnaddlevel{1}
	\scnexplanation{Парные союзы служат для соединения слов, фраз или однородных, одинаковых частей одного предложения.}
	\scnaddlevel{-1}
;союзное наречие\\
	\scnaddlevel{1}
	\scnsubset{наречие}
	\scnexplanation{Союзные наречия соединяют два независимых предложения в одно сложносочиненное, или ставятся в начало предложения для его логической связи с предыдущим предложением.}
	\scnaddlevel{-1}
}
\scnsubdividing{простой союз\\
	\scnaddlevel{1}
	\scnexplanation{Простые союзы состоят из одного корня без суффиксов или префиксов.}
	\scnaddlevel{-1}
;сложный союз\\
	\scnaddlevel{1}
	\scnexplanation{Сложные союзы образованы от других частей речи, других союзов.}
	\scnaddlevel{-1}
;составной союз\\
	\scnaddlevel{1}
	\scnexplanation{Составные союзы состоят из двух и более слов, служебных и самостоятельных частей речи. К ним также относятся парные союзы.}
	\scnaddlevel{-1}
}
\scnaddlevel{-1}
\scnhaselement{числительное}
\scnaddlevel{1}
\scnsubdividing{порядковое числительное\\
	\scnaddlevel{1}
	\scnexplanation{Порядковые числительные обозначают порядок предметов.}
	\scnaddlevel{-1}
;количественное числительное английского языка\\
	\scnaddlevel{1}
	\scnexplanation{Количественные числительные обозначают количество предметов.}
	\scnaddlevel{-1}
}
\scnaddlevel{-1}

\scnsuperset{часть речи английского языка\scnsupergroupsign}
	\scnaddlevel{1}
	\scnhaselement{наречие английского языка}
		\scnaddlevel{1}
		\scnreltoset{объединение}{наречие места английского языка\\
			\scnaddlevel{1}
			\scnsubset{наречие места}
			\scnaddlevel{-1}
		;наречие времени английского языка\\
			\scnaddlevel{1}
			\scnsubset{наречие времени}
			\scnaddlevel{-1}
		;наречие меры и степени английского языка\\
			\scnaddlevel{1}
			\scnsubset{наречие меры и степени}
			\scnaddlevel{-1}
		;наречие образа действия английского языка\\
			\scnaddlevel{1}
			\scnsubset{наречие образа действия}
			\scnaddlevel{-1}
		}
		\scnsubdividing{простое наречие английского языка\\
			\scnaddlevel{1}
			\scnsubset{простое наречие}
			\scnaddlevel{-1}
		;производное наречие английского языка\\
			\scnaddlevel{1}
			\scnsubset{производное наречие}
			\scnaddlevel{-1}
		;сложное наречие английского языка\\
			\scnaddlevel{1}
			\scnsubset{сложное наречие}
			\scnaddlevel{-1}
		;составное наречие английского языка\\
			\scnaddlevel{1}
			\scnsubset{составное наречие}
			\scnaddlevel{-1}
		}
		\scnaddlevel{-1}
	\scnhaselement{существительное английского языка}
		\scnaddlevel{1}
		\scnsubdividing{простое существительное английского языка\\
			\scnaddlevel{1}
			\scnsubset{простое существительное}
			\scnaddlevel{-1}
		;производное существительное английского языка\\
			\scnaddlevel{1}
			\scnsubset{производное существительное}
			\scnaddlevel{-1}
		;составное существительное английского языка\\
			\scnaddlevel{1}
			\scnsubset{составное существительное}
			\scnaddlevel{-1}
		}
		\scnsubdividing{имя собственное английского языка\\
			\scnaddlevel{1}
			\scnsubset{имя собственное}
			\scnaddlevel{-1}
		;имя нарицательное английского языка\\
			\scnaddlevel{1}
			\scnreltoset{объединение}{исчисляемое существительное английского языка\\
				\scnaddlevel{1}
				\scnsubdividing{конкретное исчисляемое существительное английского языка\\
					\scnaddlevel{1}
					\scnsubset{конкретное исчисляемое существительное}
					\scnaddlevel{-1}
				;абстрактное исчисляемое существительное английского языка\\
					\scnaddlevel{1}
					\scnidtf{отвлеченное исчисляемое существительное английского языка}
					\scnsubset{абстрактное исчисляемое существительное}
					\scnaddlevel{-1}
				}
				\scnsubset{исчисляемое существительное}
				\scnaddlevel{-1}
			;неисчисляемое существительное английского языка\\
				\scnaddlevel{1}
				\scnsubdividing{абстрактное неисчисляемое существительное английского языка\\
					\scnaddlevel{1}
					\scnsubset{абстрактное неисчисляемое существительное}
					\scnaddlevel{-1}
					;вещественное неисчисляемое существительное английского языка\\
					\scnaddlevel{1}
					\scnsubset{вещественное неисчисляемое существительное}
					\scnaddlevel{-1}
				}
				\scnsubset{неисчисляемое существительное}
				\scnaddlevel{-1}
			;собирательное существительное английского языка\\
				\scnaddlevel{1}
				\scnsubset{собирательное существительное}
				\scnaddlevel{-1}
			}
			\scnsubset{имя нарицательное}
			\scnaddlevel{-1}
		}
		\scnaddlevel{-1}
	\scnhaselement{глагол английского языка}
		\scnaddlevel{1}
		\scnsubdividing{простой глагол английского языка\\
			\scnaddlevel{1}
			\scnsubset{простой глагол}
			\scnaddlevel{-1}
		;производный глагол английского языка\\
			\scnaddlevel{1}
			\scnsubset{производный глагол}
			\scnaddlevel{-1}
		;сложный глагол английского языка\\
			\scnaddlevel{1}
			\scnsubset{сложный глагол}
			\scnaddlevel{-1}
		;составной глагол английского языка\\
			\scnaddlevel{1}
			\scnsubset{составной глагол}
			\scnaddlevel{-1}
		}
		\scnsubdividing{смысловой глагол английского языка\\
			\scnaddlevel{1}
			\scnidtf{самостоятельный глагол английского языка}
			\scnsubset{простой глагол}
			\scnaddlevel{-1}
		;служебный глагол английского языка\\
			\scnaddlevel{1}
			\scnsubset{служебный глагол}
			\scnaddlevel{1}
			\scnsubdividing{глагол-связка английского языка\\
				\scnaddlevel{1}
				\scnsubset{глагол-связка}
				\scnaddlevel{-1}
			;вспомогательный глагол английского языка\\
				\scnaddlevel{1}
				\scnsubset{вспомогательный глагол}
				\scnaddlevel{-1}
			;модальный глагол английского языка\\
				\scnaddlevel{1}
				\scnsubset{модальный глагол}
				\scnaddlevel{-1}
			}
			\scnaddlevel{-2}
		}
		\scnaddlevel{-1}
	\scnhaselement{прилагательное английского языка}
	\scnaddlevel{1}
	\scnsubdividing{простое прилагательное английского языка\\
		\scnaddlevel{1}
		\scnsubset{простое прилагательное}
		\scnaddlevel{-1}
	;производное прилагательное английского языка\\
		\scnaddlevel{1}
		\scnsubset{производное прилагательное}
		\scnaddlevel{-1}
	;сложное прилагательное английского языка\\
		\scnaddlevel{1}
		\scnsubset{сложное прилагательное}
		\scnaddlevel{-1}
	}
	\scnsubdividing{качественное прилагательное английского языка\\
		\scnaddlevel{1}
		\scnsubset{качественное прилагательное}
		\scnaddlevel{-1}
	;относительное прилагательное английского языка\\
		\scnaddlevel{1}
		\scnsubset{относительное прилагательное}
		\scnaddlevel{-1}
	}
	\scnaddlevel{-1}
	\scnhaselement{местоимение английского языка}
		\scnaddlevel{1}
		\scnsubdividing{личное местоимение английского языка\\
			\scnaddlevel{1}
			\scnsubset{личное местоимение}
			\scnaddlevel{-1}
		;притяжательное местоимение английского языка\\
			\scnaddlevel{1}
			\scnsubset{притяжательное местоимение}
			\scnaddlevel{-1}
		;указательное местоимение английского языка\\
			\scnaddlevel{1}
			\scnsubset{указательное местоимение}
			\scnaddlevel{-1}
		;возвратное местоимение английского языка\\
			\scnaddlevel{1}
			\scnsubset{возвратное местоимение}
			\scnaddlevel{-1}
		;взаимное местоимение английского языка\\
			\scnaddlevel{1}
			\scnsubset{взаимное местоимение}
			\scnaddlevel{-1}
		;вопросительное местоимение английского языка\\
			\scnaddlevel{1}
			\scnsubset{вопросительное местоимение}
			\scnaddlevel{-1}
		;относительное местоимение английского языка\\
			\scnaddlevel{1}
			\scnsubset{относительное местоимение}
			\scnaddlevel{-1}
		;неопределенное местоимение английского языка\\
			\scnaddlevel{1}
			\scnsubset{неопределенное местоимение}
			\scnaddlevel{-1}
		;отрицательное местоимение английского языка\\
			\scnaddlevel{1}
			\scnsubset{отрицательное местоимение}
			\scnaddlevel{-1}
		;разделительное местоимение английского языка\\
			\scnaddlevel{1}
			\scnsubset{разделительное местоимение}
			\scnaddlevel{-1}
		;универсальное местоимение английского языка\\
			\scnaddlevel{1}
			\scnsubset{универсальное местоимение}
			\scnaddlevel{-1}
		}
		\scnaddlevel{-1}
	\scnhaselement{предлог английского языка}
		\scnaddlevel{1}
		\scnsubdividing{производный предлог английского языка\\
			\scnaddlevel{1}
			\scnsubset{производный предлог}
			\scnaddlevel{-1}
		;непроизводный предлог английского языка\\
			\scnaddlevel{1}
			\scnsubset{непроизводный предлог}
			\scnaddlevel{-1}
		;сложный предлог английского языка\\
			\scnaddlevel{1}
			\scnsubset{сложный предлог}
			\scnaddlevel{-1}
		;составной предлог английского языка\\
			\scnaddlevel{1}
			\scnsubset{составной предлог}
			\scnaddlevel{-1}
		}
		\scnaddlevel{-1}
	\scnhaselement{союз английского языка}
		\scnaddlevel{1}
		\scnsubdividing{сочинительный союз английского языка\\
			\scnaddlevel{1}
			\scnsubset{сочинительный союз}
			\scnaddlevel{-1}
		;подчинительный союз английского языка\\
			\scnaddlevel{1}
			\scnsubset{подчинительный союз}
			\scnaddlevel{-1}
		;парный союз английского языка\\
			\scnaddlevel{1}
			\scnsubset{парный союз}
			\scnaddlevel{-1}
		;союзное наречие английского языка\\
			\scnaddlevel{1}
			\scnsubset{наречие английского языка}
			\scnsubset{союзное наречие}
			\scnaddlevel{-1}
		}
		\scnsubdividing{простой союз английского языка\\
			\scnaddlevel{1}
			\scnsubset{простой союз}
			\scnaddlevel{-1}
		;сложный союз английского языка\\
			\scnaddlevel{1}
			\scnsubset{сложный союз}
			\scnaddlevel{-1}
		;составной союз английского языка\\
			\scnaddlevel{1}
			\scnsubset{составной союз}
			\scnaddlevel{-1}
		}
		\scnaddlevel{-1}
	\scnhaselement{числительное английского языка}
	\scnaddlevel{1}
	\scnsubdividing{порядковое числительное английского языка\\
		\scnaddlevel{1}
		\scnsubset{порядковое числительное}
		\scnaddlevel{-1}
	;количественное числительное английского языка\\
		\scnaddlevel{1}
		\scnsubset{количественное числительное}
		\scnaddlevel{-1}
	}
	\scnaddlevel{-2}

\scnendstruct \scnendcurrentsectioncomment

\end{SCn}

\scparagraph[\scneditors{Никифоров С.А.;Бобёр  Е.С.}\protect\scnmonographychapter{Глава 2.6. Языковые средства формального описания синтаксиса и денотационной семантики различных языков в интеллектуальных компьютерных системах нового поколения}]{Предметная область и онтология синтаксиса естественных языков}
\label{sd_syntax_natural_lang}

\scparagraph[\scneditors{Никифоров С.А.;Бобёр  Е.С.}\protect\scnmonographychapter{Глава 2.6. Языковые средства формального описания синтаксиса и денотационной семантики различных языков в интеллектуальных компьютерных системах нового поколения}]{Предметная область и онтология денотационной семантики естественных языков}
\label{sd_sem_natural_lang}

\scsubsection[\scneditors{Никифоров С.А.;Шункевич Д.В.}\protect\scnmonographychapter{Глава 3.1. Формализация понятий действия, задачи, метода, средства, навыка и технологии}]{Глобальная предметная область и онтология, описывающая воздействия, действия, методы, средства и технологии}
\label{sd_actions}
\begin{SCn}

\scnsectionheader{\currentname}

\scnstartsubstruct

\scnsdmainclasssingle{действие}
\scnsdclass{информационное действие;поведенческое действие;эффекторное действие;рецепторное действие;действие в sc-памяти;действие во внешней среде ostis-системы;эффекторное действие ostis-системы;рецепторное действие ostis-системы;инициированное действие;выполняемое действие;активное действие;отложенное действие;планируемое действие;выполненное действие;успешно выполненное действие;безуспешно выполненное действие;действие, выполненное с ошибкой;приоритет действия;субъект;внутренний субъект ostis-системы;внешний субъект ostis-системы, с которым осуществляется взаимодействие;внешний субъект ostis-системы, с которым взаимодействие не происходит;класс действий;атомарный класс действий;неатомарный класс действий;конъюнкция предшествующих действий;проверка условия;задача;процедурная формулировка задачи;декларативная формулировка задачи;класс задач;вопрос;команда;класс команд;класс команд без аргументов;класс команд с одним аргументом;класс команд с двумя аргументами;класс команд с произвольным числом аргументов;атомарный класс команд;неатомарный класс команд;план;программа;программа в sc-памяти;протокол;решение}
\scnsdrelation{дейcтвие с очень высоким приоритетом';дейcтвие с высоким приоритетом';дейcтвие со средним приоритетом';дейcтвие с низким приоритетом';дейcтвие с очень низким приоритетом';декомпозиция действия*;поддействие*;последовательность действий*;последовательность действий при положительном результате*;последовательность действий при отрицательном результате*;последовательность действий в случае ошибки*;результат*;исполнитель*;класс выполняемых действий*;заказчик*;инициатор*;объект*;контекст действия*;аргумент действия';первый аргумент действия’;второй аргумент действия’;третий аргумент действия’;класс аргументов*;класс первых аргументов*;класс вторых аргументов*}
\scnrelfromvector{ключевые знаки}{действие;класс действий;метод;класс методов;деятельность;вид деятельности}



\scnendstruct \scnendcurrentsectioncomment

\end{SCn}

\scsubsubsection[\scnmonographychapter{Глава 3.1. Формализация понятий действия, задачи, метода, средства, навыка и технологии}]{Предметная область и онтология локальных предметных областей и онтологий действий}
\label{local_sd_actions}

\scsubsubsection[\scnidtf{Типология неавтоматизированных ("вручную"{} выполняемых) и автоматически выполняемых \textit{действий}, направленных на управление процессами выполнения различных \textit{сложных действий}, а также система понятий, используемая для \textit{управления сложными действиями}}]{Предметная область и онтология действий по управлению деятельностью многоагентных систем}
\label{local_sd_project_management}
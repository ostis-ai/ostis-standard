\scseparatedfragment{Титульная спецификация Стандарта OSTIS}
\begin{SCn}
    \scnsectionheader{Титульная спецификация Стандарта OSTIS}

    \begin{scnsubstruct}
        \scnrelto{титульная спецификация}{Стандарт OSTIS}
        \scnrelfrom{оглавление}{Оглавление Стандарта OSTIS}
        \scnrelfrom{общая структура}{Общая Структура Стандарта OSTIS}
        \scnrelfrom{система ключевых знаков}{Система ключевых знаков Стандарта OSTIS}
        \scnrelfrom{редакционная коллегия}{Редакционная коллегия Стандарта OSTIS}
        \scnrelfrom{авторский коллектив}{Авторский коллектив Стандарта OSTIS}
        \scnrelfrom{направления развития}{Направления развития Стандарта OSTIS}
        \scnrelfrom{правила построения}{Правила построения Стандарта OSTIS}
        \begin{scnindent}
            \scnidtf{правила построения*(Стандарт OSTIS)}
        \end{scnindent}
        \scniselement{sc-выражение}
        \scnrelfrom{правила организации развития}{Правила организации развития
            Стандарта OSTIS}
        \begin{scnrelfromset}{декомпозиция}
            \scnitem{Правила организации развития исходного текста Стандарта OSTIS}
            \scnitem{Правила организации развития Стандарта OSTIS на уровне его
                внутреннего представления в памяти Метасистемы IMS.ostis}
        \end{scnrelfromset}
        \scnsourcecommentinline{Вычитать то, что дальше}
        \scnnote{В титульную спецификацию \textit{Стандарта OSTIS} должны быть
            включены ссылки на все разделы и фрагменты этих разделов,
            где описываются правила построения и оформления всех видов информационных
            конструкций, входящих в состав \textit{Стандарта OSTIS}
            (внешних идентификаторов знаков, входящих в состав \textit{Стандарта OSTIS},
            спецификаций различного вида cущностей, описываемых в \textit{Стандарте OSTIS})

            В \textit{базе знаний ostis-систем} задаются правила унифицированного построения
            (представления, оформления) следующих видов \textit{информационных
                конструкций}:
            \begin{scnitemize}
                \item \textit{sc-идентификаторов} --- внешних идентификаторов
                    \textit{sc-элементов} следующих классов:
                    \begin{scnitemizeii}
                        \item \textit{sc-элементов} (имеются в виду общие правила идентификации
                                    любых sc-элементов) --- смотрите в разделе\scnqqi{\nameref{intro_idtf}}
                        \item \textit{sc-переменных, sc-констант}
                        \item знаков материальных сущностей
                            \begin{scnitemizeiii}
                                \item знаков персон
                                \item знаков библиографических источников
                            \end{scnitemizeiii}
                        \item знаков множеств
                            \begin{scnitemizeiii}
                                \item классов, понятий
                                \begin{itemize}
                                    \item отношений
                                    \item параметров
                                    \item структур
                                    \item знаний
                                \end{itemize}
                            \end{scnitemizeiii}
                        \item знаков файлов ostis-систем
                        \item знаков sc-знаний баз знаний
                    \end{scnitemizeii}
                \item \textit{sc-конструкций}
                \item \textit{sc.g-конструкций}
                \item \textit{sc.s-конструкций}
                \item \textit{sc.n-конструкций}
                \item базовых правил \textit{sc-спецификаций:}
                    \begin{scnitemizeii}
                        \item понятий
                        \item разделов баз знаний (титульные спецификации разделов)
                        \item файлов ostis-систем
                        \item библ. источников
                        \item предметных областей
                    \end{scnitemizeii}
                \item специализированная \textit{sc-спецификация}
                    \begin{scnitemizeii}
                        \item информационная конструкция
                            \begin{scnitemizeiii}
                                \item оглавление
                                \item система ключевых знаков
                            \end{scnitemizeiii}
                        \item понятий
                            \begin{scnitemizeiii}
                                \item пояснение
                                \item определения
                                \item теоретико-множественная окрестность
                                \item семейство утверждений
                            \end{scnitemizeiii}
                        \item сегментов баз знаний (титульная спецификация)
                        \item семейство разделов баз знаний
                    \end{scnitemizeii}
            \end{scnitemize}}

        \scnheader{Стандарт OSTIS-2021}
        \scnidtf{Издание Документации Технологии OSTIS-2021}
        \scnidtf{Первое издание (публикация) Внешнего представления Документации
            Технологии OSTIS в виде книги}
        \scniselement{публикация}
        \begin{scnindent}
            \scnidtf{библиографический источник}
        \end{scnindent}
        \scniselement{официальная версия Стандарта OSTIS}
        \scniselement{бумажное издание}
        \scniselement{научное издание}
        \scnrelfrom{рекомендация издания}{Совет БГУИР}
        \begin{scnrelfromset}{рецензенты}
            \scnitem{Курбацкий А.Н.}
            \scnitem{Дудкин А.А.}
        \end{scnrelfromset}
        \scnrelfrom{издательство}{Бестпринт}
        \scniselement{\scnnonamednode}
        \scniselement{УДК}
        \scniselement{параметр}
        \scnrelfrom{Индекс УДК}{004.8}
        \scnidtftext{ISBN}{978-985-7267-13-2}

        \scnheader{Стандарт OSTIS-2022}
        \scnidtf{Издание Документации Технологии OSTIS-2022}
        \scnidtf{Второе издание (публикация) Внешнего представления Документации Технологии OSTIS в виде книги}
        \scniselement{публикация}
        \scniselement{официальная версия Стандарта OSTIS}
        \scniselement{бумажное издание}
        \scniselement{научное издание}
        \bigskip
    \end{scnsubstruct}
    \scnsourcecommentinline{Завершили Титульную спецификацию Стандарта OSTIS}
\end{SCn}
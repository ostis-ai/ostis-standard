%Ввести новую глава начиная с сем. окретсностей, назвать "Про и онтология фрагментов баз знаний"
% Ввести главу Архитектура Экосистемы OSTIS

\scseparatedfragment{Оглавление Стандарта OSTIS}
\begin{SCn}
	\scnstartsubstruct
	
    \scnidtf{Оглавление текущей версии Стандарта OSTIS}
    \scnidtftext{пояснение}{Иерархический перечень разделов, входящих в состав
        \textit{Стандарта OSTIS}}
    \begin{scnindent}
   	\scntext{примечание}{Существенно подчеркнуть, что иерархия разделов \textit{Стандарта OSTIS} как и \textit{разделов} любой другой \textit{базы знаний} не всегда означает то, что \textit{разделы} более низкого уровня иерархии входят в состав (являются частями) соответствующих разделов более высокого уровня. Так, например, связь между \textit{разделами} разных уровней иерархии может означать то, что \textit{раздел} более низкого уровня иерархии является \textit{дочерним} разделом по отношению к соответствующему \textit{разделу} более высокого уровня, т.е. \textit{разделом}, который наследует свойства указанного \textit{раздела} более высокого уровня.}
   	\end{scnindent}
    \scntext{примечание}{Подробное описание логико-семантических связей каждого раздела \textit{Стандарта OSTIS} с другими разделами \textit{Стандарта OSTIS} приводится в рамках \textit{титульной спецификации} каждого \textit{раздела}.}
    
    \scnheader{Стандарт OSTIS}
    \begin{scnrelfromvector}{декомпозиция}
        \scnitem{Предисловие к первому изданию Стандарта OSTIS}
        \scnitem{Предисловие ко второму изданию Стандарта OSTIS}
        \scnitem{Оглавление Стандарта OSTIS}
        \scnitem{Титульная спецификация Стандарта OSTIS}
        \scnitem{Титульная спецификация второго издания Стандарта OSTIS}
        \scnitem{Часть 1. Обоснование интеллектуальных компьютерных систем нового поколения и комплексной технологии их разработки и сопровождения}
            \begin{scnindent}
                \begin{scnrelfromvector}{декомпозиция}
                    \scnitem{Глава 10. Кибернетические системы, их деятельность и эволюция}
                    \begin{scnindent}
                    	\begin{scnrelfromvector}{декомпозиция}
                    		\scnitem{\S 10.1. Предметная область и онтология кибернетических систем}
                    		\scnitem{\S 10.2. Предметная область и онтология деятельности кибернетических систем}
                    		\scnitem{\S 10.3. Предметная область и онтология эволюции кибернетических систем}
                    	\end{scnrelfromvector}
                    \end{scnindent}
                    \scnitem{Глава 11. Индивидуальные кибернетические системы и их эволюция} 
                    \begin{scnindent}
                    	\begin{scnrelfromvector}{декомпозиция}
                    		\scnitem{\S 11.1. Предметная область и онтология индивидуальных кибернетических систем}
                    		\scnitem{\S 11.2. Предметная область и онтология эволюции индивидуальных кибернетических систем}
                    		\scnitem{\S 11.3. Предметная область и онтология стимульно-реактивных индивидуальных кибернетических систем}
                    		\scnitem{\S 11.4. Предметная область и онтология индивидуальных кибернетических систем со знаковой памятью}
                    		\scnitem{\S 11.5. Предметная область и онтология задачно-ориентированных индивидуальных кибернетических систем}
                    		\scnitem{\S 11.6. Предметная область и онтология индивидуальных кибернетических систем со структурированной информационной моделью окружающей среды}
                    		\scnitem{\S 11.7. Предметная область и онтология гибридных индивидуальных кибернетических систем}
                    		\begin{scnindent}	
                   			\begin{scnrelfromvector}{детализация}
                   				\scnitem{Пункт 11.7.1. Предметная область и онтология смыслового представления информации}
                   				\scnitem{Пункт 11.7.2. Предметная область и онтология многоагентных моделей решателей задач, основанных на смысловом представлении информации}
                   			\end{scnrelfromvector}
                   			\end{scnindent}	 
                   			\scnitem{\S 11.8. Предметная область и онтология эволюции интерфейсов индивидуальных кибернетических систем}
                   			\begin{scnindent}
                   			\scnrelfrom{часть}
                   				{Пункт 11.8.1. Предметная область и онтология онтологических моделей мультимодальных физических интерфейсов интеллектуальных компьютерных систем, основанных на смысловом представлении информации}
                   			\end{scnindent}
                   			\scnitem{\S 11.9. Предметная область и онтология обучаемых индивидуальных кибернетических систем}
                    		\scnitem{\S 11.10. Предметная область и онтология самостоятельных индивидуальных кибернетических систем}
                    	\end{scnrelfromvector}
                    \end{scnindent}
                    \scnitem{Глава 12. Многоагентные кибернетические системы и их эволюция}
                    \begin{scnindent}
                    	\begin{scnrelfromvector}{декомпозиция}
                    		\scnitem{\S 12.1. Предметная область и онтология многоагентных кибернетических систем}
                    		\scnitem{\S 12.2. Предметная область и онтология эволюции многоагентных кибернетических систем}
                    		\begin{scnindent}
                    			\scnrelfrom{часть}
                    			{Пункт 12.2.1. Предметная область и онтология эволюции языков коммуникации агентов и коммуникативных интерфейсов}
                    		\end{scnindent}
                    		\scnitem{\S 12.3. Предметная область и онтология интероперабельных индивидуальных и многоагентных кибернетических систем}
                    	\end{scnrelfromvector}
                    \end{scnindent}
                    \scnitem{Глава 13. Интеллектуальные кибернетические системы, интеллектуальные компьютерные системы и технологии их разработки и сопровождения}
                    \begin{scnindent}
                    	\begin{scnrelfromvector}{декомпозиция}
                    		\scnitem{\S 13.1. Предметная область и онтология интеллектуальных индивидуальных и многоагентных кибернетических систем}
                    		\scnitem{\S 13.2. Предметная область и онтология интеллектуальных компьютерных систем нового поколения}
                    		\scnitem{\S 13.3. Предметная область и онтология эволюции технологий разработки интеллектуальных компьютерных систем}
                    		\scnitem{\S 13.4. Предметная область и онтология комплексной технологии разработки и сопровождения семантически совместимых интеллектуальных компьютерных систем нового поколения}
                    	\end{scnrelfromvector}
                    \end{scnindent}
                    \scnitem{Глава 14. Человеко-машинные системы и их эволюция}
                    \begin{scnindent}                    	
                    	\begin{scnrelfromvector}{декомпозиция}
                    		\scnitem{\S 14.1. Предметная область и онтология человеко-машинных систем}
                    		\scnitem{\S 14.2. Предметная область и онтология эволюции человеко-машинных систем}
                    		\begin{scnindent}
                    			\scnrelfrom{часть}
                    			{Пункт 14.2.1. Предметная область и онтология эволюции пользовательских интерфейсов машин (в том числе компьютерных систем)}
                    		\end{scnindent}
                    		\scnitem{\S 14.3. Предметная область и онтология компонентов интеллектуального человеко-машинного Сообщества разработчиков прикладных семантически совместимых интеллектуальных компьютерных систем нового поколения}
                    		\scnitem{\S 14.4. Предметная область и онтология компонентов интеллектуального человеко-машинного Сообщества разработчиков комплексной технологии разработки и сопровождения семантически совместимых интеллектуальных компьютерных систем нового поколения}
                    		\scnitem{\S 14.5. Предметная область и онтология компонентов интегрированного интеллектуального человеко-машинного Сообщества специалистов в области Искусственного интеллекта}
                    		\scnitem{\S 14.6. Предметная область и онтология компонентов Глобального интеллектуального человеко-машинного сообщества}  
                    	\end{scnrelfromvector}
                    \end{scnindent}
                \end{scnrelfromvector}
            \end{scnindent}
        \scnitem{Часть 2. Смысловое представление и онтологическая систематизация знаний в интеллектуальных компьютерных системах нового поколения}
            \begin{scnindent}
                \begin{scnrelfromvector}{декомпозиция}         	
                    \scnitem{Глава 20. Предметная область и онтология информационных конструкций и языков}               
                    \scnitem{Глава 21. Внутренний язык интеллектуальных компьютерных систем нового поколения и близкие ему внешние языки (языки обмена информацией)}
                    \begin{scnindent}
                        \begin{scnrelfromvector}{декомпозиция}
                            \scnitem{\S 21.1. Предметная область и онтология внутреннего языка ostis-систем}
                                \begin{scnindent}
                                    \begin{scnrelfromvector}{детализация}
                                        \scnitem{Пункт 21.1.1. Предметная область и онтология синтаксиса внутреннего языка ostis-систем}
                                        \scnitem{Пункт 21.1.2. Предметная область и онтология базовой денотационной семантики внутреннего языка ostis-систем}
                                    \end{scnrelfromvector}
                                \end{scnindent}
                            \scnitem{\S 21.2. Предметная область и онтология внешних идентификаторов знаков, входящих в информационные конструкции внутреннего языка ostis-систем}
                            \scnitem{\S 21.3. Предметная область и онтология языка внешнего графического представления информационных конструкций внутреннего языка ostis-систем}
                                \begin{scnindent}
                                    \begin{scnrelfromvector}{детализация}
                                        \scnitem{Пункт 21.3.1. Предметная область и онтология синтаксиса языка внешнего графического представления информационных конструкций внутреннего языка ostis-систем}
                                        \scnitem{Пункт 21.3.2. Предметная область и онтология денотационной семантики языка внешнего графического представления информационных конструкций внутреннего языка ostis-систем}
                                        \scnitem{Пункт 21.3.3. Предметная область и онтология иерархического семейства подъязыков семантически эквивалентных языку внешнего графического представления}
                                    \end{scnrelfromvector}
                                \end{scnindent}
                            \scnitem{\S 21.4. Предметная область и онтология языка внешнего линейного представления информационных конструкций внутреннего языка ostis-систем}
                                \begin{scnindent}
                                    \begin{scnrelfromvector}{детализация}
                                        \scnitem{Пункт 21.4.1. Предметная область и онтология синтаксиса языка внешнего линейного представления информационных конструкций внутреннего языка ostis-систем}
                                        \scnitem{Пункт 21.4.2. Предметная область и онтология денотационной семантики языка внешнего линейного представления информационных конструкций внутреннего языка ostis-систем}
                                        \scnitem{Пункт 21.4.3. Предметная область и онтология иерархического семейства подъязыков, семантически эквивалентных языку внешнего линейного представления информационных конструкций внутреннего языка ostis-систем}
                                    \end{scnrelfromvector}
                                \end{scnindent}
                            \scnitem{\S 21.5. Предметная область и онтология языка внешнего форматированного представления информационных конструкций внутреннего языка ostis-систем}
                                \begin{scnindent}
                                    \begin{scnrelfromvector}{детализация}
                                        \scnitem{Пункт 21.5.1. Предметная область и онтология синтаксиса языка внешнего форматированного представления информационных конструкций внутреннего языка ostis-систем}
                                        \scnitem{Пункт 21.5.2. Предметная область и онтология денотационной семантики языка внешнего форматированного представления информационных конструкций внутреннего языка ostis-систем}
                                        \scnitem{Пункт 21.5.3. Предметная область и онтология иерархического семейства подъязыков, семантически эквивалентных языку внешнего форматированного представления информационных конструкций внутреннего языка ostis-систем}
                                    \end{scnrelfromvector}
                                \end{scnindent}
                        \end{scnrelfromvector}
                    \end{scnindent}
                    
                    \scnitem{Глава 22. Предметная область и онтология сущностей верхнего уровня}
                    \begin{scnindent}
                        \begin{scnrelfromvector}{дочерние разделы}
                            \scnitem{\S 22.1. Предметная область и онтология множеств}
                            \scnitem{\S 22.2. Предметная область и онтология связок и отношений}
                            \scnitem{\S 22.3. Предметная область и онтология параметров, величин и шкал}
                            \scnitem{\S 22.4. Предметная область и онтология чисел и числовых структур}
                            \scnitem{\S 22.5. Предметная область и онтология структур}
                            \scnitem{\S 22.6. Предметная область и онтология темпоральных сущностей}
                                \begin{scnindent}
                                    \scnrelfrom{дочерний раздел}
                                        {Пункт 22.6.1. Предметная область и онтология ситуаций и событий, описывающих динамику баз знаний ostis-систем}
                                \end{scnindent}
                            \scnitem{\S 22.7. Предметная область и онтология пространственных сущностей различных форм}
                            \scnitem{\S 22.8. Предметная область и онтология материальных сущностей}
                    	\end{scnrelfromvector}
	          		\end{scnindent}              
                    \scnitem{Глава 23. Предметная область и онтология фрагментов баз знаний}
                    \begin{scnindent}
                   	\begin{scnrelfromvector}{дочерние разделы}
                            \scnitem{\S 23.1. Предметная область и онтология семантических окрестностей}
                            \scnitem{\S 23.2. Предметная область и онтология разделов баз знаний (структуризации баз знаний)}
                            \begin{scnindent}
                            	\begin{scnrelfromvector}{дочерние разделы}
                            	\scnitem{Пункт 23.2.1. Предметная область и онтология предметных областей}
                            	\scnitem{Пункт 23.2.2. Предметная область и онтология онтологий}
                            	\end{scnrelfromvector}
                            \end{scnindent}
                            \scnitem{\S 23.3. Предметная область и онтология логических формул, высказываний и логических sc-языков}
                            \scnitem{\S 23.4. Предметная область и онтология файлов, внешних информационных конструкций и внешних языков ostis-систем}
                        \end{scnrelfromvector}
                    \end{scnindent}
                \end{scnrelfromvector}
            \end{scnindent}
        \scnitem{Часть 3. Многоагентные решатели задач интеллектуальных компьютерных систем нового поколения}
            \begin{scnindent}
                \begin{scnrelfromvector}{декомпозиция}
                    \scnitem{Глава 30. Предметная область и онтология решателей задач ostis-систем}
                        \begin{scnindent}
                            \begin{scnrelfromvector}{декомпозиция}
                                \scnitem{\S 30.1. Предметная область и онтология действий, задач, планов, протоколов и методов, реализуемых ostis-системой, а также внутренних агентов, выполняющих эти действия}
                                \begin{scnindent}
                                	\begin{scnrelfromvector}{детализация}
                                		\scnitem{Пункт 30.1.1. Предметная область и онтология локальных предметных областей и онтологий действий}
                                		\scnitem{Пункт 30.1.2. Предметная область и онтология действий по управлению деятельностью многоагентных систем}
                                	\end{scnrelfromvector}
                                \end{scnindent}
                                \scnitem{\S 30.2. Предметная область и онтология Базового языка программирования ostis-систем}
                                    \begin{scnindent}
                                        \begin{scnrelfromvector}{детализация}
                                            \scnitem{Пункт 30.2.1. Предметная область и онтология синтаксиса Базового языка программирования ostis-систем}
                                            \scnitem{Пункт 30.2.2. Предметная область и онтология денотационной семантики Базового языка программирования ostis-систем}
                                            \scnitem{Пункт 30.2.3. Предметная область и онтология операционной семантики Базового языка программирования ostis-систем}
                                        \end{scnrelfromvector}
                                    \end{scnindent}
                                \scnitem{\S 30.3. Предметная область и онтология программ и языков программирования для ostis-систем}
                                    \begin{scnindent}
                                        \scnrelfrom{часть}{Пункт 30.3.1. Предметная область и онтология интерпретации современных языков программирования в ostis-системах}
                                    \end{scnindent}
                                \scnitem{\S 30.4. Предметная область и онтология sc-языка вопросов}
                                    \begin{scnindent}
                                        \begin{scnrelfromvector}{детализация}
                                            \scnitem{Пункт 30.4.1. Предметная область и онтология синтаксиса sc-языка вопросов}
                                            \scnitem{Пункт 30.4.2. Предметная область и онтология денотационной семантики sc-языка вопросов}
                                            \scnitem{Пункт 30.4.3. Предметная область и онтология операционной семантики sc-языка вопросов}
                                        \end{scnrelfromvector}
                                    \end{scnindent}
                                \scnitem{\S 30.5. Предметная область и онтология операционной семантики логических sc-языков}
                                \scnitem{\S 30.6. Предметная область и онтология sc-языков продукционного программирования}
                                    \begin{scnindent}
                                        \begin{scnrelfromvector}{детализация}
                                            \scnitem{Пункт 30.6.1. Предметная область и онтология синтаксиса sc-языков продукционного программирования}
                                            \scnitem{Пункт 30.6.2. Предметная область и онтология денотационной семантики sc-языков продукционного программирования}
                                            \scnitem{Пункт 30.6.3. Предметная область и онтология операционной семантики sc-языков продукционного программирования}
                                        \end{scnrelfromvector}
                                    \end{scnindent}
                                \scnitem{\S 30.7. Предметная область и онтология sc-моделей искусственных нейронных сетей}
                                    \begin{scnindent}
                                        \begin{scnrelfromvector}{детализация}
                                            \scnitem{Пункт 30.7.1. Предметная область и онтология синтаксиса sc-моделей искусственных нейронных сетей}
                                            \scnitem{Пункт 30.7.2. Предметная область и онтология денотационной семантики sc-моделей искусственных нейронных сетей}
                                            \scnitem{Пункт 30.7.3. Предметная область и онтология операционной семантики sc-моделей искусственных нейронных сетей}
                                        \end{scnrelfromvector}
                                    \end{scnindent}
                            \end{scnrelfromvector}
                        \end{scnindent}
                \end{scnrelfromvector}
            \end{scnindent}
        \scnitem{Часть 4. Онтологические модели интерфейсов интеллектуальных компьютерных систем нового поколения}
        \begin{scnindent}
            \begin{scnrelfromvector}{декомпозиция}
                \scnitem{Глава 40. Предметная область и онтология интерфейсов ostis-систем}
                \scnitem{Глава 41. Предметная область и онтология интерфейсных действий пользователей ostis-системы}
                \scnitem{Глава 42. Предметная область и онтология сообщений, входящих в ostis-систему и выходящих из неё}
                \scnitem{Глава 43. Предметная область и онтология действий и внутренних агентов пользовательского интерфейса ostis-системы}
                \scnitem{Глава 44. Предметная область и онтология естественно-языковых интерфейсов ostis-систем}
                    \begin{scnindent}
                        \begin{scnrelfromvector}{детализация}
                       		\scnitem{\S 44.1. Предметная область и онтология естественных языков}
                       		\begin{scnindent}
                       			\begin{scnrelfromvector}{детализация}
                       				\scnitem{Пункт 44.1.1. Предметная область и онтология синтаксиса естественных языков}
                       				\scnitem{Пункт 44.1.2. Предметная область и онтология денотационной семантики естественных языков}
                       			\end{scnrelfromvector}
                       		\end{scnindent}
                            \scnitem{\S 44.2. Предметная область и онтология синтаксического анализа естественно-языковых сообщений, входящих в ostis-систему}
                            \scnitem{\S 44.3. Предметная область и онтология понимания естественно-языковых сообщений, входящих в ostis-систему}
                            \scnitem{\S 44.4. Предметная область и онтология синтеза естественно-языковых сообщений ostis-системы}
                        \end{scnrelfromvector}
                    \end{scnindent}
            \end{scnrelfromvector}
        \end{scnindent}
        \scnitem{Часть 5. Методы и средства проектирования интеллектуальных компьютерных систем нового поколения}
            \begin{scnindent}
                \begin{scnrelfromvector}{декомпозиция}
                    \scnitem{Глава 50. Предметная область и онтология комплексной библиотеки многократно используемых семантически совместимых компонентов ostis-систем}
                        \begin{scnindent}
                            \begin{scnrelfromvector}{декомпозиция}
                                \scnitem{\S 50.1. Предметная область и онтология многократно используемых компонентов баз знаний ostis-систем}
                                \scnitem{\S 50.2. Предметная область и онтология многократно используемых компонентов решателей задач ostis-систем}
                                    \begin{scnindent}
                                        \begin{scnrelfromvector}{детализация}
                                            \scnitem{Пункт 50.2.1. Предметная область и онтология многократно используемых методов, хранимых в памяти ostis-систем и интерпретируемых их внутренними агентами}
                                            \scnitem{Пункт 50.2.2. Предметная область и онтология многократно используемых внутренних агентов ostis-систем}
                                        \end{scnrelfromvector}
                                    \end{scnindent}
                                \scnitem{\S 50.3. Предметная область и онтология многократно используемых компонентов интерфейсов ostis-систем}
                                \scnitem{\S 50.4. Предметная область и онтология многократно используемых встраиваемых ostis-систем}
                            \end{scnrelfromvector}
                        \end{scnindent}
                    \scnitem{Глава 51. Предметная область и онтология действий и методик проектирования ostis-систем}
                        \begin{scnindent}
                            \begin{scnrelfromvector}{декомпозиция}
                                \scnitem{\S 51.1. Предметная область и онтология действий и методик проектирования баз знаний ostis-систем}
                                \scnitem{\S 51.2. Предметная область и онтология действий и методик проектирования решателей задач ostis-систем}
                                \scnitem{\S 51.3. Предметная область и онтология действий и методик проектирования интерфейсов ostis-систем}
                            \end{scnrelfromvector}
                        \end{scnindent}
                    \scnitem{Глава 52. Предметная область и онтология действий и методик обучения проектированию ostis-систем}
                    \scnitem{Глава 53. Предметная область и онтология средств проектирования ostis-систем}
                        \begin{scnindent}
                            \begin{scnrelfromvector}{декомпозиция}
                                \scnitem{\S 53.1. Логико-семантическая модель комплекса встраиваемых ostis-систем автоматизации проектирования баз знаний ostis-систем}
                                    \begin{scnindent}
                                        \begin{scnrelfromvector}{декомпозиция}
                                            \scnitem{Пункт 53.1.1. Логико-семантическая модель ostis-системы редактирования, сборки и ввода исходных текстов различных компонентов проектируемой базы знаний в память ostis-системы}
                                            \scnitem{Пункт 53.1.2. Логико-семантическая модель ostis-системы редактирования проектируемой базы знаний ostis-системы на уровне её внутреннего представления}
                                            \scnitem{Пункт 53.1.3. Логико-семантическая модель ostis-системы обнаружения и анализа ошибок и противоречий в базе знаний ostis-системы}
                                            \scnitem{Пункт 53.1.4. Логико-семантическая модель ostis-системы обнаружения и анализа информационных дыр в базе знаний ostis-системы}
                                            \scnitem{Пункт 53.1.5. Логико-семантическая модель ostis-системы автоматизации управления взаимодействием разработчиков различных категорий в процессе проектирования базы знаний ostis-системы}
                                        \end{scnrelfromvector}
                                    \end{scnindent}
                                \scnitem{\S 53.2. Логико-семантическая модель комплекса ostis-систем автоматизации проектирования решателей задач ostis-систем}
                                    \begin{scnindent}
                                        \begin{scnrelfromvector}{декомпозиция}
                                            \scnitem{Пункт 53.2.1. Логико-семантическая модель ostis-системы автоматизации проектирования программ Базового языка программирования ostis-систем}
                                            \scnitem{Пункт 53.2.2. Логико-семантическая модель ostis-системы автоматизации проектирования внутренних агентов ostis-систем, а также коллективов таких агентов}
                                            \scnitem{Пункт 53.2.3. Логико-семантическая модель ostis-системы автоматизации проектирования искусственных нейронных сетей, семантически совместимых с базам знаний ostis-систем}
                                        \end{scnrelfromvector}
                                    \end{scnindent}
                                \scnitem{\S 53.3. Логико-семантическая модель ostis-системы автоматизации проектирования интерфейсов ostis-систем}
                                \scnitem{\S 53.4. Предметная область и онтология ostis-систем автоматизации проектирования различных классов ostis-систем}
                            \end{scnrelfromvector}
                        \end{scnindent}
                    \scnitem{Глава 54. Предметная область и онтология ostis-систем обучения проектированию ostis-систем и их компонентов}
                \end{scnrelfromvector}
            \end{scnindent}
        \scnitem{Часть 6. Платформы реализации интеллектуальных компьютерных систем нового поколения}
        \begin{scnindent}
            \scnrelfrom{часть}{Глава 60. Предметная область и онтология методов и средств производства ostis-систем}
                    \begin{scnindent}
                        \scnrelfrom{часть}{\S 60.1. Предметная область и онтология базовых интерпретаторов логико-семантических моделей ostis-систем}
                                \begin{scnindent}
                                    \begin{scnrelfromvector}{детализация}
                                        \scnitem{Пункт 60.1.1. Предметная область и онтология ассоциативных семантических компьютеров для ostis-систем}
                                        \scnitem{Пункт 60.1.2. Предметная область и онтология программных вариантов реализации базового интерпретатора логико-семантических моделей ostis-систем на современных компьютерах}
                                    \end{scnrelfromvector}
                                \end{scnindent}
                    \end{scnindent}
        \end{scnindent}
        \scnitem{Часть 7. Методы и средства реинжиниринга и эксплуатации интеллектуальных компьютерных систем нового поколения}
            \begin{scnindent}
                \scnrelfrom{часть}{Глава 70. Предметная область и онтология методов и средств поддержки жизненного цикла ostis-систем}
                        \begin{scnindent}
                            \begin{scnrelfromvector}{детализация}
                                \scnitem{\S 70.1. Предметная область и онтология методов и средств реинжиниринга ostis-систем в ходе эксплуатации}
                                \scnitem{\S 70.2. Предметная область и онтология встроенных ostis-систем поддержки использования ostis-систем конечными пользователями}
                            \end{scnrelfromvector}
                        \end{scnindent}
            \end{scnindent}
        \scnitem{Часть 8. Экосистема интеллектуальных компьютерных систем нового поколения}
            \begin{scnindent}
                \scnrelfrom{часть}{Глава 80. Предметная область и онтология компонентов Экосистемы OSTIS}
                    \begin{scnindent}
                        \begin{scnrelfromvector}{детализация}
                            \scnitem{\S 80.1. Предметная область и онтология автоматизируемых видов и областей человеческой деятельности}
                            \scnitem{\S 80.2. Предметная область и онтология технологий автоматизации различных видов и областей человеческой деятельности}
                                \begin{scnindent}
                                    \scnrelfrom{часть}{Пункт 80.2.1. Предметная область и онтология технологий компьютеризации различных видов и областей человеческой деятельности}
                                \end{scnindent}
                            \scnitem{\S 80.3. Предметная область и онтология деятельности в области Искусственного интеллекта}
                            \scnitem{\S 80.4. Логико-семантическая модель интеграции разнородных информационных ресурсов и сервисов в Экосистеме OSTIS в процессе ее расширения}
                                \begin{scnindent}
                                    \scnrelfrom{часть}{Пункт 80.4.1. Предметная область и онтология библиографических источников и других информационных ресурсов}
                                \end{scnindent}
                            \scnitem{\S 80.5. Предметная область и онтология семантически совместимых интеллектуальных ostis-порталов научных знаний}
                                \begin{scnindent}
                                    \scnrelfrom{часть}{Пункт 80.5.1. Логико-семантическая модель Метасистемы OSTIS}
                                \end{scnindent}
                            \scnitem{\S 80.6. Предметная область и онтология семантически совместимых информационно-справочных ostis-систем и интеллектуальных help-систем}
                            \scnitem{\S 80.7. Предметная область и онтология семантически совместимых интеллектуальных корпоративных ostis-систем различного назначения}
                                \begin{scnindent}
                                    \scnrelfrom{часть}{Пункт 80.7.1. Предметная область и онтология организаций}
                                \end{scnindent}
                            \scnitem{\S 80.8. Предметная область и онтология ostis-систем, являющихся персональными ассистентами пользователей, обеспечивающими организацию эффективного взаимодействия каждого пользователя с другими ostis-системами и пользователями, входящими в состав Экосистемы OSTIS}
                                \begin{scnindent}
                                    \scnrelfrom{часть}{Пункт 80.8.1. Предметная область и онтология персон}
                                \end{scnindent}
                            \scnitem{\S 80.9. Предметная область и онтология семантически совместимых ostis-систем автоматизации образовательной деятельности}
                                \begin{scnindent}
                                    \scnrelfrom{часть}{Пункт 80.9.1. Предметная область и онтология дидактических знаний}
                                \end{scnindent}
                            \scnitem{\S 80.10. Предметная область и онтология семантически совместимых ostis-систем автоматизации проектирования и управления проектированием различных объектов}
                            \scnitem{\S 80.11. Предметная область и онтология семантически совместимых ostis-систем автоматизации производственной деятельности}
                                \begin{scnindent}
                                    \scnrelfrom{часть}{Пункт 80.11.1. Предметная область и онтология семантически совместимых ostis-систем управления рецептурным производством}
                                \end{scnindent}
                            \scnitem{\S 80.12. Предметная область и онтология геоинформационных ostis-систем}
                                \begin{scnindent}
                                    \scnrelfrom{часть}
                                        {Пункт 80.12.1. Предметная область и онтология географических объектов}
                                \end{scnindent}
                            \scnitem{\S 80.13. Предметная область и онтология средств обеспечения информационной безопасности ostis-систем в рамках Экосистемы OSTIS}
                        \end{scnrelfromvector}
                    \end{scnindent}
            \end{scnindent}
        \scnitem{Библиография OSTIS}
    \end{scnrelfromvector}
    
    \scnendstruct
    \scnsourcecommentinline{Завершили \textit{Оглавление
            Стандарта OSTIS}}
    
\end{SCn}
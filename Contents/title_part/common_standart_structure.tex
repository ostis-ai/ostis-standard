\begin{SCn}
	\scnsectionheader{Общая Структура Стандарта OSTIS}
	\scnsectionheader{Стандарт OSTIS}
	\scntext{общая структура}{\textit{Основной текст Стандарта OSTIS}
		состоит из следующих частей:
		\begin{scnitemize}
			\item Анализ текущего состояния \textit{технологий Искусственного
				интеллекта} и постановка задачи на создание комплекса совместимых
			\textit{технологий Искусственного интеллекта}, ориентированного на создание и
			эксплуатацию \textit{интеллектуальных компьютерных систем нового поколения}.
			\\Данная часть \textit{Стандарта OSTIS} начинается с \textit{раздела}
			``\textbf{\textit{Предметная область и онтология кибернетических систем}} и
			заканчивается \textit{разделом} ``\textit{\textbf{Предметная область и
					онтология логико-семантических моделей компьютерных систем, основанных на
					смысловом представлении информации}}.
			\item Документация предлагаемой комплексной технологии создания и
			эксплуатации \textit{интеллектуальных компьютерных систем нового поколения},
			которая названа нами \textit{Технологией OSTIS}. Эта часть \textit{Стандарта
				OSTIS} начинается с \textit{раздела} ``\textit{\textbf{Предметная область и
					онтология предлагаемой комплексной технологии создания и эксплуатации
					интеллектуальных компьютерных систем нового поколения}}, заканчивается
			\textit{разделом} ``\textit{\textbf{Предметная область и онтология встроенных
					ostis-систем поддержки эксплуатации соответствующих ostis-систем конечными
					пользователями}} и включает в себя:
			\begin{scnitemizeii}
				\item Описание формальных структурно-функциональных
				логико-семантических моделей предполагаемых \textit{интеллектуальных
					компьютерных систем нового поколения}(такие системы названы нами
				\textit{ostis-системами}). Сюда входит:
				\begin{scnitemizeiii}
					\item описание моделей \textit{знаний} и \textit{баз знаний}, а также
					методов и средств их проектирования;
					\item описание логических и продукционных моделей обработки
					\textit{знаний}, а также методов и средств их проектирования;
					\item описание "нейросетевых"{} моделей обработки \textit{знаний}, а
					также методов и средств их проектирования;
					\item описание моделей \textit{решателей задач};
					\item описание моделей \textit{интерфейсов ostis-систем};
					\item описание онтологических моделей \textit{интерфейсов
						интеллектуальных компьютерных систем}, а также методов и средств их
					проектирования, включая описание онтологических моделей естественно-языковых
					\textit{интерфейсов интеллектуальных компьютерных систем}, а также методов и
					средств их проектирования.
				\end{scnitemizeiii}
				\item Описание методов:
				\begin{scnitemizeiii}
					\item методов проектирования \textit{баз знаний ostis-систем};
					\item методов проектирования \textit{решателей задач ostis-систем};
					\item методов проектирования \textit{интерфейсов ostis-систем};
					\item методов производства (реализации) \textit{ostis-систем};
					\item методов реинжиниринга \textit{ostis-систем};
					\item методов использования \textit{ostis-систем} конечными
					пользователями.
				\end{scnitemizeiii}
				\item Описание средств:
				\begin{scnitemizeiii}
					\item средств поддержки проектирования \textit{баз знаний
						ostis-систем};
					\item средств поддержки проектирования \textit{решателей задач
						ostis-систем};
					\item средств поддержки проектирования \textit{интерфейсов
						ostis-систем};
					\item средств производства \textit{ostis-систем} -- программных средств
					интеллектуализации логико-семантических моделей \textit{ostis-систем} и
					специально предназначенных для этого ассоциативных семантических компьютеров;
					\item средств поддержки реинжиниринга \textit{ostis-систем} в ходе их
					эксплуатации;
					\item средств поддержки использования \textit{ostis-систем} конечными
					пользователями.
				\end{scnitemizeiii}
				\item Описание реализации системы управления \textit{базами знаний
					ostis-систем} на основе системы управления графовыми базами данных.
				\item Описание аппаратной реализации графодинамической памяти, а также
				средств обработки знаний в этой памяти.
			\end{scnitemizeii}
			\item Описание продуктов, создаваемых с помощью \textit{Технологии
				OSTIS}, основным из которых является глобальная \textit{Экосистема OSTIS} --
			Экосистема семантически совместимых и активно взаимодействующих
			\textit{ostis-систем}.\\Эта часть \textit{Стандарта OSTIS} представлена
			\textit{разделом} ``\textit{\textbf{Предметная область и онтология Экосистемы
					OSTIS}}
			\item \textit{\textbf{Библиография Стандарта OSTIS}}
		\end{scnitemize}
	}
\end{SCn}
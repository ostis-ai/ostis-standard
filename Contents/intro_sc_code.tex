
\scsubsection{Введение в описание внутреннего языка ostis-систем}
\label{intro_sc_code}

\begin{SCn}

\scnsectionheader{\currentname}

\scnstartsubstruct

\scnsegmentheader{Основные положения внутреннего языка ostis-систем}
\scnstartsubstruct

\scnheader{\currentname}
\scnreltovector{конкатенация сегментов}{Основные положения внутреннего языка ostis-систем;Описание Ядра SC-кода;SC-код как синтаксическое расширение Ядра SC-кода;Использование SC-кода для формального описания собственного синтаксиса}

\scnheader{SC-код}
\scnidtf{Язык унифицированного смыслового представления знаний в памяти интеллектуальных компьютерных систем}
\scnidtf{Внутренний язык ostis-систем}
\scnrelto{внутренний язык}{ostis-система}
\scntext{эпиграф}{Информация содержится не в самих знаках, а в конфигурации связей между ними.}
\scntext{эпиграф}{Он вскочил на коня и поскакал во все стороны.}

\scntext{основной внешний идентификатор sc-элемента}{SC-код}
\scnaddlevel{1}
\scniselement{имя собственное}
\scnaddlevel{-1}
\scntext{часто используемый неосновной внешний идентификатор sc-элемента}{sc-текст}
\scnaddlevel{1}
\scniselement{имя нарицательное}
\scnaddlevel{-1}
\scnidtf{sc-конструкция}
\scniselement{абстрактный язык}
\scnaddlevel{1}
    \scnidtf{Язык, для которого не уточняется способ представления символов, входящих в состав текстов этого языка, а задается только \uline{алфавит*} этих символов, то есть семейство классов символов, считающихся синтаксически эквивалентными друг другу.}
    \scnaddlevel{1}
        \scnnote{Каждому абстрактному языку можно поставить в соответствие целое семейство \textit{реальных языков}, обеспечивающих \uline{изоморфное} реальное представление текстов указанного абстрактного языка путем уточнения способов представления (изображения, кодирования) символов, входящих в состав этих текстов, а также путем уточнения правил установления синтаксической эквивалентности этих символов. Очевидно, что во всём остальном синтаксис и денотационная семантика указанных реальных языков полностью совпадает с синтаксисом и денотационной семантикой соответствующего абстрактного языка.}
        \scnaddlevel{1}
            \scnnote{Для SC-кода как абстрактного языка необходима разработка как минимум трех синтаксически и семантически эквивалентных ему реальных языков: (1) язык кодирования текстов SC-кода в памяти традиционных компьютеров; (2) язык кодирования текстов SC-кода в семантической ассоциативной памяти; (3) Ядро SCg-кода -- язык, синтаксически и семантически эквивалентный SC-коду и обеспечивающий графическое представление текстов SC-кода.}
        \scnaddlevel{-1}
    \scnaddlevel{-1}
\scnaddlevel{-1}
\scniselement{графовый язык}
\scnaddlevel{1}
    \scnexplanation{язык, каждый текст которого (1) задается множеством входящих в него элементарных фрагментов (символов), которое, в свою очередь, состоит из (1.1) из множества узлов (вершин), возможно, синтаксически разного вида и (1.2) из множества связок, которые также могут принадлежать разным синтаксически выделяемым классам, а также (2) задается в общем случае несколькими отношениями инцидентности связок с компонентами этих связок (при этом указанными компонентами в общем случае могут быть не только вершины, но и другие связки).}
\scnaddlevel{-1}

\scnidtf{Универсальный язык, обеспечивающий внутреннее представление и структуризацию \uline{всех}(!), используемых ostis-системой в процессе своего функционирования.}
\scnidtf{Универсальный язык, являющийся результатом унификации (уточнения) синтаксиса и денотационной семантики семантических сетей.}
\scnaddlevel{1}
    \scnexplanation{Универсальность SC-кода обеспечивается и тем, что элементами текстов SC-кода могут быть знаки описываемых сущностей \uline{любого} вида, в том числе, и  знаки связей между описываемыми сущностями и/или их знаками}
    \scnaddlevel{1}
        \scntext{следствие}{Тексты SC-кода являются графовыми структурами расширенного вида, в которых знаки описываемых связей могут соединять (быть инцидентными) не только вершины (узлы) графовой структуры, но и знаки других связей.}
    \scnaddlevel{-1}
\scnaddlevel{-1}
\scnidtf{Базовый универсальный язык внутреннего представления знаний в памяти ostis-систем.}
\scnidtf{Базовый внутренний язык ostis-систем.}
\scnidtf{Максимальный внутренний язык ostis-систем, по отношению к которому все остальные (специализированные) внутренние языки являются его подъязыками (подмножествами)}
\scnidtf{Множество всевозможных текстов SC-кода}
\scnaddlevel{1}
\scniselement{имя собственное}
\scnaddlevel{-1}
\scnidtf{текст SC-кода}
\scnaddlevel{1}
\scniselement{имя нарицательное}
\scnaddlevel{-1}


\filemodetrue
\scnrelfromvector{принципы, лежащие в основе}{\textit{Знаки} (обозначения) всех \textit{сущностей}, описываемых в \textit{sc-текстах} (текстах \textit{\textbf{SC-кода}}) представляются в виде синтаксически элементарных (атомарных) фрагментов \textit{sc-текстов} и, следовательно, не имеющих внутренней структуры, не состоящих из более простых фрагментов \textit{текста}, как, например, имена (термины), которые представляют \textit{знаки} описываемых \textit{сущностей} в привычных \textit{языках} и состоят из \textit{букв}.;\textit{Имена} (термины), \textit{естественно-языковые тексты} и другие информационные конструкции, не являющиеся \textit{sc-текстами}, могут входить в состав \textit{sc-текста}, но только как \textit{файлы}, описываемые (специфицируемые) \textit{sc-текстами}. Таким образом, в состав базы знаний \textit{интеллектуальной компьютерной системы}, построенной на основе \textit{\textbf{SC-кода}}, могут входить \textit{имена} (термины), обозначающие некоторые описываемые \textit{сущности} и представленные соответствующими \textit{файлами}. Каждый \textit{sc-элемент} будем называть внутренним обозначением некоторой \textit{сущности}, а \textit{имя} этой \textit{сущности}, представленное соответствующим файлом, будем называть \textit{внешним идентификатором} (внешним обозначением) этой \textit{сущности}. При этом каждый именуемый (идентифицируемый) \textit{sc-элемент} связывается дугой, принадлежащей отношению <<\textit{\textbf{быть внешним идентификатором*}}~>>, с \textit{узлом}, содержимым которого является \textit{файл} идентификатора (в частности, \textit{имени}), обозначающего ту же \textit{сущность}, что и указанный выше \textit{sc-элемент}. \textit{Внешним идентификатором} может быть не только \textit{имя} (термин), но и иероглиф, пиктограмма, озвученное имя, жест. Особо отметим, что \textit{внешние идентификаторы} описываемых \textit{сущностей} в \textit{интеллектуальной компьютерной системе}, построенной на основе \textit{\textbf{SC-кода}}, используются только (1) для анализа информации, поступающей в эту систему из вне из различных источников, и ввода (понимания и погружения) этой информации в \textit{базу знаний}, а также (2) для синтеза различных \textit{сообщений}, адресуемых различным субъектам (в т.ч. пользователям).;Тексты \textit{\textbf{SC-кода}} (\textit{sc-тексты}) имеют в общем случае нелинейную (графовую) структуру, поскольку (1) \textit{знак} каждой описываемой сущности входит в состав \textit{sc-текста} однократно и (2) каждый такой \textit{знак} может быть инцидентен неограниченному числу других \textit{знаков}, поскольку каждая описываемая \textit{сущность} может быть связана неограниченным числом связей с другими описываемыми \textit{сущностями}.;
\textit{База знаний}, представленная текстом \textit{\textbf{SC-кода}}, является \textit{графовой структурой} специального вида, алфавит элементов которой включает в себя множество \textit{узлов}, множество \textit{ребер}, множество \textit{дуг}, множество \textit{базовых дуг} -- дуг специально выделенного типа, обеспечивающих структуризацию \textit{баз знаний}, а также множество специальных \textit{узлов}, каждый из которых имеет содержимое, являющееся \textit{файлом}, хранящимся в памяти \textit{интеллектуальной компьютерной системы}. Структурная особенность данной \textit{графовой структуры} заключается в том, что ее \textit{дуги} и \textit{ребра} могут связывать не только \textit{узел} с \textit{узлом}, но и \textit{узел} с \textit{ребром} или \textit{дугой}, \textit{ребро} или \textit{дугу} с другим \textit{ребром} или \textit{дугой}.;
\uline{Все элементы} (\textit{sc-элементы}) указанной выше \textit{графовой структуры} (текста \textit{\textbf{SC-кода}}), т.е. все ее узлы (sc-узлы), ребра (sc-ребра) и дуги (sc-дуги) являются обозначениями различных сущностей. При этом ребро является обозначением бинарной неориентированной связки между двумя сущностями, каждая из которых либо представлена в рассматриваемой графовой структуре соответствующим знаком, либо является самим этим знаком. Дуга является обозначением бинарной ориентированной связки между двумя сущностями. Дуга специального вида (\textit{\textbf{базовая дуга}}) является знаком связи между узлом, обозначающим некоторое множество элементов рассматриваемой графовой структуры, и одним из элементов этой графовой структуры, который принадлежит указанному множеству. Узел, имеющий содержимое (узел, для которого содержимое существует, но может в текущий момент быть неизвестным) является знаком файла, который является содержимым этого узла. Узел, не являющийся знаком файла, может обозначать какой-либо материальный объект, первичный абстрактный объект(например, число, точку в некотором абстрактном пространстве), какую-либо бинарную связь, какое-либо множество (в частности, понятие, структуру, ситуацию, событие, процесс). При этом сущности, обозначаемые элементами рассматриваемой графовой структуры, могут быть постоянными (существующими всегда) и временными (сущностями, которым соответствует отрезок времени их существования). Кроме того, сущности, обозначаемые элементами рассматриваемой графовой структуры, могут быть константными (конкретными) сущностями и переменными (произвольными) сущностями. Каждому элементу рассматриваемой графовой структуры, являющемуся обозначением переменной сущности, ставится в соответствие область возможных значений этого обозначения. Область возможных значений каждого переменного ребра является подмножеством множества всевозможных константных ребер, область возможных значений каждой переменной дуги является подмножеством множества всевозможных константных дуг, область возможных значений каждого переменного узла является подмножеством множества всевозможных константных узлов.;
В рассматриваемой графовой структуре, являющейся представлением базы знаний в \textit{\textbf{SC-коде}}, могут, но не должны существовать разные элементы графовой структуры, обозначающие одну и ту же сущность. Если пара таких элементов обнаруживается, то эти элементы склеиваются (отождествляются). Таким образом, синонимия внутренних обозначений в базе знаний интеллектуальной компьютерной системы, построенной на основе \textit{\textbf{SC-кода}}, запрещена. При этом синонимия внешних обозначений считается нормальным явлением. Формально это означает, что из некоторых элементов рассматриваемой графовой структуры выходит несколько дуг, принадлежащих отношению <<быть \textit{\textbf{внешним идентификатором*}}~>>. Из всех указанных дуг, принадлежащих отношению <<быть \textit{\textbf{внешним идентификатором*}}~>> и выходящих из одного элемента рассматриваемой графовой структуры, обязательно выделяется одна (очень редко две) путем включения их в число дуг, принадлежащих отношению <<быть \textit{\textbf{основным внешним идентификатором*}}~>>. Это означает, что указываемый таким образом внешний идентификатор не является омонимичным, т.е. не может быть использован как внешний идентификатор, соответствующий другому элементу рассматриваемой графовой структуры.;
Кроме файлов, представляющих различные внешние обозначения (имена, иероглифы, пиктограммы), в памяти интеллектуальной компьютерной системе, построенной на основе \textit{\textbf{SC-кода}}, могут хранится файлы различных текстов (книг, статей, документов, примечаний, комментариев, пояснений, чертежей, рисунков, схем, фотографий, видео-материалов, аудио-материалов).;
\uline{Любую сущность}, требующую описания, в тексте \textit{\textbf{SC-кода}} можно обозначить в виде \textit{sc-элемента}. Это являетс яодним из факторов, обеспечивающих универсальность \textit{\textbf{SC-кода}}. Особо подчеркнем, что sc-элементы являются не просто обозначениями различных описываемых сущностей, а обозначениями, которые являются элементарными (атомарными) фрагментами знаковой конструкции, т.е. фрагментами, детализация структуры которых не требуется для "прочтения"{} и понимания этой знаковой конструкции.;
Текст \textit{\textbf{SC-кода}}, как и любая другая графовой структура, является абстрактным математическим объектом, не требующим детализации (уточнения) его кодирования в памяти компьютерной системы (например, в виде матрицы смежности, матрицы инцидентности, списковой структуры). Но такая детализация потребуется для технической реализации памяти, в которой хранятся и обрабатываются sc-тексты.;
Важнейшим дополнительным свойством \textit{\textbf{SC-кода}} является то,что он удобен не просто для внутреннего представления знаний в памяти интеллектуальной компьютерной системы, но также и для внутреннего представления информации в памяти компьютеров, специально предназначенных для интерпретации семантических моделей интеллектуальных компьютерных систем. Т.е., \textit{\textbf{SC-код}} определяет синтаксические, семантические и функциональные принципы организации памяти компьютеров нового поколения, ориентированных на реализацию интеллектуальных компьютерных систем, -- принципы организации графодинамической ассоциативной семантической памяти.;
\textit{\textbf{SC-код}} рассматривается нами как объединение трех его подъязыков, в число которых входит \textit{\textbf{Ядро SC-кода}}, подъязык \textit{\textbf{SC-кода}}, обеспечивающий представление текстов \textit{\textbf{SC-кода}} (\textit{sc-текстов}) в форме орграфов классического вида, являющихся подразбиениями текстов \textit{\textbf{Ядра SC-кода}} и, соответственно, использующих \uline{явное} представление пар инцидентности элементов sc-текстов (sc-элементов), синтаксическое \textit{\textbf{Расширение Ядра SC-кода}}, обеспечивающее представление в памяти ostis-системы информационных конструкций инородного для \textit{\textbf{SC-кода}} вида.
}
\filemodefalse
\scnaddlevel{1}
\scnsourcecommentpar{Завершили сегмент "Описание принципов, лежащих в основе SC-кода"}
\scnaddlevel{-1}

\scnheader{SC-пространство}
\scnnote{Понятие SC-пространства наряду с понятием SC-кода играет важнейшую роль для уточнения и формализации понятия смысла информационных конструкций, для унификации смыслового представления информации и для максимально возможного исключения субъективизма в трактовке понятия смысла. Смысл информационной конструкции в конечном счете определяется (1) конфигурацией смыслового представления этой конструкции и (2) и "местоположением" (контекстом) смыслового представления указанной информационной конструкции в рамках смыслового пространства, т.е. в рамках объединенного смыслового представления \uline{всевозможных} информационных конструкций, либо в рамках объединенного смыслового представления информации, накопленной к заданному моменту времени некоторым индивидуальным субъектом или коллективом субъектов. Таким объединенным смысловым представлением информации, в частности, является смысловое представление глобальной базы всех знаний, накопленных человечеством к текущему моменту.}
\scnexplanation{Объединение (вместилище) \uline{всевозможных} унифицированных семантических сетей (текстов SC-кода)}
\scnaddlevel{1}
    \scnnote{При теоретико-множественном объединении текстов SC-кода семантически эквивалентные (синонимичные) элементы (синтаксически элементарные фрагменты) этих текстов считаются совпадающими элементами и при объединении указанных текстов "склеиваются".}
\scnaddlevel{-1}
\scnrelto{объединение}{SC-код}
\scnidtf{Унифицированное смысловое пространство}
\scntext{достоинство}{Важнейшим достоинством SC-пространства является возможность уточнения таких понятий, как понятие аналогичности (сходства и отличия) различных описываемых "внешних" сущностей, аналогичности унифицированных семантических сетей (текстов SC-кода), понятие семантической близости описываемых сущностей (в том числе, и текстов SC-кода).}

\scnendstruct

\scnsegmentheader{Описание Ядра SC-кода}
\scnstartsubstruct

\scnheader{Синтаксис Ядра SC-кода}
\scnstartsubstruct

\scnheader{Ядро SC-кода}
\scnrelfrom{синтаксис}{Синтаксис Ядра SC-кода}

\scnheader{Ядро SC-кода}
\scnrelfrom{множество всех элементов конструкция данного языка}{sc-элемент}
\scnaddlevel{1}
    \scnidtf{элемент конструкции Ядра SC-кода}
    \scnidtf{синтаксически элементарный (атомарный) фрагмент дискретной информационной конструкции, принадлежащей Ядру SC-кода}
    \scnidtf{Класс элементов конструкций Ядра SC-кода}
    \scnidtf{Множество всех элементов всевозможных конструкций Ядра SC-кода}
\scnaddlevel{-1}

\scnheader{Ядро SC-кода}
\scnrelfrom{алфавит}{Алфавит Ядра SC-кода}

\scnheader{Алфавит Ядра SC-кода}
\scnidtf{Множество (Семейство) всех классов синтаксически эквивалентных sc-элементов Ядра SC-кода}
\scnidtf{класс синтаксически эквивалентных sc-элементов Ядра SC-кода}
\scnidtf{класс синтаксически эквивалентных элементов конструкций Ядра SC-кода}
\scnidtf{элемент Алфавита Ядра SC-кода}
\scnidtf{синтаксический тип sc-элемента Ядра SC-кода}
\scnsubdividing{sc-узел общего вида;sc-ребро общего вида;sc-дуга общего вида;базовая sc-дуга}

\scnheader{sc-элемент}
\scnrelto{разбиение}{Алфавит Ядра SC-кода}
\scnaddlevel{1}
    \scnnote{Алфавит Ядра SC-кода является одним из признаков классификации sc-элементов.}
    \scnnote{В процессе обработки текстов Ядра SC-кода синтаксический тип sc-элементов может меняться -- sc-узел может трансформироваться в sc-ребро, sc-ребро -- в sc-дугу, sc-дуга общего вида -- в базовую sc-дугу.}
\scnaddlevel{-1}

\scnheader{синтаксически выделяемый класс sc-элементов в рамках Ядра SC-кода}
\scnidtf{класс sc-элементов Ядра SC-кода, определяемый на основе Алфавита Ядра SC-кода}
\scnhaselement{sc-коннектор}
\scnhaselement{sc-дуга}
\scnsuperset{Алфавит Ядра SC-кода}

\scnheader{sc-дуга}
\scnsubdividing{sc-дуга общего вида;базовая sc-дуга}

\scnheader{sc-коннектор}
\scnsubdividing{sc-ребро общего вида;sc-дуга общего вида}

\scnheader{синтаксически выделяемый класс sc-элементов в рамках Ядра SC-кода}
\scneqtoset{sc-элемент\\
\scnaddlevel{1}
    \scnsubdividing{sc-узел общего вида;sc-коннектор\\
        \scnaddlevel{1}
        \scnsubdividing{sc-узел общего вида;sc-дуга\\
        \scnaddlevel{1}
            \scnsubdividing{sc-дуга общего вида;базовая sc-дуга}
        \scnaddlevel{-1}
        }
        \scnaddlevel{-1}
    }
\scnaddlevel{-1}
}
\scnaddlevel{1}
\scnsourcecommentpar{Завершили представление Синтаксической классификации sc-элементов в рамках Ядра SC-кода}
\scnnote{Все Классы sc-элементов, входящие в состав синтаксической классификации sc-элементов являются синтаксически выделяемыми классами sc-элементов}
\scnaddlevel{-1}

\scnheader{Синтаксис Ядра SC-кода}
\scnnote{Синтаксис Ядра SC-кода задается: (1) Алфавитом Ядра SC-кода, (2) Отношением \textit{инцидентности sc-коннекторов*}, (3) Отношением \textit{инцидентности входящих sc-дуг*}}

\scnheader{инцидентность sc-коннекторов*}
\scnidtfdef{Бинарное ориентированное отношение, первым компонентом каждой ориентированной пары которого является некоторый sc-коннектор, а вторым компонентом является один из sc-элементов, соединяемых указанным sc-коннектором с некоторым другим sc-элементом, который указывается в другой паре инцидентности для этого же sc-коннектора}

\scnheader{инцидентность входящих sc-дуг*}
\scnidtfdef{Бинарное ориентированное отношение, первым компонентом каждой ориентированной пары которого является некоторая sc-дуга, а вторым компонентом --- sc-элемент, в который указанная sc-дуга входит, т.е. sc-элемент, который является вторым компонентом, соединяемым (связываемым) указанной sc-дугой}

\scnheader{Ядро SC-кода}
\scneqtoset{\scnstartsetlocal\\
\scnaddlevel{1}
\scnheaderlocal{инцидентность sc-коннекторов*}\\
\scnsuperset{инцидентность входящих sc-дуг*}
\scniselement{бинарное ориентированное отношение}
\scnendstruct
\scnaddlevel{-1}
;\scnfileitem{Для каждого sc-коннектора существует две и только две пары \textit{инцидентности sc-коннекторов*}, указанный sc-коннектор является первым связующим компонентом. При этом для каждой sc-дуги из двух указанных пар инцидентности \uline{одна} должна принадлежать отношению инцидентности \textit{входящей sc-дуги*}.}
;\scnfileitem{Пары инцидентности sc-коннекторов могут быть \uline{кратными}. То есть sc-коннектор может соединять (связывать) sc-элемент с самим собой. Такие sc-коннекторы будем называть петлевыми sc-коннекторами (петлевыми sc-ребрами и петлевыми sc-дугами).}
;\scnfileitem{Само \textit{Отношение инцидентности sc-коннекторов*} и, следовательно, \textit{Отношение инцидентности входящих sc-дуг*} не имеет кратных пар инцидентности. То есть sc-коннектор не может быть инцидентен самому себе.}
;\scnfileitem{В область определения \textit{Отношения инцидентности sc-коннекторов*} и \textit{Отношения инцидентности входящих sc-дуг*} входят не только sc-узлы общего вида, но и sc-коннекторы. Это значит, что sc-коннектор может соединять (связывать) не только sc-узел с sc-узлом, но также sc-узел с sc-коннектором и даже sc-коннектор с sc-коннектором.}}

\scnaddlevel{1}
\scnsourcecommentpar{Завершили перечень синтаксических правил Ядра SC-кода}
\scnaddlevel{-1}

\scnendstruct
\scnsourcecommentpar{Завершили изложение Синтаксиса Ядра SC-кода}

\scnheader{Денотационная семантика Ядра SC-кода}
\scnstartsubstruct

\scnheader{Ядро SC-кода}
\scnrelfrom{денотационная семантика}{Денотационная семантика Ядра SC-кода}
\scnaddlevel{1}
    \scnidtf{Описание соответствия информационных конструкций, принадлежащих Ядру SC-кода, и сущностей, описываемых этими конструкциями}
\scnaddlevel{-1}

\scnheader{параметр, заданный на множестве sc-элементов}
\scnhaselement{Алфавит Ядра SC-кода}
\scnhaselement{Алфавит SC-кода}
\scnhaselement{Структурная типология sc-элементов}
\scnhaselement{Типология sc-элементов по признаку константности}
\scnhaselement{Типология sc-элементов по признаку постоянства обозначаемой сущности}
\scnhaselement{Типология sc-элементов по признаку доступности sc-элемента в процессе эксплуатации и эволюции базы знаний}

\scnheader{Семантическая классификация sc-элементов}
\scnstartsubstruct

\scnheader{sc-элемент}
\scnidtf{обозначение описываемой сущности}
\scnrelto{разбиение}{Структурная типология sc-элементов}
\scnaddlevel{1} 
    \scneqtoset{обозначение терминальной сущности\\
    \scnaddlevel{1} 
    \scnsubdividing{обозначение материальной сущности\\
    \scnaddlevel{1} 
        \scnnote{К материальным сущностям относятся физические тела, поля, биологические объекты, системы и многое другое.}
    \scnaddlevel{-1}
    ;обозначение абстрактной терминальной сущности\\
    \scnaddlevel{1} 
        \scnnote{Примерами абстрактных терминальных сущностей являются предельно малые физические тела, точки различных пространств, числа.}
    \scnaddlevel{-1}
    ;обозначение внешней информационной конструкции\\
    \scnaddlevel{1} 
        \scnidtf{обозначение информационной конструкции, не являющейся конструкцией SC-кода и тем более Ядра SC-кода}
        \scnidtf{обозначение инородной для SC-кода информационной конструкции}
        \scnsuperset{обозначение файла}
        \scnaddlevel{1}
            \scnidtf{обозначение внешней информационной конструкции, представленной в электронной форме}
        \scnaddlevel{-1}
    \scnaddlevel{-1}
    }
    \scnaddlevel{-1}
    ;обозначение sc-множества\\
    \scnaddlevel{1}
    \scnsubdividing{обозначение sc-связки;обозначение sc-класса;обозначение sc-структуры}
    \scnaddlevel{-1}
    }
\scnaddlevel{-1}

\scnheader{sc-элемент}
\scnrelto{разбиение}{Типология sc-элементов по признаку константности}
\scnaddlevel{1}
    \scneqtoset{sc-константа\\
    \scnaddlevel{1}
        \scnidtf{константный sc-элемент}
        \scnidtf{обозначение конкретной (фиксированной) сущности}
    \scnaddlevel{-1}
    ;sc-переменная\\
    \scnaddlevel{1}
        \scnidtf{переменный sc-элемент}
        \scnidtf{обозначение произвольной сущности из некоторого множества сущностей}
        \scnidtf{sc-элемент, имеющий (принимающий) произвольное значение из некоторого множества sc-элементов}
    \scnaddlevel{-1}
    }
\scnaddlevel{-1}

\scnheader{sc-элемент}
\scnrelto{разбиение}{Типология sc-элементов по постоянства обозначаемых сущностей}
\scnaddlevel{1}
    \scneqtoset{обозначение постоянной сущности;обозначение временной сущности\\
    \scnaddlevel{1}
        \scnidtf{обозначение нестационарной сущности, факт существования которой зависит от времени}
        \scnsubdividing{обозначение прошлой сущности\\
        \scnaddlevel{1}
            \scnidtf{обозначение сущности, существовавшей до текущего момента времени}
        \scnaddlevel{-1}
        ;обозначение настоящей сущности\\
        \scnaddlevel{1}
            \scnidtf{обозначение сущности, существующей в текущий момент времени}
        \scnaddlevel{-1};обозначение будущей сущности\\
        \scnaddlevel{1}
            \scnidtf{обозначение сущности, существование которой прогнозируется или планируется в будущем}
        \scnaddlevel{-1}}
    \scnaddlevel{-1}
    }
\scnaddlevel{-1}

\scnheader{sc-элемент}
\scnrelto{разбиение}{Типология sc-элементов по признаку доступности sc-элемента в процессе эксплуатации и эволюции базы знаний}
\scnaddlevel{1}
    \scneqtoset{удаленный sc-элемент\\
        \scnaddlevel{1}
            \scnidtf{sc-элемент, считающийся логически удаленным, но присутствующим в описании истории эксплуатации и эволюции базы знаний}
        \scnaddlevel{-1};настоящий sc-элемент\\
        \scnaddlevel{1}
            \scnidtf{sc-элемент, входящий в состав эксплуатируемой части базы знаний}
        \scnaddlevel{-1};будущий sc-элемент\\
        \scnaddlevel{1}
            \scnidtf{sc-элемент, планируемый для включения в состав эксплуатируемой части базы знаний}
        \scnaddlevel{-1}}
\scnaddlevel{-1}

\scnheader{обозначение sc-связки}
\scnsubdividing{обозначение небинарной sc-связки;обозначение sc-пары\\
\scnaddlevel{1}
    \scnsubdividing{обозначение неориентированной sc-пары;обозначение ориентированной пары неизвестной направленности;обозначение ориентированной sc-пары\\
    \scnaddlevel{1}
        \scnsuperset{обозначение sc-пары принадлежности}
    \scnaddlevel{-1}
    }
\scnaddlevel{-1}
}

\scnheader{обозначение sc-пары принадлежности}
\scnsubdividing{обозначение позитивной sc-пары принадлежности;обозначение негативной sc-пары принадлежности;обозначение нечеткой sc-пары принадлежности}

\scnendstruct
\scnsourcecommentpar{Завершили представление Семантической классификации sc-элементов}


\scnheader{Соотношение между семантически и синтаксически выделяемыми классами sc-элементов в рамках Ядра SC-кода}
\scnstartstruct

\scnheader{семантически выделяемый класс sc-элементов}
\scnidtf{класс sc-элементов, определяемый сущностями, которые обозначаются этими sc-элементами, также доступностью (активностью использования) sc-элементов в процессе эксплуатации и эволюции базы знаний}
\scniselement{обозначение терминальной сущности}
\scnaddlevel{1}
\scnsuperset{sc-узел общего вида}
\scnaddlevel{-1}
\scniselement{обозначение небинарной sc-связки}
\scnaddlevel{1}
\scnsuperset{sc-узел общего вида}
\scnaddlevel{-1}
\scniselement{обозначение sc-пары}
\scnaddlevel{1}
\scnrelboth{пара пересекающихся множеств}{sc-узел общего вида}
\scnsuperset{sc-коннектор}
\scnnote{\textit{обозначение sc-пары} может быть представлено либо \textit{sc-узлом общего вида}, либо \textit{sc-коннектором}. При этом каждый \textit{sc-коннектор} представляет собой \textit{обозначение sc-пары}.}
\scnaddlevel{-1}
\scniselement{обозначение неориентированной sc-пары}
\scnaddlevel{1}
\scnidtf{обозначение бинарной неориентированной связи между sc-элементами}
\scnrelbothlist{пара пересекающихся множеств}{sc-узел общего вида;sc-ребро общего вида}
\scnnote{\textit{обозначение неориентированной sc-пары} может быть представлено либо \textit{sc-узлом общего вида}, либо \textit{sc-ребром}. При этом не каждое \textit{sc-ребро} представляет обозначение \textit{неориентированный sc-пары}. Некоторые из них представляют \textit{обозначения ориентированных sc-пар неизвестной направленности}.}
\scnaddlevel{-1}
\scniselement{обозначение ориентированной sc-пары неизвестной направленности}
\scnaddlevel{1}
\scnrelbothlist{пара пересекающихся множеств}{sc-узел общего вида;sc-ребро общего вида}
\scnaddlevel{-1}
\scniselement{обозначение ориентированной sc-пары}
\scnaddlevel{1}
\scnidtf{обозначение бинарной ориентированной связи между sc-элементами}
\scnrelbothlist{пара пересекающихся множеств}{sc-узел общего вида;sc-ребро общего вида}
\scnsuperset{sc-дуга общего вида}
\scnaddlevel{-1}
\scniselement{константная постоянная позитивная sc-пара принадлежности}
\scnaddlevel{1}
\scnrelbothlist{пара пересекающихся множеств}{sc-узел общего вида;sc-ребро общего вида;sc-дуга общего вида}
\scnsuperset{базовая sc-дуга}
\scnreltoset{пересечение множеств}{sc-константа;обозначение постоянной сущности;обозначение sc-пары принадлежности}
\scnaddlevel{-1}
\scniselement{обозначение sc-класса}
\scnaddlevel{1}
\scnsubset{sc-узел общего вида}
\scnaddlevel{-1}
\scniselement{обозначение sc-структуры}
\scnaddlevel{1}
\scnsubset{sc-узел общего вида}
\scnaddlevel{-1}

\scnendstruct
\scnsourcecommentpar{Завершили Описание Соотношения между семантически и синтаксически выделяемыми классами sc-элементов в рамках Ядра SC-кода}

\scnheader{Ядро SC-кода}
\scnrelfrom{семантические правила}{Семантические правила Ядра SC-кода}
\scnaddlevel{1}
    \scneqtoset{\scnfileitem{Каждый sc-элемент является знаком (обозначением) некоторой описываемой сущности.};\scnfileitem{Любая сущность может быть обозначена sc-элементом и, соответственно, описана в виде конструкции Ядра SC-кода.};\scnfileitem{С помощью sc-элементов можно описать любые связи между sc-элементами и/или между сущностями, которые обозначаются этими sc-элементами. При этом указанные связи трактуются как множества связываемых sc-элементов и обозначаются sc-ребрами, sc-дугами, а в случае небинарных связей -- sc-узлами.};\scnfileitem{Поскольку каждый sc-коннектор семантически трактуется как обозначение пары sc-элементов, связываемых (соединяемых) этим sc-коннектором, каждая пара инцидентности sc-коннектора семантически интерпретируется как обозначение пары принадлежности, связывающей sc-коннектор с одним из элементов обозначаемой им пары sc-элементов.};\scnfileitem{\uline{Любая} описываемая сущность может быть обозначена sc-узлом общего вида, но обратное неверно, т.к. некоторые сущности могут быть обозначены sc-ребрами общего вида, sc-дугами общего вида, базовыми sc-дугами.};\scnfileitem{Каждое sc-ребро является обозначением либо бинарной неориентированной связи между sc-элементами, либо бинарной ориентированной связи неизвестной направленности между sc-элементами.};\scnfileitem{Любая бинарная неориентированная связь между sc-элементами может быть обозначена sc-ребром, но обратное неверно.}}
\scnaddlevel{-1}

\scnheader{Правила синтаксической трансформации sc-элементов в рамках Ядра SC-кода}
\scnidtf{Правила модификации синтаксического типа sc-элементов в рамках Ядра SC-кода}
\scneqtoset{\scnfileitem{Если \textit{sc-узел общего вида} является \textit{обозначением sc-пары}, то он трансформируется в \textit{sc-коннектор}};\scnfileitem{Если \textit{sc-узел общего вида} является \textit{обозначением неориентированной sc-пары} или \textit{обозначением ориентированной sc-пары неизвестной направленности}, то он трансформируется в \textit{sc-ребро общего вида}};\scnfileitem{Если \textit{sc-узел общего вида} или \textit{sc-ребро общего вида} являются \textit{обозначением ориентированной sc-пары} и при этом дополнительно указана направленность этой sc-пары, то она трансформируется в \textit{sc-дугу общего вида}.};\scnfileitem{Если \textit{sc-узел общего вида} или \textit{sc-ребро общего вида} или \textit{sc-дуга общего вида} являются \textit{константными постоянными позитивными sc-парами принадлежности}, то они трансформируются в \textit{базовую sc-дугу}.}
}

\scnheader{следует отличать*}
\scnhaselementset{синтаксически выделяемый класс sc-элементов в рамках Ядра SC-кода;синтаксически выделяемый класс sc-элементов в рамках SC-кода;семантически выделяемый класс sc-элементов}

\scnendstruct
\scnsourcecommentpar{Завершили описание Денотационной семантики Ядра SC-кода}

\scnendstruct
\scnsourcecommentpar{Завершили сегмент "Описание Ядра SC-кода"}

\scnsegmentheader{Описание Расширения Ядра SC-кода}
\scnstartsubstruct

\scnheader{SC-код}
\scnidtf{Расширение Ядра SC-кода}
\scnidtf{Результат введения в Ядро SC-кода sc-узлов, имеющих содержимое и обозначающих файлы, хранимые в памяти ostis-системы}
\scnnote{Все файлы, представляющие собой электронные образы инородных для SC-кода информационных конструкций, можно представить в SC-кода с помощью графовых структур, в которых sc-элементы обозначают буквы текстов или пиксели изображений. Но такой вариант кодирования внешних для ostis-системы информационных конструкций не дает возможности непосредственно использовать накопленный человечеством арсенал электронных информационных ресурсов.}
\scnnote{Важнейшим видом файлов ostis-систем являются внешние идентификаторы sc-элементов (в частности, имена sc-элементов), представляющие sc-элементы в текстах внешних языков (в том числе, текстах SCs-кода и SCn-кода)} 
\scnnote{Результатом просмотренного расширения \textit{Ядра SC-кода} является расширение \textit{Алфавита Ядра SC-кода}}

\scnheader{SC-код}
\scnrelfrom{алфавит}{Алфавит SC-кода}
\scnaddlevel{1}
\scnhaselement{sc-узел}
\scnhaselement{sc-ребро}
\scnhaselement{sc-дуга}
\scnhaselement{базовая sc-дуга}
\scnhaselement{файл ostis-системы}
\scnaddlevel{-1}

\scnheader{файл ostis-системы}
\scnidtf{sc-узел с содержимым}
\scnidtf{sc-узел, имеющий содержимое}
\scnidtf{sc-узел, обозначающий файл, хранимый в памяти ostis-системы}
\scnidtf{знак файла ostis-системы}
\scnreltoset{разбиение}{ея-файл ostis-системы\\
\scnaddlevel{1}
\scnidtf{естественно-языковой файл ostis-системы}
\scnaddlevel{-1};файл ostis-системы, являющийся текстом формального языка\\
\scnaddlevel{1}
\scnsuperset{sc.g-файл ostis-системы}
\scnsuperset{sc.s-файл ostis-системы}
\scnsuperset{sc.n-файл ostis-системы}
\scnaddlevel{-1};файл ostis-системы, отражающий процесс изменения sc.g-текста;графический файл ostis-системы;файл ostis-системы, являющийся изображением;видео-файл ostis-системы;аудио-файл ostis-системы}
\scnreltoset{разбиение}{файл-экземпляр ostis-системы
\scnaddlevel{1}
\scnidtf{файл, являющийся конкретным электронным документом или электронным образом конкретной внешней информационной конструкции}
\scnaddlevel{-1};файл-класс ostis-системы
\scnaddlevel{1}
\scnidtf{файл, являющийся знаком множества всевозможных экземпляров (копий) этого файла}
\scnaddlevel{-1}
}

\scnheader{SC-код}
\scnrelfrom{синтаксис}{Cинтаксис SC-кода} 
\scnaddlevel{1}
\scnexplanation{\textit{\textbf{Синтаксис}} \textit{\textbf{SC-кода}} задается
\begin{scnitemize}
\item типологией (алфавитом) sc-элементов (атомарных фрагментов текстов sc-кода);
\item правилами соединения (инцидентности) sc-элементов (например, sc-элементы каких типов не могут быть инцидентными друг другу);
\item типологией конфигураций sc-элементов (связки, классы, структуры), связями между конфигурациями sc-элементов (в частности, теоретико-множественными)
\end{scnitemize}
}
\scnaddlevel{-1}
\scnrelfrom{денотационная семантика}{Денотационная семантика SC-кода} 
\scnaddlevel{1}
\scnexplanation{\textit{\textbf{Денотационная семантика}} \textit{\textbf{SC-кода}} задается
\begin{scnitemize}
\item
 семантической интерпретацией sc-элементов и их конфигураций;
\item
 семантической интерпретацией инцидентности sc-элементов;
\item
 иерархической системой предметных областей;
\item
 структурой используемых понятий в каждой предметной области (исследуемые классы объектов, исследуемые отношения, исследуемые классы объектов отношений из смежных предметных областей, ключевые экземпляры исследуемых классов объектов);
\item
 онтологиями предметных областей.
\end{scnitemize}
}
\scnaddlevel{-1}
\scnnote{Следует особо подчеркнуть, что  унификация и максимально возможное упрощение  \textbf{\textit{синтаксиса}} и \textbf{\textit{денотационной семантики}} внутреннего языка интеллектуальных компьютерных систем необходимы потому, что подавляющий объем \textbf{\textit{знаний}}, хранимых в составе  базы знаний интеллектуальной компьютерной системы, представляют собой \textbf{\textit{метазнания}}, описывающими свойства других знаний. Более того, по указанной причине конструктивное (формальное) развитие теории интеллектуальных компьютерных систем невозможно без уточнения (унификации, стандартизации) и обеспечения семантической совместимости различных видов знаний, хранимых в базе знаний интеллектуальной компьютерной  системы.  Очевидно, что многообразие форм представления семантически эквивалентных знаний делает разработку общей теории  интеллектуальных компьютерных систем практически невозможной. К \textit{метазнаниям}, в частности, следует отнести и различного вида логические высказывания и всевозможного вида программы, описания методов (навыков). Обеспечивающих решение различных классов информационных задач.}

\scnendstruct~
\scnsourcecomment{Завершили сегмент "Описание расширения Ядра SC-кода"}

\scnsegmentheader{Использование SC-кода для формального описания собственного синтаксиса}
\scnstartsubstruct

\bigskip
\scnfilelong{В предыдущем сегменте ``\textit{SC-код как синтаксическое расширение Ядра SC-кода}'' рассмотрен \textit{Синтаксис SC-кода} путём:
\begin{scnenumerate}
\item введения синтаксически выделяемых классов sc-элементов в рамках SC-кода\char59
\item описания \textit{теоретико-множественных связей} между указанными классами \textit{sc-элементов} (к такому описанию, в частности, относится \textit{Синтаксическая классификация sc-элементов в рамках SC-кода})\char59
\item введения двух отношений инцидентности sc-элементов -- Отношения инцидентности sc-коннекторов* и Отношения инцидентности входящих sc-дуг*\char59
\item описания \textit{Синтаксических правил SC-кода}, которые, прежде всего, описывают формальные свойства указанных выше отношений инцидентности sc-элементов.
\end{scnenumerate}

Однако для того, чтобы получить возможность \uline{все} (!) \textit{Синтаксические правила SC-кода} записать средствами самого \textit{SC-кода}, необходимо иметь \uline{явное} представление пар отношений инцидентности \textit{sc-элементов} в виде \textit{sc-дуг}, принадлежащим этим отношениям. В случае, если указанные \textit{sc-дуги} инцидентности являются \textit{sc-переменными}, логико-семантических проблем не возникнет. И этого, кстати, вполне достаточно, чтобы \textit{Синтаксические правила SC-кода}, сформулированные в виде логических высказываний, записать средствами \textit{SC-кода}. Но, если разрешить \textit{sc-дугам} инцидентности быть sc-константами, то, во-первых, в \textit{Синтаксические правила SC-кода} необходимо добавить Правило удаления константной sc-дуги инцидентности, если эта инцидентность представлена неявно, а, во-вторых, в \textit{Правила синтаксической трансформации sc-элементов} необходимо добавить Правило трансформации (замены) константной sc-дуги инцидентности на неявное представление этой инцидентности.

В теоретическом и, возможно, даже в практическом плане может быть интересна такая синтаксическая модификация (синтаксическое расширение) SC-кода, в котором: 
\begin{scnenumerate}
\item \uline{все} неявно представленные пары инцидентности sc-элементов заменяются на константные sc-дуги инцидентности -- неявно представленными парами инцидентности остаются \uline{только} пары инцидентности константных sc-дуг инцидентности с компонентами этих sc-дуг\char59
\item В \textit{Алфавит SC-кода} вводятся два новых синтаксически выделяемых класса sc-элементов -- класс sc-дуг инцидентности sc-коннекторов, а также класс sc-дуг инцидентности входящих sc-дуг.
\end{scnenumerate}

В результате такого преобразования конструкций SC-кода конструкции SC-кода перестают быть графовыми конструкциями нетрадиционного вида, в которых рёбра, гиперрёбра, дуги могут быть инцидентны другим рёбрам, гиперребрам и дугам, а становятся классическими графами с двумя типами дуг (с sc-дугами инцидентности sc-коннекторов и с sc-дугами инцидентности входящих sc-дуг) и с пятью типами вершин (с вершинами, представляющими \textit{sc-узлы общего вида}, с вершинами, представляющими sc-узлы, являющиеся знаками \textit{файлов ostis-системы}, с вершинами, представляющими \textit{sc-рёбра общего вида}, с вершинами, представляющими \textit{sc-дуги общего вида}, с вершинами, представляющими \textit{базовые sc-дуги}).

Рассмотренное преобразование конструкций SC-кода в теории графов называется подразбиением ... \scnauthorcomment{Найти ссылку, например Касяьнов-Евстигнеев}
}

\scnendstruct
\scnsourcecommentpar{Завершили Сегмент \dq{}\textit{Использование SC-кода для формального описания собственного синтаксиса}\dq{}}

\scnendstruct
\scnsourcecommentpar{Завершили раздел \currentnumber~  \dq{}\currentname\dq{}}

\newpage

\scnauthorcomment{уточнить то, что касается семантики}

\scnheader{следует отличать*}
\scnhaselementset{денотационная семантика*;описание денотационной семантики*;операционная семантика*;описание операционной семантики*}

\scnheader{следует отличать*}
\scnhaselementset{денотационная семантика знака;денотационная семантика текста;денотационная семантика языка}

\scnheader{следует отличать*}
\scnhaselementset{синтаксис
\scnaddlevel{1}
    \scnidtf{синтаксис языка}
    \scnidtf{отношение, связывающее тексты некоторого языка и соответствующие тексты того же или другого языка, описывающие синтаксическую структуру этих текстов}
\scnaddlevel{-1}
;синтаксис*
\scnaddlevel{1}
    \scnidtf{Отношение, связывающее язык и его синтаксис}
\scnaddlevel{-1}
;синтаксическая структура текста
\scnaddlevel{1}
    \scnidtf{информационная конструкция, описывающая синтаксическую структуру текста некоторого языка}
\scnaddlevel{-1}
;описание синтаксиса
\scnaddlevel{1}
    \scnidtf{информационная конструкция, описывающая синтаксис некоторого языка и его свойства, включая правила построения синтаксически корректных конструкций данного языка}
\scnaddlevel{-1}
}


\end{SCn}

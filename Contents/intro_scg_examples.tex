
\scnheader{sc.g-текст}
\scnhaselementrole{пример}{\scnfilescg{figures/intro/scg/example_triangle.png}}
\scnaddlevel{1}
\scnexplanation{Данный sc.g-текст содержит следующую информацию:
\begin{scnitemize}
\item Сущности \textit{Треугольник ABC}~~ и ~~\textit{Треугольник CDE} являются треугольниками (принадлежат классу \textit{треугольников}). При этом известно, что площадь \textit{Треугольника CDE} в 4 раза больше, чем площадь \textit{Треугольника ABC}, но конкретные значения ллощадей не известны\char59
\item Сущность \textit{Отрезок DE} является отрезком (принадлежит классу \textit{отрезков}) и является стороной \textit{Треугольника CDE}. Кроме того, у \textit{Отрезка DE} есть длина, измерение которой в сантиметрах составляет 5. Обратите внимание, что в данном случае для упрощения понимания использовано бинарное отношение \textit{длина*}, которое является \textit{неосновным понятием} и в базе знаний заменяется на \textit{базовую sc-дугу}, связывающую величину как класс эквивалентности с конкретной сущностью, входящей в данный класс, в данном случае -- \textit{Отрезок DE}\char59  
\item Сущность \textit{Треугольник AEB} является треугольником и имеет \textit{внутренний угол*}~~~ \textit{Угол AEB}. В свою очередь, \textit{Угол AEB} является \textit{углом} и имеет \textit{косинус*}, равный 0,5\char59
\item \textit{Треугольник AEB} имеет \textit{сторону*} (не указывается, какая именно из сторон имеется в виду), \textit{средней точкой*} которой является \textit{Точка O}. В свою очередь, \textit{Точка O} является центром некоторой \textit{Окружности O}, которая относится к классу \textit{окружностей}.
\end{scnitemize}
}
\scnaddlevel{-1}

\newpage

\scnheader{Пример sc.g-текста, трансформируемого по Первому направлению расширения Ядра SCg-кода}
\scneqscg{figures/intro/scg/scg_transf1.png}
\scniselement{sc.g-текст}
\scnexplanation{Здесь (в левом нижнем углу приведенного sc.g-текста) представлен \textit{sc.g-узел общего вида}, изображающий \textit{sc-узел общего вида}, которому соответствует \textit{основной sc-идентификатор*} в виде строки ``\textbf{\textit{ei}}''}
\scnrelfrom{трансформация sc.g-текста по Первому направлению расширения Ядра SCg-кода}{\scnfilescg{figures/intro/scg/scg_transf2.png}}
\scnaddlevel{1}
    \scniselement{sc.g-текст}
    \scnexplanation{\textit{sc.g-узлу общего вида} изображающему \textit{sc-узел}, внешним идентификатором которого является строка ``\textit{основной sc-идентификатор*}'' и который, соответственно является знаком \textit{бинарного ориентированного отношения}, каждая \textit{пара} которого связывает идентифицируемый \textit{sc-элемент} с его основным внешним sc-идентификатором, приписывается указанный внешний идентификатор изображаемого им \textit{sc-элемента}.}
    \scnrelfrom{трансформация sc.g-текста по Первому направлению расширения Ядра SCg-кода}{\scnfilescg{figures/intro/scg/scg_transf3.png}}
    \scnaddlevel{1}
        \scniselement{sc.g-текст} 
        \scnexplanation{В результате данной трансформации исходный \textit{sc.g-текст} трансформируется в один \textit{sc.g-общего вида}, которому приписывается \textit{основной sc-идентификатор} ``\textit{\textbf{ei}}''.}
    \scnaddlevel{-1}
\scnaddlevel{-1}

\newpage

\scnheader{Примеры sc.g-текстов, трансформируемых по Второму направлению расширения Ядра SCg-кода}
\scnstructinclusion

\scnmakeset{\scgfileitem{figures/intro/scg/scg2_ex1.png}\\
\scnaddlevel{1}
    \scnrelfrom{синтаксическая трансформация}{\scnfilescg{figures/intro/scg/scg2_ex1_1.png}}
    \scnaddlevel{1}
        \scnexplanation{Здесь вводится новый синтаксический вид \textit{sc.g-элементов} -- \textit{константный постоянный sc.g-узел общего вида}, изображаемый окружностью диаметром ***.}
    \scnaddlevel{-1}
\scnaddlevel{-1}
}

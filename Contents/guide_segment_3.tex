\scnsegmentheader{Основные принципы разработки исходных текстов Стандарта OSTIS на основе LaTex}
\scnstartsubstruct

\scnheader{принципы работы с LaTex}

\scnrelfromvector{основные принципы}{
	Основные принципы работы с командами scn-tex\\
	\scnrelfromset{этапы реализации}{
		\scnfileitem{Для того, чтобы иметь возможность, с одной стороны, формировать читабельный текст \textit{Стандарта OSTIS}, пригодный для его публикации в виде книги или включения его фрагментов в другие печатные издания, а с другой стороны, иметь формальный исходный текст \textit{Стандарта OSTIS}, который может быть автоматически протранслирован в базу знаний любой \textit{ostis-системы}, был выбран вариант разработки исходных текстов \textit{Стандарта} с использованием набора команд, разработанных на основе языка LaTex};
		\scnfileitem{Предлагаемый набор команд условно называется \textit{scn-tex}, поскольку в его основу положена идея того, чтобы разработчик писал исходный текст максимально приближенно к тому, как он будет видеть результат компиляции этого исходного текста (в SCn-коде) и при этом максимально был избавлен от необходимости учитывать особенности языка LaTex в работе. Таким образом, весь исходный текст стандарта формируется исключительно с использованием набора команд \textit{scn-tex}, запрещается использовать любые другие команды для форматирования текста, изменения шрифта, вставки внешних файлов и т.д. В рамках естественно-языковых фрагментов, входящих в состав стандарта, допускается использование команд LaTex для вставки специальных символов и математических формул. Для добавления файлов изображений в текст стандарта используются только команды \textit{scn-tex}. Также только команды \textit{scn-tex} используются для формирования нумерованных и маркированных списков, добавления закрывающих и открывающих скобок различного вида (кроме круглых)};
		\scnfileitem{Для выделения курсивом \textit{идентификаторов} в рамках естественно-языковых фрагментов, входящих в состав стандарта, используется только команда $\backslash$textit\{\}. Для выделения полужирным курсивом используется комбинация команд $\backslash$textbf\{$\backslash$textit\{\}\} (именно в таком порядке)};
		\scnfileitem{Каждая команда из набора \textit{scn-tex} начинается с префикса $\backslash$scn, после которого идет имя команды, примерно описывающее то, как связан текущий отображаемый фрагмент текста с описываемой сущностью\\
		\scntext{пример}{
			$\backslash$scnrelfrom - дуга ориентированного отношения, которая выходит из описываемой сущности в другую сущность\\
			$\backslash$scnrelto - дуга ориентированного отношения, которая входит в описываемую сущность другой\\
			$\backslash$scnnote - естественно-языковое примечание к описываемой сущности}\\
		Полный перечень команд можно увидеть в файле scn.tex, а примеры использования команд каждого типа - в исходных текстах стандарта};
		\scnfileitem{Для формирования отступов в SCn-тексте используется команда $\backslash$scnaddlevel\{x\}, где х - число уровней, на которое необходимо сместиться. Число х должно быть целым, но не обязательно положительным, как правило после смещения на определенное число уровней вправо следует смещение на то же число уровней влево. Для сброса уровня до нуля (левый край страницы) можно использовать команду $\backslash$scnresetlevel\\
		\scntext{иллюстрация примера исходного кода}{
			$\newline$
			\scnfilescg{Contents/guide_image/image1}
		}\\
		\scntext{иллюстрация результата компиляции примера}{
			$\newline$
			\scnfilescg{Contents/guide_image/image2}
		}
		}
	};
	Добавление разделов (отдельный файл и label)\\
	\scntext{пояснение}{Исходный текст каждого раздела \textit{Стандарта} настоятельно рекомендуется записывать в отдельном tex-файле для последующего удобства коллективной работы. Если раздел имеет большой размер и/или делится на фрагменты, то целесообразно разделить исходный текст раздела на несколько tex-файлов.Структура директорий с tex-файлами обычно повторяет иерархию разделов в Оглавлении \textit{Стандарта OSTIS.\\
		Добавление нового раздела осуществляется в 3 шага:
			\begin{scnitemizeii}
				\item Добавление имени раздела и его спецификации в Оглавление Стандарта OSTIS. В зависимости от уровня раздела в оглавлении для этого используются команды $\backslash$scchapter, $\backslash$scsection, $\backslash$scsubsection, $\backslash$scsubsubsection. $\backslash$scparagraph, $\backslash$scsubparagraph.\\
				Если есть необходимость специфицировать раздел прямо в оглавлении, то соответствующие команды размещаются в квадратных скобках сразу после команды scsection и других аналогичных (см. пример ниже) перед фигурными скобками, в которых записывается имя раздела;
				\item Добавление метки (label), соответствующей данному разделу. Метка нужна для того, чтобы иметь возможность обратиться к имени раздела в других местах текста не копируя его явно. Это удобно в случае переименования раздела.\\
				Имя метки рекомендуется формировать на основе англоязычных идентификаторов ключевых знаков данного раздела, например:\\  \textit{Предметная область и онтология базовых интерпретаторов логико-семантических моделей ostis-систем} -> sd\_interpreters (SD - subject domain). Как правило, имя метки совпадает с именем tex-файла, содержащего исходный текст раздела.\\
				Для вставки имени раздела в именительном падеже в какое-либо место текста необходимо использовать только команду $\backslash$nameref\{<имя метки>\}, запрещается набирать имя раздела вручную обычным текстом. Исключение составляют ситуации, когда имя раздела необходимо использовать в другом падеже или необходимо использовать только часть имени. Использование меток позволяет в случае переименования раздела автоматически изменять имя раздела во все местах текста, где осуществляется отсылка к этому имени;
				\item Подключение tex-файла с исходным текстом раздела при помощи команды include в рамках основного файла проекта (book.tex) и input в других местах.
			\end{scnitemizeii}
	}
	\scnaddlevel{1}
	\scntext{пример}{
		$\newline$
		\scnfilescg{Contents/guide_image/image3}
	}
	\scnaddlevel{-1}
	};
	Временное удаление раздела\\
		\scntext{пояснение}{
		Во время локальной работы над конкретным разделом нецелесообразно компилировать весь текст стандарта, поскольку полная компиляция занимает достаточно много времени. Для того чтобы ускорить компиляцию проекта можно временно исключить те разделы или группы разделов, которые в данный момент не важны. Проще всего это сделать закомментировав строку, в которой подключается текст соответствующего раздела или группы разделов\\
		\scnaddlevel{1}
		\scntext{пример}{
			$\newline$
			\scnfilescg{Contents/guide_image/image4}
		}
		\scntext{пример}{
			$\newline$
			\scnfilescg{Contents/guide_image/image5}
		}
	\scnaddlevel{-1}
	};
	Добавление библиографии\\
	\scnrelfromset{этапы реализации}{
		\scnfileitem{Суть задачи}
			\scntext{пояснение}{Одним из вариантов полезной и не очень объемной задачи в рамках проекта по развитию \textit{Стандарта OSTIS} является дополнение библиографии \textit{Стандарта}. Задачи такого плана хорошо подойдут для тех, кто хотел бы попробовать свои силы в работе над Стандартом и помогут адаптироваться к выполнению более сложных задач.\\
				Библиография в данном случае трактуется в самом широком смысле и включает не только перечень собственно библиографических источников, но и их достаточно подробную спецификацию, перечень проектов и систем, изучение которых представляет интерес в рамках \textit{Технологии OSTIS}, перечень конкретных персон, работавших  и работающих в областях, смежных с \textit{Технологией OSTIS} и т.д.\\
				Конкретными задачи по расширению библиографии Стандарта OSTIS являются:
				\begin{scnitemizeii}
					\item Привязка глав (разделов, параграфов) книг или конкретных статей по частям или целиком к соответствующим понятиям или конкретным сущностям из стандарта;\\
					\scntext{пример}{
					$\newline$
					\scnfilescg{Contents/guide_image/image6}}
					\scntext{пример}{
					$\newline$
					\scnfilescg{Contents/guide_image/image7}}
					\item Выделение конкретных цитат или перефразированных фрагментов каких-либо текстов и их привязка к соответствующим понятиям стандарта;\\
					\scntext{пример}{
					$\newline$
					\scnfilescg{Contents/guide_image/image8}}
					\scntext{пример}{
					$\newline$
					\scnfilescg{Contents/guide_image/image9}}
					\item Дополнение списка синонимичных идентификаторов, пояснений и примечаний для понятий стандарта с обязательным указанием того источника, откуда взят соответствующий идентификатор или фрагмент текста. В данном случае важно учитывать, чтобы описываемое понятие одинаково трактовалось как в рамках стандарта, так и в рамках того источника, откуда берется информация и не возникало противоречий в рамках стандарта. В стандарте могут быть описаны альтернативные точки зрения или противоречащие друг другу взгляды на одну и ту же проблему, но это должно быть явно помечено как противоречие;
					\item Описание сравнительного анализа различных сущностей, с указанием их сходств и отличий. Это актуально как для понятий, непосредственно описываемых в рамках каких-либо разделов стандарта, так и для сравнения \textit{Технологии OSTIS} с другими близкими технологиями, в частности, сравнения языков, разрабатываемых в рамках \textit{Технологии OSTIS}, с другими аналогичными языками, сравнения понятий, исследуемых в рамках \textit{Технологии OSTIS}, с близкими понятиями из других технологий и т.д.
				\end{scnitemizeii}
			};
		\scnfileitem{Как добавить свой библиографический источник}
			\scntext{пояснение}{Для описания библиографических источников используется средство BibTex.\\ 
				Для добавления нового библиографического источника необходимо выполнить следующие шаги:\\
				\begin{scnitemizeii}
					\item Убедиться, что нужный источник еще не присутствует в файле biblio.bib, который находится в репозитории с исходными текстами \textit{Стандарта OSTIS}. В настоящее время все библиографические источники изначально описываются в этом файле.
					\item Добавить в файл biblio.bib описание библиографического источника в соответствии с форматом описания BibTex. Для помощи в оформлении можно использовать различные бесплатные средства, например, сервис doi2bib позволяет сгенерировать bib-описание на основе идентификатора DOI, кроме того, многие онлайн-библиотеки позволяют выгрузить описание нужного источника в формат BibTex/
					\item Каждому источнику в соответствии с форматом BibTex присваивается некоторое условное имя (цитатный ключ или просто ключ), по которому затем можно процитировать соответствующий источник. В рамках \textit{Стандарта OSTIS} рекомендуется цитатные ключи источников в формате BibTex формировать путем транслитерации в латинский алфавит фамилии первого автора и добавления года издания источника.\\\scntext{пример}{
						Trudeau1993\\
						Golenkov2011\\
					}
					Если при этом возникает неоднозначность, связанная с тем, что существует несколько работ того же автора в один год, то в конце ключа рекомендуется добавлять строчные латинские буквы a, b, c и так далее.\\\scntext{пример}{
						Gribova2015a\\
						Gribova2015b\\
					}
					При формировании ключа для электронного источника или коллективной публикации, где невозможно выделить ключевого автора, рекомендуется формировать ключ из 1-2 английских слов или аббревиатур, позволяющих однозначно идентифицировать соответствующий источник. При использовании нескольких слов их можно соединять знаком нижнее подчеркивание, пробелы в ключах запрещены. При необходимости в конце ключа можно добавлять год издания.\\\scntext{пример}{
						IMS (библиографическая ссылка на сайт Метасистемы IMS.ostis)\\
						CYPHER (библиографическая ссылка на сайт с описанием языка Cypher)\\
						AIDictionary1992 (библиографическая ссылка на Словарь по исусственному интеллекту 1992 года издания)\\}
					\item Для добавленного источника необходимо описать его идентификатор, который далее будет использоваться в рамках текста Стандарта. Это делается при помощи BibTex поля shorthand, например (см. Правила идентификации библиографических источников).\\\scntext{пример}{
						shorthand = \{Trudeau.R.J.IntroGT-1993кн\}\\
						shorthand = \{Duchi.J..AdaptiveSubgradMethods-2011ст\}\\
						shorthand = \{Грибова.В.В..БазоваяТРИСОП-2015ст\}}
					\item Далее этот идентификатор может использоваться как в формальном тексте, также как и идентификатор любой другой сущности, так и в рамках естественно-языкового текста. Для автоматической вставки идентификатора библиографического источника в формальный либо естественно-языковой текст используется команда $\backslash$scncite\{<цитатный ключ>\}\\\scntext{иллюстрация примера исходного кода}{
					$\newline$
					\scnfilescg{Contents/guide_image/image10}}\\\scntext{иллюстрация результата компиляции примера}{
					$\newline$
					\scnfilescg{Contents/guide_image/image11}}
					\item Для каждого источника крайне желательно добавить его краткую аннотацию. Это делается при помощи BibTex поля annotation.\\\scntext{пример}{
					annotation = \{В этой книге представлены исследования по внедрению концептуальных основ, стратегий, методов, методологий, информационных платформ и моделей для разработки современных систем, основанных на знаниях\}
					}
					В рамках аннотации допускается использование средств форматирования естественно-языковых текстов, принятых в рамках Стандарта OSTIS, например, выделение курсивом или полужирным курсивом.\\
					Для вставки аннотации в формальный scn-текст используется команда $\backslash$scnciteannotation\{<цитатный ключ>\}.\\\scntext{иллюстрация примера исходного кода}{
					$\newline$
					\scnfilescg{Contents/guide_image/image12}}\\\scntext{иллюстрация результата компиляции примера}{
					$\newline$
					\scnfilescg{Contents/guide_image/image13}}
				\end{scnitemizeii}
			};
		\scnfileitem{Правила идентификации библиографических источников}
			\scntext{пояснение}{Идентификаторы статей, книг и других печатных работа строятся следующим образом:\\
				\begin{scnitemizeii}
					\item Пишется фамилия первого автора на том языке, на котором опубликована данная работа. Затем через точку ставятся инициал(-ы) первого автора.
					\item Если работа опубликована с участием только одного автора, то ставится одна точка, если нескольких - две точки.
					\item Пишется первое слово из названия работы на том языке, на котором опубликована данная работа. Допускается сокращение, если слово очень длинное.
					\item Перечисляются Заглавные первые буквы всех остальных слов названия работы за исключением служебных слов, таких как предлоги, частицы, артикли и т.п.
					\item Ставится дефис.
					\item Указывается год издания работы.
					\item Указывается 2-3 буквенный код, обозначающий тип работы.\\\scntext{пример}{
					кн или bk - книга\\
					ст или art - статья\\}\\\scntext{пример}{
					Мартынов.В.В.СемиологОИ-1974кн\\
					Мартынов В. В., Семиологические основы информатики, Минск, 1974\\}\\\scntext{пример}{Golenkov.V.V..MethodsTECCS-2019art\\
					Methods and tools for ensuring compatibility of computer systems / V. Golenkov [et al.] // Открытые семантические технологии проектирования интеллектуальных систем = Open Semantic Technologies for Intelligent Systems (OSTIS-2019) : материалы международной научно-технической конференции, Минск, 21 - 23 февраля 2019 г. / Белорусский государственный университет информатики и радиоэлектроники; редкол.: В. В. Голенков (гл. ред.) [и др.]. - Минск, 2019. - С. 25 - 52.\\}
					Идентификаторы электронных и прочих ресурсов формируются аналогичным образом, с учетом того, что опускается год издания и фамилии авторов, а также ставится буквенный код эл для обозначения электронного ресурса.\\\scntext{пример}{МетасистIMS-эл \\
					Cypher-2017эл }
			\end{scnitemizeii}}
		}
}




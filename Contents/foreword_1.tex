\scseparatedfragment[\scnidtf{Предисловие Стандарта OSTIS-2021}]{Предисловие к первому изданию Стандарта OSTIS}

\begin{SCn}

\scnsectionheader{\currentname}
\scntext{предисловие}{Первое издание \text{Стандарта OSTIS}, занимает особое место:
		\begin{scnitemize}
		\item Во-первых, это первый опыт издания (публикации) подобного документа, в рамках которого необходимо обеспечивать, с одной стороны, строгую формальность, а, с другой стороны, интуитивное и адекватное понимание формальных текстов со стороны читателей;
		\item Во-вторых, данный текст является описанием условно выделенной первой версии \textit{Стандарта OSTIS} (\textit{Стандарта OSTIS-2021}), в рамках которого представлены далеко не все разделы \textit{Стандарта OSTIS}. Эти разделы будут представлены в последующих версиях \textit{Стандарта OSTIS} (в \textit{Стандарте OSTIS-2022}, в \textit{Стандарте OSTIS-2023} и т.д.);
		\item Особенностью \textit{публикации} (издания) Стандарта OSTIS версии OSTIS-2021, как, впрочем, и всех последующих версий, является то, что она оформлена в виде \uline{внешнего представления} основной части \textit{базы знаний} специальной \textit{ostis-системы}, которая предназначена для комплексной поддержки проектирования \uline{семантически совместимых} \textit{ostis-систем}. Эту систему мы назвали \textit{Метасистемой IMS.ostis} (Intelligent MetaSystem for ostis-systems). Последовательность изложения материала во внешнем представлении \textit{базы знаний} не является единственно возможным маршрутом прочтения (просмотра) \textit{базы знаний}. Каждый читатель, войдя в \textit{Метасистему IMS.ostis}, может выбрать любой другой маршрут навигации по этой \textit{базе знаний}, задавая указанной метасистеме те \textit{вопросы}, которые в текущий момент его интересуют. Таким образом, читая предлагаемый вашему вниманию текст и одновременно работая с \textit{Метасистемой IMS.ostis}, можно значительно быстрее усвоить детали \textit{Технологии OSTIS} и значительно быстрее приступить к непосредственному использованию указанной технологии. Этому также способствует большое количество примеров семантических моделей различных фрагментов \textit{интеллектуальных компьютерных систем};
		\item Основной семантический вид \textit{разделов баз знаний ostis-систем} -- это формальное представление различных \textit{предметных областей} вместе с соответствующими им \textit{онтологиями}. При этом явно указываются связи между этими \textit{предметными областями} и \textit{онтологиями}. Таким образом, \textit{база знаний} \scnbigspace \textit{Метасистемы IMS.ostis}, как и любых других \textit{интеллектуальных компьютерных систем}, построенных по \textit{Технологии OSTIS}, представляет собой иерархическую систему связанных между собой формальных моделей \textit{предметных областей} и соответствующих им \textit{онтологий}. Соответственно этому структурирован и текст \textit{Стандарта OSTIS-2021};
		\item В основе \textit{Технологии OSTIS} лежит предлагаемая нами унификация \textit{интеллектуальных компьютерных систем}, основанная, в свою очередь, на \textit{смысловом представлении знаний} в \textit{памяти интеллектуальных компьютерных систем}. Таким образом, данную \textit{публикацию} \textit{Стандарта OSTIS-2021} можно рассматривать как версию \textit{стандарта} семантических моделей \textit{интеллектуальных компьютерных систем}. Последующие \textit{публикации}, посвящённые детальному описанию различных компонентов \textit{Технологии OSTIS}, будут также оформляться как внешнее представление соответствующих \textit{разделов базы знаний} \scnbigspace \textit{Метасистемы IMS.ostis} и будут отражать следующие этапы развития \textit{Технологии OSTIS}, следующие версии этой технологии, и, соответственно, следующие версии \textit{Метасистемы IMS.ostis};
		\item Все основные положения \textit{Технологии OSTIS} рассматривались и обсуждались на ежегодных \textit{конференциях OSTIS}, которые стали важным стимулирующим фактором становления и развития \textit{Технологии OSTIS}. Мы благодарим всех активных участников этих конференций;
		\item Важной задачей \textit{Стандарта OSTIS-2021} была выработка стилистики формализованного представления научно-технической информации, которая одновременно была бы понятна как человеку, так и интеллектуальной компьютерной системе. По сути это принципиально новый подход к оформлению научно-технических результатов, позволяющий:
		\begin{scnitemizeii}
			\item существенно повысить уровень автоматизации анализа качества (корректности, целостности) научно-технической информации;
			\item интеллектуальным компьютерным системам непосредственно (без какой-либо дополнительной {} доработки) использовать информацию (знания), содержащуюся в разработанных специалистами документах;
			\item существенно упростить согласование точек зрения различных специалистов, входящих в коллектив разработчиков той или иной научно-технической документации.
		\end{scnitemizeii}
		\item Для выработки стилистики и формального представления научно-технической информации нам было важно привлечь к обсуждению и анализу материала \textit{Стандарта OSTIS-2021} как можно больше коллег, участвующих в развитии и применении \textit{Технологии OSTIS}. При этом некоторых коллег мы включили в число соавторов соответствующих разделов монографии.
		Основной целью написания \textit{Стандарта OSTIS-2021} является создание технологических и организационных предпосылок к принципиально новому  подходу к организации \textit{научно-технической деятельности} в любой области и, в частности, в области создания и перманентного развития комплексной технологии проектирования и производства семантически совместимых интеллектуальных компьютерных систем (\textit{Технологии OSTIS}). Суть указанного подхода заключается  в глубокой конвергенции и интеграции результатов деятельности всех специалистов, участвующих в создании и развитии \textit{Технологии OSTIS}, путем организации коллективной разработки \textit{базы знаний}, являющейся формальным представлением полной \textit{Документации Технологии OSTIS}, отражающей текущее состояние этой технологии.
	\end{scnitemize}
В настоящее время уровень требований, предъявляемых к комплексу технологий искусственного интеллекта существенно повысился -- возникла необходимость разработки компьютерных систем, которые не только обладают высоким уровнем интеллекта, но и обладают семантической совместимостью, взаимопониманием, способностью координировать свою деятельность с другими системами при коллективном решении сложных "внештатных"{} задач. Очевидно, что эти требования предполагают существенное развитие стандартов интеллектуальных компьютерных систем. Важнейшая особенность стандарта интеллектуальной компьютерной системы, обеспечивающего их семантическую совместимость и взаимопонимание, заключается в том, что для этого интеллектуальные компьютерные системы должны использовать:
\begin{scnitemize}
	\item  один и тот же язык внутреннего представления знаний;
	\item  один и тот же язык их коммуникации;
	\item  одну и ту же систему понятий;
	\item  обладать достаточно большим количеством  общих (одинаковых) знаний.
\end{scnitemize}

Следовательно, разработка стандарта интеллектуальной компьютерной системы в той части этого стандарта, которая связана с выделением и формализацией общих (одинаковых) знаний, хранимых в памяти интеллектуальной компьютерной системы и необходимых для обеспечения их взаимопонимания, фактически осуществляет разработку достаточно большой  одинаковой части всех интеллектуальных компьютерных систем   (и не только прикладных),  что существенно  сокращает сроки их разработки.

К этому можно добавить возможность и целесообразность  одинаковой  для всех интеллектуальных компьютерных систем реализации целого ряда их способностей: 
\begin{scnitemize}
	\item способности \uline{понимать} информацию, которой они обмениваются между собой,
	\item способности  \uline{договариваться} и \uline{координировать} свои действия при коллективном решении сложных интеллектуальных задач в условиях нештатных (аномальных, нестандартных, непредусмотренных) ситуаций,
	\item способности \uline{принимать решения} на основе их глубокого  обоснования,
	\item способности \uline{обучаться} и многие другие способности, обеспечивающие необходимый уровень интеллекта разрабатываемых интеллектуальных компьютерных систем.
\end{scnitemize}

Данная монография является первым этапом на пути комплексного решения указанных выше проблем.
Предназначена она одновременно:
\begin{scnitemize}
	\item для студентов, магистрантов и аспирантов, обучающихся по специальности ``Искусственный интеллект'';
	\item для разработчиков прикладных интеллектуальных компьютерных систем;
	\item для разработчиков технологий проектирования и производства интеллектуальных компьютерных систем;
	\item для научных работников, создающих новые модели и методы решения интеллектуальных задач.
\end{scnitemize}

Данная монография сочетает:
\begin{scnitemize}
	\item строгую формализацию представляемой информации и её доступность (возможность первичного её понимания без предварительного изучения используемого \uline{формального} языка);
	\item традиционную ("пассивную"{}) форму представления материала (в "электронном"{} и "бумажном"{} виде) с "активной"{} формой в виде интеллектуальной справочной системы, когда компьютерная система не только обеспечивает оперативное редактирование информации, но и помогает пользователям  быстрее и  качественнее усваивать эту информацию (за счет возможности отвечать на широкий спектр вопросов и учитывать индивидуальные особенности, потребности и интересы пользователей). 
\end{scnitemize}

	Область искусственного интеллекта сочетает в себе как  научно-исследовательский аспект и   создание технологий разработки интеллектуальных компьютерных систем, а так и непосредственно разработку самих интеллектуальных компьютерных систем. Эта область развивается настолько быстрыми темпами, что за время обучения студентов и магистрантов ситуация в области искусственного интеллекта меняется существенно, поэтому подготовка специалистов в этой области требует особого подхода, учитывающего высокий уровень сложности этой научно-технической области, а также быстрые темпы развития теории интеллектуальных компьютерных систем, технологий их проектирования, непосредственно практика разработки конкретных прикладных интеллектуальных компьютерных систем.
	
	Если специалист в области искусственного интеллекта не будет постоянно ориентироваться в тенденциях развития каждого из этих направлений развития работ в этой области, то он быстро перестанет быть конкурентноспособным. Это значит, что специалист в области искусственного интеллекта должен быть в достаточной степени и ученым, и создателем технологий следующего поколения, и разработчиком конкретных приложений.
	
	Таким образом, подготовку специалистов в области искусственного интеллекта необходимо ориентировать не на конкретное состояние науки, технологии и практики в этой области, а на перманентный процесс эволюции всех этих направлений.
	
	Сформировать у студентов и магистрантов реальные навыки в области искусственного интеллекта можно только путем поэтапного и непосредственного их включения в  реальную деятельность в этой области (и в научно-исследовательскую деятельность, и в развитие технологий искусственного интеллекта, и в разработку прикладных интеллектуальных компьютерных систем на основе текущего состояния соответствующих технологий). Но для этого необходимо создать соответствующую научно-исследовательскую и инженерную инфраструктуру.
	
	Научно-исследовательская деятельность  в области искусственного интеллекта предполагает исследование феномена интеллекта и создание принципиально новых подходов (моделей и методов) к решению интеллектуальных задач и к разработке принципов организации соответствующих компьютерных систем.
	
	\uline{Развитие технологий искусственного интеллекта} включает в себя:
	\begin{scnitemize}
		\item разработку стандарта интеллектуальных компьютерных систем, соответствующего текущему состоянию технологий искусственного интеллекта;
		\item разработку методов, средств проектирования и реализации интеллектуальных компьютерных систем.
	\end{scnitemize}
	
	Разработка прикладных интеллектуальных компьютерных систем  предполагает грамотное применение соответствующих технологий.
	
	
	Методику обучения необходимо ориентировать не только на формирование навыков разработки прикладных интеллектуальных компьютерных систем по заданной технологии, но и на формирование навыков перманентного совершенствования  и непосредственно самих прикладных интеллектуальных компьютерных систем, и технологий их разработки, и принципов (моделей и методов) решения интеллектуальных задач и организации интеллектуальных систем.


Данная монография рассматривается нами как первый выпуск целой серии коллективных монографий, которые будут представлять последующие версии \textbf{\textit{Стандарта Технологии OSTIS}} (Open Semantic Technology for Intelligent Systems -- Стандарта технологии, ориентированной на разработку семантически совместимых интеллектуальных компьютерных систем). При этом предполагается существенное расширение авторского коллектива и организация всей работы на развитие Стандарта Технологии OSTIS как открытого проекта, целью которого является коллективное совершенствование базы знаний, посвященной детальному описанию этого стандарта.

При этом при подготовке даже данного издания текущей версии Стандарта Технологии OSTIS мы приобрели хороший опыт организации коллективной деятельности такого рода, привлекая к этой работе целый ряд аспирантов, магистрантов и студентов, а также сотрудников других организаций.

Вклад некоторых из них в ряд разделов монографии позволил включить их в число соавторов этих разделов, что отражено непосредственно в тексте монографии.

Авторы выражают благодарность:

\begin{scnitemize}
	\item Cотрудникам кафедры Интеллектуальных информационных технологий Белорусского государственного университета информатики и радиоэлектроники и кафедры Интеллектуальных информационных технологий Брестского государственного технического университета, а также сотрудникам ОАО <<Савушкин продукт>>;
	\item студентам кафедры Интеллектуальных информационных технологий Белорусского государственного университета информатики и радиоэлектроники Банцевич К.А., Бутрину С.В., Василевской А.П., Меньковой Е.А., Жмырко А.В., Григорьевой И.В., Загорскому А.Г., Марковцу В.С., Киневичу Т.О. за оказание технической помощи при подготовке текста к печати;
	\item ООО <<Интелиджент семантик системс>> и его генеральному директору Т. Грюневальду за финансовую поддержку работ по развитию \textit{Технологии OSTIS}, а также финансовую поддержку издания \textit{Стандарта OSTIS};
	\item Рецензентам -- д-ру техн. наук, профессору Александру Николаевичу Курбацкому и д-ру техн. наук, профессору Александру Арсентьевичу Дудкину;
	\item Коллегам из Советской (ныне Российской) Ассоциации Искусственного интеллекта и коллегам из Белорусского объединения специалистов в области искусственного интеллекта;
	\item Членам Программного Комитета ежегодных конференций OSTIS, а также всем участникам этих конференций за плодотворное и конструктивное обсуждение направлений развития семантических технологий и технологии OSTIS в частности.
\end{scnitemize}}

\scnidtf{Издание Документации Технологии OSTIS-2021}
\scnidtf{Первое издание (публикация) Внешнего представления Документации Технологии OSTIS в виде книги}
\scniselement{публикация}
\scnaddlevel{1}
\scnidtf{библиографический источник}
\scnaddlevel{-1}
\scniselement{официальная версия Стандарта OSTIS}
\scniselement{бумажное издание}
\scniselement{научное издание}
\scnrelfrom{рекомендация издания}{Совет БГУИР}
\scnrelfromset{рецензенты}{Курбацкий А.Н.; Дудкин А.А.}
\scnrelfrom{издательство}{Бестпринт}
\scniselement{\scnstartsetlocal\scnendstructlocal}
\scnaddlevel{1}
\scniselement{УДК}
\scnaddlevel{1}
\scniselement{параметр}
\scnaddlevel{-1}
\scnrelfrom{Индекс УДК}{004.8}
\scnaddlevel{-1}
\scnidtftext{ISBN}{978-985-7267-13-2}

\newpage

\end{SCn}
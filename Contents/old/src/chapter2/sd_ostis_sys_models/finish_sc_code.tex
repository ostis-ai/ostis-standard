\begin{SCn}

\scnsectionheader{\nameref{sd_sc_code}}

\scnstartsubstruct

\scnheader{sc-модель семантической совместимости}
\scnexplanation{Важнейшим этапом эволюции любой технологии является переход к \textbf{компонентному проектированию} на основе постоянно пополняемый \textbf{библиотеки многократно используемых компонентов}.

Основной проблемой для реализации компонентного проектирования являются 
\begin{scnitemize}
    \item унификация компонентов по форме; 
    \item разработка стандартов обеспечивающих совместимость этих компонентов.
\end{scnitemize}

Для реализации компонентного проектирования \textit{базы знаний} требуется:

\begin{scnitemize}
    \item универсальный язык представления знаний;
    \item универсальная процедура интеграции знаний в рамках указанного языка; 
    \item разработка стандарта, обеспечивающего \textbf{семантическую совместимость} интегрируемых знаний (таким стандартом является согласованная система используемых понятий).
\end{scnitemize}

Даже для смыслового представления знаний нужны, своего рода, смысловые семантические координаты, роль которых выполняет используемая система понятий (своего рода, ключевых знаков), которая, в свою очередь, описывается (специфицируется, задается) иерархической системой семантически связанных между собой \textit{онтологий}.

Другими словами, человеческие знания необходимо привести к общему "семантическому знаменателю"\ (к общей семантической системе координат), чем является постоянно уточняемая система понятий, специфицируемая в виде объединенной онтологии. Эта объединенная онтология \textbf{стратифицируется} на частные онтологии, эволюционируемые в достаточной степени \textbf{независимо} друг от друга.

Одним из критериев семантической совместимости новой информации с базой знаний, в которую эта информация погружается, можно сформулировать следующим образом. 

Все знаки, являющиеся новыми для воспринимающей базы знаний (в которую погружаются эти новые знаки) должны быть в достаточной степени специфицированы (а для новых понятий -- определены) через понятия, известные базе знаний.

Стандарт смыслового представления информации (\textit{SC-код}) дает возможность, с одной стороны, повысить уровень совместимости компьютерных систем, а с другой стороны, формально уточнить понятие интеграции компьютерных систем и их компонентов. 

При этом следует отличать: 
\begin{scnitemize}
    \item Cемантическую интеграцию двух текстов, принадлежащих языку смыслового представления информации (SC-коду). В результате такой интеграции два исходных sc-текста преобразуются в один интегрированный текст;
    \item Семантическую интеграцию двух разных моделей обработки информации, представленной в SC-коде;
    \item Модель понимания текста некоторого внешнего языка путем трансляции исходного внешнего текста в SC-код и последующbм погружением построенного sc-текста в базу знаний, представленную в SC-коде.
    \item Семантическую интеграцию двух компьютерных систем, построенных на основе SC-кода;
    \item Семантическую совместимость компьютерной системы, построенной на основе SC-кода, с ее пользователями.
\end{scnitemize}
}

\scnendstruct

\end{SCn}
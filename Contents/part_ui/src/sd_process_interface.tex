
\textbf{Аудиоинтерфейс ostis-систем}
\scnheader{Интерфейс}

\scnsuperset{речевой интерфейс}

\begin{scnindent}
\scnidtf{обеспечение человеко-машинного взаимодействия в составе современных коммуникационных,
мультимедийных и интеллектуальных систем }

\scniselement{интеллектуальные комппьютерные сети нового поколения}

\scniselement{SILK-интерфейсы пользовательских интерфейсов интеллектуальных
компьютерных систем}

\scntext{примечание}{Элементы речевого интерфейса сложной системы должны быть совместимы с более высокоуровневыми модулями обработки естественно-языковой
информации, такими как модули понимания (SLU, spoken language understanding) и генерации речи (SLG, spoken
language generation), управления диалогом (DM, dialog manager) }
\end{scnindent}

\scnsuperset{аудиоинтерфейс}

\begin{scnindent}
\scnidtf{аппаратно-программный комплекс
осуществляющий анализ и синтез сигналов во всем доступном спектре параметров носителей акустической информации}

\scniselement{интеллектуальные комппьютерные сети нового поколения}

\scniselement{SILK-интерфейсы пользовательских интерфейсов интеллектуальных
компьютерных систем}

\scntext{примечание}{Сложные системы, кроме стандартных модулей распознавания (ASR, automatic speech recognition) и синтеза (TTS, text to speech), на уровне аудиоинтерфейса должны содержать модели, определяющие наличие/отсутствие речи в аудиосигнале в сложной акустической
обстановке, классификации звуков окружающей среды, распознавание диктора и пр.}
\end{scnindent}

\begin{scnrelfromset}{тенденции развития}
\scnfileitem{экономические показатели и прогнозы развития рынка речевых технологий, текущие среднегодовые темпы
роста которого, по оценкам экспертов, составляют порядка 22\%, а совокупный объем будет равен 59,6 млрд.
долл. США к 2030}

\scnfileitem{появление широкого спектра продуктов на основе речевого интерфейса, получивших массовое распространение. В первую очередь это персональные голосовые ассистенты, такие как “Alexa” (Amazon), “Siri” (Apple),
“Сortana” (MicroSoft), “Алиса” (Yandex)}

\scnfileitem{интерес со стороны научного сообщества, выражающийся в росте публикаций в этом направлении исследований на 15\% за последние 5 лет}
\end{scnrelfromset}

\scnheader{Аудиоинтерфейс}

\scntext{цель}{рассмотрение процесса проектирования аудиоинтерфейса как интерфейсной подсистемы в рамках общего процесса разработки интеллектуальной компьютерной системы и построении ее
формальной логико-семантической модели}

\begin{scnrelfromset}{процесс создания}{
\scnfileitem{произвести декомпозицию информационной компьютерной системы на компоненты. Качество декомпозиции
при этом определяется простотой последующего синтеза общей формальной модели из формальных моделей
выделенных компонентов}

\scnfileitem{провести конвергенцию выделенных компонентов в целях построения совместимых (легко интегрируемых)
формальных моделей этих компонентов}

\scnfileitem{провести интеграцию построенных формальных моделей выделенных компонентов и получить общую формальную модель}
}

\end{scnrelfromset}

\scntext{примечание}{На основе Технологии OSTIS подсистема аудиоинтерфейса будет строиться как
\textit{многократно используемый компонент}, который в будущем будет при необходимости встраиваться в различные
ostis-системы}

\begin{scnrelfromset}{преимущества использования Технологии OSTIS}
\scnfileitem{ в рамках указанной технологии предложены унифицированные средства представления различных видов
знаний, в том числе — метазнаний, что позволяет описать всю необходимую для анализа информацию в
одной базе знаний в едином ключе (см. Давыденко И.Т.МоделМиСРГБ-2017дс)}

\scnfileitem{используемый в рамках технологии формализм позволяет специфицировать в базе знаний не только понятия,
но и любые внешние с точки зрения базы знаний файлы (например, фрагменты речевого сигнала), в том числе
— синтаксическую структуру таких файлов}

\scnfileitem{предложенный в рамках технологии подход к представлению различных видов знаний (см. Давыденко
И.Т.МоделМиСРГБ-2017дс) и моделей их обработки (см. Shunkevich.D.V.AgentMMaToC-2018art) обеспечивает
модифицируемость ostis-систем, то есть позволяет легко расширять функциональные возможности системы,
вводя новые виды знаний (новые системы понятий) и новые модели обработки знаний}

\end{scnrelfromset}

\scnheader{Задача построения пользовательского интерфейса}

\begin{scnrelfromset}{решение}

\scnfileitem{наличие sc-модели компонентов пользовательского интерфейса}

\scnfileitem{наличие интерфейсных действий
пользователей}

\scnfileitem{классификация пользовательских интерфейсов}

\end{scnrelfromset}

\scntext{примечание}{При проектировании интерфейса
используется компонентный подход, который предполагает представление всего интерфейса приложения в виде
отдельных специфицированных компонентов, которые могут разрабатываться и совершенствоваться независимо}

\scnheader{Предметная область и онтология задач аудиоинтерфейса ostis-систем}

\begin{scnhaselementrolelist}{ключевое понятие}

\scnitem{сигнал}
\scnitem{акустический сигнал}
\scnitem{аудиосигнал}
\scnitem{речевой сигнал}

\end{scnhaselementrolelist}

\scnhaselement{понятие, лежащее в семантической окрестности подпространства функционального назначения аудиоинтерфейсов и обработки аудиосигналов}
\begin{scnindent}
\begin{scnrelfromset}{разбиение}

\scnitem{анализ аудиосигнала}

\scnitem{синтез аудиосигнала}

\scnitem{кодирование аудиосигнала}

\scnitem{шумоочистка аудиосигнала}

\scnitem{классификация аудиосигнала}

\scnitem{классификация событий}

\scnitem{детектирование аномалий} 

\scnitem{идентификация положения источника в пространстве} 

\end{scnrelfromset}
\end{scnindent}

\scnhaselement{понятие, лежащее в семантической окрестности функционального назначения речевого сигнала и обработки аудиосигналов}
\begin{scnindent}
\begin{scnrelfromset}{разбиение}
\scnitem{анализ речевого сигнала}

\scnitem{детектирование наличия ключевых слов}

\scnitem{активация по ключевому слову}

\scnitem{детектирование наличия речевого сигнала}

\scnitem{синтез речевого сигнала}

\scnitem{синтез текста в речь}

\scnitem{эмоциональный синтез текста в речь}

\scnitem{cинтез пения}

\scnitem{распознавание эмоций в речевом сигнале}

\scnitem{распознавание диктора}

\scnitem{сепарация речи}

\scnitem{классификация диктора}

\scnitem{верификация диктора}

\end{scnrelfromset}
\end{scnindent}

\scnheader{сигнал}
\begin{scnrelfromset}{математическое описание обрабатываемого сигнала}

\scnitem{аналоговый сигнал}

\scnitem{дискретный сигнал}

\scnitem{цифровой сигнал}

\scnitem{периодический сигнал}

\scnitem{апериодический сигнал}

\scnitem{гармонический сигнал}

\scnitem{тональный сигнал}

\scnitem{шумовой сигнал}

\scnitem{импульсный сигнал}

\end{scnrelfromset}


\scnheader{речевой сигнал}
 \begin{scnrelfromset}{сегментная и надсегментная характеристика}
\scnitem{амплитуда сигнала}
    
\scnitem{частота сигнала}

\scnitem{фаза сигнала}

\scnitem{интенсивность сигнала}

\scnitem{длительность сигнала}

\scnitem{мощность/энергия сигнала}

\scnitem{осцилограмма сигнала}

\scnitem{спектр сигнала}

\scnitem{частота дискретизации сигнала}

\scnitem{степень квантования сигнала}
\end{scnrelfromset}


\textbf{Предметная область и онтология моделей параметрического представления
сигнала}


\scntext{примечание}{Все вышеперечисленные задачи взаимосвязаны, поскольку относятся к одному и тому же объекту исследования
— речевому сигналу. Решение каждой из них непосредственно либо косвенно зависит от эффективности моделирования речи как сложного феномена в различных аспектах: параметрическое представление речевого сигнала
и выделение его свойств, моделирование процесса фонации, восприятия и интерпретации содержания речевого сообщения (в том числе фонетического, смыслового, эмоционального). Это делает создание универсальных
способов обработки речевых сигналов перспективным научным направлением.}

\begin{scnrelfromset}{В контексте перечисленных задач
моделирование речи можно условно разделить на три уровня
}
\scnfileitem{моделирование сигнала в общем виде, используя отсчеты во временной или частотной области;}

\scnfileitem{моделирование характеристик сигнала, являющихся специфическими для речи и связанных с процессом
фонации (таких как частота основного тона, последовательность возбуждения и огибающая амплитудного
спектра);
}

\scnfileitem{моделирование высокоуровневых речевых характеристик (голос, акцент, экспрессия, фонетическое и семантическое содержание речевого сообщения). Каждый следующий уровень основывается на предыдущем и
подразумевает использование специальных методов параметрического описания.}
\end{scnrelfromset}

\begin{scnindent}
\scntext{примечание}{К первым двум уровням относятся широко известные в цифровой обработке речевых сигналов модели на основе
линейного предсказания, кепстральных коэффициентов и синусоидальных параметров.}
\end{scnindent}

\begin{scnrelfromset}{Среди подходов, использующих синусоидальное описание сигнала, в настоящее время наиболее перспективными
являются смешанные (гибридные) модели, учитывающие возможность разных режимов фонации
}
\scnfileitem{с участием голосовых связок (вокализованная речь)}

\scnfileitem{без участия голосовых связок (невокализованная речь)}
\end{scnrelfromset}


\begin{scnrelfromset}{Вокализованная речь рассматривается как квазипериодический (детерминистский) сигнал, в то время как невокализованная — как непериодический (стохастический) сигнал. Наиболее известной среди существующих моделей
является модель гармоники+шум, которая используется для решения таких сложных задач, как
}
\scnitem{создание речевых интерфейсов}

\scnitem{распознавание речи}
\scnitem{создание речевых интерфейсов}
\scnitem{конверсия голоса}
\scnitem{синтез речи по тексту}
\scnitem{шумоподавление}
\scnitem{повышение
разборчивости и субъективного качества речевых сигналов}
\scnitem{коррекция акцента}
\end{scnrelfromset}

\scntext{примечание}{Преимуществом данной модели
является теоретическая возможность воссоздания вокализованных звуков в виде непрерывных функций с изменяющимися параметрами, что позволяет получить эффективное описание процесса фонации и избежать наложения
смежных фрагментов, разрыва фаз при синтезе речи. Недостатком модели является высокая сложность алгоритмов
анализа и синтеза, обусловленная нестационарностью речевого сигнала.}






\scnheader{Моделирование речевого сигнала на основе линейного предсказания}
\begin{scnrelfromset}{преимущества}
\scnfileitem{ раздельное описание сигнала в виде огибающей спектра и сигнала возбуждения;}
\scnfileitem{низкая вычислительная сложность;}
\end{scnrelfromset}
\begin{scnrelfromset}{недостатки}
\scnfileitem{не обеспечивает эффективных способов для
параметрической обработки сигнала возбуждения и непрерывного синтеза выходного сигнала;}
\scnfileitem{каждый речевой
фрагмент (кадр) сигнала представляет собой отдельную независимую единицу и при синтезе возникает проблема
согласования соседних кадров;}
\scnfileitem{несогласованное изменение огибающей амплитудного и фазового спектра при переходе от кадра к кадру вызывает появление слышимых артефактов;}
\scnfileitem{оценка огибающей спектра при
помощи классических методов линейного предсказания представляет собой усреднение по всему кадру, вследствие чего ее точность ограничена;}
\scnfileitem{оценка огибающей спектра представляет собой усреднение по всему кадру, вследствие чего ее точность ограничена;}
\end{scnrelfromset}

\begin{scnrelfromset}{ В последнее время предпочтение отдается моделям, использующим синусоидальное
представление сигнала и в первую очередь это касается приложений, подразумевающих синтез речевого сигнала
с измененными параметрами, таких как
}
\scnitem{изменение интонации}
\scnitem{конверсия голоса}
\scnitem{синтез речи по тексту}
\end{scnrelfromset}

\scntext{примечание}{ Использование кепстральных коэффициентов для моделирования речевых сигналов также является классическим
подходом. Наиболее хорошо разработанной системой моделирования речевых сигналов, использующей кепстральные коэффициенты, является TANDEM–STRAIGHT. Так же как и для классических способов анализа на основе линейного предсказания, при оценке кепстральных коэффициентов предполагается стационарность сигнала на протяжении интервала наблюдения.
Оценка огибающей амплитудного спектра требует сглаживания и также недостаточно точна по сравнению с моделями на основе синусоидальных.}


\begin{scnrelfromset}{ В зависимости от приложения процесс обработки речевого сигнала с использованием той или иной модели обычно
включает
}
\scnitem{анализ (определение параметров модели)}
\scnitem{модификацию (изменение параметров модели в зависимости от цели приложения)}
\scnitem{синтез (формирование нового сигнала из измененных параметров модели)}
\end{scnrelfromset}

\begin{scnrelfromset}{Для решения многих современных прикладных задач требуется не только наличие возможности описания речевого
сигнала или процесса фонации, но и использование высокоуровневых речевых характеристик, определяющих
персональный голос диктора, экспрессию, фонетику и так далее. К таким задачам относятся}
\scnitem{конверсия голоса}
\scnitem{синтез речи по тексту}
\scnitem{верификация диктора}
\end{scnrelfromset}

\scntext{примечание}{Высокоуровневое моделирование речи является
очень сложной предметной областью, поскольку требует использования интеллектуальных моделей и методов
машинного обучения. На данный момент не существует единого универсального способа, применяемого для
разных приложений.}


\scnheader{Параметрическая модель сигнала}
\scntext{определение}{математическое выражение, используемое для представления отсчетов сигнала во временной или частотной
области}
\scnsuperset{параметрическая модель речевого сигнала}
\begin{scnindent}
\scntext{определение}{математическое описание характеристик сигнала, являющихся специфическими для речи и связанных с
процессом фонации (таких как частота основного тона, последовательность возбуждения и огибающая
амплитудного спектра)}
\end{scnindent}

\scntext{примечание}{к основным моделям речевого сигнала относят: модели на основе линейного предсказания; на основе
кепстрального представления; синусоидальные и гибридные модели. Среди гибридных моделей наиболее
известна модель гармоники+шум}




\scnheader{Заключение}
\scntext{примечание}{в главе изложены идеи, лежащие в основе оригинального подхода к проектированию аудиоинтерфейсов интеллектуальных компьютерных систем на основе онтологического проектирования и формализации системы понятий
из соответствующей предметной области, с использованием Технологии OSTIS. Изложены основные принципы
лежащие в основе данного подхода, а также их отличительные особенности от общепринятых}

\begin{scnrelfromset}{К ограничениям предлагаемого подхода можно отнести следующие основные факторы}
\scnfileitem{для реализации задач по формализации любой предметной области, в том числе и
аудиоинтерфейсов, требуется в первую очередь большое количество источников знаний для их пополнения. Для
преодоления данной проблемы требуется привлечение большого количества экспертов, обладающих соответствующими компетенциями и знаниями в предметной области, либо же разработка механизмов надежного автоматического извлечения этих знаний из имеющихся источников}
\scnfileitem{Прямой доступ к знаниям экспертов весьма ограничен, поскольку это требует значимых усилий по подбору репрезентативной выборки таких экспертов, выстраиванию эффективных и интероперабельных взаимоотношений между
сторонами процесса, что зачастую зависит от большого количества субъективных факторов, — соответственно,
требует большого количества временных и материальных ресурсов}

\scnfileitem{Существенное количество информации, накопленной человечеством, хранится в виде текстов на
естественных языках. Процесс извлечения данной информации и представления ее в формализованном виде — в
виде знаний, также выглядит нетривиальным}
\end{scnrelfromset}


\begin{scnrelfromset}{Исходя из природы данных проблем, по мнению авторов, видятся следующие основные направления их преодоления и, как следствие, две основные стратегии развития предложенного подхода}
\scnfileitem{создание для экспертов, работающих в домене аудио и речевых интерфейсов, специализированных инструментальных средств по формализации и представлению знаний из данной предметной области, фиксации их в
виде стандартов единой формы. Подобные инструменты должны обладать качественно новыми функциональными возможностями, обеспечивающими высокий уровень совместимости и интероперабельности в процессе
накопления и стандартизации знаний, чтобы сами эксперты были заинтересованы в применении и широком
распространении данной технологии для представления знаний. Данный пункт является одной из ключевых
задач Технологии OSTIS и Стандарта OSTIS}
\scnfileitem{Создание автоматизированных и автоматических средств извлечения знаний из существующих источников
информации, в первую очередь текстов на естественном языке. К видам документов, в которых содержится
уже структурированная и отчасти формализованная информация, относятся в первую очередь: стандарты,
протоколы, рекомендации (RFC), инструкции и так далее. Следовательно процесс автоматизации извлечения
знаний должен быть направлен в первую очередь на формализацию уже существующих отраслевых стандартов
разработки аудиоинтерфейсов, систем обработки и кодирования аудиоинформации, систем обработки речевых
сигналов, таких как стандарты серии ISO, IEEE и AES (Audio Engineering Society): ISOIECITCoAVO-2005art,
ISOIECITMAT-2020art, IEEESfSGAC-2020el, IECAVaRE-2015el, AEST3250-2004art}
\end{scnrelfromset}
\begin{scnindent}
\scntext{примечание}{Реализация подхода, предложенного в данной главе, позволит обеспечить свойства унификации, семантической
совместимости и интероперабельности, при разработке аудио и речевых интерфейсов (своеобразный аналог Модели OSI/ISO в области проектирования интерфейсов интеллектуальных компьютерных систем), что в итоге
позволит существенным образом сократить издержки при создании интеллектуальных компьютерных систем нового поколения для решении сложных комплексных задач}
\end{scnindent}




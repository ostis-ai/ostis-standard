\begin{SCn}
		\scntext{аннотация}{В данной главе рассматривается подход к реализации естественно-языковых интерфейсов ostis-систем, построенных по Технологии OSTIS, а также предлагается модель контекста диалога. В данном подходе все этапы анализа, включая лексический, синтаксический и семантический анализ могут производиться непосредственно в базе знаний такой системы. Такой подход позволит эффективно решать такие задачи как управление глобальным и локальным контекстами диалога, а также разрешение языковых явлений таких как анафоры, омонимия и эллиптические фразы.}
		\begin{scnrelfromlist}{подраздел}
			\scnitem{§ 4.2.1. Синтаксический анализ естественно-языковых сообщений, входящих в ostis-систему}
			\scnitem{§ 4.2.2. Понимание естественно-языковых сообщений, входящих в ostis-систему}
			\scnitem{§ 4.2.3. Методика и средства разработки естественно-языковых интерфейсов}
		\end{scnrelfromlist}
		\begin{scnrelfromlist}{ключевое знак}
			\scnitem{Абстрактный sc-агент лексического анализа}
			\scnitem{Абстрактный sc-агент синтаксического анализа}
			\scnitem{Абстрактный sc-агент понимания сообщения}
		\end{scnrelfromlist}
		\begin{scnrelfromlist}{ключевое понятие}   
			\scnitem{естественно-языковой интерфейс}
			\scnitem{речевой интерфейс}
			\scnitem{действие. лексический анализ естественно-языкового сообщения}
			\scnitem{действие. синтаксический анализ естественно-языкового сообщения}
			\scnitem{действие. понимание естественно-языкового сообщения}
			\scnitem{контекст}
			\scnitem{контекст диалога}
		\end{scnrelfromlist}
		\begin{scnrelfromlist}{ключевое отношение}
			\scnitem{потенциально эквивалентная структура*}
			\scnitem{множество тематических контекстов диалога*}
		\end{scnrelfromlist}
		\begin{scnrelfromlist}{ключевое знание}
			\scnitem{Структура решателя задач естественно-языкового интерфейса ostis-систем}
			\scnitem{Синтаксический анализ естественно-языкового сообщения}
			\scnitem{Понимание естественно-языкового сообщения}
			\scnitem{Разрешение контекста}
			\scnitem{Принципы построения естественно-языкового интерфейса ostis-систем для китайского языка}
		\end{scnrelfromlist}
		\begin{scnrelfromlist}{библиографическая ссылка}
			\scnitem{GlobaVAMbT-2019el}
			\scnitem{Pais S..NLPBPaaSaBR-2022art}
			\scnitem{Trajanov ..Surve oNLPiPMT-2022art}
			\scnitem{Khurana D..NaturLPSotA-2022art}
			\scnitem{Strubell E..Energ aPCfDL-2019art}
			\scnitem{LargeLMaNML-2021el}
			\scnitem{AmazoAOSWiA-2022el}
			\scnitem{Siri-2022el}
			\scnitem{GooglAYOP-2022el}
			\scnitem{Cortana-2022el}
			\scnitem{Hoy M.B.AlexaSCamItVA-2018art}
			\scnitem{Jackendoff R..XS aSoPS-1977bk}
			\scnitem{Davydenko I.T.SemanMMaToKB-2018art}
			\scnitem{Shunkevich.D.V.AgentMMaToC-2018art}
			\scnitem{Давыденко И.Т.МоделМиСРГБ-2017дс}
		\end{scnrelfromlist}
		\scnheader{большое количество различных интерфейсов компьютерных систем}
		\begin{scnrelfromlist}{усложнение}
			\scnitem{интероперабельность между компьютерными системами и людьми.}
		\end{scnrelfromlist}
		\newpage
		\scnheader{ostis-система}
		\begin{scnrelfromlist}{основная особенность}
			\scnitem{пользовательский интерфейс}
			\begin{scnindent}
				\scntext{примечание}{пользовательский интерфейс ostis-sistem должен быть способен обеспечить эффективное взаимодействие пользователя с системой в условиях его общей профессиональной неподготовленности.}
			\end{scnindent}
		\end{scnrelfromlist}
		\scnheader{речь}
		\scnidtf{одна из наиболее естественных и удобных форм передачи информации между людьми.}
		\scnidtf{форма взаимодействия человека и машины.}
		\begin{scnrelfromlist}{важная роль в}
			\scnitem{взаимодействие с различными компьютерными системами.}
		\end{scnrelfromlist}
		\scnheader{Разработка естественно-языковых интерфейсов для современных интеллектуальных систем, основанных на знаниях.}
		\begin{scnrelfromset}{аспекты}
			\scnitem{особенности обрабатываемого естественного языка.}
			\scnitem{диапазон базы знаний интеллектуальных систем, то есть широта знаний в базах знаний интеллектуальных систем.}
		\end{scnrelfromset}
		\begin{scnrelfromset}{методы обработки естественного языка.}
			\scnitem{методы на основе правил и лингвистических знаний.}
			\scnitem{методы машинного обучения, основанные на математической статистике и теории информации.}
		\end{scnrelfromset}
		\begin{scnrelfromlist}{основа многих подходов}
			\scnitem{машинное обучение}
			\begin{scnindent}
				\begin{scnrelfromset}{недостатки}
					\scnitem{проблемы при работе с различными областями.}
					\scnitem{наличиe большого объема данных.}
					\scnitem{модели представляют собой "черный ящик".}
					\scnitem{модель решает только свой узкий класс задач.}
				\end{scnrelfromset}
			\end{scnindent}
		\end{scnrelfromlist}
		\scntext{примечание}{Данные недостатки используемых методов являются причиной части недостатков современных систем, реализующих естественно-языковой интерфейс, так, несмотря на то, что сейчас существует большое количество речевых ассистентов, создаваемых разными компаниями (см. AmazoAOSWiA-2022el, Siri-2022el,GooglAYOP-2022el, Cortana-2022el). Они обладают схожими недостатками, например, исключительно распределенной реализацией, в силу недостаточной для запуска ресурсоемких моделей производительности устройств конечных пользователей. Это в свою очередь ведет к проблемам с приватностью (см. Hoy M.B.AlexaSCamItVA-2018art).}
		\scnheader{несовместимость способов представления результатов}
		\begin{scnrelfromlist}{причина}
			\scnitem{разные форматы представления результатов.}
		\end{scnrelfromlist}
		\begin{scnindent}
			\scntext{пояснение}{Для представления результатов промежуточных этапов обработки используются иные форматы, модули которые их реализуют не имеют какой-либо единой основы и взаимодействуют посредством специализированных программных интерфейсов между ними, что приводит к несовместимости способов представления результатов на различных этапах обработки и конечного результата обработки текстов.}
		\end{scnindent}
		\begin{scnrelfromlist}{решение}
			\scnitem{использование естественного языка на основе его формальной модели в виде набора онтологий.}
			\begin{scnindent}
				\scntext{пояснение}{В качестве решения проблемы совместимости предлагается использование подхода к обработке естественного языка на основе его формальной модели в виде набора онтологий, сформированных с использованием универсальных средств представления знаний, что будет способствовать интероперабельности как компонента по обработке естественного языка в целом с другими компонентами системы, так и между составляющими самого данного компонента.}
			\end{scnindent}
		\end{scnrelfromlist}
		\scnheader{естественно-языковой интерфейс}
		\scnsuperset{речевой интерфейс}
		\scnheader{речевой интерфейс}
		\scnidtf{SILK-интерфейс, обмен информацией в котором происходит за счет диалога, в процессе которого компьютерная система и пользователь общаются с помощью речи.}
		\scnidtf{Вид интерфейса наиболее приближен к естественному общению между людьми.}
		\scnheader{естественный язык}
		\begin{scnrelfromset}{этапы обработки}
			\scnitem{лексический анализ}
			\begin{scnindent}
				\begin{scnrelfromset}{включение}
					\scnitem{декомпозиция текста на токены.}
					\scnitem{сопоставление токенов с лексемами.}
				\end{scnrelfromset}
			\end{scnindent}
			\scnitem{синтаксический анализ.}
			\scnitem{понимание сообщения.}
		\end{scnrelfromset}
		\scnheader{Решатель задач естественно-языкового интерфейса}
		\begin{scnrelfromset}{декомпозиция абстрактного sc-агента}
			\scnitem{Абстрактный sc-агент лексического анализа}
			\begin{scnindent}
				\begin{scnrelfromset}{декомпозиция абстрактного sc-агента}
					\scnitem{Абстрактный sc-агент декомпозиции текста на токены}
					\scnitem{Абстрактный sc-агент сопоставления токенов с лексемами}
				\end{scnrelfromset}
			\end{scnindent}
			\scnitem{Абстрактный sc-агент синтаксического анализа}
			\scnitem{Абстрактный sc-агент понимания сообщения}
			\begin{scnindent}
				\begin{scnrelfromset}{декомпозиция абстрактного sc-агента}
					\scnitem{Абстрактный sc-агент генерации вариантов значения сообщения}
					\scnitem{Абстрактный sc-агент выбора и обновления контекста}
					\begin{scnindent}
						\begin{scnrelfromset}{декомпозиция абстрактного sc-агента}
							\scnitem{Абстрактный sc-агент разрешения контекста}
							\scnitem{Абстрактный sc-агент выбора смысла сообщения на основе контекста}
							\scnitem{Абстрактный sc-агент погружения сообщения в контекст}
						\end{scnrelfromset}
					\end{scnindent}
				\end{scnrelfromset}
			\end{scnindent}
		\end{scnrelfromset}
		\scnheader{действие. лексический анализ естественно-языкового сообщения}
		\begin{scnrelfromset}{обобщенная декомпозиция}
			\scnitem{действие. декомпозиция текста на токены}
			\scnitem{действие. сопоставление токенов с лексемами}
		\end{scnrelfromset}
		\scnheader{естественно-языковой текст}
		\begin{scnrelfromlist}{с точки зрения ostis-системы}
			\scnitem{файл}
		\end{scnrelfromlist}
		\scnheader{Лексический анализ}
		\scnidtf{декомпозиция текста на последовательность токенов и сопоставление лексем с получившимися при данной декомпозиции токенами.}
		\scntext{примечание}{данные токены при необходимости могут сопоставляться не с лексемами, а с их подмножествами, входящими в ее морфологическую парадигму, соответствующими определенным грамматическим категориям: падежу, числу, роду и так далее.}
		\begin{scnrelfromlist}{помощь в осуществлении}
			\scnitem{словарь, содержащий лексемы и их различные формы.}
		\end{scnrelfromlist}
		\scnheader{лексема}
		\scnidtf{единица словарного состава языка, которая представляет собой множество всех форм некоторого слова}
		\scnidtf{носитель морфологической парадигмы и ключевой элемент анализа}
		\scnheader{единица сегментации}
		\scntext{определение}{базовая единица для обработки китайского языка, имеющая определенные семантические или грамматические свойства}
		\scnsubset{файл}
		\scnheader{Китайский язык}
		\scntext{примечание}{В китайском языке отсутствуют четкие показатели категорий числа, падежа и рода, в отличие от русского языка и других европейских языков. Функцию слова в китайском языке можно определить не на основании морфемного состава, а при помощи анализа связей этого слова с другими словами. В связи с этим, в процессе анализа текстов китайского языка сначала необходимо выполнить лексический анализ, разбивающий поток иероглифов в тексте китайского языка на отдельные значимые единицы сегментации. При этом из-за невозможности разрешения структурной неоднозначности на этапе синтаксического анализа, результатом работы агента синтаксического анализа в общем случае будет являться множество потенциальных синтаксических структур}
		\scnheader{действие. понимание естественно-языкового сообщения}
		\begin{scnrelfromset}{обобщенная декомпозиция}
			\scnitem{действие. генерация вариантов значения сообщения}
			\scnitem{действие. выбор и обновление контекста}
			\begin{scnindent}
				\begin{scnrelfromset}{обобщенная декомпозиция}
					\scnitem{действие. разрешение контекста}
					\scnitem{действие. выбор смысла сообщения на основе контекста}
					\scnitem{действие. погружение сообщения в контекст}
				\end{scnrelfromset}
			\end{scnindent}
		\end{scnrelfromset}
		\scnheader{действие. генерация вариантов значения сообщения}
		\scnidtf{действие, в ходе которого осуществляется формирование строгой дизъюнкции потенциально эквивалентных структур.}
		\scnheader{потенциально эквивалентная структура*}.
		\scnidtf{бинарное ориентированное отношение, связывающее структуру и множество структур, которые потенциально могут быть эквивалентны ей, однако для достоверного определения факта требуются дополнительные действия.}
		\scnheader{контекст}
		\scnidtf{sc-структура, содержащая знания, которыми оперирует система в ходе одного или нескольких диалогов.}
		\scnheader{контекст диалога}
		\scnsubset{контекст}
		\begin{scnrelfromset}{разбиение}
			\begin{scnindent}
				\begin{scnrelfromset}{Типология контекстов диалога по глобальности}
					\scnitem{тематический контекст}
					\scnitem{пользовательский контекст}
					\scnitem{глобальный контекст}
				\end{scnrelfromset}
			\end{scnindent}
		\end{scnrelfromset}
		\scnheader{тематический контекст}
		\scnidtf{контекст диалога, содержащий специфические для темы сведения}
		\scntext{примечание}{специфическими сведениями являются сведения, полученные во время ведения диалога, на определенную тематику, например, при диалоге об определенном наборе сущностей.}
		\scnheader{множество тематических контекстов диалога*}
		\scnidtf{бинарное ориентированное отношение, диалог с ориентированным множеством его тематических контекстов.}
		\scnheader{пользовательский контекст}
		\scnidtf{контекст диалога, содержащие специфические для пользователя сведения, которые могут быть использованы в диалоге с ним на любую тематику.}
		\scntext{примечание}{В общем случае пользовательский контекст имеет пересечение с согласованной частью базы знаний (предварительно занесенная в базу знаний достоверная информация о пользователе, прошедшая необходимую модерацию), но не включается в нее целиком (часть, полученная в ходе диалога в которой мы не уверены).}
		\scnheader{глобальный контекст}
		\scnidtf{контекст диалога, содержащий сведения, которые могут быть необходимы при ведении диалога с любым пользователем.}
		\scnidtf{подмножество согласованной части базы знаний, содержащее те сведения, что допустимо использовать в диалоге.}
		\begin{scnrelfromlist}{пример}
			\scnitem{диалог с определенным пользователем}
			\begin{scnindent}
				\begin{scnrelfromset}{запрет на использование}
					\scnitem{служебная информация}
					\scnitem{части пользовательских контекстов иных пользователей}
				\end{scnrelfromset}
			\end{scnindent}
		\end{scnrelfromlist}
		\scnheader{контекст диалога}
		\begin{scnrelfromset}{разбиение}
			\begin{scnindent}
				\begin{scnrelfromset}{Типология контекство по сроку достоверности знаний}
					\scnitem{неизменяемый в ходе работы системы контекст диалога}
					\begin{scnindent}
						\scntext{примечание}{содержит в себе знания, необходимые для обеспечения выполнения системой своих функций, которые были заложены в нее априорно ее разработчиками и/или администраторами и не изменяются в ходе ее функционирования на постоянной основе.}
					\end{scnindent}
					\scnitem{изменяемый в ходе работы системы контекст диалога}
					\begin{scnrelfromset}{разбиние}
						\begin{scnindent}
							\begin{scnrelfromset}{Типология изменяемых в ходе работы системы контекстов по источнику знаний}
								\scnitem{контекст диалога, содержащий знания из внешних источников}
								\scnitem{контекст диалога, содержащий знания, полученные в ходе диалога}
							\end{scnrelfromset}
						\end{scnindent}
					\end{scnrelfromset}
					\begin{scnrelfromset}{разбиние}
						\begin{scnindent}
							\begin{scnrelfromset}{Типология изменяемых контекстов по степени их достоверности}
								\scnitem{достоверный контекст диалога}
								\scnitem{недостоверный контекст диалога}
							\end{scnrelfromset}
						\end{scnindent}
					\end{scnrelfromset}
				\end{scnrelfromset}
			\end{scnindent}
		\end{scnrelfromset}
		\newpage
		\scnheader{действие. разрешение контекста}
		\begin{scnrelfromlist}{сведение}
			\scnitem{сопоставлению каждому варианту его значения соответствующего контекста}
		\end{scnrelfromlist}
		\begin{scnrelfromlist}{основание выбора}
			\scnitem{значение функции Fctd(T,C)}
			\begin{scnindent}
				\begin{scnrelfromset}{пояснение}
					\scnitem{T — вариант трансляции}
					\scnitem{C — тематический контекст}
				\end{scnrelfromset}
			\end{scnindent}
		\end{scnrelfromlist}
		\scntext{примечание}{Подходящим контекстом для варианта трансляции считается тот, для которого значение этой функции максимально. В случае, если подходящий контекст не найден, генерируется новый. Пример результата данного действия представлен на SCg-текст.}
		\scnheader{действие. погружение сообщения в контекст}
		\scnidtf{погружение полученного смысла сообщения в контекст}
		\scntext{примечание}{Кроме выбранного смысла сообщения, в контекст может добавляться и иная необходимая для обработки сообщения информация. Кроме того, на данном этапе на основе хранящихся в контексте сведений также должно выполняться разрешение местоимений}
		\scnheader{методика разработки естественно-языковых интерфейсов}
		\begin{scnrelfromlist}{этапы}
			\scnitem{Этап 1. Формирование требований и особенностей конкретного естественного языка}
			\begin{scnindent}
				\begin{scnrelfromset}{задачи}
					\scnitem{рассмотрение особенностей конкретного естественного языка}
					\scnitem{разработка базы знаний по обработке конкретного естественного языка и соответствующих решателей задач для выполнения обработки}
				\end{scnrelfromset}
				\scntext{примечание}{После определения конкретного естественного языка существует вероятность того, что в составе библиотеки компонентов уже есть реализованный вариант требуемой базы знаний и соответствующих решателей. В противном случае, тем не менее, у разработчика появляется возможность включить разработанную базу знаний по обработке конкретного естественного языка и соответствующие решатели задач в библиотеку компонентов для последующего использования.}
			\end{scnindent}
			\scnitem{Этап 2. Разработка базы знаний по обработке конкретного естественного языка}
			\begin{scnindent}
				\scntext{уточнение}{На данном этапе при разработке базы знаний используются общие принципы согласованного построения и модификации гибридных баз знаний.}
			\end{scnindent}
			\scnitem{Этап 3. Разработка решателя задач естественно-языкового интерфейса}
			\begin{scnindent}
				\begin{scnrelfromset}{направления решателей}
					\scnitem{приобретение фактографических знаний}
					\scnitem{генерация текстов конкретного естественного языка}
				\end{scnrelfromset}
				\scntext{уточнение}{На данном этапе используются общие принципы согласованного построения и модификации гибридных решателей задач}
			\end{scnindent}
			\scnitem{Этап 4. Верификация разработанных каждых компонентов}
			\begin{scnindent}
				\scntext{уточнение}{На данном этапе выполняется верификация разработанных компонентов (базы знаний по обработке конкретного естественного языка и соответствующего решателя задач) конкретного естественно-языкового интерфейса.}
			\end{scnindent}
			\scnitem{Этап 5. Отладка разработанных каждых компонентов. Исправление ошибок}
		\end{scnrelfromlist}
		\scntext{примечение}{Как правило, этапы 4 и 5 могут выполняться циклически до тех пор, пока разработанные компоненты не будут соответствовать предъявляемым требованиям}
		\scnheader{Библиотека многократно используемых компонентов}
		\scnidtf{важнейшее понятием в рамках Технологии OSTIS} 
		\begin{scnrelfromlist}{обеспечивает возможность}
			\scnitem{выбрать уже разработанные компоненты и включить их в разрабатываемый естественно-языковой интерфейс других ostis-систем.}
		\end{scnrelfromlist}
		\scntext{примечание}{разработанные компоненты естественно-языковых интерфейсов могут быть повторно использованы при разработке естественно-языковых интерфейсов в других ostis-системах.}
		\begin{scnrelfromset}{разбиение}
			\scnitem{Библиотека многократно используемых компонентов базы знаний естественно-языковых интерфейсов}
			\begin{scnindent}
				\scnidtf{Библиотека многократно используемых компонентов лингвистической базы знаний}
				\scnidtf{Библиотека многократно используемых компонентов базы знаний по обработке естественного языка}
			\end{scnindent}
			\scnitem{Библиотека многократно используемых компонентов решателей задач естественно-языковых интерфейсов}
			\begin{scnindent}
				\scnidtf{Библиотека многократно используемых компонентов решателей задач для обработки естественного языка}
			\end{scnindent}
		\end{scnrelfromset}
		\begin{scnrelfromlist}{преставляет возможность}
			\scnitem{дополнение знаниями о конкретных естественных языках}
		\end{scnrelfromlist}
		\scnheader{Библиотека многократно используемых компонентов китайско-языкового интерфейса}
		\begin{scnrelfromset}{разбиение}
			\scnitem{Библиотека многократно используемых компонентов базы знаний китайско-языкового интерфейса}
			\begin{scnindent}
				\scnidtf{Библиотека многократно используемых компонентов базы знаний по обработке китайского языка}
			\end{scnindent}
			\scnitem{Библиотека многократно используемых компонентов решателей задач китайско-языкового интерфейса}
			\begin{scnindent}
				\scnidtf{Библиотека многократно используемых компонентов решателей задач для обработки китайского языка}
			\end{scnindent}
		\end{scnrelfromset}
		\newpage
\end{SCn}
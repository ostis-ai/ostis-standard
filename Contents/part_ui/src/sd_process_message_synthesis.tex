\begin{SCn}
    \scnsectionheader{Предметная область и онтология синтеза естественно-языковых сообщений ostis-системы}
    \begin{scnsubstruct}
    \begin{scnrelfromlist}{соавтор}
        \scnitem{Никифоров С.А.}
        \scnitem{Гойло А.А.}
        \scnitem{Цянь Л.}
    \end{scnrelfromlist}

    \scnheader{Предметная область синтеза естественно-языковых сообщений ostis-системы}
    \scnhaselementrole{максимальный класс объектов исследования}{методика разработки естественно-языковых интерфейсов}
    \begin{scnhaselementrolelist}{класс объектов исследования}
        \scnitem{методика разработки естественно-языковых интерфейсов}
        \scnitem{Библиотека многократно используемых компонентов естественно-языковых интерфейсов}
        \scnitem{Библиотека многократно используемых компонентов китайско-языкового интерфейса}
    \end{scnhaselementrolelist}
    
    \scnheader{методика разработки естественно-языковых интерфейсов}
    \scntext{примечание}{Методика разработки естественно-языковых интерфейсов включает несколько этапов, в которых необходимо учитывать методику построения и модификации гибридных баз знаний и гибридных решателей задач.}
    \begin{scnindent}
        \begin{scnrelfromlist}{источник}
            \scnitem{\scncite{Davydenko2018}}
            \scnitem{\scncite{Shunkevich2018}}
        \end{scnrelfromlist}
    \end{scnindent}
    \scnrelfrom{иллюстрация}{\scnfileimage[40em]{Contents/part_ui/src/images/sd_ui/method.png}}
    \begin{scnindent}
        \scnidtf{Рисунок. Этапы процесса разработки естественно-языкового интерфейса}
    \end{scnindent}
    \scntext{примечание}{Данная методика может быть применена при разработке конкретного естественно-языкового интерфейса по конкретной предметной области.}
    \begin{scnrelfromvector}{этапы}
        \scnitem{Этап 1. Формирование требований с учетом особенностей конкретного естественного языка}
        \begin{scnindent}
            \scntext{пояснение}{На данном этапе необходимо четко рассматривать особенности конкретного \textit{естественного языка}. Затем можно разработать базу знаний по обработке конкретного \textit{естественного языка} и соответствующие \textit{решатели задач} для выполнения обработки. После определения конкретного \textit{естественного языка} существует вероятность того, что в составе библиотеки компонентов уже есть реализованный вариант требуемой базы знаний и соответствующих решателей. В противном случае, тем не менее, у разработчика появляется возможность включить разработанную \textit{базу знаний} по обработке конкретного естественного языка и соответствующие \textit{решатели задач} в \textit{библиотеку компонентов} для последующего использования.}
        \end{scnindent}
        \scnitem{Этап 2. Разработка базы знаний по обработке конкретного естественного языка}
        \begin{scnindent}
            \scntext{пояснение}{На данном этапе при разработке базы знаний используются общие принципы согласованного построения и модификации \textit{гибридных баз знаний}.}
            \begin{scnindent}
                \scnrelfrom{источник}{\scncite{Davydenko2017}}
            \end{scnindent}
        \end{scnindent}
        \scnitem{Этап 3. Разработка решателей задач естественно-языковых интерфейсов}
        \begin{scnindent}
            \scntext{пояснение}{Для разработки \textit{решателей задач} \textit{естественно-языкового интерфейса}, направленных на приобретение фактографических знаний и генерацию текстов конкретного \textit{естественного языка}, используются общие принципы согласованного построения и модификации гибридных \textit{решателей задач}.}
            \begin{scnindent}
                \scnrelfrom{источник}{\scncite{Shunkevich2018}}
            \end{scnindent}
        \end{scnindent}
        \scnitem{Этап 4. Верификация разработанных компонентов}
        \begin{scnindent}
            \scntext{пояснение}{На данном этапе выполняется верификация разработанных \textit{компонентов} (\textit{базы знаний} по обработке конкретного естественного языка и соответствующего \textit{решателя задач}) конкретного \textit{естественно-языкового интерфейса}.}
        \end{scnindent}
        \scnitem{Этап 5. Отладка разработанных компонентов. Исправление ошибок}
    \end{scnrelfromvector}
    \begin{scnindent}
        \scntext{примечание}{Как правило, этапы 4 и 5 могут выполняться циклически до тех пор, пока разработанные компоненты не будут соответствовать предъявляемым требованиям.}
    \end{scnindent}

    \scnheader{Библиотека многократно используемых компонентов естественно-языковых интерфейсов}
	\begin{scnrelfromset}{разбиение}
		\scnitem{Библиотека многократно используемых компонентов базы знаний естественно-языковых интерфейсов}
		\begin{scnindent}
			\scnidtf{Библиотека многократно используемых компонентов лингвистической базы знаний}
			\scnidtf{Библиотека многократно используемых компонентов базы знаний по обработке естественного языка}
		\end{scnindent}
		\scnitem{Библиотека многократно используемых компонентов решателей задач естественно-языковых интерфейсов}
		\begin{scnindent}
			\scnidtf{Библиотека многократно используемых компонентов решателей задач для обработки естественного языка}
		\end{scnindent}
	\end{scnrelfromset}
    \scntext{примечание}{\textit{Библиотека многократно используемых компонентов} является важнейшим понятием в рамках \textit{Технологии OSTIS}. Библиотека многократно используемых компонентов естественно-языковых интерфейсов позволяет выбрать компоненты уже разработанные компоненты и включить их в разрабатываемый \textit{естественно-языковой интерфейс} других \textit{ostis-систем}, то есть разработанные компоненты \textit{естественно-языковых интерфейсов} могут быть повторно использованы при разработке естественно-языковых интерфейсов в других \textit{ostis-системах}. Ниже предложена структура библиотеки многократно используемых компонентов \textit{естественно-языковых интерфейсов}. При необходимости, библиотеку многократно используемых компонентов естественно-языковых интерфейсов можно дополнять знаниями о конкретных естественных языках.}
    \begin{scnindent}
        \scnrelfrom{источник}{Предметная область и онтология комплексной библиотеки многократно используемых семантически совместимых компонентов ostis-систем}
    \end{scnindent}

    \scnheader{Библиотека многократно используемых компонентов китайско-языкового интерфейса}
	\begin{scnrelfromset}{разбиение}
		\scnitem{Библиотека многократно используемых компонентов базы знаний китайско-языкового интерфейса}
		\begin{scnindent}
			\scnidtf{Библиотека многократно используемых компонентов базы знаний по обработке китайского языка}
		\end{scnindent}
		\scnitem{Библиотека многократно используемых компонентов решателей задач китайско-языкового интерфейса}
		\begin{scnindent}
			\scnidtf{Библиотека многократно используемых компонентов решателей задач для обработки китайского языка}
		\end{scnindent}
	\end{scnrelfromset}

    \end{scnsubstruct}
\end{SCn}
    

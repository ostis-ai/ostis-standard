\begin{SCn}
    \scnsectionheader{Предметная область и онтология понимания естественно-языковых сообщений, входящих в ostis-систему}
    \begin{scnsubstruct}
    \begin{scnrelfromlist}{соавтор}
        \scnitem{Никифоров С.А.}
        \scnitem{Гойло А.А.}
        \scnitem{Цянь Л.}
    \end{scnrelfromlist}

    \scnheader{Предметная область понимания естественно-языковых сообщений, входящих в ostis-систему}
    \scnhaselementrole{максимальный класс объектов исследования}{действие. понимание естественно-языкового сообщения}
    \begin{scnhaselementrolelist}{класс объектов исследования}
        \scnitem{действие. разрешение контекста}
        \scnitem{действие. выбор смысла сообщения}
        \scnitem{действие. погружение сообщения в контекс}
        \scnitem{действие. переход от результата синтаксического анализа к потенциально эквивалентным сообщению структурам}
        \scnitem{смысл сообщении}
        \scnitem{контекст}
        \scnitem{контекст диалога}
        \scnitem{тематический контекст}
        \scnitem{пользовательский контекст}
        \scnitem{глобальный контекст}
        \scnitem{неизменяемый в ходе работы системы контекст диалога}
        \scnitem{изменяемый в ходе работы системы контекст диалога}
    \end{scnhaselementrolelist}

    \begin{scnhaselementrolelist}{исследуемое отношение}
        \scnitem{потенциально эквивалентная структура*}
        \scnitem{множество тематических контекстов диалога*}
    \end{scnhaselementrolelist}
    
    \scnheader{действие. понимание естественно-языкового сообщения}
    \begin{scnrelfromset}{обобщенная декомпозиция}
        \scnitem{действие. генерация вариантов значения сообщения}
        \begin{scnindent}
            \scntext{определение}{\textbf{\textit{действие. генерация вариантов значения сообщения}} --- \textit{действие}, в ходе которого осуществляется формирование \textit{строгой дизъюнкции} потенциально эквивалентных структур}
        \end{scnindent}
        \scnitem{действие. выбор и обновление контекста}
        \begin{scnindent}
            \begin{scnrelfromset}{обобщенная декомпозиция}
                \scnitem{действие. разрешение контекста}
                \scnitem{действие. выбор смысла сообщения на основе контекста}
                \scnitem{действие. погружение сообщения в контекст}
            \end{scnrelfromset}
        \end{scnindent}
    \end{scnrelfromset}

    \scnheader{потенциально эквивалентная структура*}
    \scntext{определение}{\textbf{\textit{потенциально эквивалентная структура*}} --- \textit{бинарное ориентированное отношение}, связывающее структуру и множество структур, которые потенциально могут быть эквивалентны ей, однако для достоверного определения факта требуются дополнительные \textit{действия}}
      
    \scnheader{правила перехода от результата синтаксического анализа к потенциально эквивалентным сообщению структурам}
    \scnrelfrom{источник}{Предметная область и онтология информационных конструкций и языков}
    \scnrelfrom{пример}{\scnfileimage[40em]{Contents/part_ui/src/images/sd_ui/transition_to_semanic_rule.png}}
    \begin{scnindent}
        \scnidtf{SCg-текст. Иллюстрация правила перехода от синтаксической структуры к семантике}
    \end{scnindent}
    
    \scnheader{действие. переход от результата синтаксического анализа к потенциально эквивалентным сообщению структурам}
    \begin{scnrelfromvector}{примечание}
        \scnfileitem{В результате действия перехода от результата синтаксического анализа к потенциально эквивалентным сообщению структурам в базе знаний формируется структура, описывающая возможные варианты смысла сообщения. Наличие нескольких таких структур объясняется тем, что в общем случае на этапе синтаксического анализа выполняется генерация нескольких вариантов синтаксической структуры. Выбор корректного значения сообщения будет осуществлен в ходе выполнения последующих действий.}
        \begin{scnindent}
            \scnrelfrom{источник}{\scncite{Jackendoff1977}}
            \scnrelfrom{смотрите}{SCg-текст. Иллюстрация конструкции, описывающей потенциальные смыслы сообщения}
        \end{scnindent}
        \scnfileitem{Следует отметить, что при необходимости смысл сообщения может быть сгенерирован не только на основании его синтаксической структуры в терминах грамматики составляющих, но и других знаний о данном сообщении, например выделенных из текста данного сообщения троек вида субъект-отношение-объект, результата его классификации и тому подобные.}
        \scnfileitem{Дальнейшие этапы процесса понимания сообщения выполняются на основе контекста.}
    \end{scnrelfromvector}

    \scnheader{смысл сообщении}
    \scnrelfrom{пример}{SCg-текст. Иллюстрация конструкции, описывающей потенциальные смыслы сообщения}
    
    \scnheader{SCg-текст. Иллюстрация конструкции, описывающей потенциальные смыслы сообщения}
	\scnrelfrom{иллюстрация}{\scnfileimage[40em]{Contents/part_ui/src/images/sd_ui/messsage_meaning_variants.png}}
    
    \scnheader{контекст диалога}
    \scnsubset{контекст}
    \begin{scnindent}
        \scntext{определение}{\textbf{\textit{контекст}} --- \textit{sc-структура}, содержащая знания, которыми оперирует система в ходе одного или нескольких диалогов}
        \begin{scnindent}
            \scntext{пояснение}{В общем случае, данные знания включают в себя как предварительно занесенные в \textit{базу знаний}, так и полученные в ходе работы с сенсоров и/или диалога.}
        \end{scnindent}
    \end{scnindent}
    \scnrelfrom{разбиение}{\scnkeyword{Типология контекстов диалога по глобальности\scnsupergroupsign}}
    \begin{scnindent}
        \begin{scneqtoset}
            \scnitem{тематический контекст}
            \scnitem{пользовательский контекст}
            \scnitem{глобальный контекст}
        \end{scneqtoset}
    \end{scnindent}
    \scnrelfrom{разбиение}{\scnkeyword{Типология контекство по сроку достоверности знаний\scnsupergroupsign}}
    \begin{scnindent}
        \begin{scneqtoset}
            \scnitem{неизменяемый в ходе работы системы контекст диалога}
            \scnitem{изменяемый в ходе работы системы контекст диалога}
        \end{scneqtoset}
    \end{scnindent}
    \scntext{примечание}{Подмножество \textit{контекста} может включаться в согласованную часть \textit{базы знаний}, например, если речь идет о каких-то предварительно занесенных в \textit{базу знаний} биографических сведениях --- дате рождения и тому подобном.}
    \scntext{прммечание}{В каждый момент времени с пользователем связан 1 пользовательский диалоговый контекст (содержащий, по крайней мере известные заранее факты о нем: имя, возраст и тому подобное) и несколько тематических.}
    \scnrelfrom{пример спецификации контекстов}{\scnfileimage[40em]{Contents/part_ui/src/images/sd_ui/user_context.png}}
    \begin{scnindent}
        \scnidtf{SCg-текст. Иллюстрация спецификации контекстов}
    \end{scnindent}
    \scntext{пояснение}{Актуальная информация собирается в тематический контекст, объединив который с контекстом пользователя и глобальным контекстом можно получить общий контекст, на основании которого должны осуществляться требуемые действия системы, включая генерацию ответа системы.}

    \scnheader{тематический контекст}
    \scntext{определение}{\textbf{\textit{тематический контекст}} --- \textit{контекст диалога}, содержащий специфические для темы сведения (сведения, полученные во время ведения диалога, на определенную тематику, например, при диалоге об определенном наборе сущностей)}

    \scnheader{множество тематических контекстов диалога*}
    \scntext{определение}{\textbf{\textit{множество тематических контекстов диалога*}} --- \textit{бинарное ориентированное отношение}, диалог с ориентированным множеством его тематических контекстов}

    \scnheader{пользовательский контекст}
    \scntext{определение}{\textbf{\textit{пользовательский контекст}} --- \textit{контекст диалога}, содержащие специфические для пользователя сведения, которые могут быть использованы в диалоге с ним на любую тематику}
    \scntext{примечание}{В общем случае пользовательский \textit{контекст} имеет пересечение с согласованной частью \textit{базы знаний} (предварительно занесенная в \textit{базу знаний} достоверная информация о пользователе, прошедшая необходимую модерацию), но не включается в нее целиком (часть, полученная в ходе диалога в которой мы не уверены).}
    \scnrelfrom{пример}{\scnfileimage[40em]{Contents/part_ui/src/images/sd_ui/context_in_KB.png}}
    \begin{scnindent}
        \scnidtf{Рисунок. Соотношение контекстов с согласованной частью баз знаний}
    \end{scnindent}

    \scnheader{глобальный контекст}
    \scntext{определение}{\textbf{\textit{глобальный контекст}} --- \textit{контекст диалога}, содержащий сведения, которые могут быть необходимы при ведении диалога с любым пользователем}
    \scntext{определение}{\textbf{\textit{глобальный контекст}} --- подмножество согласованной части \textit{базы знаний}, содержащее те сведения, что допустимо использовать в диалоге}
    \scntext{пример}{В диалоге с определенным пользователем не нужно использовать:
        \begin{itemize}
            \item находящуюся в базе знаний служебную информацию, необходимую для работы системы, но не предназначенную для использования в диалоге;
            \item части пользовательских контекстов иных пользователей.
        \end{itemize}}

    \scnheader{неизменяемый в ходе работы системы контекст диалога}
    \scntext{пояснение}{\textbf{\textit{Неизменяемый в ходе работы системы контекст диалога}} содержит в себе знания, необходимые для обеспечения выполнения системой своих функций,  которые были заложены в нее априорно ее разработчиками и/или администраторами и не изменяются в ходе ее функционирования на постоянной основе.}
    
    \scnheader{изменяемый в ходе работы системы контекст диалога}
    \scntext{пояснение}{\textbf{\textit{Изменяемый в ходе работы системы контекст диалога}} содержит в себе знания, необходимые для обеспечения выполнения системой своих функций,  которые были ей получены в ходе ее работы и/или достоверность которых скоротечна.}
    \scnrelfrom{разбиение}{\scnkeyword{Типология изменяемых в ходе работы системы контекстов по источнику знаний\scnsupergroupsign}}
    \begin{scnindent}
        \begin{scneqtoset}
            \scnitem{контекст диалога, содержащий знания из внешних источников}
            \scnitem{контекст диалога, содержащий знания, полученные в ходе диалога}
        \end{scneqtoset}
    \end{scnindent}
    \scnrelfrom{разбиение}{\scnkeyword{Типология изменяемых контекстов по степени их достоверности\scnsupergroupsign}}
    \begin{scnindent}
        \begin{scneqtoset}
            \scnitem{достоверный контекст диалога}
            \scnitem{недостоверный контекст диалога}
        \end{scneqtoset}
    \end{scnindent}

    \scnheader{действие. разрешение контекста}
    \scntext{примечание}{\textbf{\textit{действие. разрешение контекста}} сводится к сопоставлению каждому варианту его значения соответствующего \textit{контекста}. Выбор производится на основании значения функции \textit{$F\underscore{CTD}(T, C)$}, где \textit{T} --- \textit{вариант трансляции}, \textit{C} --- \textit{тематический контекст}. Подходящим контекстом для варианта трансляции считается тот, для которого значение этой функции максимально. В случае, если подходящий контекст не найден, генерируется новый.}
    \scnrelfrom{пример результата}{\scnfileimage[40em]{Contents/part_ui/src/images/sd_ui/relevant_contexts.png}}
    \begin{scnindent}
        \scnidtf{SCg-текст. Иллюстрация сообщения, всем вариантам значения которого сопоставлен контекст}
    \end{scnindent}

    \scnheader{действие. выбор смысла сообщения}
    \scntext{примечание}{\textbf{\textit{действие. выбор смысла сообщения}} представляет собой выбор из множества вариантов трансляции и соответствующих им \textit{контекстов} одной пары и обозначение ее как эквивалентной сообщению конструкции. В простейшем случае, на данном этапе допустимо выполнить выбор в соответствии с рассчитанными на предыдущем этапе для пар потенциально эквивалентных структур и соответствующих им \textit{контекстов} значениями функции \textit{$F\underscore{CTD}(T, C)$} и выбрать пару, для которой оно максимально, однако при необходимости также возможно введение и отдельной функции.}
	\scnrelfrom{пример результата}{\scnfileimage[40em]{Contents/part_ui/src/images/sd_ui/message_equivalent_structure.png}}
    \begin{scnindent}
        \scnidtf{SCg-текст. Иллюстрация конструкции, описывающей эквивалентную сообщению структуру}
    \end{scnindent}

    \scnheader{действие. погружение сообщения в контекс}
    \scntext{примечание}{\textbf{\textit{действие. погружение сообщения в контекст}} представляет собой погружение полученного смысла сообщения в \textit{контекст}. Кроме выбранного смысла сообщения, в контекст может добавляться и иная необходимая для обработки сообщения информация. Кроме того, на данном этапе на основе хранящихся в контексте сведений также должно выполняться разрешение местоимений.}
    \begin{scnrelfromset}{пример}
        \scnitem{SCg-текст. Иллюстрация контекста до погружения в него сообщения}
        \scnitem{SCg-текст. Иллюстрация контекста после погружения в него сообщения}
    \end{scnrelfromset}

    \scnheader{SCg-текст. Иллюстрация контекста до погружения в него сообщения}
    \scnrelfrom{иллюстрация}{\scnfileimage[40em]{Contents/part_ui/src/images/sd_ui/context_1.png}}

    \scnheader{SCg-текст. Иллюстрация контекста после погружения в него сообщения}
    \scnrelfrom{иллюстрация}{\scnfileimage[40em]{Contents/part_ui/src/images/sd_ui/context_2.png}}


    \end{scnsubstruct}
\end{SCn}
    

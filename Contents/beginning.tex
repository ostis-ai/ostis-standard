\begin{SCn}

\scsuperchapter

\scnsectionheader{Стандарт OSTIS}
\label{super_char}
\scnstartsubstruct

\scnidtf{Документация Технологии OSTIS}
\scnidtf{Документация открытой технологии онтологического проектирования, производства и эксплуатации семантически совместимых гибридных интеллектуальных компьютерных систем}
\scnidtf{Описание \textit{Технологии OSTIS} (Open Semantic Technology for Intelligent Systems), представленное в виде раздела \textit{базы знаний ostis-системы} на внутреннем языке \textit{ostis-систем} и обладающее достаточной полнотой для использования этой технологии разработчиками \textit{интеллектуальных компьютерных систем}}
\scnidtf{Полное описание текущего состояния \textit{Технологии OSTIS}, представленное в виде раздела \textit{базы знаний}, построенной по \textit{Технологии OSTIS}}
\scnidtf{Основной раздел \textit{базы знаний} \scnbigspace \textit{Метасистемы IMS.ostis}, которая предназначена для комплексной поддержки онтологического проектирования семантически совместимых \textit{гибридных интеллектуальных компьютерных систем}}
\scniselement{раздел базы знаний}
    \scnaddlevel{1}
    \scnidtf{раздел внутреннего представления \textit{базы знаний ostis-системы} -- \textit{интеллектуальной компьютерной системы}, построенной по \textit{Технологии OSTIS}}
    \scnaddlevel{-1}
%\scnrelfromset{основные авторы}{Голенков В.В.;Гулякина Н.А.;Шункевич Д.В.}
%\scnrelfrom{научный редактор}{Голенков В.В.}
\scnrelfromset{рецензенты}{Курбацкий А.Н.;Дудкин А.А.}
\scnrelfrom{финансовая поддержка}{Intelligent Semantic Systems Ltd.}
\scnaddlevel{1}
\scnaddlevel{-1}
\scnreltovector{конкатенация подразделов}{Вводный раздел Документации Технологии OSTIS;Обоснование Технологии OSTIS;Предметная область и онтология Технологии OSTIS;Заключительная часть Документации Технологии OSTIS;Библиографическая часть Документации Технологии OSTIS}

\scnheader{Стандарт OSTIS}
\scntext{эпиграф}{From data science to knowledge science}
\scntext{аннотация}{В настоящее время информатика преодолевает важнейший этап своего развития --- переход от информатики данных (data science) к информатике знаний (knowledge science), где акцентируется внимание на \uline{семантических} аспектах представления и обработки \textit{знаний}.\\
Без фундаментального анализа такого перехода невозможно решить многие проблемы, связанные с управлением \textit{знаниями}, экономикой \textit{знаний}, с \textit{семантической совместимостью интеллектуальных компьютерных систем}.\\
Основной особенностью \textit{Технологии OSTIS} является ориентация на использование компьютеров нового поколения, специально предназначенных для  реализации семантически совместимых гибридных \textit{интеллектуальных компьютерных систем}. Предлагаемая \textit{Документация Технологии OSTIS} оформлена в виде \textit{раздела базы знаний} специальной интеллектуальной компьютерной \textit{Метасистемы IMS.ostis} (Intelligent MetaSystem for ostis-systems), которая построена по Технологии OSTIS и представляет собой постоянно совершенствуемый интеллектуальный \textit{портал научно-технических знаний}, который поддерживает перманентную эволюцию \textit{Документации Технологии OSTIS}, а также разработку различных \textit{ostis-систем} (интеллектуальных компьютерных систем, построенных по \textit{Технологии OSTIS}).}
\scnidtf{Процесс перманентной эволюции \textit{Стандарта OSTIS}, совмещенного (интегрированного) с комплексными учебно-методическим обеспечением подготовки специалистов в области Искуссственного интеллекта и представленного в виде специального раздела базы знаний}
\scnnote{Подчеркнем, что \textit{Стандарт OSTIS} -- это не описание некоторого состояния \textit{Технологии OSTIS}, а \uline{динамическая} информационная модель процесса эволюции этой технологии}
\scnidtf{Стандарт Технологии OSTIS}
\scnidtf{Документация \textit{Технологии OSTIS}, полностью отражающая \uline{текущее} состояние \textit{Технологии OSTIS} и представленная соответствующим \textit{разделом базы знаний} специальной \textit{ostis-системы}, которая ориентирована на поддержку проектирования, производства, эксплуатации и эволюции (реинжиниринга) \textit{ostis-систем}, а также на поддержку эволюции самой \textit{Технологии OSTIS} и которая названа нами \textit{Метасистемой IMS.ostis}}
\scnidtf{Максимальный раздел \textit{Стандарта OSTIS}, т.е. раздел, в состав которого входят все остальные \textit{разделы} (подразделы) \textit{Стандарта OSTIS}}
\scnidtf{Раздел базы знаний, текущее состояние которого отражает текущее состояние (текущую версию) перманентно эволюционируемого Стандарта Комплексной Технологии OSTIS}
\scnidtf{Представленное в форме раздела базы знаний специальной ostis-системы (Метасистемы IMS.ostis) полное описание (спецификация, документация) текущего состояния Технологии OSTIS}
\scnidtf{Мы рассматриваем \textit{Стандарт OSTIS} (Документацию Стандарта Технологии OSTIS) как продукт \textit{научно-технической деятельности}, к которому предъявляются \uline{высокие требования} по полноте, согласованности, непротиворечивости, практической значимости разрабатываемой документации, описывающей текущее состояние \textit{Технологии OSTIS}}

\scnheader{официальная версия Стандарта OSTIS}
\scnidtf{официально издаваемая (публикуемая в бумажном и/или электронном виде) \textit{версия Стандарта OSTIS}}
\scnhaselement{Стандарт OSTIS-2021}
	\scnaddlevel{1}
	\scnidtf{Официальная \textit{версия Стандарта OSTIS}, издаваемая непостредственно перед началом \textit{конференции OSTIS-2021}}
	\scnaddlevel{-1}

\scnheader{Стандарт OSTIS}
\scnrelto{формальная спецификация}{Технология OSTIS}
	\scnaddlevel{1}
	\scnidtf{Перманентно развиваемый в рамках открытого проекта комплекс моделей, методов и средств, ориентированных на онтологическое проектирование, производство, экплуатацию и реинжиниринг семантически совместимых гибридных интеллектуальных компьютерных систем, способных самостоятельно взаимодействовать друг с другом}
	\scnidtf{Технология разработки семантически совместимых и самостоятельно взаимодействующих интеллектуальных компьютерных систем}
	\scnexplanation{\textit{Технология OSTIS} -- это технология принципиально нового уровня, это обусловлено:
		\begin{scnitemize}
		\item высоким качеством интеллектуальных компьютерных систем (ostis-систем), разрабатываемых на ее основе -- их семантической совместимостью, способностью к самостоятельному взаимодействию, способностью адаптироваться к пользователям и способностью адаптировать (обучать) самих пользователей более эффективному взаимодействию с интеллектальными компьютерными системами;
		\item высоким качеством самой \textit{технологии} -- возможностью интегрировать самые различные \textit{виды знаний} и самые различные \textit{модели решения задач}, неразрывной связью процесса разработки интеллектуальных компьютерных систем и процесса повышения квалификации разработчиков.
		\end{scnitemize}}
	\scnaddlevel{-1}
\bigskip
\scnrelfromvector{ключевые знаки}{Технология OSTIS
	\scnaddlevel{1}
	\scnidtf{Open Semantic Technology for Intelligent Systems}
	\scnidtf{Открытая технология онтологического проектирования, производства и эксплуатации семантически совместимых гибридных интеллектуальных компьютерных систем}
	\scnaddlevel{-1}
;технология\\
\scnaddlevel{1}
	\scnidtf{комплекс моделей, методов и средств, обеспечивающих решение соответствующего класса задач, выполнение соответствующего класса действий, реализацию соответствующего вида деятельности}
	\scnsuperset{технология разработки интеллектуальных компьютерных систем}
		\scnaddlevel{1}
		\scnhaselement{Технология OSTIS}
			\scnaddlevel{1}
		\scnidtf{Технология разработки ostis-систем}
		\scnidtf{Технология разработки семантически совместимых интеллектуальных компьютерных систем, способных к самостоятельному взаимодействию}
			\scnaddlevel{-1}
		\scnsuperset{технология проектирования интеллектуальных компьютерных систем}
		\scnsuperset{технология производства интеллектуальных компьютерных систем}
			\scnaddlevel{1}
			\scnidtf{технологии сборки и установки интеллектуальных компьютерных систем}
			\scnaddlevel{-1}
		\scnsuperset{технология эксплуатации интеллектуальных компьютерных систем}
	 	\scnsuperset{технология реинжинеринга интеллектуальных компьютерных систем}
	 		\scnaddlevel{1}
			\scnidtf{технология совершенствования эволюции интеллектуальных компьютерных систем}
			\scnaddlevel{-1}
		\scnaddlevel{-1}
	\scnsuperset{технология эволюции технологии}
		\scnaddlevel{1}
		\scnidtf{метатехнология эволюции}
		\scnaddlevel{-1}
	\scnaddlevel{-1}
	\scnaddlevel{1}
	\scnsuperset{открытая технология}
		\scnaddlevel{1}
		\scnidtf{технология, доступная не только для тех, кто желает ее использовать, но и для тех, кто желает участвовать в ее развитии -- для того, чтобы технология быстро развивалась, она должна иметь широкий круг своих пользователей и разработчиков}
		\scnexplanation{технология с достаточно свободным формированием её авторского коллектива и пользовательского коллектива}
		\scnsuperset{свободно распространяемая технология}
		\scnsuperset{свободно эволюционируемая технология}
			\scnaddlevel{1}
			\scnidtf{технология эволюционируемая (совершенствуемая) с помощью свободно формируемого авторского коллектива и, соответственно, развиваемая в рамках открытого проекта}
			\scnaddlevel{-1}
		\scnaddlevel{-1}
	\scnsuperset{технология проектирования}
	\scnsuperset{технология производства}
	\scnsuperset{технология эксплуатации}
	\scnsuperset{технология эволюции}
	\scnsuperset{онтологическая технология}
		\scnaddlevel{1}
		\scnidtf{технология, в основе которой лежит формальное представление иерархической системы предметных областей и онтологий}
		\scnidtf{технология выполнения соответствующего вида деятельности, в основе которой лежит иерархическая система формальных онтологий, обеспечивающая четкую стратификацию указанной деятельности и наследование свойств между различными уровнями детализации этой
деятельности}
		\scnsuperset{технология онтологического проектирования}
		\scnsuperset{технология онтологического производства}
		\scnsuperset{технология онтологической эксплуатации}
		\scnaddlevel{-1}
		\scnaddlevel{-1}
;интеллектуальная компьютерная система
			\scnaddlevel{1}
			\scnidtf{искусственная интеллектуальная система}
			\scnsubset{интеллектуальная система}
				\scnaddlevel{1}
				\scnidtf{интеллектуальная кибернетическая система}
				\scnsubset{кибернетическая система}
				\scnaddlevel{-1}
		\scnsuperset{гибридная интеллектуальная компьютерная система}
		\scnaddlevel{1}
		\scnidtf{интеллектуальная компьютерная система, в которой глубоко интегрированы различные виды знаний и различные модели решения задач}
		\scnaddlevel{-1}
	\scnaddlevel{-1}
;семантическая совместимость интеллектуальных компьютерных систем\scnsupergroupsign
	\scnaddlevel{1}
	\scnidtf{свойство, определяющее степень (уровень) семантической совместимости каждой пары \textit{интеллектуальных компьютерных систем}\scnsupergroupsign}
	\scnaddlevel{-1}
;семейство семантически совместимых интеллектуальных компьютерных систем*
	\scnaddlevel{1}
	\scnidtf{семейство интеллектуальных компьютерных
систем, все пары которых имеют одинаковый уровень семантической совместимости (взаимопонимания)}
	\scnaddlevel{-1}
;семантическая унификация интеллектуальных компьютерных систем 
	\scnaddlevel{1}
	\scnidtf{процесс обеспечения высокого уровня семантической совместимости \textit{интеллектуальных компьютерных систем} в ходе их проектирования, эксплуатации и реинжиниринга}
	\scnaddlevel{-1}
;ostis-система
	\scnaddlevel{1}
	\scnidtf{интеллектуальная компьютерная система, разработанная по \textit{Технологии OSTIS}}
	\scnsubset{интеллектуальная компьютерная система}
	\scnaddlevel{-1}
;ostis-документация
	\scnaddlevel{1}
	\scnidtf{документация соответствующего объекта, представленная в виде \textit{раздела базы знаний} некоторой \textit{ostis-системы}}
	\scnhaselement{Документация Технологии OSTIS}
	\scnaddlevel{-1}
;публикация ostis-документации
	\scnaddlevel{1}
	\scnidtf{внешнее представление \textit{ostis-документации}, доступное широкому кругу читателей}
	\scnsubset{внешнее представление фрагмента базы знаний ostis-системы}
	\scnhaselement{Публикация Документации Технологии OSTIS-2021}
		\scnaddlevel{1}
		\scnnote{Здесь имеется в виду Документации Технологии OSTIS версии 2021 года}
		\scnaddlevel{-1}
	\scnaddlevel{-1}
;публикация ostis-документации*
	\scnaddlevel{1}
	\scnidtf{быть публикацией заданной \textit{ostis-документации}*}
	\scnidtfexp{Бинарное ориентированное \textit{отношение}, каждая \textit{пара} которого связывает знак некоторого \textit{раздела базы знаний} со знаком \textit{файла}, который является внешним представлением указанного раздела, а также является либо копией электронной публикации материалов этого раздела, либо оригинал-макетом бумажной публикации указанных материалов*}
	\scnaddlevel{-1}
;оглавление публикации ostis-документации\\
	\scnaddlevel{1}
	\scnhaselement{Оглавление Публикации Документации Технологии OSTIS-2021}
	\scnaddlevel{-1}
;оглавление публикации ostis-документации*
	\scnaddlevel{1}
	\scnidtf{быть оглавлением заданной публикации соответствующей ostis-документации*}
	\scnidtfexp{Бинарное ориентированное \textit{отношение}, каждая \textit{пара} которого связывает знак некоторого \textit{раздела базы знаний} либо знак \textit{файла}, содержащего некоторый \textit{документ}, с описанием иерархии \uline{всех} \textit{разделов}, входящих в состав указанного \textit{раздела базы знаний} либо указанного \textit{документа}*}
	\scnaddlevel{-1}
;база знаний ostis-системы
;раздел базы знаний \textit{ostis-системы}
%% Текст в кружке, не знаю как выделить, пока поставил textit
;конкатенация подразделов*\\
	\scnaddlevel{1}
	\scnexplanation{Бинарное ориентированное \textit{отношение}, каждая \textit{пара} которого связывает \textit{знак} некоторого \textit{раздела базы знаний} либо знак \textit{файла}, содержащего некоторый \textit{документ}, с упорядоченным множеством всех \uline{непосредственных} подразделов указанного \textit{раздела базы знаний} или указанного \textit{документа}*}
	\scnaddlevel{-1}
;Искусственный интеллект\\
	\scnaddlevel{1}	
	\scnidtf{Научно-техническая дисциплина, направленная на изучение \textit{интеллектуальных систем} для построения \textit{искусственных интеллектуальных систем}}
	\scniselement{научно-техническая дисциплина}
	\scnaddlevel{-1}	
;интеллектуальная система\\
	\scnaddlevel{1}
	\scnidtf{кибернетическая система, имеющая достаточно высокий уровень \textit{интеллекта}}
	\scnsubset{кибернетическая система}
	\scnaddlevel{1}
	\scnidtf{система, в основе функционирования которой лежит обработка \textit{информации}}
	\scnaddlevel{-1}
	\scnsuperset{интеллектуальная компьютерная система}
	\scnaddlevel{1}
	\scnidtf{искусственная \textit{интеллектуальная система}}
	\scnidtf{интеллектуальная \textit{компьютерная система}}
	\scnaddlevel{-1}	
	\scnaddlevel{-1}	
;SC-код
	\scnaddlevel{1}
	\scnidtf{Semantic Computer Code}
	\scnidtf{Внутренний язык \textit{ostis-систем}}
	\scnaddlevel{-1}
;sc-конструкция
	\scnaddlevel{1}
	\scnidtf{информационная конструкция, принадлежащая \textit{SC-коду}}
	\scnsuperset{sc-текст}
	\scnaddlevel{1}
	\scnidtf{синтаксически целостная sc-конструкция}
	\scnsuperset{sc-знание}
	\scnaddlevel{1}
	\scnidtf{семантически целостный sc-текст}
	\scnaddlevel{-1}
	\scnaddlevel{-1}
	\scnaddlevel{-1}
;sc-элемент
	\scnaddlevel{1}
	\scnidtf{\textit{знак}, входящий в в состав \textit{sc-конструкции}}
	\scnaddlevel{-1}
;sc-идентификатор
	\scnaddlevel{1}
	\scnidtf{внешний идентификатор \textit{sc-элемента}}
	\scnidtf{внешний идентификатор знака, входящего в текст Внутреннего языка \textit{ostis-системы}}	
	\scnaddlevel{-1}
;внешний язык ostis-систем
	\scnaddlevel{1}
	\scnidtf{язык коммуникации \textit{ostis-систем} с их пользователями и другими \textit{ostis-системами}}
	\scnhaselement{SCg-код}
		\scnaddlevel{1}
		\scnidtf{Semantic Computer Code graphical}
		\scnidtf{Язык графического представления знаний \textit{ostis-систем}}
		\scnaddlevel{-1}
	\scnhaselement{SCs-код}
		\scnaddlevel{1}
		\scnidtf{Semantic Computer Code string}
		\scnidtf{Язык линейного представления знаний \textit{ostis-систем}}
		\scnaddlevel{-1}
	\scnhaselement{SCn-код}
		\scnaddlevel{1}
		\scnidtf{Semantic Computer Code natural}
		\scnidtf{Язык структурированного представления знаний \textit{ostis-систем}}
		\scnaddlevel{-1}
	\scnaddlevel{-1}
;знание ostis-системы\\
	\scnaddlevel{1}
	\scnsuperset{внутреннее представление знания ostis-системы}
	\scnsuperset{внешнее представление знания ostis-системы}
	\scnaddlevel{-1}
;предметная область
	\scnaddlevel{1}
	\scnidtf{предметная область, представленная в памяти \textit{ostis-системы}}
	\scnidtf{\textit{sc-модель} предметной области}
	\scnaddlevel{-1}
;онтология
	\scnaddlevel{1}
	\scnidtf{формальная онтология, представленная в памяти \textit{ostis-системы}}
	\scnidtf{\textit{sc-модель} онтологии}
	\scnaddlevel{-1}
;предметная область и онтология
	\scnaddlevel{1}
	\scnidtf{объединение \textit{предметной области} и онтологии}
	\scnaddlevel{-1}
;кибернетическая система\\
	\scnaddlevel{1}
	\scnsuperset{интеллектуальная система}
	\scnsuperset{компьютерная система}
		\scnaddlevel{1}
		\scnsuperset{интеллектуальная компьютерная система}
		\scnaddlevel{-1}
	\scnaddlevel{-1}
;решатель задач кибернетической системы
;интерфейс кибернетической системы
;база знаний
;традиционная компьютерная технология
	\scnaddlevel{1}
	\scnidtf{современная информационная технология}
	\scnaddlevel{-1}
;технология искусственного интеллекта
;логико-семантическая модель ostis-системы
	\scnaddlevel{1}
	\scnidtf{формальная логико-семантическая модель \textit{ostis-системы}, представленная в \textit{SC-коде}}
	\scnidtf{sc-модель \textit{ostis-системы}}
	\scnidtf{результат проектирования \textit{ostis-системы}}
	\scnidtf{стартовое состояние \textit{базы знаний} проектируемой \textit{ostis-системы}}
	\scnaddlevel{-1}
;семантическая сеть
;семантический язык\\
	\scnaddlevel{1}
	\scnidtf{\textit{язык}, построенный на основе \textit{семантических сетей}}
	\scnaddlevel{-1}
;семантическая модель базы знаний\\
	\scnaddlevel{1}
	\scnidtf{\textit{база знаний}, представленная на \textit{семантическом языке}}
	\scnsuperset{sc-модель базы знаний}
	\scnaddlevel{-1}
;семантическая модель решателя задач\\
\scnaddlevel{1}
\scnsuperset{sc-модель решателя задач}
\scnaddlevel{-1}
;семантическая модель интерфейса кибернетической системы\\
\scnaddlevel{1}
\scnsuperset{sc-модель интерфейса кибернетической системы}
\scnaddlevel{-1}
;семантическая окрестность\\
	\scnaddlevel{1}
	\scnidtf{семантическая спецификация}
	\scnidtf{sc-окрестность}
	\scnaddlevel{-1}
;логическая формула\\
\scnaddlevel{1}
	\scnidtf{sc-модель логической формулы}
\scnaddlevel{-1}
;логическая онтология\\
	\scnaddlevel{1}
	\scnsubset{онтология}
	\scnaddlevel{-1}
;внешняя информационная конструкция ostis-системы\\
	\scnaddlevel{1}
	\scnidtf{информационная конструкция, записанная не в SC-коде}
	\scnaddlevel{-1}
;файл ostis-системы\\
;свойство\\
	\scnaddlevel{1}
	\scnidtf{параметр}
	\scnidtf{семейство множеств сущностей, эквивалентных в заданном семействе}
	\scnsubset{класс классов}
	\scnaddlevel{-1}
;величина\\
	\scnaddlevel{1}
	\scnidtf{значение свойства}
	\scnaddlevel{-1}
;шкала
;структура\\
	\scnaddlevel{1}
	\scnidtf{фрагмент базы знаний \textit{ostis-системы}}
	\scnaddlevel{-1}
;теоретико-множественная онтология\\
	\scnaddlevel{1}
	\scnsubset{онтология}
	\scnaddlevel{-1}
;материальная сущность
;пространственная сущность
;темпоральная сущность\\
	\scnaddlevel{1}
	\scnidtf{временная сущность}
	\scnidtf{временно существующая сущность}
	\scnaddlevel{-1}
;темпоральная сущность базы знаний ostis-системы\\
	\scnaddlevel{1}
	\scnsubset{темпоральная сущность}
	\scnaddlevel{-1}
;действие
;задача
;план
;протокол
;метод\\
	\scnaddlevel{1}
	\scnidtf{метод решения задач заданного класса}
	\scnaddlevel{-1}
;задачная онтология\\
	\scnaddlevel{1}
	\scnsubset{онтология}
	\scnidtf{онтология методов решения задач в рамках заданной предметной области}
	\scnaddlevel{-1}
;внутренний агент ostis-системы
;Язык SCP\\
	\scnaddlevel{1}
	\scnidtf{Базовый язык программирования ostis-систем}
	\scnidtf{Semantic Code Programming}
	\scnaddlevel{-1}
;денотационная семантика языка программирования
;операционная семантика языка программирования
;информационно-поисковый агент ostis-системы
;логическое исчисление
;логический агент ostis-системы
;сообщение
;интерфейсное действие ostis-системы
;агент пользовательского интерфейса ostis-системы
;знаковая конструкция
;естественный язык
;методика разработки ostis-систем
;средство разработки ostis-систем
;базовый интерпретатор логико-семантических моделей ostis-систем
;семантический ассоциативный компьютер
;Библиотека многократно используемых компонентов ostis-систем\\
	\scnaddlevel{1}
	\scnidtf{Библиотека \textit{Технологии OSTIS}}
	\scnaddlevel{-1}
;встраиваемая ostis-система
;понимание информации
;противоречие в базе знаний ostis-системы
;информационная дыра в базе знаний ostis-системы\\
	\scnaddlevel{1}
	\scnidtf{неполнота в базе знаний \textit{ostis-системы}}
	\scnaddlevel{-1}
;разработчик баз знаний ostis-систем
;Экосистема OSTIS
;интеллектуальный портал научно-технических знаний
;интеллектуальная справочная система
;интеллектуальная help-система
;интеллектуальная корпоративная система
;интеллектуальная система в сфере образования\\
	\scnaddlevel{1}
	\scnsuperset{интеллектуальная обучающая система}
	\scnaddlevel{-1}
;интеллектуальная система автоматизации проектирования
;интеллектуальная система управления проектированием
;интеллектуальная система управления производством
;Метасистема IMS.ostis
;ostis-система\\
\scnaddlevel{1}
\scnidtf{интеллектуальная компьютерная система, построенная на Технологии OSTIS}
\scnaddlevel{-1}
;интеллектуальная компьютерная система\\
\scnaddlevel{1}
\scnidtf{компьютерная система, основанная на знаниях}
\scnaddlevel{-1}
;база знаний
\scnaddlevel{1}
\scnidtf{система знаний интеллектуальной компьютерной системы}
\scnaddlevel{-1}
}

\scnheader{Публикация Документации Технологии OSTIS-2021}
\scnrelto{публикация ostis-документации}{Документация Технологии OSTIS}
\scnidtf{Предлагаемое Вашему вниманию издание внешнего представления \textit{раздела базы знаний} \scnbigspace \textit{Метасистемы IMS.ostis}, посвященного комплексному описанию \textit{Технологии OSTIS} и отражающего версию указанного раздела, соответствующую весеннему периоду 2021 года}

\bigskip

\scnaddlevel{1}
\scnheaderlocal{публикация ostis-документации*}
\scnidtfexp{Бинарное ориентированное \textit{отношение}, каждая \textit{пара} которого связывает знак некоторого \textit{раздела базы знаний} со знаком \textit{файла}, который является внешним представлением указанного раздела, а также является либо копией электронной публикации материалов этого раздела, либо оригинал-макетом бумажной публикации указанных материалов*}
\scnaddlevel{-1}

\scnheader{Публикация Документации Технологии OSTIS-2021}
\scnidtf{Издание Документации Технологии OSTIS-2021}
\scnidtf{Первое издание (публикация) Внешнего представления Документации Технологии OSTIS в виде книги}
\scniselement{публикация}
	\scnaddlevel{1}
	\scnidtf{библиографический источник}
	\scnaddlevel{-1}
\scniselement{бумажное издание}
\scniselement{научное издание}
\scnrelfrom{рекомендация издания}{***}
\scnrelfrom{издательство}{***}
\scniselementrole{УДК}{\scnfileshort{***}}
\scniselementrole{ББК}{\scnfileshort{***}}
\scniselementrole{ISBN}{\scnfileshort{***}}
\scnrelfrom{технический редактор}{***}
\scnrelfrom{художественный редактор}{***}
\scnrelfrom{корректор}{***}
\scnrelfrom{верстка}{***}
\scnrelfrom{дата подписания в печать}{***}
\scnrelfrom{тираж}{***}
\scnrelfrom{оглавление публикации ostis-документации}{Оглавление Публикации Документации Технологии OSTIS-2020}

\bigskip

\scnaddlevel{1}
\scnheaderlocal{оглавление публикации ostis-документации*}
\scnidtfexp{Бинарное ориентированное \textit{отношение}, каждая \textit{пара} которого связывает знак некоторого \textit{раздела базы знаний} либо знак \textit{файла}, содержащего некоторый \textit{документ}, с описанием иерархии \uline{всех} \textit{разделов}, входящих в состав указанного \textit{раздела базы знаний} либо указанного \textit{документа}*}

\bigskip

\scnheaderlocal{следует отличать*}
\scnhaselementset{конкатенация подразделов*;оглавление*}
	\scnaddlevel{1}
	\scnexplanation{Отношение \textit{конкатенация подразделов*} описывает декомпозицию заданного раздела или документа только на один шаг --- на непосредственные подразделы заданного раздела базы знаний или документа.}
	\scnaddlevel{-1}

\scnaddlevel{-1}

\newpage

\scnstructheader{Оглавление Публикации Документации Технологии OSTIS-2021}
\scnstartfile

%\vspace{-3\baselineskip}

\end{SCn}

%\DeactivateBG

\doublespacing
\normalsize 

\begingroup
\let\clearpage\relax
\tableofcontents
\endgroup

\singlespacing

\begin{SCn}
\scnendfile \scninlinesourcecommentpar{Завершили \textit{Оглавление Публикации Документации Технологии OSTIS-2021}}
\end{SCn}

\newpage
%\ActivateBG

\begin{SCn}

\scnnote{Подчеркнем, что в \textit{Оглавлении Публикации Документации Технологии OSTIS-2021} отсутствуют номера разделов.\\
 Это обусловлено тем, что количество и порядок разделов, а также номера их страниц актуальны только для текущего состояния данного внешнего текста базы знаний. База знаний эволюционирует независимо от вводимых в неё знаний, которые в общем случае разрабатываются разными авторами и могут иметь разный объем. В результате такой эволюции появляются новые разделы базы знаний, некоторые разделы могут "переместиться"\ и, соответственно, поменять приписываемый им номер.\\
Это обусловлено эволюцией самой базы знаний, а также тем, какой раздел базы знаний отображается в виде внешнего текста}

\bigskip
\scnaddlevel{1}
\scnheaderlocal{следует отличать*}
\scnhaselementset{Документация Технологии OSTIS;публикация Документации Технологии OSTIS;файл Документации Технологии OSTIS
\scnaddlevel{1}
    \scnidtf{файл соответствующей версии Документации Технологии OSTIS}
    \scnidtf{файл соответствующей версии стандарта Технологии OSTIS}
\scnaddlevel{-1}
}
\scnaddlevel{-1}

\end{SCn}


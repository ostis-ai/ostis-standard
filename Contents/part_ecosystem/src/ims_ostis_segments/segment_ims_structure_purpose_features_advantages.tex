\begin{SCn}
	\scnsectionheader{Сегмент. Структура, назначение, особенности и достоинства Метасистемы OSTIS}
	
	\begin{scnsubstruct}
		
	\scnheader{Метасистема OSTIS}
	\scniselement{ostis-система}
	\scntext{назначение}{Эффективность любой технологии, в том числе и \textit{\textbf{Технологии OSTIS}} определяется не только сроками создания искусственных систем соответствующего класса, но и темпами совершенствования самой технологии (темпами совершенствования средств автоматизации и темпами совершенствования системы стандартов, лежащих в основе технологии).\\
	Для фиксации текущего состояния \textit{Технологии OSTIS}, а также для организации ее эффективного использования и ее перманентного совершенствования с участием ученых, работающих в области искусственного интеллекта, и инженеров, разрабатывающих семантические компьютерные системы различного назначения, в состав \textit{Экосистемы OSTIS} вводится \textit{Метасистема OSTIS} (\cite{IMS}), назначение которой делает ее \uline{ключевой} \textit{ostis-системой} в рамках \textit{Экосистемы OSTIS}.}
	\scnrelto{форма реализации}{Технология OSTIS}
	\scnidtf{Интеллектуальная Метасистема, являющаяся формой (вариантом) реализации (представления, оформления) \textit{Технологии OSTIS} в виде \textit{ostis-системы}}
	\scntext{примечание}{Тот факт, что \textit{Технология OSTIS} реализуется в виде \textit{ostis-системы}, является весьма важным для эволюции \textit{Технологии OSTIS}, поскольку методы и средства эволюции (перманентного совершенствования) \textit{Технологии OSTIS} становятся фактически совпадающими с методами и средствами разработки любой (!) \textit{ostis-системы} на всех этапах их жизненного цикла. Другими словами, эволюция Технологии OSTIS осуществляется методами и средствами самой этой технологии.}
	\scnidtf{Система комплексной автоматизации (информационной и инструментальной поддержки) проектирования и реализации ostis-систем, которая сама реализована также в виде ostis-системы}
	\scnidtf{Портал знаний по \textit{Технологии OSTIS}, интегрированный с с.а.п.р. ostis-систем и реализованный в виде ostisсистемы}
	\scniselement{портал научно-технических знаний}
	\scnidtf{Интеллектуальная метасистема комплексной информационной и инструментальной поддержки проектирования совместимых семантических компьютерных систем, которая является формой реализации общей теории и технологии проектирования семантических компьютерных систем и которая поддерживает высокий темп эволюции указанной теории и технологии}
	\scnidtf{Интеллектуальная метасистема, построенная по стандартам Технологии OSTIS и предназначенная (1) для инженеров \textit{ostis-систем} --- для поддержки проектирования. Реализации и обновления (реинжиниринга) \textit{ostis-систем} и для разработчиков \textit{Технологии OSTIS} --- для поддержки коллективной деятельности по развитию стандартов и библиотек \textit{Технологии OSTIS}}
	\scnidtf{интеллектуальная система, предназначенная для комплексной информационной и инструментальной поддержки проектирования семантически совместимых компьютерных систем, на назначение которых не накладывается никаких ограничений}
	\scnidtf{Универсальная базовая (предметно-независимая) ostis-система автоматизации проектирования ostis-систем любых ostis-систем)}
	\scnidtf{Intelligent MetaSystem for intelligent systems design}
	\scnidtf{OSTIS}
	\scnidtf{ostis-система автоматизации проектирования ostis-систем}
	\scnidtf{Фреймворк интеллектуальных систем}
	\scnidtf{Интеллектуальная метасистема комплексной поддержки проектирования совместимых семантических компьютерных систем по Технологии OSTIS}
	\scnidtf{Фреймворк ostis-систем}
	\scnidtf{Фреймворк OSTIS}
	\scntext{url}{http://OSTIS.net}
	\scntext{назначение}{\textit{Метасистема OSTIS} является в \textit{Экосистеме OSTIS} ключевой интеллектуальной системой, которая поддерживает не только проектирование новых интеллектуальных систем и не только замену устаревших компонентов в интеллектуальных системах, входящих в состав \textit{Экосистемы OSTIS}, но и включение (интеграцию) в состав \textit{Экосистемы OSTIS} новых создаваемых интеллектуальных систем.\\
		\textit{Метасистема OSTIS} ориентирована на разработку и практическое внедрение методов и средств \textbf{компонентного проектирования} семантически совместимых интеллектуальных систем, которая предоставляет возможность быстрого создания интеллектуальных приложений различного назначения. Подчеркнем при этом, что сферы практического применения методики компонентного проектирования семантически совместимых интеллектуальных систем ничем не ограничены.}
	\scnidtf{реализация технологии проектирования семантически совместимых компьютерных систем в виде метасистемы, построенной по той же технологии и обеспечивающей комплексную информационную и инструментальную поддержку проектирования семантически совместимых компьютерных систем}
	\scntext{декомпозиция}{\begin{scnitemize}
			\item полное описание самой Технологии OSTIS;
			\item история эволюции Технологии OSTIS;
			\item описание правил использования Технологии OSTIS;
			\item описание организационной инфраструктуры, направленной на развитие Технологии OSTIS;
			\item библиотека многократно используемых и семантически совместимых компонентов ostis-систем;
			\item методы и инструментальные средства проектирования различного вида компонентов ostis-систем;
			\item технические средства координации деятельности участников проекта, направленные на постоянное совершенствование Технологии OSTIS.
	\end{scnitemize}}
	\scntext{примечание}{При разработке \textit{Teхнологии OSTIS} средством автоматизации этой деятельности является не вся \textit{Метасистема OSTIS}, а только ее часть --- входящая в состав \textit{Метасистемы OSTIS}, \textit{Встраиваемая ostis-система поддержки реижиниринга ostis-систем}, которая поддерживает деятельность разработчиков базы знаний \textit{Метасистемы OSTIS}. Это обусловлено тем, что вся деятельность по разработке \textit{Teхнологии OSTIS} сводится к разработке инженирингу) и обновлению (совершенствованию, реинжинирингу) \textit{Базы знаний Метасистемы OSTIS}).}
	\scntext{новизна}{Новизна \textit{Метасистемы OSTIS} заключается в унификации представления различного вида информации в памяти компьютерных систем на основе смыслового (семантического) представления этой информации, что обеспечивает:
		\begin{scnitemize}
			\item устранение дублирования одной и той же информации в разных интеллектуальных системах и в разных компонентах одной и той же системы;
			\item семантическую совместимость различных компонентов интеллектуальных систем и различных интеллектуальных систем в целом;
			\item существенное расширение библиотек совместимых многократно используемых компонентов компьютерных систем за счет крупных\ компонентов и, в частности, типовых подсистем.
	\end{scnitemize}}
	\begin{scnrelfromlist}{обеспечивает}
		\scnfileitem{Комплексную информационную поддержку всех этапов \textit{жизненного цикла интеллектуальных компьютерных систем нового поколения}.}
		\scnfileitem{Автоматизацию проектирования всех компонентов \textit{интеллектуальных компьютерных систем нового поколения}.}
		\scnfileitem{Комплексную автоматизацию всех этапов жизненного цикла \textit{интеллектуальных компьютерных систем нового поколения}.}
	\end{scnrelfromlist}
	\scntext{примечание}{\textit{Метасистема OSTIS} --- метасистема, являющаяся:
		\begin{scnitemize}
			\item{\textit{корпоративной ostis-системой}, обеспечивающей организацию (координацию) деятельности \textit{Консорциума OSTIS}.}
			\item{формой представления реализации и фиксации текущего состояния \textit{Ядра Технологии OSTIS}.}
			\item{\textit{корпоративной ostis-системой}, взаимодействующей со всеми корпоративными ostis-системами, каждая из которых координирует развитие соответствующей специализированной \textit{ostis-технологии}.}
	\end{scnitemize}}
	\scnrelto{продукт}{Проект OSTIS}
	
	\scnheader{Проект OSTIS}
	\begin{scnrelfromset}{подзадачи}
		\scnfileitem{Разработать \textit{Метасистему OSTIS}, обеспечивающую быстрое компонентное проектирование семантически совместимых компьютерных систем различного назначения.}
		\scnfileitem{Разработать методы и средства, обеспечивающие интенсивное развитие рынка семантически совместимых прикладных интеллектуальных систем, созданных на основе \textit{Метасистемы OSTIS}.}
		\scnfileitem{Разработать методы и средства, обеспечивающие стимулирование интенсивного развития самой \textit{Метасистемы OSTIS}.}
	\end{scnrelfromset}
	\scntext{принципы организации}{Организация \textit{Проекта OSTIS} реализуется в форме взаимодействия \textit{Метасистемы OSTIS} с его пользователями и основана на следующих принципах:\begin{scnitemize}
			\item Решатель задач и пользовательский интерфейс \textit{Метасистемы OSTIS} обеспечивают поддержку всего комплекса проектных задач, решаемых разработчиками прикладных интеллектуальных систем, а также разработчиками самой \textit{Метасистемы OSTIS}.\item Для стимулирования развития рынка совместимых прикладных интеллектуальных систем, разработанных с помощью \textit{Метасистемы OSTIS} и развития самой этой метасистемы используются технические средства анализа и оценки объекта и значимости персонального вклада каждого разработчика в специальных условных единицах.
			\item Для стимулирования развития рынка совместимых прикладных интеллектуальных систем, разработанных с помощью \textit{Метасистемы OSTIS}, за каждую такую интеллектуальную систему, зарегистрированную и специфицированную в рамках \textit{Метасистемы OSTIS}, разработчикам выделяется вознаграждение в используемых условных единицах после того, как эта прикладная система будет протестирована на предмет семантической совместимости с другими системами, разработанными с помощью \textit{Метасистемы OSTIS}. При этом \textit{Метасистема OSTIS} становится площадкой для рекламы и распространения интеллектуальных систем, разработанных с ее помощью.
			\item Стимулирование развития самой \textit{Метасистемы OSTIS} осуществляется следующим образом. Участие в развитии \textit{Метасистемы OSTIS} носит открытый характер, для чего достаточно соответствующим образом зарегистрироваться. Авторские права каждого разработчика \textit{Метасистемы OSTIS} защищаются и каждый его вклад в зависимости от его ценности автоматически измеряется и фиксируется в используемых условных единицах.\item Участие в развитии \textit{Метасистемы OSTIS} может иметь самые различные формы (в простейшем случае, это может быть указание на конкретные ошибки, на конкретные трудности, с которыми пользователь столкнулся, формулировка конкретных пожеланий; более сложным вкладом является добавление в базу знаний метасистемы новых знаний, новых компонентов в библиотеку многократно используемых компонентов). При этом автор нового многократно используемого компонента, включенного в библиотеку \textit{Метасистемы OSTIS}, может выбрать любую лицензию для его распространения и, в том числе, назначить ему любую цену.
			\item Ознакомление зарегистрированными пользователями с \textit{Метасистемой OSTIS} носит бесплатный открытый характер. При коммерческой разработке прикладных интеллектуальных систем стоимость каждого обращения к библиотекам \textit{Метасистемы OSTIS} вполне доступна, но существенно снижается в зависимости от степени активности пользователя в развитии \textit{Метасистемы OSTIS}. Это еще один механизм стимулирования участия в развитии \textit{Метасистемы OSTIS}.
		\end{scnitemize}
		Таким образом, указанные принципы организации \textit{Метасистемы OSTIS} обеспечивают на постоянной основе привлечение к разработке \textit{Метасистемы OSTIS} и к формированию рынка семантически совместимых прикладных интеллектуальных систем неограниченные научные, технические и финансовые ресурсы и, в частности, привлечение любых специалистов, желающих участвовать в этом открытом проекте.}
	
	\scnheader{Метасистема OSTIS}
	\scntext{принципы реализации}{Принципы технической реализации \textit{Метасистемы OSTIS} полностью совпадают с принципами технической реализации прикладных интеллектуальных систем, разрабатываемых с помощью этой метасистемы.}
	\begin{scnrelfromset}{декомпозиция}
		\scnitem{полное описание самой Технологии OSTIS}
		\scnitem{история эволюции Технологии OSTIS}
		\scnitem{описание правил использования Технологии OSTIS}
		\scnitem{описание организационной инфраструктуры, направленной на развитие Технологии OSTIS}
		\scnitem{библиотека многократно используемых компонентов ostis-систем}
		\scnitem{методы и инструментальные средства проектирования различного вида компонентов ostis-систем}
		\scnitem{технические средства координации деятельности участников проекта, направленные на постоянное совершенствование Технологии OSTIS}
	\end{scnrelfromset}
	\begin{scnrelfromset}{декомпозиция ostis-системы}
		\scnitem{SC-модель Метасистемы OSTIS}
		\begin{scnindent}
			\begin{scnrelfromset}{декомпозиция sc-модели ostis-системы}
				\scnitem{База знаний Метасистемы OSTIS}
				\scnitem{Решатель задач Метасистемы OSTIS}
				\scnitem{Пользовательский интерфейс Метасистемы OSTIS}
			\end{scnrelfromset}
		\end{scnindent}
		\scnitem{Программный вариант реализации ostis-платформы}
	\end{scnrelfromset}
	\begin{scnrelfromlist}{подсистема}
		\scnitem{Библиотека Метасистемы OSTIS}
		\scnitem{Средства разработки компонентов ostis-систем}
		\begin{scnindent}
			\begin{scnrelfromset}{декомпозиция}
				\scnitem{Средства поддержки проектирования баз знаний ostis-систем}
				\scnitem{Средства поддержки проектирования решателей задач ostis-систем}
				\scnitem{Средства поддержки проектирования пользовательских интерфейсов ostis-систем}
			\end{scnrelfromset}
		\end{scnindent}
	\end{scnrelfromlist}
	
	\scnheader{База знаний Метасистемы OSTIS}
	\begin{scnrelfromset}{декомпозиция}
		\scnitem{Стандарт OSTIS}
		\scnitem{Раздел Проект OSTIS. История, текущее состояние и перспективы эволюции и применения Технологии OSTIS}
		\scnitem{Документация Метасистемы OSTIS}
		\scnitem{История и текущие процессы эксплуатации Метасистемы OSTIS}
		\scnitem{Раздел Проект OSTIS. История, текущие процессы и план развития Метасистемы OSTIS}
	\end{scnrelfromset}
	
	\scnheader{Решатель задач Метасистемы OSTIS}
	\scntext{примечание}{Решатель задач собственно \textit{Метасистемы OSTIS}, без учета подсистем, включает в себя набор агентов информационного поиска, реализующих базовые механизмы навигации по базе знаний.}
	\begin{scnrelfromset}{декомпозиция}
		\scnitem{Абстрактный sc-агент поиска всех входящих константных позитивных стационарных sc-дуг принадлежности}
		\scnitem{Абстрактный sc-агент поиска всех идентификаторов заданного sc-элемента}
		\scnitem{Абстрактный sc-агент поиска полной семантической окрестности заданного элемента}
		\scnitem{Абстрактный sc-агент поиска связок декомпозиции для заданного sc-элемента}
		\scnitem{Абстрактный sc-агент поиска всех известных сущностей, являющихся общими по отношению к заданной}
		\scnitem{Абстрактный sc-агент поиска определения или пояснения для заданного объекта}
		\scnitem{Абстрактный sc-агент поиска всех известных сущностей, являющихся частными по отношению к заданной}
		\scnitem{Абстрактный sc-агент поиска всех выходящих константных позитивных стационарных sc-дуг принадлежности с их ролевыми отношениями}
		\scnitem{Абстрактный sc-агент поиска всех выходящих константных позитивных стационарных sc-дуг принадлежности}
		\scnitem{Абстрактный sc-агент поиска всех входящих константных позитивных стационарных sc-дуг принадлежности с их ролевыми отношениями}
	\end{scnrelfromset}
	
	\scnheader{Средства поддержки проектирования баз знаний ostis-систем}
	\scnidtf{Встраиваемая ostis-система комплексной поддержки проектирования баз знаний ostis-систем}
	\scnidtf{Встраиваемая типовая интеллектуальная система комплексной автоматизации проектирования, а также управления процессом коллективного проектирования и совершенствования баз знаний интеллектуальных систем на всех этапах их жизненного цикла}
	\scnidtf{Интеллектуальная система автоматизированного проектирования баз знаний}
	\scnidtf{Встраиваемая интеллектуальная система, поддержки проектирования и совершенствования баз знаний интеллектуальных систем на всех этапах их жизненного цикла}
	\scnidtf{Интеллектуальный компьютерный фреймворк баз знаний интеллектуальных систем, разрабатываемых по Технологии OSTIS}
	\scntext{пояснение}{Известно, что разработка базы знаний интеллектуальной системы является весьма трудоемким процессом, во много определяющим качество интеллектуальной системы. Очевидно также, что сокращение сроков разработки базы знаний возможно путем организации коллективной разработки, но при условии решения ряда задач, например:
		\begin{scnitemize}
			\item Как в рамках коллектива разработчиков одной и той же базы знаний предотвратить синдром лебедя, рака и щуки или синдром семи нянек и как снизить накладные расходы на согласование их деятельности по созданию качественной базы знаний.
			\item Как обеспечить возможность включения любых уже формализованных знаний в базу знаний любой интеллектуальной системы (если они там необходимы) без какой-либо ручной корректировки этих знаний и тем самым полностью исключить повторную разработку и адаптацию этих знаний.
	\end{scnitemize}}
	\scntext{назначение}{\textit{Средства поддержки проектирования баз знаний ostis-систем} осуществляют:
		\begin{scnitemize}
			\item мониторинг деятельности каждого участника процесса проектирования баз знаний, что необходимо для защиты его авторских прав, для оценки объема и значимости его вклада в проектную деятельность, для оценки его профессиональной квалификации, для качественного распределения новых проектных работ с учетом его текущей квалификации и планируемого направления ее повышения, для реализации откатов, то есть отмены ошибочных решений, принятых администраторами или менеджерами проектируемой базы знаний;
			\item контроль версий проектируемой базы знаний, реализацию необходимых возвратов к предшествующим версиям;
			\item контроль исполнительской дисциплины;
			\item анализ текущего состояния и динамики процесса проектирования, выявление критических ситуаций;
			\item семантический анализ корректности результатов проектных работ всех участников;
			\item оценку объема и значимости деятельности каждого участника проекта;
			\item оценку текущего состояния и динамики развития квалификационного портрета каждого участника проекта;
			\item формирование рекомендаций по повышению квалификации каждого участника проекта;
			\item контроль качества (непротиворечивости, целостности, полноты, чистоты) текущего состояния проектируемой и совершенствуемой базы знаний.
	\end{scnitemize}}
	\scntext{принципы функционирования}{Каждый участник процесса проектирования базы знаний может выполнять различные виды проектных работ:
		\begin{scnitemize}
			\item предложить новый фрагмент в согласованную часть базы знаний или некоторую корректировку (удаление, изменение) в этой части базы знаний;
			\item высказать согласие или несогласие с предложенной кем-то корректировкой или добавлением в согласованную часть базы знаний;
			\item провести верификацию, тестирование, рецензирование предложенной кем-то корректировки или добавления в согласованную часть базы знаний и написать замечания к доработке этого предложения;\item предложить формулировку нового проектного задания, например, на устранение указываемого противоречия (ошибки), на заполнение указываемой информационной дыры;
			\item высказать конструктивные критические замечания к формулировке нового проектного задания;
			\item предложить исполнителя или группу исполнителей для выполнения пока не исполняемого проектного задания;
			\item высказать конструктивные критические замечания к предложенным исполнителям некоторого свободного проектного задания.
	\end{scnitemize}}
	\begin{scnrelfromset}{декомпозиция}
		\scnitem{База знаний средств поддержки проектирования баз знаний ostis-систем}
		\scnitem{Решатель задач средств поддержки проектирования баз знаний ostis-систем}
		\begin{scnindent}
			\begin{scnrelfromset}{декомпозиция}
				\scnitem{Абстрактный sc-агент верификации баз знаний}
				\scnitem{Абстрактный sc-агент редактирования баз знаний}
				\scnitem{Абстрактный sc-агент автоматизации деятельности разработчиков баз знаний}
				\scnitem{Абстрактный sc-агент автоматизации деятельности администраторов баз знаний}
				\scnitem{Абстрактный sc-агент автоматизации деятельности менеджеров баз знаний}
				\scnitem{Абстрактный sc-агент автоматизации деятельности экспертов базы знаний}
				\scnitem{Абстрактный sc-агент оценки качества базы знаний}
			\end{scnrelfromset}
		\end{scnindent}
		\scnitem{Пользовательский интерфейс средств поддержки проектирования баз знаний ostis-систем}
	\end{scnrelfromset}
	\begin{scnindent}
		\scnrelfrom{источник}{\cite{Davydenko2018}}
	\end{scnindent}
	
	\scnheader{Средства поддержки проектирования решателей задач ostis-систем}
	\begin{scnrelfromset}{декомпозиция}
		\scnitem{Средства поддержки проектирования программ Базового языка программирования ostis-систем}
		\scnitem{Средства поддержки проектирования коллективов внутренних агентов ostis-систем}
		\scnitem{Интеллектуальная среда проектирования искусственных нейронных сетей, семантически совместимых с базами знаний ostis-систем}
	\end{scnrelfromset}
	
	\scnheader{Средства поддержки проектирования коллективов внутренних агентов ostis-систем}
	\begin{scnrelfromset}{декомпозиция}
		\scnitem{База знаний средств поддержки проектирования коллективов внутренних агентов ostis-систем}
		\scnitem{Решатель задач средств поддержки проектирования коллективов внутренних агентов ostis-систем}
		\begin{scnindent}
			\begin{scnrelfromset}{декомпозиция}
				\scnitem{Абстрактный sc-агент верификации sc-агентов}
				\begin{scnindent}
					\begin{scnrelfromset}{декомпозиция}
						\scnitem{Абстрактный sc-агент верификации спецификации sc-агента}
						\scnitem{Абстрактный sc-агент проверки неатомарного sc-агента на непротиворечивость его спецификации спецификациям более частных sc-агентов в его составе}
					\end{scnrelfromset}
				\end{scnindent}
				\scnitem{Абстрактный sc-агент отладки коллективов sc-агентов}
				\begin{scnindent}
					\begin{scnrelfromset}{декомпозиция}
						\scnitem{Абстрактный sc-агент поиска всех выполняющихся процессов, соответствующих заданному sc-агенту}
						\scnitem{Абстрактный sc-агент инициирования заданного sc-агента на заданных аргументах}
						\scnitem{Абстрактный sc-агент активации заданного sc-агента}
						\scnitem{Абстрактный sc-агент деактивации заданного sc-агента}
						\scnitem{Абстрактный sc-агент установки блокировки заданного типа для заданного процесса на заданный sc-элемент}
						\scnitem{Абстрактный sc-агент снятия всех блокировок заданного процесса}
						\scnitem{Абстрактный sc-агент снятия всех блокировок с заданного sc-элемента}
					\end{scnrelfromset}
				\end{scnindent}
			\end{scnrelfromset}
		\end{scnindent}
		\scnitem{Пользовательский интерфейс средств поддержки проектирования коллективов внутренних агентов ostis-систем}
	\end{scnrelfromset}
	\begin{scnindent}
		\scnrelfrom{источник}{\cite{Shunkevich2018}}
	\end{scnindent}
	
	\scnheader{Средства поддержки проектирования программ Базового языка программирования ostis-систем}
	\begin{scnrelfromset}{декомпозиция}
		\scnitem{База знаний средств поддержки проектирования программ Базового языка программирования ostis-систем}
		\begin{scnindent}
			\scntext{пояснение}{\textit{База знаний средств поддержки проектирования программ Базового языка программирования ostis-систем} включает в себя, в частности, типологию некорректностей в scp-программах, способов их выявления и устранения.}
		\end{scnindent}
		\scnitem{Решатель задач средств поддержки проектирования программ Базового языка программирования ostis-систем}
		\begin{scnindent}
			\begin{scnrelfromset}{декомпозиция}
				\scnitem{Абстрактный sc-агент верификации scp-программ}
				\scnitem{Абстрактный sc-агент отладки scp-программ}
				\begin{scnindent}
					\begin{scnrelfromset}{декомпозиция}
						\scnitem{Абстрактный sc-агент запуска заданной scp-программы для заданного множества входных данных}
						\scnitem{Абстрактный sc-агент запуска заданной scp-программы для заданного множества входных данных в режиме пошагового выполнения}
						\scnitem{Абстрактный sc-агент поиска всех scp-операторов в рамках scp-программы}
						\scnitem{Абстрактный sc-агент поиска всех точек останова в рамках scp-процесса}
						\scnitem{Абстрактный sc-агент добавления точки останова в scp-программу}
						\scnitem{Абстрактный sc-агент удаления точки останова из scp-программы}
						\scnitem{Абстрактный sc-агент добавления точки останова в scp-процесс}
						\scnitem{Абстрактный sc-агент удаления точки останова из scp-процесса}
						\scnitem{Абстрактный sc-агент продолжения выполнения scp-процесса на один шаг}
						\scnitem{Абстрактный sc-агент продолжения выполнения scp-процесса до точки останова или завершения}
						\scnitem{Абстрактный sc-агент просмотра информации об scp-процессе}
						\scnitem{Абстрактный sc-агент просмотра информации об scp-операторе}
					\end{scnrelfromset}
				\end{scnindent}
			\end{scnrelfromset}
		\end{scnindent}
		\scnitem{Пользовательский интерфейс средств поддержки проектирования программ Базового языка программирования ostis-систем}
		
		\scnheader{Метасистема OSTIS}
		\scntext{назначение}{\textit{Метасистема OSTIS} является в \textit{Экосистеме OSTIS} ключевой интеллектуальной системой, которая поддерживает не только проектирование новых интеллектуальных систем и не только замену устаревших компонентов в интеллектуальных системах, входящих в состав \textit{Экосистемы OSTIS}, но и включение (интеграцию) в состав \textit{Экосистемы OSTIS} новых создаваемых интеллектуальных систем.}
		\scntext{примечание}{Описываемая \textit{Метасистема OSTIS} является:
		\begin{scnitemize}
			\item{Системой информационной и инструментальной поддержки всех этапов жизненного цикла и.к.с. нового поколения (\textit{ostis-систем}) самого различного назначения.}
			\item{Порталом знаний по \textit{Технологии OSTIS}, обеспечивающим координацию работ по развитию \textit{Технологии OSTIS} и автоматизацию анализа качества \textit{Стандарта OSTIS}.}
		\end{scnitemize}
	То есть \textit{Метасистема OSTIS} является системой управления \textit{Проектом создания и развития Стандарта OSTIS}.}
	\end{scnrelfromset}
	\scntext{примечание}{Важнейшим направлением \textit{Метасистемы OSTIS} и, соответственно, важнейшим направлением применения \textit{Стандарта OSTIS} является использование их в качестве комплексного интегрированного компьютерного учебного пособия по специальности “Искуственный интеллект”. Для этого устанавливается связь между разделами Стандарта OSTIS и программами различных учебных дисциплин указанной специальности. Важно подчеркнуть при этом: \textit{Стандарт OSTIS} содержит достаточно полный сравнительный анализ с различными альтернативными подходами, то есть ни в коем случае не ограничивается рассмотрением только \textit{Технологией OSTIS}.}
	\scntext{примечание}{\textit{Метасистема OSTIS} взаимодействует не только со своими разработчиками и конечными пользователями, но и с другими ostis-системами, которые созданы с помощью \textit{Технологии OSTIS} и представляют собой ее \textit{дочерние системы*}.}
	\scntext{примечание}{\textit{Метасистема OSTIS} для своих дочерних систем может:
	\begin{scnitemize}
		\item{Осуществлять автоматическую сборку \textit{дочерних ostis-систем} стартовых версий по инструкциям. Таким образом, генерировать новые \textit{дочерние ostis-системы}.}
		\item{Включать в \textit{дочерние ostis-системы} новые многократно используемые компоненты из постоянно пополняемой \textit{библиотеки многократно используемых семантически совместимых компонентов ostis-систем}.}
		\item{Заменять в \textit{дочерних ostis-системах} устаревшие версии многократно используемых компонентов на новые версии из \textit{Библиотеки Метасистемы OSTIS}.}
		\item{Включать в \textit{дочерние ostis-системы} подсистему совершенствования своей расширенной базы знаний и, при необходимости, подсистему улучшения ее интегрированной машины обработки знаний и пользовательского интерфейса.}
	\end{scnitemize}
	Таким образом, после появления \textit{дочерней ostis-системы} ее связь с \textit{Метасистемой OSTIS} не прерывается и она становится постоянным участником процесса совершенствования всех \textit{дочерних ostis-систем}.}
	\scntext{примечание}{\textit{Метасистема OSTIS} является одновременно и системой автоматизации проектирования ostis-систем, и интеллектуальной системой, обучающей методам и средствам проектирования \textit{ostis-систем}. Этот факт существенно повышает качество проектирования прикладных \textit{ostis-систем}, расширяет контингент разработчиков ostis-систем и интегрирует проектную (инженерную) деятельность в области искусственного интеллекта с образовательной деятельностью в этой области.}
	\scntext{примечание}{Все опубликованные материалы о \textit{Технологии OSTIS} в формализованном виде входят в \textit{Базу знаний Метасистемы OSTIS}.}
	
		\bigskip
	\end{scnsubstruct}
	\scnsourcecomment{Завершили \scnqqi{Сегмент. Структура, назначение, особенности и достоинства Метасистемы OSTIS}}
\end{SCn}
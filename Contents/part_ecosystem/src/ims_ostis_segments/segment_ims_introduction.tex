\begin{SCn}
	\scnsectionheader{Сегмент. Ввдение в Логико-семантическую модель Метасистемы OSTIS}
	
	\begin{scnsubstruct}
		\scnheader{Логико-семантическая модель Метасистемы OSTIS}
		\begin{scnrelfromlist}{введение}
		\scnfileitem{В основе каждой развитой сферы человеческой деятельности лежит ряд стандартов, формально описывающих различные ее аспекты --- систему понятий (включая терминологию), типологию и последовательность действий, выполняемых в процессе применения соответствующих методов и средств.}
		\begin{scnindent}
			\scnrelfrom{источник}{\scncite{Golenkov2019}}
		\end{scnindent}
		\scnfileitem{Стандарты в самых различных областях являются важнейшим видом знаний, главной целью которых является обеспечение совместимости различных видов деятельности. Несмотря на развитие информационных технологий, в настоящее время подавляющее большинство стандартов представлено либо в виде традиционных линейных документов, либо в виде web-ресурсов содержащих набор статических страниц, связанных гиперссылками. Для того чтобы стандарты выполняли свою главную функцию, они должны постоянно совершенствоваться.}
		\scnfileitem{Текущее оформление стандартов имеет ряд недостатков, которые мешают эффективному и грамотному использованию стандартов в различных областях:
		\begin{scnitemize}
			\item{дублирование информации в рамках документа, описывающего стандарт;}
			\item{трудоемкость сопровождения самого стандарта, обусловленная в том числе дублированием информации, в частности, трудоемкость изменения терминологии;}
			\item{проблема интернационализации стандарта --- фактически перевод стандарта на несколько языков приводит к необходимости поддержки и согласования независимых версий стандарта на разных языках;}
			\item{неудобство применения стандарта, в частности, трудоемкость поиска необходимой информации. Как следствие --- трудоемкость изучения стандарта;}
			\item{несогласованность формы различных стандартов между собой, как следствие --- трудоемкость автоматизации процессов развития и применения стандартов;}
			\item{трудоемкость автоматизации проверки соответствия объектов или процессов требования того или иного стандарта;}
			\item{и другие.}
		\end{scnitemize}
		Перечисленные проблемы связаны в основном с формой представления стандартов.}
		\begin{scnindent}
			\begin{scnrelfromlist}{источник}
				\scnitem{\scncite{Serenkov2004}}
				\scnitem{\scncite{Uglev2012}}
			\end{scnrelfromlist}
		\end{scnindent}
		\scnfileitem{Задачей любого стандарта в общем случае является описание согласованной системы понятий (и соответствующих терминов), бизнес-процессов, правил и других закономерностей, способов решения определенных классов задач и так далее. Для формального описания информации такого рода с успехом применяются онтологии. Более того, в настоящее время в ряде областей вместо разработки стандарта в виде традиционного документа разрабатывается соответствующая онтология. Такой подход дает очевидные преимущества в плане автоматизации процессов согласования и использования стандартов.}
		\scnfileitem{Актуальной остается проблема, связанная не с формой, а с сутью (семантикой) стандартов --- проблема несогласованности системы понятий и терминов между различными стандартами, которая актуальна даже для стандартов в рамках одной и той же сферы деятельности.}
		\scnfileitem{В настоящее время Информатика преодолевает важнейший этап своего развития — переход от информатики данных (data science) к информатике знаний (knowledge science), где акцентируется внимание на семантических аспектах представления и обработки знаний. Без фундаментального анализа такого перехода невозможно решить многие проблемы, связанные с управлением знаниями, экономикой знаний, с семантической совместимостью интеллектуальных компьютерных систем.}
		\scnfileitem{С семантической точки зрения каждый стандарт есть иерархическая онтология, уточняющих структуру и систем понятий соответствующих им предметных областей, которая описывает структуру и функционирование либо некоторого класса технических или иных искусственных систем, либо некоторого класса организаций, либо некоторого вида деятельности.}
		\scnfileitem{Наиболее перспективным подходом к решению перечисленных проблем является преобразование каждого конкретного стандарта в базу знаний, в основе которой лежит набор онтологий, соответствующих данному стандарту. Такой подход позволяет в значительной мере автоматизировать процессы развития стандарта и его применения. В рамках Технологии OSTIS данный подход используется при построении Стандарта OSTIS.}
		\scnfileitem{Предлагаемый Стандарт OSTIS оформлен в виде семейства разделов базы знаний специальной интеллектуальной компьютерной Метасистемы OSTIS (Intelligent MetaSystem for ostis-systems), которая построена по Технологии OSTIS и представляет собой постоянно совершенствуемый интеллектуальный портал научно-технических знаний, который поддерживает перманентную эволюцию Стандарта OSTIS, а также разработку различных ostis-систем (интеллектуальных компьютерных систем, построенных по Технологии OSTIS).}
		\begin{scnindent}
			\scnrelfrom{источник}{\scncite{IMS2017}}
		\end{scnindent}
		\end{scnrelfromlist}
		
		
		\bigskip
	\end{scnsubstruct}
	\scnsourcecomment{Завершили \scnqqi{Сегмент. Ввдение в Логико-семантическую модель Метасистемы OSTIS}}
\end{SCn}
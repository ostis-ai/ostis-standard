\begin{SCn}
    \scnsectionheader{Предметная область и онтология семантически совместимых интеллектуальных ostis-порталов научных знаний}
    \begin{scnsubstruct}
    \scnrelfrom{дочерний раздел}{\nameref{ims_ostis_model}}
    \scniselement{раздел базы знаний}
    \scnhaselementrole{ключевой sc-элемент}{Предметная область семантически совместимых интеллектуальных порталов научно-технических знаний}

    \begin{scnrelfromlist}{библиографическая ссылка}
        \scnitem{\scncite{Van2005}}
        \scnitem{\scncite{Mack2001}}
    \end{scnrelfromlist}
    
    \scnheader{Предметная область семантически совместимых интеллектуальных порталов научно-технических знаний}
    \scniselement{предметная область}
    \begin{scnhaselementrolelist}{максимальный класс объектов исследования}
        \scnitem{портал научных знаний}
    \end{scnhaselementrolelist}
    \begin{scnhaselementrolelist}{класс объектов исследования}
        \scnitem{портал знаний}
        \scnitem{ostis-портал знаний}
    \end{scnhaselementrolelist}
    
    \scnheader{портал научных знаний}
    \scntext{примечание}{Понятие \textit{портала знаний} представляет собой один из способов создания централизованного доступа к информации, которая может быть необходима для решения задач, связанных с работой в организации. Такие порталы могут содержать информацию, связанную с процессами, документацией, процедурами, обучающими материалами, а также ответы на часто задаваемые вопросы.}
    \begin{scnindent}
        \begin{scnrelfromset}{смотрите}
            \scnitem{\scncite{Van2005}}
            \scnitem{\scncite{Mack2001}}
        \end{scnrelfromset}
    \end{scnindent}
    \scntext{примечание}{Одним из ключевых преимуществ \textit{порталов знаний} является их способность к сбору и хранению информации из различных источников, таких как базы данных, системы управления документами, системы управления проектами и так далее. Это позволяет пользователям получать полную и актуальную информацию в одном месте.}
    \scntext{примечание}{На основе \textit{портала знаний} обеспечивается возможность взаимодействия между пользователями путем создания форумов, обсуждений и коллективного редактирования документов. Это может способствовать обмену знаниями и опытом между сотрудниками организации и повысить эффективность их работы.}
    \scntext{примечание}{При создании \textit{порталов знаний} возникают проблемы, связанные с организацией и управлением информацией. Например, необходимо обеспечить корректное и структурированное хранение информации, ее поиск и обновление. Также необходимо учитывать потребности пользователей и обеспечить удобный и интуитивно понятный интерфейс.}
    \begin{scnrelfromlist}{цель}
        \scnfileitem{Ускорение погружения каждого человека в новые для него научные области при постоянном сохранении общей целостной картины Мира (образовательная цель).}
        \scnfileitem{Фиксация в систематизированном виде новых научных результатов так, чтобы все основные связи новых результатов с известными были четко обозначены.}
        \scnfileitem{Автоматизация координации работ по рецензированию новых результатов.}
        \scnfileitem{Автоматизация анализа текущего состояния базы знаний.}
    \end{scnrelfromlist}
    \scntext{пояснение}{Создание интеллектуальных \textbf{\textit{порталов научных знаний}}, обеспечивающих повышение темпов интеграции и согласования различных точек зрения, --- это способ существенного повышения темпов эволюции научно-технической деятельности.\\
        Совместимые \textbf{\textit{порталы научных знаний}}, реализованные в виде \textit{ostis-систем}, входящих в \textit{Экосистему OSTIS}, являются основой новых принципов организации научной деятельности, в которой
        \begin{scnitemize}
            \item результатами этой деятельности являются не статьи, монографии, отчеты и другие научно-технические документы, а фрагменты глобальной базы знаний, разработчиками которых являются свободно формируемые научные коллективы, состоящие из специалистов в соответствующих научных дисциплинах,
            \item с помощью \textbf{\textit{порталов научных знаний}} осуществляется
            \begin{scnitemizeii}
                \item координация процесса рецензирования новой научно-технической информации, поступающей от научных работников в базы знаний этих порталов,
                \item процесс согласования различных точек зрения ученых (в частности, введению и семантической корректировке понятий, а также введению и корректировке терминов, соответствующих различным сущностям).
            \end{scnitemizeii}
        \end{scnitemize}
        Реализация семейства семантически совместимых порталов научных знаний в виде совместимых \textit{\mbox{ostis-систем}}, входящих в состав \textit{Экосистемы OSTIS}, предполагает разработку иерархической системы семантически согласованных формальных онтологий, соответствующих различным научно-техническим дисциплинам, с четко заданным наследованием свойств описываемых сущностей и с четко заданными междисциплинарными связями, которые описываются связями между соответствующими формальными онтологиями и специфицируемыми ими предметными областями.\\
        Реализация \textbf{\textit{порталов научных знаний}} в виде семейства семантически совместимых \textit{ostis-систем} означает также попытку преодолеть вавилонское столпотворение\ многообразия научно-технических языков, не меняя сути научно-технических знаний, а сводя эти знания к единой универсальной форме смыслового представления знаний в памяти порталов научных знаний, т.е. к форме которая в достаточной степени понятна как \textit{ostis-системам}, так и любым потенциальным их пользователям.}
    \scntext{пример}{Примером \textbf{\textit{портала научных знаний}}, построенного в виде \textit{ostis-системы} является \textit{Метасистема OSTIS}, содержащая все известные на текущий момент знания и навыки, входящие в состав \textit{Технологии OSTIS}.}
    \begin{scnrelfromlist}{преимущества}
        \scnfileitem{Использование методов семантической обработки информации, что позволяет более точно и эффективно организовывать и искать информацию на портале знаний.}
        \scnfileitem{Высокий уровень гибкости и расширяемости, что позволяет адаптировать \textit{ostis-порталы знаний} под различные нужды и требования пользователей.}
        \scnfileitem{Автоматическая интеграция \textit{ostis-порталов знаний} с другими \textit{ostis-системами} в рамках \textit{Экосистемы OSTIS}, что позволяет создать централизованный доступ к информации из различных источников.}
        \scnfileitem{Возможность создания персонализированного \textit{ostis-портала знаний}, который учитывает интересы и потребности каждого пользователя, что позволяет более эффективно использовать знания \textit{ostis-систем}.}
        \scnfileitem{Возможность производить \textit{ostis-порталы знаний} быстро и с минимальными затратами благодаря использованию существующих компонентов и инструментов.}
    \end{scnrelfromlist}
        \scntext{примечание}{Реализация \textit{порталов знаний} на основе \textit{Технологии OSTIS} позволяет создать более эффективную и гибкую систему для хранения, организации и поиска знаний, которая может быть адаптирована под различные требования пользователей и организаций.}

    \end{scnsubstruct}
    \scnendcurrentsectioncomment
\end{SCn}

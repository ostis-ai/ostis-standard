\begin{SCn}
    \scnsectionheader{Предметная область и онтология Экосистемы OSTIS}
    \begin{scnsubstruct}
        \begin{scnrelfromlist}{дочерний раздел}
            \scnitem{\nameref{sd_learning}}
            \scnitem{\nameref{sd_assistants}}
            \scnitem{\nameref{sd_portals}}
            \scnitem{\nameref{sd_ecosys_enterprise}}
        \end{scnrelfromlist}
        
    \scnheader{Предметная область Экосистемы OSTIS}
    \scniselement{предметная область}
    \begin{scnhaselementrole}{максимальный класс объектов исследования}
        {Экосистема OSTIS}
    \end{scnhaselementrole}
    \begin{scnhaselementrolelist}{класс объектов исследования}
        \scnitem{ostis-система}
        \scnitem{самостоятельная ostis-система}
        \scnitem{поддержка совместимости между компьютерными системами и их пользователями в Экосистеме OSTIS}
    \end{scnhaselementrolelist}
    
    \scnheader{Проект OSTIS}
    \scnidtf{Проект, направленный на создание \textit{Технологии OSTIS} и, в частности, на разработку \textit{Стандарта OSTIS}}
    \begin{scnrelfromlist}{продукт}
        \scnitem{Технология OSTIS}
        \scnitem{Метасистема OSTIS}
            \begin{scnindent}
                \scnidtf{Метасистема OSTIS}
            \end{scnindent}
        \scnitem{Стандарт OSTIS}
        \scnitem{Экосистема OSTIS}
    \end{scnrelfromlist}
    \begin{scnrelfromlist}{подпроект}
        \scnitem{Проект разработки Технологии OSTIS}
        \scnitem{Проект разработки Метасистемы OSTIS}
        \scnitem{Проект разработки Стандарта OSTIS}
        \scnitem{Проект разработки Экосистемы OSTIS}
    \end{scnrelfromlist}
    \begin{scnrelfromlist}{библиографический источник}
        \scnitem{\cite{DeNicola2021}}
        \scnitem{\cite{Alrehaili2021}}
        \scnitem{\cite{Alrehaili2017}}
        \scnitem{\cite{Shahzad2021}}
    \end{scnrelfromlist}
    
    \scnheader{Экосистема OSTIS}
    \scnidtf{Социотехническая экосистема, представляющая собой коллектив взаимодействующих семантических компьютерных систем и осуществляющая перманентную поддержку эволюции и семантической совместимости всех входящих в нее систем, на протяжении всего их жизненного цикла.}
    \scnidtf{Неограниченно расширяемый коллектив постоянно эволюционируемых семантических компьютерных систем, которые взаимодействуют между собой и с пользователями для корпоративного решения сложных задач и для постоянной поддержки высокого уровня совместимости и взаимопонимания во взаимодействии как между собой, так и с пользователями.}
    \scntext{пояснение}{Поскольку \textit{Технология OSTIS} ориентирована на разработку \textit{семантических компьютерных систем}, обладающих высоким уровнем \textit{обучаемости} и, в частности, высоким уровнем семантической \textit{совместимости}, и поскольку обучаемость и совместимость есть только \uline{способность} к обучению (т.е. к высоким темпам расширения и совершенствования своих знаний и навыков), а также \uline{способность} к обеспечению высокого уровня взаимопонимания (согласованности), необходима некая среда, социотехническая инфраструктура, в рамках которой были бы созданы максимально комфортные условия для реализации указанных выше способностей. Такая среда названа нами \textit{\textbf{Экосистемой OSTIS}}, которая представляет собой коллектив взаимодействующих (через сеть Интернет):
        \begin{scnitemize}
            \item самих \textit{ostis-систем};
            \item пользователей указанных \textit{ostis-систем} (как конечных пользователей, так и разработчиков);
            \item некоторых компьютерных систем, не являющихся \textit{ostis-системами}, но рассматриваемых ими в качестве дополнительных информационных ресурсов или сервисов.
        \end{scnitemize}}
    \scntext{основная задача}{Обеспечить постоянную поддержку совместимости компьютерных систем, входящих в \textit{Экосистему OSTIS} как на этапе их разработки, так и в ходе их эксплуатации. Проблема здесь заключается в том, что в ходе эксплуатации систем, входящих в \textit{Экосистему OSTIS}, они могут изменяться из-за чего совместимость может нарушаться.\\
        Задачами \textit{Экосистемы OSTIS} являются:
        \begin{scnitemize}
            \item оперативное внедрение всех согласованных изменений стандарта \textit{ostis-систем} (в том числе, и изменений систем используемых понятий и соответствующих им терминов);
            \item перманентная поддержка высокого уровня взаимопонимания всех систем, входящих в \textit{Экосистему OSTIS}, и всех их пользователей;
            \item корпоративное решение различных сложных задач, требующих координации деятельности нескольких (чаще всего, априори неизвестных) \textit{ostis-систем}, а также, возможно, некоторых пользователей.
        \end{scnitemize}}
    \scntext{примечание}{\textit{Экосистема OSTIS} --- это переход от самостоятельных (автономных, отдельных, целостных) \textit{ostis-систем} к коллективам самостоятельных \textit{ostis-систем}, т.е. к распределенным \textit{ostis-системам}.}
    
    \scnheader{ostis-система}
    \begin{scnsubdividing}
        \scnitem{самостоятельная ostis-система}
            \begin{scnindent}
                \scnidtf{целостная \textit{ostis-система}, которая должна самостоятельно решать соответствующее множество задач и, в частности, взаимодействовать с внешней средой (как вербально --- с пользователями и другими компьютерными системами, так и невербально)}
            \end{scnindent}
        \scnitem{встроенная ostis-система}
            \begin{scnindent}
                \scnidtf{интеллектуальная компьютерная подсистема, разработанная по \textit{Технологии OSTIS} и реализующая часть функционала \textit{ostis-системы} более высокого уровня иерархии}
                \scnidtf{\textit{ostis-система}, интегрированная в состав \textit{самостоятельной ostis-системы}}
                \begin{scnsubdividing}
                    \scnitem{атомарная встроенная ostis-система}
                        \begin{scnindent}
                        \scnidtf{\textit{встроенная ostis-система}, не включающая в себя какие-либо другие \textit{встроенные ostis-системы}}
                        \end{scnindent}
                    \scnitem{неатомарная встроенная ostis-система}
                        \begin{scnindent}
                            \scnsuperset{интерфейс ostis-системы}
                        \end{scnindent}
                \end{scnsubdividing}
            \end{scnindent}
        \scnitem{коллектив ostis-систем}
            \begin{scnindent}
                \scnidtf{группа общающихся ostis-систем, в состав которой могут входить не только самостоятельные ostis-системы, но и коллективы ostis-систем}
                \scnidtf{распределенная ostis-система}
            \end{scnindent}
    \end{scnsubdividing}
    
    \scnheader{самостоятельная ostis-система}
    \scntext{пояснение}{Подчеркнем, что к \textit{\textbf{самостоятельным ostis-системам}}, входящим в состав \textit{Экосистемы OSTIS}, предъявляются особые требования:
        \begin{scnitemize}
            \item Они должны обладать всеми необходимыми знаниями и навыками для обмена сообщениями и целенаправленной организации взаимодействия с другими \textit{ostis-системам}и, входящими в \textit{Экосистему OSTIS}.
            \item В условиях постоянного изменения и эволюции \textit{ostis-систем}, входящих в \textit{Экосистему OSTIS}, каждая из них должна \uline{сама следить за состоянием своей совместимости} (согласованности) со всеми остальными \textit{ostis-системами},  т.е. должна самостоятельно поддерживать эту совместимость, согласовывая с другими ostis-системами все требующие согласования изменения, происходящие у себя и в других системах.
            \item Каждая система, входящая в состав \textit{Экосистемы OSTIS}, должна:
            \begin{scnitemizeii}
                \item Интенсивно, активно и целенаправленно обучаться (как с помощью учителей-разработчиков, так и самостоятельно).
                \item Сообщать всем другим системам о предлагаемых или окончательно утвержденных изменениях в \textit{онтологиях} и, в частности, в наборе используемых \textit{понятий}.
                \item Принимать от других \textit{ostis-систем} предложения об изменениях в \textit{онтологиях} (в том числе в наборе используемых понятий) для согласования или утверждения этих предложений.
                \item Реализовывать утвержденные изменения в \textit{онтологиях}, хранимых в ее базе знаний.
                \item Способствовать поддержанию высокого уровня семантической совместимости не только с другими \textit{ostis-системами}, входящими в \textit{Экосистему OSTIS}, но и со своими \textit{пользователями} ( т.е. обучать их, информировать их об изменениях в онтологиях).
            \end{scnitemizeii}
        \end{scnitemize}}
    
    \scnheader{Экосистема OSTIS}
    \scntext{пояснение}{\textit{Экосистема OSTIS} является формой реализации, совершенствования и применения \textit{Технологии OSTIS} и, следовательно, является формой создания, развития, самоорганизации рынка семантически совместимых компьютерных систем  и включает в себя все необходимые для этого ресурсы ---  информационные, технологические, кадровые, организационные, инфраструктурные.\\
    \textit{Экосистеме OSTIS} ставится в соответствие ее \textit{\textbf{объединенная база знаний}}, которая представляет собой \textbf{виртуальное объединение} \textit{баз знаний} всех \textit{ostis-систем}, входящих в состав \textit{Экосистемы OSTIS}. Качество этой \textit{базы знаний} (полнота, непротиворечивость, чистота) является постоянной заботой всех самостоятельных \textit{ostis-систем}, входящих в состав \textit{Экосистемы OSTIS}. Соответственно этому каждой указанной \textit{ostis-системе} ставится в соответствие своя \textit{база знаний} и своя иерархическая система \textit{sc-агентов}.\\
        По назначению \textit{ostis-системы}, входящие в \textit{Экосистему OSTIS}, могут быть:
        \begin{scnitemize}
            \item ассистентами конкретных пользователей или конкретных пользовательских коллективов;
            \item типовыми встраиваемыми подсистемами \textit{ostis-систем};
            \item системами информационной и инструментальной поддержки проектирования различных компонентов и различных классов \textit{ostis-систем};
            \item системами информационной и инструментальной поддержки проектирования или производства различных классов технических и других искусственно создаваемых систем;
            \item порталами знаний по самым различным научным дисциплинам;
            \item системами автоматизации управления различными сложными объектами (производственными предприятиями, учебными заведениями, кафедрами вузов, конкретными обучаемыми);
            \item интеллектуальными справочными и help-системами;
            \item интеллектуальными обучающими системами, семантическими электронными учебными пособиями;
            \item интеллектуальными робототехническими системами.
        \end{scnitemize}}
    
    \scnheader{поддержка совместимости между компьютерными системами и их пользователями в Экосистеме OSTIS}
    \scntext{пояснение}{Есть три аспекта поддержки совместимости и взаимопонимания в \textit{Экосистеме OSTIS}:
        \begin{scnitemize}
            \item поддержка совместимости между самими \textit{ostis-системами}, входящими в \textit{Экосистему OSTIS} в процессе их эволюции;
            \item поддержка совместимости между каждой ostis-системой и текущим состоянием Технологии OSTIS в процессе эволюции этой технологии;
            \item поддержка совместимости и взаимопонимания между \textit{ostis-системами}, входящими в \textit{Экосистему OSTIS}, и их пользователями при активном стимулировании со стороны \textit{Экосистемы OSTIS} того, чтобы каждый пользователь \textit{Экосистемы OSTIS} был одновременно не только активным ее конечным пользователем, но и активным ее разработчиком.
        \end{scnitemize}
        Таким образом, для обеспечения высокой эффективности эксплуатации и высоких темпов эволюции \textit{Экосистемы OSTIS}, необходимо постоянно повышать уровень информационной совместимости (уровень взаимопонимания) не только между компьютерными системами, входящими в состав \textit{Экосистемы OSTIS}, но также между этими системами и их пользователями. Одним из направлений обеспечения такой совместимости является стремление к тому, чтобы \textit{база знаний} (картина мира) каждого пользователя стала частью (фрагментом) \textbf{\textit{Объединенной базы знаний Экосистемы OSTIS}}. Это значит, что каждый пользователь должен знать, как устроена структура каждой научно-технической дисциплины (объекты исследования, предметы исследования, определения, закономерности и т.д.), как могут быть связаны между собой различные дисциплины.\\
        Формирование таких навыков системного построения картины Мира необходимо начинать со средней школы. Для этой цели необходимо создать комплекс совместимых интеллектуальных обучающих систем по всем дисциплинам среднего образования с четко описанными междисциплинарными связями (\cite{Bashmakov}, \cite{Taranchuk2015}). Благодаря этому можно предотвратить формирование у пользователей мозаичной картины Мира как множества слабо связанных между собой дисциплин. А это, в свою очередь, означает существенное повышение качества образования, которое абсолютно необходимо для качественной эксплуатации компьютерных систем следующего поколения --- \textit{семантических компьютерных систем}.\\
        Пользователи и, первую очередь, разработчики \textit{Экосистемы OSTIS} должны иметь высокий уровень:
        \begin{scnitemize}
            \item Математической культуры (культуры формализации) при построении формальной модели среды, в которой функционирует интеллектуальная система, формальных моделей решаемых ею задач и формальных моделей различных используемых ею способов решения задач.
            \item Системной культуры, позволяющей адекватно оценивать качество разрабатываемых систем с точки зрения общей теории систем и, в частности, оценивать общий уровень автоматизации, реализуемый с помощью этих систем. Системная культура предполагает стремление и умение избегать эклектики, стремление и умение обеспечить качественную стратифицированность, гибкость, рефлексивность, а также качественное сопровождение, высокий уровень обучаемости и комфортный пользовательский интерфейс разрабатываемых систем.
            \item Технологической культуры, обеспечивающей совместимость разрабатываемых систем и их компонентов, а также постоянное расширение библиотеки многократно используемых компонентов создаваемых систем и предполагающей высокий уровень проектной дисциплины.
            \item Умения работать в команде разработчиков наукоемких систем, что предполагает высокий уровень умения работать на междисциплинарных стыках, высокий уровень коммуникабельности и \uline{договороспособности}, т.е. способности не столько отстаивать свою точку зрения, сколько согласовывать ее с точками зрения других разработчиков в интересах развития \textit{Экосистемы OSTIS}.
            \item Активности и ответственности за общий результат --- высокие темпы эволюции \textit{Экосистемы OSTIS} в целом.
        \end{scnitemize}
        Таким образом высокие темпы эволюции \textit{Экосистемы OSTIS} обеспечиваются не только профессиональной квалификацией пользователей (знаниями о \textit{Технологии OSTIS}, о текущем состоянии и проблемах \textit{Экосистемы OSTIS} и навыками использования \textit{Технологии OSTIS} и интеллектуальных систем, входящих в \textit{Экосистему OSTIS}), но и соответствующими человеческими качествами. Очевидно, что современный уровень \uline{договороспособности, активности и ответственности} не может быть основой для эволюции таких систем, как \textit{Экосистема OSTIS}.\\
        Поддержка совместимости \textit{Экосистемы OSTIS} с ее пользователями осуществляется следующим образом:
        \begin{scnitemize}
            \item в каждую \textit{ostis-систему} включаются встроенные ostis-системы, ориентированные
            \begin{scnitemizeii}
                \item на перманентный мониторинг деятельности конечных пользователей и разработчиков этой \textit{\mbox{ostis-системы}},
                \item на анализ качества и, в первую очередь, корректности этой деятельности,
                \item на перманентное ненавязчивое персонифицированное обучение, направленное на повышение качества деятельности пользователей, т.е. на повышение их квалификации;
            \end{scnitemizeii}
            \item в состав \textit{Экосистемы OSTIS} включаются \textit{ostis-системы}, специально предназначенные для обучения пользователей \textit{Экосистемы OSTIS} базовым общепризнанным знаниям и навыкам решения соответствующих классов задач. Сюда входят и знания, соответствующие уровню среднего образования, и знания соответствующие базовым дисциплинам высшего образования в области информатики (и, в том числе, в области искусственного интеллекта), и базовые знания по \textit{Технологии OSTIS} и об \textit{Экосистеме OSTIS}.
        \end{scnitemize}}
    
    \scnheader{Экосистема OSTIS}
    \scntext{обоснование}{Проблема создания рынка совместимых компьютерных систем --- \textbf{вызов современной науке и технике}. От ученых, работающих в области искусственного интеллекта требуется умение коллективно работать над решением междисциплинарных проблем и доводить эти решения до общей интегрированной теории интеллектуальных систем, предполагающей интеграцию всех направлений искусственного интеллекта, и до технологий, доступных широкому кругу инженеров. От инженеров интеллектуальных систем требуется активное участие в развитии соответствующих технологий и существенное повышение уровня математической, системный, технологической и организационно-психологической культуры.\\
        Но главной задачей здесь является снижение барьера между научными исследованиями в области искусственного интеллекта и инженерией в области разработки интеллектуальных систем. Для этого наука должна стать конструктивной и ориентированной на интеграцию своих результатов в форме комплексной технологии разработки интеллектуальных систем, а инженерия, осознав наукоемкость своей деятельности, должна активно участвовать в разработке технологий.\\
        Особый акцент в \textit{Экосистеме OSTIS} делается на постоянный процесс согласования \textit{онтологий} (и, в первую очередь, на согласование семейства всех используемых понятий и терминов, соответствующих этим понятиям) между \uline{всеми} (!) активными субъектами \textit{Экосистемы OSTIS} --- между всеми \textit{ostis-системами} и всеми пользователями.\\
        При наличии \textit{ostis-систем}, являющихся персональными ассистентами пользователей во взаимодействии с \textit{Экосистемой OSTIS}, вся эта Экосистема будет восприниматься пользователями как единая интеллектуальная система, объединяющая все имеющиеся в \textit{Экосистеме OSTIS} информационные ресурсы и сервисы.\\
        Принципы организации \textit{Экосистемы OSTIS} создают все необходимые условия для привлечения к разработке и совершенствованию \textit{Технологии OSTIS} научные, организационные и финансовые ресурсы, которые будут направлены на развитие методов и средств искусственного интеллекта и на формирование рынка семантически совместимых интеллектуальных систем.}
    \begin{scnrelfromlist}{автор}
    \scnitem{Загорский А.С.}
    \scnitem{Голенков В.В.}
    \scnitem{Шункевич Д.В.}
    \end{scnrelfromlist}
\scntext{аннотация}{Рассмотрена структура цифровой экосистемы интеллектуальных компьютерных систем на основе Технологии OSTIS. Уточнена формальная трактовка таких понятий как ostis-система, ostis-сообщество, выделена
типология ostis-систем, что в совокупности позволяет определить       структуру Экосистемы OSTIS.}
\begin{scnrelfromlist}{дочерний раздел}
    \scnitem{§ 7.3.1. Иерархическая система взаимодействующих ostis-сообществ}
    \scnitem{§ 7.3.2. Семантически совместимые интеллектуальные ostis-порталы знаний}
    \scnitem{§ 7.3.3. Семантически совместимые интеллектуальные корпоративные ostis-системы различного
назначения}
 \scnitem{§ 7.3.4. Персональные ostis-ассистенты пользователей}
\end{scnrelfromlist}
\begin{scnrelfromlist}{ключевой знак}
    \scnitem{экосистема OSTIS}
\end{scnrelfromlist}
\begin{scnrelfromlist}{ключевое понятие}
    \scnitem{распределённая система}
    \scnitem{цифровая экосистема}
\end{scnrelfromlist}
\begin{scnrelfromlist}{библиографическая ссылка}
    \scnitem{Zagorskiy А..Princ fItEoNGICS-2022art}
    \scnitem{Van B..KnowlSiaENoP-2005art}
    \scnitem{Mack R..KnowlPatEDKW-2001art}
    \scnitem{Ameri F..ProduLMCtKL-2005art}
    \scnitem{Gerhard D.ProduLMCoC-2017art}
    \scnitem{Meurisch C..ReferMoNGDP-2017art}
    \scnitem{Meurisch C..ExploUEoPAI-2020art}
    \scnitem{Jeni P..CanDPAP-2022art}
    \scnitem{Awais A.Proac iIPAaSB-2022art}
    \scnitem{Briscoe B..DigitESOoEA-2008art}
    \scnitem{Boley H..DigitEPaS-2007art}
    \scnitem{Masaharu T..aRevie otECTCE-2018art}
    \scnitem{Mohseni M..aModelDAfSW-2021art}
    \scnitem{Berners-Lee T..SemanW-2001art}
    \scnitem{Kirrane S..PrivaSaPaRoP-2018art}
    \scnitem{McCooI R.Rethi tSWP2-2006art}
    \scnitem{Nacer H..SemanWSSAC-2014art}
    \scnitem{Burstrom T..SoftwENaitFaDS-2022art}
\end{scnrelfromlist}
\scnheader{цифровая экосистема}
    \scnidtf{совокупность цифровых продуктов и сервисов, которые
взаимодействуют друг с другом и с внешней средой, образуя единую среду обитания. }
 \scnidtf{децентрализованная структура, которая обеспечивает более гибкое и устройчивое управление, нежели полностью иерархически контролируемые системы.}
 \begin{scnrelfromlist}{источник}
    \scnitem{Masaharu T..aRevie otECTCE-2018art}
\end{scnrelfromlist}
\begin{scnrelfromlist}{влияние}
    \scnitem{ключевой аспект для достижения целей бизнеса и общества}
\end{scnrelfromlist}
\begin{scnrelfromlist}{процесс реализации}
    \scnitem{формирование распределённой системы}
    \begin{scnrelfromset}{недостатки}
    \scnitem{неоптимальность}
    \scnitem{неуправляемость}
    \scnitem{непредсказуемость поведения}
\end{scnrelfromset}
 \begin{scnrelfromset}{преимущества}
    \scnitem{высокий уровень адаптивности}
    \scnitem{высокий уровень устройчивости}
    \scnitem{высокий уровень связности}
\end{scnrelfromset}
\begin{scnindent}
\begin{scnrelfromlist}{источник}
    \scnitem{Briscoe B..DigitESOoEA-2008art}
    \scnitem{Boley H..DigitEPaS-2007art}
    \scnitem{Burstrom T..SoftwENaitFaDS-2022art}
\end{scnrelfromlist}
\end{scnindent}
\end{scnrelfromlist}
\scnheader{традиционные подходы к решению проблемы формирования цифровой экосистемы}
\scntext{возникающая проблема}{низкий уровень интероперабельности таких систем}
\begin{scnindent}
\begin{scnrelfromlist}{источник}
    \scnitem{Li W..DigitECaP-2012art}
\end{scnrelfromlist}
\end{scnindent}
\scntext{возникающая проблема}{неэффективность таких подходов ввиду наличия у каждой из систем специализированного программного интерфейса и формата данных для взаимодействия}
\scnheader{современные подходы к формированию цифровой экосистемы}
\begin{scnrelfromlist}{разбиение}
    \scnitem{открытые стандарты}
    \scnitem{протоколы взаимодействия}
\end{scnrelfromlist}
\scntext{основная цель}{упрощение задача иобеспечения интероперабельности}
\scntext{основная цель}{повышение эффективности и экономической целесообразности проектов цифровой трансформации}
\scntext{основная цель}{снижение временных и финансовых затрат на разработку и поддержку цифровой экосистемы}
\begin{scnindent}
\begin{scnrelfromlist}{источник}
    \scnitem{Mohseni M..aModelDAfSW-2021art}
\end{scnrelfromlist}
\end{scnindent}
\scnheader{технология OSTIS}
\scntext{возможность}{создание цифровых экосистем ввиду эффективного управления данными и знаниями}
\scntext{возможность}{создание цифровых экосистем ввиду автоматической обработки информации}
\scntext{возможность}{создание интеллектуальных систем, позволяющих обмениваться данными и знаниями между собой}
\scnheader{иерархическая система взаимодействующих ostis-сообществ}
\begin{scnrelfromlist}{ключевое понятие}
    \scnitem{ostis-система}
    \scnitem{самостоятельная ostis-система}
    \scnitem{встроенная ostis-система}
    \scnitem{коллектив ostis-систем}
\end{scnrelfromlist}
\begin{scnrelfromlist}{библиографическая ссылка}
    \scnitem{Zagorskiy А..Princ fItEoNGICS-2022art}
\end{scnrelfromlist}
\scnheader{проблемы создания цифровой экосистемы}
\scnsuperset{проблемы, связанные с обеспечением высокого уровня интероперабельности}
\begin{scnindent}
\scntext{Способ решения}{переход к универсальным сообществам индивидуальных интеллектуальных кибернетических систем}
\end{scnindent}
\scnheader{экосистема OSTIS}
\scneq{максимальное иерархическое ostis-сообщество, обеспечивающее комплексную автоматизайию всех видов человеческой деятельности}
\scnidtf{социотехническая экосистема, представляющая собой коллектив взаимодействующих семантических компьютерных систем и осуществляющая перманентную поддержку эволюции и семантической совместимости всех входящих в нее систем, на протяжении всего их жизненного цикла.}
\scnidtf{переход от самостоятельных ostis-систем к коллективам ostis-систем, то есть к
распределенным ostis-системам.}
\scntext{основная цель}{обеспечить постоянную поддержку совместимости компьютерных систем, входящих в Экосистему OSTIS как на этапе их разработки, так и в ходе их эксплутации.}
\scntext{основная цель}{повышение качества информационной
внешней среды для всех субъектов, входящих в состав Экосистемы OSTIS.}
\begin{scnrelfromlist}{основная задача}
    \scnitem{оперативное внедрение всех согласованных изменений стандарта ostis-систем (в том числе, и изменений
систем используемых понятий и соответствующих им терминов)}
    \scnitem{перманентная поддержка высокого уровня взаимопонимания всех систем, входящих в Экосистему OSTIS,
и всех их пользователей}
    \scnitem{корпоративное решение различных комплексных задач, требующих координации деятельности нескольких ostis-систем, а также, возможно, некоторых пользователей}
    \scnitem{обеспечение поддержки информационной экологии человеческого общества}
\end{scnrelfromlist}
\scnsuperset{система}
\begin{scnindent}
\scntext{основное требование}{интенсивно, активно и целенаправленно обучаться, как с помощью учителей-разработчиков, так и самостоятельно.}
\scntext{основное требование}{сообщать всем другим системам о предлагаемых или окончательно утвержденных изменениях в онтологиях
и, в частности, в наборе используемых понятий.}
\scntext{основное требование}{принимать от других ostis-систем предложения об изменениях в онтологиях, в том числе в наборе используемых понятий, для согласования или утверждения этих предложений.}
\scntext{основное требование}{реализовывать утвержденные изменения в онтологиях, хранимых в ее базе знаний.}
\scntext{основное требование}{способствовать поддержанию высокого уровня семантической совместимости не только с другими ostisсистемами, входящими в Экосистему OSTIS, но и со своими пользователями (обучать их, информировать
их об изменениях в онтологиях).
}
\scntext{основное требование}{она должна обладать всеми необходимыми знаниями и навыками для обмена сообщениями и целенаправленной организации взаимодействия с другими ostis-системами, входящими в Экосистему OSTIS.}
\scntext{основное требование}{в условиях постоянного изменения и эволюции ostis-систем, входящих в Экосистему OSTIS, она должна
сама следить за состоянием своей совместимости (согласованности) со всеми остальными ostis-системами,
то есть должна самостоятельно поддерживать эту совместимость, согласовывая с другими ostis-системами
все требующие согласования изменения, происходящие у себя и в других системах.}
\end{scnindent}
\begin{scnrelfromlist}{поддержка совместимости с пользователями}
    \scnitem{в каждую ostis-систему включаются встроенные ostis-системы, ориентированные на перманентный мониторинг деятельности конечных пользователей и разработчиков этой ostis-системы,}
    \scnitem{в каждую ostis-систему включаются встроенные ostis-системы, ориентированные на анализ качества и, в первую очередь, корректности этой деятельности.}
    \scnitem{в каждую ostis-систему включаются встроенные ostis-системы, ориентированные на повышение квалификации пользователей (персонифицированное обучение).}
     \scnitem{в состав Экосистемы OSTIS включаются ostis-системы, специально предназначенные для обучения пользователей Экосистемы OSTIS базовым общепризнанным знаниям и навыкам решения соответствующих классов
задач.
}
\end{scnrelfromlist}
\begin{scnrelfromlist}{основные принципы}
    \scnitem{Экосистема OSTIS представляет собой сеть ostis-сообществ}
    \scnitem{Каждому ostis-сообществу взаимно однозначно соответствует корпоративная ostis-система этого ostisсообщества}
    \scnitem{Каждое ostis-сообщество может входить в состав любого другого ostis-сообщества по своей инициативе. Формально это означает, что корпоративная ostis-система первого ostis-сообщества является членом другого
ostis-сообщества}
     \scnitem{Каждому специалисту, входящему в состав Экосистемы OSTIS ставится во взаимнооднозначное соответствие
его персональный ostis-ассистент, который трактуется как корпоративная ostis-система вырожденного ostisсообщества, состоящего из одного человека.}
\end{scnrelfromlist}
\begin{scnrelfromlist}{уровни иерархии}
    \scnitem{индивидуальные кибернетические системы (индивидуальные ostis-системы и люди, являющиеся конечными
пользователями ostis-систем)}
    \scnitem{иерархическая система ostis-сообществ, членами каждого из которых могут быть индивидуальные ostisсистемы, люди, а также другие ostis-сообщества}
    \scnitem{Максимальное ostis-сообщество Экосистемы OSTIS, не являющееся членом никакого другого ostis-сообщества,
входящего в состав Экосистемы OSTIS}
\end{scnrelfromlist}
\scnheader{ostis-система}
\scnidtf{система, построенная в соответствии с требованиями и стандартами Технологии OSTIS,}
\begin{scnrelfromset}{разбиение}
    \scnitem{самостоятельная ostis-система}
    \scnitem{встроенная ostis-система}
    \scnitem{коллектив ostis-систем}
\end{scnrelfromset}
\begin{scnrelfromset}{классификация по назначению}
    \scnitem{ассистенты конкретных пользователей или конкретных пользовательских коллективов}
    \scnitem{типовые встраиваемые подсистемы ostis-систем}
    \scnitem{системы информационной и инструментальной поддержки проектирования различных компонентов и различных классов ostis-систем}
     \scnitem{системы информационной и инструментальной поддержки проектирования или производства различных
классов технических и других искусственно создаваемых систем}
    \scnitem{порталы знаний по самым различным научным дисциплинам}
    \scnitem{системы автоматизации управления различными сложными объектами (производственными предприятиями,
учебными заведениями, кафедрами вузов, конкретными обучаемыми)}
     \scnitem{интеллектуальные справочные и help-системы}
    \scnitem{интеллектуальные робототехнические системы}
\end{scnrelfromset}
\scnheader{Участники коллектива Экосистемы OSTIS}
\begin{scnrelfromlist}{основная характеристика}
    \scnitem{семантическая совместимость}
    \scnitem{непрерывная индивидуальная эволюция}
    \scnitem{непрерывная поддержка совместимости с другими участниками в ходе своей индивидуальной эволюции}
    \scnitem{способность к децентрализованной коррдинации своей деятельности}
\end{scnrelfromlist}
\scnheader{агент Экосистемы OSTIS}
\begin{scnrelfromset}{разбиение}
    \scnitem{индивидуальная ostis-система Экосистемы OSTIS}
    \begin{scnindent}
    \begin{scnrelfromset}{разбиение}
    \scnitem{самостоятельная ostis-система Экосистемы OSTIS}
    \scnitem{встроенная ostis-система Экосистемы OSTIS}
\end{scnrelfromset}
\end{scnindent}
    \scnitem{пользователь Экосистемы OSTIS}
    \scnitem{ostis-сообщество}
     \begin{scnindent}
     \scnidtf{коллектив ostis-систем}
     \scnidtf{определенный коллектив людей (пользователей и разработчиков соответствующих ostis-систем)}
     \scnidtf{устойчивый
фрагмент Экосистемы OSTIS, обеспечивающий комплексную автоматизацию определенной части коллективной человеческой деятельности и перманентное повышение ее эффективности}
    \begin{scnrelfromset}{разбиение}
    \scnitem{простое ostis-сообщество}
    \scnitem{иерархическое ostis-сообщество}
    \begin{scnindent}
    \scnidtf{ostis-сообщество, по крайней мере одним из членов которого является некоторое другое ostis-сообщество}
    \end{scnindent}
\end{scnrelfromset}
\begin{scnrelfromset}{разбиение}
    \scnitem{минимальное ostis-сообщество}
    \scnitem{коллектив ostis-систем}
    \begin{scnindent}
     \scnidtf{все ostis-сообщества, кроме минимальных ostis-сообществ}
     \end{scnindent}
\end{scnrelfromset}
\end{scnindent}
\end{scnrelfromset}
\begin{scnrelfromset}{правила поведения}
    \scnitem{согласовывать денотационную семантику всех используемых знаков (в первую очередь понятий)}
    \scnitem{согласовывать терминологию, устранять противоречия и информационные дыры}
    \scnitem{постоянно бороться с синонимией и омонимией как на уровне sc-элементов (внутренних знаков), так и на
уровне соответствующих им терминов и прочих внешних идентификаторов (внешних обозначений)}
    \scnitem{каждый агент Экосистемы OSTIS по своей инициативе может стать членом любого ostis-сообщества Экосистемы OSTIS после соответствующей регистрации.}
\end{scnrelfromset}
\scnheader{ostis-ассистент}
 \scnidtf{личный (персональный) ассистент, который взаимно однозначно соответствует каждой персоне, входящей в состав Экосистемы OSTIS}
 \scnidtf{ostis-система, являющаяся персональным ассистентом пользователя в рамках Экосистемы OSTIS}
 \begin{scnrelfromlist}{возможности}
    \scnitem{анализ деятельности пользователя и формирование рекомендаций по ее оптимизации}
    \scnitem{адаптация под настроение пользователя, его личностные качества, общую окружающую обстановку, задачи,
которые чаще всего решает пользователь}
    \scnitem{перманентное обучение самого ассистента в процессе решения новых задач, при этом обучаемость потенциально не ограничена}
    \scnitem{ведение диалога с пользователем на естественном языке, в том числе в речевой форме}
    \scnitem{возможность отвечать на вопросы различных классов, при этом если системе что-то не понятно, то она сама может задавать
встречные вопросы}
    \scnitem{автономное получение информации от всей окружающей среды, а не только от пользователя (в текстовой
или речевой форме)}
\end{scnrelfromlist}
\begin{scnrelfromlist}{достоинства}
    \scnitem{пользователю нет необходимости хранить разную информацию в разной форме в разных местах, вся информация хранится в единой базе знаний компактно и без дублирований}
    \scnitem{благодаря неограниченной обучаемости ассистенты могут потенциально автоматизировать практически любую деятельность, а не только самую рутинную}
    \scnitem{благодаря базе знаний, ее структуризации и средствам поиска информации в базе знаний пользователь может
получить более точную информацию более быстро}
\end{scnrelfromlist}
 \scnheader{минимальное ostis-сообщество}
  \scnidtf{коллектив, состоящий из персоны и соответствующего ей персонального ostis-ассистента}
\scnheader{корпоративная ostis-система}  
\scnidtf{центральная ostis-система, осуществляющая координацию, организацию, а также поддержку эволюции деятельности членов соответствующего ostis-сообщества}
\scnidtf{представитель соответствующего ostis-сообщества в других ostis-сообществах, членом которых оно
является}
\scntext{основное назначение}{организация общего взаимодействия при выполнении самых различных видов и областей человеческой деятельности, которые могут быть либо
полностью автоматизированными, либо частично автоматизированными, либо вообще неавтоматизированными.}
\scnsuperset{база знаний Корпоративной системы Экосистемы OSTIS}
\begin{scnindent}
    \scntext{содержание}{общая формальная теория человеческой деятельности, включающая в себя типологию видов и областей человеческой
деятельности, а также общая методология этой деятельности}
\end{scnindent}
\begin{scnrelfromlist}{принципы действия}
    \scnitem{интеллектуальные подсистемы (агенты) упорядочивают структуру данных таким образом, что актуальная
информация всегда доступна, а устаревшая информация автоматически архивируется или удаляется в соответствии с законами о хранении и защите данных в режиме реального времени}
    \scnitem{запросы к системе выполняются в виде простых инструкций, система помогает менеджерам вводить необходимую информацию, осуществляет частичную или полную автоматизацию обновления информации из баз
данных, доступных через Интернет}
    \scnitem{интеллектуальные подсистемы выполняют структуризацию и классификацию документов и информации для
принятия быстрых и правильных решений, автоматически обрабатывает документы и доступные базы данных
для отбора ключевой информации, необходимой в данный момент и в будущем}
    \scnitem{существующее системное окружение на предприятии может быть легко подключено к системе через открытые
интерфейсы, вся информация остается доступной}
    \scnitem{все ключевые системы данных синхронизируются с основной системой, данные постоянно сравниваются друг
с другом, чтобы избежать потерь}
    \scnitem{вся информация доступна в базе знаний, которая является источником данных для рабочих процессов, отчетности и комплексных проверок}
\end{scnrelfromlist}
\begin{scnrelfromlist}{достоинства внедрения}
    \scnitem{помощь сбора и оценки информации без преднамеренных искажений или ошибок, связанных с человеческим
фактором}
    \scnitem{предоставление возможности полного контроля своих данных}
    \scnitem{система предоставляет только высококачественные, достоверные и актуальные данные}
    \scnitem{цифровое представление всех процессов сообщества обеспечивает интегрированную обработку информации
внутри сообщества, что дает полную прозрачность управления, облегчает доступ ко всей информации и ее
анализ}
    \scnitem{благодаря поддержке интеллектуальных подсистем все необходимые данные из документов, процессов и
внешних источников могут быть извлечены, структурированы и грамотно оценены}
\end{scnrelfromlist}
\begin{scnrelfromlist}{области применения}
    \scnitem{медицина и здравоохранение}
    \scnitem{образовательная деятельность широкого профиля}
    \scnitem{страховой бизнес}
    \scnitem{промышленная деятельность}
    \scnitem{административная деятельность}
     \scnitem{недвижимость}
      \scnitem{транспорт}
\end{scnrelfromlist}
\scnheader{Деятельность в области Искусственного интеллекта, осуществляемая на основе Технологии OSTIS}
\begin{scnrelfromlist}{основной продукт}
    \scnitem{Экосистема OSTIS}
\end{scnrelfromlist}
\begin{scnrelfromlist}{подпроект}
    \scnitem{Проект Метасистемы OSTIS}
    \scnitem{Проект программной реализации абстрактной sc-машины}
     \scnitem{Проект разработки универсального sc-компьютера}
\end{scnrelfromlist}
\scnheader{ostis-система, являющаяся агентом Экосистемы OSTIS} 
\scnsuperset{персональный ostis-ассистент}
\scnsuperset{корпоративная ostis-система}
\scnsuperset{ostis-портал знаний}
\scnsuperset{ostis-система автоматизации проектирования}
\scnsuperset{ostis-система автоматизации производства}
\scnsuperset{ostis-система автоматизации образовательной деятельности}
\begin{scnindent}
\scnsuperset{обучающаяся ostis-система}
\scnsuperset{корпоративная ostis-система виртуальной кафедры}
\end{scnindent}
\scnsuperset{ostis-система автоматизации бизнес-деятельности}
\scnsuperset{ostis-система автоматизации управления}
\begin{scnindent}
\scnsuperset{ostis-система управления проектами соответствующего вида}
\scnsuperset{ostis-система сенсомоторной координации при выполнении определенного вида сложных действий во
внешней среде}
\begin{scnindent}
\scnsuperset{ostis-система управления самостоятельным перемещением}
\scnsuperset{робота по пересеченной местности}
\end{scnindent}
\end{scnindent}
\scnheader{семантически совместимые интеллектуальные ostis-порталы знаний} 
\begin{scnrelfromlist}{ключевое понятие}
    \scnitem{портал знаний}
    \scnitem{ostis-портал знаний}
\end{scnrelfromlist}
\begin{scnrelfromlist}{библиографическая ссылка}
    \scnitem{Van B..KnowlSiaENoP-2005art}
    \scnitem{Mack R..KnowlPatEDKW-2001art}
\end{scnrelfromlist}
\scnheader{портал знаний}
\scnidtf{один из способов создания централизованного доступа к информации, которая может быть необходима для решения задач, связанных с работой в организации}
\scntext{содержание}{информация, связанная с процессами, документацией, процедурами, обучающими материалами и ответами на часто задаваемые вопросы}
\scntext{основное преимущество}{х способность к сбору и хранению информации из
различных источников, таких как базы данных, системы управления документами, системы управления проектами
и так далее.}
\begin{scnrelfromlist}{проблемы при создании}
    \scnitem{необходимость обеспечить корректное и структурированное хранение информации, ее поиск и обновление}
    \scnitem{необходимость учитывать потребности пользователей и обеспечить удобный и интуитивно понятный интерфейс}
\end{scnrelfromlist}
\begin{scnrelfromlist}{основные цели}
    \scnitem{ускорение погружения каждого человека в новые для него области при постоянном сохранении общей целостной картины Мира (образовательная цель)}
    \scnitem{фиксация в систематизированном виде новых результатов так, чтобы все основные связи новых результатов
с известными были четко обозначены}
     \scnitem{автоматизация координации работ по рецензированию новых результатов}
      \scnitem{автоматизация анализа текущего состояния базы знаний.}
\end{scnrelfromlist}
\begin{scnrelfromlist}{преимущества создания и реализации}
    \scnitem{способ существенного повышения темпов эволюции научно-технической деятельности}
    \scnitem{создание основы для новых принципов организации научной деятельности, в которой результатами этой деятельности являются не статьи, монографии, отчеты и другие научно-технические документы, а фрагменты глобальной
базы знаний, разработчиками которых являются свободно формируемые научные коллективы, состоящие из специалистов в соответствующих научных дисциплинах}
\scnitem{осуществляется как координация процесса рецензирования новой научно-технической информации, поступающей от научных работников
в базы знаний этих порталов, так и процесс согласования различных точек зрения ученых (в частности, введению и
семантической корректировке понятий, а также введению и корректировке терминов, соответствующих различным
сущностям)}
\end{scnrelfromlist}
\scnheader{ostis-портал знаний}
\begin{scnrelfromlist}{преимущества реализации}
    \scnitem{использование методов семантической обработки информации, что позволяет более точно и эффективно
организовывать и искать информацию на портале знаний}
    \scnitem{высокий уровень гибкости и расширяемости, что позволяет адаптировать ostis-порталы знаний под различные
нужды и требования пользователей}
\scnitem{автоматическая интеграция ostis-порталов знаний с другими ostis-системами в рамках Экосистемы OSTIS,
что позволяет создать централизованный доступ к информации из различных источников}
 \scnitem{возможность создания персонализированного ostis-портала знаний, который учитывает интересы и потребности каждого пользователя, что позволяет более эффективно использовать знания ostis-систем}
 \scnitem{возможность производить ostis-порталы знаний быстро и с минимальными затратами благодаря использованию существующих компонентов и инструментов}
\end{scnrelfromlist}
\begin{scnrelfromlist}{пример}
    \scnitem{Метасистема OSTIS}
\end{scnrelfromlist}
\scnheader{Семантически совместимые интеллектуальные корпоративные ostis-системы различного назначения}
\begin{scnrelfromlist}{ключевое понятие}
    \scnitem{корпоративная система}
    \scnitem{корпоративная ostis-система}
\end{scnrelfromlist}
\begin{scnrelfromlist}{библиографическая ссылка}
    \scnitem{Ameri F..ProduLMCtKL-2005art}
    \scnitem{Gerhard D.ProduLMCoC-2017art}
\end{scnrelfromlist}
\scnheader{корпоративная система}
\scnidtf{программное решение, предназначенное для автоматизации бизнеспроцессов и управления ресурсами и данными внутри организации}
\scnsuperset{управление отношения с клиентами}
\scnsuperset{управление контентом}
\scnsuperset{управление проектами}
\scnsuperset{управление ресурсами предприятия}
\scnsuperset{управление документами}
 \scntext{ключевая роль}{обеспечение эффективного управления
бизнес-процессами и ресурсами}
\scntext{ключевая роль}{повышение производительности и качества работы}
\scntext{ключевая роль}{обеспечении прозрачности и оперативности принятия решений на основе актуальных данных}
\begin{scnrelfromlist}{основные цели}
    \scnitem{автоматизация многих рутинных задач, таких как обработка заказов, управление складом, учет финансовых
операций и так далее. Это позволяет сократить время на выполнение задач и уменьшить количество ошибок}
    \scnitem{сбор, хранение и обработка данных о бизнес-процессах и ресурсах организации. Это позволяет увеличить
точность и оперативность принятия решений, а также обеспечить прозрачность в управлении организацией}
    \scnitem{эффективное управление ресурсами организации, такими как финансы, трудовые ресурсы, материальные и
технические ресурсы и так далее. Это позволяет сократить затраты на управление ресурсами и повысить
эффективность их использования}
    \scnitem{управление отношениями с клиентами, автоматизация процессов продаж и обслуживания, а также анализ
данных о клиентах. Это позволяет повысить удовлетворенность клиентов и увеличить объемы продаж}
    \scnitem{управление проектами, планирование и отслеживание выполнения работ, управление ресурсами и расписание
проектов. Это позволяет повысить эффективность выполнения проектов, уменьшить сроки выполнения работ
и снизить затраты на проекты}
    \scnitem{управление документами, контроль версиями, автоматизация процессов редактирования и утверждения документов. Это позволяет повысить эффективность работы с документами и обеспечить безопасность их
хранения и передачи}
\end{scnrelfromlist}
\begin{scnrelfromlist}{проблемы внедрения и эксплуатации}
    \scnitem{Внедрение корпоративных систем может быть дорогостоящим и трудоемким процессом, который требует
значительных ресурсов и экспертизы. Кроме того, многие системы могут потребовать изменения бизнеспроцессов и требовать адаптации культуры организации}
    \scnitem{Корпоративные системы могут столкнуться с проблемами совместимости с другими системами, используемыми в организации. Это может привести к проблемам с обменом данными и снижению эффективности
работы}
    \scnitem{Корпоративные системы могут стать мишенью для кибератак, поэтому важно обеспечить безопасность хранения и передачи данных, используемых в системах}
    \scnitem{Корпоративные системы могут потребовать значительных затрат на обслуживание и поддержку, включая
установку обновлений, устранение ошибок и техническую поддержку}
    \scnitem{Внедрение новых корпоративных систем может потребовать обучения персонала, что может быть трудоемким
и затратным процессом}
    \scnitem{Внедрение корпоративных систем может потребовать изменения бизнес-процессов, что может быть сложным
и вызвать сопротивление со стороны сотрудников}
\end{scnrelfromlist}
\scnheader{создание семантически совместимых интеллектуальных корпоративных систем}
\begin{scnrelfromlist}{основные цели}
    \scnitem{высокая степень гибкости}
    \scnitem{высокая степень масштабируемости}
    \scnitem{высокая степень автоматизации}
    \scnitem{высокая степень интеграции}
    \begin{scnrelfromlist}{средства для достижения целей}
    \scnitem{аналитика данных}
    \scnitem{машинное обучение}
    \scnitem{искусственный интеллект}
    \scnitem{технологии распределённых вычислений}
\end{scnrelfromlist}
\end{scnrelfromlist}
\scnheader{Персональные ostis-ассистенты пользователей}
\begin{scnrelfromlist}{ключевое понятие}
    \scnitem{персональный ассистент}
    \scnitem{персональный ostis-ассистент}
\end{scnrelfromlist}
\begin{scnrelfromlist}{библиографическая ссылка}
    \scnitem{Meurisch C..ReferMoNGDP-2017art}
    \scnitem{Meurisch C..ExploUEoPAI-2020art}
     \scnitem{Jeni P..CanDPAP-2022art}
    \scnitem{Awais A.Proac iIPAaSB-2022art}
\end{scnrelfromlist}
\scnheader{общество}
\scntext{основная задача}{обеспечивать персональную поддержку каждому человеку, учитывая его индивидуальные особенности}
\begin{scnindent}
     \begin{scnrelfromlist}{основные цели}
    \scnitem{достижение максимального уровня физического здоровья, активности и долголетия}
    \scnitem{достижение максимального уровня физического комфорта, личного пространства и материального благосостояния}
    \scnitem{достижение максимального уровня социального комфорта и защиты прав и свобод}
     \begin{scnrelfromlist}{способы достижения}
    \scnitem{персональный мониторинг каждой личности по всем направлениям}
    \scnitem{диагностика и устранение нежелательных отклонений}
    \scnitem{оказание своевременной персональной помощи в уточнении направлений дальнейшей эволюции каждой личности}
\end{scnrelfromlist}
\end{scnrelfromlist}
\end{scnindent}
\scnheader{цифровой персональный ассистент}
\scnidtf{программа, основанная на технологиях искусственного интеллекта и
машинного обучения, которая помогает пользователям в выполнении повседневных задач, таких как составление
расписания, управление контактами, поиск информации, напоминание о важных событиях}
\begin{scnindent}
\begin{scnrelfromlist}{источник}
    \scnitem{Meurisch
C..ReferMoNGDP-2017art}
 \scnitem{Meurisch C..ExploUEoPAI-2020art}
 \scnitem{Jeni P..CanDPAP-2022art}
 \scnitem{Awais A.Proac iIPAaSB2022art}
\end{scnrelfromlist}
\end{scnindent}
\scntext{основное требование}{должен быть семантически совместимыми с целью понимания друг друга}
\scntext{основное требование}{должен обладать способностью самостоятельно взаимодействовать в рамках
различных корпоративных систем, представляя интересы своих пользователей}
\scntext{основная проблема}{необходимость точного понимания запросов и задач, поступающих от пользователя}
\scnheader{Заключение к Главе 7.3.}
\scnheader{Экосистема OSTIS}
\scnidtf{саморазвивающаяся сеть ostis-систем, которая обеспечивает комплексную
автоматизацию всевозможных видов и областей человеческой деятельности}
\scnidtf{следующий этап развития человеческого общества, обеспечивающий существенное
повышение уровня общественного, коллективного интеллекта путем преобразования человеческого общества в
экосистему, состоящую из людей и семантически совместимых интеллектуальных систем}
\scnidtf{предлагаемый подход к реализации smart-общества или Общества 5.0, построенного на основе Технологии OSTIS}
\scntext{сверхзадача}{комплексная автоматизация всех видов человеческой деятельности (только тех видов деятельности, автоматизация которых целесообразна) и существенное повышение
уровня интеллекта различных человеко-машинных сообществ и всего человеческого общества в целом}


\begin{scnrelfromlist}{автор}
    \scnitem{Загорский А.С.}
    \scnitem{Таранчук В. Б.}
    \scnitem{Шункевич Д.В.}
     \scnitem{Соловьев А. М.}
      \scnitem{Коршунов Р. А.}
       \scnitem{Савенок В. А.}
\end{scnrelfromlist}
\scntext{аннотация}{В главе описываются общие принципы интеграции, а также принципы интеграции Экосистемы OSTIS с
разнородными сервисами и структурированными информационными ресурсами. Также описывается интеграция инструментов компьютерной алгебры в ostis-системы. Эта глава будет полезна для специалистов в
области программной инженерии, искусственного интеллекта и системного анализа, которые заинтересованы
в интеграции различных сервисов и ресурсов в интеллектуальные системы.}
\begin{scnrelfromlist}{дочерний раздел}
    \scnitem{ § 7.4.1. Общие принципы интеграции Экосистемы OSTIS с современными сервисами и
информационными ресурсами}
    \scnitem{ § 7.4.2. Интеграция инструментов компьютерной алгебры в приложения OSTIS}
\end{scnrelfromlist}
\begin{scnrelfromlist}{ключевой знак}
    \scnitem{Экосистема OSTIS}
    \end{scnrelfromlist}
    \begin{scnrelfromlist}{ключевое понятие}
    \scnitem{интеграция}
    \scnitem{информационный ресурс}
    \scnitem{сервис}
\end{scnrelfromlist}
\begin{scnrelfromlist}{библиографическая ссылка}
    \scnitem{Valdez O.How tDaDEaPF-2019art}
    \scnitem{Li W..DigitECaP-2012art}
    \scnitem{Caldarola E..Appro tOIfORiK-2015art}
    \scnitem{Bork D..aOpenPfMMCtO-2019art}
    \scnitem{Kroshchanka A..aNeuraSAtCV-2022art}
    \scnitem{RDFCAS-2023el}
    \scnitem{RDB tRML-2012el}
    \scnitem{EasilGHKG-2022el}
    \scnitem{ЧтоТОД-2023эл}
    \scnitem{Дьяконов В.ЭнцикКА-2022кн}
    \scnitem{Аладьев В.З..Введе вСПM22-1999кн}
    \scnitem{Аладьев В.З..МодулПMvMaVV-2011кн}
    \scnitem{List oCAS-2023el}
    \scnitem{СистеКАОС-эл}
    \scnitem{СистеКАM-эл}
    \scnitem{Стахин Н.А.ОсновРсСАС-2008кн}
    \scnitem{AXIOM.tSCS-эл}
    \scnitem{Дьяконов В.MatlaПС-2022кн}
    \scnitem{Особе иПВM-эл}
    \scnitem{MathWMtLoT-el}
    \scnitem{MapleB-el}
    \scnitem{Дьяконов В.Maple вМР-2022кн}
    \scnitem{Аладьев В.З..СистеКАMИП-2006кн}
    \scnitem{MaplePH-el}
    \scnitem{Дьяконов В.MatheПР-2022кн}
    \scnitem{Stephen-2023el}
    \scnitem{tSemanRoPM-2023el}
    \scnitem{Wolfram-2023эл}
    \scnitem{Wolfram-2018el}
    \scnitem{MatheИВ-эл}
    \scnitem{MatheQRH-el}
    \scnitem{Таранчук В.Б.Метод иТРПС-2019ст}
    \scnitem{Таранчук В.Б.ИнтелВАВБ-2019ст}
\end{scnrelfromlist}
\scntext{ключевая цель}{дать
всесторонний анализ и представить аргументы в пользу внедрения современных систем и сервисов в Экосистему
OSTIS.}
\scnheader{интеграция современных сервисов и информационных ресурсов с Экосистемой OSTIS}
\scntext{основная задача}{важнейший элемент для продвижения технологических инноваций в современном мире.}
\scnheader{технология OSTIS}
\scntext{основная задача}{важнейший элемент для продвижения технологических инноваций в современном мире.}
\begin{scnrelfromlist}{преимущества в использовании}
    \scnitem{обеспечивает основу для разработки сложных систем}
    \scnitem{обеспечивает основу для  их беспрепятственной интеграции c
существующими услугами и ресурсами}
\end{scnrelfromlist}
\scnidtf{мощный инструмент для разработки и реализации цифровых экосистем,
которые могут интегрироваться с различными сервисами и информационными ресурсами }
\scnheader{Общие принципы интеграции цифровой экосистемы с современными сервисами и информационными ресурсами}
\begin{scnrelfromlist}{разбиение}
    \scnitem{стандартизация и совместимость, что достигается путем использования стандартизованных протоколов и
форматов обмена данными}
    \scnitem{открытость и доступность цифровой экосистемы для различных участников, что обеспечениватся открытыми и удобными интерфейсами;}
    \scnitem{безопасность и конфиденциальность, что достигается путем использования криптографических методов
защиты данных и контроля доступа к ресурсам}
    \scnitem{автоматизация и масштабируемость, что позволит обеспечить эффективность и производительность цифровой экосистемы при работе с большим количеством сервисов и ресурсов}
    \scnitem{анализ и управление данными, что поможет определить эффективность интеграции и улучшить ее в дальнейшем}
\end{scnrelfromlist}
\scnheader{ostis-система}
\begin{scnrelfromlist}{преимущества в использовании}
    \scnitem{возможность выполнять роль системных интеграторов различных ресурсов и сервисов, реализованных
современными компьютерными системами}
    \scnitem{уровень интеллекта, позволяющий с любой
степенью детализации специфицировать интегрируемые компьютерные системы и, следовательно, достаточно
адекватно понимать, что знает и/или умеет каждая из систем}
\scnitem{способность достаточно качественно
координировать деятельность стороннего ресурса и сервиса и обеспечивать релевантный поиск нужного ресурса
или сервиса}
\scnitem{возможность выполнять роль интеллектуальных help-систем}
\scnitem{высокий уровень интероперабельности}
\scnitem{монолитная структура, что позволяет упростить
процесс внедрения новых сервисов и sc-агентов в ostis-систему}
\end{scnrelfromlist}
\begin{scnrelfromlist}{перспективы развития}
    \scnitem{научиться выполнять миссию активной интероперабельной надстройки над различными
современными средствами автоматизации}
    \scnitem{ставить перед современными средствами автоматизации выполнимые для них задачи, обеспечивая их непосредственное участие в решении сложных комплексных задач и организуя
управление взаимодействием различных средств автоматизации в процессе коллективного решения сложных комплексных задач}
\end{scnrelfromlist}
\scnheader{help-система}
\scnidtf{помощник и консультант по
вопросам эффективной эксплуатации различных компьютерных систем со сложными функциональными возможностями, имеющий пользовательский интерфейс с нетривиальной семантикой и использующийся в сложных
предметных областях}
\scnheader{сервис}
\scnidtf{приложение,
программа, веб-сервис и другие информационные системы, которые предоставляют определенный функционал,
механизм преобразования информации в соответствии с заданной функцией.}
\begin{scnindent}
\scntext{способ представления}{программный интерфейс}
\end{scnindent}
\begin{scnrelfromset}{проблемы интеграции в цифровых экосистемах}
    \scnitem{различные форматы данных и протоколы обмена, которые могут привести к ошибкам при обмене информацией,
что затрудняет взаимодействие между сервисами}
    \scnitem{несовместимость версий приложений, что может привести к конфликтам при обмене информацией}
\scnitem{разные уровни безопасности, что может стать причиной утечки конфиденциальной информации}
\scnitem{отсутствие единой точки управления, что затрудняет мониторинг и управление процессами интеграции}
\scnitem{отсутствие механизмов для анализа и управления информацией, что затрудняет контроль над процессами
обмена информации}
\end{scnrelfromset}
\begin{scnindent}
\begin{scnrelfromlist}{источник}
    \scnitem{Valdez O.How tDaDEaPF-2019art}
    \scnitem{Li W..DigitECaP2012art}
\end{scnrelfromlist}
\begin{scnrelfromlist}{последствия}
    \scnitem{усложнение разработки самих сервисов}
    \scnitem{значительное увеличение
временных и материальных затрат}
\end{scnrelfromlist}
\begin{scnrelfromlist}{способы решения}
    \scnitem{использование стандартных протоколов и форматов обмена данных, таких как XML, JSON и другие, что
позволяет сделать обмен информацией более надежным и универсальным}
    \scnitem{разработка единой схемы данных и правил доступа, что позволяет сделать интеграцию более простой и
управляемой}
\scnitem{реализация механизмов для автоматической обработки ошибок и конфликтов, что позволяет снизить количество ошибок и улучшить надежность цифровой экосистемы}
    \scnitem{использование инструментов и технологий для анализа и управления информацией, таких как системы бизнесаналитики и управления информацией, что позволяет контролировать процессы обмена информации и оптимизировать их работу}
\end{scnrelfromlist}
\begin{scnindent}
\begin{scnrelfromlist}{источник}
    \scnitem{Caldarola E..Appro tOIfORiK-2015art}
    \scnitem{Bork D..aOpenPfMMCtO-2019art}
\end{scnrelfromlist}
\end{scnindent}
\end{scnindent}
\scnheader{интеграция экосистемы OSTIS с сервисом}
\scnidtf{возможность использовать функционал сервиса
для изменения внутреннего состояния базы знаний Экосистемы OSTIS.}
\scnheader{экосистема OSTIS}
\begin{scnrelfromset}{уровни интеграции}
    \scnitem{Полная интеграция}
    \begin{scnindent}
    \scnidtf{исполнение функции сервиса на платформонезависимом уровне, где вся программа исполняется в самой базе знаний Экосистемы OSTIS.}
\end{scnindent}
    \scnitem{Частичная интеграция}
    \begin{scnindent}
    \scnidtf{реализация взаимодействия и изменения состояния базы знаний Экосистемы
OSTIS на этапах исполнения функции сервиса}
\end{scnindent}
\end{scnrelfromset}
\begin{scnindent}
\begin{scnrelfromlist}{минимальные требования для эффективой интеграции}
    \scnitem{спецификация входной конструкции в базе знаний системы: определение структуры в базе знаний системы,
которая будет преобразовываться в формат данных, совместимый с сервисом}
    \scnitem{спецификация выходной конструкции в базе знаний системы: определение структуры в базе знаний системы,
которая будет формироваться из исходной структуры впоследствии преобразования данных сервиса в знания}
    \scnitem{реализация sc-агента, который преобразует конструкцию базы знаний в формат, который может быть использован в сервисе, а также погружать результаты работы сервиса обратно в базу знаний системы в соответствии
со спецификацией}
\end{scnrelfromlist}
\end{scnindent}
\begin{scnrelfromset}{внедрение стороннего сервиса}
    \scnitem{анализ требований к интегрируемому сервису, определение необходимого функционала, форматов входных
и выходных данных, и других характеристик сервиса}
    \scnitem{разработка sc-агента, который будет обеспечивать взаимодействие между базой знаний и сторонним сервисом
в соответствии со спецификацией интеграции}
\scnitem{разработка спецификации интеграции, которая будет определять форматы данных и правила взаимодействия
между базой знаний Экосистемы OSTIS и сторонним сервисом}
    \scnitem{тестирование и отладка}
\scnitem{внедрение в Экосистему OSTIS, что позволит использовать возможности интегрированного стороннего сервиса в различных ostis-системах}
\end{scnrelfromset}
\begin{scnindent}
\begin{scnrelfromlist}{источник}
    \scnitem{Kroshchanka A..aNeuraSAtCV-2022art}
\end{scnrelfromlist}
\end{scnindent}
\begin{scnrelfromset}{методы взаимодействия с с.к.а.}
    \scnitem{Интеграция по принципу "черного ящика", когда в базе знаний ostis-системы присутствует спецификация
используемой функции ядра системы компьютерной алгебры, а также спецификация способа вызова данной
функции (например, указание через какой программный интерфейс осуществляется взаимодействие с данной
внешней системой). Такой вариант интеграции является наиболее простым в плане реализации и в целом
обладает перечисленными выше достоинствами. В то же время, данный вариант обладает и недостатком,
связанным с тем, что ostis-система не содержит средств анализа и объяснения того, как был выполнен
конкретный шаг решения задачи, реализуемый используемой функцией с.к.а}
    \scnitem{Более тесная интеграция, при которой конкретная функция по-прежнему остается частью сторонней с.к.а.,
когда в базу знаний ostis-системы погружается не просто результат ее выполнения, а и всевозможная его
спецификация, например, объяснение хода решения задачи, указание конкретных алгоритмов и формул, которые могут быть задействованы в решении, описание возможных альтернативных вариантов решения, оценка
эффективности решения и так далее. В данном варианте интеграции ostis-система получает больше возможностей по анализу и объяснению хода решения задачи. (Следует отметить, что конкретно в с.к.а. Wolfram
Mathematica уже присутствуют подробные пояснения хода решения задачи и допустим режим пошагового
выполнения).}
\scnitem{Полная интеграция, при которой осуществляется трансляция используемых функций системы компьютерной
алгебры с внутреннего языка этой системы в ostis-систему. Данный вариант является наиболее трудоемким и
сложным с точки зрения актуализации реализации возможностей систем компьютерной алгебры в соответствующих ostis-системах с учетом их постоянного развития. В то же время такой вариант интеграции по сравнению
в двумя предыдущими обладает важным достоинством — он обеспечивает платформенную независимость
полученного решения и позволяет использовать при решении конкретной задачи все достоинства предлагаемых в рамках Технологии OSTIS подходов, в частности, возможность многопользовательской, параллельной
обработки знаний и возможность оптимизации плана решения задачи или его фрагментов непосредственно в
ходе решения}
\end{scnrelfromset}
\scnheader{sc-агент}
\begin{scnrelfromset}{обобщённый алгоритм}
    \scnitem{извлечение из базы знаний необходимых структур знаний, соответствующих требованиям функционального
сервиса}
    \scnitem{преобразование извлеченных знаний в формат, необходимый для подачи на вход функциональному сервису}
\scnitem{отправка запроса на функциональный сервис и ожидание его ответа}
    \scnitem{формирование структур знаний на основе полученных данных от функционального сервиса}
\scnitem{погружение новых структур знаний в базу знаний интеллектуальной системы, с целью обеспечения их
дальнейшей использования}
\end{scnrelfromset}
\scntext{способ внедрения}{реализация отдельной ostis-системы, в рамках которой будет
интегрирована функция сервиса.}
\begin{scnindent}
\begin{scnrelfromlist}{основные преимущества}
    \scnitem{возможность перейти к использованию микросервисной архитектуры, что
характеризуется распределенным взаимодействием ostis-систем.}
\scnitem{гибкость}
\scnitem{возможность масштабирования}
\end{scnrelfromlist}
\end{scnindent}
\scnsuperset{абстрактный sc-агент}
\begin{scnindent}
\begin{scnrelfromlist}{этапы работы}
    \scnitem{интеграция с использованием готовых правил}
    \begin{scnindent}
 \scnidtf{применение готовых правил интеграций, хранящихся в базе знаний, ко всем сгенерированным тройкам}
\end{scnindent}
\scnitem{интеграция с сохранением исходной схемы}
 \begin{scnindent}
 \scnidtf{преобразование оставшихся троек с сохранением той структуры отношения, в которой находились участвовавшие в нём сущности}
\end{scnindent}
\scnitem{дополнительные преобразования}
\begin{scnindent}
 \scnidtf{оставшиеся интеграционные преобразования, которым не нашлось места в предыдущих
пунктах, но которые необходимы для завершения процесса интеграции}
\end{scnindent}
\end{scnrelfromlist}
\end{scnindent}
\scnheader{микросервисная архитектура ostis-систем}
\begin{scnrelfromlist}{случаи использования}
    \scnitem{функциональный сервис обладает сложной структурой}
\scnitem{требуется масштабирование и гибкость всей системы}
\end{scnrelfromlist}
\scnheader{интеграция интеллектуальной системы с информационными источниками}
\begin{scnrelfromlist}{основные причины}
    \scnitem{Обеспечение полноты и точности данных: Интеллектуальная система создается для обработки больших объемов данных и принятия решений на их основе. Информационные источники являются основой для этих
данных, и интеграция системы с ними гарантирует полноту и точность данных}
\scnitem{Уменьшение времени и усиление эффективности работы: Интеграция информационных источников в интеллектуальную систему обеспечивает быстрый и удобный доступ к нужной информации. Это уменьшает время
на поиск и обработку данных, что повышает эффективность работы системы и уменьшает количество ошибок}
\scnitem{Расширение возможностей системы: Информационные источники обладают большим количеством данных,
которые могут быть полезны для работы интеллектуальной системы. Интеграция системы с различными
источниками расширяет возможности системы и позволяет ей повысить качество работы.}
\scnitem{Повышение надежности: Разнообразные источники данных обеспечивают резервирование и возможность
сравнения, что позволяет интеллектуальной системе работать более надежно и безопасно в случае сбоя
одного или нескольких источников.}
\scnitem{Улучшение качества прогнозирования: Интеграция информационных источников с интеллектуальной системой способствует улучшению качества прогнозирования, так как позволяет объединять данные из различных
источников и анализировать их вместе для получения более точных результатов.}
\end{scnrelfromlist}
\scnheader{источники получения информации}
\begin{scnrelfromlist}{случаи использования}
    \scnitem{Интернет — сайты, блоги, форумы, социальные сети, новостные порталы и другие ресурсы в сети.}
\scnitem{Книги и учебники — доступные в библиотеках, книжных магазинах или в электронном формате.}
  \scnitem{СМИ — телевизионные программы, радио, газеты, журналы и другие источники новостей.}
\scnitem{Официальные документы и отчеты — включая законы, правительственные статистические данные, отчеты об
исследованиях и другие официальные документы}
\end{scnrelfromlist}
\scnheader{интеграция с ресурсами на основе RDF}
\scnidtf{Один из наиболее востребованных подходов в направлении пополнения базы знаний Экосистемы OSTIS новыми знаниями}
\scnheader{RDF-модель}
\begin{scnrelfromlist}{основные принципы}
    \scnitem{играет ключевую роль в
организации связей между различными ресурсами}
\scnitem{используется для описания ресурсов в сети Интернет}
  \scnitem{является основой для построения семантических веб-приложений, таких как Linked Open Data.
}
\end{scnrelfromlist}
\begin{scnrelfromset}{основная структура}
    \scnitem{субъект}
\scnitem{предикат}
  \scnitem{объект}
\end{scnrelfromset}
\begin{scnindent}
\scnidtf{Граф RDF}
\begin{scnindent}
\begin{scnrelfromlist}{основные свойства}
    \scnitem{атемпорален(представляет собой статический снимок информации)}
\scnitem{может выражать информацию о событиях и временных аспектах других сущностей, учитывая соответствующие
термины из словаря}
  \scnitem{определён как математический набор}
\end{scnrelfromlist}
\end{scnindent}
\end{scnindent}
\begin{scnrelfromlist}{поддерживаемые типы данных}
    \scnitem{строковый (string)}
\scnitem{логический (boolean)}
  \scnitem{числовые (integer,
double, float и другие)}
\scnitem{временные}
\end{scnrelfromlist}
\begin{scnrelfromlist}{сфера использования}
    \scnitem{оформление баз знаний в рамках различных
проектов во множестве институтов, университетов и иных организаций}
\scnitem{повышение информативности ссылок
на сайты в результатах поиска в веб-сфере}
  \scnitem{описание свойств страниц, так же позволяющих красиво оформить ссылку на нее в записи пользователя
социальной сети}
\end{scnrelfromlist}
\scnheader{Узел}
\begin{scnrelfromlist}{тип}
    \scnitem{IRI. Представляет собой короткую последовательность символов, идентифицирующую абстрактный или физический ресурс на любом языке мира. IRI представляе собой обобщение URI}
\scnitem{Литерал. Представляя собой структуру, состоящую из лексической формы (UNICODE-строка) и типа данных}
  \scnitem{Пустой узел. Представляет собой локальный идентификатор, который используются в некоторых конкретных
синтаксисах RDF или реализациях хранилища RDF}
\end{scnrelfromlist}
\scnheader{словарь RDF}
\scnidtf{совокупность IRI, ссылающихся на
другие графы с классами, литералами и так далее}
\scnheader{подходы к интеграции информационных ресурсов на основе RDF с
другими системами}
\begin{scnrelfromlist}{разбиение}
    \scnitem{R2RML — это стандарт W3C для выражения настраиваемых отображений из реляционных БД в RDF. Такие отображения предоставляют возможность просматривать существующие реляционные
данные в модели данных RDF, выраженные в структуре и целевом словаре по выбору автора сопоставления}
\begin{scnindent}
\begin{scnrelfromlist}{источник}
    \scnitem{RDB tRML-2012el}
\end{scnrelfromlist}
\end{scnindent}
    \scnitem{R2RML.io — это open-source проект, разрабатываемый с 2013 года. Данная технология
предназначена для генерации базы знаний на основе данных из полуструктурированных источников}
\begin{scnindent}
\begin{scnrelfromlist}{источник}
    \scnitem{EasilGHKG-2022el}
\end{scnrelfromlist}
\end{scnindent}
    \scnitem{“Озеро данных”  — это централизованное хранилище, которое позволяет хранить
все структурированные и неструктурированные данные в любом масштабе. “Семантическое озеро данных”
— это особая форма озер данных, в которых верхний семантический слой обогащает и связывает данные
семантически. Семантический уровень преодолевает разрозненность данных и обеспечивает семантический
поиск по всем данным}
\begin{scnindent}
\begin{scnrelfromlist}{источник}
    \scnitem{ЧтоТОД-2023эл}
\end{scnrelfromlist}
\end{scnindent}
\end{scnrelfromlist}
\begin{scnrelfromlist}{достоинства}
    \scnitem{способность осуществлять интеграцию знаний в своей памяти на высоком уровне}
\scnitem{возможность интегрировать различные виды знаний}
  \scnitem{возможность интегрировать различные модели решения задач}
\end{scnrelfromlist}
\scnheader{компьютерная алгебра}
 \scnidtf{возникшее в середине 20 века и интенсивно развивающееся фундаментальное научное направление}
 \scnidtf{наука об эффективных алгоритмах вычислений математических
объектов}
\begin{scnrelfromset}{разбиение}
    \scnitem{теория}
     \scnitem{технологии}
      \scnitem{программные средства}
\end{scnrelfromset}
\scntext{основной продукт}{программные системы компьютерной алгебры — с.к.а.}
\begin{scnindent}
\begin{scnrelfromlist}{решаемые задачи}
    \scnitem{разработка алгоритмов вычисления топографических инвариантов многообразий}
    \scnitem{разработка алгоритмов вычисления узлов}
    \scnitem{разработка алгоритмов вычисления алгебраических кривых}
    \scnitem{разработка алгоритмов вычисления когомологий различных математических объектов}
    \scnitem{разработка алгоритмов вычисления арифметических инвариантов колец целых элементов в полях алгебраических чисел}
    \scnitem{квантовые алгоритмы, имеющие иногда полиномиальную сложность, тогда как существующие классические алгоритмы
имеют экспоненциальную}
\end{scnrelfromlist}
\scntext{Поэтапная интеграция с Экосистемой OSTIS}{описание спецификации основных
функций выбранной системы компьютерной алгебры средствами Технологии OSTIS}
\scntext{Основное назначение}{работа с математическими выражениями в символьной форме}
\begin{scnrelfromset}{базовые типы данных}
    \scnitem{числа}
    \begin{scnindent}
\begin{scnrelfromlist}{разбиение}
    \scnitem{короткие целые}
    \scnitem{длинные целые}
    \scnitem{рациональные числа}
    \scnitem{комплексные числа}
    \scnitem{алгебраические числа}
\end{scnrelfromlist}
\end{scnindent}
    \scnitem{математические выражения}
      \begin{scnindent}
\begin{scnrelfromlist}{разбиение}
    \scnitem{арифметика}
    \scnitem{функции}
    \scnitem{уравнения}
    \scnitem{производные}
    \scnitem{интегралы}
     \scnitem{векторы}
    \scnitem{матрицы}
    \scnitem{тензоры}
\end{scnrelfromlist}
\end{scnindent}
\scnitem{другие объекты}
  \begin{scnindent}
\begin{scnrelfromlist}{разбиение}
    \scnitem{функциональные, дифференциальные поля, допускающие показательные, логарифмические,
тригонометрические функции}
    \scnitem{матричные кольца}
\end{scnrelfromlist}
\end{scnindent}
\end{scnrelfromset}
\begin{scnrelfromlist}{принцип работы}
    \scnitem{математические объекты (алгебраические выражения, ряды, уравнения, векторы, матрицы и так далее) и
указания, что с ними делать, задаются пользователем на входном языке системы в виде символьных выражений}
    \scnitem{интерпретатор анализирует и переводит символьные выражения во внутреннее представление}
     \scnitem{символьный процессор системы выполняет требуемые преобразования или вычисления и выдает ответ в
математической нотации}
\end{scnrelfromlist}
\begin{scnrelfromlist}{классификационные признаки}
    \scnitem{функциональное назначение}
    \scnitem{тип архитектуры}
     \scnitem{средства реализации}
      \scnitem{область применения}
     \scnitem{интегральные оценки качества}
\end{scnrelfromlist}
\scnsuperset{с.к.а. общего назначения}
\begin{scnindent}
\scntext{основная функция}{обеспечивают
решение задач для большинства основных разделов символьной математики}
\begin{scnrelfromset}{основные примеры}
    \scnitem{Derive}
    \scnitem{Mathematica}
     \scnitem{Maple}
      \scnitem{Macsyma
и ее потомок Maxima}
     \scnitem{Scratchpad и ее потомок Axiom}
     \scnitem{Reduce}
\end{scnrelfromset}
\end{scnindent}
\scnsuperset{специализированные с.к.а.}
\begin{scnindent}
\scntext{основная функция}{Решение задач одного или нескольких смежных разделов символьной математики}
\begin{scnrelfromset}{основные примеры}
    \scnitem{GAP (теория групп)}
    \scnitem{Cadabra (тензорная алгебра)}
     \scnitem{KANT
(алгебра и теория чисел)}
      \scnitem{Singular (полиномиальные вычисления с акцентом на нужды коммутативной алгебры,
алгебраическая геометрия)}
     \scnitem{Calc3D (для работы с 3D матрицами, векторами, комплексными числами)}
     \scnitem{GRTensorII
(дифференциальная геометрия)}
\end{scnrelfromset}
\end{scnindent}
\scnsuperset{Система помощи}
\begin{scnindent}
\begin{scnrelfromset}{разбиение}
    \scnitem{обучающие
материалы с разделением по категориям пользователей}
    \scnitem{интерактивные учебные курсы решения математических
задач в среде системы}
     \scnitem{консультант-репетитор, выполняющий пошаговое решение примеров с поясняющими комментариями}
\end{scnrelfromset}
\end{scnindent}
\begin{scnrelfromlist}{классы по типу архитектуры}
    \scnitem{с.к.а. классической архитектуры: системное ядро + прикладные расширения, примеры: Axiom, Maple,
Mathematica}
    \scnitem{Программный пакет для расширения базовой прикладной математической системы, примеры: ядро Maple для
MATLAB и MathCAD}
     \scnitem{Встраиваемое расширение (плагин) для языка и / или системы программирования, примеры: MathEclipse /
Symja — Java-библиотека}
      \scnitem{Open Source, GNU GPL, мультиплатформные с.к.а., примеры: Maxima (Lisp), PARI/GP (C)}
\end{scnrelfromlist}
\scnsuperset{Справочная система}
\begin{scnindent}
\scntext{основная функция}{содержит и обеспечивает пользователей описаниями функциональных возможностей и демонстрационными примерами работы, информационными сообщениями о текущем состоянии системы,
а также сведениями о математических основах алгоритмов.}
\begin{scnindent}
\scnsuperset{краткая контекстная справка о
функциональном назначении выбранного элемента}
\scnsuperset{информация о синтаксисе и семантике операторов и функций
языка с поясняющими примерами}
\scnsuperset{описание реализованных вариантов решения}
\end{scnindent}
\end{scnindent}
\begin{scnrelfromlist}{классы по типу архитектуры}
    \scnitem{с.к.а. классической архитектуры: системное ядро + прикладные расширения, примеры: Axiom, Maple,
Mathematica}
    \scnitem{Программный пакет для расширения базовой прикладной математической системы, примеры: ядро Maple для
MATLAB и MathCAD}
     \scnitem{Встраиваемое расширение (плагин) для языка и / или системы программирования, примеры: MathEclipse /
Symja — Java-библиотека}
      \scnitem{Open Source, GNU GPL, мультиплатформные с.к.а., примеры: Maxima (Lisp), PARI/GP (C)}
\end{scnrelfromlist}
\begin{scnrelfromlist}{разбиение}
    \scnitem{ядро системы — содержит машинные коды реализаций операторов и встроенных функций с.к.а., обеспечивающих выполнение аналитических (символьных) преобразований математических выражений на основе системы
определенных правил}
\begin{scnindent}
\begin{scnrelfromset}{разбиение}
    \scnitem{реализации операторов}
    \scnitem{реализации встроенных функций, обеспечивающих выполнение аналитических преобразований математических выражений на основе системы определенных правил}
\end{scnrelfromset}
\end{scnindent}
    \scnitem{интерфейсная оболочка — обеспечивают поддержку всех функций, необходимых для информационных и
управляющих взаимодействий между с.к.а. и пользователями (людьми, программами, аппаратными средствами)}
     \scnitem{библиотеки специализированных программных модулей и функций — содержат каталогизированные (по типам обрабатываемых абстрактных объектов — числа, функции, алгебры и тому подобные и/или методам
вычислений — аналитические, численные, смешанные) реализации алгоритмов решения типовых математических задач; они функционально расширяют ядро с.к.а.}
      \scnitem{пакеты расширения — обеспечивают различные формы адаптации с.к.а. к классам математических задач,
внешнему ПО (операционным системам, графическим пакетам и тому подобных) и целям пользователей}
       \scnitem{справочная система — содержит описание функциональных возможностей и примеров работы в с.к.а., информационные сообщения о текущем состоянии системы, а также сведения о математических основах алгоритмов
с.к.а.}
\end{scnrelfromlist}
\begin{scnrelfromlist}{сфера работы}
    \scnitem{различные аппаратные платформы}
    \scnitem{под управлением разных операционных
систем}
\end{scnrelfromlist}
\begin{scnrelfromlist}{задачи, которые можно выполнять в аналитической форме}
    \scnitem{упрощение выражений или приведение к стандартному виду}
    \scnitem{подстановки символьных и численных значений в выражения}
     \scnitem{выделение общих множителей и делителей}
    \scnitem{раскрытие произведений и степеней, факторизацию}
     \scnitem{разложение на простые дроби}
    \scnitem{нахождение пределов функций и последовательностей}
     \scnitem{операции с рядами}
    \scnitem{дифференцирование в полных и частных производных}
     \scnitem{нахождение неопределенных и определенных интегралов}
    \scnitem{анализ функций на непрерывность}
     \scnitem{поиск экстремумов функций и их асимптот}
    \scnitem{операции с векторами}
     \scnitem{матричные операции}
    \scnitem{нахождение решений линейных и нелинейных уравнений}
     \scnitem{символьное решение задач оптимизации}
    \scnitem{алгебраическое решение дифференциальных уравнений}
     \scnitem{интегральные преобразования}
    \scnitem{прямое и обратное быстрое преобразование Фурье}
    \scnitem{интерполяция, экстраполяция и аппроксимация}
    \scnitem{статистические вычисления}
    \scnitem{машинное доказательство теорем}
\end{scnrelfromlist}
\begin{scnrelfromlist}{преимущества использования}
    \scnitem{числовые операции произвольной точности}
    \scnitem{целочисленную арифметику для больших чисел}
     \scnitem{вычисление фундаментальных констант с произвольной точностью}
    \scnitem{поддержку функций теории чисел}
     \scnitem{редактирование математических выражений в двумерной форме}
    \scnitem{построение графиков аналитически заданных функций}
     \scnitem{построение графиков функций по табличным значениям}
    \scnitem{построение графиков функций в двух или трех измерениях}
     \scnitem{анимацию формируемых графиков разных типов}
      \scnitem{использование пакетов расширения специального назначения}
     \scnitem{программирование на встроенном языке}
    \scnitem{автоматическую формальную верификацию}
     \scnitem{синтез программ}
\end{scnrelfromlist}
\scntext{основная особенность}{преимущественно интерактивный характер работы — пользователь не знает заранее ни
размера, ни формы результатов и поэтому должен иметь возможность корректировать ход вычислений на всех
этапах, задавать режим пошагового выполнения с выводом промежуточных результатов}
\begin{scnrelfromlist}{лидеры}
\scnidtf{мощные системы с собственными ядрами, оснащенные развитым
пользовательским интерфейсом и обладающие разнообразными графическими и редакторскими возможностями}
    \scnitem{Mathematica}
    \scnitem{Maple}
\end{scnrelfromlist}
\end{scnindent}
\scnheader{Техника символьных вычислений}
\begin{scnrelfromlist}{Содержание}
    \scnitem{внутреннее представление математического выражения в системе символьных вычислений — синтаксическое
дерево (список списков)}
    \scnitem{суть символьных вычислений (аналитических преобразований) — переписывание терма с помощью последовательного применения правил из определенного пользователем или системой списка}
     \scnitem{преобразование из внешнего представления во внутреннее и обратно обеспечивается дополнительными инструментальными средствами}
\end{scnrelfromlist}
\scnheader{системы компьютерной математики(с.к.м.)}
\begin{scnrelfromset}{разбиение}
    \scnitem{табличные процессоры, например, Microsoft Excel, Lotus Symphony Spreadsheets, Gnumeric, OpenOffice.org Calc}
     \scnitem{системы для статистических расчетов, например, STATISTICA, PASW Statistics (первоначальное название SPSS
Statistics)}
      \scnitem{системы для моделирования, анализа и принятия решений, например, GPSS, AnyLogic, DSS}
      \scnitem{системы компьютерной алгебры}
      \scnitem{универсальные математические системы}
\end{scnrelfromset}
\scnheader{Подход к решению задач интеллектуализации образовательной деятельности}
\scntext{основное преимущество}{При разработке ostis-систем исключается необходимость программировать многие функции, которые уже
реализованы, оттестированы и апробированы в с.к.а. Это принципиально, так как системы компьютерной
алгебры разрабатываются высококвалифицированными специалистами в соответствующих областях, реализация аналогичных функций в ostis-системах может потребовать значительных финансовых и временных
затрат}
\scntext{основное преимущество}{Конкретная ostis-система, использующая отдельные функции с.к.а., благодаря подходу к разработке гибридных решателей задач в Технологии OSTIS (см. Главу 3.3. Агентно-ориентированные модели гибридных
решателей задач ostis-систем) получает возможность самостоятельно самостоятельно планировать ход решения задач при
условии, что некоторые его этапы будут реализованы при помощи присоединяемых функций. С точки зрения
подхода, предлагаемого в рамках Технологии OSTIS, каждая функция системы компьютерной алгебры становится методом решения задач некоторого класса. Этот класс задач описывается в базе знаний ostis-системы
и позволяет ей при решении конкретной задачи самостоятельно делать вывод о целесообразности применения той или иной функции с.к.а. Такая интеграция с ostis-системами позволит устранить сформулированный
ранее возможный недостаток систем компьютерной алгебры (определяется тем, какие с.к.а. используются
— отдельно поясняется ниже в обзоре систем компьютерной математики, условий их применения и доступа
к отдельным компонентам).}
\scnheader{Пользовательский интерфейс}
\begin{scnrelfromset}{виды реализации}
    \scnitem{текстовый (поле ввода символьных
строк, поле вывода символьных строк)}
     \scnitem{графический (ячейки/секции ввода данных / вывода результатов, окна
отображения графиков)}
      \scnitem{командные (меню и кнопки управления с.к.а., панели библиотек функций, индикаторы
состояний с.к.а.)}
\end{scnrelfromset}
\scnheader{Неккомерческие универсальные с.к.а.}
\scnidtf{программный продукт, который с изменениями или без них не имеет ограничений применения, копирования и передачи
другим пользователям, за плату или безвозмездно}
\scntext{пример}{Maxima}
\begin{scnindent}
    \scnidtf{свободная полнофункциональная система компьютерной алгебры, потомок системы Macsyma, разрабатывавшейся в рамках проекта создания искусственного интеллекта в Массачусетском Технологическом Институте
с 1968 по 1982 годы, будучи первой системой аналитических вычислений,
произвела в свое время переворот в компьютерной алгебре и оказала влияние на многие другие системы, в числе
которых Mathematica и Maple}
\scnidtf{консольная программа, которая "отрисовывает" все математические формулы обычными текстовыми символами}
\begin{scnrelfromset}{основные возможности}
    \scnitem{операции с многочленами, списками, векторами, матрицами и тензорами, множествами, рациональными функциями, обобщенными функциями Дирака и Хэвисайда}
     \scnitem{синтаксические, алгебраические и подстановки по шаблону}
      \scnitem{преобразования тригонометрических и выражений со степенями и логарифмами, выносить за скобки, а также
раскрывать скобки, упрощение выражений}
 \scnitem{нахождение пределов в конечных точках (в том числе — поиск односторонних пределов), на бесконечности}
     \scnitem{вычисление сумм ряда}
      \scnitem{дифференцирование, интегрирование}
     \scnitem{нахождение разложений в ряд, вычетов}
      \scnitem{преобразование Лапласа}
     \scnitem{вычисление длины кривых, площади и объема двух-, трех- и многомерных фигур}
\end{scnrelfromset}
\begin{scnrelfromset}{возможности при использовании ядра и дополнительных пакетов}
    \scnitem{уравнения, системы линейных алгебраических уравнений (алгоритмы численного решения задач линейной
алгебры почти соответствуют популярной системе компьютерной математики MATLAB)}
     \scnitem{аналитическими методами обыкновенные дифференциальные уравнения первого и второго порядка, в частности, линейные и нелинейные дифференциальные уравнения первого порядка, линейные дифференциальные
уравнения второго порядка и системы линейных дифференциальных уравнений первого порядка}
      \scnitem{приближенными методами широкий класс обыкновенных дифференциальных уравнений (разложение в ряд
Тейлора и три метода возмущений для решения, классические алгоритмы Рунге-Кутта а также алгоритмы
решения жестких дифференциальных уравнений)}
 \scnitem{интегральные уравнения с фиксированными и переменными пределами интегрирования}
     \scnitem{задачи теории вероятностей, математической статистики и статистической обработки данных}
\end{scnrelfromset}
\begin{scnrelfromset}{основные преимущества}
    \scnitem{производит численные расчеты высокой точности, используя точные дроби, целые числа и числа с плавающей точкой произвольной точности}
      \scnitem{позволяет иллюстрировать функции и статистические данные в двух и трех измерениях}
 \scnitem{реализованы возможности получения качественных иллюстраций, включая параметрические графики кривых и поверхностей, а также графики векторных полей и анимацию.}
     \scnitem{Настройки и управление сгруппированы в простых интерфейсных диалогах, при работе с графическими объектами возможны: вращение, преобразование, увеличение, включение/выключение перспективы и осей.}
      \scnitem{Число настраиваемых атрибутов в системе большое.
Например, типичный трехмерный график имеет около 200 атрибутов, которые можно менять по предпочтениям
пользователя.}
     \scnitem{В рабочем документе можно производить анимацию положения камеры,
цветов, освещения, планов и других атрибутов}
      \scnitem{Графику можно экспортировать в основные векторные и растровые
форматы.}
\scnitem{имеет средства процедурного программирования и программирования по заданному
правилу}
\scnitem{имеет открытую архитектуру, большинство команд, хранящихся в командных файлах (с расширением .mac) могут быть прочитаны и изменены пользователем}
\scnitem{Пользователь может программировать свои команды,
пополняя библиотеку}
\scnitem{успешно работает на всех
современных операционных системах: Windows (готовые сборки доступны на сайте проекта), Linux и UNIX, Mac
OS и даже под управлением Windows CE/Mobile}
\end{scnrelfromset}
\begin{scnrelfromset}{оболочка}
    \scnitem{TeXmacs}
    \begin{scnindent}
    \scnidtf{самостоятельная программа, которую классифицируют как научный WYSIWYG-редактор}
    \begin{scnrelfromset}{основные преимущества}
    \scnitem{текстовый (поле ввода символьных
строк, поле вывода символьных строк)}
     \scnitem{графический (ячейки/секции ввода данных / вывода результатов, окна
отображения графиков)}
      \scnitem{командные (меню и кнопки управления с.к.а., панели библиотек функций, индикаторы
состояний с.к.а.)}
\end{scnrelfromset}
\end{scnindent}
     \scnitem{wxMaxima}
\end{scnrelfromset}
\scnsuperset{отладчик}
\scnsuperset{справочное руководство}
\begin{scnindent}
    \scnidtf{онлайновый справочник, внедрённый в систему, оснащённый средствами поиска с описанием практически всех встроенных функций.}
\end{scnindent}
\end{scnindent}
\scntext{пример}{Axiom}
\begin{scnindent}
    \scnidtf{свободная система компьютерной алгебры.}
    \scnsuperset{среда интерпретатора}
\scnsuperset{компилятор}
\scnsuperset{библиотека, описывающая строгую, математически правильную иерархию типов}
\begin{scnrelfromset}{ветки с открытым исходным кодом}
    \scnitem{OpenAxiom}
     \scnitem{FriCAS}
     \begin{scnindent}
\scntext{основное преимущество}{наличие развитой иерархии типов, соответствующей реальным математическим структурам.}
\end{scnindent}
\end{scnrelfromset}
\end{scnindent}
\scnheader{MATLAB}
     \scnidtf{одна из старейших, тщательно проработанных и проверенных временем систем автоматизации математических расчетов, построенная на расширенном представлении и применении матричных операций.}
     \scntext{разработчик}{The MathWorks, Inc.}
     \begin{scnrelfromset}{основные преимущества}
    \scnitem{включает инструменты разработки сложных программ
с развитым графическим интерфейсом}
     \scnitem{является эффективной средой для проведения исследований, создания
моделей, решения естественнонаучных и инженерных задач}
 \scnitem{поддерживает 64-разрядные микропроцессоры и многоядерные микропроцессоры, например Intel Core 2 Duo и Quad}
 \scnitem{богатая
библиотека команд и свой язык программирования}
\end{scnrelfromset}
\begin{scnrelfromset}{знаковые позиции}
    \scnitem{MATLAB 7.14 R2012a — была последняя версия для поддержки 32-битного Linux}
     \scnitem{MATLAB 8.2 R2013b — добавлен тип данных таблицы, среда выполнения Java обновлена до версии 7}
 \scnitem{MATLAB 8.4 R2014b — добавлены улучшенная пользовательская панель инструментов, новые функции и
пакеты, такие как py (для использования Python), счетчик веб-страниц, гистограммы, TCP-клиент и другие}
 \scnitem{MATLAB 8.6 R2015b — для работы с графиками добавлен новый механизм исполнения (LXE) и новые классы,
такие как графики и орграфы}
 \scnitem{MATLAB 9.1 R2016b — официальный движок MATLAB для Java, новые функции кодирования и декодирования для JSON, добавлен новый “строковый” тип данных; алгоритмы для обработки данных, не помещающихся
в оперативной памяти, включая алгоритмы понижения размерности, описательной статистики, кластеризации
методом к-средних, линейной регрессии, логистической регрессии и дискриминантного анализа; Байесова оптимизация для автоматической настройки параметров алгоритмов машинного обучения, анализ окрестности
компонента (NCA) для выбора функций модели машинного обучения}
 \scnitem{MATLAB 9.5 R2018b — реализовано взаимодействие осей графиков, что обеспечивает эффективный анализ данных с панорамированием, изменением масштаба; добавлены функции: удаление выбросов в массиве,
таблице или расписании; задание локального окружения о каждом элементе во входных данных}
 \scnitem{MATLAB 9.6 R2019a — добавлены Функци задания местоположения отсутствующего значения, обнаружения
выбросов с помощью процентилей; реализованы улучшения для искусственного интеллекта и аналитики}
 \scnitem{MATLAB 9.7 R2019b — включает обновления по искусственному интеллекту (новые возможности позволяют пользователям обучать продвинутые сетевые архитектуры с использованием пользовательских циклов
обучения, автоматического дифференцирования, общих весов и пользовательских функций потерь; пользователи могут создавать генеративные состязательные сети GAN, сиамские сети, вариационные автокодеры и
сети внимания; Deep Learning Toolbox также теперь может экспортировать в сети формата ONNX, которые
объединяют слои CNN и LSTM, и сети, которые включают 3D-слои CNN)}
 \scnitem{MATLAB 9.11 R2021b — добавлены: набор инструментов для статистики и машинного обучения (анализ
сигналов и изображений, предварительная обработка и извлечение параметров с помощью вейвлет-методов
и интерактивных приложений для моделей искусственного интеллекта), кластеризация k-средних в реальной
задачах}
 \scnitem{MATLAB 9.11 R2021b (2021 год)}
  \scnitem{MATLAB 9.13 R2022b включает в себя обновления по искусственному интеллекту, набор инструментов
идентификации системы — создавайте нелинейные модели пространства состояний на основе глубокого
обучения, используя нейронные обыкновенные дифференциальные уравнения (ОДУ); методы машинного
обучения и глубокого обучения также могут представлять нелинейную динамику в нелинейных моделях ARX
и Хаммерштейна-Винера}
\end{scnrelfromset}
\scnsuperset{интерпретатор команд}
\scnsuperset{графическая оболочка}
\scnsuperset{редактор-отладчик}
\scnsuperset{профилировщик}
\scnsuperset{компилятор}
\scnsuperset{символьное ядро с.к.а. Maple для проведения аналитических вычислений}
\scnsuperset{математические
библиотеки и библиотеки Toolboxes, предназначенные для работы со специальными классами задач}
\scnsuperset{Язык MATLAB}
\begin{scnindent}
    \scnidtf{высокоуровневый интерпретируемый язык программирования, включающий основанные на матрицах структуры данных, широкий спектр функций, интегрированную среду разработки, объектно-ориентированные возможности и интерфейсы к программам, написанным на других языках программирования}
    \end{scnindent}
    \begin{scnindent}
    \begin{scnrelfromset}{типы программ}
    \scnitem{функции}
    \begin{scnindent}
     \scnidtf{программы, имеющие входные и выходные аргументы, собственное рабочее пространство для хранения промежуточных результатов вычислений и переменных. }
\end{scnindent}
     \scnitem{скрипты}
     \begin{scnindent}
     \scnidtf{программы, использующие общее рабочее пространство}
\end{scnindent}
\end{scnrelfromset}
\begin{scnindent}
     \scntext{Средство для сборки}{MATLAB Compiler}
\end{scnindent}
\end{scnindent}
\scnheader{Maple}
\begin{scnrelfromset}{разбиение}
    \scnitem{редактор для подготовки и изменения документов и программных модулей}
    \scnitem{ядро алгоритмов и правил преобразования математических выражений}
     \scnitem{численный и символьный процессоры}
      \scnitem{язык программирования}
       \scnitem{многооконный пользовательский интерфейс с возможностью работы в диалоговом режиме}
        \scnitem{справочную систему с пояснениями всех функций и опций}
        \scnitem{систему диагностики}
        \scnitem{библиотеки встроенных и дополнительных функций}
        \scnitem{пакеты функций сторонних производителей}
        \scnitem{поддержку нескольких других языков программирования}
\end{scnrelfromset}
\begin{scnindent}
\begin{scnrelfromlist}{источники}
    \scnitem{Аладьев В.З..МодулПMvMaVV-2011кн}
    \scnitem{Дьяконов В.Maple вМР-2022кн}
\end{scnrelfromlist}
\end{scnindent}
 \scntext{основной документ}{Worksheet}
 \begin{scnindent}
     \begin{scnrelfromset}{основные преимущества}
    \scnitem{Текст можно форматировать на уровне абзацев, оформляя их различными стилями, или символов}
     \scnitem{Секция может состоять из
различных объектов: текстовых комментариев, строк ввода, строк вывода, графиков и других секций (подсекций)}
      \scnitem{Наличие активной строки ввода, воспринимающей команды пользователя}
       \scnitem{реализована концепция "умных" документов, доступны
инструменты, которые обеспечивают создание самодокументируемых контекстных меню}
\scnitem{интерфейс Maple организован в формате интуитивно понятного и дружественного
диалога, что облегчает и ускоряет освоение системы}
\end{scnrelfromset}
\begin{scnrelfromset}{интегрируемые языки}
    \scnitem{Входной или язык общения с системой}
    \begin{scnindent}
        \scnidtf{Интерпретирующий язык сверхвысокого уровня, ориентированный на решение математических задач практически любой сложности в диалоговом режиме.}
    \end{scnindent}
     \scnitem{Язык реализации}
      \scnitem{Язык программирования}
\end{scnrelfromset}
\end{scnindent}
\scnheader{Mathematica}
\scnidtf{система компьютерной алгебрыкомпании Wolfram Research, которая является одним из наиболее мощных и
широко применяемых интегрированных программных комплексов мультимедиа-технологии.}
\scnidtf{платформа для разработки, полностью интегрирующая вычисления в рабочий процесс от
начала до конца, плавно проводя пользователя от первоначальных идей до развернутых индивидуальных и промышленных решений.}
\scnidtf{тип программного средства, предназначенного для манипулирования математическими формулами}
\scntext{Основная цель}{автоматизация
зачастую утомительных и в целом ряде случаев трудных алгебраических преобразований.}
    \begin{scnrelfromset}{преимущества использования}
    \scnitem{реализованы и доступны пользователям практически все возможности аналитических преобразований и численных расчетов}
     \scnitem{поддерживает
работу с базами данных, графикой и звуком}
      \scnitem{дает пользователю возможности работать, анализировать,
манипулировать, иллюстрировать графиками практически все функции чистой и прикладной математики}
\scnitem{обеспечивает расчеты с любой заданной точностью; построение двух- и трехмерных графиков, их анимацию,
рисование геометрических фигур; импорт, обработку, экспорт изображений и звука}
\end{scnrelfromset}
\scntext{Встроенный язык программирования}{Wolfram Language}
\begin{scnindent}
    \scnidtf{Язык программирования, включающий средства создания программ и пользовательских интерфейсов, подключения внешних dll, параллельных вычислений.}
    \scnsuperset{средства визуальноориентированного программирования на основе применения шаблонов математических символов}
    \begin{scnindent}
        \begin{scnrelfromset}{разбиение}
            \scnitem{знаки интеграла}
\scnitem{знаки суммирования}
\scnitem{знаки произведения}
        \end{scnrelfromset}
    \end{scnindent}
    \begin{scnrelfromset}{поддерживаемые форматы}
        \scnitem{3DS}
        \scnitem{ACO}
        \scnitem{Affymetrix}
        \scnitem{AgilentMicroarray}
        \scnitem{AIFF}
        \scnitem{ApacheLog}
        \scnitem{AU}
        \scnitem{AVI}
        \scnitem{Base64}
        \scnitem{BDF}
        \scnitem{Binary}
        \scnitem{Bit}
        \scnitem{BMP}
         \scnitem{BSON}
        \scnitem{Byte}
        \scnitem{BYU}
        \scnitem{BZIP2}
        \scnitem{CDED}
        \scnitem{CDF}
        \scnitem{Character16}
        \scnitem{Character8}
        \scnitem{CIF}
        \scnitem{Complex128}
        \scnitem{Complex256}
        \scnitem{Complex64}
        \scnitem{CSV}
        \scnitem{CUR}
        \scnitem{DAE}
        \scnitem{DBF}
        \scnitem{DICOM}
        \scnitem{DIF}
        \scnitem{DIMACS}
         \scnitem{Directory}
        \scnitem{DOT}
        \scnitem{DXF}
        \scnitem{EDF}
        \scnitem{EML}
        \scnitem{EPS}
        \scnitem{ExpressionJSON}
        \scnitem{ExpressionML}
        \scnitem{FASTA}
        \scnitem{FASTQ}
        \scnitem{FCS}
        \scnitem{FITS}
        \scnitem{FLAC}
        \scnitem{GenBank}
        \scnitem{GeoJSON}
        \scnitem{GeoTIFF}
        \scnitem{GIF}
        \scnitem{GPX}
        \scnitem{Graph6}
        \scnitem{Graphlet}
        \scnitem{GraphML}
        \scnitem{GRIB}
         \scnitem{GTOPO30}
        \scnitem{GXL}
        \scnitem{GZIP}
        \scnitem{HarwellBoeing}
        \scnitem{HDF}
        \scnitem{HDF5}
         \scnitem{HIN}
        \scnitem{HTML}
        \scnitem{HTTPRequest}
        \scnitem{HTTPResponse}
        \scnitem{ICC}
        \scnitem{ICNS}
        \scnitem{ICO}
        \scnitem{ICS}
        \scnitem{Ini}
        \scnitem{Integer128}
        \scnitem{Integer16}
        \scnitem{Integer24}
        \scnitem{Integer32}
        \scnitem{Integer64}
        \scnitem{Integer8}
        \scnitem{JavaProperties}
        \scnitem{JavaScriptExpression}
        \scnitem{JCAMP-DX}
        \scnitem{JPEG}
        \scnitem{JPEG2000}
        \scnitem{JSON}
         \scnitem{JVX}
        \scnitem{KML}
        \scnitem{LaTeX}
        \scnitem{LEDA}
        \scnitem{List}
        \scnitem{LWO}
        \scnitem{M4A}
        \scnitem{MAT}
        \scnitem{MathML}
        \scnitem{MBOX}
        \scnitem{MCTT}
         \scnitem{MDB}
        \scnitem{MESH}
        \scnitem{MGF}
        \scnitem{MIDI}
        \scnitem{MMCIF}
        \scnitem{MO}
        \scnitem{MOL}
        \scnitem{MOL2}
        \scnitem{MP3}
        \scnitem{MPS}
        \scnitem{MTP}
        \scnitem{MTX}
        \scnitem{MX}
        \scnitem{MXNet}
        \scnitem{NASACDF}
        \scnitem{NB}
        \scnitem{NDK}
        \scnitem{NetCDF}
        \scnitem{NEXUS}
        \scnitem{NOFF}
        \scnitem{OBJ}
         \scnitem{ODS}
        \scnitem{OFF}
        \scnitem{OGG}
        \scnitem{OpenEXR}
        \scnitem{Package}
        \scnitem{Pajek}
        \scnitem{PBM}
        \scnitem{PCAP}
        \scnitem{PCX}
        \scnitem{PDB}
         \scnitem{PDF}
        \scnitem{PGM}
        \scnitem{PHPIni}
        \scnitem{PLY}
        \scnitem{PNG}
        \scnitem{PNM}
        \scnitem{PPM}
        \scnitem{PXR}
        \scnitem{PythonExpression}
        \scnitem{QuickTime}
        \scnitem{Raw}
         \scnitem{RawBitmap}
        \scnitem{RawJSON}
        \scnitem{Real128}
        \scnitem{Real32}
        \scnitem{Real64}
        \scnitem{RIB}
        \scnitem{RLE}
        \scnitem{RSS}
        \scnitem{RTF}
        \scnitem{SCT}
        \scnitem{SDF}
        \scnitem{SDTS}
        \scnitem{SDTSDEM}
        \scnitem{SFF}
        \scnitem{SHP}
        \scnitem{SMA}
        \scnitem{SME}
        \scnitem{SMILES}
        \scnitem{SND}
        \scnitem{SP3}
        \scnitem{Sparse6}
         \scnitem{STL}
        \scnitem{String}
        \scnitem{SurferGrid}
        \scnitem{SXC}
        \scnitem{Table}
        \scnitem{TAR}
        \scnitem{ TerminatedString}
        \scnitem{TeX}
        \scnitem{Text}
        \scnitem{TGA}
        \scnitem{TGF}
        \scnitem{TIFF}
        \scnitem{TIGER}
        \scnitem{TLE}
        \scnitem{TSV}
        \scnitem{UBJSON}
        \scnitem{UnsignedInteger128}
        \scnitem{UnsignedInteger16}
         \scnitem{UnsignedInteger24}
        \scnitem{UnsignedInteger32}
        \scnitem{UnsignedInteger64}
        \scnitem{UnsignedInteger8}
        \scnitem{USGSDEM}
        \scnitem{UUE}
        \scnitem{VCF}
        \scnitem{VCS}
        \scnitem{VTK}
        \scnitem{WARC}
        \scnitem{WAV}
         \scnitem{Wave64}
        \scnitem{WDX}
        \scnitem{WebP}
        \scnitem{WLNet}
        \scnitem{Wolfram MathematicaLF}
        \scnitem{WXF}
        \scnitem{XBM}
        \scnitem{XHTML}
        \scnitem{XHTMLMathML}
        \scnitem{XLS}
        \scnitem{XLSX}
        \scnitem{XML}
        \scnitem{XPORT}
        \scnitem{XYZ}
        \scnitem{ZIP}
    \end{scnrelfromset}
    \begin{scnrelfromset}{основные группы функций}
        \scnitem{Structural
Text Manipulation}
\begin{scnindent}
    \begin{scnrelfromset}{основные функции}
        \scnitem{TextCases}
        \begin{scnindent}
            \scnidtf{extract symbolically specified
elements (TextCases[text, form] gives a list of all cases of text identified as being of type form that appear in text)}
        \end{scnindent}
        \scnitem{TextSentences}
         \begin{scnindent}
            \scnidtf{extract a list of sentences (TextSentences["string"] gives a list of the runs of characters identified as
sentences in string)}
        \end{scnindent}
        \scnitem{TextWords}
         \begin{scnindent}
            \scnidtf{extract a list of words (TextWords["string"] gives a list of the runs of characters
identified as words in string)}
        \end{scnindent}
        \scnitem{SequenceAlignment}
         \begin{scnindent}
            \scnidtf{find matching sequences in text; TextStructure["text"] generates a
nested collection of TextElement objects representing the grammatical structure of natural language text}
        \end{scnindent}
    \end{scnrelfromset}
\end{scnindent}
        \scnitem{Text Analysis}
        \scnitem{Natural Language Processing}
    \end{scnrelfromset}
     \begin{scnrelfromset}{основные функции для работы с WDR}
            \scnitem{Entity}
\scnitem{EntityClass}
\scnitem{EntityValue}
\scnitem{Transformations}
\scnitem{Computations on Entity Classes}
\scnitem{Standard Properties}
\scnitem{Specific Domains}
\scnitem{Setting Up Custom Entity
Stores}
\scnitem{Wolfram Data Repository}
\scnitem{Wolfram Data Drop}
\scnitem{Setting Up Custom Entity Stores}
\scnitem{External Knowledgebases}
\scnitem{External Database Connectivity}
\scnitem{Web Content}
\scnitem{Textual Question Answering}
\begin{scnindent}
    \begin{scnrelfromset}{разбиение}
        \scnitem{FindTextualAnswer attempt to find answers to questions from text}
        \scnitem{SemanticInterpretation convert free-form linguistics to Wolfram Language for; SemanticInterpretation["string"]
attempts to give the best semantic interpretation of the specified free-form string as a Wolfram Language expression;
• SemanticImport import data, converting entities etc. to Wolfram Language form}
        \scnitem{SemanticImport import data, converting entities etc. to Wolfram Language form}
        \scnitem{Interpreter interpret input of various types (e.g. “City”, “Date”, etc.); Interpreter attempt to interpret strings of a wide
variety of types; Interpreter[form] represents an interpreter object that can be applied to an input to try to interpret it
as an object of the specified form}
    \end{scnrelfromset}
\end{scnindent}
\scnitem{System Configuration.}
        \end{scnrelfromset}
\end{scnindent}
\begin{scnrelfromset}{поддерживаемые парадигмы}
            \scnitem{функциональное}
\scnitem{структурное}
\scnitem{объектно-ориентированное}
\scnitem{математическое}
\scnitem{логическое}
\scnitem{рекурсивное}
        \end{scnrelfromset}
        \scnsuperset{NB документ}
        \begin{scnindent}
            \scnidtf{Блокнот, с которым пользователь работает в системе.}
            \scnsuperset{интерфейс}
            \begin{scnindent}
                \scnsuperset{много палитр (меню) и графических инструментов для создания, редактирования, просмотра документов, отправки и получения данных к ядру и от ядра}
                \scnsuperset{одна или
набор секций, которые при необходимости можно объединять в группы}
\begin{scnindent}
\scnidtf{Область, где пользователь вводит или размещает
команды, комментарии, объекты мультимедиа, они могут быть исполняемыми и другими; исполняемые секции
обрабатываются — система возвращает результаты.
}
                \scnsuperset{одна или более строк текста}
                \scnsuperset{одна или более строк формул}
                \scnsuperset{цифровой объект аудио}
                \scnsuperset{видео}
            \end{scnindent}
            \end{scnindent}
        \end{scnindent}
        \scntext{Дата выпуска}{июнь 1988 года}
        \begin{scnrelfromset}{знаковые позиции}
            \scnitem{Mathematica 1.0, 23 июня 1988 — первый выпуск Mathematica}
\scnitem{Версия 1.2, август 1989 года — интерфейс под Macintosh, поддержка удаленных ядер, добавлены стандартные
пакеты Statistics и Graphics}
\scnitem{Версия 2.0, январь 1991 года — Notebook интерфейс, протокол MathLink межпроцессорного и сетевого взаимодействия, поддержка звука, поддержка наборов букв не только латинского алфавита, добавлен
ParametricPlot3D}
\scnitem{Версия 2.1, июнь 1992 года — MathLink под Macintosh, поддержка Windows 3.1, синтез звука}
\scnitem{Версия 2.2, июнь 1993 года — реализация для Linux, трехмерное построение контурных графиков, вариационное исчисление, музыка, онлайновые руководства для Windows и браузер функций для Macintosh и NeXT.}
\scnitem{Версия 3.0, сентябрь 1996 года — интерактивная система математического набора, интервальная арифметика}
\scnitem{Версия 4.0, май 1999 года — публикация документов в ряд форматов, прямой импорт и экспорт в более чем
20 форматов графических, звуковых файлов и файлов стандартных данных}
\scnitem{Версия 4.1, ноябрь 2000 года — интеграция с Java посредством J/Link 1.1, поддержка управления в реальном
времени трехмерной графики под Linux и Unix}
\scnitem{Версия 4.2, ноябрь 2002 года — XML-расширения, которые позволяют сохранять файлы и выражения
Mathematica как XML}
\scnitem{Версия 5.0, июнь 2003 года — включена .NET/Link, обеспечивающая полную интеграцию с Microsoft’s .NET
Framework.}
\scnitem{Версия 5.1, октябрь 2004 года — встроенная подключаемость к универсальной базе данных, интегрированная
поддержка веб-сервисов, интерфейс GUIKit и встроенный разработчик программ}
scnitem{Версия 5.2, июнь 2005 года — поддержка многоядерности на основных платформах, MathematicaMark 5.2 —
обеспечивает работу с grid и кластерами}
\scnitem{Версия 6, май 2007 года — язык для интеграции данных, включая автоматическую интеграцию сотен стандартных форматов данных, объединение активных графиков и элементов управления с поточным текстом и
вводом}
\scnitem{Версия 7, ноябрь 2008 года — встроенная поддержка параллельных высокопроизводительных вычислений,
визуализация векторных полей, полная поддержка сплайнов, включая неоднородный рациональный В-сплайн,
интегрированные геодезические и GIS данные}
\scnitem{Версия 8, ноябрь 2010 года — интеграция с Wolfram|Alpha, вейвлет-анализ, встроенная поддержка CUDA и
OpenCL, автоматическое генерирование кода С, расширенная двух- и трехмерная графика, включающая отображение текстур и аппаратное ускорение 3D рендеринга, интерактивный мастер создания CDF-документов}
\scnitem{Версия 9, ноябрь 2012 года — набор функций анализа социальных сетей, включая выявление сообществ;
встроенная связь с Facebook, LinkedIn, Twitter; расширенная поддержка случайных процессов; универсальная
платформа для моделирования систем, которые случайным образом изменяются во времени, включая поддержку построения реализаций, оценивание параметров (калибровку), нахождение распределений временных
срезов; значительно расширенный набор функций для задач теории вероятностей и статистики, включая критерии статистической независимости, новые тесты для проверки статистических гипотез, поддержка взвешенных данных, параметрические и вторичные распределения вероятностей; поддержка графов и сетей, новые и
оптимизированные распределения вероятностей на графах, функции расчета транспортных сетей; встроенная
интеграция с языком R, обеспечение использования кода на языке R в процессе работы в системе Mathematica,
обмена данными между системой Mathematica и средой R путем выполнения R кода непосредственно из блокнота Mathematica; поддержка объемных операций с 3D изображениями; поддержка полного спектра интернет
доступа — доступ к интернету со стороны клиента для обмена информацией с удаленными серверами, и для
работы с программными интерфейсами веб-приложений, асинхронное соединение для программирования в
стиле AJAX.}
\scnitem{Версии 10.0 (2014/12/10), 10.1, 10.2, 10.3, 10.4 (2016/03/02), в указанных и следующих версиях можно пользоваться, как настольным, так и облачным вариантом Mathematica Online. Избранные составляющие: существенные обновления в разделах Математические структуры, Решение дифференциальных уравнений, Структурные
и семантические данные, Расширения базового языка программирования, Вычисления, связанные со временем,
Анализ случайных процессов, Визуализации и графика, Обработка изображений, Инженерные вычисления,
Работа с внешними объектами; новое: Геометрические вычисления (Символьная геометрия, Именные и формульные геометрические области, Сеточные геометрические области), Машинное обучение (функции Classify
и Predict, Машинное обучение с высокой степенью автоматизации, Встроенный комплект классификаторов,
Автоматический анализ временных рядов), Географические вычисления (Географические визуализации, Свойства, связанные с геоположением, Георасчеты с использованием логических объектов)}
\scnitem{Версии 11.0 (2016/10/01), 11.1 (2017/03/18), 11.2 (2017/09/14), 11.3 (2018/03/09). Избранные составляющие,
существенные обновления в разделах: Работа в интернете, Облачные данные, Географические вычисления и
визуализации, Подключение к внешним сервисам (Facebook, Twitter, Instagram, ArXiv, Reddit и многим другим).
3D печать. В части машинного обучения— новые функции позволяют пользователям извлекать признаки моделируемых объектов, уменьшать их размер, группировать данные, оптимизировать гиперпараметры и строить
интерпретируемые модели; функциональные возможности извлечения признаков могут быть использованы
для визуализации данных или для создания семантических расстояний для поисковых систем; доступны:
вычисление нейронных сетей, аудио интеграция и вычислительная обработка лингвистических данных. Значительно усилено машинное зрение (ImageIdentify может определить более 10000 объектов), реализованы
возможности встроенного распознавания объектов на изображениях, обучение собственного идентификатора
изображений. Сформированы базы знаний Образование, сведения о Вселенной, ряд других}
\scnitem{Версии 12.0 (2019/04/16), 12.1 (2020/03/18), 12.2 (2020/12/16), 12.3 (2021/05/20). Существенно модифицированы функции, расширены возможности работы в разделах: Символьные и числовые вычисления, Визуализация
и графика, Геометрия и география, Наука о данных и вычисления, Изображение и аудио (новые функции
и возможности: Вычисление изображений, Аудио вычисления, Вычисление изображений для микроскопии,
Машинное обучение (новые функции и возможности: Суперфункции машинного обучения, Фреймворк нейронной сети, Машинное обучение для изображений, Машинное обучение для аудио, Обработка естественного
языка). Отдельно следует отметить: 25 новых типов сетей, включая популярную языковую модель рендеринга
BERT и генератор преформированного преобразования 2, используемый для систем генерации текста; импорт
новых реализаций нейронных сетей становится немного проще, так как версия 12.1 поддерживает ONNX,
открытый формат для представления моделей машинного обучения; работающие с обработкой изображений, получают дополнительную помощь с такими дополнениями, как FindImageText, который обнаруживает
текст на изображении и маркирует его, аудиофилы могут воспользоваться преимуществами SpeechInterpreter
и SpeechCases. Разработаны и доступны функции работы с базами знания, в частности: открыто, наполнено
хранилище Wolfram Knowledgebase — в открытом доступе в облаке, включает: Язык запросов к базе знаний,
Объекты астрономии и науки о космосе, Биологические и медицинские объекты, Математические объекты,
Географические объекты, Объекты питания и нутрициологии, Физические и химические объекты, Финансовые
и социально-экономические объекты, Культурные и исторические объекты}
\scnitem{Версия 13.0 (2021/12/13), 13.1 (2022/06/29), 13.2 (2022/12/14). Существенно модифицированы функции, расширены возможности работы в разделах: Символьные и числовые вычисления (Непрерывное и дискретное исчисление, Асимптотика), Визуализация и графика (Векторная и комплексная визуализация, Многопанельная
и многоосевая визуализация, Графическое освещение, наполнители и шейдеры), Графы, деревья и геометрия
(Графы и сети, Деревья, Геометрическое вычисление), Оптимизация, дифференциальное уравнение в частных
производных(PDE) и системное моделирование (Математическая оптимизация, PDE моделирование, Системное моделирование и системы управления), Данные и наука о данных (Машинное обучение и нейронные сети,
База знаний, Дата и время, Пространственная статистика), Видео, карты и молекулы (Видео, изображения и
аудио, География, Молекулы и биомолекулярные последовательности).}
        \end{scnrelfromset}
        \begin{scnrelfromlist}{помощь пользователям}
            \scnitem{Центр документации (Documentation Center)}
            \begin{scnindent}
                \scntext{основная задача}{обеспечение помощи пользователям в изучении программного языка и функциональных возможностей системы Mathematica.}
            \end{scnindent}
             \scnitem{Навигатор по функциям (Function Navigator)}
              \scntext{основная задача}{обеспечение возможности иерархического просмотра справочных материалов по категориям}
              \scnitem{Виртуальный учебник
(Virtual Book)}
\scntext{основная задача}{ энциклопедический источник информации для
пользователей всех уровней подготовки, желающих получить практические навыки и более детальную информацию, знания по аспектам взаимодействия с системой и выполнения функций Mathematica}
        \end{scnrelfromlist}
        \scnheader{естественный язык}
        \scnidtf{речь и текст, с помощью которых люди взаимодействуют друг с другом.}
        \scnheader{обработка естественного языка}
         \scnidtf{процесс манипулирования речью текста людьми с помощью Искусственного интеллекта, чтобы компьютеры могли их понимать.}
          \scnheader{обработка естественного языка(NLP)}
          \scnidtf{ процесс манипулирования речью текста людьми с помощью Искусственного интеллекта, чтобы компьютеры могли их понимать.}
          \scntext{Пример извлечения знаний из статей Википедии}{Данные Википедии используют программный интерфейс MediaWiki для извлечения содержимого статей и категорий, а также метаданных из Википедии. Статья может быть указана, как строка или объект языка Wolfram. Извлечение статей, ассоциированных с сущностями языка, обеспечивает функция Wolfram Mathematica TextSentences,
в частности, можно работать с ресурсами Википедии. Ниже на Рисунок. Сведения из Wikipedia о космонавте
Leonov представлен результат выполнения функции TextSentences, с параметрами WikipediaData, Entity, “Person”,
“AlexeiLeonov”.}
\scnheader{Заключение к главе 7.4.}
\scntext{вывод}{Целесообразно интегрировать систему компьютерной алгебры Wolfram Mathematica с ostis-системами, входящими в состав Экосистемы OSTIS}
\scntext{вывод}{При интеграции различных сервисов и информационных ресурсов с Экосистемой OSTIS пользователь может
получить ряд значительных преимуществ: 
\begin{itemize}
    \item[\square] повышение точности и качества данных, получение более точной и полной информации, что может повысить
качество принимаемых решений (это особенно важно в условиях быстро меняющейся среды, когда точность
и качество данных играют решающую роль)
 \item[\square] улучшение управления и контроля процессов обмена информацией, что может помочь в принятии быстрых и
правильных решений
 \item[\square] снижение затрат на разработку и поддержку приложений, улучшение скорости разработки новых приложений,
а также повышение качества их реализации (это связано с тем, что Экосистема OSTIS позволяет использовать
уже существующие компоненты, что сокращает время и затраты на разработку)
\end{itemize}
    \bigskip
    \end{scnsubstruct}
    \scnendcurrentsectioncomment
\end{SCn}

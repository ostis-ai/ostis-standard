\begin{SCn}
    \bigskip
    \scnsectionheader{Предметная область и онтология семантически совместимых ostis-систем автоматизации образовательной деятельности}
    \begin{scnsubstruct}
    \scniselement{раздел базы знаний}
    \scnhaselementrole{ключевой sc-элемент}{Предметная область методов и средств реализации целенаправленного и персонифицированного процесса обучения пользователей для каждой ostis-системы, входящей в состав Экосистемы OSTIS}

    \scnheader{Предметная область методов и средств реализации целенаправленного и персонифицированного процесса обучения пользователей для каждой ostis-системы, входящей в состав Экосистемы OSTIS}
    \scniselement{предметная область}
    \begin{scnhaselementrolelist}{максимальный класс объектов исследования}
        \scnitem{обучаемость}
    \end{scnhaselementrolelist}
    \begin{scnhaselementrolelist}{класс объектов исследования}
        \scnitem{обучаемость пользователя компьютерной системы}
        \scnitem{интеллектуальная компьютерная система}
        \scnitem{подсистема обучения пользователей интеллектуальных систем}
        \scnitem{интеллектуальная обучающая система}
        \scnitem{интеллектуальная обучающая ostis-система}
    \end{scnhaselementrolelist}
    
    \scnheader{обучаемость}
    \scnidtf{способность системы приобретать новые знания и навыки}
    \scnsuperset{неограниченная обучаемость}
    \begin{scnindent}
        \scnrelto{требование}{интеллектуальная компьютерная система}
        \scnidtf{степень обучаемости, при которой ненакладывается никаких ограничений на типологию приобретенных знаний и навыков}
        \scntext{пояснение}{Говоря другими словами, система,обладающая неограниченной обучаемостью, при необходимости может с течением времени приобрести любое знание и способность решать любую задачу}
        \begin{scnindent}
            \scntext{примечание}{Уточним, что это не означает, что одна конкретная система будет уметь решать любую задачу, это означает, что система может приобрести способность решать нужную ей задачу, при этом нет принципиальных ограничений на класс таких задач.}
        \end{scnindent}
    \end{scnindent}
    
    \scnheader{обучение пользователя компьютерной системы}
    \scnsubset{процесс}
    \begin{scnsubdividing}
        \scnitem{обучение пользователя компьютерной системы принципам работы с этой компьютерной системой}
        \begin{scnindent}
            \scnidtf{обучение конечного пользователя компьютерной системы}
        \end{scnindent}
        \scnitem{обучение пользователя компьютерной системы принципам устройства и эволюции этой компьютерной системы}
        \begin{scnindent}
            \scnidtf{обучение разработчика компьютерной системы}
        \end{scnindent}
        \scnitem{обучение пользователя компьютерной системы знаниям из некоторой предметной области, заложенным в эту компьютерную систему}
    \end{scnsubdividing}
    \begin{scnindent}
    	\scntext{примечание}{Для реализации третьего из перечисленных аспектов обучения существует отдельный класс систем, называемых \textit{обучающие системы}. В то же время первые два из перечисленных аспектов не менее важны, поскольку неумение конечного пользователя работать с компьютерной системой приводит к ее неэффективному использованию, а незнание разработчиком принципов работы системы приводит к дополнительным накладным расходам при ее эволюции, а иногда и к невозможности или нецелесообразности такой эволюции --- проще переделать систему, чем разобраться в том, как она устроена и как ее улучшить. В особенности это актуально для интеллектуальных систем, принципы работы с которыми и принципы функционирования которых сложнее, чем у традиционных компьютерных систем.}
    \end{scnindent}
    
    \scnheader{интеллектуальная компьютерная система}
    \scnidtf{сложная техническая система, разработка и даже использование которой требует высоких профессиональных качеств}
    \begin{scnrelfromset}{Проблемы текущего состояния}
        \scnfileitem{недостаточная эффективность использования современных интеллектуальных систем, трудоемкость их внедрения и сопровождения, которые в значительной мере определяются высоким порогом вхождения конечных пользователей в интеллектуальные системы.}
        \scnfileitem{Пользователь часто не использует значительную часть функций даже традиционных компьютерных систем просто по той причине, что не знает об их наличии и не имеет простого механизма, позволяющего о них узнать. Для интеллектуальных систем данная проблема стоит еще более остро.}
        \scnfileitem{Высоки затраты на обучение разработчиков интеллектуальных систем, на их адаптацию под особенности устройства конкретной интеллектуальной системы.}
    \end{scnrelfromset}
    \begin{scnindent}
        \scntext{примечание}{Перечисленные трудности связаны не только с естественной сложностью интеллектуальных компьютерныхсистем по сравнению с традиционными компьютерными системами, но с низким уровнем документациидля таких систем, неудобством использования такойдокументации, трудоемкостью локализации средстви области решения той или иной задачи, как дляконечного пользователя, так и для разработчика.}
        \scntext{предлагаемый подход}{Для решения указанных проблем, предлагается подход, предполагающий дополнение каждой интеллектуальной системы модулем, представляющим собой интеллектуальную обучающую подсистему, целью которой является обучение конечного пользователя и разработчика основной системы принципам работы с ней, принципам ее функционирования и развития.}
        \begin{scnindent}
            \scntext{основная идея}{Независимо от того, для решения каких задач разрабатывается интеллектуальная система, она должна обладать некоторыми функциями обучающей системы, даже если система изначально не является обучающей.}
            \begin{scnindent}
                \begin{scnrelfromlist}{следствие}
                    \scnfileitem{пользователь должен иметь возможностьобучаться как принципам работы с интеллектуальной системой, так и иметь возможность получать новые знания о той предметной области, для которой создается интеллектуальная система}
                    \scnfileitem{разработчик интеллектуальных систем должен иметь возможность обучаться принципам внутреннего устройства системы, принципам ее функционирования, назначению конкретных компонентов системы, иметь возможность локализовать ту часть системы, в которой он должен разобраться для внесения изменений в функциональные возможности системы.}
                \end{scnrelfromlist}
                \scntext{примечание}{Для реализации данной идеи интеллектуальная система должна содержать не только знания о той предметной области, для которой она разработана, но и:
                    \begin{scnitemize}
                        \item знания о самой себе, своей архитектуре, компонентах, функциях, принципах работы и т.д.;
                        \item знания о пользователе, его опыте, навыках, предпочтениях, интересах;
                        \item знания о задачах, которые решает сама система втекущий момент и задачах которые планируются к решению в будущем;
                        \item знания об актуальных задачах по развитию системы и ее сопровождению.
                    \end{scnitemize}}
                \scnrelfrom{реализация}{подсистема обучения пользователей интеллектуальных компьютерных систем}
                \begin{scnindent}
                    \scnidtf{подсистема обучения конечных пользователей и разработчиков интеллектуальных компьютерных систем}
                    \begin{scnrelfromlist}{реализуемая функция}
                        \scnitem{обучение пользователя компьютерной системы принципам работы с этой компьютерной системой}
                        \scnitem{обучение пользователя компьютерной системы принципам устройства и эволюции этой компьютерной системы}
                    \end{scnrelfromlist}
                \end{scnindent}
            \end{scnindent}
        \end{scnindent}
    \end{scnindent}
    
    \begin{scnset}
        \scnheader{подсистема обучения конечных пользователей и разработчиков интеллектуальных компьютерных систем}
        \begin{scnindent}
            \scnrelfrom{технологическая основа}{Технология OSTIS}
        \end{scnindent}
    \end{scnset}
    \begin{scnrelfromset}{преимущества}
        \scnfileitem{В основе Технологии OSTIS лежит \textit{SC-код}, который позволяет в унифицированном (одинаковом) виде представить любую информацию, что позволит сделать предлагаемый подход универсальным и подходящим для любого класса интеллектуальных систем.}
        \scnfileitem{\textit{Технология OSTIS} и, в частности, \textit{SC-код}, легко интегрируется с любыми современными технологиями, что позволит применить предлагаемый подход для большого числа уже разработанных интеллектуальных систем.}
        \scnfileitem{\textit{SC-код} позволяет хранить и описывать в базе знаний ostis-системы любую внешнюю (инородную)по отношению к \textit{SC-коду} информацию в виде внутренних файлов \textit{ostis-систем}. Таким образом, база знаний обучающей подсистемы может содержать в явном виде фрагменты уже имеющейся документации к системе, представленной в любой форме.}
        \scnfileitem{В рамках \textit{Технологии OSTIS} уже разработаны модели баз знаний \textit{ostis-систем}, решателей задач \textit{ostis-систем} и пользовательских интерфейсов \textit{ostis-систем}, предполагающие полное их описание в базе знаний системы. Таким образом для \textit{ostis-систем} предлагаемый подход к обучению конечных пользователей и разработчиков реализуется значительно проще и дает дополнительные преимущества.}
        \scnfileitem{Одним из основных принципов \textit{Технологии OSTIS} является обеспечение гибкости (модифицируемости) систем, разрабатываемых на ее основе. Таким образом, использование \textit{Технологии OSTIS} обеспечит возможность эволюции самой интеллектуальной обучающей подсистемы.}
    \end{scnrelfromset}
            
    \scnheader{подсистема обучения пользователей интеллектуальных систем}
    \scntext{пояснение}{Для реализации взаимодействия \textit{подсистемы обучения пользователей интеллектуальных систем}, реализуемой на основе \textit{Технологии OSTIS} с основной \textit{интеллектуальной системой}, которая в общем случае может быть реализована на основе какой-либо другой технологии, предполагается разработка интерфейсного компонента, который также является частью подсистемы. Важно отметить, что для разных \textit{интеллектуальных систем} такие компоненты будут в значительной степени пересекаться, что, в свою очередь, позволит снизить затраты на интеграцию подсистемы обучения и основной \textit{интеллектуальной системы}.}
    \begin{scnindent}
        \scnrelfrom{иллюстрация}{\scnfileimage[20em]{Contents/part_ecosystem/src/images/sd_learning/system_arch.png}}
        \scntext{примечание}{В случае, если рассматриваемая интеллектуальная система является \textit{ostis-системой}, ее интеграция с подсистемой обучения пользователей интеллектуальных систем осуществляется более глубоко. Компоненты \textit{подсистемы обучения пользователей интеллектуальных систем} просто дополняют уже существующие в основной ostis-системе компоненты, что позволяет максимально снизить затраты на интеграцию \textit{подсистемы обучения пользователей интеллектуальных систем} и основной \textit{ostis-системы}.}
        \begin{scnindent}
        	\scnrelfrom{иллюстрация}{\scnfileimage[20em]{Contents/part_ecosystem/src/images/sd_learning/subsystem_arch.png}}
        \end{scnindent}
    \end{scnindent}

    \scnheader{интеллектуальная обучающая система}
    \scnidtf{ИОС}
    \scnsubset{интеллектуальная система}
    \scntext{примечание}{Такого рода системы по сравнению с традиционными системами электронного обучения (например, электронными учебниками) предоставляют обладают рядом существенных преимуществ.}
    \scnsubset{интеллектуальная справочная система}
    \begin{scnindent}
        \scntext{пояснение}{Каждая \textit{интеллектуальная обучающая система} в качестве простейшего средства изучения учебного материала предполагает наличие средств навигации по этому материалу и средств задания по нему различных вопросов. Системы, обладающие только таким ограниченным набором возможностей, названы \textit{интеллектуальными справочными системами}. Таким образом, можно сказать, что \textit{интеллектуальная обучающая система} обязательно реализует в себе функции \textit{интеллектуальной справочной системы}.}
    \end{scnindent}
    \scnsuperset{интеллектуальная обучающая ostis-система}
    \begin{scnindent}
        \scnidtf{интеллектуальная обучающая система, построенная на основе Технологии OSTIS}
    \end{scnindent}
        
    \scnheader{интеллектуальная обучающая ostis-система}
    \begin{scnrelfromlist}{обобщенная часть}
        \scnitem{база знаний интеллектуальной обучающей ostis-системы}
        \begin{scnindent}
            \begin{scnrelfromset}{преимущества}
                \scnfileitem{SC-код позволяет представлять знания любого рода, в том числе конкретные факты, логические утверждения (аксиомы, теоремы, определения), текстовые и мультимедийные иллюстрации и комментарии, примеры конкретных задач с решениями, в том числе доказательства и т.д.}
                \scnfileitem{Пользователю становятся доступны достаточно полные сведения об изучаемой предметной области, отражены все ее аспекты, благодаря явному помещению в базу знаний всех предметных закономерностей и взаимосвязей понятий.}
                \scnfileitem{База знаний системы рассматривается как иерархия предметных областей и соответствующих им онтологий, то есть позволяет произвести семантическую структуризацию предлагаемого учащемуся материала, что существенно облегчает процесс обучения за счет систематизации знаний на основе именно их семантики, а не каких-либо других сторонних факторов. Кроме этого, знания в базе могут делиться на логические разделы, каждый из которых соответствует какому-либо фрагменту излагаемого материала. База знаний позволяет осуществлять свободную навигацию по любым ассоциативным связям, изучая таким образом материал в той последовательности, какая кажется более логичной для самого обучаемого. С другой стороны, такой подход позволяет указать рекомендуемую последовательность изучения материала. При необходимости структура предметных областей может быть легко перестроена.}
                \scnfileitem{Пользователю в явном виде представляется семантическая структура изучаемого учебного материала и изучаемой предметной области. При этом обеспечивается наглядная визуализация любого уровня указанной семантической структуры.}
                \scnfileitem{Знания из различных областей представляются в сходном виде, что позволяет говорить не о семействе не связанных между собой обучающих систем по различным предметным областям, а о глобальном смысловом пространстве, объединяющем в себе знания всего семейства разрабатываемых систем. В свою очередь, наличие такого смыслового пространства обеспечивает ряд дополнительных возможностей:
                \begin{scnitemize}
                    \item каждая система при необходимости может использовать знания, относящиеся к другим системам, что позволяется задавать не только вопросы, касающиеся конкретной предметной области, но и вопросы, носящие междисциплинарный характер
                    \item в рамках глобального смыслового пространства можно выделить часть знаний, которые имеют отношение ко многим системам из всего комплекса, например базовые знания из области математики, логики и т.д. Концепция глобального смыслового пространства позволяет записывать такие фрагменты знаний только в одной из систем, а затем использовать их во всех остальных, что существенно уменьшает количество дублирований, сокращает сроки разработки систем и снижает накладные расходы.
                \end{scnitemize}}
                \scnfileitem{Унифицированное представление знаний позволяет не ограничивать номенклатуру пользовательских запросов только специально выделенными для этого командами, а задавать произвольный запрос системе с использованием универсального языка отображения знаний, что делает перечень возможных запросов зависящим только от количества и разнообразия знаний, внесенных в базу знаний системы.}
            \end{scnrelfromset}
        \end{scnindent}
        \scnitem{решатель задач интеллектуальной обучающей ostis-системы}
        \begin{scnindent}
            \scntext{преимущество}{Пользователю предоставляется возможность задавать системе любые вопросы и задачи по изучаемой предметной области. Это достигается включением в ИОС решателя задач, способного решать задачи по их формулировкам, в том числе, введенным пользователем. При этом указанный решатель задач может находить путь решения задачи даже, если соответствующий способ решения (например, алгоритм) ему неизвестен.}
        \end{scnindent}
        \scnitem{пользовательский интерфейс интеллектуальной обучающей ostis-системы}
        \begin{scnindent}
            \begin{scnrelfromset}{преимущества}
                \scnfileitem{Унификация моделей пользовательских интерфейсов позволяет отображать знания различного рода в унифицированном виде независимо от предметной области, к которой эти знания относятся. Таким образом, все разрабатываемые системы будут обладать пользовательским интерфейсом, построенным по одним и тем же принципам, что позволит существенно сократить срок ознакомления учащегося со всем семейством систем. Данный факт не отрицает возможность и необходимость разработки отдельных компонентов интерфейса, ориентированных на конкретную предметную область, например, редактора геометрических чертежей, виртуальной лаборатории для проведения химических опытов и т.д.}
                \scnfileitem{ИОС имеет интеллектуальный пользовательский интерфейс с компьютерными (виртуальными) моделями различных объектов изучаемой предметной области, что позволяет системе понимать  смысл (анализировать семантику) пользовательских действий по преобразованию этих объектов. Все это существенно повышает уровень интерактивной виртуальной лабораторной среды электронного учебника.}
                \scnfileitem{Каждый компонент пользовательского интерфейса также является отображением определенного элемента из базы знаний, что позволяет, во-первых, легко менять интерфейс системы даже во время ее работы, а, во-вторых, позволяет пользователю задавать системе вопросы не только касательно предметной области, которой посвящена данная система, но и касательно любого из компонентов интерфейса и других частей системы. Таким образом, пользователю достаточно научиться задавать системе несколько простейших вопросов, чтобы в дальнейшем изучить все тонкости работы системой уже в процессе общения с ней.}
                \scnfileitem{При общении с системой пользователю предоставляется свобода в выборе любого из множества синонимичных терминов (идентификаторов), зарегистрированных в базе знаний системы. При этом указанные термины могут принадлежать различным естественным языкам.}
                \scnfileitem{Появляется принципиальная возможность реализации естественно-языкового интерфейса с пользователем (благодаря широким возможностям семантического анализа пользовательских сообщений и возможностям синтеза на семантическом уровне сообщений, адресуемых пользователям).}
                \scnfileitem{Достаточно легко осуществляется переориентация ИОС на обслуживание пользователей с другим естественным языком (т.к. основная часть базы знаний ИОС, непосредственно описывающая семантику соответствующей предметной области, абсолютно не зависит от внешнего языка, в т.ч. от естественного).}
            \end{scnrelfromset}
        \end{scnindent}
    \end{scnrelfromlist}
    \begin{scnrelfromset}{преимущества}
        \scnfileitem{Помимо возможности чтения текстов и иллюстративных материалов учебника предоставляется возможность навигации по семантическому пространству предметной области.}
        \scnfileitem{Пользователю предоставляется возможность под контролем системы тренироваться (приобретать практические навыки) в решении самых различных задач по изучаемой предметной области. При этом система\\
            \begin{scnitemize}
                \item осуществляет семантический анализ правильности решения задач как по свободно конструируемым ответам (результатам), так и по протоколам решения
                \item локализует допущенные пользователем ошибки в решении задач, определяет их причину и выдает соответствующие рекомендации пользователю.
            \end{scnitemize}}
        \scnfileitem{Пользователю предоставляется полная свобода в выборе последовательности изучения учебного материала (маршрута навигации по учебному материалу), но соответствующие рекомендации выдаются.}
        \scnfileitem{Пользователю предоставляется полная свобода в выборе решаемых им задач (в сборнике задач и лабораторных работ), но соответствующие рекомендации выдаются. Эти рекомендации направлены на то, чтобы минимизировать число решаемых задач, обеспечивающих приобретение требуемых практических навыков.}
        \scnfileitem{Достаточно легко осуществляется интеграция нескольких самостоятельных ИОС по смежным дисциплинам в единый учебник, что, в частности, предоставляет возможность задавать вопросы и задачи на стыке этих дисциплин.}
        \scnfileitem{Пользователь ИОС работает под наблюдением и контролем интеллектуального help-а, который помогает пользователю быстро и эффективно освоить возможности системы. По сути это не что иное, как руководство пользователя ИОС, оформленное как семантический электронный учебник.}
        \scnfileitem{При проектировании базы знаний ИОС появляется уникальная возможность проверять семантическую корректность формируемого информационного ресурса:\\
            \begin{scnitemize}
                \item корректность определений и утверждений
                \item корректность использования различных понятий
                \item корректность алгоритмов
                \item корректность доказательств теорем
                \item и т.д.
            \end{scnitemize}}
    \end{scnrelfromset}
    \begin{scnindent}
        \scntext{примечание}{Часть из перечисленных возможностей (а в предельном случае и все их них) могут быть реализованы в рамках \textit{подсистемы обучения пользователей интеллектуальных систем}.}
    \end{scnindent}

    \bigskip
    \end{scnsubstruct}
    \scnendcurrentsectioncomment
\end{SCn}

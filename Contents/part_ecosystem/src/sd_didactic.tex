\begin{SCn}
    \bigskip
    \scnsectionheader{Предметная область и онтология дидактических знаний}
    \begin{scnsubstruct}
    	\begin{scnrelfromlist}{ключевое понятие}
    		\scnitem{дидактическая информация}
    		\scnitem{учебный вопрос}
    		\scnitem{учебная задача}
    	\end{scnrelfromlist}
    	
    	\bigskip
    	
    	\begin{scnrelfromlist}{библиографическая ссылка}
    		\scnitem{\scncite{OntologySummit2020}}
    		\scnitem{\scncite{Belkin1982}}
    		\scnitem{\scncite{Gavrilova2008}}
    		\scnitem{\scncite{Gavrilova2009b}}
    		\scnitem{\scncite{Golenkov2001b}}
    		\scnitem{\scncite{Grakova2015}}
    		\scnitem{\scncite{Davydenko2013b}}
    		\scnitem{\scncite{Davydenko2015}}
    		\scnitem{\scncite{Zalivako2011}}
    		\scnitem{\scncite{Zimin2016}}
    		\scnitem{\scncite{Kolb2011}}
    		\scnitem{\scncite{Kolb2013}}
    		\scnitem{\scncite{Li2019}}
    		\scnitem{\scncite{Muromtsev2020}}
    		\scnitem{\scncite{Osin2004}}
    		\scnitem{\scncite{Rusetski2013}}
    		\scnitem{\scncite{Solovov2006}}
    		\scnitem{\scncite{Solovov2007}}
    		\scnitem{\scncite{Solovov2021}}
    		\scnitem{\scncite{Taranchuk2019}}
    		\scnitem{\scncite{Shunkevich2013b}}
    	\end{scnrelfromlist}
    	\bigskip
    \scnheader{важнейший критерий качества ostis-систем}
    \scnrelfrom{является}{быстрое приобретение требуемой квалификации с помощью эксплуатируемой системы}
    \scnrelfrom{следствие}{каждая система должна уметь обучать своих пользователей}
    \scnheader{отсутствие взаимопонимания между системой и пользователем}
    \scnrelfrom{является}{нарушение требования интероперабельности}
   \scnheader{современная документация}
    \begin{scnrelfromlist}{не способствует пониманию со стороны}
    	\scnitem{интеллектуальная компьютерная система}
    	\scnitem{человек, к которому адресована документация}
    	\scnrelfrom{следствие}{требование написания учебных пособий}
    	
    \end{scnrelfromlist}


 \scnheader{отсутствие взаимопонимания между системой и пользователем}
 
  \scnheader{база знаний каждой ostis-системы}
\begin{scnrelfromlist}{должна уметь}
	\scnitem{отвечать на вопросы о себе}
	\scnitem{анализировать действия пользователей}
	\scnitem{давать советы пользователям исходя из их действий}

	
\end{scnrelfromlist}

    \scnheader{дидактическая информация}
    \scnrelfrom{должна быть в базе знаний ostis-системы для}{выполнение образовательной функции}
    
    \scnidtf{информация, хранимая в \textit{базе знания ostis-системы}, предназначенная для ее пользователей и помогающая им быстрее и адекватнее усвоить денотационную семантику знаний, хранимых в памяти этой системы}
    
    
    \scnidtf{\textit{знание ostis-системы}, предназначенное для ее пользователей и помогающее им быстрее и адекватнее усвоить денотационную семантику знаний хранимых в памяти этой системы}
    \scnidtf{\textit{знание}, входящее в состав \textit{базы знаний} и помогающее пользователю быстрее и качественнее (адекватнее) понять смысл (\textit{денотационную семантику}) \textit{базы знаний}}
    \scnidtf{информация, способствующая более глубокому пониманию и усвоению смысла различного рода сущностей (в том числе, и различных знаний)}
    \scnidtfexp{информация, которая при ее представлении в памяти \textit{ostis-систем} способствует установлению взаимопонимания между ostis-системами и их пользователями, которая ускоряет процесс формирования у пользователей требуемой квалификации в различных областях}
    \scntext{примечание}{"дидактический"{} эффект дидактической информации обеспечивается:
    	\begin{scnitemize}
    		\item достаточной детализацией изучаемой сущности (полнотой семантической окрестности, описывающей связи этой сущности с другими сущностями)
    		\begin{scnitemize}
    			\item декомпозицией рассматриваемых сущностей;
    			\item указанием аналогов (сходных сущностей в различных смыслах);
    			\item указанием метафор (эпиграфов);
    			\item указанием антиподов (сущностей, отличающихся в различных смыслах);
    		\end{scnitemize}
    		\item упражнениями --- решением различных задач с использованием изучаемых сущностей;
    		\item ссылками на знания, хранимые в рамках той же \textit{базы знаний};
    		\item ссылками на библиографические источники.
    \end{scnitemize}}
    \scnnote{Типология описываемых (рассматриваемых, исследуемых, изучаемых) сущностей (объектов) ничем не ограничивается и включает в себя как конкретные материальные и абстрактные объекты и процессы, конкретные связи и структуры, конкретные информационные конструкции и, в частности, различного вида знания, так и различные классы указанных объектов}
    
    
    
    
    \scnheader{спецификация}
    \scnidtftext{часто используемый sc-идентификатор}{\textit{семантическая окрестность}}
    \scnsuperset{однозначная спецификация}
    \begin{scnindent}
    	\scnidtf{спецификация, которая однозначно соответствует специфицируемой сущности}
    	\scnsuperset{высказывание необходимости и достаточности}
    	\begin{scnindent}
    		\scnidtf{высказывание о необходимых и достаточных условиях принадлежности сущностей соответствующему специфицируемому \textit{множеству}}
    	\end{scnindent}
    	\scnsuperset{определение}
    	\begin{scnindent}
    		\scnnote{Большинство определений являются высказываниями о необходимости и достаточности}
    		\scnsuperset{формальное определение}
    		\begin{scnindent}
    			\scnidtf{определение, представленное на формальном языке (в частности, в SC-коде)}
    		\end{scnindent}
    		\scnsuperset{строгое естественно-языковое определение}
    		\begin{scnindent}
    			\scnidtf{естественно-языковое определение, которое достаточно легко и адекватно перевести на формальный язык (в том числе, на SC-код)}
    		\end{scnindent}
    		\scnsuperset{нестрогое естественно-языковое определение}
    	\end{scnindent} 
    \end{scnindent}
    \scnsuperset{пояснение}
    \begin{scnindent}
    	\scnidtf{нестрогая спецификация, которая например, используется для спецификаций неопределяемых понятий и обычно представляется на естественном языке}
    \end{scnindent}
    \scnsuperset{представление множества}
    \begin{scnindent}
    	\scnidtf{перечисление всех или многих (желательно разнообразных) элементов специфицируемого множества}
    \end{scnindent}
    \scnsuperset{представление иерархии множеств}
    \scnsuperset{представление разбиения}
    \scnsuperset{представление иерархии разбиений}
    \begin{scnindent}
    	\scnidtf{представление иерархического разбиения специфицируемого множества по различным признакам разбиения}
    	\scnidtf{текст разбиения множества}
    	\scnsuperset{классификация}
    	\begin{scnindent}
    		\scnidtf{представление иерархической классификации специфицируемого класса по различным признакам}
    		\scnidtf{текст классификации}
    	\end{scnindent}
    \end{scnindent}
    \scnsuperset{представление декомпозиции}
    \begin{scnindent}
    	\scnsuperset{представление пространственной декомпозиции}
    	\scnsuperset{представление темпоральной декомпозиции}
    \end{scnindent} 
    \scnsuperset{представление иерархии декомпозиции}
    \begin{scnindent}
    	\scnidtf{представление иерархической декомпозиции специфицируемого объекта на компоненты (части) по различным критериям}
    	\scnidtf{иерархическое описание структуры специфицируемого объекта}
    \end{scnindent}
    \scnsuperset{представление логической формулы}
    \begin{scnindent}
    	\scnidtf{полное представление логической формулы до уровня входящих в нее атомарных логических формул}
    \end{scnindent}
    \scnsuperset{логическая структура понятия}
    \scnsuperset{иерархия логической структуры понятия}
    \begin{scnindent}
    	\scnidtf{логическая структура специфицируемого понятия, соответствующая его определению и иерархии определений всех используемых понятий (вплоть до неопределяемых понятий)}
    \end{scnindent}
    \scnsuperset{иерархическая структура доказательства}
    \begin{scnindent}
    	\scnidtf{иерархическая структура доказательства специфицируемого высказывания (теоремы), в каждом шаге которого указывается (1) используемое правило вывода, (2) высказывания, на основе которых осуществляется шаг логического вывода, (3) высказывание, являющееся результатом (следствием) этого шага вывода}
    \end{scnindent}
\scnheader{перечисленные виды спецификаций}
\scnrelfrom{используются в}{дидактиеских целях}
\scnrelfrom{соответствуют}{различным классам специфицируемых сущностей}
\scnrelfrom{предлагают}{использование ряда других понятий}

\scnheader{дополнительно используемые понятия}

\begin{scnrelfromset}{включение}


\scnitem{однозначная спецификация*}
\scnidtf{быть однозначной спецификацией заданной сущности*}




\scnitem{высказывание о необходимости и достаточности*}
\scnidtf{быть высказыванием о необходимости и достаточности принадлежности заданному множеству*}




\scnitem{определение*}
\scnidtf{быть определением заданного класса*}
\scnidtf{быть определением заданного понятия*}





\scnitem{пояснение*}
\scnidtf{быть пояснением заданной сущности*}





\scnitem{принадлежность*}
\scnidtf{быть элементом заданного множества*}
\scnidtf{элемент*}




\scnitem{подмножество*}
\scnidtf{\textit{включение*}}





\scnitem{разбиение*}
\scnidtf{разбиение множества*}




\scnitem{покрытие*}
\scnidtf{покрытие множества*}





\scnitem{представление множества*}



\scnitem{представление иерархии множеств*}


\scnitem{представление разбиения*}





\scnitem{представление иерархии разбиения*}
\scnidtf{быть представлением разбиения заданного множества*}
\scnsuperset{классификация*}



\scnitem{часть*}
\scnsuperset{пространственная часть*}
\scnsuperset{темпоральная часть*}



\scnitem{декомпозиция*}
\scnsuperset{пространственная декомпозиция*}
\scnsuperset{темпоральная декомпозиция*}



\scnitem{представление декомпозиции*}
\scnidtf{быть представлением декомпозиции заданной сущности*}



\scnitem{представление иерархии декомпозиций*}



\scnitem{определение*}
\scnidtf{быть определением заданного понятия*}



\scnitem{логическая структура понятия*}
\scnidtf{быть логической структурой заданного понятия*}
\scnidtf{быть семейством понятий, используемых в определении заданного понятия*}



\scnitem{иерархия логической структуры понятия*}



\scnitem{шаг логического вывода*}
\scnidtf{быть семейством высказываний, из которых логически следует заданное высказывание*}


\scnitem{иерархическая структура доказательства*}
\scnidtf{быть иерархической структурой доказательства заданного высказывания*}
\end{scnrelfromset}
\scnheader{структуризация сущностей и знаний в базе знаний ostis-системы}
\scnrelfrom{является}{важнейший вид дидактической информации}
\scnheader{ключевой вид структуризации}
\scnrelfrom{является}{выделение различных предметных областей и соответствующих им онтологий}

\scnheader{понятия для представления спецификаций библиографических источников}
\begin{scnrelfromset}{включение}
	
	
	\scnitem{официальный библиографический идентификатор*}
	\scnidtf{библиографическая запись, соответствующая заданному информационному ресурсу*}





\scnitem{автор*}
\scnidtf{быть одним из авторов данного знания*}




\scnitem{редактор*}




\scnitem{подраздел*}





\scnitem{цитата*}





\scnitem{библиографическая ссылка*}
\scnidtf{семантически близкий библиографический источник*}





\scnitem{рассматриваемый вопрос*}


\scnitem{аннотация*}





\scnitem{эпиграф*}


\scnitem{ключевой знак*}
\scnsuperset{ключевая сущность, не являющаяся понятием*}
\scnsuperset{ключевое понятие*}
\begin{scnindent}
	\scnsuperset{ключевое понятие, не являющееся ни отношением, ни параметром*}
	\scnsuperset{ключевое отношение*}
	\scnsuperset{ключевой параметр*}
\end{scnindent}
\scnsuperset{\textbf{ключевое знание*}}
\begin{scnindent}
	\scnidtf{основной тезис*}
	\scnsuperset{основное положение*}
	\begin{scnindent}
		\scnidtf{основной вывод (результат)*}
	\end{scnindent}
\end{scnindent}
	
	\end{scnrelfromset}
	
	
	
	\scnheader{cредства логико-семантической систематизации используемых информационных
		конструкций}
	\begin{scnrelfromset}{разбиение}
		\scnitem{средства явного описания синонимии и омонимии идентификаторов}
		\scnitem{средства явного описания семантической эквивалентности, семантического включения, семантической смеж-
			ности и других семантических связей между используемыми информационными конструкциями}
		\scnitem{средства систематизации информационных ресурсов, представленных в дидактически эффективных (нагляд-
			ных, мультимедийных) формах в виде графиков, таблиц, диаграмм, фотографий, рисунков, 3D–изображений,
			аудио-записей, видео-записей лекций, семинаров, конференций, бесед, средств виртуальной реальности}
		\end{scnrelfromset}
	\scnheader{cредства указания и описания сходств, отличий, типовых экземпляров}
	\begin{scnrelfromlist}{эпиграф}
		\scnitem{Все познается в сравнении}
		\scnitem{Важнейшим проявлением интеллекта является умение "видеть" сходства в различных объектах и различия в сходных}
	\end{scnrelfromlist}
\scnheader{понятия для представления сходств, отличий типовых экземпляров}
\begin{scnrelfromset}{включение}
	\scnitem{аналог*}
	\scnidtf{быть аналогичной сущностью*}
	\scnidtf{быть похожей сущностью*}
	\scnidtf{быть аналогом*}
	\scnidtf{быть аналогом заданной сущности*}
	\scnidtf{бинарное неориентированное отношение, каждая пара которого связывает знаки двух аналогичных (в том или ином смысле) сущностей*}
	\scnidtf{пара аналогичных сущностей*}



\scnitem{аналоги*}
\scnidtf{множество аналогичных сущностей*}
\scnsuperset{аналог*}
\begin{scnindent}
	\scnidtf{быть аналогом заданной сущности*}
\end{scnindent}

\scnitem{близкий аналог*}
\scnidtf{сущность-близнец*}
\scnidtf{быть очень похожей сущностью*}

\scnitem{следует отличать*}
\scnidtf{быть семейством похожих, но отличающихся сущностей, которые не следует путать*}
\scnsubset{аналоги*}





\scnitem{аналогичность*}
\scniselement{параметр}
\scnnote{степень аналогичности может быть разной}
\scnidtf{степень аналогичности*}
\scnidtf{Параметр, заданный на множестве пар Отношения "быть аналогом"*}
\begin{scnrelbothlist}{следует отличать}
	\scnitem{\textit{сходство}*}
	\begin{scnindent}
		\scnidtf{описание того, в чем конкретно заключается аналогичность (сходства)}
	\end{scnindent} 
	\scnitem{\textit{следует отличать}*}
	\begin{scnindent}
		\scnidtf{указание факта наличие отличий между перечисленными сущностями*}
	\end{scnindent} 
\end{scnrelbothlist}

\scnitem{антипод*}
\scnidtf{пара принципиально отличающихся сущностей*}


\scnitem{сходство*}
\scnidtf{сходство сущностей, принадлежащих заданному семейству*}
\scnidtf{уточнение того, в чем заключается аналогичность заданного семейства сущностей*}
\scnidtf{аналогия*}

\scnitem{отличие*}
\scnidtf{описание отличий двух заданных сущностей*}

\scnitem{сравнение*}
\scnidtf{текст, описывающий сходства и отличия сущностей, принадлежащих заданному семейству}

\scnitem{сравнительный анализ*}
\scnidtf{сравнительный анализ заданной сущности*}
\scnitem{пример\scnrolesign}
\scnsubset{(включение* $\bigcup$ принадлежность*)}

\scnitem{типичный экземпляр\scnrolesign}
\scnidtf{экземпляр (элемент) заданного класса, аналогичный большинству экземпляров этого класса \scnrolesign}
\scnidtf{типичный представитель заданного класса\scnrolesign}
\scnidtf{типичный пример\scnrolesign}
\scnsubset{пример\scnrolesign}

\scnitem{примечание*}
\scnidtf{неформальное описание отличительной особенности заданной сущности*}

\scnitem{афоризм}
\scnidtf{краткое высказывание отражающее существующего свойства описываемых объектов}

\scnitem{метафора}
\scnidtf{иносказание, указывающее аналогию некоторых объектов}

\scnitem{эпиграф*}
\scnidtf{афоризм, соответствующий заданной сущности (чаще всего, тексту)*}
	\end{scnrelfromset}


\scnheader{указание факта аналогичности групп понятий, используемых для представления дидактической информации вида структуризации и систематизации описываемых объектов}
\scnrelfrom{пример}{}


\scnheader{аналоги*}
\begin{scnhaselementset}
	\scnitem{%
		\begin{scnvector}
			\begin{scnlist}
				\scnitem{представление множества*}
				\scnitem{представление множества}
				\scnitem{принадлежность\scnrolesign}
				\scnitem{множество}
			\end{scnlist}			
	\end{scnvector}}
	\scnitem{%
		\begin{scnvector}
			\begin{scnlist}
				\scnitem{представление разбиения*}
				\scnitem{представление разбиения}
				\scnitem{разбиение*}
				\scnitem{множество}
			\end{scnlist}			
	\end{scnvector}}
	\scnitem{%
		\begin{scnvector}
			\begin{scnlist}
				\scnitem{классификация*}
				\scnitem{классификация}
				\begin{scnindent}
					\scnidtf{представление классификации}
					\scnidtf{текст классификации}
				\end{scnindent} 
				\scnitem{разбиение*}
				\scnitem{класс}
				\begin{scnindent}
					\scnsubset{множество}
				\end{scnindent}
			\end{scnlist}		
	\end{scnvector}}
	\scnitem{%
		\begin{scnvector}
			\begin{scnlist}
				\scnitem{представление декомпозиции*}
				\scnitem{представление декомпозиции}
				\scnitem{декомпозиция*}
				\scnitem{сущность}
			\end{scnlist}			
	\end{scnvector}}
	\scnitem{%
		\begin{scnvector}
			\begin{scnlist}
				\scnitem{представление логической формулы*}
				\scnitem{представление логической формулы}
				\scnitem{принадлежность\scnrolesign}
				\scnitem{логическая формула}
				\begin{scnindent}
					\scnsubset{множество}
				\end{scnindent} 
			\end{scnlist}		
	\end{scnvector}}
	\scnitem{%
		\begin{scnvector}
			\begin{scnlist}
				\scnitem{иерархическая структура определения*}
				\scnitem{иерархическая структура определения}
				\scnitem{определяющее понятие*}
				\begin{scnindent}
					\scnidtf{понятие, используемое в определении заданного понятия*}
				\end{scnindent} 
				\scnitem{определение}
			\end{scnlist}
	\end{scnvector}}
	\scnitem{%
		\begin{scnvector}
			\begin{scnlist}			
				\scnitem{иерархическая структура доказательства*}
				\scnitem{иерархическая структура доказательства}
				\scnitem{посылка шага вывода*}
				\scnitem{шаг вывода}
			\end{scnlist}			
	\end{scnvector}}
\end{scnhaselementset}



\scnheader{правила построения сущностей различных классов}
\scnidtf{принципы, лежащие в основе заданного продукта некоторой деятельности или продуктов, принадлежащих заданному классу*}
\scnidtf{требования, предъявляемые к заданному продукту или к продуктам заданного класса*}
\scnidtf{свойства и характеристики правильно построенного (хорошо построенного) заданного продукта или правильно построенных продуктов заданного класса*}
\scnidtf{быть правилом построения сущностей заданного класса*}
\scnsuperset{правила оформления*}
\begin{scnindent}
	\scnidtf{правила придания окончательной формы (вида), правила "упаковки"{} заданного создаваемого продукта или создаваемых продуктов заданного класса*}
\end{scnindent}
\scnexplanation{С формальной точки зрения правила построения сущностей заданного класса --- это ядро (аксиоматическая система) \textit{логической онтологии}, которая специфицирует \textit{предметную область}, классом объектов исследования которой является указанный выше (заданный) класс сущностей}
\scnsuperset{правила идентификации*}
\begin{scnindent}
	\scnidtf{правила построения \textit{sc-идентификаторов} для заданного класса \textit{sc-элементов}}
\end{scnindent}
\scnsuperset{правила спецификации*}
\begin{scnindent}
	\scnidtf{правила построения \textit{sc-спецификаций} для заданного класса \textit{sc-элементов}}


	\scnnote{Специфицируемыми \textit{sc-элементами} и, соответственно, обозначаемыми ими сущностями могут быть \textit{персоны}, \textit{библиографические источники}, разделы и сегменты \textit{баз знаний}, \textit{файлы} ostis-систем и многое другое}
\end{scnindent}
	
	\scnheader{понятия для представления методологической спецификации тенденций эволюции знаний}

\begin{scnrelfromset}{включение}
	
	\scnitem{результат анализа*}
	\scnidtf{результат анализа заданного объекта или класса объектов*}
	\scnsuperset{оценка качества*}
	\begin{scnindent}
		\scnidtf{анализ качества заданного объекта}
	\end{scnindent}

\scnitem{принципы, лежащие в основе*}
\scnidtf{принципы, лежащие в основе заданной сущности или всех сущностей, принадлежащих заданному классу*}
\scnidtf{основные свойства и характеристики заданной сущности или сущностей заданного класса*}
\scnidtf{основные положения (свойства, закономерности), присущие заданной (описываемой) сущности*}
\scnidtf{принципы, лежащие в основе заданной сущности (объекта системы, информационной конструкции) или сущностей заданного класса*}
\scnidtf{ключевые особенности*}

\scnitem{достоинства*}
\scnidtf{преимущества*}

\scnitem{недостатки*}

\scnitem{назначение*}
\scnidtf{предъявляемые требования*}
\scnidtf{требования, которым должны удовлетворять сущности заданного класса}
\scnidtf{требования, предъявляемые к заданным сущностям*}

\scnitem{проблемы*}
\scnidtf{проблемы заданного объекта или класса объектов*}
\scnidtf{в чем заключается проблема*}
\scnidtf{проблема, ассоциируемая с заданной сущностью*}

\scnitem{актуальные задачи*}
\scnidtf{актуальные задачи совершенствования заданного объекта или класса объектов*}

\scnitem{известные варианты решения заданных проблем*}

\scnitem{предлагаемый подход к решению заданных проблем*}
\scnidtf{принципы, лежащие в основе предлагаемого подхода к решению заданных проблем*}

\scnitem{новизна предлагаемого подхода*}
\scnidtf{бинарное ориентированное отношение, каждая пара которого связывает некоторую сущность с описанием того, чем она принципиально отличается от предшествующих ей аналогов. Такой сущностью может быть новая техническая система, новый метод решения некоторого класса задач и другое}
\scnsuperset{научная новизна предлагаемого подхода*}
\begin{scnindent}
	\scnidtf{"изюминка"{} предполагаемой системы, метода, принципов, подхода к решению}
\end{scnindent} 

\scnitem{реализация предлагаемого подхода*}

\scnitem{оценка качества реализации предлагаемого подхода*}

\scnitem{направления дальнейшего развития*}
\scnidtf{направления дальнейшего развития заданного объекта*}
\begin{scnsubdividing}
	\scnitem{перманентные направления развития*}
	\scnitem{тактические направления развития текущего состояния}
	\begin{scnindent}
		\scnidtf{план ближайших работ по развитию данного объекта*}
	\end{scnindent} 
	\scnitem{стратегические направления развития текущего состояния*}
	\begin{scnindent}
		\scnidtf{перспективный план развития данного объекта*}
	\end{scnindent} 
\end{scnsubdividing}


	\end{scnrelfromset}


\scnheader{понятия для представления учебных заданий и учебно-методической структуризации баз знаний ostis-систем}

\begin{scnrelfromset}{включение}
	
	
	
	\scnitem{учебный вопрос}
	\scnidtf{вопрос для самопроверки или проверки качества усвоения знаний}
	\scnsuperset{аттестационный вопрос}

\scnitem{учебный вопрос*}
\scnidtf{учебный вопрос для заданного раздела (контекста) учебного материала*}
\scnsuperset{аттестационный вопрос*}

\scnitem{учебная задача}
\scnidtf{упражнение}
\begin{scnsubdividing}
	\scnitem{учебная задача не требующая дополнительных средств для ее выполнения}
	\scnitem{учебная лабораторная работа}
\end{scnsubdividing}
\scnsuperset{\textbf{аттестационная задача}}

\scnitem{учебная задача*}
\scnidtf{бинарное ориентированное отношение, любая пара которого связывает предметную область и онтологию с соответствующими упражнениями}
\scnsuperset{\textbf{аттестационная задача*}}
\scnnote{Необходимо разбиение множества упражнений по уровню сложности, по тематике, по используемым методам, а также задание порядка упражнений.}

\scnitem{аттестационный вопрос}
\scnidtf{тестовый вопрос контроля знаний}
\scnidtf{вопрос, который рекомендуется задавать при проверке усвоения знаний}

\scnitem{аттестационный вопрос*}
\scnidtf{быть аттестационным вопросом для заданного контекста (например, для заданной предметной области и онтологии)}
	
	
	
	\end{scnrelfromset}

\scnheader{eчебно-методическая структуризация баз знаний ostis-систем}
\begin{scnrelfromlist}{требует}
	\scnitem{ установление соответствия учебных вопросов и учебных задач с соответствующими разделами баз знаний}
	\scnitem{включения в состав баз	знаний рекомендаций о последовательности изучения разделов учебного материала и о последовательности решения учебных задач}
	\end{scnrelfromlist}
    
    	
   
    	\end{scnsubstruct}
\end{SCn}


\begin{SCn}

\scnsectionheader{Завершение введения в описание внутреннего языка ostis-систем и близких ему внешних языков}

\scnstartsubstruct

\scnnote{Приведем сравнительный анализ основных понятий, связанных с языками представления знаний в ostis-системах}

\scnheader{язык представления знаний в ostis-системах}
\scnsubset{формальный язык}
\scnsubset{универсальный язык}
\scnhaselements{\textit{SC-код}; \textit{SCg-код}; \textit{SCs-код}; \textit{SCn-код}}
\scnnote{Следует отличать
\begin{scnitemize}
\item саму описываемую сущность;
\item текст, являющийся описанием некоторой сущности;
\item тест, являющийся описанием некоторого другого текста, а возможно и самого себя (т.е. текст может быть описываемой сущностью);
\item знак (обозначение) описываемой сущности в рамках заданного текста;
\item обозначение описываемой сущности в SC-тексте (это всегда sc-элемент того или иного вида);
\item коммуникативный (внешний для ostis-системы) уникальный (основной) идентификатор (чаще всего строковый идентификатор-имя), обозначающий соответствующую описываемую сущность и являющийся внешним идентификатором (именем) для соответствующего синонимичного ему sc-элемента. Такие идентификаторы взаимно однозначно соответствуют sc-элементам, которые имеют такие идентификаторы;
\item вспомогательные (неосновные) внешние идентификаторы sc-элементов. Такие идентификаторы и свойством омонимии (когда один идентификатор соответствует нескольким sc-элементам) и синонимии (когда разные идентификаторы соответствует одному sc-элементу);
\item обозначение описываемой сущности в sc.g-тексте (это всегда графически представленный sc.g-элемент, являющийся \uline{изображением} соответствующего sc-элемента);
\item обозначение описываемой сущности в sc.s-предложении и в sc.n-предложении – это всегда строка символов (либо \uline{омонимичное} изображение sc-коннекторов различного семантического типа, либо \uline{основной} строковый идентификатор, соответствующий некоторому sc-элементу, либо выражение, являющееся \uline{неатомарным} идентификатором, содержащим некоторую информацию о соответствующей именуемой сущности).
\end{scnitemize}

Подчеркнем, что каждое \uline{обозначение} описываемой сущности в SCg-коде, SCs-коде; SCn-коде рассматривается нами как \uline{изображение} соответствующего ему (синонимичного ему) sc-элемента, обозначающего ту же описываемую сущность. Таким образом, указанные языки (SCg-код, SCs-код; SCn-код) рассматриваются нами как различные варианты изображения текстов SCg-кода.}
\scnnote{Для формального описания рассматриваемого нами семейства языков (SCg-код, SCg-код, SCs-код, SCn-код) и не только их используется целый ряд метаязыковых понятий.

Перечислим некоторые из них: \textit{идентификатор}, \textit{класс синтаксически эквивалентных идентификаторов}, \textit{имя}, \textit{простое имя}, \textit{выражение}, \textit{внешний идентификатор*}, \textit{алфавит*}, \textit{разделители*}, \textit{ограничители*}, \textit{предложения*}}

\scnheader{алфавит*}
\scnidtf{быть алфавитом для заданного множества текстов*}
\scnidtf{быть семейством максимальных множеств синтаксически однотипных элементарных (атомарных) фрагментов текстов, принадлежащих заданному множеству текстов*}

\scnheader{ограничители*}
\scnidtf{Отношение, связывающее заданный класс информационных конструкций с соответствующим классом их ограничителей}
\scnidtf{быть ограничителями, используемыми в заданном множестве информационных конструкций*}

\scnheader{ограничитель}
\scnsuperset{sc.g-ограничитель}
\scnsuperset{sc.s-ограничитель}
\scnsuperset{sc.n-ограничитель}
\scnsuperset{ограничитель, используемый в ея-файлах ostis-систем}

\scnheader{SCg-код}
\scnrelfrom{ограничители}{sc.g-ограничитель}
\scnaddlevel{1}
\scnidtf{Множество ограничителей, используемых в sc.g-текстах}
\scnaddlevel{-1}

\scnheader{разделители*}
\scnidtf{быть разделителями, используемыми в заданном множестве информационных конструкций*}
\scnrelfrom{второй домен}{разделитель}
\scnaddlevel{1}
\scnsuperset{sc.g-разделитель}
\scnaddlevel{1}
\scneq{sc.g-коннектор}
\scnaddlevel{-1}
\scnsuperset{sc.s-разделитель}
\scnsuperset{sc.n-разделитель}
\scnsuperset{разделитель, используемый в ея-файлах ostis-систем}
\scnaddlevel{-1}

\scnheader{идентификатор}
\scnsuperset{sc.s-идентификатор}
\scnidtf{cтруктурированный знак соответствующей (обозначаемой) сущности, который чаще всего представляет собой строку (цепочку символов), которую будем называть именем соответствующей сущности.} 
\scnnote{В формальных текстах (в том числе текстах SC-кода, SCg-кода, SCs-кода, SCn-кода) основные используемые идентификаторы не должны быть омонимичными, то есть должны \uline{однозначно} соответствовать идентифицируемым сущностям. Следовательно, каждая пара идентификаторов, имеющих \uline{одинаковую} структуру, должны обозначать одну и ту же сущность.}

\scnheader{следует отличать*}
\scnhaselementset{идентификатор\\
\scnaddlevel{1}
\scnidtf{Множество всевозможных конкретных \uline{экземпляров}, конкретных вхождений идентификаторов, имеющих различную структуру, во всевозможные тексты}
\scnaddlevel{-1}
;класс синтаксически эквивалентных идентификаторов\\
\scnaddlevel{1}
\scnidtf{класс идентификаторов, имеющих одинаковую структуру}
\scnidtf{Семейство всевозможных множеств, каждое из которых является максимальным множеством синтаксически эквивалентных идентификаторов}
\scnaddlevel{-1}
}

\scnheader{имя}
\scnsubset{идентификатор}
\scnidtf{строковый идентификатор}
\scnidtf{идентификатор, представляющий собой строку (цепочку) символов}
\scnsubdividing{простое имя\\
\scnaddlevel{1}
\scnidtf{атомарное имя}
\scnidtf{имя, в состав которого другие имена не входят}
\scnaddlevel{-1}
;выражение\\
\scnaddlevel{1}
\scnidtf{неатомарное имя}
\scnaddlevel{-1}
}
\scnsuperset{sc.s-идентификатор}

\scnheader{внешний идентификатор*}
\scniselement{отношение}
\scnidtf{Бинарное ориентированное отношение, каждая связка (sc-дуга) которого связывает некоторый элемент с файлом, содержимым которого является внешний идентификатор (чаще всего, имя), соответствующий указанному элементу}
\scnidtf{быть внешним идентификатором*}
\scnidtf{внешний идентификатор sc-элемента*}
\scnrelfrom{второй домен}{идентификатор}
\scnnote{Понятие внешнего идентификатора является понятием относительным и важным для ostis-систем, поскольку внутреннее для ostis-систем представление информации (в виде текстов SC-кода) оперирует не идентификаторами описываемых сущностей, а знаками, структура которых никакого значения не имеет}

\scnheader{следует отличать*}
\scnhaselementset{sc-элемент, обозначающий файл ostis-системы;sc.g-элемент, обозначающий файл ostis-системы;простой sc.s-идентификатор, обозначающий файл ostis-системы\\
\scnaddlevel{1}
\scnidtf{простое имя файла ostis-системы}
\scnaddlevel{-1}
;изображение файла ostis-системы, ограниченное sc.g-рамкой;изображение файла ostis-системы, ограниченное sc.n-рамкой;изображение строки символов, ограниченное квадратными скобками}
\scnheader{следует отличать*}
\scnhaselementset{файл-экземпляр;файл-класс\\
\scnaddlevel{1}
\scnidtf{файл, обозначающий класс файлов-экземпляров, синтаксически эквивалентных заданному образцу}
\scnaddlevel{-1}
}

\scnheader{предложения*}
\scnidtf{быть множеством всех предложений заданного текста, не являющихся встроенными предложениями, то есть предложениями, входящими в состав других предложений*}
\scnrelfrom{второй домен}{предложение}

\scnheader{предложение}
\scnexplanation{минимальный семантически целостный фрагмент текста, представляющий собой конфигурацию знаков, входящих в этот фрагмент и связываемых между собой отношениями инцидентности (в частности, отношением непосредственной последовательности в строке), а также различного вида разделителями и ограничителями}


\newpage


\scnheader{sc.g-текст}
\scnhaselementrole{пример}{\scnfilescg{figures/intro/scg/example_triangle.png}}
\scnaddlevel{1}
\scnexplanation{Данный sc.g-текст содержит следующую информацию:
\begin{scnitemize}
\item Сущности \textit{Треугольник ABC}~~ и ~~\textit{Треугольник CDE} являются треугольниками (принадлежат классу \textit{треугольников}). При этом известно, что площадь \textit{Треугольника CDE} в 4 раза больше, чем площадь \textit{Треугольника ABC}, но конкретные значения ллощадей не известны\char59
\item Сущность \textit{Отрезок DE} является отрезком (принадлежит классу \textit{отрезков}) и является стороной \textit{Треугольника CDE}. Кроме того, у \textit{Отрезка DE} есть длина, измерение которой в сантиметрах составляет 5. Обратите внимание, что в данном случае для упрощения понимания использовано бинарное отношение \textit{длина*}, которое является \textit{неосновным понятием} и в базе знаний заменяется на \textit{базовую sc-дугу}, связывающую величину как класс эквивалентности с конкретной сущностью, входящей в данный класс, в данном случае -- \textit{Отрезок DE}\char59  
\item Сущность \textit{Треугольник AEB} является треугольником и имеет \textit{внутренний угол*}~~~ \textit{Угол AEB}. В свою очередь, \textit{Угол AEB} является \textit{углом} и имеет \textit{косинус*}, равный 0,5\char59
\item \textit{Треугольник AEB} имеет \textit{сторону*} (не указывается, какая именно из сторон имеется в виду), \textit{средней точкой*} которой является \textit{Точка O}. В свою очередь, \textit{Точка O} является центром некоторой \textit{Окружности O}, которая относится к классу \textit{окружностей}.
\end{scnitemize}
}
\scnaddlevel{-1}

\newpage

\scnheader{Пример sc.g-текста, трансформируемого по Первому направлению расширения Ядра SCg-кода}
\scneqscg{figures/intro/scg/scg_transf1.png}
\scniselement{sc.g-текст}
\scnexplanation{Здесь (в левом нижнем углу приведенного sc.g-текста) представлен \textit{sc.g-узел общего вида}, изображающий \textit{sc-узел общего вида}, которому соответствует \textit{основной sc-идентификатор*} в виде строки ``\textbf{\textit{ei}}''}
\scnrelfrom{трансформация sc.g-текста по Первому направлению расширения Ядра SCg-кода}{\scnfilescg{figures/intro/scg/scg_transf2.png}}
\scnaddlevel{1}
    \scniselement{sc.g-текст}
    \scnexplanation{\textit{sc.g-узлу общего вида} изображающему \textit{sc-узел}, внешним идентификатором которого является строка ``\textit{основной sc-идентификатор*}'' и который, соответственно является знаком \textit{бинарного ориентированного отношения}, каждая \textit{пара} которого связывает идентифицируемый \textit{sc-элемент} с его основным внешним sc-идентификатором, приписывается указанный внешний идентификатор изображаемого им \textit{sc-элемента}.}
    \scnrelfrom{трансформация sc.g-текста по Первому направлению расширения Ядра SCg-кода}{\scnfilescg{figures/intro/scg/scg_transf3.png}}
    \scnaddlevel{1}
        \scniselement{sc.g-текст} 
        \scnexplanation{В результате данной трансформации исходный \textit{sc.g-текст} трансформируется в один \textit{sc.g-общего вида}, которому приписывается \textit{основной sc-идентификатор} ``\textit{\textbf{ei}}''.}
    \scnaddlevel{-1}
\scnaddlevel{-1}

\newpage

\scnheader{Примеры sc.g-текстов, трансформируемых по Второму направлению расширения Ядра SCg-кода}
\scnstructinclusion

\scnmakeset{\scgfileitem{figures/intro/scg/scg2_ex1.png}\\
\scnaddlevel{1}
    \scnrelfrom{синтаксическая трансформация}{\scnfilescg{figures/intro/scg/scg2_ex1_1.png}}
    \scnaddlevel{1}
        \scnexplanation{Здесь вводится новый синтаксический вид \textit{sc.g-элементов} -- \textit{константный постоянный sc.g-узел общего вида}, изображаемый окружностью диаметром ***.}
    \scnaddlevel{-1}
\scnaddlevel{-1}
}


\newpage

\scnendstruct

\end{SCn}

\scsection{Введение в описание внутреннего языка ostis-систем{\normalfont (}компьютерных систем, построенных по Технологии OSTIS{\normalfont)} и близких ему внешних языков, используемых для представления исходных текстов баз знаний}
\label{intro_lang}

\begin{SCn}

\scnsectionheader{Введение в описание внутреннего языка ostis-систем и близких ему внешних языков}
\scnstartsubstruct

\scsectionbeginningname{Начало Введения в описание внутреннего языка ostis-систем и близких ему внешних языков}
\scnstartsubstruct

\scnheader{ostis-система}
\scnidtf{компьютерная система, построенная по Технологии OSTIS}

\scnheader{Введение в описание внутреннего языка ostis-систем и близких ему внешних языков}
\scniselement{раздел базы знаний}
\scnreltovector{конкатенация}{Начало Введения в описание внутреннего языка ostis-систем и близких ему внешних языков;\nameref{intro_sc_code};\nameref{intro_idtf};\nameref{intro_scg};\nameref{intro_scs};\nameref{intro_scn}}

\bigskip
\bigskip
\scnfilelong{Поскольку предлагаемая Вашему вниманию Публикация Документации Технологии OSTIS-2020 представляет собой внешний текст основной части \textit{базы знаний} ostis-системы (\textit{Метасистемы IMS.ostis}), необходимо сразу во вводном разделе этой Документации пояснить основные принципы, лежащие в основе внутреннего представления \textit{баз знаний} в памяти \textit{ostis-систем}, а также некоторые правила и условные обозначения, используемые в оформлении внешних текстов \textit{баз знаний ostis-систем}.

Подчеркнем, что все эти принципы, правила и условные обозначения детально рассмотрены в соответствующих разделах \textit{Документации Технологии OSTIS}, но некоторые из них необходимо пояснить до начала ознакомления с основными положениями \textit{Технологии OSTIS}. Фактически речь идет о кратком руководстве конечных пользователей \textit{ostis-систем}.

Прежде, чем рассматривать \textit{внутренний язык*} и \textit{внешние языки*}, используемые \textit{ostis-системами}, необходимо уточнить понятие \textit{информационной конструкции}, понятие \textit{знака}, понятие \textit{текста}, понятие \textit{языка}. Более подробно указанные понятия рассмотрены в разделе, посвященном описанию соответствующей предметной области и онтологии.}

\scnheader{информационная конструкция}
\scnidtf{информация}
\scnidtf{конструкция (структура), содержащая некоторые сведения о некоторых сущностях}
\scnnote{Форма представления ("изображения"{}, "материализации"), форма структуризации (синтаксическая структура), а также смысл (денотационная семантика) информационных конструкций могут быть самыми различными.}

\scnheader{дискретная информационная конструкция}
\scnsubset{информационная конструкция}
\scnexplanation{Дискретная информационная конструкция — это информационная конструкция, смысл которой задается (1) множеством элементов (синтаксически атомарных фрагментов) этой информационной конструкции, (2) алфавитом этих элементов — семейством классов синтаксически эквивалентных элементов информационной конструкции, (3) принадлежностью каждого элемента информационной конструкции соответствующему классу синтаксически эквивалентных элементов информационной конструкции, (4) конфигурацией связей инцидентности между элементами информационной конструкции.}
    \scnaddlevel{1}
    \scntext{следствие}{Форма представления элементов дискретной информационной конструкции для анализа её смысла не требует уточнения. Главным является (1) наличие простой процедуры выделения (сегментации) элементов дискретной информационной конструкции (2) наличие простой процедуры установления синтаксической эквивалентности разных элементов дискретной информационной конструкции и (3) наличие простой процедуры установления принадлежности каждого элемента дискретной информационной конструкции соответствующему классу синтаксически эквивалентных элементов (т.е. соответствующему элементу алфавита).}
    \scnaddlevel{-1}

\scnheader{элемент дискретной информационной конструкции}
\scnidtf{синтаксически атомарный фрагмент (символ), входящий в состав дискретной информационной конструкции}
\scnnote{Поскольку дискретные информационные конструкции могут иметь общие элементы (атомарные фрагменты) и даже некоторые из них могут быть частями других информационных конструкций, элемент дискретной информационной конструкции может входить в состав сразу нескольких информационных конструкций.}

\scnheader{отношение, заданное на множестве элементов дискретных информационных конструкций\textasciicircum}
\scnhaselement{синтаксическая эквивалентность элементов дискретных информационных конструкций*}
	\scnaddlevel{1}
	\scnidtf{быть синтаксически эквивалентными элементами (атомарными фрагментами) одной и той же или разных дискретных информационных конструкций, т.е. элементами, принадлежащими одному и тому же классу синтаксически эквивалентных элементов дискретных информационных конструкций*}
	\scnaddlevel{-1}
\scnhaselement{инцидентность элементов дискретных информационных конструкций*}
	\scnaddlevel{1}
	\scniselement{бинарное ориентированное отношение\textasciicircum}
	\scnnote{Для линейных информационных конструкций — это последовательность элементов (символов), входящих в состав этих конструкций.\\
	Для дискретных информационных конструкций, конфигурация которых имеет нелинейный характер, отношение инцидентности их элементов может быть разбито на несколько частных отношений инцидентности, каждое из которых является \uline{подмножеством} объединенного отношения инцидентности. Например, для двухмерных дискретных информационных конструкций это (1) инцидентность элементов информационных конструкций "по горизонтали"{} и (2) инцидентность элементов информационных конструкций "по вертикали".}
	\scnaddlevel{-1}

\scnheader{отношение, заданное на множестве дискретных информационных конструкций\textasciicircum}
\scnhaselement{элемент дискретной информационной конструкции*}
	\scnaddlevel{1}
	\scnidtf{быть элементарным (синтаксически атомарным) фрагментом заданной дискретной информационной конструкции*}
	\scnrelfrom{второй домен}{элемент дискретной информационной конструкции}
	\scnaddlevel{-1}
\scnhaselement{неэлементарный фрагмент дискретной информационной конструкции*}
	\scnaddlevel{1}
	\scnidtf{быть дискретной информационной конструкцией, которая является \uline{подструктурой} для заданной дискретной информационной конструкции, в состав которой входит (1) подмножество элементов заданной информационной конструкции и, соответственно, (2) подмножество пар инцидентности элементов заданной информационной конструкции.}
	\scnaddlevel{-1}
\scnhaselement{алфавит дискретной информационной конструкции*}
	\scnaddlevel{1}
	\scnidtf{быть семейством \uline{классов} синтаксически эквивалентных элементов заданной дискретной информационной конструкции*}
	\scnaddlevel{-1}
\scnhaselement{первичная синтаксическая структура дискретной информационной конструкции*}
	\scnaddlevel{1}
	\scnidtf{быть графовой структурой, которая полностью описывает конфигурацию заданной дискретной информационной конструкции и которая включает в себя: (1) знаки всех тех классов синтаксически эквивалентных элементов, которым принадлежат элементы описываемой дискретной информационной конструкции, (2) знаки всех элементов (атомарных фрагментов) описываемой информационной конструкции, (3) пары, описывающие инцидентность элементов описываемой информационной конструкции, (4) пары, описывающие принадлежность элементов описываемой информационной конструкции соответствующим классам синтаксически эквивалентных элементов этой информационной конструкции.}
	\scnaddlevel{-1}
\scnhaselement{синтаксическая эквивалентность дискретных информационных конструкций*}
	\scnaddlevel{1}
	\scndefinition{Дискретные информационные конструкции $T_i$ и $T_j$ являются синтаксически эквивалентными в том и только в том случае, если между конструкцией $T_i$ и конструкцией $T_j$ существует \uline{изоморфизм}, в рамках которого каждому элементу конструкции $T_i$ соответствует синтаксически эквивалентный элемент конструкции $T_j$, т.е. элемент, принадлежащий тому же классу синтаксически эквивалентных элементов дискретных информационных конструкций. И наоборот.}
	\scnaddlevel{-1}
\scnhaselement{копия дискретной информацион ной конструкции*}
	\scnaddlevel{1}
	\scnsubset{синтаксическая эквивалентность дискретных информационных конструкций*}
	\scnidtfdef{быть дискретной информационной конструкцией, которая не только синтаксически эквивалентна заданной, но и содержит информацию о форме представления элементов копируемой информационной конструкции*}
	\scnaddlevel{-1}
\scnhaselement{семантическая эквивалентность дискретных информационных конструкций*}
	\scnaddlevel{1}
	\scndefinition{Информационная конструкция $T_i$ и информационная конструкция $T_j$ являются \uline{семантически эквивалентными} в том и только в том случае, если \uline{каждая} сущность (в том числе, и каждая связь между сущностями), описываемая в информационной конструкции $T_i$ описывается также и в информационной конструкции $T_j$. И наоборот.}
	\scnaddlevel{-1}
\scnhaselement{семантическое расширение дискретной информационной конструкции*}
	\scnaddlevel{1}
	\scndefinition{Информационная конструкция $T_j$ является семантическим расширением информационной конструкции $T_i$ в том и только в том случае, если \uline{каждая} сущность, описываемая в $T_i$, описывается также и в $T_j$, но обратное неверно.}
	\scnaddlevel{-1}
	
\scnheader{параметр, заданный на множестве дискретных информационных конструкций\textasciicircum}
\scnhaselement{размерность дискретных информационных конструкций\textasciicircum}
	\scnaddlevel{1}
	\scnidtf{типология дискретных информационных конструкций, определяемая их размерностью}
	\scnexplanation{Это параметр дискретных информационных конструкций, определяющий характер \uline{инцидентности} элементов таких конструкций.}
	\scnhaselement{линейная информационная конструкция}
		\scnaddlevel{1}
		\scnidtfexp{дискретная информационная конструкция, каждый элемент которой может иметь не более двух инцидентных ему элементов (один слева, другой справа)}
		\scnaddlevel{-1}
	\scnhaselement{двухмерная информационная конструкция}
		\scnaddlevel{1}
		\scnidtfexp{дискретная информационная конструкция, каждый элемент которой может иметь не более четырех инцидентных ему элементов (слева-справа, сверху-снизу)}
		\scnaddlevel{-1}
	\scnhaselement{трехмерная информационная конструкция}
		\scnaddlevel{1}
		\scnidtfexp{дискретная информационная конструкция, каждый элемент которой может иметь не более шести инцидентных ему элементов (слева-справа, сверху-снизу, сзади-спереди)}
		\scnaddlevel{-1}
	\scnhaselement{четырехмерная информационная конструкция}
		\scnaddlevel{1}
		\scnidtfexp{дискретная информационная конструкция, каждый элемент которой может иметь не более восьми инцидентных ему элементов (например, слева-справа, сверху-снизу, сзади-спереди, раньше-позже)}
		\scnaddlevel{-1}
	\scnhaselement{графовая информационная конструкция}
		\scnaddlevel{1}
		\scnidtfexp{дискретная информационная конструкция, множество элементов которой разбивается на два подмножества — связки и узлы. При этом узлы могут иметь \uline{неограниченное} количество инцидентных им связок}
			\scnaddlevel{1}
			\scnnote{А в некоторых графовых информационных конструкциях и связки могут иметь неограниченное количество инцидентных им других связок}
			\scnaddlevel{-1}
		\scnaddlevel{-1}
	\scnaddlevel{-1}
\scnhaselement{типология дискретных информационных конструкций, определяемая их носителем\textasciicircum}
	\scnaddlevel{1}
	\scnhaselement{аудио-сообщение}
		\scnaddlevel{1}
		\scnidtf{речевое сообщение}
		\scnidtf{информационная конструкция, представленная в звуковой форме}
		\scnaddlevel{-1}
	\scnhaselement{информационная конструкция, представленная на языке жестов}
	\scnhaselement{информационная конструкция, представленная в письменной форме}
		\scnaddlevel{1}
		\scnnote{Конкретный вид носителя для письменной формы представления информации может быть разным — бумага, папирус, береста, камень...}
		\scnaddlevel{-1}
	\scnhaselement{файл}
		\scnaddlevel{1}
		\scnidtf{"электронная"{} форма (формат) представления и хранения информационной конструкции}
		\scnnote{Представление информационных конструкций в виде файлов ориентировано на представление \uline{дискретных} (!) информационных конструкций. Поэтому "файловое"{} представление недискретных информационных конструкций (например, различного рода сигналов) предполагает "дискретизацию"{} таких конструкций, т.е. преобразование их в дискретные. Так преобразуются аудио-сигналы (в частности, речевые сообщения), изображения, видео-сигналы и др.}
		\scnaddlevel{-1}
	\scnaddlevel{-1}
\scnhaselement{уровень унификации представления синтаксически эквивалентных элементов дискретных информационных конструкций\textasciicircum}
	\scnaddlevel{1}
	\scnidtf{уровень "членораздельности"{} дискретных информационных конструкций}
	\scnnote{Чем выше уровень унификации представления элементов дискретных информационных конструкций, тем проще реализуется (1) процедура выделения (сегментации) элементов дискретной информационной конструкции, (2) процедура установления синтаксической эквивалентности этих элементов и (3) процедура их распознавания, т.е. процедура установления их принадлежности соответствующим классам синтаксически эквивалентных элементов.}
	\scnhaselement{дискретная информационная конструкция с низким уровнем унификации представления элементов}
		\scnaddlevel{1}
		\scnsuperset{аудио-сообщение}
		\scnsuperset{информационная конструкция, представленная на языке жестов}
		\scnsuperset{рукопись или её копия}
		\scnaddlevel{-1}
	\scnhaselement{дискретная информационная конструкция с высоким уровнем унификации представления элементов}
		\scnaddlevel{1}
		\scnsuperset{печатный текст}
		\scnsuperset{файл}
		\scnaddlevel{-1}
	\scnaddlevel{-1}
\scnaddlevel{-1}	

\scnheader{знак}
\scnexplanation{фрагмент информационной конструкции, который условно представляет (изображает) некоторую описываемую сущность, которую называют денотатом знака}
\scnsubdividing{знак, являющийся элементом дискретной информационной конструкции;знак, являющийся неатомарным фрагментом дискретной информационной конструкции}

\scnheader{отношение, заданное на множестве знаков\textasciicircum}
\scnnote{Поскольку все знаки являются дискретными информационными конструкциями, множество знаков является областью задания всех отношений, заданных на множестве дискретных информационных конструкций. Тем не менее есть как минимум одно отношение, специфичное для множества знаков.}
\scnhaselement{синонимия знаков*}
	\scnaddlevel{1}
	\scndefinition{Знаки являются синонимичными в том и только в том случае, если они обозначают одну и ту же сущность.}
	\scnnote{Синонимичные знаки могут быть синтаксически эквивалентными, а могут и не быть таковыми.}
	\scnaddlevel{-1}

\scnheader{знаковая конструкция}
\scnsubset{дискретная информационная конструкция}
\scnidtfexp{дискретная информационная конструкция, которая в общем случае представляет собой конфигурацию знаков и специальных фрагментов информационной конструкции, обеспечивающих структуризацию конфигурации знаков — различного вида разделителей и ограничителей}
	\scnaddlevel{1}
	\scnnote{Для некоторых знаковых конструкций и даже для некоторых языков необходимость в разделителях и ограничителях может отсутствовать.}
	\scnaddlevel{-1}	
	
\scnheader{отношение, заданное на множестве знаковых конструкций\textasciicircum}
\scnhaselement{знак*}
	\scnaddlevel{1}
	\scnidtf{быть знаком для заданной знаковой конструкции*}
	\scnaddlevel{-1}
\scnhaselement{разделитель знаковой конструкции*}
\scnhaselement{разделители знаковой конструкции*}
	\scnaddlevel{1}
	\scnidtf{Множество всех разделителей, входящих в состав заданной знаковой конструкции*}
	\scnaddlevel{-1}
\scnhaselement{ограничитель знаковой конструкции*}
\scnhaselement{ограничители знаковой конструкции*}
\scnhaselement{семантическая смежность знаковых конструкций*}
	\scnaddlevel{1}
	\scndefinition{Знаковые конструкции $T_i$ и $T_j$ являются смежными в том и только в том случае, если существуют синонимичные знаки $t_i$ и $t_j$, один из которых входит в состав конструкции $T_i$, а второй — в состав конструкции $T_j$}
	\scnaddlevel{-1}
\scnhaselement{конкатенация знаковых конструкций*}
	\scnaddlevel{1}
	\scnidtf{декомпозиция заданной знаковой конструкции на последовательность знаковых конструкций*}
	\scnaddlevel{-1}
	
\scnheader{класс знаковых конструкций\textasciicircum}
\scnhaselement{семантически элементарная знаковая конструкция}
	\scnaddlevel{1}
	\scnidtf{знаковая конструкция, описывающая некоторую (одну) связь между некоторыми сущностями}
	\scnaddlevel{-1}
\scnhaselement{семантически связная знаковая конструкция}
	\scnaddlevel{1}
	\scnidtfdef{знаковая конструкция, которую можно представить в виде конкатенации семантически элементарных знаковых конструкций, каждая из которых семантически смежна предшествующей и последующей семантически элементарной знаковой конструкции}
	\scnaddlevel{-1}

\scnheader{параметр, заданный на множестве знаковых конструкций\textasciicircum}
\scnhaselement{семантическая связность знаковых конструкций\textasciicircum}
	\scnaddlevel{1}
	\scnhaselement{семантически связная знаковая конструкция}
	\scnhaselement{семантически несвязная знаковая конструкция}
	    \scnaddlevel{1}
    	\scnhaselementrole{пример}{\scnfilelong{В огороде бузина, а в Киеве дядька}}
    	\scnaddlevel{-1}
	\scnaddlevel{-1}
\scnhaselement{наличие разделителей и ограничителей\textasciicircum}
	\scnaddlevel{1}
	\scnhaselement{знаковая конструкция, содержащая разделители и-или ограничители}
	\scnhaselement{знаковая конструкция без разделителей и ограничителей}

\scnheader{язык}
\scnidtfexp{класс знаковых конструкций, для которого существуют (1) общие правила их построения и (2) общие правила их соотнесения с теми сущностями и конфигурациями сущностей, которые описываются (отражаются) указанными знаковыми конструкциями}
\scnidtf{класс знаковых конструкций, эквивалентных с точки зрения правил их построения и правил их семантической интерпретации}
\scnsubdividing{язык, у которого все знаки, входящие в состав его знаковых конструкций, являются элементарными фрагментами этих конструкций\\
	\scnaddlevel{1}
	\scnnote{Для языков такого типа существенно упрощаются методы обработки их текстов.}
	\scnaddlevel{-1}
;язык, у которого знаки, входящие в состав его знаковых конструкций, в общем случае не являются элементарными фрагментами этих конструкций}
\scnsubdividing{язык, знаковые конструкции которого содержат разделители и ограничители;язык, знаковые конструкции которого не содержат разделителей и ограничителей\\
	\scnaddlevel{1}
	\scntext{следствие}{Знаковые конструкции такого языка состоят \uline{только} из знаков.}
	\scnaddlevel{-1}}

\scnheader{отношение, заданное на множестве языков\textasciicircum}
\scnhaselement{текст заданного языка*}
	\scnaddlevel{1}
	\scnidtf{знаковая конструкция, принадлежащая заданному языку*}
	\scnidtf{синтаксически правильная (правильно построенная) знаковая конструкция заданного языка*}
	\scnidtf{синтаксически корректная и целостная знаковая конструкция для заданного языка*}
	\scneq{{\normalfont(}синтаксически корректная знаковая конструкция для заданного языка* $\cap$ синтаксически целостная знаковая конструкция для заданного языка*{\normalfont)}}
	\scnaddlevel{-1}
\scnhaselement{синтаксически корректная знаковая конструкция для заданного языка*}
	\scnaddlevel{1}
	\scnidtf{знаковая конструкция, не содержащая синтаксических ошибок для заданного языка}
	\scnaddlevel{-1}
\scnhaselement{синтаксически целостная знаковая конструкция для заданного языка*}
\scnhaselement{синтаксически неправильная знаковая конструкция для заданного языка*}
	\scnaddlevel{1}
	\scneq{{\normalfont(}синтаксически некорректная знаковая конструкция для заданного языка* $\cup$ синтаксически нецелостная знаковая конструкция для заданного языка*{\normalfont)}}
	\scnsuperset{синтаксически некорректная знаковая конструкция для заданного языка*}
	\scnsuperset{синтаксически нецелостная знаковая конструкция для заданного языка*}
	\scnaddlevel{-1}
\scnhaselement{знание, представленное на заданном языке*}
	\scnaddlevel{1}
	\scnidtf{семантически правильный текст заданного языка*}
	\scneq{(семантически корректный текст заданного языка* $\cap$ семантически целостный текст заданного языка*)}
	\scnidtf{истинный текст заданного языка*}
	\scnidtf{истинное высказывание, представленное на заданном языке*}
	\scnaddlevel{-1}
\scnhaselement{семантически корректный текст заданного языка*}
	\scnaddlevel{1}
	\scnidtf{текст заданного языка, не содержащий семантических ошибок, противоречащих признанным закономерностям и фактам*}
	\scnaddlevel{-1}
\scnhaselement{семантически целостный текст заданного языка*}
	\scnaddlevel{1}
	\scnidtf{текст заданного языка, содержащий достаточную информацию для установления его истинности*}
	\scnaddlevel{-1}
\scnhaselement{семантически неправильный текст для заданного языка*}
	\scnaddlevel{1}
	\scneq{(семантически некорректный текст для заданного языка* $\cup$ семантически нецелостный текст для заданного языка*)}
	\scnsuperset{семантически некорректный текст для заданного языка*}
	\scnsuperset{семантически нецелостный текст для заданного языка*}
	\scnaddlevel{-1}
\scnhaselement{алфавит*}
	\scnaddlevel{1}
	\scnidtf{алфавит заданной информационной конструкции или заданного языка*}
	\scnidtf{семейство классов, синтаксически эквивалентных элементов (элементарных фрагментов) заданной информационной конструкции или информационных конструкций заданного языка*}
	\scnaddlevel{-1}
\scnhaselement{семейство классов синтаксически эквивалентных разделителей*}
	\scnaddlevel{1}
	\scnidtf{семейство классов синтаксически эквивалентных разделителей, входящих в состав заданной информационной конструкции или в состав информационных конструкций заданного языка*}
	\scnaddlevel{-1}
\scnhaselement{семейство классов синтаксически эквивалентных ограничителей*}
	\scnaddlevel{1}
	\scnidtf{семейство классов синтаксически эквивалентных ограничителей, входящих в состав заданной информационной конструкции или в состав информационных конструкций заданного языка*}
	\scnaddlevel{-1}	
\scnhaselement{синтаксис языка*}
	\scnaddlevel{1}
	\scnidtf{быть теорией правильно построенных информационных конструкций, принадлежащих заданному языку*}
	\scnidtf{определение понятия правильно построенной информационной конструкции заданного языка*}
	\scnaddlevel{-1}
\scnhaselement{денотационная семантика*}
	\scnaddlevel{1}
	\scnidtf{быть теорией морфизмов, связывающих правильно построенные информационные конструкции заданного языка с описываемыми конфигурациями описываемых сущностей*}
	\scnaddlevel{-1}
\scnidtf{отношение, область определения которого включает в себя множество всевозможных языков}
\scnhaselement{семантическая эквивалентность языков*}
	\scnaddlevel{1}
	\scnidtf{быть семантически эквивалентными языками*}
	\scndefinition{Язык $\bm{L_i}$ и язык $\bm{L_j}$ будем считать \textit{семантически эквивалентными языками*} в том и только в том случае, если для каждого текста, принадлежащего языку $\bm{L_i}$, существует \textit{семантически эквивалентный текст*}, принадлежащий языку $\bm{L_j}$, и наоборот.}
	\scnaddlevel{-1}
\scnhaselement{семантическое расширение языка*}
	\scnaddlevel{1}
	\scnrelboth{обратное отношение}{семантическое сужение языка*}
	\scndefinition{Язык $\bm{L_j}$ будем считать \textit{семантическим расширением*} языка $\bm{L_i}$ в том и только в том случае, есть ли для каждого текста, принадлежащего языку $\bm{L_i}$, существует \textit{семантически эквивалентный текст*}, принадлежащий языку $\bm{L_j}$, но обратное неверно.}
	\scnaddlevel{-1}
\scnhaselement{синтаксическое расширение языка*}
	\scnaddlevel{1}
	\scnidtf{быть семантически эквивалентным надмножеством заданного языка*}
	\scndefinition{Язык $\bm{L_j}$ будем считать \textit{синтаксическим расширением*} языка $\bm{L_i}$ в том и только в том случае, если:
    \begin{scnitemize}
    \item $\bm{L_j} \supset \bm{L_i}$ (то есть все тексты языка $\bm{L_i}$ являются также и текстами языка $\bm{L_j}$, но обратное неверно);
    \item Язык $\bm{L_j}$ и язык $\bm{L_i}$ являются \textit{семантически эквивалентными языками*}.
    \end{scnitemize}
    }
	\scnaddlevel{-1}
\scnhaselement{синтаксическое ядро языка*}
	\scnaddlevel{1}
	\scnidtf{быть синтаксическим ядром заданного языка*}
	\scnidtf{быть семантически эквивалентным подмножеством заданного языка, имеющим минимальную синтаксическую сложность*}
	\scnaddlevel{-1}
\scnhaselement{направление синтаксического расширения ядра заданного языка*}
	\scnaddlevel{1}
	\scnidtf{быть правилом трансформации информационных конструкций, принадлежащих заданному языку, которое описывает одно из направлений перехода от множества конструкций ядра этого языка ко множеству всех принадлежащих ему информационных конструкций*}
	\scnaddlevel{-1}
\scnhaselement{внутренний язык*}
	\scnaddlevel{1}
	\scnidtf{быть внутренним языком для заданной системы, основанной на обработке информации, или заданного множества таких систем*}
	\scnidtf{быть языком внутреннего представления информации в памяти заданной системы, основанной на обработке информации или заданного класса таких систем*}
	\scnaddlevel{-1}
\scnhaselement{внешний язык*}
	\scnaddlevel{1}
	\scnidtf{быть внешним языком для заданной системы, основанной на обработке информации, или заданного множества таких систем*}
	\scnidtf{быть языком, используемым для обмена информацией заданной системы, основанной на обработке информации, или заданного множества таких систем с другими системами, основанными на обработке информации, (в том числе, и с себе подобными)*}
	\scnaddlevel{-1}
\scnhaselement{используемый язык*}
	\scnaddlevel{1}
	\scneq{{\normalfont(}внутренний язык* $\cup$ внешний язык*{\normalfont)}}
	\scnidtf{язык, используемый заданной системой, основанной на обработке информации или заданного множества таких систем*}
	\scnidtf{язык, носителем которого является (которым владеет) данная система, основанная на обработке информации}
	\scnaddlevel{-1}

\scnheader{параметр, заданный на множестве языков\textasciicircum}
\scnidtf{признак классификации языков}
\scnidtf{свойство, которым обладают языки}
\scnidtf{свойство языков, на основании которого можно осуществлять их классификацию}
\scnidtf{семейство классов эквивалентности языков, трактуемой в контексте того или иного свойства (характеристики), присущего языкам}
\scnhaselement{семантическая мощность языка\textasciicircum}
	\scnaddlevel{1}
	\scnidtf{класс языков, семантически эквивалентных друг другу}
	\scnhaselement{универсальный язык}
		\scnaddlevel{1}
		\scnidtf{класс всевозможных универсальных языков}
		\scnnote{Очевидно, что все универсальные языки (если они действительно таковыми являются, а не только претендуют на это) семантически эквивалентны друг другу, т.е. имеют одинаковую семантическую мощность.}
		\scnaddlevel{-1}
	\scnaddlevel{-1}
\scnhaselement{уровень синтаксической сложности представления знаков в текстах языка\textasciicircum}
	\scnaddlevel{1}
	\scnhaselement{язык, в текстах которого все знаки представлены синтаксически элементарными фрагментами}
	\scnhaselement{язык, в текстах которого знаки в общем случае представлены синтаксически неэлементарными фрагментами}
	\scnaddlevel{-1}
\scnhaselement{использование разделителей и ограничителей в текстах языка\textasciicircum}
	\scnaddlevel{1}
	\scnhaselement{язык, в текстах которого не используются разделители и ограничители}
	\scnhaselement{язык, в текстах которого используются разделители и ограничители}
	\scnaddlevel{-1}
\scnhaselement{уровень сложности процедуры установления синонимии знаков в текстах языка\textasciicircum}
	\scnaddlevel{1}
	\scnhaselement{язык, в рамках каждого текста которого синонимичные знаки отсутствуют}
		\scnaddlevel{1}
		\scnexplanation{В текстах такого языка знак каждой описываемой сущности входит \uline{однократно}.}
		\scnaddlevel{-1}
	\scnhaselement{язык, в рамках которого синонимичные знаки представлены синтаксически эквивалентными фрагментами текстов}
	\scnhaselement{флективный язык}
		\scnaddlevel{1}
		\scnidtf{язык, в рамках которого синонимичные знаки могут быть представлены синтаксически неэквивалентными фрагментами, но фрагментами, являющимися модификациями некоторого "ядра"{} этих фрагментов (при склонении и спряжении этих знаков).}
		\scnaddlevel{-1}
	\scnhaselement{язык, в рамках которого синонимичные знаки в общем случае могут быть представлены синтаксически неэквивалентными текстовыми фрагментами, структура которых носит непредсказуемый характер}
	\scnaddlevel{-1}
\scnhaselement{наличие омонимии в текстах языка\textasciicircum}
	\scnaddlevel{1}
	\scnhaselement{язык, в текстах которого присутствует омонимия знаков}
		\scnaddlevel{1}
		\scnidtf{язык, в текстах которого присутствуют синтаксически эквивалентные, не синонимичные знаки}
		\scnaddlevel{-1}
	\scnhaselement{язык, в текстах которого омонимия знаков отсутствует}
	\scnaddlevel{-1}

\scnheader{язык ostis-системы}
\scnidtf{язык, используемый ostis-системами}
\scnidtf{Множество языков, которыми "владеют" ostis-системы}
\scnrelto{языки}{ostis-система}

\scnhaselement{SC-код}
\scnaddlevel{1}
\scnidtf{Semantic Computer Code}
\scnrelto{внутренний язык}{ostis-система}
\scniselement{универсальный язык}
\scnaddlevel{-1}

\scnhaselement{SCg-код}
\scnaddlevel{1}
\scnidtf{Semantic Code graphical}
\scnidtf{\textit{внешний язык*} ostis-систем, тексты которого представляют собой графовые структуры общего вида с точно заданной \textit{денотационной семантикой*}}
\scnrelto{внешний язык}{ostis-система}
\scniselement{универсальный язык}
\scnaddlevel{-1}

\scnhaselement{SCs-код}
\scnaddlevel{1}
\scnidtf{Semantic Code string}
\scnidtf{\textit{внешний язык*} ostis-систем, тексты которого представляют собой строки (цепочки) символов}
\scnrelto{внешний язык}{ostis-система}
\scniselement{универсальный язык}
\scnaddlevel{-1}

\scnhaselement{SCn-код}
\scnaddlevel{1}
\scnidtf{Semantic Code natural}
\scnidtf{\textit{внешний язык*} ostis-систем, тексты которого представляют собой двухмерные матрицы символов, являющиеся результатом форматирования, двухмерной структуризации текстов SCs-кода.}
\scnrelto{внешний язык}{ostis-система}
\scniselement{универсальный язык}
\scnaddlevel{-1}
\scnexplanation{Для представления \textit{баз знаний ostis-систем} используется целый ряд как \textit{универсальных языков}, так и \textit{специализированных языков}, как \textit{формальных языков}, так и \textit{естественных языков}, как \textit{внутренних языков}, обеспечивающих представление информации в памяти \textit{ostis-систем}, так и \textit{внешних языков}, обеспечивающих представление информации, вводимой в память \textit{ostis-систем}, либо выводимой из этой памяти. \textit{Естественные языки} используются исключительно для представления \textit{файлов}, хранимых в памяти \textit{ostis-системы} и формально специфицируемых в рамках \textit{базы знаний} этой \textit{ostis-системы}. 

Для эксплуатации \textit{интеллектуальных компьютерных систем}, построенных на основе \textit{SC-кода}, кроме способа абстрактного внутреннего представления баз знаний (\textit{SC-кода}) потребуются несколько способов внешнего изображения абстрактных \textit{sc-текстов}, удобных для пользователей и используемых при оформлении исходных текстов \textit{баз знаний} указанных интеллектуальных компьютерных систем и исходных текстов фрагментов этих \textit{баз знаний}, а также используемых для отображения пользователям различных фрагментов \textit{баз знаний} по пользовательским запросам. В качестве таких способов внешнего отображения \textit{sc-текстов} и предлагаются указанные выше внешние языки ostis-систем (\textit{SCg-код}, \textit{SCS-код} и  \textit{SCn-код}).

Для описания перечисленных \textit{языков}, используемых \textit{ostis-системами}, в каждом из них мы выделим \textit{ядро языка*}, которое является \textit{семантически эквивалентным языком*} для соответствующего языка и имеет минимальную синтаксическую сложность. Описание каждого из указанных языков строится как описание нескольких направлений синтаксического расширения выделенного \textit{языка-ядра}.
}

\scnnote{Все основные внешние формальные языки, используемые ostis-системами (SCg-код, SCs-код, SCn-код) являются различными вариантами внешнего представления текстов внутреннего языка ostis-систем -- SC-кода. Указанные языки являются универсальными и, следовательно, \textit{семантически эквивалентными языками*}.}
\scnnote{Каждая ostis-система может приобрести способность использовать любой внешний язык (как универсальный, так и специализированный, как естественный, так и искусственный), если синтаксис и денотационная семантика этого языка будут описаны в памяти ostis-системы на ее внутреннем языке (SC-коде).}

\scnheader{следует отличать*}
\scnhaselementset{семантическое расширение языка*\\
\scnaddlevel{1}
\scnidtf{расширение семантической мощности языка*}
\scnaddlevel{-1}
;синтаксическое расширение языка*\\
\scnaddlevel{1}
\scnidtf{расширение синтаксиса языка при сохранении его семантической мощности*}
\scnaddlevel{-1}
}
\scnhaselementset{синтаксическое расширение языка*;направление синтаксического расширения ядра заданного языка*\\
\scnaddlevel{1}
\scnidtf{одно из (или группа) правил синтаксической трансформации текстов заданного языка, расширяющих множество синтаксически правильных текстов этого языка}
\scnnote{Таких направлений синтаксического расширения заданного языка может быть несколько.}
\scnaddlevel{-1}
}


\scnendstruct

\end{SCn}


\scsubsection{Введение в описание внутреннего языка ostis-систем}
\label{intro_sc_code}

\begin{SCn}

\scnsectionheader{\currentname}

\scnstartsubstruct

\scnsegmentheader{Первый сегмент Введения описание внутреннего языка ostis-систем}
\scnstartsubstruct

\scnheader{\currentname}
\scnreltovector{конкатенация сегментов}{Первый сегмент Введения описание внутреннего языка ostis-систем;Описание Ядра SC-кода;Описание Расширения Ядра SC-кода}

\scnheader{SC-код}
\scnidtf{Внутренний язык ostis-систем}
\scnidtf{Множество sc-текстов}
\scnidtf{sc-текст}
\scnidtf{Множество sc-конструкций}
\scnidtf{Язык унифицированного смыслового представления знаний в памяти интеллектуальных компьютерных систем}
\filemodetrue
\scnrelfromvector{принципы, лежащие в основе}{Знаки (обозначения) всех сущностей, описываемых в \textit{sc-текстах} (текстах SC-кода) представляются в виде синтаксически элементарных (атомарных) фрагментов \textit{sc-текстов} и, следовательно, не имеющих внутренней структуры, не состоящих из более простых фрагментов текста, как, например, имена (термины), которые представляют знаки описываемых сущностей в привычных языках и состоят из букв.;Имена (термины), естественно-языковые тексты и другие информационные конструкции, не являющиеся \textit{sc-текстами}, могут входить в состав \textit{sc-текста}, но только как файлы, описываемые (специфицируемые) \textit{sc-текстами}. Таким образом, в состав базы знаний \textit{интеллектуальной компьютерной системы}, построенной на основе \textit{SC-кода}, могут входить имена (термины), обозначающие некоторые описываемые сущности и представленные соответствующими файлами. Каждый sc-элемент будем называть внутренним обозначением некоторой сущности, а имя этой сущности, представленное соответствующим файлом, будем называть внешним идентификатором (внешним обозначением) этой сущности. При этом каждый именуемый (идентифицируемый) \textit{sc-элемент} связывается дугой, принадлежащей отношению "\textit{\textbf{быть внешним идентификатором*}}, с узлом, содержимым которого является файл идентификатора (в частности, имени), обозначающего ту же сущность, что и указанный выше \textit{sc-элемент}. Внешним обозначением может быть не только имя (термин), но и иероглиф, пиктограмма, озвученное имя, жест. Особо отметим, что внешние обозначения описываемых сущностей в интеллектуальной компьютерной системе, построенной на основе \textit{SC-кода}, используются только (1) для анализа информации, поступающей в эту систему из вне из различных источников, и ввода (понимания и погружения) этой информации в базу знаний, а также (2) для синтеза различных сообщений, адресуемых различным субъектам (в т.ч. пользователям).;Тексты \textit{SC-кода} (sc-тексты) имеют в общем случае нелинейную (графовую) структуру, поскольку (1) знак каждой описываемой сущности в ходит в состав sc-текста однократно и (2) каждый такой знак может быть инцидентен неограниченному числу других знаков, поскольку каждая описываемая сущность может быть связана неограниченным числом связей с другими описываемыми сущностями.;
База знаний, представленная текстом \textit{SC-кода}, является графовой структурой специального вида, алфавит элементов которой включает в себя множество узлов, множество ребер, множество дуг, множество базовых дуг -- дуг специально выделенного типа, обеспечивающих структуризацию баз знаний, а также множество специальных узлов, каждый из которых имеет содержимое, являющееся файлом, хранящимся в памяти интеллектуальной компьютерной системы. Структурная особенность данной графовой структуры заключается в том, что ее дуги и ребра могут связывать не только узел с узлом, но и узел с ребром или дугой, ребро или дугу с другим ребром или дугой.;
\uline{Все элементы} указанной выше графовой структуры (текста SC-кода), т.е. все ее узлы, ребра и дуги являются обозначениями различных сущностей. При этом ребро является обозначением бинарной неориентированной связки между двумя сущностями, каждая из которых либо представлена в рассматриваемой графовой структуре соответствующим знаком, либо является самим этим знаком. Дуга является обозначением бинарной ориентированной связки между двумя сущностями. Дуга специального вида (\textit{\textbf{базовая дуга}}) является знаком связи между узлом, обозначающим некоторое множество элементов рассматриваемой графовой структуры, и одним из элементов этой графовой структуры, который принадлежит указанному множеству. Узел, имеющий содержимое (узел, для которого содержимое существует, но может в текущий момент быть неизвестным) является знаком файла, который является содержимым этого узла. Узел, не являющийся знаком файла, может обозначать какой-либо материальный объект, первичный абстрактный объект(например, число, точку в некотором абстрактном пространстве), какую-либо бинарную связь, какое-либо множество (в частности, понятие, структуру, ситуацию, событие, процесс). При этом сущности, обозначаемые элементами рассматриваемой графовой структуры, могут быть постоянными (существующими всегда) и временными (сущностями, которым соответствует отрезок времени их существования). Кроме того, сущности, обозначаемые элементами рассматриваемой графовой структуры, могут быть константными (конкретными) сущностями и переменными (произвольными) сущностями. Каждому элементу рассматриваемой графовой структуры, являющемуся обозначением переменной сущности, ставится в соответствие область возможных значений этого обозначения. Область возможных значений каждого переменного ребра является подмножеством множества всевозможных константных ребер, область возможных значений каждой переменной дуги является подмножеством множества всевозможных константных дуг, область возможных значений каждого переменного узла является подмножеством множества всевозможных константных узлов.;
В рассматриваемой графовой структуре, являющейся представлением базы знаний в SC-коде, могут, но не должны существовать разные элементы графовой структуры, обозначающие одну и ту же сущность. Если пара таких элементов обнаруживается, то эти элементы склеиваются (отождествляются). Таким образом, синонимия внутренних обозначений в базе знаний интеллектуальной компьютерной системы, построенной на основе \textit{SC-кода,} запрещена. При этом синонимия внешних обозначений считается нормальным явлением. Формально это означает, что из некоторых элементов рассматриваемой графовой структуры выходит несколько дуг, принадлежащих отношению "\textit{\textbf{быть внешним идентификатором*}}". Из всех указанных дуг, принадлежащих отношению "\textit{\textbf{быть внешним идентификатором*}}" и выходящих из одного элемента рассматриваемой графовой структуры, обязательно выделяется одна (очень редко две) путем включения их в число дуг, принадлежащих отношению "\textit{\textbf{быть основным внешним идентификатором*}}". Это означает, что указываемый таким образом внешний идентификатор не является омонимичным, т.е. не может быть использован как внешний идентификатор, соответствующий другомуэлементу рассматриваемой графовой структуры.;
Кроме файлов, представляющих различные внешние обозначения (имена, иероглифы, пиктограммы), в памяти интеллектуальной компьютерной системе, построенной на основе \textit{SC-кода,} могут хранится файлы различных текстов (книг, статей, документов, примечаний, комментариев, пояснений, чертежей, рисунков, схем, фотографий, видео-материалов, аудио-материалов).;
\uline{Любую сущность}, требующую описания, можно обозначить в виде sc-элемента. Особо подчеркнем, что sc-элементы являются не просто обозначениями различных описываемых сущностей, а обозначениями, которые являются элементарными (атомарными) фрагментами знаковой конструкции, т.е. фрагментами, детализация структуры которых не требуется для "прочтения" и понимания этой знаковой конструкции.;
Текст \textit{\textbf{SC-кода}}, как и любая другая графовой структура, является абстрактным математическим объектом, не требующим детализации (уточнения) его кодирования в памяти компьютерной системы (например, в виде матрицы смежности, матрицы инцидентности, списковой структуры). Но такая детализация потребуется для технической реализации памяти, в которой хранятся и обрабатываются sc-тексты.;
Важнейшим дополнительным свойством \textit{\textbf{SC-кода}} является то,что он удобен не просто для внутреннего представления знаний в памяти интеллектуальной компьютерной системы, но также и для внутреннего представления информации в памяти компьютеров, специально предназначенных для интерпретации семантических моделей интеллектуальных компьютерных систем. Т.е., SC-код определяет синтаксические, семантические и функциональные принципы организации памяти компьютеров нового поколения, ориентированных на реализацию интеллектуальных компьютерных систем, -- принципы организации графодинамической ассоциативной семантической памяти.;
SC-код рассматривается нами как объединение нескольких его подъязыков, в число которых входит ядро SC-кода и его расширение, обеспечивающее ввод и вывод информации для ostis-системы на всевозможных внешних языках.
}
\filemodefalse
\scnaddlevel{1}
\scnsourcecomment{Завершили описание принципов SC-кода}
\scnaddlevel{-1}

\scnendstruct

\scnsegmentheader{Описание Ядра SC-кода}
\scnstartsubstruct

\scnheader{Ядро SC-кода}
\scnrelfrom{алфавит}{Алфавит Ядра SC-кода}
\scnaddlevel{1}
\scnhaselement{sc-узел}
\scnhaselement{sc-ребро}
 \scnaddlevel{1}
 \scnidtf{обозначение бинарной неориентированной связи между sc-элементами}
 \scnaddlevel{-1}
\scnhaselement{sc-дуга}
\scnaddlevel{1}
 \scnidtf{обозначение бинарной ориентированной связи между sc-элементами}
 \scnaddlevel{-1}
\scnhaselement{базовая sc-дуга}
 \scnaddlevel{1}
 \scnidtf{sc-дуга константной позитивной стационарной принадлежности}
 \scnidtf{знак константной позитивной стационарной пары принадлежности}
 \scnaddlevel{-1}
\scnnote{Подчеркнем, что с помощью указанных типов sc-элементов можно описать любые связи между sc-элементами, трактуя эти связи как множества связываемых sc-элементов и используя некоторые sc-узлы как знаки этих множеств.}
\scnaddlevel{-1}

\scnendstruct

\scnsegmentheader{Описание Расширения Ядра SC-кода}
\scnstartsubstruct

\scnheader{SC-код}
\scnidtf{Расширение Ядра SC-кода}
\scnidtf{Результат введения в Ядро SC-кода sc-узлов, имеющих содержимое и обозначающих файлы, хранимые в памяти ostis-системы}
\scnnote{Все файлы, представляющие собой электронные образы инородных для SC-кода информационных конструкций, можно представить в SC-кода с помощью графовых структур, в которых sc-элементы обозначают буквы текстов или пиксели изображений. Но такой вариант кодирования внешних для ostis-системы информационных конструкций не дает возможности непосредственно использовать накопленный человечеством арсенал электронных информационных ресурсов.}
\scnnote{Важнейшим видом файлов ostis-систем являются внешние идентификаторы sc-элементов (в частности, имена sc-элементов), представляющие sc-элементы в текстах внешних языков (в том числе, текстах SCs-кода и SCn-кода)} 
\scnnote{Результатом просмотренного расширения \textit{Ядра SC-кода} является расширение \textit{Алфавита Ядра SC-кода}}

\scnheader{SC-код}
\scnrelfrom{алфавит}{Алфавит SC-кода}
\scnaddlevel{1}
\scnhaselement{sc-узел}
\scnhaselement{sc-ребро}
\scnhaselement{sc-дуга}
\scnhaselement{базовая sc-дуга}
\scnhaselement{файл ostis-системы}
\scnaddlevel{-1}

\scnheader{файл ostis-системы}
\scnidtf{sc-узел с содержимым}
\scnidtf{sc-узел, имеющий содержимое}
\scnidtf{sc-узел, обозначающий файл, хранимый в памяти ostis-системы}
\scnidtf{знак файла ostis-системы}
\scnreltoset{разбиение}{ея-файл ostis-системы\\
\scnaddlevel{1}
\scnidtf{естественно-языковой файл ostis-системы}
\scnaddlevel{-1};файл ostis-системы, являющийся текстом формального языка\\
\scnaddlevel{1}
\scnsuperset{sc.g-файл ostis-системы}
\scnsuperset{sc.s-файл ostis-системы}
\scnsuperset{sc.n-файл ostis-системы}
\scnaddlevel{-1};файл ostis-системы, отражающий процесс изменения sc.g-текста;графический файл ostis-системы;файл ostis-системы, являющийся изображением;видео-файл ostis-системы;аудио-файл ostis-системы}
\scnreltoset{разбиение}{файл-экземпляр ostis-системы
\scnaddlevel{1}
\scnidtf{файл, являющийся конкретным электронным документом или электронным образом конкретной внешней информационной конструкции}
\scnaddlevel{-1};файл-класс ostis-системы
\scnaddlevel{1}
\scnidtf{файл, являющийся знаком множества всевозможных экземпляров (копий) этого файла}
\scnaddlevel{-1}
}

\scnheader{SC-код}
\scnrelfrom{синтаксис}{Cинтаксис SC-кода} 
\scnaddlevel{1}
\scnexplanation{\textit{\textbf{Синтаксис}} \textit{\textbf{SC-кода}} задается
\begin{scnitemize}
\item типологией (алфавитом) sc-элементов (атомарных фрагментов текстов sc-кода);
\item правилами соединения (инцидентности) sc-элементов (например, sc-элементы каких типов не могут быть инцидентными друг другу);
\item типологией конфигураций sc-элементов (связки, классы, структуры), связями между конфигурациями sc-элементов (в частности, теоретико-множественными)
\end{scnitemize}
}
\scnaddlevel{-1}
\scnrelfrom{денотационная семантика}{Денотационная семантика SC-кода} 
\scnaddlevel{1}
\scnexplanation{\textit{\textbf{Денотационная семантика}} \textit{\textbf{SC-кода}} задается
\begin{scnitemize}
\item
 семантической интерпретацией sc-элементов и их конфигураций;
\item
 семантической интерпретацией инцидентности sc-элементов;
\item
 иерархической системой предметных областей;
\item
 структурой используемых понятий в каждой предметной области (исследуемые классы объектов, исследуемые отношения, исследуемые классы объектов отношений из смежных предметных областей, ключевые экземпляры исследуемых классов объектов);
\item
 онтологиями предметных областей.
\end{scnitemize}
}
\scnaddlevel{-1}
\scnnote{Следует особо подчеркнуть, что  унификация и максимально возможное упрощение  \textbf{\textit{синтаксиса}} и \textbf{\textit{денотационной семантики}} внутреннего языка интеллектуальных компьютерных систем необходимы потому, что подавляющий объем \textbf{\textit{знаний}}, хранимых в составе  базы знаний интеллектуальной компьютерной системы, представляют собой \textbf{\textit{метазнания}}, описывающими свойства других знаний. Более того, по указанной причине конструктивное (формальное) развитие теории интеллектуальных компьютерных систем невозможно без уточнения (унификации, стандартизации) и обеспечения семантической совместимости различных видов знаний, хранимых в базе знаний интеллектуальной компьютерной  системы.  Очевидно, что многообразие форм представления семантически эквивалентных знаний делает разработку общей теории  интеллектуальных компьютерных систем практически невозможной. К \textit{метазнаниям}, в частности, следует отнести и различного вида логические высказывания и всевозможного вида программы, описания методов (навыков). Обеспечивающих решение различных классов информационных задач.}

\scnendstruct~
\scnsourcecomment{Завершили сегмент "Описание расширения Ядра SC-кода"}

\scnendstruct~
\scnsourcecomment{Завершили раздел "\currentname"}

\end{SCn}

%
\scsubsection{Введение в описание внешних идентификаторов sc-элементов}
\label{intro_idtf}

\begin{SCn}

\scnsectionheader{\currentname}

\scnstartsubstruct

\scnsegmentheader{Понятие внешнего идентификатора sc-элемента}

\scnstartsubstruct

\scnheader{\currentname}
\scnreltovector{конкатенация сегментов}{Понятие внешнего идентификатора sc-элемента;Понятие простого идентификатора sc-элемента;Понятие сложного идентификатора sc-элемента}

\scnheader{sc-идентификатор}
\scnidtf{внешний идентификатор sc-элемента}
\scntext{пояснение}{Внешние идентификаторы \textit{sc-элементов} (или, сокращенно \scnkeyword{sc-идентификаторы}) необходимы \textit{ostis-системе} исключительно для того, чтобы осуществлять обмен информацией с другими \textit{ostis-системами} или со своими пользователями. Для того чтобы представить свою \textit{базу знаний}, решать самые различные \textit{задачи}, связанные с анализом текущего состояния и эволюцией своей \textit{базы знаний}, задачи, связанные с анализом текущего состояния (текущих ситуаций) окружающей среды, принятием соответствующих решений (целей) и организацией соответствующих \textit{действий}, направленных на выполнение принятых решений (на достижение поставленных целей), \textit{ostis-системе} не нужны никакие внешние идентификаторы (в частности, имена) соответствующие \textit{sc-элементам}. Но для \uline{понимания} сообщений, принимаемых от других субъектов (что для \textit{ostis-системы} означает построение \textit{sc-текста},~~ \textit{семантически эквивалентного} принятому сообщению) и для анализа сообщений, передаваемых другим субъектам (что для \textit{ostis-системы} означает синтез \textit{внешнего текста},~~\textit{семантически эквивалентного} заданному \textit{sc-тексту} и удовлетворяющего некоторым дополнительным требованиям, например, эмоционального характера) \textit{ostis-системе} необходимо знать, как в принимаемом или передаваемом сообщении изображаются (представляются) \textit{знаки}, \uline{синонимичные sc-элементам}, которые уже хранятся или могут храниться в составе \textit{базы знаний}~~\textit{ostis-системы}. В качестве внешних идентификаторов \textit{sc-элементов} чаще всего используются имена (термины) соответствующих (обозначаемых) сущностей, представленные отдельными словами или словосочетаниями на различных естественных языках, но также могут использоваться иероглифы, условные обозначения, пиктограммы.

В общем случае \textit{sc-элементу} может соответствовать несколько синонимичных ему имен на разных \textit{естественных языках}. Более того, \textit{sc-элементу} может соответствовать несколько синонимичных ему имен на одном и том же \textit{естественном языке}. В этом случае одно из этих имен объявляется как основной внешний идентификатор для соответствующего \textit{sc-элемента} и соответствующего \textit{естественного языка}. Основное требование, предъявляемое к таким внешним идентификаторам это отсутствие как синонимов, так и омонимов в рамках множества основных внешних идентификаторов sc-элементов для каждого естественного языка. 

Каждый внешний идентификатор \textit{sc-элемента}, используемый ostis-системой, описывается (представляется) в её памяти в виде \textit{внутреннего файла ostis-системы}, т.е. в виде электронного образа всевозможных вхождений данного внешнего идентификатора во всевозможные внешние тексты соответствующего внешнего языка.}
\scnidtf{внешний идентификатор sc-элемента}
\scnsubdividing{простой sc-идентификатор\\
\scnaddlevel{1}
\scnidtf{простой внешний идентификатор sc-элемента}
\scnaddlevel{-1}
;sc-выражение\\
\scnaddlevel{1}
\scnidtf{сложный внешний идентификатор sc-элемента, в состав которого входит один или несколько идентификаторов других sc-элементов} 
\scnaddlevel{-1}
}
\scnsubdividing{основной sc-идентификатор\\
\scnaddlevel{1}
\scnidtf{основной sc-идентификатор для носителей дополнительно указываемого языка общения (например, соответствующего естественного языка)}
\scnsuperset{основной международный sc-идентификатор}
\scnaddlevel{1}
\scntext{примечание}{В качестве \textit{основных sc-идентификаторов} могут использоваться также общепринятые международные условные обозначения некоторых сущностей, например, обозначения часто используемых функций (sin, cos, tg, log, и т.д.), единиц измерения, денежных единиц и многое другое. Формально каждый основной международный sc-идентификатор считается основным sc-идентификатором также и для каждого естественного языка, несмотря на то, что символы, используемые в основных международных sc-идентификаторах, могут не соответствовать алфавиту некоторых или даже всех естественных языков.}
\scnaddlevel{-1}
\scnsuperset{системный sc-идентификатор}
\scnaddlevel{1}
\scnidtf{основной sc-идентификатор для языка общения между ostis-системами}
\scnaddlevel{1}
\scntext{примечание}{В качестве указанного языка общения между ostis-системами используется SCs-код.}
\scnaddlevel{-1}
\scntext{примечание}{системный sc-идентификатор часто совпадает с основным англоязычным sc-идентификатором.}
\scnaddlevel{-2}
;неосновной sc-идентификатор\\
\scnaddlevel{1}
\scntext{примечание}{С помощью неосновных sc-идентификаторов указываются возможные \textit{синонимы*} соответствующего \textit{основного sc-идентификатора}, которые в частности, могут пояснять или даже определять обозначаемую сущность, указывает на важные свойства этой сущности.}
\scnsuperset{{\normalfont(}неосновной sc-идентификатор $\cap$ пояснение{\normalfont)}}
\scnaddlevel{1}
\scnidtf{неосновной sc-идентификатор, являющийся одновременно и пояснением обозначаемой сущности}
\scnsuperset{{\normalfont(}неосновной sc-идентификатор $\cap$ определение{\normalfont)}}
\scnaddlevel{1}
\scnidtf{неосновной sc-идентификатор, являющийся одновременно и определением обозначаемого понятия}
\scnaddlevel{-2}
\scnsuperset{неосновной часто используемый sc-идентификатор}
\scnaddlevel{1}
\scntext{пояснение}{Для некоторых sc-элементов могут часто использоваться не только основные, но и неосновные sc-идентификаторы (особенно в неформальных текстах -- в пояснениях, примечаниях и т.п.). Явное выделение такого класса sc-идентификаторов позволяет упростить семантический анализ исходных текстов баз знаний.}
\scnaddlevel{-2}
}
\scntext{примечание}{Каждому sc-элементу может соответствовать целое семейство внешних идентификаторов этого sc-элемента, которые обычно являются терминами, именующими обозначаемую сущность. Среди этих внешних идентификаторов для каждого идентифицируемого sc-элемента выделяется один как основной идентификатор. А неосновные термины (имена), соответствующие этим sc-элементам (в том числе и sc-классам), поясняют денотационную семантику указанного sc-элемента.}

\bigskip
\scnstartset
\scnheader{основной sc-идентификатор}
\scnsubset{файл-образец ostis-системы}
\scnendstruct

\scnrelboth{семантическая эквивалентность}{\scnfilelong{Все основные идентификаторы sc-элементов в памяти ostis-системы оформляются в виде копируемых фалов-образцов ostis-системы.}}
\scnaddlevel{1}
\scntext{пояснение}{Копии основных sc-идентификаторов входят в состав внешних текстов различных языков (SCg-кода, SCs-кода, SCn-кода), а также в различных падежах, склонения, спряжениях в состав файлов ostis-систем.}
\scnaddlevel{-1}
\scntext{примечание}{Аналогичное утверждение справедливо и для неосновных часто используемых sc-идентификаторов. Все остальные неосновные sc-идентификаторы считаются вспомогательными файлами-экземплярами.}

\scnheader{sc-идентификатор}
\scnsubdividing{строковый sc-идентификатор\\
\scnaddlevel{1}
\scnidtf{sc-идентификатор, представленный строкой символов, которая является именем обозначаемой сущности}
\scnidtf{имя сущности, обозначаемой идентифицируемым sc-элементом}
\scnidtf{имя (термин, словосочетание), синонимичное соответствующему (идентифицируемому) sc-элементу и представленное в соответствующем алфавите символов}
\scnaddlevel{-1}
;sc-идентификатор, представленный иероглифами;sc-идентификатор, представленный условным обозначением или пиктограммой}
\scntext{примечание}{Введенные нами sc-идентификаторы используются во всех внешних языках, близких SC-коду -- в SCg-коде, в SCs-коде и в SCn-коде.}

\scnheader{строковый sc-идентификатор}
\scnidtf{имя, приписываемое идентифицируемому sc-элементу}
\scnidtf{имя сущности, обозначаемой идентифицируемым sc-элементом}
\scnidtf{строка символов, синонимичная соответствующему идентифицируемому sc-элементу}
\scnsuperset{основной строковый sc-идентификатор}
\scnaddlevel{1}
\scnidtf{уникальное для каждого естественного языка внешнее имя, приписываемое идентифицируемому sc-элементу}
\scnsuperset{основной русскоязычный sc-идентификатор}
\scnsuperset{системный sc-идентификатор}
\scnsuperset{основной англоязычный sc-идентификатор}
\scnsuperset{основной германоязычный sc-идентификатор}
\scnsuperset{основной франкоязычный sc-идентификатор}
\scnsuperset{основной италоязычный sc-идентификатор}
\scnsuperset{основной китайскоязычный sc-идентификатор}
\scnaddlevel{-1}
\scnheader{sc-идентификатор}
\scntext{примечание}{Представление знаков в виде неатомарных фрагментов информационных конструкций (в частности, в виде имен обозначаемых сущностей, построенных в фиксированном алфавите), взаимно однозначно соответствующих обозначенным сущностям, необходимо только для того, чтобы иметь простую процедуру установления синонимии знаков, входящих в состав одной или разных информационных конструкций.}

\scnendstruct

\scnsegmentheader{Понятие простого идентификатора sc-элемента}

\scnstartsubstruct

\scnheader{простой идентификатор sc-элемента}
\scntext{правила построения}{Правила построения простых sc-идентификаторов.\\
Данные правила включают в себя:
\begin{scnitemize}
    \item Символы, используемые в простых sc-идентификаторах (в том числе, специальные символы);
    \item Специальные предикаты, используемые в простых sc-идентификаторах;
    \item Специальные суффиксы, используемые в простых sc-идентификаторах;
    \item Разделители, используемые в простых sc-идентификаторах;
    \item Ограничители, используемые в простых sc-идентификаторах;
    \item Правила построения простых sc-идентификаторов, определяемые различными классами идентифицируемых сущностей;
    \item Правила построения sc-имен собственных и sc-имен нарицательных.
\end{scnitemize}
Общим правилом построения простых sc-идентификаторов является стремление максимально возможным образом использовать сложившуюся терминологию. Но при этом следует подчеркнуть, что необходимость исключения омонимии в sc-идентификаторах требует строгого формального \uline{уточнения} семантической интерпретации каждого используемого термина. Особо подчеркнем то, что в ostis-системах процесс построения новых терминов (sc-идентификаторов) и процесс совершенствования существующей терминологии по отношению к процессу развития ostis-систем, баз знаний, представленных в SC-коде, с технической точки зрения абсолютно не зависят друг от друга. Кроме того, следует помнить, что \uline{далеко не все} sc-элементы, входящие в состав базы знаний ostis-системы, должны иметь соответствующие им sc-идентификаторы (быть идентифицированными). Очевидно, что идентифицированными (именованными) должны быть все используемые понятия, вводимые в соответствующих предметных областях и специфицируемые соответствующими онтологиями. Идентифицированными также должны быть обладающие особыми свойствами ключевые экземпляры (элементы) некоторых понятий, различные социально значимые объекты (персоны, населенные пункты, географические объекты, страны, организации, библиографические источники и многое другое).\\
Рассмотрим правила построений простых sc-идентификаторов, определеяемые различными классами идентифицируемых сущностей:
\begin{scnitemize}
    \item Первым символом каждого простого sc-идентификатора и каждого сложного sc-идентификатора, идентифицирующего sc-переменную (переменный sc-элемент), является подчеркнутый пробел;
    \item Последним символом простого sc-идентификатора, идентифицирующего sc-узел, обозначающий неролевое отношение, заданное на множестве sc-элементов, является крестик в виде верхнего индекса;
    \item Последним символом простого sc-идентификатора, идентифицирующего sc-узел, обозначающий заданное на множестве sc-элементов ролевое отношение (т.е. отношение, являющееся подмножеством отношения принадлежности), является апостроф (черточка в виде верхнего индекса);
    \item Последним символом простого sc-идентификатора, идентифицирующего sc-узел, обозначающий понятие, не являющееся отношением, (таковыми, в частности, являются различного рода параметры — длина, площадь, объем, масса) является звездочка в виде верхнего индекса;
    \item В рамках SCs-кода целесообразно вводить правила унифицированного построения простых sc-идентификаторов и целого ряда других классов идентифицируемых сущностей — персон, библиографических источников (публикаций), разделов баз знаний ostis-систем, файлов ostis-систем, самих ostis-систем.
\end{scnitemize}}

\scnheader{имя нарицательное}
\scnidtf{имя, которое может быть приписано \uline{любому} экземпляру некоторого класса и которое обозначает указанный класс}
\scntext{примеры}{треугольник; город; персона; отношение; параметр; константа; переменная}
\scnnote{\textit{имя нарицательное} всегда начинается с маленькой буквы}

\scnheader{имя собственное}
\scnidtf{имя, которое либо не является обозначением какого-либо класса сущностей, либо является обозначением (именем) некоторого класса сущностей, но построенным без использования нарицательного имени этого класса, либо является именем некоторого класса сущностей, построенным с использованием нарицательного имени этого класса (1) путем преобразования имени нарицательного во множественное число или (2) путем дополнительного использования в начале формируемого имени собственнного таких терминов, как "Класс...", "Множество...", "Множество всевозможных...".}
\scntext{примеры}{Москва; Иванов Иван Сергеевич; Точка А; Город Минск; SC-код; Русский язык; Множество всевозможных sc-текстов; Класс sc-текстов; Класс русскоязычных текстов.}
\scnnote{имя собственное всегда начинается с большой буквы}

\scnendstruct

\scnsegmentheader{Понятие сложного идентификатора sc-элемента}

\scnstartsubstruct

\scnheader{сложный sc-идентификатор}
\scnsuperset{сложный sc-идентификатор, идентифицирующее sc-коннектор}
\scnsuperset{сложный sc-идентификатор, ограничиваемое фигурными скобками и обозначающее множество sc-элементов, все sc-идентификаторы которых перечисляются}
\scnsuperset{сложный sc-идентификатор, ограничиваемое фигурными скобками и обозначающее множество sc-элементов, входящих в состав sc-текста, который семантически эквивалентен тому тексту (sc.s-тексту, sc.g-тексту, ея-тексту и т.д.), который ограничен указанными фигурными скобками}
\scnsuperset{сложный sc-идентификатор, ограничиваемое квадратными скобками и обозначающее файл-экземпляр ostis-системы}
    \scnaddlevel{1}
    \scnrelfrom{смотрите}{Принципы SC-кода}
        \scnaddlevel{1}
        \scniselement{секция раздела базы знаний}
        \scnaddlevel{-1}
    \scnaddlevel{-1}
\scnsuperset{сложный sc-идентификатор, ограничиваемое квадратными скобками и дополнительными вертикальными линиями (или кавычками) и обозначающее файл-класс ostis-системы}
    \scnaddlevel{1}
    \scnrelfrom{смотрите}{Принципы SC-кода}
        \scnaddlevel{1}
        \scniselement{секция раздела базы знаний}
        \scnaddlevel{-1}
    \scnaddlevel{-1}
\scnsuperset{сложный sc-идентификатор, использующее знаки алгебраических операций}
    \scnaddlevel{1}
    \scntext{примеры}{($s_i \cup s_j \cup s_k$); ($s_i \cap s_j \cap s_k$); ($s_i \backslash s_j$); ($x+y+z$); ($x \times y \times z$)}
    \scnaddlevel{-1}
\scnsuperset{сложный sc-идентификатор, обозначающее второй компонент пары указываемого ориентированного бинарного или квазибинарного отношения для указываемых аргументов\\
    \scnaddlevel{1}
    \scntext{примеры}{объединение*($s_i$; $s_j$; $s_k$); пересечение*($s_i$; $s_j$; $s_k$); разность множеств*($s_i$; $s_j$); сложение*($x$; $y$; $z$); умножение*($x$; $y$; $z$); sin*($x$); cos*($x$)}
    \scnaddlevel{-1}}
\scnexplanation{Использование сложныйх sc-идентификаторов позволяет существенно сократить число "придумываемых"\ sc-идентификаторов, каковыми в конечном счете становятся только простые sc-идентификаторы, поскольку, зная то, как связан идентифицируемый sc-элемент с теми sc-элементами, которые уже имеют sc-идентификаторы, во многих случаях можно построить сложный sc-идентификатор, идентифицирующее указанный sc-элемент. Кроме того, каждое сложный sc-идентификатор, являясь внешним идентификатором, является также и \uline{транслируемым} формальным текстом, содержащим некоторую информацию об обозначаемой ею сущности.}

\scnheader{сложный sc-идентификатор, идентифицирующий sc-коннектор}
\scnnote{Поскольку кратные sc-коннекторы одного и того же вида встречаются редко, сложный sc-идентификатор, идентифицирующий sc-коннектор, чаще всего \uline{однозначно} идентифицирует соответствующий sc-коннектор. В случае кратных sc-коннекторов одинакового семантического вида, отражаемого типом sc.s-коннектора, можно, например, sc-идентификаторам, идентифицирующим разные кратные sc-коннекторы, приписывать разные номера. Пусть, например, sc-элементы $e_i$ и $e_j$ соединены двумя \uline{кратными} константными постоянными sc-дугами. Тогда указанные sc-дуги можно идентифицировать следующими sc-идентификаторами:\\
($e_i => e_j$)1\\
($e_i => e_i$)2\\}

\scnheader{Файл-рисунок на странице 125}
\scnexplanation{В данном файле ostis-системы приведено определение понятия сложный sc-идентификатор, идентифицирующий sc-коннектор, представленное в SCg-коде и на Языке Бэкуса-Наура.}

\scnheader{Таблица.Типология сложных sc-идентифкаторов}
\scneqfile{\\
\includegraphics[width=1\linewidth]{figures/intro/scs-statements.pdf}\\
}

\scnendstruct \scninlinesourcecommentpar{Завершили представление Сегмента "\textit{Понятие сложного идентификатора sc-элемента}"}

\scnendstruct \scninlinesourcecommentpar{Завершили \textit{Введение в описание внешних идентификаторов sc-элементов}}

\end{SCn}
    
%
\scsubsection{Неформальное введение в язык визуального представления баз знаний ostis-систем}

\begin{SCn}

\scnsectionheader{\currentname}

\end{SCn}

%
\scsubsection{Введение в язык линейного представления баз знаний ostis-систем}
\label{intro_scs}

\begin{SCn}

\scnsectionheader{\currentname}

\scnstartsubstruct

\scnsegmentheader{Первый сегмент Введения в язык графического представления баз знаний ostis-систем}

\scnstartsubstruct

\scnheader{Введение в язык графического представления баз знаний ostis-систем}
\scnidtf{Введение в SCs-код}
\scnreltovector{конкатенация сегментов}{Первый сегмент Введения в SCs-код;Описание Алфавита SCs-кода;Описание sc.s-разделителей;Описание sc.s-ограничителей;Описание sc.s-предложений;Описание Ядра SCs-кода и различных направлений его расширения}

\scnheader{SCs-код}
\scnidtf{Semantic Code string}
\scnidtf{Язык линейного представления баз знаний ostis-систем}
\scnidtf{Множество всевозможных текстов SCs-кода}
\scnidtf{Тексты SCs-кода}	
\scnaddlevel{1}
\scniselement{имя собственное}
\scnaddlevel{-1}
\scnidtf{текст SCs-кода}	
\scnaddlevel{1}
\scniselement{имя нарицательное}
\scnaddlevel{-1}
\scnidtf{sc.s-текст}
\scniselement{линейный язык}
\scnrelfrom{алфавит}{Алфавит SCs-кода}
\scnrelfrom{разделители}{sc.s-разделитель}
\scnrelfrom{ограничители}{sc.s-ограничитель}
\scnrelfrom{предложения}{sc.s-предложение}
\scnrelfrom{идентификаторы}{sc.s-идентификатор}
\scnrelfrom{неоднозначные обозначения описываемых сущностей}{неоднозначное sc.s-изображение sc-элемента}
\scnexplanation{Множество линейных текстов (sc.s-текстов), каждый из которых состоит из предложений (sc.s-предложений), разделенных друг от друга двойной точкой с запятой (разделителем sc.s-предложений). При этом sc.s-предложение представляет собой последовательность sc.s-идентификаторов, являющихся именами описываемых сущностей и разделяемых между собой различными sc.s-разделителями и sc.s-ограничителями}

\scnheader{sc.s-идентификатор}
\scnidtf{идентификатор, построенный по правилам SCs-кода}
\scnidtf{имя, построенное по правилам SCs-кода, обозначающее (идентифицирующее) соответствующую сущность, а так же одновременно идентифицирующее синонимичный этому имени sc-элемент, обозначающий ту же сущность, что и указанное имя}

\scnaddlevel{1}
	\scnidtf{строковый идентификатор}
	\scnsubset{идентификатор}
	\scnexplanation{обозначение описываемой сущности, структура которого в формальных языках однозначно соответствует этой сущности}
\scnaddlevel{-1}
\scnsubdividing{
простой sc.s-идентификатор\\
\scnaddlevel{1}
\scnidtf{атомарный sc.s-идентификатор}
\scnidtf{sc.s-идентификатор некоторого sc-элемента, в состав которого другие sc.s-идентификаторы (sc.s-идентификаторы других sc-элементов) не входят}
\scnidtf{простой (атомарный) внешний строковый идентификатор sc-элемента}
\scnaddlevel{-1}
;sc.s-выражение\\
\scnaddlevel{1}
\scnidtf{неатомарный sc.s-идентификатор}
\scnidtf{sc.s-идентификатор, представляющий собой в общем случае иерархическую конфигурацию sc.s-идентификаторов, связываемых между собой соответствующими sc.s-разделителями и sc.s-ограничителями}
\scnaddlevel{-1}
}

\scnheader{неоднозначное sc.s-изображение sc-элемента}
\scnidtf{условное обозначение не именуемой (неидентифицируемой) сущности}
\scnsuperset{sc.s-коннектор}
\scnaddlevel{1}
    \scnidtf{неоднозначное sc.s-изображение sc-коннектора, являющееся также одновременно одним из видов sc.s-разделителей}
    \scnsubset{sc.s-разделитель}
\scnaddlevel{-1}
\scnsuperset{неоднозначное sc.s-изображение sc-узла}
\scnaddlevel{1}
    \scnsuperset{условное обозначение неименуемого множества sc-элементов}
    \scnaddlevel{1}
        \scnexplanation{условное обозначение неименуемого множества sc-элементов в SCs-коде представляется строкой из двух символов -- левой фигурной скобки и правой фигурной скобки. В SCs-коде, это соответствует кружочку с точкой внутри}
    \scnaddlevel{-1}
    \scnsuperset{условное обозначение неименуемого кортежа sc-элементов}
    \scnaddlevel{1}
        \scnexplanation{В SCs-коде такое обозначение представляется двух-символьной строкой левой угловой скобки и правой угловой скобки}
    \scnaddlevel{-1}
	\scnsuperset{условное обозначение неименуемого файла-экземпляра ostis-системы}
	\scnaddlevel{1}
		\scnexplanation{В SCs-коде такое обозначение представляется двух-символьной строкой левой квадратной скобки и правой квадратной скобки}
	\scnaddlevel{-1}
	\scnsuperset{условное обозначение не именуемого файла-класса ostis-системы}
	\scnaddlevel{1}
		\scnexplanation{В SCs-коде такое обозначение представляется трех-символьной строкой левой квадратной скобки, вертикальной черты (***), правой квадратной скобки}
	\scnaddlevel{-1}
\scnaddlevel{-1}
	
\scnendstruct

\scnsegmentheader{Описание Алфавита SCs-кода}
\scnstartsubstruct

\scnheader{Алфавит SCs-кода}
\scnidtf{Алфавит символов SCs-кода}
\scnidtf{множество символов SCs-кода}
\scnidtf{символ, используемый в текстах SCs-кода}
\scnreltoset{объединение}{символ, используемый в sc.s-разделителях;символ, используемый в sc.s-ограничителях;символ, используемый в простых sc.s-индетификаторах}

\scnheader{символ, используемый в sc.s-разделителях}
\scnhaselements{\textit{подчеркнутый пробел};\textit{дефис};\textit{запятая};\textit{пробел};\textit{точка};\textit{точка с запятой};\textit{двоеточие};\textit{“мячик”};\textit{знак равенства}}
\scnhaselement{знак инцидентности правого sc-коннектора}
\scnaddlevel{1}
\scneqfileclass{|-}
\scnaddlevel{-1}
\scnhaselement{знак инцидентности левого sc-коннектора}
\scnaddlevel{1}
\scneqfileclass{-|}
\scnaddlevel{-1}
\scnhaselement{знак инцидентности входящей sc-дуги справа}
\scnaddlevel{1}
\scneqfileclass{|<}
\scnaddlevel{-1}
\scnhaselement{знак инцидентности входящей sc-дуги слева}
\scnaddlevel{1}
\scneqfileclass{>|}
\scnaddlevel{-1}

\scnheader{символ, используемый в sc.s-разделителях}
\scnsuperset{символ, используемый в sc,s-коннекторах}
\scnaddlevel{1}
\scnhaselements{\scnfileclass{$\in$};\scnfileclass{$\ni$};\scnfileclass{$\notin$};\scnfileclass{$\not \ni$};\scnfileclass{***};\scnfileclass{***};\scnfileclass{$\sim$};\textit{знак подчеркивания};\textit{знак тире};\textit{знак равенства};\scnfileclass{>};\scnfileclass{<};\scnfileclass{$\subseteq$};\scnfileclass{$\supseteq$};\scnfileclass{$\subset$};\scnfileclass{$\supset$};\scnfileclass{$\leq$};\scnfileclass{$\geq$};\textit{двоеточие}}
\scnaddlevel{-1}

\scnheader{символ, используемый в sc.s-разделителях}
\scnhaselement{вертикальная черта}
\scnhaselement{двойная вертикальная черта}
\scnaddlevel{1}
\scnidtf{знак операции конкатенации}
\scnaddlevel{-1}
\scnhaselement{знак операции объединения множеств}
\scnaddlevel{1}
\scnidtf{$\cup$}
\scnaddlevel{-1}
\scnhaselement{знак пересечения множеств}
\scnaddlevel{1}
\scnidtf{$\cap$}
\scnaddlevel{-1}
\scnhaselement{знак операции разности множеств}
\scnaddlevel{1}
\scnidtf{$\backslash$}
\scnaddlevel{-1}
\scnhaselement{знак операции сложения}	
\scnaddlevel{1}
\scnidtf{+}
\scnaddlevel{-1}
\scnhaselement{знак операции умножение}	
\scnaddlevel{1}
\scnidtf{*}
\scnaddlevel{-1}
\scnhaselement{знак операции вычитания}	
\scnaddlevel{1}
\scnidtf{‑}
\scnaddlevel{-1}

\scnheader{символ, используемый в sc.s-ограничителях}
\scnhaselement{прямые кавычки}
\scnaddlevel{1}
\scnidtf{"}
\scnaddlevel{-1}
\scnhaselement{левая круглая скобка}
\scnaddlevel{1}
\scnidtf{(}
\scnaddlevel{-1}
\scnhaselement{правая круглая скобка}
\scnaddlevel{1}
\scnidtf{)}
\scnaddlevel{-1}
\scnhaselement{левая фигурная скобка}
\scnhaselement{правая фигурная скобка}
\scnhaselement{левая угловая скобка}
\scnhaselement{правая угловая скобка}
\scnhaselement{левая квадратная скобка}
\scnhaselement{правая квадратная скобка}
\scnhaselement{косая черта}
\scnhaselement{звездочка вверху}
\scnhaselement{левая цитатная кавычка}
\scnhaselement{правая цитатная кавычка}

\scnheader{символ, используемый в простых sc.s-идентификаторах}
\scnsuperset{буква Русского языка}
\scnaddlevel{1}
\scnnote{и заглавная, и строчная}
\scnaddlevel{-1}
\scnsuperset{буква Английского языка}	\scnaddlevel{1}
\scnnote{и заглавная, и строчная}
\scnaddlevel{-1}
\scnsuperset{цифра}
\scnsuperset{спецсимвол, используемый в простых sc.s-идентификаторах}
\scnaddlevel{1}
\scneqtoset{знак подчеркивания;тире;дефис;запятая;	пробел;точка;двойная точка;прямая кавычка;звездочка вверху;крестик вверху;апостроф;круглая скобка}
\scnaddlevel{-1}

\scnendstruct \scnsourcecomment{Завершили Описание Алфавита SCs-кода}

\scnsegmentheader{Описание sc.s-разделителей}
\scnstartsubstruct

\scnheader{sc.s-разделитель}	\scnidtf{разделитель, используемый в sc.s-текстах}
\scnsubdividing{sc.s-разделитель, используемый в простых sc.s-идентификаторах;sc.s-разделитель, используемый в sc.s-выражениях\\
\scnaddlevel{1}
    \scnsuperset{точка с запятой}
    \scnaddlevel{1}
        \scneqfileclass{;}
    \scnaddlevel{-1}
    \scnsuperset{"мячик"}
    \scnaddlevel{1}
        \scneqfileclass{$\bullet$}
    \scnaddlevel{-1}
\scnaddlevel{-1}
;sc.s-разделитель, используемый для структуризации sc.s-предложений\\
\scnaddlevel{1}
    \scnsuperset{sc.s-коннектор}
    \scnsuperset{sc.s-разделитель, изображающий связь инцидентности sc-элементов}
    \scnsuperset{двоеточие}
    \scnaddlevel{1}
        \scneqfileclass{:}
        \scnnote{Разделяет ***}
    \scnaddlevel{-1}
\scnaddlevel{-1}
;sc.s-разделитель sc.s-предложений\\
\scnaddlevel{1}
    \scneqfileclass{;;}
    \scnidtf{двойная точка с запятой}
\scnaddlevel{-1}
}

\scnheader{sc.s-разделитель, используемый в простых sc.s-идентификаторах}
\scnsuperset{пробел}
\scnaddlevel{1}
    \scnnote{используется в качестве разделителя между словами, входящими в состав простого sc.s-иден\-ти\-фи\-ка\-то\-ра}
\scnaddlevel{-1}
\scnsuperset{подчеркнутый пробел}
\scnsuperset{запятая}
\scnaddlevel{1}
    \scnnote{используется в качестве разделителя между некоторыми словами, входящими в состав простого sc.s-иден\-ти\-фи\-ка\-то\-ра}
\scnaddlevel{-1}
\scnsuperset{точка}
\scnaddlevel{1}
    \scneqfileclass{.}
    \scnnote{используется в конце аббревиатур, например, в инициалах}
\scnaddlevel{-1}
\scnsuperset{двойная точка}
\scnaddlevel{1}
    \scneqfileclass{..}
    \scnnote{используется в качестве разделителя между содержательно разными частями простого sc.s-иден\-ти\-фи\-ка\-то\-ра, состоящего из нескольких фраз}
\scnaddlevel{-1}
\scnsuperset{двойная точка}
\scnaddlevel{1}
    \scneqfileclass{-}
    \scntext{примеры использования}{sc-текст; sc.s-разделитель}
\scnaddlevel{-1}

\scnheader{sc.s-коннектор}
\scnidtf{изображение sc-коннектора во внешнем тексте SCs-кода или SCn-кода}
\scnsubset{sc.s-разделитель}
\scnnote{типология sc.g-коннекторов и, тем более, sc-коннекторов, т.к. она она учитывает устоявшиеся традиции изображения связок целого ряда конкретных отношений}
\scnsubdividing{ориентированный sc.s-коннектор;неориентированный sc.s-коннектор}
\scnsubdividing{sc.s-коннектор, соответствующий sc.g-дуге принадлежности\\
\scnaddlevel{1}
    \scnrelfrom{таблица}{Таблица. Алфавит sc.s-коннекторов, соответствующих sc.g-дугам принадлежности}
\scnaddlevel{-1}
;sc.s-коннектор, соответствующий sc.g-коннектору, который не является sc.g-дугой принадлежности\\
\scnaddlevel{1}
    \scnaddhind{-1}
    \scnrelfrom{таблица}{Таблица. Алфавит sc.s-коннекторов, соответствующих sc.g-коннекторам, которые не являются sc.g-дугами принадлежности}
\scnaddlevel{-1}
}

\scnheader{Таблица. Алфавит sc.s-коннекторов, соответствующих sc.g-дугам принадлежности}
\scneqfile{\\
\includegraphics[width=1\linewidth]{figures/intro/scs_membership_connectors.pdf}\\
}

\scnheader{Таблица. Алфавит sc.s-коннекторов, соответствующих sc.g-коннекторам, которые не являются sc.g-дугами принадлежности}
\scneqfile{\\
\includegraphics[width=1\linewidth]{figures/intro/scs_non_membership_connectors.pdf}\\
}

\scnheader{знак равенства}
\scneqfileclass{=}
\scnidtf{связь синонимии идентификаторов}
\scnnote{знак равенства является sc.s-разделителем двух sc.s-идентификаторов, которые идентифицируют (именуют) одну и ту же сущность и, соответственно, являются внешними идентификаторами* (внешними уникальными изображениями) одного и того же sc-элемента. При этом из указанных двух sc.s-идентификаторов чаще всего один является простым sc.s-идентификатором, а второй -- sc.s-выражением. Реже оба эти sc.s-идентификатора являются sc.s-выражениями. И совсем редко оба они являются простыми sc.s-идентификаторами. Последнее обозначает то, что оба эти sc.s-идентификатора являются основными внешними идентификаторами* одного и того же sc-элемента. Пример:

SC-код = sc.s-текст;;

Здесь первый sc.s-идентификатор является именем собственным, а второй -- именем нарицательным.

При трансляции sc.s-текста в SC-код знаку равенства на некотором этапе может быть поставлено в соответствие sc-ребро, принадлежащее отношению синонимии* sc-элементов, идентифицируемых sc.s-идентификаторами, связанными знаком равенства. Но на последующем этапе указанное sc-ребро \uline{удаляется}, а связанные им sc-элементы \uline{склеиваются}. Таким образом sc-ребро, принадлежащее отношению синонимии* sc-элементов, т.е. имеет не только денотационную, но и операционную семантику.}

\scnheader{знак равенства с включением}
\scneq{{\normalfont(}\scnfileclass{$\supset$=} $\cup$ \scnfileclass{=$\subset$}{\normalfont)}}
\scnidtf{изображение sc-дуги, принадлежащей отношению погружения*, связывающей два sc-узла, обозначающих sc-тексты, первый из которых является погружающим, а второй (в который указанная sc-дуга входит) является погружаемым, вводимым в состав первого sc-текста}
\scnnote{sc-дуга, принадлежащая отношению погружения*, интерпретируется как команда погружения одного sc-текста в состав другого. При выполнении этой команды (1) все sc-элементы погружаемого sc-текста становятся элементами, принадлежащими погружающему sc-тексту, (2) все синонимичные sc-элементы, оказавшиеся в составе погружающего sc-текста, склеиваются, (3) sc-узел, обозначающий погружаемый sc-текст, а так же спецификация этого sc-текста (включая перечень всех его sc-элементов) погружается в историю эволюции базы знаний вместе со спецификацией события погружения рассматриваемого sc-текста в состав базы знаний.}

\scnheader{{\normalfont(}знак равенства $\cup$ знак равенства с включением{\normalfont)}}
\scnnote{указанные sc.s-коннекторы отличаются от остальных sc.s-коннекторов тем, что они и соответствующие им sc-коннекторы (sc-ребра, принадлежащих отношению синонимии sc-элементов и sc-дуги, принадлежащие отношению погружения одного sc-текста в состав другого) имеют не только денотационную, но и операционную семантику, т.к. являются командами склеивания и командами погружения.}

\scnheader{sc.s-разделитель, изображающий связь инцидентности sc-элементов}
\scnsubdividing{знак инцидентности “правого” sc-коннектора\\
\scnaddlevel{1}
\scnidtf{знак инцидентности sc-коннектора, sc.s-идентификатор которого находится справа}
\scneqfileclass{|-}
\scnaddlevel{-1}
;знак инцидентности “левого” sc-коннектора\\
\scnaddlevel{1}
\scnidtf{знак инцидентности sc-коннектора, sc.s-идентификатор которого находится слева}
\scneqfileclass{-|}
\scnaddlevel{-1}
;знак инцидентности входящей sc-дуги справа\\
\scnaddlevel{1}
\scnidtf{знак инцидентности sc-дуги, sc.s-идентификатор который находится справа}
\scneqfileclass{|<}
\scnaddlevel{-1}
;знак инцидентности входящей sc-дуги слева\\
\scnaddlevel{1}
\scnidtf{знак инцидентности sc-дуги, sc.s-идентификатор который находится слева}
\scneqfileclass{>|}
\scnaddlevel{-1}}
\scnexplanation{На множестве sc-элементов задано бинарное ориентированное отношение инцидентности sc-элементов, а так же подмножество этого отношения -- отношение инцидентности входящих sc-дуг, каждая пара которого связывает sc-дугу с тем sc-элементом, в который она входит.
В SC-коде sc-коннекторы могут соединять между собой не только sc-узел с sc-узлами, но и sc-узел с sc-коннектором и даже sc-коннектор с sc-коннектором. В последнем случае, указывая инцидентность sc-коннекторов, необходимо уточнить, какой из них является соединяемым (связываемым), а какой-соединяющим (связующим). Поэтому отношение инцидентности, заданное на множестве sc-элементов является ориентированным. Первый компонент пары этого отношения -- связующий sc-коннектор, а второй -- связуемый sc-элемент. Очевидно, что связующий sc-элемент всегда является sc-коннектором, а sc-узел может быть только связуемым.}

\scnheader{sc.s-разделитель, изображающий связь инцидентности sc-элементов}
\scnnote{указанные sc.s-разделители с точки зрения синтаксической структуры sc.s-предложений аналогичны sc.s-коннекторам, но с точки зрения их денотационной семантики в отличие от sc.s-коннекторов они не являются изображениями соответствующих sc-коннекторов}

\scnendstruct \scnsourcecomment{Завершили Описание sc.s-разделителей}

\scnsegmentheader{Описание sc.s-ограничителей}
\scnstartsubstruct

\scnheader{sc.s-ограничитель}
\scnsubdividing{левый sc.s-ограничитель\\
    \scnaddlevel{1}
    \scnidtf{начальный sc.s-ограничитель}
    \scnidtf{открывающий sc.s-ограничитель}
    \scnaddlevel{-1}
;правый sc.s-ограничитель\\
    \scnaddlevel{1}
    \scnidtf{конечный sc.s-ограничитель}
    \scnidtf{закрывающий sc.s-ограничитель}
    \scnaddlevel{-1}}
\scnsubdividing{sc.s-ограничитель, используемый в простых sc.s-идентификаторах\\
    \scnaddlevel{1}
    \scnrelboth{пара пересекающихся множеств}{прямые кавычки}
        \scnaddlevel{1}
        \scneqfileclass{"}
        \scnidtf{ограничитель метафорических словосочетаний}
        \scnaddlevel{-1}
    \scnrelto{пара пересекающихся множеств}{круглая скобка}
        \scnaddlevel{1}
        \scneq{{\normalfont(}\scnfileclass{(} $\cup$ \scnfileclass{)}{\normalfont)}}
        \scnaddlevel{-1}
    \scnnote{Данные ограничители используются не только в простых sc.s-идентификаторах: прямые кавычки — в нетранслируемых комментариях и в ея-файлах ostis-систем, а круглые скобки — везде (и в sc.s-выражениях, и в нетранслируемых комментариях, и в ея-файлах ostis-систем.}
    \scnaddlevel{-1}
;sc.s-ограничитель в sc.s-выражении;sc.s-ограничитель, используемый в sc.s-выражениях\\
    \scnaddlevel{1}
    \scnrelboth{пара пересекающихся множеств}{фигурная скобка}
    \scnaddlevel{1}
        \scneq{{\normalfont(}\scnfileclass{\{} $\cup$ \scnfileclass{\}}{\normalfont)}}
        \scnexplanation{В sc.s-тексте фигурные скобки ограничивают sc.s-выражение, обозначающее либо множество sc-элементов, sc.s-идентификаторы которых перечисляются через точку с запятой, либо множество sc-элементов, входящих в состав sc-текста, который является результатом трансляции в SC-код sc.s-текста или даже ея-текста, ограниченного фигурными скобками.}
    \scnaddlevel{-1}
    \scnrelboth{пара пересекающихся множеств}{квадратная скобка}
        \scnaddlevel{1}
        \scnexplanation{В sc.s-тексте квадратные скобки ограничивают sc.s-выражение, обозначающее текстовый файл-экземпляр ostis-системы, содержимое (тело) которого изображается и ограничивается указанными квадратными скобками.}
        \scnaddlevel{-1}
    \scnrelboth{пара пересекающихся множеств}{квадратная скобка с вертикальной чертой}
        \scnaddlevel{1}
        \scnexplanation{В sc.s-тексте данный sc.s-ограничитель ограничивает sc.s-выражение, обозначающее текстовый файл-класс ostis-системы, содержимое (тело) которого изображается и ограничивается указанными sc.s-ограничителями.}
        \scnaddlevel{-1}
    \scnrelboth{пара пересекающихся множеств}{круглая скобка}
        \scnaddlevel{1}
        \scnexplanation{В sc.s-выражениях круглые скобки ограничивают перечень (через точку с запятой) sc.s-идентификаторов, обозначающих аргументы заданного квазибинарного ориентированного отношения, sc.s-идентификатор которого указывается перед левой круглой скобкой.}
        \scnaddlevel{-1}
    \scnaddlevel{-1}
;sc.s-ограничитель нетранслируемого комментария\\
    \scnaddlevel{1}
    \scnrelboth{пара пересекающихся множеств}{косая черта со звездочкой}
    \scnaddlevel{1}
        \scneq{{\normalfont(}\scnfileclass{/*} $\cup$ \scnfileclass{*/}{\normalfont)}}
    \scnaddlevel{-1}
    \scnaddlevel{-1}
;sc.s-ограничитель, используемый в ея-файлах ostis-систем}

\scnsourcecomment{Завершили перечень видов sc.s-ограничителей}

\scnendstruct \scnsourcecomment{Завершили Описание sc.s-ограничителей}

\scnsegmentheader{Описание sc.s-предложений}
\scnstartsubstruct

\scnheader{sc.s-предложение}
\scnidtf{минимальный семантически целостный фрагмент sc.s-текста}
\scnidtf{минимальный sc.s-текст}
\scnsubset{sc.s-текст}
\scnsuperset{простое sc.s-предложение\\
    \scnaddlevel{1}
    \scnidtf{sc.s-предложение, \uline{состоящее} из двух sc.s-идентификаторов, соединенных между собой sc.s-коннектором или sc.s-разделителем, изображающим связь инцидентности sc-элементов, и завершающееся двойной точкой с запятой}
    \scnaddlevel{-1}}
\scnnote{Признаком завершения любого sc.s-предложения, т.е. последними его символами является двойная точка с запятой, которую, следовательно можно считать разделителем sc.s-предложений.}
\scnrelfromlist{заданная операция}{Операция конверсии sc.s-предложения\\
    \scnaddlevel{1}
    \scnexplanation{Каждое sc.s-предложение (в том числе, и простое sc.s-предложение) можно преобразовать в семантически эквивалентное ему sc.s-предложение путем конверсии ("разворота") цепочки компонентов sc.s-предложения. Так, например, при конверсии ("развороте") простого sc.s-предложения (1) первый его sc.s-идентификатор (первый компонент этого sc.s-предложения) становится третьим компонентом конвертированного sc.s-предложения, (2) второй его sc.s-идентификатор (третий компонент исходного sc.s-предложения) становится первым компонентом "конвертированного"\ sc.s-предложения и (3) второй компонент исходного sc.s-предложения (sc.s-коннектор или sc.s-разделитель, изображающий связь инцидентности sc-элементов, соединяющий указанные выше компоненты) остается вторым компонентом конвертированного sc.s-предложения, но меняет направленность ("$\ni$"\ заменяется на "$\in$"\ и наоборот, "$\supset$"\ на "$\subset$"\ и наоборот, "$=>$"\ на "$<=$"\ и наоборот и т.д.)}
    \scnnote{Можно говорить не только о конверсии sc.s-предложения, но и о конверсии sc.s-коннектора, о конверсии sc.s-разделителя, изображающего связь инцидентности sc.s-элементов.}
    \scnaddlevel{-1}
;Операция соединения двух sc.s-предложений при совпадении последнего компонента первого предложения с первым компонентом второго\\
    \scnaddlevel{1}
    \scnexplanation{В результате выполнения данной операции два исходных sc.s-предложения соединяются в одно sc.s-предложение путем "склеивания"\ указанных совпадающих компонентов и удаления двойной точки с запятой, разделяющей исходные два предложения.}
    \scnaddlevel{-1}
;Операция присоединения простого sc.s-предложения к sc.s-предложению, у которого последний sc.s-коннектор совпадает с sc.s-коннектором простого sc.s-предложения, а предшествующий указанному sc.s-коннектору sc.s-идентификатор совпадает с первым sc.s-идентификатором простого sc.s-предложения\\
    \scnaddlevel{1}
    \scnexplanation{В результате выполнения этой операции совпадающие sc.s-идентификаторы и sc.s-коннекторы соединяемых sc.s-предложений "склеиваются", а последние sc.s-иден\-ти\-фи\-ка\-то\-ры соединяемых sc.s-предложений становятся последними компонентами объединенного sc.s-предложения,
    разделенными точкой с запятой. Аналогичным образом можно присоединять сколько угодно простых sc.s-предложений.}
    \scnaddlevel{-1}
;Операция разложения sc.s-предложений на простые sc.s-предложения\\
    \scnaddlevel{1}
    \scnexplanation{Каждое sc.s-предложение можно разложить на множество простых sc.s-предложений, т.е. представить в виде последовательности простых sc.s-предложений}
    \scnaddlevel{-1}
;Операция разложения sc.s-предложений на простые sc.s-предложения с sc.s-разделителем, изображающим связь инцидентности sc-элементов\\
    \scnaddlevel{1}
    \scnexplanation{Каждое sc.s-предложение (в том числе и простое sc.s-предложение с sc.s-коннектором) можно представить в виде семантически эквивалентной последовательности простых sc.s-предложений с sc.s-разделителем, изображающим связь инцидентности sc-элементов.}
    \scnnote{Данная операция осуществляет \uline{однозначное} (!) формирование множества простых sc.s-предложений указанного вида.}
    \scnaddlevel{-1}
    }

\scnheader{sc.s-предложение}
\scnnote{Операции, заданные на множестве sc.s-предложений можно разделить на три группы:
    \begin{scnitemize}
        \item группа операций конверсии sc.s-предложений, состоящая из одной операции;
        \item группа операций соединения sc.s-предложений;
        \item группа операций декомпозиции sc.s-предложений и, в частности, операций разложения sc.s-предложений.
    \end{scnitemize}
Очевидно, что операции соединения sc.s-предложений и операции декомпозиции sc.s-предложений являются обратными друг другу операциями.}

\scnheader{компонент sc.s-предложения*}
\scnexplanation{Каждое sc.s-предложение представляет собой последовательность (1) sc.s-идентификаторов, (2) sc.s-коннекторов или sc.s-разделителей, изображающих связь инцидентности sc-элементов, (3) точек с запятыми, завершаемая двойной точкой с запятой. При этом непосредственно соседствовать друг с другом не могут ни sc.s-идентификаторы, ни sc.s-коннекторы, ни, очевидно, точки с запятыми.\\
Между sc.s-идентификаторами в рамках sc.s-предложения может находиться либо точка с запятой, либо sc.s-коннектор, либо sc.s-разделитель, изображающий связь инцидентности sc-элементов. Слева и справа от sc.s-коннектора и от sc.s-разделителя, изображающего связь инцидентности sc-элементов, должны находиться sc.s-идентификаторы.

Указанные sc.s-идентификаторы, sc.s-коннекторы и sc.s-разделители, изображающие связь инцидентности sc-элементов, считаются компонентами соответствующего sc.s-предложения. Понятие "быть компонентом sc.s-предложения"\ является относительным понятием (отношением), т.к. в состав некоторых компонентов sc.s-предложения (в состав sc.s-идентификаторов, являющихся sc.s-выражениями, ограничиваемыми фигурными или квадратными скобками) могут входить других sc.s-предложения, состоящие из своих компонентов.}
    
\scnheader{sc.s-предложение}
\scntext{денотационная семантика}{С семантической точки зрения sc.s-предложение представляет собой описание некоторого \uline{маршрута} в соответствующем sc-тексте, который является графовой структурой специального вида и структура которого описывается (изображается) с помощью sc.s-предложений. Указанный маршрут "проводится"\ по sc-коннекторам и по связям инцидентности sc-элементов, если маршрут проходит через инцидентные sc-коннекторы. В описании указанного маршрута могут дополнительно указываться множества (чаще всего отношения), которым принадлежат sc-коннекторы, входящие в описываемый маршрут. Кроме того, указанный маршрут в начале и/или в конце может иметь разветвления, когда какой-либо sc-элемент \uline{одинаково} инцидентен нескольким \uline{однотипным} sc-коннекторам, соединяющим указанный sc-элемент с некоторыми другими sc-элементами.

Таким образом каждое указанное разветвление состоит из неограниченного числа ветвей, каждая из которых состоит из одного sc-коннектора и одного связываемого им sc-элемента.}

\scnheader{sc.s-модификатор*}
\scnsubset{компонент sc.s-предложения*}
\scnexplanation{Это дополнительный вид компонентов sc.s-предложений. Каждый sc.s-модификатор, являющийся компонентом некоторого sc.s-предложения, представляет собой sc.s-идентификатор, обозначающий множество (чаще всего, отношение), которому принадлежит sc-коннектор, изображенный sc.s-коннектором, который предшествует указанному sc.s-идентификатору. Признаком sc.s-модификатора является двоеточие, которое ставится после sc.s-модификатора и отделяет его либо от следующего за ним другого sc.s-модификатора для этого же sc.s-коннектора, либо от следующего за ним sc.s-идентификатора, соответствующего sc-элементу, который инцидентен sc-коннектору, изображенному sc.s-коннектором, находящимся левее рассматриваемого sc.s-идентификатора после одного или нескольких sc.s-модификаторов.}

\scnheader{sc.s-текст}
\scnidtf{конкатенация sc.s-предложений}
\scnidtf{последовательность sc.s-предложений, разделяемых двойными точками с запятой}
\scnsuperset{максимальный исходный sc.s-текст}
    \scnaddlevel{1}
    \scnidtf{конкатенция sc.s-предложений, слева и справа от которой отсутствуют какие-либо символы SCs-кода}
    \scnaddlevel{-1}
\scnsuperset{максимальный встроенный sc.s-текст}
    \scnaddlevel{1}
    \scnidtf{конкатенция всех sc.s-предложений, входящих в состав sc.s-выражения, ограничиваемого фигурными скобками}
    \scnaddlevel{-1}
\scnsuperset{исходный sc.s-текст}
    \scnaddlevel{1}
    \scnidtf{часть цепочки sc.s-предложений, входящих в состав максимального исходного sc.s-текста}
    \scnsuperset{исходное sc.s-предложение}
    \scnaddlevel{-1}
\scnsuperset{встроенный sc.s-текст}
    \scnaddlevel{1}
    \scnidtf{часть цепочки sc.s-предложений, входящих в состав максимального встроенного sc.s-текста}
    \scnsuperset{встроенное sc.s-предложение}
    \scnaddlevel{-1}
\scnnote{sc.s-предложение является минимальным sc.s-текстом.}
\scntext{свойство}{Смысл исходного sc.s-текста, а также встроенного sc.s-текста независит от порядка sc.s-предложений в этих sc-текстах. Т.е. перестановка sc.s-предложений в рамках таких sc.s-текстов смысла этих sc.s-текстов не меняет (т.е. приводит к семантически эквивалентным sc.s-текстам), но сильно влияет на трудоемкость человеческого восприятия (на "читабельность") этих текстов.}

\scnendstruct \scnsourcecomment{Завершили Описание sc.s-предложений}

\scnsegmentheader{Описание Ядра SCs-кода и различных направлений его расширения}
\scnstartsubstruct

\scnheader{Ядро SCs-кода}
\scnidtf{Подъязык SCs-кода, который использует минимальный набор семантических средств, но при этом имеет семантическую мощность, эквивалентную мощности SCs-кода в целом}
\scntext{принципы}{В Ядре SCs-кода:
\begin{scnitemize}
    \item используются только простые sc.s-идентификаторы (sc.s-выражения не используются);
    \item используются только sc.s-разделители, изображающие связь инцидентности sc-элементов (sc.s-коннекторы не используются);
    \item не используются sc.s-модификаторы и, соответственно, двоеточия, являющиеся признаком завершения sc.s-модификаторов;
    \item используются только простые sc.s-предложения, которые, как следует из вышеуказанных свойств Ядра SCs-кода, состоят из двух простых sc.s-идентификаторов, соединяемых sc.s-разделителем, изображающим связь инцидентности sc-элементов.
\end{scnitemize}

Из перечисленных свойств Ядра SCs-кода следует, что для представления (изображения) любого sc-текста средствами Ядра SCs-кода необходимо для \uline{всех} (!) sc-элементов этого sc-текста построить соответствующие им простые sc.s-идентификаторы, т.е. необходимо проименовать все указанные sc-элементы.}
\scntext{примечание}{Очевидно, что практического значения Ядро SCs-кода не имеет. Для обеспечения возможности практического использования необходимы синтаксические расширения Ядра SCs-кода в целях:
\begin{scnitemize}
    \item минимизации числа идентифицируемых (именуемых) sc-элементов путем введения sc.s-выражений и ликвидации необходимости идентифицировать (именовать) \uline{все} (!) sc-элементы;
    \item сокращения текста путем минимизации числа повторений одного и того же sc.s-идентификатора путем соединения sc.s-предложений;
    \item повышение уровня наглядности, "читабельности"\ sc.s-текстов.
\end{scnitemize}}

\scnheader{Первое направление расширения Ядра SCs-кода}
\scnidtf{Первое направление расширения Ядра SCs-кода \uline{и всех иных его расширений}}
\scntext{принципы}{По сравнению с Ядром SCs-кода в Первом направлении расширения Ядра SCs-кода вместо sc.s-идентификаторов, являющихся идентификаторами (именами), которые взаимно однозначно соответствуют синонимичным им (представляемым ими) sc-коннекторам, вводятся sc.s-коннекторы, каждый из которых соответствует не одному конкретному sc-коннектору, а некоторому классу однотипных sc-коннекторов. Очевидно, что это ликвидирует необходимость \uline{каждому} sc-коннектору приписывать уникальный sc.s-идентификатор. Кроме того, Алфавит sc.s-коннекторов включает в себя элементы этого Алфавита (классы \uline{синтаксически} эквивалентных sc.s-коннекторов), которые соответствуют \uline{всем} (!) элементам Алфавита sc-коннекторов, но при этом дополнительно включают в себя и другие элементы Алфавита sc.s-коннекторов, которые соответствуют часто используемым \uline{семантически} явно выделяемым (с помощью sc-элементов) классам sc-коннекторов. К таким дополнительно вводимым классам sc.s-коннекторов относятся константные sc.s-коннекторы включения множеств ("$\supset$"\ или "$\subset$"), переменные sc.s-коннекторы включения множеств ("$\_\supset$"\ или "$\_\subset$"), sc.s-коннектор равенства ("$=$"), sc.s-коннектор равенства с включением ("$=\subset$"\ или "$\supset=$") и др.\\
Заметим, что указанное расширение Алфавита sc.s-коннекторов аналогично расширенному Алфавиту sc.g-коннекторов в SCg-коде и ликвидирует необходимость (как и в SCs-коде) явно специфицировать (средствами SCs-кода) синтаксически выделяемые классы sc.s-коннекторов.}

\scnheader{Второе направление расширения Ядра SCs-кода}
\scntext{принципы}{Во Втором направлении расширения Ядра SCs-кода вводятся модификаторы sc.s-коннекторов (sc.s-модификаторы), которые позволяют достаточно компактно дополнительно специфицировать sc-коннекторы, изображаемые (представляемые) соответствующими sc.s-коннекторами. Речь идет о такой часто востребованной форме спецификации sc-коннекторов, как указание множества (возможно, нескольких множеств), которому принадлежит специфицируемый  sc-коннектор (чаще всего, таким множеством является бинарное или квазибинарное отношение).}

\scnheader{sc.s-модификатор*}
\scniselement{отношение}
    \scnaddlevel{1}
    \scnidtf{относительное понятие}
    \scnaddlevel{-1}
\scnidtf{модификатор sc.s-коннектора*}
\scnidtf{sc.s-идентификатор, который (1) находится либо между sc.s-коннектором и двоеточием, либо между двоеточиями и (2) обозначает множество (чаще всего, отношение), которому принадлежит sc-коннектор, изображаемый ближайшим предшествующим sc.s-коннектором. Очевидно, что, если не использовать sc.s-модификаторы, указанного вида спецификация sc-коннекторов средствами SCs-кода будет выглядеть значительно более громоздкой.}

\scnheader{Третье направление расширения Ядра SCs-кода}
\scntext{принципы}{В третьем направлении расширения Ядра SCs-кода осуществляется переход от использования только простых sc.s-идентификаторов к использованию как простых sc.s-идентификаторов, так и sc.s-выражений, а также к использованию sc.s-представления некоторых неидентифицируемых sc-узлов. Это существенно сокращает число придумываемых простых sc.s-идентификаторов, т.к. каждое sc.s-выражение в конечном счете — это комбинация простых sc.s-идентификаторов, построенная по правилам, которые достаточно легко семантически интерпретируются. Если проводить аналогию с SCg-кодом, то очевидно, что sc.s-выражение, ограничиваемое фигурными скобками есть не что иное, как информационная конструкция, ограничиваемая sc.g-контуром, а sc.s-выражение, ограничиваемое квадратными скобками есть не что иное, как информационная конструкция, ограничиваемая sc.g-рамкой. Отличие здесь заключается в том, что круглыми и квадратными скобками можно ограничивать только линейные информационные конструкции (цепочки символов).}

\scnheader{sc.s-представление неидентифицируемого sc-узла}
\scnidtf{изображение (представление) неидентифицируемого (неименуемого) sc-узла в sc.s-тексте}
\scnidtf{sc.s-обозначение неименуемой сущности, не являющейся парой}
\scnidtf{sc.s-представление sc-узла, не являющееся sc.s-идентификатором (именем этого sc-узла)}
\scnreltoset{разбиение}{sc.s-обозначение неименуемой структуры\\
    \scnaddlevel{1}
    \scnidtf{конкатенация левой фигурной скобки и правой фигурной скобки}
    \scnaddlevel{-1}
;sc.s-обозначение неименуемой неориентированной связки\\
    \scnaddlevel{1}
    \scnidtf{конкатенация левой фигурной скобки, дефиса и правой фигурной скобки}
    \scnaddlevel{-1}
;sc.s-обозначение неименуемого кортежа\\
    \scnaddlevel{1}
    \scnidtf{конкатенация левой угловой скобки, дефиса и правой угловой скобки}
    \scnaddlevel{-1}
;sc.s-обозначение неименуемого файла-экземпляра\\
    \scnaddlevel{1}
    \scnidtf{конкатенация левой квадратной скобки  правой квадратной скобки}
    \scnaddlevel{-1}
;sc.s-обозначение неименуемого файла-класса\\
    \scnaddlevel{1}
    \scnidtf{конкатенация левой квадратной скобки, левой цитатной кавычки и правой квадратной скобки}
    \scnaddlevel{-1}
;sc.s-обозначение неименуемой терминальной сущности\\
    \scnaddlevel{1}
    \scnidtf{конкатенация левой круглой скобки, буквы "о"\ и правой круглой скобки}
    \scnaddlevel{-1}}
\scntext{примечание}{Если одно и то же обозначение неименуемой сущности встречается в \uline{разных} sc.s-предложениях, то считается, что это обозначения \uline{разных} сущностей, т.е. изображения \uline{разных} sc-узлов.}

\scnheader{Четвертое направление расширения Ядра SCs-кода}
\scntext{принципы}{В Четвертом направлении расширения Ядра SCs-кода осуществляется переход от использования только простых sc.s-предложений к использованию как простых, так и соединенных sc.s-предложений, построенных с помощью операций соединения sc.s-предложений. В результате этого, благодаря "склеиванию"\ одинаковых sc.s-идентификаторов, а также "склеиванию"\ синтаксически эквивалентных sc.s-коннекторов с одинаковыми sc.s-модификаторами (несмотря на то, что эти "склеиваемые"\ sc.s-коннекторы соответствуют \uline{разным} sc-коннекторам), существенно сокращается число копий используемых sc.s-идентификаторов и sc.s-коннекторов с их sc.s-модификаторами.}

\scnendstruct \scnsourcecomment{Завершили Описание Ядра SCs-кода и различных направлений его расширения}

\scnsegmentheader{Последний сегмент введения в SCs-код}
\scnstartsubstruct

\scnheader{следует отличать*}
\scnhaselementset{sc-элемент;sc.s-идентификатор\\
    \scnaddlevel{1}
    \scnidtf{внешний идентификатор sc-элемента}
    \scnidtf{имя, соответствующее (приписываемое) sc-элементу}
    \scnaddlevel{-1}}
\scnhaselementset{sc.s-идентификатор;sc.s-представление неидентифицируемого sc-элемента}
\scnhaselementset{sc.s-представление неидентифицируемого sc-узла;sc.s-коннектор\\
    \scnaddlevel{1}
    \scnidtf{sc.s-представление неидентифицируемого sc-коннектора}
    \scnaddlevel{-1}}
\scnhaselementset{sc-коннектор;sc.s-коннектор}
\scnhaselementset{sc.s-коннектор;sc.s-модификатор*\\
    \scnaddlevel{1}
    \scnidtf{модификатор sc.s-коннектора*}
    \scniselement{отношение}
    \scnaddlevel{-1}}
\scnhaselementset{простой sc.s-идентификатор;sc.s-выражение}
\scnhaselementset{sc.s-выражение, ограничиваемое фигурными скобками;sc.s-выражение, ограничиваемое квадратными скобками}
    \scnaddlevel{1}
    \scnnote{Для каждого sc.s-выражения, ограничиваемого фигурными скобками, существует sc.s-выражение, отличающееся от первого только заменой фигурных скобок на квадратные.}
    \scnaddlevel{-1}
\scnhaselementset{простое sc.s-предложение;соединенное sc.s-предложение}
\scnhaselementset{sc.s-идентификатор;Правила построения sc.s-идентификаторов}
\scnhaselementset{sc.s-коннектор;Правила построения sc.s-коннекторов}
\scnhaselementset{sc.s-предложение;Правила построения sc.s-предложений}
\scnhaselementset{sc.s-коннектор;sc.g-коннектор}
\scnhaselementset{sc.s-текст;sc.g-текст}

\scnendstruct \scnsourcecomment{Завершили сегмент "Последний сегмент введения в SCs-код"}

\scnendstruct \scnsourcecomment{Завершили раздел \ref{intro_scs} "\nameref{intro_scs}"}

%END SCs
\scnendstruct

\end{SCn}
%
\scsubsection{Неформальное введение в язык гипертекстового представления баз знаний ostis-систем}

\begin{SCn}

\scnsectionheader{\currentname}

\end{SCn}


\scsectionfinish{intro_lang}
\begin{SCn}

\scnsectionheader{Завершение введения в описание внутреннего языка ostis-систем и близких ему внешних языков}

\scnstartsubstruct

\scnnote{Приведем сравнительный анализ основных понятий, связанных с языками представления знаний в ostis-системах}

\scnheader{язык представления знаний в ostis-системах}
\scnsubset{формальный язык}
\scnsubset{универсальный язык}
\scnhaselements{\textit{SC-код}; \textit{SCg-код}; \textit{SCs-код}; \textit{SCn-код}}
\scnnote{Следует отличать
\begin{scnitemize}
\item саму описываемую сущность;
\item текст, являющийся описанием некоторой сущности;
\item тест, являющийся описанием некоторого другого текста, а возможно и самого себя (т.е. текст может быть описываемой сущностью);
\item знак (обозначение) описываемой сущности в рамках заданного текста;
\item обозначение описываемой сущности в SC-тексте (это всегда sc-элемент того или иного вида);
\item коммуникативный (внешний для ostis-системы) уникальный (основной) идентификатор (чаще всего строковый идентификатор-имя), обозначающий соответствующую описываемую сущность и являющийся внешним идентификатором (именем) для соответствующего синонимичного ему sc-элемента. Такие идентификаторы взаимно однозначно соответствуют sc-элементам, которые имеют такие идентификаторы;
\item вспомогательные (неосновные) внешние идентификаторы sc-элементов. Такие идентификаторы и свойством омонимии (когда один идентификатор соответствует нескольким sc-элементам) и синонимии (когда разные идентификаторы соответствует одному sc-элементу);
\item обозначение описываемой сущности в sc.g-тексте (это всегда графически представленный sc.g-элемент, являющийся \uline{изображением} соответствующего sc-элемента);
\item обозначение описываемой сущности в sc.s-предложении и в sc.n-предложении – это всегда строка символов (либо \uline{омонимичное} изображение sc-коннекторов различного семантического типа, либо \uline{основной} строковый идентификатор, соответствующий некоторому sc-элементу, либо выражение, являющееся \uline{неатомарным} идентификатором, содержащим некоторую информацию о соответствующей именуемой сущности).
\end{scnitemize}

Подчеркнем, что каждое \uline{обозначение} описываемой сущности в SCg-коде, SCs-коде; SCn-коде рассматривается нами как \uline{изображение} соответствующего ему (синонимичного ему) sc-элемента, обозначающего ту же описываемую сущность. Таким образом, указанные языки (SCg-код, SCs-код; SCn-код) рассматриваются нами как различные варианты изображения текстов SCg-кода.}
\scnnote{Для формального описания рассматриваемого нами семейства языков (SCg-код, SCg-код, SCs-код, SCn-код) и не только их используется целый ряд метаязыковых понятий.

Перечислим некоторые из них: \textit{идентификатор}, \textit{класс синтаксически эквивалентных идентификаторов}, \textit{имя}, \textit{простое имя}, \textit{выражение}, \textit{внешний идентификатор*}, \textit{алфавит*}, \textit{разделители*}, \textit{ограничители*}, \textit{предложения*}}

\scnheader{алфавит*}
\scnidtf{быть алфавитом для заданного множества текстов*}
\scnidtf{быть семейством максимальных множеств синтаксически однотипных элементарных (атомарных) фрагментов текстов, принадлежащих заданному множеству текстов*}

\scnheader{ограничители*}
\scnidtf{Отношение, связывающее заданный класс информационных конструкций с соответствующим классом их ограничителей}
\scnidtf{быть ограничителями, используемыми в заданном множестве информационных конструкций*}

\scnheader{ограничитель}
\scnsuperset{sc.g-ограничитель}
\scnsuperset{sc.s-ограничитель}
\scnsuperset{sc.n-ограничитель}
\scnsuperset{ограничитель, используемый в ея-файлах ostis-систем}

\scnheader{SCg-код}
\scnrelfrom{ограничители}{sc.g-ограничитель}
\scnaddlevel{1}
\scnidtf{Множество ограничителей, используемых в sc.g-текстах}
\scnaddlevel{-1}

\scnheader{разделители*}
\scnidtf{быть разделителями, используемыми в заданном множестве информационных конструкций*}
\scnrelfrom{второй домен}{разделитель}
\scnaddlevel{1}
\scnsuperset{sc.g-разделитель}
\scnaddlevel{1}
\scneq{sc.g-коннектор}
\scnaddlevel{-1}
\scnsuperset{sc.s-разделитель}
\scnsuperset{sc.n-разделитель}
\scnsuperset{разделитель, используемый в ея-файлах ostis-систем}
\scnaddlevel{-1}

\scnheader{идентификатор}
\scnsuperset{sc.s-идентификатор}
\scnidtf{cтруктурированный знак соответствующей (обозначаемой) сущности, который чаще всего представляет собой строку (цепочку символов), которую будем называть именем соответствующей сущности.} 
\scnnote{В формальных текстах (в том числе текстах SC-кода, SCg-кода, SCs-кода, SCn-кода) основные используемые идентификаторы не должны быть омонимичными, то есть должны \uline{однозначно} соответствовать идентифицируемым сущностям. Следовательно, каждая пара идентификаторов, имеющих \uline{одинаковую} структуру, должны обозначать одну и ту же сущность.}

\scnheader{следует отличать*}
\scnhaselementset{идентификатор\\
\scnaddlevel{1}
\scnidtf{Множество всевозможных конкретных \uline{экземпляров}, конкретных вхождений идентификаторов, имеющих различную структуру, во всевозможные тексты}
\scnaddlevel{-1}
;класс синтаксически эквивалентных идентификаторов\\
\scnaddlevel{1}
\scnidtf{класс идентификаторов, имеющих одинаковую структуру}
\scnidtf{Семейство всевозможных множеств, каждое из которых является максимальным множеством синтаксически эквивалентных идентификаторов}
\scnaddlevel{-1}
}

\scnheader{имя}
\scnsubset{идентификатор}
\scnidtf{строковый идентификатор}
\scnidtf{идентификатор, представляющий собой строку (цепочку) символов}
\scnsubdividing{простое имя\\
\scnaddlevel{1}
\scnidtf{атомарное имя}
\scnidtf{имя, в состав которого другие имена не входят}
\scnaddlevel{-1}
;выражение\\
\scnaddlevel{1}
\scnidtf{неатомарное имя}
\scnaddlevel{-1}
}
\scnsuperset{sc.s-идентификатор}

\scnheader{внешний идентификатор*}
\scniselement{отношение}
\scnidtf{Бинарное ориентированное отношение, каждая связка (sc-дуга) которого связывает некоторый элемент с файлом, содержимым которого является внешний идентификатор (чаще всего, имя), соответствующий указанному элементу}
\scnidtf{быть внешним идентификатором*}
\scnidtf{внешний идентификатор sc-элемента*}
\scnrelfrom{второй домен}{идентификатор}
\scnnote{Понятие внешнего идентификатора является понятием относительным и важным для ostis-систем, поскольку внутреннее для ostis-систем представление информации (в виде текстов SC-кода) оперирует не идентификаторами описываемых сущностей, а знаками, структура которых никакого значения не имеет}

\scnheader{следует отличать*}
\scnhaselementset{sc-элемент, обозначающий файл ostis-системы;sc.g-элемент, обозначающий файл ostis-системы;простой sc.s-идентификатор, обозначающий файл ostis-системы\\
\scnaddlevel{1}
\scnidtf{простое имя файла ostis-системы}
\scnaddlevel{-1}
;изображение файла ostis-системы, ограниченное sc.g-рамкой;изображение файла ostis-системы, ограниченное sc.n-рамкой;изображение строки символов, ограниченное квадратными скобками}
\scnheader{следует отличать*}
\scnhaselementset{файл-экземпляр;файл-класс\\
\scnaddlevel{1}
\scnidtf{файл, обозначающий класс файлов-экземпляров, синтаксически эквивалентных заданному образцу}
\scnaddlevel{-1}
}

\scnheader{предложения*}
\scnidtf{быть множеством всех предложений заданного текста, не являющихся встроенными предложениями, то есть предложениями, входящими в состав других предложений*}
\scnrelfrom{второй домен}{предложение}

\scnheader{предложение}
\scnexplanation{минимальный семантически целостный фрагмент текста, представляющий собой конфигурацию знаков, входящих в этот фрагмент и связываемых между собой отношениями инцидентности (в частности, отношением непосредственной последовательности в строке), а также различного вида разделителями и ограничителями}


\newpage


\scnheader{sc.g-текст}
\scnhaselementrole{пример}{\scnfilescg{figures/intro/scg/example_triangle.png}}
\scnaddlevel{1}
\scnexplanation{Данный sc.g-текст содержит следующую информацию:
\begin{scnitemize}
\item Сущности \textit{Треугольник ABC}~~ и ~~\textit{Треугольник CDE} являются треугольниками (принадлежат классу \textit{треугольников}). При этом известно, что площадь \textit{Треугольника CDE} в 4 раза больше, чем площадь \textit{Треугольника ABC}, но конкретные значения ллощадей не известны\char59
\item Сущность \textit{Отрезок DE} является отрезком (принадлежит классу \textit{отрезков}) и является стороной \textit{Треугольника CDE}. Кроме того, у \textit{Отрезка DE} есть длина, измерение которой в сантиметрах составляет 5. Обратите внимание, что в данном случае для упрощения понимания использовано бинарное отношение \textit{длина*}, которое является \textit{неосновным понятием} и в базе знаний заменяется на \textit{базовую sc-дугу}, связывающую величину как класс эквивалентности с конкретной сущностью, входящей в данный класс, в данном случае -- \textit{Отрезок DE}\char59  
\item Сущность \textit{Треугольник AEB} является треугольником и имеет \textit{внутренний угол*}~~~ \textit{Угол AEB}. В свою очередь, \textit{Угол AEB} является \textit{углом} и имеет \textit{косинус*}, равный 0,5\char59
\item \textit{Треугольник AEB} имеет \textit{сторону*} (не указывается, какая именно из сторон имеется в виду), \textit{средней точкой*} которой является \textit{Точка O}. В свою очередь, \textit{Точка O} является центром некоторой \textit{Окружности O}, которая относится к классу \textit{окружностей}.
\end{scnitemize}
}
\scnaddlevel{-1}

\newpage

\scnheader{Пример sc.g-текста, трансформируемого по Первому направлению расширения Ядра SCg-кода}
\scneqscg{figures/intro/scg/scg_transf1.png}
\scniselement{sc.g-текст}
\scnexplanation{Здесь (в левом нижнем углу приведенного sc.g-текста) представлен \textit{sc.g-узел общего вида}, изображающий \textit{sc-узел общего вида}, которому соответствует \textit{основной sc-идентификатор*} в виде строки ``\textbf{\textit{ei}}''}
\scnrelfrom{трансформация sc.g-текста по Первому направлению расширения Ядра SCg-кода}{\scnfilescg{figures/intro/scg/scg_transf2.png}}
\scnaddlevel{1}
    \scniselement{sc.g-текст}
    \scnexplanation{\textit{sc.g-узлу общего вида} изображающему \textit{sc-узел}, внешним идентификатором которого является строка ``\textit{основной sc-идентификатор*}'' и который, соответственно является знаком \textit{бинарного ориентированного отношения}, каждая \textit{пара} которого связывает идентифицируемый \textit{sc-элемент} с его основным внешним sc-идентификатором, приписывается указанный внешний идентификатор изображаемого им \textit{sc-элемента}.}
    \scnrelfrom{трансформация sc.g-текста по Первому направлению расширения Ядра SCg-кода}{\scnfilescg{figures/intro/scg/scg_transf3.png}}
    \scnaddlevel{1}
        \scniselement{sc.g-текст} 
        \scnexplanation{В результате данной трансформации исходный \textit{sc.g-текст} трансформируется в один \textit{sc.g-общего вида}, которому приписывается \textit{основной sc-идентификатор} ``\textit{\textbf{ei}}''.}
    \scnaddlevel{-1}
\scnaddlevel{-1}

\newpage

\scnheader{Примеры sc.g-текстов, трансформируемых по Второму направлению расширения Ядра SCg-кода}
\scnstructinclusion

\scnmakeset{\scgfileitem{figures/intro/scg/scg2_ex1.png}\\
\scnaddlevel{1}
    \scnrelfrom{синтаксическая трансформация}{\scnfilescg{figures/intro/scg/scg2_ex1_1.png}}
    \scnaddlevel{1}
        \scnexplanation{Здесь вводится новый синтаксический вид \textit{sc.g-элементов} -- \textit{константный постоянный sc.g-узел общего вида}, изображаемый окружностью диаметром ***.}
    \scnaddlevel{-1}
\scnaddlevel{-1}
}


\newpage

\scnendstruct

\end{SCn}
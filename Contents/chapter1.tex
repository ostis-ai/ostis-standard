%%%%%%%%%%%%%%%%%%%%% chapter.tex %%%%%%%%%%%%%%%%%%%%%%%%%%%%%%%%%
%
% sample chapter
%
% Use this file as a template for your own input.
%
%%%%%%%%%%%%%%%%%%%%%%%% Springer-Verlag %%%%%%%%%%%%%%%%%%%%%%%%%%
%\motto{Use the template \emph{chapter.tex} to style the various elements of your chapter content.}
\scchapter{Обоснование Технологии OSTIS}
\label{chap_justification} 

\scsection{Предметная область и онтология кибернетических систем}
\begin{SCn}

\bigskip
\scnsegmentheader{Комплекс свойств, определяющих уровень интеллекта кибернетической системы}
\scnstartsubstruct

\scnheader{интеллект}
\scniselement{свойство}
\scniselement{упорядоченное свойство}
\scnidtf{уровень (степень, величина) интеллекта кибернетической системы}
\scnidtf{Семейство классов \textit{кибернетических систем}, обладающих эквивалентным (одинаковым) уровнем интеллекта -- от низкого до высокого уровня интеллекта}
\scnidtf{свойство кибернетических систем, характеризующее эффективность их взаимодействия со своей средой (средой их "жизнедеятельности"{})}
\scnrelfrom{область определения}{кибернетическая система}
\scnexplanation{С формальной точки зрения интеллектуальность -- это семейство классов кибернетических систем, в каждый из которых входят кибернетические системы, эквивалентные по уровню и характеру проявления интеллектуальных свойств (в том числе способностей).\\
Таким образом, характер (вид) интеллектуальных свойств кибернетических систем и уровень их развития для разных кибернетических систем может быть разным. В соответствии с этим кибернетические системы можно сравнивать между собой.}
\scnnote{Основным свойством (характеристикой, качеством, параметром) кибернетической системы является уровень (степень) ее интеллекта, который является \uline{интегральной} характеристикой, определяющей уровень эффективности взаимодействия кибернетической системы со средой своего существования.}
\scnidtf{комплексное свойство (качество) кибернетической системы, определяющее уровень ее "выживаемости"{} во внешней среде и предполагающее возможность воздействия на эту среду и даже возможность ее преобразования}
\scnidtf{интеллектуальный потенциал кибернетической системы}
\scnidtf{спектр знаний, навыков и способностей к обучению кибернетической системы}
\scnidtf{интеллектуальность кибернетической системы}
\scnnote{Процесс эволюции \textit{кибернетических систем} следует рассматривать как процесс повышения уровня их качества по целому ряду свойств (характеристик) и, в первую очередь, как процесс повышения уровня их \textit{интеллекта}. При этом можно говорить об эволюции каждой конкретной \textit{кибернетической системы} в процессе своей "жизнедеятельности"{}, а также об эволюции целого класса \textit{кибернетических систем}, когда новые экземпляры этого класса являются более интеллектуальными, чем их предшественники. В таком аспекте, в частности, можно рассматривать эволюцию \textit{компьютерных систем} (искусственных кибернетических систем).}
\scnnote{Очень важно уточнить, какими иными свойствами \textit{кибернетических систем} определяется уровень и характер их интеллектуальности. Подчеркнем, что \uline{любая} \textit{кибернетическая система} обладает соответствующим уровнем интеллектуальности. Пусть даже и достаточно низким. Существенным является уточнение того, за счет чего уровень интеллектуальности \textit{кибернетической системы} может быть повышен. Нет смысла проводить четкую границу между \textit{интеллектуальными кибернетическими системами} и неинтеллектуальными. Но есть смысл уточнять направления повышения уровня интеллектуальности \textit{кибернетических систем.}}
\scntext{эпиграф}{Никто не может провести линию, отделяющую атмосферу от космоса, или черту, за которой начинается жизнь, или границу электронного облака. Все дело в степени проявления свойства.}
	\scnaddlevel{1}
	\scnrelfrom{автор}{Барт Коско}
	\scnaddlevel{-1}
\scnnote{Прежде, чем говорить о требованиях, предъявляемых к \textit{технологии проектирования и производства интеллектуальных компьютерных систем (искусственных кибернетических систем}, обладающих высоким уровнем \textit{интеллекта)}, необходимо уточнить (детализировать) \textit{свойства}, присущие указанным системам и являющиеся предпосылками, обеспечивающими высокий уровень \textit{интеллекта}. Подчеркнем, что указанные \textit{свойства}, уточняющие (детализирующие, обеспечивающие, определяющие) \textit{свойства} \scnbigspace \textit{интеллектуальных систем}\scnbigspace (\textit{свойства}, определяющие уровень \textit{интеллекта} этих систем) должны быть общими как для искусственных кибернетических систем (\textit{компьютерных систем}), так и для \textit{естественных кибернетических систем.}}
\scnidtf{интегральное качество информационного обеспечения и информационных процессов в кибернетической системе}
\scnidtf{интегральное качество кибернетической системы, определяемое:
	\begin{scnitemize}
		\item уровнем ее образованности -- качеством накопленных к заданному моменту знаний и умений (навыков);
		\item уровнем ее обучаемости -- способностью \uline{самостоятельно} повышать уровень свой образованности.
	\end{scnitemize}}
\scnrelfromlist{свойство-предпосылка}{образованность кибернетической системы
;обучаемость кибернетической системы
;социализация кибернетической системы\\
	\scnaddlevel{1}
	\scnnote{интеллект \textit{кибернетической системы}, как и лежащий в его основе познавательный процесс, выполняемый кибернетической системой, имеет социальный характер, поскольку наиболее эффективно формируется и развивается в форме взаимодействия \textit{кибернетической} системы с другими \textit{кибернетическими системами}}
	\scnaddlevel{-1}}
\bigskip

\scnheader{ограниченность обучения кибернетической системы}
\scnexplanation{Данное свойство определяет границу между теми знаниями и навыками, которые соответствующая \textit{кибернетическая система} принципиально может приобрести, и теми знаниями и навыками, которые указанная кибернетическая система не сможет приобрести никогда. Данное свойство определяет максимальный уровень потенциальных возможностей соответствующей кибернетической системы. Очевидно, что максимальная степень отсутствия ограничений в приобретении новых знаний и навыков -- это полное отсутствие ограничений, т.е. полная универсальность возможностей соответствующих кибернетических систем, которые всё могут познать и всё могут сотворить.}
\scnidtf{максимум того, чему кибернетическая система может обучиться}
\scnidtf{максимальная перспектива обучения кибернетической системы}
\scnidtf{максимальный уровень качества, который кибернетическая система может достичь в процессе обучения}
\scnrelfromlist{частное свойство}{максимальный объём знаний, которые кибернетическая система может приобрести в процессе обучения;максимальный объём навыков, которые кибернетическая система может приобрести в процессе обучения}

\scnheader{максимальный объём знаний, которые кибернетическая система может приобрести в процессе обучения}
\scnidtf{граница приобретаемых знаний, за пределы которой кибернетическая система принципиально не может перейти в процессе своего обучения}
\scnidtf{максимум того, чему можно научить соответствующую кибернетическую систему}
\scnidtf{максимальный объём знаний, которые кибернетическая система принципиально может приобрести}
\scnrelto{свойство-предпосылка}{обучаемость}
\scnnote{чем больше \textit{максимальный объём знаний, которые кибернетическая система принципиально может приобрести}, тем выше уровень \textit{обучаемости} кибернетической системы}

\scnheader{познавательная активность кибернетической системы}
\scnidtf{познавательная мотивированность}
\scnidtf{познавательная пассионарность}
\scnidtf{любознательность}
\scnidtf{активность и самостоятельность в приобретении новых знаний и навыков}
\scnidtf{стремление, активная целевая установка к постоянному совершенствованию (повышению качества) и пополнению собственной базы знаний}
\scnnote{Следует отличать
	\begin{scnitemize}
		\item способность (возможность) приобретать новые знания и навыки и совершенствовать приобретенные знания и навыки
		\item от желания (стремления) это делать.
	\end{scnitemize}}
\scnnote{желание (целевая установка) научиться решать те или иные задачи может быть сформулировано кибернетической системой либо самостоятельно, либо извне (некоторым учителем).}

\scnrelfromlist{частное свойство}{активность в изучении внешней среды;активность в анализе качества информации, хранимой в собственной памяти;активность в анализе собственных действий и действий других кибернетических систем}
\scnrelfromlist{свойство-предпосылка}{способность кибернетической системы к синтезу познавательных целей и процедур;способность кибернетической системы к самоорганизации собственного обучения;способность кибернетической системы к экспериментальным действиям}

\scnheader{способность кибернетической системы к синтезу познавательных целей и процедур}
\scnidtf{способность планировать своё обучение и управлять процессом обучения}
\scnidtf{умение задавать вопросы или целенаправленные последовательности вопросов самому себе или другим субъектам как важнейший фактор обучаемости}
\scnidtf{способность генерировать (формулировать, задавать) вопросы, адресуемые либо самому себе, либо некоторому внешнему источнику знаний и направленные на повышение качества собственных знаний и навыков}
\scnidtf{способность генерировать четкую спецификацию своей информационной потребности}
\scnidtf{способность кибернетической системы четко формулировать то, что она не знает (в частности, не умеет), но хотела бы знать и уметь}
\scnidtf{способность к формированию спецификаций информационных баз в своих знаниях}
\scnidtf{способность кибернетической системы самостоятельно генерировать цели на приобретение знаний и навыков, обеспечивающих решение различных классов задач}

\scnheader{способность кибернетической системы к самоорганизации собственного обучения}
\scnidtf{способность осуществлять управление своим обучением}
\scnidtf{способность кибернетической системы самой выполнять роль своего учителя, организующего процесс своего обучения}

\scnheader{способность кибернетической системы к экспериментальным действиям}
\scnidtf{способность к отклонениям от составленных планов своих действий для повышения качества результата или сохранении целенаправленности этих действий}
\scnidtf{способность к экспромтам и импровизации}

\scnheader{способность кибернетической системы к самосохранению}
\scnidtf{способность кибернетической системы к выявлению и устранению угроз, направленных на снижение её качества и даже на её уничтожение, что означает полную потерю необходимого качества}
\scnidtf{уровень безопасности (защищенности) кибернетической системы}
\scnexplanation{Данное свойство кибернетических систем является необходимым фактором высокого уровня обучаемости кибернетических систем. Чем выше уровень безопасности кибернетической системы, тем выше её уровень обучаемости.}
\scnidtf{способность кибернетической системы к обеспечению собственной безопасности}
\scnrelfromlist{свойство-предпосылка}{способность кибернетической системы анализировать смысл задач, инициированных извне, и отказываться от решения вредных задач}
\scnaddlevel{1}
\scntext{эпиграф}{Прежде, чем выполнять приказ, подумай}
\scnexplanation{Примером вредной задачи для \textit{ostis-системы} является запрос всех хранимых в памяти \textit{sc-элементов}}
\scnexplanation{Подчеркнем, что в современных компьютерных системах и интеллектуальных компьютерных системах подходы к обеспечению их информационной безопасности имеют принципиальные отличия, связанные, прежде всего с тем интеллектуальные компьютерные системы обладают более мощными средствами семантического и контекстного анализа приобретаемой информации.}

\end{SCn}
\input{Contents/chapter1/sd_sys_inform.tex}
\label{sd_sys_inform}

\scsubsection{Предметная область и онтология интеллектуальных систем}
\input{Contents/chapter1/sd_int_sys.tex}

\scsubsection{Предметная область и онтология компьютерных систем}
\input{Contents/chapter1/sd_comp_sys.tex}
\scsubsubsection{Предметная область и онтология решателей задач компьютерных систем}
\begin{SCn}

\scnsectionheader{Предметная область и онтология решателей задач компьютерных систем}

\scnstartsubstruct

\scnheader{Предметная область решателей задач современных интеллектуальных компьютерных систем}
\scnsdmainclasssingle{решатель задач компьютерных систем}
\scnsdclass{***}
\scnsdrelation{***}

\scnheader{решатель задач компьютерных систем}
\scnexplanation{Одним из ключевых компонентов интеллектуальной системы, обеспечивающим возможность решать широкий круг задач, является решатель задач. Особенностью решателей задач интеллектуальных систем по сравнению с другими современными
программными системами является необходимость решать задачи в условиях, когда сведения, необходимые для решения задачи, не локализованы явно в базе знаний интеллектуальной системы и должны быть найдены в процессе решения задачи на основании каких-либо критериев}
\scnsuperset{объединенный решатель задач}
\scnaddlevel{1}
\scnexplanation{В общем случае решатель задач обеспечивает возможность решения задач, связанных как с непосредственно основной функциональностью системы, так и с обеспечением эффективности работы такой системы, а также с обеспечением автоматизации развития самой этой системы. Решатель задач, обеспечивающий выполнение всех перечисленных функций, будем называть \textbf{\textit{объединенным решателем задач}} указанной интеллектуальной системы}
\scnaddlevel{-1}
\scntext{проблемы разработки}{Несмотря на то что в настоящее время существует большое число моделей решения задач, многие из которых реализованы и успешно используются на практике в различных системах, остается актуальной проблема низкой согласованности принципов, лежащих в основе реализации таких моделей, и отсутствия единой унифицированной основы для реализации и интеграции различных моделей, что приводит к тому, что:
\begin{scnitemize}
\item затруднена возможность одновременного использования различных моделей решения задач в рамках одной системы при решении одной и той же комплексной задачи; практически невозможно комбинировать различные модели с целью решения задачи, для которой априори отсутствует алгоритм ее
решения;
\item практически невозможно использовать технические решения, реализованные в одной системе, в других системах, т. е. возможности использования компонентного подхода при построении решателей задач сильно ограничены. Как следствие, велико количество дублирований аналогичных решений в разных системах;
\item фактически отсутствуют комплексные методики и средства построения решателей задач, которые бы обеспечивали возможность проектирования, реализации и отладки решателей различного вида.
\end{scnitemize}

Следствиями указанных проблем являются:
\begin{scnitemize}
\item высокая трудоемкость разработки каждого решателя, увеличение сроков их разработки, а значит, и увеличение затрат на разработку и поддержку соответствующих интеллектуальных систем;
\item высокая трудоемкость внесения изменений в уже разработанные решатели, т. е. отсутствует или сильно затруднена возможность дополнения уже разработанного решателя новыми компонентами и внесения изменений в уже существующие компоненты в процессе эксплуатации системы. Таким образом, высока трудоемкость поддержки разработанных решателей;
\item высокий уровень профессиональных требований к разработчикам решателей задач, что обусловлено, в частности:
\begin{scnitemizeii}
\item высокой сложностью существующих формализмов в области решения задач, рассчитанных на их интерпретацию компьютерной системой, а не человеком;
\item отсутствием возможности рассматривать разрабатываемые решатели на разных уровнях детализации, выделения на каждом уровне достаточно независимых компонентов, что затрудняет процесс проектирования, тестирования и отладки таких решателей, а также снижает эффективность попыток объединения разработчиков решателей в коллективы по причине увеличения накладных расходов на согласование их деятельности;
\item низким уровнем информационной поддержки разработчиков и автоматизации их деятельности.
\end{scnitemizeii}
\end{scnitemize}
}

\scnheader{модель решения задач}
\scnsubdividing{модель решения задач на основе хранимых программ\\
\scnaddlevel{1}
\scnsuperset{модель решения задач на основе декларативных программ}
\scnsuperset{модель решения задач на основе императивных программ}
\scnsuperset{модель решения задач на основе генетических алгоритмов}
\scnauthorcomment{сложно сказать, генетический алгоритм это программа или нет, механизм логического вывода тоже наверное можно считать программой в таком случае}
\scnsuperset{модель решения задач на основе нейросетевых моделей}
\scnaddlevel{-1}
;модель решения задач в условиях, когда программа решения не известна\\
\scnaddlevel{1}
\scnsuperset{стратегия решения задач}
\scnaddlevel{1}
\scnhaselement{стратегия поиска пути решения задач в ширину}
\scnhaselement{стратегия поиска пути решения задач в глубину}
\scnaddlevel{-1}
\scnsuperset{модель логического вывода}
\scnaddlevel{1}
\scnsuperset{модель дедуктивного вывода}
\scnsuperset{модель индуктивного вывода}
\scnsuperset{модель трансдуктивного вывода}
\scnsuperset{модель нечеткого вывода}
\scnsuperset{модель на основе темпоральных логик}
\scnsuperset{модель на основе логик умолчаний}
\scnaddlevel{-2}
}

\scnheader{многоагентная система}
\scntext{достоинства}{
\begin{scnitemize}
\item автономность (независимость) агентов в рамках такой системы, что позволяет локализовать изменения, вносимые в решатель при его эволюции, и снизить соответствующие трудозатраты;
\item децентрализация обработки, т.е. отсутствие единого контролирующего центра, что также позволяет локализовать вносимые в решатель изменения.
\end{scnitemize}
}
\scntext{структура}{В общем случае для построения некоторой конкретной многоагентной системы необходимо уточнить следующие ее компоненты:
\begin{scnitemize}
\item модель собственно агента, входящего в состав такой системы, включая классификацию таких агентов и набор понятий, характеризующих каждый агент в рамках системы. В настоящее время наиболее популярной является модель BDI (belief-desire-intention), в рамках которой предполагается описывать на соответствующих языках <<убеждения>>, <<желания>> и <<намерения>> каждого агента системы. 
\item модель среды, в рамках которой находятся агенты, на события в которой они реагируют и в рамках которой могут осуществлять некоторые преобразования. Обзор разновидностей сред для многоагентных систем приводится в работе \cite{Weyns2007}.
\item модель коммуникации агентов, в рамках которой уточняется язык взаимодействия агентов (структура и классификация сообщений) и способ передачи сообщений между агентами. В настоящее время существует ряд стандартов, описывающих языки взаимодействия агентов, например, KQML \cite{Finin1994} и ACL \cite{ACL}. 
\item модель координации агентов, регламентирующая принципы их деятельности, в том числе, механизмы решения возможных конфликтов. В настоящее время основное число работ в области многоагентных систем направлено именно на разработку механизмов координации агентов, в числе которых выделение агентов более высокого уровня (метаагентов) \cite{Hartung2008}, различные социально-психологические модели \cite{Vasconcelos2009,Rumbell2012}, поведение на основе онтологий \cite{Gorodetsky2015} и другие.
\end{scnitemize}}
\scntext{проблемы разработки}{
\begin{scnitemize}
\item жесткая ориентация большинства средств построения многоагентных систем на модель BDI (Belief-Desire-Intention) приводит к существенным накладным расходам, связанным с необходимостью выражения конкретной практической задачи в системе понятий BDI. В то же время, ориентация на модель BDI неявно провоцирует искусственное разделение языков, описывающих собственно компоненты BDI и знания агента о внешней среде, что приводит к отсутствию унификации представления и, соответственно, дополнительным накладным расходам.
\item большинство современных средств построения многоагентных систем ориентированы на представление знаний агента при помощи узкоспециализированных языков, зачастую не предназначенных для представления знаний в широком смысле. Речь при этом идет как о знаниях агента о себе самом (например, в соответствии с моделью BDI), так и знаниях о внешней среде. В некоторых подходах вначале строится онтология, для создания которой, однако, часто используются средства с низкой выразительной способностью, не предназначенные для построения онтологий \cite{Evertsz2004,JADE2017}. В конечном итоге такой подход приводит к сильной ограниченности возможностей разработанных многоагентных систем и их несовместимости.
\item абсолютное большинство современных средств построения многоагентных систем предполагает, что взаимодействие агентов осуществляется путем обмена сообщениями непосредственно от агента к агенту. Такой подход обладает существенным недостатком, связанным с тем, что в этом случае каждый агент системы должен иметь достаточно полную информацию о других агентах в системе, что приводит к дополнительным затратам ресурсов, кроме того добавление или удаление одного или нескольких агентов приводит к необходимости оповещения об этом других агентов. Данная проблема решается путем организации общения агентов по принципу <<доски объявлений>> \cite{Jagannathan1989}, предполагающему, что сообщения помещаются в некоторую общую для всех агентов область, при этом каждый агент в общем случае может не знать, какому из агентов адресовано сообщение и от какого из агентов получено то или иное сообщение. Однако, данный подход не исключает проблему, связанную с необходимостью разработки специализированного языка взаимодействия агентов, который в общем случае не связан с языком, на котором описываются знания агента о решаемых задачах и окружающей среде.
\item многие средства построения многоагентных систем построены таким образом, что логический уровень взаимодействия агентов жестко привязан к физическому уровню реализации многоагентной системы. Например, при передаче сообщений от агента к агенту разработчику многоагентной системы необходимо помимо семантически значимой информации указывать ip-адрес компьютера, на котором расположен агент-получатель, кодировку, с помощью которой закодирован текст сообщения и другую техническую информацию, обусловленную исключительно особенностями текущей реализации средств.
\item в большинстве подходов среда, с которой взаимодействуют агенты, уточняется отдельно разработчиком для каждой многоагентной системы, что с одной стороны, расширяет возможности применения соответствующих средств, но с другой стороны приводит к существенным накладным расходам и несовместимости таких многоагентных систем. Кроме того, в ряде случаев разработчик также обязан учитывать особенности технической реализации средств разработки в плане их стыковки с предполагаемой средой, в роли которой может выступать, например, локальная или глобальная сеть.
\end{scnitemize}}

\scnendstruct

\end{SCn}
\scsubsubsection{Предметная область и онтология интерфейсов компьютерных систем}
\begin{SCn}

\scnsectionheader{Предметная область и онтология интерфейсов компьютерных систем}

\scnstartsubstruct

\scnheader{Предметная область интерфейсов компьютерных систем}
\scnsdmainclasssingle{интерфейс компьютерной системы}
\scnsdclass{***}
\scnsdrelation{***}

\scnheader{интерфейс компьютерной системы}
\scnsuperset{пользовательский интерфейс}
\scnaddlevel{1}
\scnsuperset{естественно-языковой интерфейс}
\scnaddlevel{1}
\scnsuperset{речевой интерфейс}
\scnaddlevel{-1}
\scnaddlevel{-1}

\scnheader{Пользовательский интерфейс}
	\scntext{База пользовательсокго интерфейса}{
\begin{scnitemize}
\item Описание процессов, относящихся к прошлому, настоящему и будущему пользовательского 				интерфейса. Под прошлым пользовательского интерфейса подразумеваются история его 					эксплуатации, а также эволюция интерфейса, под настоящим - текущее состояние 						пользовательского интерфейса, под будущим - планы развития пользовательского 						интерфейса.Анализ изложенных временных процессов позволяет делать оценки эффективности 				развития интерфейса и обеспечивает версионность при проектировании пользовательских 				интерфейсов.

\item Модели пользователей, содержащие информацию об их особенностях, возможностях и 						предпочтениях, что в свою очередь позволяет интерфейсу быть гибким и адаптироваться к 				пользователю, обеспечивая максимально эффективное взаимодействие.

\item Типология действий пользователей и ostis-систем, позволяющая
		описать принципы организации взаимодействия пользовательского интерфейса с пользователями 			на всех уровнях интерфейсного взаимодействия.
\item Типология объектов этих действий, позволяющая произвести уни-
		фикацию и согласование компонентов пользовательских интерфейсов, а
		также составить их иерархию.
\item Формальное описание внешних языков представления конструкций SC-кода, как универсальных, 			так и специализированных.
\end{scnitemize}
}
\scnendstruct
}
\scnheader{Предметная область и онтология пользовательских интерфейсов}
\scnexplanation{Под проектированием пользовательских интерфейсов будет подразумеваться пользовательский интерфейс ostis-системы, поэтому все приводимые далее принципы будут характеризовать именно данный вид интерфейсов. Таким образом, sc-модель такого интерфейса строится в соответствии с общими принципами построения ostis-систем:
}
\scnreltovector{пользовательский интерфейс}{
sc-модель базы знаний пользовательского интерфейса
ostis-системы;
sc-модель машины обработки знаний пользовательского
интерфейса ostis-системы
}
\scnrelfromvector{пользовательский интерфейс}{
командный пользовательский интерфейс IMS;\\
графический пользовательский интерфейс\\
\scnaddlevel{1}
	\scntext{SILK-интерфейс}{\\
	\scnaddlevel{1}
		\scntext{естественно-языковой интерфейс}{
		речевой интерфейс
		}
		\scnaddlevel{-1}
	}
	\scnaddlevel{-1}
}


\scnendstruct

\end{SCn}
\scsubsubsection{Предметная область речевых интерфейсов компьютерных систем}
\begin{SCn}
 
\scnsectionheader{\currentname}
    
\scnstartsubstruct
    
\scnheader{Предметная область речевых интерфейсов компьютерных систем}
\scniselement{предметная область}
    
\scnsdmainclasssingle{интерфейс компьютерной системы}
\scnsdclass{естественно-языковой интерфейс}
\scnsdrelation{диалоговый интерфейс}

\scnheader{автоматическое распознавание речевого собщения (Automatic Speech Recognition -- ASR)}
\scnsubset{обработка естественно языкового сообщения (Natural Language Processing -- NLP)}
\scnexplanation{процесс автоматического анализа речевого сигнала и получения данных о том, что было сказано пользователем, без определения смысловой составляющей \textit{(синтаксический уровень)}. Наиболее часто применяется для преобразования информацию из речевой в текстовую форму.}

\scnheader{понимание речевого сообщения (Spoken Language Understanding -- SPU)}
\scnsubset{понимание естественно языковго сообщения (Natural Language Understanding -- NLU)}
\scnexplanation{процесс \textit{автоматического респознавания речевого сообещения} а также выделние из сообщения данных и знаний о смысле высказывания \textit{(семантический уровень)}. Применяется для семантического анализа речевого сообщения.}

\scnheader{понимание речевого сообщения}
\filemodetrue
\scnrelfromset{ограничения существующих систем}{
Большинство современных систем понимания смысла построены на основе трехуровневой архитектуры, когда речевое сообщение последовательно проходит этапы акустического анализа речевого сигнала, лингвистического анализа, в результате которого получается текстовая форма представления исходного сообщения, а уже только потом производится его семантический анализ. Однако, из работ по психолингвистике и когнитивной психологии известно, что процессы восприятия и понимания в человеческом сознании протекают непрерывно [21], [27], и в общем случае нет необходимости в предварительном приведении речевого сообщения к текстовой форме для выполнения смыслового анали за его содержимого. Устная и письменная формы речи с равным успехом могут быть обработаны сенсорной и когнитивной системами человека [22]. Поэтому вопрос создания методов и систем в которых осуществляется непосредственный переход от обработки сообщения в речевой форме к анализу смыслового его содержимого (семантико-акустического анализа) является весьма актуальным.
;Классический трехуровневый подход (акустический, лингвистический, семантический анализ), в случае решения задачи понимания речевых сообщений, обладает рядом существенных недостатков:
\scnaddlevel{1}
;введение промежуточного этапа преобразования речевого сигнала в текст, влечет дополнительные накладные расходы, связанные с необходимостьюлингвистической обработки, увеличивая тем самым общую вычислительную сложность алгоритма;
;наличие текстового этапа обработки привносит дополнительные ошибки и искажения в следствие ограничений и неполного соответствия лингвистических моделей описываемому процессу, используемых для перехода к текстовому представлению информации на различных стадиях преобразования (фонема-морфема, морфема-лексема, лексема-словосочетание и т.д.) [14];
;при переводе речевого сигнала в текст теряется часть информации, которая может оказаться важной для понимания смысла сообщения, например, громкость, продолжительность звучания, интонация, паузы между словами, которые в тексте могут не всегда однозначно выражаться знаками препинания и др. Особенно актуальной эта проблема становится при анализе сообщений, которые не являются полными предложениями, но при этом могут быть интерпретированы слушателем. Так, например, в повседневной речи предложение, состоящее из одного только звука [a] в зависимости от громкости, интонации и продолжительности звучания может выражать боль, удивление, вопрос, выступать союзом или частицей («ааа, ну его...», «а кто это?», «а если бы сделали по-другому...») [26]
\scnaddlevel{-1}
;Перевод звукового сигнала в текст делает невозможным анализ аудиофрагментов, не являющихся речевыми сообщениями, но несущих потенциально
важную для системы информацию, например:
\scnaddlevel{1}
;условных сигналов, издаваемых объектами внешней среды, в частности, оборудованием на производстве, автомобилями на дороге и др;
;звуков, которые могут соответствовать нештатным ситуациям или сигнализировать об опасности (грохот, лязг, шипение, взрывы, т.д.)
;других звуков, которые потенциально несут информацию о состоянии окружающей среды автоматизированной системы
\scnaddhind{-1}
;Отсутствие средств анализа такого рода сигналов сильно ограничивает возможности автоматизированных систем, ориентированных на работу в постоянно меняющейся среде, в том числе - трудно предсказуемой.
}
\filemodefalse

\scnheader{семантико-акустический анализ}
\scnexplanation{процесс подразумевает первичный разбор речевого сообщения с использованием специальных техник обработки сигнала. В ходе их применения про-
изводится вычленение из потока отдельных «акустических образов» слов, которые в свою очередь будут соответствовать определенным узлам (знакам конкретных сущностей или понятий) в семантической сети.
Предполагается, что результаты этапа акустического анализа будут итерационно корректироваться с учетом информации хранящейся в базе знаний системы, в том числе за счет семантического анализа контекстно-зависимой информации.
}

\scnendstruct

\end{SCn}
    

\scsubsection{Предметная область и онтология интеллектуальных компьютерных систем}
\begin{SCn}

\scnsectionheader{\currentname}

\scnstartsubstruct

\scnheader{Предметная область современных интеллектуальных компьютерных систем}
\scnsdmainclasssingle{интеллектуальная компьютерная система}
\scnsdclass{***}
\scnsdrelation{***}

\scnheader{интеллектуальная система}

\scnexplanation{В процессе эволюции \textit{систем, основанных на обработке информации}, в процессе развития свойств этих систем (увеличение \textit{объема памяти}, повышение \textit{скорости обработки информации}, повышение \textit{уровня гибкости}, уровня структуризации хранимой информации (уровня систематизации обрабатываемой информации, \textit{уровня рефлексивности}, \textit{уровня ассоциативности доступа к хранимой информации}, \textit{уровня стратифицированности}) появляется свойство \textit{интеллектуальности} (разумности) системы, основанной на обработке информации. Это свойство проявляется в целом ряде способностей системы:
\begin{scnitemize}
\item в способности строить \textit{систематизированную информационную модель среды}, в которой функционирует система, причем указанная среда включает в себя как \textit{внешнюю среду} (внешнее материальное окружение, внешние субъекты, с которыми необходимо взаимодействовать, собственное материальное воплощение), так и \textit{внутреннюю среду} (т.е. информацию, хранимую в памяти системы);
\item в способности к рассуждениям...

\scnauthorcomment{см. Финна и Литвинцева новую книгу!!}

\item в способности быстро обучаться

\item гибкость...

\item стратифицированность...

\item рефлексивность...

\end{scnitemize}
}

\scnendstruct \scnendcurrentsectioncomment

\end{SCn}
\scsubsubsection{Предметная область и онтология баз знаний}
\begin{SCn}

\scnsectionheader{Предметная область и онтология баз знаний}

\scnstartsubstruct

\scnheader{Предметная область баз знаний современных интеллектуальных компьютерных систем}
\scnsdmainclasssingle{база знаний}
\scnsdclass{***}
\scnsdrelation{***}

\scnheader{знание}
\scnsuperset{фактографическое знание}
\scnsuperset{закономерность}
\scnsuperset{программа}
\scnsuperset{алгоритм}
\scnsuperset{ситуация}
\scnsuperset{событие}
\scnsuperset{способ решения задачи}
\scnsuperset{онтология}

\scnheader{база знаний}

\scnheader{модель представления знаний}

\scnheader{язык представления знаний}
\scnsubdividing{универсальный язык представления знаний;специализированный язык представления знаний}
\scnhaselement{CycL}
\scnhaselement{IDEF5}
\scnaddlevel{1}
\scnidtf{Integrated Definitions for Ontology Description Capture Method}
\scnaddlevel{-1}
\scnhaselement{Prolog}
\scnhaselement{CLIPS}

\scnheader{онтология}
\scnidtf{эксплицитная спецификация концептуализации}
\scnidtf{формальная спецификация предметной области, включающая описания классов объектов исследования и отношений, заданных на объектах исследования}
\scnsuperset{онтология представления}
\scnsuperset{онтология верхнего уровня}
\scnsuperset{онтология предметной области}
\scnsuperset{прикладная онтология}
\scnsuperset{тезаурус}
\scnauthorcomment{есть много классификаций онтологий, не знаю, стоит ли приводить}

\scnheader{онтология верхнего уровня}
\scnexplanation{В онтологии верхнего уровня представлена систематизация знаний о реальном мире безотносительно к
какой-либо конкретной предметной области. Основной функцией, которая возлагалась на онтологии верхнего уровня, является поддержка семантической совместимости онтологий предметных областей и прикладных онтологий. Поддержка предполагает создание общей точки для формулирования определений. Термины предметно-ориентированных онтологий подчинены терминам онтологии более высокого уровня}
\scnhaselement{OpenCyc}
\scnhaselement{SUMO}
\scnhaselement{DOLCE}
\scnhaselement{Онтология Джона Совы}
\scnhaselement{WordNet}
\scnaddlevel{1}
\scniselement{тезаурус}
\scnaddlevel{-1}

\scnheader{язык представления знаний}
\scnsuperset{язык описания онтологий}
\scnaddlevel{1}
\scnhaselement{OWL}
\scnaddlevel{1}
\scnidtf{Ontology Web Language}
\scnaddlevel{-1}
\scnhaselement{OWL 2}
\scnaddlevel{-1}

\scnendstruct

\end{SCn}
\scsubsubsection{Предметная область и онтология решателей задач, основанных на использовании баз знаний}
\scsubsubsection{Предметная область и онтология интерфейсов компьютерных систем, основанных на использовании баз знаний}

\scsection{Предметная область и онтология человеческой деятельности и соответствующих технологий}
\begin{SCn}

\scnsectionheader{Предметная область и онтология человеческой деятельности и соответствующих технологий}
\scnrelfromlist{подраздел}{Предметная область и онтология традиционных компьютерных
технологий;Предметная область и онтология технологий искусственного интеллекта}

\scnstartsubstruct

\scnheader{Предметная область человеческой деятельности и соответствующих технологий}
\scnsdmainclasssingle{***}
\scnsdclass{технология;информационная технология}
\scnsdrelation{***}

\scnheader{технология}
\scnexplanation{Система организации деятельности, направленной на решение сложных задач соответствующего класса и включающая в себя объекты деятельности, уточняемых до необходимого уровня детализации, а также программы (методы) выполнения соответствующих действий и инструменты (в т.ч. средства автоматизации), с помощью которых эти действия выполняются.}

\scnheader{компьютерная технология}

\scnsubdividing{технология проектирования компьютерных систем;технология сборки компьютерных систем;технология эксплуатации компьютерных систем;технология обновления компьютерных систем}
\scnsubdividing{традиционная компьютерная технология;технология искусственного интеллекта}

\scnheader{компьютеризация научной деятельности}
\scnidtf{автоматизация научной деятельности}
\scnproblems{Очевидно, что высшей формой информационной деятельности является \textbf{научная деятельность} и, следовательно, высшим уровнем развития компьютерных систем являются системы, непосредственно и активно участвующие в этой деятельности. Научная деятельность направлена на повышение качества наших знаний об окружающем нас Мире и, следовательно, связана с анализом, обработкой и систематизацией этих знаний. Совершенно очевидно что, если компьютерные системы, направленные на автоматизацию научной деятельности, будут \textbf{понимать} обрабатываемые ими научные знания и, следовательно, будут становиться не пассивными исполнителями, а партнерами научной деятельности, способными самостоятельно анализировать, систематизировать научные знания и использовать их в решении различных задач, то уровень автоматизации научной деятельности будет существенно повышен.

Важнейшими факторами сдерживания научно-технического прогресса в настоящее время являются:
\begin{scnitemize}
    \item многообразие ("вавилонское столпотворение") как естественных, так и формальных языков, используемых для оформления результатов научно-технических исследований;
    \item привязка научно-технических текстов к естественным языкам (монографии, отчеты, статьи);
    \item принципиальное противоречие между принципами эволюции естественных языков как основного средства коммуникации и требованиями, предъявляемыми к научно-техническим языкам.
\end{scnitemize}

Для решения указанных проблем необходимо:
\begin{scnitemize}
    \item построить строгую формальную систему научно-технических языков;
    \item построить четкую связь между научно-техническими и естественными языками; 
    \item обеспечить оформление научно-технических текстов на совместимых формальных языках, понятных и удобных как людям, так и компьютерным системам; 
    \item обеспечить поддержку эволюции всего этого мультиязыкового комплекса.
\end{scnitemize}

Важнейшим направлением повышения эффективности научно-технической деятельности (и, в частности, повышения темпов научно-технического развития) является переход от традиционного варианта оформления результатов этой деятельности (в форме отчетов, статей, монографий, справочников) к представлению научно-технической информации в виде энциклопедической системы взаимосвязанных баз знаний по различным научно-техническим дисциплинам. Формальным результатом любой научной дисциплины должна стать база знаний, отражающая текущее состояние этой дисциплины. 
Для прикладных научных дисциплин дополнительным результатом должна стать доступная для инженеров компьютерная система автоматизации проектирования искусственных систем соответствующего класса.

Представление о трудностях такого перехода сильно преувеличивается, т. к. современные средства инженерии знаний уже готовы к реализации таких проектов. Этому препятствует: 

\begin{scnitemize}
    \item боязнь нового, непривычного; 
    \item необходимость пересмотра организации научно-технической деятельности.
\end{scnitemize}

Но перспективой является переход на качественно новый уровень культуры научно-технического прогресса. Социальная значимость такого перехода заключается в следующем: 

\begin{scnitemize}
    \item Существенно повысятся темпы эволюции научных знаний благодаря тому, что добываемые научные знания представляются в форме, удобной как для людей, так и для компьютерных систем, а также благодаря автоматизации их интеграции, анализа, структуризации и согласования различных точек зрения. 
    \item Существенно повысится эффективность использования научных знаний в разрабатываемых компьютерных системах, благодаря тому, что отпадает необходимость этапа формализации этих знаний для включения их в состав баз знаний.
    \item Возможность непосредственного участия студентов в совершенствовании тех знаний, которые соответствуют изучаемым ими учебным дисциплинам, существенно повысит качество такого обучения, т.к. способствует индивидуальному, активному и системному усвоению учебного материала.
\end{scnitemize}

Основной проблемой развития научно-технической деятельности и, соответственно, ее информатизации является необходимость глубокой \textbf{конвергенции} различных научных дисциплин, о чем говорится в целом ряде работ~\cite{Palagin, Yankovskaya}.

Важной проблемой также является снижение времени и трудоемкости при организации информационного взаимодействия между научными работниками при \textbf{согласовании точек зрения}, при совместном выполнении каких-либо исследований, при совместной работе над статьями или монографиями, при рецензировании. 

При этом следует помнить, что любая точка зрения всегда имеет недостатки (неполноту, нечеткость и т.п.). Поэтому методологически необходимо переходить от практики противостояния точек зрения к практике интеграции точек зрения (в том числе и тех, которые кажутся альтернативными, противоречащими друг другу). Только так при разработке сложных систем можно добиться синергетического эффекта, в основе которого лежит компенсация недостатков одних точек зрения достоинствами других.

Так и должна быть устроена организация коллективного творческого процесса. Автоматизация такого процесса предполагает фиксацию множественности точек зрения и управление процессом согласования этих точек зрения.\\}
\scnaddlevel{1}
\scnsolutionapproach{Анализ проблем эволюции компьютерных систем разного уровня сложности, разного уровня обучаемости и интеллектуальности, разного назначения показывает, что проклятие "вавилонского столпотворения"\ и, как следствие, несовместимость, дублирование и субъективизм согласовываемых информационных ресурсов и моделей их обработки нас преследует везде:

\begin{scnitemize}
\item и в развитии традиционных компьютерных систем;
\item и в развитии технологий искусственного интеллекта;
\item и в развитии методов и средств компьютеризации научной и инженерной деятельности.
\end{scnitemize}

Рассматривая проблему обеспечения совместимости информационных ресурсов и моделей их обработки, следует говорить о разных аспектах решения этой проблемы:

\begin{scnitemize}
\item об обеспечении совместимости между различными компонентами компьютерных систем, а также между целостными компьютерными системами, входящими в коллективы компьютерных систем;
\item об обеспечении совместимости, т.е. высокого уровня взаимопонимания между различными компьютерными системами и их пользователями;
\item об обеспечении междисциплинарной совместимости, т.е. конвергенции различных областей знаний;
\item о методах и средствах постоянного мониторинга и восстановления совместимости в условиях интенсивной эволюции компьютерных систем и их пользователей, которая часто нарушает достигнутую совместимость (согласованность) и требует дополнительных усилий на ее восстановление.
\end{scnitemize}}
\scnaddlevel{-1}

\scnheader{компьютерная технология}
\scnexplanation{Ключевая на текущий момент проблема развития компьютерных технологий в целом и технологий искусственного интеллекта в частности - \textbf{проблема обеспечения информационной совместимости} компьютерных систем и в том числе интеллектуальных компьютерных систем.

Актуальность решения этой проблемы обусловлена тем, что:
\begin{scnitemize}
    \item информационная совместимость компьютерных систем существенно \textbf{повысит уровень их обучаемости} благодаря более эффективному восприятию опыта (знаний и навыков) от других  компьютерных систем;
    \item появится возможность существенно \textbf{расширять многообразие} используемых в компьютерной системе знаний и навыков без необходимости разработки специальных средств их согласования. Это также повышает уровень обучаемости компьютерных систем и позволяет переходить к \textbf{гибридным, синергетическим} компьютерным системам;
    \item появится возможность создания \textbf{коллективов компьютерных систем}, использующих универсальные принципы организации взаимодействия между компьютерными системами на содержательном уровне;
    \item появится возможность не только разрабатывать совместимые компьютерные системы, но и автоматизировать процесс постоянной \textbf{поддержки совместимости компьютерных систем}. Необходимость указанной поддержки вызвана тем, что совместимость компьютерных систем в ходе их эксплуатации и эволюции может нарушаться. Следовательно, должны существовать средства перманентного восстановления совместимости компьютерных систем в условиях их постоянного изменения;
    \item появится возможность автоматизации процесса постоянной поддержки (восстановления) информационной \textbf{совместимости} компьютерных систем не только с другими компьютерными системами, но и \textbf{с их пользователями};
    \item появится возможность существенно сократить сроки разработки новых компьютерных систем с помощью постоянно расширяемой \textbf{библиотеки многократно используемых компонентов компьютерных систем}, имеющих разный уровень сложности (вплоть до типовых встроенных подсистем) и различный вид (типовые встраиваемые знания, например, онтологии, широко используемые навыки, в частности, программы,  интерфейсные подсистемы, обеспечивающие обмен сообщениями с внешними субъектами на заданном внешнем языке).
\end{scnitemize}}

\scnheader{компьютерная система}
\scnevolutiondirections{В эволюции компьютерных систем можно выделить два общих направления.

\textbf{Первое направление} - это 

\begin{scnitemize}
    \item \textbf{расширение множества и многообразия задач}, решаемых компьютерной системой; 
    \item повышение \textbf{сложности этих задач} вплоть до трудно формализуемых (трудно решаемых) задач, интеллектуальных задач, решаемых в условиях неполноты, неточности, нечеткости и т.д.;
    \item повышение \textbf{качества решения задач} либо путем более эффективного использования известных моделей решения задач (например, путем разработки более качественных алгоритмов), либо путем использования принципиально новых моделей решения задач;
    \item расширение \textbf{многообразия используемых видов информации} (знаний);
    \item расширение \textbf{многообразия используемых моделей решения задач}.
\end{scnitemize}

Очевидно, что расширение множества решаемых задач в условиях пусть и большой, но всегда конечной памяти компьютерной системы делает все более и более актуальным переход от частных методов и моделей решения задач к их обобщениям (или, как отмечал Д.А. Поспелов, от связки "ключей"\ к набору "отмычек").

Очевидно также, что многообразие видов задач, решаемых компьютерными системами, многообразие используемых моделей решения задач приводит: 
\begin{scnitemize}
    \item к интегрированным информационным ресурсам;
    \item к интегрированным решателям задач;
    \item к интегрированным компьютерным системам;
    \item к коллективам компьютерных систем.
\end{scnitemize}

Проблема здесь заключается не в самой интеграции, а в ее качестве. Интеграция может быть \textbf{эклектичной}, если не обеспечить совместимость интегрируемых компонентов, а в случае такой совместимости интеграция может привести к новому качеству, к дополнительному расширению множества решаемых задач. Это будет означать переход от эклектичности к гибридности, синергетичности.

\textbf{Второе общее направление} эволюции компьютерных систем -- это повышение уровня их \textbf{обучаемости} и, как следствие, темпов их эволюции.

\textbf{Обучаемость} компьютерной системы определяется:

\begin{scnitemize}
    \item \textbf{трудоемкостью} и темпами приобретения (расширения) и совершенствования активно используемых знаний и навыков;
    \item \textbf{уровнем ограничений}, накладываемых на вид приобретаемых и используемых знаний и навыков (фактически, это ограничения на множество всех тех задач, которые принципиально могут быть решены данной компьютерной системой).
\end{scnitemize}

В свою очередь, \textbf{трудоемкость и темпы расширения и совершенствования} знаний и навыков компьютерной системы определяется:
\begin{scnitemize}
    \item \textbf{гибкостью} -- многообразием и трудоемкостью возможных изменений, вносимых в систему в процессе пополнения системы новыми знаниями и навыками и совершенствования уже приобретенных знаний и навыков;
    \item \textbf{стратифицированностью} -- четким разделением системы на достаточно независящие друг от друга уровни иерархии, т.е. возможностью локализации фрагментов компьютерной системы, не выходя за пределы которых, \uline{априори достаточно} проводить анализ последствий тех или иных вносимых в систему изменений;
    \item \textbf{рефлексивностью} –- способностью анализировать собственное состояние и свою деятельность;
    \item \textbf{гибридностью} -- способностью приобретать и использовать широкое (а в идеале -- неограниченное) многообразие знаний и навыков;
    \item \textbf{уровнем самообучаемости} -- уровнем активности, самостоятельности, целеустремленности в процессе своего обучения, т.е. уровнем способности к обучению \uline{без учителя}, уровнем автоматизации приобретения новых знаний и навыков, а также совершенствования уже приобретенных знаний и навыков;
    \item \textbf{совместимостью} -- трудоемкостью интеграции;
    \item \textbf{способностью к постоянному мониторингу и поддержке своей совместимости} как с другими компьютерными системами, так и со своими пользователями в условиях интенсивной эволюции этих компьютерных систем и их пользователей. 
\end{scnitemize}

\textbf{Совместимость} (трудоемкость интеграции) компьютерных систем может рассматриваться в двух аспектах:

\begin{scnitemize}
    \item в аспекте \textbf{глубокой интеграции} компьютерных систем, что предполагает преобразование нескольких компьютерных систем в одну целостную компьютерную систему путем объединения информационных и функциональных ресурсов интегрируемых компьютерных систем;
    \item в аспекте преобразования нескольких компьютерных систем в \textbf{коллектив взаимодействующих компьютерных систем}, способных к совместному корпоративному решению сложных задач.
\end{scnitemize}

Совместимость (трудоемкость интеграции) компьютерных систем определяется:
\begin{scnitemize}
    \item совместимостью различного вида информации (знаний), хранимой в памяти компьютерной системы;
    \item совместимостью различных моделей решения задач;
    \item совместимостью встроенных (в т.ч. типовых) подсистем, входящих в состав компьютерных систем;
    \item совместимостью внешней информации, поступающей на вход компьютерной системе, с информацией, хранимой в памяти компьютерной системы (трудоемкостью понимания внешней информации -- трансляции, погружения, выравнивания понятий);
    \item коммуникационной (в т.ч. семантической) совместимостью с пользователями и с другими компьютерными системами.
\end{scnitemize}

Важнейшая форма обучения компьютерной системы это приобретение новых знаний и навыков в "готовом"\ виде, т.е. в виде некоторых знаковых конструкций, вводимых в память компьютерной системы, поскольку приобретение знаний и навыков из внешних достоверных источников требует существенно меньшего времени по сравнению с их приобретением собственными силами, на основе собственного опыта и собственных ошибок. Но для того, чтобы указанная форма обучения была эффективной, необходимо максимально возможным образом упростить и формализовать механизм (процедуру) погружения новых знаний в память компьютерной системы.

Для решения этой задачи ключевое значение имеет создание удобного для этой цели способа кодирования различного вида информации в памяти компьютерной системы.

Поскольку основным каналом обучения компьютерных систем является приобретение ими знаний и навыков от других субъектов -- от других компьютерных систем и от пользователей (от разработчиков-учителей и от конечных пользователей), важнейшим фактором обучаемости компьютерной системы является превращение компьютерной системы в коммуникативную систему, способную эффективно общаться с внешними субъектами. Следовательно, уровень обучаемости компьютерных систем определяется также уровнем ее совместимости с самими этими внешними субъектами, с приобретаемыми ею знаниями и навыками, т.е. степенью того, как компьютерная система вместе с теми субъектами, с которыми она обменивается информацией, решает проблему "вавилонского столпотворения".\\}
\scnaddlevel{1}
\scnsolutionapproach{Суть предлагаемого нами подхода к решению проблем эволюции компьютерных систем заключается, во-первых, в объединении всех указанных выше направлений эволюции компьютерных систем (как общих направлений, так и частных) и, во-вторых, в трактовке проблемы обеспечения \textbf{совместимости} различных видов знаний, различных моделей решения задач, различных компьютерных систем как \textbf{ключевой проблемы} эволюции компьютерных систем, решение которой существенно упростит решение и многих других проблем.

Так, например, без обеспечения совместимости информационных ресурсов, используемых в разных компьютерных системах, а также информационных ресурсов, представляющих знания различного семантического вида невозможно:

\begin{scnitemize}
\item ни создавать \textbf{коллективы компьютерных систем}, способные координировать свои действия при кооперативном расширении сложных задач;
\item ни создавать \textbf{гибридные компьютерные системы}, которые способны при решении сложных комплексных задач использовать всевозможные сочетания разных видов знаний и разных моделей решения задач;
\item ни использовать \textbf{компонентную методику проектирования} компьютерных систем \textbf{на всех уровнях} иерархии проектируемых систем.
\end{scnitemize}

О какой информационной совместимости и взаимопонимании (в т.ч. между специалистами) можно говорить при наличии ужасающей понятийной и терминологической неряшливости, терминологического псевдотворчества, в том числе, в области информатики.

Говоря о \textbf{совместимости} компьютерных систем и их компонентов, а также совместимости компьютерных систем с пользователями, следует отметить неоднозначность трактовки термина ``совместимость''. В этой связи следует отличать:

\begin{scnitemize}
    \item совместимость как один из факторов обучаемости, как \textbf{способность} к быстрому повышению уровня согласованности (интеграции, взаимопонимания).
    Сравните обучаемость как \textbf{способность} к быстрому расширению знаний и навыков, но никак не характеристика объема и качества приобретенных знаний и навыков;
    \item совместимость как характеристика достигнутого уровня согласованности (интеграции, взаимопонимания).
\end{scnitemize}

Аналогичным образом интеллект компьютерной системы с одной стороны можно трактовать как \textbf{уровень} (объем и качество) приобретенных знаний и навыков, а с другой стороны как \textbf{способность} к быстрому расширению и совершенствованию знаний и навыков, т.е. как \textbf{скорость} повышения уровня знаний и навыков.

Кроме того, следует говорить не только о \textbf{способности} к быстрому повышению уровня согласованности и не только о достигнутом уровне согласованности, но и о самом \textbf{процессе} повышения уровня согласованности и, прежде всего, о перманентном процессе восстановления (поддержки, сохранения) достигнутого уровня согласованности, поскольку в ходе эволюции компьютерных систем и их пользователей (т. е. в ходе расширения и повышения качества их знаний и навыков) уровень их согласованности может понижаться.\\}
\scnaddlevel{-1}

\scnendstruct

\end{SCn}

\scsubsection{Предметная область и онтология традиционных компьютерных технологий}
\begin{SCn}

\scnsectionheader{Предметная область и онтология традиционных компьютерных технологий}

\scnstartsubstruct

\scnheader{Предметная область традиционных компьютерных технологий}
\scnsdmainclasssingle{традиционная компьютерная технология}
\scnsdclass{***}
\scnsdrelation{***}

\scnheader{традиционная компьютерная технология}
\scntext{текущее состояние}{Современное состояние \textbf{\textit{традиционных компьютерных технологий}} в целом можно охарактеризовать как:
\begin{scnitemize}
    \item иллюзию благополучия;
    \item иллюзию всесилия финансовых ресурсов в решении сложных технических задач;
    \item "вавилонское столпотворение"\ различных технических решений, о совместимости которых никто серьезно не задумывается;
    \item отсутствие комплексного системного подхода к автоматизации сложных видов проектной деятельности;
    \item отсутствие осознания того, что недостатки современных компьютерных технологий имеют фундаментальный, системный характер.
\end{scnitemize}}

\scntext{недостатки}{К недостаткам традиционных компьютерных технологий можно отнести:
\begin{scnenumerate}
\item Многообразие синтаксических форм представления одной и той же информации, т.е. многообразие семантически эквивалентных форм (языков) представления (кодирования) обрабатываемой информации (знаний) в памяти компьютерных систем. Отсутствие унификации представления различного вида знаний в памяти современных компьютерных систем приводит:
\begin{scnitemizeii}
	\item к многообразию семантически эквивалентных моделей решения задач (как процедурных, так и непроцедурных – функциональных, логических и т.д.), т.е. к дублированию моделей обработки информации, отличающихся не сутью способов решения задач, а формой представления обрабатываемой информации и формой представления способов (навыков) решения различных классов задач;
	\item к дублированию семантически эквивалентных информационных компонентов компьютерных систем;
	\item к многообразию форм технической реализации каждой используемой модели решения задач;
	\item к семантической несовместимости компьютерных систем и, следовательно, к высокой трудоемкости их интеграции в системы более высокого уровня иерархии, требующей дополнительных усилий на трансляцию (конвертирование) информации, которой обмениваются разные интегрируемые системы и, следовательно, существенно ограничивающей эффективность совместного решения задач коллективом взаимодействующих компьютерных систем. Трудоемкость процесса интеграции может быть существенно снижена за счет приведения интегрируемых компьютерных систем к некоторой унифицированной форме, поскольку в этом случае интеграция может быть осуществлена универсальным и автоматизированным способом;
	\item к существенному снижению эффективности применения методики компонентного проектирования компьютерных систем на основе библиотек многократно используемых компонентов (особенно, если речь идет о "крупных"\ компонентах, в частности, о типовых подсистемах)~\cite{Borisov2014}.
\end{scnitemizeii}

\item Недостаточно высокую степень обучаемости современных компьютерных систем в ходе их эксплуатации, следствием чего является высокая трудоемкость их сопровождения и совершенствования, а также недостаточно длительный их жизненный цикл. 

\item Отсутствие возможности у экспертов реально влиять на качество разрабатываемых компьютерных систем. Опыт разработки сложных компьютерных систем показывает, что посредничество программистов между экспертами и проектируемыми компьютерными системами существенно искажает вклад экспертов. При разработке компьютерных систем следующего поколения доминировать должны не программисты, а эксперты, способные точно излагать свои знания.

\item Отсутствие семантической (смысловой) унификации интерфейсной деятельности пользователей компьютерных систем, что вместе с многообразием форм реализации пользовательских интерфейсов приводит к серьезным накладным расходам на усвоение пользовательских интерфейсов новых компьютерных систем.

\item Документация компьютерной системы не является важным компонентом самой компьютерной системы, определяющим качество функционирования этой системы, следствием чего является недостаточно высокая эффективность эксплуатации компьютерной системы из-за неполного и неэффективного использования возможностей эксплуатируемой компьютерной системы.
\end{scnenumerate}

Преодолеть указанные недостатки можно только путем фундаментального переосмысления архитектуры и принципов организации сложных компьютерных систем. Основой такого переосмысления является устранение многообразия форм представления (кодирования) информации в памяти компьютерных систем.

Результатом такого переосмысления должен стать новый этап развития компьютерных технологий.

Преодоление недостатков современных компьютерных систем предполагает:
\begin{scnitemize}
\item унификацию представления обрабатываемой информации;
\item функциональную унификацию (унификацию принципов обработки информации).
\end{scnitemize}}

\scnheader{традиционная компьютерная система}
\scnevolution{Расширение областей применения компьютерных систем требует перехода от традиционных компьютерных систем к системам, ориентированным на обработку широкого многообразия структурированной информации, а также на решение все более и более сложных задач. Следовательно, переход от традиционных компьютерных систем к интеллектуальным системам неизбежен. Более того, этот переход давно уже происходит. Это подтверждают такие направления эволюции компьютерных систем, как:

\begin{scnitemize}
    \item переход от доминирования программ к доминированию обрабатываемой информации, т. е. компьютерным системам, управляемым данными;
    \item от слабоструктурированных данных к структурированным и независящим от программ, обрабатывающих эти данные, т. е. к базам данных;
    \item от данных к знаниям путем расширения семантических видов обрабатываемой информации, и далее к компьютерным системам, управляемым структурированными знаниями, и к компьютерным системам, управляемыми базами знаний;
    \item переход от неконтекстного решения задач, исходные данные для которых априори точно заданы, к решению задач с активным использованием контекста этих задач, т. е. знаний о той предметной области, в рамках которой задача решается;
    \item переход и процедурных языков программирования низкого уровня к процедурным языкам программирования высокого уровня, и к непроцедурным языкам программирования (функциональным, логическим);
    \item переход от последовательных программ к параллельным;
    \item переход от синхронной обработки информации к асинхронной;
    \item переход от программ к исчислениям, к "мягким"\ вычислениям (нечетким логикам, генетическим алгоритмам, искусственным нейросетям);
    \item переход от программ, ориентированных на обработку данных, структуризация которых определяется соответствующими программами, к программам, ориентированным на обработку баз данных и далее баз знаний;
    \item переход от адресной памяти к ассоциативной памяти;
    \item переход от линейной памяти к нелинейной (структурно перестраиваемый, реконфигурируемой, графодинамической) памяти, в которой обработка информации сводится не только к изменению состояния элементов в памяти, но и к изменению конфигурации связей между ними;
    \item переход от традиционных компьютерных систем к компьютерным системам, способным решать широкое многообразие сложных (трудно формализуемых) задач и, в том числе интеллектуальных задач, к компьютерным системам с гибридной хорошо структурированной базой знаний высокого качества, с гибридным решателем задач, с гибридным (мультимодальным) интерфейсом (как вербальным, так и невербальным);
    \item переход от необучаемых компьютерных систем к обучаемым.
\end{scnitemize}

Следовательно, интеллектуализация компьютерных систем -- это естественное направление их эволюции.}

\scnendstruct

\end{SCn}

\scsubsection{Предметная область и онтология технологий искусственного интеллекта}
\begin{SCn}

\scnsectionheader{Предметная область и онтология технологий искусственного интеллекта}

\scnstartsubstruct

\scnheader{Предметная область технологий искусственного интеллекта}
\scnsdmainclasssingle{технология искусственного интеллекта}
\scnsdclass{***}
\scnsdrelation{***}

\scnheader{технология искусственного интеллекта}
\scntext{текущее состояние}{До настоящего времени \textit{традиционные компьютерные технологии} и \textbf{\textit{технологии искусственного интеллекта}} развивались \textbf{\uline{независимо друг от друга}}.

Сейчас настало время \textbf{фундаментального переосмысления} опыта использования и эволюции \textit{традиционных компьютерных технологий} и их \textbf{интеграции} с \textit{технологиями искусственного интеллекта}. Это необходимо для устранения целого ряда недостатков современных компьютерных технологий.

Опыт использования компьютерных систем для автоматизации различных видов человеческой деятельности показывает, что автоматизация беспорядка приводит к еще большему беспорядку, а безграмотная автоматизация хуже ее отсутствия. При этом, если автоматизация требует применения методов и средств искусственного интеллекта, то последствия безграмотной автоматизации могут быть еще более разрушительны.

Это значит, что прежде, чем приступить к автоматизации какой-либо деятельности (и, тем более, с применением средств искусственного интеллекта), необходимо построить качественную формальную модель этой деятельности (т.е. достаточно детальное целостное ее описание, но без излишеств).}

\scnheader{технология искусственного интеллекта}
\scntext{направления эволюции}{Расширение областей применения компьютерных систем приводит к расширению многообразия автоматизируемых видов деятельности - управление предприятиями различного вида, управление организациями, управление сложными техническими системами, мультисенсорная интеграция и первичный анализ невербальной информации, распознавание, проектирование искусственных объектов различного вида, проектирование систем бизнес-процессов, направленных на воспроизводство спроектированных искусственных объектов, общение с пользователями (на естественных языках в текстовый и речевой форме, с помощью средств когнитивной графики), обучение пользователей, комплексное информационное обслуживание пользователей.

В свою очередь, расширение многообразия автоматизируемых видов деятельности приводит к расширению многообразия видов решаемых задач, видов методов и средств решения задач, видов используемой информации (видов знаний).

Так, например, повышение уровня автоматизации различных предприятий приводит к знание-ориентированной организации их деятельности, а в перспективе -- к знание-ори\-ен\-ти\-ро\-ван\-ной экономике. Это означает, что основой автоматизации предприятия становятся средства управления знаниями. 

Из этого, в свою очередь, следует, что в перспективных системах управления предприятиями необходимо переходить от баз данных, обеспечивающих представление достаточно простых (фактографических) видов знаний, к базам знаний, в состав которых могут входить знания самого различного вида.}

\scnheader{технология искусственного интеллекта}
\scnevolutiondirections{К числу современных наиболее активно развиваемых направлений развития теории интеллектуальных компьютерных систем и технологий искусственного интеллекта можно отнести:

\begin{scnitemize}
    \item управление знаниями и онтологический инжиниринг \cite{Gavrilova2016}, Semantic Web~\cite{W3C};
    \item формальные логики (четкие, нечеткие, дедуктивные, индуктивные, абдуктивные, дескриптивные, темпоральные, пространственные и т.д.);
    \item искусственные нейросети, байесовские сети, генетические алгоритмы (Machine learning в узком смысле);
    \item компьютерная лингвистика (Natural Language Processing, NLP), семантический анализ текстов естественного языка;
    \item speech processing, семантический анализ речевых сообщений;
    \item image processing – техническое зрение, семантический анализ изображений; 
    \item многоагентные системы, коллективы интеллектуальных систем \cite{Wooldridge2009, Tarasov2002, Yarushkina2007};
    \item гибридные интеллектуальные системы, синергетические интеллектуальные системы \cite{Kolesnikov2001}.
\end{scnitemize}
}

\scnheader{технология искусственного интеллекта}
\scnevolutionproblems{Несмотря на наличие серьезных \textbf{научных результатов} в области искусственного интеллекта, темпы \textbf{развития рынка интеллектуальных систем} не столь впечатляющи.

Причин тому несколько:
\begin{scnitemize}
    \item имеет место большой разрыв между научными исследованиями в области искусственного интеллекта и созданием качественных технологий разработки интеллектуальных систем. Научные исследования в области искусственного интеллекта в основном сконцентрированы на разработку новых методов решения интеллектуальных задач;
    \item указанные исследования разрозненны и не осознана необходимость их интеграции и создания общей формальной теории интеллектуальных систем, т. е. имеет место "вавилонское столпотворение"\ различных моделей, методов и средств, используемых в искусственном интеллекте при отсутствии осознания проблемы обеспечения их совместимости. Без решения этой проблемы не может быть создана ни общая теория интеллектуальных систем, ни, следовательно, комплексная технология разработки интеллектуальных систем, доступная инженерам и \textbf{экспертам};
    \item указанная интеграция моделей и методов искусственного интеллекта весьма сложна, т. к. носит междисциплинарный характер;
    \item интеллектуальные системы как объекты проектирования имеют значительно более высокий уровень сложности по сравнению со всеми техническими системами, которыми человечество имело дело;
    \item как следствие вышесказанного, имеет место большой разрыв между научными исследованиями и инженерной практикой в этой области. Заполнить этот разрыв можно только путем создания эволюционируемой технологии разработки интеллектуальных систем, развитие которой осуществляется путем активного сотрудничества ученых и инженеров;
    \item качество разработки прикладных интеллектуальных систем в большой степени зависит от взаимопонимания экспертов и инженеров знаний. Инженеры знаний, не владея тонкостями прикладной области, могут вносить серьезные ошибки в разрабатываемые базы знаний. Посредничество инженеров знаний между экспертами и разрабатываемой базой знаний существенно снижает качество разрабатываемых интеллектуальных систем. Для решения этой проблемы необходимо, чтобы \textit{язык представления знаний} в базе знаний был удобен не только \textit{интеллектуальной системе} и \textit{инженерам знаний}, \textbf{но и \textit{экспертам}}.
    
\end{scnitemize}

Текущее состояние технологий искусственного интеллекта можно охарактеризовать следующим образом:
\begin{scnitemize}
    \item Есть большой набор частных технологий искусственного интеллекта с соответствующими инструментальными средствами, но отсутствует общая теория интеллектуальных систем и, как следствие, отсутствует общая комплексная технология проектирования интеллектуальных систем  (см. конференции «Artificial General Intelligence»~\cite{AGI2018});
    \item Совместимость частных технологий искусственного интеллекта практически не осуществляется и более того, отсутствует осознание такой необходимости.
\end{scnitemize}


Развитие технологий искусственного интеллекта существенным образом затрудняется следующими социально-методологическими обстоятельствами:
\begin{scnitemize}
    \item Высокий социальный интерес к результатам работ в области искусственного интеллекта и большая сложность этой науки порождает поверхностность и нечистоплотность при разработке и рекламе различных приложений. Серьезная наука перемешивается с безответственным маркетингом, понятийной и терминологической неряшливостью и безграмотностью, вбрасыванием новых абсолютно ненужных эффектных терминов, запутывающих суть дела, но создающих иллюзию принципиальной новизны.
    \item Междисциплинарный характер исследований в области искусственного интеллекта существенно затрудняет эти исследования, т.к. работа на стыках научных дисциплин требует высокой культуры и квалификации.
\end{scnitemize}}

\scnaddlevel{1}
\scntext{предлагаемый подход к решению}{Для решения указанных выше проблем развития \textbf{\textit{технологий искусственного интеллекта}}:
\begin{scnitemize}
    \item Продолжая разрабатывать новые формальные модели решения интеллектуальных задач и совершенствовать существующие модели (логические, нейросетевые, продукционные), необходимо обеспечить совместимость этих моделей как между собой, так и с традиционными моделями решения задач, не попавших в число интеллектуальных задач. Другими словами, речь идет о разработке принципов организации гибридных интеллектуальных систем, обеспечивающих решение \textbf{комплексных задач}, требующих совместного использования и в непредсказуемых комбинациях самых различных видов знаний и самых различных моделей решения задач.
    \item Необходим переход от эклектичного построения сложных интеллектуальных систем, использующих различные виды знаний и различные виды моделей решения задач, к глубокой их интеграции, когда одинаковые модели представления и модели обработки знаний реализуется в разных системах и подсистемах одинаково. 
    \item Необходимо сократить дистанцию между современным уровнем теории интеллектуальных систем и практики их разработки. 
    \item Необходимо существенно повысить уровень согласованности действий лиц, участвующих в процессе постоянного совершенствования баз знаний.
    \item Надо, чтобы в решении этой проблемы совместимости интеллектуальных систем активно участвовали сами системы, а не только их разработчики. Системы должны сами заботиться о поддержке своей совместимости с другими системами в условиях активного изменения этих систем с помощью механизма автоматизированного согласования используемых понятий между интеллектуальными системами.
\end{scnitemize}}
\scnaddlevel{-1}
\scnendstruct

\end{SCn}
\scsubsubsection{Предметная область и онтология технологий разработки интеллектуальных компьютерных систем}
\scparagraph{Предметная область и онтология технологий разработки баз знаний}
\begin{SCn}

\scnsectionheader{Предметная область и онтология технологий разработки баз знаний}

\scnstartsubstruct

\scnheader{Предметная область технологий разработки баз знаний}
\scnsdmainclasssingle{технология разработки баз знаний}
\scnsdclass{***}
\scnsdrelation{***}

\scnheader{методология разработки баз знаний}
\scnexplanation{Методология разработки онтологий представляет собой набор инструкций и руководств, описывающих процесс выполнения сложных процедур разработки онтологий. Она детализирует различные задачи, как они должны быть выполнены, в каком порядке и каким образом осуществлять документирование работы по созданию онтологий.}
\scnsubdividing{методология, поддерживающая совместную коллективную разработку онтологии;методология, не поддерживающая совместную коллективную разработку онтологии}
\scnsubdividing{методология, зависимая от инструментария;методология, частично зависимая от инструментария;методология, независимая от инструментария}
\scnsubdividing{методология без указания модели жизненного цикла онтологии;методология с итеративной моделью жизненного цикла онтологии; методология с моделью жизненного цикла онтологии на основе эволюционного прототипирования;методология с совпадающей с моделью жизненного цикла приложения}
\scnsubdividing{методология, предусматривающая способы формализации;методология, не предусматривающая способы формализации}
\scnsubdividing{методология, поддерживающая повторное использование разрабатываемых онтологий;методология, не поддерживающая повторного использования разрабатываемых онтологий}
\scnsubdividing{методология, использующая стратегию "cнизу вверх"{} (bottom-up) для выделения концептов предметной области;методология, использующая стратегию "сверху вниз"{} (top-down) для выделения концептов предметной области;методология, использующая стратегию "от середины"{} (middle-out) для выделения концептов предметной области; методология, сочетающая различные стратегии для выделения концептов предметной области}
\scnhaselement{скелетная методология Ушолда и Кинга}
\scnhaselement{методология Грюнингера и Фокса (TOVE)}
\scnhaselement{METHONTOLOGY}
\scnhaselement{On-To-Knowledge (OTK)}
\scnhaselement{KACTUS}
\scnhaselement{DILIGENT}
\scnhaselement{SENSUS}
\scnhaselement{UPON}
\scnnote{Большинство методологий не поддерживают совместную разработку баз знаний, поддержку совместимости разрабатываемых баз знаний и, как следствие, поддержку повторного использования уже разработанных баз знаний и их компонентов. 
Кроме того, подавляющее большинство методологий разработки баз знаний описывают процесс разработки в общих чертах, не регламентируя действия участников на каждом этапе разработки онтологии, не уточняя принципы согласования новых понятий с уже существующими, высоким оказывается субъективное влияние разработчиков.}

\scnheader{средство разработки баз знаний}
\scnsuperset{среда разработки онтологий}
\scnaddlevel{1}
\scnsuperset{инструмент создания онтологий}
\scnaddlevel{1}
\scnexplanation{Данный класс средств обеспечивает процесс создания базы знаний "с нуля"{}.}
\scnnote{Помимо редактирования и просмотра данный класс средств обеспечивает поддержку документирования онтологий, импорт/экспорт онтологий в различные форматы и языки, управление библиотеками онтологий.}
\scnhaselement{Protege}
\scnhaselement{NeON}
\scnhaselement{Co4}
\scnhaselement{Ontolingua}
\scnhaselement{OntoEdit}
\scnhaselement{OilEd}
\scnhaselement{WebOnto}
\scnaddlevel{-1}
\scnsuperset{инструмент отображения, выравнивания и объединения онтологий}
\scnaddlevel{1}
\scnexplanation{Данный класс инструментов помогает пользователям найти сходство и различие между исходными онтологиями и создают результирующую онтологию, которая содержит элементы исходных онтологий.}
\scnhaselement{PROMPT}
\scnhaselement{Chimaera}
\scnhaselement{OntoMerge}
\scnhaselement{OntoMorph}
\scnhaselement{OBSERVER}
\scnhaselement{FCAMerge}
\scnhaselement{ONION}
\scnaddlevel{-1}
\scnsuperset{инструмент аннотирования на основе онтологий}
\scnaddlevel{1}
\scnhaselement{MnM}
\scnhaselement{SHOE}
\scnhaselement{Knowledge Annotator}
\scnaddlevel{-2}
\scnsuperset{библиотека многократно используемых компонентов баз знаний}
\scnaddlevel{1}
\scnhaselement{Protege ontology library}
\scnhaselement{Ontaria ontology directory}
\scnaddlevel{-1}
\scnsuperset{средство коллективной разработки баз знаний}

\scnheader{средство коллективной разработки баз знаний}
\scnhaselement{Collaborative Protege}
\scnaddlevel{1}
	\scntext{основные возможности}{Основные поддерживаемые возможности для коллективной разработки онтологий Collaborative Protege:
	\begin{scnitemize}
    \item \textit{аннотирование компонентов онтологии} (таких как классы, свойства, отдельные лица);
    \item \textit{аннотация изменений} (создание классов, переименование и т. д.);
    \item \textit{обсуждения}. Пользователь может ответить на комментарий другого пользователя об определенном компоненте онтологии;
    \item \textit{предложения и голосование}. Пользователь может начать предложение об изменении онтологии;
    \item \textit{поиск и фильтрация аннотаций на основе различных критериев};
    \item \textit{живое обсуждение (чат)}. Пользователи, подключенные одновременно к одному и тому же серверу Protege, могут общаться друг с другом. Сообщения чата транслируются всем пользователям.
\end{scnitemize}}
	\scnaddlevel{-1}
\scnhaselement{NeON}
\scnaddlevel{1}
	\scntext{особенности}{Особенностями платформы NeOn являются:
\begin{scnitemize}
    \item поддержка "жизненного цикла"{}, включая взаимодействие активностей периода разработки и исполнения;
    \item ориентация на онтологический инжиниринг и использование онтологий;
    \item расширяемость архитектуры на всех уровнях.
\end{scnitemize}}
	\scnnote{Наиболее важные варианты использования проекта NeOn:
	\begin{scnitemize}
	\item \textit{Поиск}. При редактировании онтологии редактор онтологии может выполнять поиск по всем редактируемым онтологиям;
	\item \textit{Запрос ответа}. При редактировании онтологии редактор онтологии может выполнять запросы в редактируемых онтологиях;
	\item \textit{Управление многоязычностью}. Редактор онтологии занимается добавлением языков к онтологии, выполнением проверки орфографии, управлением многоязычными метками, выбором рабочего языка, а также преодолением специфических особенностей перевода;
	\item \textit{Экспорт} онтологии в другие форматы;
	\item \textit{Преобразование} онтологий из других форматов;
	\item \textit{Управление сопоставлениями}. Создание выравниваний между онтологиями вручную и полуавтоматическим способом. Создаются сопоставления между понятиями или модулями в различных онтологиях;
	\item \textit{Визуализация}. Визуализируются отображения и отношения между концепциями и модулями в сетевых онтологиях;
	\item \textit{Модульность}. Работа с модулями онтологии; создание модулей вручную, полуавтоматически и объединение модулей;
	\item \textit{Управление статистикой}.  Пользователи могут видеть историю изменений, просматривать статистику использования и статистику онтологии (глубина, число дочерних узлов, число отношений и свойств, количество понятий на «ветке»);
	\item \textit{Оценка и проверка онтологии}. Редактор онтологии может проверить качество разработки онтологии, проверить наличие дубликатов в онтологии, провести сравнение с другими онтологиями и оценить структурные свойства онтологии.
	\item \textit{Документирование}. Автоматическое создание соответствующих метаданных, генерация UML-диаграмм и документации, касающейся используемых отношений и свойств.
	\end{scnitemize}}
	\scnaddlevel{-1}
\scnhaselement{Co4}
\scnaddlevel{1}
	\scnexplanation{\textit{Co4} – инфраструктура, обеспечивающая совместное конструирование базы знаний в интернет-среде. При этом разработка баз знаний трактуется как социальный процесс, в который вовлечено сообщество множества агентов, а система Co4 поддерживает разработку с помощью экспертов, которые являются равноправными участниками коллективной разработки.}
	\scntext{особенность}{Какое-либо изменение базы знаний принимается только после согласования со всеми участниками процесса разработки}
	\scnaddlevel{1}
	\scntext{следовательно}{Каждый из участников проекта может модифицировать только свое личное рабочее пространство, но не всю базу знаний.}
	\scnaddlevel{-1}
	\scntext{назначение}{Целью Co4 является создание баз знаний, которые отвечают требованиям:
\begin{scnitemize}
    \item \textit{согласованные} – любое добавление в базу знаний происходит после принятия всеми пользователями, участвующими в разработке;
    \item \textit{совместно построенные} – сотрудничество пользователей в целях создания базы знаний;
    \item \textit{последовательные} – поскольку они хранятся в официальном хранилище знаний и проверяются на согласованность.
\end{scnitemize}}	
	\scnaddlevel{-1}
\scnexplanation{В онтологическом инжиниринге любая онтология рассматривается как результат согласованной деятельности группы специалистов о модели некоторой области знаний. Исходя из этого с развитием методов и средств в области инженерии знаний все большее внимание стало уделяться инструментальной поддержке процесса коллективной разработки баз знаний и онтологий.}
\scntext{назначение}{
\begin{scnitemize}
\item управление взаимодействием и коммуникацией между разработчиками;
\item контроль за доступом к текущим результатам совместного проектирования;
\item фиксация авторских прав на экспертные знания, переданные в общее пользование;
\item обнаружение ошибок проектирования и управление коррекцией ошибок;
\item конкурентное управление изменениями.
\end{scnitemize}
}
\scntext{проблемы}{
\begin{scnitemize}
\item отсутствие развитых средств автоматического редактирования и верификации баз знаний, в том числе оценки полноты и избыточности;
\item отсутствие единого механизма коллективного создания баз знаний, включающего в себя средства согласования вносимых изменений между разработчиками разного уровня ответственности, типологию ролей разработчиков;
\item недостаточный уровень расширяемости инструментов разработки.
\end{scnitemize}
}

\scnheader{технология разработки баз знаний}
\scnsuperset{Wiki-технология}
	\scnaddlevel{1}
	\scnexplanation{Wiki-технология позволяет накапливать знания, которые представляются в интероперабельной форме, обеспечивая навигацию по знаниям.}
	\scnnote{Wiki-технология предоставляет своим пользователям средства хранения, структуризации текста, гипертекста, файлов и мультимедиа. Использует в качестве инструмента платформу MediaWiki, которая позволяет осуществлять информационное взаимодействие, обеспечивать доступ к информационным ресурсам всем участникам процесса разработки системы, организовывать управление и наблюдение за разработкой.}
	\scntext{достоинства}{Среди достоинств данной технологии можно выделить:
	\begin{scnitemize}
	\item простоту Wiki-разметки;
	\item движки Wiki-сайтов, которые поддерживают онтологическое представление знаний и семантическую разметку ресурсов, что позволяет включать семантические аннотации в Wiki-разметку в виде OWL и RDF и явно разделять структурированную и неструктурированную информацию;
	\item коммуникативные возможности, которые реализуются через совместное редактирование страниц, а также посредством электронных обсуждений в Wiki или дополнительных средах, таких как чат или форум.
	\end{scnitemize}	   
Проектный характер работы, сотрудничество, формирование единого продукта совместной деятельности обеспечивают содержательное взаимодействие, обмен знаниями, оценку и постоянное совершенствование работ.}
\scntext{проблемы}{Кроме указанных достоинств Wiki как технология имеет ряд недостатков:  
	\begin{scnitemize}
	\item дублирование информации на различных страницах;
	\item невозможность структурирования знаний ввиду отсутствия иерархии гиперссылок и отсутствия унификации представления информации;
	\item отсутствие возможности автоматической верификации;
	\item технология рассчитана на работу только со структурированными естественно-языковыми текстами.
	\end{scnitemize}}
	\scnaddlevel{-1}
\scntext{проблемы}{Несмотря на достигнутые успехи в области создания баз знаний, остаются актуальными следующие проблемы:

\begin{scnitemize}
\item трудоемкость одновременного использования моделей представления различных видов знаний;
\item несовместимость уже разработанных компонентов баз знаний приводит к необходимости повторной разработки уже существующих решений;
\item изменения, вносимые в базу знаний, могут повлечь необходимость внесения существенных изменений в саму структуру базы знаний, особенно в случае динамических баз знаний;
\item несмотря на наличие достаточно развитых средств создания баз знаний, они не в полной мере обеспечивают комплексную поддержку (в том числе – информационную) коллектива разработчиков на всех стадиях проектирования базы знаний, а также не обладают достаточной гибкостью и расширяемостью;
\item существующие средства ориентированы, как правило, на какой-либо конкретный формат хранения знаний, что затрудняет перенос уже разработанной базы знаний на другую платформу интерпретации модели.
\end{scnitemize}

Основной причиной всех указанных проблем является отсутствие в рамках базы знаний интеллектуальной системы совместимости различных видов знаний, в том числе метазнаний. Совместимость различных видов знаний включает два аспекта: синтаксическую совместимость, что подразумевает унификацию формы представления знаний, и семантическую совместимость, что подразумевает однозначную и единую для всех фрагментов базы знаний трактовку используемых понятий. Кроме того, при модификации и расширении база знаний должна сохранять свою целостность и непротиворечивость.

Существующие подходы к разработке баз знаний, как правило, предполагают решение задачи обеспечения синтаксической совместимости знаний путем соединения разнородных моделей представления знаний, а также разработки новых интегрированных моделей и новых языков представления знаний. Разработка базы знаний таким способом приводит к дополнительным накладным расходам при интеграции и обработке разнородных знаний и, как следствие, к резкому увеличению трудозатрат при модификации таких баз знаний и добавлении новых видов знаний.

Попытки решения задачи обеспечения семантической совместимости раличных видов знаний в рамках разрабатываемых баз знаний связаны с построением онтологий верхнего уровня, однако, отсутствие единой формальной основы, обеспечивающей однозначную интерпретацию представляемых знаний и вводимых новых понятий, не привело к решению указанной задачи. Кроме того, существующие средства создания баз знаний предполагают, что процессы разработки и модификации базы знаний осуществляются отдельно от процесса ее использования, что приводит к дополнительному усложнению решения задачи обеспечения совместимости знаний различного вида.}

\scnheader{модель представления знаний}
\scnsubdividing{продукционная модель;логическая модель;фреймовая модель;семантическая сеть}
\scntext{примечание}{На сегодняшний день существуют десятки моделей представления знаний, однако большинство из них базируются на основных четырех моделях, представленных выше.}
\scntext{примечание}{Отдельное внимание следует уделить рассмотрению средств для представления знаний, предлагаемых в рамках направления \textit{Semantic Web}, по причине их проработанности и распространенности.}


\scnheader{продукционная модель}
\scnexplanation{\textit{продукционная модель} является системой продукций, представляющих собой конструкции типа "Если (условие), то (действие)"{}.}
\scntext{примечание}{Продукционные модели удобны для представления логических взаимосвязей между фактами. Чаще всего данные модели используются для представления знаний в экспертных системах.}


\scnheader{логическая модель}
\scnexplanation{\textit{логическая модель} основывается на классическом исчислении предикатов и его расширениях, позволяет описывать свойства предметной области в виде набора аксиом и правил вывода.}
\scntext{примечание}{\textit{Логические модели}, как и \textit{продукционные модели}, удобны для представления логических взаимосвязей между фактами. Их отличительной особенностью являются строгость и формализованность.}


\scnheader{cемантическая сеть}
\scnexplanation{\textit{семантическая сеть} -- модель представления знаний в виде графовой структуры, вершинами которой являются информационные единицы, а дуги обозначают связи между ними.}
\scntext{достоинство}{Главной особенностью семантических сетей является соединение в одном представлении синтаксического и семантического аспектов описаний знаний предметной области, что значительно снижает вычислительную сложность обработки знаний}
\scntext{примечание}{\textit{Семантическая сеть} является весьма распространенным способом представления знаний в интеллектуальных системах.}


\scnheader{фреймовая модель}
\scnexplanation{\textit{Фреймовые модели} представляют собой системы взаимосвязанных фреймов.}
	\scnaddlevel{1}
	\scntext{примечание}{Под фреймом объекта или явления понимается его минимальное описание, содержащее всю существенную информацию об этом объекте или явлении и обладающее таким свойством, что удаление из описания любой его части приводит к потере существенной информации, без которой описание объекта или явления не может быть достаточным для их идентификации. Фрейм задается именем и набором слотов, описывающих свойства объекта или явления.}
		\scnaddlevel{1}
		\scnrelfrom{автор}{Минский М.}
		\scnaddlevel{-1}	
	\scnaddlevel{-1}


\scnheader{средства Semantic Web}
\scnexplanation{\textit{cредства Semantic Web} представляют собой набор методов и технологий, предназначенных для представления информации в виде, пригодном для машинной обработки.}
\scntext{примечание}{Информация представляется в виде \textit{семантической сети}, специфицируемой посредством \textit{онтологий}. Стандартизация представления информации позволяет компьютерной системе получать различную фактографическую информацию и делать на ее основе логические заключения.}
\scnrelfromset{основные инструменты}{модель описания ресурсов RDF;языки, обеспечивающие представление RDF-данных;средства представления метаданных RDF Schema;принципы представления знаний в виде онтологий;языки описания онтологий\\
	\scnaddlevel{1}
	\scnhaselement{OWL Lite}
	\scnhaselement{OWL DL}
	\scnhaselement{OWL Full}
	\scnaddlevel{1}
	\scntext{примечание}{Онтологии, представленные с помощью OWL, включают описание классов, их свойств, обеспечивающих связи между классами, и экземпляров этих классов.}
	\scntext{примечание}{Практика позволила выявить ограниченность выразительных способностей OWL и недостатки технического характера:
	\begin{scnitemize}
	\item сложность синтаксического разбора;
	\item невозможность обнаружить опечатки в именах.
	\end{scnitemize} 
Это привело к созданию новой версии языка — \textit{OWL 2}, целью создания которого являлось устранение указанных недостатков}
	\scnaddlevel{1}
	\scntext{примечание}{В настоящий момент OWL 2 является общепризнанным стандартом для представления онтологий.}
	\scnaddlevel{-1}
	\scnaddlevel{-1}
	\scnaddlevel{-1}
;хранилища баз знаний на основе RDF\\
	\scnaddlevel{1}		
	\scnhaselement{Sesame}
	\scnhaselement{HyperGraphDB}
	\scnhaselement{Neo4j}
	\scnhaselement{Virtuoso}
	\scnhaselement{AllegroGraph}
	\scnaddlevel{1}
	\scntext{примечание}{Данные хранилища обеспечивают хранение и доступ к данным средствами языка запросов SPARQL}
	\scnaddlevel{-1}
	\scnaddlevel{-1}
;язык запросов к хранилищам RDF-данных SPARQL}
\scntext{недостатки}{К недостаткам cредств представления знаний, предлагаемых в рамках подхода Semantic Web можно отнести:
	\begin{scnitemize}
	\item OWL запрещает наличие дуг, инцидентных другим дугам, что вынуждает вводить дополнительную вершину, обозначающую связку, и далее связывать с ней все необходимые вершины соответствующими отношениями.Такой подход имеет ряд недостатков, в частности, приводит к необходимости:
	\begin{scnitemizeii}
	\item вводить новые отношения, которые связывают такую дополнительную вершину с остальными;
	\item преобразовывать указанным образом все связки рассматриваемого отношения, в противном случае связки одного и того же отношения в разных конструкциях будут представлены по-разному, что сильно затруднит обработку представленных таким образом знаний.
	\end{scnitemizeii}	
	 \item отсутствие возможности описания метасвязей;
	 \item отсутствие средств описания нечетких знаний;
	 \item отсутствие возможности описания свойств целых классов сущностей;
	 \item невозможность описания исключений из некоторых правил;
	 \item возможность задания метасвязей только для отдельных связей;
	 \item и др.
	\end{scnitemize}}
\scntext{примечание}{Таким образом, подход к созданию баз знаний на основе Semantic Web предоставляет средства формального представления знаний и доступа к ним, которые, однако, не позволяют в унифицированном виде представлять все виды знаний, необходимые для функционирования современных интеллектуальных систем.}

\scnheader{язык представления знаний}
\scntext{примечание}{Каждой \textit{модели представления знаний} соответствует некоторое множество \textit{языков представления знаний}, реализующих эти модели.}
\scnexplanation{В языках представления знаний, как правило, разделяется синтаксическая и семантическая составляющие. Синтаксис задает правила, по которым строятся конструкции данного языка, а семантика определяет правила интерпретации указанных конструкций.}
\scnhaselement{CycL}
	\scnaddlevel{1}
	\scnexplanation{\textit{CycL} -- язык представления знаний, основанный на онтологиях и используемый в рамках проекта Cyc.}
	\scnaddlevel{-1}
\scnhaselement{IDEF5}
	\scnaddlevel{1}
	\scnidtf{Integrated  Definitions  for  Ontology  Description  Capture  Method}
	\scnexplanation{\textit{IDEF5} -- стандарт онтологического исследования для наглядного представления данных, полученных в результате обработки онтологических запросов в простой естественной графической форме.}
	\scnaddlevel{-1}
\scnhaselement{Prolog}
	\scnaddlevel{1}
	\scnexplanation{\textit{Prolog} -- язык, основанный на языке предикатов математической логики дизъюнктов Хорна, представляющей собой подмножество логики предикатов первого порядка}
	\scnaddlevel{-1}
\scnhaselement{CLIPS}
	\scnaddlevel{1}
	\scnidtf{C Language Integrated Production System}
	\scnexplanation{\textit{CLIPS} -- язык представления знаний, основанный на логических правилах, использующийся одноименной программной оболочкой для создания экспертных систем}
	\scnaddlevel{-1}}


\scnheader{знание}
\scnexplanation{Под \textit{знаниями} в широком смысле понимается совокупность сведений, которые формируют целостное описание некоторого объекта, явления или проблемы}
		\scnaddlevel{1}
		\scnrelfrom{автор}{Аверкин А.Н.}
		\scnaddlevel{-1}
\scnexplanation{Знания определяется как хорошо структурированные данные, или данные о данных (т. е. метаданные).}
		\scnaddlevel{1}
		\scnrelfrom{автор}{Гаврилова Т.А.}
		\scnaddlevel{-1}
\scnnote{Понятие \textit{знаний} тесно связано с понятием \textit{предметной области}.}


\scnheader{предметная область}
\scnexplanation{Под \textit{предметной областью} в инженерии знаний понимают совокупность реальных или абстрактных объектов (сущностей), связей и отношений между этими объектами, а также процедур преобразования этих объектов для решения задач, возникающих в предметной области.}
		\scnaddlevel{1}
		\scnrelfrom{автор}{Аверкин А.Н.}
		\scnaddlevel{-1}
		
\scnheader{знание}
\scnexplanation{\textit{Знания} о некоторой \textit{предметной области} представляют собой совокупность сведений об объектах этой \textit{предметной области}, их существенных свойствах и связывающих их отношениях, процессах, протекающих в данной предметной области, а также методах анализа возникающих в ней ситуаций и способах разрешения ассоциируемых с ними проблем.}
		\scnaddlevel{1}
		\scnrelfrom{автор}{Гаврилова Т.А.}
		\scnaddlevel{-1}
\scnsubdividing{декларативное знание\\
	\scnaddlevel{1}
	\scnexplanation{Под \textit{декларативными знаниями} понимаются знания, которые записаны в памяти интеллектуальной системы так, что они непосредственно доступны для использования после обращения к соответствующему полю памяти.}
	\scntext{примечание}{В виде \textit{декларативных знаний} обычно записывается информация о свойствах предметной области, фактах, имеющих в ней место и тому подобная информация.}
	\scnaddlevel{-1}
	;процедурное знание\\
	\scnaddlevel{1}
	\scnexplanation{\textit{процедурные знания} – это знания, которые хранятся в памяти интеллектуальной системы в виде описаний процедур, с помощью которых их можно получить.}
	\scntext{примечание}{В виде \textit{процедурных знаний} обычно описываются информация о предметной области, характеризующая способы решения задач в этой области, а также различные инструкции, методики и тому подобная информация.}
	\scnaddlevel{-1}}
\scntext{примечание}{В инженерии знаний известны следующие признаки классификации знаний:
\begin{scnitemize}
  \item по глубине;
  \item по владельцу;
  \item по форме;
  \item по источнику получения;
  \item по сфере применения.
\end{scnitemize}}
\scnnote{База знаний чаще всего содержит три уровня \textit{знаний}:
\begin{scnitemize}
  \item общие, или абстрактные \textit{знания}, которые описывают закономерности, общие для большого числа \textit{предметных областей} (знания о теоретико-множественных связях, знания о терминах, знания о логических моделях предметных областей, знания о базовых математических отношениях и операциях и др.);
  \item \textit{знания} о конкретной \textit{предметной области} (domain-specific knowledge). Например, знания по геометрии, истории, медицине и др.;
  \item конкретные \textit{знания}, добавляемые в \textit{базу знаний} пользователями или программными агентами.
\end{scnitemize}}

\scnheader{структуризация базы знаний}
\scnexplanation{\textbf{\textit{cтруктуризация базы знаний}} -- выделение в базе различных связанных между собой фрагментов.}
\scnnote{\textit{Структуризация базы знаний} необходима по следующим причинам:
	\begin{scnitemize}   
	\item для повышения эффективности обработки баз знаний путем указания областей решения задач;   
	\item для выделения независимых фрагментов базы знаний с целью организации распределения работ по проектированию (когда разным исполнителям поручается разработка разных фрагментов базы знаний, имеющих достаточно четкие границы);   
	\item для дидактических целей (человеку, усваивающему некоторые знания, желательно иметь своего рода оглавление этих знаний, что позволяет планировать их усвоение и рассматривать их с различной степенью детализации), что является немаловажным фактором при работе с системами, основанных на знаниях. 
	\end{itemize}}
\scnrelfrom{цель}{декомпозиция базы знаний на множество фрагментов, связанных друг с другом тем или иным набором отношений}
\scnrelfromset{принципы, лежащие в основе}{
\scnfileitem{В основе методов структурирования информации традиционно используется \textit{иерархический подход} как методологический прием разделения формально описанной системы на уровни (блоки, или модули).}
	\scnaddlevel{1}
	\scnnote{На верхних уровнях иерархии представляются описания наименьшей степени детализации, которые отражают общие особенности предметной области (или системы), на следующих уровнях степень детализации описания увеличивается, при этом предметная область (или система) рассматривается не целиком, а отдельными частями.}  
	\scntext{следовательно}{Преимуществом данного подхода является сведение исходной задачи к подзадачам, которые должны быть решены в рамках предметной области (или системы)}
	\scnaddlevel{-1}
;\scnfileitem{В качестве одной из исторически первых моделей, применимых для структуризации знаний, также рассматривается модель клубных систем}
	\scnaddlevel{1}
	\scnnote{Данная модель позволяет выделять отдельные фрагменты заданного множества и рассматривать отношения между ними, и далее при необходимости осуществлять аналогичные действия уже с выделенными фрагментами, опускаясь таким образом на необходимый уровень детализации.}
	\scnaddlevel{1}
	\scntext{следовательно}{Такая модель позволяет, с одной стороны, специфицировать целые фрагменты исследуемого множества, рассматривая их как отдельные сущности, с другой стороны – абстрагироваться от детального рассмотрения тех фрагментов, для которых в текущий момент этого не требуется.}
	\scnaddlevel{-2}
;\scnfileitem{Подход к структуризации знаний, основанный на методологии объектно-структурного анализа}
	\scnaddlevel{1}
	\scnnote{Данный подход основывается на разбиении предметной области на восемь слоев (страт) в зависимости от вида знаний, рассматриваемого на том или ином слое}
	\scnaddlevel{-1}
;\scnfileitem{Cтратифицированная фрактальная модель (или ФС-модели)}
	\scnaddlevel{1}
	\scnnote{В основе данного подхода лежит концептуальная модель структурирования знаний, базирующаяся на представлении различных видов знаний как объектов расслоенного пространства}
	\scnnote{Графически ФС-модель представляется в виде совокупности вложенных сферических оболочек. Точка на одной из сфер, условно обозначающая информационный объект, в свою очередь может быть расслоена при необходимости более детального рассмотрения данного объекта.}
	\scnaddlevel{1}
	\scntext{следовательно}{ФС-модель определяется как совокупность непересекающихся слоев (информационных миров) и их отображений в информационном пространстве. Каждому уровню соответствует свой слой этого пространства и, следовательно, свой информационный мир.}
	\scnaddlevel{-2}}
\scnnote{Актуальной остается задача обеспечения возможности использования различных подходов к структуризации в рамках одной базы знаний одновременно.}	
\scnnote{На сегодняшний день наиболее эффективным средством формализации и структуризации различных областей знаний являются онтологии.}

\scnheader{онтология}
\scnexplanation{\textit{\textbf{онтология}} трактуется как эксплицитная спецификация концептуализации}
		\scnaddlevel{1}
		\scnrelfrom{автор}{Грубер Т.}
		\scnaddlevel{-1}
\scnnote{Применительно к интеллектуальным системам под \textit{онтологией} понимается формальная спецификация \textit{предметной области}, включающая описания классов объектов исследования и отношений, заданных на объектах исследования}
\scnnote{В качестве признаков классификации онтологий используются такие признаки, как:
	\begin{scnitemize}
	\item цель создания;
	\item степень формальности;
	\item содержимое.
	\end{scnitemize}}
\scnrelfrom{разбиение}{Классификация онтологий по цели создания}
\scnaddlevel{1}
\scneqtoset{онтология представления\\
	\scnaddlevel{1}
	\scntext{назначение}{
	\begin{scnitemize}
	\item описание области представления знаний;
	\item создание языка для спецификации онтологий более низких уровней.
	\end{scnitemize}}
	\scnaddlevel{-1}
;онтология верхнего уровня\\
	\scnaddlevel{1}
	\scntext{назначение}{Создание онтологий для общих предметных областей, свойства которых исследуются онтологиями более низкого уровня.}
	\scnnote{Онтологии верхнего уровня могут быть повторно используемы вместе с соответствующими им онтологиями более низкого уровня.}
	\scnaddlevel{-1}
;онтология предметной области\\
	\scnaddlevel{1}
	\scnnote{Область охвата данных онтологий ограничена одной предметной областью.}
	\scntext{назначение}{Обобщение понятия, использующиеся в некоторых задачах домена, абстрагируясь от самих задач.}
	\scnnote{Данный вид онтологии повторно используемы внутри одной предметной области}
	\scnaddlevel{-1}
;прикладная онтология\\
	\scnaddlevel{1}
	\scntext{назначение}{Описание концептуальной модели конкретной задачи или приложения.}
	\scnnote{Данный вид онтологии нет возможности использовать повторно.}
	\scnaddlevel{-1}}
\scnaddlevel{-1}
\scnsubdividing{неформальная онтология\\
	\scnaddlevel{1}
	\scnidtf{онтология, описываемая на естественном языке}
	\scnaddlevel{-1}
;более формализованная онтология\\
	\scnaddlevel{1}
	\scnidtf{онтология, основанная на отношениях таксономии}
	\scnaddlevel{-1}
;сильно формализованная онтология\\
	\scnaddlevel{1}
	\scnidtf{онтология, которая  задает формальную семантику понятий в разрешенных языком точных и непротиворечивых выражениях}
	\scnaddlevel{-1}}
\scnsubdividing{общая онтология\\
	\scnaddlevel{1}
	\scnidtf{онтология, описывающая сущности, события, пространство, время}
	\scnaddlevel{-1}
;онтология задач\\
	\scnaddlevel{1}
	\scnidtf{онтология, описывающая типологию классов задач и их спецификацию}
	\scnaddlevel{-1}
;предметная онтология\\
	\scnaddlevel{1}
	\scnidtf{онтология, описывающая множества предметов}
	\scnaddlevel{-1}}	
\scnsubdividing{простая онтология;многоуровневая онтология}
\scnsubdividing{легкая онтология;весомая онтология}
\scnsubdividing{статическая онтология;динамическая онтология}
\scnrelfrom{разбиение}{Классификация онтологий в зависимости от набора используемых отношений}
\scnaddlevel{1}
\scneqtoset{словарь понятий\\
	\scnaddlevel{1}
	\scnexplanation{\textit{словарь понятий} -- явно определяется смысл терминов словаря с помощью соответствующей функции интерпретации.}
	\scnaddlevel{-1}
;пассивный словарь\\
	\scnaddlevel{1}
	\scnexplanation{\textit{пассивный словарь} -- включает множество интерпретируемых понятий предметной области, множество интерпретирующих терминов и соответствующие им функции интерпретации.}
	\scnaddlevel{-1}
;таксономия понятий\\
	\scnaddlevel{1}
	\scnexplanation{\textit{таксономия понятий} -- описывает отношения типа "is a"{} между понятиями предметной области.}
	\scnaddlevel{-1}
;мерономия понятий\\
	\scnaddlevel{1}
	\scnexplanation{\textit{мерономия понятий} -- описывает отношения типа "part of"{} между понятиями предметной области.}
	\scnaddlevel{-1}
;метасистема понятий\\
	\scnaddlevel{1}
	\scnexplanation{\textit{метасистема понятий} -- описывает отношения "is a"{} и "part of"{} между понятиями предметной области.}
	\scnaddlevel{-1}
;онтология с ограничениями\\
	\scnaddlevel{1}
	\scnexplanation{\textit{онтология с ограничениями} -- описывает отношения "is a"{}, "part of"{} и другие, дополнительно уточняемые отношения между понятиями предметной области.}
	\scnaddlevel{-1}
;полная онтология\\
	\scnaddlevel{1}
	\scnexplanation{\textit{полная онтология} -- включает описания отношений "is a"{}, "part of"{} и других, дополнительно уточняемых отношений между понятиями предметной области, а также соответствующие им функции интерпретации}
	\scnaddlevel{-1}}
\scnnote{Онтологии позволяют сформировать понятийный базис рассматриваемой предметной области, что является ключевым фактором в процессе структуризации знаний. Онтологии являются основой любой базы знаний и используются для интеграции различных баз знаний и их частей.}


\scnheader{онтология верхнего уровня}
\scnexplanation{\textit{онтология верхнего уровня} ориентирована на описание фундаметальных понятий, таких, как "сущность"{}, "явление"{}, "отношение"{}, "действие"{} и др.}
\scnnote{В \textit{онтологии верхнего уровня} представлена систематизация знаний о реальном мире безотносительно к какой-либо конкретной \textit{предметной области}.}
\scntext{назначение}{Поддержка семантической совместимости онтологий предметных областей и прикладных онтологий}
	\scnaddlevel{1}
	\scnnote{Поддержка предполагает создание общей точки для формулирования определений. Термины предметно-ориентированных онтологий подчинены терминам онтологии более высокого уровня.}
	\scnaddlevel{-1}
\scnhaselement{OpenCyc}
	\scnaddlevel{1}
	\scnexplanation{\textit{OpenCyc} -- открытая для общего пользования часть коммерческого проекта Cyc, на текущий момент наиболее масштабной и детализированной онтологии в области общего знания.}
	\scnnote{Структурно \textit{база знаний OpenCyc} состоит из констант (терминов) и правил (формул), оперирующих этими константами. 
	Правила делятся на два вида: 
	\begin{scnitemize}
	\item аксиомы -- утверждения, которые были явно и вручную введены в базу знаний экспертами, а не появились там в результате работы машины вывода;
	\item выводимые утверждения.	
	\end{scnitemize} 
  Все утверждения или формулы в базе знаний OpenCyc фиксируются на языке CycL, выразительно эквивалентном исчислению предикатов первого порядка.}
	\scnaddlevel{-1}
\scnhaselement{DOLCE}
	\scnaddlevel{1}
	\scnidtf{Descriptive Ontology for Linguistic and Cognitive Engineering}
	\scnexplanation{\textit{DOLCE} -- базовая онтология проекта WonderWeb.}
	\scntext{назначение}{Обеспечение согласования между интеллектуальными агентами, использующими разную терминологию.}
	\scnnote{Онтология имеет когнитивный уклон, поскольку в основном фиксируются онтологические категории естественного языка и знания "здравого смысла"{}. Онтология не претендует на звание универсальной, стандартной или общей. 
	В основе данной онтологии лежит разделение всех сущностей на универсальные, которые могут иметь экземпляры, и индивидные (частные), которые не могут иметь экземпляры.}
	\scnaddlevel{-1}
\scnhaselement{SUMO}
	\scnaddlevel{1}
	\scnidtf{Suggested Upper Merged Ontology}
	\scnexplanation{SUMO -- онтология верхнего уровня, разработанная в рамках проекта рабочей группы IEEE Standard Upper Ontology Working Group и Teknowledge.}
	\scntext{назначение}{Содействие улучшению интероперабельности данных, извлечения и поиска информации, автоматического вывода (доказательства), обработки естественного языка.}
	\scnnote{Онтология SUMO содержит наиболее общие и самые абстрактные концепты, имеет исчерпывающую иерархию фундаментальных понятий (около 1000 понятий), а также набор аксиом (примерно 4000), определяющих эти понятия. Cоздатели SUMO предоставляют лишь информацию, которая может обрабатываться программно и включаться в качестве составной части в различные приложения и обрабатываться средствами этих приложений.}
	\scnaddlevel{-1}
\scnhaselement{WordNet}
	\scnaddlevel{1}
	\scnexplanation{WordNet -- один из наиболее полно разработанных тезаурусов общего назначения.}
	\scnnote{Центральным объектом в WordNet является синсет, множество синонимов (или синонимический ряд). WordNet содержит около 70 тыс. синсетов, организованных в иерархию по отношению "надкласс-подкласс"{}.}
	\scnreltoset{недостатки текущего состояния}{
\scnfileitem{Отсутствие отношений между частями речи}
;\scnfileitem{Различия значений в WordNet слишком тонки для компьютерных приложений}
;\scnfileitem{Нехватка отношений между синсетами, относящимися к одной и той же тематической области}}
	\scnaddlevel{-1}}
\scnaddlevel{1}
\scnnote{В целом попытки создать универсальную онтологию верхнего уровня пока не привели к ожидаемым результатам. Многие онтологии верхнего уровня содержат одни и те же понятия, однако их трактовка и принципы организации иерархии отличаются в разных онтологиях. Так, например, во всех онтологиях проводится разделение сущностей на (1) абстрактные и реально существующие, (2) на постоянные и временные сущности, (3) деление на объект и процесс.
В то же время даже на верхних уровнях наблюдаются существенные различия. В онтологии SUMO первично разделение на абстрактные и материальные сущности, а разделение на постоянные и временные – вторично. В DOLCE на верхнем уровне производится разделение на постоянные, временные, абстрактные и качественные сущности.}
\scnaddlevel{-1}
\scnnote{Онтологии верхнего уровня были призваны решить задачу обеспечения семантической совместимости представляемых знаний в базах знаний, однако отсутствие единой формальной основы, обеспечивающей однозначную интерпретацию представляемых знаний и вводимых новых понятий, не привело к решению указанной проблемы. Отсутствие удовлетворительного решения этой задачи приводит к несовместимости компонентов баз знаний, разрабатываемых для разных систем, и невозможности их повторного использования в других системах. Как следствие, имеет место многократная повторная разработка содержательно одних и тех же компонентов для разных баз знаний.}

\scnendstruct

\end{SCn}
\scparagraph{Предметная область и онтология технологий разработки решателей задач интеллектуальных компьютерных систем}
\scparagraph{Предметная область и онтология технологий разработки интерфейсов интеллектуальных компьютерных систем}

\scsubsubsection{Предметная область и онтология технологий эксплуатации интеллектуальных компьютерных систем}
\scsubsubsection{Предметная область и онтология технологий обновления интеллектуальных компьютерных систем}

\scsection{Предметная область и онтология логико-семантических моделей компьютерных систем}
\label{sem_mod_comp_sys}
\begin{SCn}

\scnsectionheader{Предметная область и онтология семантических моделей компьютерных систем}
\scnrelfromlist{подраздел}{Предметная область и онтология семантических сетей, семантических языков и семантических моделей баз знаний; Предметная область и онтология агентно-ориентированных семантических моделей решателей задач;Предметная область и онтология семантических моделей интерфейсов компьютерных систем}

\scnstartsubstruct

\scnheader{Предметная область и онтология семантических моделей компьютерных систем}
\scnsdmainclasssingle{семантическая модель компьютерной системы}
\scnsdclass{стандарт}
\scnsdrelation{***}

\scnheader{семантическая модель компьютерной системы}
\scnexplanation{Главным фактором обеспечения совместимости различных видов знаний, различных моделей решения задач и различных компьютерных систем в целом является 
\begin{scnitemize}
    \item унификация (стандартизация) представления информации в памяти компьютерных систем;
    \item унификация принципов организации обработки информации в памяти компьютерных систем.
\end{scnitemize}

Унификация представления информации, используемой в компьютерных системах, предполагает:
\begin{scnitemize}
    \item синтаксическую унификацию используемой информации – унификацию формы представления (кодирования) этой информации. При этом следует отличать:
    \begin{scnitemizeii}
    	\item кодирование информации в памяти компьютерной системы (внутреннее представление информации);
    	\item внешнее представление информации, обеспечивающее однозначность интерпретации (понимания, трактовки) этой информации разными пользователями и разными компьютерными системами;
    \end{scnitemizeii}
    \item семантическую унификацию используемой информации в основе которой лежит согласование и точная спецификация всех (!) используемых понятий (концептов) с помощью иерархической системы формальных онтологий.
\end{scnitemize}}

\scnresetlevel
\scnheader{стандарт}
\scnidtf{знания о структуре и принципах функционирования искусственных систем соответствующего класса}
\scnidtf{онтология искусственных систем некоторого класса}
\scnidtf{теория искусственных систем некоторого класса}
\scnexplanation{Важно отметить, что грамотная унификация (стандартизация) должна не ограничивать творческую свободу разработчика, а гарантировать \textbf{совместимость} его результатов с результатами других разработчиков. Подчеркнем также, что текущая версия любого \textbf{стандарта} -- это не догма, а только опора для дальнейшего его совершенствования.

Целью качественного стандарта является не только обеспечения совместимости технических решений, но и минимализация дублирования (повторения) таких решений. Один из важных критериев качества стандарта -- ничего лишнего.

Стандарты, как и другие важные для человечества знания, должны быть формализованы и должны постоянно совершенствоваться с помощью специальных интеллектуальных компьютерных систем, поддерживающих процесс эволюции стандартов путем согласования различных точек зрения.}
\scnsuperset{стандарт семантических моделей компьютерных систем}
\scnaddlevel{1}
\scntext{примечание}{Предлагаемый нами \textit{стандарт семантических моделей компьютерных систем} представлен разделом \textit{Предметная область и онтология семантических моделей ostis-систем}.}
\scnaddlevel{-1}

\scnendstruct

\end{SCn}

%\scsubsection{Предметная область и онтология семантических сетей, семантических языков и семантических моделей баз знаний}
\scsubsection{Предметная область и онтология смыслового представления информации и семантических моделей баз знаний}
%\begin{SCn}

\scnsectionheader{Предметная область и онтология семантических сетей, семантических языков и семантических моделей баз знаний}

\scnstartsubstruct

\scnheader{Предметная область семантических сетей, семантических языков и семантических моделей баз знаний}
\scnsdmainclasssingle{***}
\scnsdclass{смысловое представление информации}
\scnsdrelation{***}

\scnheader{смысловое представление информации}
\scnexplanation{Объективным ориентиром для \textbf{унификации представления информации} в памяти компьютерных систем и ключом к решению многих проблем эволюции компьютерных систем и технологий является \textbf{формализация смысла представляемой информации}.

Уточнение принципов \textbf{смыслового представления информации} основано, во-первых, на четком противопоставление \textbf{внутреннего языка компьютерной системы}, используемого для хранения информации в памяти компьютера, и \textbf{внешних языков компьютерной системы}, используемых для общения (обмена сообщениями) компьютерной системы с пользователями и другими компьютерными системами (смысловое представление используется исключительно для \textbf{внутреннего представления} информации в памяти компьютерной системы), и, во-вторых, на максимально возможном упрощении синтаксиса внутреннего языка компьютерной системы при обеспечении универсальности  путем исключения из такого внутреннего универсального языка средств, обеспечивающих коммуникационную функцию языка (т. е. обмен сообщениями).

Так, например, для внутреннего языка компьютерной системы излишними являются такие коммуникационные средства языка, как союзы, предлоги, разделители, ограничители, склонения, спряжения и другие.

Внешние языки компьютерной системы могут быть как близки ее внутреннему языку, так и весьма далеки от него (как, например, естественные языки).

\textbf{Смысл} – это \textbf{абстрактная} знаковая конструкция, принадлежащая внутреннему языку компьютерной системы, являющаяся \textbf{инвариантом} максимального класса семантически эквивалентных знаковых конструкций (текстов), принадлежащих самым разным языкам, и удовлетворяющая следующим требованиям:
\begin{scnitemize}
    \item \textbf{универсальность} - возможность представления любой информации;
    \item \textbf{отсутствие синонимии знаков} (многократного вхождения знаков с одинаковыми денотатами);
    \item \textbf{отсутствие дублирования информации} в виде семантически эквивалентных текстов (не путать с логической эквивалентностью);
    \item \textbf{отсутствие омонимичных знаков} (в том числе местоимений);
    \item \textbf{отсутствие у знаков внутренней структуры} (атомарный характер знаков);
    \item \textbf{отсутствие склонений, спряжений} (как следствие отсутствия у знаков внутренней структуры);
    \item \textbf{отсутствие фрагментов} знаковой конструкции, \textbf{не являющихся знаками} (разделителей, ограничителей, и т.д.);
    \item \textbf{выделение знаков связей}, компонентами которых могут быть любые знаки, с которыми знаки связей связываются синтаксически задаваемыми отношениями инцидентности.
\end{scnitemize}

Следствием указанных принципов смыслового представления информации в памяти компьютерной системы является то, что знаки сущностей, входящие в смысловое представление информации, \textbf{не являются именами} (терминами) и, следовательно, не привязаны ни к какому естественному языку и не зависят от субъективных терминотворческих пристрастий различных авторов. Это значит, что при коллективной разработке смыслового представления каких-либо информационных ресурсов терминологические споры исключены.

Следствием указанных принципов смыслового представления информации  является также то, что эти принципы приводят к нелинейным знаковым конструкциям (к графовым структурам), что усложняет реализацию памяти компьютерных систем, но существенно упрощает ее логическую организацию (в частности, ассоциативный доступ).

Нелинейность смыслового представления информации обусловлена тем, что: 
\begin{scnitemize}
    \item каждая описываемая сущность, т.е. сущность, имеющая соответствующий ей знак, может иметь неограниченное число связей с другими описываемыми сущностями;
    \item каждая описываемая сущность в смысловом представлении имеет единственный знак, т.к. синонимия знаков здесь запрещена;
    \item все связи между описываемыми сущностями описываются (отражаются, моделируются) связями между знаками этих описываемых сущностей.
\end{scnitemize}

Суть \textbf{универсального смыслового представления информации} можно сформулировать в виде следующих положений:
\begin{scnitemize}
    \item Смысловая знаковая конструкция трактуется как множество знаков, взаимно-однозначно обозначающих различные сущности (денотаты этих знаков) и множество связей между этими знаками;
    \item Каждая связь между знаками трактуется, с одной стороны, как множество знаков, связываемых этой связью, а, с другой стороны, как описание (отражение, модель) соответствующей связи, которая связывает денотаты указанных знаков или денотаты одних знаков непосредственно с другими знаками, или сами эти знаки. Примером первого вида связи между знаками является связь между знаками материальных сущностей, одна из которых является частью другой. Примером второго вида связи между знаками является связь между знаком множества знаков и одним из знаком, принадлежащих этому множеству, а также связь между знаком и знаком файла, являющегося электронным отражением структуры представления указанного знака во внешних знаковых конструкциях. Примерами третьего вида связи между знаками является связь между синонимичными знаками;
    \item Денотатами знаков могут быть (1) не только конкретные (константные, фиксированные), но и произвольные (переменные, нефиксированные) сущности, "пробегающие"\ различные множества знаков (возможных значений), 
    (2) не только реальные (материальные), но и абстрактные сущности (например, числа, точки различных абстрактных пространств), 
    (3) не только "внешние"\,, но и "внутренние"\ сущности, являющиеся множествами знаков, входящих в состав той же самой знаковой конструкции.
\end{scnitemize}

Ключевым свойством языка смыслового представления информации является однозначность представления информации в памяти каждой компьютерной системы, т. е. отсутствие семантически эквивалентных знаковых конструкций, принадлежащих смысловому языку и хранимых в одной смысловой памяти. При этом логическая эквивалентность таких знаковых конструкций допускается и используются, например, для компактного представления некоторых знаний, хранимых в смысловой памяти.

Тем не менее, логической эквивалентностью хранимых в памяти знаковых конструкций увлекаться не следует, т.к. \textbf{логически эквивалентные} знаковые конструкции -- это представление одного и того же знания, но с помощью \textbf{разных наборов понятий}. В отличие от этого \textbf{семантически эквивалентные} знаковые конструкции -- это представление одного и того же знания с помощью одних и тех же понятий. Очевидно, что многообразие возможных вариантов представления одних и тех же знаний в памяти компьютерной системы существенно усложняет решение задач. Поэтому, полностью исключив \textbf{семантическую эквивалентность} в смысловой памяти, необходимо стремиться к минимизации \textbf{логической эквивалентности}. Для этого необходимо грамотное построение системы используемых понятий в виде иерархической системы формальных онтологий ~\cite{Davydenko2018}.

Важным этапом создания универсального формального способа смыслового кодирования знаний был разработанный В.В. Мартыновым Универсальный Семантический Код (УСК)~\cite{Martynov}.

В качестве \textbf{стандарта} универсального смыслового представления информации \textbf{в памяти компьютерных систем} нами предложен \textbf{\textit{SC-код}} (Semantic Computer Code). В отличие от УСК В.В. Мартынова он, во-первых, носит нелинейный характер и, во-вторых, специально ориентирован на кодирование информации в памяти компьютеров нового поколения, ориентированных на разработку семантически совместимых интеллектуальных систем и названных нами \textbf{семантическими ассоциативными компьютерами}. Более подробно это понятие (\textbf{\textit{SC-код}}) рассмотрено в разделе \textit{Предметная область и онтология внутреннего языка ostis-систем -- SC-кода}. Таким образом, основным лейтмотивом предлагаемого нами смыслового представления информации является ориентация на формальную модель памяти нефоннеймановского компьютера, предназначенного для реализации интеллектуальных систем, использующих смысловое представление информации. Особенностями такого представления являются следующие:
\begin{scnitemize}
    \item ассоциативность;
    \item вся информация заключена в конфигурации связей, т.е. переработка информации сводится к реконфигурации связей (к графодинамическим процессам);
    \item прозрачная семантическая интерпретируемость и, как следствие, семантическая совместимость.
\end{scnitemize}

Неявная привязка к фоннеймановским компьютерам присутствует во всех известных моделях представления знаний. Одним из примеров такой зависимости, является, например, обязательность именования описываемых объектов.}

\scnheader{смысловое представление информации}
\scnadvantages{Почему целесообразен переход к \textit{смысловому представлению информации} в памяти \textit{компьютерной системы}: 
\begin{scnitemize}
    \item \textit{смысловое представление информации} есть \uline{объективный}, не зависящий от субъективизма и многообразия синтаксических решений, способ представления информации;
    \item в рамках смыслового представления существенно упрощается процедура интеграции знаний и погружения новых знаний в \textit{базу знаний};
    \item cущественно упрощается процедура приведения различного вида знаний к общему виду (к согласованной системе используемых понятий);
    \item cущественно упрощается процедура интеграции различных \textit{~решателей задач~} и целых \textit{компьютерных систем}; 
    \item существенно упрощается автоматизация перманентного процесса поддержки семантической совместимости (согласованности понятий и онтологий) для \textit{компьютерных систем} в условиях их постоянного совершенствования;
    \item в рамках смыслового представления информации достаточно легко осуществляется переход от информационных конструкций к информационным метаконструкциям путем введения узлов семантической сети, обозначающих информационные конструкции, а также дуг, связывающих эти узлы со всеми элементами обозначаемой им информационной конструкции;
    \item на основе \textit{стандарта смыслового представления информации} существенно упрощается интеграция различных дисциплин в области искусственного интеллекта, т.е. построение общей формальной теории интеллектуальных компьютерных систем, так как для построения общей формальной модели интеллектуальных компьютерных систем необходим базовый язык, в рамках которого можно было бы легко переходить от информации (от знаний) к \textbf{метаинформации} (к метазнаниям, к спецификациям исходных знаний).  Это подтверждается тем, что:
    \begin{scnitemizeii}
        \item подавляющее число понятий искусственного интеллекта носит метаязыковой характер;
        \item формальное смысловое уточнение почти каждого понятия искусственного интеллекта требует предшествующего формального уточнения соответствующего языка-объекта. Так, например, как можно строго говорить о языке онтологий (т.е. языке спецификации предметных областей), не уточнив язык представления самих этих предметных областей. Как можно строго говорить о языке описания способов обработки информации, не уточнив язык представления самой этой обрабатываемой информации.
    \end{scnitemizeii}
\end{scnitemize}}

\scnendstruct

\end{SCn}
\begin{SCn}

\scnsectionheader{\currentname}

\scnstartsubstruct

\scnrelto{частная предметная область и онтология}{Предметная область и онтология информационных конструкций}
\scnaddlevel{1}
\scnsourcecommentpar{Раздел 2.1.2.0}
\scnaddlevel{-1}

\scnsdmainclasssingle{смысловое представление информации}

\scnsdclass{семантическая сеть\\
	\scnaddlevel{1}
	\scnsubdividing{нерафинированная семантическая сеть;рафинированная семантическая сеть}
	\scnsubdividing{абстрактная семантическая сеть\\
		\scnaddlevel{1}
		\scnidtf{семантическая сеть, абстрагирующаяся от того, как физически представлены ее элементарные (атомарные) фрагменты, а также связи инцидентности между этими фрагментами}
		\scnaddlevel{-1}
	;графически представленная семантическая сеть\\
		\scnaddlevel{1}
		\scnidtf{нарисованная семантическая сеть}
		\scnaddlevel{-1}
	;семантическая сеть, хранимая в графодинамической памяти\\
		\scnaddlevel{1}
		\scnrelboth{следует отличать}{представление семантической сети в адресной памяти}
			\scnaddlevel{1}
			\scnnotsubset{семантическая сеть}
			\scnidtf{представление семантической сети в виде линейной информационной конструкции, которая хранится в адресной памяти и которая, строго говоря, уже не является семантической сетью, но является информационной конструкцией, семантически эквивалентной соответствующей (представляемой) семантической сети}			
			\scnaddlevel{-1}
		\scnaddlevel{-1}}
	\scnaddlevel{-1}
;язык семантических сетей\\
	\scnaddlevel{1}
	\scnidtf{язык, все тексты которого являются семантическими сетями}
	\scnsubdividing{специализированный язык семантических сетей;универсальный язык семантических сетей}
	\scnsuperset{язык рафинированных семантических сетей}
	\scnaddlevel{-1}}

\scnrelfromvector{рассматриваемые вопросы}{
\scnfileitem{Что такое семантические сети и в чем их принципиальное отличие от других вариантов представления информации}
;\scnfileitem{До какой степени можно минимизировать алфавит элементов семантических сетей}
;\scnfileitem{Можно ли все описываемые связи свести к бинарным связям и почему это целесообразно}
;\scnfileitem{Можно ли разработать \uline{универсальный} язык семантических сетей}
;\scnfileitem{До какой степени можно упростить синтаксические структуры семантических сетей до, условно говоря, рафинированного вида}
;\scnfileitem{Какими достоинствами обладает семантические сети}}

\scnrelfromlist{ссылка}{Понятие Технологии OSTIS\\
	\scnaddlevel{1}
	\scnsourcecommentpar{Сегмент 3 Раздела 0.2}
	\scntext{аннотация}{В данном сегменте \textit{Документации Технологии OSTIS} рассматриваются принципы, лежащие в основе \textit{Технологии OSTIS}, основным из которых является ориентация на использование \textit{\uline{универсального} языка рафинированных семантических сетей} в качестве внутреннего языка \textit{интеллектуальных компьютерных систем}}
	\scnaddlevel{-1}
;Описание внутреннего языка ostis-систем\\
	\scnaddlevel{1}
	\scnsourcecommentpar{Раздел 0.3.1}	
	\scntext{аннотация}{В данном разделе \textit{Документации Технологии OSTIS} рассматриваются принципы, лежащие в основе \textit{универсального языка рафинированных семантических сетей}, используемого в качестве внутреннего языка \textit{ostis-систем} -- \textit{интеллектуальных компьютерных систем} следующего поколения}
	\scnaddlevel{-1}
;Описание языка графического представления знаний ostis-систем\\
	\scnaddlevel{1}
	\scnsourcecommentpar{Раздел 0.3.3}
	\scntext{аннотация}{В данном разделе \textit{Документации Технологии OSTIS} рассматриваются принципы, лежащие в основе универсального языка графически представленных семантических сетей, используемого в \textit{пользовательском интерфейсе ostis-систем}}
	\scnaddlevel{-1}
;Бирюков Б.В. ТеориСГФ-1960ст;Гладун В.П.;Скороходько;Мартынов;Шенк;Мельчук-Жолковский Смысл-Текст;Кузнецов Игорь}
	
\scnauthorcomment{Дооформить библиографию}	

\bigskip
\scnfragmentcaption

\scnheader{знак}
\scnidtf{фрагмент информационной конструкции, обладающий свойством, \uline{обозначать} некоторую сущность (объект), которая наряду с другими сущностями описывается указанной информационной конструкцией}
\scnnote{\uline{Форма} представления знаков в известной степени условна и является результатом соглашения между носителями соответствующего языка. Знак может быть, например, представлен:
	\begin{scnitemize}
	\item  в виде фрагмента речевого сообщения (последовательностью фонем);
	\item в виде строки символов (последовательности букв) в заданном алфавите;
	\item в виде иероглифа, пиктограммы;
	\item в виде жеста.
	\end{scnitemize}}
\scniselementrole{ключевой знак}{Предметная область и онтология информационных конструкций}
	\scnaddlevel{1}
	\scnsourcecommentpar{Раздел 2.1.2.0}
	\scnhaselement{раздел Базы знаний IMS.ostis}
	\scnaddlevel{-1}
\scntext{характеристика элементов данного множества}{Знаки, используемые в различных языках, характеризуются:
	\begin{scnitemize}
	\item синтаксической структурой, по совпадению (изоморфизму) которых для разных знаокв предполагается их синонимия;
	\item денотационной семантикой, т.е. той сущностью, которая обозначается соответствующим знаком;
	\item типом (классом) обозначаемой сущности, которая может быть:
	 	\begin{scnitemizeii}
		\item материальным(физическим) элементом (точкой) абстрактного пространства, множеством, которое может быть:
			\begin{scnitemizeiii}
			\item связью;
			\item классом;
			\item структурой;
			\end{scnitemizeiii}
		\item реальной и вымышленной сущностью;
		\item константной (конкретной) и переменной (произвольной) сущностью;
		\item постоянно существующей и временно существующей сущностью (прошлой, настоящей, будущей);		
		\end{scnitemizeii}
	\item множеством тех связей, которые связывают сущность, обозначаемую данным знаком с другими сущностями, а также, если данный знак обозначает некоторую связь, множеством сущностей, которые связаны этой связью, т.е. сущностей, являющихся компонентом этой связи;
	\item текущим статусом самого знака в памяти кибернетической системы, который может быть:
		\begin{scnitemizeii}
			\item логически удаленным знаком;
			\item настоящим знаком;
			\item предлагаемым (возможно, будущим) знаком.
		\end{scnitemizeii}
	\end{scnitemize}}
	
\scnheader{денотат*}
\scnidtf{денотат заданного знака*}
\scnidtf{объект, обозначаемый заданным знаком*}
\scnidtf{денотационная семантика заданного знака*}
\scnidtf{смысл заданного знака*}
\scnidtf{Бинарное ориентированное отношение, каждая пара которого связывает:
	\begin{scnitemize}
			\item некоторый знак, представленный в той или иной форме в тексте исследуемого языка;
			\item \uline{со знаком} той сущности, которая обозначается указанным выше знаком в рамках используемого метаязыка.
		\end{scnitemize}}
\scnnote{Данное отношение используется, когда с помощью одного языка необходимо описать денотационную семантику другого языка. Фактически речь идет о переводе заданного знака, входящего в состав некоторого рассматриваемого текста, принадлежащего некоторому исследуемому языку (языку-объекту), на некоторый метаязык (в нашем случае на SC-код), денотационная семантика которого нам считается априори известной. Указанный перевод есть связь заданного знака с синонимичным ему знаком, входящим в состав текста, принадлежащего указанному метаязыку.}
\scnrelboth{обратное отношение}{внешний sc-идентификатор*}
\scnaddlevel{1}
\scnidtf{быть знаком, обозначающим заданную сущность*}
\scnaddlevel{-1}
\scniselementlist{ключевой знак}{Описание внешних идентификаторов знаков, входящих в тексты внутреннего языка ostis-систем\\
	\scnaddlevel{1}
	\scnsourcecommentpar{Раздел 0.3.2}
	\scniselement{раздел Базы знаний IMS.ostis}
	\scnaddlevel{-1}
;Предметная область и онтология знаков, входящих в тексты внутреннего языка ostis-систем\\
	\scnaddlevel{1}
	\scnsourcecommentpar{Раздел 2.1.1.2}	
	\scniselement{раздел Базы знаний IMS.ostis}
	\scnaddlevel{-1}}
	
\scnheader{информационная конструкция}
\scnidtf{информация}
\scnnote{В общем случае информационная конструкция представляет собой сложную иерархическую структуру, каждому уровню иерархии которой соответствует определенный класс информационных конструкций}
\scnsuperset{синтаксически элементарный фрагмент информационной конструкции}
	\scnaddlevel{1}
	\scnidtf{атомарный фрагмент информационной конструкции}
	\scnidtf{элемент информационной конструкции}
	\scnnote{Примерами таких элементарных фрагментов информационных конструкций являются буквы}
	\scnsuperset{буква}
	\scnaddlevel{-1}
\scnsuperset{простой знак}
	\scnaddlevel{1}
	\scnidtf{семантически элементарный фрагмент информационной конструкции}
	\scnsubset{знак}
	\scnaddlevel{-1}
\scnsuperset{выражение}
\scnaddlevel{1}
	\scnidtf{сложный (непростой) знак}
	\scnidtf{знак, являющийся одновременно некоторым знанием обозначаемой сущности (спецификацией этой сущности)}
	\scnidtf{знак, построенный как выражение вида "тот, который..."{}}
	\scnidtf{знак, в состав которого входят другие знаки}
	\scnsubset{знак}
	\scnaddlevel{-1}
\scnsuperset{простой текст}
	\scnaddlevel{1}
	\scnidtf{минимальная синтаксически целостная и корректная (правильная) информационная конструкция, включающая в себя:
	\begin{scnitemize}
	\item знак некоторой описываемой связи;
	\item минимальную спецификацию указанного знака связи (указание отношения, которому это связь принадлежит);
	\item указание \uline{всех} компонентов описываемой связи (знаков всех сущностей, связываемых этой связью, и/или всех знаков, связываемых этой связью -- описываемая связь может связывать не только "внешние"{} описываемые сущности, но и сами знаки);
	\item если описываемая связь не является бинарной, то связи с её компонентами могут потребовать явного представления знаков этих связей с дополнительным указанием роли этих компонентов.
	\end{scnitemize}}
	\scnsubset{текст}
	\scnaddlevel{-1}
\scnsuperset{сложный текст}
	\scnaddlevel{1}
	\scnidtf{информационная конструкция, являющаяся результатом соединения нескольких простых текстов}
	\scnsubset{текст}
	\scnaddlevel{-1}
\scnsuperset{простое знание}
	\scnaddlevel{1}
	\scnidtf{минимальная семантические целостная информационная конструкция}
	\scnidtf{знание, в состав которого не входят другие знания}
	\scnsubset{знание}
	\scnaddlevel{-1}	
\scnsuperset{сложное знание}
	\scnaddlevel{1}
	\scnidtf{информационная конструкция, являющаяся результатом соединения нескольких простых знаний}
	\scnidtf{знание, в состав которого не входят другие знания}
	\scnsubset{знание}
	\scnaddlevel{-1}	
\scniselementrole{ключевой знак}{Предметная область и онтология информационных конструкций}
\scnaddlevel{1}
	\scnsourcecommentpar{Раздел 2.1.2.0}
\scnaddlevel{-1}
\end{SCn}

\begin{SCn}
	
\scnheader{Смысловое представление информации}
\scnidtf{смысловая форма представления информации}
\scnidtf{смысловое представление информационной конструкции}
\scnidtf{знаковая конструкция(текст), представленная в смысловой форме}
\scnidtf{смысловое представление информационной конструкции}
\scnidtftext{часто используемый sc-идентификатор}{смысл}
\scnidtf{смысловое представление}
\scnidtf{семантическое представление информации}


\scnrelfrom{основной принцип}{Как можно меньше лишнего, не имеющего отношения к смыслу представляемой информации.}
\scnidtf{такое представление информационной конструкции, которое существенно прощает соответствие между структурой самой этой информационной конструкции и описываемой(отображаемой) ею конфигурацией связей между рассматриваемыми(исследуемыми) сущностями}
\scnidtf{смысловое представление знаковой конструкции}
\scnidtf{абстрактная знаковая конструкция, являющаяся \uline{инвариантом} соответствующего максимального класса семантически эквивалентных знаковых конструкций}
\scnidtf{смысл информационной конструкции}
\scnidtf{денотационная семантика информационной конструкции}
\scnidtf{смысловое представление информационной конструкции}


\scnnote{Суть(смысл,денотационная семантика) любой информационной конструкции(информационной модели) сводится к описанию системы(конфигурации) связей между списываемыми(рассматриваемыми) сущностями. Важно, чтобы эта суть не была \uline{закамуфлирована} различными "синтаксическими" деталями, не имеющими никакого отношения к указанному смыслу(синтаксическая структура знаков, многократное повторение одного и того же знака, синонимия, омонимия, местоимения, предлоги, знаки препинания, разделители, ограничители, падежи и т.п.) а обусловленными \uline{формой} представления информационных конструкций, например, их линейностью.}


\scnexplanation{Смысловое представление любой информации в конечном счете сводится:
	\scnaddlevel{1}
	\begin{scnitemize}
		\item{к перечню знаков конкретных описываемых сущностей - как первичных сущностей, так и вторичных сущностей, которые сами являются информационными конструкциями(фрагментами данной конструкции)};
		\item{к явному описанию связи между знаками вторичных сущностей и самими этими сущностями (т.е. фрагментами информационной конструкции)};
		\item{к описанию других связей между описываемыми сущностями}
	\end{scnitemize}
}
\scnaddlevel{-1}


\scnexplanation{Формализация смысла представляемой информации, т.е. строгое уточнение того, что такое \textit{смысловое представление информации}, является объективной основой для \uline{унификации} представления информации в \textit{памяти компьютерных систем} и \uline{ключом} к решению многих проблем семантической совместимости и эволюции компьютерных систем и технологий.}

\scnnote{Грамотная унификация(стандартизация) \textit{смыслового представления информации} не должна привести к ограничению творческой свободы авторов различного вида публикуемых научно-технических знаний (и, в том числе, разработчиков \textit{баз знаний}), не должна гарантировать \textit{семантическую совместимость} различных \textit{знаний}, представленных различными авторами(разумеется, при условии соблюдения соответствующих правил построения этих \textit{знаний}). При этом любые \textit{стандарты} (в том числе и принятые стандарты \textit{смыслового представления информации}) должны постоянно эволюционировать. Текущая версия любого стандарта должна быть не догмой, а точкой опоры для дальнейшего совершенствования этого стандарта.}

\scnsuperset{УСК}
\scnaddlevel{1}
\scnidtf{Универсальный Семантический Код}
\scnrelfrom{автор}{Мартынов В. В.}
\scnnote{Разработанный Мартыновым В. В. Универсальный Семантический Код стал важнейшим этапом создания универсальных формальных средств смыслового представления знаний. Основная методологическая идея \textit{Мартынова В. В.}, касающаяся построения \textit{языка смыслового представления знаний}, заключается в том, чтобы выделить смысловые "кирпичики", имеющие достаточно общий характер, а многообразие конкретных смыслов конструировать комбинаторно за счёт различных комбинаций(конфигураций) из этих "кирпичей". Это можно назвать принципом минимизации типов атомарных смысловых фрагментов}

\scnrelto{ключевой знак}{Книга УСК}


\scnheader{смысловое представление информации*}
\scnexplanation{\textit{Бинарное ориентированное отношение}, каждая \textit{пара} которого связывает некоторую \textit{информационную конструкцию} со смысловым представлением этой \textit{информационной конструкции*}}

\scnsubset{формализация*}


\end{SCn}

\begin{SCn}
\scnheader{формализация*}\\
\scniselement{ключевой знак*:}{Начало Предметной области и онтологии кибернетических систем}\\
\scniselement{начало раздела Базы знаний IMS ostis}
\scniselement{бинарное ориентированное отношение}
\scnidtf{формализация информации*}
\scnidtf{пара, связывающая менее формализованное и более формализованное представление некоторой информации*}
\scnidtf{формализация информационной модели некоторой описываемой (моделируемой) системы взаимосвязанных сущностей*}
\scnidtf{Бинарное ориентированное отношение, каждая \textit{пара} которого, связывает два \textit{семантически эквивалентных} знания, второе из которых является более точным (более точно сформированным) знанием по сравнению с первым \textit{знанием}.}
\scnexplanation{Повышение точности (строгости) формулировки знания - минимизация (а в идеале -- исключение) \uline{неоднозначной} семантической интерпретации этой формулировки, т.е. несоответствия того, что хотел "сказать" автор формулировки, и того, как его поняли. Формализация знаний предполагает (1) точное (строгое) описание \textit{синтаксиса и денотационной семантики} того \textit{языка}, на котором формулируются \textit{знания} и (2) максимально возможное \uline{упрощение} синтаксических и семантических принципов, лежащих в основе указанного \textit{языка}. Очевидно, что \textit{естественные языки} указанным требованиям не удовлетворяют и, следовательно, не могут быть основой для точной формулировки \textit{научно-технических знаний} и, соответственно, для представления этих \textit{знаний} в \textit{памяти интеллектуальных компьютерных систем}. Очевидно также, что разработка \textit{\uline{универсального} языка} формального представления научно-технических знаний является \uline{основой} для глубокой конвергенции различных научно-технических дисциплин, для расширения областей применения современной математики и даже для появления новых разделов математики, которые, например, изучают общие свойства \textit{универсального смыслового пространства} и, в частности, свойство семантического расстояния(семантической близости) как между различными \textit{знаками}, так и между различными \textit{знаковыми конструкциями}(конфигурациями знаков).}
\scnaddlevel{1}
\scnnote{Слово "математика" означает "точное знание".}
\scnaddlevel{1}
\scnrelto{цитата}{\textit{Арнольд В.И. Что TM--2012кн-c.4}}
\scnaddlevel{-2}
\scnnote{формализация информационной модели есть не что иное как "движение" в сторону семантического (смыслового) представления этой модель, т.е. переход к такому представлению этой модели, в котором мы избавляемся от всего, не имеющего отношения к сути моделируемой системы и касающегося только способа построения этой модели (т.е. её синтаксической структуры). }
\scnnote{Нет проблемы записать любое \textit{знание} в компьютерную \textit{память}. Для этого надо придумать соответствующий формат их кодирования. Но есть проблема представить это \textit{знание} так, чтобы с ним было легко работать, чтобы с использованием этого \textit{знания} можно было достаточно удобно (без лишних накладных расходов, обусловленных выбранным способом представления) решать самые различные информационные \textit{задачи} (задачи интеграции знаний, информационного поиска по базе знаний, верификации и оптимизации баз знаний, логического вывода, поиска способов решения задач, хранимых в базе знаний и т. д.).
	Какими характеристиками должно обладать удобное представление знаний, удовлетворяющее указанным требованиям. Очевидно, что такое представление есть не что иное, как формальная(математическая) модель, семантически эквивалентная этим знаниям. Т.е. удобно представить знание - это фактически построить соответствующую этому знанию \textit{математическую модель}.
	Для интеллектуальных компьютерных систем важно не просто приобрести знания, но и представить их в такой форме, которая была бы удобна не только для человека(пользователя и разработчика), но и для различных компьютерных систем, т.е. не требовала бы переоформление (перезаписи) этих знаний для различных компьютерных систем. Очевидно, что такая форма записи(представления) знаний должна быть абсолютно не зависящий от различных компьютерных платформ.
	Это и есть главная цель формализации знаний, обеспечивающей эффективную автоматизацию обработки этих знаний.}
\scnheader{формальное представление информации}\\
\scnsubset{информация}
\scnaddlevel{1}
\scnidtf{информационная конструкция}
\scnaddlevel{-1}
\scntext{вопрос}{Почему разработка и использование формальных моделей (математических моделей) представления \textit{информации} является важнейшим этапом развития любой научной и научно-технической дисциплины.
}\scnaddlevel{1}
\scnrelfromset{ответ}{\scnfileitem{Формализация любой \textit{предметной области} даёт возможность более конструктивно накапливать, интегрировать, понимать и систематизировать новые \textit{знания} об этой \textit{предметной области}};
\scnfileitem{Формализация \textit{предметной области} обеспечивает более строгую верификацию, обоснование (аргументацию, доказательство) и согласование различных точек зрения};
\scnfileitem{Формализация \textit{предметной области} создает условия для разработки строгих и легко воспроизводимых (реализуемых) \textit{методов} решения различных \textit{классов задач}}}
\scnaddlevel{1}
\scniselement{конъюнкция*}
\scnrelto{достоинства}{формальное представление информации}
\scnaddlevel{-2}
\scnidtf{формальное (формализованное) представление информационной конструкции}
\scnsubset{смысловое представления информации}
\scnnote{Высшим уровнем качества \textit{формального представления информации} является смысловое представление этой информации}
\scnidtf{формальная модель системы описываемых взаимосвязанных сущностей}
\scnidtf{математическая модель системы описываемых взаимосвязанных сущностей}
\scnidtf{формула}
\scnnote{Сам термин \textit{"формальное представление информации"} свидетельствует о том, что при таком представлении \textit{информации} сама \uline{форма} представляемой информационной конструкции (т.е. синтаксическая структура этой конструкции) имеет очевидную аналогию с описываемой конфигурацией связей между соответствующими соответствующими описываемыми \textit{сущностями}.
	В предельном "идеальном" случае указанная аналогия между формой и смыслом информационной конструкции должна быть изоморфизмом.}
\scnnote{Формализация формализации рознь и, соответственно, степень приближения формы представления информации к "идеальному" смысловому представлению может быть различной. Разработка такого "идеального" \textit{языка смыслового представления информации} должна руководствоваться следующими основными критериями:
	\begin{scnitemize}		
		\item максимально возможное упрощения синтаксиса (как можно меньше синтаксических излишеств и синтаксического разнообразия).
		\item обеспечение универсальности языка.
	\end{scnitemize}
\bigskip
Подчеркнем, что обеспечение универсальности\textit{ языка смыслового представления информации} является весьма нетривиальной задачей, т.к. сложно одновременно достигнуть две противоречащие друг другу цели- обеспечить простоту синтаксиса языка и его неограниченную семантическую мощность. Косвенным подтверждением этого является большое количество созданных человечеством специализированных \textit{формальных языков, языков смыслового представления информации} и даже \textit{языков семантических сетей}, что свидетельствует о востребованности \textit{смыслового представления информации}.
}
\scnsubdividing{формальное представление информации, не являющееся смысловыми
;смысловое представление информации, не являющееся семантической сетью
;нерафинированная семантическая сеть
\scnaddlevel{1}
\scnidtf{смысловые представления информации 2-го уровня}
\scnaddlevel{-1}
;рафинированная семантическая сеть
\scnaddlevel{1}
\scnidtf{смысловое представление информации 3-го уровня}
\scnaddlevel{-1}}




\end{SCn}
\input{Contents/chapter1/1.3.1_part_4.tex}
\input{Contents/chapter1/1.3.1_part_10.tex}

\scsubsection{Предметная область и онтология агентно-ориентированных семантических моделей решателей задач}
\begin{SCn}

\scnsectionheader{Предметная область и онтология агентно-ориентированных семантических моделей решателей задач}

\scnstartsubstruct

\scnheader{Предметная область агентно-ориентированных семантических моделей решателей задач}
\scnsdmainclasssingle{***}
\scnsdclass{***}
\scnsdrelation{***}

\scnheader{совместимость моделей решения задач*}
\scnexplanation{Предлагаемый нами подход к существенному повышению уровня совместимости (интегрируемости) различных \textbf{моделей решения задач} заключается в следующем:

\begin{scnitemize}
    \item Вся информация, хранимая в памяти каждого \textit{\textbf{решателя задач}} (как собственно обрабатываемая информация, так и хранимые в памяти интерпретируемые навыки, например, различного вида программы), представляется в форме смыслового представления этой информации; 
    \item Собственно решение каждой задачи осуществляется коллективом агентов, работающих над общей для них смысловой (семантической) памятью и выполняющих интерпретацию хранимых в этой же памяти навыков;
    \item Интеграция двух разных моделей решения задач сводится:
    \begin{scnitemizeii}
        \item к объединению памяти первой модели с памятью второй модели;
        \item к интеграции всей информации, хранимой в памяти первой модели, с информацией, хранимой в памяти второй модели (эта интеграция осуществляется путем взаимного погружения соответствующих информационных конструкций друг в друга, т.е. путем склеивания синонимов, а также путем выравнивания используемых ими понятий);
        \item к объединению множества агентов, входящих в состав первой модели, со множеством агентов, входящих во вторую модель решения задач.
    \end{scnitemizeii}
\end{scnitemize}

Таким образом, унификация моделей решения задач путем приведения этих моделей к виду семантических моделей (т. е. моделей обработки информации, представленной в смысловой форме) повышает уровень совместимости этих моделей благодаря наличию прозрачной процедуры интеграции информации, представленной в смысловой форме, и тривиальной процедуры объединения множеств \textit{агентов}. Простота процедуры объединения множеств \textit{агентов}, соответствующих разным моделям решения задач, обусловлена тем, что непосредственного взаимодействия между этими агентами нет, а инициирование каждого из них определяется им самим, а также \uline{текущим состоянием} хранимой в памяти информации.

Таким образом, в качестве основы унификации принципов обработки информации в компьютерных системах предлагается использовать \textbf{многоагентный подход}. Ориентация на многоагентный подход обусловлена следующими основными преимуществами такого подхода \cite{Wooldridge2009}:

\begin{scnitemize}
    \item автономность (независимость) агентов, что позволяет локализовать изменения, вносимые в систему при ее эволюции, и снизить соответствующие трудозатраты;
    \item децентрализация обработки, т.е. отсутствие единого контролирующего центра, что также позволяет локализовать вносимые в систему изменения.
\end{scnitemize}

Но современные принципы построения \textit{\textbf{многоагентных систем}} при их применении для многоагентной обработки \textit{баз знаний} имеют ряд недостатков:
\begin{scnitemize}
    \item знания агента представляются при помощи узкоспециализированных языков, зачастую не предназначенных для представления знаний в широком смысле и онтологий в частности;
    \item большинство современных многоагентных систем предполагает, что взаимодействие агентов осуществляется путем обмена сообщениями непосредственно от агента к агенту;
    \item логический уровень взаимодействия агентов жестко привязан к физическому уровню реализации многоагентной системы;
    \item среда, с которой взаимодействуют агенты, уточняется отдельно разработчиком для каждой многоагентной системы, что приводит к существенным накладным расходам и несовместимости таких многоагентных систем.
\end{scnitemize}

Перечисленные недостатки предлагается устранять за счет использования следующих принципов:
\begin{scnitemize}
    \item коммуникацию агентов предлагается осуществлять путем спецификации (в общей памяти компьютерной системы) действий (процессов), выполняемых агентами и направленных на решение задач;
    \item в роли внешней среды для агентов выступает та же общая память;
    \item спецификация каждого агента описывается средствами языка представления знаний в той же памяти;
    \item синхронизацию деятельности агентов предлагается осуществлять на уровне выполняемых ими процессов;
    \item каждый информационный процесс в любой момент времени имеет ассоциативный доступ к необходимым фрагментам базы знаний, хранящейся в общей памяти.
\end{scnitemize}
}
\scnnote{\textbf{\textit{совместимость моделей решения задач}} -- это возможность использования разными моделями решения задач одних и тех же информационных ресурсов.}

\scnendstruct

\end{SCn}

\scsubsection{Предметная область и онтология семантических моделей интерфейсов компьютерных систем}

\scsectionfinish{sem_mod_comp_sys}

\begin{SCn}

\scnsectionheader{\currentname}

\scnstartsubstruct

\scnheader{семантическая совместимость компьютерных систем}
\scnexplanation{Уровень совместимости \textit{компьютерных систем} определяется трудоемкостью реализации процедур интеграции (объединения, соединения знаний этих систем), а также трудоемкостью и глубиной интеграции входящих в эти системы \textit{решателей задач} (навыков и интерпретаторов этих навыков). Подчеркнем при этом, что интеграция интеграции рознь -- от эклектики до гибридности и синергетичности дистанция огромного размера.

Совместимые \textit{компьютерные системы} -- это компьютерные системы, для которых существует автоматически выполняемая процедура их интеграции, превращающая эти системы в единую \textbf{\textit{гибридную систему}}, в рамках которой каждая исходная компьютерная система в процессе своего функционирования может свободно использовать любые необходимые информационные ресурсы (знания и навыки), входящие в состав другой исходной компьютерной системы.

Целостную \textit{компьютерную систему} можно рассматривать как решатель задач, интегрировавший несколько моделей решения задач и обладающий средствами взаимодействия с внешней для себя средой (с другими компьютерными системами, с пользователями).

Таким образом, для того, чтобы повысить уровень совместимости \textit{компьютерных систем}, необходимо преобразовать их к виду \textit{многоагентных систем}, работающих над общей семантической памятью, в которой информация представлена семантическими сетями. Такие \textit{компьютерные системы} не всегда целесообразно непосредственно объединять (интегрировать) в более крупные \textit{компьютерные системы}. Иногда целесообразнее их объединять в \textit{коллективы взаимодействующих компьютерных систем}. Но при создании таких коллективов компьютерных систем унификация и совместимость таких систем также очень важны, т.к. существенно упрощают обеспечение высокого уровня их взаимопонимания. Так, например, противоречия между компьютерными системами, входящими в коллектив, можно обнаруживать путем анализа на непротиворечивость \textbf{\textit{виртуальной объединенной базы знаний}} этого коллектива. Более того, непротиворечивость указанной виртуальной базы знаний можно считать одним из критериев семантической совместимости систем, входящих в соответствующий коллектив.}

\scnheader{семантическая компьютерная система}
\scnexplanation{Предлагаемое нами устранение проблем современных информационных технологий путем перехода к \textit{смысловому представлению информации} в памяти компьютерных систем фактически преобразует современные компьютерные системы (в том числе и современные интеллектуальные системы) в \textit{\textbf{семантические компьютерные системы}}, которые, следовательно, являются не альтернативной ветвью развития \textit{компьютерных систем}, а естественным этапом их эволюции, направленным на обеспечение высокого уровня их обучаемости и, в первую очередь, \textbf{совместимости}.

Архитектура \textit{семантических компьютерных систем} (см. \textit{Рис. Архитектура семантических компьютерных систем}) практически совпадает с архитектурой интеллектуальных компьютерных систем, основанных на знаниях. Отличие здесь заключаются в том, что в \textit{семантических компьютерных системах}:
\begin{scnitemize}
    \item база знаний имеет смысловое представление;
    \item интерпретатор знаний и навыков представляет собой коллектив \textit{агентов}, осуществляющих обработку \textit{базы знаний}.
\end{scnitemize}

Как следствие этого, \textit{семантические компьютерные системы} обладают высоким уровнем обучаемости, т.е. способностью быстро приобретать новые и совершенствовать уже приобретенные знания и навыки и при этом не иметь никаких ограничений на вид приобретаемых и совершенствуемых ею знаний и навыков, а также на их совместное использование.

Более того, при согласовании соответствующих стандартов, а также при перманентном совершенствовании этих стандартов и при грамотной их поддержке в условиях интенсивной эволюции как самих стандартов, так и \textit{семантических компьютерных систем} (речь идет о постоянной поддержке соответствия между текущим состоянием компьютерных систем и текущим состоянием эволюционируемых стандартов), \textit{семантические компьютерные системы} и их компоненты обладают весьма высокой степенью совместимости.

Это, в свою очередь, практически исключает дублирование инженерных решений и дает возможность существенно ускорить разработку \textit{семантических компьютерных систем} с помощью постоянно расширяемой библиотеки многократно используемых и совместимых между собой компонентов. 

Основным лейтмотивом перехода от современных компьютерных систем (в том числе интеллектуальных) к семантическим компьютерным системам, т.е. компьютерным системам, основанным на смысловом представлении всей информации, хранимой в ее памяти, является создание \textit{\textbf{общей семантической теории компьютерных систем}}, включающей в себя:
\begin{scnitemize}
    \item cемантическую теорию знаний и баз знаний;
    \item семантическую теорию задач и моделей их решения;
    \item cемантическую теорию взаимодействия информационных процессов;
    \item cемантическую теорию пользовательских и, в том числе, естественно языковых интерфейсов;
    \item cемантическую теорию невербальных сенсорно-эффекторных интерфейсов;
    \item теорию универсальных интерпретаторов семантических моделей компьютерных систем и, в частности, теорию семантических компьютеров.
\end{scnitemize}

Эпицентром следующего этапа развития информационных технологий является решение проблемы обеспечения \textbf{семантической совместимости} \textit{компьютерных систем} и их компонентов. Для решения этой проблемы необходим
\begin{scnitemize}
    \item переход от традиционных компьютерных систем и от современных интеллектуальных систем к \textit{семантическим компьютерным системам};
    \item разработка \textit{стандарта семантических компьютерных систем}.
\end{scnitemize}    
    
Очевидно, что \textit{семантические компьютерные системы} являются компьютерными системами нового поколения, устраняющие многие недостатки современных компьютерных систем. Но для массовой разработки таких систем необходима соответствующая технология, которая должна включать в себя  

\begin{scnitemize}        
    \item теорию \textit{семантических компьютерных систем} и комплекс всех стандартов, обеспечивающих совместимость разрабатываемых систем;
    \item методы и средства проектирования \textit{семантических компьютерных систем};
    \item методы и средства перманентного совершенствования самой технологии.
\end{scnitemize}
}

\scnheader{Рис. Архитектура семантических компьютерных систем}
\scneqfile{\\\includegraphics[width=0.5\linewidth]{figures/arch.pdf}\\}

\scnendstruct

\end{SCn}


\scsectionfamily{Часть 1 Стандарта OSTIS. Обоснование интеллектуальных компьютерных систем нового поколения и соответствующей им комплексной технологии}
\label{part_intro}
\scntext{аннотация}{***}

\scsection{Глава 1. Кибернетические системы, их деятельность и эволюция}
\label{sd_sys_inform}
\begin{SCn}
	\begin{scnrelfromset}{содержание}
		\scnitem{\S 1.1. Предметная область и онтология кибернетических систем}
		\scnitem{\S 1.2. Предметная область и онтология деятельности кибернетических систем}
		\scnitem{\S 1.3. Предметная область и онтология эволюции кибернетических систем}
	\end{scnrelfromset}
\end{SCn}

\scsubsection{\S 1.1. Предметная область и онтология кибернетических систем}
\label{cyb_systems}
\begin{SCn}
	\scnsectionheader{Предметная область и онтология кибернетических систем}

	\begin{scnsubstruct}

		\scnheader{Предметная область кибернетических систем}
		\scniselement{предметная область}
		\begin{scnhaselementrolelist}{класс объектов исследования}
			\scnitem{кибернетическая система}
		\end{scnhaselementrolelist}

		\begin{scnhaselementrolelist}{класс объектов исследования}
			\scnitem{искусственная сущность}
			\scnitem{компьютерная система}
			\scnitem{простая кибернетическая система}
			\scnitem{индивидуальная кибернетическая система}
			\scnitem{кибернетическая система, встроенная в индивидуальную
				кибернетическую систему}
			\scnitem{многоагентная система}
			\scnitem{одноуровневый коллектив кибернетических систем}
			\scnitem{коллектив индивидуальных кибернетических систем}
			\scnitem{иерархический коллектив индивидуальных кибернетических систем}
			\scnitem{информация, хранимая в памяти кибернетической системы}
			\scnitem{абстрактная память кибернетической системы}
			\scnitem{решатель задач кибернетической системы}
			\scnitem{действие кибернетической системы}
			\scnitem{задача}
			\scnitem{задача, решаемая кибернетической системой}
			\scnitem{навык}
			\scnitem{интерфейс кибернетической системы}
			\scnitem{физическая оболочка кибернетической системы}
			\scnitem{память кибернетической системы}
			\scnitem{процессор кибернетической системы}
			\scnitem{компьютер}
			\scnitem{качество кибернетической системы}
			\scnitem{гибридная кибернетическая система}
			\scnitem{приспособленность кибернетической системы к её
				совершенствованию}
			\scnitem{гибкость кибернетической системы}
			\scnitem{производительность кибернетической системы}
			\scnitem{надежность кибернетической системы}
			\scnitem{качество физической оболочки кибернетической системы}
			\scnitem{качество памяти кибернетической системы}
			\scnitem{интеллект}
			\scnitem{образованность кибернетической системы}
			\scnitem{интеллектуальная система}
			\scnitem{кибернетическая система, основанная на знаниях}
			\scnitem{кибернетическая система, управляемая знаниями}
			\scnitem{целенаправленная кибернетическая система}
			\scnitem{обучаемая кибернетическая система}
			\scnitem{социально ориентированная кибернетическая система}
			\scnitem{интеллектуальная компьютерная система}
			\scnitem{информация}
			\scnitem{сенсорная информация}
			\scnitem{качество решателя задач кибернетической системы}
			\scnitem{обучаемость кибернетической системы}
			\scnitem{стратифицированность кибернетической системы}
			\scnitem{рефлексивность кибернетической системы}
			\scnitem{синергетическая кибернетическая система}
			\scnitem{интероперабельность кибернетической системы}
		\end{scnhaselementrolelist}

		\begin{scnhaselementrolelist}{исследуемое отношение}
			\scnitem{информация, хранимая в памяти кибернетической системы*}
			\scnitem{задача, решаемая кибернетической системой*}
			\scnitem{внешняя среда кибернетической системы*}
			\scnitem{среда кибернетической системы*}
			\scnitem{агент*}
		\end{scnhaselementrolelist}

		\scnidtf{Иерархическая система свойств (характеристик) кибернетических
			систем, определяющих общий (интегральный) уровень их качества}
		\scnidtf{Эволюционный подход к определению качества и, в частности,
			уровня интеллекта кибернетической системы}
		\scntext{аннотация}{Рассмотрена иерархическая система свойств (в т.ч.
			способностей) кибернетических систем, определяющих их качество и позволяющих
			сформулировать требования, которым должна удовлетворять высокоинтеллектуальная
			система (кибернетическая система с сильным интеллектом).Уровень качества
			кибернетических систем определяется достаточно большим набором свойств
			(параметров, характеристик) кибернетических систем, каждое из которых
			определяет уровень качества кибернетической системы в соответствующем аспекте
			(ракурсе), указывая (задавая) уровень развития конкретных  способностей и
			возможностей кибернетической системы. При этом важно подчеркнуть следующее:
			\begin{scnitemize}
				\item существенное значение имеет не столько сам набор свойств, а
				иерархия этих свойств, позволяющая уточнять (детализировать) направления
				проявления (реализации) каждого свойства
				\item существенное значение также имеет \uline{баланс} уровней развития
				различных свойств --- вклад разных свойств, обеспечивающих (определяющих)
				значение одного и того же свойства более высокого уровня иерархии, а значение
				этого свойства более высокого уровня может быть разным. Из этого следует, что
				не всегда следует акцентировать внимание на развитие некоторых свойств
				(характеристик). Нужен целостный, коллективный подход
				\item рассмотренная иерархия свойств кибернетических систем является
				общей как для естественных, так и для искусственных кибернетических систем
				\item приведенная иерархическая детализация свойств кибернетических
				систем (с помощью отношения \scnqqi{\textit{частное свойство*}} и отношения
				\textit{свойство-предпосылка*}), определяющих качество таких систем, (1) дает
				возможность четко определить направления совершенствования (развития)
				кибернетических систем и (2) дает ориентир (систему критериев) для обоснования
				конкретных предложений по совершенствованию компьютерных систем, а также для
				сравнения различных альтернативных предположений
				\item особое значение для развития кибернетических систем имеют такие
				их свойства, как стратифицированность, рефлексивность и социализация
				\item важное значение имеет не только совершенствование кибернетических
				систем в соответствии с иерархической системой их свойств, но и
				совершенствование (в том числе, детализация) самой этой иерархической системы
				свойств.
			\end{scnitemize}}

		\scntext{предисловие}{Свойства (способности), которым должны
			удовлетворять \textit{интеллектуальные системы}, рассматриваются в целом ряде
			публикаций. Тем не менее, для \uline{практической} реализации
			\textit{компьютерных систем}, обладающих указанными свойствами (способностями),
			т.е. \textit{интеллектуальных компьютерных систем}, необходимо детализировать
			(уточнить) эти \textit{свойства}, пытаясь свести их к более конструктивным,
			прозрачным и понятным для реализации свойствам.}

		\begin{scnrelfromset}{рассматриваемые вопросы}
			\scnfileitem{По каким свойствам (параметрам, характеристикам,
					способностям) кибернетических систем можно оценивать уровень их качества.}
			\scnfileitem{Можно ли считать уровень развития какого-либо
					свойства (способности) кибернетической системы, т.е. значение какого-либо ее
					параметра (характеристики) оценкой уровня качества кибернетической системы по
					соответствующему аспекту.}
			\scnfileitem{Может ли какое-либо свойство кибернетических
					систем определять (влиять на) значение сразу нескольких свойств более высокого
					уровня иерархии.}
			\scnfileitem{Какими отношениями свойства кибернетических
					систем связаны со свойствами более низкого и, соответственно, более высокого
					уровня иерархии.}
			\scnfileitem{Зачем нужна такая иерархия свойств, определяющих
					качество кибернетических систем и позволяющих детализировать (уточнять) то,
					какими свойствами определяется уровень (степень) развития каждого свойства
					(значение каждого свойства) за исключением свойств, которые условно можно
					считать элементарными, не требующими детализации (по крайнем мере, пока).}
			\scnfileitem{Может ли иерархия свойств, определяющих качество
					кибернетических систем, быть критерием оценки и выбора того или иного подхода к
					построению интеллектуальных компьютерным систем.}
			\scnfileitem{Какими свойствами (способностями) должна обладать
					кибернетическая система, имеющая высокий уровень интеллекта.}
			\scnfileitem{Какими свойствами определяется уровень интеллекта
					многоагентной кибернетической системы.}
			\scnfileitem{Как связан уровень интеллекта многоагентной
					системы с уровнем интеллекта агентов, входящих в ее состав.}
			\scnfileitem{Почему, например, не каждый коллектив
					высокоинтеллектуальных людей демонстрирует высокий уровень интеллекта самого
					коллектива.}
			\scnfileitem{Какими дополнительными свойствами кроме
					достаточно высокого уровня интеллекта должны обладать агенты многоагентных
					систем для обеспечения высокого уровня интеллекта самой многоагентной системы
					как самостоятельной целостной кибернетической системы.}
			\scnfileitem{Как зависит уровень интеллекта многоагентной
					системы от организации взаимодействия между агентами, например, от
					использования централизованного или децентрализованного управления.}
		\end{scnrelfromset}

		\begin{scnrelfromvector}{ключевые знаки}
			\scnitem{кибернетическая система}
			\begin{scnindent}
				\begin{scnsubdividing}
					\scnitem{естественная кибернетическая система}
					\scnitem{компьютерная система}
                    \begin{scnindent}
                        \scnidtf{искусственная кибернетическая система}
                    \end{scnindent}
					\scnitem{естественно-искусственная кибернетическая система}
                    \begin{scnindent}
                        \scnidtf{кибернетическая система, являющаяся симбиозом компонентов как
                            естественного, так и искусственного происхождения}
                    \end{scnindent}
				\end{scnsubdividing}
            \end{scnindent}
			\scnitem{качество кибернетической системы}
			\scnitem{физическая оболочка кибернетической системы}
			\scnitem{качество физической оболочки кибернетической системы}
			\scnitem{интеллект}
				\begin{scnindent}
					\scnidtf{уровень интеллекта кибернетическойсистемы}
					\scnidtf{интеллектуальность}
				\end{scnindent}
			\scnitem{интеллектуальная система}
				\begin{scnindent}
					\scnidtf{интеллектуальная кибернетическая система}
					\scnsuperset{интеллектуальная компьютерная система}
				\end{scnindent}
			\scnitem{информация, хранимая в памяти кибернетической системы}
			\scnitem{качество информации, хранимой в памяти кибернетической
				системы}
			\scnitem{база знаний}
			\scnitem{смысловое представление информации в памяти кибернетической
				системы}
			\scnitem{решатель задач кибернетической системы}
			\scnitem{качество решателя задач кибернетической системы}
			\scnitem{память кибернетической системы}
			\scnitem{качество памяти кибернетической системы}
			\scnitem{обучаемость кибернетической системы}
			\scnitem{гибкость кибернетической системы}
			\scnitem{стратифицированность кибернетической системы}
			\scnitem{рефлексивность кибернетической системы}
				\begin{scnindent}
					\scnidtf{уровень рефлексии кибернетической системы}
				\end{scnindent}
			\scnitem{многоагентная система}
			\scnitem{качество многоагентной системы}
			\scnitem{унифицированность агентов многоагентной системы}
			\scnitem{семантическая совместимость агентов многоагентной системы}
			\scnitem{интероперабельность кибернетической системы}
				\begin{scnindent}
					\scnidtf{способность кибернетической системы своей внутренней и внешней деятельностью обеспечивать
						высокий уровень интеллекта тех многоагентных систем, членом (агентом) которых
						она является}
				\end{scnindent}
		\end{scnrelfromvector}

		\begin{scnrelfromvector}{библиография}
			\scnitem{\scncite{Viner1952}}
			\scnitem{\scncite{Pospelov1989}}
			\scnitem{\scncite{Finn2008}}
			\scnitem{\scncite{YarushinaHS}}
			\scnitem{\scncite{RedkoV2019}}
			\scnitem{\scncite{Fry2002}}
			\scnitem{\scncite{Glushkov1979}}
			\scnitem{\scncite{Nilsson2005}}
			\scnitem{\scncite{Kerr2006}}
			\scnitem{\scncite{Antsyferov2013}}
			\scnitem{\scncite{Sherif1988}}
			\scnitem{\scncite{Cho2019}}
			\scnitem{\scncite{Melekhova2018}}
			\scnitem{\scncite{Gao2002}}
			\scnitem{\scncite{Laird2009}}
			\scnitem{\scncite{Zagorskiy2022b}}
			\scnitem{\scncite{Finn2021}}
			\scnitem{\scncite{Dorri2018}}
			\scnitem{\scncite{Ferber2003}}
			\scnitem{\scncite{Hadzic2009}}
			\scnitem{\scncite{Balaji2010}}
			\scnitem{\scncite{Ouksel1999}}
			\scnitem{\scncite{Lopes2022}}
			\scnitem{\scncite{Hamilton2006}}
			\scnitem{\scncite{Neiva2016}}
		\end{scnrelfromvector}

		\begin{scnreltovector}{конкатенация сегментов}
			\scnitem{Уточнение понятия кибернетической системы}
			\scnitem{Комплекс свойств, определяющий общий уровень качества
				кибернетической системы}
			\scnitem{Комплекс свойств, определяющих качество физической оболочки
				кибернетической системы}
			\scnitem{Комплекс свойств, определяющих уровень интеллекта
				кибернетической системы}
			\scnitem{Комплекс свойств, определяющий качество информации, хранимой в
				памяти кибернетической системы}
			\scnitem{Комплекс свойств, определяющих качество решателя задач
				кибернетической системы}
			\scnitem{Комплекс свойств, определяющих уровень обучаемости
				кибернетической системы}
			\scnitem{Комплекс свойств, определяющих качество многоагентной системы}
			\scnitem{Комплекс свойств, определяющих уровень интероперабельности
				кибернетической системы как фактора существенного повышения уровня ее
				обучаемости, а также фактора существенного повышения качества всех тех
				многоагентных систем, в состав которых входит данная кибернетическая система}
			\scnitem{Направления эволюции компьютерных систем}
		\end{scnreltovector}

		\newpage
		
		\scnsegmentheader{Уточнение понятия кибернетической системы}
\begin{scnsubstruct}
    \scnheader{кибернетическая система}
    \scnidtf{cистема, которая способна \uline{управлять} своими \uline{действиями},
        адаптируясь к изменениям состояния внешней среды (среды своего обитания) в
        целях самосохранения (сохранения своей целостности и комфортности
        существования путем удержания своих жизненно  важных параметров в определенных
        рамках комфортности) и/или в целях формирования определенных реакций
        (воздействий на внешнюю среду) в ответ на определенные стимулы (на определенные
        ситуации или события во внешней среде), а также которая способна (при
        соответствующем уровне развития) эволюционировать в направлении:
        \begin{scnitemize}
            \item изучения своей внешней среды как минимум для предсказания последствий
            своих воздействий на внешнюю среду, а также для предсказания изменений внешней
            среды, которые не зависят от собственных воздействий;
            \item изучения самой себя и, в частности, своего взаимодействия с внешней
            средой;
            \item создания технологий (методов и средств), обеспечивающих изменение своей
            внешней среды (условий своего существования) в собственных интересах.
        \end{scnitemize}
    }
    \scnidtf{адаптивная система}
    \scnidtf{целенаправленная (целеустремленная) система}
    \scnidtf{активный субъект самостоятельной деятельности}
    \scnidtf{материальная сущность, способная целенаправленно (в своих интересах)
        воздействовать	на среду своего обитания  как минимум для сохранения своей
        целостности, жизнеспособности, безопасности}
    \scntext{примечание}{Уровень (степень) адаптивности, целенаправленности, активности у
        систем, основанных на обработке информации может быть самым
        различным.}
    \scnidtf{система, организация функционирования которой основано на
        обработке информации о той среде, в которой существует эта система}
    \scnidtf{материальная сущность, способная к активной  целенаправленной
        деятельности, которая  на определенном уровне развития указанной сущности
        становится осмысленной, планируемой, преднамеренной деятельностью}
    \scnidtf{субъект, способный на самостоятельное выполнение некоторых внутренних
        и внешних  действий либо порученных извне, либо инициированных самим субъектом}
    \scnidtf{сущность, способная выполнять роль субъекта деятельности}
    \scnidtf{естественная или искусственно созданная система, способная мониторить
        и анализировать свое состояние и состояние окружающей среды, а также способная
        достаточно активно воздействовать на собственное на собственное состояние и на
        состояние окружающей среды}
    \scnidtf{система, способная в достаточной степени самостоятельно
        взаимодействовать со своей средой, решая различные задачи}
    \scnidtf{система, основанная на обработке информации}
    \scnrelto{ключевой знак}{\scncite{Glushkov1979}}
    \begin{scnindent}
        \scniselement{статья}
    \end{scnindent}
    \scntext{примечание}{\scnkeyword{кибернетическая система} динамически сопоставляет полученную информацию с выбранными действиями,
        относящимися к задаче, которая определяет основную цель системы.}
        \scntext{источник}{\scncite{Fry2002}}
    \bigskip

    \scnsegmentheader{Типология кибернетических систем}
    \begin{scnsubstruct}
        \scnheader{кибернетическая система}
        \scnrelfrom{разбиение}{Признак естественности или искусственности кибернетических систем}
        \begin{scnindent}
            \begin{scneqtoset}
                \scnitem{естественная кибернетическая система}
                \begin{scnindent}
                    \scnidtf{кибернетическая система естественного происхождения}
                    \scnsuperset{человек}
                \end{scnindent}
                \scnitem{компьютерная система}
                \begin{scnindent}
                    \scnidtf{искусственная кибернетическая система}
                    \scnidtf{кибернетическая система искусственного происхождения}
                    \scnidtf{технически реализованная кибернетическая система}
                \end{scnindent}
                \scnitem{симбиоз естественных и искусственных кибернетических систем}
                \begin{scnindent}
                    \scnidtf{кибернетическая система, в состав которой входят компоненты
                        как естественного, так и искусственного происхождения}
                    \scnsuperset{сообщество компьютерных систем и людей}
                \end{scnindent}
            \end{scneqtoset}
        \end{scnindent}
        
        \scnheader{искусственная сущность}
        \scnidtf{артефакт}
        \scnidtf{сущность, являющаяся либо результатом человеческой деятельности, либо
            частью самой этой деятельности}
        \scnidtf{сущность искусственного происхождения}
        \scnidtf{антропогенная сущность}
        \scnsuperset{научно-техническое знание}
        \scnidtf{знание, приобретенное в результате научно-технической деятельности
            человеческого общества}
        \scnsuperset{материальная искусственная сущность}
        \scnsuperset{компьютерная система}
        \scnheader{компьютерная система}
        \scnidtf{искусственная кибернетическая система}
        \scntext{примечание}{Особенностью компьютерных систем является то, что они могут
            выполнять роль	не только продуктов соответствующих действий по реализации этих
            систем, но и сами являются \textit{субъектами*}, способными выполнять
            (автоматизировать) широкий спектр действий. При этом интеллектуализация этих
            систем существенно расширяет этот спектр. \textit{См. интеллектуальная
                компьютерная система}.}\scnidtf{технически реализованная кибернетическая
            система}
        \scnidtf{искусственная кибернетическая система}
        \scnsubset{кибернетическая система}
        \scnsuperset{современная компьютерная система традиционного вида}
        \scnsuperset{современная интеллектуальная компьютерная система}
        \scnsuperset{интеллектуальная компьютерная система следующего поколения}
        \scnsuperset{ostis-система}
        \scntext{примечание}{Основной тенденцией эволюции компьютерных систем является
            повышение уровня их интеллектуальности.}
        \begin{scnrelfromset}{особенность}
            \scnfileitem{Ориентация на принципиально новые компьютеры}
            \scnfileitem{Cущественное повышение уровня интеллекта}
        \end{scnrelfromset}
        
        \scnheader{кибернетическая система}
        \scnrelfrom{разбиение}{Структурная классификация кибернетических
            систем}
            \begin{scnindent}
                \begin{scneqtoset}
                    \scnitem{простая кибернетическая система}
                    \scnitem{индивидуальная кибернетическая система}
                    \scnitem{многоагентая система}
                    \begin{scnindent}
                        \begin{scnsubdividing}
                            \scnitem{одноуровневый коллектив кибернетических систем}
                                \begin{scnindent}
                                    \scnidtf{многоагентная система, агентами которой не могут быть многоагентные системы}
                                \end{scnindent}
                            \scnitem{иерархический коллектив кибернетических систем}
                                \begin{scnindent}
                                    \scnidtf{многоагентная система, по крайней мере одним	агентом которой является многоагентная система}
                                \end{scnindent}
                        \end{scnsubdividing}
                        \begin{scnsubdividing}
                            \scnitem{коллектив из простых кибернетических систем}
                                \begin{scnindent}
                                    \scntext{примечание}{Такой коллектив может быть либо одноуровневым, либо иерархическим коллективом}
                                \end{scnindent}
                            \scnitem{коллектив из индивидуальных кибернетических систем}
                            \scnitem{коллектив из индивидуальных и простых кибернетических систем}
                        \end{scnsubdividing}
                    \end{scnindent}
                \end{scneqtoset}
            \end{scnindent}
        \scnrelfrom{разбиение}{Классификация кибернетических систем по признаку наличия
            надсистемы и роли в рамках этой надсистемы}
            \begin{scnindent}
                \begin{scneqtoset}
                    \scnitem{кибернетическая система, не являющаяся частью никакой другой
                        кибернетической системы}
                        \begin{scnindent}
                            \scnidtf{кибернетическая система, не имеющая надсистем}
                        \end{scnindent}
                    \scnitem{кибернетическая система, встроенная в индивидуальную кибернетическую
                        систему}
                    \scnitem{агент многоагентной системы}
                        \begin{scnindent}
                            \scnidtf{кибернетическая система, являющаяся агентом одной или нескольких многоагентных систем}
                        \end{scnindent}
                \end{scneqtoset}
            \end{scnindent}

        \scnheader{простая кибернетическая система}
        \scnidtf{\textit{кибернетическая система}, уровень развития которой находится
            ниже уровня \textit{индивидуальных кибернетических систем} и которая является
            специализированным средством обработки информации специализированным решателем
            задач, реализующим (интерпретирующим) чаще всего один \textit{метод} решения
            задач и, соответственно, решающим только \textit{задачи} заданного
            \textit{класса задач}}
        \scnidtf{специализированный \textit{решатель задач}}
        \scntext{примечание}{\textit{простая кибернетическая система} может быть
            \textit{компонентом*}, встроенным в \textit{индивидуальную кибернетическую
            систему}, а также может быть \textit{агентом*} \textit{многоагентной системы}, являющейся коллективом из простых
            кибернетических систем}
            
        \scnheader{индивидуальная кибернетическая система}
        \scnidtf{условно выделенный уровень развития \textit{кибернетических систем}, в
            основе которого лежит переход от \textit{специализированного решателя задач к
            индивидуальному решателю}, обеспечивающему интерпретацию произвольного
            (нефиксированного) набора \textit{методов} (программ) решения задач при
            условии, если эти \textit{методы} введены (загружены, записаны) в
            \textit{память} \textit{кибернетической системы}}
        \scnidtf{кибернетическая система, способная быть самостоятельной}
        \scntext{пояснение}{Признаками индивидуальных кибернетических систем
            являются:
            \begin{scnitemize}
                \item наличие \textit{памяти}, предназначенной для хранения как минимум
                интерпретируемых \textit{методов} (программ)  и обеспечивающей корректировку
                (редактирование) хранимых \textit{методов}, а также их удаление  из
                \textit{памяти} и ввод (запись) в \textit{память} новых \textit{методов};
                \item легкая возможность перепрограммировать  \textit{кибернетическую систему}
                на решение других задач, что обеспечивается наличием \textit{универсальной
                    модели решения задач} и, соответственно, \textit{универсальным интерпретатором
                    \uline{любых} моделей}, представленных (записанных) на соответствующем
                \textit{языке};
                \item наличие пусть даже простых средств коммуникации (обмена информацией) с
                другими \textit{кибернетическими системами} (например, с людьми);
                \item способность входить в различные \textit{коллективы кибернетических
                    систем}.
            \end{scnitemize}
        }\scntext{примечание}{класс \textit{индивидуальных кибернетических систем}  это
            определенный этап эволюции кибернетических систем, означающий переход к
            кибернетическим системам, которые способны самостоятельно
            выживать}\scnidtf{самостоятельная автономная, целостная кибернетическая
            системам}
        \scnidtf{субъект деятельности}
        \scntext{примечание}{\textit{индивидуальная кибернетическая система} может быть
            агентом (членом) многоагентной системы (членом коллектива индивидуальных
            кибернетических систем), но некоторые многоагентные системы могут состоять из
            агентов , не являющихся  \textit{индивидуальными кибернетическими системами},
            представляющих собой простые специализированные кибернетические системы,
            выполняющие достаточно простые действия}
            \begin{scnindent}
                \begin{scnrelfromlist}{источник}
                    \scnitem{\scncite{Stefanuk}}
                    \scnitem{\scncite{fonNeuman}}
                \end{scnrelfromlist}
            \end{scnindent}
        \scnidtf{кибернетическая система, которая обладает
            достаточной самостоятельностью (целостностью), но не является коллективом таких
            самостоятельных  кибернетических систем}
        \scnidtf{минимальная самостоятельная (самодостаточная, в известной степени
            автономная) кибернетическая система}
        \scnidtf{индивидуальный субъект}
        \scnheader{кибернетическая система, встроенная в индивидуальную кибернетическую
            систему}
        \scnrelsuperset{пример}{sc-агент ostis-системы}
        \scnrelsuperset{пример}{решатель задач ostis-системы}
        \scnheader{многоагентная система}
        \scnidtf{коллектив взаимодействующих автономных кибернетических систем, имеющих
            общую среду обитания (жизнедеятельности)}
        \begin{scnsubdividing}
            \scnitem{коллектив из простых кибернетических систем}
            \scnitem{коллектив индивидуальных кибернетических систем}
            \scnitem{коллектив индивидуальных и простых кибернетических систем}
        \end{scnsubdividing}
        \begin{scnsubdividing}
            \scnitem{одноуровневый коллектив кибернетических систем}
            \begin{scnindent}
                \scnidtf{многоагентная система, агентами которой не могут быть многоагентные системы}
            \end{scnindent}
            \scnitem{иерархический коллектив кибернетических систем}
            \begin{scnindent}
                \scnidtf{многоагентная система, по крайней мере одним агентом которой является многоагентная система}
            \end{scnindent}
        \end{scnsubdividing}
        \scnheader{одноуровневый коллектив кибернетических систем}
        \scnidtf{специализированное средство решения задач, реализующее либо
            \uline{одну} модель параллельного (распределенного) решения задач
            соответствующего класса, либо комбинацию \uline{фиксированного числа} разных и
            параллельно реализованных моделей решения задач}
        \begin{scnsubdividing}

            \scnitem{одноуровневая однородная многоагентная система}
            \scnitem{одноуровневая неоднородная многоагентная система}

        \end{scnsubdividing}
        \scnheader{коллектив индивидуальных кибернетических систем}
        \scnsubset{многоагентная система}
        \scnidtf{многоагентная система, агентами (членами) которой являются
            \uline{индивидуальные}(!) кибернетические системы}
        \begin{scnsubdividing}

            \scnitem{коллектив людей}
                \begin{scnindent}
                    \scnidtf{человеческое сообщество}
                \end{scnindent}
            \scnitem{сообщество компьютерных систем и людей}

        \end{scnsubdividing}
        \scnheader{иерархический коллектив индивидуальных кибернетических систем}
        \scnidtf{многоагентная система, агентами (членами) которой могут быть:
            \begin{scnitemize}

                \item индивидуальные кибернетические системы;
                \item коллективы индивидуальных кибернетических систем;
                \item коллективы, состоящие из индивидуальных кибернетических систем и
                коллективов индивидуальных кибернетических систем и т.д.
            \end{scnitemize}
        }
        \bigskip
    \end{scnsubstruct}

    \scnsegmentheader{Структура кибернетической системы}
    \begin{scnsubstruct}
        \scnheader{кибернетическая система}
        \begin{scnrelfromset}{обобщенная декомпозиция}

            \scnitem{информация, хранимая в памяти кибернетической системы}
            \scnitem{абстрактная память кибернетической системы}
            \scnitem{решатель задач кибернетической системы}
            \scnitem{физическая оболочка кибернетической системы}

        \end{scnrelfromset}
        \scnheader{информация, хранимая в памяти кибернетической системы}
        \scnidtf{информация, хранимая в памяти \textit{кибернетической системы} и
            представляющая собой информационную модель среды, в которой действует
            (существует, функционирует) эта \textit{кибернетическая система}}
        \scnidtf{текущее состояние памяти кибернетической системы}
        \scnidtf{текущее состояние внутренней (информационной) среды кибернетической
            системы}
        \scnrelto{второй домен}{информация, хранимая в памяти кибернетической системы*}
        \begin{scnindent}
            \scniselement{бинарное отношение}
            \scniselement{ориентированное отношение}
        \end{scnindent}
        \scnheader{абстрактная память кибернетической системы}
        \scnidtf{внутренняя абстрактная информационная среда \textit{кибернетической системы},
            представляющая собой динамическую \textit{информационную  конструкцию}, каждое состояние
            которой есть не что иное, как \textit{информация, хранимая в памяти кибернетической
            системы} в соответствующий момент времени}
        \scnidtf{абстрактная динамическая модель памяти кибернетической системы}
        \scnsubset{динамическая информационная конструкция}
        \begin{scnindent}
            \scnidtf{процесс преобразования информационной конструкции}
        \end{scnindent}
        \scnheader{решатель задач кибернетической системы}
        \scnidtf{совокупность всех навыков (умений), приобретенных кибернетической
            системой к рассматриваемому моменту}
        \scnidtf{встроенный в кибернетическую систему субъект, способный выполнять
            целенаправленные (осознанные) действия во внешней среде этой кибернетической
            системы, а также в её внутренней среде (в абстрактной памяти)}
        \scnheader{действие кибернетической системы}
        \scnsubset{действие}
        \scnidtf{целенаправленное (осознанное) действие, выполняемое кибернетической
            системой, а точнее, её решателем задач}
        \begin{scnsubdividing}

            \scnitem{внешнее действие кибернетической системы}
            \begin{scnindent}
                \scnidtf{действие, выполняемое кибернетической системой в её внешней среде}
                \scnidtf{поведенческое действие}
            \end{scnindent}
            \scnitem{действие кибернетической системы, выполняемое в собственной физической
                оболочке}
            \scnitem{действие кибернетической системы, выполняемое в собственной
                абстрактной памяти}
            \begin{scnindent}
                \scnidtf{действие, направленное на
                    преобразование информации, хранимой в памяти, но никак не на преобразование
                    физической памяти (физической оболочки абстрактной памяти)}
            \end{scnindent}
        \end{scnsubdividing}
        \scntext{примечание}{Каждое \uline{сложное} действие,выполняемое
            кибернетической системой вне собственный абстрактной памяти, включает в себя
            поддействия, выполняемые в указанной \textit{абстрактной памяти кибернетической системы}. Это означает, что все
            \textit{внешние действия кибернетической системы} \uline{управляются} внутренними её
            действиями (действиями в абстрактной памяти).}
        
        \scnheader{задача}
        \scnidtf{спецификация действия}
        \scnidtf{формулировка задачи с различной степенью детализации (уточнения)
            специфицируемого (описываемого) действия, в состав которой может входить:
            \begin{scnitemize}

                \item описание цели (целевой ситуации);
                \item указание объектов (аргументов) действия;
                \item указание типа действия (класса действий, которому принадлежит данное
                действие);
                \item указание субъекта действия;
                \item указание инструмента (средств) выполненного действия;
                \item и др.
            \end{scnitemize}
        }
        \scntext{примечание}{Процесс решения задачи и действие, специфицируемое этой задачей
            (точнее, процесс выполнения этого действия) суть одно и то
            же.}
        \scnheader{задача, решаемая кибернетической системой}
        \scnidtf{задача, решаемая соответствующей кибернетической системой}
        \scnidtf{Второй домен отношения быть задачей, решаемой заданной кибернетической
            системой*}
        \scnrelboth{следует отличать}{задача, решаемая кибернетической системой*}
        \begin{scnindent}
            \scnidtf{быть задачей, решаемой заданной кибернетической системой*}
        \end{scnindent}
        \begin{scnsubdividing}

            \scnitem{задача, решаемая кибернетической системой во внешней среде}
                \begin{scnindent}
                    \scnidtf{внешняя задача кибернетической системы}
                    \scnidtf{задача, направленная на изменение состояния внешней среды
                        соответствующей кибернетической системы, но включающая в себя (в качестве
                        подзадач) задачи, решаемые в памяти кибернетической системы, например:
                        \begin{scnitemize}

                            \item интерфейсные задачи (анализ первичный информации о текущем состоянии
                            внешней среды),
                            \item cенсо-моторную координацию выполнения сложных действий во внешней среде,
                            состоящих из большого количества частных (более простых) действий, находящихся
                            на разных уровнях иерархии,
                            \item задачи планирования целенаправленного поведения во внешней среде,
                            \item задачи принятия решений.
                        \end{scnitemize}
                    }
                \end{scnindent}
            \scnitem{задача, решаемая кибернетической системой в собственной физической
                оболочке}
            \scnitem{задача решаемая кибернетической системой в абстрактной
                памяти}
                \begin{scnindent}
                    \scnidtf{задача, полностью решаемая в памяти кибернетической системы и
                        направленная на изменение состояния информации, хранимой в памяти
                        кибернетической системы}
                    \scnidtf{внутренняя задача кибернетической системы}
                \end{scnindent}

        \end{scnsubdividing}
        \scnheader{навык}
        \scnsubset{знание}
        \scntext{пояснение}{знание частного вида, содержащее (1) некоторый метод --
            знание о том, как можно решать задачи, принадлежащие соответствующему множеству
            задач, (2) полное знание о том, как указанный метод следует интерпретировать
            (реализовывать), декомпозируя исходные задачи на подзадачи и, в конечном счёте
            на элементарные действия, выполняемые \textit{процессором кибернетической
                системы}}\scnidtf{умение}
        \scnidtf{методы и средства, обеспечивающие способность \textit{кибернетической
                системы} решать некоторое множество задач (выполнять некоторое множество
            действий)}
        \scnheader{интерфейс кибернетической системы}
        \scnidtf{условно выделяемый компонент \textit{решателя задач кибернетической
                системы}, обеспечивающий решение \textit{интерфейсных задач}, направленных на
            \uline{непосредственную} реализацию взаимодействия \textit{кибернетической
                системы} с её \textit{внешней средой}}
        \scnidtf{решатель интерфейсных задач кибернетической системы}
        \scnrelto{обобщенная часть}{решатель задач кибернетической системы}
        \scnrelboth{следует отличать}{физическое обеспечение интерфейса кибернетической
            системы}
            \begin{scnindent}
                \scnrelto{обобщенная часть}{физическая оболочка кибернетической системы}
            \end{scnindent}

        \scnheader{физическая оболочка кибернетической системы}
        \scnidtf{часть кибернетической системы, являющаяся посредником	между её
            внутренней средой (памятью, в которой хранится и обрабатывается информация
            кибернетической системы) и её внешней средой}
        \scnrelto{второй домен}{физическая оболочка кибернетической системы*}
        \begin{scnindent}
            \scniselement{бинарное отношение}
            \scniselement{ориентированное отношение}
        \end{scnindent}

        \scnheader{кибернетическая система}
        \begin{scnrelfromset}{обобщенная декомпозиция}
            \scnitem{память кибернетической системы}
            \scnitem{процессор кибернетической системы}
            \scnitem{физическое обеспечение интерфейса кибернетической системы}
            \begin{scnindent}   
                \scnidtf{аппаратное обеспечение интерфейса кибернетической системы с её
                    внешней средой}
                \begin{scnrelfromset}{обобщенная декомпозиция}
                    \scnitem{сенсорная подсистема физической оболочки кибернетической системы}
                    \scnitem{эффекторная подсистема физической оболочки кибернетической системы}
                \end{scnrelfromset}
            \end{scnindent}
            \scnitem{корпус кибернетической системы}
        \end{scnrelfromset}
        
        \scnheader{память кибернетической системы}
        \scnidtf{физическая оболочка (реализация) абстрактной \textit{памяти
                кибернетической системы} --- внутренней среды \textit{кибернетической системы},
            в рамках которой \textit{кибернетическая система} формирует и использует
            (обрабатывает) информационную модель своей внешней среды}
        \scntext{примечание}{Не каждая \textit{кибернетическая система} имеет
            \textit{память}. В \textit{кибернетических системах}, которые не имеют
            \textit{памяти}, обработка информации сводится к обмену сигналами между
            компонентами этих систем. Появление в \textit{кибернетических системах} памяти
            как среды для централизованного  хранения и обработки \textit{информации}
            является важнейшим этапом их эволюции. Дальнейшая эволюция
            \textit{кибернетических систем} во многом определяется:
            \begin{scnitemize}

                \item \textit{качеством памяти} как среды для хранения и обработки информации;
                \item качеством информации (информационной модели), хранимой в памяти
                кибернетической системы;
            \end{scnitemize}
        }\scnidtf{компонент \textit{кибернетической системы}, в рамках которого
            \textit{кибернетическая система} осуществляет отображение (формирование
            информационной модели) среды своего существования, а также использование этой
            информационной модели для управления собственным поведением в указанной среде}
        \scnidtf{физическая оболочка для хранения информации, которую кибернетическая
            система приобретает и обрабатывает (т.е. меняет состояния этой информации)}
        \scnidtf{физическая (аппаратная) реализация \uline{внутренней} среды
            кибернетической системы, каковой является среда существования  информации,
            накапливаемой и непосредственно используемой решателем задач этой
            кибернетической системы}
        \scntext{примечание}{Сам факт появления в \textit{кибернетической системе} памяти,
            которая (1) обеспечивает представление различного виды информации о среде, в
            рамках которой \textit{кибернетическая} система решает различные задачи (выполняет
            различные действия), (2) обеспечивает хранение достаточно полной информационной
            модели указанной среды (достаточно полной для реализации своей деятельности),
            (3) обеспечивает высокую степень гибкости указанной хранимой в памяти
            информационной модели среды жизнедеятельности (т.е. лёгкость внесения изменений
            в эту информационную модель), существенно повышает уровень адаптивности
            \textit{кибернетической системы} к различным изменениям своей
            среды.}
        \scntext{примечание}{появление  \uline{\textit{памяти}} в кибернетических
            системах является основным признаком перехода от простых  автоматов к
            компьютерным системам, от роботов 1-го поколения к роботам следующих
            поколений}\scnidtf{физическая реализация хранилища информации, которую
            приобрела (накопила) к текущему моменту соответствующая кибернетическая
            система}
        \scnidtf{физическая оболочка внутренней абстрактной информационной среды
            кибернетической системы}
        \scnidtf{среда хранения и обработки информации}
        \scnidtf{запоминающая среда}
        \scnidtf{среда хранения и обработки информационных конструкций}
        \scntext{примечание}{Принципы организации памяти \textit{кибернетической системы} могут быть
            разными (ассоциативная, адресная, структурно фиксированная/структурно
            перестраиваемая, нелинейная/линейная). От организации памяти \textit{кибернетической системы} во многом зависит
            её качество.}
    
        \scnheader{кибернетическая система}
        \scnrelfrom{уровни эволюции}{Уровни структурной эволюции кибернетических систем}
        \begin{scnindent}
            \begin{scneqtovector}

                \scnitem{простая кибернетическая система, не имеющая памяти}
                \scnitem{простая кибернетическая система, имеющая память}
                \scnitem{одноуровневый коллектив, не имеющий общей памяти и состоящий из
                    простых кибернетических систем, не имеющих памяти}
                \scnitem{одноуровневый коллектив, не имеющий общей памяти и состоящий из
                    простых кибернетических систем, имеющих память}
                \scnitem{иерархический коллектив,  имеющий общую память и состоящий из простых
                    кибернетических систем}
                \scnitem{индивидуальная кибернетическая система}
                    \begin{scnindent}
                        \scntext{примечание}{Каждая \textit{индивидуальная кибернетическая система} содержит \textit{память},
                                имеющую достаточно высокий уровень качества}
                    \end{scnindent}
                \scnitem{одноуровневый коллектив индивидуальных кибернетических систем, не
                    имеющий общей памяти}
                \scnitem{одноуровневый коллектив индивидуальных кибернетическая систем, имеющий
                    общую память }
                \scnitem{иерархический коллектив из индивидуальных кибернетических систем, не
                    имеющий общей памяти}
                \scnitem{иерархический коллектив из индивидуальных кибернетических систем,
                    имеющий общую память}
            \end{scneqtovector}
        \end{scnindent}

        \scnheader{процессор кибернетической системы}
        \scnidtf{физически (аппаратно реализованный) интерпретатор хранимых в \textit{памяти
            кибернетической системы методов (программ)}, соответствующих базовой (для данной
            кибернетической системы) \textit{модели решения задач}, т.е. такой модели решения задач,
            которая для данной \textit{кибернетической системы} является \textit{моделью решения задач}
            самого нижнего уровня и, следовательно, не может быть интерпретирована с
            помощью другой \textit{модели решения задач}, используемой этой же \textit{кибернетической
            системой}, а может быть проинтерпретирована либо путем аппаратной реализации
            такого интерпретатора, либо путём его программной реализации, например, на
            современных компьютерах, но в последнем случае, кроме собственного
            интерпретатора, необходимо также построить модель \textit{памяти} реализуемой
            кибернетической системы}
        \scnidtf{физически  реализованные средства, обеспечивающие выполнение
            элементарных  действий, направленных на изменение состояния памяти
            кибернетической системы (на изменение информации, хранимой в этой памяти)}
        \scnidtf{движок(мотор) кибернетической системы}
        \scnrelto{второй домен}{процессор кибернетической системы*}
        \begin{scnindent}
            \scnidtftext{пояснение}{бинарное ориентированное отношения, каждая пара
                которого связывает знак кибернетической системы со знаком её процессора}
            \scniselement{бинарное отношение}
            \scniselement{ориентированное отношение}
        \end{scnindent}

        \scnheader{компьютер}
        \scnsubset{физическая оболочка кибернетической системы}
        \scnidtf{физическая оболочка искусственной кибернетической системы}
        \scnidtf{аппаратное обеспечение компьютерной системы}
        \scnidtf{hardware of computer system}
        \scnsuperset{компьютер для интеллектуальных систем}
        \begin{scnindent}
            \scnidtf{компьютер, ориентированный на реализацию интеллектуальных компьютерных
                систем}
            \scntext{примечание}{Развитие рынка \textit{интеллектуальных компьютерных систем} существенно
                сдерживается неприспособленностью современного поколения компьютеров к
                реализации на их основе \textit{интеллектуальных компьютерных систем}. Попытки создания
                компьютеров, приспособленных к реализации \textit{интеллектуальных компьютерных систем},
                не привели к успеху, т.к. эти проекты были направлены на выполнение отдельных
                (частных) требований, предъявляемых к физическому (аппаратному) уровню
                \textit{интеллектуальных компьютерных систем}, что неминуемо приводило к приспособленности
                создаваемых компьютеров к реализации не всего многообразия \textit{интеллектуальных
                компьютерных систем}, а только некоторых подмножеств таких систем. Указанные
                подмножества \textit{интеллектуальных компьютерных систем} в основном определялись
                ориентацией на конкретные используемые \textit{модели решения интеллектуальных задач},
                тогда, как важнейшим фактором, определяющим уровень \textit{интеллекта кибернетических
                систем} (в том числе, и компьютерных систем), является их универсальность в
                плане многообразия используемых моделей решения задач. Следовательно, компьютер
                для \textit{интеллектуальных компьютерных систем} должен быть эффективным аппаратным
                интерпретатором любых моделей решения задач (как интеллектуальных задач, так и
                достаточно простых задач, т.к. интеллектуальная система должна уметь решать
                любые задачи).}\scnidtf{компьютер, приспособленный к реализации
                интеллектуальных компьютерных систем}
            \scnidtf{универсальный компьютер для интеллектуальных систем}
            \scnidtf{компьютер, обеспечивающий интерпретацию любых моделей решения задач}
        \end{scnindent}
        \bigskip
    \end{scnsubstruct}
    \scnsegmentheader{Семейство отношений, заданных на множестве кибернетических
        систем}
    \begin{scnsubstruct}
        \scnheader{отношение, заданное на множестве кибернетических систем}
        \scnhaselement{память кибернетической системы*}
        \scnhaselement{процессор кибернетической системы*}
        \scnhaselement{член коллектива*}
        \scnhaselement{внешняя среда кибернетической системы*}
        \scnhaselement{сенсор кибернетической системы*}
        \scnhaselement{эффектор кибернетической системы*}
        \scnhaselement{физическая оболочка кибернетической системы*}
        \scnhaselement{информация, хранимая в памяти кибернетической системы*}
        \scnhaselement{абстрактная память кибернетической системы*}
        \scnhaselement{часть*}
        \begin{scnindent}
            \scnsuperset{встроенная кибернетическая система*}
        \end{scnindent}

        \scnheader{информация, хранимая в памяти кибернетической системы*}
        \scnidtf{\textit{информационная модель среды*}, в которой существует
            (осуществляет деятельность) соответствующая кибернетическая система*}
        \scntext{примечание}{От того, насколько полна, адекватна (корректна) и
            систематизирована (структурирована) внутренняя среда кибернетической системы,
            зависит уровень интеллектуальности и эффективность соответствующей
            кибернетической системы.}
            
        \scnheader{следует отличать*}
        \begin{scnhaselementset}

            \scnitem{задача, решаемая кибернетической системой*}
            \scnitem{решатель задач кибернетической системы}
            \begin{scnindent}
                \scnidtf{иерархическая система моделей решения задач}
                    \scnrelfrom{обобщённая часть}{процессор кибернетической системы}
                    \begin{scnindent}
                        \scntext{пояснение}{Это реализация модели решения задач, обеспечивающей
                        интерпретацию всех используемых моделей решения задач верхнего уровня}
                    \end{scnindent}
            \end{scnindent}
        \end{scnhaselementset}

        \scnheader{задача, решаемая кибернетической системой*}
        \scnidtf{быть задачей, решаемой заданной кибернетической системой*}
        \scnsuperset{задача, решаемая в памяти кибернетической системы*}
        \begin{scnindent}
            \scnidtf{внутренняя задача кибернетической системы*}
        \end{scnindent}
        \scnsuperset{задача, решаемая во внешней среде кибернетической системы*}
        
        \scnheader{\textit{внешняя среда кибернетической системы*}}
        \scnidtf{внешняя среда*}
        \scntext{примечание}{Понятие \textit{внешней среды кибернетической системы*} является
            понятием относительным, т.к. (1) разные \textit{кибернетические системы} имеют в общем
            случае разную внешнюю среду и (2) одна \textit{кибернетическая система} может входить в
            состав внешней среды другой кибернетической системы}
        \scnidtf{быть внешней
            средой для заданной кибернетической системы*}
        \scniselement{бинарное отношение}
        \scniselement{ориентированное отношение}
        \scnrelfrom{первый домен}{кибернетическая система}
        \scnsuperset{внешняя информационная среда кибернетической системы*}
        \begin{scnindent}
            \scnidtf{совокупность всевозможных информационных конструкций, к которым данная
                кибернетическая система имеет доступ и которые представлены самым различным
                образом (в том числе, и в памяти тех кибернетических систем (субъектов), с
                которыми данная система взаимодействует)*}
        \end{scnindent}

        \scnheader{среда кибернетической системы*}
        \scnidtf{быть средой существования (жизнедеятельности) заданной (указанной,
            соответствующей) кибернетической системы*}
        \scntext{примечание}{В общем случае среда жизнедеятельности \textit{кибернетической
                системы} включает в себя (1) \textit{внешнюю среду*} этой системы, (2)
            \textit{физическую оболочку*} этой системы и (3) её \textit{абстрактную
                память}, т.е. внутреннюю среду*, которая является хранилищем информационной
            модели всей среды}\begin{scnsubdividing}

            \scnitem{внешняя среда*}
            \scnitem{физическая оболочка*}
            \scnitem{абстрактная память*}

        \end{scnsubdividing}
        \bigskip
    \end{scnsubstruct}
\end{scnsubstruct}
\scnsourcecomment{Завершили Сегмент \scnqqi{Уточнение понятия кибернетической системы}}

		\newpage
\scnsegmentheader{Комплекс свойств, определяющий общий уровень качества кибернетической системы}
\begin{scnsubstruct}
    \scnheader{качество кибернетической системы}
    \scnidtf{интегральный уровень качества кибернетической системы в заданный момент}
    \scnidtf{комплексная оценка (характеристика) уровня качества кибернетической системы}
    \scntext{пояснение}{Для того, чтобы уточнить (детализировать) понятие \textit{качества кибернетической системы}, необходимо
        \begin{scnitemize}
            \item задать метрику \textit{качества кибернетических систем} и
            \item построить иерархическую систему свойств (параметров, признаков), определяющих \textit{качество кибернетической системы}.
        \end{scnitemize}
    }
    \scniselement{упорядоченное свойство}
    \scnidtf{эволюционный уровень кибернетической системы}
    \scnidtf{интегральная (комплексная) оценка уровня развития (совершенства) кибернетической системы}
    \scntext{пояснение}{\textit{Качество кибернетической системы} --- это такое свойство (характеристика) \textit{кибернетических систем}, такой признак их классификации, который позволяет разместить эти системы по ступенькам некоторой условной эволюционной лестницы. На каждую такую ступеньку попадают \textit{кибернетические системы}, имеющие одинаковый уровень развития, каждому их которых соответствует свой набор значений дополнительных свойств \textit{кибернетических систем}, которые уточняют (детализируют, специализируют) соответствующий уровень развития \textit{кибернетических систем}. Такой эволюционный подход к рассмотрению \textit{кибернетических систем} даёт возможность, во-первых, детализировать направления эволюции \textit{кибернетических систем} и, во-вторых, уточнить то место этой эволюции, где и благодаря чему осуществляется переход от неинтеллектуальных \textit{кибернетических систем} к интеллектуальным. Фактически речь идёт об эволюционной теории качества \textit{кибернетических систем}, рассматривающей эволюцию \textit{кибернетических систем} как в рамках жизненного цикла каждой из них, так и в рамках эволюции целой популяции при переходе от одного поколения \textit{кибернетических систем} к другому поколению (в частности, от одного поколения \textit{компьютерных систем} к другому).
        В основе эволюционного подхода к рассмотрению многообразия \textit{кибернетических систем} лежит положение о том, что идеальных \textit{кибернетических систем} не существует, но существует постоянное стремление к идеалу, к большему совершенству. При этом важно уточнить, что конкретно в каждой \textit{кибернетической системе} следует изменить, чтобы привести эту систему к более совершенному виду.
        \\Эволюционный подход к рассмотрению \textit{кибернетических систем} имеет важное практическое значение для развития (совершенствования) каждой конкретной \textit{компьютерной системы} (искуственной \textit{кибернетической системы}), а также для развития \textit{технологий} разработки \textit{компьютерных систем}. Так, например, развитие технологий разработки \textit{компьютерных систем} должно быть направлено на переход к таким новым архитектурным и функциональным принципам, лежащим в основе \textit{компьютерных систем}, которые
        \begin{scnitemize}
            \item обеспечивают существенное снижение трудоемкости их разработки и сокращение сроков разработки, а также
            \item обеспечивают существенное повышение уровня \textit{интеллекта} и, в частности, уровня \textit{обучаемости} разрабатываемых \textit{компьютерных систем}, например, путём перехода от поддержки обучения с учителем к реализации эффективного самообучения (к автоматизации организации самостоятельного обучения).
        \end{scnitemize}}
    \scntext{примечание}{В эволюции \textit{кибернетических систем} (и, в частности, \textit{компьютерных систем}) можно выделить целый ряд этапов:
        \begin{scnitemize}
            \item переход от стимульно-реактивного поведения к поведению, предполагающему учёт постоянно накапливаемого собственного опыта, означает переход от протопамяти, которая просто фиксирует связи между стимулами и соответствующими реакциями и которая не предполагает изменения этих связей, к \textit{памяти}, которая становится средой существования информации, отражающей  собственный опыт \textit{кибернетической системы} (а в перспективе и многое другое) и которая обеспечивает высокую степень \textit{гибкости} хранимой \textit{информации}, т.е. широкие возможности изменения (корректировки) этой \textit{информации} в процессе функционирования \textit{кибернетической системы}. Таким образом, \textit{память кибернетической системы} вместе с хранимой в ней \textit{информацией} становится управляемым самой этой \textit{кибернетической системой} гибким коммутатором между её стимулами и реакциями, учитывающим не только накапливаемый собственный опыт, но и контекст (дополнительные обстоятельства) выполняемых \textit{действий} (реакций), рассматривающий выполняемые \textit{действия} с самых разных аспектов;
            \item включение в состав \textit{информации, хранимой в памяти компьютерной системы}, \textit{программ}, описывающих различные \textit{методы} обработки этой \textit{информации} и интерпретируемых \textit{процессором} указанной \textit{компьютерной системы};
            \item переход от указанной выше коммутационной трактовки \textit{информации, хранимой в памяти кибернетической системы} к её трактовке как мощной и постоянно совершенствуемой информационной модели внешней среды, в которой существует указанная \textit{кибернетическая система}. Это означает
                \begin{scnitemizeii}
                    \item переход \textit{информации, хранимой в памяти кибернетической системы} на уровень \textit{базы знаний}, которой ставится в \textit{соответствие} достаточно чёткая \textit{денотационная семантика}, и
                    \item переход \textit{программ}, хранимых в \textit{памяти кибернетической системы}, на уровень \textit{программ}, которые ориентированы на обработку \textit{базы знаний} и которые сами являются частью обрабатываемой \textit{базы знаний};
                \end{scnitemizeii}
            \item существенное расширение \textit{семантической мощности баз знаний} и многообразия используемых \textit{моделей решения задач}, в том числе, моделей, способных работать в условиях неполноты (недостаточности), нечеткости и недостоверности обрабатываемых \textit{знаний}.
        \end{scnitemize}}
    \scntext{примечание}{Повышение качества искусственных\textit{ кибернетических систем} (\textit{компьютерных систем}) потребует формирования таких свойств (характеристик, способностей) \textit{компьютерных систем}, которые аналогичны психическим свойствам людей. Таким образом, дальнейшее развитие \textit{Искусственного интеллекта} (теории и практики создания \textit{интеллектуальных компьютерных систем} --- интеллектуальных искусственных \textit{кибернетических систем}) настоятельно потребует обобщения современной психологии (психологии биологических индивидов и их коллективов --- \textit{психологии естественных кибернетических систем}) и создания \textit{общей психологии кибернетических систем} (как естественных, так и искусственных) основанной на высоком уровне формализации.}
    \scntext{примечание}{Проблема выделения критериев интеллектуальности компьютерных систем рассмотрена в ряде работ. Были предложены различные системные показатели для измерения качества компьютерных систем. Поскольку системы становятся все более сложными и включают множество подсистем или компонентов, измерение их качества в нескольких измерениях становится сложной задачей. Метрики качества включают в себя набор мер, которые могут описывать атрибуты системы в терминах, не зависящих от структуры, которая приводит к этим атрибутам. Эти меры должны быть выражены количественно и должны иметь значительный уровень точности и надежности.}
    \begin{scnindent}
        \begin{scnrelfromlist}{источник}
            \scnitem{\scncite{Cho2019}}
            \scnitem{\scncite{Sherif1988}}
            \scnitem{\scncite{Finn2021}}
            \scnitem{\scncite{Nilsson2005}}
            \scnitem{\scncite{Kerr2006}}
            \scnitem{\scncite{Antsyferov2013}}
        \end{scnrelfromlist}
    \end{scnindent}

    \scnheader{качество кибернетической системы}
    \begin{scnrelfromlist}{cвойство-предпосылка}

        \scnitem{качество физической оболочки кибернетической системы}
        \scnitem{качество решателя задач кибернетической системы}
        \begin{scnindent}
            \scnrelfrom{cвойство-предпосылка}{качество информации, хранимой в памяти кибернетической системы}
        \end{scnindent}
        \scnitem{качество информации, хранимой в памяти кибернетической системы}
        \scnitem{гибридность кибернетической системы}
        \begin{scnindent}
            \scnidtf{степень многообразия (1) видов знаний, хранимых в памяти кибернетической системы, (2) используемых моделей решения задач, (3) видов сенсоров и эффекторов}
            \begin{scnrelfromlist}{частное свойство}

                \scnitem{многообразие видов знаний, хранимых в памяти кибернетической системы}
                \scnitem{многообразие моделей решения задач}
                \scnitem{многообразие видов сенсоров и эффекторов}

            \end{scnrelfromlist}
        \end{scnindent}
        \scnitem{приспособленность кибернетической системы к её совершенствованию}
        \scnitem{производительность кибернетической системы}
        \begin{scnindent}
            \scnidtf{cкорость решения задач кибернетической системы}
        \end{scnindent}
        \scnitem{надежность кибернетической системы}
        \scnitem{интероперабельность кибернетической системы}

    \end{scnrelfromlist}
    \scnheader{гибридность кибернетической системы}
    \begin{scnrelfromlist}{частное свойство}

        \scnitem{многообразие видов знаний, хранимых в памяти кибернетической системы}
        \scnitem{многообразие моделей решения задач}
        \scnitem{многообразие видов сенсоров и эффекторов}

    \end{scnrelfromlist}
    \scnheader{гибридная кибернетическая система}
    \scnidtf{кибернетическая система, использующая многообразие рецепторных и/или эффекторных подсистем, и/или многообразие видов обрабатываемой информации, и/или многообразие способов решения задач}
    \scnsuperset{гибридная компьютерная система}
    \begin{scnindent}
        \scnidtf{\textit{компьютерная система}, способная решать \textit{комплексные задачи}, требующие использования многообразия различных видов обрабатываемой информации и различных \textit{моделей решения задач}}
    \end{scnindent}

    \scnheader{приспособленность кибернетической системы к её совершенствованию}
    \scnidtf{приспособленность кибернетической системы к эволюции, к повышению уровня своего качества}
    \begin{scnrelfromset}{комплекс свойств-предпосылок}

        \scnitem{обучаемость кибернетической системы}
        \begin{scnindent}
            \scnidtf{способность кибернетической системы самостоятельно повышать уровень своего качества}
            \scnidtf{способность кибернетической системы к самоэволюции, саморазвитию, устранению своих недостатков}
        \end{scnindent}
        \scnitem{приспособленность кибернетической системы к её совершенствованию, осуществляемому извне}
        \begin{scnindent}    
            \scnidtf{приспособленность кибернетической системы к её совершенствованию, осуществляемому внешними субъектами}
            \scnidtf{удобство совершенствования кибернетической системы для её создателей}
            \scntext{примечание}{Важнейшим фактором качества каждой \textit{технологии разработки кибернетических систем} является гибкость и стратифицированность разрабатываемых кибернетических систем при их совершенствовании, осуществляемом руками разработчиков}
        \end{scnindent}
    \end{scnrelfromset}
    \begin{scnrelfromset}{комплекс свойств-предпосылок}
        \scnitem{гибкость кибернетической системы}
        \scnitem{стратифицированность кибернетической системы}
        \begin{scnindent}
            \scnidtf{уровень стратифицированности кибернетической системы}
            \scnidtf{качество разделения (декомпозиции) кибернетической системы на в достаточной степени независимые части (компоненты), определенные виды изменений которых не предполагают внесения изменений в другие части системы}
        \end{scnindent}
    \end{scnrelfromset}

    \scnheader{гибкость кибернетической системы}
    \scnidtf{реконфигурируемость кибернетической системы}
    \scnidtf{модифицируемость кибернетической системы}
    \scnidtf{реформируемость кибернетической системы }
    \scnidtf{трансформируемость кибернетической системы}
    \scnidtf{пластичность кибернетической системы}
    \scnidtf{легкость реализации различного вида изменений в кибернетической системе}
    \scnidtf{степень трансформенности кибернетической системы}
    \scnidtf{простота внесения изменений в кибернетическую систему и многообразие видов возможных таких изменений}
    \scnidtf{модифицируемость кибернетической системы}
    \scnidtf{трансформируемость кибернетической системы}
    \scnidtf{реконфигурируемость кибернетической системы}
    \scnidtf{приспособленность к реинжинирингу кибернетической системы}
    \scnidtf{мягкость}
    \scnidtf{softness}
    \scnidtf{приспособленность к внесению изменений}
    \scnidtf{\uline{легкость} внесения изменений}
    \scntext{примечание}{Чем легче вносить изменения в кибернетическую систему, тем выше скорость ее эволюции}
    \scntext{примечание}{изменения могут вноситься (1) полностью самостоятельно (без учителя) (2) с помощью учителя-тренера (терапевта) путем создания определенных условий для совершенствования системы (3) хирургически --- путем непосредственного вмешательства извне (например, вмешательства разработчика)}
    \scntext{примечание}{Чем выше \textit{гибкость кибернетической системы} --- тем ниже трудоемкость и меньше сроки внесения различных изменений в систему в направлении ее совершенствования (приближения к идеалу)}
    \begin{scnrelfromset}{комплекс свойств-предпосылок}
        \scnitem{простота внесения изменений в кибернетическую систему}
        \begin{scnindent}
            \scnrelfrom{свойство-предпосылка}{стратифицированность кибернетической системы}
        \end{scnindent}
        \scnitem{многообразие возможных изменений, вносимых в кибернетическую систему}
    \end{scnrelfromset}
    \begin{scnrelfromset}{комплекс частных свойств}
        \scnitem{гибкость информации, хранимой в памяти кибернетической системы}
        \scnitem{гибкость решателя задач кибернетической системы}
        \scnitem{гибкость физической оболочки кибернетической системы}
        \begin{scnindent}
            \scnrelfrom{частное свойство}{гибкость памяти кибернетической системы}
        \end{scnindent}
        \scnitem{гибкость интерфейса кибернетической системы}
    \end{scnrelfromset}
    \begin{scnrelfromset}{комплекс частных свойств}
        \scnitem{гибкость кибернетической системы при ее совершенствовании, осуществляемом извне}
        \scnitem{гибкость возможных самоизменений кибернетической системы}
        \begin{scnindent}
            \scnrelto{свойство-предпосылка}{обучаемость кибернетической системы}
        \end{scnindent}
    \end{scnrelfromset}

    \scnheader{приспособленность кибернетической системы к её совершенствованию, осуществляемому извне}
    \scnidtf{приспособленность кибернетическиой системы к хирургическим методам её совершенствования, реализуемым разработчиками}
    \scnidtf{насколько легко осуществлять обновление, перепроектирование, тестирование, ремонт (исправление ошибок) кибернетической системы}
    \begin{scnrelfromlist}{свойство-предпосылка}

        \scnitem{простота внесения изменений в кибернетическую систему, реализуемых извне}
        \begin{scnindent}
            \scnrelfrom{свойство-предпосылка}{стратифицированность кибернетической системы}
        \end{scnindent}
        \scnitem{многообразие возможных изменений кибернетической системы, реализуемых извне}

    \end{scnrelfromlist}

    \scnheader{производительность кибернетической системы}
    \scnidtf{быстродействие кибернетической системы}
    \scnidtf{интегральная оценка скорости решения задач, время реакции кибернетической системы на задачные ситуации}
    \begin{scnrelfromlist}{частное свойство}

        \scnitem{производительность базового интерпретатора логико-семантической модели кибернетической системы}
        \scnitem{качество используемых кибернетической системой методов и моделей решения задач}

    \end{scnrelfromlist}

    \scnheader{надежность кибернетической системы}
    \scnidtf{способность кибернетической системы при соответствующих условиях ее функционирования сохранять (и, точнее, не снижать) уровень всех свойств и способностей, определяющих общее (комплексное) качество кибернетической системы}
    \begin{scnrelfromlist}{свойство-предпосылка}

        \scnitem{безотказность кибернетической системы}
        \scnitem{долговечность кибернетической системы}
        \scnitem{ремонтопригодность кибернетической системы}
        \begin{scnindent}    
        \scnrelfrom{основной sc-идентификатор}{ремонтопригодность кибернетических систем}
            \begin{scnindent}
                \scntext{примечание}{Здесь слово ремонтопригодность взято в кавычки, т.к. речь идет не только об искусственных (технических) кибернетических системах}
            \end{scnindent}
        \end{scnindent}
    \end{scnrelfromlist}
    \bigskip
\end{scnsubstruct}

		\newpage
\scnsegmentheader{Комплекс свойств, определяющих качество физической оболочки
    \textit{кибернетической системы}}
\begin{scnsubstruct}
    \scnheader{качество физической оболочки кибернетической системы}
    \scnidtf{интегральное качество аппаратной (физической) основы \textit{кибернетической
        системы}}
    \scnidtf{hardware кибернетической системы}
    \begin{scnrelfromlist}{свойство-предпосылка}

        \scnitem{качество памяти кибернетической системы}
        \scnitem{качество процессора кибернетической системы}
        \scnitem{качество сенсоров кибернетической системы}
        \scnitem{качество эффекторов кибернетической системы}
        \scnitem{приспособленность физической оболочки кибернетической системы к ее
            совершенствованию}
        \scnitem{удобство транспортировки кибернетической системы}
        \scnitem{надежность физической оболочки кибернетической системы}

    \end{scnrelfromlist}
    \scnheader{качество памяти кибернетической системы}
    \begin{scnreltolist}{свойство-предпосылка}

        \scnitem{качество информации, хранимой в памяти кибернетической системы}
        \scnitem{качество решателя задач кибернетической системы}

    \end{scnreltolist}
    \begin{scnrelfromlist}{свойство-предпосылка}

        \scnitem{способность памяти кибернетической системы обеспечить хранение
            высококачественной информации}
        \scnitem{способность памяти кибернетической системы обеспечить функционирование
            высококачественного решателя задач}
        \scnitem{объём памяти}

    \end{scnrelfromlist}
    \scnheader{память кибернетической системы}
    \scnidtf{компонент \textit{кибернетической системы}, представляющий собой
        внутреннюю среду \textit{кибернетической системы}, в которой она хранит
        (запоминает) и преобразует \textit{информационную модель} своей \textit{внешней
            среды}. При этом важно, чтобы память обеспечивала высокий уровень
        \textit{гибкости} указанной \textit{информационной модели}. Важно также, чтобы
        эта \textit{информационная модель} была моделью не только \textit{внешней
            среды} \bigskip \textit{кибернетической системы}, но также и моделью самой этой
        \textit{информационной модели} --- описанием её \textit{текущей ситуации},
        предыстории, закономерностей. Таким образом, \textit{кибернетическая система},
        имеющая \textit{память}, функционирует в двух средах --- во внешней, в которой
        существуют и преобразуются внешние(материальные) сущности, и во внутренней, в
        которой существуют и преобразуются(обрабатываются) внутренние
        \textit{информационные конструкции}.}
    \scntext{примечание}{\textit{Кибернетические системы}, находящиеся на низком уровне
        развития(качества) \textit{памяти} не имеют. Адаптационные механизмы такой
        кибернетической системы жестко запаяны в связях между блоками обработчика
        \textit{сигналов} при переходе от \textit{сигналов}, вырабатываемых
        \textit{сенсорами} к \textit{сигналам}, которые управляют
        \textit{эффекторами}.}\scnidtf{внутренняя среда кибернетической системы,
        обеспечивающая хранение и преобразование(обработку) информационной модели
        внешней среды кибернетической системы}
    \scntext{примечание}{Сам факт возникновения памяти в \textit{кибернетической системе}
        является важнейшим этапом её эволюции. Дальнейшее развитие \textit{памяти
            кибернетической системы}, обеспечивающее:\begin{scnitemize}

            \item хранение все более качественной информации, хранимой в памяти
            \item все более качественную организацию обработки этой информации, т.е.
            переход на поддержку(обеспечение) все более качественных моделей обработки
            информации\end{scnitemize}
        является важнейшим фактором эволюции \textit{кибернетических
            систем}.}

    \scnheader{способность памяти кибернетической системы обеспечить хранение высококачественной информации}
    \begin{scnrelfromlist}{свойство-предпосылка}

        \scnitem{способность системы обеспечить компактное хранение
            сложноструктурированных баз знаний}
            \begin{scnindent}
                \scntext{примечание}{Здесь имеется в виду необходимость перехода от
                    линейной организации, памяти на физическом уровне (как последовательности ячеек
                    памяти) к нелинейной, графодинамической памяти.}
            \end{scnindent}
        \scnitem{способность памяти кибернетической системы обеспечить хранение
            широкого многообразия знаний}
            \begin{scnindent}
                \scntext{примечание}{имеется в виду хранение гибридных баз знаний}
            \end{scnindent}
    \end{scnrelfromlist}

    \scnheader{способность памяти кибернетической системы обеспечить
        функционирование высококачественного решателя задач}
    \begin{scnrelfromlist}{свойство-предпосылка}

        \scnitem{качество доступа к информации, хранимой памяти кибернетической системы}
        \begin{scnindent}
            \scntext{примечание}{Здесь имеется в виду необходимость перехода от адресного к
                ассоциативному доступу, причем, с расширением многообразия видов реализуемых
                запросов, в частности, к реализации запросов фрагментов баз знаний по заданному
                образцу произвольного размера и произвольной конфигурации.}
        \end{scnindent}
        \scnitem{логико-семантическая гибкость памяти кибернетической системы}
        \scnitem{способность памяти кибернетической системы обеспечить интерпретацию
            широкого многообразия моделей решения задач}

    \end{scnrelfromlist}

    \scnheader{логико-семантическая гибкость памяти кибернетической системы}
    \scnidtf{степень близости физической организации памяти кибернетической системы
        к реализуемым ею базовым семантически целостным действиям над информацией,
        хранимой в памяти}
    \scnidtf{простота реализации базовых семантически целостных действий над
        информацией, хранимой в памяти кибернетической системы}
    \scntext{примечание}{Важен переход от мелких действий, к элементарным действиям,
        имеющим логико-семантический смысл (целостность, законченность)}
    
    \scnheader{качество процессора кибернетической системы}
    \scnrelto{свойство-предпосылка}{качество решателя задач кибернетической
        системы}
    \begin{scnrelfromlist}{свойство-предпосылка}
        \scnitem{способность процессора кибернетической системы обеспечить функционирования высококачественного решателя задач}
    \end{scnrelfromlist}
    \begin{scnrelfromlist}{свойство-предпосылка}
        \scnitem{многообразие моделей решения задач, интерпретируемых процессором
            кибернетической системы}
        \scnitem{простота и качество интерпретации процессором системы широкого
            многообразия моделей решения задач}
        \begin{scnindent}
            \scntext{примечание}{Указанная простота определяется степенью близости
                интерпретируемых моделей решения задач к физическому уровню организации
                процессора кибернетической системы.}
        \end{scnindent}
        \scnitem{обеспечение процессором кибернетической системы качественного
            управления информационными процессами в памяти}
        \begin{scnindent}
            \scntext{примечание}{Речь идет о грамотном сочетание таких аспектов управление
                процессами, как централизация и децентрализация, синхронность и асинхронность,
                последовательность и параллельность.}\scnrelfrom{свойство-предпосылка}{уровень
                параллелизма обработки информации в памяти кибернетической системы}
            \scnidtf{максимальное количество одновременно выполняемых информационных
                процессов в памяти кибернетической системы}
        \end{scnindent}
        \scnitem{быстродействие процессора кибернетической системы}
    \end{scnrelfromlist}

    \scnheader{многообразие моделей решения задач, интерпретируемых
        процессором кибернетической системы}
    \scntext{примечание}{Максимальным уровнем качества процессора кибернетической системы
        по данном параметру является его универсальность, т.е. его принципиальная
        возможность интерпретировать любую модель решения как интеллектуальных, так и
        неинтеллектуальных задач (алгоритмизацию, процедурную параллельную синхронную,
        процедруную параллельную асинхронную, продукционную, нейросетевую,
        генетическую, функциональную, целое семейство моделей). Простота определяется степенью близости интерпретируемых моделей решения задач к
        \scnqq{физическому} уровню организации процессора кибернетической системы. Качественное управление информационными
        процессами в памяти подразумевает грамотное сочетание таких аспектов управление процессами, как
        централизация и децентрализация, синхронность и асинхронность, последовательность и
        параллельность}
        \begin{scnindent}
            \begin{scnrelfromlist}{источник}
                \scnitem{\scncite{Melekhova2018}}
            \end{scnrelfromlist}
        \end{scnindent}
        
    \scnheader{качество сенсоров кибернетической системы}
    \scnrelfrom{свойство-предпосылка}{многообразие видов сенсоров кибернетической системы}
    \begin{scnindent}    
        \scnidtf{многообразие средств восприятия (отображения) информации о текущем
                состоянии внешней среды кибернетической системы и её собственной физической
                оболочки}
    \end{scnindent}
    

    \scnheader{качество эффекторов кибернетической системы}
    \scnrelfrom{свойство-предпосылка}{многообразие видов эффекторов кибернетической системы}
    \begin{scnindent}    
        \scnidtf{многообразие средств воздействия на собственную физическую оболочку
            кибернетической системы и через нее на внешнюю среду этой системы}
        \scntext{примечание}{Эффекторы кибернетической системы являются инструментами
            воздействия кибернетической системы на свою внешнюю среду.}
    \end{scnindent}

    \scnheader{приспособленность физической оболочки кибернетической системы к её
        совершенствованию}
    \scnidtf{приспособленность кибернетической системы к повышению качества её
        физической оболочки}
    \scnidtf{простота ремонта и совершенствования таких компонентов кибернетической
        системы как память, процессор, сенсоры, эффекторы}
    \scnrelfrom{частное свойство}{ремонтопригодность физической оболочки
        кибернетической системы}
    \begin{scnrelfromset}{группа свойств-предпосылок}

        \scnitem{гибкость физической оболочки кибернетической системы}
        \scnitem{стратифицированность физической оболочки кибернетической системы}
        \begin{scnindent}    
            \scnidtf{мобильность физической оболочки кибернетической системы}
            \scnidtf{легкость сохранения целостности физической оболочки кибернетической
                системы при внесении различных изменений (локализация области учета последствий
                внесения изменений, предсказуемость последствий)}
        \end{scnindent}

    \end{scnrelfromset}
    \bigskip
\end{scnsubstruct}
\scnsourcecomment{Завершили Сегмент \scnqqi{Комплекс свойств, определяющих качество физической оболочки кибернетической системы}}

		\scnsegmentheader{Комплекс свойств, определяющих уровень интеллекта
    кибернетической системы}
\begin{scnsubstruct}
    \scnheader{интеллект}
    \scniselement{свойство}
    \scniselement{упорядоченное свойство}
    \scnidtf{уровень (степень, величина) интеллекта кибернетической системы}
    \scnidtf{Семейство классов \textit{кибернетических систем}, обладающих
        эквивалентным (одинаковым) уровнем интеллекта --- от низкого до высокого уровня
        интеллекта}
    \scnidtf{свойство кибернетических систем, характеризующее эффективность их
        взаимодействия со своей средой (средой их жизнедеятельности)}
    \scnrelfrom{область определения}{кибернетическая система}
    \scntext{пояснение}{С формальной точки зрения интеллектуальность --
        это семейство классов кибернетических систем, в каждый из которых входят
        кибернетические системы, эквивалентные по уровню и характеру проявления
        интеллектуальных свойств (в том числе способностей).
        \\Таким образом, характер (вид) интеллектуальных свойств кибернетических
        систем и уровень их развития для разных кибернетических систем может быть
        разным. В соответствии с этим кибернетические системы можно сравнивать между
        собой.}\scntext{примечание}{Основным свойством (характеристикой, качеством,
        параметром) кибернетической системы является уровень (степень) ее интеллекта,
        который является \uline{интегральной} характеристикой, определяющей уровень
        эффективности взаимодействия кибернетической системы со средой своего
        существования.}
        \begin{scnindent}
            \scntext{источник}{\scncite{Zagorskiy2022b}}
        \end{scnindent}
    \scnidtf{комплексное свойство (качество) кибернетической
        системы, определяющее уровень ее выживаемости во внешней среде и предполагающее
        возможность воздействия на эту среду и даже возможность ее преобразования}
    \scnidtf{интеллектуальный потенциал кибернетической системы}
    \scnidtf{спектр знаний, навыков и способностей к обучению кибернетической
        системы}
    \scnidtf{интеллектуальность кибернетической системы}
    \scntext{примечание}{Процесс эволюции \textit{кибернетических систем} следует
        рассматривать как процесс повышения уровня их качества по целому ряду свойств
        (характеристик) и, в первую очередь, как процесс повышения уровня их
        \textit{интеллекта}. При этом можно говорить об эволюции каждой конкретной
        \textit{кибернетической системы} в процессе своей жизнедеятельности, а также об
        эволюции целого класса \textit{кибернетических систем}, когда новые экземпляры
        этого класса являются более интеллектуальными, чем их предшественники. В таком
        аспекте, в частности, можно рассматривать эволюцию \textit{компьютерных систем}
        (искусственных кибернетических систем).}\scntext{примечание}{Очень важно уточнить,
        какими иными свойствами \textit{кибернетических систем} определяется уровень и
        характер их интеллектуальности. Подчеркнем, что \uline{любая}
        \textit{кибернетическая система} обладает соответствующим уровнем
        интеллектуальности. Пусть даже и достаточно низким. Существенным является
        уточнение того, за счет чего уровень интеллектуальности \textit{кибернетической
            системы} может быть повышен. Нет смысла проводить четкую границу между
        \textit{интеллектуальными кибернетическими системами} и неинтеллектуальными. Но
        есть смысл уточнять направления повышения уровня интеллектуальности
        \textit{кибернетических систем.}}\scntext{эпиграф}{Никто не может провести
        линию, отделяющую атмосферу от космоса, или черту, за которой начинается жизнь,
        или границу электронного облака. Все дело в степени проявления свойства.}
    \begin{scnindent}
        \scnrelfrom{автор}{Барт Коско}
    \end{scnindent}
    \scntext{примечание}{Исследователи Искусственного интеллекта определяют интеллект как неотъемлемое свойство машины.
        Их целью является построение систем, которые демонстрируют весь спектр когнитивных способностей, которые мы обнаруживаем у людей}
        \begin{scnindent}
            \begin{scnrelfromlist}{источник}
                \scnitem{\scncite{Gao2002}}
                \scnitem{\scncite{Laird2009}}
            \end{scnrelfromlist}
        \end{scnindent}
    \scntext{примечание}{Прежде, чем говорить о требованиях, предъявляемых к
        \textit{технологии проектирования и производства интеллектуальных компьютерных
            систем (искусственных кибернетических систем}, обладающих высоким уровнем
        \textit{интеллекта)}, необходимо уточнить (детализировать) \textit{свойства},
        присущие указанным системам и являющиеся предпосылками, обеспечивающими высокий
        уровень \textit{интеллекта}. Подчеркнем, что указанные \textit{свойства},
        уточняющие (детализирующие, обеспечивающие, определяющие) \textit{свойства}
        %\bigspace
        \textit{интеллектуальных систем}
        %\bigspace
        (\textit{свойства}, определяющие уровень \textit{интеллекта} этих систем)
        должны быть общими как для искусственных кибернетических систем
        (\textit{компьютерных систем}), так и для \textit{естественных кибернетических
            систем.}}\scnidtf{интегральное качество информационного обеспечения и
        информационных процессов в кибернетической системе}
    \scnidtf{интегральное качество кибернетической системы,
        определяемое:\begin{scnitemize}
            \item уровнем ее образованности --- качеством накопленных к заданному моменту
            знаний и умений (навыков);
            \item уровнем ее обучаемости --- способностью \uline{самостоятельно} повышать
            уровень своей образованности.\end{scnitemize}
    }

    \scnheader{уровень интеллекта кибернетической системы}
    \begin{scnrelfromlist}{свойство-предпосылка}
        \scnitem{образованность кибернетической системы}
        \scnitem{обучаемость кибернетической системы}
        \scnitem{интероперабельность кибернетической системы}
        \begin{scnindent}    
            \scntext{примечание}{Интеллект \textit{кибернетической системы}, как и лежащий в
                его основе познавательный процесс, выполняемый кибернетической системой, имеет
                социальный характер, поскольку наиболее эффективно формируется и развивается в
                форме взаимодействия \textit{кибернетической} системы с другими
                \textit{кибернетическими системами}.}
        \end{scnindent}
    \end{scnrelfromlist}

    \scnheader{образованность кибернетической системы}
    \scnidtf{уровень навыков (умений), а также иных знаний, приобретенных \textit{кибернетической системой} к заданному моменту}
    \begin{scnrelfromlist}{свойство-предпосылка}
    \scnitem{\textbf{качество навыков, приобретенных кибернетической системой}}
        \begin{scnindent}
            \scnidtf{качество умений, которыми владеет кибернетическая система в
                текущий момент}
            \scnrelfrom{свойство-предпосылка}{\textbf{качество информации, хранимой в памяти кибернетической системы}}
                \begin{scnindent}
                    \scnidtf{качество знаний, приобретенных кибернетической системой к заданному моменту}
                \end{scnindent}
        \end{scnindent}
    \scnitem{\textbf{качество информации, хранимой в памяти кибернетической системы}}
        \begin{scnindent}
            \scntext{примечание}{Следует обратить внимание на то, что \textit{качество 
                информации, хранимой в памяти кибернетической системы}, является фактором,
                обеспечивающим не только \textit{качество навыков, приобретенных
                кибернетической системой}, но и общий \textit{уровень качества кибернетической системы}.}
        \end{scnindent}
    \end{scnrelfromlist}

    \scnheader{кибернетическая система}
    \scnrelto{объединение}{Признак интеллектуальности кибернетических систем}
    \begin{scnindent}
        \begin{scneqtoset}
            \scnitem{неинтеллектуальная кибернетическая система}
            \scnitem{интеллектуальная система}
                \begin{scnindent}
                \scnidtf{интеллектуальная кибернетическая система}
                \begin{scnreltoset}{объединение}
                    \scnitem{слабоинтеллектуальная система}
                        \begin{scnindent}
                            \scnidtf{кибернетическая система со слабым интеллектом}
                            \scnidtf{кибернетическая система с низким уровнем интеллекта}
                            \scnidtf{кибернетическая система с элементами интеллекта}
                        \end{scnindent}
                    \scnitem{высокоинтеллектуальная система}
                        \begin{scnindent}
                            \scnidtf{идеальная интеллектуальная система}
                            \scnidtf{кибернетическая система с сильным интеллектом}
                            \scnidtf{кибернетическая система с высоким уровнем интеллекта}
                            \scnidtf{действительно интеллектуальная система}
                        \end{scnindent}
                \end{scnreltoset}
            \end{scnindent}
        \end{scneqtoset}
    \end{scnindent}

    \scnheader{Признак интеллектуальности кибернетических систем}
    \scntext{примечание}{Данный признак классификации кибернетических систем формально
        является не разбиением, а покрытием множества \textit{кибернетических систем},
        так как отсутствует четкая грань между неинтеллектуальными и интеллектуальными
        кибернетическими системами, а также между слабоинтеллектуальными и
        высокоинтеллектуальными кибернетическими системами.}
        
    \scnheader{интеллектуальная система}
    \scnidtf{интеллектуальная кибернетическая система}
    \scntext{примечание}{В этом термине слово кибернетическая можно опустить, так как
        интеллектуальными могут быть только \textit{кибернетические системы}}
    \scntext{примечание}{Интеллектуальные кибернетические системы могут быть
        \textit{естественными интеллектуальными системами}, искусственными
        интеллектуальными системами (которые будем называть \textit{интеллектуальными
        компьютерными системами}), а также естественно-искусственными интеллектуальными
        системами, состоящими из компонентов как естественного, так и искусственного
        происхождения. Важнейшим примером естественно-искусственных интеллектуальных
        систем являются человеко-машинные системы, представляющие собой коллективы
        (многоагентные системы), состоящие из \textit{интеллектуальных компьютерных
        систем} и людей (конечных пользователей и разработчиков этих компьютерных систем).}
    \scntext{примечание}{Вводя понятие \textit{интеллектуальной
        системы}, важно, во-первых, уточнить понятие \textit{кибернетической системы} и
        определить те свойства, которые присущи \uline{всем} кибернетическим системам,
        и, во-вторых, локализовать ту условную \uline{грань} перехода от
        неинтеллектуальных \textit{кибернетических систем} к интеллектуальным, а также
        \uline{грань} перехода от слабоинтеллектуальных к высокоинтеллектуальным
        кибернетическим системам. В этом и заключается уточнение феномена
        \textit{интеллекта} (интеллектуальности) кибернетических систем.}
    \scntext{примечание}{Все \textit{свойства} (в том числе способности и
        активности), присущие \textit{кибернетическим системам}, в различных
        \textit{кибернетических системах} могут иметь самый различный уровень (уровень
        развития). Более того, в некоторых \textit{кибернетических системах} некоторые
        из этих свойств могут вообще отсутствовать. При этом в кибернетических
        системах, которые условно будем называть \textit{\textbf{интеллектуальными
        системами}}, \uline{все} указанные выше свойства должны быть представлены в
        достаточно развитом виде. Заметим также, что мы называем
        \textit{интеллектуальными системами}, иногда называют кибернетическими
        системами с сильным интеллектом (с высоким уровнем интеллекта),
        противопоставляя их кибернетическим системам со слабым интеллектом (с низким
        уровнем интеллекта).}
    \scnsubset{образованная кибернетическая система}
    \begin{scnindent}
        \scnidtf{кибернетическая система, имеющая высокий уровень образованности}
        \scnidtf{кибернетическая система, обладающая высоким уровнем знаний и навыков}
        \scnsubset{кибернетическая система, основанная на знаниях}
        \scnsubset{кибернетическая система, управляемая знаниями}
        \scnsubset{целенаправленная кибернетическая система}
        \scnsubset{гибридная кибернетическая система}
        \scnsubset{потенциально универсальная кибернетическая система}
    \end{scnindent}
    \scnsubset{обучаемая кибернетическая система}
    \begin{scnindent}
        \scnidtf{когнитивная кибернетическая система}
        \scnidtf{кибернетическая система, имеющая высокий уровень обучаемости}
        \scnsubset{кибернетическая система с высоким уровнем стратифицированности своих знаний и навыков}
        \scnsubset{рефлексивная кибернетическая система}
        \scnsubset{самообучаемая кибернетическая система}
        \scnsubset{кибернетическая система с высоким уровнем познавательной активности}
    \end{scnindent}
    \scnsubset{социально ориентированная кибернетическая система}
    \begin{scnindent}
        \scnidtf{кибернетическая система, имеющая высокий уровень интероперабельности}
        \scnsubset{кибернетическая система, способная устанавливать и поддерживать
            высокий уровень семантической совместимости и взаимопонимания с другими системами}
        \scnsubset{договороспособная кибернетическая система}
        \begin{scnindent}
            \scnidtf{кибернетическая система, способная координировать (согласовывать) свою
                деятельность с другими системами}
        \end{scnindent}
    \end{scnindent}

    \scnheader{кибернетическая система, основанная на знаниях}
    \scnidtf{кибернетическая система, в основе которой лежит формируемая в ее
        памяти, постоянно совершенствуемая и структурированная информационная модель
        той среды, в рамках которой она существует и решает соответствующие задачи}
    \scnidtf{кибернетическая система, в основе которой лежит ее база знаний --
        систематизированная совокупность всех используемых ею знаний}
    \scnidtf{кибернетическая система, формирующая в своей памяти
        систематизированную информационную модель среды своего обитания и использующая
        эту модель для организации своего целенаправленного поведения}

    \scnheader{кибернетическая система, управляемая знаниями}
    \scnidtf{кибернетическая система, в которой выполняемые ею действия
        инициируются соответствующими ситуациями и/или событиями, возникающими в ее
        базе знаний}

    \scnheader{целенаправленная кибернетическая система}
    \scnidtf{субъект, осознанно и целенаправленно осуществляющий свою деятельность,
        ведающий то, что он творит}

    \scnheader{обучаемая кибернетическая система}
    \scnidtf{когнитивная система}
    \scnidtf{кибернетическая система, способная познавать (изучать) среду своего
        обитания, то есть строить и постоянно уточнять в своей памяти информационную
        модель (описание) этой среды, а также использовать эту модель для решения
        различных задач (для организации своей деятельности (поведения)) в указанной
        среде}
    \scnidtf{кибернетическая система, способная к самосовершенствованию}

    \scnheader{социально ориентированная кибернетическая система}
    \scnidtf{кибернетическая система, имеющая достаточно высокий уровень
        интеллекта, чтобы быть полезным членом различных, в том числе, и
        человеко-машинных сообществ}
    \scntext{примечание}{Определенный уровень социально значимых качеств является
        необходимым условием интеллектуальности кибернетической системы. Это, своего
        рода, модификация теста Тьюринга --- важна не имитация, не иллюзия
        человекоподобия, а \uline{реальная} польза в процессе коллективного решения
        сложных задач.}
        
    \scnheader{интеллектуальная компьютерная система}
    \scnidtf{искусственная интеллектуальная система}
    \scnidtf{искусственная кибернетическая система, обладающая высоким уровнем
        интеллекта (высоким уровнем знаний и умений), а также высоким уровнем обучаемости}
    \scnsubset{компьютерная система}
    \begin{scnindent}
        \scnsubset{кибернетическая система}
    \end{scnindent}
    \scnidtftext{основной sc-идентификатор}{интеллектуальная компьютерная система}
    \begin{scnindent}
        \scntext{сокращение}{и.к.с.}
    \end{scnindent}
    \scnidtf{система искусственного интеллекта}
    \scnidtf{искусственная интеллектуальная система}
    \scntext{примечание}{Все свойства, присущие кибернетическим системам, в различных кибернетических системах могут иметь самый
    различный уровень. Более того, в некоторых кибернетических системах некоторые из этих свойств могут вообще
    отсутствовать. При этом в кибернетических системах, которые условно будем называть интеллектуальными
    системами, все указанные выше свойства должны быть представлены в достаточно развитом виде.}
    \scnsubset{интеллектуальная система}
    \begin{scnindent}
        \scnsubset{кибернетическая система}
    \end{scnindent}
    \bigskip
\end{scnsubstruct}
\scnsourcecomment{Завершили Сегмент \scnqqi{Комплекс свойств, определяющих уровень интеллекта кибернетической системы}}

		\scnsegmentheader{Комплекс свойств, определяющих качество информации, хранимой
    в памяти кибернетической системы}
\begin{scnsubstruct}
    \scnheader{информация}
    \scnidtf{информационная конструкция}
    \scnidtf{информационная модель, состоящая из некоторого множества различных
        \textit{знаков}, обозначающих моделируемые (описываемые) \textit{сущности}
        любого вида и, в частности, \textit{знаков}, обозначающих различного вида
        \textit{связи} между \textit{знаками} описываемых \textit{сущностей} (такие
        \textit{связи} чаще всего являются отражениями (моделями) \textit{связей} между
        \textit{сущностями}, которые обозначаются связываемыми \textit{знаками})}
    \begin{scnindent}
        \scntext{примечание}{Подчеркнем, что \textit{связи} между \textit{знаками}
            описываемых \textit{сущностей} сами также могут быть описываемыми
            \textit{сущностями}, но для этого указанные \textit{связи} в рамках
            информационной модели должны быть представлены своими \textit{знаками}. Не все
            \textit{связи} между \textit{знаками} являются описываемыми
            \textit{сущностями}. Такими неописываемыми связями являются связи инцидентности
            знаков.}
    \end{scnindent}
    \scnidtf{конфигурация знаков}
    \scnidtf{знаковая конструкция}
    \scnidtf{текст}
    \scnidtf{описание (отражение) некоторого множества (1) первичных сущностей, (2)
        понятий, (3) связей между ними, (4) связей между связями, (5) фрагментов
        данного описания, (6) связей между этими фрагментами}
    \scnsuperset{дискретная информационная конструкция}
    \begin{scnindent}
        \scnidtf{информационная конструкция, у которой все входящие в неё знаки имеют
            чёткие границы}
        \scnsuperset{дискретная информационная конструкция, у которой входящие в неё
            знаки имеют \uline{условную} структуру}
    \end{scnindent}
    \scnidtf{информационная модель}
    \scnidtf{информационная модель (отражение, описание) некоторого множества
        связей между некоторым описываемыми (рассматриваемыми, исследуемыми,
        изучаемыми) сущностями}
    \scntext{определение}{Множество всевозможных информационных конструкций
        (понятие информационной конструкции) представляет собой множество, на котором
        задано

        \begin{scnitemize}

            \item Отношение \uline{синтаксической} эквивалентности и, соответственно,
            семейство классов синтаксической эквивалентности информационных конструкций

            \item Отношение \uline{семантической} эквивалентности и, ответственно,
            семейство классов семантической эквивалентности информационных конструкций

            \item Отношение \uline{логической} эквивалентности и, соответственно, семейство
            классов логической эквивалентности информационных конструкций.

        \end{scnitemize}
        При этом можно говорить об инварианте каждого класса синтаксически
        эквивалентных информационных конструкций, об инварианте каждого класса
        семантически эквивалентных информационных конструкций и об инварианте каждого
        класса логически эквивалентных информационных конструкций синтаксически
        эквивалентные информационные конструкции могут отличаться вариантами
        изображения букв (различным почерком, разными шрифтами), вариантами разрезания
        текста на страницы и на строчки.Семантически эквивалентные информационные
        конструкции могут отличаться разными именами, обозначающими одни и те же
        сущности, разным порядком размещения этих имён.}

    \scnheader{денотационная семантика информационной конструкции}
    \scntext{пояснение}{Каждая информационная конструкция имеет денотационную
        семантику, описывающую то, как связаны входящие в информационную конструкцию
        знаки с соответствующими им денотатами (т.е. сущностями, обозначаемыми этими
        знаками).}
        
    \scnheader{сенсорная информация}
    \scnsubset{информация}
    \scnidtf{первичная информация, приобретаемая кибернетической системы с помощью
        её сенсоров (рецепторов)}
    \scnidtf{первичная информация}
    \scntext{примечание}{Подчеркнем, что \textit{сенсорная информация}
        %\bigspace
        \textit{кибернетической системы} с точки зрения её \textit{денотационной
            семантики} является простейшим видом \textit{знаковой конструкции}, в которой
        \textit{внешняя среда}
        %\bigspace
        \textit{кибернетической системы} описывается

        \begin{scnitemize}

            \item путём задания параметрического пространства (множество параметров,
            признаков, \textit{свойства}, характеристик), с помощью которого описываются
            состояние элементарных (атомарных) фрагментов \textit{внешней среды}, которые
            непосредственно являются смежными (соприкасаются с) чувствительными
            поверхностями \textit{сенсоров кибернетической системы};
            \item путём пространственной декомпозиции наблюдаемой \textit{внешней среды} с
            выделением указанных выше элементарных фрагментов этой среды (элементарных с
            точки зрения  \textit{сенсоров кибернетической системы}) и с явным описанием
            пространственных связей между указанными элементарными фрагментами (эти связи
            соответствует пространственным связям между сенсорами);
            \item путём темпоральной декомпозиции наблюдаемой \textit{внешней среды},
            которая предполагает фиксацию моментов времени для каждого события по изменению
            состояния измеряемого параметра каждого элементарного фрагмента наблюдаемой
            \textit{внешней среды}
        \end{scnitemize}
    }
    \scntext{примечание}{Качество (в частности, информативность) \textit{сенсорной
            информации} обеспечивается:
        \begin{scnitemize}

            \item качеством используемого параметрического пространства
            \begin{scnitemizeii}

                \item многообразием видов \textit{сенсоров}, т.е. многообразием параметров
                (свойств), с помощью которых описывается внешняя среда

                \item информативностью каждого из указанных параметров

                \item целостностью (полнотой, достаточностью) всего набора рассматриваемых
                параметров

                \item отсутствием избыточности в наборе этих параметров
            \end{scnitemizeii}

            \item общим количеством сенсоров и количеством сенсоров, соответствующих
            каждому параметру

            \item способностью кибернетической системы перемещать сенсоры в пространстве

        \end{scnitemize}
    }\scntext{примечание}{\textit{сенсорная информация} обеспечивает формирование
        первичного описания состояния и динамики изменения не только \textit{внешней
            среды кибернетической системы}, но также и её физической оболочки, которую
        можно рассматривать как часть всей \textbf{\textit{физической среды
                кибернетической системы}}, противопоставляя такую \textit{физическую среду
            кибернетической системы} её внутренней (информационной, \uline{абстрактной})
        среде, в которой хранится и обрабатывается \textit{информация}, используемая
        \textit{кибернетической системой}. Указанную абстрактную внутреннюю среду
        кибернетической системы будем называть \textbf{\textit{абстрактной памятью
                кибернетической системы}}.}
                
    \scnheader{язык}
    \scnidtf{множество информационных конструкции, построенных по общим
        синтаксическим и семантическим правилам}
    \scnsuperset{внутренний язык кибернетической системы}
    \begin{scnindent}
        \scnidtf{язык, используемый кибернетической системой для представления
            информации, хранимой в её памяти}
    \end{scnindent}

    \scnheader{информация, хранимая в памяти кибернетической системы}
    \scnidtf{совокупность \uline{всей} информации, хранимой в памяти
        кибернетической системы}
    \scnsubset{информация}

    \scnheader{качество информации, хранимой в памяти кибернетической системы}
    \scnidtf{качество знаний, приобретенных кибернетической системой к текущему
        моменту}
    \scnidtf{уровень качества хранимой информации}
    \scnidtf{качество информационной модели среды кибернетической системы, хранимой
        в её памяти}
    \scnidtf{уровень качества хранимых в памяти кибернетической системы внутренней
        информационной модели среды существования (жизнедеятельности) этой
        кибернетической системы}
    \scnidtf{интегральное качество знаний, накопленных кибернетической системой к
        текущему моменту}
    \scnidtf{степень приближения информации, хранимой в памяти кибернетической
        системы к качественной информационной модели той среды, в которой существует
        кибернетическая система, к систематизированной базе знаний, описывающей все
        свойства этой среды, необходимые для функционирования этой кибернетической
        системы}
    \scnidtf{качество хранимой в памяти кибернетической системы информационной
        модели среды жизнедеятельности этой системы}
    \scntext{примечание}{Качество информационной модели среды обитания  кибернетической
        системы, в частности, определяется
        \begin{scnitemize}

            \item корректностью этой модели (отсутствием в ней ошибок);

            \item адекватностью этой модели;

            \item полнотой --- достаточностью находящейся в ней информации для эффективного
            функционирования кибернетической системы;

            \item структурированностью, систематизированностью.

        \end{scnitemize}
        Важнейшим этапом эволюции информационной модели среды кибернетической системы
        является переход от недостаточно полной и несистематизированной информационные
        модели среды к \textit{базе знаний}. Именно поэтому важнейшим этапом повышения
        уровня интеллектуальности компьютерной систем является переход от традиционных
        компьютерных систем к компьютерным системам, основанным на знаниях.}
    \scnrelfrom{комплекс свойств-предпосылок}{не-фактор}
    \begin{scnrelfromlist}{свойство-предпосылка}

        \scnitem{семантическая мощность языка представления информации в памяти
            кибернетической системы}
        \scnitem{ объём информации, загруженной в память кибернетической системы}
        \scnitem{ степень конвергенции и интеграции различного вида знаний, хранимых в
            памяти кибернетической системы}
        \scnitem{ стратифицированность информации, хранимой в памяти кибернетической
            системы}
        \scnitem{простота и локальность выполнения семантически целостных операций над
            информацией, хранимой в памяти кибернетической системы}

    \end{scnrelfromlist}

    \scnheader{не-фактор}
    \scnidtf{группа семантических свойств, определяющих качество информации,
        хранимой в памяти кибернетической системы}
    \begin{scneqtoset}

        \scnitem{корректность/некорректность информации, хранимой в памяти
            кибернетической системы}
        \scnitem{однозначность/неоднозначность информации, хранимой в памяти
            кибернетической системы}
        \scnitem{целостность/нецелостность информации, хранимой в памяти
            кибернетической системы}
        \scnitem{чистота/загрязненность информации, хранимой в памяти кибернетической
            системы}
        \scnitem{достоверность/недостоверность информации, хранимой в памяти
            кибернетической системы}
        \scnitem{точность/неточность информации, хранимой в памяти кибернетической
            системы}
        \scnitem{четкость/нечеткость информации, хранимой в памяти кибернетической
            системы}
        \scnitem{определенность/недоопределенность информации, хранимой в памяти
            кибернетической системы}

    \end{scneqtoset}
    \scntext{пояснение}{Ярушкина.Н.Г.НечетГС-2007кн.-стр.10-28}
    \begin{scnindent}
        \scnrelto{цитата}{\cite{YarushinaHS}}
    \end{scnindent}

    \scnheader{корректность/некорректность информации, хранимой в памяти
        кибернетической системы}
    \scnidtf{уровень адекватности хранимой информации той среде, в которой
        существует кибернетическая система и информационной моделью которой эта
        хранимая информация является}

    \scnheader{непротиворечивость/противоречивость информации, хранимой в памяти
        кибернетической системы}
    \scnidtf{уровень присутствия в хранимой информации различного вида противоречий
        и, в частности, ошибок}

    \scnheader{противоречие*}
    \scnidtf{пара противоречащих друг другу фрагментов информации, хранимой в
        памяти кибернетической системы*}
    \scntext{примечание}{Чаще всего противоречащими друг другу информационными
        фрагментами являются:

        \begin{scnitemize}

            \item явно представленная в памяти некоторая закономерность (некоторое правило)

            \item информационный фрагмент, не соответствующий (противоречащий) указанной
            закономерности

        \end{scnitemize}
        \bigskip
        В этом случае некорректность может присутствовать:

        \begin{scnitemize}

            \item либо в информационном фрагменте, который противоречит указанной
            закономерности;

            \item либо в самой этой закономерности;

            \item либо и там и там.

        \end{scnitemize}
    }
    \scnheader{информационная ошибка}
    \scnidtftext{определение}{противоречие, заключающееся в нарушении некоторой
        закономерности (некоторого правила), которая не подвергается сомнению}
    
    \scnheader{информационная ошибка}
    \scntext{примечание}{Ошибки (ошибочные фрагменты) в хранимой информации могут быть
        синтаксическими и семантическими, противоречащими некоторым правилам
        (закономерностям), которые явно в памяти могут быть не представлены и считаются
        априори истинными.}
        
    \scnheader{полнота/неполнота информации, хранимой в памяти кибернетической системы}
    \scnidtf{уровень того, насколько информация, хранимая в памяти кибернетической
        системы, описывает среду существования этой системы и используемые ею методы
        решения задач достаточно полно (достаточно детально) для того, чтобы
        кибернетическая система могла действительно решать все множество
        соответствующих ей задач}
    \scnidtf{уровень соответствия хранимой информации объёму задач (действий),
        которые соответствующая кибернетическая система желает уметь решать
        (выполнять)}
    \scnidtf{степень достаточности информации, хранимой в памяти кибернетической
        системы, для достижения целей этой системы, для выполнения своих обязанностей}
    \scntext{примечание}{Чем полнее информация, хранимая в памяти кибернетической
        системы, чем полнее \uline{информационное обеспечение деятельности этой
            системы} это системы, тем эффективнее (качественнее) сама эта
        деятельность.}\begin{scnrelfromlist}{свойство-предпосылка}

        \scnitem{многообразие видов знаний, хранимых в памяти кибернетической системы}
        \scnitem{структурированность информации, хранимой в памяти кибернетической
            системы}

    \end{scnrelfromlist}

    \scnheader{однозначность/неоднозначность информации, хранимой в кибернетической
        системе}
    \begin{scnrelfromlist}{свойство-предпосылка}

        \scnitem{многообразие форм дублирования информации, хранимой в памяти
            кибернетической системы}
        \scnitem{частота дублирования информации, хранимой в памяти кибернетической
            системы}

    \end{scnrelfromlist}

    \scnheader{целостность/нецелостность информации, хранимой в памяти
        кибернетической системы}
    \scnidtf{уровень содержательной информативности информации, хранимой в памяти
        кибернетической системы}
    \scnidtf{уровень того, насколько содержательно (семантически) \uline{связной}
        является информация, хранимая в памяти кибернетической системы, насколько полно
        специфицированы \uline{все} описываемые в памяти сущности (путём описания
        необходимого набора связей этих сущностей с другими описываемыми сущностями),
        насколько редко или часто в рамках хранимой информации встречаются
        \textit{информационные дыры}, соответствующие явной недостаточности некоторых
        спецификаций}
    \scnidtf{известность/неизвестность информации, хранимой в памяти
        кибернетической системы}
    \scnidtf{многообразие форм и частота присутствия \textit{информационных дыр} в
        информации, хранимой в памяти кибернетической системы}

    \scnheader{информационная дыра в информации, хранимой в памяти кибернетической
        системы}
    \scnidtf{информация, отсутствие которой в памяти кибернетической системы
        существенно усложняет деятельность этой системы}
    \scntext{примечание}{Примерами информационных дыр являются:
        \begin{scnitemize}
            \item отсутствующий метод решения часто встречающихся задач;
            \item отсутствующее определение используемого определяемого понятия;
            \item недостаточно подробная спецификация часто рассматриваемых сущностей
        \end{scnitemize}
    }
    
    \scnheader{чистота/загрязненность информации, хранимой в памяти
        кибернетической системы}
    \scnidtf{многообразие форм и общее количество информационного мусора, входящего
        в состав информации, хранимой в памяти кибернетической системы}
    
    \scnheader{информационный мусор, входящий в состав информации, хранимой в
        памяти кибернетической системы}
    \scnidtf{информационный фрагмент, входящий в состав информации, хранимой в
        памяти кибернетической системы, удаление которого существенно \uline{не}
        усложнит деятельность кибернетической системы}
    \scntext{примечание}{Примерами информационного мусора являются:
        \begin{scnitemize}
            \item информация, которая нечасто востребована, но при необходимости может быть легко логически выведена
            \item информация, актуальность которой истекла
        \end{scnitemize}
    }
    
    \scnheader{семантическая мощность языка представления информации в памяти
        кибернетической системы}
    \scnidtf{семантическая мощность внутреннего языка кибернетической системы}
    \scnrelfrom{свойство-предпосылка}{гибридность информации, хранимой в памяти
        кибернетической системы}
    \newpage\scntext{примечание}{Универсальность внутреннего языка кибернетической
        системы является важнейшим фактором её
        интеллектуальности}
        
    \scnheader{универсальный язык}
    \scnidtf{язык, информационные конструкции которого могут представить (описать)
        \uline{любую} конфигурацию \uline{любых} связей между \uline{любыми}
        сущностями}
    
    \scnheader{гибридность информации, хранимой в памяти кибернетической системы}
    \begin{scnrelfromlist}{свойство-предпосылка}

        \scnitem{многообразие видов знаний, хранимых в памяти кибернетической системы}
        \scnitem{степень конвергенции и интеграции различного вида знаний, хранимых в
            памяти кибернетической системы}

    \end{scnrelfromlist}
    
    \scnheader{многообразие видов знаний, хранимых в памяти кибернетической системы}
    \begin{scnrelfromlist}{частное свойство}

        \scnitem{рефлексивность информации, хранимой в памяти кибернетической системы}
        \begin{scnindent}    
            \scnidtf{многообразие видов метаинформации (метазнаний), хранимых в
                    памяти кибернетической системы}
        \end{scnindent}
        
        \scnitem{многообразие моделей решения задач, используемых кибернетической
            системой}
        \scnitem{многообразие видов целей, анализируемых или синтезируемых
            кибернетической системой}
        \scnitem{многообразие планов решения задач, решаемых кибернетической системой}
        \scnitem{многообразие протоколов решения задач, решаемых кибернетической
            системой}

    \end{scnrelfromlist}

    \scnheader{объем информации, хранимой в памяти кибернетической системы}
    \scnidtf{объем знаний, приобретенных кибернетической системой к текущему
        моменту}
    \scnidtf{содержательная совокупность всех знаний, хранимых в текущий момент в
        памяти кибернетической системы}
    \scntext{примечание}{Чем больше кибернетическая система знает, тем при прочих равных
        условиях выше уровень её качества}
        
    \bigskip
    \scnheader{степень конвергенции и интеграции различного вида знаний, хранимых в памяти кибернетической системы}
    \scnidtf{уровень бесшовной  интеграции различного вида знаний кибернетической
        системы}
    \scntext{примечание}{Максимальный уровень конвергенции и интеграции знаний (в том
        числе,	и знаний различного вида) предполагает:
        \begin{scnitemize}

            \item использование универсального базового языка, по отношению к которому всем
            используемым видам знаний соответствуют специализированные языки, являющиеся
            подъязыками указанного базового языка
            \item построение четкой иерархии указанных специализированных языков по
            принципу язык-подъязык
            \item явное введение семейства отношений, заданных на множестве различных
            знаний и, в том числе, связывающих знания различного вида
        \end{scnitemize}
    }\scnrelfrom{свойство-предпосылка}{уровень формализованности информации,
        хранимой в памяти кибернетической системы}

    \scnheader{уровень формализованности информации, хранимой в памяти
        кибернетической системы}
    \scnidtf{степень приближения информации, хранимой в памяти кибернетической
        системы, к максимально простой и компактной форме представления информационной
        модели некоторого множества описываемых сущностей, которая является отражением
        определенной конфигурации связей между указанными сущностями}
    \scntext{примечание}{Высшим уровнем формализации информации, хранимой в памяти
        кибернетической системы, является смысловое представление информации в форме
        семантических сетей. Смотрите Раздел \scnqqi{\textit{Предметная область и онтология
            семантических сетей, семантических языков и семантических моделей баз
            знаний}}.}
    \scnrelboth{следует отличать}{формализация*}
    \begin{scnindent}
        \scnidtf{Бинарное ориентированное отношение, каждая пара которого связывает
            некоторую информационную конструкцию с другой информационной конструкцией,
            которая семантически эквивалентна первой, но имеет более высокий уровень
            формализованности}
        \scntext{примечание}{Приобретение навыков формального представления информации не
            является простой проблемой даже для человека. По сути совокупность таких
            навыков --- это основа математической культуры, культуры точного изложения своих
            соображений. Некоторые примеры, иллюстрирующие нетривиальность проблемы
            смотрите в \cite{Arnold2012}}
    \end{scnindent}
    \scnrelboth{следует отличать}{формализация}
    \begin{scnindent}
        \scnidtf{деятельность, направленная на повышение уровня формализованности
            представление информации}
        \scntext{метафора}{сближение синтаксиса с семантикой --- сближение
            синтаксической структуры информационной конструкции с её смысловой структурой}
    \end{scnindent}
    \scnidtf{уровень способности кибернетической системы к формальному
        представлению знаний и используемых понятий, к рационализации идей}
    \scnidtf{степень близости языка внутреннего представления (способа внутреннего
        кодирования) информации в памяти кибернетической системы к смысловому
        представлению информации}
    \scnidtf{степень близости к изоморфизму соответствия между: (1) синтаксической
        структурой внутреннего представления информации в памяти кибернетической
        системы и (2) конфигурацией связей описываемых сущностей}
    \begin{scnrelfromlist}{свойство-предпосылка}

        \scnitem{многообразие форм дублирования информации, хранимой в памяти
            кибернетической системы}
        \scnitem{относительный объём дублирования информации, хранимой в памяти
            кибернетической системы }
        \scnitem{многообразие фрагментов хранимой информации, не являющихся ни знаками,
            ни конфигурациями знаков }
        \scnitem{компактность представления информации, хранимой в памяти
            кибернетической системы}

    \end{scnrelfromlist}
    \scnheader{смысловое представление информации}
    \scnidtftext{пояснение}{способ представления информации, в котором
        минимизируются чисто синтаксические  аспекты представления информационных
        конструкций, не имеющие непосредственной семантической интерпретации}
    \scntext{примечание}{Примерами чисто синтаксических  аспектов представления
        информационных конструкций являются:
        \begin{scnitemize}

            \item буквы, которые входят в состав слов и которые, следовательно, не являются
            знаками описываемых сущностей;
            \item алфавиты букв различных языков;
            \item знаки препинания (разделители и ограничители);
            \item инцидентность (порядок, последовательность) букв и других символов,
            входящих в состав информационной конструкции.
        \end{scnitemize}
    }
    \begin{scnindent}
        \scntext{следовательно}{Информационная конструкция, представленная на
            каком-либо привычном для нас языке, является достаточно громоздкой
            информационной конструкцией, смысл которой (т.е знаки описываемых сущностей и
            семантически интерпретируемые связи между знаками, отражающие соответствующие
            связи между обозначаемыми сущностями) сильно закамуфлирован. Это существенно
            усложняет обработку информации. если пытаться реализовывать осмысленные  модели
            решения задач, для которых смысловые  аспекты обрабатываемой информации
            являются ключевыми.}
    \end{scnindent}
    \scntext{примечание}{Существенно подчеркнуть, что приближение внутреннего
        представления информации в памяти кибернетической системы к смысловому
        представлению информации является важнейшим фактором упрощения решателя задач
        кибернетической системы при реализации сложных моделей решения задач, требующих
        глубокого анализа смысла обрабатываемой информации. А это, в свою очередь,
        является важнейшим фактором качества решателя задач кибернетической
        системы.}
        
    \newpage\scnheader{многообразие форм дублирования информации, хранимой
        в памяти кибернетической системы}
    \scnidtf{многообразие видов семантической эквивалентности фрагментов
        информации, хранимой в памяти кибернетической системы}
    \scntext{примечание}{Простейшим видом семантической эквивалентности является
        синонимия знаков, когда два разных фрагмента хранимой информации являются
        знаками, имеющими один и тот же денотат (т. е обозначающими одну и ту же
        сущность).}
        
    \scnheader{относительный объем дублирования информации, хранимой в
        памяти кибернетической системы}
    \scnidtf{частота присутствия в хранимой информации семантически эквивалентных
        информационных фрагментов и, в частности, синонимичных знаков}
    
    \scnheader{многообразие фрагментов хранимой информации, не являющихся ни
        знаками, ни конфигурациями знаков}
    \scntext{примечание}{Примерами фрагментов хранимой информации, не являющихся знаками
        или конфигурациями знаков, являются:
        \begin{scnitemize}

            \item буквы, входящие в состав слов
            \item слова, входящие в состав словосочетаний
            \item различного вида разделители, знаки препинания
            \item различного вида ограничители.
        \end{scnitemize}
    }
    
    \scnheader{компактность представления информации, хранимой в памяти
        кибернетической системы}
    \scntext{примечание}{Должно уменьшаться число элементов памяти, используемых для
        представления информации, т.е. необходим переход к более компактным, но
        семантически эквивалентным информационным
        конструкциям.}
        
    \scnheader{стратифицированность информации, хранимой в памяти
        кибернетической системы}
    \scnrelfrom{свойство-предпосылка}{структурированность информации, хранимой в
        памяти кибернетической системы}
    \scnidtf{способность кибернетической системы выделять такие разделы информации,
        хранимой в памяти этой системы, которые бы ограничивали области действия
        агентов решателя задач кибернетической системы, являющиеся достаточными для
        решения заданных задач}
    \begin{scnindent}
        \scntext{примечание}{Существует правило, позволяющее каждой заданной задаче поставить
            в соответствие априори известный (выделенный) раздел хранимой информации,
            являющийся областью действия агентов решателя, осуществляющих решение заданной
            задачи. Основными видами такого рода разделов хранимой информации являются
            \textit{предметные области} и \textit{онтологии}.}
    \end{scnindent}
    \scnrelfrom{свойство-предпосылка}{рефлексивность
        информации, хранимой в памяти кибернетической системы}
    \scnidtf{уровень систематизации знаний, хранимых в памяти кибернетической
        системы}
    \scnidtf{уровень перехода от неструктурированных или слабоструктурированных
        данных к хорошо структурированным базам знаний}
    \scnidtf{уровень перехода от первичной информации к метаинформации,
        метаметаинформации и т.д.}
    
    \scnheader{рефлексивность информации,хранимой в памяти кибернетической системы}
    \scnidtf{уровень применения средств самоописания (метаязыковых средств) в
        информации, хранимой в памяти кибернетической системы}
    \scnidtf{относительный, объём и многообразие метаинформации, хранимой в памяти
        кибернетической системы}
    \scntext{примечание}{рефлексивность информации, хранимой в памяти кибернетической
        системы, т.е. наличие метаязыковых средств, является фактором, обеспечивающим
        не только структуризацию хранимой информации, но возможность описания
        синтаксиса и семантики самых различных языков, используемых кибернетической
        системой.}
        
    \newpage\scnheader{простота и локальность выполнения семантически
        целостных операций над информацией, хранимой в памяти кибернетической системы}
    \scntext{примечание}{Данное свойство касается не самой информации, хранимой в памяти,
        а язык кодирования (представления) информации в памяти кибернетической
        системы}\scnidtf{гибкость выполнения семантически целостных операций над
        информацией, хранимой в памяти кибернетической системы}
    
    \scnheader{база знаний}
    \scnidtf{база знаний кибернетической системы}
    \scnsubset{информация, хранимая в памяти кибернетической системы}
    \scnidtftext{пояснение}{информация, хранимая в памяти кибернетической системы
        и имеющая высокий уровень качества по всем показателям и, в частности, высокий
        уровень:
        \begin{scnitemize}

            \item \textit{семантической мощности языка представления информации хранимой в
                памяти кибернетической системы} (в базе знаний указанный язык должен быть
            универсальным);
            \item \textit{гибридности информации, хранимой в памяти кибернетической
                системы};
            \item \textit{многообразия видов знаний, хранимых в памяти кибернетической
                системы};
            \item формализованности информации, хранимой в памяти кибернетической системы;
            \item \textit{структурированности информации, хранимой в памяти кибернетической
                системы}
        \end{scnitemize}
    }
    \scntext{примечание}{Переход \textit{информации, хранимой в памяти кибернетической
            системы} на уровень качества, соответствующий \textit{базам знаний}, является
        важнейшим этапом эволюции \textit{кибернетических систем}. Подчеркнем при этом,
        что \textit{базы знаний} по уровню своего качества могут сильно отличаться друг
        от друга.}
    \bigskip
\end{scnsubstruct}
\scnsourcecomment{Завершили Сегмент \scnqqi{Комплекс свойств, определяющих качество информации, хранимой в памяти кибернетической системы}}
		\scnsegmentheader{Комплекс свойств, определяющих качество решателя задач
    кибернетической системы}

\begin{scnsubstruct}
    \scnheader{качество решателя задач кибернетической системы}
    \scnidtf{интегральная качественная оценка множества задач (действий), которые
        кибернетическая система способна выполнять в заданный момент}
    \scnidtf{качество навыков, приобретенных кибернетической системой}
    \scntext{примечание}{Основным свойством и назначением \textit{решателя задач
            кибернетической системы} является способность решать \textit{задачи} на основе
        накапливаемых (приобретаемых) \textit{кибернетической системой} различного вида
        \textit{навыков} с использованием \textit{процессора кибернетической системы},
        являющегося универсальным интерпретатором всевозможных накопленных
        \textit{навыков}. При этом качество (уровень развития, уровень совершенства)
        указанной способности определяется целым рядом дополнительных факторов
        (свойств).}
    \scnidtf{интеллектуальный уровень качества решателя задач
        кибернетической системы}
    \scnidtf{интегральное качество умений (навыков), приобретенных
        \textit{кибернетической системой} к текущему моменту}

    \begin{scnrelfromlist}{свойство-предпосылка}

        \scnitem{общая характеристика решателя задач кибернетической системы}
        \scnitem{качество логико-семантической организации памяти кибернетической
            системы}
        \scnitem{качество решения интерфейсных задач в кибернетической системе}

    \end{scnrelfromlist}
    
    \scnheader{общая характеристика решателя задач кибернетической системы}
    \begin{scnrelfromlist}{свойство-предпосылка}

        \scnitem{общий объем задач, решаемых кибернетической системой}
        \scnitem{многообразие видов задач, решаемых кибернетической системой}
        \scnitem{способность кибернетической системы к анализу решаемых задач}
        \scnitem{способность кибернетической системы к решению задач, методы решения
            которых в текущий момент известны}
        \scnitem{способность кибернетической системы к решению задач, методы решения
            которых ей в текущий момент не известны}
        \scnitem{множество навыков, используемых кибернетической системой}
        \scnitem{степень конвергенции и интеграции различного вида моделей решения
            задач, используемых кибернетической системой}
        \scnitem{качество организации взаимодействия процессов решения задач в
            кибернетической системе}
        \scnitem{быстродействие решателя задач кибернетической системы}
        \scnitem{способность кибернетической системы решать задачи, предполагающие
            использование информации, обладающей различного рода не-факторами}
        \scnitem{многообразие и качество решения задач информационного поиска}
        \scnitem{способность кибернетической системы генерировать ответы на вопросы
            различного вида в случае, если они целиком или частично отсутствуют в текущем
            состоянии информации, хранимой в памяти}
        \scnitem{способность кибернетической системы к рассуждениям различного вида}
        \scnitem{качество целеполагания}
        \scnitem{качество реализации планов собственных действий}
        \scnitem{способность кибернетической системы к локализации такой области
            информации,хранимой в ее памяти, которой достаточно для обеспечения решения
            заданной задачи}
        \scnitem{способность кибернетической системы к выявлению существенного в
            информации, хранимой в ее памяти}
        \scnitem{активность кибернетической системы}

    \end{scnrelfromlist}

    \scnheader{общий объем задач, решаемых кибернетической системой}
    \scnidtf{общий объем задач, которые кибернетическая система способна решать}
    \scnidtf{общий объем (множество), задач (действий), которые кибернетическая
        система способна (может, умеет) решать (выполнять) в заданный (в том числе, в
        текущий) момент}
    \scnrelfrom{свойство-предпосылка}{мощность языка представления задач, решаемых
        кибернетической системой}

    \scnheader{мощность языка представления задач, решаемых кибернетической системой}
    \scnidtf{мощность языка спецификации (описания) различного вида действий,
        выполняемых кибернетической системой}
    \scntext{примечание}{\textit{мощность языка представления задач} прежде всего
        определяется многообразием видов представляемых задач (многообразием видов
        описываемых действий).}\scnrelto{свойство-предпосылка}{многообразие видов
        задач, решаемых кибернетической системой}

    \begin{scnrelfromlist}{частное свойство}

        \scnitem{мощность языка представления задач, решаемых в памяти кибернетической системы}
        \begin{scnindent}
            \scnrelto{свойство-предпосылка}{многообразие видов задач, решаемых в
                памяти кибернетической системы}
        \end{scnindent}
        \scnitem{мощность языка представления задач, решаемых во внешней среде кибернетической системы}
        \begin{scnindent}
            \scnrelto{свойство-предпосылка}{многообразие видов
                задач, решаемых во внешней среде кибернетической системы}
        \end{scnindent}

        \scnitem{мощность языка представления задач, решаемых в рамках физической оболочки кибернетической системы}
        \begin{scnindent}
            \scnrelto{свойство-предпосылка}{многообразие
                видов задач, решаемых в рамках физической оболочки кибернетической системы}
        \end{scnindent}

    \end{scnrelfromlist}

    \scnheader{многообразие видов задач, решаемых кибернетической системой}
    \scnidtf{многообразие видов действий, которые кибернетическая система способна
        выполнять}
    \scntext{примечание}{Подчеркнем, что каждая задача есть спецификация соответствующего
        (описываемого) действия. Поэтому рассмотрение многообразия видов задач,
        решаемых кибернетической системой, полностью соответствует многообразию видов
        деятельности, осуществляемой этой системой. Важно заметить, что есть виды
        деятельности кибернетической системы, которые определяют качество и, в
        частности, уровень интеллекта кибернетической
        системы.}\scnrelfrom{свойство-предпосылка}{мощность языка представления задач в
        памяти кибернетической системы}

    \begin{scnrelfromset}{комплекс частных свойств}

        \scnitem{многообразие видов задач, решаемых в памяти кибернетической системы}
        \scnitem{многообразие видов задач, решаемых во внешней среде кибернетической
            системы}
        \scnitem{многообразие видов задач, решаемых в рамках физической оболочки
            кибернетической системы}

    \end{scnrelfromset}

    \scnheader{способность кибернетической системы к анализу решаемых задач}
    \scnidtf{способность кибернетической системы осмысливать (ведать) то, что она
        творит}
    \scnidtf{способность анализировать свои цели и, соответственно, решаемые задачи
        на предмет:
        \begin{scnitemize}

            \item сложности достижения;
            \item целесообразности достижения (нужности, важности, приоритетности);
            \item соответствия цели существующим нормам (правилам) соответствующей
            деятельности
        \end{scnitemize}
    }

    \scnheader{способность кибернетической системы к решению задач, методы решения
        которых ей в текущий момент известны}
    \scntext{примечание}{Указанными методами могут быть не только алгоритмы, но также и
        функциональные программы, продукционные системы, логические исчисления,
        генетические алгоритмы, искусственные нейронные сети различного вида.}
    \begin{scnrelfromlist}{свойство-предпосылка}

        \scnitem{способность кибернетической системы к поиску хранимых в своей памяти
            методов решения инициированных задач}
        \scnitem{способность кибернетической системы к интерпретации хранимых в своей
            памяти методов решения задач}

    \end{scnrelfromlist}

    \scnheader{способность кибернетической системы к решению задач, методы решения
        которых ей в текущий момент не известны}
    \scnidtf{способность кибернетической системы к решению задач, для которых не
        найдены соответствующие (релевантные) им методы их решения}
    \scnidtf{способность кибернетической системы строить цепочку цель-план
        достижения цели-система действий}
    \scntext{примечание}{Задачи, для которых не находятся соответствующие им методы,
        решаются с помощью метаметодов (стратегий) решения задач, направленных:
        \begin{scnitemize}

            \item на генерацию нужных исходных данных (нужного контекста), необходимых для
            решения каждой задачи;
            \item на генерацию плана решения задачи, описывающего сведение исходной задачи
            к подзадачам (до тех подзадач, методы решения которых системы известны);
            \item на сужение области решения задачи (на сужения контекста задачи,
            достаточного для ее решения).
        \end{scnitemize}
    }
    
    \scnheader{множество навыков, используемых кибернетической системой}
    \scnidtf{объем и многообразие навыков, приобретенных кибернетической системой к
        текущему моменту (с помощью учителей-разработчиков или полностью
        самостоятельно)}
    \scnidtf{возможности, навыки, приобретенные кибернетической системой}
    \scnidtf{опыт, приобретенный кибернетической системой}
    \scntext{примечание}{Новые навыки могут приобретаться кибернетической системой либо
        полностью самостоятельно, либо с помощью учителей, которые в простейшем случае
        просто сообщают обучаемой системе полностью сформулированные навыки. Для
        компьютерных систем учителями является их разработчики.}\bigskip
    \begin{scnrelfromlist}{частное свойство}

        \scnitem{множество методов решения задач, используемых кибернетической
            системой}
        \scnitem{множество моделей решения задач, используемых кибернетической
            системой}
        \scnitem{мощность языка представления в памяти кибернетической системы методов
            и моделей решения задач}

    \end{scnrelfromlist}
    
    \scnheader{множество методов решения задач, используемых кибернетической
        системой}
    \scnidtf{множество методов решения задач, используемых кибернетической системой
        и хранимых в ее памяти}
    \scnrelto{частное свойство}{многообразие видов знаний, хранимых в памяти
        кибернетической системы}
    
        \scnheader{метод решения задач}
    \scntext{пояснение}{\textbf{\textit{метод решения задач}} --- это \textit{вид
            знаний}, хранимых в \textit{памяти кибернетической системы} и содержащих
        информацию, которой достаточно либо для сведения каждой \textit{задачи} из
        соответствующего \textit{класса задач} к \textit{полной системе подзадач*},
        решение которых гарантирует решение исходной \textit{задачи}, \uline{либо} для
        окончательного решения этой \textit{задачи} из указанного \textit{класса задач}}
            
    \scnheader{множество моделей решений задач, используемых кибернетической
        системой}
    \scnidtf{способность кибернетической системы к использованию различных видов
        методов решения задач, соответствующих различным моделям решения задач}
    \scnidtf{многообразие методов решения задач, используемых кибернетической
        системой}
    \scnrelfrom{свойство-предпосылка}{мощность языка представления в памяти
        кибернетической системы методов и моделей решения задач}
    
    \scnheader{множество моделей решения задач, используемых кибернетической
        системой}
    \begin{scnrelfromset}{примечание}
        \scnitem{следует отличать*}
        \begin{scnindent}
            \begin{scnhaselementset}
                \scnitem{вид задач}
                \scnitem{модель решения задач}
                \begin{scnindent}
                    \scntext{пояснение}{каждая \textit{модель решения задач} задается
                        \begin{scnitemize}
                            \item \textit{языком}, обеспечивающим представление в \textit{памяти
                                кибернетической системы} некоторого класса \textit{методов решения задач}
                            \item интерпретатором указанных \textit{методов}, определяющим
                            \textit{операционную семантику} указанного \textit{языка}
                        \end{scnitemize}
                    }
                \end{scnindent}
                \scnitem{метод решения задач}
                \scnitem{класс задач}
                \begin{scnindent}
                    \scnidtf{\textit{множество} всех тех и только тех
                        \textit{задач}, которые решаются с помощью соответствующего \textit{метода}}
                \end{scnindent}
            \end{scnhaselementset}
        \end{scnindent}
    \end{scnrelfromset}

    \scnheader{Степень конвергенции и интеграции различного вида моделей решения
        задач, используемых кибернетической системой}
    \scntext{примечание}{Необходим переход от эклектики никак не связанных друг с другом
        \textit{моделей решения задач} к их \textit{конвергенции}, это предполагает:
        \begin{scnitemize}
            \item разработку общего (базового) для всех \textit{моделей решения задач}
            языка описания \textit{операционной семантики} языков описания методов,
            соответствующих различным \textit{моделям решения задач};
            \item включение всех языков описания \textit{методов решения задач} в общую
            систему языков, связанных между собой отношением \scnqqi{язык-подъязык*}.
        \end{scnitemize}
    }
    
    \scnheader{качество организации взаимодействия процессов решения задач в
        кибернетической системе}
    \begin{scnrelfromlist}{частное свойство}

        \scnitem{качество управления информационным процессом в памяти кибернетической системы}
        \begin{scnindent}
            \scnrelfrom{свойство-предпосылка}{обеспечение процессором
                кибернетической системы качественного управления информационными процессами в
                памяти}
        \end{scnindent}
        
        \scnitem{качество организации взаимодействия процессов решения задач во внешней
            среде или в физической оболочке кибернетической системы}
        \begin{scnindent}
            \begin{scnrelfromlist}{свойство-предпосылка}

                \scnitem{последовательность/параллельность процессов решения задач в
                    кибернетической системе}
                \scnitem{ синхронность/асинхронность процессов решения задач в кибернетической
                    системе}
                \scnitem{ централизованной/децентрализованность управления процессами решения
                    задач в кибернетической системе}

            \end{scnrelfromlist}
        \end{scnindent}

    \end{scnrelfromlist}
    \scntext{примечание}{Качество решения каждой \textit{задачи} определяется:
        \begin{scnitemize}

            \item временем её решения (чем быстрее \textit{задача} решается, тем выше
            качество её решения);
            \item полнотой и корректностью результата решения \textit{задачи};
            \item затраченными для решения \textit{задачи} ресурсами памяти (объемом
            фрагмента хранимой информации, используемой для решения задачи);
            \item затраченным для решения \textit{задачи} ресурсами решателя задач
            (количеством используемых внутренних агентов).
        \end{scnitemize}
        Таким образом, повышение качества процесса решения каждой конкретной
        \textit{задачи}, а также каждого \textit{класса задач} (путем совершенствования
        соответствующего метода, в частности, алгоритма) является важным фактором
        повышения качества \textit{решателя задач} в
        целом.}
        
    \scnheader{агентно-ориентированная модель обработки информации в памяти}
    \scnidtf{агентно-ориентированная модель управления действиями кибернетической
        системы, выполняемыми ею в своей памяти}
    \scntext{пояснение}{Перспективным вариантом построения \textit{решателя задач
            кибернетической системы} является реализация \textit{агентно-ориентированной
            модели обработки информации}, т.е. построение \textit{решателя задач} в виде
        \textit{многоагентной системы}, агенты которой осуществляют обработку
        \textit{информации, хранимой в памяти} кибернетической системы, и управляются
        этой информацией (точнее, её текущим состоянием). Особое место среди этих
        \textit{агентов} занимают сенсорные (рецепторные) и эффекторные
        \textit{агенты}, которые, соответственно, воспринимают информацию о текущем
        состоянии \textit{внешней среды} и воздействуют на \textit{внешнюю среду}, в
        частности, путем изменения состояния \textit{физической оболочки
            кибернетической системы}.
            \\Подчеркнем, что указанная агентно-ориентированная
        модель организации взаимодействия процессов решения задач в
        \textit{кибернетической системе} по сути есть не что иное, как модель
        ситуационного управления процессами решения задач, решаемых
        \textit{кибернетической системой} как в своей \uline{внешней среде}, так и в
        своей памяти.}
        
    \scnheader{модель инициирования действий кибернетической системы}
    \scnidtf{модель управления поведением кибернетической системы}

    \begin{scnsubdividing}

        \scnitem{стимульно-реактивная модель инициирования действий}
        \begin{scnindent}
            \scntext{пояснение}{от комбинации \textit{исходных сигналов},
                формируемых, например, априори известным набором сенсоров (рецепторов) к
                комбинации выходных \textit{сигналов}, управляющих, например, априори известным набором эффекторов}
        \end{scnindent}
        \scnitem{ситуационная модель инициирования действий без учета предыстории ситуаций и событий}
        \begin{scnindent}    
            \scntext{пояснение}{действие инициируется возникновением
                в памяти \textit{ситуации} априори известной конфигурации или априори
                известного события}
        \end{scnindent}
        \scnitem{ситуационная модель инициирования действий с учетом предыстории ситуаций и событий}
        \begin{scnindent}
            \scntext{пояснение}{действие инициируется не только
                текущей \textit{ситуацией} но и предшествующими \textit{ситуациями}, т.е.
                событиями перехода от одних \textit{ситуаций} к другим}
        \end{scnindent}

    \end{scnsubdividing}
    \scntext{примечание}{Речь идет о действиях, выполняемых \textit{кибернетической
            системой} как во внешней среде, так и в своей внутренней \textit{среде} (в
        своей памяти).}
        
    \scnheader{последовательность/параллельность процессов решения
        задач в кибернетической системе}
    \scnidtf{способность одновременно решать несколько разных задач, некоторые из
        которых могут быть подзадачами одной и той же задачи}
    \scnidtf{способность одновременно решать несколько разных задач, некоторые из
        которых могут быть подзадачами одной и той же задачи}

    \begin{scnrelfromlist}{свойство-предпосылка}

        \scnitem{максимально возможное количество действий, одновременно выполняемых
            кибернетической системой}
        \scnitem{способность кибернетической системы к одновременному выполнению
            взаимосвязанных действий}
        \begin{scnindent}    
            \scnidtf{способность кибернетической системы к
                одновременному выполнению действий, выполнение каждого из которых может
                помешать выполнению другого}
            \scnidtf{способность кибернетической системы к эквилибристике}
        \end{scnindent}

    \end{scnrelfromlist}

    \begin{scnrelfromset}{комплекс частных свойств}

        \scnitem{физическая последовательность/параллельность процессов решения задач в
            кибернетической системе}
        \scnitem{ логическая последовательность/параллельность процессов решения задач
            в кибернетической системе}
        \begin{scnindent}
            \scntext{пояснение}{Логическая параллельность выполняемых процессов
                (действий) предполагает возможность существования \uline{выполняемых} процессов
                в двух режимах:
                \begin{scnitemize}

                    \item в активном режиме --- в режиме непосредственного выполнения
                    \item в режиме прерывания --- в режиме ожидания	условий (событий и/или
                    ситуаций) при возникновении которых прерванный процесс переходит в режим
                    активного процесса.
                \end{scnitemize}
            }
        \end{scnindent}

    \end{scnrelfromset}

    \begin{scnrelfromset}{комплекс частных свойств}

        \scnitem{последовательность/параллельность информационных процессов в памяти
            кибернетической системы}
        \scnitem{ последовательность/параллельность процессов решения задач во внешней
            среде или в физической оболочке кибернетической системы}

    \end{scnrelfromset}
    \scntext{примечание}{Подчеркнем, что есть целый ряд задач, решаемых кибернетической
        системой, процессы решения которых носят перманентный (постоянный) характер. К
        таким задачам относятся:
        \begin{scnitemize}

            \item поддержка высокого качества базы знаний (устранение противоречий,
            информационного мусора);
            \item поддержка семантической совместимости с другими компьютерными системами;
            \item мониторинг и анализ состояния внешней среды;
            \item обеспечение собственной безопасности;
            \item самообучение.
        \end{scnitemize}
    }
    
    \scnheader{быстродействие решателя задач кибернетической системы}
    \scnidtf{скорость решения задач в кибернетической системе}
    \scnidtf{быстродействие решателя задач кибернетической системы}
    \scnidtf{скорость реакции кибернетической системы на различные задачные
        ситуации}
    \scnrelfrom{свойство-предпосылка}{быстродействие процессора кибернетической
        системы}
    
    \scnheader{способность кибернетической системы решать задачи, предполагающие
        использование информации, обладающей различного рода не-факторами}
    \scnidtf{способность кибернетических систем решать задачи, которые:
        \begin{scnitemize}

            \item либо нечетко сформулированы (делай то, не знаю что);
            \item либо решаются в условиях неполноты, неточности, противоречивости исходных
            данных;
            \item либо являются задачами, принадлежащими классам задач, для которых
            практически невозможно построить соответствующие алгоритмы.
        \end{scnitemize}
    }
    \scnidtf{способность кибернетической системы решать труднорешаемые,
        трудноформализуемые задачи}
    \scnidtf{способность решать интеллектуальные (трудноформализуемые) задачи, для
        которых характерна:
        \begin{scnitemize}

            \item неточность и недостоверность исходных данных;
            \item отсутствие критерия качества результата;
            \item невозможность или высокая трудоемкость разработки алгоритма;
            \item необходимость учета контекста задачи.
        \end{scnitemize}
    }

    \scnheader{задача, предполагающая использование информации, обладающей
        различного рода не-факторами}
    \scnidtf{трудноформализуемая задача}
    \scnsuperset{задача проектирования}
    \scnsuperset{задача распознавания}
    \scnsuperset{задача прогнозирования}
    \scnsuperset{задача целеполагания}
    \scnsuperset{задача планирования}
    
    \scnheader{многообразие и качество решения задач информационного поиска}
    \scnrelfrom{свойство-предпосылка}{семантический уровень доступа к информации,
        хранимой в памяти кибернетической системы}
    \scnrelto{частное свойство}{многообразие видов задач, решаемых кибернетической
        системой}
    \scnidtf{способность кибернетической системы качественно решать широкое
        многообразие задач информационного поиска в рамках текущего состояния хранимой
        информации}
    \scnidtf{способность кибернетической системы находить в текущем состоянии
        хранимой информации релевантные ответы на запросы (вопросы) самого различного
        вида}
    
    \scnheader{вопрос}
    \scnidtf{запрос}
    \scnsuperset{запрос изоморфных или гомоморфных фрагментов хранимой информации
        по заданному образцу с указанием знаков известных сущностей}
    \begin{scnindent}
        \scnrelfrom{класс частных вопросов}{запрос всех связок различных отношений,
            обязывающих заданную сущность с другими}
            \begin{scnindent}
                \scnrelfrom{класс частных вопросов}{запрос всех связок заданных отношений,
                    связывающих заданную сущность с другими}
            \end{scnindent}
    \end{scnindent}
    \scnsuperset{вопрос типа \scnqqi{как связаны между собой заданные две сущности}}
    \begin{scnindent}
    \scntext{пояснение}{Две сущности будем считать связанными в том и только в
        том случае, если существует маршрут, соединяющий указанные две сущности, в
        состав которого входят связки, принадлежащие в общем случаем разным
        отношениям}\scntext{примечание}{Здесь принципиально важным является учет
        \textit{семантической силы связей} между сущностями, которая определяется
        \textit{семантической силой отношений}, которым принадлежат связки, входящие в
        состав связей (маршрутов) между сущностями.}
    \scnrelto{класс частных
        вопросов}{вопрос типа \scnqqi{как связаны между собой заданные сущности}}
        \begin{scnindent}
            \scntext{примечание}{Здесь имеется в виду произвольное количество связываемых
                сущностей, а это предполагает, что ответом на данный запрос является
                \uline{связный граф}, вершинами которого являются знаки заданных сущностей.}
        \end{scnindent}
    \end{scnindent}
    \scnsuperset{вопрос типа \scnqqi{что это такое}}
    \begin{scnindent}
        \scnidtf{запрос спецификации (описания) заданной сущности}
        \scnrelfrom{класс частных вопросов}{запрос определения}
        \begin{scnindent}
            \scnidtf{запрос определения заданного понятия}
        \end{scnindent}
        \scnrelfrom{класс частных вопросов}{запрос документации заданного объекта}
    \end{scnindent}
    \scnsuperset{почему-вопрос}
    \begin{scnindent}
        \scnsuperset{запрос причины возникновения заданной ситуации или события}
        \scnsuperset{запрос логического обоснования заданного высказывания}
            \begin{scnindent}
                \scnidtf{запрос объяснения корректности заданного высказывания, которое, в
                    частности, может быть порождено (сгенерировано) в процессе решения некоторой
                    задачи с помощью некоторого метода (алгоритма, искусственной нейронной сети
                    логического исчисления и т.п.)}
                \scnsuperset{запрос доказательства заданной теоремы}
            \end{scnindent}
    \end{scnindent}
    \scnsuperset{запрос возможных последствий заданной ситуации или события}
    \scnsuperset{запрос того, что логически следует из заданного высказывания}
    \scnsuperset{запрос метода решения данной задачи}
    \scnsuperset{запрос плана решения данной задачи}
    \begin{scnindent}
        \scnidtf{запрос декомпозиции данной задачи на систему и/или подзадач}
    \end{scnindent}
    \scnsuperset{зачем-вопрос}
    \begin{scnindent}
        \scnidtf{каково назначение заданной сущности}
        \scnidtf{для решения какой задачи (для чего, достижения какой цели) нужна
            данная сущность}
    \end{scnindent}
    \scnsuperset{запрос аналогов заданной сущности}
    \scnsuperset{запрос антиподов заданной сущности}
    \scnsuperset{запрос сходств и отличий двух связанных сущностей}
    \scnsuperset{запрос сравнительного анализа заданной сущности}
    \begin{scnindent}
        \scnsuperset{запрос достоинств заданной сущности}
        \scnsuperset{запрос недостатков заданной сущности}
    \end{scnindent}
    \scnsuperset{где-вопрос}
    \begin{scnindent}
        \scnidtf{запрос информации о местоположении заданной пространственной сущности
            примечание}
        \scntext{примечание}{Здесь запрашивается любая информация о пространственных связях
            заданной сущности}
    \end{scnindent}    
    \scnsuperset{когда-вопрос}
    \begin{scnindent}
        \scnidtf{запрос информации о темпоральных свойствах и связях заданной временной
            сущности (о моменте начала, о моменте завершения, о длительности)}
    \end{scnindent}

   \scnheader{cпособность кибернетической системы генерировать ответы на вопросы
        различного вида в случае, если они целиком или частично отсутствуют в текущем
        состоянии информации, хранимой в памяти}
    \scnidtf{способность кибернетической системы генерировать (порождать, строить,
        синтезировать, выводить) ответы на самые различные вопросы и, в частности, на
        вопросы типа \scnqqi{что это такое}, на почему-вопросы, это означает способность
        кибернетической системы \uline{объяснять} (обосновывать корректность) своих
        действий}

    \begin{scnrelfromlist}{свойство-предпосылка}

        \scnitem{семантическая гибкость информации, хранимой в памяти кибернетической
            системы}
        \scnitem{ способность кибернетической системы к рассуждениям различного вида}

    \end{scnrelfromlist}

    \scnheader{способность кибернетической системы к рассуждениям различного вида}
    \scnidtf{способность кибернетической системы к целенаправленному порождению
        (генерации) новых истинных или правдоподобных знаний (следствий) на основе
        имеющихся знаний (посылок)}

    \begin{scnrelfromlist}{частное свойство}

        \scnitem{способность кибернетической системы к дедуктивному выводу}
        \scnitem{способность кибернетической системы к индуктивному выводу}
        \scnitem{способность кибернетической системы к абдуктивному выводу}

    \end{scnrelfromlist}

    \scnheader{качество целеполагания}
    \scnidtf{качество реализации первого этапа решения сложных задач --- этапа
        генерации (построения) планов решения сложных задач}
    \scnidtf{качество генерации планов выполнения сложных действий:
        \begin{scnitemize}

            \item как внутренних действий (в памяти кибернетической системы), так и внешних
            действий (во внешней среде)
            \item как собственных действий, так и действий других субъектов
        \end{scnitemize}
    }
    \scnidtf{качество генерации планов действий кибернетической системы и, в
        частности, трудоемкость процесса генерации этих планов}
    \scnidtf{качество организации целенаправленной деятельности кибернетической
        системы}
    \scnidtf{качество построения цепочек цель-план-действие }
    \scnidtf{качество генерации, анализа и инициирования собственных целей
        (собственных задач)}
    \scnidtf{способность кибернетической системы к целеполаганию}

    \begin{scnrelfromlist}{свойство-предпосылка}
        \scnitem{самостоятельность целеполагания}
        \begin{scnindent}
            \scnidtf{самостоятельность генерации
                и инициирования целей (задач), направленных на создание условий достижения
                соответствующих стратегических целей (сверхзадач)}
        \end{scnindent}
        \scnitem{целенаправленность целеполагания}
        \begin{scnindent}
            \scnidtf{степень соответствия
                (степень полезности) генерируемых целей (задач) для достижения соответствующих
                стратегических целей (сверхзадач)}
        \end{scnindent}
        \scnitem{сбалансированность целеполагания}
        \begin{scnindent}
            \scnidtf{качество расстановки
                приоритетов у сгенерированных и инициированных целей (задач) для обеспечения
                баланса между тактическими и стратегическими целями}
        \end{scnindent}
    \end{scnrelfromlist}

    \scnheader{самостоятельность целеполагания}
    \scnidtf{способность кибернетической системы генерировать, инициировать и
        решать задачи, которые не являются подзадачами, инициированными внешними
        (другими) субъектами, а также способность на основе анализа своих возможностей
        отказаться от выполнения задачи, инициированной извне, переадресовав её другой
        кибернетической системе, либо на основе анализа самой этой задачи обосновать её
        нецелесообразность или некорректность}
    \scnidtf{способность к самостоятельному целеполаганию (генерации идей) и к
        инициированию процессов их достижения (т.е. к принятию решений), способность
        свободно (в определенных рамках) выбирать (ставить перед собой цели)}
    \scnidtf{уровень самостоятельности}
    \scnidtf{способность решать задачи в комплексе, включая создание всех
        необходимых условий для их решения с учетом конкретных обстоятельств}
    \scnidtf{умение решать задачи в условиях сильных помех (в осложненных
        обстоятельствах)}
    \scntext{примечание}{Повышение уровня самостоятельности существенно расширяет
        возможности кибернетической системы, т.е. объем тех задач, которые она может
        решать не только в идеальных  условиях, но и в реальных (осложненных)
        обстоятельствах.}\scnidtf{степень свободы выбора целей, подлежащих достижению,
        а также свободы генерации целей, не являющихся подцелями извне поставленных
        целей}

    \scnheader{целенаправленность целеполагания}
    \scnidtf{целеустремленность}
    \scnidtf{целенаправленность}
    \scnidtf{степень целостности деятельности}
    \scnidtf{степень соответствия между тактическими и стратегическими уровнями
        деятельности}
    \scnidtf{общее соотношение между временем, затраченным на лишние  (ненужные,
        нецелесообразные, нецеленаправленные) действия и полезные действия}
    \scnidtf{целесообразность деятельности}
    \scnidtf{способность адекватно расставлять приоритеты своим целям и не
        распыляться  на достижение неприоритетных (несущественных) целей}
    
        \scnheader{качество реализации планов собственных действий}
    \scnidtf{качество реализации целенаправленной деятельности на основе
        построенных планов}
    \scnidtf{качество реализации построенных в памяти кибернетической системы
        планов выполнения сложных собственных действий, которые могут предполагать
        участие других субъектов}
    
    \scnheader{способность кибернетической системы к локализации такой области
        информации, хранимой в ее памяти, которой достаточно для обеспечения решения
        заданной задачи}
    \scnidtf{способность кибернетической системы к сужению области решения каждой
        решаемой ею задачи, что существенно минимизирует затраты кибернетической
        системы на учет и анализ факторов, априори незначимых (несущественных) для
        решения каждой решаемой задачи}
    \scntext{примечание}{Для реализации данной способности важное значение имеет
        качественная стратификация базы знаний кибернетической системы на предметные
        области и соответствующие им онтологии.}
        
    \newpage\scnheader{способность кибернетической системы к выявлению существенного в информации, хранимой в ее памяти}
    \scnidtf{способность к выявлению (обнаружению, выделению) таких фрагментов
        информации, хранимой в памяти кибернетической системы, которые существенны
        (важны) для достижения соответствующих целей}
    \scntext{примечание}{Понятие существенного (важного) фрагмента информации, хранимой в
        памяти кибернетической системы, относительно и определяется соответствующей
        задачей. Тем не менее, есть важные перманентно (постоянно) решаемые задачи, в
        частности задачи анализа качества информации, хранимой в памяти кибернетической
        системы. Существенные фрагменты хранимой информации, выделяемые в процессе
        решения этих задач, являются относительными не столько по отношению к решаемой
        задаче, сколько по отношению к текущему состоянию хранимой информации.
        Примерами таких фрагментов являются:
        \begin{scnitemize}
            \item обнаруженные противоречия (ошибки) с явным указанием того, что чему
            противоречит;
            \item обнаруженные информационные дыры, точнее точная спецификация этих дыр;
            \item обнаруженные мусорные фрагменты, которые либо носят вспомогательный
            характер, либо могут быть легко восстановлены (воспроизведены).
        \end{scnitemize}
    }
    
    \scnheader{следует отличать*}
    \begin{scnhaselementset}
        \scnitem{способность кибернетической системы к выявлению существенного в
            информации, хранимой в ее памяти}
        \begin{scnindent}    
            \scntext{примечание}{Здесь кибернетическая система
                выделяет информацию, которая необходима, но не обязательно достаточна для
                решения соответствующей задачи.}
        \end{scnindent}
        \scnitem{способность кибернетической системы к локализации такой области
            информации, хранимой в ее памяти, которой достаточно для обеспечения решения
            заданной задачи}
        \begin{scnindent}    
            \scntext{примечание}{Здесь кибернетическая система отбрасывает
                (исключает) информацию, которая априори несущественна для решения
                соответствующей (заданной) задачи.}
        \end{scnindent}
    \end{scnhaselementset}

    \scnheader{активность кибернетической системы}
    \scnidtf{уровень активности кибернетической системы}
    \scnidtf{уровень мотивации к деятельности в различных направлениях}
    \scnidtf{уровень желания  действовать}
    \scnidtf{активность/пассивность кибернетической системы}
    \scnidtf{уровень инициативности, пассионарности, мотивированности}
    \scntext{примечание}{Уровень активности кибернетической системы может быть разным для
        разных решаемых задач, для разных классов выполняемых действий, для разных
        видов деятельности.}\scntext{примечание}{Следует отличать уровень активности
        (мотивации, желания) и направленность этой активности.}\scntext{примечание}{Чем выше
        активность кибернетической системы, тем (при прочих равных условиях) она больше
        успевает сделать, следовательно, тем выше ее качество
        (эффективность).}\scnrelboth{обратное свойство}{пассивность}
    \begin{scnindent}
        \scnidtf{уровень бездеятельности, медлительности, вялости, ленивости}
    \end{scnindent}
    \begin{scnrelfromlist}{частное свойство}
        \scnitem{познавательная активность}
        \scnitem{ социальная активность}
    \end{scnrelfromlist}
    
    \scnheader{качество логико-семантической организации памяти
        кибернетической системы}
    \scnidtf{качество базовых семантически целостных действий в памяти
        кибернетической системы}
    \scnidtf{качество семантически элементарных (законченных, целостных)
        информационных процессов, выполняемых кибернетической системой в своей памяти}
    \scnidtf{интегральная оценка того, насколько способствует (насколько близка)
        организация памяти кибернетической системы реализации осмысленных
        преобразований, хранимых в памяти знаний}
    \scnidtf{степень приспособленности решателя задач кибернетической системы к
        обработке сложноструктурированных баз знаний}
    \scnidtf{степень приспособленности решателя задач кибернетической системы к
        обработке хранимой в её памяти информации, имеющий высокий уровень качества как
        по форме представления информации, так и по её содержанию --- по многообразию
        представляемых знаний и по уровню их конвергенции и интеграции}

    \begin{scnrelfromlist}{свойство-предпосылка}
        \scnitem{семантический уровень доступа к информации, хранимой в памяти
            кибернетической системы}
        \scnitem{семантическая гибкость информации, хранимой в памяти кибернетической
            системы}
        \scnitem{степень конвергенции и интеграции представления навыков, хранимых в
            памяти кибернетической системы, с представлением обрабатываемой информации}
    \end{scnrelfromlist}

    \scnheader{семантический уровень доступа к информации, хранимой в памяти
        кибернетической системы}
    \scnidtf{степень ассоциативности доступа к информации, хранимой в памяти
        кибернетической системы}
    \scnidtf{способность кибернетической системы локализовывать (находить)
        требуемый (запрашиваемый) фрагмент информации, хранимой в её памяти, не на
        основании известного адреса запрашиваемой информации (её местоположения в
        памяти), а на основании:
        \begin{scnitemize}
            \item известного типа запрашиваемой информации;
            \item известных сущностей, знаки которых входят в состав запрашиваемой
            информации;
            \item полностью или частично известной конфигурации запрашиваемой информации
            (т.е. конфигурации связей между известными и искомыми сущностями)
        \end{scnitemize}
    }
    \scntext{пояснение}{\textit{уровень доступа к информации, хранимой в памяти
            кибернетической системы} определяется тем, что нам достаточно знать об искомой
        в памяти кибернетической системы информации (в частности, об искомом знаке
        некоторой интересующей нас сущности). Мы можем знать место в памяти (ячейку
        памяти, область памяти), где находится интересующая нас информация. Такой
        доступ называется \uline{адресным}. Мы можем знать имя интересующей нас
        сущности, но не знать, где находится информация, описывающая эту сущность. Мы
        можем не знать имени интересующей нас сущности, но знать, как эта сущность
        связана с другими известными нам сущностями.}\scntext{пояснение}{Пусть нам
        необходимо локализовать (выделить) хранимую в памяти информацию, описывающую
        известные нам сущности, связанные известными нам отношениями, но
        местонахождение этой информации в памяти нам не известно. Если организация
        памяти нам представляет такую возможность, то такую память будем называть
        ассоциативной, т.е. памятью, обеспечивающей семантический доступ к хранимой в
        ней информации.}\scntext{примечание}{Для того, чтобы построить информационную модель
        среды, в которой действует (функционирует) кибернетическая система, необходимо,
        с одной стороны, разложить  эту информационную модель по полочкам , превратить
        её в некую систему из компонентов этой информационной модели, а, с другой
        стороны, обеспечить быстрый поиск нужного фрагмента указанной информационной
        модели, не зная, на каких полочках   находятся компоненты этого искомого
        фрагмента, который при этом может иметь произвольную конфигурацию и
        произвольный размер. Это и есть высший уровень \uline{ассоциативности} доступа
        к информации, хранимой в памяти кибернетической
        системы.}\scntext{пояснение}{Данное свойство, данная характеристика
        организации информации, хранимой в памяти кибернетической системы, является
        важнейшей характеристикой \uline{внутреннего} языка представления информации в
        памяти кибернетической системы. Указанная характеристика внутреннего языка
        определяется \uline{простотой процедур поиска} востребованных (запрашиваемых)
        фрагментов хранимой информации --- например, процедуры поиска знаков всех
        сущностей, каждая из которых связана с заданными (известными) сущностями
        связями заданных (известных) типов, процедуры поиска (выделения) знаков всех
        сущностей, которые связаны с заданной (известной) сущностью связью неважно
        какого типа, процедуры поиска информационного фрагмента заданному образцу
        (шаблону) произвольного размера и конфигурации, в котором выделены знаки
        известных сущностей и условные обозначения искомых
        сущностей.}\scnrelfrom{свойство-предпосылка}{степень близости языка внутреннего
        представления информации в памяти кибернетической системы к смысловому
        представлению информации}

    \newpage\scnheader{семантическая гибкость информации, хранимой в памяти
        кибернетической системы}
    \scnrelfrom{свойство-предпосылка}{степень близости языка внутреннего
        представления информации в памяти кибернетической системы к смысловому
        представлению информации}
    \scnidtf{простота реализации базовых (элементарных), но семантически целостных
        (семантически значимых, осмысленных) действий (операций) преобразования
        (обработки) информации, хранимой в памяти кибернетической системы}
    
    \scnheader{базовое семантически целостное действие над информацией, хранимой в
        памяти кибернетической системы}
    \scnidtf{элементарная семантически значимая (осмысленная) операция над
        информацией, хранимой в памяти кибернетической системы}
    \scntext{примечание}{Здесь принципиальной является семантическая целостность
        (осмысленность) действия над хранимой информацией. Так, например, операция
        адресного доступа к требуемому фрагменту хранимой информации не является
        семантически целостной, так как смысл искомого (запрашиваемого) фрагмента
        хранимой информации не уточняется.}\scntext{примечание}{Разные кибернетические
        системы могут использовать разные наборы классов базовых семантически целостных
        действий над информацией, хранимой в их памяти.}\scntext{примечание}{Примерами
        \textit{базовых семантически целостных действий над информацией, хранимой в
            памяти кибернетической системы}, в частности, являются:
        \begin{scnitemize}

            \item операции поиска, генерации, удаления или замены связок между знаками
            известных сущностей;
            \item операции поиска, генерации, удаления или замены имен, приписываемых
            знакам известных сущностей.
        \end{scnitemize}
        Существенно подчеркнуть, что простота реализации такого рода операций (т.е.
        гибкость хранимой в памяти информации) во многом обеспечивается стремлением к
        локальности выполнения этих операций. Такая локальность означает то, что при
        выполнении \uline{каждой} из указанных операций меняется только обрабатываемый
        фрагмент хранимой информации и не требуется никакого переразмещения в памяти
        остальной части хранимой информации.}
        
    \scnheader{степень конвергенции и интеграции представления навыков, хранимых в памяти кибернетической системы, с
        представлением обрабатываемой информации}
    \scnrelto{частное свойство}{степень конвергенции и интеграции различного вида
        знаний, хранимых в памяти кибернетической системы}
    \begin{scnindent}
        \scnrelfrom{свойство-предпосылка}{степень близости языка внутреннего
            представления информации в памяти кибернетической системы к смысловому
            представлению информации}
    \end{scnindent}
    \scntext{примечание}{Навыки кибернетической системы являются частным видом знаний,
        хранимых в её памяти, поэтому степень конвергенции навыков и обрабатываемых
        знаний определяется глубиной  и объемом  \uline{общих} (одинаковых) принципов,
        лежащих в основе как представления навыков, так представления обрабатываемых
        знаний.}
        
    \scnheader{качество решения интерфейсных задач в
        кибернетической системе}
    \begin{scnrelfromlist}{частное свойство}
        \scnitem{способность кибернетической системы к пониманию сенсорной информации}
        \scnitem{способность кибернетической системы к пониманию принимаемых сообщений}
        \scnitem{способность кибернетической системы к самостоятельной деятельности во
            внешней среде}
            \begin{scnindent}
                \scnidtf{способность кибернетической системы к воздействию на
                внешнюю среду и к управлению своим поведением во внешней среде}
            \end{scnindent}
    \end{scnrelfromlist}

    \scnheader{интерфейсная задача}
    \scnsuperset{задача анализа введенной информации}
    \scnsuperset{задача анализа сенсорной информации}
    \begin{scnindent}
        \scnidtf{задача анализа информации, порождаемой (генерируемой) непосредственно
            сенсорами кибернетической системы}
        \scnsuperset{задача синтаксического анализа сенсорной информации}
        \scnsuperset{задача семантического анализа сенсорной информации}
        \begin{scnindent}
            \scnidtf{задача анализа сенсорной информации, направленного на
                \uline{понимание} этой информации --- на выявление (распознавание) в этой
                информации отображения (сенсорного описания) объектов, важных для
                кибернетической системы (т.е. объектов, описанных в базе знаний этой системы и,
                соответственно, представленных в этой базе знаний своими знаками либо знаками
                классов, которым эти объекты принадлежат), а также важных для кибернетических
                связей между указанными объектами}
            \scnidtf{задача генерации фрагмента базы знаний кибернетической системы,
                являющегося логическим следствием заданной сенсорной информации и
                представляющегося собой важную для кибернетической системы информацию}
            \scnidtf{задача извлечения из сенсорной информации (первичной информации)
                важной для кибернетической системы вторичной информации}
            \scnsuperset{задача анализа принимаемого вербального сообщения}
            \scnidtf{задача анализа введенных знаковых конструкций}
            \scnidtf{задача анализа сообщений, введенных в кибернетическую систему}
            \scnidtf{задача анализа внешних знаковых конструкций}
            \scnsuperset{задача синтаксического анализа принимаемого вербального сообщения}
            \scnsuperset{задача трансляции принимаемого вербального сообщения на внутренний
                язык кибернетической системы}
            \scnsuperset{задача погружения нового фрагмента в состав согласованной части
                базы знаний}
            \begin{scnindent}
                \scnidtf{задача интеграции (встраивания) нового фрагмента базы знаний в состав
                    базы знаний}
                \scnidtf{задача понимания нового фрагмента базы знаний в контексте её текущего
                    состояния, что, прежде всего, требует обеспечения семантической совместимости
                    (согласования понятий) между базой знаний и интегрируемым новым фрагментом}
            \end{scnindent}
        \end{scnindent}
    \end{scnindent}
    \scnsuperset{задача управления эффекторами кибернетической системы при
        выполнении сложных воздействий на внешнюю среду и/или физическую оболочку этой
        кибернетической системы}
    \begin{scnindent}
        \scnidtf{задача целенаправленной сенсорно-эффекторной (в частности, сенсомоторной) координации}
    \end{scnindent}

    \scnheader{сенсорная информация}
    \scnidtf{информация, генерируемая непосредственно некоторой группой
        (конфигурацией) сенсоров (рецепторов) кибернетической системы}
    \scnidtf{рецепторная информация}
    \scnidtf{первичная информация, получаемая (приобретаемая) кибернетической
        системой}
    \scnidtf{первичная знаковая конструкция, которая описывает те или иные свойства
        текущего состояния физической окружающей среды (внешней среды и физической
        оболочки) кибернетической системы}

    \scnheader{сенсор кибернетической системы}
    \scnidtf{рецептор кибернетической системы}
    \scntext{пояснение}{Компонент кибернетической системы, генерирующий в памяти
        этой системы информацию о текущем значении соответствующего этому компоненту
        свойства (характеристики, параметра) того фрагмента физической окружающей среды
        кибернетической системы, который непосредственно смежен (пограничен) указанному
        компоненту.}
        
    \scnheader{эффектор кибернетической системы}
    \scnidtf{компонент кибернетической системы, который способен менять своё
        состояние в целях непосредственного воздействия на свою физическую оболочку и
        на внешнюю среду}

    \scnheader{способность кибернетической системы к пониманию сенсорной
        информации}
    \scnidtf{способность к синтаксическому и семантическому анализу информации,
        формируемой сенсорами кибернетической системы, а также к погружению  этой
        информации в состав общей информационной модели внешней среды кибернетической
        системы (в состав общей картины внешнего мира)}
    \scnidtf{способность кибернетической системы к переходу от первичной
        (сенсорной) информации ко вторичной информации, которая описывает связи между
        вторичными объектами, каждый из которых представлен (описан) в первичной
        информации конфигурацией знаков своих частей с дополнительным описанием свойств
        каждой из этих частей}

    \newpage\scnheader{способность кибернетической системы к самостоятельной
        деятельности во внешней среде}
    \begin{scnrelfromlist}{свойство-предпосылка}
        \scnitem{уровень развития эффекторов, обеспечивающих самостоятельное
            перемещение кибернетической системы}
        \begin{scnindent}
            \begin{scnrelfromlist}{частное свойство}
                \scnitem{уровень развития эффекторов, обеспечивающих локальное перемещение
                    сенсоров кибернетической системы}
                \scnitem{уровень развития эффекторов, обеспечивающих функционирование
                    манипуляторов кибернетической системы}
                \scnitem{уровень развития эффекторов, обеспечивающих перемещение всей
                    физической оболочки кибернетической системы}
            \end{scnrelfromlist}
        \end{scnindent}
        \scnitem{качество управления поведением кибернетической системы во внешней
            среде}
        \begin{scnindent}
            \scnidtf{качество сенсорно-эффекторной координации действий
                кибернетической системы при выполнении сложных действий во внешней среде}
        \end{scnindent}
    \end{scnrelfromlist}
    \bigskip

\end{scnsubstruct}
\scnsourcecomment{Завершили Сегмент \scnqqi{Комплекс свойств, определяющих качество решателя задач кибернитической системы}}
		\scnsegmentheader{Комплекс свойств, определяющих уровень обучаемости кибернетической системы}
\begin{scnsubstruct}
    \scnheader{обучаемость кибернетической системы}
    \scnidtf{способность \textit{кибернетической системы} повышать своё качество, адаптируясь к решению новых задач, \textit{качество внутренней информации модели своей среды}, \textit{качество} своего \textit{решателя задач} и даже \textit{качество} своей \textit{физической оболочки}.}
    \scnidtf{способность кибернетической системы к самосовершенствованию с различной степенью самостоятельности (с учителем, с экспертом, с внешними источниками информации, только на собственном опыте)}
    \scnrelboth{следует отличать}{приспособленность кибернетической системы к её совершенствованию, осуществляемому извне}
    \scnidtf{способность кибернетической системы к самостоятельному повышению уровня (качества) своих знаний, навыков, а также уровня своей обучаемости}
    \scnidtf{способность кибернетической системы к самостоятельному самосовершенствованию}
    \scnidtf{скорость эволюции кибернетической системы}
    \scnidtf{уровень (степень) обучаемости кибернетической системы}
    \scnidtf{способность кибернетической системы к совершенствованию (к эволюции, к повышению уровня своего качества)}
    \scntext{примечание}{Максимальный уровень обучаемости кибернетической системы --- это её способность эволюционировать (повышать уровень своего качества) максимально быстро и \uline{в любом}(!) направлении, т.е. способность быстро и без каких-либо ограничений приобретать \uline{любые}(!) новые \textit{знания} и \textit{навыки}.}
    \scnidtf{способность кибернетической системы к повышению своего качества (в том числе, путем устранения своих недостатков, выявленных в результате самоанализа (рефлексии), в частности, в результате работы над своими ошибками, разбора собственных полетов)}
    \scnidtf{способность кибернетической системы к обучению}
    \scnidtf{умение кибернетической системы учиться}
    \scnidtf{способность кибернетической системы обучаться}
    \scntext{примечание}{Реализация способности кибернетической системы обучаться, т.е. решать перманентно инициированную сверхзадачу самообучения, накладывает \uline{дополнительные требования}, предъявляемые к \textit{информации, хранимой в памяти кибернетической системы}, \textit{к решателю задач кибернетической системы}, а в перспективе также и к \textit{физической оболочке кибернетической системы}.}
    \scnidtf{способность кибернетической системы повышать уровень своего интеллекта --- (1) общий (интегральный) уровень качества информации, хранимый в собственной памяти, (2) общий уровень качества своих приобретаемых навыков, (3) уровень своей обучаемости.}
    \scnidtf{способность кибернетической системы к максимально возможной \uline{самостоятельной эволюции}, в процессе которой кибернетическая система сама постоянно заботится о своей эволюции и о повышении темпов этой эволюции}
    \scntext{примечание}{Важнейшей характеристикой \textit{кибернетической системы} является не только то, какой уровень интеллекта (интеллектуальных возможностей) \textit{кибернетическая система} имеет в текущий момент, какое множество действий (задач) она способна выполнять, но и то, насколько быстро этот уровень может повышаться.}
    
    \scnheader{следует отличать*}
    \begin{scnhaselementset}
        \scnitem{образованность кибернетической системы}
        \begin{scnindent}
            \scnidtf{навыки и другие знания, которые кибернетическая приобрела (с учителем, экспертом или самостоятельно) к заданному моменту}
            \scnidtf{результат, который кибернетическая система достигла в процессе своей эволюции к заданному моменту}
        \end{scnindent}
        \scnitem{обучаемость кибернетической системы}
        \begin{scnindent}    
            \scnidtf{скорость повышения уровня образованности кибернетической системы}
            \scnidtf{скорость эволюции кибернетической системы}
        \end{scnindent}
        \scnitem{скорость повышения уровня обучаемости кибернетической системы}
        \begin{scnindent}
            \scnidtf{ускорение повышения уровня образованности кибернетической системы}
            \scntext{примечание}{с увеличением объема и качества приобретаемых кибернетической системой новых навыков и знаний и, в первую очередь, при грамотной их систематизации скорость обучения кибернетической системы существенно возрастает.}
        \end{scnindent}
    \end{scnhaselementset}

    \scnheader{обучаемость кибернетической системы}
    \scnrelfrom{комплекс свойств-предпосылок}{Комплекс свойств, определяющих обучаемость кибернетических систем по уровню обучаемости различных их компонентов}
    \scnrelfrom{комплекс частных свойств}{Комплекс свойств кибернетических систем, определяющих их обучаемость по различным формам обучения}
    
    \scnheader{Комплекс свойств, определяющих обучаемость кибернетических систем по уровню их гибкости, стратифицированности, рефлексивности, активности}
    \begin{scneqtoset}
        \scnitem{гибкость кибернетической системы}
        \scnitem{стратифицированность кибернетической системы}
        \scnitem{рефлексивность кибернетической системы}
        \scnitem{ограниченность обучения кибернетической системы}
        \scnitem{познавательная активность кибернетической системы}
        \scnitem{способность кибернетической системы к самосохранению}
    \end{scneqtoset}

    \scnheader{гибкость возможных самоизменений кибернетической системы}
    \scnidtf{гибкость кибернетической системы при выполнении ею изменений над самой этой системой}
    \begin{scnrelfromset}{комплекс свойств-предпосылок}

        \scnitem{простота возможных самоизменений кибернетической системы}
        \scnitem{многообразие возможных самоизменений кибернетической системы }

    \end{scnrelfromset}
    \begin{scnrelfromset}{комплекс частных свойств}

        \scnitem{семантическая гибкость обработки информации, хранимой в памяти кибернетической системы}
        \scnitem{семантическая гибкость возможных самоизменений решателя задач кибернетической системы}
        \scnitem{гибкость возможных изменений физической оболочки кибернетической системы, осуществляемых самой системой}

    \end{scnrelfromset}

    \scnheader{гибкость возможных самоизменений кибернетической системы}
    \scnidtf{гибкость кибернетической системы при её самосовершенствовании}
    \scnrelto{частное свойство}{гибкость кибернетической системы}
    \begin{scnindent}
        \scntext{примечание}{Поскольку обучение всегда сводится к внесению тех или иных изменений в обучаемую кибернетическую систему, без высокого уровня гибкости этой системы не может быть высокого уровня её обучаемости.}
        \scnrelto{свойство-предпосылка}{обучаемость кибернетической системы}
    \end{scnindent}
    
    \scnheader{простота возможных самоизменений кибернетической системы}
    \scnidtf{легкость (трудоемкость) внесения различных изменений в кибернетическую систему, осуществляемых самой этой кибернетической системой}
    \scnidtf{приспособленность кибернетической системы к самостоятельному внесению различных изменений в саму себя}
    
    \scnheader{стратифицированность кибернетической системы}
    \scnidtf{иерархическая декомпозиция кибернетической системы на такие подсистемы, структура и функционирование которых минимально возможным образом связаны друг с другом, что существенным образом сужает область учета последствий различных изменений вносимых в систему, а также область поиска причин всевозможных ошибок}
    \scnidtf{модульность кибернетической системы}
    \scnidtf{возможность разделить кибернетическую систему на такие части (страты), эволюция (изменения) которых может осуществляться независимо друг от друга.}
    \scntext{примечание}{Уровень стратифицированности определяется \begin{scnitemize}
            \item количеством страт;\item степенью зависимости страт друг от друга. \end{scnitemize}
    }\scntext{примечание}{При наличии стратифицированности кибернетической системы появляется возможность четкого определения области действия различных изменений, вносимых в кибернетическую систему, т.е. возможность четкого ограничения тех частей кибернетической системы, за пределы которых нет необходимости выходить для учета последствий внесенных в систему первичных изменений, т.е. осуществлять \uline{дополнительные} изменения, являющиеся последствиями первичных изменений.}\scntext{примечание}{Стратификация кибернетической системы --- это не просто её структуризация (прежде всего, структуризация информации, хранимой в памяти кибернетической системы), а такая её структуризация, которая четко определяет границы учета возможных последствий вносимых в систему изменений различного вида.}\begin{scnrelfromlist}{частное свойство}

        \scnitem{стратифицированность информации, хранимой в памяти кибернетической системы}
        \scnitem{стратифицированность решателя задач кибернетической системы}
        \scnitem{стратифицированность физической оболочки кибернетической системы}

    \end{scnrelfromlist}

    \scnheader{рефлексивность кибернетической системы}
    \scnidtf{уровень (степень) рефлексивности кибернетической системы}
    \scnidtf{способность кибернетической системы к самоанализу (к анализу интегрального уровня своего качества и, в том числе, уровня своего интеллекта)}
    \scnidtf{способность кибернетической системы самостоятельно анализировать (оценивать) свое качество}
    \scnidtf{уровень рефлексии кибернетической системы}
    \scnidtf{способность кибернетической системы к самоанализу --- к анализу своих знаний, навыков, своих действий во внутренней и внешней среде}
    \scnidtf{способность кибернетической системы к самонаблюдению и самоанализу}
    \scnidtf{способность кибернетической системы к рефлексии}
    \scnidtf{способность кибернетической системы к анализу своего качества}
    \scnidtf{Способность кибернетической системы к самоанализу (к анализу самой себя во всевозможных аспектах).}
    \scntext{примечание}{Конструктивным результатом рефлексии кибернетической системы является генерация в её памяти спецификации различных негативных или подозрительных особенностей, которые следует учитывать для повышения качества кибернетической системы. Такими особенностями (недостатками) могут быть выявленные противоречия (ошибки), выявленные пары синонимичных знаков, омонимичные знаки, информационные дыры и многое другое.}
    \begin{scnrelfromlist}{частное свойство}
        \scnitem{способность кибернетической системы к анализу качества информации, хранимой в её памяти}
        \scnitem{способность кибернетической системы к анализу качества своего решателя задач}
        \begin{scnindent}
            \scnrelfrom{частное свойство}{способность кибернетической системы к анализу качества своего поведения во внешней среде}
        \end{scnindent}
        \scnitem{пособность кибернетической системы к анализу качества своей физической оболочки}
        \begin{scnindent}    
            \scnrelfrom{частное свойство}{способность кибернетической системы к анализу качества физического обеспечения своего интерфейса с внешней средой}
        \end{scnindent}
    \end{scnrelfromlist}

    \scnheader{ограниченность обучения кибернетической системы}
    \scntext{пояснение}{Данное свойство определяет границу между теми знаниями и навыками, которые соответствующая \textit{кибернетическая система} принципиально может приобрести, и теми знаниями и навыками, которые указанная кибернетическая система не сможет приобрести никогда. Данное свойство определяет максимальный уровень потенциальных возможностей соответствующей кибернетической системы. Очевидно, что максимальная степень отсутствия ограничений в приобретении новых знаний и навыков --- это полное отсутствие ограничений, т.е. полная универсальность возможностей соответствующих кибернетических систем, которые всё могут познать и всё могут сотворить.}\scnidtf{максимум того, чему кибернетическая система может обучиться}
    \scnidtf{максимальная перспектива обучения кибернетической системы}
    \scnidtf{максимальный уровень качества, который кибернетическая система может достичь в процессе обучения}
    \begin{scnrelfromlist}{частное свойство}
        \scnitem{максимальный объём знаний, которые кибернетическая система может приобрести в процессе обучения}
        \scnitem{максимальный объём навыков, которые кибернетическая система может приобрести в процессе обучения}
    \end{scnrelfromlist}

    \scnheader{максимальный объём знаний, которые кибернетическая система может приобрести в процессе обучения}
    \scnidtf{граница приобретаемых знаний, за пределы которой кибернетическая система принципиально не может перейти в процессе своего обучения}
    \scnidtf{максимум того, чему можно научить соответствующую кибернетическую систему}
    \scnidtf{максимальный объём знаний, которые кибернетическая система принципиально может приобрести}
    \scnrelto{свойство-предпосылка}{обучаемость}
    \scntext{примечание}{чем больше \textit{максимальный объём знаний, которые кибернетическая система принципиально может приобрести}, тем выше уровень \textit{обучаемости} кибернетической системы}
    
    \scnheader{познавательная активность кибернетической системы}
    \scnidtf{познавательная мотивированность}
    \scnidtf{познавательная пассионарность}
    \scnidtf{любознательность}
    \scnidtf{активность и самостоятельность в приобретении новых знаний и навыков}
    \scnidtf{стремление, активная целевая установка к постоянному совершенствованию (повышению качества) и пополнению собственной базы знаний}
    \scntext{примечание}{Следует отличать
        \begin{scnitemize}
            \item способность (возможность) приобретать новые знания и навыки и совершенствовать приобретенные знания и навыки
            \item от желания (стремления) это делать.
        \end{scnitemize}}
    \scntext{примечание}{желание (целевая установка) научиться решать те или иные задачи может быть сформулировано кибернетической системой либо самостоятельно, либо извне (некоторым учителем).}\begin{scnrelfromlist}{частное свойство}

        \scnitem{активность в изучении внешней среды}
        \scnitem{активность в анализе качества информации, хранимой в собственной памяти}
        \scnitem{активность в анализе собственных действий и действий других кибернетических систем}

    \end{scnrelfromlist}
    \begin{scnrelfromlist}{свойство-предпосылка}

        \scnitem{способность кибернетической системы к синтезу познавательных целей и процедур}
        \scnitem{способность кибернетической системы к самоорганизации собственного обучения}
        \scnitem{способность кибернетической системы к экспериментальным действиям}

    \end{scnrelfromlist}

    \scnheader{способность кибернетической системы к синтезу познавательных целей и процедур}
    \scnidtf{способность планировать своё обучение и управлять процессом обучения}
    \scnidtf{умение задавать вопросы или целенаправленные последовательности вопросов самому себе или другим субъектам как важнейший фактор обучаемости}
    \scnidtf{способность генерировать (формулировать, задавать) вопросы, адресуемые либо самому себе, либо некоторому внешнему источнику знаний и направленные на повышение качества собственных знаний и навыков}
    \scnidtf{способность генерировать четкую спецификацию своей информационной потребности}
    \scnidtf{способность кибернетической системы четко формулировать то, что она не знает (в частности, не умеет), но хотела бы знать и уметь}
    \scnidtf{способность к формированию спецификаций информационных баз в своих знаниях}
    \scnidtf{способность кибернетической системы самостоятельно генерировать цели на приобретение знаний и навыков, обеспечивающих решение различных классов задач}
    
    \scnheader{способность кибернетической системы к самоорганизации собственного обучения}
    \scnidtf{способность осуществлять управление своим обучением}
    \scnidtf{способность кибернетической системы самой выполнять роль своего учителя, организующего процесс своего обучения}
    
    \scnheader{способность кибернетической системы к экспериментальным действиям}
    \scnidtf{способность к отклонениям от составленных планов своих действий для повышения качества результата или сохранении целенаправленности этих действий}
    \scnidtf{способность к экспромтам и импровизации}
    
    \scnheader{способность кибернетической системы к самосохранению}
    \scnidtf{способность кибернетической системы к выявлению и устранению угроз, направленных на снижение её качества и даже на её уничтожение, что означает полную потерю необходимого качества}
    \scnidtf{уровень самообеспечения безопасности (защищенности) кибернетической системы}
    \scntext{пояснение}{Данное свойство кибернетических систем является необходимым фактором высокого уровня обучаемости кибернетических систем. Чем выше уровень безопасности кибернетической системы, тем выше её уровень обучаемости.}\scnidtf{способность кибернетической системы к обеспечению собственной безопасности}
    \begin{scnrelfromlist}{свойство-предпосылка}
        \scnitem{способность кибернетической системы анализировать смысл задач, инициированных извне, и отказываться от решения вредных задач}
    \end{scnrelfromlist}
    \begin{scnindent}
        \scntext{эпиграф}{Прежде, чем выполнять приказ, подумай}
        \scntext{пояснение}{Примером вредной задачи для \textit{ostis-системы} является запрос всех хранимых в памяти \textit{sc-элементов}}
        \scntext{пояснение}{Подчеркнем, что в современных компьютерных системах и интеллектуальных компьютерных системах подходы к обеспечению их информационной безопасности имеют принципиальные отличия, связанные, прежде всего с тем интеллектуальные компьютерные системы обладают более мощными средствами семантического и контекстного анализа приобретаемой информации.}
        \begin{scnindent}
            \scntext{детализация}{\nameref{sd_inf_security}}
        \end{scnindent}
    \end{scnindent}

    \bigskip\scnheader{Комплекс свойств, определяющих обучаемость кибернетических систем по уровню обучаемости различных их компонентов}
    \begin{scneqtoset}
        \scnitem{способность кибернетической системы к повышению качества информации хранимой в её памяти}
        \scnitem{способность кибернетической системы к повышению качества своего решателя задач}
        \scnitem{способность кибернетической системы к повышению качества своей физической оболочки}
    \end{scneqtoset}

    \scnheader{способность кибернетической системы к повышению качества информации, хранимой в её памяти}
    \scnidtf{способность кибернетической системы к постоянному пополнению и совершенствованию информации, хранимой в её памяти, по всевозможным направлениям и, в первую очередь, в направлении повышения уровня адекватности (корректности) и полноты описания своей внешней среды и своей физической оболочки}
    \begin{scnrelfromlist}{свойство-предпосылка}

        \scnitem{семантическая гибкость информации, хранимой в памяти кибернетической системы}
        \scnitem{стратифицированность информации, хранимой в памяти кибернетической системы}
        \scnitem{способность кибернетической системы к повышению уровня структуризации информации, хранимой в памяти кибернетической системы}
        \scnitem{способность кибернетической системы к анализу качества информации, хранимой в её памяти}
        \scnitem{способность кибернетической системы к устранению противоречий, обнаруженных в информации, хранимой в её памяти}
        \scnitem{способность кибернетической системы к устранению информационных дыр, обнаруженных в информации, хранимой в её памяти}
        \scnitem{способность кибернетической системы к удалению информационного мусора, обнаруженного в информации, хранимой в её памяти}
        \scnitem{способность кибернетической системы к погружению новых \textit{знаний} в состав информации, хранимой в её памяти}
        \scnitem{способность кибернетической системы к обнаружению сходств в знаниях, хранимых в её памяти}
        \scnitem{способность кибернетической системы к конвергенции знаний, хранимых в её памяти}
        \scnitem{способность кибернетической системы к интеграции знаний, хранимых в её памяти}
        \scnitem{способность кибернетической системы к обобщениям и формированию новых понятий}
        \scnitem{способность кибернетической системы к генерации гипотез и обнаружению закономерностей в информации, хранимой в её памяти}
        \scnitem{способность кибернетической системы к обоснованию или опровержению знаний, хранимых в её памяти}
        \scnitem{способность кибернетической системы к экспериментальному подтверждению или опровержению гипотез о свойствах динамических систем с помощью имитационных моделей этих систем}
        \scnitem{способность кибернетической системы к коррекции теорий, хранимых в её памяти}

    \end{scnrelfromlist}

    \scnheader{семантическая гибкость информации, хранимой в памяти кибернетической системы}
    \scnidtf{гибкость информации, хранимой в памяти кибернетической системы, при её обработке на семантическом уровне}
    \scnidtf{гибкость возможных действий (операций), выполняемых кибернетической системой над информацией, хранимой в её памяти, и осуществляемых на семантическом (осмысленном) уровне представления этой информации}
    \scnidtf{трудоёмкость содержательного редактирования информации, хранимой в памяти кибернетической системы (поиска, удаления, вставки, замены различных фрагментов информации), при соблюдении семантической целостности и корректности всей редактируемой информации}
    \scntext{примечание}{Обработка информации на семантическом уровне предполагает такие операции над хранимой информацией, как:\begin{scnitemize}
            \item замена имени некоторой сущности \item поиск связи заданного вида между знаками заданных сущностей и корректировка этой связи\item поиск семантической окрестности знака заданной сущности, то есть поиск всех известных связей, инцидентных этому знаку и, соответственно, всех смежных ему знаков\item поиск фрагмента хранимой информации, релевантного заданному семантическому образцу ---  конфигурации знаков сущностей и связей между ними\item удаление или генерация (порождение) связи между заданными знаками\end{scnitemize}
    }\scntext{примечание}{Все операции семантического уровня обработки информации рассматривают обрабатываемую информацию на абстрактном уровне знаков описываемых сущностей и знаков связей между описываемыми сущностями. При этом указанные связи рассматриваются как частный вид описываемых (и, соответственно, обозначаемых) сущностей.}\scnidtf{простота и многообразие редактирования информации, хранимой в памяти кибернетической системы}
    \scnidtf{простота и многообразие внесения изменений в информацию, хранимую в памяти кибернетической системы}
    \scntext{пояснение}{\textit{Гибкость обработки информации, хранимой в памяти кибернетической системы}, определяется не столько трудоемкостью непосредственно самой операции редактирования, сколько теме дополнительными действиями, которые являются обязательными последствиями каждой такой операции редактирования. Так, например, изменение имени какой-либо описываемой сущности требует внесения этого изменения во всех местах, где это имя упоминается, удаление какой-либо связи между известными описываемыми сущностями требует внесения этого изменения везде, где удаляемая связь упоминается.}
    
    \scnheader{стратифицированность информации, хранимой в памяти кибернетической системы}
    \scnidtf{логико-семантическая стратифицированность информации, хранимой в памяти кибернетической системы}
    \scnrelfrom{свойство-предпосылка}{структуризация информации, хранимой в памяти кибернетической системы}
    \scnrelfrom{свойство-предпосылка}{качество метаязыковых средств представления информации, хранимой в памяти кибернетической системы}
    \begin{scnindent}
        \scnidtf{уровень развития метаязыковых средств кодирования (внутреннего представления) информации, хранимой в памяти кибернетической системы}
    \end{scnindent}

    \scnheader{способность кибернетической системы к повышению уровня структуризации информации, хранимой в памяти указанной системы}
    \scnrelboth{следует отличать}{структурированность информации, хранимой в памяти кибернетической системы}
    \begin{scnindent}
        \scnidtf{уровень структуризации информации, хранимой в памяти кибернетической системы}
        \scntext{примечание}{Качественная структуризация информации, хранимой в памяти кибернетической системы, то есть качественное разложение  этой информации по семантическим полочкам  существенно упрощает и, следовательно, ускоряет повышение качества самой этой информации.}\scnrelboth{следует отличать}{структуризация информации, хранимой в памяти кибернетической системы}
        \begin{scnindent}
            \scnsubset{действие, выполняемое кибернетической системой в своей памяти}
            \begin{scnindent}
                \scnsubset{процесс}
            \end{scnindent}
        \end{scnindent}
    \end{scnindent}

    \scnheader{способность кибернетической системы к анализу качества информации, хранимой в её памяти}
    \scnidtf{способность кибернетической системы к анализу информации, хранимой в собственной памяти, для последующего повышения качества этой информации}
    \scnrelto{частное свойство}{способность кибернетической системы к рефлексии}
    \scntext{примечание}{Рефлексия кибернетической системы, то есть анализ собственного качества, включает в себя не только анализ качества информации, хранимой в её памяти, но и анализ собственной деятельности как во внешней среде, так и в собственной памяти. При этом анализ собственной деятельности сводится к анализу описания этой деятельности, представленного в собственной памяти.}
    \begin{scnrelfromlist}{свойство-предпосылка}
        \scnitem{качество метаязыковых средств описания в памяти кибернетической системы качества информации, хранимой в её памяти}
        \scnitem{способность кибернетической системы к обнаружению противоречий в информации, хранимой в её памяти}
        \begin{scnindent}
            \begin{scnrelfromlist}{частное свойство}
                \scnitem{способность кибернетической системы к обнаружению пар синонимичных знаков, входящих в состав информации, хранимой в её памяти}
                \scnitem{способность кибернетической системы к обнаружению семантически эквивалентных фрагментов, входящих в состав информации, хранимой в её памяти}
                \scnitem{способность кибернетической системой к обнаружению омонимичных знаков в информации, хранимой в её памяти}
            \end{scnrelfromlist}
        \end{scnindent}
        \scnitem{способность кибернетической системы к обнаружению информационных дыр в информации, хранимой в её памяти}
        \scnitem{способность кибернетической системой к обнаружению информационного мусора в информации, хранимой в её памяти}
    \end{scnrelfromlist}

    \scnheader{способность кибернетической системы к устранению противоречий, обнаруженных в информации, хранимой в её памяти}
    \begin{scnrelfromlist}{частное свойство}

        \scnitem{способность кибернетической системы к устранению синонимии знаков, входящих в состав информации, хранимой в памяти указанной системы}
        \scnitem{способность кибернетической системы к устранению семантической эквивалентности фрагментов, входящих в состав информации, хранимой в памяти указанной системы}
        \scnitem{способность кибернетической системы к устранению омонимичных знаков, входящих в состав информации, хранимой в памяти указанной системы}
        \scnitem{способность кибернетической системы к устранению противоречий, обнаруженных в информации, хранимой в памяти указанной системы, и не являющихся обнаруженной синонимией, семантической эквивалентностью или омонимией}

    \end{scnrelfromlist}

    \scnheader{способность кибернетической системы к устранению семантической эквивалентности фрагментов, входящих в состав информации, хранимой в памяти указанной системы}
    \scnidtf{способность кибернетической системы к устранению дублирования информации в рамках памяти указанной системы}
    
    \scnheader{следует отличать*}
    \begin{scnhaselementset}
        \scnitem{семантическая эквивалентность*}
        \begin{scnindent}    
            \scnidtf{эквивалентность информационных конструкций по смыслу (содержанию)*}
        \end{scnindent}
        \scnitem{синтаксическая эквивалентность*}
        \begin{scnindent}
            \scnidtf{эквивалентность информационных конструкций по форме*}
        \end{scnindent}
        \scnitem{логическая эквивалентность*}
        \begin{scnindent}
            \scnidtf{пары информационных конструкций, первая из которых логически следует из второй и наоборот*}
            \scntext{примечание}{Если с семантической эквивалентности в памяти кибернетической системы можно и нужно бороться, то без логической эквивалентности обойтись трудно (как минимум из-за необходимости вводить определяемые понятия и, соответственно, формулировать определения). Тем не менее, логической эквивалентностью и, в частности, расширением числа определяемых понятий увлекаться не следует. Так, например, если определение нового понятия не является громоздким (в частности, понятия, являющегося теоретико-множественным объединением или пересечением ранее введенных понятий), то явно вводить это новое понятие не следует.}
        \end{scnindent}
    \end{scnhaselementset}

    \scnheader{способность кибернетической системы к устранению информационных дыр, обнаруженных в информации, хранимой в ее памяти}
    \scnrelfrom{свойство-предпосылка}{способность кибернетической системы генерировать ответы на вопросы различного вида в случае, если они целиком или частично отсутствуют в текущем состоянии информации, хранимой в памяти}
    \scntext{примечание}{Формальным результатом обнаружения информационной дыры является формулировка запроса на недостающую информацию, которую необходимо сгенерировать.}
    
    \scnheader{способность кибернетической системы к удалению информационного мусора, обнаруженного в информации, хранимой в ее памяти}
    \scnidtf{способность кибернетической системы к забыванию (стиранию, удалению) ненужной (лишней, отработанной ) информации, которая, например, играет роль информационных лесов  при решении различных задач}
    \scntext{примечание}{Критериями информационного мусора может быть:\begin{scnitemize}
            \item завершение решения задачи, для которой данная информация является вспомогательной и востребованной только в рамках решения соответствующей задачи;\item истечение срока давности хранения;\item легкая воспроизводимость (при необходимости).\end{scnitemize}
    }
    
    \scnheader{способность кибернетической системы к семантическому погружению новых знаний в состав информации, хранимой в ее памяти}
    \scntext{примечание}{Новая введенная в память информационная конструкция трактуется как конструкция, у которой входящие в нее знаки являются потенциальными синонимами тем знакам, которые уже присутствуют в хранимой информации. Поэтому для всех этих знаков надо проверить наличие их синонимов. После этого синонимичные знаки должны быть отождествлены. Отождествление знаков осуществляется либо путем приписывания им одинаковых идентификаторов (имен), либо путем физического  склеивания этих знаков.}\scntext{примечание}{Новой информацией, погружаемой (вводимой) в состав информации, хранимой в памяти кибернетической системы, может быть:\begin{scnitemize}
            \item либо принятое сообщение, поступившее от другой кибернетической системы и переведенное на внутренний язык данной системы;\item либо информация, сгенерированная в результате решения какой-либо задачи.\end{scnitemize}
    }
    
    \scnheader{способность кибернетической системы к обнаружению сходств в знаниях, хранимых в ее памяти}
    \scntext{примечание}{Сходства в знаниях могут иметь самый разнообразный вид и далеко не всегда являются очевидными.}\scntext{примечание}{Умение видеть  сходство в различном и различие в сходном является важнейшим признаком интеллекта.}
    
    \scnheader{способность кибернетической системы к конвергенции знаний, хранимых в ее памяти}
    \scnidtf{способность кибернетической системы к увеличению сходств в знаниях хранимых в ее памяти}
    \scnrelfrom{свойство-предпосылка}{способность к увеличению числа общих понятий для различных фрагментов информации, хранимой в памяти кибернетической системы, без ущерба качеству этих фрагментов}
    \scnidtf{способность к сближению  знаний путем:\begin{scnitemize}
            \item увеличения числа общих понятий, используемых в сближаемых  знаниях;\item преобразования исходных знаний к их логически эквивалентным вариантам в целях получения фрагментов как можно большего размера и как можно в большем количестве, которые были бы:\begin{scnitemizeii}
            \item либо синтаксически изоморфными и содержащими как можно большее число общих понятий;\item либо синтаксически изоморфными и одновременно семантически эквивалентными.\end{scnitemizeii}
        \end{scnitemize}
    }

    \scnheader{способность кибернетической системы к интеграции знаний, хранимых в ее памяти}
    \scnidtf{способность объединять имеющиеся знания и формировать целостную картину различных исследуемых объектов, систем, процессов, явлений}
    \scnrelfrom{свойство-предпосылка}{способность кибернетической системы к конвергенции знаний, хранимых в ее памяти}
    \scntext{примечание}{Качество (глубина) интеграции знаний определяется тем, насколько качественно до этого была проведена конвергенция интегрируемых знаний.}\scntext{примечание}{Качественная (бесшовная , глубокая) интеграция различных знаний, хранимых в памяти кибернетической системы, дает возможность существенно снизить количество хранимых в памяти методов решения задач, так как позволяет некоторые ранее различные классы задач объединить в один класс задач. При этом очевидно, что такая интеграция знаний, хранимых в памяти кибернетической системы, требует разработки \uline{общих} (базовых) синтаксических и семантических принципов представления знаний различного вида.}
    
    \scnheader{конвергенция и интеграция знаний}
    \scntext{примечание}{Мы вынуждены смотреть на окружающую нас внешнюю среду (внешний мир) через замочную скважину  своих сенсоров (рецепторов), своих персональных точек зрения, мировоззрения различных научных дисциплин. Но необходимо помнить, что целостную картину внешней среды (картину мира) можно построить только путем сближения (конвергенции) и соединения (интеграции) самых различных точек зрения, самых различных научных дисциплин и направлений. Мир не делится на различные дисциплины --- он един. Для эффективного решения задач конвергенции и интеграции знаний необходимо построить искусственную (рукотворную) среду (память), в которой было бы удобно не только хранить самые различные знания, но и осуществлять конвергенцию и интеграцию этих знаний. При этом очень важно, чтобы формируемая таким образом информационная модель окружающей нас внешней среды (информационной картины мира) была общедоступна как для просмотра (ознакомления), причем, без каких бы то ни было замочных скважин , так и для ввода новых знаний, представляющих (отражающих) точку зрения их авторов.}\scnrelfrom{эпиграф}{Древнеиндийская притча о слоне и слепцах}
    
    \scnheader{следует отличать*}
    \begin{scnhaselementset}

        \scnitem{конвергенция\scnsupergroupsign}
        \begin{scnindent}
            \scnidtf{Свойство, определяющее степень близости (уровень конвергенции) между двумя заданными сущностями и, в частности, знаниями}
        \end{scnindent}
        \scnitem{конвергенция*}
        \begin{scnindent}
            \scnidtf{Множество пар близких (аналогичных, сходных) сущностей*}
        \end{scnindent}
        \scnitem{конвергенция}
        \begin{scnindent}
            \scnidtf{Множество \uline{процессов} сближения различных пар сущностей}
        \end{scnindent}

    \end{scnhaselementset}
    \begin{scnhaselementset}

        \scnitem{интеграция*}
        \begin{scnindent}
            \scnidtf{Квазибинарное \uline{отношение}, каждая пара которого связывает множество интегрируемых сущностей с результатом интеграции*}
        \end{scnindent}
        \scnitem{интеграция}
        \begin{scnindent}
            \scnidtf{Множество \uline{процессов} интеграции множества заданных сущностей}
        \end{scnindent}

    \end{scnhaselementset}

    \scnheader{способность кибернетической системы к обобщениям и формированию новых понятий}
    \scntext{примечание}{Важным примером обобщения является переход от задач к классам часто решаемых задач.}
    
    \scnheader{cпособность кибернетической системы к генерации гипотез и обнаружению закономерностей в информации, хранимой в ее памяти}
    \scntext{примечание}{Данная способность кибернетической системы является важнейшим фактором эволюции информации, хранимой в памяти кибернетической системы, в направлении перехода от данных (от фактографической информации) к знаниям.}
    
    \scnheader{cпособность кибернетической системы к обоснованию или опровержению знаний, хранимых в ее памяти}
    \scntext{примечание}{Примерами знаний, подлежащих обоснованию или опровержению, являются:\begin{scnitemize}
            \item любое введенное в кибернетическую систему сообщение (любая новая информация, поступающая от любого субъекта);\item формулировка какой-либо задачи, предлагаемой для решения;\item формулировка какого-либо гипотетического утверждения (теоремы), подлежащего доказательству.\end{scnitemize}
    }\scnidtf{способность к объяснению (обоснованию, аргументации) корректности, важности и целесообразности использовать (обратить внимание на) указываемое знание}
    \scnidtf{способность либо находить в текущем состоянии базы знаний, либо генерировать (строить) ответы на \textit{почему-вопросы}}
    
    \scnheader{способность кибернетической системы к экспериментальному подтверждению или опровержению гипотез о свойствах динамических систем с помощью имитационных моделей этих систем}
    \scntext{примечание}{Создание динамических информационных моделей сложных динамических систем и проведение различного рода мысленных  экспериментов с такими моделями является весьма перспективным и мощным методом исследования сложных динамических систем.}
    
    \scnheader{способность кибернетической системы к коррекции теорий, хранимых в ее памяти}
    \scnidtf{способность к адаптации накопленных знаний к различным изменениям условий и жизненных ситуаций}
    \scntext{примечание}{В основе данного свойства кибернетической системы лежит:\begin{scnitemize}
            \item постоянная готовность кибернетической системы подвергнуть сомнению любое знание, хранимое в ее памяти;\item постоянное уточнение степени достоверности каждого знания, хранимого в памяти кибернетической системы.\end{scnitemize}
    }
    \bigskip\scnheader{способность кибернетической системы к повышению качества своего решателя задач}
    \scnidtf{способность кибернетической системы повышать качество своих приобретаемых навыков}
    \begin{scnrelfromlist}{свойство-предпосылка}

        \scnitem{способность кибернетической системы к повышению качества информации, хранимой в ее памяти}
        \scnitem{семантическая гибкость возможных самоизменений решателя задач кибернетической системы}
        \scnitem{стратифицированность решателя задач кибернетической системы}
        \scnitem{способность кибернетической системы к анализу качества своего решателя задач}
        \scnitem{способность кибернетической системы к целенаправленной коррекции своей деятельности}
        \scnitem{способность кибернетической системы к оптимизации хранимых в памяти методов решения задач}
        \scnitem{способность кибернетической системы к генерации новых методов решения задач}
        \scnitem{способность кибернетической системы интегрировать у себя новые приобретаемые извне методы и модели решения задач}

    \end{scnrelfromlist}
    \scnheader{семантическая гибкость возможных самоизменений решателя задач кибернетической системы}
    \scnidtf{простота реализации решателем задач кибернетической системы различного рода изменений самого себя}
    \scntext{примечание}{Очевидно, что семантическая гибкость решателя задач кибернетической системы во многом определяется процессором кибернетической системы (прежде всего, его универсальностью и близостью реализуемой им модели обработки информации к смысловому уровню). Но, поскольку решатель задач кибернетической системы кроме процессора включает в себя хранимые в памяти кибернетической системы методы решения различного вида задач (в том числе, и методы интерпретации методов высокого уровня), семантическая гибкость решателя задач определяется также \textit{семантической гибкостью информации, хранимой в памяти кибернетической системы}.}\scnrelfrom{свойство-предпосылка}{семантическая гибкость информации, хранимой в памяти кибернетической системы}
    \scnheader{стратифицированность решателя задач кибернетической системы}
    \begin{scnrelfromlist}{частное свойство}

        \scnitem{стратифицированность методов и навыков решения задач, представленных в памяти кибернетической системы}
        \scnitem{стратифицированность технологий, соответствующих различным видам деятельности}
        \scnitem{стратифицированность различного вида действий, классов действий и видов деятельности}

    \end{scnrelfromlist}
    \begin{scnindent}
        \scnrelfrom{частное свойство}{стратифицированность различного вида информационных процессов, выполняемых в памяти кибернетической системы}
    \end{scnindent}

    \scnheader{качество внутренних языковых средств кибернетической системы для описания качества собственного решателя задач}
    \scnrelfrom{свойство-предпосылка}{качество внутренних языковых средств кибернетической системы для описания качества собственных действий}
    
    \scnheader{способность кибернетической системы к анализу качества своего решателя задач}
    \scnidtf{способность кибернетической системы к анализу (к оценке качества) своей деятельности в собственной внутренней среде (в своей памяти), а также в своей внешней среде}
    \scntext{примечание}{Анализ качества решателя задач включает в себя:\begin{scnitemize}
            \item анализ качества используемых методов и технологий решения задач;\item анализ качества используемых моделей решения задач;\item анализ полноты набора постоянно инициированных целей (задач), направленных на эволюцию и на борьбу с деградацией (снижением качества) кибернетической системы;\item анализ качества выполняемых действий (процессов решения задач).\end{scnitemize}
    }\scnidtf{способность кибернетической системы к описанию (к построению в своей памяти информационной модели) собственных действий, выполняемых в собственной памяти, а также к анализу и оценке этих действий}
    \scnidtf{способность кибернетической системы к анализу своего поведения в своей внутренней среде (в своей памяти), а также в своей внешней среде и в своей физической оболочке}
    \begin{scnrelfromlist}{свойство-предпосылка}
        \scnitem{качество внутренних языковых средств кибернетической системы для описания качества собственного решателя задач}
        \scnitem{способность кибернетической системы к анализу собственной деятельности}
        \begin{scnindent}
            \begin{scnrelfromlist}{частное свойство}
                \scnitem{способность кибернетической системы к анализу качества информационных процессов, выполняемых в собственной памяти}
                \begin{scnindent}
                    \scnidtf{способность кибернетической системы к анализу качества своего поведения (действий, информационных процессов) в собственной внутренней среде --- своих действий, сводящихся к поиску, генерации, удалению и преобразованию информационных конструкций, хранимых в собственной памяти}
                \end{scnindent}
                \scnitem{способность кибернетической системы к анализу качества своего поведения во внешней среде}
            \end{scnrelfromlist}
        \end{scnindent}
        \scnitem{способность кибернетической системы к анализу качества методов, хранимых в собственной памяти}
        \begin{scnindent}    
            \begin{scnrelfromlist}{частное свойство}
                \scnitem{способность кибернетической системы к анализу качества методов и технологий, используемых ею для выполнения сложных действий в собственной памяти}
                \scnitem{способность кибернетической системы к анализу качества методов и технологий, используемых ею для выполнения сложных действий во внешней среде}
            \end{scnrelfromlist}
        \end{scnindent}
    \end{scnrelfromlist}

    \scnheader{качество внутренних языковых средств кибернетической системы для описания качества собственных действий}
    \scntext{примечание}{В данном свойстве кибернетической системы имеется в виду описание собственных действий, выполняемых кибернетической системой как в своей внутренней среде (в собственной памяти), так и в своей внешней среде.}
    
    \scnheader{способность кибернетической системы к анализу качества своего поведения во внешней среде}
    \scnidtf{способность кибернетической системы к анализу соответствия между тем, что планировалось сделать во внешней среде и тем, что реально получилось}
    \scntext{примечание}{Поведение кибернетической системы во внешней среде рассматривается ею как эксперимент , подтверждающий или опровергающий ее представление о внешней среде.}
    \scnidtf{способность кибернетической системы к анализу своего опыта взаимодействия с внешней средой и, в частности, к выявлению своих ошибок}
   
    \scnheader{способность кибернетической системы к целенаправленной коррекции своей деятельности}
    \scnidtf{способность кибернетической системы к коррекции своего поведения в целях повышения его качества (эффективности)}
    \scnidtf{способность кибернетической системы учиться на ошибках своей деятельности на основе анализа этих ошибок}
    \scnrelfrom{свойство-предпосылка}{способность кибернетической системы к анализу собственной деятельности}
    
    \scnheader{способность кибернетической системы к оптимизации хранимых в памяти методов решения задач}
    \scntext{примечание}{Хранимые в памяти \textit{методы} решения задач разбиваются на следующие классы:
        \begin{scnitemize}
            \item \textit{методы верхнего уровня} --- интерпретируемые методы;
            \item \textit{методы базового уровня}, представленные на базовом языке программирования, который интерпретируется непосредственно процессором кибернетической системы;
            \item \textit{метаметоды}, описывающие интерпретацию методов верхнего уровня.
        \end{scnitemize}
    }
    
    \scnheader{способность кибернетической системы к генерации новых методов решения задач}
    \scntext{примечание}{Целесообразность генерации нового метода решения задач возникает, когда кибернетической системе приходится часто решать эквивалентные задачи некоторого класса. Генерация соответствующего метода и последующая его оптимизация позволяет существенно сократить время решения задач.}\scnidtf{способность кибернетической системы расширять множество используемых ею методов решения задач}
    \scntext{примечание}{Если добавляемые методы соответствуют используемым моделям решения задач, то, кроме добавления самих методов, желательно, чтобы в стратифицированной кибернетической системе никакие другие изменения не потребовались. Если добавляемый метод соответствует новой (ранее не известной) модели решения задач, то желательно, чтобы в стратифицированной кибернетической системе никакие другие изменения не потребовались, кроме добавления агентов, обеспечивающих интерпретацию (описание операционной семантики) методов нового класса.}\scntext{примечание}{Речь идет о методах решения как внутренних задач, решаемых в памяти кибернетической системы, так и внешних задач, решаемых во внешней среде путем управления деятельностью эффекторов и рецепторов кибернетической системы.}\scnrelfrom{свойство-предпосылка}{способность расширять множество использованных моделей решения задач}
    
    \scnheader{способность кибернетической системы интегрировать у себя новые приобретаемые извне методы и модели решения задач}
    \scntext{примечание}{Для обеспечения такой способности необходима:\begin{scnitemize}
            \item разработка универсальной базовой модели решения задач, для которой соответствующие ей методы решения задач интерпретируются процессором кибернетической системы;\item разработка семейства классов методов верхнего уровня, что предполагает:\begin{scnitemizeii}
                \item разработку языков представления методов для каждого класса методов верхнего уровня;\item разработку интерпретаторов для каждого класса методов верхнего уровня на основе указанной выше базовой модели решения задач.\end{scnitemizeii}
        \end{scnitemize}
    }
    \bigskip\scnheader{способность кибернетической системы к повышению качества своей физической оболочки}
    \scnidtf{способность кибернетической системы к самостоятельному совершенствованию (эволюции) своей физической оболочки}
    \scntext{примечание}{Данная способность кибернетической системы накладывает определенные требования к построению ее физической оболочки.}\begin{scnrelfromlist}{свойство-предпосылка}

        \scnitem{гибкость возможных изменений физической оболочки кибернетической системы}
        \scnitem{стратифицированность физической оболочки кибернетической системы}
        \scnitem{способность кибернетической системы к анализу качества своей физической оболочки}
        \begin{scnindent}
            \scnrelfrom{свойство-предпосылка}{качество внутренних языковых средств кибернетической системы для описания качества собственной физической оболочки}
        \end{scnindent}
        \scnitem{способность кибернетической системы расширять и/или совершенствовать набор собственных сенсоров и эффекторов}

    \end{scnrelfromlist}

    \bigskip\scnheader{комплекс свойств кибернетических систем, определяющих их обучаемость по различным формам обучения}
    \begin{scneqtoset}

        \scnitem{обучаемость с учителем}
        \scnitem{самообучаемость с экспертом}
        \scnitem{самообучаемость на основе внешних информационных источников}
        \scnitem{самообучаемость без внешних информационных источников}

    \end{scneqtoset}

    \scnheader{обучаемость с учителем}
    \scnidtf{уровень способности к обучению под управлением внешнего субъекта-учителя}
    \scnidtf{способность кибернетической системы к эффективному обучению с помощью учителя, осуществляющего управление процессом обучения}
    \scnidtf{способность заданной кибернетической системы эффективно обучаться с помощью внешней кибернетической системы (внешнего субъекта, внешнего активного учителя), осуществляющей организацию обучения заданной кибернетической системы на основе различных методик обучения, учитывающих особенности обучаемой системы и определяющих характер (в том числе последовательность) передачи знаний и новыков, а также тестирование качества их усвоения}
    
    \scnheader{самообучаемость с экспертом}
    \scnidtf{способность кибернетической системы к самообучению в диалоге с экспертом-консультантом}
    \scnidtf{способность кибернетической системы не просто задавать нужные для собственного обучения вопросы (информационные цели), но и вести вопросно-ответный диалог с другими субъектами (кибернетическими системами), которые являются экспертами в соответствующей области (указанные эксперты  это своего рода пассивные учителя , которые много знают и умеют в соответствующей области, могут отвечать на вопросы, но не желают управлять процессом передачи этих знаний и умений другим кибернетическим системам)}
    \scnidtf{эффективность самообразования кибернетической системы, в основе которого лежит диалог, управляемый этой обучаемой системой и осуществляемый с кибернетической системой, являющейся носителем востребованных знаний и навыков}
    \scnidtf{эффективность самообучения, осуществляемого в форме консультации}
    \scnidtf{способность управлять процессом самообучения путем формирования последовательности вопросов (познавательных целей), адресуемых внешним субъектам}
    \scnrelfrom{свойство-предпосылка}{способность кибернетической системы к синтезу познавательных целей и процедур}
    
    \scnheader{обучаемость на основе пассивных внешних информационных источников}
    \scnidtf{способность кибернетической системы к извлечению информации, содержащейся во внешних информационных источниках, к поиску нужных внешних источников и к построению на этой основе систематизированной картины мира}
    \scnidtf{эффективность самообучения, основанного на анализе \uline{пассивных} источников информации (документов различного вида, публикаций, текстов, которые необходимо находить в различного рода библиотеках, читать и \uline{понимать})}
    
    \scnheader{самообучаемость без внешних информационных источников}
    \scnidtf{способность кибернетической системы формировать систематизированную модель (картину) окружающей среды, используя для ее непосредственного восприятия и изучения только собственные сенсоры и эффекторы, а также некоторые дополнительные средства, усиливающие возможности сенсоров и эффекторов}
    \scnidtf{эффективность самообучения кибернетической системы, основанного исключительно на собственном опыте, на анализе собственной деятельности и собственных ошибок}
    \scntext{примечание}{Данная способность кибернетической системы является необходимым, но явно недостаточным фактором ее высокого качества. Учиться только на собственном опыте --- существенно понизить уровень своего интеллекта. В этом смысле познавательный процесс социален.}
    \bigskip
\end{scnsubstruct}
\scnsourcecomment{Завершили Сегмент \scnqqi{Комплекс свойств, определяющих уровень обучаемости кибернитической системы}}

		\scnsegmentheader{Комплекс свойств, определяющих качество многоагентной
    системы}
\begin{scnsubstruct}
    \scnheader{многоагентная система}
    \scntext{пояснение}{Переход от \textit{кибернетических систем} к коллективам
        взаимодействующих между собой \textit{кибернетических систем}, т.е. к
        социальной организации кибернетических систем, является важнейшим фактором
        эволюции \textit{кибернетических систем}.}\scnsubset{кибернетическая система}
    \begin{scnsubdividing}

        \scnitem{моногенная многоагентная система}
        \begin{scnindent}    
            \scnidtf{однородная \textit{многоагентная система}, состоящая из однотипных \textit{агентов}}
        \end{scnindent}
        \scnitem{гетерогенная многоагентная система}
        \begin{scnindent}
            \scnidtf{неоднородная \textit{многоагентная система}, состоящая из \textit{агентов} разного типа}
        \end{scnindent}

    \end{scnsubdividing}
    \begin{scnsubdividing}

        \scnitem{простая многоагентная система}
        \begin{scnindent}    
            \scnidtf{многоагентная система, \textit{агенты} которой не являются \textit{многоагентными системами}}
        \end{scnindent}
        \scnitem{иерархическая многоагентная система}
        \begin{scnindent}
            \scnidtf{многоагентная система, некоторые или все \textit{агенты} которой являются \textit{многоагентнымисистемами}}
        \end{scnindent}

    \end{scnsubdividing}
    \scntext{примечание}{Агенты \textit{многоагентной системы} могут (но вовсе не
        обязательно должны) быть \textit{интеллектуальными системами}. Так, например,
        агенты интеллектуального решателя задач, имеющего агентно-ориентированную
        архитектуру, не являются интеллектуальными системами.}
        
    \scnheader{агент*}
    \scnidtf{быть агентом данной многоагентной системы*}
    \scnidtf{быть кибернетической системой, входящей в состав данной многоагентной
        системы*}
    \scntext{примечание}{Агентом иерархической многоагентной системы может быть другая
        многоагентная система}\scnsuperset{член многоагентной системы*}
    \begin{scnindent}
        \scnidtf{агент многоагентной системы, не являющийся агентом другого агента этой системы*}
        \scnidtf{непосредственный (ближайший) агент многоагентной системы*}
    \end{scnindent}

    \scnheader{кибернетическая система}
    \begin{scnsubdividing}
        \scnitem{индивидуальная кибернетическая система}
        \begin{scnindent}
            \scnidtftext{пояснение}{минимальная целостная \textit{кибернетическая система} обладающая достаточно высоким уровнем самостоятельности и способности выживать  в своей \textit{внешней среде}}
        \end{scnindent}
        \scnitem{кибернетическая система, являющаяся минимальным компонентом
            индивидуальной кибернетической системы}
        \begin{scnindent}
            \scntext{пояснение}{Это такой компонент, в состав которого не входят \textit{кибернетические системы}}
        \end{scnindent}
        \scnitem{кибернетическая система, являющаяся комплексом компонентов
            соответствующей индивидуальной кибернетической системы}
        \scnitem{сообщество индивидуальных кибернетических систем}
        \begin{scnindent}    
        \begin{scnsubdividing}
                \scnitem{простое сообщество индивидуальных кибернетических систем}
                \scnitem{иерархическое сообщество индивидуальных кибернетических систем}
            \end{scnsubdividing}
        \end{scnindent}
    \end{scnsubdividing}

    \scnheader{многоагентная система}
    \scnidtf{коллектив взаимодействующих  кибернетических систем}
    \begin{scnsubdividing}

        \scnitem{сообщество индивидуальных кибернетических систем}
        \scnitem{индивидуальная кибернетическая система, реализованная в виде многоагентной системы}
            \begin{scnindent}
                \scnsubset{кибернетическая система, являющаяся комплексом компонентов
                    соответствующей индивидуальной кибернетической системы}
                \scntext{пояснение}{Такая внутренняя	\textit{многоагентная система} в
                    индивидуальной кибернетической системе появляется, когда на определенном этапе
                    её эволюции \textit{решатель задач} \textit{индивидуальной кибернетической системы} переходит  на
                    \textit{агентно-ориентированную модель обработки информации} в памяти \textit{индивидуальной компьютерной системы}}
            \end{scnindent}

    \end{scnsubdividing}
    \scnidtf{кибернетическая система, представляющая собой коллектив
        взаимодействующих кибернетических систем, обладающих определенной степенью
        самостоятельности (самодостаточности, свободы выбора)}

    \scnheader{многоагентная система с централизованным управлением}
    \scnidtf{многоагентная система, в которой специально выделяются агенты, которые
        принимают решения в определенной области деятельности многоагентной системы и
        обеспечивают выполнение этих решений  путем управления деятельностью остальных
        агентов, входящих в состав этой системы}
    \scnsubset{многоагентная система}
    
    \scnheader{сообщество интеллектуальных систем с децентрализованным управлением}
    \scnidtf{многоагентная система с децентрализованным управлением, агентами
        которой являются интеллектуальные системы}
    \scnidtf{многоагентная система, в которой решения принимаются коллегиально и
        автоматически  (\uline{решения} о признании новой кем-то предложенной
        информации --- в том числе, об инициировании некоторой задачи, \uline{решения} о
        коррекции (уточнении) уже ранее признанной (одобренной, согласованной)
        информации) \uline{на основе} четко продуманной и постоянно совершенствуемой
        методики, а также \uline{на основе} активного участия всех агентов в
        формировании новых предложений, подлежащих признанию (одобрению, согласованию)}
    \scnsubset{многоагентная система}
    \scntext{примечание}{В такой многоагентной системе все агенты участвуют в управлении
        этой системы}\scnhaselement{Экосистема OSTIS}
    \scntext{пояснение}{В такой многоагентной системе отсутствуют специально
        назначенные  агенты, которые обязаны  принимать решения о том, какую
        коллективно решаемую задачу надо инициировать, и о том, как распределить между
        агентами подзадачи указанной инициированной задачи.}\scnsubset{многоагентная
        система с децентрализованным управлением}
    \scnsubset{сообщество интеллектуальных систем}
    \scntext{примечание}{Примером такой системы является оркестр, способный играть без
        дирижера. При этом подчеркнем, что каждый музыкант такого
        оркестра:\begin{scnitemize}

            \item должен иметь квалификацию не только музыканта, но и дирижера и даже
            композитора
            \item должен быть договороспособным --- уметь согласовывать свои действия с
            действиями коллег\end{scnitemize}
        Аналогичным примером децентрализованной многоагентной системы является
        строительная бригада, способная построить дом без бригадира, прораба,
        архитектора.}
    
    \scnheader{синергетическая кибернетическая система}
    \scnidtf{эволюционная многоагентная система}
    \scnidtf{многоагентная система, состоящая из когнитивных агентов}
    \scnidtf{многоагентная система, обладающая высоким уровнем коллективного
        интеллекта, атомарными агентами которой являются индивидуальные
        интеллектуальные системы, имеющие высокий уровень социализации}
    \scnrelfrom{пояснение}{Ярушкина.Н.Г.НечетГС-2007кн.-стр.88-101}
    \begin{scnindent}
        \scnrelto{цитата}{\cite{YarushinaHS}}
    \end{scnindent}
    \begin{scnrelfromlist}{пояснение}

        \scnitem{\cite{Tarasov1997}}
        \scnitem{\cite{Tarasov1998}}

    \end{scnrelfromlist}
    \scntext{примечание}{Очевидным примером синергетической кибернетической системы
        является творческий коллектив, реализующий сложный наукоемкий проект. Огромная
        сложность создания таких коллективов является главной причиной медленного
        развития целого ряда весьма актуальных научно-технических проектов, таких как
        создание принципиально нового технологического уровня автоматизации
        человеческой деятельности на основе интеллектуальных семантически совместимых
        компьютерных систем, способных самостоятельно взаимодействовать друг с
        другом.}
        
    \scnheader{многоагентная система}
    \scntext{примечание}{Переход к \textit{многоагентным системам} является важнейшим
        фактором повышения \textit{качества} (и, в частности, уровня
        \textit{интеллекта}) \textit{кибернетических систем}, т.к. уровень интеллекта
        \textit{многоагентной системы} может быть значительно выше уровня интеллекта
        каждого входящего в неё агента. Но это бывает далеко не всегда, поскольку
        важнейшим фактором качества многоагентных систем является не только качество
        входящих в неё агентов, но и организация взаимодействия агентов и, в частности,
        переход от централизованного к децентрализованному управлению. Количество
        далеко не всегда переходит в новое качество.
        \\Повышение уровня интеллекта многоагентной системы
        обеспечивается\begin{scnitemize}

            \item не только повышением уровня интеллекта и, в первую очередь, уровня
            \textit{социализации} ее агентов;
            \item не только переходом от централизованного к децентрализованного управлению
            деятельности управлению деятельностью агентов;
            \item но и качеством общей базы знаний всей многоагентной
            системы.\end{scnitemize}
    }
    
    \scnheader{социализация кибернетической системы}
    \scntext{примечание}{Когда мы говорим о \textit{социализации кибернетических систем},
        речь идет только об \textit{индивидуальных кибернетических системах}, т.е. о
        \textit{кибернетических  системах}, достигших некоторого уровня целостности и
        автономности и способных входить в состав различных коллективов. Соответственно
        этому, качество \textit{индивидуальных кибернетических систем} определяется,
        кроме всего прочего тем, насколько большой вклад \textit{индивидуальная
            кибернетическая система} вносит в повышение качества тех коллективов, в состав
        которых она входит. Указанное свойство \textit{индивидуальных кибернетических
            систем} будем называть уровнем их \textit{социализации}. Прежде, чем
        детализировать это свойство, целесообразно рассмотреть то, чем определяется
        качество коллектива кибернетических систем, например, качество творческого
        сообщества компьютерных систем и людей.}
        
    \scnheader{качество сообщества компьютерных систем и людей}
    \scntext{пояснение}{Эффективность творческого коллектива (например в области
        научно-технической деятельности) определяется:\begin{scnitemize}

            \item согласованностью мотивации (целевой установки) всего коллектива и каждого
            его члена:\begin{scnitemizeii}

                \item не должно быть синдрома лебедя, рака и щуки;
                \item не должно быть противоречий между целью коллектива и творческой
                самореализацией каждого его члена;\end{scnitemizeii}

            \item эффективной организацией децентрализованного управления деятельностью
            членов сообщества;
            \item четкой, оперативной и доступной всем фиксацией документации текущего
            состояния содеянного и направлений его дальнейшего развития;
            \item уровнем трудоемкости оперативности фиксации индивидуальных результатов в
            рамках коллективно создаваемого общего результата;
            \item уровнем структурированности и, прежде всего, стратифицированности
            обобщенной документации  (базы знаний);
            \item эффективностью ассоциативного доступа к фрагментам документации;
            \item гибкостью коллективно создаваемой базы;
            \item автоматизацией анализа содеянного и управления проектом.
        \end{scnitemize}
    }
    
    \scnheader{качество многоагентной системы}
    \begin{scnrelfromlist}{свойство-предпосылка}

        \scnitem{средний уровень интеллекта членов многоагентной системы}
        \scnitem{средний уровень социализации членов многоагентной системы}
        \scnitem{минимальный уровень социализации членов многоагентной системы}
            \begin{scnindent}
                \scntext{примечание}{Члены многоагентной системы, имеющие низкий уровень
                    социализации, существенно снижают качество системы.}
            \end{scnindent}
        \scnitem{качество организации взаимодействия членов многоагентной системы}
            \begin{scnindent}
            \scntext{примечание}{Высший уровень качества организации взаимодействия агентов
                многоагентной системы обеспечивается:
                \begin{scnitemize}
                    \item введением дополнительного специального (корпоративного) агента,
                    выполняющего функцию хранителя интегратора общих (корпоративных) знаний
                    многоагентной системы
                    \item реализацией децентрализованного взаимодействия агентов, управляемого
                    текущим состоянием информации, хранимой в памяти корпоративного агента.
                \end{scnitemize}}
            \end{scnindent}

    \end{scnrelfromlist}
    \bigskip
    \begin{scnset}
        \scnheader{ostis-система}
        \begin{scnindent}
            \scnsubset{многоагентная система, управляемая общей базой знаний}
        \end{scnindent}
    \end{scnset}
    \scntext{примечание}{Агенты \textit{ostis-системы} (sc-системы) являются
        \uline{специализированными} \textit{кибернетическими системами},
        \uline{действия} каждой из которых (кроме \textit{сенсорных sc-агентов})
        инициализируются определенного вида ситуациями и/или событиями в памяти
        \textit{ostis-системы} и \uline{заключаются} (за исключением
        \textit{эффекторных sc-агентов}) в преобразовании текущего состояния
        информации, хранимой в этой памяти. Таким образом, sc-агенты не являются
        интеллектуальными системами.}
    \bigskip
\end{scnsubstruct}
    \scnsourcecomment{Завершили Сегмент \scnqqi{Комплекс свойств, определяющих качество многоагентной системы}}

		\scnsegmentheader{Комплекс свойств, определяющих уровень социализации
    кибернетической системы как фактора существенного повышения уровня ее
    обучаемости, а также фактора существенного повышения качества всех тех
    многоагентных систем, в состав которых входит данная кибернетическая система}

\begin{scnsubstruct}
    \scnidtf{Комплекс свойств \textit{кибернетической системы}, определяющих
        необходимые требования к тем \textit{кибернетическим системам}, которые могут
        входить в состав \textit{синергетических кибернетических систем}}
    \scnheader{социализация кибернетической системы}
    \scnidtf{способность кибернетической системы взаимодействовать с другими
        кибернетическими системами в целях создания коллектива кибернетических систем
        (\textit{многоагентных систем}), уровень качества и, в частности, уровень
        \textit{интеллекта} которого выше уровня качества каждой
        \textit{кибернетической системы}, входящей в состав этого коллектива)}
    \scnidtf{комплекс способностей кибернетической системы, которые определяют ее
        вклад в уровень коллективной (социальной) интеллектуальности, т.е. в уровень
        интеллектуальности того коллектива кибернетических систем, членом которого
        данная кибернетическая система является (в уровень интеллектуальности
        соответствующей многоагентной системы)}
    \scnidtf{уровень вклада \textit{кибернетической системы} в обеспечение
        \textit{интеллекта} тех многоагентных систем, в состав которых эта
        \textit{кибернетическая система} входит}
    \scnidtf{уровень социализации кибернетической системы}
    \scnidtf{социализация}
    \scntext{примечание}{Уровень \textit{интеллекта} коллектива кибернетических систем
        (\textit{многоагентной системы}) может быть значительно ниже уровня
        \textit{интеллекта} самого глупого  члена этого коллектива, но может быть и
        значительно выше уровня \textit{интеллекта} самого умного  члена указанного
        коллектива. Для того, чтобы количество \textit{интеллектуальных систем}
        переходило в существенно более интеллектуальное качество коллектива таких
        систем, все объединяемые в коллектив \textit{интеллектуальные системы} должны
        иметь высокий уровень \textit{социализации}, что накладывает
        \uline{дополнительные требования}, предъявляемые к \textit{информации, хранимой
            в памяти}, а также к \textit{решателям задач}
        %\bigspace
        \textit{интеллектуальных систем}, объединяемых в
        коллектив.}\scntext{примечание}{Коллектив \textit{кибернетических систем} может иметь
        значительно более высокий уровень качества, в том числе, уровень интеллекта,
        чем уровень качества \textit{кибернетических систем}, являющихся членами этого
        коллектива. Но так бывает не всегда. Для того, чтобы количество членов
        коллектива \textit{кибернетической системы} перешло в более высокое качество
        самого коллектива, члены коллектива должны обладать дополнительными
        способностями, которые будем называть свойствами \textit{социализации}.
        Основными такими свойствами являются способность устанавливать и поддерживать
        достаточный уровень \textit{семантической совместимости} (взаимопонимания) с
        другими кибернетическими системами и \textit{договороспособность} (способность
        согласовывать свои действия с другими).}\scntext{примечание}{Целенаправленный обмен
        информацией между \textit{кибернетическими системами} существенно ускоряет
        процесс их обучения (процесс накопления знаний и навыков). Следовательно,
        способность эффективно использовать указанный канал накопления знаний и навыков
        существенно повышает уровень \textit{обучаемости}
        %\bigspace
        \textit{кибернетических систем}. В этом смысле можно сказать, что
        познавательный процесс социален.}\scnidtf{уровень развития социально значимых
        качеств кибернетической системы}
    \scntext{примечание}{Повышение уровня \textit{социализации}
        %\bigspace
        \textit{кибернетической системы} является, с одной стороны, дополнительным
        повышением уровня \textit{интеллекта} самой этой \textit{кибернетической
            системы}, а также фактором повышения уровня \textit{интеллекта} тех
        коллективов, тех \textit{многоагентных систем}, в состав которых эта
        \textit{кибернетическая система} входит.}\scntext{примечание}{Переход к
        \textit{многоагентным системам} не только является важным фактором повышения
        качества \textit{кибернетических систем}, но также имеет и обратную сторону
        медали	-- появление целого ряда угроз, связанного с возможными
        целенаправленными вредоносными воздействиями на \textit{многоагентную систему}
        (со стороны некоторых ее \textit{агентов}), существенно снижающими уровень ее
        качества. Наличие таких \textit{вредоносных целей} у соответствующих
        \textit{агентов} свидетельствует о нижайшем уровне \textit{социализации} этих
        \textit{агентов}.}\scnidtf{умение согласовывать (синхронизировать) свою
        деятельность с деятельностью других кибернетических систем в процессе решения
        задач, требующих коллективных усилий}
    \scnidtf{умение участвовать в децентрализованном процессе распределения
        подзадач некоторой коллективно (распределенно) решаемой задачи между членами
        заданного коллектива кибернетических систем и умение участвовать в управлении
        коллективного решения указанной задачи}
    \begin{scnindent}
        \scntext{примечание}{Речь идет о децентрализованном асинхронном управлении деятельностью коллектива кибернетических систем}
    \end{scnindent}
    \scnidtf{способность и готовность кибернетической системы к координации своей деятельности в рамках
        коллектива кибернетических систем, в состав которого она входит в целях:
        \begin{scnitemize}

            \item эффективного решения тактических задач, решаемых указанным коллективом;
            \item решения главной стратегической задачи этого коллектива --- обеспечения как
            можно более высокой скорости роста уровня интеллекта указанного коллектива.
        \end{scnitemize}
    }
    \scntext{примечание}{Подчеркнем, что повышение уровня интеллекта коллектива
        кибернетической системы (многоагентной системы) имеет свои особенности:
        \begin{scnitemize}

            \item во-первых, это забота о семантической совместимости кибернетических
            систем входящих в состав коллектива;
            \item во-вторых, это переход от виртуальной распределенной базы знаний
            коллектива к реально поддерживаемым базам знаний и к порталам корпоративных
            знаний, реализованных в виде индивидуальных кибернетических систем, через
            которые осуществляются все процессы координации и согласования деятельности
            соответствующих членов коллектива
        \end{scnitemize}
    }
    \begin{scnrelfromlist}{свойство-предпосылка}

        \scnitem{договороспособность кибернетической системы}
        \begin{scnindent}
            \scnidtftext{часто используемый sc-идентификатор}{договороспособность}
        \end{scnindent}
        \scnitem{социальная ответственность кибернетической системы}
            \begin{scnindent}
                \scnidtftext{часто используемый sc-идентификатор}{социальная ответственность}
            \end{scnindent}
        \scnitem{социальная активность кибернетической системы}
            \begin{scnindent}
                \scnidtftext{часто используемый sc-идентификатор}{социальная активность}
            \end{scnindent}

    \end{scnrelfromlist}
    \bigskip\scnheader{договороспособность кибернетической системы}

    \begin{scnrelfromlist}{свойство-предпосылка}

        \scnitem{способность кибернетической системы к пониманию принимаемых сообщений}
        \scnitem{способность кибернетической системы к формированию передаваемых
            сообщений, понятных адресатам}
        \scnitem{семантическая совместимость кибернетической системы с партнёрами }
        \scnitem{способность кибернетической системы к обеспечению семантической
            совместимости с партнёрами }
        \scnitem{коммуникабельность кибернетической системы }
        \scnitem{способность кибернетической системы к обсуждению и согласованию целей
            и планов коллективной деятельности }
        \scnitem{способность кибернетической системы брать на себя выполнение
            актуальных задач в рамках согласованных планов коллективной деятельности}

    \end{scnrelfromlist}

    \scnheader{способность кибернетической системы к пониманию принимаемых
        сообщений}
    \scnidtf{способность кибернетической системы к пониманию информации,
        поступающей извне от других кибернетических систем}
    \scnidtf{способность кибернетической системы к отображению принимаемых
        сообщений в семантически эквивалентные фрагменты собственной базы знаний}

    \begin{scnrelfromset}{комплекс частных свойств}

        \scnitem{способность кибернетической системы к пониманию принимаемых вербальных
            сообщений }
        \scnitem{способность кибернетической системы к пониманию принимаемых
            невербальных сообщений}   

    \end{scnrelfromset}
    \scnrelfrom{свойство-предпосылка}{способность кибернетической системы к
        обеспечению семантической совместимости с партнёрами}
    
    \scnheader{сообщение}
    \scnidtf{информация, передаваемая (пересылаемая) от одной кибернетической
        системы к другой или к другим кибернетическим системам}
    \scntext{примечание}{Каждому \textit{сообщению} ставится в соответствие одна
        \textit{кибернетическая система}, являющаяся \textbf{\textit{источником
                сообщения*}} и одна или несколько \textit{кибернетических систем}, являющихся
        \textbf{\textit{адресатами сообщения*}}. В соответствии с этим для каждой
        \textit{кибернетической системы} те сообщения, \textit{источником*} которых она
        является, будем называть \textbf{\textit{передаваемыми сообщениями*}}, а те
        сообщения, \textit{адресатами*} которых она является, будем называть
        \textbf{\textit{принимаемыми сообщениями*}}.}
    \begin{scnsubdividing}

        \scnitem{вербальное сообщение}
        \begin{scnindent}
            \scnidtf{передаваемая словесная информация}
        \end{scnindent}
        \scnitem{невербальное сообщение}
        \begin{scnindent}    
            \scntext{примечание}{Примерами невербальных сообщений являются пересылаемые фото-документы, видео-материалы}
        \end{scnindent}

    \end{scnsubdividing}

    \begin{scnrelfromset}{обобщённая декомпозиция}

        \scnitem{спецификация сообщения}
            \begin{scnindent}
                \begin{scnrelfromset}{обобщённая декомпозиция}

                    \scnitem{указание источника специфицируемого сообщения}
                    \scnitem{указание множества адресатов специфицируемого сообщения}
                    \scnitem{отметка момента времени отправления специфицируемого сообщения}
                    \scnitem{указание прагматического типа специфицируемого сообщения}
                    \scnitem{указания запроса, ответом на который является специфицируемое сообщение}
                        \begin{scnindent}
                            \scntext{примечание}{Если специфицируемое сообщение является ответом на некоторый запрос}
                        \end{scnindent}
                    \scnitem{указание раздела баз знаний адресатов, которому соответствует специфицируемое сообщение}
                    \scnitem{указание способа представления тела сообщения}
                        \begin{scnindent}   
                            \scntext{примечание}{Для вербальных сообщений это указание используемого  внешнего языка}
                        \end{scnindent}
                \end{scnrelfromset}
            \end{scnindent}
        \scnitem{тело сообщения}
            \begin{scnindent}    
                \scnidtf{собственно само сообщение}
            \end{scnindent}

    \end{scnrelfromset}
    \scnrelfrom{разбиение}{прагматический тип сообщения}

    \begin{scneqtoset}

        \scnitem{повествовательное сообщение}
            \begin{scnindent}
                \scnsuperset{ответ на запрос}
            \end{scnindent}
        \scnitem{вопросительное сообщение}
        \scnitem{команда редактирования баз знаний адресатов}
        \scnitem{команда, инициирующая действие адресатов в их внешней среде}

    \end{scneqtoset}

    \scnheader{следует отличать*}
    \begin{scnhaselementset}
        \scnitem{вербальная информация}
        \scnitem{файл, содержащий вербальную информацию}
            \begin{scnindent}
                \scnidtf{вербальная информация, представленная в виде файла}
            \end{scnindent}
        \scnitem{вербальное сообщение}
    \end{scnhaselementset}

    \scnheader{вербальная информация}
    \scnidtf{знаковая конструкция, которая имеет в общем случае произвольную
        денотационную семантику и которая может либо поступать на вход кибернетической
        системы через соответствующие ее сенсоры (рецепторы), либо через
        соответствующие эффекторы передаваться (пересылаться) в качестве сообщения
        другим кибернетическим системам}

    \scnheader{следует отличать*}
    \begin{scnhaselementset}
        \scnitem{вербальная информация}
        \scnitem{сенсорная информация}
    \end{scnhaselementset}
    \begin{scnindent}
        \scntext{примечание}{И \textit{вербальная информация} и \textit{сенсорная информация}
            являются \textit{знаковыми конструкциями}, но, во-первых, \textit{вербальная
            информация} может быть как внешней знаковой конструкцией, так и внутренней
            знаковой конструкцией, хранимой в памяти кибернетической системы, а
            \textit{сенсорная информация} всегда является внутренней \textit{знаковой
            конструкцией} кибернетической системы и, во-вторых, \textit{сенсорная
            информация} описывает только пограничную  для \textit{кибернетической системы}
            физическую \textit{окружающую среду}, тогда, как \textit{вербальная информация}
            может описывать все, что угодно.}
    \end{scnindent}

    \scnheader{следует отличать*}
    \begin{scnhaselementset}
        \scnitem{невербальная информация}
        \scnitem{файл, содержащий невербальную информацию}
            \begin{scnindent}    
                \scnidtf{файл, содержимым которого является электронный образ некоторой невербальной информации}
            \end{scnindent}
        \scnitem{невербальное сообщение}
            \begin{scnindent}
            \scnidtf{невербальная информация, представленная в виде файла и передаваемая
                (пересылаемая) от одной кибернетической системы к другой}
            \end{scnindent}
        \scnitem{сенсорная информация}
            \begin{scnindent}
                \scnidtf{информация, формируемая сенсорами кибернетической системы}
            \end{scnindent}
    \end{scnhaselementset}\

    \scnheader{невербальная информация}
    \scnsuperset{музыкальное произведение}
    \scnsuperset{танец}
    \scnsuperset{произведение изобразительного искусства}
    \begin{scnindent}
        \scnsuperset{живопись}
        \scnsuperset{скульптура}
        \scnsuperset{графика}
    \end{scnindent}
    \scnsuperset{статическое изображение}
    \scnsuperset{динамическое изображение}

    \scnheader{способность кибернетической системы к пониманию принимаемых
        вербальных сообщений}
    \scnidtf{способность кибернетической системы к пониманию вербальной информации,
        поступающей извне из разных источников}
    \scntext{примечание}{Понимание информации, поступающей извне, включает в себя:
        \begin{scnitemize}

            \item перевод этой информации на внутренний язык кибернетической системы;
            \item локальную верификацию вводимой информации;
            \item погружение (конвергенцию, размещение) текста, являющегося результатом
            указанного перевода в состав хранимой информации (в частности, в состав базы
            знаний)
        \end{scnitemize}
    }\scntext{примечание}{Погружение вводимой информации в состав базы знаний
        кибернетической системы сводится к выявлению и устранению противоречий,
        возникающих между погружаемым текстом и текущего состояния базы знаний. Первым
        уровнем таких противоречий являются появляющиеся при интеграции погружаемого
        текста с текущим состоянием базы знаний \textit{омонимичные знаки} и пары
        \textit{синонимичных знаков}. Омонимичные знаки появляются в результате
        ошибочного отождествления знака, входящего в состав погружаемого текста, со
        знаком, входящим в состав погружаемого текста, со знаком, входящим в состав
        текущего состояния базы знаний. Появление пар синонимичных знаков, один из
        которых входит в погружаемый текст, а второй --- в текущее состояние базы
        знаний, при погружении вводимого текста является штатным  противоречием,
        устранение которого осуществляется путем отождествления (склеивания )
        синонимичных знаков.}\scntext{примечание}{Сложность проблемы понимания вводимой
        вербальной информации заключается не только в сложности непротиворечивого
        погружения вводимой информации в текущее состояние базы знаний, но и в
        сложности трансляции этой информации с внешнего языка на внутренний язык
        кибернетической системы, т. е. в сложности генерации текста внутреннего языка,
        семантически эквивалентного вводимому тексту внешнего языка. Очевидно, что для
        естественных языков указанная трансляция является сложной задачей, так как в
        настоящее время проблема формализации синтаксиса и семантики естественных
        языков не решена.}
        
    \scnheader{семантическая совместимость кибернетической системы с партнерами}
    \scnidtf{уровень взаимопонимания кибернетической системой со своими партнерами}
    \scnidtf{степень конвергенции (близости) базы знаний кибернетической системы с базами знаний своих партнеров}
    
    \scnheader{семантическая совместимость кибернетической системы с партнерами}
    \scnrelto{частное свойство}{\textit{совместимость кибернетических систем}}
    \scntext{пояснение}{\textit{семантическая совместимость кибернетических
            систем} определяется
        \begin{scnitemize}

            \item количеством знаков, которые хранятся в памяти одной заданной
            кибернетической системы и денотационная семантика которых совпадает с
            денотационной семантикой знаков, хранимых в памяти другой заданной
            кибернетической системы (другими словами, это количество сущностей, которые
            описывают как в памяти первой кибернетической системы, так и в памяти второй
            кибернетической системы),
            \item тем, согласованы ли между двумя заданными кибернетическими системами факт
            совпадения денотационной семантики указанных выше знаков сущностей, описываемых
            в памяти как первой, так и второй кибернетической системы (такое согласование
            осуществляется путем согласования уникальных внешних идентификаторов (имен),
            которые приписываются указанным знакам сущностей и которые используются
            указанными кибернетическими системами при обмене сообщениями между ними).
        \end{scnitemize}
    }\scntext{примечание}{Прежде всего семантическая совместимость двух заданных
        кибернетических систем определяется согласованностью систем понятий,
        используемых обеими взаимодействующими кибернетическими системами, (т.е.
        совпадением семантической трактовки всех этих понятий) и включением в число
        таких общих понятий всех или почти всех неопределяемых понятий, а также тех
        определяемых понятий, которые обеими кибернетическими системами часто
        используются при определении остальных определяемых
        понятий.}\scntext{примечание}{Высокий уровень семантической совместимости даже для
        кибернетических систем с высоким уровнем интеллекта (например, для людей)
        встречается значительно реже, чем хотелось бы. Очевидно, что проблема
        обеспечения перманентной поддержки семантической совместимости
        взаимодействующих кибернетических систем является необходимым условием
        обеспечения высокого уровня взаимопонимания кибернетических систем и, как
        следствие, эффективного их взаимодействия.}
        
    \scnheader{способность кибернетической системы к обеспечению семантической совместимости с партнерами}
    \scnidtf{способность кибернетической системы к обеспечению взаимопонимания со своими партнерами.}
    \begin{scnrelfromset}{комплекс частных свойств}
        \scnitem{способность кибернетической системы к обеспечению семантической
            совместимости собственной базы знаний с базами знаний своих партнеров}
        \scnitem{способность кибернетической системы к обеспечению коммуникационной совместимости со своими партнерами}
            \begin{scnindent}
                \scntext{примечание}{Речь идет о согласовании внешних языков, используемых кибернетическими системами при их общении.}
            \end{scnindent}
    \end{scnrelfromset}
    \begin{scnrelfromset}{комплекс частных свойств}
        \scnitem{уровень предварительной семантической совместимости кибернетической системы с партнерами}
            \begin{scnindent}
                \scntext{примечание}{Речь идет об обеспечении начальной (стартовой) семантической совместимости.}
            \end{scnindent}
        \scnitem{способность кибернетической системы к перманентной поддержке
            семантической совместимости с партнерами}
        \begin{scnindent}
            \scntext{примечание}{Речь идет о перманентном процессе поддержки
                необходимого уровня семантической совместимости(взаимопонимания) в условиях
                постоянной эволюции всех взаимодействующих кибернетических систем.}
        \end{scnindent}
    \end{scnrelfromset}

    \scnheader{уровень предварительной семантической совместимости кибернетической
        системы с партнерами}
    \scnidtf{унификация представления информации, хранимой в памяти всевозможных кибернетических систем}
    \scnidtf{максимально возможная конвергенция, стандартизация, согласованность
        представления информации, хранимой в памяти всевозможных кибернетических систем}
    \scntext{примечание}{речь идет об использовании всеми кибернетическими системами
        общего универсального языка внутреннего представления знаний и о согласовании используемых ими понятий}
        
    \scnheader{способность кибернетической системы к перманентной поддержке семантической совместимости с партнерами}
    \scnidtf{способность кибернетической системы к согласованию денотационной
        семантики знаков (и, в первую очередь, знаков понятий), используемых в
        собственной базе знаний с денотационной семантике тех знаков, которые входят в
        состав информации поступающей от других кибернетических систем-партнеров}
    \scnidtf{способность кибернетической системы к повышению уровня семантической
        совместимости и взаимопонимания с другими системами (в том числе, с
        компьютерными системами, с людьми) в условиях перманентного процесса
        собственной эволюции (следствием которой является появление новых знаковых
        понятий и других описываемых сущностей, а также уточнение денотационной
        семантики используемых знаков), перманентной эволюции партнерских
        кибернетических систем и перманентной эволюции коллективно согласованной
        картины мира}
    \scntext{примечание}{Рассматриваемое свойство (способность) кибернетической системы
        заключается в \uline{самостоятельной} реализацией перманентного (постоянного)
        процесса обеспечения поддержки своей семантической совместимости \uline{со
            всеми}(!) кибернетическими системами, с которыми данная кибернетическая система
        взаимодействует в текущий момент времени. Подчеркнем при этом, что условия
        поддержки семантической совместимости постоянно меняются --- меняется состав
        партнеров , меняются (эволюционируют) сами партнеры , эволюционирует и сама
        данная кибернетическая система}
        
    \scnheader{следует отличать*}
    \begin{scnhaselementset}
        \scnitem{cпособность кибернетической системы к обеспечению семантической
            совместимости с партнерами}
            \begin{scnindent}
                \scniselement{свойство}
                \scnrelfrom{область определения}{кибернетическая система}
            \end{scnindent}
        \scnitem{cемантическая совместимость кибернетической системы с партнерами}
            \begin{scnindent}    
                \scniselement{свойство}
                \scnrelfrom{область определения}{множество всевозможных неориентированных пар кибернетических систем*}
                    \begin{scnindent}
                        \scnidtf{множество всевозможных сочетаний кибернетических систем по две*}
                        \scnidtf{множество всевозможных двухмощных множеств кибернетических систем*}
                    \end{scnindent}
                \scnidtf{степень (уровень) семантической совместимости различных пар кибернетических систем}
            \end{scnindent}
    \end{scnhaselementset}

    \scnheader{коммуникабельность кибернетической системы}
    \scnidtftext{часто используемый sc-идентификатор}{коммуникабельность}
    \scnidtf{способность кибернетической системы к установлению взаимовыгодных
        контактов с другими кибернетическими системами (в том числе, с коллективами
        интеллектуальных систем) путем честного выявления взаимовыгодных общих целей
        (интересов).}
    \scnidtf{способность кибернетической системы к формированию новых партнерских
        связей с другими кибернетическими системами}
    
    \scnheader{способность кибернетической системы к обсуждению и согласованию
        целей и планов коллективной деятельности}
    \scnidtf{способность активно участвовать в коллективном (в согласовании
        каких-либо предложений) --- т.е. в подтверждении (признании) этих предложений,
        либо в их отклонении с указанием причин или предлагаемых доработок}
    
    \scnheader{способность кибернетической системы брать на себя выполнение
        актуальных задач в рамках согласованных планов коллективной деятельности}
    \scntext{примечание}{Данная способность кибернетической системы предполагает:
        \begin{scnitemize}
            \item учет приоритета актуальных задач;
            \item учет собственных возможностей;
            \item согласование распределения актуальных задач по исполнителям;
            \item публикацию момента начала и предполагаемого момента завершения выполнения
            указанной актуальной задачи
        \end{scnitemize}
    }
    
    \scnheader{социальная ответственность кибернетической системы}
    \begin{scnrelfromlist}{свойство-предпосылка}
        \scnitem{способность кибернетической системы выполнять качественно и в срок
            взятые на себя обязательства в рамках соответствующих коллективов}
        \scnitem{ способность кибернетической системы адекватно оценивать свои
            возможности при распределении коллективной деятельности}
        \scnitem{ альтруизм/эгоизм кибернетической системы}
        \scnitem{ отсутствие/наличие действий, которые по безграмотности
            кибернетической системы снижают качество коллективов, в состав которых она
            входит}
        \scnitem{ отсутствие/наличие осознанных , мотивированных действий, снижающих
            качество коллективов, в состав которых кибернетическая система входит}
    \end{scnrelfromlist}

    \scnheader{альтруизм/эгоизм кибернетической системы}
    \scntext{примечание}{уровень мотивации к повышеннию качества коллективов, в состав
        которых кибернетическая система входит}\scntext{эпиграф}{Надо любить науку, а
        не себя в науке.}
    \scntext{эпиграф}{Ты играешь и всем своим видом показываешь: \scnqqi{Смотрите, как я
        красиво играю}, а надо играть и показывать красоту самой музыки.}
    
    \scnheader{социальная активность кибернетической системы}
    \scnidtftext{часто используемый sc-идентификатор*}{социальная активность}
    \scnidtf{пассионарность}
    \begin{scnrelfromlist}{свойство-предпосылка}
        \scnitem{способность кибернетической системы к генерации предлагаемых целей и
            планов коллективной деятельности}
        \scnitem{активность кибернетической системы в экспертизе результатов других
            участников коллективной деятельности}
        \scnitem{способность кибернетической системы к анализу качества всех
            коллективов, в состав которых она входит, а также всех членов этих коллективов}
        \scnitem{способность кибернетической системы к участию в формировании новых
            коллективов}
        \scnitem{количество и качество тех коллективов, в состав которых
            кибернетическая система входит или входила}
    \end{scnrelfromlist}

    \scnheader{способность кибернетической системы к участию в формировании коллективов}
    \scnidtf{уровень способности в создании таких коллективов кибернетических
        систем, в состав которых входит данная кибернетическая система и которые
        направлены на коллективное решение соответствующего актуального класса сложных
        комплексных задач, с каждой из которых не может справиться любая из имеющихся
        кибернетических систем.}
    \scntext{примечание}{Формирование специализированного коллектива кибернетических
        систем сводится к тому, что в памяти каждой кибернетической системы, входящей в
        коллектив, генерируется спецификация этого коллектива, включающая в себя:
        \begin{scnitemize}
            \item перечень весь членов коллектива;
            \item способности (возможности) каждого из них;
            \item обязанности в рамках коллектива;
            \item спецификацию всего множества задач (вида деятельности), для решения
            (выполнения) которых сформирован данный коллектив кибернетических систем
        \end{scnitemize}
    }\scntext{примечание}{Каждая кибернетическая система может входить в состав большого
        количества коллективов, выполняя при этом в разных коллективах в общем случае
        разные должностные обязанности , разные
        бизнес-процессы...}\scntext{примечание}{Рассмотренный принцип формирования
        специализированного коллектива, состоящего из компьютерных систем и людей,
        фактически означает автоматизацию системной интеграции компьютерных систем и
        децентрализованный (горизонтальный) характер такой интеграции, это очевидно
        предполагает наличие достаточно высокого уровня интеллекта у интегрируемых
        компьютерных систем и людей.}
    \scnheader{количество и качество тех коллективов, в состав которых кибернетическая система входит или входила}
    \scntext{пояснение}{Данная характеристика кибернетической системы уточняет
        спектр ее социальной активности}\scntext{примечание}{Чем умнее (интеллектуальнее)
        многоагентные системы, членом которых является данная кибернетическая система,
        тем выше ее социальный статус  и перспективы быть умнее --- есть у кого
        учиться}\bigskip
\end{scnsubstruct}
\scnsourcecomment{Завершили Сегмент \scnqqi{Комплекс свойств, определяющих уровень социализации кибернетической системы как фактора существенного повышения уровня ее обучаемости, а также фактора существенного повышения качества всех тех многоагентных систем, в состав которых входит данная кибернетическая система}}

		\scnsegmentheader{Направления эволюции компьютерных систем}

\begin{scnsubstruct}
    \scntext{эпиграф}{From data science to knowledge science.}

    \scnheader{эволюция компьютерных систем}
    \begin{scnsubdividing}
        \scnitem{первое направление эволюции компьютерных систем}
        \scnitem{второе направление эволюции компьютерных систем}
    \end{scnsubdividing}

    \scnheader{первое направление эволюции компьютерных систем}
    \scntext{примечание}{Первое направление эволючии включает в себя следующее:
    \begin{scnitemize}
        \item{расширение множества и многообразия задач, решаемых компьютерной системой}
        \item{повышение сложности этих задач вплоть до трудно формализуемых (трудно решаемых) задач, интеллектуальных задач, решаемых в условиях неполноты, неточности, нечеткости и так далее}
        \item{повышение качества решения задач либо путем более эффективного использования известных моделей решения задач (например, путем разработки более качественных алгоритмов), либо путем использования принципиально новых моделей решения задач}
        \item{расширение многообразия используемых видов информации (знаний)}
        \item{расширение многообразия используемых моделей решения задач}
    \end{scnitemize}}

    \scntext{примечание}{Очевидно, что расширение множества решаемых задач в условиях пусть и большой, но всегда конечной памяти компьютерной системы делает все более и более актуальным переход от частных методов и моделей решения задач к их обобщениям (или, как отмечал Д. А. Поспелов, от связки \scnqq{ключей} к набору \scnqq{отмычек})}

    \scntext{примечание}{Очевидно также, что многообразие видов задач, решаемых компьютерными системами, многообразие используемых моделей решения задач приводит:
    \begin{scnitemize}
        \item{к интегрированным информационным ресурсам}
        \item{к интегрированным решателям задач}
        \item{к интегрированным компьютерным системам}
        \item{к коллективам компьютерных систем}
    \end{scnitemize}}

    \scntext{примечание}{Проблема здесь заключается не в самой интеграции, а в ее качестве. Интеграция может быть эклектичной, если не обеспечить совместимость интегрируемых компонентов, а в случае такой совместимости интеграция может привести к новому качеству, к дополнительному расширению множества решаемых задач. Это будет означать переход от эклектичности к гибридности, синергетичности.}

    \scnheader{второе направление эволюции компьютерных систем}
    \scnidtf{повышение уровня обучаемости компьютерных систем и, как следствие, темпов их эволюции}

    \scnheader{обучаемость компьютерной системы}
    \begin{scnrelfromset}{определяется}
        \scnitem{трудоемкость и темпы расширения и совершенствования знаний и навыков компьютерной системы}
        \scnitem{уровень ограничений, накладываемых на вид приобретаемых и используемых знаний и навыков}
        \begin{scnindent}
            \scnidtf{ограничения на множество всех тех задач, которые принципиально могут быть решены данной компьютерной системой}
        \end{scnindent}
    \end{scnrelfromset}

    \scnheader{трудоемкость и темпы расширения и совершенствования знаний и навыков компьютерной системы}
    \begin{scnrelfromset}{определяется}
        \scnitem {гибкость}
        \begin{scnindent}
            \scnidtf{многообразие и трудоемкость возможных изменений, вносимых в систему в процессе пополнения системы новыми знаниями и навыками и совершенствования уже приобретенных знаний и навыков}
        \end{scnindent}
        \scnitem{стратифицированность}
        \begin{scnindent}
            \scnidtf{четкое разделение системы на достаточно независящие друг от друга уровни иерархии, то есть возможность локализации фрагментов компьютерной системы, не выходя за пределы которых, априори достаточно проводить анализ последствий тех или иных вносимых в систему изменений}
        \end{scnindent}
        \scnitem{рефлексивность} 
        \begin{scnindent}
            \scnidtf{способность анализировать собственное состояние и свою деятельность}
        \end{scnindent}
        \scnitem{гибридность} 
        \begin{scnindent}
            \scnidtf{способность приобретать и использовать широкое (а в идеале — неограниченное) многообразие знаний и навыков}
        \end{scnindent}
        \scnitem{уровень самообучаемости}
        \begin{scnindent}
            \scnidtf{уровень активности, самостоятельности, целеустремленности в процессе своего обучения, то есть уровень способности к обучению без учителя, уровень автоматизации приобретения новых знаний и навыков, а также совершенствование уже приобретенных знаний и навыков}
        \end{scnindent}
        \scnitem{совместимость} 
        \begin{scnindent}
            \scnidtf{трудоемкость интеграции}
        \end{scnindent}
        \scnitem{способность к постоянному мониторингу и поддержке своей совместимости} 
        \begin{scnindent}
            \scntext{примечание}{поддержка совместимести как с другими компьютерными системами, так и со своими пользователями в условиях интенсивной эволюции этих компьютерных систем и их пользователей}
        \end{scnindent}
    \end{scnrelfromset}

    \scnheader{совместимость компьютерных систем}
    \begin{scnrelfromset}{аспекты}
        \scnitem{глубокая интеграция компьютерных систем} 
        \begin{scnindent}
            \scnidtf{преобразование нескольких компьютерных систем в одну целостную компьютерную систему путем объединения информационных и функциональных ресурсов интегрируемых компьютерных систем}
        \end{scnindent}
        \scnitem{преобразование нескольких компьютерных систем в коллектив взаимодействующих компьютерных систем, способных к совместному корпоративному решению сложных задач}
    \end{scnrelfromset}
    \begin{scnrelfromset}{определяется}
        \scnitem{совместимость различного вида информации (знаний), хранимой в памяти компьютерной системы}
        \scnitem{совместимость различных моделей решения задач}
        \scnitem{совместимость встроенных (в том числе типовых) подсистем, входящих в состав компьютерных систем}
        \scnitem{совместимость внешней информации, поступающей на вход компьютерной системе, с информацией, хранимой в памяти компьютерной системы (трудоемкость понимания внешней информации — трансляция, погружение, выравнивание понятий)}
        \scnitem{коммуникационная (в том числе семантическая) совместимость с пользователями и с другими компьютерными системами}
    \end{scnrelfromset}

    \scnheader{обучение компьютерных систем}
    \scntext{примечание}{Важнейшая форма обучения компьютерной системы это приобретение новых знаний и навыков в \scnqq{готовом} виде, то есть в виде некоторых знаковых конструкций, вводимых в память компьютерной системы, поскольку приобретение знаний и навыков из внешних достоверных источников требует существенно меньшего времени по сравнению с их приобретением собственными силами, на основе собственного опыта и собственных ошибок. Но для того, чтобы указанная форма обучения была эффективной, необходимо максимально возможным образом упростить и формализовать механизм (процедуру) погружения новых знаний в память компьютерной системы. Для решения этой задачи ключевое значение имеет создание удобного для этой цели способа кодирования различного вида информации в памяти компьютерной системы.}
    \scntext{примечание}{Поскольку основным каналом обучения компьютерных систем является приобретение ими знаний и навыков от других субъектов — от других компьютерных систем и от пользователей (от разработчиков-учителей и от конечных пользователей), важнейшим фактором обучаемости компьютерной системы является превращение компьютерной системы в коммуникативную систему, способную эффективно общаться с внешними субъектами. Следовательно, уровнь обучаемости компьютерных систем определяется также уровнем ее совместимости с самими этими внешними субъектами, с приобретаемыми ею знаниями и навыками, то есть степенью того, как компьютерная система вместе с теми субъектами, с которыми она обменивается информацией, решает проблему \scnqq{вавилонского столпотворения}.}

    \scnheader{эволюция компьютерных систем}
    \scntext{примечание}{Таким образом, этапы эволюции традиционных компьютерных систем, в основе которых лежит их интерпретация на машинах фон Неймана, направлены на повышение качества этих систем и, в частности, на повышение уровня их интеллекта.}
    \scntext{примечание}{В качестве примера рассмотрим эволюцию языков программирования компьютерных систем:
    \begin{scnitemize}
        \item{Исходная особенность языков программирования заключается в том, что язык представления обрабатываемых программами данных (его синтаксис и денотационная семантика) не задается и фактически для любой программы или для семейства программ разрабатывается свой такой язык. (Языки программирования \scnqq{хромают} на одну ногу.)}
        \item{Данные преобразуются в базы данных, которые становятся общими для программ заданного языка программирования и изменение которых не может быть обусловлено и предусмотрено каждой из этих программ. Такие языки становятся языками программирования, ориентированными на обработку баз данных, а базам данных ставится в соответствие общий язык представления баз данных (с соответствующим синтаксисом и денотационной семантикой)}
        \item{Разные языки программирования (с разной денотационной и операционной семантикой) ориентируются на обработку баз данных, которым соответствует один и тот же язык представления баз данных (т.е. языки становятся совместимыми по обрабатываемым данными).}
        \item{Языки представления баз данных становятся универсальными и \scnqq{превращаются} в универсальные языки представления баз знаний (заметим, что продукционные и фреймовые языки представления знаний не являются универсальными)}
        \item{Разные языки программирования, ориентированные на обработку баз знаний, становятся подъязыками универсального языка представления баз знаний, т.е. становятся совместимыми не только по обрабатываемым базам знаний, но также и по своему синтаксису.}
        \item{Расширяется многообразие языков программирования, реализующих различные модели решателей задач:
            \begin{scnitemizeii}
                \item{алгоритмические языки программирования низкого и высокого уровня}
                \item{последовательные и параллельные процедурные языки программирования (синхронные и асинхронные)}
                \item{функциональные языки программирования}
                \item{логические языки программирования}
                \item{продукционные языки программирования}
                \item{объектно-ориентированные языки программирования}
                \item{генетические алгоритмы}
            \end{scnitemizeii}
        }
        \item{Создаются языки семантической спецификации программ, языки формулировки задач и стратегии поиска пути решения задач на основе заданного пакета программ различных языков программирования}
    \end{scnitemize}
    }
    \scntext{примечание}{Эволюция языков программирования подробнее рассматривается в работах Ершова А.П., Капитоновой Ю.В., Летичевского А.А., Непейводы Н.Н., Мак-Карти Дж. (язык LISP), Ковальского Р. (язык Рrolog) и других.}
\end{scnsubstruct}
		\scnsegmentheader{Итоговый сегмент Раздела Предметная область и онтология
    кибернетических систем}
\begin{scnsubstruct}
    \scnheader{качество кибернетической системы}
    \scntext{резюме}{}
    \bigskip
\end{scnsubstruct}
  %  \scnsourcecomment{Завершили Раздел \scnqqi{Предметная область и онтология кибернетических систем}}


		\bigskip
	\end{scnsubstruct}
\end{SCn}

\scsubsection{\S 1.2. Предметная область и онтология деятельности кибернетических систем}
\label{cyb_systems_activity}

\scsubsection{\S 1.3. Предметная область и онтология эволюции кибернетических систем}
\label{cyb_systems_evolution}

\scsection{Глава 2. Индивидуальные кибернетические системы и их эволюция}
\label{individual_syst_evolution}
\begin{SCn}
	\begin{scnrelfromset}{содержание}
		\scnitem{\S 2.1. Предметная область и онтология индивидуальных кибернетических систем}
		\scnitem{\S 2.2. Предметная область и онтология эволюции индивидуальных кибернетических систем}
		\scnitem{\S 2.3. Предметная область и онтология стимульно-реактивных индивидуальных кибернетических систем}
		\scnitem{\S 2.4. Предметная область и онтология индивидуальных кибернетических систем со знаковой памятью}
		\scnitem{\S 2.5. Предметная область и онтология задачно-ориентированных индивидуальных кибернетических систем}
		\scnitem{\S 2.6. Предметная область и онтология индивидуальных кибернетических систем со структурированной информационной моделью окружающей среды}
		\scnitem{\S 2.7. Предметная область и онтология гибридных индивидуальных кибернетических систем}
		\scnitem{\S 2.8. Предметная область и онтология обучаемых индивидуальных кибернетических систем}
		\scnitem{\S 2.9. Предметная область и онтология самостоятельных индивидуальных кибернетических систем}
	\end{scnrelfromset}
\end{SCn}

\scsubsection{\S 2.1. Предметная область и онтология индивидуальных кибернетических систем}
\label{individual_cyber_systems}

\scsubsection{\S 2.2. Предметная область и онтология эволюции индивидуальных кибернетических систем}
\label{individual_cyber_systems_evolution}

\scsubsection{\S 2.3. Предметная область и онтология стимульно-реактивных индивидуальных кибернетических систем}
\label{stimreact_individual_cyber_systems}

\scsubsection{\S 2.4. Предметная область и онтология индивидуальных кибернетических систем со знаковой памятью}
\label{sign_mem_individual_cyber_systems}

\scsubsection{\S 2.5. Предметная область и онтология задачно-ориентированных индивидуальных кибернетических систем}
\label{taskorient_individual_cyber_systems}

\scsubsection{\S 2.6. Предметная область и онтология индивидуальных кибернетических систем со структурированной информационной моделью окружающей среды}
\label{env_model_individual_cyber_systems}

\scsubsection{\S 2.7. Предметная область и онтология гибридных индивидуальных кибернетических систем}
\label{hybrid_individual_cyber_systems}

\scsubsection{\S 2.8. Предметная область и онтология обучаемых индивидуальных кибернетических систем}
\label{learnable_individual_cyber_systems}

\scsubsection{\S 2.9. Предметная область и онтология самостоятельных индивидуальных кибернетических систем}
\label{independent_individual_cyber_systems}

\scsection{Глава 3. Многоагентные кибернетические системы и их эволюция}
\label{multiagent_syst_tech}
\begin{SCn} 
	\begin{scnrelfromset}{содержание}
		\scnitem{\S 3.1. Предметная область и онтология многоагентных кибернетических систем}
		\scnitem{\S 3.2. Предметная область и онтология эволюции многоагентных кибернетических систем}
		\scnitem{\S 3.3. Предметная область и онтология интероперабельных индивидуальных и многоагентных кибернетических систем}
	\end{scnrelfromset} 
\end{SCn}

\scsubsection{\S 3.1. Предметная область и онтология многоагентных кибернетических систем}
\label{multiagent_cyber_systems}

\scsubsection{\S 3.2. Предметная область и онтология эволюции многоагентных кибернетических систем}
\label{multiagent_cyber_systems_evolution}

\scsubsection{\S 3.3. Предметная область и онтология интероперабельных индивидуальных и многоагентных кибернетических систем}
\label{interoperable_cyber_systems}

\scsection{Глава 4. Интеллектуальные кибернетические системы, интеллектуальные компьютерные системы и соответствующие им технологии}
\label{int_syst_tech}
\begin{SCn}
	\begin{scnrelfromset}{содержание}
		\scnitem{\S 4.1. Предметная область и онтология интеллектуальных индивидуальных и многоагентных кибернетических систем}
		\scnitem{\S 4.2. Предметная область и онтология интеллектуальных компьютерных систем нового поколения}
		\scnitem{\S 4.3. Предметная область и онтология эволюции технологий разработки интеллектуальных компьютерных систем}
		\scnitem{\S 4.4. Предметная область и онтология комплексной технологии разработки и сопровождения семантически совместимых интеллектуальных компьютерных систем нового поколения}
	\end{scnrelfromset}
\end{SCn}

\scsubsection{\S 4.1. Предметная область и онтология интеллектуальных индивидуальных и многоагентных кибернетических систем}
\label{intelligent_cyber_systems}

\scsubsection{\S 4.2. Предметная область и онтология интеллектуальных компьютерных систем нового поколения}
\label{intelligent_comp_systems_ng}
\begin{SCn}
	\begin{scnrelfromset}{содержание}
		\scnitem{Пункт 4.2.1. Предметная область и онтология смыслового представления информации}
		\scnitem{Пункт 4.2.2. Предметная область и онтология многоагентных моделей решателей задач, основанных на смысловом представлении информации}
		\scnitem{Пункт 4.2.3. Предметная область и онтология онтологических моделей интерфейсов интеллектуальных компьютерных систем, основанных на смысловом представлении информации}
	\end{scnrelfromset}
\end{SCn}
\begin{SCn}
	\scnsectionheader{Предметная область и онтология интеллектуальных компьютерных систем нового поколения}
	\begin{scnsubstruct}

		\begin{scnrelfromlist}{дочерний раздел}
			\scnitem{\nameref{sd_sem_inf_rep}}
			\scnitem{\nameref{sd_agent_solvers}}
			\scnitem{\nameref{sd_sem_ui}}
		\end{scnrelfromlist}

		\scntext{аннотация}{Данный раздел и дочерние ему разделы являются
			уточнением и обоснованием наших предложений, направленных на построение
			компьютерных систем следующего поколения, основанных на смысловом представлении
			обрабатываемой информации.}
		\scntext{основной тезис}{Для \uline{любой} \textit{компьютерной
				системы} можно построить эквивалентную ей логико-семантическую модель,
			основанную на смысловом представлении обрабатываемой информации}

		\scnheader{логико-семантическая модель компьютерной системы}
		\scntext{пояснение}{Главным фактором обеспечения совместимости
			различных видов знаний, различных моделей решения задач и различных
			компьютерных систем в целом является
			\begin{scnitemize}
				\item унификация (стандартизация) представления информации в памяти
				компьютерных систем;
				\item унификация принципов организации обработки информации в памяти
				компьютерных систем.
			\end{scnitemize}
			Унификация представления информации, используемой в компьютерных
			системах, предполагает:
			\begin{scnitemize}
				\item синтаксическую унификацию используемой информации  унификацию
				формы представления (кодирования) этой информации. При этом следует отличать:
				\begin{scnitemizeii}
					\item кодирование информации в памяти компьютерной системы (внутреннее
					представление информации);
					\item внешнее представление информации, обеспечивающее однозначность
					интерпретации (понимания, трактовки) этой информации разными пользователями и
					разными компьютерными системами;
				\end{scnitemizeii}
				\item семантическую унификацию используемой информации, в основе
				которой лежит согласование и точная спецификация всех (!) используемых понятий
				(концептов) с помощью иерархической системы формальных онтологий.
			\end{scnitemize}}

		\scnheader{стандарт}
		\scnhaselement{Стандарт OSTIS}
		\begin{scnindent}
			\scnidtf{Предлагаемый нами стандарт логико-семантических моделей
				компьютерных систем,  основанных на смысловом представлении информации, и
				технологии разработки таких моделей и соответствующих компьютерных систем}
		\end{scnindent}
		\scnidtf{знания о структуре и принципах функционирования искусственных
			систем соответствующего класса}
		\scnidtf{онтология искусственных систем некоторого класса}
		\scnidtf{теория искусственных систем некоторого класса}
		\scntext{пояснение}{Важно отметить, что грамотная унификация
			(стандартизация) должна не ограничивать творческую свободу разработчика, а
			гарантировать \uline{совместимость} его результатов с результатами других
			разработчиков. Подчеркнем также, что текущая версия любого \textit{стандарта}
			-- это не догма, а только опора для дальнейшего его совершенствования.Целью
			качественного \textit{стандарта} является не только обеспечения совместимости
			технических решений, но и минимизация дублирования (повторения) таких решений.
			Один из важных критериев качества \textit{стандарта} --- ничего
			лишнего.\textit{Стандарты}, как и другие важные для человечества
			\textit{знания}, должны быть формализованы и должны постоянно
			совершенствоваться с помощью специальных \textit{интеллектуальных компьютерных
				систем}, поддерживающих процесс эволюции стандартов путем согласования
			различных точек зрения.}
			
		\scnheader{семантическая совместимость компьютерных систем}
		\scntext{пояснение}{Уровень совместимости \textit{компьютерных
				систем} определяется трудоемкостью реализации процедур интеграции (объединения,
			соединения знаний этих систем), а также трудоемкостью и глубиной интеграции
			входящих в эти системы \textit{решателей задач} (интеграции навыков и
			интерпретаторов этих навыков). Подчеркнем при этом, что интеграция интеграции
			рознь --- от эклектики до гибридности и синергетичности дистанция огромного
			размера.
			
			Совместимые \textit{компьютерные системы} --- это компьютерные системы,
			для которых существует автоматически выполняемая процедура их интеграции,
			превращающая эти системы в единую \textit{гибридную систему}, в рамках которой
			каждая интегрируемая компьютерная система в процессе своего функционирования
			может свободно использовать любые необходимые информационные ресурсы (знания и
			навыки), входящие в состав другой интегрируемой компьютерной системы.
			
			Целостную \textit{компьютерную систему} можно рассматривать как решатель задач,
			интегрировавший несколько моделей решения задач и обладающий средствами
			взаимодействия с внешней для себя средой (с другими компьютерными системами, с
			пользователями).
			
			Таким образом, для того, чтобы повысить уровень совместимости
			\textit{компьютерных систем}, необходимо преобразовать их к виду
			\textit{многоагентных систем}, работающих над общей семантической памятью.
			Такие \textit{компьютерные системы} не всегда целесообразно непосредственно
			объединять (интегрировать) в более крупные \textit{компьютерные системы}.
			Иногда целесообразнее их объединять в \textit{коллективы взаимодействующих
			компьютерных систем}. Но при создании таких коллективов компьютерных систем
			унификация и совместимость таких систем также очень важны, т.к. существенно
			упрощают обеспечение высокого уровня их взаимопонимания. Так, например,
			противоречия между компьютерными системами, входящими в коллектив, можно
			обнаруживать путем анализа на непротиворечивость \textit{виртуальной
			объединенной базы знаний} этого коллектива. Более того, непротиворечивость
			указанной виртуальной базы знаний можно считать одним из критериев
			семантической совместимости систем, входящих в соответствующий
			коллектив.}
			
		\scnheader{компьютерная система, основанная на смысловом представлении информации}
		\scntext{пояснение}{Предлагаемое нами устранение проблем современных
			информационных технологий путем перехода к \textit{смысловому представлению
				информации} в памяти компьютерных систем фактически преобразует современные
			компьютерные системы (в том числе и современные интеллектуальные компьютерные
			системы) в \textit{компьютерные системы, основанные на смысловом представлении
				информации}, которые являются не альтернативной ветвью развития
			\textit{компьютерных систем}, а естественным этапом их эволюции, направленным
			на обеспечение высокого уровня их \textit{обучаемости} и, в первую очередь,
			\textit{совместимости}.
			
			Архитектура \textit{компьютерных систем, основанных на
			смысловом представлении информации} (см. \textit{Рис. Архитектура компьютерных
			систем, основанных на смысловом представлении информации}) практически
			совпадает с архитектурой \textit{интеллектуальных компьютерных систем},
			основанных на знаниях. Отличие здесь заключаются в том, что в
			\textit{компьютерных системах, основанных на смысловом представлении
			информации}:
			\begin{scnitemize}
				\item база знаний имеет смысловое представление;
				\item интерпретатор знаний и навыков представляет собой коллектив
				\textit{агентов}, осуществляющих обработку \textit{базы знаний}.
			\end{scnitemize}
			Как следствие этого, \textit{компьютерные системы, основанная на
			смысловом представлении информации}, обладают высоким уровнем
			\textit{обучаемости}, т.е. способностью быстро приобретать новые и
			совершенствовать уже приобретенные знания и навыки и при этом не иметь никаких
			ограничений на вид приобретаемых и совершенствуемых ею знаний и навыков, а
			также на их совместное использование.
			
			Более того, при согласовании соответствующих стандартов, а также при перманентном совершенствовании этих
			стандартов и при грамотной их поддержке в условиях интенсивной эволюции как
			самих стандартов, так и \textit{компьютерных систем, основанных на смысловом
			представлении информации} (речь идет о постоянной поддержке соответствия между
			текущим состоянием компьютерных систем и текущим состоянием эволюционируемых
			стандартов), \textit{компьютерные системы, основанные на смысловом
			представлении информации} и их компоненты обладают весьма высокой степенью
			\textit{совместимости}.
			
			Это, в свою очередь, практически исключает дублирование
			инженерных решений и дает возможность существенно ускорить разработку
			\textit{компьютерных систем, основанных на смысловом представлении информации}
			с помощью постоянно расширяемой библиотеки многократно используемых и
			совместимых между собой компонентов. 
			
			Основным лейтмотивом перехода от современных компьютерных систем (в том числе интеллектуальных) к
			\textit{компьютерным системам, основанным на смысловом представлении
				информации}, хранимой в ее памяти, является создание \textbf{\textit{общей
					семантической теории компьютерных систем}}, включающей в себя:
			\begin{scnitemize}
				\item cемантическую теорию \textit{знаний} и \textit{баз знаний};
				\item семантическую теорию \textit{задач} и \textit{моделей решения
					задач};
				\item cемантическую теорию \textit{взаимодействия информационных
					процессов};
				\item cемантическую теорию пользовательских и, в том числе,
				естественно-языковых интерфейсов;
				\item cемантическую теорию невербальных (сенсорно-эффекторных)
				интерфейсов;
				\item теорию универсальных интерпретаторов \textit{логико-семантических
					моделей компьютерных систем} и, в частности, теорию семантических компьютеров.
			\end{scnitemize}
			Эпицентром следующего этапа развития информационных технологий является
			решение проблемы обеспечения \textbf{\textit{семантической совместимости}}
			\textit{компьютерных систем} и их компонентов. Для решения этой проблемы
			необходим
			\begin{scnitemize}
				\item переход от традиционных компьютерных систем и от современных
				интеллектуальных компьютерных систем к \textit{компьютерным системам,
					основанным на смысловом представлении информации};
				\item разработка \textit{стандарта компьютерных систем, основанных на
					смысловом представлении информации}.
			\end{scnitemize}
			Очевидно, что \textit{компьютерные системы, основанных на смысловом
				представлении информации} являются компьютерными системами нового поколения,
			устраняющие многие недостатки современных компьютерных систем. Но для массовой
			разработки таких систем необходима соответствующая технология, которая должна
			включать в себя
			\begin{scnitemize}
				\item теорию \textit{компьютерных систем, основанных на смысловом
					представлении информации} и комплекс всех стандартов, обеспечивающих
				совместимость разрабатываемых систем;
				\item методы и средства проектирования \textit{компьютерных систем,
					основанных на смысловом представлении информации};
				\item методы и средства перманентного совершенствования самой
				технологии.
			\end{scnitemize}}

		\scnheader{Рис. Архитектура компьютерных систем, \textit{основанных на
				смысловом представлении информации}}
		\scneqfile{\includegraphics{Contents/part_intro/src/images/arch.pdf}}
		\bigskip

		\scnsegmentheader{Предметная область и онтология требований, предъявляемых к интеллектуальным компьютерным системам нового поколения}

\begin{scnsubstruct}

    \begin{scnrelfromlist}{ключевое понятие}
    	\scnitem{интеллектуальная компьютерная система нового поколения}
    	\scnitem{интероперабельная интеллектуальная компьютерная система}
    	\scnitem{гибридная интеллектуальная компьютерная система}
    \end{scnrelfromlist}
    
    \begin{scnrelfromlist}{ключевое отношение}
    	\scnitem{соединение интеллектуальных компьютерных систем*}
    	\begin{scnindent}
    		\scnidtf{преобразование множества интеллектуальных компьютерных систем в коллектив, членами (агентами) которого являются эти системы*}
    	\end{scnindent}
    	\scnitem{глубокая интеграция интеллектуальных компьютерных систем*}
    	\begin{scnindent}
    		\scnidtf{быть результатом преобразования множества индивидуальных интеллектуальных компьютерных систем в одну интегрированную индивидуальную интеллектуальную компьютерную систему*}
    	\end{scnindent}
    \end{scnrelfromlist}
    
    \begin{scnrelfromlist}{ключевой параметр}
    	\scnitem{интероперабельность интеллектуальных компьютерных систем\scnsupergroupsign}
    	\scnitem{семантическая совместимость пар интеллектуальных компьютерных систем\scnsupergroupsign}
    \end{scnrelfromlist}
    
    \begin{scnrelfromlist}{ключевое знание}
    	\scnitem{Требования, предъявляемые к интеллектуальным компьютерным системам нового поколения}
    	\scnitem{Принципы, лежащие в основе интеллектуальных компьютерных систем нового поколения}
    	\scnitem{Отличие данных от знаний}
    \end{scnrelfromlist}

    \scnheader{уровень интероперабельности интеллектуальных компьютерных систем}
    \scntext{примечание}{Создание различных комплексов взаимодействующих интеллектуальных компьютерных систем \uline{требует} повышения качества не только самих этих систем, но также и качества их взаимодействия. Интеллектуальные компьютерные системы нового поколения должны иметь высокий уровень интероперабельности.}
    \scnidtf{уровень коммуникационной (социальной) совместимости интеллектуальных компьютерных систем, позволяющей им самостоятельно формировать коллективы интеллектуальных компьютерных систем и их пользователей, а также самостоятельно согласовывать и координировать свою деятельность в рамках этих коллективов при решении сложных задач в частично предсказуемых условиях}
    \scnidtf{уровень способности к эффективному, целенаправленному взаимодействию с себе подобными и с пользователями в процессе коллективного (распределенного) и децентрализованного решения сложных задач}
    \begin{scnindent}
        \begin{scnrelfromset}{источник}
            \scnitem{\scncite{Yaghoobirafi2022}}
            \scnitem{\scncite{Ouksel1999}}
            \scnitem{\scncite{Lanzenberger2008}}
            \scnitem{\scncite{Neiva2016}}
            \scnitem{\scncite{Pohl2004}}
            \scnitem{\scncite{Waters2009}}
        \end{scnrelfromset}
    \end{scnindent}
    \scnidtf{уровень \scnqq{социализации} интеллектуальных компьютерных систем, полезности в рамках различных априори неизвестных сообществ (коллективов) \textit{интеллектуальных систем}}
    \scntext{примечание}{Повышение уровня \textit{интероперабельности} интеллектуальных компьютерных систем определяет переход к \textbf{\textit{интеллектуальным компьютерным системам нового поколения}}, без которых невозможна реализация таких проектов, как \textit{интеллектуальное-предприятие}, \textit{интеллектуальная-больница}, \textit{интеллектуальная-школа}, \textit{интеллектуальный-университет}, \textit{интеллектуальная-кафедра}, \textit{интеллектуальный-дом}, \textit{интеллектуальный-город}, \textit{интеллектуаль\-ное-общество}.}
        \begin{scnindent}
            \begin{scnrelfromset}{источник}
                \scnitem{\scncite{Lopes2022}}
                \scnitem{\scncite{Hamilton2006}}
            \end{scnrelfromset}
        \end{scnindent}
    
    \scnheader{интеллектуальная компьютерная система}
    \scnidtf{интеллектуальная искусственная кибернетическая система}
    \begin{scnrelfromset}{разбиение}
    	\scnitem{индивидуальная интеллектуальная компьютерная система}
    	\scnitem{интеллектуальный коллектив интеллектуальных компьютерных систем}
    	\begin{scnindent}
    		\scnidtf{интеллектуальная \textit{многоагентная система}, агенты которой являются \textit{интеллектуальными компьютерными системами}}
    		\scntext{примечание}{Не каждый \textit{коллектив интеллектуальных компьютерных систем} может оказаться интеллектуальным, поскольку уровень интеллекта такого коллектива определяется не только уровнем интеллекта его членов, но также и эффективностью (качеством) \uline{их взаимодействия}.}
    		\begin{scnrelfromset}{разбиение}
    			\scnitem{интеллектуальный коллектив \uline{индивидуальных} интеллектуальных компьютерных систем}
    			\scnitem{иерархический интеллектуальный коллектив интеллектуальных компьютерных систем}
    			\begin{scnindent}
    				\scnidtf{\textit{интеллектуальный коллектив интеллектуальных компьютерных систем}, по крайней мере одним из членов которого является \textit{интеллектуальный коллектив интеллектуальных компьютерных систем}}
    			\end{scnindent}
    		\end{scnrelfromset}
    	\end{scnindent}
    \end{scnrelfromset}
    
    \scnheader{интеллектуальные компьютерные системы нового поколения}
    \begin{scnrelfromlist}{предъявляемые требования}
    	\scnitem{высокий уровень \textit{интероперабельности}}
    	\scnitem{высокий уровень \textit{обучаемости}}
    	\scnitem{высокий уровень \textit{гибридности}}
    	\scnitem{высокий уровень способности решать \textit{интеллектуальные задачи}}
        \begin{scnindent}
            {\textit{задачи}, \textit{методы} решения которых и/или требуемая для их решения исходная информация априори неизвестны}
        \end{scnindent}
        \scnitem{высокий уровень \textit{синергетичности}}
    \end{scnrelfromlist}
    
    \scnheader{интероперабельность\scnsupergroupsign}
    \scnidtf{способность к эффективному (целенаправленному) взаимодействию с другими самостоятельными субъектами}
    \scnidtf{способность к партнерскому взаимодействию в решении \textit{комплексных задач}, требующих \textit{коллективной деятельности}}
    \scnidtf{способность работать в коллективе (в команде)}
    \scnidtf{уровень социализации}
    \scnidtf{social skills}
    
    \scnheader{высокий уровень интероперабельности}
    \begin{scnrelfromlist}{обеспечивается}
    	\scnfileitem{высоким уровнем \textit{взаимопонимания}}
    	\begin{scnindent}
    		\begin{scnrelfromlist}{обеспечивается}
    			\scnfileitem{высоким уровнем \textbf{\textit{семантической совместимости}} заданного субъекта с другими субъектами заданного коллектива}
    			\scnfileitem{высоким уровнем \textit{способности понимать} сообщения и поведение партнеров}
    			\scnfileitem{высоким уровнем \textit{способности быть понятной} для партнеров:}
                \begin{scnindent}
                    \begin{scnrelfromlist}{обеспечивается}
                        \scnfileitem{способностью понятно и обоснованно формулировать свои предложения и информацию, полезную для решения текущих задач}
    			        \scnfileitem{способностью действовать и комментировать свои действия так, чтобы они и их мотивы были понятны партнерам}
                \end{scnrelfromlist}
                \end{scnindent}
    			\scnfileitem{высоким уровнем \textit{способности к повышению уровня семантической совместимости} со своими партнерами}
    		\end{scnrelfromlist}
    	\end{scnindent}
    	\scnfileitem{высоким уровнем \textit{договороспособности}, то есть способности согласовывать с партнерами свои планы и намерения в целях своевременного обеспечения высокого качества коллективного результата}
    	\scnfileitem{высоким уровнем \textit{способности к децентрализованной координации} своих действий с действиями партнеров в непредсказуемых (нештатных) обстоятельствах}
    	\scnfileitem{высоким уровнем способности разделять ответственность с партнерами}
    	\scnfileitem{высоким уровнем \textit{способности к минимизации негативных последствий конфликтных ситуаций} с другими субъектами}
    	\begin{scnindent}
    		\begin{scnrelfromlist}{обеспечивается}
    			\scnfileitem{высоким уровнем \textit{способности к предотвращению возникновения конфликтных ситуаций}}
    			\scnfileitem{\textit{соблюдением этических норм} и правил, препятствующих возникновению разрушительных последствий конфликтных ситуаций}
    			\scnfileitem{высоким уровнем \textit{способности разделять ответственность} с партнерами за своевременное и качественное достижение общей цели}
    		\end{scnrelfromlist}
    	\end{scnindent}
    \end{scnrelfromlist}
    
    \scnheader{семантическая совместимость\scnsupergroupsign}
    \scnidtf{степень согласованности (совпадения) систем \textit{понятий} и других \textit{ключевых знаков}, используемых заданными взаимодействующими субъектами}
    \scntext{примечание}{Обеспечение \textit{семантической совместимости} требует формализации \textit{смыслового представления информации}.}
    
    \scnheader{способность разделять ответственность с партнёрами}
    \scnidtf{необходимое условие децентрализованного управления коллективной деятельностью}
    \begin{scnrelfromlist}{обеспечивается}
    	\scnfileitem{\textit{способностью к мониторингу} и анализу коллективно выполняемой деятельности}
    	\scnfileitem{\textit{способностью оперативно информировать партнеров} о неблагоприятных ситуациях, событиях, тенденциях, а также инициировать соответствующие коллективные действия}
    \end{scnrelfromlist}
    
    \scnheader{высокий уровень обучаемости интеллектуальной компьютерной системы нового поколения}
    \scnexplanation{Важнейшим направлением повышения уровня автоматизации человеческой деятельности является повышение уровня автоматизации не только проектирования интеллектуальной компьютерной системы, но и комплексной поддержки всех остальных этапов жизненного цикла \textit{интеллектуальной компьютерной системы}. В частности, это касается модернизации (совершенствования, реинжиниринга) интеллектуальной компьютерной системы непосредственно в ходе их эксплуатации. Для того, чтобы обеспечить высокий уровень автоматизации такой модернизации, необходимо существенно повысить \textbf{\textit{уровень самообучаемости}} \textit{интеллектуальной компьютерной системы} для того, что они сами (самостоятельно) могли себя модернизировать (самосовершенствовать) в ходе своего целевого функционирования.}
    
    \scnheader{высокий уровень обучаемости}
    \begin{scnrelfromlist}{обеспечивается}
    	\scnfileitem{высоким уровнем \textit{гибкости информации}, хранимой в памяти интеллектуальной системы}
    	\scnfileitem{высоким уровнем \textit{качества} \textit{стратификации информации}, хранимой в памяти интеллектуальной системы (стратифицированностью \textit{базы знаний})}
    	\scnfileitem{высоким уровнем \textit{рефлексивности} интеллектуальной системы}
    	\scnfileitem{высоким уровнем \textit{способности исправлять свои ошибки} (в том числе устранять противоречия в своей \textit{базе знаний})}
    	\scnfileitem{высоким уровнем \textit{познавательной активности}}
    	\scnfileitem{низким уровнем \textit{ограничений на вид приобретаемых знаний и навыков} (отсутствие таких ограничений означает потенциальную \textit{универсальность} интеллектуальной системы и предполагает высокий уровень ее гибридности)}
    \end{scnrelfromlist}
    
    \scnheader{обучаемость\scnsupergroupsign}
    \scnidtf{способность быстро и качественно приобретать новые \textit{знания} и \textit{навыки}, а также совершенствовать уже приобретенные \textit{знания} и \textit{навыки}}
    
    \scnheader{гибридность\scnsupergroupsign}
    \scnidtf{степень многообразия используемых \textit{видов знаний} и \textit{моделей решения задач} и уровень эффективности их совместного использования}
    \scnidtf{индивидуальная способность решать \textit{комплексные задачи}, требующие использования различных \textit{видов знаний}, а также различных комбинаций различных \textit{моделей решения задач}}
    \scntext{пояснение}{\textit{Гибридность} и \textit{интероперабельность} \textit{интеллектуальных компьютерных систем нового поколения} предполагает отказ от известной парадигмы \scnqq{черных ящиков}, поскольку:
    \begin{scnitemize}
        \item все многообразие моделей решения задач \textit{гибридной интеллектуальной компьютерной системы} должно интерпретироваться на одной общей \textit{универсальной платформе};
    	\item
    	доступность информации о том, как устроен каждый используемый метод, модель решения задач, каждый субъект существенно повышает качество их \textit{координации} при \textit{совместном решении комплексных задач};
    	\item
    	появляется возможность некоторые методы, модели решения задач и целые субъекты (например, \textit{интеллектуальные компьютерные системы}) использовать для совершенствования (повышения качества) других методов, моделей и субъектов.
    \end{scnitemize}}
    
    \scnheader{высокий уровень гибридности}
    \begin{scnrelfromlist}{обеспечивается}
    	\scnfileitem{высокой степенью многообразия используемых \textit{видов знаний} и \textit{моделей решения задач}}
    	\scnfileitem{высокой степенью \textit{конвергенции} и глубокой \textit{интеграции} (степенью взаимопроникновения) различных \textit{видов знаний} и \textit{моделей решения задач}}
    	\scnfileitem{способностью неограниченно расширять уровень своей \textit{гибридности}}
    \end{scnrelfromlist}
    
    \scnheader{характеристики \textit{интеллектуальных компьютерных систем нового поколения}}
    \begin{scnhassubset}
        \scnfileitem{\textbf{\textit{Степень}} \textbf{\textit{конвергенции}}, унификации и стандартизации \textit{интеллектуальных компьютерных систем} и их компонентов и соответствующая этому \textbf{\textit{степень интеграции}} (глубина интеграции) \textit{интеллектуальных компьютерных систем} и их компонентов.}
        \scnfileitem{\textbf{\textit{Семантическая совместимость}} между \textit{интеллектуальными компьютерными системами} в целом и \textit{семантическая совместимость} между компонентами каждой \textit{интеллектуальной компьютерной системы} (в частности, совместимость между различными \textit{видами знаний} и различными \textit{моделями обработки знаний}), которые являются основными показателями степени \textbf{\textit{конвергенции}} (сближения) между \textit{интеллектуальными компьютерными системами} и их компонентами.}
    \end{scnhassubset}
    \scntext{пояснение}{Особенность указанных характеристик \textit{интеллектуальных компьютерных систем} их компонентов заключается в том, что они играют важную роль при решении всех ключевых задач современного этапа развития \textit{Искусственного интеллекта} и тесно связаны друг с другом.}
    \scntext{пояснение}{Перечисленные требования, предъявляемые к \textit{интеллектуальным компьютерным системам нового поколения}, направлены на преодоление проклятия \textit{вавилонского столпотворения} как внутри \textit{интеллектуальных компьютерных систем нового поколения} (между внутренними \textit{информационными процессами} решения различных задач), так и между взаимодействующими самостоятельными \textit{интеллектуальными компьютерными системами нового поколения} в процессе коллективного решения \textit{комплексных задач}.}
    
    \scnheader{интеллектуальная компьютерная система нового поколения}
    \scntext{примечание}{На современном этапе эволюции \textit{интеллектуальных компьютерных систем} для существенного расширения областей их применения и качественного повышения уровня автоматизации человеческой деятельности:
        \begin{scnitemize}
            \item{необходим переход к созданию \uline{семантически совместимых} \textbf{интеллектуальных компьютерных систем \uline{нового поколения}}, ориентированных не только на индивидуальное, но и на \uline{коллективное} (совместное) решение \textit{комплексных задач}, требующих скоординированной деятельности нескольких самостоятельных интеллектуальных компьютерных систем и использования различных моделей и методов в непредсказуемых комбинациях, что необходимо для существенного расширения сфер применения \textit{интеллектуальных компьютерных систем}, для перехода от автоматизации локальных видов и областей \textit{человеческой деятельности} к комплексной автоматизации более крупных (объединенных) видов и областей этой деятельности;}
            \item{необходима разработка \textbf{Общей формальной теории и стандарта интеллектуальных компьютерных систем нового поколения};}
            \item{необходима разработка \textbf{Технологии комплексной поддержки жизненного цикла интеллектуальных компьютерных систем нового поколения}, которая включает в себя поддержку \textit{проектирования} этих систем (как начального этапа их жизненного цикла) и обеспечение их \textit{совместимости} на всех этапах их жизненного цикла;}
            \item{необходима \textbf{конвергенция} и \textbf{унификация} \textit{интеллектуальных компьютерных систем нового поколения} и их компонентов;}
            \item{необходима реализация \scnqq{бесшовной}, {диффузной}, взаимопроникающей, \textbf{глубокой интеграции семантически смежных компонентов интеллектуальных компьютерных систем}, то есть интеграции, при которой отсутствуют четкие границы (\scnqq{швы}) интегрируемых (соединяемых) компонентов, и которая может осуществляться \uline{автоматически}. Это означает переход к \textbf{\uline{гибридным} интеллектуальным компьютерным системам};}
            \item{необходимо соблюдение \textbf{Принципа бритвы Оккама} — максимально возможное структурное упрощение \textit{интеллектуальных компьютерных систем нового поколения}, исключение \uline{эклектичных} решений;}
            \item{необходима ориентация на потенциально \textbf{универсальные} (то есть способные быстро приобретать \uline{любые} знания и навыки), \textbf{синергетические} \textit{интеллектуальные компьютерные системы} с \scnqq{сильным} интеллектом}
        \end{scnitemize}}
    \begin{scnrelfromlist}{принципы, лежащие в основе}
        \scnfileitem{\textit{смысловое представление знаний} в памяти \textit{интеллектуальных компьютерных систем}, предполагающее отсутствие \textit{омонимических знаков}, которые в разных контекстах обозначают разные сущности, а также отсутствие \textit{синонимии}, то есть пар синонимичных \textit{знаков}, которые обозначают одну и ту же сущность}
        \scnfileitem{смысловое представление информационной конструкции в общем случае имеет нелинейный (графовый) характер представления информации, который является \textit{рафинированной семантической сетью}}
        \scnfileitem{фрактальный характер (масштабируемое самоподобие) структуризации представляемых знаний в базах знаний}
        \scnfileitem{использование \uline{общего} для всех интеллектуальных компьютерных систем \textit{универсального языка смыслового представления знаний} в памяти \textit{интеллектуальных компьютерных систем}, обладающий максимально простым \textit{синтаксисом}, обеспечивающий представление любых \textit{видов знаний} и имеющий неограниченные возможности перехода от \textit{знаний} к \textit{метазнаниям}. Простота синтаксиса \textit{информационных конструкций} указанного \textit{языка} позволяет называть эти конструкции \textit{рафинированными семантическими сетями}}
        \scnfileitem{\textit{структурно-перестраиваемая (графодинамическая) организация памяти} интеллектуальных компьютерных систем, при которой обработка знаний сводится не столько к изменению состояния хранимых \textit{знаков}, сколько к изменению конфигурации связей между этими \textit{знаками}}
        \scnfileitem{\textit{семантически неограниченный ассоциативный доступ к информации}, хранимой в памяти \textit{интеллектуальных компьютерных систем}, по заданному образцу произвольного размера и произвольной конфигурации}
        \scnfileitem{универсальная ситуационная многоагентная модель обработки знаний, ориентированная на обработку смыслового представления информации в ассоциативной графодинамической памяти, \textit{децентрализованное ситуационное управление информационными процессами} в памяти \textit{интеллектуальных компьютерных систем}, реализованное с помощью \textit{агентно-ориентированной модели обработки баз знаний}, в котором \textit{инициирование} новых \textit{информационных процессов} осуществляется не путем передачи управления соответствующим априори известным процедурам, а в результате возникновения соответствующих \textit{ситуаций} или \textit{событий} \textit{в памяти интеллектуальной компьютерной системы}, поскольку \scnqqi{основная проблема компьютерных систем состоит не в накоплении знаний, а в умении активизировать нужные знания в процессе решения задач} (Поспелов Д.~А.). Такой многоагентный процесс обработки информации представляет собой \textit{деятельность}, выполняемую некоторым коллективом \uline{самостоятельных} \textit{информационных агентов} (агентов обработки информации), условием инициирования каждого из которых является появление в текущем состоянии \textit{базы знаний} соответствующей этому агенту \textit{ситуации} и/или \textit{события}.
            \scnqqi{Выбор многоагентных технологий объясняется тем, что в настоящее время любая сложная производственная, логистическая или другая система может быть представлена набором взаимодействий более простых систем до любого уровня детальности, что обеспечивает фрактально-рекурсивный принцип построения многоярусных систем, построенных как открытые цифровые колонии и экосистемы ИИ. В основе многоагентных технологий лежит распределенный или децентрализованный подход к решению задач, при котором динамически обновляющаяся информация в распределенной сети интеллектуальных агентов обрабатывается непосредственно у агентов вместе с локально доступной информацией от \scnqq{соседей}. При этом существенно сокращаются как ресурсные и временные затраты на коммуникации в сети, так и время на обработку и принятие решений в центре системы (если он все-таки есть).}}
        \scnfileitem{агентно-ориентированная модель обработки знаний в памяти интеллектуальной компьютерной системы, обеспечивающая высокую степень \textit{интероперабельности} между внутренними агентами индивидуальной интеллектуальной компьютерной системы, взаимодействующими через общую память (это, фактически, \scnqq{внутренняя} интероперабельность интеллектуальной компьютерной системы нового поколения)}
        \scnfileitem{Переход к \textit{семантическим} \textit{моделям решения задач}, в основе которых лежит учет не только синтаксических (структурных) аспектов обрабатываемой информации, но также и \uline{семантических} (смысловых) аспектов этой информации - \scnqqi{From data science to knowledge science}}
        \scnfileitem{\textbf{\textit{онтологическая модель баз знаний}} \textit{интеллектуальных компьютерных систем}, то есть онтологическая структуризация всей информации, хранимой в памяти \textit{интеллектуальной компьютерной системы}, предполагающая четкую \textit{стратификацию базы знаний} в виде иерархической системы \textit{предметных областей} и соответствующих им \textit{онтологий}, каждая из которых обеспечивает семантическую \textit{спецификацию} всех \textit{понятий}, являющихся ключевыми в рамках соответствующей \textit{предметной области}}
        \scnfileitem{\textbf{\textit{онтологическая локализация решения задач}} в \textit{интеллектуальных компьютерных системах}, предполагающая \uline{локализацию} \textit{области действия} каждого хранимого в памяти \textit{метода} и каждого \textit{информационного агента} в соответствии с \textit{онтологической моделью} обрабатываемой \textit{базы знаний}. Чаще всего, такой \textit{областью действия} является одна из \textit{предметных областей} либо одна из \textit{предметных областей} вместе с соответствующей ей \textit{онтологии}}
        \scnfileitem{\textbf{\textit{онтологическая модель интерфейса}} \textit{интеллектуальной компьютерной системы}}
        \begin{scnindent}
            \begin{scnrelfromlist}{входить в состав}
                \scnfileitem{онтологическое описание \textit{синтаксиса} всех языков, используемых \textit{интеллектуальной компьютерной системой} для общения с \textit{внешними субъектами}}
                \scnfileitem{онтологическое описание \textit{денотационной семантики} каждого языка, используемого \textit{интеллектуальной компьютерной системой} для \textit{общения} с внешними \textit{субъектами}}
                \scnfileitem{семейство \textit{информационных агентов}, обеспечивающих \textit{синтаксический анализ}, \textit{семантический анализ} (перевод на внутренний смысловой язык) и \textit{понимание} (погружение в \textit{базу знаний}) любого введенного \textit{сообщения}, принадлежащего любому \textit{внешнему языку}, полное онтологическое описание которого находится в базе знаний \textit{интеллектуальной компьютерной системы}}
                \scnfileitem{семейство \textit{информационных агентов}, обеспечивающих \textit{синтез сообщений}, которые (1) адресуются внешним субъектам, с которыми общается \textit{интеллектуальная компьютерная система}, (2) \textit{семантически эквивалентны} заданным \textit{фрагментам базы знаний} интеллектуальной компьютерной системы, определяющим \textit{смысл} передаваемых \textit{сообщений}, (3) принадлежат одному из \textit{внешних языков}, полное онтологическое описание которого находится в \textit{базе знаний} интеллектуальной компьютерной системы}
            \end{scnrelfromlist}
        \end{scnindent}
        \scnfileitem{\textit{семантически дружественный характер пользовательского интерфейса}, обеспечиваемый (1) формальным описание в базе знаний средства управления пользовательским интерфейсом и (2) введением в состав \textit{интеллектуальной компьютерной системы} соответствующих help-подсистем, обеспечивающих существенное снижение языкового барьера между пользователями и \textit{интеллектуальными компьютерными системами}, что существенно повысит эффективность \textit{эксплуатации интеллектуальных компьютерных систем}}
        \scnfileitem{\textit{минимизация негативного влияния человеческого фактора} на эффективность \textit{эксплуатации} \textit{интеллектуальных компьютерных систем} благодаря реализации интероперабельного (партнерского) стиля взаимодействия не только между самими \textit{интеллектуальными компьютерными системами}, но также и между \textit{интеллектуальными компьютерными системами} и их пользователями. Ответственность за качество совместной деятельности должно быть распределено между всеми партнерами}
        \scnfileitem{\textbf{\textit{мультимодальность}} (гибридный характер) \textit{интеллектуальной компьютерной системы}}
        \begin{scnindent}
            \begin{scnrelfromlist}{предполагает}
                \scnfileitem{многообразие \textit{видов знаний}, входящих в состав \textit{базы знаний} интеллектуальной компьютерной системы}
                \scnfileitem{многообразие \textit{моделей решения задач}, используемых \textit{решателем задач} интеллектуальной компьютерной системы}
                \scnfileitem{многообразие \textit{сенсорных каналов}, обеспечивающих \textit{мониторинг} состояния \textit{внешней среды} интеллектуальной компьютерной системы}
                \scnfileitem{многообразие \textit{эффекторов}, осуществляющих \textit{воздействие на внешнюю среду}}
                \scnfileitem{многообразие \textit{языков общения} с другими субъектами (с пользователями, с интеллектуальными компьютерными системами)}
            \end{scnrelfromlist}
        \end{scnindent}
        \scnfileitem{\textbf{\textit{внутренняя семантическая совместимость}} между компонентами \textit{интеллектуальной компьютерной системы} (то есть максимально возможное введение общих, совпадающих \textit{понятий} для различных фрагментов хранимой \textit{базы знаний}), являющаяся формой \textbf{\textit{конвергенции}} и \textit{глубокой интеграции} внутри \textit{интеллектуальной компьютерной системы} для различного вида \textit{знаний} и различных \textit{моделей решения задач}, что обеспечивает эффективную реализацию \textit{мультимодальности интеллектуальной компьютерной системы}}
        \scnfileitem{\textbf{\textit{внешняя семантическая совместимость}} между различными \textit{интеллектуальными компьютерными системами}, выражающаяся не только в общности используемых \textit{понятий}, но и в общности базовых \textit{знаний} и являющаяся необходимым условием обеспечения высокого уровня \textit{интероперабельности} интеллектуальных компьютерных систем}
        \scnfileitem{ориентация на использование \textit{интеллектуальных компьютерных систем} как \textit{когнитивных агентов} в составе \textbf{\textit{иерархических многоагентных систем}}}
        \scnfileitem{фрактальный характер (масштабируемое самоподобие) структуризации иерархических коллективов интеллектуальных компьютерных систем нового поколения}
        \scnfileitem{\textbf{\textit{платформенная независимость} интеллектуальных компьютерных систем}}
        \begin{scnindent}
            \begin{scnrelfromlist}{предполагает}
                \scnfileitem{четкую \textit{стратификацию} каждой \textit{интеллектуальной компьютерной системы} (1) на \textit{логико-семантическую модель}, представленную ее \textit{базой знаний}, которая содержит не только \textit{декларативные знания}, но и знания, имеющие \textit{операционную семантику}, и (2) на \textit{платформу}, обеспечивающую \textit{интерпретацию} указанной \textit{логико-семантической модели}}
                \scnfileitem{универсальность указанной \textit{платформы} интерпретации \textit{логико-семантической модели интеллектуальной компьютерной системы}, что дает возможность каждой такой \textit{платформе} обеспечивать интерпретацию любой \textit{логико-семантической модели интеллектуальной компьютерной системы}, если эта модель представлена на том же \textit{универсальном языке смыслового представления информации}}
                \scnfileitem{многообразие вариантов реализации \textit{платформ интерпретации логико-семантических моделей интеллектуальных компьютерных систем} — как вариантов, программно реализуемых на \textit{современных компьютерах}, так и вариантов, реализуемых в виде \textit{универсальных компьютеров нового поколения}, ориентированных на использование в \textit{интеллектуальных компьютерных системах нового поколения} (такие компьютеры мы назвали \textit{ассоциативными семантическими компьютерами})}
                \scnfileitem{легко реализуемую возможность переноса (переустановки) логико-семантической модели (\textit{базы знаний}) любой \textit{интеллектуальной компьютерной системы} на любую другую \textit{платформу интерпретации логико-семантических моделей}}
            \end{scnrelfromlist}
        \end{scnindent}
        \scnfileitem{изначальная ориентация \textit{интеллектуальных компьютерных систем нового поколения} на использование \textbf{\textit{универсальных ассоциативных семантических компьютеров}} (компьютеров нового поколения) в качестве \textit{платформы интерпретации логико-семантических моделей} (баз знаний) \textit{интеллектуальных компьютерных систем}}
    \end{scnrelfromlist}
    \scntext{примечание}{В настоящее время разработано большое количество различного вида моделей решения задач, моделей представления и обработки знаний различного вида. Но в разных \textit{интеллектуальных компьютерных системах} могут быть востребованы разные комбинации этих моделей. При разработке и реализации различных \textit{интеллектуальных компьютерных систем} соответствующие методы и средства должны гарантировать \textit{логико-семантическую совместимость} разрабатываемых компонентов и, в частности, их способность использовать общие \textit{информационные ресурсы}. Для этого, очевидно, необходима \textit{унификация} указанных моделей.}
    \scntext{примечание}{\uline{Многообразие} различных видов интеллектуальных компьютерных систем и, соответственно, многообразие используемых ими комбинаций моделей представления знаний и решения задач определяется:
        \begin{scnitemize}
            \item{многообразием назначения интеллектуальных компьютерных систем и вида окружающей их среды;}
            \item{многообразием различных видов хранимых знаний;}
            \item{многообразием моделей обработки знаний и решений задач;}
            \item{многообразием различных видов сенсорных и эффекторных подсистем.}
        \end{scnitemize}}

    \scnheader{аспекты \textit{совместимости} моделей представления и обработки знаний в \textit{интеллектуальных компьютерных системах}}
    \scnsuperset{синтаксический аспект}
    \scnsuperset{семантический аспект}
    \begin{scnindent}
    \scntext{примечание}{Cогласованность систем понятий, их денотационной семантики}
    \end{scnindent}
    \scnsuperset{функциональный аспект}
    \begin{scnindent}
        \scneq{операционный аспект}
    \end{scnindent}
    
    \scnheader{следует отличать*}
    \begin{scnhaselementset}
    	\scnitem{\textit{совместимость} между компонентами \textit{интеллектуальных компьютерных систем}}
        \scnitem{\textit{совместимость} между верхним логико-семантическим уровнем используемых моделей представления и обработки знаний и различными уровнями их интерпретации вплоть до аппаратного уровня}
        \scnitem{\textit{совместимость} между индивидуальными интеллектуальными компьютерными системами}
    	\scnitem{\textit{совместимость} между индивидуальными интеллектуальными компьютерными системами и их пользователями}
    	\scnitem{\textit{совместимость} между коллективами интеллектуальных компьютерных системам}
    \end{scnhaselementset}

	\scnheader{следует отличать*}
	\begin{scnhaselementset}
		\scnitem{данные}
		\begin{scnindent}
			\scnidtf{информационная конструкция, обрабатываемая с помощью программы традиционного языка программирования}
		\end{scnindent}
		\scnitem{знание}
		\begin{scnindent}
			\scnidtf{семантически целостный фрагмент базы знаний}
		\end{scnindent}
	\end{scnhaselementset}
	\begin{scnindent}
		\scntext{отличие}{Для каждого знания всегда известен язык, на котором это знание представлено и денотационная семантика которого задана. При этом указанный язык имеет достаточно большую семантическую мощность, а в идеале является универсальным языком. В отличие от этого структуризация данных для традиционных программ осуществляется в целях упрощения самих этих программ и, следовательно, для разных программ в общем случае осуществляется по-разному. Таким образом, при разработке традиционных программ представление обрабатываемых данных осуществляется в общем случае на разных языках, денотационная семантика которых нигде не документируется и известна только разработчикам программ. Другими словами, данные для разных программ имеют денотационную семантику не только разную, но еще и априори неизвестную. По сути это форма проявления \textit{вавилонского столпотворения} в традиционных языках программирования, которые образно говоря \scnqq{хромают на одну ногу}, формализуя методы обработки информации, но не формализуя семантику обрабатываемой информации.}
	\end{scnindent} 

\end{scnsubstruct}

		\scnsegmentheader{Предметная область и онтология принципов, лежащие в основе онтологических моделей 
    мультимодальных интерфейсов интеллектуальных компьютерных систем нового поколения}

\begin{scnsubstruct}
    \begin{scnrelfromlist}{ключевое понятие}
    	\scnitem{смысловая память}
    	\scnitem{графодинамическая память}
    	\scnitem{ассоциативная память с информационным доступом по образцу произвольного размера и
            конфигурации}
        \scnitem{система ситуационного децентрализованного управления информационными процессами}
    	\scnitem{многоагентная система обработки информации в общей памяти}
    	\scnitem{язык смыслового представления задач}
        \scnitem{универсальный язык смыслового представления знаний}
    	\scnitem{язык смыслового представления методов}
        \begin{scnindent}
    		\scnidtf{интегрированный язык смыслового представления различного вида программ}
    	\end{scnindent}
    	\scnitem{инсерционная программа}
    \end{scnrelfromlist}
   
    \begin{scnrelfromlist}{ключевое знание}
    	\scnitem{Принципы, лежащие в основе решателей задач индивидуальных интеллектуальных компьютерных
            систем нового поколения}
    \end{scnrelfromlist}

    \scnheader{решатель задач интеллектуальных компьютерных систем нового поколения}
    \begin{scnrelfromlist}{предъявляемые требования}
        \scnitem{решатель задач интеллектуальных компьютерных систем нового поколения должен уметь решать 
            интеллектуальные задачи}
            \begin{scnrelfromlist}{виды задач}
                \scnitem{некачественно сформулированная задача}
                \begin{scnindent}
                    \scnidtf{задача, формулировка которой содержит различные не-факторы (неполнота, нечеткость,
                        противоречивость (некорректность) и так далее)}
                \end{scnindent}
                \scnitem{задача, для решения которой, кроме самой формулировки задачи и соответствующего метода ее
                    решения необходима дополнительная, но априори неизвестно какая информация об объектах, указанных
                    в формулировке (постановке) задачи. При этом указанная дополнительная информация
                    может присутствовать, а может и отсутствовать в текущем состоянии базы знаний интеллектуальных
                    компьютерных систем. Кроме того, для некоторых задач может быть задана (указана) та область
                    базы знаний, использования которой достаточно для поиска или генерации (в частности, логического
                    вывода) указанной дополнительной требуемой информации. Такую область базы знаний будем
                    называть областью решения соответствующей задачи}
                \scnitem{задача, для которой соответствующий метод ее решения в текущий момент не известен}
                \begin{scnrelfromlist}{решение}
                    \scnitem{переформулировать задачу, то есть сгенерировать (логически вывести) логически эквивалентную
                        формулировку исходной задачи, для которой метод ее решения в текущий момент является
                        известным}
                    \scnitem{свести исходную задачу к семейству подзадач, для которых методы их решения в текущий
                        момент известны.}
                \end{scnrelfromlist}
            \end{scnrelfromlist}
        \scnitem{процесс решения задач в интеллектуальных компьютерных системах нового поколения реализуется коллективом
            информационных агентов, обрабатывающих базу знаний интеллектуальных компьютерных систем}
        \scnitem{управление информационными процессами в памяти интеллектуальных компьютерных систем нового
            поколения осуществляется децентрализованным образом по принципам ситуационного управления}
    \end{scnrelfromlist}

    \scnheader{ситуационное управление}
    \scnidtf{ситуационно-событийное управление}
    \scntext{пояснение}{управление последовательностью выполнения действий, при котором условием (scnqq{триггером}) инициирования
        указанных действий является:
        \begin{scnitemize}
            \item{возникновение некоторых ситуаций (условий, состояний);}
            \item{и/или возникновение некоторых событий.}
        \end{scnitemize}}
        
    \scnheader{ситуация}
    \scnidtf{структура, описывающая некоторую временно существующую конфигурацию связей между некоторыми
        сущностями}
    \scnidtf{описание временно существующего состояния некоторого фрагмента (некоторой части) некоторо
        динамической системы}

    \scnheader{событие}
    \scnsuperset{возникновение временной сущности}
    \begin{scnindent}
        \scnidtf{появление, рождение, начало существования некоторой временной сущности}
    \end{scnindent}
    \scnsuperset{исчезновение временной сущности}
    \begin{scnindent}
        \scnidtf{прекращение, завершение существования некоторой временной сущности}
    \end{scnindent}
    \scnsuperset{переход от одной ситуации к другой}
    \begin{scnindent}
        \scntext{примечание}{Здесь учитывается не только факт возникновения новой ситуации, но и ее предыстория — то есть та
        ситуация, которая ей непосредственно предшествует. Так, например, реагируя на аномальное значение
        какого-либо параметра, нам важно знать:
        \begin{scnitemize}
            \item{какова динамика изменения этого параметра (увеличивается он или уменьшается и с какой скоростью);}
            \item{какие меры были предприняты ранее для ликвидации этой аномалии.}
        \end{scnitemize}}
    \end{scnindent}

    \scnheader{решатель задач индивидуальной интеллектуальной компьютерной системы нового поколения}
    \begin{scnrelfromlist}{принципы, лежащие в основе}
        \scnitem{смысловое представление обрабатываемых знаний}
        \scnitem{семантически неограниченный ассоциативный доступ к различным фрагментам знаний, хранимым в
            памяти интеллектуальных компьютерных систем нового поколения (доступ по заданному образцу произвольного
            размера и произвольной конфигурации)}
        \scnitem{графодинамический характер обработки знаний в памяти, при котором обработка знаний сводится не
            только к изменению состояния атомарных фрагментов (ячеек) памяти, но и к изменению конфигурации
            связей между этими атомарными фрагментами}
        \scnitem{ситуационное децентрализованное управление процессом обработки знаний, а также процессом организации
            взаимодействия интеллектуальных компьютерных систем с внешней средой}
        \scnitem{использование семантически мощного языка задач, обеспечивающего представление формулировок самых
            различных задач, которые могут решаться либо в рамках памяти интеллектуальной компьютерной
            системы, либо во внешней среде и которые осуществляют инициирование соответствующих процессов
            решения задач}
        \scnitem{многоагентный характер реализации процессов решения инициированных задач, в основе которого лежит
            иерархическая система агентов, каждый из которых активизируются при возникновении в памяти
            интеллектуальной компьютерной системы соответствующий ситуации или соответствующего события}
    \end{scnrelfromlist}
\end{scnsubstruct}

		\scnsegmentheader{Принципы, лежащие в основе онтологических моделей мультимодальных
    интерфейсов интеллектуальных компьютерных систем нового поколения}

\begin{scnsubstruct}
    \begin{scnrelfromlist}{ключевое понятие}
    	\scnitem{мультимодальный интерфейс}
        \scnitem{вербальный интерфейс}
        \scnitem{естественно-языковой интерфейс}
        \scnitem{внешний язык}
    	\begin{scnindent}
    		\scnidtf{язык обмена сообщениями}
    	\end{scnindent}
        \scnitem{внутренний язык}
    	\begin{scnindent}
    		\scnidtf{язык представления информации в памяти кибернетической системы}
    	\end{scnindent}
        \scnitem{синтаксис внешнего языка}
        \scnitem{денотационная семантика внешнего языка}
        \scnitem{интерфейсная задача}
        \scnitem{понимание сообщения}
        \scnitem{синтез сообщения}
        \scnitem{невербальный интерфейс}
        \scnitem{сенсор}
    	\begin{scnindent}
    		\scnidtf{рецептор}
    	\end{scnindent}
        \scnitem{сенсорная подсистема}
        \scnitem{мультисенсорная подсистема}
        \scnitem{сенсорная информация}
        \scnitem{эффектор}
        \scnitem{мультиэффекторная подсистема}
        \scnitem{сенсо-моторная координация}
    \end{scnrelfromlist}
   
    \begin{scnrelfromlist}{ключевое знание}
    	\scnitem{Принципы, лежащие в основе интерфейсов интеллектуальных компьютерных систем нового
            поколения}
    \end{scnrelfromlist}

    \scnheader{интерфейс интеллектуальной компьютерной системы нового поколения}
    \begin{scnrelfromlist}{принципы, лежащие в основе}
        \scnfileitem{интерфейс \textit{интеллектуальной компьютерной системы нового поколения} рассматривается как решатель
        задач частного вида — \textit{интерфейсных задач}}
        \begin{scnrelfromlist}{основные зачачи}
            \scnfileitem{задачи понимания вербальной информации, приобретаемой интеллектуальной компьютерной системой 
                (синтаксический анализ, семантический анализ и погружение в базу знаний интеллектуальной
                компьютерной системы)}
            \scnfileitem{задачи понимания невербальной информации, воспринимаемой сенсорными подсистемами
                интеллектуальной компьютерной системы (анализ изображений, анализ аудио-сигналов, погружение 
                результатов анализа в базу знаний интеллектуальной компьютерной системы)}
            \scnfileitem{задачи синтеза сообщений, адресуемых внешним субъектам (кибернетическим системам)}
        \end{scnrelfromlist}
        \scnfileitem{тот факт, что интерфейс \textit{интеллектуальной компьютерной системы нового поколения} является 
            решателем частного вида \textit{задач интеллектуальной компьютерной системы нового поколения}, свойства,
            лежащие в основе решателей \textit{задач интеллектуальной компьютерной систем нового поколения}, наследуются
            интерфейсами \textit{интеллектуальной компьютерной систем нового поколения}}
        \begin{scnrelfromlist}{принципы, лежащие в основе}
            \scnfileitem{смысловое представление накапливаемых (приобретаемых знаний)}
            \scnfileitem{трактовка семантического анализа приобретаемой вербальной информации как процесса перевода
                этой информации на внутренний язык смыслового представления знаний с последующим погружением
                (вводом, интеграцией) результата этого перевода в состав текущего состояния базы знаний
                \textit{интеллектуальной компьютерной системы нового поколения}}
            \scnfileitem{трактовка синтеза сообщений, адресуемых внешними субъектами как процесса обратного перевода
                некоторого фрагмента базы знаний с внутреннего языка смыслового представления информации на
                внешний язык, используемый для общения с заданным субъектом}
            \scnfileitem{агентно-ориентированная организация решения интерфейсных задач, реализуемая соответствующим
                коллективов внутренних агентов \textit{интерфейса интеллектуальных компьютерных систем нового 
                поколения}, взаимодействующих через общедоступную для них базу знаний \textit{интеллектуальной
                компьютерной системы нового поколения}}
        \end{scnrelfromlist}
        \scnfileitem{интерфейс \textit{интеллектуальной компьютерной системы нового поколения} трактуется как специализированная
            встроенная \textit{интеллектуальная компьютерная система нового поколения}, входящая в состав
            указанной выше интеллектуальной компьютерной системы, база знаний которой включает в себя:
            \begin{scnitemize}
                \item{онтологию синтаксиса внутреннего языка смыслового преставления информации}
                \item{онтологию денотационной семантики внутреннего языка смыслового представления информации}
                \item{онтологию синтаксиса всех внешних языков, используемых для общения с внешними субъектами}
                \item{онтологии денотационной семантики всех внешних языков, используемых для общения с внешними субъектами (каждая такая онтология с формальной точки зрения является описанием соответствия между текстами внешних языков и семантически эквивалентными им текстами внутреннего языка смыслового представления информации)}
            \end{scnitemize}}
        \scntext{примечание}{Подчеркнем при этом, что все указанные онтологии, входящие в состав базы знаний интерфейса
            интеллектуальных компьютерных систем нового поколения, как и вся остальная информация, входящая в
            состав этой базы знаний, представляется на внутреннем языке смыслового представления информации,
            который, соответственно используется в данном случае как метаязык}
    \end{scnrelfromlist}

    \scnheader{интерфейс индивидуальной интеллектуальной компьютерной системы нового поколения}
    \begin{scnrelfromlist}{принципы, лежащие в основе}
        \scnfileitem{интерфейс индивидуальной интеллектуальной компьютерной системы нового поколения является
            специализированным компонентом решателя задач интеллектуальной компьютерной системы нового поколения,
            то есть специализированной \uline{встроенной} (в индивидуальную интеллектуальную компьютерную систему
            нового поколения) интеллектуальной компьютерной системой нового поколения, ориентированной на
            решение интерфейсных задач, к которым относятся:
            \begin{scnitemize}
                \item{понимание принятых сообщений (их перевод на язык внутреннего смыслового представления информации
                    и погружения в текущее состояние базы знаний)}
                \item{синтез передаваемых сообщений (перевод сформированного сообщения с внутреннего языка смыслового
                    представления на используемый внешний язык)}
                \item{первичный анализ приобретаемой сенсорной информации, предполагающий распознавание некоторого
                    семейства первичных образов и сцен}
                \scnfileitem{сенсомоторная координация действий, выполняемых эффекторами интеллектуальной компьютерной системы}
            \end{scnitemize}}
        \scnfileitem{мультимодальный характер интерфейса — многообразие внешних языков, видов сенсоров и эффекторов}
        \scnfileitem{формальное онтологическое описание на языке внутреннего смыслового представления информации}
        \begin{scnrelfromlist}{виды информации}
            \scnfileitem{синтаксиса и денотационной семантики всех используемых внешних языков}
            \scnfileitem{первичных образов и сцен (ситуаций), являющихся результатом первичного анализа приобретаемой
                сенсорной информации}
            \scnfileitem{методов низкого уровня, непосредственно интерпретируемых эффекторами интеллектуальной компьютерной системы}
        \end{scnrelfromlist}
    \end{scnrelfromlist}
    \scntext{примечание}{Разговоры о дружественном и, в частности, адаптивном \textit{пользовательском интерфейсе} ведутся давно, но это, чаще
        всего, касается формы (\scnqq{синтаксической} стороны) \textit{пользовательского интерфейса}, а не смыслового содержания
        взаимодействия с пользователями. В настоящее время \textit{пользовательские интерфейсы} компьютерных систем (в
        том числе и \textit{интеллектуальных компьютерных систем}) для широкого контингента пользователей не являются
        семантически (содержательно) дружественными (семантически комфортными). Организация взаимодействия
        пользователей с компьютерными системами (в том числе и с \textit{интеллектуальными компьютерными системами})
        является \scnqq{узким местом}, оказывающим существенное влияние на эффективность \textit{автоматизации человеческой
        деятельности}. В основе современной организации взаимодействия пользователя с компьютерной системой лежит
        парадигма \uline{грамотного} пользователя, который знает, чего он хочет от используемого им инструмента и несет полную
        ответственность за качество взаимодействия с этим инструментом. Эта парадигма лежит в основе деятельности
        лесоруба во взаимодействии с топором, всадника во взаимодействии с лошадью, автоводителя, летчика во взаимодействии
        с соответствующим транспортным средством, оператора атомной электростанции, железнодорожного диспетчера и так далее.}
    \scntext{примечание}{На современном этапе развития \textit{Искусственного интеллекта} для повышения эффективности взаимодействия
        необходим переход \uline{от парадигмы грамотного управления} используемым инструментом \uline{к парадигме равноправного
        сотрудничества}, партнерскому взаимодействию интеллектуальной компьютерной системы со своим пользователем.
        \textit{Интеллектуальная компьютерная система} должна повернуться \scnqq{лицом} к пользователю. Семантическая дружественность 
        пользовательского интерфейса должна заключаться в адаптивности к особенностям и квалификации пользователя, исключении 
        любых проблем для пользователя в процессе диалога с \textit{интеллектуальной компьютерной системой}, в перманентной заботе о 
        совершенствовании коммуникационных навыков пользователя.}
    \scntext{примечание}{При организации взаимодействия пользователя с \textit{Глобальной сетью} компьютерным системам необходимо перейти
        от парадигмы \scnqq{многооконного} интерфейса, в каждом \scnqq{окне} которого свои \scnqq{правила игры}, к парадигме \scnqq{одного
        окна}. Пользователь не должен знать, какое \scnqq{окно} ему надо \scnqq{открыть} (в какую систему ему надо войти) для
        удовлетворения той или иной его потребности.
        Пользователь не должен знать, какая конкретно система будет решать его задачу. Пользователь должен уметь с
        помощью \uline{универсальных} средств сформулировать свою задачу, а соответствующая компьютерная система, входящая 
        в \textit{Глобальную сеть} и способная решить эту задачу, должна сама инициироваться, реагируя на факт появления
        указанной задачи. Таким образом пользовательский интерфейс должен быть интерфейсом пользователя не с 
        конкретной компьютерной системой, а в целом со всей \textit{Глобальной сетью компьютерных систем}.}
\end{scnsubstruct}

		\scnsegmentheader{Достоинства предлагаемых принципов, лежащие в основе
интеллектуальных компьютерных систем нового поколения}

\begin{scnsubstruct}
    \scnheader{принципы, лежащих в основе интеллектуальных компьютерных систем нового поколения}
    \scntext{достоинство}{\textbf{смысловое представление информации} в памяти \textit{интеллектуальных компьютерных систем} обеспечивает
        устранение дублирования информации, хранимой в памяти \textit{интеллектуальной компьютерной системы}, то есть
        устранение многообразия форм представления одной и той же информации, запрещение появления в одной памяти 
        \textit{семантически эквивалентных информационных конструкций} и, в том числе, синонимичных \textit{знаков}. Это
        существенно снижает сложность и повышает качество:
    \begin{scnitemize}
        \item{разработки различных \textit{моделей обработки знаний} (так как нет необходимости учитывать многообразие форм
            представления одного и того же знания);}
        \item{\textit{семантического анализа} и \textit{понимания} информации, поступающей (передаваемой) от различных внешних
            субъектов (от пользователей, от разработчиков, от других \textit{интеллектуальных компьютерных систем});}
        \item{\textit{конвергенции} и \textit{интеграции} различных видов знаний в рамках каждой \textit{интеллектуальной компьютерной
            системы};}
        \item{обеспечения \textit{семантической совместимости} и \textit{взаимопонимания} между различными \textit{интеллектуальными
            компьютерными системами}, а также между \textit{интеллектуальными компьютерными системами} и их пользователями}
    \end{scnitemize}}

    \scntext{достоинство}{Понятие \textit{семантической сети} нами рассматривается не как красивая метафора сложноструктурированных
        \textit{знаковых конструкций}, а как формальное уточнение понятия \textit{смыслового представления информации}, как принцип
        представления информации, лежащей в основе нового поколения \textit{компьютерных языков} и самих \textit{компьютерных 
        систем} — \textit{графовых языков} и \textit{графовых компьютеров}. \textit{Семантическая сеть} — это нелинейная (графовая)
        \textit{знаковая конструкция}, обладающая следующими свойствами:
        \begin{scnitemize}
            \item{все элементы (то есть синтаксически элементарные фрагменты) этой \textit{графовой структуры} (узлы и связки)
                являются знаками описываемых сущностей и, в частности, \textit{знаками связей} между этими сущностями;}
            \item{все знаки, входящие в эту \textit{графовую структуру}, не имеют \textit{синонимов} в рамках этой структуры;}
            \item{\scnqq{внутреннюю} структуру (строение) \textit{знаков}, входящих в семантическую сеть не требуется учитывать при ее
                \textit{семантическом анализе} (понимании);}
            \item{смысл \textit{семантической сети} определяется денотационной семантикой всех входящих в нее знаков и конфигурацией
                \textit{связей инцидентности} этих знаков;}
            \item{из двух \textit{инцидентных знаков}, входящих в \textit{семантическую сеть}, по крайней мере один является знаком связи.}
        \end{scnitemize}}

    \scntext{достоинство}{\textit{рафинированная семантическая сеть} — это \textit{семантическая сеть}, имеющая максимально простую 
        \textit{синтаксическую структуру}, в которой, в частности,
        \begin{scnitemize}
            \item{используется \uline{конечный} \textit{алфавит} элементов \textit{семантической сети}, то есть конечное число синтаксически
                выделяемых типов (синтаксических меток), приписываемых этим элемента;}
            \item{внешние идентификаторы (в частности, имена), приписываемые элементам \textit{семантической сети} используются
                \uline{только} для ввода/вывода информации}
        \end{scnitemize}}

    \scntext{достоинство}{\textit{агентно-ориентированная модель обработки информации} в сочетании с \textit{децентрализованным ситуационным
        управлением процессом обработки информации}, а также со \textit{смысловым представлением информации} в памяти
        \textit{интеллектуальной компьютерной системы} существенно снижает сложность и повышает качество интеграции
        \begin{scnitemize}
        \item{беспечивает автоматизацию решения сложных комплексных задач, для которых требуется создание
            временных или постоянных \uline{коллективов};}
        \item{превращает \textit{интеллектуальные компьютерные системы} в \uline{самостоятельные} активные \textit{субъекты}, способные
            инициировать различные комплексные задачи и, собственно, инициировать для этого работо-
            способные коллективы, состоящие из людей и \textit{интероперабельных интеллектуальных компьютерных
            систем} требуемой квалификации}
        \end{scnitemize}}

    \scntext{достоинство}{Высокий уровень семантической гибкости информации, хранимой в памяти интеллектуальной компьютерной
        системы нового поколения, обеспечивается тем, что каждое удаление или добавление синтаксически элементарного
        фрагмента хранимой информации, а также удаление или добавление каждой связи инцидентности между такими
        элементами имеет четкую семантическую интерпретацию.}

    \scntext{достоинство}{Высокий уровень стратифицированности информации, хранимой в памяти интеллектуальной компьютерной
        системы нового поколения, обеспечивается онтологически ориентированной структуризацией базы знаний интеллектуальной
        компьютерной системы нового поколения.}

    \scntext{достоинство}{Высокий уровень индивидуальной обучаемости интеллектуальных компьютерных систем нового поколения (то
        есть их способности к быстрому расширению своих знаний и навыков) обеспечивается:
        \begin{scnitemize}
            \item{семантической гибкостью информации, хранимой в их памяти;}
            \item{стратифицированностью этой информации;}
            \item{рефлексивностью интеллектуальных компьютерных систем нового поколения.}
        \end{scnitemize}}

    \scntext{достоинство}{Высокий уровень коллективной обучаемости интеллектуальных компьютерных систем нового поколения
        обеспечивается высоким уровнем их интероперабельности (их социализации, способности к эффективному участию в
        деятельности различных коллективов, состоящих из интеллектуальных компьютерных систем нового поколения и
        людей) и, прежде всего, высоким уровнем их взаимопонимания.}

    \scntext{достоинство}{Высокий уровень интероперабельности интеллектуальных компьютерных систем нового поколения
        принципиально меняет характер взаимодействия компьютерных систем с людьми, автоматизацию деятельности которых они
        осуществляют, — от управления этими средствами автоматизации к равноправным партнерским осмысленным
        взаимоотношениям}

    \scntext{достоинство}{Каждая интеллектуальная компьютерная система нового поколения способна:
        \begin{scnitemize}
            \item{самостоятельно или по приглашению войти в состав коллектива, состоящего из интеллектуальных компьютерных 
                систем нового поколения и/или людей. Такие коллективы создаются на временной или постоянной основе
                для коллективного решения сложных задач;}
            \item{участвовать в распределении (в том числе в согласовании распределения) задач — как \scnqq{разовых} задач, так и
                долгосрочных задач (обязанностей);}
            \item{мониторить состояние всего процесса коллективной деятельности и координировать свою деятельность с
                деятельностью других членов коллектива при возможных непредсказуемых изменениях условий (состояния)
                соответствующей среды.}
        \end{scnitemize}}

    \scntext{достоинство}{Высокий уровень интеллекта интеллектуальных компьютерных систем нового поколения и, соответственно,
        высокий уровень их самостоятельности и целенаправленности позволяет им быть полноправными членами самых
        различных сообществ, в рамках которых интеллектуальные компьютерные системы нового поколения получают
        права самостоятельно инициировать (на основе детального анализа текущего положения дел и, в том числе,
        текущего состояния плана действий сообщества) широкий спектр действий (задач), выполняемых другими членами
        сообщества, и тем самым участвовать в согласовании и координации деятельности членов сообщества. Способность
        интеллектуальной компьютерной системы нового поколения согласовывать свою деятельность с другими
        подобными системами, а также корректировать деятельность всего коллектива интеллектуальных компьютерных
        систем нового поколения, адаптируясь к различного вида изменениям среды (условий), в которой эта деятельность
        осуществляется, позволяет существенно автоматизировать деятельность системного интегратора как на этапе
        создания коллектива интеллектуальных компьютерных систем нового поколения, так и на этапе его обновления
        (реинжиниринга)}
    
    \scntext{примечание}{Достоинства интеллектуальных компьютерных систем нового поколения обеспечиваются:
        \begin{scnitemize}
            \item{достоинствами языка внутреннего смыслового кодирования информации, хранимой в памяти этих систем;}
            \item{достоинствами организации графодинамической ассоциативной смысловой памяти интеллектуальных
                компьютерных систем нового поколения;}
            \item{достоинствами смыслового представления баз знаний интеллектуальных компьютерных систем нового
                поколения и средствами онтологической структуризации баз знаний этих систем;}
            \item{достоинствами агентно-ориентированных моделей решения задач, используемых в интеллектуальных
                компьютерных системах нового поколения в сочетании с децентрализованным управлением процессом обработки
                информации.}
        \end{scnitemize}}

    \begin{scnrelfromlist}{основные положения}
    \scnfileitem{основным практически значимым направлением развития современных интеллектуальных компьютерных
        систем является переход к интероперабельным интеллектуальным компьютерным системам, способным к эффективному
        взаимодействию между собой и с пользователями, что:
        \begin{scnitemize}
            \item{обеспечивает автоматизацию решения сложных комплексных задач, для которых требуется создание
                временных или постоянных \uline{коллективов;}}
            \item{превращает интеллектуальные компьютерные системы в \uline{самостоятельные} активные субъекты, способные
                инициировать различные комплексные задачи и, собственно, инициировать для этого работоспособные
                коллективы, состоящие из людей и интероперабельных интеллектуальных компьютерных систем требуе-
                мой квалификации.}
        \end{scnitemize}}
    \scnfileitem{коллективы, состоящие из самостоятельных \textit{интероперабельных интеллектуальных компьютерных систем}
        и людей, имеют хорошие перспективы стать \textit{синергетическими} системами}
    \scnfileitem{\textit{интероперабельность интеллектуальных компьютерных систем} обеспечивается:
        \begin{scnitemize}
            \item{высоким уровнем взаимопонимания и, соответственно, семантической совместимостью;}
            \item{высоким уровнем договороспособности, то есть способности предварительно согласовывать свои действия
                с действиями других субъектов;}
            \item{высоким уровнем способности оперативно координировать свои действия с действиями других субъектов
                в ходе их выполнения}
        \end{scnitemize}}
    \scnfileitem{к числу принципов, лежащих в основе построения \textit{интероперабельных интеллектуальных компьютерных
        систем}, относятся:
        \begin{scnitemize}
            \item{смысловое представление знаний в памяти \textit{интеллектуальных компьютерных систем} в виде рафинированных 
                семантических сетей;}
            \item{использование универсального языка внутреннего смыслового представления знаний;}
            \item{графодинамическая организация обработки знаний;}
            \item{агентно-ориентированные модели решения задач;}
            \item{структуризация и стратификация баз знаний в виде иерархической системы формальных онтологий;}
            \item{семантически дружественный пользовательский интерфейс.}
        \end{scnitemize}}
    \scnfileitem{для разработки большого количества интероперабельных семантически совместимых \textit{интеллектуальных
        компьютерных систем}, обеспечивающих переход на принципиально новый уровень автоматизации \textit{человеческой
        деятельности}, необходимо создание технологии, обеспечивающей массовое производство таких \textit{интеллектуальных
        компьютерных систем}, участие в котором доступно широкому контингенту разработчиков (в том
        числе разработчиков средней квалификации и начинающих разработчиков). Основными положениями такой
        технологии являются
        \begin{scnitemize}
            \item{стандартизация \textit{интероперабельных интеллектуальных компьютерных систем};}
            \item{широкое использование \textit{компонентного проектирования} на основе мощной библиотеки семантически
                совместимых многократно используемых (типовых) компонентов \textit{интероперабельных интеллектуальных
                компьютерных систем}}
        \end{scnitemize}}
    \scnfileitem{эффективная эксплуатация \textit{интероперабельных интеллектуальных компьютерных систем} требует создания
        не только \textit{технологии проектирования} таких систем, но также и семейства технологий поддержки всех
        остальных этапов их жизненного цикла. Особенно это касается технологии перманентной поддержки \textit{семантической
        совместимости} всех взаимодействующих \textit{интероперабельных интеллектуальных компьютерных систем} в ходе их эксплуатации}
    \end{scnrelfromlist}
\end{scnsubstruct}

	\end{scnsubstruct}
\end{SCn}
\scnsourcecomment{Завершили Раздел \scnqqi{Предметная область и онтология интеллектуальных компьютерных систем нового поколения}}


\scsubsubsection{Пункт 4.2.1. Предметная область и онтология смыслового представления информации}
\label{sd_sem_inf_rep}
\begin{SCn}
	\scnsectionheader{Предметная область и онтология смыслового представления информации}

	\begin{scnsubstruct}

		\scnrelto{частная предметная область и онтология}{Предметная область и
			онтология информационных конструкций}
		\begin{scnhaselementrolelist}{класс объектов исследования}
			\scnitem{смысловое представление информации}
			\begin{scnindent}
				\scnidtf{смысл}
			\end{scnindent}
		\end{scnhaselementrolelist}

		\begin{scnhaselementrolelist}{класс объектов исследования}
			\scnitem{семантическая сеть}
				\begin{scnindent}
					\begin{scnsubdividing}
						\scnitem{нерафинированная семантическая сеть}
						\scnitem{рафинированная семантическая сеть}
					\end{scnsubdividing}
					\begin{scnsubdividing}
						\scnitem{абстрактная семантическая сеть}
							\begin{scnindent}
								\scnidtf{семантическая сеть, абстрагирующаяся от того, как
									физически представлены ее элементарные (атомарные) фрагменты, а также связи
									инцидентности между этими фрагментами}
							\end{scnindent}
						\scnitem{графически представленная семантическая сеть}
							\begin{scnindent}
								\scnidtf{нарисованная семантическая сеть}
							\end{scnindent}
						\scnitem{семантическая сеть, хранимая в графодинамической памяти}
							\begin{scnindent}
								\scnrelboth{следует отличать}{представление семантической сети
									в адресной памяти}
								\scnnotsubset{семантическая сеть}
								\scnidtf{представление семантической сети в виде линейной
									информационной конструкции, которая хранится в адресной памяти и которая,
									строго говоря, уже не является семантической сетью, но является информационной
									конструкцией, семантически эквивалентной соответствующей (представляемой)
									семантической сети}
							\end{scnindent}
					\end{scnsubdividing}
				\end{scnindent}
			\scnitem{граф знаний}
			\begin{scnindent}
				\scnidtf{представление сложноструктурированного знания в виде графовой структуры}
			\end{scnindent}
			\scnitem{язык семантических сетей}
				\begin{scnindent}
					\scnidtf{язык, все тексты которого являются семантическими
						сетями}
					\begin{scnsubdividing}
						\scnitem{специализированный язык семантических сетей}
						\scnitem{универсальный язык семантических сетей}
					\end{scnsubdividing}
					\scnsuperset{язык рафинированных семантических сетей}
				\end{scnindent}
		\end{scnhaselementrolelist}

		\begin{scnrelfromvector}{рассматриваемые вопросы}
			\scnfileitem{Что такое семантические сети и в чем их принципиальное
				отличие от других вариантов представления информации}
			\scnfileitem{До какой степени можно минимизировать алфавит элементов
				семантических сетей}
			\scnfileitem{Можно ли все описываемые связи свести к бинарным связям и
				почему это целесообразно}
			\scnfileitem{Можно ли разработать \uline{универсальный} язык
				семантических сетей}
			\scnfileitem{До какой степени можно упростить синтаксические структуры
				семантических сетей до, условно говоря, рафинированного вида}
			\scnfileitem{Какими достоинствами обладает семантические сети}
		\end{scnrelfromvector}

		\begin{scnrelfromlist}{ссылка}
			\scnitem{Понятие Технологии OSTIS}
				\begin{scnindent}
					\scntext{аннотация}{В указанном сегменте \textit{Стандарта
							OSTIS} рассматриваются принципы, лежащие в основе \textit{Технологии OSTIS},
						основным из которых является ориентация на использование
						\textit{\uline{универсального} языка рафинированных семантических сетей} в
						качестве внутреннего языка \textit{интеллектуальных компьютерных систем}}
				\end{scnindent}
			\scnitem{Описание внутреннего языка ostis-систем}
				\begin{scnindent}
					\scntext{аннотация}{В указанном разделе \textit{Стандарта
						OSTIS} рассматриваются принципы, лежащие в основе \textit{универсального языка
						рафинированных семантических сетей}, используемого в качестве внутреннего языка
						\textit{ostis-систем} --- \textit{интеллектуальных компьютерных систем}
						следующего поколения}
				\end{scnindent}
			\scnitem{Описание языка графического представления знаний ostis-систем}
				\begin{scnindent}
					\scntext{аннотация}{В указанном разделе \textit{Стандарта
						OSTIS} рассматриваются принципы, лежащие в основе универсального языка
						графически представленных семантических сетей, используемого в
						\textit{пользовательском интерфейсе ostis-систем}}
				\end{scnindent}
			\scnitem{\cite{Birukov1960}}
			\scnitem{\cite{Morris2001}}
			\scnitem{\cite{Pirs2009}}
			\scnitem{\cite{Stepanov1971}}
			\scnitem{\cite{MelchukST}}
		\end{scnrelfromlist}

		\bigskip
		\scnheader{знак}
		\scnidtf{фрагмент информационной конструкции, обладающий свойством,
			\uline{обозначать} некоторую сущность (объект), которая наряду с другими
			сущностями описывается указанной информационной конструкцией}
		\scntext{примечание}{\uline{Форма} представления знаков в известной степени
			условна и является результатом соглашения между носителями соответствующего
			языка. Знак может быть, например, представлен:
			\begin{scnitemize}
				\item  в виде фрагмента речевого сообщения (последовательностью
				фонем);
				\item в виде строки символов (последовательности букв) в
				заданном алфавите;
				\item в виде иероглифа, пиктограммы;
				\item в виде жеста.
			\end{scnitemize}}
		\scniselementrole{ключевой знак}{Предметная область и онтология
			информационных конструкций}
			\begin{scnindent}
				\scnhaselement{раздел Базы знаний IMS.ostis}
			\end{scnindent}
		\scntext{характеристика элементов данного множества}{Знаки,
			используемые в различных языках, характеризуются:
			\begin{scnitemize}
				\item синтаксической структурой, по совпадению (изоморфизму)
				которых для разных знаокв предполагается их синонимия;
				\item денотационной семантикой, т.е. той сущностью, которая
				обозначается соответствующим знаком;
				\item типом (классом) обозначаемой сущности, которая может
				быть:
				\begin{scnitemizeii}
					\item материальным(физическим) элементом (точкой)
					абстрактного пространства, множеством, которое может быть:
					\begin{scnitemizeiii}
						\item связью;
						\item классом;
						\item структурой;
					\end{scnitemizeiii}
					\item реальной и вымышленной сущностью;
					\item константной (конкретной) и переменной
					(произвольной) сущностью;
					\item постоянно существующей и временно существующей
					сущностью (прошлой, настоящей, будущей);
				\end{scnitemizeii}
				\item множеством тех связей, которые связывают сущность,
				обозначаемую данным знаком с другими сущностями, а также, если данный знак
				обозначает некоторую связь, множеством сущностей, которые связаны этой связью,
				т.е. сущностей, являющихся компонентом этой связи;
				\item текущим статусом самого знака в памяти кибернетической
				системы, который может быть:
				\begin{scnitemizeii}
					\item логически удаленным знаком;
					\item настоящим знаком;
					\item предлагаемым (возможно, будущим) знаком.
				\end{scnitemizeii}
			\end{scnitemize}}

		\bigskip
		\scnheader{денотат*}
		\scnidtf{денотат заданного знака*}
		\scnidtf{объект, обозначаемый заданным знаком*}
		\scnidtf{денотационная семантика заданного знака*}
		\scnidtf{смысл заданного знака*}
		\scnidtf{Бинарное ориентированное отношение, каждая пара которого
			связывает:
			\begin{scnitemize}
				\item некоторый знак, представленный в той или иной форме в тексте
				исследуемого языка;
				\item \uline{со знаком} той сущности, которая обозначается указанным
				выше знаком в рамках используемого метаязыка.
			\end{scnitemize}}

		\scntext{примечание}{Данное отношение используется, когда с помощью одного
			языка необходимо описать денотационную семантику другого языка. Фактически речь
			идет о переводе заданного знака, входящего в состав некоторого рассматриваемого
			текста, принадлежащего некоторому исследуемому языку (языку-объекту), на
			некоторый метаязык (в нашем случае на SC-код), денотационная семантика которого
			нам считается априори известной. Указанный перевод есть связь заданного знака с
			синонимичным ему знаком, входящим в состав текста, принадлежащего указанному
			метаязыку.}
		\scnrelboth{обратное отношение}{внешний sc-идентификатор*}
			\begin{scnindent}
				\scnidtf{быть знаком, обозначающим заданную сущность*}
			\end{scnindent}

		\scnheader{информационная конструкция}
		\scnidtf{информация}
		\scntext{примечание}{В общем случае информационная конструкция представляет
			собой сложную иерархическую структуру, каждому уровню иерархии которой
			соответствует определенный класс информационных конструкций.}
		\scnsuperset{синтаксически элементарный фрагмент информационной конструкции}
			\begin{scnindent}
				\scnidtf{атомарный фрагмент информационной конструкции}
				\scnidtf{элемент информационной конструкции}
				\scntext{примечание}{Примерами таких элементарных фрагментов информационных
					конструкций являются буквы}
				\scnsuperset{буква}
			\end{scnindent}
		\scnsuperset{простой знак}
			\begin{scnindent}
				\scnidtf{семантически элементарный фрагмент информационной конструкции}
				\scnsubset{знак}
			\end{scnindent}
		\scnsuperset{выражение}
			\begin{scnindent}
				\scnidtf{сложный (непростой) знак}
				\scnidtf{знак, являющийся одновременно некоторым знанием обозначаемой
					сущности (спецификацией этой сущности)}
				\scnidtf{знак, построенный как выражение вида тот, который... }
				\scnidtf{знак, в состав которого входят другие знаки}
				\scnsubset{знак}
			\end{scnindent}
		\scnsuperset{простой текст}
			\begin{scnindent}
				\scnidtf{минимальная синтаксически целостная и корректная (правильная)
					информационная конструкция, включающая в себя:
					\begin{scnitemize}
						\item знак некоторой описываемой связи;
						\item минимальную спецификацию указанного знака связи (указание
						отношения, которому это связь принадлежит);
						\item указание \uline{всех} компонентов описываемой связи (знаков всех
						сущностей, связываемых этой связью, и/или всех знаков, связываемых этой связью
						-- описываемая связь может связывать не только внешние	описываемые сущности,
						но и сами знаки);
						\item если описываемая связь не является бинарной, то связи с её
						компонентами могут потребовать явного представления знаков этих связей с
						дополнительным указанием роли этих компонентов.
					\end{scnitemize}}
				\scnsubset{текст}
			\end{scnindent}
		\scnsuperset{сложный текст}
			\begin{scnindent}
				\scnidtf{информационная конструкция, являющаяся результатом соединения
					нескольких простых текстов}
				\scnsubset{текст}
			\end{scnindent}
		\scnsuperset{простое знание}
			\begin{scnindent}
				\scnidtf{минимальная семантические целостная информационная конструкция}
				\scnidtf{знание, в состав которого не входят другие знания}
				\scnsubset{знание}
			\end{scnindent}
		\scnsuperset{сложное знание}
			\begin{scnindent}
				\scnidtf{информационная конструкция, являющаяся результатом соединения
					нескольких простых знаний}
				\scnidtf{знание, в состав которого не входят другие знания}
				\scnsubset{знание}
			\end{scnindent}
		\scniselementrole{ключевой знак}{Предметная область и онтология информационных конструкций}

		\scnheader{стандартизация моделей представления и обработки информации}
		\scnidtf{предлагаемый подход к решению проблем, препятствующих дальнейшей эволюции компьютерных систем и технологий}
		\scntext{примечание}{Анализ проблем эволюции компьютерных систем разного уровня сложности, разного уровня обучаемости и
		интеллектуальности, разного назначения показывает, что проклятие \scnqq{вавилонского столпотворения} и, как следствие,
		несовместимость, дублирование и субъективизм согласовываемых информационных ресурсов и моделей их обработки 
		нас преследует везде:
		\begin{scnitemize}
			\item{и в развитии традиционных компьютерных систем;}
			\item{и в развитии технологий искусственного интеллекта;}
			\item{и в развитии методов и средств информатизации научной и инженерной деятельности.}
		\end{scnitemize}}
		

		\scnheader{проблема обеспечения совместимости информационных ресурсов и моделей их обработки}
		\begin{scnrelfromlist}{аспекты решения}
			\scnitem{обеспечение совместимости между различными компонентами компьютерных систем, а также между
			целостными компьютерными системами, входящими в коллективы компьютерных систем}
			\scnitem{обеспечение совместимости, то есть высокого уровня взаимопонимания между различными компьютерными 
			системами и их пользователями}
			\scnitem{беспечение междисциплинарной совместимости, то есть конвергенции различных областей знаний}
			\scnitem{методы и средства постоянного мониторинга и восстановления совместимости в условиях интенсивной
			эволюции компьютерных систем и их пользователей, которая часто нарушает достигнутую совместимость
			(согласованность) и требует дополнительных усилий на ее восстановление}
		\end{scnrelfromlist}

		\scnheader{подход к решению проблем эволюции компьютерных систем}
		\scntext{примечание}{Суть предлагаемого нами подхода к решению проблем эволюции компьютерных систем заключается:
		\begin{scnitemize}
			\item{в объединении всех указанных выше направлений эволюции компьютерных систем (как общих направлений, так
				и частных)}
			\item{в трактовке проблемы обеспечения \textbf{совместимости} различных видов знаний, различных
				моделей решения задач, различных компьютерных систем как \textbf{ключевой проблемы} эволюции компьютерных
				систем, решение которой существенно упростит решение и многих других проблем}
		\end{scnitemize}}

		\scnheader{совместимость}
		\scntext{примечание}{Без обеспечения совместимости информационных ресурсов, используемых в разных компьютерных
		системах, а также информационных ресурсов, представляющих знания различного семантического вида невозможно:
		\begin{scnitemize}
			\item{ни создавать \textbf{коллективы компьютерных систем}, способные координировать свои действия при кооперативном
				расширении сложных задач;}
			\item{ни создавать \textbf{гибридные компьютерные системы}, которые способны при решении сложных комплексных
				задач использовать всевозможные сочетания разных видов знаний и разных моделей решения задач;}
			\item{ни использовать \textbf{компонентную методику проектирования} компьютерных систем \textbf{на всех уровнях} иерархии
				проектируемых систем.}
		\end{scnitemize}
		О какой информационной совместимости и взаимопонимании (в том числе между специалистами) можно говорить
		при наличии ужасающей понятийной и терминологической неряшливости, терминологического псевдотворчества,
		в том числе, в области информатики.}
		
		\scnheader{следует отличать*}
		\begin{scnhaselementset}
			\scnitem{совместимость как один из факторов обучаемости, как \textbf{способность} к быстрому повышению уровня
				согласованности (интеграции, взаимопонимания). Сравните обучаемость как \textbf{способность} к быстрому расширению
				знаний и навыков, но никак не характеристика объема и качества приобретенных знаний и навыков}
			\scnitem{совместимость как характеристика достигнутого уровня согласованности (интеграции, взаимопонимания)}
		\end{scnhaselementset}

		\scnheader{следует отличать*}
		\begin{scnhaselementset}
			\scnitem{интеллект компьютерной системы как \textbf{уровень} (объем и качество) приобретенных знаний и навыков}
			\scnitem{интеллект компьютерной системы как \textbf{способность} к быстрому расширению и
				совершенствованию знаний и навыков, то есть как \textbf{скорость} повышения уровня знаний и навыков}
		\end{scnhaselementset}

		\scnheader{следует говорить*}
		\begin{scnhaselementset}
			\scnitem{о \textbf{способности} к быстрому повышению уровня согласованности}
			\scnitem{о достигнутом уровне согласованности}
			\scnitem{о самом \textbf{процессе} повышения уровня согласованности}
			\scnitem{о перманентном процессе восстановления (поддержки, сохранения) достигнутого уровня согласованности,
				поскольку в ходе эволюции компьютерных систем и их пользователей (то есть в ходе расширения и повышения
				качества их знаний и навыков) уровень их согласованности может понижаться}
		\end{scnhaselementset}

		\scnheader{обеспечения совместимости различных видов знаний, различных моделей решения задач и различных компьютерных систем}
		\begin{scnrelfromlist}{главный фактор}
			\scnitem{унификация представления информации в памяти компьютерных систем}
				\begin{scnindent}
					\scnidtf{стандартизация представления информации в памяти компьютерных систем}
				\end{scnindent}
			\scnitem{унификация принципов организации обработки информации в памяти компьютерных систем}
		\end{scnrelfromlist}

		\scnheader{унификация представления информации в памяти компьютерных систем}
		\begin{scnrelfromset}{предполагает}
			\scnitem{синтаксическая унификация используемой информации}
			\begin{scnindent}
				\scnidtf{унификация формы представления (кодирования) этой информации}
			\end{scnindent}
			\scnitem{семантическая унификация используемой информации}
			\begin{scnindent}
				\scnidtf{согласование и точная спецификация всех (!) используемых понятий 
					(концептов) с помощью иерархической системы формальных онтологий}
			\end{scnindent}
		\end{scnrelfromset}
		\scntext{примечание}{Важно отметить, что грамотная унификация (стандартизация) должна не ограничивать творческую свободу 
		разработчика, а гарантировать \textbf{совместимость} его результатов с результатами других разработчиков.}

		\scnheader{следует отличать*}
		\begin{scnhaselementset}
			\scnitem{внутреннее представление информации}
			\begin{scnindent}
				\scnidtf{кодирование информации в памяти компьютерной системы}
			\end{scnindent}
			\scnitem{внешнее представление информации}
			\begin{scnindent}
				\scnidtf{обеспечение однозначности интерпретации (понимания, трактовки) этой информации
					разными пользователями и разными компьютерными системами}
			\end{scnindent}
		\end{scnhaselementset}

		\scnheader{стандарт}
		\scntext{примечание}{Подчеркнем, что текущая версия любого \textbf{стандарта} — это не догма, а только опора для дальнейшего его совершенствования.}
		\begin{scnrelfromlist}{цель}
			\scnitem{обеспечения совместимости технических решений}
			\scnitem{минимализация дублирования (повторения) решений}
		\end{scnrelfromlist}
		\scntext{критерий качества}{ничего лишнего}
		\scntext{примечание}{Стандарты, как и другие важные для человечества знания, должны быть формализованы и должны постоянно
		совершенствоваться с помощью специальных интеллектуальных компьютерных систем, поддерживающих процесс
		эволюции стандартов путем согласования различных точек зрения.}

		\bigskip\scnheader{смысловое представление информации}
		\scnidtf{смысловая форма представления информации}
		\scnidtf{смысловое представление информационной конструкции}
		\scnidtf{знаковая конструкция (текст), представленная в смысловой
			форме}
		\scnidtf{запись (представление) информационной конструкции на смысловом уровне}
		\scnidtf{информационная конструкция синтаксическая структура которой близка ее смыслу, то есть близка 
			описываемой конфигурации связей между описываемыми сущностями}
		\scnidtftext{часто используемый sc-идентификатор}{смысл}
		\scnidtf{смысловое представление}
		\scnidtf{семантическое представление информации}
		\scntext{основной принцип}{Как можно меньше лишнего, не имеющего
			отношения к смыслу представляемой информации.}
		\scnidtf{такое представление информационной конструкции, которое
			существенно упрощает соответствие между структурой самой этой информационной
			конструкции и описываемой (отображаемой) ею конфигурацией связей между
			рассматриваемыми (исследуемыми) сущностями}
		\scnidtf{смысловое представление знаковой конструкции}
		\scnidtf{абстрактная знаковая конструкция, являющаяся
			\uline{инвариантом} соответствующего максимального класса семантически
			эквивалентных знаковых конструкций}
		\scnidtf{смысл информационной конструкции}
		\scnidtf{денотационная семантика информационной конструкции}
		\scntext{примечание}{Суть (смысл, денотационная семантика) любой
			информационной конструкции (информационной модели) сводится к описанию системы
			(конфигурации) связей между списываемыми (рассматриваемыми) сущностями. Важно,
			чтобы эта суть не была \uline{закамуфлирована} различными синтаксическими
			деталями, не имеющими никакого отношения к указанному смыслу (синтаксическая
			структура знаков, многократное повторение одного и того же знака, синонимия,
			омонимия, местоимения, предлоги, знаки препинания, разделители, ограничители,
			падежи и т.п.) а обусловленными \uline{формой} представления информационных
			конструкций, например, их линейностью.}
		\scntext{пояснение}{Смысловое представление любой информации в
			конечном счете сводится:
			\begin{scnitemize}
				\item к перечню знаков конкретных описываемых сущностей - как первичных
				            сущностей, так и вторичных сущностей, которые сами являются информационными
				            конструкциями (фрагментами данной конструкции);
				\item к явному описанию связи между знаками вторичных сущностей и
				            самими этими сущностями (т.е. фрагментами информационной конструкции);
				\item к описанию других связей между описываемыми сущностями
			\end{scnitemize}
			\vspace{-0.6\baselineskip}}
		\newpage
		\scntext{пояснение}{Формализация смысла представляемой информации,
			т.е. строгое уточнение того, что такое \textit{смысловое представление
				информации}, является объективной основой для \uline{унификации} представления
			информации в \textit{памяти компьютерных систем} и \uline{ключом} к решению
			многих проблем семантической совместимости и эволюции компьютерных систем и
			технологий.
			
			Согласно \textit{Мартынову В. В.} (\cite{Martynov}), <<фактически
			всякая мыслительная деятельность человека (не только научная), как полагают
			многие ученые, использует \uline{внутренний семантический код}, на который
			переводят с естественного языка и с которого переводят на естественный язык.
			Поразительная способность человека к идентификации огромного множества
			структурно различных фраз с одинаковым \textit{смыслом} и способность
			\uline{запомнить смысл вне этих фраз} убеждает нас в этом.>>
			
			Приведем также слова \textit{Мельчука И. А.} (\cite{MelchukST}):
			
			<<Идея была следующая --- язык надо описывать следующим образом: надо уметь записывать смыслы фраз. \uline{Не
			фразы, а их \textit{смыслы}}, что отдельно. Плюс построить систему, которая по
			смыслу строит фразу. Это та область или тот поворот исследований, при котором
			интуиция способного лингвиста работает лучше всего: как выразить на данном
			языке данный смысл. Это --- то, для чего лингвистов учат.
			
			Лингвистический	\textit{смысл} научного текста --- это совсем не то, что ты, читая его, из него
			извлекаешь. Это, очень грубо говоря, инвариант синонимических перифраз. Ты
			можешь один и тот же смысл выразить очень многими способами. Когда ты говоришь,
			то можешь сказать по-разному: \scnqqi{Сейчас я налью тебе вина}, или: \scnqqi{Дай, я тебе
			предложу вина}, или: \scnqqi{Не выпить ли нам по бокалу?}, --- все это имеет один и тот
			же смысл. И вот можно придумать, как записывать этот \textit{смысл}. Именно
			его. Не фразу, а \textit{смысл}. И работать надо от этого \textit{смысла} к
			реальным фразам. Синтаксис там по дороге тоже нужен, но он нужен именно по
			дороге, он не может быть ни конечной целью, ни начальной точкой. Это --
			промежуточное дело.>> (\cite{Melchuk}).}
		\scntext{примечание}{Грамотная унификация (стандартизация) \textit{смыслового
				представления информации} не должна привести к ограничению творческой свободы
			авторов различного вида публикуемых научно-технических знаний (и, в том числе,
			разработчиков \textit{баз знаний}), не должна гарантировать
			\textit{семантическую совместимость} различных \textit{знаний}, представленных
			различными авторами (разумеется, при условии соблюдения соответствующих правил
			построения этих \textit{знаний}). При этом любые \textit{стандарты} (в том
			числе и принятые стандарты \textit{смыслового представления информации}) должны
			постоянно эволюционировать. Текущая версия любого стандарта должна быть не
			догмой, а точкой опоры для дальнейшего совершенствования этого стандарта.}
		\scnsuperset{УСК}
			\begin{scnindent}
				\scnidtf{Универсальный Семантический Код}
				\scnrelfrom{автор}{Мартынов В. В.}
				\scntext{примечание}{Разработанный Мартыновым В. В. Универсальный
					Семантический Код стал важнейшим этапом создания универсальных формальных
					средств смыслового представления знаний. Основная методологическая идея
					\textit{Мартынова В. В.}, касающаяся построения \textit{языка смыслового
					представления знаний}, заключается в том, чтобы выделить смысловые кирпичики ,
					имеющие достаточно общий характер, а многообразие конкретных смыслов
					конструировать комбинаторно за счёт различных комбинаций (конфигураций) из этих
					кирпичей. Это можно назвать принципом минимизации типов атомарных смысловых
					фрагментов.}
				\scnrelto{ключевой знак}{\cite{Martynov1984}}
			\end{scnindent}
		\scnsuperset{семантическая сеть}
			\begin{scnindent}
				\scnsuperset{рафинированная семантическая сеть}	
			\end{scnindent}
		\scntext{примечание}{Уточнение принципов \textbf{смыслового представления информации} основано, во-первых, на четком
			противопоставление \textbf{внутреннего языка компьютерной системы}, используемого для хранения информации в памяти
			компьютера, и \textbf{внешних языков компьютерной системы}, используемых для общения (обмена сообщениями) компьютерной
			системы с пользователями и другими компьютерными системами (смысловое представление используется
			исключительно для \textbf{внутреннего представления} информации в памяти компьютерной системы), и, во-вторых,
			на максимально возможном упрощении синтаксиса внутреннего языка компьютерной системы при обеспечении
			универсальности путем исключения из такого внутреннего универсального языка средств, обеспечивающих
			коммуникационную функцию языка (то есть обмен сообщениями). Так, например, для внутреннего языка компьютерной 
			системы излишними являются такие коммуникационные средства языка, как союзы, предлоги, разделители, ограничители, 
			склонения, спряжения и другие. Внешние языки компьютерной системы могут быть как близки ее внутреннему языку, так и 
			весьма далеки от него (как, например, естественные языки)}

		\scnheader{смысловое представление информации*}
		\scnidtftext{пояснение}{\textit{Бинарное ориентированное отношение},
			каждая \textit{пара} которого связывает некоторую \textit{информационную
				конструкцию} со смысловым представлением этой \textit{информационной
				конструкции*}.}
		\scnsubset{формализация*}
		\bigskip
		
		\scnheader{смысл}
		\scnidtf{\textbf{абстрактная} знаковая конструкция, принадлежащая внутреннему языку компьютерной системы,
			являющаяся \textbf{инвариантом} максимального класса семантически эквивалентных знаковых конструкций (текстов),
			принадлежащих самым разным языкам, и удовлетворяющая следующим требованиям:
			\begin{scnitemize}
				\item{\textbf{универсальность} — возможность представления любой информации;}
				\item{\textbf{отсутствие синонимии знаков} (многократного вхождения знаков с одинаковыми денотатами);}
				\item{\textbf{отсутствие дублирования информации} в виде семантически эквивалентных текстов (не путать с логической
				эквивалентностью);}
				\item{\textbf{отсутствие омонимичных знаков} (в том числе местоимений);}
				\item{\textbf{отсутствие у знаков внутренней структуры} (атомарный характер знаков);}
				\item{\textbf{отсутствие склонений, спряжений} (как следствие отсутствия у знаков внутренней структуры);}
				\item{\textbf{отсутствие фрагментов} знаковой конструкции, не являющихся знаками (разделителей, ограничителей, и
				так далее);}
				\item{\textbf{выделение знаков связей}, компонентами которых могут быть любые знаки, с которыми знаки связей
				связываются синтаксически задаваемыми отношениями инцидентности.}
			\end{scnitemize}}

		\scnheader{принципы смыслового представления информации в памяти компьютерной системы}
		\scntext{следствие}{Знаки сущностей, входящие в смысловое представление информации, \textbf{не являются именами}
			(терминами) и, следовательно, не привязаны ни к какому естественному языку и не зависят от субъективных
			терминотворческих пристрастий различных авторов. Это значит, что при коллективной разработке смыслового
			представления каких-либо информационных ресурсов терминологические споры исключены.}
		\scntext{следствие}{Эти принципы приводят к нелинейным знаковым конструкциям (к графовым структурам), что усложняет реализацию памяти
			компьютерных систем, но существенно упрощает ее логическую организацию (в частности, ассоциативный доступ).}

		\scnheader{нелинейность смыслового представления информации}
		\scntext{обусловлено}{
			\item{Каждая описываемая сущность, то есть сущность, имеющая соответствующий ей знак, может иметь неограниченное
				число связей с другими описываемыми сущностями;}
			\item{каждая описываемая сущность в смысловом представлении имеет единственный знак, так как синонимия
				знаков здесь запрещена;}
			\item{все связи между описываемыми сущностями описываются (отражаются, моделируются) связями между знаками
				этих описываемых сущностей.}
		}

		\scnheader{универсальное смысловое представления информации}
		\scntext{примечание}{Суть можно сформулировать в виде следующих положений:
		\begin{scnitemize}
			\item{Смысловая знаковая конструкция трактуется как множество знаков, взаимно-однозначно обозначающих различные
				сущности (денотаты этих знаков) и множество связей между этими знаками;}
			\item{Каждая связь между знаками трактуется, с одной стороны, как множество знаков, связываемых этой связью,
				а, с другой стороны, как описание (отражение, модель) соответствующей связи, которая связывает денотаты
				указанных знаков или денотаты одних знаков непосредственно с другими знаками, или сами эти знаки. Примером
				первого вида связи между знаками является связь между знаками материальных сущностей, одна из
				которых является частью другой. Примером второго вида связи между знаками является связь между знаком
				множества знаков и одним из знаком, принадлежащих этому множеству, а также связь между знаком и знаком
				файла, являющегося электронным отражением структуры представления указанного знака во внешних знаковых
				конструкциях. Примерами третьего вида связи между знаками является связь между синонимичными знаками;}
			\item{Денотатами знаков могут быть (1) не только конкретные (константные, фиксированные), но и произвольные
				(переменные, нефиксированные) сущности, \scnqq{пробегающие} различные множества знаков (возможных значений),
				(2) не только реальные (материальные), но и абстрактные сущности (например, числа, точки различных
				абстрактных пространств), (3) не только \scnqq{внешние}, но и \scnqq{внутренние} сущности, являющиеся множествами
				знаков, входящих в состав той же самой знаковой конструкции.}
		\end{scnitemize}}
			
		\scnheader{формализация*}
		\scniselement{бинарное ориентированное отношение}
		\scnidtf{формализация информации*}
		\scnidtf{пара, связывающая менее формализованное и более
			формализованное представление некоторой информации*}
		\scnidtf{формализация информационной модели некоторой описываемой
			(моделируемой) системы взаимосвязанных сущностей*}
		\scnidtf{Бинарное ориентированное отношение, каждая \textit{пара}
			которого, связывает два \textit{семантически эквивалентных} знания, второе из
			которых является более точным (более точно сформированным) знанием по сравнению
			с первым \textit{знанием}*.}
		\scntext{пояснение}{Повышение точности (строгости) формулировки
			знания --- минимизация (а в идеале --- исключение) \uline{неоднозначной}
			семантической интерпретации этой формулировки, т.е. несоответствия того, что
			хотел сказать  автор формулировки, и того, как его поняли. Формализация знаний
			предполагает (1) точное (строгое) описание \textit{синтаксиса и денотационной
				семантики} того \textit{языка}, на котором формулируются \textit{знания} и (2)
			максимально возможное \uline{упрощение} синтаксических и семантических
			принципов, лежащих в основе указанного \textit{языка}. Очевидно, что
			\textit{естественные языки} указанным требованиям не удовлетворяют и,
			следовательно, не могут быть основой для точной формулировки
			\textit{научно-технических знаний} и, соответственно, для представления этих
			\textit{знаний} в \textit{памяти интеллектуальных компьютерных систем}.
			Очевидно также, что разработка \textit{\uline{универсального} языка}
			формального представления научно-технических знаний является \uline{основой}
			для глубокой конвергенции различных научно-технических дисциплин, для
			расширения областей применения современной математики и даже для появления
			новых разделов математики, которые, например, изучают общие свойства
			\textit{универсального смыслового пространства} и, в частности, свойство
			семантического расстояния(семантической близости) как между различными
			\textit{знаками}, так и между различными \textit{знаковыми конструкциями}
			(конфигурациями знаков).}
			\begin{scnindent}
				\scntext{примечание}{Слово математика  означает точное знание .}
					\begin{scnindent}
						\scnrelto{цитата}{\cite{Arnold2012}}
					\end{scnindent}
			\end{scnindent}
		\scntext{примечание}{Формализация информационной модели есть не что иное как
			движение  в сторону семантического (смыслового) представления этой модель, т.е.
			переход к такому представлению этой модели, в котором мы избавляемся от всего,
			не имеющего отношения к сути моделируемой системы и касающегося только способа
			построения этой модели (т.е. её синтаксической структуры). }
		\scntext{примечание}{Нет проблемы записать любое \textit{знание} в
			компьютерную \textit{память}. Для этого надо придумать соответствующий формат
			их кодирования. Но есть проблема представить это \textit{знание} так, чтобы с
			ним было легко работать, чтобы с использованием этого \textit{знания} можно
			было достаточно удобно (без лишних накладных расходов, обусловленных выбранным
			способом представления) решать самые различные информационные \textit{задачи}
			(задачи интеграции знаний, информационного поиска по базе знаний, верификации и
			оптимизации баз знаний, логического вывода, поиска способов решения задач,
			хранимых в базе знаний и т. д.).Какими характеристиками должно обладать удобное
			представление знаний, удовлетворяющее указанным требованиям. Очевидно, что
			такое представление есть не что иное, как формальная (математическая) модель,
			семантически эквивалентная этим знаниям. Т.е. удобно представить знание --- это
			фактически построить соответствующую этому знанию \textit{математическую
				модель}.Для интеллектуальных компьютерных систем важно не просто приобрести
			знания, но и представить их в такой форме, которая была бы удобна не только для
			человека (пользователя и разработчика), но и для различных компьютерных систем,
			т.е. не требовала бы переоформление (перезаписи) этих знаний для различных
			компьютерных систем. Очевидно, что такая форма записи (представления) знаний
			должна быть абсолютно не зависящий от различных компьютерных платформ.Это и
			есть главная цель формализации знаний, обеспечивающей эффективную автоматизацию
			обработки этих знаний.}

		\scnheader{формальное представление информации}
		\scnsubset{информация}
			\begin{scnindent}
				\scnidtf{информационная конструкция}
			\end{scnindent}
		\scntext{вопрос}{Почему разработка и использование формальных моделей
			(математических моделей) представления \textit{информации} является важнейшим
			этапом развития любой научной и научно-технической дисциплины.}
		\begin{scnindent}
			\begin{scnrelfromset}{ответ}
				\scnfileitem{Формализация любой \textit{предметной области} даёт
					возможность более конструктивно накапливать, интегрировать, понимать и
					систематизировать новые \textit{знания} об этой \textit{предметной области}}
				\scnfileitem{Формализация \textit{предметной области} обеспечивает
					более строгую верификацию, обоснование (аргументацию, доказательство) и
					согласование различных точек зрения}
				\scnfileitem{Формализация \textit{предметной области} создает условия
					для разработки строгих и легко воспроизводимых (реализуемых) \textit{методов}
					решения различных \textit{классов задач}}
			\end{scnrelfromset}
		\end{scnindent}
		\scnrelto{достоинства}{формальное представление информации}
		\scnidtf{формальное (формализованное) представление информационной
			конструкции}
		\scnsubset{смысловое представления информации}
		\scntext{примечание}{Высшим уровнем качества \textit{формального
				представления информации} является смысловое представление этой информации}
		\scnidtf{формальная модель системы описываемых взаимосвязанных
			сущностей}
		\scnidtf{математическая модель системы описываемых взаимосвязанных
			сущностей}
		\scnidtf{формула}
		\scntext{примечание}{Сам термин \scnqqi{\textit{формальное представление
				информации}} свидетельствует о том, что при таком представлении
			\textit{информации} сама \uline{форма} представляемой информационной
			конструкции (т.е. синтаксическая структура этой конструкции) имеет очевидную
			аналогию с описываемой конфигурацией связей между соответствующими
			соответствующими описываемыми \textit{сущностями}.В предельном идеальном
			случае указанная аналогия между формой и смыслом информационной конструкции
			должна быть изоморфизмом.}
		\scntext{примечание}{Формализация формализации рознь и, соответственно,
			степень приближения формы представления информации к идеальному  смысловому
			представлению может быть различной. Разработка такого идеального  \textit{языка
				смыслового представления информации} должна руководствоваться следующими
			основными критериями:
			\begin{scnitemize}
				\item максимально возможное упрощения синтаксиса (как можно меньше
				синтаксических излишеств и синтаксического разнообразия);
				\item обеспечение \uline{универсальности} языка.
			\end{scnitemize}
			Подчеркнем, что обеспечение универсальности \textit{языка смыслового
				представления информации} является весьма нетривиальной задачей, т.к. сложно
			одновременно достигнуть две противоречащие друг другу цели- обеспечить простоту
			синтаксиса языка и его неограниченную семантическую мощность. Косвенным
			подтверждением этого является большое количество созданных человечеством
			специализированных \textit{формальных языков}, \textit{языков смыслового
				представления информации} и даже \textit{языков семантических сетей}, что
			свидетельствует о востребованности \textit{смыслового представления
				информации}.}
		\begin{scnsubdividing}
			\scnitem{формальное представление информации, не являющееся смысловым}
			\scnitem{смысловое представление информации, не являющееся
				семантической сетью}
			\scnitem{нерафинированная семантическая сеть}
				\begin{scnindent}
					\scnidtf{смысловое представления информации 2-го уровня}
				\end{scnindent}
			\scnitem{рафинированная семантическая сеть}
				\begin{scnindent}
					\scnidtf{смысловое представление информации 3-го уровня}
				\end{scnindent}
		\end{scnsubdividing}

		\bigskip
		\scnheader{язык смыслового представления информации}
		\begin{scnrelfromlist}{ключевое свойство}
			\scnitem{однозначность представления информации в памяти каждой компьютерной системы}
			\begin{scnindent}
				\scnidtf{отсутствие семантически эквивалентных знаковых	конструкций, принадлежащих 
					смысловому языку и хранимых в одной смысловой памяти}
				\scntext{примечание}{При этом логическая эквивалентность таких знаковых конструкций 
					допускается и используются, например, для компактного представления некоторых знаний,
					хранимых в смысловой памяти.}
			\end{scnindent}
		\end{scnrelfromlist}

		\scnheader{смысловое представление информации, не являющееся семантической сетью}
		\scntext{примечание}{Данному уровню смыслового представления информации
			соответствуют предлагаемые нами универсальные формальные языки SCs-код и SCn-код}
		\scnsuperset{SCs-код}
			\begin{scnindent}
				\scniselement{универсальный формальный язык}
				\scniselementrole{ключевой знак}{Описание языка линейного представления знаний ostis-систем}
			\end{scnindent}
		\scnsuperset{SCn-код}
		\begin{scnindent}
			\scniselement{универсальный формальный язык}
			\scniselementrole{ключевой знак}{Описание языка структурированного представления знаний ostis-систем}
		\end{scnindent}
		\begin{scnreltovector}{принципы, лежащие в основе}
			\scnfileitem{В состав \textit{смыслового представления информации, не
				являющегося семантической сетью} могут входить все уровни иерархии
				представления информационной конструкции --
				\begin{scnitemize}
					\item синтаксически элементарные фрагменты информационной конструкции,
					из которых строятся простые знаки описываемых сущностей, а также разделители и
					ограничители
					\item простые знаки
					\item выражения
					\item простые тексты
					\item сложные тексты
					\item простые знания
					\item сложные знания.
				\end{scnitemize}}

			\scnfileitem{Множество всех описываемых сущностей, \uline{не являющихся
					связями}, разбивается на два подмножества:
				\begin{scnitemize}
					\item каждой сущности, принадлежащей первому подмножеству,
					\uline{взаимно однозначно} соответствует множество \uline{синтаксически
						эквивалентных} (синтаксически одинаковых) \textit{простых знаков}, каждый из
					которых обозначает указанную сущность
					\item каждой сущности, принадлежащей второму подмножеству,
					соответствует в общем случае \uline{семейство} множеств, кажо из которых
					является максимальным множеством синтаксически эквивалентных выражений,
					обозначающих указанную сущность.
				\end{scnitemize}}
				\begin{scnindent}
					\scntext{следовательно}{Здесь синонимия \textit{простых знаков},
						имеющих \uline{разную} синтаксическую структуру, отсутствует, а вот синонимия
						\textit{выражений}, имеющих разную синтаксическую структуру, вполне возможна.
						Подчеркнем при этом, что в рамках \textit{смыслового представления информации,
							не являющегося семантической сетью},
						% \bigspace
						\textit{знаки} (как \textit{простые знаки}, так и \textit{выражения}),
						имеющие одинаковую синтаксическую структуру, считаются также и семантически
						эквивалентными, т.е. обозначающими одну и ту же сущность. Это означает
						отсутствие омонимии синтаксически эквивалентных знаков.}
					\scntext{следовательно}{В рамках \textit{смыслового представления
							информации, не являющегося семантической сетью}, простые знаки, обозначающие
						\uline{разные} сущности, имеют легко устанавливаемое отличие своих
						синтаксических структур, а простые знаки, обозначающие одну и ту же сущность
						имеют легко устанавливаемое сходство своих синтаксических структур. Таким
						образом, в рамках \textit{смыслового представления информации, не являющегося
							семантической сетью},
						% \bigspace
						\uline{дублирование знаков}, т.е. многократное вхождение
						\textit{знаков} одной и той же сущности, \uline{допускается}.}
				\end{scnindent}
			\scnfileitem{Связи как вид описываемых сущностей имеют очень важные
				особенности:
				\begin{scnitemize}
					\item каждой описываемой \textit{связи} \uline{однозначно}, а в
					подавляющем числе случаев и \uline{взаимно однозначно} соответствует
					\textit{простой текст}, являющийся контекстом (спецификацией) этой
					\textit{связи}
					\item весьма редки \textit{кратные связи}, т.е. \textit{свзяи},
					принадлежащие одному и тому же \textit{отношению} и связывающие одинаковым
					образом одни и те же \textit{сущности}
					\item довольно редко \textit{связи} являются компонентами других
					\textit{связей}.
				\end{scnitemize}}
				\begin{scnindent}
					\scntext{следовательно}{Для подавляющего числа описываемых
						\textit{связей} нет никакой необходимости вводить обозначающие их
						\textit{знаки}, если эти \textit{связи} описываются соответствующими
						\textit{простыми текстами}. Вместо таких \textit{знаков} можно ввести условные
						представления этих \textit{связей}, отражающие их вид и направленность. Такие
						условные представления (изображения) описываемых \textit{связей} можно считать
						\textit{знаками}, но \textit{знаками}, семантические свойства которых
						принципиально отличаются от тех \textit{знаков} описываемых \textit{сущностей},
						которые мы рассматривали выше. Любые данного вида разные \textit{знаки}
						описываемых \textit{связей} даже, если, они являются \textit{синтаксически
							эквивалентными}, т.е. имеют одинаковую структуру, считаются \textit{знаками}
						\uline{разных} описываемых \textit{связей}. Синонимия таких \textit{знаков}
						принципиально возможна, но только в том случае, если \textit{простые тексты},
						описывающие соответствующие \textit{связи}, будут полностью
						\uline{продублированы}.}
				\end{scnindent}
			\scnfileitem{Для описания связей между описываемыми сущностями в
				смысловом представлении информации нет необходимости использовать такие приемы
				естественных языков, как склонения, спряжения, семантическая значимость
				последовательности знаков.}
			\scnfileitem{В случае, если с помощью \textit{простых текстов}
				необходимо описать контекст (спецификацию) нескольких \uline{кратных}
				\textit{связей}, все эти \textit{связи} необходимо обозначить \textit{знаками}
				первого типа --- знаками, \textit{синтаксическая структура} каждого из которых
				\uline{уникальна.}Кроме этого, необходимо ввести знак, который обозначает
				\textit{связь инцидентности} между описываемой \textit{связью} и компонентом
				этой \textit{связи}, и который относится к числу \textit{знаков} второго типа
				-- \textit{знаков}, разные экземпляры (разные вхождения) которого считаются
				обозначениями \uline{разных} \textit{связей}}
			\scnfileitem{Для явного указания синонимии двух разных \textit{знаков}
				первого типа, имеющих разную \textit{синтаксическую структуру}, вводится
				фиктивная \textit{связь равенства}, которая сама не является описываемой
				\textit{связью}, а только указывает факт синонимии двух \textit{знаков}, по
				крайней мере один из которых должен быть \textit{выражением}.}
			\scnfileitem{Каждая описываемая \textit{сущность} должна быть
				специфицирована путем указания типа этой \textit{сущности}. Описываемая
				\textit{сущность} может быть:
				\begin{scnitemize}
					\item \textit{материальной сущностью}
					\item \textit{точкой абстрактного пространства}
					\item \textit{множеством}:
						\begin{scnitemizeii}
							\item \textit{связью}
							\item \textit{классом}
							\item \textit{структурой}
						\end{scnitemizeii}
					\item \textit{реальной сущностью}
					\item \textit{вымышленной сущностью}
					\item \textit{константой}
					\item \textit{переменной}
					\item \textit{постоянной сущностью}
					\item \textit{временной сущностью}:
						\begin{scnitemizeii}
							\item \textit{прошлой сущностью}
							\item \textit{настоящей сущностью}
							\item \textit{будующей сущностью}.
						\end{scnitemizeii}
				\end{scnitemize}
				Кроме того, сам \textit{знак} описываемой сущности может иметь
				следующий статус:
				\begin{scnitemize}
					\item \textit{логически удаленный знак}
					\item \textit{настоящий знак}
					\item \textit{будущий знак}.
				\end{scnitemize}
			}
			\scnfileitem{Возможно дублирование информации, т.е. могут
				присутствовать семантически эквивалентные информационные конструкции, входящие
				в остав одной информационной конструкции (например, в состав информации,
				хранимой в памяти одной компьютерной системы). Но при этом есть принципиальная
				возможность обнаружить такое дублирование информации.}
		\end{scnreltovector}

		\bigskip
		\scnheader{графовая структура}
		\scnidtftext{определение}{абстрактная (математическая) структура, которая задается:
			\begin{scnitemize}
				\item множеством ее элементов:
				            \begin{scnitemizeii}
					            \item множеством ее вершин (узлов);
					            \item множеством ее связок:
					                        \begin{scnitemizeiii}
						                        \item множеством ее ребер (неориентированных пар
						                        элементов графовой структуры);
						                        \item множеством ее дуг (ориентированных пар элементов
						                        графовой структуры);
						                        \item множеством ее гиперребер, каждое из которых
						                        является конечным множеством элементов графовой структуры, имеющим мощность
						                        больше двух
					                        \end{scnitemizeiii}
				            \end{scnitemizeii}
				\item бинарным ориентированным отношением инцидентности, связывающим
				каждую связку графовой структуры с каждым компонентом (элементом) этой связки.
			\end{scnitemize}
		}
		\scnheader{следует отличать*}
		\begin{scnhaselementset}
			\scnfileitem{\textit{графовую структуру} как абстрактный математический
				объект, в рамках которого не уточняется то, как выглядят (представляются,
				изображаются) элементы графовой структуры и связи их инцидентности}
			\scnfileitem{представление (изображение) \textit{графовой структуры} --
				ее рисунок, ее представление в компьютерной памяти в виде матрицы
				инцидентности, матрицы смежности, списковой структуры}
		\end{scnhaselementset}

		\scnheader{графовая структура}
		\scnidtftext{часто используемый sc-идентификатор}{дискретная
			информационная конструкция}
		\scntext{примечание}{Поскольку любая \textit{графовая структура} является
			дискретной математической моделью, которая может описывать любое множество
			\textit{сущностей}, связанных между собой заданным множеством \textit{связей},
			все \textit{графовые структуры} с полным основанием можно считать дискретными
			\textit{информационными конструкциями}. Более того, любая дискретная
			\textit{информационная конструкция} (в том числе, и обычная цепочка символов) с
			формальной точки зрения является \textit{графовой структурой}. Тот факт, что
			теория графов рассматривает синтаксические  свойства \textit{графовых структур}
			с точностью до их изоморфизма, не лишает \textit{графовые структуры}
			соответствующих семантических  свойств.}
		\scntext{пояснение}{С семантической точки зрения графовая структура
			-- это нелинейная (в общем случае) знаковая конструкция, в состав которой могут
			входить знаки \uline{любых} сущностей. При этом указанные знаки
			\uline{синтаксически} разбиваются на два класса --
			\begin{scnitemize}
				\item на \textit{знаки} сущностей, которые не являются \uline{связями}
				между сущностями --- в теории графов такие знаки называются узлами (вершинами);
				\item на знаки \uline{связей} между \textit{сущностями} --- к таким
				\textit{знакам} относятся ребра неориентированных графов, гиперребра
				гиперграфов, дуги ориентированных графов.
			\end{scnitemize}
			Кроме того, на множестве знаков \textit{сущностей}, входящих в состав
			\textit{графовой структуры}, задаются \textit{отношения инцидентности}, которые
			связывают \textit{знаки} связей, входящих  в состав \textit{графовой
			структуры}, со знаками тех \textit{сущностей} которые являются компонентами
			указанных \textit{связей}.
			
			Теория графов рассматривает только синтаксические
			аспекты \textit{графовых структур}.Семантика \textit{графовой структуры}
			задается \textit{онтологией}, специфицирующей систему понятий, экземплярами
			которых являются элементы этой графовой структуры, т.е. \textit{знаки},
			входящие в состав этой \textit{графовой структуры}.}

		\scnheader{семантическая сеть}
		\scnidtf{\textit{графовая структура}, являющаяся \uline{формальным
				уточнением} одного из видов \textit{смыслового представления информации}}
		\scnsubset{графовая структура}
		\scnsubset{смысловое представление информации}
		\begin{scnindent}
			\scnsubset{знаковая структура}
		\end{scnindent}
		\scnidtf{графовая структура, \uline{вершины} (узлы) которой трактуются
			как знаки некоторых описываемых сущностей, а \uline{связки} (ребра, дуги,
			гиперребра, гипердуги) которой трактуются как знаки связей между описываемыми
			сущностями и/или знаками этих сущностей}
		\scnidtf{\uline{абстрактная} графовая и в общем случае нелинейная
			знаковая конструкция (знаковая структура), являющаяся вариантом
			\uline{смыслового} представления соответствующей информации}
		\scnidtftext{explanation}{информационная конструкция, в которой
			\uline{явно} выделены знаки \uline{всех} описываемых сущностей, а также знаки
			связей, которые также считаются описываемыми сущностями и которые связывают
			либо сами описываемые сущности, либо описываемые сущности со знаками других
			описываемых сущностей, либо знаки описываемых сущностей}
		\scntext{примечание}{Теоретико-графовая трактовка (уточнение)
			\textit{смыслового представления информации} является вполне естественной, т.к.
			любая описываемая сущность может иметь неограниченное количество связей с
			другими описываемыми сущностями, и очень часто анализ свойств какой-либо
			описываемой сущности предполагает анализ всех представленных (описанных) связей
			этой сущности с различными другими сущностями. Более того, для любых
			описываемых сущностей существует связывающая их связь (все в Мире
			взаимосвязано). Вопрос в том, какая это связь и нужно ли ее описывать. Далеко
			не все то, что можно описывать, целесообразно описывать.}
		\begin{scnrelfromvector}{общие предпосылки}
			\scnfileitem{Информация в знаковой конструкции содержится не в самих
				знаках, а в конфигурации связей между знаками, обозначающими описываемые
				сущности}
			\scnfileitem{Конфигурация связей между описываемыми сущностями \uline{в
					общем случае} \uline{не} являются линейной}
			\scnfileitem{Идеальным \textit{смысловым представлением информации}
				следует считать такую знаковую конструкцию, синтаксическая конфигурация связей
				между знаками которой \uline{изоморфна} конфигурации связей между описываемыми
				сущностями}
		\end{scnrelfromvector}

		\scntext{примечание}{Понятие семантической сети является основным понятием
			для \textit{Технологии OSTIS}. Ранее семантические сети рассматривались не как
			основа технологии разработки интеллектуальных компьютерных систем, а как
			наглядная иллюстрация представления знаний, не имеющая практической перспективы
			из-за сложности реализации, не обладающая универсализмом.Для нас семантические
			сети --- это
			\begin{scnitemize}
				\item формальный подход к построению знаковых конструкций;
				\item формальный подход, позволяющий создавать целое \uline{семейство}
				языков и в том числе языков \uline{универсальных};
				\item основа организации памяти нового типа --- структурно
				перестраиваемой (реконфигурируемой) памяти, обработка информации в которой
				сводится к реконфигурации связей между ее элементами.
			\end{scnitemize}}
		\begin{scnrelfromlist}{достоинства}
			\scnfileitem{\textit{Семантическая сеть} наряду с системами правил
				является весьма распространенным способом представления знаний в
				интеллектуальных системах. Особое значение этот способ представления знаний
				приобретает в связи с развитием сети интернет. Кроме ряда особенностей,
				позволяющих применять семантические сети в тех случаях, когда системы правил не
				применимы, \textit{семантические сети} обладают следующим важным свойством: они
				дают возможность \uline{соединения в одном представлении синтаксиса и
					семантики} или синтаксического и семантического аспектов описаний знаний
				предметной области. Происходит это благодаря тому, что в семантических сетях
				наряду с переменными для обозначения тех или иных объектов (элементов множеств,
				некоторых конструкций из них) присутствуют и сами эти элементы и конструкции присутствуют и связи, сопоставляющие тем или иным переменным
				множества допустимых интерпретаций. Эти обстоятельства позволяют во многих
				случаях резко \uline{уменьшить реальную вычислительную сложность решаемых
					задач}.
				\newline Помимо изобразительных возможностей, семантические сети
				обладают более серьезными достоинствами. То обстоятельство, что \uline{вся
					информация об индивиде} представлена в единственном месте --- в одной вершине --
				означает, что вся эта информация непосредственно доступна в этой вершине, что,
				в свою очередь, \uline{сокращает время поиска}, в частности, при выполнении
				унификации и подстановки в задачах логического вывода.Существует еще одна,
				более тонкая особенность расширенных семантических сетей --- они позволяют
				интегрировать в одном представлении \textit{синтаксис} и \textit{семантику}
				(т.е. интерпретацию) клаузальных форм. Это позволяет в процессе вывода
				обеспечивать взаимодействие синтаксических и семантических, теоретико-модельных
				подходов, что, в свою очередь, также является фактором, зачастую делающим вывод
				практически более эффективным}
				\begin{scnindent}
				\scnrelto{цитата}{Осипов.Г.С.МетодыИИ-2015кн,с.43-54}
					\begin{scnindent}	
						\scnrelto{часть}{\cite{Osipov2015}}
					\end{scnindent}
				\end{scnindent}
			\scnfileitem{Все связи между \textit{знаками}, входящими в состав
				\textit{семантической сети} представляются с помощью специальных связующих
				элементов \textit{семантической сети} (дуг, ребер) и, следовательно, для
				описания указанных связей в \textit{семантической сети} нет необходимости
				использовать такие средства, как предлоги, союзы, падежи, склонения, спряжения,
				различные разделители и ограничители, что существенно упрощает обработку
				\textit{знаний}.}
			\scnfileitem{Соединение синтаксических и семантических аспектов в
				\textit{семантической сети} проявляется в том, что дуга или ребро,
				синтаксически  соединяющая элементы \textit{семантической сети} описывает
				наличие соответствующей \textit{связи} между \textit{сущностями}, обозначаемыми
				указанными элементами \textit{семантической сети}.}
		\end{scnrelfromlist}

		\bigskip
		\scnheader{нерафинированная семантическая сеть}
		\scntext{примечание}{Переход от смыслового представления информации, не
			являющегося семантической сетью, к нерафинированным семантическим сетям
			представляет собой переход к информационным конструкциям, имеющим более простую
			синтаксическую структуру и денотационную семантику. К нерафинированным
			семантическим сетям можно отнести тексты предлагаемого нами универсального
			формального SCg-кода, а также используемые в Semantic Web
			rdf-графы}	
		\scnsuperset{SCg-код}
			\begin{scnindent}
				\scnhaselement{универсальный формальный язык}
				\scnhaselementrole{ключевой знак}{Описание языка графического представления знаний ostis-систем}
			\end{scnindent}
		\scnsuperset{rdf-граф}
		\begin{scnreltovector}{принципы, лежащие в основе}
			\scnfileitem{Поскольку в \textit{информационной конструкции} информация
				содержится не в самих \textit{знаках} (если не считать \textit{знаки},
				являющиеся \textit{выражениями}), а в конфигурации связей между
				\textit{знаками}, очень важно \uline{явно} формально представить саму эту
				конфигурацию \textit{знаков}. И как нельзя лучше для этого подходит понятие
				\textit{графовой структуры} и, соответственно, понятие \textit{семантической
					сети}.\\Что касается \textit{выражений}, то каждое из них легко
				трансформируется в \textit{семантически эквивалентную} информационную
				конструкцию, не являющиюся \textit{выражением}. Заметим, что \textit{выражения}
				используются исключительно для минимизации числа вводимых \textit{знаков}
				(имен) с уникальной синтаксической структурой.}
			\scnfileitem{\uline{Все} элементы, входящие в состав нерафинированной
				семантической сети и представленные узлами, ребрами или дугами, являются
				\textit{знаками}, обозначающими соответствующие описываемые \textit{сущности},
				причём \textit{знаками} второго типа, которые, обозначая соответствующую
				\textit{сущность}, входят в \textit{информационную конструкцию}
				\uline{однократно} (отсутствует многократное вхождение \textit{знаков},
				обозначающих одну и ту же \textit{сущность}). Также \textit{знаки} могут иметь
				синтаксическую структуру, которая не является уникальной для обозначаемой
				\textit{сущности}, а отражает только принадлежность этой сущности к
				соответствующих классам.Таким образом, в \textit{нерафинированной семантической
					сети} в отличие от \textit{смыслового представления информации не являющегося
					семантической сетью}, доминируют не \textit{знаки} первого типа, а
				\textit{знаки} второго типа, которыми в \textit{нерафинированной семантической
					сети} представлены (обозначены) \uline{все} описываемые \textit{сущности}, а в
				\textit{смысловом представлении информации, не являющемся семанитеской сетью},
				представлены \uline{только} \textit{бинарные связи} \uline{и то не все}.}
			\scnfileitem{\uline{Все} ребра \textit{нерафинированной семантической
					сети} являются знаками \textit{бинарных неориентированных связей} и формально
				трактуются как знаки \textit{двухмощных множеств}, каждым \textit{элементом}
				которых являются либо знак \textit{сущности}, соединяемой указанной
				\textit{бинарной связью}, либо \textit{знак}, который сам является
				\textit{сущностью}, соединяемой этой \textit{бинарной связью}. Более того,
				\uline{все} \textit{двухмощные множества}, не являющиеся \textit{кортежами}
				(ориентированными парами) в \textit{нерафинированной семантической сети}
				обозначаются \textit{ребрами} этой сети.}
			\scnfileitem{\uline{Все} дуги \textit{нерафинированной семантической
					сети} являются знаками \textit{бинарных ориентированных связей} и формально
				трактуются как знаки \textit{двухмощных кортежей} (ориентированных пар), каждым
				\textit{элементом} которых является либо знак \textit{сущности}, соединяемой
				указанной \textit{бинарной связи}, либо \textit{знак}, который сам является
				\textit{сущностью}, соединяемой этой \textit{бинарной связью}. Более того,
				\uline{все} \textit{ориентированные пары} в \textit{нерафинированной
					семантической сети} обозначаются \textit{дугами} этой сети.}
			\scnfileitem{\uline{Каждая} небинарная связь, описываемая в
				нерафинированной семантической сети, трактуется как множество, мощность
				которого не равна двум и обозначается соответствующим узлом этой сети, который
				соединяется дугами, принадлежащими отношению принадлежности со всеми знаками,
				которые либо обозначаются сущности, связывающие рассматриваемой небинарной
				связью, либо сами являются такими сущностями. Для описания ориентированных
				небинарных связей (в частности, небинарных кортежей) выделяется несколько
				подмножеств отношения принадлежности, соответствующих различным ролям элементов
				(компонентов) ориентированных небинарных связей.}
			\scnfileitem{В рамках нерафинированной семантической сети \uline{все}
				рассматриваемые связи между описываемыми сущностями представляются \uline{явно}
				в виде знаков, обозначающих эти связи.}
			\scnfileitem{В рамках нерафинированной семантической сети не
				используются такие средства, как разделители, ограничители и др.}
			\scnfileitem{Узлами \textit{нерафинированной семантической сети},
				которые обозначают различного вида \uline{ключевые} описываемые
				\textit{сущности} (прежде всего, различные \textit{понятия}) приписываются
				уникальные \textit{знаки} (имена) этих \textit{ключевых сущностей}. Очевидно,
				что каждый такой \textit{узел} и приписываемое ему \textit{имя} --- это
				\textit{синонимичные знаки}, обозначающие одну и ту же \textit{сущность}, но
				являющиеся \textit{знаками} двух разных типов --- (1) \textit{знаком}, который
				\uline{однократно} представлен в рамках \textit{информационной конструкции}
				(2) \textit{знаком}, синтаксическая структура которого \uline{взаимно
					однозначно} соответствует обозначаемой им \textit{сущности}.}
			\scnfileitem{Большинству узлов, обозначающих небинарные связи,
				большинству ребер и дуг, а также некоторым другим узлам нерафинированной
				семантической сети могут быть приписаны уникальные знаки (в частности, имена)
				понятий (чаще всего, отношений), которым принадлежат указанные узлы, ребра и
				дуги.}
		\end{scnreltovector}

		\bigskip
		\scnheader{рафинированная семантическая сеть}
		\scntext{основной принцип}{Абсолютно ничего лишнего, не имеющего
			отношения к смыслу представляемой информации}
		\scnidtf{\uline{предельно} компактная (сжатая) смысловая информационная
			модель соответствующей системы рассматриваемых (описываемых, исследуемых,
			моделируемых) сущностей}
		\scntext{примечание}{Указанная система рассматриваемых сущностей представляет
			собой конфигурацию связей между этими сущностями. Подчеркнем при этом, что
			указанные связи между рассматриваемыми сущностями также входят в число
			рассматриваемых сущностей.}\scnidtf{\textit{информационная конструкция},
			являющаяся результатом максимально возможного упрощения ее
			\textit{синтаксической структуры} при обеспечении представления \uline{любой}
			\textit{информации}, что приводит к фактическому слиянию синтаксических и
			семантических аспектов представления \textit{информации}}
		\scnidtf{\textit{семантическая сеть} внутреннего\ потребления,
			используемая для \textit{смыслового представления информации} в памяти
			\textit{компьютерных систем}}
		\scnidtf{уточнение принципов \textit{смыслового представления
				информации}, которое основано, \uline{во-первых}, на четком противопоставлении
			\textit{внутреннего языка компьютерной системы}, используемого для хранения
			информации в памяти компьютера, и \textit{внешних языков компьютерной системы},
			используемых для общения (обмена сообщений) \textit{компьютерной системы} с
			пользователями и другими \textit{компьютерными системами} (рафинированная
			семантическая сеть используется исключительно для \textit{внутреннего
				представления информации} в памяти \textit{компьютерной системы}), и,
			\uline{во-вторых} на максимально возможном упрощении \textit{синтаксиса
				внутреннего языка компьютерной системы} при обеспечении \uline{универсальности}
			путем исключения из такого внутреннего универсального языка средств,
			обеспечивающих коммуникационную функцию \textit{языка} (т.е. обмен
			сообщениями).
			\newline Так, например, для \textit{внутреннего языка компьютерной
				системы} излишними являются такие коммуникационные средства \textit{языка}, как
			союзы, предлоги, разделители, ограничители, склонения, спряжения и другие.
			\newline\textit{Внешние языки компьютерной системы} могут быть как
			близки ее внутреннему языку, так и весьма далеки от него (как, например,
			\textit{естественные языки}).}
		\scnidtf{\uline{абстрактная} знаковая конструкция, принадлежащая
			\uline{универсальному} внутреннему языку компьютерных систем и являющаяся
			\uline{инвариантом} соответствующего максимального множества семантически
			эквивалентных знаковых конструкций (текстов), принадлежащих самым различным
			языкам}

		\begin{scnrelfromvector}{принципы лежащие в основе}
			\scnfileitem{Каждый фрагмент \textit{рафинированной семантической сети}
				является либо \textit{знаком} (элементарным фрагментом, представленным либо
				\textit{узлом}, либо \textit{ребром}, либо \textit{дугой}), либо множеством
				\textit{знаков}, связанных между собой отношением \textit{инцидентности}
				элементов \textit{рафинированной семантической сети}. Указанное отношение
				\textit{инцидентности} является \textit{бинарным ориентированным отношением},
				связывающим \textit{знаки} описываемых \textit{связей} (которые представляются
				\textit{ребрами}, \textit{дугами} и \textit{узлами}, если описываемая связь
				является небинарной) со \textit{знаками}, которые либо обозначают связываемые
				\textit{сущности}, либо сами являются такими сущностями}
				\begin{scnindent} 
					\scntext{следовательно}{В состав \textit{рафинированной
							семантической сети} не входят такие средства синтаксической структуризации
						знаковых конструкций, как \textit{разделители} и \textit{ограничители}. Любая
						структуризация \textit{рафинированных семантических сетей} описывается явно с
						помощью метаязыковых средств путем:
						\begin{scnitemize}
							\item введения узлов \textit{рафинированной семантической
							сети}, обозначающих различные \uline{не\-э\-ле\-мен\-тар\-ные} фрагменты этой
							семантической сети, являющиеся \textit{множествами} узлов, ребер и дуг,
							входящих в состав обозначаемого фрагмента
							\item введения \textit{дуг принадлежности}, связывающих
							введенные \textit{узлы}, обозначающие неэлементарные фрагменты
							\textit{рафинированной семантической сети}, с элементами обозначаемых ими \textit{множеств}
							\item введения целого ряда \textit{отношений}, связывающих
							неэлементарные фрагменты \textit{рафинированной семантической сети} с другими
							фрагментами, а также с сущностями других видов
							\item введения различных классов неэлементарных фрагментов
							\textit{рафинированной семантической сети}.
						\end{scnitemize}}
				\end{scnindent}
			\scnfileitem{Абсолютно все \textit{знаки}, входящие в состав
				\textit{рафинированной семантической сети}, являются синтаксически
				элементарными (атомарными) фрагментами \textit{рафинированной семантической
					сети}, т.е. фрагментами, внутренняя\ структура которых не имеет никакого
				значения для семантического анализа и понимания \textit{рафинированной
					семантической сети}. Множеству \textit{знаков}, входящих в
				\textit{рафинированную семантическую сеть}, как и множеству \textit{букв},
				входящих в обычный \textit{текст}, ставится в соответствие \textit{алфавит},
				определяющий \uline{синтаксическую типологию} таких элементарных фрагментов
				\textit{рафинированной семантической сети}. При этом, если \textit{алфавит}
				букв обычного \textit{текста} не имеет никакой семантической интерпретации, то
				\textit{алфавит} элементарных фрагментов \textit{рафинированной семантической
					сети} имеет четкую семантическую интерпретацию --- каждый элемент этого
				\textit{алфавита} обозначает класс знаков \textit{сущности},
				\uline{синтаксический тип} которых соответствует указанному элементу
				\textit{алфавита} (задается этим элементом \textit{алфавита знаков}, входящих в
				состав \textit{рафинированной семантической сети}).}
				\begin{scnindent}
					\scntext{следовательно}{Таким образом, \textit{знаки}, входящие
						в \textit{рафинированную семантическую сеть}, не являются \textit{именами}
						(терминами) и, следовательно, не привязаны ни к какому \textit{естественному
							языку} и не зависят от субъективных терминотворческих пристрастий различных
						авторов. Это значит, что при коллективной разработке \textit{рафинированных
							семантических сетей}, соответствующих каким-либо информационным ресурсам,
						терминологические споры практически исключены.}
					\scntext{следовательно}{В \textit{рафинированной семантической
							сети} нет необходимости использовать синтаксически элементарные фрагменты,
						\uline{не} являющиеся знаками описываемых \textit{сущностей}, т.е. фрагменты
						\textit{информационной конструкции}, из которых сторятся \textit{простые
							знаки}, \textit{выражения}, а также различные разделители и ограничители. Более
						того, в \textit{рафинированной семантической сети} нет необходимости
						противопоставлять \textit{простые знаки} и \textit{выражения}. Как
						\textit{простым знакам}, так и \textit{выражениям} в \textit{рафинированной
							семантической сети} соответствуют элементы этой сети, имеющие аналогичные
						\textit{денотаты}. Но при этом \textit{выражениям} дополнительно соответствуют
						семантически эквивалентные неэлементарные фрагменты \textit{рафинированной
							семантической сети}, которые специфицируют \textit{сущности}, обозначаемые
						этими \textit{выражениями}.}
				\end{scnindent}
			\scnfileitem{Абсолютно ве описываемые \textit{связи} между описываемыми
				сущностями в \textit{рафинированной семантической сети} представляются
				\uline{явно} в виде соответствующих \textit{знаков}, обозначающих эти
				\textit{связи} и инцидентных знакам связываемых \textit{сущностей}. Для
				бинарных связей, связывающих \uline{две} описываемые сущности, \textit{знаком}
				связей являются \textit{ребра} или \textit{дуги} \textit{рафинированной
					семантической сети}.}
				\begin{scnindent}
					\scntext{следовательно}{В \textit{рафинированных семантических
							сетях} нет необходимости использовать такие средства, как склонения, спряжения,
						род (мужской, женский, средний), семантически интерпретируемая
						последовательность слов.}
				\end{scnindent}
			\scnfileitem{Все \textit{знаки}, входящие в состав
				\textit{рафинированной семантической сети}, входят в нее \uline{однократно}.
				Т.е. в рамках \textit{рафинированной семантической сети} отсутствуют пары
				\textit{синонимичных знаков}, т.е. \textit{знаков}, имеющих один и тот же
				\textit{денотат}. Таким образом, разные элементы \textit{рафинированной
					семантической сети} априори считаются знаками \uline{разных} сущностей. При
				этом эти знаки могут принадлежать одному и тому же синтаксическому типу, т.е.
				одному и тому же элементу алфавита соответствующего языка
				\textit{рафинированных семантических сетей}. Таким образом, в
				\textit{рафинированных семантических сетях} отсутствует синонимия не только
				\textit{знаков}, имеющих одинаковую синтаксическую структуру, не только знаков,
				имеющих одинаковый синтаксический тип, но также и просто \uline{разных} знаков.}
				\begin{scnindent}
					\scntext{следовательно}{Появление в рафинированной
						семантической сети синонимичных знаков превращает эту семантическую сеть в
						некорректную и требует отождествления (склеивания) обнаруженных синонимичных
						знаков.}
				\end{scnindent}
			\scnfileitem{В рамках \textit{рафинированной семантической сети}
				отсутствуют \textit{синонимичные знаки}, т.е. \textit{знаки}, которые имеют не
				один, а несколько \textit{денотатов}, каждому из которых соответствует свой
				контекст (ракурс) семантической трактовки этого \textit{знака}.}
				\begin{scnindent}
					\scntext{примечание}{Когда речь идет об омонимии знаков в привычных
						нам языках, имеется в виду омонимия \uline{разных} знаков, имеющих одинаковую
						синтаксическую структуру, т.е. омонимия разных вхождений, разных экземпляров
						\uline{синтаксически эквивалентных}, но семантически различных знаков.
						Очевидным примером такого рода омонимии являются различного вида местоимения.}
				\end{scnindent}
			\scnfileitem{В рамках каждой \textit{рафинированной семантической сети}
				отсутствует дублирование информации не только в виде многократного вхождения
				\textit{синонимичных знаков}, т.е. \textit{знаков} с одинаковыми денотатами, но
				также и в виде многократного вхождения \textit{семантически эквивалентных}
				\textit{рафинированных семантических сетей}. Две \textit{рафинированные
					семантические сети} являются \textit{семантически эквивалентными} в том и
				только в том случае, если:
				\begin{scnitemize}
					\item они \textit{изоморфны}
					\item пары соответствия указанного \textit{изоморфизма}
					связывают \textit{синонимичные знаки}.
				\end{scnitemize}
				Таким образом, полное исключение \textit{омонимии знаков}
				является необходимым и достаточным условием исключения \textit{семантически
					эквивалентных рафинированных семантических сетей}. Подчеркнем при этом, что
				запрет \textit{семантической эквивалентности} в рамках \textit{рафинированной
					семантической сети} не означает запрета \textit{логической эквивалентности}
				фрагментов \textit{рафинированной семантической сети}. Логическая
				эквивалентность необходима для обеспечения компактности представления некоторых
				знаний. Тем не менее, логической эквивалентностью хранимых в памяти знаковых
				конструкций увлекаться не следует, т.к. \uline{\textit{логически
						эквивалентные}} знаковые конструкции --- это представление одного и того же
				\textit{знания}, но с помощью \uline{\textit{разных наборов понятий}}. В
				отличие от этого \uline{\textit{семантически эквивалентные}} \textit{знаковые
					конструкции} --- это представление одного и того же \textit{знания} с помощь
				одних и тех же \textit{понятий}. Очевидно, что многообразие возможных вариантов
				представления одних и тех же \textit{знаний} в памяти компьютерной системы
				существенно усложняет решение \textit{задач}. Поэтому, полностью исключив
				\textit{семантическую эквивалентность} в смысловой памяти, необходимо
				стремиться к минимизации \textit{логической эквивалентности}. Для этого
				необходимо грамотное построение системы используемых \textit{понятий} в виде
				иерархической системы формальных \textit{онтологий}.}
				\begin{scnindent}
					\scntext{следовательно}{Интеграция (соединение, объединение)
						двух \textit{рафинированных семантических сетей}, в результате чего могут
						появиться семантически эквивалентные фрагменты, сводится к тому, чтобы
						результат такого соединения был приведен в соответствие с требованием
						отсутствия синонимии элементов и семантической эквивалентности фрагментов
						\textit{рафинированной семантической сети}.}
				\end{scnindent}
			\scnfileitem{\textit{Рафинированные семантические сети} должны быть
				\uline{универсальными}, т.е. должны обеспечивать представление \uline{любой}
				информации, в том числе, и \textit{метаинформации}, обеспечивающей описание
				различных связей, свойств и закономерностей самих \textit{рафинированных
					семантических сетей}, на множестве которых, в частности, заданно
				\textit{отношение} быть подструктурой*\, которое связывает
				\textit{рафинированные семантические сети} с их фрагментами (частями), т.е. с
				теми \textit{рафинированными семантическими сетями}, которые входят в их
				состав.\newline Каждая \textit{рафинированная семантическая сеть} трактуется
				как множество \textit{знаков} \uline{взаимно однозначно} соответствующих
				обозначаемым ими \textit{сущностям} (денотатам этих \textit{знаков}) и
				множество \textit{связей} между этими \textit{знаками}.\newline Каждая
				\textit{связь} между \textit{знаками} трактуется, с одной стороны, как
				множество \textit{знаков}, связываемых этой \textit{связью}, а, с другой
				стороны, как описание (отражение, модель) соответствующей \textit{связи},
				которая связывает денотаты указанных \textit{знаков} или денотаты одних
				\textit{знаков} непосредственно с другими \textit{знаками}, или сами эти
				\textit{знаки}. Примером первого вида \textit{связи} между \textit{знаками}
				является связь между \textit{знаками} \textit{материальных сущностей}, одна из
				которых является частью другой. Примером второго вида \textit{связи} между
				\textit{знаками} является \textit{связь} между знаком, входящим в состав
				внутреннего смыслового представления информации, и знаком файла, являющегося
				электронным отражением структуры представления указанного \textit{знака} во
				внешних \textit{знаковых конструкциях}. Примерами третьего вида \textit{связи}
				между \textit{знаками} является \textit{связь} между синонимичными
				знаками.\newline Денотатами \textit{знаков} могут быть \uline{любые}
				описываемые сущности, причем: (1) не только конкретные (константные,
				фиксированные), но и произвольные (переменные, нефиксированные)  сущности,
				пробегающие\ различные множества знаков (возможных значений), (2) не только
				реальные (материальные), но и абстрактные сущности (например, числа, точки
				различных абстрактных пространств), (3) не только внешние\, но и внутренние
				сущности, являющиеся множествами знаков, входящих в состав той же самой
				знаковой конструкции, хранимой в памяти компьютерной системы.}
			\scnfileitem{Поскольку \textit{рафинированные семантические сети}
				ориентированы на \textit{смысловое представление информации} в памяти
				\textit{компьютеров нового поколения}, необходимо, с одной стороны,
				использовать накопленный полезный опыт представления информации в
				\textit{современных компьютерах}, а, с другой стороны, обеспечить
				взаимодействие \textit{компьютерных систем}, построенных на \textit{современных
					компьютерах}, с \textit{компьютерными системами}, построенными на
				\textit{компьютерах нового поколения}. Для этой цели в памяти
				\textit{компьютеров нового поколения} можно и нужно обеспечить обработку и
				хранение различного вида \textit{информационных конструкций}, представленных в
				различных широко используемых форматах. И ничто не препятствует такие
				\textit{информационные конструкции}, хранимые в памяти \textit{компьютера
					нового поколения} и не являющиеся \textit{рафинированными семантическими
					сетями}, рассматривать как \textit{сущности}, описываемые
				\textit{рафинированной семантической сетью}, хранимой в памяти этого
				\textit{компьютера нового поколения}. Такой вид \textit{сущностей}, описываемых
				\textit{рафинированной семантической сетью} и хранимых в той же
				\textit{памяти}, будем называть \textit{файлами}, описываемыми соответствующуей
				\textit{рафинированной семантической сетью}, т.е. электронными	\ образами
				(копиями) соответствующих \textit{информационных конструкций}. Таким образом,
				среди \textit{узлов рафинированной семантической сети} появляются
				\textit{узлы}, являющиеся знаками \textit{файлов}, т.е. \textit{узлы}, денотаты
				(обозначаемые \textit{сущности}) которых находятся (хранятся) в той же памяти,
				что и обозначающие их \textit{узлы}.}
				\begin{scnindent}
					\scntext{следовательно}{Ничто не мешает в виде \textit{файла},
						описываемого \textit{рафинированной семантической сетью}, хранить \textit{имя}
						(термин) какой-либо \textit{сущности}, описываемой этой же семантической сетью,
						а также связать это \textit{имя} (точнее, узел, обозначающий это \textit{имя})
						с тем элементом \textit{рафинированной семантической сети}, который обозначает
						ту же описываемую \textit{сущность}.}
				\end{scnindent}
			\scnfileitem{Следствием указанных принципов \textit{рафинированных
					семантических сетей} является также то, что эти принципы приводят к нелинейным
				\textit{знаковым конструкциям} (к \textit{графовым структурам}), что усложняет
				реализацию \textit{памяти компьютерных систем}, но существенно упрощает ее
				логическую организацию (в частности, ассоциативный доступ).
				\newline Нелинейность \textit{рафинированных семантических
					сетей} обусловлена тем, что:
				\begin{scnitemize}
					\item каждая описываемая \textit{сущность}, т.е.
					\textit{сущность}, имеющая соответствующий ей \textit{знак}, может иметь
					неограниченное число \textit{связей} с другими описываемыми \textit{сущностями}
					\item каждая описываемая \textit{сущность} в смысловом
					представлении имеет единственный \textit{знак}, т.к. синонимия \textit{знаков}
					здесь запрещена
					\item все \textit{связи} между описываемыми \textit{сущностями}
					описываются (отражаются, моделируются) \textit{связями} между \textit{знаками}
					этих описываемых \textit{сущностей}.
				\end{scnitemize}}
				\begin{scnindent}
					\scntext{примечание}{Напомним, что нелинейность информационных
						конструкций характерна не только для рафинированных, но и для нерафинированных
						семантических сетей.}
				\end{scnindent}
		\end{scnrelfromvector}

		\scnsuperset{SC-код}
		\begin{scnindent}
			\scnidtf{Semantic Computer Code}
			\scniselement{универсальный формальный язык}
			\scniselementrole{ключевой знак}{Описание внутреннего языка
				ostis-сиcтем}

			\scntext{пояснение}{В качестве \textit{стандарта}
				\uline{универсального} \textit{смыслового представления информации} \textit{в
					памяти компьютерных систем} нами предложен SC-код (Semantic Computer Code). В
				отличие от УСК \textit{Мартынова В.В.}, он, во-первых, носит нелинейный
				характер и, во-вторых, специально ориентирован на кодирование информации в
				памяти компьютеров \uline{нового поколения}, ориентированных на разработку
				семантически совместимых \textit{интеллектуальных компьютерных систем} и
				названных нами \textit{семантическими ассоциативными компьютерами}. Более
				подробно это понятие (\textit{SC-код}) рассмотрено в разделе \textit{Предметная
					область и онтология внутреннего языка osts-систем}. Таким образом, основым
				лейтмотивом предлагаемого нами \textit{смыслового представления информации}
				является ориентация на формальную модель памяти \textit{компьютерных}
				\uline{не}фон-неймановского \textit{компьютера}, предназначенного для
				реализации \textit{интеллектуальных систем}, использующих \textit{смысловое
					представление информации}. Особенностями такого представления являются
				следующие:
				\begin{scnitemize}
					\item ассоциативность памяти;
					\item поскольку при смысловом представлении информациия содержится в
					конфигурации связей между знаками, переработка информации сводится к
					реконфигурации этих связей (к графодинамическим процессам);
					\item прозрачная семантическая интерпретируемость и, как следствие,
					\textit{семантическая совместимость}.
				\end{scnitemize}
				Подчеркнем что, неявная привязка к фон-неймановским
				\textit{компьютерам} присутствует во всех известных \textit{моделях
					представления знаний}. Одним из примеров такой зависимости, является, например,
				обязательность именования описываемых объектов.}
		\end{scnindent}
			\begin{scnrelfromset}{достоинства}
				\scnfileitem{рафинированная семантическая сеть есть
					\uline{объективный}, не зависящий от субъективизма и многообразия
					синтаксических решений, способ представления информации}
				\scnfileitem{в рамках \textit{рафинированной семантической сети}
					существенно упрощается процедура \textit{интеграции знаний} и погружения новых
					знаний в \textit{базу знаний}}
				\scnfileitem{существенно упрощается процедура приведения различного
					вида \textit{знаний} к общему виду (к согласованной системе используемых
					\textit{понятий})}
				\scnfileitem{существенно упрощается процедура интеграции различных
					\textit{решателей задач} и целых \textit{компьютерных систем}}
				\scnfileitem{существенно упрощается автоматизация перманентного
					процесса \textit{поддержки семантической совместимости} (согласованности
					\textit{понятий} и \textit{онтологий}) для \textit{компьютерных систем} в
					условиях их постоянного совершенствования}
				\scnfileitem{в рамках \textit{рафинированных семантических сетей}
					достаточно легко осуществляется переход от информационных конструкций к
					информационным \uline{мета}конструкциям путем введения узлов
					\textit{семантической сети}, обозначающих \textit{информационные конструкции},
					а также дуг, связывающих эти узлы со всеми элементами обозначаемой ими
					\textit{информационной конструкции}}
				\scnfileitem{на основе \textit{рафинированных семантических сетей}
					существенно упрощается интеграция различных дисциплин в области
					\textit{Искуственного интеллекта}, т.е. построение \textit{Общей формальной
						теории интеллектуальных компьютерных систем}, так как для построения общей
					формальной модели \textit{интеллектуальных компьютерных систем} необходим
					базовый \textit{язык}, в рамках которого можно было бы легко переходить от
					информации (от \textit{знаний}) к \textit{метаинформации} (к метазнаниям, к
					спецификациям исходных \textit{знаний}). Это потверждается тем, что:
					\begin{scnitemize}
						\item подавляющее число \textit{понятий}
						            %\bigspace
						            \textit{Искусственного интеллекта} носит метаязыковой характер
						\item формальное смысловое уточнение почти каждого \textit{понятия}
						            %\bigspace
						            \textit{Искусственного интеллекта} требует предшествующего формального
						            уточнения соответсвующего языка-объекта. Так, например, как можно строго
						            говорить о \textit{языке онтологий} (т.е. \textit{языке} спецификации
						            \textit{предметных областей}), не уточнив \textit{язык} представления самих
						            этих \textit{предметных областей}. как можно строго говорить о \textit{языке}
						            описания способов обработки \textit{информации}, не уточнив \textit{язык
						            }представления самой этой обрабатываемой \textit{информации}.
					\end{scnitemize}}
			\end{scnrelfromset}

		\bigskip
		\scnheader{язык смыслового представления информации}
		\scnidtf{смысловой язык}
		\scnidtf{семантический язык}
		\begin{scnsubdividing}
			\scnitem{язык смыслового представления информации, не являющийся языком
				семантических сетей}
			\scnitem{язык семантических сетей}
		\end{scnsubdividing}

		\scnheader{язык семантических сетей}
		\scntext{пояснение}{Несмотря на то, что синтаксическая структура
			семантической сети во многом носит \uline{объективный} характер, поскольку
			определяется конфигурацией описываемых связей между описываемыми сущностями.
			Тем не менее, можно говорить о разных \textit{языках семантических сетей},
			каждому из которых соответствует свой \textit{алфавит*} элементов
			(синтаксически атомарных фрагментов) \textit{семантических сетей}. При атом
			языки семантических сетей могут быть как специализированными, так и
			универсальными. Задача каждого из этих \textit{языков} --- обеспечить в рамках
			\textit{языка} полное отсутствие многообразия синтаксических форм представления
			одной и той же информации.}
		\scnsubset{язык}
		\begin{scnindent}
			\scnidtf{множество информационных конструкций, для которого существуют,
				причем не обязательно в формализованном виде, (1) правила построения
				синтаксически корректных информационных конструкций, а также (2) правила,
				позволяющие установить семантическую корректность правильно построенных
				(синтаксически корректных) информационных конструкций}
		\end{scnindent}
		\scnidtf{язык, информационными конструкциями которого являются
			семантические сети и в рамках которого обеспечивается полное отсутствие
			многообразия форм представления одной и той же информации}
		\scnidtf{графовый (нелинейный) язык смыслового представления
			информации}
		\begin{scnsubdividing}
				\scnitem{специализированный язык семантических сетей}
				\begin{scnindent}
					\scnidtf{язык семантических сетей, семантическая мощность
						которого ограничена соответствующей предметной областью}
				\end{scnindent}
			\scnitem{универсальный язык емантических сетей}
				\begin{scnindent}
					\scntext{примечание}{Человечество давно и широко использует
						различные специализированные языки семантических сетей --- язык принципиальных
						электрических схем, язык блок-схем программ, язык генеалогических деревьев и
						др. Но в настоящее время актуальным является создание такого
						\textit{универсального языка семантических сетей}
						\begin{scnitemize}
							\item синтаксис и семантика которого были бы максимально просты
							\item по отношению к которому все используемые
							специализированные языки были бы его подъязыками*
							\item который был бы приспособлен к использованию в качестве
							внутреннего языка интеллектуальных компьютерных систем и компьютеров следующего
							поколения
							\item который был бы удобной основой как для обмена информацией
							между интеллектуальными компьютерными системами, так и для общения
							интеллектуальных компьютерных систем с их пользователями.
						\end{scnitemize}
					}
				\end{scnindent}
		\end{scnsubdividing}
		\begin{scnsubdividing}
			\scnitem{язык нерафинированных семантических сетей}
			\scnitem{язык рафинированных семантических сетей}
		\end{scnsubdividing}

		\scnheader{следует отличать*}
		\begin{scnhaselementset}
			\scnitem{язык семантических сетей}
				\begin{scnindent}
					\scnidtf{язык семантических сетей, рассматриваемых как
						\uline{абстрактные} графовые структуры, в которых не уточняется способ их кодирования}
					\scnhaselement{SC-код}
				\end{scnindent}
			\scnitem{графодинамический язык семантических сетей}
				\begin{scnindent}
					\scnidtf{язык графического изображения (визуализации) семантических сетей}
					\scnidtf{язык, текстами которого являются рисунки семантичеких сетей}
					\scnhaselement{SCg-код}
				\end{scnindent}
		\end{scnhaselementset}

		\scnheader{универсальный язык семантических сетей}
		\scntext{примечание}{Если ставить задачу разработки \uline{универсального}(!)
			языка, текстами которого являются графовые структуры, то классических графовых
			структур явно недостаточно. Так, например:
			\begin{scnitemize}
				\item по аналогии с переходом от ребер к ребрам и гиперребрам
				необходим переход от дуг к ориентированным связкам, связывающим более чем два
				компонента и в рамках которых эти компоненты могут иметь разные роли, которые
				необходимо явно указывать (классическим видом таких связок являются кортежи);
				\item в семантических сетях, представляющих некоторые виды
				знаний, некоторые связки (ребра, дуги, гиперребра, ориентированные связки,
				связывающие более двух компонентов) могут быть компонентами других связок;
				\item в семантических сетях, представляющих различного вида
				метазнания необходимо вводить узлы, обозначающие целые фрагменты (подграфы)
				этих же семантических сетей, и, соответственно, вводить дуги, связывающие
				каждый из этих узлов со всеми элементами подграфа, обозначаемого этим узлом.
			\end{scnitemize}}
		\bigskip

		\scnheader{семантическая модель базы знаний}
		\scnidtftext{пояснение}{смысловое представление всей \textit{базы
				знаний} \textit{интеллектуальной компьютерной системы} в виде
			\textit{семантической сети}, принадлежащей \textit{универсальному языку
				семантических сетей}}
		\scntext{пояснение}{Для того, чтобы семантические сети могли быть
			использованы в качестве средства представления \textit{знаний} в памяти
			\textit{интеллектуальной компьютерной системы} необходимо:
			\begin{scnitemize}
				\item рассмотреть \textit{семантические сети} как тексты,
				представляющие \uline{различного вида} \textit{знания};
				\item уточнить синтаксис и семантику \uline{универсального} (!)
				\textit{языка представления знаний}, текстами которого являются
				\textit{семантические сети} \cite{Inform2008} --- стр.195.Считается, что
				\textit{семантические сети} являются теоретической моделью
				\textit{представления знаний}, не используемой на практике. \cite{Inform2008}
				-- стр. 207. Однако, если реализовать \textit{графодинамическую память} и
				разработать \textit{языки программирования}, ориентированные на обработку
				информации в такой памяти, то уникальные достоинства \textit{семантических
					сетей} будут практически использованы в полной мере.
			\end{scnitemize}}

		\scnheader{семантическая модель базы знаний}
		\begin{scnrelfromvector}{достоинства}
			\scnfileitem{\textit{семантическая модель базы знаний}, построенная на
				основе \textit{универсального языка семантических сетей}, обеспечивает высокий
				уровень ассоциативности доступа к требуемым фрагментам \textit{базы знаний}
				благодаря широкому многообразию реализуемых видов запросов и существенному
				снижению реальной вычислительной сложности алгоритмов доступа (информационного
				поиска)}
			\scnfileitem{\textit{семантическая модель базы знаний} позволяет
				реализовать эффективную семантическую навигацию по текущему состоянию
				\textit{базы знаний} (при просмотре \textit{базы знаний}) путем отображения
				различного вида \textit{семантических окрестностей} для указываемых
				\textit{элементов семантической сети}. При этом \textit{семантическая модель
					базы знаний} позволяет реализовать \uline{наглядную} двумерную или трехмерную
				визуализацию просматриваемого фрагмента \textit{базы знаний} (просматриваемой
				\textit{семантической сети}). Таким образом, \textit{семантическая сеть}
				является средством представления \textit{знаний}, удобным как для самой
				\textit{интеллектуальной компьютерной системы}, так и для её пользователей.}
			\scnfileitem{сущностями, вписываемыми в \textit{семантической модели
					базы знаний} и, соответственно, обозначаемыми \textit{знаками} этих сущностей,
				могут быть не только \textit{части внешней среды} соответствующие
				\textit{интеллектуальной компьютерной системе}, но и \textit{части} (фрагменты)
				самой \textit{базы знаний}.  Это дает возможность \textit{базе знаний} включать
				в себя описание собственной структуры с любой степенью детализации и
				рассматривающее самые разные аспекты такой структуризации.}
			\scnfileitem{\textit{семантическая модель базы знаний} позволяет
				\uline{явно} выделить фрагменты \textit{базы знаний}, представляющие различные
				\textit{предметные области} и соответствующие им \textit{онтологии}, а также
				\uline{явно} описать иерархию выделенных \textit{предметных областей} и иные
				связи мужду ними (например, различного рода морфизмы). Такая семантическая
				структуризация \textit{базы знаний} позволяет осуществлять локализацию области
				действия каждой конкретной операции обработки \textit{базы знаний}, что
				существенно упрощает реализацию этих операций. Каждая \textit{предметная
					область} выделенная в рамках \textit{семантической модели базы знаний},
				описывающая соответствующий класс исследуемых (описываемых) \textit{сущностей}
				(объектов исследования) и соответствующих подклассов этого \textit{класса} с
				помощью соответствующего набора \textit{отношений} (в том числе,
				\textit{функций} и \textit{алгебраических операций}) и соответствующего набора
				\textit{свойств} (параметров), представляет собой результат интеграции
				(соединения) текстов соответствующего \textit{специализированного языка
					семантической сети}.}
			\scnfileitem{размещение каждой информации, хранимой в составе
				\textit{семантической модели базы знаний}, и, соответственно, доступ к этой
				информации (поиск её в \textit{базе знаний}) определяются \uline{исключительно}
				семантическими характеристиками этой информации (т.е. её смыслом) и не зависят
				от особенностей реализации памяти \textit{интеллектуальной компьютерной
					системы}. Т.е. смысл информации \uline{однозначно} определяет её местоположение
				в \textit{семантической модели базы знаний}, а, точнее, её связи с остальной
				частью этой \textit{базы знаний}}
			\scnfileitem{если объем представления \textit{информационной
					конструкции} определить как пару, состоящую (1) из количества синтаксически
				элементарных (атомарных) фрагментов этой конструкции и (2) из числа элементов
				\textit{алфавита} указанных элементарных фрагментов, и если рассмотреть
				множество всевозможных \uline{\textit{семантически эквивалентных*}}
				представлений каждой \textit{информационной конструкции}, то наиболее
				\uline{компактным} (сжатым) её представлением окажется представление в виде
				\textit{рафинированной семантической сети}. Более того, при расширении
				\textit{базы знаний} (при увеличении числа описываемых \textit{сущностей} и, в
				частности, числа описываемых \textit{связей}) компактность
				\textit{рафинированных семантических сетей} повышается, т.к. новые описываемые
				\textit{связи} далеко не всегда рассматривают связи между \uline{новыми}
				\textit{сущностями}, которые до этого не описывались.}
			\scnfileitem{\textit{семантические сети} позволяют говорить о
				принципиально ином характере соединения (конкатенации\, интеграции) двух
				текстов в один интегрированный текст. \textit{Интеграция} двух
				\textit{семантических сетей} предполагает склеивание (отождествление)
				\uline{синонимичных} элементов интегрируемых \textit{семантических сетей}.
				Такая \textit{интеграция}, в частности, происходит при вводе (погружении) новой
				информации в состав \textit{семантической модели базы знаний}.}
			\scnfileitem{семантические модели баз знаний дают возможность:
				\begin{scnitemize}
					\item конструктивно осуществлять анализ \textit{семантической
						связности базы знаний}, путем уточнения понятия семантической силы связи между
					различными элементами и фрагментами базы знаний
					\item конструктивно осуществлять кластеризацию баз знаний
					\item задавать метрику семантического расстояния между знаками,
					входящими в состав базы знаний
					\item осуществлять в рамках базы знаний описания различного
					вида соответствий (морфизмов) между различными фрагментами базы знаний
					(изоморфизмов, гомоморфизмов, аналогий и отличий различного вида). Так,
					например, большое значение имеет исследование таких соответствий между
					различными \textit{предметными областями}
					\item широко использовать мощный арсенал теоретико-графовых
					\textit{алгоритмов} для выполнения различного рода операций обработки
					\textit{баз знаний}.
				\end{scnitemize}}
		\end{scnrelfromvector}

		\bigskip
		\scnheader{следует отличать*}
		\begin{scnhaselementset}
			\scnitem{предельно омонимичный класс синтаксически эквивалентных знаков}
				\begin{scnindent}
					\scnidtf{класс синтаксически эквивалентных знаков, все экземпляры (все
						вхождения) которого являются знаками \uline{разных} сущностей}
					\scntext{следовательно}{В рамках рассматриваемого класса знаков
						синонимия знаков отсутствует}
					\scnidtf{максимальный класс синтаксически эквивалентных знаков,
						среди которых отсутствуют синонимичные знаки}
				\end{scnindent}
			\scnitem{частично омонимичный класс синтаксически эквивалентных знаков}
				\begin{scnindent}
					\scnidtf{класс синтаксически эквивалентных знаков, среди экземпляров
						которого встречаются как синонимичные знаки, так и знаки \uline{разных}
						сущностей}
				\end{scnindent}
			\scnitem{неомонимичный класс синтаксически эквивалентных знаков}
				\begin{scnindent}
					\scnidtf{класс синтаксически эквивалентных знаков, \uline{все} экземпляры
						которого являются знаками одной и той же сущности}
					\scnidtf{класс синтаксически эквивалентных знаков,
						синтаксическая структура которых однозначно идентифицирует (соответствует)
						обозначаемую ими сущность}
				\end{scnindent}
			\scnitem{множество особенностей (характеристик), которыми обладает
				сущность, обозначаемая заданным знаком*}
		\end{scnhaselementset}

		\bigskip
		\begin{scnhaselementset}
			\scnitem{смысловое представление информации*}
				\begin{scnindent}
					\scnidtftext{часто используемый	sc-идентификатор}{смысл*}
				\end{scnindent}
			\scnitem{смысловое представление информации}
				\begin{scnindent}
					\scnrelto{второй домен}{смысловое представление информации*}
				\end{scnindent}
			\scnitem{синтаксическая структура информационной конструкции*}
			\scnitem{синтаксическая структура информационной конструкции}
				\begin{scnindent}
					\scnrelto{второй домен}{синтаксическая структура информационной конструкции*}
				\end{scnindent}
			\scnitem{денотационная семантика информационной конструкции*}
				\begin{scnindent}
					\scnidtf{\textit{соответствие} (морфизм) между синтаксической
						структурой заданной информационной конструкции и ее \textit{смысловым
							представлением*}}
					\scntext{примечание}{\textit{соответствие} между знаками входящими в
						состав \textit{рафинированной семантической сети} и их \textit{денотатами*}
						(обозначаемыми сущностями) являются \uline{взаимно однозначными}}
				\end{scnindent}
		\end{scnhaselementset}

		\begin{scnhaselementset}
			\scnitem{смысловое пространство}
				\begin{scnindent}
					\scnhaselement{SC-пространство}
					\scnidtf{семантическое пространство}
					\scntext{пояснение}{объединение (соединение) всевозможных
						корректных абстрактных семантических сетей, принадлежащих некоторому языку
						абстрактных \textit{рафинированных семантических сетей}}
					\scnidtf{глобальная (максимальная) абстрактная
						\textit{рафинированная семантическая сеть}, включающая в себя всевозможные
						абстрактные рафинированные семантические сети соответствующего языка}
					\scnidtf{абстрактное смысловое пространство}
				\end{scnindent}
			\scnitem{абстрактная смысловая память}
				\begin{scnindent}
					\scnidtf{абстрактная семантическая память}
					\scnidtf{среда, обеспечивающая хранение абстрактных
						рафинированных семантических сетей, а также редактирование этих семантических
						сетей и при этом абстрагирующаяся от деталей этих процессов}
					\scnidtf{абстрактная графодинамическая память, обеспечивающая
						хранение и редактирование абстрактных рафинированных семантических сетей}
				\end{scnindent}
			\scnitem{реальная смысловая память}
				\begin{scnindent}
					\scnidtf{физическая реализация абстрактной смысловой памяти}
					\scnrelboth{следует отличать}{программная реализация \uline{модели} абстрактной смысловой памяти на современных компьютерах}
				\end{scnindent}
		\end{scnhaselementset}
		\bigskip
	\end{scnsubstruct}
\end{SCn}
%\scnsourcecomment{Завершили Раздел \scnqq{Предметная область и онтология смыслового представления информации}}


\scsubsubsection{Пункт 4.2.1. Предметная область и онтология многоагентных моделей решателей задач, основанных на смысловом представлении информации}
\label{sd_agent_solvers}
\begin{SCn}
	\scnsectionheader{Предметная область и онтология многоагентных моделей решателей задач, основанных на смысловом представлении информации}

	\begin{scnsubstruct}

		\scnheader{Предметная область многоагентных онтологических моделей решателей задач, основанных на смысловом представлении информации}
		\scniselement{предметная область}
		\begin{scnhaselementrolelist}{класс объектов исследования}
			\scnitem{многоагентный подход к обработке информации}
		\end{scnhaselementrolelist}

		\begin{scnhaselementrolelist}{класс объектов исследования}
			\scnitem{интеграция решателей задач}
		\end{scnhaselementrolelist}

		\begin{scnhaselementrolelist}{исследуемое отношение}
			\scnitem{совместимость моделей решения задач*}
		\end{scnhaselementrolelist}

		\scnheader{агентно-ориентированный подход к обработке информации}
		\scntext{примечание}{В качестве основы унификации принципов обработки
			информации в компьютерных системах предлагается использовать
			\textit{агентно-ориентированный подход к обработке информации}, обладающий
			рядом важных достоинств.}
		\begin{scnrelfromset}{достоинства}
			\scnfileitem{Автономность (независимость) агентов, что позволяет
				локализовать изменения, вносимые в систему при ее эволюции, и снизить
				соответствующие трудозатраты.}
			\scnfileitem{Децентрализация обработки, т.е. отсутствие единого
				контролирующего центра, что также позволяет локализовать вносимые в систему
				изменения.}
			\scnfileitem{Возможность параллельной работы разных информационных
				процессов, соответствующих как одному агенту, так и разным агентам, как
				следствие, --- возможность распределенного решения задач. Однако возможность
				параллельного выполнения информационных процессов подразумевает наличие средств
				синхронизации такого выполнения, разработка которых является отдельной
				задачей.}
			\scnfileitem{Активность агентов и многоагентной системы в целом, дающая
				возможность при общении с такой системой не указывать явно способ решения
				поставленной задачи, а формулировать задачу в \uline{декларативном ключе}.}
		\end{scnrelfromset}

		\begin{scnindent}
			\scnrelfrom{источник}{\scncite{Wooldridge2009}}
		\end{scnindent}

		\begin{scnrelfromset}{недостатки современного состояния}
			\scnfileitem{Знания агента представляются при помощи
				узкоспециализированных языков, зачастую не предназначенных для представления
				знаний в широком смысле и онтологий в частности.}
			\scnfileitem{Большинство современных многоагентных систем предполагает,
				что взаимодействие агентов осуществляется путем обмена сообщениями
				непосредственно от агента к агенту.}
			\scnfileitem{Логический уровень взаимодействия агентов жестко привязан
				к физическому уровню реализации многоагентной системы.}
			\scnfileitem{Среда, с которой взаимодействуют агенты, уточняется
				отдельно разработчиком для каждой многоагентной системы, что приводит к
				существенным накладным расходам и несовместимости таких многоагентных систем.}
		\end{scnrelfromset}
		\begin{scnindent}
			\begin{scnrelfromset}{принципы устранения}
				\scnfileitem{Коммуникацию агентов предлагается осуществлять путем
					спецификации (в общей памяти компьютерной системы) действий (процессов),
					выполняемых агентами и направленных на решение задач.}
					\begin{scnindent}
						\scntext{детализация}{Коммуникацию агентов предлагается
							осуществлять по принципу \scnqqi{доски объявлений}, однако в отличие от классического
							подхода в роли сообщений выступают спецификации в общей семантической памяти
							выполняемых агентами действий (процессов), направленных на решение каких-либо
							задач, а в роли среды коммуникации выступает сама эта семантическая память.
							Такой подход позволяет:
							\begin{scnitemize}
								\item исключить необходимость разработки специализированного
											языка для обмена сообщениями
								\item обеспечить \scnqq{обезличенность} общения, т. е. каждый из
								агентов в общем случае не знает, какие еще агенты есть в системе, кем
								сформулирован и кому адресован тот или иной запрос. Таким образом, добавление
								или удаление агентов в систему не приводит к изменениям в других агентах, что
								обеспечивает модифицируемость всей системы
								\item агентам, в том числе конечному пользователю,
											формулировать задачи в \uline{декларативном ключе}, т. е. не указывать для
											каждой задачи способ ее решения. Таким образом, агенту заранее не нужно знать,
											каким образом система решит ту или иную задачу, достаточно лишь специфицировать
											конечный результат
								\item сделать средства коммуникации агентов и синхронизации их
											деятельности более понятными разработчику и пользователю системы, не требующими
											изучения специальных низкоуровневых типов данных и форматов сообщений. Таким
											образом повышается доступность предлагаемых решений широкому кругу
											разработчиков.
							\end{scnitemize}
							Следует отметить, что такой подход позволяет при необходимости
							организовать обмен сообщениями между агентами напрямую и, таким образом, может
							являться основой для моделирования многоагентных систем, предполагающих другие
							способы взаимодействия между агентами.}
						\end{scnindent}	
				\scnfileitem{В роли внешней среды для агентов выступает та же общая
					память, в которой формулируются задачи и посредством которой осуществляется
					взаимодействие агентов. Такой подход обеспечивает унификацию среды для
					различных систем агентов, что, в свою очередь, обеспечивает их совместимость.}
				\scnfileitem{Спецификация каждого агента описывается средствами языка
					представления знаний в той же памяти, что позволяет:
					\begin{scnitemize}
						\item минимизировать число специализированных средств, необходимых для
						спецификации агентов, как языковых, так и инструментальных
						\item с одной стороны --- минимизировать необходимую в общем случае
						спецификацию агента, которая включает условие его инициирования и программу,
						описывающую алгоритм работы агента, с другой стороны --- обеспечить возможность
						произвольного расширения спецификации для каждого конкретного случая, в том
						числе возможность реализации различных современных моделей спецификации агента.
					\end{scnitemize}}
				\scnfileitem{Синхронизацию деятельности агентов предполагается
					осуществлять на уровне выполняемых ими процессов, направленных на решение тех
					или иных задач в общей семантической памяти. Таким образом, каждый агент
					трактуется как некий абстрактный процессор, способный решать задачи
					определенного класса. При таком подходе необходимо решить задачу обеспечения
					взаимодействия параллельных асинхронных процессов в общей семантической памяти,
					для решения которой можно заимствовать и адаптировать решения, применяемые в
					традиционной линейной памяти. При этом вводится дополнительный класс агентов --
					метаагенты, задачей которых является решение возникающих проблемных ситуаций,
					таких как взаимоблокировки}
				\scnfileitem{Каждый информационный процесс в любой момент времени имеет
					ассоциативный доступ к необходимым фрагментам базы знаний, хранящейся в
					семантической памяти, за исключением фрагментов, заблокированных другими
					процессами в соответствии с соответствующим механизмом синхронизации. Таким
					образом, с одной стороны, исключается необходимость хранения каждым агентом
					информации о внешней среде, с другой стороны, каждый агент, как и в
					классических многоагентных системах, обладает только частью всей информации,
					необходимой для решения задачи.\\Важно отметить, что в общем случае невозможно
					априори предсказать, какие именно знания, модели и способы решения задач
					понадобятся системе для решения конкретной задачи. В связи с этим необходимо
					обеспечить, с одной стороны, возможность доступа ко всем необходимым фрагментам
					базы знаний (в пределе --- ко всей базе знаний), с другой стороны --- иметь
					возможность локализовать область поиска пути решения задачи, например, рамками
					одной \textit{предметной области}.\\Каждый из агентов обладает набором ключевых
					элементов (как правило, понятий), которые он использует в качестве отправных
					точек при ассоциативном поиске в рамках базы знаний. Набор таких элементов для
					каждого агента уточняется на этапах проектирования многоагентной системы в
					соответствии с рассматриваемой ниже методикой. Уменьшение числа ключевых
					элементов агента делает его более универсальным, однако снижает эффективность
					его работы за счет необходимости выполнения дополнительных  операций.}
			\end{scnrelfromset}	
				\begin{scnindent}
					\scntext{примечание}{Предлагаемый подход позволяет рассматривать решатель
						задач как иерархическую систему. Некий целостный коллектив агентов, реализующий
						какую-либо подсистему решателя (например, машину дедуктивного вывода,
						подсистему верификации базы знаний и т. д.), может рассматриваться как единый
						неатомарный агент, поскольку коллективы агентов и отдельные агенты работают в
						соответствии с одними и теми же принципами.}
				\end{scnindent}
		\end{scnindent}

		\scnheader{совместимость моделей решения задач*}
		\scntext{примечание}{\textbf{\textit{совместимость моделей решения задач*}}
			-- это возможность одновременного использования разными моделями решения задач
			одних и тех же информационных ресурсов.}

		\begin{scnrelfromset}{принципы реализации}
			\scnfileitem{Вся информация, хранимая в памяти каждой ostis-системы и
				используемая \textit{\textbf{решателем задач}} (как собственно обрабатываемая
				информация, так и хранимые в памяти интерпретируемые методы, например,
				различного вида программы), записывается в форме смыслового представления этой
				информации}
			\scnfileitem{Собственно решение каждой задачи осуществляется
				коллективом агентов, работающих над общей для них смысловой (семантической)
				памятью и выполняющих интерпретацию хранимых в этой же памяти
				\textit{методов}.}
		\end{scnrelfromset}

		\scnheader{интеграция решателей задач}
		\scnsubset{процесс}
		\begin{scnrelfromvector}{алгоритм реализации}
			\scnfileitem{Объединение множества методов первого решателя и множества
				методов второго решателя}
			\scnfileitem{Интеграция множества методов первого решателя и множества
				методов второго решателя путем взаимного погружения соответствующих
				информационных конструкций друг в друга, т.е. путем склеивания синонимов, а
				также путем выравнивания используемых ими понятий.}
			\scnfileitem{Объединение множества агентов, входящих в состав первого
				решателя, со множеством агентов, входящих во второй решатель задач.}
		\end{scnrelfromvector}

		\scntext{пояснение}{Таким образом, унификация моделей решения задач
			путем приведения этих моделей к виду семантических моделей (т. е. моделей
			обработки информации, представленной в смысловой форме) повышает уровень
			совместимости этих моделей благодаря наличию прозрачной процедуры интеграции
			информации, представленной в смысловой форме, и тривиальной процедуры
			объединения множеств \textit{агентов}. Простота процедуры объединения множеств
			\textit{агентов}, соответствующих разным решателя задач, обусловлена тем, что
			непосредственного взаимодействия между этими агентами нет, а инициирование
			каждого из них определяется им самим, а также \uline{текущим состоянием}
			хранимой в памяти информации.}
		\bigskip
	\end{scnsubstruct}
\end{SCn}
%\scnsourcecomment{Завершили Раздел \scnqq{Предметная область и онтология многоагентных моделей решения задач, основанных на смысловом представлении информации}}


\scsubsubsection{Пункт 4.2.3. Предметная область и онтология онтологических моделей интерфейсов интеллектуальных компьютерных систем, основанных на смысловом представлении информации}
\label{sd_sem_ui}
\begin{SCn}
	\scnsectionheader{Предметная область и онтология онтологических моделей интерфейсов интеллектуальных компьютерных систем, основанных на смысловом представлении информации}

	\begin{scnsubstruct}

		\scnrelfrom{соавтор}{Садовский М.Е.}
		\scnheader{Предметная область онтологических моделей интерфейсов
			интеллектуальных компьютерных систем, основанных на смысловом представлении
			информации}
		\scniselement{предметная область}

		\begin{scnhaselementrolelist}{класс объектов исследования}
			\scnitem{подход к построению пользовательского интерфейса}
		\end{scnhaselementrolelist}

		\begin{scnhaselementrolelist}{класс объектов исследования}
			\scnitem{подход к построению пользовательского интерфейса на
				основе специализированных языков описания}
			\scnitem{контекстно-зависимый подход к построению
				пользовательского интерфейса;подход к построению пользовательского интерфейса
				на основе данных}
			\scnitem{онтологический подход к построению пользовательского
				интерфейса}
			\scnitem{онтологический подход к построению пользовательского
				интерфейса на основе логико-семантической модели}
		\end{scnhaselementrolelist}

		\scnheader{подход к построению пользовательского интерфейса}
		\scnsuperset{подход к построению пользовательского интерфейса на основе
			специализированных языков описания}
		\scnsuperset{контекстно-зависимый подход к построению пользовательского
			интерфейса}
		\scnsuperset{подход к построению пользовательского интерфейса на основе
			данных}
		\scnsuperset{онтологический подход к построению пользовательского
			интерфейса}
			\begin{scnindent}
				\scnsuperset{онтологический подход к построению пользовательского
					интерфейса на основе логико-семантической модели}
			\end{scnindent}

		\scnheader{подход к построению пользовательского интерфейса на основе
			специализированных языков описания}

		\scntext{пояснение}{подход на основе специализированных языков
			описания предполагает представление конкретного пользовательского интерфейса в
			платформенно независимом виде. В качестве примеров языков описания интерфейса
			можно привести UIML (\cite{ABRAMS19991695}), UsiXML (\cite{UsiXML}), XForms
			(\cite{XForms}) и FXML (\cite{fxml}). Ключевой идеей представленных языков
			является создание модели диалогов и форм интерфейса в независимом от
			используемой технологии виде, описание визуальных элементов, а также
			взаимосвязей между ними и их свойств для создания конкретного пользовательского
			интерфейса.}
		\begin{scnrelfromset}{недостатки современного состояния}
			\scnfileitem{как  правило,  спецификация  модели  интерфейса является
				неполной,  что	приводит  к  сложности автоматизации процесса генерации
				пользовательского интерфейса}
			\scnfileitem{как правило, созданные модели специфичны для конкретной
				платформы либо конкретной реализации пользовательского интерфейса, что
				препятствует их повторному использованию для других целей.}
			\scnfileitem{решения,	которые   предлагают   платформенно независимое
				описание,  позволяют  генерировать лишь  простые  ограниченные  по
				функционалу пользовательские интерфейсы (приложения-опросники, диаграммы и
				т.д.).}
		\end{scnrelfromset}

		\scnheader{контекстно-зависимый подход к построению пользовательского
			интерфейса}
		\scntext{пояснение}{Контекстно-зависимый подход интегрирует
			использование структурного описания интерфейса на основе языков описания с
			поведенческой спецификацией, то есть генерация интерфейса основывается на
			действиях пользователя. В рамках подхода специфицируются переходы между
			различными видами конкретного пользовательского интерфейса. В качестве примеров
			языков, реализующих идеи такого подхода можно привести CAP3 (\cite{CAP3}) и
			MARIA (\cite{MARIA}).}
		\scnheader{подход к построению пользовательского интерфейса на основе
			данных}
		\scntext{пояснение}{подход на основе данных или моделеориентированный
			подход использует модель предметной области в качестве основы для создания
			пользовательских интерфейсов. Указанный подход реализуется в таких проектах,
			как JANUS (\cite{JANUS}) и Mecano (\cite{Mecano}).}
		\scnheader{онтологический подход к построению пользовательского
			интерфейса}
		\scntext{пояснение}{cуществующие онтологические подходы как правило
			основаны на представленных ранее подходах и используют онтологии в качестве
			способа представления информации о конкретном пользовательском интерфейсе.
			Например, по аналогии с подходом на основе специализированных языков описания,
			был предложен фреймворк  (\cite{ui_model-based-approach}), использующий
			онтологию для описания пользовательского интерфейса на основе понятий,
			хранящихся в базе знаний. По аналогии с контекстно-зависимым подходом в рамках
			работы \cite{gaulke} используется модель предметной области совместно с моделью
			пользовательского интерфейса, ассоциированная с онтологией действий. Проект
			ActiveRaUL (\cite{ActiveRaUL}) совмещает UIML с моделеориентированным подходом.
			В рамках данного проекта онтологическая модель предметной области
			сопоставляется с онтологическим представлением пользовательского интерфейса.
			Подход, предложенный в работе \cite{hitz}, совмещает данные приложения с
			онтологией пользовательского интерфейса для создания единого описания в базе
			знаний с целью последующей автоматической генерации различных вариантов
			интерфейса для приложений-опросников с готовыми сценариями взаимодействия с
			пользователем. Следует также отметить работы \cite{vladivostok1} и
			\cite{vladivostok2}, в рамках которых предложена концепция, позволяющая
			объединить однородную по содержанию информацию в компоненты модели интерфейса,
			освободить разработчика интерфейса от кодирования и формировать информацию для
			каждого компонента модели интерфейса с помощью редакторов, управляемых
			соответствующими моделями онтологий.}
		\begin{scnrelfromset}{недостатки современного состояния}
			\scnfileitem{актуальна проблема совместимости различных онтологий в
				рамках единой системы}
			\scnfileitem{отсутствие способности адаптироватьсяк запросам
				пользователя и анализировать его действия длясамостоятельного
				совершенствования.}
		\end{scnrelfromset}

		\begin{scnrelfromset}{достоинства}
			\scnfileitem{позволяет интегрировать ранее предложенные подходы за счет
				единого способа представления знаний.}
			\scnfileitem{позволяет создать наиболее полное описание различных
				аспектов пользовательского интерфейса.}
			\scnfileitem{упрощает повторное использование интерфейса.}
		\end{scnrelfromset}

		\scnheader{онтологический подход к построению пользовательского
			интерфейса на основе логико-семантической модели}
		\scntext{примечание}{для проектирования пользовательских интерфейсов
			предлагается использовать \textbf{\textit{онтологический подход к построению
					пользовательского интерфейса на основе логико-семантической модели}},
			обладающий рядом важных достоинств.}
		\begin{scnrelfromset}{достоинства}
			\scnfileitem{возможность переноса пользовательских интерфейсов с одной
				платформы реализации на другую.}
			\scnfileitem{наличие общих    принципов построения пользовательских
				интерфейсов, что позволяет повторно использовать уже разработанные компоненты
				и  снижает сроки  обучения  пользователя  новым  для  него пользовательским
				интерфейсам.}
			\scnfileitem{возможность модификации пользовательского интерфейса в
				процессе работы.}
			\scnfileitem{возможность гибкой адаптации пользовательского интерфейса
				под нужды конкретного пользователя.}
		\end{scnrelfromset}

		\scntext{пояснение}{подход предполагает создание полной семантической
			модели интерфейса, содержащей  лексическое описание  интерфейса(описание
			компонентов, из которых формируется интерфейс), синтаксическое	описание
			интерфейса(правила  формирования  корректного  и  полного интерфейса из его
			компонентов), но также и его семантическое описание (знание о том, знаком какой
			сущности является отображаемый компонент). При этом семантическое описание
			также включает всебя назначение, область применения компонентов интерфейса,
			описание интерфейсной деятельности пользователя.}
		\begin{scnrelfromset}{основные принципы}
			\scnfileitem{пользовательский интерфейс представляет собой
				специализированную ostis-систему, ориентированную на решение интерфейсных
				задач,и имеющую в составе базу знаний и машину обработки знаний
				пользовательского интерфейса,что даёт возможность пользователю адресовать
				пользовательскому интерфейсу различного рода вопросы}
			\scnfileitem{используется онтологический подход к проектированию
				пользовательского интерфейса, чтоспособствует чёткому разделению деятельности
				различных разработчиков пользовательских интерфейсов, а также унификации
				принципов проектирования}
			\scnfileitem{используется SC-код в качестве формальногоязыка
				внутреннего представления знаний (онтологий, предметных областей и др.),
				благодарячему обеспечивается легкость интерпретацииэтих знаний и системой, и
				человеком - пользователем или разработчиком, а также однозначность восприятия
				этой информации ими}
			\scnfileitem{средствами SC-кода с помощью соответствующих онтологий
				описываются синтаксис и семантика всевозможных используемых внешнихязыков}
			\scnfileitem{трансляции с внутреннего языка на внешний иобратно
				организовываются так, чтобы механизмы трансляции не зависели от внешнего языка,
				для реализации нового специализированногоязыка в таком случае необходимо будет
				толькоописать его синтаксис и семантику, универсальная же модель трансляции не
				будет зависеть отданного описания}
			\scnfileitem{предполагается выбор стилей визуализации,осуществляемый в
				зависимости от вида отображаемых знаний (например, использование различных
				элементов визуализации для одних видов знаний и других - для других), что
				позволит пользователю быстрее обучаться новымспециализированным языкам, а также
				сделатьпростым и понятным отображение знаний}
			\scnfileitem{модель пользовательского интерфейса строитсянезависимо от
				реализации платформы интерпретации такой модели, что позволяет легкопереносить
				разработанную модель на разныеплатформы.}
		\end{scnrelfromset}
		\bigskip
	\end{scnsubstruct}
\end{SCn}
%\scnsourcecomment{Завершили Раздел \scnqq{Предметная область и онтология логико-семантических моделей интерфейсов компьютерных систем, основанных на смысловом представлении информации}}


\scsubsection{\S 4.3. Предметная область и онтология эволюции технологий разработки интеллектуальных компьютерных систем}
\label{intelligent_comp_systems_tech_evolution}

\scsubsection{\S 4.4. Предметная область и онтология комплексной технологии разработки и сопровождения семантически совместимых интеллектуальных компьютерных систем нового поколения}
\label{sd_ostis_tech}
\label{sd_ostis_tech}
\begin{SCn}

    \scnsectionheader{Предметная область и онтология комплексной технологии поддержки жизненного цикла интеллектуальных компьютерных систем нового поколения}
    
    \begin{scnsubstruct}
    
    \scnheader{Предметная область и онтология комплексной технологии поддержки жизненного цикла интеллектуальных компьютерных систем нового поколения}
    \scniselement{предметная область}
    \begin{scnrelfromlist}{ключевой знак}
    	\scnitem{Технология OSTIS}
    	\scnitem{Стандарт ostis-систем}
    	\scnitem{Метасистема OSTIS}
    	\scnitem{Стандарт OSTIS}
    	\scnitem{Экосистема OSTIS}
    \end{scnrelfromlist}
    
    \begin{scnrelfromlist}{ключевое понятие}
    	\scnitem{ostis-система}
    \end{scnrelfromlist}
    
    \begin{scnrelfromlist}{ключевое знание}
    	\scnitem{Обобщенный жизненный цикл ostis-систем}
    	\scnitem{Принципы, лежащие в основе Технологии OSTIS}
    \end{scnrelfromlist}
    
    \scntext{аннотация}{В главе рассмотрены принципы построения комплексной технологии разработки и поддержки жизненного цикла интеллектуальных компьютерных систем нового поколения --- \textit{Технологии OSTIS}.}
    
    \end{scnsubstruct}


    \scnsegmentheader{Технология OSTIS (Open Semantic Technology for Intelligent Systems)}

    \begin{scnsubstruct}
    \scnheader{жизненный цикл интеллектуальной компьютерной системы нового поколения}
    \scnhaselement{проектирование интеллектуальной компьютерной системы нового поколения}
    \begin{scnindent}
        \scnhaselement{проектирование базы знаний интеллектуальной компьютерной системы нового поколения}
    	\scnhaselement{проектирование решателя задач интеллектуальной компьютерной системы нового поколения}
    	\scnhaselement{проектирование интерфейса интеллектуальной компьютерной системы нового поколения}
    \end{scnindent}
    \scnhaselement{реализацию интеллектуальной компьютерной системы нового поколения}
    \scnhaselement{начальное обучение интеллектуальной компьютерной системы нового поколения}
    \scnhaselement{мониторинг качества интеллектуальной компьютерной системы нового поколения}
    \scnhaselement{поддержка требуемого уровня интеллектуальной компьютерной системы нового поколения}
    \scnhaselement{реинжиниринг интеллектуальной компьютерной системы нового поколения}
    \scnhaselement{обеспечение безопасности интеллектуальной компьютерной системы нового поколения}
    \scnhaselement{эксплуатация интеллектуальной компьютерной системы нового поколения}
    
    
    \scnheader{Построение \textit{Технологии} \textit{комплексной поддержки жизненного цикла интеллектуальных компьютерных систем нового поколения}}
    \begin{scnrelfromlist}{предполагает}
    	\scnitem{четкое описание текущей версии \textit{стандарта интеллектуальных компьютерных систем нового поколения}, обеспечивающего семантическую совместимость разрабатываемых систем}
    	\scnitem{создание мощных библиотек семантически совместимых и многократно (повторно) используемых компонентов разрабатываемых \textit{интеллектуальных компьютерных систем}}
    	\scnitem{уточнение требований, предъявляемых к создаваемой комплексной технологии и обусловленных особенностями \textit{интеллектуальных компьютерных систем нового поколения}, разрабатываемых и эксплуатируемых с помощью указанной технологии}
    \end{scnrelfromlist}
    
    
    \scnheader{Создание инфраструктуры, обеспечивающей интенсивное перманентное развитие \textit{Технологии} \textit{комплексной поддержки жизненного цикла интеллектуальных компьютерных систем нового поколения}}
    \begin{scnrelfromlist}{предполагает}
    	\scnitem{обеспечение низкого порога вхождения в \textit{технологию проектирования интеллектуальных компьютерных систем} как для пользователей технологии (то есть разработчиков прикладных или специализированных интеллектуальных компьютерных систем), так и для разработчиков самой технологии}
    	\scnitem{обеспечение высоких темпов развития \textit{технологии} за счет учета опыта разработки различных приложений путем активного привлечения авторов приложений к участию в развитии (совершенствовании) \textit{технологии}}
    \end{scnrelfromlist}
    
    
    \scnheader{Технология комплексной поддержки жизненного цикла интеллектуальных компьютерных систем нового поколения}
    \scnrelfrom{принципы, лежащие в основе}{В основе создания предлагаемой \textbf{\textit{технологии комплексной поддержки жизненного цикла интеллектуальных компьютерных систем нового поколения}} лежат следующие положения}
    \begin{scnindent}
        \begin{scneqtovector}
            \scnitem{Реализация предлагаемой \textit{технологии} разработки и сопровождения \textit{интеллектуальных компьютерных систем нового поколения} в виде \textbf{\textit{интеллектуальной компьютерной метасистемы}}, которая полностью соответствует \textit{стандартам} предлагаемых \textit{интеллектуальных компьютерных систем нового поколения}, разрабатываемым по предлагаемой \textit{технологии}.}
            \begin{scnindent}
                \scntext{пояснение}{В состав такой \textit{интеллектуальной компьютерной метасистемы}, реализующей предлагаемую технологию входит:
                \begin{scnitemize}
                \item  формальное онтологическое описание текущей версии \textit{стандарта интеллектуальных компьютерных систем нового поколения}
        		\item  формальное онтологическое описание текущей версии \textit{методов и средств проектирования, реализации, сопровождения, реинжиниринга и эксплуатации интеллектуальных компьютерных систем нового поколения}
            \end{scnitemize}}
            \end{scnindent}  
            \scnitem{\textbf{\textit{Унификация}} и \textbf{\textit{стандартизация} интеллектуальных компьютерных систем нового поколения}, а также \textit{методов} их \textit{проектирования, реализации, сопровождения, реинжиниринга и эксплуатации}}
            \scnitem{Перманентная эволюция \textbf{\textit{стандарта интеллектуальных компьютерных систем нового поколения}}, а также \textit{методов} их \textit{проектирования, реализации, сопровождения, реинжиниринга и эксплуатации;}}
            \scnitem{\textbf{\textit{Онтологическое проектирование} интеллектуальных компьютерных систем нового поколения}}
            \begin{scnindent}
                \scntext{пояснение}{Онтологическое проектирование интеллектуальных компьютерных систем предполагает:
                \begin{scnitemize}
                    \item  четкое согласование и оперативную формализованную фиксацию (в виде \textit{формальных онтологий}) утвержденного \textit{текущего состояния} иерархической системы всех \textit{понятий}, лежащих в основе перманентно эволюционируемого \textit{стандарта интеллектуальных компьютерных систем нового поколения}, а также в основе каждой разрабатываемой \textit{интеллектуальной компьютерной системы}
    
    		      \item  достаточно полное и оперативное документирование текущего состояния каждого проекта
    
    		      \item  использование \textit{методики проектирования} \textit{"сверху-вниз"{}}
                \end{scnitemize}}
            \end{scnindent}
            \scnitem{\textbf{\textit{Компонентное проектирование} интеллектуальных компьютерных систем нового поколения}, то есть проектирование, ориентированное на сборку \textit{интеллектуальных компьютерных систем} из готовых компонентов на основе постоянно расширяемых библиотек \textit{многократно используемых компонентов}}
            \scnitem{\textbf{\textit{Комплексный характер}} предлагаемой \textit{технологии}}
            \begin{scnindent}
                \scntext{пояснение}{Комплексный характер технологии осуществляет:
                \begin{scnitemize}
                    \item поддержку \textit{проектирования} не только \textit{компонентов} \textit{интеллектуальных компьютерных систем нового поколения} (различных \textit{фрагментов баз знаний, баз знаний} в целом, различных \textit{методов решения задач}, различных \textit{внутренних информационных агентов, решателей задач} в целом, формальных онтологических описаний различных \textit{внешних языков}, \textit{интерфейсов} в целом), но также и \textit{интеллектуальных компьютерных систем} в целом как самостоятельных \textit{объектов проектирования} с учетом специфики тех классов, которым принадлежат проектируемые \textit{интеллектуальных компьютерных системы}
    	          \item поддержку не только \textit{комплексного} \textit{проектирования} \textit{интеллектуальных компьютерных систем} \textit{нового поколения}, но также и поддержку их реализации (сборки, воспроизводства), сопровождения, реинжиниринга в ходе эксплуатации и непосредственно самой эксплуатации
                \end{scnitemize}}
            \end{scnindent}
        \end{scneqtovector}
    \end{scnindent}
    
    
    \scnheader{База знаний Метасистемы OSTIS}
    \begin{scnsubdividing}
        \scnitem{Формальную теорию \textit{ostis-систем}}
        \scnitem{Стандарт \textit{ostis-систем}}
        \begin{scnindent}
            \begin{scnsubdividing}
                \scnitem{Стандарт баз знаний \textit{ostis-систем}}
                \begin{scnindent}
                    \begin{scnsubdividing}
            			\scnitem{Стандарт внутреннего универсального языка смыслового представления знаний в памяти \textit{ostis-систем}}
            			\scnitem{Стандарт внутреннего представления онтологий верхнего уровня в памяти \textit{ostis-систем}}
            			\scnitem{Стандарт представления исходных текстов баз знаний \textit{ostis-систем}}
                    \end{scnsubdividing}
                \end{scnindent}
                \scnitem{Стандарт решателей задач \textit{ostis-систем}}
                \begin{scnindent}
                    \begin{scnsubdividing}
                        \scnitem{Стандарт базового языка программирования \textit{ostis-систем}}
            			\scnitem{Стандарт языков программирования высокого уровня для \textit{ostis-систем}}
            			\scnitem{Стандарт представления искусственных нейронных сетей в памяти \textit{ostis-систем}}
            			\scnitem{Стандарт внутренних информационных агентов в \textit{ostis-систем}}
                    \end{scnsubdividing}
                \end{scnindent}
                \scnitem{Стандарт интерфейсов \textit{ostis-систем}}
                \begin{scnindent}
                    \begin{scnsubdividing}
                        \scnitem{Стандарт внешних языков \textit{ostis-систем}, близких к внутреннему универсальному языку смыслового представления знаний}
                    \end{scnsubdividing}
                \end{scnindent}
            \end{scnsubdividing}
            \scnitem{Стандарт методик и средств поддержки жизненного цикла \textit{ostis-систем}}
            \begin{scnsubdividing}
                \scnitem{Ядро Библиотеки многократно используемых компонентов \textit{ostis-систем} (\textbf{\textit{Библиотеки OSTIS}})}
        		\scnitem{Методики \textit{поддержки жизненного цикла} \textit{ostis-систем} и их компонентов}
        		\scnitem{Инструментальные средства поддержки жизненного цикла \textit{ostis-систем}}
            \end{scnsubdividing}
        \end{scnindent}    
    \end{scnsubdividing}
    
    
    
    
    \scnheader{Технология OSTIS}
    \scnidtf{Open Semantic Technology for Intelligent Systems}
    \scnidtf{Открытая семантическая технология комплексной поддержки жизненного цикла \textbf{\textit{семантически совместимых}} интеллектуальных компьютерных систем нового поколения}
    \scnidtf{Модели, методики, методы и средства комплексной поддержки жизненного цикла интеллектуальных компьютерных систем нового поколения}
    \scnidtf{Теория интеллектуальных компьютерных систем нового поколения и практика компьютерной поддержки их жизненного цикла}
    \scnidtf{Технологический комплекс (моделей, методик, автоматизированных методов и средств), соответствующий интеллектуальным компьютерным системам нового поколения (интероперабельным и семантически совместимым компьютерным системам)}
    \scnidtf{Предлагаемая нами комплексная технология поддержки всех этапов жизненного цикла всех компонентов для всех классов (видов) интеллектуальных компьютерных систем нового поколения при перманентной поддержке их семантической совместимости}
    \begin{scnrelfromlist}{принципы, лежащие в основе}
        \scnitem{комплексный характер технологии, заключающийся в том, что осуществляется поддержкa всех этапов жизненного цикла создаваемых продуктов, для всех компонентов интеллектуальных компьютерных систем нового поколения, для всех классов интеллектуальных компьютерных систем нового поколения}
        \scnitem{обеспечивается перманентная поддержка семантической совместимости между всеми создаваемыми интеллектуальными компьютерными системами нового поколения}
        \scnitem{ориентация на комплексную автоматизацию всего многообразия человеческой деятельности}
        \scnitem{реализация технологии и, соответственно, комплексная автоматизация поддержки жизненного цикла интеллектуальных компьютерных систем нового поколения (со всеми их компонентами и классами) осуществляется в виде семейства интеллектуальных компьютерных систем нового поколения, построенных по той же технологии}
    \end{scnrelfromlist}
    
    
    \scnheader{Стандарт OSTIS}
    \scnidtf{Стандарт Технологии OSTIS}
    \scnidtf{Основная часть базы знаний Метасистемы OSTIS}
    \scnhaselement{Стандарт ostis-систем}
    \scnhaselement{Стандарт методик и средств поддержки жизненного цикла ostis-систем}

\end{scnsubstruct}
\end{SCn}


\begin{SCn}
\scnsegmentheader{Семантически совместимые ostis-системы}

\begin{scnsubstruct}


    \scnheader{база знаний ostis-системы}
    \scnidtf{sc-конструкция, которая в текущий момент времени хранится в памяти ostis-системы}
    \scnsuperset{база знаний индивидуальной ostis-системы}
    \begin{scnindent}
        \scnsuperset{база знаний корпоративной ostis-системы}
    \end{scnindent}
    \scnsuperset{распределенная база знаний коллектива ostis-систем}
    \begin{scnindent}
        \scnhaselement{База знаний Экосистемы OSTIS}
    \end{scnindent}
    \scnnote{Каждый \textit{sc-элемент} (знак, хранимый в базе знаний ostis-системы) по отношению к базе знаний \textit{ostis-системы} считается временной сущностью, поскольку каждый \textit{sc-элемент} в какой-то момент вводится в состав \textit{базы знаний} и в какой-то момент может быть из нее удален, но при этом следует отличать временный характер самого \textit{sc-элемента} от временного или постоянного характера обозначаемой им сущности}


	\scnheader{ostis-система}
	\scnidtf{интеллектуальная компьютерная система нового поколения, построенная по \textit{Технологии OSTIS}}
	\scnidtf{предлагаемое нами уточнение понятия интеллектуальной компьютерной системы нового поколения}
	\begin{scnsubdividing}
		\scnitem{ostis-субъект}
		\begin{scnindent}
			\scnidtf{самостоятельная \textit{ostis-система}}
			\scnidtf{интероперабельная ostis-система}
			\begin{scnsubdividing}
				\scnitem{индивидуальная ostis-система}
				\scnitem{коллективная ostis-система}
			\end{scnsubdividing}
		\end{scnindent}
		\scnitem{встроенная ostis-система}
		\begin{scnindent}
			\scnidtf{\textit{ostis-система}, являющаяся частью некоторой \textit{индивидуальной ostis-системы}}
		\end{scnindent}
	\end{scnsubdividing}

	\scnheader{интеллектуальная компьютерная система}
	\scnsuperset{интероперабельная интеллектуальная компьютерная система}
	\begin{scnindent}
		\scnidtf{интеллектуальная компьютерная система нового поколения}
		\scnsuperset{ostis-субъект}
		\begin{scnindent}
			\scnidtf{предлагаемый нами вариант построения интероперабельных интеллектуальных компьютерных систем}
		\end{scnindent}
	\end{scnindent}

	\scnheader{индивидуальная ostis-система}
	\scnidtf{минимальная самостоятельная \textit{ostis-система}}
	\begin{scnsubdividing}
		\scnitem{персональный ostis-ассистент}
		\begin{scnindent}
			\scnidtf{\textit{ostis-система}, осуществляющая комплексное адаптивное обслуживание конкретного пользователя по \textit{всем} вопросам, касающимся его взаимодействия с любыми другими \textit{ostis-системами}, а также представляющая интересы этого пользователя во всей глобальной сети \textit{ostis-систем}}
		\end{scnindent}
		\scnitem{корпоративная ostis-система}
		\begin{scnindent}
			\scnidtf{\textit{ostis-система}, осуществляющая координацию совместной деятельности \textit{ostis-систем} в рамках соответствующего коллектива \textit{ostis-систем}, осуществляющая мониторинг и реинжиниринг соответствующего множества \textit{ostis-систем} и представляющая интересы этого коллектива в рамках других коллективов \textit{ostis-систем}}
		\end{scnindent}
		\scnitem{индивидуальная ostis-система, не являющаяся ни персональным ostis-ассистентом, ни корпоративной ostis-системой}
	\end{scnsubdividing}

	\scnheader{коллективная ostis-система}
	\scnidtf{многоагентная система, представляющая собой коллектив индивидуальных и коллективных \textit{ostis-систем}, деятельность которого координируется соответствующей корпоративной \textit{ostis-системой}}
	\scntext{примечание}{В состав коллектива \textit{ostis-систем} могут входить индивидуальные \textit{ostis-системы} могут входить индивидуальные \textit{ostis-системы} любого вида --- в том числе, корпоративные \textit{ostis-системы}, представляющие интересы других коллективов \textit{ostis-систем}}

    \scnheader{Метасистема OSTIS}
    \scnidtf{Индивидуальная ostis-система, являющаяся реализацией Ядра Технологии OSTIS}
    \scnidtf{Интеллектуальная компьютерная система нового поколения, построенная по \textit{Технологии OSTIS} и обеспечивающая автоматизацию компьютерную поддержку жизненного цикла интеллектуальной компьютерной системы нового поколения, создаваемых также по \textit{Технологии OSTIS}}
    \scniselement{ostis-система}
    \scnrelto{предлагаемая форма реализации}{Технология OSTIS}
    
    \scnheader{Экосистема OSTIS}
    \scnidtf{Коллективная ostis-система, представляющая собой глобальный коллектив \uline{всех} \textit{ostis-систем}, взаимодействующих между собой и осуществляющих комплексную автоматизацию человеческой деятельности}
    \scnidtf{Глобальная сеть взаимодействующих \textit{ostis-систем}, ориентированная на перманентно расширяемую комплексную автоматизацию самых различных видов и областей человеческой деятельности}
    \scnnote{Это основной продукт \textit{Технологии OSTIS}, который можно рассматривать как предлагаемый нами подход к реализации \textit{Общества 5.0}, \textit{Науки 5.0}, \textit{Индустрии 5.0}}

\end{scnsubstruct}
\end{SCn}



\scsection{Глава 5. Человеко-машинные системы и их эволюция}
\label{human_machine_sys}
\begin{SCn}
	\begin{scnrelfromset}{содержание}
		\scnitem{\S 5.1. Предметная область и онтология человеко-машинных систем}
		\scnitem{\S 5.2. Предметная область и онтология эволюции человеко-машинных систем}
		\scnitem{\S 5.3. Предметная область и онтология компонентов интеллектуального человеко-машинного сообщества разработчиков прикладных семантически совместимых интеллектуальных компьютерных систем нового поколения}
		\scnitem{\S 5.4. Предметная область и онтология компонентов интеллектуального человеко-машинного сообщества разработчиков комплексной технологии разработки и сопровождения семантически совместимых интеллектуальных компьютерных систем нового поколения}
		\scnitem{\S 5.5. Предметная область и онтология компонентов интегрированного интеллектуального человеко-машинного сообщества специалистов в области Искусственного интеллекта}
		\scnitem{\S 5.6. Предметная область и онтология глобального глобального интеллектуального человеко-машинного сообщества}  
	\end{scnrelfromset}
\end{SCn}

\scsubsection{\S 5.1. Предметная область и онтология человеко-машинных систем}
\label{human_machine_systems}

\scsubsection{\S 5.2. Предметная область и онтология эволюции человеко-машинных систем}
\label{human_machine_systems_evolution}

\scsubsection{\S 5.3. Предметная область и онтология компонентов интеллектуального человеко-машинного сообщества разработчиков прикладных семантически совместимых интеллектуальных компьютерных систем нового поколения}
\label{community_systems_developers}

\scsubsection{\S 5.4. Предметная область и онтология компонентов интеллектуального человеко-машинного сообщества разработчиков комплексной технологии разработки и сопровождения семантически совместимых интеллектуальных компьютерных систем нового поколения}
\label{community_tech_developers}

\scsubsection{\S 5.5. Предметная область и онтология компонентов интегрированного интеллектуального человеко-машинного сообщества специалистов в области Искусственного интеллекта}
\label{community_ai_specialists}

\scsubsection{\S 5.6. Предметная область и онтология глобального глобального интеллектуального человеко-машинного сообщества}
\label{global_community}
\begin{SCn}

\scnsectionheader{Предметная область и онтология традиционных компьютерных технологий}

\scnstartsubstruct

\scnheader{Предметная область традиционных компьютерных технологий}
\scnsdmainclasssingle{традиционная компьютерная технология}
\scnsdclass{***}
\scnsdrelation{***}

\scnheader{традиционная компьютерная технология}
\scntext{текущее состояние}{Современное состояние \textbf{\textit{традиционных компьютерных технологий}} в целом можно охарактеризовать как:
\begin{scnitemize}
    \item иллюзию благополучия;
    \item иллюзию всесилия финансовых ресурсов в решении сложных технических задач;
    \item "вавилонское столпотворение"\ различных технических решений, о совместимости которых никто серьезно не задумывается;
    \item отсутствие комплексного системного подхода к автоматизации сложных видов проектной деятельности;
    \item отсутствие осознания того, что недостатки современных компьютерных технологий имеют фундаментальный, системный характер.
\end{scnitemize}}

\scntext{недостатки}{К недостаткам традиционных компьютерных технологий можно отнести:
\begin{scnenumerate}
\item Многообразие синтаксических форм представления одной и той же информации, т.е. многообразие семантически эквивалентных форм (языков) представления (кодирования) обрабатываемой информации (знаний) в памяти компьютерных систем. Отсутствие унификации представления различного вида знаний в памяти современных компьютерных систем приводит:
\begin{scnitemizeii}
	\item к многообразию семантически эквивалентных моделей решения задач (как процедурных, так и непроцедурных – функциональных, логических и т.д.), т.е. к дублированию моделей обработки информации, отличающихся не сутью способов решения задач, а формой представления обрабатываемой информации и формой представления способов (навыков) решения различных классов задач;
	\item к дублированию семантически эквивалентных информационных компонентов компьютерных систем;
	\item к многообразию форм технической реализации каждой используемой модели решения задач;
	\item к семантической несовместимости компьютерных систем и, следовательно, к высокой трудоемкости их интеграции в системы более высокого уровня иерархии, требующей дополнительных усилий на трансляцию (конвертирование) информации, которой обмениваются разные интегрируемые системы и, следовательно, существенно ограничивающей эффективность совместного решения задач коллективом взаимодействующих компьютерных систем. Трудоемкость процесса интеграции может быть существенно снижена за счет приведения интегрируемых компьютерных систем к некоторой унифицированной форме, поскольку в этом случае интеграция может быть осуществлена универсальным и автоматизированным способом;
	\item к существенному снижению эффективности применения методики компонентного проектирования компьютерных систем на основе библиотек многократно используемых компонентов (особенно, если речь идет о "крупных"\ компонентах, в частности, о типовых подсистемах)~\cite{Borisov2014}.
\end{scnitemizeii}

\item Недостаточно высокую степень обучаемости современных компьютерных систем в ходе их эксплуатации, следствием чего является высокая трудоемкость их сопровождения и совершенствования, а также недостаточно длительный их жизненный цикл. 

\item Отсутствие возможности у экспертов реально влиять на качество разрабатываемых компьютерных систем. Опыт разработки сложных компьютерных систем показывает, что посредничество программистов между экспертами и проектируемыми компьютерными системами существенно искажает вклад экспертов. При разработке компьютерных систем следующего поколения доминировать должны не программисты, а эксперты, способные точно излагать свои знания.

\item Отсутствие семантической (смысловой) унификации интерфейсной деятельности пользователей компьютерных систем, что вместе с многообразием форм реализации пользовательских интерфейсов приводит к серьезным накладным расходам на усвоение пользовательских интерфейсов новых компьютерных систем.

\item Документация компьютерной системы не является важным компонентом самой компьютерной системы, определяющим качество функционирования этой системы, следствием чего является недостаточно высокая эффективность эксплуатации компьютерной системы из-за неполного и неэффективного использования возможностей эксплуатируемой компьютерной системы.
\end{scnenumerate}

Преодолеть указанные недостатки можно только путем фундаментального переосмысления архитектуры и принципов организации сложных компьютерных систем. Основой такого переосмысления является устранение многообразия форм представления (кодирования) информации в памяти компьютерных систем.

Результатом такого переосмысления должен стать новый этап развития компьютерных технологий.

Преодоление недостатков современных компьютерных систем предполагает:
\begin{scnitemize}
\item унификацию представления обрабатываемой информации;
\item функциональную унификацию (унификацию принципов обработки информации).
\end{scnitemize}}

\scnheader{традиционная компьютерная система}
\scnevolution{Расширение областей применения компьютерных систем требует перехода от традиционных компьютерных систем к системам, ориентированным на обработку широкого многообразия структурированной информации, а также на решение все более и более сложных задач. Следовательно, переход от традиционных компьютерных систем к интеллектуальным системам неизбежен. Более того, этот переход давно уже происходит. Это подтверждают такие направления эволюции компьютерных систем, как:

\begin{scnitemize}
    \item переход от доминирования программ к доминированию обрабатываемой информации, т. е. компьютерным системам, управляемым данными;
    \item от слабоструктурированных данных к структурированным и независящим от программ, обрабатывающих эти данные, т. е. к базам данных;
    \item от данных к знаниям путем расширения семантических видов обрабатываемой информации, и далее к компьютерным системам, управляемым структурированными знаниями, и к компьютерным системам, управляемыми базами знаний;
    \item переход от неконтекстного решения задач, исходные данные для которых априори точно заданы, к решению задач с активным использованием контекста этих задач, т. е. знаний о той предметной области, в рамках которой задача решается;
    \item переход и процедурных языков программирования низкого уровня к процедурным языкам программирования высокого уровня, и к непроцедурным языкам программирования (функциональным, логическим);
    \item переход от последовательных программ к параллельным;
    \item переход от синхронной обработки информации к асинхронной;
    \item переход от программ к исчислениям, к "мягким"\ вычислениям (нечетким логикам, генетическим алгоритмам, искусственным нейросетям);
    \item переход от программ, ориентированных на обработку данных, структуризация которых определяется соответствующими программами, к программам, ориентированным на обработку баз данных и далее баз знаний;
    \item переход от адресной памяти к ассоциативной памяти;
    \item переход от линейной памяти к нелинейной (структурно перестраиваемый, реконфигурируемой, графодинамической) памяти, в которой обработка информации сводится не только к изменению состояния элементов в памяти, но и к изменению конфигурации связей между ними;
    \item переход от традиционных компьютерных систем к компьютерным системам, способным решать широкое многообразие сложных (трудно формализуемых) задач и, в том числе интеллектуальных задач, к компьютерным системам с гибридной хорошо структурированной базой знаний высокого качества, с гибридным решателем задач, с гибридным (мультимодальным) интерфейсом (как вербальным, так и невербальным);
    \item переход от необучаемых компьютерных систем к обучаемым.
\end{scnitemize}

Следовательно, интеллектуализация компьютерных систем -- это естественное направление их эволюции.}

\scnendstruct

\end{SCn}
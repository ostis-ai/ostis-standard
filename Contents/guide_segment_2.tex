\scnsegmentheader{Основные принципы работы с исходными текстами Стандарта OSTIS посредством Github}
\scnstartsubstruct

\scnheader{работа с исходными текстами Стандарта OSTIS посредством Github}

\scnrelfromvector{основные принципы}{
	Начало работы с Git. Клонирование репозитория\\
	\scnrelfromset{этапы реализации}{
		\scnfileitem{Устанавливаем GitHub Desktop, SmarGit или любой другой графический клиент для Git. Можно работать и через терминал. Дальнейшие шаги показаны на примере GitHub Desktop};
		\scnfileitem{После установки в появившемся окне нажимаем “Sign in to GitHub.com”};
		\scnfileitem{В открывшемся окне браузера вводим в форму свои данные, как при регистрации, и нажимаем “Sign in”. Или же создаем новый аккаунт.\\
			\scntext{иллюстрация}{
				$\newline$
				\scnfilescg{Contents/guide_image/image14}}};
		\scnfileitem{Если браузер запросит, то подтвердить, что нужно “Открыть приложение GitHub Desktop”.\\
			\scntext{иллюстрация}{
				$\newline$
				\scnfilescg{Contents/guide_image/image15}
				$\newline$
				\scnfilescg{Contents/guide_image/image16}}};
		\scnfileitem{Далее регистрационные данные перенесутся в форму конфигурации (настроек) Git - нажимаем “Finish”.\\
			\scntext{иллюстрация}{
				$\newline$
				\scnfilescg{Contents/guide_image/image17}}};
		\scnfileitem{Далее видим начальное окно GitHub Desktop.\\
			\scntext{иллюстрация}{
				$\newline$
				\scnfilescg{Contents/guide_image/image18}}
		\scntext{пояснение}{
			\textbf{“Create a tutorial repository…“} - создать обучающий репозиторий.\\
			\textbf{“Clone repository from the Internet…“} - клонировать (скопировать/скачать) репозиторий из GitHub к себе на компьютер.\\
			\textbf{“Create a New Repository on your hard drive…“} - создать новый репозиторий на вашем жестком диске (на вашем компьютере) и добавить систему Git в проект.\\
			\textbf{“Add an Existing Repository from your hard drive…“} - добавить на GitHub репозиторий, который уже есть на вашем компьютере и использует Git.}
		Справа будут отображаться ваши репозитории, которые уже загружены на GitHub, но если только что зарегистрировались, то список будет пуст.};
		\scnfileitem{В меню выбираем File > Clone Repository.\\
			\scntext{иллюстрация}{
				$\newline$
				\scnfilescg{Contents/guide_image/image19}}};
		\scnfileitem{Выбираем URL и вставляем адрес проекта по разработке Стандарта Технологии OSTIS:  \uline{https://github.com/ostis-ai/ostis-standard}. Указываем папку назначения для проекта.\\
			\scntext{иллюстрация}{
				$\newline$
				\scnfilescg{Contents/guide_image/image20}}};
		\scnfileitem{Репозиторий успешно склонирован, теперь у вас есть локальная копия.}};
	Работа с TeX, компиляция проекта в PDF\\
	\scnrelfromset{этапы реализации}{
		\scnfileitem{ Непосредственно в рамках репозитория есть файл readme.md, в котором описана инструкция для установки дистрибутива Tex и компиляции проекта для ОС на базе Linux. Для ОС Windows установка осуществляется аналогично, далее процесс установки и настройки будет показан на примере дистрибутива MiKTex и среды TeXstudio};
		\scnfileitem{Для работы с TeX необходимо скачать и установить дистрибутив TeX, например, MiKTeX или Texlive. Инструкция по установке MiKTeX здесь. Для работы под ОС на базе Linux рекомендуется использовать дистрибутив Texlive, при этом обязательно установить полную версию дистрибутива со всеми пакетами (texlive-full)};
		\scnfileitem{Далее устанавливаем любой удобный редактор LaTeX, например TeXstudio. Руководство пользователя TeXstudio находится на официальном сайте};
		\scnfileitem{После установки MiKTex и Texstudio, запускаем Texstudio};
		\scnfileitem{Файл -> Открыть (Ctrl + O). Находим папку с проектом, указанную на 8 шаге. Выбираем book.tex.\\
			\scntext{иллюстрация}{
				$\newline$
				\scnfilescg{Contents/guide_image/image21}}};
		\scnfileitem{Для компиляции и одновременного просмотра проекта нажимаем F5 (двойная зеленая стрелочка в меню) или же F6 для компиляции (одинарная зеленая стрелка).}}
		\scnrelfromset{примечание}{
		\scnfileitem{Для компиляции полной pdf-версии стандарта с полным оглавлением и вертикальными фоновыми линиями необходимо произвести полную компиляцию 2-3 раза};
		\scnfileitem{В процессе компиляции стандарта могут в большом количестве возникать  ошибки (например, “no line here to end”), связанные с недоработками в реализации текущей версии команд LaTex, используемых при разработке стандарта. Эти ошибки не влияют на результат компиляции. Если PDF не отображается автоматически, его можно увидеть, выполнив команду View (F7)};
		\scnfileitem{Для корректной компиляции библиографических источников иногда требуется при первой сборке выполнить команду latexmk. Ниже показано расположение команды в меню в среде TexStudio.\\
			\scntext{иллюстрация}{
				$\newline$
				\scnfilescg{Contents/guide_image/image22}}}
	};
	Внесение локальных изменений в исходный текст Стандарта OSTIS\\
	\scntext{пояснение}{
		Все изменения обязательно делаются в отдельной ветке репозитория.
		После каждого логически законченного изменения делается коммит. В сообщении коммита предпочтительно на английском языке пишется какие разделы стандарта были изменены и в чем заключается суть изменений.};
	Внесение изменений в основной репозиторий Стандарта OSTIS\\
	\scntext{пояснение}{
		Поскольку репозиторий \textit{Стандарта OSTIS} является открытым, и принять участие в работе над \textit{Стандартом} потенциально может любой желающий, то работа с репозиторием осуществляется через механизм fork и pull-request. Почитать об этом подробнее можно здесь.\\
		Каждый pull-request должен пройти рецензирование как минимум:\\
		\begin{scnitemizeiii}
			\item хотя бы одним редактором того раздела, в который вносятся изменения;
			\item хотя бы одним членом \textit{Редколлегии Стандарта OSTIS}.
		\end{scnitemizeiii}\\
		Рецензентов автор pull-request-a может назначить самостоятельно, чтобы ускорить процесс рецензирования и исключить необходимость всем участникам процесса регулярно просматривать все приходящие pull-request-ы.   \\
		В процессе работы и обсуждения pull-request-а  допускается наличие временных коммитов с временными условными сообщениями. После принятия pull-request-а история коммитов приводится в порядок при помощи force push и осуществляется слияние (merge). Слияние выполняет член \textit{Редколлегии Стандарта OSTIS} после проверки истории коммитов.\\
		Для опытных авторов, внесших значительный вклад в развитие \textit{Стандарта} может рассматриваться вопрос о добавлении в основной репозиторий в качестве коллаборатора. В этом случае работа ведется в отдельной ветке и pull-request делается из этой ветки в основную ветку (master).\\}
}



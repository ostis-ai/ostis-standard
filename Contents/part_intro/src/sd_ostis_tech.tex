\label{sd_ostis_tech}
\begin{SCn}

    \scnsectionheader{Предметная область и онтология комплексной технологии поддержки жизненного цикла интеллектуальных компьютерных систем нового поколения}
    
    \begin{scnsubstruct}
    
    \scnheader{Предметная область и онтология комплексной технологии поддержки жизненного цикла интеллектуальных компьютерных систем нового поколения}
    \scniselement{предметная область}
    \begin{scnrelfromlist}{ключевой знак}
    	\scnitem{Технология OSTIS}
    	\scnitem{Стандарт ostis-систем}
    	\scnitem{Метасистема OSTIS}
    	\scnitem{Стандарт OSTIS}
    	\scnitem{Экосистема OSTIS}
    \end{scnrelfromlist}
    
    \begin{scnrelfromlist}{ключевое понятие}
    	\scnitem{ostis-система}
    \end{scnrelfromlist}
    
    \begin{scnrelfromlist}{ключевое знание}
    	\scnitem{Обобщенный жизненный цикл ostis-систем}
    	\scnitem{Принципы, лежащие в основе Технологии OSTIS}
    \end{scnrelfromlist}
    
    \scntext{аннотация}{В главе рассмотрены принципы построения комплексной технологии разработки и поддержки жизненного цикла интеллектуальных компьютерных систем нового поколения --- \textit{Технологии OSTIS}.}
    
    \end{scnsubstruct}


    \scnsegmentheader{Технология OSTIS (Open Semantic Technology for Intelligent Systems)}

    \begin{scnsubstruct}
    \scnheader{жизненный цикл интеллектуальной компьютерной системы нового поколения}
    \scnhaselement{проектирование интеллектуальной компьютерной системы нового поколения}
    \begin{scnindent}
        \scnhaselement{проектирование базы знаний интеллектуальной компьютерной системы нового поколения}
    	\scnhaselement{проектирование решателя задач интеллектуальной компьютерной системы нового поколения}
    	\scnhaselement{проектирование интерфейса интеллектуальной компьютерной системы нового поколения}
    \end{scnindent}
    \scnhaselement{реализацию интеллектуальной компьютерной системы нового поколения}
    \scnhaselement{начальное обучение интеллектуальной компьютерной системы нового поколения}
    \scnhaselement{мониторинг качества интеллектуальной компьютерной системы нового поколения}
    \scnhaselement{поддержка требуемого уровня интеллектуальной компьютерной системы нового поколения}
    \scnhaselement{реинжиниринг интеллектуальной компьютерной системы нового поколения}
    \scnhaselement{обеспечение безопасности интеллектуальной компьютерной системы нового поколения}
    \scnhaselement{эксплуатация интеллектуальной компьютерной системы нового поколения}
    
    
    \scnheader{Построение \textit{Технологии} \textit{комплексной поддержки жизненного цикла интеллектуальных компьютерных систем нового поколения}}
    \begin{scnrelfromlist}{предполагает}
    	\scnitem{четкое описание текущей версии \textit{стандарта интеллектуальных компьютерных систем нового поколения}, обеспечивающего семантическую совместимость разрабатываемых систем}
    	\scnitem{создание мощных библиотек семантически совместимых и многократно (повторно) используемых компонентов разрабатываемых \textit{интеллектуальных компьютерных систем}}
    	\scnitem{уточнение требований, предъявляемых к создаваемой комплексной технологии и обусловленных особенностями \textit{интеллектуальных компьютерных систем нового поколения}, разрабатываемых и эксплуатируемых с помощью указанной технологии}
    \end{scnrelfromlist}
    
    
    \scnheader{Создание инфраструктуры, обеспечивающей интенсивное перманентное развитие \textit{Технологии} \textit{комплексной поддержки жизненного цикла интеллектуальных компьютерных систем нового поколения}}
    \begin{scnrelfromlist}{предполагает}
    	\scnitem{обеспечение низкого порога вхождения в \textit{технологию проектирования интеллектуальных компьютерных систем} как для пользователей технологии (то есть разработчиков прикладных или специализированных интеллектуальных компьютерных систем), так и для разработчиков самой технологии}
    	\scnitem{обеспечение высоких темпов развития \textit{технологии} за счет учета опыта разработки различных приложений путем активного привлечения авторов приложений к участию в развитии (совершенствовании) \textit{технологии}}
    \end{scnrelfromlist}
    
    
    \scnheader{Технология комплексной поддержки жизненного цикла интеллектуальных компьютерных систем нового поколения}
    \scnrelfrom{принципы, лежащие в основе}{В основе создания предлагаемой \textbf{\textit{технологии комплексной поддержки жизненного цикла интеллектуальных компьютерных систем нового поколения}} лежат следующие положения}
    \begin{scnindent}
        \begin{scneqtovector}
            \scnitem{Реализация предлагаемой \textit{технологии} разработки и сопровождения \textit{интеллектуальных компьютерных систем нового поколения} в виде \textbf{\textit{интеллектуальной компьютерной метасистемы}}, которая полностью соответствует \textit{стандартам} предлагаемых \textit{интеллектуальных компьютерных систем нового поколения}, разрабатываемым по предлагаемой \textit{технологии}.}
            \begin{scnindent}
                \scntext{пояснение}{В состав такой \textit{интеллектуальной компьютерной метасистемы}, реализующей предлагаемую технологию входит:
                \begin{scnitemize}
                \item  формальное онтологическое описание текущей версии \textit{стандарта интеллектуальных компьютерных систем нового поколения}
        		\item  формальное онтологическое описание текущей версии \textit{методов и средств проектирования, реализации, сопровождения, реинжиниринга и эксплуатации интеллектуальных компьютерных систем нового поколения}
            \end{scnitemize}}
            \end{scnindent}  
            \scnitem{\textbf{\textit{Унификация}} и \textbf{\textit{стандартизация} интеллектуальных компьютерных систем нового поколения}, а также \textit{методов} их \textit{проектирования, реализации, сопровождения, реинжиниринга и эксплуатации}}
            \scnitem{Перманентная эволюция \textbf{\textit{стандарта интеллектуальных компьютерных систем нового поколения}}, а также \textit{методов} их \textit{проектирования, реализации, сопровождения, реинжиниринга и эксплуатации;}}
            \scnitem{\textbf{\textit{Онтологическое проектирование} интеллектуальных компьютерных систем нового поколения}}
            \begin{scnindent}
                \scntext{пояснение}{Онтологическое проектирование интеллектуальных компьютерных систем предполагает:
                \begin{scnitemize}
                    \item  четкое согласование и оперативную формализованную фиксацию (в виде \textit{формальных онтологий}) утвержденного \textit{текущего состояния} иерархической системы всех \textit{понятий}, лежащих в основе перманентно эволюционируемого \textit{стандарта интеллектуальных компьютерных систем нового поколения}, а также в основе каждой разрабатываемой \textit{интеллектуальной компьютерной системы}
    
    		      \item  достаточно полное и оперативное документирование текущего состояния каждого проекта
    
    		      \item  использование \textit{методики проектирования} \textit{"сверху-вниз"{}}
                \end{scnitemize}}
            \end{scnindent}
            \scnitem{\textbf{\textit{Компонентное проектирование} интеллектуальных компьютерных систем нового поколения}, то есть проектирование, ориентированное на сборку \textit{интеллектуальных компьютерных систем} из готовых компонентов на основе постоянно расширяемых библиотек \textit{многократно используемых компонентов}}
            \scnitem{\textbf{\textit{Комплексный характер}} предлагаемой \textit{технологии}}
            \begin{scnindent}
                \scntext{пояснение}{Комплексный характер технологии осуществляет:
                \begin{scnitemize}
                    \item поддержку \textit{проектирования} не только \textit{компонентов} \textit{интеллектуальных компьютерных систем нового поколения} (различных \textit{фрагментов баз знаний, баз знаний} в целом, различных \textit{методов решения задач}, различных \textit{внутренних информационных агентов, решателей задач} в целом, формальных онтологических описаний различных \textit{внешних языков}, \textit{интерфейсов} в целом), но также и \textit{интеллектуальных компьютерных систем} в целом как самостоятельных \textit{объектов проектирования} с учетом специфики тех классов, которым принадлежат проектируемые \textit{интеллектуальных компьютерных системы}
    	          \item поддержку не только \textit{комплексного} \textit{проектирования} \textit{интеллектуальных компьютерных систем} \textit{нового поколения}, но также и поддержку их реализации (сборки, воспроизводства), сопровождения, реинжиниринга в ходе эксплуатации и непосредственно самой эксплуатации
                \end{scnitemize}}
            \end{scnindent}
        \end{scneqtovector}
    \end{scnindent}
    
    
    \scnheader{База знаний Метасистемы OSTIS}
    \begin{scnsubdividing}
        \scnitem{Формальную теорию \textit{ostis-систем}}
        \scnitem{Стандарт \textit{ostis-систем}}
        \begin{scnindent}
            \begin{scnsubdividing}
                \scnitem{Стандарт баз знаний \textit{ostis-систем}}
                \begin{scnindent}
                    \begin{scnsubdividing}
            			\scnitem{Стандарт внутреннего универсального языка смыслового представления знаний в памяти \textit{ostis-систем}}
            			\scnitem{Стандарт внутреннего представления онтологий верхнего уровня в памяти \textit{ostis-систем}}
            			\scnitem{Стандарт представления исходных текстов баз знаний \textit{ostis-систем}}
                    \end{scnsubdividing}
                \end{scnindent}
                \scnitem{Стандарт решателей задач \textit{ostis-систем}}
                \begin{scnindent}
                    \begin{scnsubdividing}
                        \scnitem{Стандарт базового языка программирования \textit{ostis-систем}}
            			\scnitem{Стандарт языков программирования высокого уровня для \textit{ostis-систем}}
            			\scnitem{Стандарт представления искусственных нейронных сетей в памяти \textit{ostis-систем}}
            			\scnitem{Стандарт внутренних информационных агентов в \textit{ostis-систем}}
                    \end{scnsubdividing}
                \end{scnindent}
                \scnitem{Стандарт интерфейсов \textit{ostis-систем}}
                \begin{scnindent}
                    \begin{scnsubdividing}
                        \scnitem{Стандарт внешних языков \textit{ostis-систем}, близких к внутреннему универсальному языку смыслового представления знаний}
                    \end{scnsubdividing}
                \end{scnindent}
            \end{scnsubdividing}
            \scnitem{Стандарт методик и средств поддержки жизненного цикла \textit{ostis-систем}}
            \begin{scnsubdividing}
                \scnitem{Ядро Библиотеки многократно используемых компонентов \textit{ostis-систем} (\textbf{\textit{Библиотеки OSTIS}})}
        		\scnitem{Методики \textit{поддержки жизненного цикла} \textit{ostis-систем} и их компонентов}
        		\scnitem{Инструментальные средства поддержки жизненного цикла \textit{ostis-систем}}
            \end{scnsubdividing}
        \end{scnindent}    
    \end{scnsubdividing}
    
    
    
    
    \scnheader{Технология OSTIS}
    \scnidtf{Open Semantic Technology for Intelligent Systems}
    \scnidtf{Открытая семантическая технология комплексной поддержки жизненного цикла \textbf{\textit{семантически совместимых}} интеллектуальных компьютерных систем нового поколения}
    \scnidtf{Модели, методики, методы и средства комплексной поддержки жизненного цикла интеллектуальных компьютерных систем нового поколения}
    \scnidtf{Теория интеллектуальных компьютерных систем нового поколения и практика компьютерной поддержки их жизненного цикла}
    \scnidtf{Технологический комплекс (моделей, методик, автоматизированных методов и средств), соответствующий интеллектуальным компьютерным системам нового поколения (интероперабельным и семантически совместимым компьютерным системам)}
    \scnidtf{Предлагаемая нами комплексная технология поддержки всех этапов жизненного цикла всех компонентов для всех классов (видов) интеллектуальных компьютерных систем нового поколения при перманентной поддержке их семантической совместимости}
    \begin{scnrelfromlist}{принципы, лежащие в основе}
        \scnitem{комплексный характер технологии, заключающийся в том, что осуществляется поддержкa всех этапов жизненного цикла создаваемых продуктов, для всех компонентов интеллектуальных компьютерных систем нового поколения, для всех классов интеллектуальных компьютерных систем нового поколения}
        \scnitem{обеспечивается перманентная поддержка семантической совместимости между всеми создаваемыми интеллектуальными компьютерными системами нового поколения}
        \scnitem{ориентация на комплексную автоматизацию всего многообразия человеческой деятельности}
        \scnitem{реализация технологии и, соответственно, комплексная автоматизация поддержки жизненного цикла интеллектуальных компьютерных систем нового поколения (со всеми их компонентами и классами) осуществляется в виде семейства интеллектуальных компьютерных систем нового поколения, построенных по той же технологии}
    \end{scnrelfromlist}
    
    
    \scnheader{Стандарт OSTIS}
    \scnidtf{Стандарт Технологии OSTIS}
    \scnidtf{Основная часть базы знаний Метасистемы OSTIS}
    \scnhaselement{Стандарт ostis-систем}
    \scnhaselement{Стандарт методик и средств поддержки жизненного цикла ostis-систем}

\end{scnsubstruct}
\end{SCn}


\begin{SCn}
\scnsegmentheader{Семантически совместимые ostis-системы}

\begin{scnsubstruct}


    \scnheader{база знаний ostis-системы}
    \scnidtf{sc-конструкция, которая в текущий момент времени хранится в памяти ostis-системы}
    \scnsuperset{база знаний индивидуальной ostis-системы}
    \begin{scnindent}
        \scnsuperset{база знаний корпоративной ostis-системы}
    \end{scnindent}
    \scnsuperset{распределенная база знаний коллектива ostis-систем}
    \begin{scnindent}
        \scnhaselement{База знаний Экосистемы OSTIS}
    \end{scnindent}
    \scnnote{Каждый \textit{sc-элемент} (знак, хранимый в базе знаний ostis-системы) по отношению к базе знаний \textit{ostis-системы} считается временной сущностью, поскольку каждый \textit{sc-элемент} в какой-то момент вводится в состав \textit{базы знаний} и в какой-то момент может быть из нее удален, но при этом следует отличать временный характер самого \textit{sc-элемента} от временного или постоянного характера обозначаемой им сущности}


	\scnheader{ostis-система}
	\scnidtf{интеллектуальная компьютерная система нового поколения, построенная по \textit{Технологии OSTIS}}
	\scnidtf{предлагаемое нами уточнение понятия интеллектуальной компьютерной системы нового поколения}
	\begin{scnsubdividing}
		\scnitem{ostis-субъект}
		\begin{scnindent}
			\scnidtf{самостоятельная \textit{ostis-система}}
			\scnidtf{интероперабельная ostis-система}
			\begin{scnsubdividing}
				\scnitem{индивидуальная ostis-система}
				\scnitem{коллективная ostis-система}
			\end{scnsubdividing}
		\end{scnindent}
		\scnitem{встроенная ostis-система}
		\begin{scnindent}
			\scnidtf{\textit{ostis-система}, являющаяся частью некоторой \textit{индивидуальной ostis-системы}}
		\end{scnindent}
	\end{scnsubdividing}

	\scnheader{интеллектуальная компьютерная система}
	\scnsuperset{интероперабельная интеллектуальная компьютерная система}
	\begin{scnindent}
		\scnidtf{интеллектуальная компьютерная система нового поколения}
		\scnsuperset{ostis-субъект}
		\begin{scnindent}
			\scnidtf{предлагаемый нами вариант построения интероперабельных интеллектуальных компьютерных систем}
		\end{scnindent}
	\end{scnindent}

	\scnheader{индивидуальная ostis-система}
	\scnidtf{минимальная самостоятельная \textit{ostis-система}}
	\begin{scnsubdividing}
		\scnitem{персональный ostis-ассистент}
		\begin{scnindent}
			\scnidtf{\textit{ostis-система}, осуществляющая комплексное адаптивное обслуживание конкретного пользователя по \textit{всем} вопросам, касающимся его взаимодействия с любыми другими \textit{ostis-системами}, а также представляющая интересы этого пользователя во всей глобальной сети \textit{ostis-систем}}
		\end{scnindent}
		\scnitem{корпоративная ostis-система}
		\begin{scnindent}
			\scnidtf{\textit{ostis-система}, осуществляющая координацию совместной деятельности \textit{ostis-систем} в рамках соответствующего коллектива \textit{ostis-систем}, осуществляющая мониторинг и реинжиниринг соответствующего множества \textit{ostis-систем} и представляющая интересы этого коллектива в рамках других коллективов \textit{ostis-систем}}
		\end{scnindent}
		\scnitem{индивидуальная ostis-система, не являющаяся ни персональным ostis-ассистентом, ни корпоративной ostis-системой}
	\end{scnsubdividing}

	\scnheader{коллективная ostis-система}
	\scnidtf{многоагентная система, представляющая собой коллектив индивидуальных и коллективных \textit{ostis-систем}, деятельность которого координируется соответствующей корпоративной \textit{ostis-системой}}
	\scntext{примечание}{В состав коллектива \textit{ostis-систем} могут входить индивидуальные \textit{ostis-системы} могут входить индивидуальные \textit{ostis-системы} любого вида --- в том числе, корпоративные \textit{ostis-системы}, представляющие интересы других коллективов \textit{ostis-систем}}

    \scnheader{Метасистема OSTIS}
    \scnidtf{Индивидуальная ostis-система, являющаяся реализацией Ядра Технологии OSTIS}
    \scnidtf{Интеллектуальная компьютерная система нового поколения, построенная по \textit{Технологии OSTIS} и обеспечивающая автоматизацию компьютерную поддержку жизненного цикла интеллектуальной компьютерной системы нового поколения, создаваемых также по \textit{Технологии OSTIS}}
    \scniselement{ostis-система}
    \scnrelto{предлагаемая форма реализации}{Технология OSTIS}
    
    \scnheader{Экосистема OSTIS}
    \scnidtf{Коллективная ostis-система, представляющая собой глобальный коллектив \uline{всех} \textit{ostis-систем}, взаимодействующих между собой и осуществляющих комплексную автоматизацию человеческой деятельности}
    \scnidtf{Глобальная сеть взаимодействующих \textit{ostis-систем}, ориентированная на перманентно расширяемую комплексную автоматизацию самых различных видов и областей человеческой деятельности}
    \scnnote{Это основной продукт \textit{Технологии OSTIS}, который можно рассматривать как предлагаемый нами подход к реализации \textit{Общества 5.0}, \textit{Науки 5.0}, \textit{Индустрии 5.0}}

\end{scnsubstruct}
\end{SCn}


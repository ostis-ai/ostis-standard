\scnsegmentheader{Предметная область и онтология принципов, лежащие в основе онтологических моделей 
    мультимодальных интерфейсов интеллектуальных компьютерных систем нового поколения}

\begin{scnsubstruct}
    \begin{scnrelfromlist}{ключевое понятие}
    	\scnitem{смысловая память}
    	\scnitem{графодинамическая память}
    	\scnitem{ассоциативная память с информационным доступом по образцу произвольного размера и
            конфигурации}
        \scnitem{система ситуационного децентрализованного управления информационными процессами}
    	\scnitem{многоагентная система обработки информации в общей памяти}
    	\scnitem{язык смыслового представления задач}
        \scnitem{универсальный язык смыслового представления знаний}
    	\scnitem{язык смыслового представления методов}
        \begin{scnindent}
    		\scnidtf{интегрированный язык смыслового представления различного вида программ}
    	\end{scnindent}
    	\scnitem{инсерционная программа}
    \end{scnrelfromlist}
   
    \begin{scnrelfromlist}{ключевое знание}
    	\scnitem{Принципы, лежащие в основе решателей задач индивидуальных интеллектуальных компьютерных
            систем нового поколения}
    \end{scnrelfromlist}

    \scnheader{решатель задач интеллектуальных компьютерных систем нового поколения}
    \begin{scnrelfromlist}{предъявляемые требования}
        \scnitem{решатель задач интеллектуальных компьютерных систем нового поколения должен уметь решать 
            интеллектуальные задачи}
            \begin{scnrelfromlist}{виды задач}
                \scnitem{некачественно сформулированная задача}
                \begin{scnindent}
                    \scnidtf{задача, формулировка которой содержит различные не-факторы (неполнота, нечеткость,
                        противоречивость (некорректность) и так далее)}
                \end{scnindent}
                \scnitem{задача, для решения которой, кроме самой формулировки задачи и соответствующего метода ее
                    решения необходима дополнительная, но априори неизвестно какая информация об объектах, указанных
                    в формулировке (постановке) задачи. При этом указанная дополнительная информация
                    может присутствовать, а может и отсутствовать в текущем состоянии базы знаний интеллектуальных
                    компьютерных систем. Кроме того, для некоторых задач может быть задана (указана) та область
                    базы знаний, использования которой достаточно для поиска или генерации (в частности, логического
                    вывода) указанной дополнительной требуемой информации. Такую область базы знаний будем
                    называть областью решения соответствующей задачи}
                \scnitem{задача, для которой соответствующий метод ее решения в текущий момент не известен}
                \begin{scnrelfromlist}{решение}
                    \scnitem{переформулировать задачу, то есть сгенерировать (логически вывести) логически эквивалентную
                        формулировку исходной задачи, для которой метод ее решения в текущий момент является
                        известным}
                    \scnitem{свести исходную задачу к семейству подзадач, для которых методы их решения в текущий
                        момент известны.}
                \end{scnrelfromlist}
            \end{scnrelfromlist}
        \scnitem{процесс решения задач в интеллектуальных компьютерных системах нового поколения реализуется коллективом
            информационных агентов, обрабатывающих базу знаний интеллектуальных компьютерных систем}
        \scnitem{управление информационными процессами в памяти интеллектуальных компьютерных систем нового
            поколения осуществляется децентрализованным образом по принципам ситуационного управления}
    \end{scnrelfromlist}

    \scnheader{ситуационное управление}
    \scnidtf{ситуационно-событийное управление}
    \scntext{пояснение}{управление последовательностью выполнения действий, при котором условием (scnqq{триггером}) инициирования
        указанных действий является:
        \begin{scnitemize}
            \item{возникновение некоторых ситуаций (условий, состояний);}
            \item{и/или возникновение некоторых событий.}
        \end{scnitemize}}
        
    \scnheader{ситуация}
    \scnidtf{структура, описывающая некоторую временно существующую конфигурацию связей между некоторыми
        сущностями}
    \scnidtf{описание временно существующего состояния некоторого фрагмента (некоторой части) некоторо
        динамической системы}

    \scnheader{событие}
    \scnsuperset{возникновение временной сущности}
    \begin{scnindent}
        \scnidtf{появление, рождение, начало существования некоторой временной сущности}
    \end{scnindent}
    \scnsuperset{исчезновение временной сущности}
    \begin{scnindent}
        \scnidtf{прекращение, завершение существования некоторой временной сущности}
    \end{scnindent}
    \scnsuperset{переход от одной ситуации к другой}
    \begin{scnindent}
        \scntext{примечание}{Здесь учитывается не только факт возникновения новой ситуации, но и ее предыстория — то есть та
        ситуация, которая ей непосредственно предшествует. Так, например, реагируя на аномальное значение
        какого-либо параметра, нам важно знать:
        \begin{scnitemize}
            \item{какова динамика изменения этого параметра (увеличивается он или уменьшается и с какой скоростью);}
            \item{какие меры были предприняты ранее для ликвидации этой аномалии.}
        \end{scnitemize}}
    \end{scnindent}

    \scnheader{решатель задач индивидуальной интеллектуальной компьютерной системы нового поколения}
    \begin{scnrelfromlist}{принципы, лежащие в основе}
        \scnitem{смысловое представление обрабатываемых знаний}
        \scnitem{семантически неограниченный ассоциативный доступ к различным фрагментам знаний, хранимым в
            памяти интеллектуальных компьютерных систем нового поколения (доступ по заданному образцу произвольного
            размера и произвольной конфигурации)}
        \scnitem{графодинамический характер обработки знаний в памяти, при котором обработка знаний сводится не
            только к изменению состояния атомарных фрагментов (ячеек) памяти, но и к изменению конфигурации
            связей между этими атомарными фрагментами}
        \scnitem{ситуационное децентрализованное управление процессом обработки знаний, а также процессом организации
            взаимодействия интеллектуальных компьютерных систем с внешней средой}
        \scnitem{использование семантически мощного языка задач, обеспечивающего представление формулировок самых
            различных задач, которые могут решаться либо в рамках памяти интеллектуальной компьютерной
            системы, либо во внешней среде и которые осуществляют инициирование соответствующих процессов
            решения задач}
        \scnitem{многоагентный характер реализации процессов решения инициированных задач, в основе которого лежит
            иерархическая система агентов, каждый из которых активизируются при возникновении в памяти
            интеллектуальной компьютерной системы соответствующий ситуации или соответствующего события}
    \end{scnrelfromlist}
\end{scnsubstruct}

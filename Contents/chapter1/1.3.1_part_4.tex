\begin{SCn}

\scnheader{смысловое представление информации, не являющееся семантической сетью}
\scnnote{Данному уровню смыслового представления информации соответствуют предлагаемые нами универсальные формальные языки SCs-код и SCn-код}

\scnsuperset{SCs-код}
\scnaddlevel{1}
\scniselement{универсальный формальный язык}
\scniselementrole{ключевой знак}{Описание языка линейного представления знаний ostis-систем}
\scnaddlevel{-1}
\scnsuperset{SCn-код}
\scnaddlevel{1}
\scniselement{универсальный формальный язык}
\scniselementrole{ключевой знак}{Описание языка структурированного представления знаний ostis-систем}
\scnaddlevel{-1}

\scnreltovector{принципы, лежащие в основе}{\scnfileitem{В состав \textit{смыслового представления информации, не являющегося семантической сетью} могут входить все уровни иерархии представления информационной конструкции --
\begin{scnitemize}
		\item синтаксически элементарные фрагменты информационной конструкции, из которых строятся простые знаки описываемых сущностей, а также разделители и ограничители;
		\item простые знаки;
		\item выражения;
		\item простые тексты;
		\item сложные тексты;
		\item простые знания;
		\item сложные знания.
\end{scnitemize}};
\scnfileitem{Множество всех описываемых сущностей, \uline{не являющихся связями}, разбивается на два подмножества:
\begin{scnitemize}
	\item каждой сущности, принадлежащей первому подмножеству, \uline{взаимно однозначно} соответствует множество \uline{синтаксически эквивалентных} (синтаксически одинаковых) \textit{простых знаков}, каждый из которых обозначает указанную сущность;
	\item каждой сущности, принадлежащей второму подмножеству, соответствует в общем случае \uline{семейство} множеств, кажо из которых является максимальным множеством синтаксически эквивалентных выражений, обозначающих указанную сущность.
\end{scnitemize}
}
\scnaddlevel{1}
\scntext{следовательно}{Здесь синонимия \textit{простых знаков}, имеющих \uline{разную} синтаксическую структуру, отсутствует, а вот синонимия \textit{выражений}, имеющих разную синтаксическую структуру, вполне возможна. Подчеркнем при этом, что в рамках \textit{смыслового представления информации, не являющегося семантической сетью}, \scnbigspace \textit{знаки} (как \textit{простые знаки}, так и \textit{выражения}), имеющие одинаковую синтаксическую структуру, считаются также и семантически эквивалентными, т.е. обозначающими одну и ту же сущность. Это означает отсутствие омонимии синтаксически эквивалентных знаков}
\scntext{следовательно}{В рамках \textit{смыслового представления информации, не являющегося семантической сетью}, простые знаки, обозначающие \uline{разные} сущности, имеют легко устанавливаемое отличие своих синтаксических структур, а простые знаки, обозначающие одну и ту же сущность имеют легко устанавливаемое сходство своих синтаксических структур. Таким образом, в рамках \textit{смыслового представления информации, не являющегося семантической сетью}, \scnbigspace \uline{дублирование знаков}, т.е. многократное вхождение \textit{знаков} одной и той же сущности, \uline{допускается}}
\scnaddlevel{-1};
\scnfileitem{Связи как вид описываемых сущностей имеют очень важные особенности:
\begin{scnitemize}
	\item каждой описываемой \textit{связи} \uline{однозначно}, а в подавляющем числе случаев и \uline{взаимно однозначно} соответствует \textit{простой текст}, являющийся контекстом (спецификацией) этой \textit{связи};
	\item весьма редки \textit{кратные связи}, т.е. \textit{свзяи}, принадлежащие одному и тому же \textit{отношению} и связывающие одинаковым образом одни и те же \textit{сущности};
	\item довольно редко \textit{связи} являются компонентами других \textit{связей}.
\end{scnitemize}}
\scnaddlevel{1}
\scntext{следовательно}{Для подавляющего числа описываемых \textit{связей} нет никакой необходимости вводить обозначающие их \textit{знаки}, если эти \textit{связи} описываются соответствующими \textit{простыми текстами}. Вместо таких \textit{знаков} можно ввести условные представления этих \textit{связей}, отражающие их вид и направленность. Такие условные представления (изображения) описываемых \textit{связей} можно считать \textit{знаками}, но \textit{знаками}, семантические свойства которых принципиально отличаются от тех \textit{знаков} описываемых \textit{сущностей}, которые мы рассматривали выше. Любые данного вида разные \textit{знаки} описываемых \textit{связей} даже, если, они являются \textit{синтаксически эквивалентными}, т.е. имеют одинаковую структуру, считаются \textit{знаками} \uline{разных} описываемых \textit{связей}. Синонимия таких \textit{знаков} принципиально возможна, но только в том случае, если \textit{простые тексты}, описывающие соответствующие \textit{связи}, будут полностью \uline{продублированы}.}
\scnaddlevel{-1};
\scnfileitem{Для описания связей между описываемыми сущностями в смысловом представлении информации нет необходимости использовать такие приемы естественных языков, как склонения, спряжения, семантическая значимость последовательности знаков.};
\scnfileitem{В случае, если с помощью \textit{простых текстов} необходимо описать контекст (спецификацию) нескольких \uline{кратных} \textit{связей}, все эти \textit{связи} необходимо обозначить \textit{знаками} первого типа -- знаками, \textit{синтаксическая структура} каждого из которых \uline{уникальна.}
Кроме этого, необходимо ввести знак, который обозначает \textit{связь инцидентности} между описываемой \textit{связью} и компонентом этой \textit{связи}, и который относится к числу \textit{знаков} второго типа -- \textit{знаков}, разные экземпляры (разные вхождения) которого считаются обозначениями \uline{разных} \textit{связей}};
\scnfileitem{Для явного указания синонимии двух разных \textit{знаков} первого типа, имеющих разную \textit{синтаксическую структуру}, вводится фиктивная \textit{связь равенства}, которая сама не является описываемой \textit{связью}, а только указывает факт синонимии двух \textit{знаков}, по крайней мере один из которых должен быть \textit{выражением}.};
\scnfileitem{Каждая описываемая \textit{сущность} должна быть специфицирована путем указания типа этой \textit{сущности}. Описываемая \textit{сущность} может быть:
\begin{scnitemize}
	\item \textit{материальной сущностью};
		  \newline
		  \textit{точкой абстрактного пространства};
		  \newline
		  \textit{множеством}:
		  \begin{scnitemizeii}
		  	\item \textit{связью};
		  	\item \textit{классом};
		  	\item \textit{структурой};
		  \end{scnitemizeii}
	\item \textit{реальной сущностью};
		  \newline
		  \textit{вымышленной сущностью};
	\item \textit{константой};
		  \newline
		  \textit{переменной};
	\item \textit{постоянной сущностью};
		  \newline
		  \textit{временной сущностью}:
		  \begin{scnitemizeii}
		  	\item \textit{прошлой сущностью};
		  	\item \textit{настоящей сущностью};
		  	\item \textit{будующей сущностью}.
		  \end{scnitemizeii}
\end{scnitemize}
Кроме того, сам \textit{знак} описываемой сущности может иметь следующий статус:
\begin{scnitemize}
	\item \textit{логически удаленный знак};
	\item \textit{настоящий знак};
	\item \textit{будущий знак}.
\end{scnitemize}};
\scnfileitem{Возможно дублирование информации, т.е. могут присутствовать семантически эквивалентные информационные конструкции, входящие в остав одной информационной конструкции (например, в состав информации, хранимой в памяти одной компьютерной системы). Но при этом есть принципиальная возможность обнаружить такое дублирование информации}
}
\end{SCn}

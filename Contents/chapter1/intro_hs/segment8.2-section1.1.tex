\bigskip
\scnsegmentheader{Комплекс свойств, определяющих уровень социализации кибернетической системы как фактора существенного повышения уровня ее обучаемости, а также фактора существенного повышения качества всех тех многоагентных систем, в состав которых входит данная кибернетическая система}
\scnstartsubstruct

\scnidtf{Комплекс свойств \textit{кибернетической системы}, определяющих необходимые требования к тем \textit{кибернетическим системам}, которые могут входить в состав \textit{синергетических кибернетических систем}}

\scnheader{социализация кибернетической системы}
\scnidtf{способность кибернетической системы взаимодействовать с другими кибернетическими системами в целях создания коллектива кибернетических систем (\textit{многоагентных систем}), уровень качества и, в частности, уровень \textit{интеллекта} которого выше уровня качества каждой \textit{кибернетической системы}, входящей в состав этого коллектива)}
\scnidtf{комплекс способностей кибернетической системы, которые определяют ее вклад в уровень коллективной (социальной) интеллектуальности, т.е. в уровень интеллектуальности того коллектива кибернетических систем, членом которого данная кибернетическая система является(в уровень интеллектуальности соответствующей многоагентной системы)}
\scnidtf{уровень вклада \textit{кибернетической системы} в обеспечение  \textit{интеллекта} тех многоагентных систем, в состав которых эта \textit{кибернетическая система} входит}
\scnidtf{уровень социализации кибернетической системы}
\scnidtf{социализация}
\scnnote{Уровень \textit{интеллекта} коллектива кибернетических систем (\textit{многоагентной системы}) может быть значительно ниже уровня \textit{интеллекта} самого “глупого”\ члена этого коллектива, но может быть и значительно выше уровня \textit{интеллекта} самого “умного”\ члена указанного коллектива. Для того, чтобы количество \textit{интеллектуальных систем} переходило в существенно более интеллектуальное качество коллектива таких систем, все объединяемые в коллектив \textit{интеллектуальные системы} должны иметь высокий уровень \textit{социализации}, что накладывает \uline{дополнительные требования}, предъявляемые к \textit{информации, хранимой в памяти}, а также к \textit{решателям задач} \bigskip \textit{интеллектуальных систем}, объединяемых в коллектив.}
\scnnote{Коллектив \textit{кибернетических систем} может иметь значительно более высокий уровень качества, в том числе, уровень интеллекта, чем уровень качества \textit{кибернетических систем}, являющихся членами этого коллектива. Но так бывает не всегда. Для того, чтобы количество членов коллектива \textit{кибернетической системы} перешло в более высокое качество самого коллектива, члены коллектива должны обладать дополнительными способностями, которые будем называть свойствами \textit{социализации}. Основными такими свойствами являются способность устанавливать и поддерживать достаточный уровень \textit{семантической совместимости} (взаимопонимания) с другими кибернетическими системами и \textit{договороспособность} (способность согласовывать свои действия с другими).}
\scnnote{Целенаправленный обмен информацией между \textit{кибернетическими системами} существенно ускоряет процесс их обучения (процесс накопления знаний и навыков). Следовательно, способность эффективно использовать указанный канал накопления знаний и навыков существенно повышает уровень \textit{обучаемости} \bigskip \textit{кибернетических систем}. В этом смысле можно сказать, что познавательный процесс социален.}
\scnidtf{уровень развития социально значимых качеств кибернетической системы}
\scnnote{Повышение уровня \textit{социализации} \bigskip \textit{кибернетической системы} является, с одной стороны, дополнительным повышением уровня \textit{интеллекта} самой этой \textit{кибернетической системы}, а также фактором повышения уровня \textit{интеллекта} тех коллективов, тех \textit{многоагентных систем}, в состав которых эта \textit{кибернетическая система} входит.}
\scnnote{Переход к \textit{многоагентным системам} не только является важным фактором повышения качества \textit{кибернетических систем}, но также имеет и обратную “сторону модели”\ - появление целого ряда угроз, связанного с возможными целенаправленными вредоносными воздействиями на \textit{многоагентную систему} (со стороны некоторых ее \textit{агентов}), существенно снижающими уровень ее качества. Наличие таких \textit{вредоносных целей} у соответствующих \textit{агентов} свидетельствует о нижайшем уровне \textit{социализации} этих \textit{агентов}.}
\scnidtf{умение согласовывать (синхронизировать) свою деятельность с деятельностью других кибернетических систем в процессе решения задач, требующих коллективных усилий}
\scnidtf{умение участвовать в децентрализованном процессе распределения подзадач некоторой коллективно (распределенно) решаемой задачи между членами заданного коллектива кибернетических систем и умение участвовать в управлении коллективного решения указанной задачи}
\scnaddlevel{1}
\scnnote{Речь идет о децентрализованном асинхронном управлении деятельностью коллектива кибернетических систем}
\scnaddlevel{-1}

\scnidtf{способность и готовность кибернетической системы к координации своей деятельности в рамках коллектива кибернетических систем, в состав которого она входит в целях:
	\begin{scnitemize}
		\item эффективного решения тактических задач, решаемых указанным коллективом;
		\item решения главной стратегической задачи этого коллектива - обеспечения как можно более высокой скорости роста уровня интеллекта указанного коллектива.
	\end{scnitemize}}
\scnaddlevel{1}
\scnnote{Подчеркнем, что повышение уровня интеллекта коллектива кибернетической системы (многоагентной системы) имеет свои особенности: 
	\begin{scnitemize}
		\item во-первых, это забота о семантической совместимости кибернетических систем входящих в состав коллектива;
		\item во-вторых, это переход от виртуальной распределенной базы знаний коллектива к реально поддерживаемым базам знаний и к порталам корпоративных знаний, реализованных в виде индивидуальных кибернетических систем, через которые осуществляются все процессы координации и согласования деятельности соответствующих членов коллектива
	\end{scnitemize}
\scnaddlevel{-1}
	}

\scnrelfromlist{свойство-предпосылка}{
договороспособность кибернетической системы\\
\scnaddlevel{1}
\scntext{часто используемый sc-идентификатор}{договороспособность}
\scnaddlevel{-1}
;социальная ответственность кибернетической системы\\
\scnaddlevel{1}
\scntext{часто используемый sc-идентификатор}{социальная ответственность}
\scnaddlevel{-1}
;социальная активность кибернетической системы\\
\scnaddlevel{1}
\scntext{часто используемый sc-идентификатор}{социальная активность}
\scnaddlevel{-1}}

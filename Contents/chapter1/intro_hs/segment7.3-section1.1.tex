\bigskip
\scnsegmentheader{Комплекс свойств, определяющих обучаемость кибернетических систем по уровню обучаемости различных их компонентов}
\scneqtoset{
способность кибернетической системы к повышению качества информации хранимой в её памяти;
способность кибернетической системы к повышению качества своего решателя задач;
способность кибернетической системы к повышению качества своей физической оболочки
}

\scnheader{способность кибернетической системы к повышению качества информации, хранимой в её памяти}
\scnidtf{способность кибернетической системы к постоянному пополнению и совершенствованию информации, хранимой в её памяти, по всевозможным направлениям и, в первую очередь, в направлении повышения уровня адекватности (корректности) и полноты описания своей внешней среды и своей физической оболочки}
\scnrelfromlist{свойство-предпосылка}{
семантическая гибкость информации, хранимой в памяти кибернетической системы;
стратифицированность информации, хранимой в памяти кибернетической системы;
способность кибернетической системы к повышению уровня структуризации информации, хранимой в памяти кибернетической системы;
способность кибернетической системы к анализу качества информации, хранимой в её памяти;
способность кибернетической системы к устранению противоречий, обнаруженных в информации, хранимой в её памяти;
способность кибернетической системы к устранению информационных дыр, обнаруженных в информации, хранимой в её памяти;
способность кибернетической системы к удалению информационного мусора, обнаруженного в информации, хранимой в её памяти;
способность кибернетической системы к погружению новых \textit{знаний} в состав информации, хранимой в её памяти;
способность кибернетической системы к обнаружению сходств в знаниях, хранимых в её памяти;
способность кибернетической системы к конвергенции знаний, хранимых в её памяти;
способность кибернетической системы к интеграции знаний, хранимых в её памяти;
способность кибернетической системы к обобщениям и формированию новых понятий;
способность кибернетической системы к генерации гипотез и обнаружению закономерностей в информации, хранимой в её памяти;
способность кибернетической системы к обоснованию или опровержению знаний, хранимых в её памяти;
способность кибернетической системы к экспериментальному подтверждению или опровержению гипотез о свойствах динамических систем с помощью имитационных моделей этих систем;
способность кибернетической системы к коррекции теорий, хранимых в её памяти}

\scnheader{семантическая гибкость информации, хранимой в памяти кибернетической системы}
\scnidtf{гибкость информации, хранимой в памяти кибернетической системы, при её обработке на семантическом уровне}
\scnidtf{гибкость возможных действий (операций), выполняемых кибернетической системой над информации, хранимой в её памяти, и осуществляемых на семантическом (осмысленном) уровне представления этой информации}
\scnidtf{трудоёмкость содержательного редактирования информации, хранимой в памяти кибернетической системы (поиска, удаления, вставки, замены различных фрагментов информации), при соблюдении семантической целостности и корректности всей редактируемой информации}
\scnnote{обработка информации на семантическом уровне предполагает такие операции над хранимой информации, как:
\begin{scnitemize}
		\item замена имени некоторой сущности 
		\item поиск связи заданного вида между знаками заданных сущностей и корректировка этой связи
		\item поиск семантической окрестности знака заданной сущности, то есть поиск всех известных связей, инцидентный этому знаку и, соответственно, всех смежных ему знаков
		\item поиск фрагмента хранимой информации, релевантного заданному семантическому образцу -  конфигурации знаков сущностей и связей между ними
		\item удаление или генерация (порождение) связи между заданными знаками
	\end{scnitemize}}
\scnnote{Все операции семантического уровня обработки информации рассматривают обрабатываемую информацию на абстрактном уровне знаков описываемых сущностей и знаков связей между описываемыми сущностями. При этом указанные связи рассматриваются как частный вид описываемых (и, соответственно, обозначаемых) сущностей.}
\scnidtf{простота и многообразие редактирования информации, хранимой в памяти кибернетической системы}
\scnidtf{простота и многообразие внесения изменений в информацию, хранимую в памяти кибернетической системы}
\scnexplanation{\textit{Гибкость обработки информации, хранимой в памяти кибернетической системы}, определяется не столько трудоемкостью непосредственно самой операции редактирования, сколько теме дополнительными действиями, которые являются обязательными последствиями каждой такой операции редактирования. Так, например, изменение имени какой-либо описываемой сущности требует внесения этого изменения во всех местах, где это имя упоминается, удаление какой-либо связи между известными описываемыми сущностями требует внесения этого изменения везде, где удаляемая связь упоминается.}

\scnheader{стратифицированность информации, хранимой в памяти кибернетической системы}
\scnidtf{логико-семантическая стратифицированность информации, хранимой в памяти кибернетической системы}
\scnrelfrom{свойство-предпосылка}{структуризация информации, хранимой в памяти кибернетической системы}
\scnrelfrom{свойство-предпосылка}{качество метаязыковых средств представления информации, хранимой в памяти кибернетической системы
\scnidtf{уровень развития метаязыковых средств кодирования (внутреннего представления) информации, хранимой в памяти кибернетической системы}}

\scnheader{способность кибернетической системы к повышению уровня структуризации информации, хранимой в памяти указанной системы}
\scnrelboth{следует отличать}{структурированность информации, хранимой в памяти кибернетической системы
\scnidtf{уровень структуризации информации, хранимой в памяти кибернетической системы}
\scnnote{Качественная структуризация информации, хранимой в памяти кибернетической системы, то есть качественное "разложение"\ этой информации по семантическим "полочкам"\ существенно упрощает и, следовательно, ускоряет повышение качества самой этой информации.}
\scnrelboth{следует отличать}{структуризация информации, хранимой в памяти кибернетической системы\\
\scnsubset{действие, выполняемое кибернетической системой в своей памяти}\\
\scnaddlevel{1}
\scnsubset{процесс}\\
\scnaddlevel{-1}
}}

\scnheader{способность кибернетической системы к анализу качества информации, хранимой в её памяти}
\scnidtf{способность кибернетической системы к анализу информации, хранимой в собственной памяти, для последующего повышения качества этой информации}
\scnrelto{частное свойство}{способность кибернетической системы к рефлексии\\
\scnnote{Рефлексия кибернетической системы, то есть анализ собственного качества, включает в себя не только анализ качества информации, хранимой в её памяти, но и анализ собственной деятельности как во внешней среде, так и в собственной памяти. При этом анализ собственной деятельности сводится к анализу описания этой деятельности, представленного в собственной памяти.}
}
\scnrelfromlist{свойство-предпосылка}{
качество метаязыковых средств описания в памяти кибернетической системы качества информации, хранимой в её памяти;
способность кибернетической системы к обнаружению противоречий в информации, хранимой в её памяти\\
\scnrelfromlist{частное свойство}{
способность кибернетической системы к обнаружению пар синонимичных знаков, входящих в состав информации, хранимой в её памяти;
способность кибернетической системы к обнаружению семантически эквивалентных фрагментов, входящих в состав информации, хранимой в её памяти;
способность кибернетической системой к обнаружению омонимичных знаков в информации, хранимой в её памяти;};
способность кибернетической системы к обнаружению информационных дыр в информации, хранимой в её памяти;
способность кибернетической системой к обнаружению информационного мусора в информации, хранимой в её памяти}

\scnheader{способность кибернетической системы к устранению противоречий, обнаруженных в информации, хранимой в её памяти}
\scnrelfromlist{частное свойство}{
способность кибернетической системы к устранению синонимии знаков, входящих в состав информации, хранимой в памяти указанной системы;
способность кибернетической системы к устранению семантической эквивалентности фрагментов, входящих в состав информации, хранимой в памяти указанной системы;
способность кибернетической системы к устранению омонимичных знаков, входящих в состав информации, хранимой в памяти указанной системы;
способность кибернетической системы к устранению противоречий, обнаруженных в информации, хранимой в памяти указанной системы, и не являющихся обнаруженной синонимией, семантической эквивалентностью или омонимией}

\scnheader{способность кибернетической системы к устранению семантической эквивалентности фрагментов, входящих в состав информации, хранимой в памяти указанной системы}
\scnidtf{способность кибернетической системы к устранению дублирования информации в рамках памяти указанной системы}
\scnheaderlocal{следует отличать*}
\scnhaselementset{
семантическая эквивалентность*
\scnaddlevel{1}\scnidtf{эквивалентность информационных конструкций по смыслу (содержанию)*}
\scnaddlevel{-1}
;синтаксическая эквивалентность* 
\scnaddlevel{1}
\scnidtf{эквивалентность информационных конструкций по форме*}
\scnaddlevel{-1}
;логическая эквивалентность*
\scnaddlevel{1}
\scnidtf{пары информационных конструкций, первая из которых логически следует из второй и наоборот*}
\scnnote{Если с семантической эквивалентности в памяти кибернетической системы можно и нужно бороться, то без логической эквивалентности обойтись трудно (как минимум из-за необходимости вводить определяемые понятия и, соответственно, формулировать определения). Тем не менее, логической эквивалентностью и, в частности, расширением числа определяемых понятий увлекаться не следует. Так, например, если определение нового понятия не является громоздким (в частности, понятия, являющегося теоретико-множественным объединением или пересечением ранее введенных понятий), то явно вводить это новое понятие не следует.}
\scnaddlevel{-1}
}

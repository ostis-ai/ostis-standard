\bigskip

\scnheader{ограниченность обучения кибернетической системы}
\scnexplanation{Данное свойство определяет границу между теми знаниями и навыками, которые соответствующая \textit{кибернетическая система} принципиально может приобрести, и теми знаниями и навыками, которые указанная кибернетическая система не сможет приобрести никогда. Данное свойство определяет максимальный уровень потенциальных возможностей соответствующей кибернетической системы. Очевидно, что максимальная степень отсутствия ограничений в приобретении новых знаний и навыков -- это полное отсутствие ограничений, т.е. полная универсальность возможностей соответствующих кибернетических систем, которые всё могут познать и всё могут сотворить.}
\scnidtf{максимум того, чему кибернетическая система может обучиться}
\scnidtf{максимальная перспектива обучения кибернетической системы}
\scnidtf{максимальный уровень качества, который кибернетическая система может достичь в процессе обучения}
\scnrelfromlist{частное свойство}{максимальный объём знаний, которые кибернетическая система может приобрести в процессе обучения;максимальный объём навыков, которые кибернетическая система может приобрести в процессе обучения}

\scnheader{максимальный объём знаний, которые кибернетическая система может приобрести в процессе обучения}
\scnidtf{граница приобретаемых знаний, за пределы которой кибернетическая система принципиально не может перейти в процессе своего обучения}
\scnidtf{максимум того, чему можно научить соответствующую кибернетическую систему}
\scnidtf{максимальный объём знаний, которые кибернетическая система принципиально может приобрести}
\scnrelto{свойство-предпосылка}{обучаемость}
\scnnote{чем больше \textit{максимальный объём знаний, которые кибернетическая система принципиально может приобрести}, тем выше уровень \textit{обучаемости} кибернетической системы}

\scnheader{познавательная активность кибернетической системы}
\scnidtf{познавательная мотивированность}
\scnidtf{познавательная пассионарность}
\scnidtf{любознательность}
\scnidtf{активность и самостоятельность в приобретении новых знаний и навыков}
\scnidtf{стремление, активная целевая установка к постоянному совершенствованию (повышению качества) и пополнению собственной базы знаний}
\scnnote{Следует отличать
	\begin{scnitemize}
		\item способность (возможность) приобретать новые знания и навыки и совершенствовать приобретенные знания и навыки
		\item от желания (стремления) это делать.
	\end{scnitemize}}
\scnnote{желание (целевая установка) научиться решать те или иные задачи может быть сформулировано кибернетической системой либо самостоятельно, либо извне (некоторым учителем).}

\scnrelfromlist{частное свойство}{активность в изучении внешней среды;активность в анализе качества информации, хранимой в собственной памяти;активность в анализе собственных действий и действий других кибернетических систем}
\scnrelfromlist{свойство-предпосылка}{способность кибернетической системы к синтезу познавательных целей и процедур;способность кибернетической системы к самоорганизации собственного обучения;способность кибернетической системы к экспериментальным действиям}

\scnheader{способность кибернетической системы к синтезу познавательных целей и процедур}
\scnidtf{способность планировать своё обучение и управлять процессом обучения}
\scnidtf{умение задавать вопросы или целенаправленные последовательности вопросов самому себе или другим субъектам как важнейший фактор обучаемости}
\scnidtf{способность генерировать (формулировать, задавать) вопросы, адресуемые либо самому себе, либо некоторому внешнему источнику знаний и направленные на повышение качества собственных знаний и навыков}
\scnidtf{способность генерировать четкую спецификацию своей информационной потребности}
\scnidtf{способность кибернетической системы четко формулировать то, что она не знает (в частности, не умеет), но хотела бы знать и уметь}
\scnidtf{способность к формированию спецификаций информационных баз в своих знаниях}
\scnidtf{способность кибернетической системы самостоятельно генерировать цели на приобретение знаний и навыков, обеспечивающих решение различных классов задач}

\scnheader{способность кибернетической системы к самоорганизации собственного обучения}
\scnidtf{способность осуществлять управление своим обучением}
\scnidtf{способность кибернетической системы самой выполнять роль своего учителя, организующего процесс своего обучения}

\scnheader{способность кибернетической системы к экспериментальным действиям}
\scnidtf{способность к отклонениям от составленных планов своих действий для повышения качества результата или сохранении целенаправленности этих действий}
\scnidtf{способность к экспромтам и импровизации}

\scnheader{способность кибернетической системы к самосохранению}
\scnidtf{способность кибернетической системы к выявлению и устранению угроз, направленных на снижение её качества и даже на её уничтожение, что означает полную потерю необходимого качества}
\scnidtf{уровень безопасности (защищенности) кибернетической системы}
\scnexplanation{Данное свойство кибернетических систем является необходимым фактором высокого уровня обучаемости кибернетических систем. Чем выше уровень безопасности кибернетической системы, тем выше её уровень обучаемости.}
\scnidtf{способность кибернетической системы к обеспечению собственной безопасности}
\scnrelfromlist{свойство-предпосылка}{способность кибернетической системы анализировать смысл задач, инициированных извне, и отказываться от решения вредных задач}
\scnaddlevel{1}
\scntext{эпиграф}{Прежде, чем выполнять приказ, подумай}
\scnexplanation{Примером вредной задачи для \textit{ostis-системы} является запрос всех хранимых в памяти \textit{sc-элементов}}
\scnexplanation{Подчеркнем, что в современных компьютерных системах и интеллектуальных компьютерных системах подходы к обеспечению их информационной безопасности имеют принципиальные отличия, связанные, прежде всего с тем интеллектуальные компьютерные системы обладают более мощными средствами семантического и контекстного анализа приобретаемой информации.}

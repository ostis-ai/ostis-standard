\scnsegmentheaderbeginning{Комплекс свойств, определяющих качество информации, хранимой в памяти кибернетической системы}

\scnstartsubstruct

\scnheader{информация}
\scnidtf{информационная конструкция}
\scnidtf{информационная модель, состоящая из некоторого множества различных \textit{знаков}, обозначающих моделируемые (описываемые) \textit{сущности} любого вида и, в частности, \textit{знаков}, обозначающих различного вида \textit{связи} между \textit{знаками} описываемых \textit{сущностей} (такие \textit{связи} чаще всего являются отражениями (моделями) \textit{связей} между \textit{сущностями}, которые обозначаются связываемыми \textit{знаками})}
\scnaddlevel{1}
\scnnote{Подчеркнем, что \textit{связи} между \textit{знаками} описываемых \textit{сущностей} сами также могут быть описываемыми \textit{сущностями}, но для этого указанные \textit{связи} в рамках информационной модели должны быть представлены своими \textit{знаками}. Не все \textit{связи} между \textit{знаками} являются описываемыми \textit{сущностями}. Такими неописываемыми связями являются связи инцидентности знаков.} 
\scnaddlevel{-1}
\scnidtf{конфигурация знаков}
\scnidtf{знаковая конструкция} 
\scnidtf{текст}
\scnidtf{описание (отражение) некоторого множества (1) первичных сущностей, (2) понятий, (3) связей между ними, (4) связей между связями, (5) фрагментов данного описания, (6) связей между этими фрагментами}
\scnsuperset{дискретная информационная конструкция} 
\scnaddlevel{1}
\scnidtf{информационная конструкция, у которой все входящие в неё знаки имеют чёткие границы}

\scnsuperset{дискретная информационная конструкция, у которой входящие в неё знаки имеют \uline{условную} структуру}
\scnaddlevel{-1}  

\scnsubdividing {внутренняя информационная конструкция
\scnaddlevel{1}
\scnidtf{информационная конструкция, хранимая в памяти некоторой кибернетической \textit{системы}, и непосредственно интерпретируемая (понимаемая) решателем задач этой системы}
\scnaddlevel{-1}
;внешняя информационная конструкция   
\scnaddlevel{1}
\scnidtf{информационная конструкция, представленная на каком-либо внешнем носителе или в памяти другой кибернетической системы} 
\scnaddlevel{-1}
;файл   
\scnaddlevel{1}
\scnidtf{первичный электронный образ некоторой внешней информационной конструкции} 
\scnaddlevel{-1}}
\scnexplanation {Начало раздела \ref{intro_lang}}   
\scnidtf{информационная модель} 
\scnidtf{информационная модель (отражение, описание) некоторого множества связей между некоторым описываемыми (рассматриваемыми, исследуемыми, изучаемыми) сущностями}
\scntext{определение}{Множество всевозможных информационных конструкций (понятие информационной конструкции) представляет собой множество, на котором задано 
	\begin{scnitemize}
	\item Отношение \uline{синтаксической} эквивалентности и, соответственно, семейство классов синтаксической эквивалентности информационных конструкций
	\item Отношение \uline{семантической} эквивалентности и, ответственно, семейство классов семантической эквивалентности информационных конструкций 
	\item Отношение \uline{логической} эквивалентности и, соответственно, семейство классов логической эквивалентности информационных конструкций.
\end{scnitemize} 
При этом можно говорить об инварианте каждого класса синтаксически эквивалентных информационных конструкций, об инварианте каждого класса семантически эквивалентных информационных конструкций и об инварианте каждого класса логически эквивалентных информационных конструкций 
синтаксически эквивалентные информационные конструкции могут отличаться вариантами изображения букв (различным почерком, разными шрифтами), вариантами "разрезания"{} текста на страницы и на строчки.
Семантически эквивалентные информационные конструкции могут отличаться разными именами, обозначающими одни и те же сущности, разным порядком размещения этих имён.}
 
\scnheader{денотационная семантика информационной конструкции}\\
\scnexplanation{Каждая информационная конструкция имеет денотационную семантику, описывающую то, как связаны входящие в информационную конструкцию знаки с соответствующими им денотатами (т.е. сущностями, обозначаемыми этими знаками.}

\scnheader{сенсорная информация}\\
\scnsubset{информация} \\
\scnidtf{первичная информация, приобретаемая кибернетической системы с помощью её сенсоров (рецепторов)}
\scnidtf{первичная информация}
\scnnote{Подчеркнем, что \textit{сенсорная информация} \scnbigspace \textit{кибернетической системы} с точки зрения её \textit{денотационной семантики} является простейшим видом \textit{знаковой конструкции}, в которой \textit{внешняя среда} \scnbigspace \textit{кибернетической системы} описывается
\begin{scnitemize}
	\item путём задания параметрического пространства (множество параметров, признаков, \textit{свойства}, характеристик), с помощью которого описываются состояние элементарных (атомарных) фрагментов \textit{внешней среды}, которые непосредственно являются смежными (соприкасаются с) чувствительными поверхностями \textit{сенсоров кибернетической системы}; 
	\item путём пространственной декомпозиции наблюдаемой \textit{внешней среды} с выделением указанных выше элементарных фрагментов этой среды (элементарных с "точки зрения"{} \textit{сенсоров кибернетической системы}) и с явным описанием пространственных связей между указанными элементарными фрагментами (эти связи соответствует пространственным связям между сенсорами);
	\item путём темпоральной декомпозиции наблюдаемой \textit{внешней среды}, которая предполагает фиксацию моментов времени для каждого события по изменению состояния измеряемого параметра каждого элементарного фрагмента наблюдаемой \textit{внешней среды} 
\end{scnitemize}}

\scnnote{Качество (в частности, информативность) \textit{сенсорной информации} обеспечивается:
\begin{scnitemize}
\item качеством используемого параметрического пространства 
\begin{scnitemizeii}
\item многообразием видов \textit{сенсоров}, т.е. многообразием параметров (свойств), с помощью которых описывается внешняя среда
\item информативностью каждого из указанных параметров 
\item целостностью (полнотой, достаточностью) всего набора рассматриваемых параметров 
\item отсутствием избыточности в наборе этих параметров 
\end{scnitemizeii}
\item общим количеством сенсоров и количеством сенсоров, соответствующих каждому параметру
\item способностью кибернетической системы перемещать сенсоры в пространстве 
\end{scnitemize}
}
\scnnote{\textit{сенсорная информация} обеспечивает формирование первичного описания состояния и динамики изменения не только \textit{внешней среды кибернетической системы}, но также и её физической оболочки, которую можно рассматривать как часть всей \textbf{\textit{физической среды кибернетической системы}}, противопоставляя такую \textit{физическую среду кибернетической системы} её внутренней (информационной, \uline{абстрактной}) среде, в которой хранится и обрабатывается \textit{информация}, используемая \textit{кибернетической системой}. Указанную абстрактную внутреннюю среду кибернетической системы будем называть \textbf{\textit{абстрактной памятью кибернетической системы}}.}

\scnheader{язык}  
\scnidtf{множество информационных конструкции, построенных по общим синтаксическим и семантическим правилам}
\scnsuperset{внутренний язык кибернетической системы} 
\scnaddlevel{1}
\scnidtf{язык, используемый кибернетической системой для представления информации, хранимой в её памяти}
\scnaddlevel{-1}

\scnheader{информация, хранимая в памяти кибернетической системы} 
\scnidtf{совокупность \uline{всей} информации, хранимой в памяти кибернетической системы}
\scnsubset{информация} 

\scnheader{качество информации, хранимой в памяти кибернетической системы} 
\scnidtf{качество знаний, приобретенных кибернетической системой к текущему моменту}
\scnidtf{уровень качества хранимой информации} 
\scnidtf{качество информационной модели среды кибернетической системы, хранимой в её памяти}
\scnidtf{уровень качества хранимых в памяти кибернетической системы внутренней информационной модели среды существования (жизнедеятельности) этой кибернетической системы} 
\scnidtf{интегральное качество знаний, накопленных кибернетической системой к текущему моменту} 
\scnidtf{степень приближения информации, хранимой в памяти кибернетической системы к качественной информационной модели той среды, в которой существует кибернетическая система, к систематизированной базе знаний, описывающей все свойства этой среды, необходимые для функционирования этой кибернетической системы}
\scnidtf{качество хранимой в памяти кибернетической системы информационной модели среды жизнедеятельности этой системы}
\scnnote{Качество информационной модели среды "обитания"{} кибернетической системы, в частности, определяется 
\begin{scnitemize}
	\item корректностью этой модели (отсутствием в ней ошибок);
	\item адекватностью этой модели;
	\item полнотой -- достаточностью находящейся в ней информации для эффективного функционирования кибернетической системы;
	\item структурированностью, систематизированностью.
\end{scnitemize}
Важнейшим этапом эволюции информационной модели среды кибернетической системы является переход от недостаточно полной и несистематизированной информационные модели среды к \textit{базе знаний}. Именно поэтому важнейшим этапом повышения уровня интеллектуальности компьютерной систем является переход от традиционных компьютерных систем к компьютерным системам, основанным на знаниях.}
\scnrelfrom{комплекс свойств-предпосылок}{не-фактор} 
\scnrelfromlist{свойство-предпосылка}{семантическая мощность языка представления информации в памяти кибернетической системы; 
объём информации, в память кибернетической системы; 
степень конвергенции и интеграции различного вида знаний, хранимых в памяти кибернетической системы; 
стратифицированность информации, хранимой в памяти кибернетической системы;
простота и локальность выполнения семантически целостных операций над информацией, хранимой в памяти кибернетической системы}

\scnheader{не-фактор}
\scnidtf{группа семантических свойств, определяющих качество информации, хранимой в памяти кибернетической системы}
\scneqtoset{корректность/некорректность информации, хранимой в памяти кибернетической системы;
однозначность/неоднозначность информации, хранимой в памяти кибернетической системы;
целостность/нецелостность информации, хранимой в памяти кибернетической системы;
чистота/загрязненность информации, хранимой в памяти кибернетической системы;
достоверность/недостоверность информации, хранимой в памяти кибернетической системы;
точность/неточность информации, хранимой в памяти кибернетической системы;
четкость/нечеткость информации, хранимой в памяти кибернетической системы;
определенность/недоопределенность информации, хранимой в памяти кибернетической системы}
\scnexplanation{Ярушкина Н.Г. ред.2007кн-НечетГС-стр.10-28}
\scnaddlevel{1}
\scnrelto{цитата}{Ярушкина Н.Г. ред.2007кн-НечетГС}
\scnaddlevel{-1}

\scnauthorcomment{Дооформить библиографию}

\scnheader{корректность/некорректность информации, хранимой в памяти кибернетической системы}
\scnidtf{уровень адекватности хранимой информации той среде, в которой существует кибернетическая система и информационной моделью которой эта хранимая информация является}

\scnheader{непротиворечивость/противоречивость информации, хранимой в памяти кибернетической системы}
\scnidtf{уровень присутствия в хранимой информации различного вида противоречий и, в частности, ошибок}

\scnheader{противоречие*}
\scnidtf{пара противоречащих друг другу фрагментов информации, хранимой в памяти кибернетической системы*}
\scnnote{Чаще всего противоречащими друг другу информационными фрагментами являются:
\begin{scnitemize}
\item явно представленная в памяти некоторая закономерность (некоторое правило)
\item информационный фрагмент, не соответствующий (противоречащий) указанной закономерности
\end{scnitemize}
\bigskip
В этом случае некорректность может присутствовать:
\begin{scnitemize}
\item либо в информационном фрагменте, который противоречит указанной закономерности;
\item либо в самой этой закономерности;
\item либо и там и там.
\end{scnitemize}
}

\scnheader{информационная ошибка}
\scnidtfdef{противоречие, заключающееся в нарушении некоторой закономерности (некоторого правила), которая не подвергается сомнению}

\scnheader{информационная ошибка}
\scnnote{Ошибки (ошибочные фрагменты) в хранимой информации могут быть синтаксическими и семантическими, противоречащими некоторым правилам (закономерностям), которые явно в памяти могут быть не представлены и считаются априори истинными.}

\scnheader{полнота/неполнота информации, хранимой в памяти кибернетической системы}
\scnidtf{уровень того, насколько информация, хранимая в памяти кибернетической системы, описывает среду существования этой системы и используемые ею методы решения задач достаточно полно (достаточно детально) для того, чтобы кибернетическая система могла действительно решать все множество соответствующих ей задач}
\scnidtf{уровень соответствия хранимой информации объёму задач (действий), которые соответствующая кибернетическая система желает уметь решать (выполнять)}
\scnidtf{степень достаточности информации, хранимой в памяти кибернетической системы, для достижения целей этой системы, для выполнения своих "обязанностей"}
\scnnote{Чем полнее информация, хранимая в памяти кибернетической системы, чем полнее \uline{информационное обеспечение деятельности этой системы} это системы, тем эффективнее (качественнее) сама эта деятельность.}
\scnrelfromlist{свойство-предпосылка}{многообразие видов знаний, хранимых в памяти кибернетической системы;
структурированность информации, хранимой в памяти кибернетической системы}

\scnheader{однозначность/неоднозначность информации, хранимой в кибернетической системе}
\scnrelfromlist{свойство-предпосылка}{многообразие форм дублирования информации, хранимой в памяти кибернетической системы;
частота дублирования информации, хранимой в памяти кибернетической системы}

\scnheader{целостность/нецелостность информации, хранимой в памяти кибернетической системы}
\scnidtf{уровень содержательной информативности информации, хранимой в памяти кибернетической системы}
\scnidtf{уровень того, насколько содержательно (семантически) \uline{связной} является информация, хранимая в памяти кибернетической системы, насколько полно специфицированы \uline{все} описываемые в памяти сущности (путём описания необходимого набора связей этих сущностей с другими описываемыми сущностями), насколько редко или часто в рамках хранимой информации встречаются \textit{информационные дыры}, соответствующие явной недостаточности некоторых спецификаций}
\scnidtf{известность/неизвестность информации, хранимой в памяти кибернетической системы}
\scnidtf{многообразие форм и частота присутствия \textit{информационных дыр} в информации, хранимой в памяти кибернетической системы}

\scnheader{информационная дыра в информации, хранимой в памяти кибернетической системы}
\scnidtf{информация, отсутствие которой в памяти кибернетической системы существенно усложняет деятельность этой системы}
\scnnote{Примерами информационных дыр являются:
\begin{scnitemize}
	\item отсутствующий метод решения часто встречающихся задач;
	\item отсутствующее определение используемого определяемого понятия;
	\item недостаточно подробная спецификация часто рассматриваемых сущностей
\end{scnitemize}}

\scnheader{чистота/загрязненность информации, хранимой в памяти кибернетической системы}
\scnidtf{многообразие форм и общее количество информационного мусора, входящего в состав информации, хранимой в памяти кибернетической системы}

\scnheader{информационный мусор, входящий в состав информации, хранимой в памяти кибернетической системы}
\scnidtf{информационный фрагмент, входящий в состав информации, хранимой в памяти кибернетической системы, удаление которого существенно \uline{не} усложнит деятельность кибернетической системы}
\scnnote{Примерами информационного мусора являются:
\begin{scnitemize}
	\item{информация, которая нечасто востребована, но при необходимости может быть легко логически выведена}
	\item{информация, актуальность которой истекла}
\end{scnitemize}}

\scnheader{семантическая мощность языка представления информации в памяти кибернетической системы}
\scnidtf{семантическая мощность внутреннего языка кибернетической системы}
\scnrelfrom{свойство-предпосылка}{гибридность информации, хранимой в памяти кибернетической системы}
\scnnote{Универсальность внутреннего языка кибернетической системы является важнейшим фактором её интеллектуальности}

\scnheader{универсальный язык}
\scnidtf{язык, информационные конструкции которого могут представить (описать) \uline{любую} конфигурацию \uline{любых} связей между \uline{любыми} сущностями}

\scnheader{гибридность информации, хранимой в памяти кибернетической системы}
\scnrelfromlist{свойство-предпосылка}{многообразие видов знаний, хранимых в памяти кибернетической системы;
степень конвергенции и интеграции различного вида знаний, хранимых в памяти кибернетической системы}

\scnheader{многообразие видов знаний, хранимых в памяти кибернетической системы}
\scnrelfromlist{частное свойство}{рефлексивность информации, хранимой в памяти кибернетической системы
\scnaddlevel{1}
\scnidtf{многообразие видов метаинформации (метазнаний), хранимых в памяти кибернетической системы}
\scnaddlevel{-1};
многообразие моделей решения задач, используемых кибернетической системой;
многообразие видов целей, анализируемых или синтезируемых кибернетической системой;
многообразие планов решения задач, решаемых кибернетической системой;
многообразие протоколов решения задач, решаемых кибернетической системой
}

\scnheader{объем информации, хранимой в памяти кибернетической системы}
\scnidtf{объем знаний, приобретенных кибернетической системой к текущему моменту}
\scnidtf{содержательная совокупность всех знаний, хранимых в текущий момент в памяти кибернетической системы}
\scnnote{Чем больше кибернетическая система знает, тем при прочих равных условиях выше уровень её качества}

\bigskip
\scnfragmentcaption

\scnheader{степень конвергенции и интеграции различного вида знаний, хранимых в памяти кибернетической системы} 
\scnidtf{уровень "бесшовной"{} интеграции различного вида знаний кибернетической системы}
\scnnote{Максимальный уровень конвергенции и интеграции знаний (в том числе,  и знаний различного вида) предполагает:
\begin{scnitemize} 
	\item использование универсального базового языка, по отношению к которому всем используемым видам знаний соответствуют специализированные языки, являющиеся подъязыками указанного базового языка
	\item построение четкой иерархии указанных специализированных языков по принципу "язык-подъязык"{}
	\item явное введение семейства отношений, заданных на множестве различных знаний и, в том числе, связывающих знания различного вида
\end{scnitemize}
}
\scnrelfrom{свойство-предпосылка}{уровень формализованности информации, хранимой в памяти кибернетической системы}
\scnheader{уровень формализованности информации, хранимой в памяти кибернетической системы}
\scnidtf{степень приближения информации, хранимой в памяти кибернетической системы, к максимально простой и компактной форме представления информационной модели некоторого множества описываемых сущностей, которая является отражением определенной конфигурации связей между указанными сущностями}
\scnnote{Высшим уровнем формализации информации, хранимой в памяти кибернетической системы, является смысловое представление информации в форме семантических сетей. Смотрите Раздел ``\textit{Предметная область и онтология семантических сетей, семантических языков и семантических моделей баз знаний}''.}
\scnrelboth{следует отличать}{формализация*}
	\scnaddlevel{1}
		\scnidtf{Бинарное ориентированное отношение, каждая пара которого связывает некоторую информационную конструкцию с другой информационной конструкцией, которая семантически эквивалентна первой, но имеет более высокий уровень формализованности}
		\scnnote{Приобретение навыков формального представления информации не является простой проблемой даже для человека. По сути совокупность таких навыков -- это основа математической культуры, культуры точного изложения своих соображений. Некоторые примеры, иллюстрирующие нетривиальность проблемы смотрите в Арнольд В. И. 2012кн-ЧтоТМ-стр. 75-76}
	\scnaddlevel{-1}

\scnauthorcomment{дооформить ссылку}
	
\scnrelboth{следует отличать}{формализация}
	\scnaddlevel{1}
		\scnidtf{деятельность, направленная на повышение уровня формализованности представление информации}
		\scntext{метафора}{сближение синтаксиса с семантикой -- сближение синтаксической структуры информационной конструкции с её смысловой структурой}
	\scnaddlevel{-1}
\scnidtf{уровень способности кибернетической системы к формальному представлению знаний и используемых понятий, к рационализации идей}
\scnidtf{степень близости языка внутреннего представления (способа внутреннего кодирования) информации в памяти кибернетической системы к смысловому представлению информации}
\scnidtf{степень близости к изоморфизму соответствия между: (1) синтаксической структурой внутреннего представления информации в памяти кибернетической системы и (2) конфигурацией связей описываемых сущностей}
\scnrelfromlist{свойства-предпосылка}{многообразие форм дублирования информации, хранимой в памяти кибернетической системы
;относительный объём дублирования информации, хранимой в памяти кибернетической системы 
;многообразие фрагментов хранимой информации, не являющихся ни знаками, ни конфигурациями знаков 
;компактность представления представление информации, хранимой в памяти кибернетической системы}

\scnheader{смысловое представление информации}
\scnidtfexp{способ представления информации, в котором минимизируются "чисто синтаксические"{} аспекты представления информационных конструкций, не имеющие непосредственной семантической интерпретации}
\scnaddlevel{1}
\scnnote{Примерами "чисто синтаксических"{} аспектов представления информационных конструкций являются:
\begin{scnitemize}
	\item буквы, которые входят в состав слов и которые, следовательно, не являются знаками описываемых сущностей;
	\item алфавиты букв различных языков;
	\item знаки препинания (разделители и ограничители);
	\item инцидентность (порядок, последовательность) букв и других символов, входящих в состав информационной конструкции.
\end{scnitemize}
}
	\scnaddlevel{1}
		\scntext{следовательно}{Информационная конструкция, представленная на каком-либо привычном для нас языке, является достаточно громоздкой информационной конструкцией, смысл которой (т.е знаки описываемых сущностей и семантически интерпретируемые связи между знаками, отражающие соответствующие связи между обозначаемыми сущностями) сильно закамуфлирован. Это существенно усложняет обработку информации. если пытаться реализовывать "осмысленные"{} модели решения задач, для которых "смысловые"{} аспекты обрабатываемой информации являются ключевыми.}
	\scnaddlevel{-2}
\scnnote{Существенно подчеркнуть, что приближение внутреннего представления информации в памяти кибернетической системы к смысловому представлению информации является важнейшим фактором упрощения решателя задач кибернетической системы при реализации сложных моделей решения задач, требующих глубокого анализа смысла обрабатываемой информации. А это, в свою очередь, является важнейшим фактором качества решателя задач кибернетической системы.}

\scnheader{многообразие форм дублирования информации, хранимой в памяти кибернетической системы}
\scnidtf{многообразие видов семантической эквивалентности фрагментов информации, хранимой в памяти кибернетической системы}
\scnnote{Простейшим видом семантической эквивалентности является синонимия знаков, когда два разных фрагмента хранимой информации являются знаками, имеющими один и тот же денотат (т. е обозначающими одну и ту же сущность).}

\scnheader{относительный объем дублирования информации, хранимой в памяти кибернетической системы}
\scnidtf{частота присутствия в хранимой информации семантически эквивалентных информационных фрагментов и, в частности, синонимичных знаков}

\scnheader{многообразие фрагментов хранимой информации, не являющихся ни знаками, ни конфигурациями знаков}
\scnnote{Примерами фрагментов хранимой информации, не являющихся знаками или конфигурациями знаков, являются:
\begin{scnitemize}
	\item буквы, входящие в состав слов
	\item слова, входящие в состав словосочетаний
	\item различного вида разделители, знаки препинания
	\item различного вида ограничители.
\end{scnitemize}
}

\scnheader{компактность представления информации, хранимой в памяти кибернетической системы}
\scnnote{Должно уменьшаться число элементов памяти, используемых для представления информации, т.е. необходим переход к более компактным, но семантически эквивалентным информационным конструкциям.}

\scnheader{стратифицированность информации, хранимой в памяти кибернетической системы}
\scnrelfrom{свойство-предпосылка}{структурированность информации, хранимой в памяти кибернетической системы}
\scnidtf{способность кибернетической системы выделять такие разделы информации, хранимой в памяти этой системы, которые бы ограничивали области действия агентов решателя задач кибернетической системы, являющиеся достаточными для решения заданных задач}
	\scnaddlevel{1}
		\scnnote{Существует правило, позволяющее каждой заданной задаче поставить в соответствие априори известный (выделенный) раздел хранимой информации, являющийся областью действия агентов решателя, осуществляющих решение заданной задачи. Основными видами такого рода разделов хранимой информации являются \textit{предметные области} и \textit{онтологии}.}
	\scnaddlevel{-1}
\scnrelfrom{свойство-предпосылка}{рефлексивность информации, хранимой в памяти кибернетической системы}
\scnidtf{уровень систематизации знаний, хранимых в памяти кибернетической системы}
\scnidtf{уровень перехода от неструктурированных или слабоструктурированных данных к хорошо структурированным базам знаний}
\scnidtf{уровень перехода от первичной информации к метаинформации, метаметаинформации и т.д.}

\scnheader{рефлексивность информации,хранимой в памяти кибернетической системы}
\scnidtf{уровень применения средств самоописания (метаязыковых средств) в информации, хранимой в памяти кибернетической системы}
\scnidtf{относительный, объём и многообразие метаинформации, хранимой в памяти кибернетической системы}
\scnnote{рефлексивность информации, хранимой в памяти кибернетической системы, т.е. наличие метаязыковых средств, является фактором, обеспечивающим не только структуризацию хранимой информации, но возможность описания синтаксиса и семантики самых различных языков, используемых кибернетической системой.}

\scnheader{простота и локальность выполнения семантически целостных операций над информацией, хранимой в памяти кибернетической системы}
\scnnote{Данное свойство касается не самой информации, хранимой в памяти, а язык кодирования (представления) информации в памяти кибернетической системы}
\scnidtf{гибкость выполнения семантически целостных операций над информацией, хранимой в памяти кибернетической системы}

\scnheader{база знаний}
\scnidtf{база знаний кибернетической системы}
\scnsubset{информация, хранимая в памяти кибернетической системы}
\scnidtfexp{информация, хранимая в памяти кибернетической системы и имеющая высокий уровень качества по всем показателям и, в частности, высокий уровень:
\begin{scnitemize}
	\item \textit{семантической мощности языка представления информации хранимой в памяти кибернетической системы} (в базе знаний указанный язык должен быть универсальным);
	\item \textit{гибридности информации, хранимой в памяти кибернетической системы};
	\item \textit{многообразия видов знаний, хранимых в памяти кибернетической системы};
	\item формализованности информации, хранимой в памяти кибернетической системы;
	\item \textit{структурированности информации, хранимой в памяти кибернетической системы}
\end{scnitemize}
}
\scnnote{Переход \textit{информации, хранимой в памяти кибернетической системы} на уровень качества, соответствующий \textit{базам знаний}, является важнейшим этапом эволюции \textit{кибернетических систем}. Подчеркнем при этом, что \textit{базы знаний} по уровню своего качества могут сильно отличаться друг от друга.}

\bigskip

\scnendstruct \scninlinesourcecommentpar{Завершили Сегмент ``\textit{Комплекс свойств, определяющих качество информации, хранимой в памяти кибернетической системы}''}




\scnsegmentheader{Комплекс свойств, определяющих уровень обучаемости кибернетической системы}

\scnstartsubstruct

\scnauthorcomment{Статью 2018 включить сюда}

\scnheader{обучаемость кибернетической системы}
\scnidtf{способность кибернетической системы повышать своё качество, адаптируясь к решению новых задач, качество внутренней информации модели своей среды, качество своего решателя задач и даже качество своей физической оболочки.}
\scnidtf{способность кибернетической системы к самосовершенствованию с различной степенью самостоятельности (с учителем, с экспертом, с внешними источниками информации, только на собственном опыте)}
\scnrelboth{следует отличать}{приспособленность кибернетической системы к её совершенствованию, осуществляемому извне}
\scnidtf{способность кибернетической системы к самостоятельному повышению уровня (качества) своих знаний, навыков, а также уровня своей обучаемости}
\scnidtf{способность кибернетической системы к самостоятельному самосовершенствованию}
\scnidtf{скорость эволюции кибернетической системы}
\scnidtf{уровень (степень) обучаемости кибернетической системы}
\scnidtf{способность кибернетической системы к совершенствованию (к эволюции, к повышению уровня своего качества)}
\scnnote{Максимальный уровень обучаемости кибернетической системы -- это её способность эволюционировать (повышать уровень своего качества) максимально быстро и \uline{в любом}(!) направлении, т.е. способность быстро и без каких-либо ограничений приобретать \uline{любые}(!) новые знания и навыки.}
\scnidtf{способность кибернетической системы к повышению своего качества (в том числе, путем устранения своих недостатков, выявленных в результате самоанализа (рефлексии), в частности, в результате работы над своими ошибками, разбора собственных "полетов")}
\scnidtf{способность кибернетической системы к обучению}
\scnidtf{умение кибернетической системы учиться}
\scnidtf{способность кибернетической системы обучаться}
\scnnote{Реализация способности кибернетической системы обучаться, т.е. решать перманентно инициированную сверхзадачу самообучения, накладывает \uline{дополнительные требования}, предъявляемые к информации, хранимой в памяти кибернетической системы, к решателю задач кибернетической системы, а в перспективе также и к физической оболочке кибернетической системы.}
\scnidtf{способность кибернетической системы повышать уровень своего интеллекта -- (1) общий (интегральный) уровень качества информации, хранимый в собственной памяти, (2) общий уровень качества своих приобретаемых навыков, (3) уровень своей обучаемости.}
\scnidtf{способность кибернетической системы к максимально возможной \uline{самостоятельной эволюции}, в процессе которой кибернетическая система сама постоянно заботится о своей эволюции и о повышении темпов этой эволюции}
\scnnote{Важнейшей характеристикой кибернетической системы является не только то, какой уровень интеллекта (интеллектуальных возможностей) кибернетическая система имеет в текущий момент, какое множество действий (задач) она способна выполнять, но и то, насколько быстро этот уровень может повышаться.}

\scnheader{следует отличать*}
\scnhaselementset{образованность кибернетической системы
\scnaddlevel{1}
    \scnidtf{навыки и другие знания, которые кибернетическая приобрела (с учителем, экспертом или самостоятельно) к заданному моменту}
    \scnidtf{результат, который кибернетическая система достигла в процессе своей эволюции к заданному моменту}
\scnaddlevel{-1};
обучаемость кибернетической системы
\scnaddlevel{1}
    \scnidtf{скорость повышения уровня образованности кибернетической системы}
    \scnidtf{скорость эволюции кибернетической системы}
\scnaddlevel{-1};
скорость повышения уровня обучаемости кибернетической системы
\scnaddlevel{1}
    \scnidtf{ускорение повышения уровня образованности кибернетической системы}
    \scnnote{с увеличением объема и качества приобретаемых кибернетической системой новых навыков и знаний и, в первую очередь, при грамотной их систематизации скорость обучения кибернетической системы существенно возрастает.}
\scnaddlevel{-1}
}

\scnheader{обучаемость кибернетической системы}
\scnaddhind{1}
\scnrelfrom{комплекс свойств-предпосылок}{Комплекс свойств, определяющих обучаемость кибернетических систем по уровню обучаемости различных их компонентов}
\scnrelfrom{комплекс частных свойств}{Комплекс свойств кибернетических систем, определяющих их обучаемость по различным формам обучения}

\bigskip
\scnfragmentcaption

\scnheader{Комплекс свойств, определяющих обучаемость кибернетических систем по уровню их гибкости, стратифицированности, рефлексивности, активности}
\scneqtoset{гибкость кибернетической системы;стратифицированность кибернетической системы;рефлексивность кибернетической системы;ограниченность обучения кибернетической системы;познавательная активность кибернетической системы;способность кибернетической системы к самосохранению}

\scnheader{гибкость возможных самоизменений кибернетической системы}
\scnidtf{гибкость кибернетической системы при выполнении ею изменений над самой этой системой}
\scnrelfromset{комплекс свойств-предпосылок}{
простота возможных самоизменений кибернетической системы;
многообразие возможных самоизменений кибернетической системы 
}
\scnrelfromset{комплекс частных свойств}{
семантическая гибкость обработки информации, хранимой в памяти кибернетической системы;
семантическая гибкость возможных самоизменений решателя задач кибернетической системы;
гибкость возможных изменений физической оболочки кибернетической системы, осуществляемых самой системой
}

\scnheader{гибкость возможных самоизменений кибернетической системы}
\scnidtf{гибкость кибернетической системы при её самосовершенствовании}
\scnrelto{частное свойство}{гибкость кибернетической системы}
\scnaddlevel{1}
    \scnnote{Поскольку обучение всегда сводится к внесению тех или иных изменений в обучаемую кибернетическую систему, без высокого уровня гибкости этой системы не может быть высокого уровня её обучаемости.}
    \scnrelto{свойство-предпосылка}{обучаемость кибернетической системы}
\scnaddlevel{-1}

\scnheader{простота возможных самоизменений кибернетической системы}
\scnidtf{легкость (трудоемкость) внесения различных изменений в кибернетическую систему, осуществляемых самой этой кибернетической системой}
\scnidtf{приспособленность кибернетической системы к самостоятельному внесению различных изменений в саму себя}

\scnheader{стратифицированность кибернетической системы}
\scnidtf{иерархическая декомпозиция кибернетической системы на такие подсистемы, структура и функционирование которых минимально возможным образом связаны друг с другом, что существенным образом сужает область учета последствий различных изменений вносимых в систему, а также область поиска причин всевозможных ошибок}
\scnidtf{модульность кибернетической системы}
\scnidtf{возможность разделить кибернетическую систему на такие части (страты), эволюция (изменения) которых может осуществляться независимо друг от друга.}
\scnnote{Уровень стратифицированности определяется 
\begin{scnitemize}
    \item количеством страт;
    \item степенью зависимости страт друг от друга. 
\end{scnitemize}}
\scnnote{При наличии стратифицированности кибернетической системы появляется возможность четкого определения области действия различных изменений, вносимых в кибернетическую систему, т.е. возможность четкого ограничения тех частей кибернетической системы, за пределы которых нет необходимости выходить для учета последствий внесенных в систему первичных изменений, т.е. осуществлять \uline{дополнительные} изменения, являющиеся последствиями первичных изменений.}
\scnnote{Стратификация кибернетической системы -- это не просто её структуризация (прежде всего, структуризация информации, хранимой в памяти кибернетической системы), а такая её структуризация, которая четко определяет границы учета возможных последствий вносимых в систему изменений различного вида.}
\scnrelfromlist{частное свойство}{стратифицированность информации, хранимой в памяти кибернетической системы; стратифицированность решателя задач кибернетической системы; стратифицированность физической оболочки кибернетической системы}

\scnheader{рефлексивность кибернетической системы}
\scnidtf{уровень (степень) рефлексивности кибернетической системы}
\scnidtf{способность кибернетической системы к самоанализу (к анализу интегрального уровня своего качества и, в том числе, уровня своего интеллекта)}
\scnidtf{способность кибернетической системы самостоятельно анализировать (оценивать) свое качество}
\scnidtf{уровень рефлексии кибернетической системы}
\scnidtf{способность кибернетической системы к самоанализу -- к анализу своих знаний, навыков, своих действий во внутренней и внешней среде}
\scnidtf{способность кибернетической системы к самонаблюдению и самоанализу}
\scnidtf{способность кибернетической системы к рефлексии}
\scnidtf{способность кибернетической системы к анализу своего качества}
\scnidtf{Способность кибернетической системы к самоанализу (к анализу самой себя во всевозможных аспектах).}
\scnnote{Конструктивным результатом рефлексии кибернетической системы является генерация в её памяти спецификации различных негативных или подозрительных особенностей, которые следует учитывать для повышения качества кибернетической системы. Такими особенностями (недостатками) могут быть выявленные противоречия (ошибки), выявленные пары синонимичных знаков, омонимичные знаки, информационные дыры и многое другое.}
\scnrelfromlist{частное свойство}{способность кибернетической системы к анализу качества информации, хранимой в её памяти;  способность кибернетической системы к анализу качества своего решателя задач
\scnaddlevel{1}
    \scnrelfrom{частное свойство}{способность кибернетической системы к анализу качества своего поведения во внешней среде}
\scnaddlevel{-1}; 
способность кибернетической системы к анализу качества своей физической оболочки
\scnaddlevel{1}
    \scnrelfrom{частное свойство}{способность кибернетической системы к анализу качества физического обеспечения своего интерфейса с внешней средой}
\scnaddlevel{-1}}
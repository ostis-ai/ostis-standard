\bigskip
\scnfragmentcaption

\scnheader{способность кибернетической системы к повышению качества своего решателя задач}
\scnidtf{способность кибернетической системы повышать качество своих приобретаемых навыков}
\scnrelfromlist{свойство-предпосылка}{способность кибернетической системы к повышению качества информации, хранимой в ее памяти
;семантическая гибкость возможных самоизменений решателя задач кибернетической системы
;стратифицированность решателя задач кибернетической системы
;способность кибернетической системы к анализу качества своего решателя задач
;способность кибернетической системы к целенаправленной коррекции своей деятельности
;способность кибернетической системы к оптимизации хранимых в памяти методов решения задач
;способность кибернетической системы к генерации новых методов решения задач
;способность кибернетической системы интегрировать у себя новые приобретаемые извне методы и модели решения задач}


\scnheader{семантическая гибкость возможных самоизменений решателя задач кибернетической системы}
\scnidtf{простота реализации решателем задач кибернетической системы различного рода изменений самого себя}
\scnnote{Очевидно, что семантическая гибкость решателя задач кибернетической системы во многом определяется процессором кибернетической системы (прежде всего, его универсальностью и близостью реализуемой им модели обработки информации к смысловому уровню). Но, поскольку решатель задач кибернетической системы кроме процессора включает в себя хранимые в памяти кибернетической системы методы решения различного вида задач (в том числе, и методы интерпретации методов высокого уровня), семантическая гибкость решателя задач определяется также \textit{семантической гибкостью информации, хранимой в памяти кибернетической системы}.}
\scnrelfrom{свойство-предпосылка}{семантическая гибкость информации, хранимой в памяти кибернетической системы}


\scnheader{стратифицированность решателя задач кибернетической системы}
\scnrelfromlist{частное свойство}{стратифицированность методов и навыков решения задач, представленных в памяти кибернетической системы
;стратифицированность технологий, соответствующих различным видам деятельности
;стратифицированность различного вида действий, классов действий и видов деятельности}
	\scnaddlevel{1}
	\scnaddhind{1}
	\scnrelfrom{частное свойство}{стратифицированность различного вида информационных процессов, выполняемых в памяти кибернетической системы}
	\scnaddlevel{-1}
	
	
\scnheader{качество внутренних языковых средств кибернетической системы для описания качества собственного решателя задач}
\scnaddhind{1}
\scnrelfrom{свойство-предпосылка}{качество внутренних языковых средств кибернетической системы для описания качества собственных действий}
	
	
\scnheader{способность кибернетической системы к анализу качества своего решателя задач}
\scnidtf{способность кибернетической системы к анализу (к оценке качества) своей деятельности в собственной внутренней среде (в своей памяти), а также в своей внешней среде}
\scnnote{Анализ качества решателя задач включает в себя:
			\begin{scnitemize}
			\item анализ качества используемых методов и технологий решения задач;
			\item анализ качества используемых моделей решения задач;
			\item анализ полноты набора постоянно инициированных целей (задач), направленных на эволюцию и на борьбу с деградацией (снижением качества) кибернетической системы;
			\item анализ качества выполняемых действий (процессов решения задач).
			\end{scnitemize}}
\scnidtf{способность кибернетической системы к описанию (к построению в своей памяти информационной модели) собственных действий, выполняемых в собственной памяти, а также к анализу и оценке этих действий}
\scnidtf{способность кибернетической системы к анализу своего поведения в своей внутренней среде (в своей памяти), а также в своей внешней среде и в своей физической оболочке}
\scnrelfromlist{свойство-предпосылка}{качество внутренних языковых средств кибернетической системы для описания качества собственного решателя задач
;способность кибернетической системы к анализу собственной деятельности\\
		\scnaddlevel{1}
		\scnrelfromlist{частное свойство}{способность кибернетической системы к анализу качества информационных процессов, выполняемых в собственной памяти
			\scnaddlevel{1}
			\scnidtf{способность кибернетической системы к анализу качества своего поведения (действий, информационных процессов) в собственной внутренней среде -- своих действий, сводящихся к поиску, генерации, удалению и преобразованию информационных конструкций, хранимых в собственной памяти}
			\scnaddlevel{-1}
;способность кибернетической системы к анализу качества своего поведения во внешней среде}
		\scnaddlevel{-1}
;способность кибернетической системы к анализу качества методов, хранимых в собственной памяти\\
		\scnaddlevel{1}
		\scnrelfromlist{частное свойство}{способность кибернетической системы к анализу качества методов и технологий, используемых ею для выполнения сложных действий в собственной памяти;способность кибернетической системы к анализу качества методов и технологий, используемых ею для выполнения сложных действий во внешней среде}
		\scnaddlevel{-1}}


\scnheader{качество внутренних языковых средств кибернетической системы для описания качества собственных действий}
\scnnote{В данном свойстве кибернетической системы имеется в виду описание собственных действий, выполняемых кибернетической системой как в своей внутренней среде (в собственной памяти), так и в своей внешней среде.}


\scnheader{способность кибернетической системы к анализу качества своего поведения во внешней среде}
\scnidtf{способность кибернетической системы к анализу соответствия между тем, что планировалось сделать во внешней среде и тем, что реально получилось}
\scnnote{Поведение кибернетической системы во внешней среде рассматривается ею как "эксперимент"{}, подтверждающий или опровергающий ее представление о внешней среде.}
\scnidtf{способность кибернетической системы к анализу своего опыта взаимодействия с внешней средой и, в частности, к выявлению своих ошибок}


\scnheader{способность кибернетической системы к целенаправленной коррекции своей деятельности}
\scnidtf{способность кибернетической системы к коррекции своего поведения в целях повышения его качества (эффективности)}
\scnidtf{способность кибернетической системы учиться на ошибках своей деятельности на основе анализа этих ошибок}
\scnrelfrom{свойство-предпосылка}{способность кибернетической системы к анализу собственной деятельности}


\scnheader{способность кибернетической системы к оптимизации хранимых в памяти методов решения задач}
\scnnote{Хранимые в памяти \textit{методы} решения задач разбиваются на следующие классы:
	\begin{scnitemize}
	\item \textit{методы верхнего уровня} -- интерпретируемые методы;
	\item \textit{методы базового уровня}, представленные на базовом языке программирования, который интерпретируется непосредственно процессором кибернетической системы;
	\item \textit{метаметоды}, описывающие интерпретацию методов верхнего уровня.
	\end{scnitemize}}


\scnheader{способность кибернетической системы к генерации новых методов решения задач}
\scnnote{Целесообразность генерации нового метода решения задач возникает, когда кибернетической системе приходится часто решать эквивалентные задачи некоторого класса. Генерация соответствующего метода и последующая его оптимизация позволяет существенно сократить время решения задач.}
\scnidtf{способность кибернетической системы расширять множество используемых ею методов решения задач}
\scnnote{Если добавляемые методы соответствуют используемым моделям решения задач, то, кроме добавления самих методов, желательно, чтобы в стратифицированной кибернетической системе никакие другие изменения не потребовались. Если добавляемый метод соответствует новой (ранее не известной) модели решения задач, то желательно, чтобы в стратифицированной кибернетической системе никакие другие изменения не потребовались, кроме добавления агентов, обеспечивающих интерпретацию (описание операционной семантики) методов нового класса.}
\scnnote{Речь идет о методах решения как внутренних задач, решаемых в памяти кибернетической системы, так и внешних задач, решаемых во внешней среде путем управления деятельностью эффекторов и рецепторов кибернетической системы.}
\scnrelfrom{свойство-предпосылка}{способность расширять множество использованных моделей решения задач}


\scnheader{способность кибернетической системы интегрировать у себя новые приобретаемые извне методы и модели решения задач}
\scnnote{Для обеспечения такой способности необходима:
	\begin{scnitemize}
	\item разработка универсальной базовой модели решения задач, для которой соответствующие ей методы решения задач интерпретируются процессором кибернетической системы;
	\item разработка семейства классов методов верхнего уровня, что предполагает:
		\begin{scnitemizeii}
		\item разработку языков представления методов для каждого класса методов верхнего уровня;
		\item разработку интерпретаторов для каждого класса методов верхнего уровня на основе указанной выше базовой модели решения задач.
		\end{scnitemizeii}
	\end{scnitemize}}


\bigskip
\scnfragmentcaption

\scnheader{способность кибернетической системы к повышению качества своей физической оболочки}
\scnidtf{способность кибернетической системы к самостоятельному совершенствованию (эволюции) своей физической оболочки}
\scnnote{Данная способность кибернетической системы накладывает определенные требования к построению ее физической оболочки.}
\scnrelfromlist{свойство-предпосылка}{гибкость возможных изменений физической оболочки кибернетической системы
;стратифицированность физической оболочки кибернетической системы
;способность кибернетической системы к анализу качества своей физической оболочки\\
	\scnaddlevel{1}
	\scnrelfrom{свойство-предпосылка}{качество внутренних языковых средств кибернетической системы для описания качества собственной физической оболочки}
	\scnaddlevel{-1}
;способность кибернетической системы расширять и/или совершенствовать набор собственных сенсоров и эффекторов}


\bigskip
\scnfragmentcaption

\scnheader{комплекс свойств кибернетических систем, определяющих их обучаемость по различным формам обучения}
\scneqtoset{обучаемость с учителем;самообучаемость с экспертом;самообучаемость на основе внешних информационных источников;самообучаемость без внешних информационных источников}


\scnheader{обучаемость с учителем}
\scnidtf{уровень способности к обучению под управлением внешнего субъекта-учителя}
\scnidtf{способность кибернетической системы к эффективному обучению с помощью учителя, осуществляющего управление процессом обучения}
\scnidtf{способность заданной кибернетической системы эффективно обучаться с помощью внешней кибернетической системы (внешнего субъекта, внешнего активного учителя), осуществляющей организацию обучения заданной кибернетической системы на основе различных методик обучения, учитывающих особенности обучаемой системы и определяющих характер (в том числе последовательность) передачи знаний и новыков, а также тестирование качества их усвоения}


\scnheader{самообучаемость с экспертом}
\scnidtf{способность кибернетической системы к самообучению в диалоге с экспертом-консультантом}
\scnidtf{способность кибернетической системы не просто задавать нужные для собственного обучения вопросы (информационные цели), но и вести вопросно-ответный диалог с другими субъектами (кибернетическими системами), которые являются экспертами в соответствующей области (указанные эксперты – это своего рода "пассивные учителя"{}, которые много знают и умеют в соответствующей области, могут отвечать на вопросы, но не желают управлять процессом передачи этих знаний и умений другим кибернетическим системам)}
\scnidtf{эффективность самообразования кибернетической системы, в основе которого лежит диалог, управляемый этой обучаемой системой и осуществляемый с кибернетической системой, являющейся носителем востребованных знаний и навыков}
\scnidtf{эффективность самообучения, осуществляемого в форме консультации}
\scnidtf{способность управлять процессом самообучения путем формирования последовательности вопросов (познавательных целей), адресуемых внешним субъектам}
\scnrelfrom{свойство-предпосылка}{способность кибернетической системы к синтезу познавательных целей и процедур}
	

\scnheader{обучаемость на основе пассивных внешних информационных источников}
\scnidtf{способность кибернетической системы к извлечению информации, содержащейся во внешних информационных источниках, к поиску нужных внешних источников и к построению на этой основе систематизированной картины мира}
\scnidtf{эффективность самообучения, основанного на анализе \uline{пассивных} источников информации (документов различного вида, публикаций, текстов, которые необходимо находить в различного рода библиотеках, читать и \uline{понимать})}
	

\scnheader{самообучаемость без внешних информационных источников}
\scnidtf{способность кибернетической системы формировать систематизированную модель (картину) окружающей среды, используя для ее непосредственного восприятия и изучения только собственные сенсоры и эффекторы, а также некоторые дополнительные средства, усиливающие возможности сенсоров и эффекторов}
\scnidtf{эффективность самообучения кибернетической системы, основанного исключительно на собственном опыте, на анализе собственной деятельности и собственных ошибок}
\scnnote{Данная способность кибернетической системы является необходимым, но явно недостаточным фактором ее высокого качества. Учиться только на собственном опыте -- существенно понизить уровень своего интеллекта. В этом смысле познавательный процесс социален.}

\bigskip
\scnendstruct \scninlinesourcecommentpar{Завершили \textit{Сегмент 7 Начала Раздела \ref{sd_sys_inform}} ``\nameref{sd_sys_inform}''}
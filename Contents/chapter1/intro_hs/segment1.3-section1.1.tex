\bigskip

\scnheader{память кибернетической системы}

\scnidtf{физическая оболочка реализация абстрактной памяти кибернетической системы внутренней среды кибернетической системы, в рамках которой кибернетическая система формирует и использует (обрабатывает) информационную модель своей внешней среды} 
\scnnote{Не каждая кибернетическая система имеет память. В кибернетических системах, которые не имеют памяти, обработка информации сводится к обмену сигналами между компонентами этих систем. Появление в кибернетических системах памяти как среды для "централизованного" хранения и обработки информации является важнейшим этапом их эволюции. Дальнейшее эволюция кибернетических систем во многом определяется:
	\begin{scnitemize}

	\item качеством памяти как среды для хранения и обработки информации;
	\item качеством информации (информационные модели), хранимой в памяти кибернетической системы;
	 \end{scnitemize}}
\scnidtf{компонент кибернетической системы, в рамках которого теперь кибернетическая система осуществляет отображение (формировании информационной модели) среды своего существования, а также использование этой информационной модели для управления собственным поведением в указанной среде}
	 
\scnidtf{физическая оболочка для хранения информации, которую кибернетическая система приобретает и обрабатывает (т.е. меняет состояния этой информации)}
\scnidtf{физическая (аппаратная) реализация внутренней среды кибернетическая система, каковой является среда "существования" информации, накапливаемой и непосредственно используемой решателем задач этой кибернетической системы}

\scnnote{Сам факт появления в кибернетической системе памяти, которая (1) обеспечивает представление различного виды информации о среде, в рамках которой кибернетическая система решает различные задачи (выполняет различные действия), (2) обеспечивает хранение достаточно полной информационной модели указанной среды (достаточно полной для реализации своей деятельности), (3) обеспечивает высокую степень гибкости указанной хранимой в памяти информационной модели среды жизнедеятельности (т.е. лёгкость внесения изменений в эту информационную модель), существенно повышает уровень адаптивности кибернетической системы к различным изменениям своей среды}
\scnnote{"появление" \textit{памяти} кибернетических системах является основным признаком перехода от "простых" автоматов к компьютерным системам, от роботов 1-го поколение к роботам следующих поколений}
\scnidtf{физическая реализация хранилища информации, которую приобрела (накопила) к текущему моменту соответствующая кибернетическая система}
\scnidtf{физическая оболочка внутренней абстрактной информационной среды кибернетической системы}
\scnidtf{среда хранения и обработки информации}
\scnidtf{запоминающая среда}
\scnidtf{среда хранения и обработки информационных конструкций}
\scnnote{Принципы организации памяти кибернетической системы могут быть разными(ассоциативная, адресная, структурно фиксированная/структурно перестраиваемая, нелинейная/линейная). От организации памяти во многом зависит её качество}
\filemodefalse

\scntext{уровни эволюции}{
Уровни структурной эволюции кибернетических систем}
\scneqtovector{
простая кибернетическая система, не имеющая память;
простая кибернетическая система, имеющая память;
одноуровневый коллектив, не имеющий общей памяти и  одноуровневый коллектив, не имеющий общей памяти и состоящий из простых кибернетических систем, имеющих память;
иерархический коллектив,  имеющий общую памяти и состоящий из простых кибернетических систем;\\
индивидуальная кибернетическая система\\
\scnaddlevel{1}
\scnnote{Каждая индивидуальная кибернетическая система содержит память, имеющую достаточно высокий уровень качества
одноуровневый коллектив индивидуальных кибернетических систем, не имеющий общей памяти}
 \scnaddlevel{-1}
;одноуровневый коллектив индивидуальных кибернетическая систем, имеющий общую память 
;иерархический коллектив из индивидуальных кибернетических систем, не имеющий общей памяти 
;иерархический коллектив из индивидуальных кибернетических систем, имеющий общую память}\\
\scnheader{процессор кибернетической системы}\\
\scnidtf{физически (аппаратно реализованный) интерпретатор хранимых в памяти кибернетической системы методов (программ), соответствующих базовой (для данной кибернетической системы) модели решения задач, то есть такой модели решения задач, которая для данной кибернетической системы является моделью решения задач самого нижнего уровня и, следовательно, не может быть интерпретирована с помощью другой модели решения задач, используемой этой же кибернетической системой, а может быть проинтерпретирована либо путем аппаратной реализации такого интерпретатора, путём его программной реализации, например, на современных компьютерах, но в последнем случае, кроме собственного интерпретатора, необходимо также построить модель памяти реализуемой кибернетической системы}

\scnidtf{"физически" реализованные средства, обеспечивающее выполнение "элементарных" действий, направленных на изменение состояния памяти кибернетической системы (на изменение информации, хранимой в этой памяти)}
\scnidtf{"движок"("мотор") кибернетической системы}
\scnrelto{второй домен}{\textit{процессор кибернетической системы*}\\
\scnidtfexp{бинарное ориентированное отношения, каждая пара которого связывает знак кибернетической системы со знаком её процессора}
\scniselement{бинарное отношение}
\scniselement{ориентированное отношение}
}
\scnheader{компьютер}\\
\scnsubset{физическая оболочка кибернетической системы}
\scnidtf{физическая оболочка искусственной кибернетической системы} \scnidtf{аппаратное обеспечение компьютерной системы}
\scnidtf{hardware of computer system}

\scnsuperset{компьютер для интеллектуальных систем}
\scnaddlevel{1}
\scnidtf{компьютер, ориентированный на реализацию интеллектуальных компьютерных систем}
\scnnote{Развитие рынка интеллектуальных компьютерных систем существенно сдерживается неприспособленностью современного поколения компьютеров к реализации на их основе интеллектуальных компьютерных систем.Попытки создания компьютеров, приспособленных к реализации интеллектуальных компьютерных систем, не привели к успеху, т.к. эти проекты были направлены на выполнение отдельных (частных) требований, предъявляемых к физическому (аппаратному) уровню интеллектуальных систем, что неминуемо приводило к приспособленности создаваемых компьютеров к реализации не всего многообразия интеллектуальных компьютерных систем, а только некоторых подмножеств таких систем. Указанные подмножества интеллектуальных компьютерных систем в основном определялись
Ориентацией на конкретные используемые модели решения интеллектуальных задач, тогда, как важнейшим фактором, определяющим уровень интеллекта кибернетических систем (в том числе, и компьютерных систем), является их универсальность в плане многообразие используемых моделей решения задач. Следовательно, компьютер для интеллектуальных компьютерных систем должен быть эффективным аппаратным интерпретатором любых моделей решения задач (как интеллектуальных задач, так и достаточно простых задач, т.к. интеллектуальная система должна уметь решать любые задачи).} 
\scnidtf{компьютер, приспособленный к реализации интеллектуальных компьютерных систем}
\scnidtf{универсальный компьютер для интеллектуальных систем}
\scnidtf{компьютер, обеспечивающий интерпретацию любых моделей решения задач}
\scnaddlevel{-1}
\scnheader{ \textit{Семейство отношений, заданных на множестве кибернетических систем}}\\
 \scnstartsubstruct
\\
 отношений, заданное на множестве кибернетических систем\\
\scnhaselement{память кибернетической системы*}
\scnhaselement{процессор кибернетической системы*}
\scnhaselement{член коллектива}
\scnhaselement{внешняя среда кибернетической системы*}
\scnhaselement{сенсор кибернетической системы*}
\scnhaselement{эффектор кибернетической системы*}
\scnhaselement{физическая оболочка кибернетической системы*}
\scnhaselement{информация, хранимая в памяти кибернетической системы*} \scnhaselement{абстрактная память кибернетической системы*}
\scnhaselement{часть*}
\scnaddlevel{1}
\scnsuperset{встроенная кибернетическая система}
\scnaddlevel{-1}
 информация, хранимая в памяти кибернетической системы*\\ 
\scnidtf{информационная модель среды*, в которой существует (осуществляет деятельность) соответствующая кибернетическая система} 
\scnnote{От того, насколько полна, адекватна (корректна) и систематизирована (структурирована) внутренняя среда кибернетической системы, зависит уровень интеллектуальности и эффективность соответствующей кибернетической системы.}
\textit{следует отличать*}\\
\scnhaselementset{решатель задач кибернетической системы*;
решатель задач кибернетической системы\\
\scnaddlevel{1}
\scnidtf{иерархическая система моделей решения задач}
\scntext{обобщённая часть}{\textit{процессор кибернетической системы}}
\scnaddlevel{1}
\scnexplanation{Это реализация модели решения задач, обеспечивающей интерпретацию всех используемых моделей решения задач верхнего уровня} 
}\\
\scnaddlevel{-2}
\scnheader{задача, решаемые кибернетической системой} 
\scnidtf{быть задачей, решаемой заданной кибернетической системой*} \scnsuperset{задача, решаемая в памяти кибернетической системы*}
\scnaddlevel{1}
\scnidtf{внутренняя задача кибернетической системы задача*}
\scnaddlevel{-1}
\scnsuperset{задача, решаемая во внешней среде кибернетической системы*}\\
\scnheader{\textit{внешняя среда кибернетической системы}}
\scnidtf{внешняя среда} 
\scnnote{Понятие внешней среды кибернетической системы является понятием относительным, т.к. (1) разные кибернетические системы имеют в общем случае разную внешнюю среду и (2) одна кибернетическая система может входить в состав внешней среды другой кибернетической системы}
\scnidtf{быть внешней средой для заданной кибернетической системы*} \scniselement{бинарное отношение}
\scniselement{ориентированное отношение} 
\scntext{первый домен}{кибернетическая система}
\scnsuperset{внешняя информационная среда кибернетической системы*}
\scnaddlevel{1} \scnidtf{совокупность всевозможных информационных, к которым данная кибернетическая система имеет доступ и которые представлены самым различным образом (в том числе, и в памяти тех кибернетических систем (субъектов), с которыми данная система взаимодействует)}
\scnaddlevel{-1}
\scnheader{\textit{среда кибернетической системы*}}\\
\scnidtf{быть средой существования (жизнедеятельности) заданной (указанной, соответствующей) кибернетической системы}
\scnnote{В общем случае среда жизнедеятельности кибернетической системы включает в себя (1) \textit{внешнюю среду*} этой системы, (2) \textit{физическую оболочку*} этой системы и (3) её абстрактную память, т.е. внутреннюю среду*, которая является хранилищем информационной модели всей среды} 
\scnsubdividing{внешняя среда*; физическая оболочка*; абстрактная память*}
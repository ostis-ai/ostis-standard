\bigskip

\scnsegmentheader{Комплекс свойств, определяющих качество решателя задач кибернетической системы}
\scnstartsubstruct

\scnheader{качество решателя задач кибернетической системы}
\scnidtf{интегральная качественная оценка множества задач (действий), которые кибернетическая система способна выполнять в заданный момент}
\scnidtf{качество навыков, приобретенных кибернетической системой}
\scnnote{Основным свойством и назначением \textit{решателя задач кибернетической системы} является способность решать \textit{задачи} на основе накапливаемых (приобретаемых) \textit{кибернетической системой} различного вида \textit{навыков} с использованием \textit{процессора кибернетической системы}, являющегося универсальным интерпретатором всевозможных накопленных \textit{навыков}. При этом качество (уровень развития, уровень совершенства) указанной способности определяется целым рядом дополнительных факторов (свойств).}
\scnidtf{интеллектуальный уровень качества решателя задач кибернетической системы}
\scnidtf{интегральное качество умений (навыков), приобретенных \textit{кибернетической системой} к текущему моменту}
\scnrelfromlist{свойство-предпосылка}{общая характеристика решателя задач кибернетической системы;качество логико-семантической организации памяти кибернетической системы;качество решения интерфейсных задач в кибернетической системе}

\bigskip
\scnfragmentcaption

\scnheader{общая характеристика решателя задач кибернетической системы}
\scnrelfromlist{свойство-предпосылка}{общий объем задач, решаемых кибернетической системой;многообразие видов задач, решаемых кибернетической системой;способность кибернетической системы к анализу решаемых задач;способность кибернетической системы к решению задач, методы решения которых в текущий момент известны;способность кибернетической системы к решению задач, методы решения которых ей в текущий момент не известны;множество навыков, используемых кибернетической системой;степень конвергенции и интеграции различного вида моделей решения задач, используемых кибернетической системой;качество организации взаимодействия процессов решения задач в кибернетической системе;быстродействие решателя задач кибернетической системы;способность кибернетической системы решать задачи, предполагающие использование информации, обладающей различного рода не-факторами;многообразие и качество решения задач информационного поиска;способность кибернетической системы генерировать ответы на вопросы различного вида в случае, если они целиком или частично отсутствуют в текущем состоянии информации, хранимой в памяти;способность кибернетической системы к рассуждениям различного вида;качество целеполагания;качество реализации планов собственных действий;способность кибернетической системы к локализации такой области информации,хранимой в ее памяти, которой достаточно для обеспечения решения заданной задачи;способность кибернетической системы к выявлению существенного в информации, хранимой в ее памяти;активность кибернетической системы}
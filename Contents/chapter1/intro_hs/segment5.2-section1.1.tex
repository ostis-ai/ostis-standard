\scnheader{не-фактор}
\scnidtf{группа семантических свойств, определяющих качество информации, хранимой в памяти кибернетической системы}
\scneqtoset{корректность/некорректность информации, хранимой в памяти кибернетической системы;
однозначность/неоднозначность информации, хранимой в памяти кибернетической системы;
целостность/нецелостность информации, хранимой в памяти кибернетической системы;
чистота/загрязненность информации, хранимой в памяти кибернетической системы;
достоверность/недостоверность информации, хранимой в памяти кибернетической системы;
точность/неточность информации, хранимой в памяти кибернетической системы;
четкость/нечеткость информации, хранимой в памяти кибернетической системы;
определенность/недоопределенность информации, хранимой в памяти кибернетической системы}
\scnexplanation{Ярешкина Н.Г. ред.2007кн-НечетГС-стр.10-28}
\scnaddlevel{1}
\scnrelto{цитата}{Ярушкина Н.Г. ред.2007кн-НечетГС}
\scnaddlevel{-1}

\scnheader{корректность/некорректность информации, хранимой в памяти кибернетической системы}
\scnidtf{уровень адекватности хранимой информации в той среде, в которой существует кибернетическая система и информационной моделью которой эта хранимая информация является}
%% Сложно понять что написано, возможно я где-то в склонениях перепутал

\scnheader{непротиворечивость/противоречивость информации, хранимой в памяти кибернетической системы}
\scnidtf{уровень присутствия в хранимой информации различного вида противоречий и, в частности, ошибок}

\scnheader{противоречие*}
\scnidtf{пара противоречащих друг другу фрагментов информации, хранимой в памяти кибернетической системы*}
\scnnote{Чаще всего противоречащями друг другу информационными фрагментами являются:
\begin{scnitemize}
\item{явно представленная в памяти некоторая закономерность (некоторое правило);}
\item{информационный фрагмент, не соответствующий (противоречащий) указанной закономерности}
\end{scnitemize}
\bigskip
В этом случае некорректность может присутствовать:
\begin{scnitemize}
\item{либо в информационном фрагменте, который противоречит указанной закономерности;}
\item{либо в самой этой закономерности;}
\item{либо и там и там.}
\end{scnitemize}
}

\scnheader{информационная ошибка}
\scnidtfdef{противоречие, заключающееся в нарушении некоторой закономерности (некоторого правила), которая не подвергается сомнению}

\scnheader{информационная ошибка}
\scnnote{Ошибки (ошибочные фрагменты) в хранимой информации могут быть синтаксическими и семантическими, противоречащими некоторым правилам (закономерностям), которые явно в памяти могут быть не представлены и считаются априори истинными}

\scnheader{полнота/неполнота информации, хранимой в памяти кибернетической системы}
\scnidtf{уровень того, насколько информация, хранимая в памяти кибернетической системы, описывает среду существования этой системы и используемые ею методы решения задач достаточно полно (достаточно детально) для того, чтобы кибернетическая система могла действительно решать все множество соответствующих ей задач}
\scnidtf{уровень соответствия хранимой информации объёму задач (действий), которые соответствующиая кибернетическоая система желает уметь решать (выполнять)}

\scnheader{полнота/неполнота информации, хранимой в памяти кибернетической системы}
\scnidtf{степень достаточности информации, хранимой в памяти кибернетической системы, для достижения целей этой системы, для выполнения своих "обязанностей"}
\scnnote{Чем полнее информация, хранимая в памяти кибернетической системы, чем полнее \uline{информационное обеспечение деятельности этой системы} это системы, тем эффективнее (качественее) сама эта деятельность}
\scnrelfromlist{свойство-предпосылка}{многообразие видов знаний, хранимых в памяти кибернетической системы;
структурированность информации, хранимой в памяти кибернетической системы}

\scnheader{однозначность/неоднозначность информации, хранимой в кибернетической системе}
\scnrelfromlist{свойство-предпосылка}{многообразие форм дублирования информации, хранимой в памяти кибернетической системы;
частота дублирования информации, хранимой в памяти кибернетической системы}

\scnheader{целостность/нецелостность информации, ъранимой в памяти кибернетической системы}
\scnidtf{уровень содержательной информативности информации, хранимой в памяти кибернетической системы}
\scnidtf{уровень того, насколько содержательно (семантически) \uline{связной} является информация, хранимая в памяти кибернетической системы, насколько полно специфицированы \uline{все} описываемые в памяти сущности (путём описания необходимого набора связей этих сущностей с другими описываемыми сущностями), насколько редко или часто в рамках хранимой информации встречаются \textit{информационные дыры}, соответствующие явной недостаточности некоторых спецификаций}

\scnheader{целостность/нецелостность информации, хранимой в памяти кибернетической системы}
\scnidtf{известность/неизвестность информации, хранимой в памяти кибернетической системы}
\scnidtf{многообразие форм и частота присутствия \textit{информационных дыр} в информации, хранимой в памяти кибернетической системы}
\bigskip
\scnheader{информационная дыра в информации, хранимой в памяти кибернетической системы}
\scnidtf{информация, отсутствие которой в памяти кибернетической системы существенно усложняет деятельность этой системы}
\scnnote{Примерами информационных дыр являются:
\begin{scnitemize}
	\item{отсутствующий метод решения часто встречающихся задач;}
	\item{отсутствующее определение используемого определяемого понятия;}
	\item{недостаточно подробная спецификация часто рассматриваемых сущностей}
\end{scnitemize}}

\scnheader{чистота/загрязненность информации, хранимой в памяти кибернетической системы}
\scnidtf{многообразие форм и общее количество информационного мусора, входящего в состав информации, хранимой в памяти кибернетической системы}

\scnheader{информационный мусор, входящий в состав информации, хранимой в памяти кибернетической системы}
\scnidtf{информационный фрагмент, входящий в состав информации, хранимой в памяти кибернетической системы, удаление которого существенно \uline{не} усложнит деятельность кибернетической системы}
\scnnote{Примерами информационного мусора являются:
\begin{scnitemize}
	\item{информация, которая нечасто востребована, но при необходимости может быть легко логически выведена}
	\item{информация, актуальность которой истекла}
\end{scnitemize}}

\scnheader{семантическая мощность языка представления информации в памяти кибернетической системы}
\scnidtf{семантическая мощность внутреннего языка кибернетической системы}
\scnrelfrom{свойство-предпосылка}{гибридность информации, хранимой в памяти кибернетической системы}
\scnnote{Универсальность внутреннего языка кибернетической системы является важнейщим фактором её интеллектуальности}

\scnheader{универсальный язык}
\scnidtf{язык, информационные конструкции которого могут представить (описать) \uline{любую} конфигурацию \uline{любых} связей между \uline{любыми} сущностями}

\scnheader{гибридность информации, хранимой в памяти кибернетической системы}
\scnrelfromlist{свойство-предпосылка}{многообразие видов знаний, хранимых в кибернетической системе;
степень конвергенции и интеграции различного вида знаний, хранимых в памяти кибернетической системы}

\scnheader{многообразие видов знаний, хранимых в памяти кибернетической системы}
\scnrelfromlist{частное свойство}{рефлексивность информации, хранимой в памяти кибернетической системы
\scnaddlevel{1}
\scnidtf{многообразие видов метаинформации (метазнаний), хранимых в памяти кибернетической системы}
\scnaddlevel{-1};
многообразие моделей решения задач, используемых кибернетической системой;
многообразие видов целей, анализируемых или синтезируемых кибернетической системой;
многообразие планов решения задач, решаемых кибернетической системой;
многообразие протоколов решения задач, решаемых кибернетической системой
}

\scnheader{объем информации, хранимой в памяти кибернетической системы}
\scnidtf{объем знаний, приобретенных кибернетической системой к текущему моменту}
\scnidtf{содержательная совокупность всех знаний, хранимых в текущий момент в памяти кибернетической системы}
\scnnote{Чем больше кибернетическая система знает, тем при прочих равных условиях выше уровень её качества}


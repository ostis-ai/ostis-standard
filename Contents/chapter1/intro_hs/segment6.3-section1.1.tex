
\scnheader{Степень конвергенции и интеграции различного вида моделей решения задач, используемых кибернетической системой}
\scnnote{Необходим переход от эклектики никак не связанных друг с другом \textit{моделей решения задач} к их \textit{конвергенции}, это предполагает: 
\begin{scnitemize}
    \item разработку общего (базового) для всех \textit{моделей решения задач} языка описания \textit{операционной семантики} языков описания методов, соответствующих различным \textit{моделям решения задач};
    \item включение всех языков описания \textit{методов решения задач} в общую систему языков, связанных между собой отношением ``язык-подъязык*''.
\end{scnitemize}
}

\scnheader{качество организации взаимодействия процессов решения задач в кибернетической системе}
\scnrelfromlist{частное свойство}{качество управления информационным процессом в памяти кибернетической системы\\
\scnaddlevel{1}
    \scnrelfrom{свойство-предпосылка}{обеспечение процессором кибернетической системы качественного управления информационными процессами в памяти}
\scnaddlevel{-1};
качество организации взаимодействия процессов решения задач во внешней среде или в физической оболочке кибернетической системы\\
\scnaddlevel{1}
    \scnrelfromlist{свойство-предпосылка}{последовательность/параллельность процессов решения задач в кибернетической системе; синхронность/асинхронность процессов решения задач в кибернетической системе; централизованной/децентрализованность управления процессами решения задач в кибернетической системе}
\scnaddlevel{-1}}
\scnnote{Качество решения каждой \textit{задачи} определяется:
\begin{scnitemize}
    \item временем её решения (чем быстрее \textit{задача} решается, тем выше качество её решения);
    \item полнотой и корректностью результата решения \textit{задачи};
    \item затраченными для решения \textit{задачи} ресурсами памяти (объемом фрагмента хранимой информации, используемой для решения задачи);
    \item затраченным для решения \textit{задачи} ресурсами решателя задач (количеством используемых внутренних агентов).
\end{scnitemize}
Таким образом, повышение качества процесса решения каждой конкретной \textit{задачи}, а также каждого \textit{класса задач} (путем совершенствования соответствующего метода, в частности, алгоритма) является важным фактором повышения качества \textit{решателя задач} в целом.}

\scnheader{агентно-ориентированная модель обработки информации в памяти}
\scnidtf{агентно-ориентированная модель управления действиями кибернетической системы, выполняемыми ею в своей памяти}
\scnrelfrom{семантическая окрестность}{Понятие Технологии OSTIS}
\scnaddlevel{1}
\scnsourcecommentpar{Сегмент 3 Раздела 0.2}
\scnaddlevel{-1}

\scnexplanation{Перспективным вариантом построения \textit{решателя задач кибернетической системы} является реализация \textit{агентно-ориентированной модели обработки информации}, т.е. построение \textit{решателя задач} в виде \textit{многоагентной системы}, агенты которой осуществляют обработку \textit{информации, хранимой в памяти} кибернетической системы, и управляются этой информацией (точнее, её текущим состоянием). Особое место среди этих \textit{агентов} занимают сенсорные (рецепторные) и эффекторные \textit{агенты}, которые, соответственно, воспринимают информацию о текущем состоянии \textit{внешней среды} и воздействуют на \textit{внешнюю среду}, в частности, путем изменения состояния \textit{физической оболочки кибернетической системы}.\\
Подчеркнем, что указанная агентно-ориентированная модель организации взаимодействия процессов решения задач в \textit{кибернетической системе} по сути есть не что иное, как модель ситуационного управления процессами решения задач, решаемых \textit{кибернетической системой} как в своей \uline{внешней среде}, так и в своей памяти.}

\scnheader{модель инициирования действий кибернетической системы}
\scnidtf{модель управления поведением кибернетической системы}
\scnsubdividing{стимульно-реактивная модель инициирования действий\\
\scnaddlevel{1}
    \scnexplanation{от комбинации \textit{исходных сигналов}, формируемых, например, априори известным набором сенсоров (рецепторов) к комбинации выходных \textit{сигналов}, управляющих, например, априори известным набором эффекторов}
\scnaddlevel{-1};
ситуационная модель инициирования действий без учета предыстории ситуаций и событий\\
\scnaddlevel{1}
\scnexplanation{действие инициируется возникновением в памяти \textit{ситуации} априори известной конфигурации или априори известного события}
\scnaddlevel{-1};
ситуационная модель инициирования действий с учетом предыстории ситуаций и событий\\
\scnaddlevel{1}
    \scnexplanation{действие инициируется не только текущей \textit{ситуацией} но и предшествующими \textit{ситуациями}, т.е. событиями перехода от одних \textit{ситуаций} к другим}
\scnaddlevel{-1}}
\scnnote{Речь идет о действиях, выполняемых \textit{кибернетической системой} как во внешней среде, так и в своей внутренней \textit{среде} (в своей памяти).}

\scnheader{последовательность/параллельность процессов решения задач в кибернетической системе}
\scnidtf{способность одновременно решать несколько разных задач, некоторые из которых могут быть подзадачами одной и той же задачи}
\scnidtf{способность одновременно решать несколько разных задач, некоторые из которых могут быть подзадачами одной и той же задачи}
\scnrelfromlist{свойство-предпосылка}{максимально возможное количество действий, одновременно выполняемых кибернетической системой;способность кибернетической системы к одновременному выполнению взаимосвязанных действий\\
	\scnaddlevel{1}
    \scnidtf{способность кибернетической системы к одновременному выполнению действий, выполнение каждого из которых может помешать выполнению другого}
    \scnidtf{способность кибернетической системы к “эквилибристике”}
\scnaddlevel{-1}}
\scnrelfromset{комплекс частных свойств}{физическая последовательность/параллельность процессов решения задач в кибернетической системе; логическая последовательность/параллельность процессов решения задач в кибернетической системе\\
\scnaddlevel{1}
    \scnexplanation{Логическая параллельность выполняемых процессов (действий) предполагает возможность существования \uline{выполняемых} процессов в двух режимах: 
    \begin{scnitemize}
    \item в активном режиме -- в режиме непосредственного выполнения;
    \item в режиме прерывания -- в режиме ожидания соответствующих условий (событий и/или ситуаций) при возникновении которых прерванный процесс переходит в режим активного процесса.
    \end{scnitemize}}
\scnaddlevel{-1}}

\scnrelfromset{комплекс частных свойств}{последовательность/параллельность информационных процессов в памяти кибернетической системы; последовательность/параллельность процессов решения задач во внешней среде или в физической оболочке кибернетической системы}

\scnnote{Подчеркнем, что есть целый ряд задач, решаемых кибернетической системой, процессы решения которых носят перманентный (постоянный) характер. К таким задачам относятся:
\begin{scnitemize}
    \item поддержка высокого качества базы знаний (устранение противоречий, информационного мусора);
    \item поддержка семантической совместимости с другими компьютерными системами;
    \item мониторинг и анализ состояния внешней среды;
    \item обеспечение собственной безопасности;
    \item самообучение.
\end{scnitemize}
}

\scnheader{быстродействие решателя задач кибернетической системы}
\scnidtf{скорость решения задач в кибернетической системе}
\scnidtf{быстродействие решателя задач кибернетической системы}
\scnidtf{скорость реакции кибернетической системы на различные задачные ситуации}
\scnrelfrom{свойство-предпосылка}{быстродействие процессора кибернетической системы}

\scnheader{способность кибернетической системы решать задачи, предполагающие использование информации, обладающей различного рода не-факторами}
\scnidtf{способность кибернетических систем решать задачи, которые: 
\begin{scnitemize}
    \item либо нечетко сформулированы ("делай то, не знаю что");
    \item либо решаются в условиях неполноты, неточности, противоречивости исходных данных;
    \item либо являются задачами, принадлежащими классам задач, для которых практически невозможно построить соответствующие алгоритмы.
\end{scnitemize}}
\scnidtf{способность кибернетической системы решать труднорешаемые, трудноформализуемые задачи}
\scnidtf{способность решать интеллектуальные (трудноформализуемые) задачи, для которых характерна:
\begin{scnitemize}
    \item неточность и недостоверность исходных данных;
    \item отсутствие критерия качества результата;
    \item невозможность или высокая трудоемкость разработки алгоритма;
    \item необходимость учета контекста задачи.
\end{scnitemize}}

\scnheader{задача, предполагающая использование информации, обладающей различного рода не-факторами}
\scnidtf{трудноформализуемая задача}
\scnsuperset{задача проектирования}
\scnsuperset{задача распознавания}
\scnsuperset{задача прогнозирования}
\scnsuperset{задача целеполагания}
\scnsuperset{задача планирования}

\scnheader{многообразие и качество решения задач информационного поиска}
\scnrelfrom{свойство-предпосылка}{семантический уровень доступа к информации, хранимой в памяти кибернетической системы}
\scnrelto{частное свойство}{многообразие видов задач, решаемых кибернетической системой}
\scnidtf{способность кибернетической системы качественно решать широкое многообразие задач информационного поиска в рамках текущего состояния хранимой информации}
\scnidtf{способность кибернетической системы находить в текущем состоянии хранимой информации релевантные ответы на запросы (вопросы) самого различного вида}

\scnheader{вопрос}
\scnidtf{запрос}
\scnsuperset{запрос изоморфных или гомоморфных фрагментов хранимой информации по заданному образцу с указанием знаков известных сущностей}
\scnaddlevel{1}
    \scnrelfrom{класс частных вопросов}{запрос всех связок различных отношений, обязывающих заданную сущность с другими}
    \scnaddlevel{1}
	    \scnrelfrom{класс частных вопросов}{запрос всех связок заданных отношений, обязывающих заданную сущность с другими}
    \scnaddlevel{-1}
\scnaddlevel{-1}

\scnsuperset{вопрос типа "как связаны между собой заданные две сущности"{}}
\scnaddlevel{1}
    \scnexplanation{Две сущности будем считать связанными в том и только в том случае, если существует маршрут, состоящий из связок, принадлежащих в общем случаем разным отношениям, и соединяющий указанные две сущности}
    \scnnote{Здесь принципиально важным является учет \textit{семантической силы связей} между сущностями, которая определяется \textit{семантической силой отношений}, которым принадлежат связки, входящие в состав связей (маршрутов) между сущностями.}
    \scnrelto{класс частных вопросов}{вопрос типа "как связаны между собой заданные сущности"{}}
    \scnaddlevel{1}
        \scnnote{Здесь имеется в виду произвольное количество связываемых сущностей, а это предполагает, что отсветом на данный запрос является \uline{связный граф}, вершинами которого являются знаки заданных сущностей.}
    \scnaddlevel{-1}
\scnaddlevel{-1}

\scnsuperset{вопрос типа "что это такое"{}}
\scnaddlevel{1}
    \scnidtf{запрос спецификации (описания) заданной сущности}
    \scnrelfrom{класс частных вопросов}{запрос определения}
    \scnaddlevel{1}
        \scnidtf{запрос определения заданного понятия}
    \scnaddlevel{-1}
    \scnrelfrom{класс частных вопросов}{запрос документации заданного объекта}
\scnaddlevel{-1}

\scnsuperset{почему-вопрос}
\scnaddlevel{1}
    \scnsuperset{запрос причины возникновения заданной ситуации или события}
    \scnsuperset{запрос логического обоснования заданного высказывания}
    \scnaddlevel{1}
        \scnidtf{запрос объяснения корректности заданного высказывания, которое, в частности, может быть порождено (сгенерировано) в процессе решения некоторой задачи с помощью некоторого метода (алгоритма, искусственной нейронной сети логического исчисления и т.п.)}
        \scnsuperset{запрос доказательства заданной теоремы}
    \scnaddlevel{-1}
\scnaddlevel{-1}

\scnsuperset{запрос возможных последствий заданной ситуации или события}
\scnsuperset{запрос того, что логически следует из заданного высказывания}
\scnsuperset{запрос метода решения данной задачи}
\scnsuperset{запрос плана решения данной задачи}
\scnaddlevel{1}
    \scnidtf{запрос декомпозиции данной задачи на систему и/или подзадач}
\scnaddlevel{-1}

\scnsuperset{зачем-вопрос}
\scnaddlevel{1}
    \scnidtf{каково назначение заданной сущности}
    \scnidtf{для решения какой задачи (для чего, достижения какой цели) нужна данная сущность}
\scnaddlevel{-1}

\scnsuperset{запрос аналогов заданной сущности}
\scnsuperset{запрос антиподов заданной сущности}
\scnsuperset{запрос сходств и отличий двух связанных сущностей}
\scnsuperset{запрос сравнительного анализа заданной сущности}
\scnaddlevel{1}
    \scnsuperset{запрос достоинств заданной сущности}
    \scnsuperset{запрос недостатков заданной сущности}
\scnaddlevel{-1}

\scnsuperset{где-вопрос}
\scnaddlevel{1}
    \scnidtf{запрос информации о местоположении заданной пространственной сущности примечание}
    \scnnote{Здесь запрашивается любая информация о пространственных связях заданной сущности}
\scnaddlevel{-1}

\scnsuperset{когда-вопрос}
\scnaddlevel{1}
    \scnidtf{запрос информации о темпоральных свойствах и связях заданной временной сущности (о моменте начала, о моменте завершения, о длительности)}
\scnaddlevel{-1}

\scnheader{cпособность кибернетической системы генерировать ответы на вопросы различного вида в случае, если они целиком или частично отсутствуют в текущем состоянии информации, хранимой в памяти}
\scnidtf{способность кибернетической системы генерировать (порождать, строить, синтезировать, выводить) ответы на самые различные вопросы и, в частности, на вопросы типа ``что это такое'', на почему-вопросы, это означает способность кибернетической системы \uline{объяснять} (обосновывать корректность) своих действий}
\scnrelfromlist{свойство-предпосылка}{семантическая гибкость информации, хранимой в памяти кибернетической системы; способность кибернетической системы к рассуждениям различного вида}

\scnheader{способность кибернетической системы к рассуждениям различного вида}
\scnidtf{способность кибернетической системы к целенаправленному порождению (генерации) новых истинных или правдоподобных знаний (следствий) на основе имеющихся знаний (посылок)}
\scnrelfromlist{частное свойство}{способность кибернетической системы к дедуктивному выводу;способность кибернетической системы к индуктивному выводу;способность кибернетической системы к абдуктивному выводу}

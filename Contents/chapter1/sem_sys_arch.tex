\begin{SCn}

\scnsectionheader{\currentname}

\scnstartsubstruct

\scnheader{семантическая совместимость компьютерных систем}
\scnexplanation{Уровень совместимости \textit{компьютерных систем} определяется трудоемкостью реализации процедур интеграции (объединения, соединения знаний этих систем), а также трудоемкостью и глубиной интеграции входящих в эти системы \textit{решателей задач} (навыков и интерпретаторов этих навыков). Подчеркнем при этом, что интеграция интеграции рознь -- от эклектики до гибридности и синергетичности дистанция огромного размера.

Совместимые \textit{компьютерные системы} -- это компьютерные системы, для которых существует автоматически выполняемая процедура их интеграции, превращающая эти системы в единую \textbf{\textit{гибридную систему}}, в рамках которой каждая исходная компьютерная система в процессе своего функционирования может свободно использовать любые необходимые информационные ресурсы (знания и навыки), входящие в состав другой исходной компьютерной системы.

Целостную \textit{компьютерную систему} можно рассматривать как решатель задач, интегрировавший несколько моделей решения задач и обладающий средствами взаимодействия с внешней для себя средой (с другими компьютерными системами, с пользователями).

Таким образом, для того, чтобы повысить уровень совместимости \textit{компьютерных систем}, необходимо преобразовать их к виду \textit{многоагентных систем}, работающих над общей семантической памятью, в которой информация представлена семантическими сетями. Такие \textit{компьютерные системы} не всегда целесообразно непосредственно объединять (интегрировать) в более крупные \textit{компьютерные системы}. Иногда целесообразнее их объединять в \textit{коллективы взаимодействующих компьютерных систем}. Но при создании таких коллективов компьютерных систем унификация и совместимость таких систем также очень важны, т.к. существенно упрощают обеспечение высокого уровня их взаимопонимания. Так, например, противоречия между компьютерными системами, входящими в коллектив, можно обнаруживать путем анализа на непротиворечивость \textbf{\textit{виртуальной объединенной базы знаний}} этого коллектива. Более того, непротиворечивость указанной виртуальной базы знаний можно считать одним из критериев семантической совместимости систем, входящих в соответствующий коллектив.}

\scnheader{семантическая компьютерная система}
\scnexplanation{Предлагаемое нами устранение проблем современных информационных технологий путем перехода к \textit{смысловому представлению информации} в памяти компьютерных систем фактически преобразует современные компьютерные системы (в том числе и современные интеллектуальные системы) в \textit{\textbf{семантические компьютерные системы}}, которые, следовательно, являются не альтернативной ветвью развития \textit{компьютерных систем}, а естественным этапом их эволюции, направленным на обеспечение высокого уровня их обучаемости и, в первую очередь, \textbf{совместимости}.

Архитектура \textit{семантических компьютерных систем} (см. \textit{Рис. Архитектура семантических компьютерных систем}) практически совпадает с архитектурой интеллектуальных компьютерных систем, основанных на знаниях. Отличие здесь заключаются в том, что в \textit{семантических компьютерных системах}:
\begin{scnitemize}
    \item база знаний имеет смысловое представление;
    \item интерпретатор знаний и навыков представляет собой коллектив \textit{агентов}, осуществляющих обработку \textit{базы знаний}.
\end{scnitemize}

Как следствие этого, \textit{семантические компьютерные системы} обладают высоким уровнем обучаемости, т.е. способностью быстро приобретать новые и совершенствовать уже приобретенные знания и навыки и при этом не иметь никаких ограничений на вид приобретаемых и совершенствуемых ею знаний и навыков, а также на их совместное использование.

Более того, при согласовании соответствующих стандартов, а также при перманентном совершенствовании этих стандартов и при грамотной их поддержке в условиях интенсивной эволюции как самих стандартов, так и \textit{семантических компьютерных систем} (речь идет о постоянной поддержке соответствия между текущим состоянием компьютерных систем и текущим состоянием эволюционируемых стандартов), \textit{семантические компьютерные системы} и их компоненты обладают весьма высокой степенью совместимости.

Это, в свою очередь, практически исключает дублирование инженерных решений и дает возможность существенно ускорить разработку \textit{семантических компьютерных систем} с помощью постоянно расширяемой библиотеки многократно используемых и совместимых между собой компонентов. 

Основным лейтмотивом перехода от современных компьютерных систем (в том числе интеллектуальных) к семантическим компьютерным системам, т.е. компьютерным системам, основанным на смысловом представлении всей информации, хранимой в ее памяти, является создание \textit{\textbf{общей семантической теории компьютерных систем}}, включающей в себя:
\begin{scnitemize}
    \item cемантическую теорию знаний и баз знаний;
    \item семантическую теорию задач и моделей их решения;
    \item cемантическую теорию взаимодействия информационных процессов;
    \item cемантическую теорию пользовательских и, в том числе, естественно языковых интерфейсов;
    \item cемантическую теорию невербальных сенсорно-эффекторных интерфейсов;
    \item теорию универсальных интерпретаторов семантических моделей компьютерных систем и, в частности, теорию семантических компьютеров.
\end{scnitemize}

Эпицентром следующего этапа развития информационных технологий является решение проблемы обеспечения \textbf{семантической совместимости} \textit{компьютерных систем} и их компонентов. Для решения этой проблемы необходим
\begin{scnitemize}
    \item переход от традиционных компьютерных систем и от современных интеллектуальных систем к \textit{семантическим компьютерным системам};
    \item разработка \textit{стандарта семантических компьютерных систем}.
\end{scnitemize}    
    
Очевидно, что \textit{семантические компьютерные системы} являются компьютерными системами нового поколения, устраняющие многие недостатки современных компьютерных систем. Но для массовой разработки таких систем необходима соответствующая технология, которая должна включать в себя  

\begin{scnitemize}        
    \item теорию \textit{семантических компьютерных систем} и комплекс всех стандартов, обеспечивающих совместимость разрабатываемых систем;
    \item методы и средства проектирования \textit{семантических компьютерных систем};
    \item методы и средства перманентного совершенствования самой технологии.
\end{scnitemize}
}

\scnheader{Рис. Архитектура семантических компьютерных систем}
\scneqfile{\\\includegraphics[width=0.5\linewidth]{figures/arch.pdf}\\}

\scnendstruct

\end{SCn}
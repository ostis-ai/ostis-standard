\begin{SCn}

\scnheader{рафинированная семантическая сеть}
\scntext{основной принцип}{Абсолютно ничего лишнего, не имеющего отношения к смыслу представляемой информации}
\scnidtf{\uline{предельно} компактная (сжатая) смысловая информационная модель соответствующей системы рассматриваемых (описываемых, исследуемых, моделируемых) сущностей}
\scnnote{Указанная система рассматриваемых сущностей представляет собой конфигурацию связей между этими сущностями. Подчеркнем при этом, что указанные связи между рассматриваемыми сущностями также входят в число рассматриваемых сущностей.}

\scnidtf{\textit{информационная конструкция}, являющаяся результатом максимально возможного упрощения ее \textit{синтаксической структуры} при обеспечении представления \uline{любой} \textit{информации}, что приводит к фактическому слиянию синтаксических и семантических аспектов представления \textit{информации}}
\scnidtf{\textit{семантическая сеть} "внутреннего"\ потребления, используемая для \textit{смыслового представления информации} в памяти \textit{компьютерных систем}}
\scnidtf{уточнение принципов \textit{смыслового представления информации}, которое основано, \uline{во-первых}, на четком противопоставлении \textit{внутреннего языка компьютерной системы}, используемого для хранения информации в памяти компьютера, и \textit{внешних языков компьютерной системы}, используемых для общения (обмена сообщений) \textit{компьютерной системы} с пользователями и другими \textit{компьютерными системами} (рафинированная семантическая сеть используется исключительно для \textit{внутреннего представления информации} в памяти \textit{компьютерной системы}), и, \uline{во-вторых} на максимально возможном упрощении \textit{синтаксиса внутреннего языка компьютерной системы} при обеспечении \uline{универсальности} путем исключения из такого внутреннего универсального языка средств, обеспечивающих коммуникационную функцию \textit{языка} (т.е. обмен сообщениями). 
\newline
Так, например, для \textit{внутреннего языка компьютерной системы} излишними являются такие коммуникационные средства \textit{языка}, как союзы, предлоги, разделители, ограничители, склонения, спряжения и другие.
\newline
\textit{Внешние языки компьютерной системы} могут быть как близки ее внутреннему языку, так и весьма далеки от него (как, например, \textit{естественные языки}).}
\scnidtf{\uline{абстрактная} знаковая конструкция, принадлежащая \uline{универсальному} внутреннему языку компьютерных систем и являющаяся \uline{инвариантом} соответствующего максимального множества семантически эквивалентных знаковых конструкций (текстов), принадлежащих самым различным языкам}

\scnrelfromvector{принципы лежащие в основе}{\scnfileitem{Каждый фрагмент \textit{рафинированной семантической сети} является либо \textit{знаком} (элементарным фрагментом, представленным либо \textit{узлом}, либо \textit{ребром}, либо \textit{дугой}), либо множеством \textit{знаков}, связанных между собой отношением \textit{инцидентности} элементов \textit{рафинированной семантической сети}. Указанное отношение \textit{инцидентности} является \textit{бинарным ориентированным отношением}, связывающим \textit{знаки} описываемых \textit{связей} (которые представляются \textit{ребрами}, \textit{дугами} и \textit{узлами}, если описываемая связь является небинарной) со \textit{знаками}, которые либо обозначают связываемые \textit{сущности}, либо сами являются такими сущностями}
\scnaddlevel{1}
\scntext{следовательно}{В состав \textit{рафинированной семантической сети} не входят такие средства синтаксической структуризации знаковых конструкций, как \textit{разделители} и \textit{ограничители}. Любая структуризация \textit{рафинированных семантических сетей} описывается явно с помощью метаязыковых средств путем:
\begin{scnitemize}
	\item введения узлов \textit{рафинированной семантической сети}, обозначающих различные \uline{не\-э\-ле\-мен\-тар\-ные} фрагменты этой семантической сети, являющиеся \textit{множествами} узлов, ребер и дуг, входящих в состав обозначаемого фрагмента;
	\item введения \textit{дуг принадлежности}, связывающих введенные \textit{узлы}, обозначающие неэлементарные фрагменты \textit{рафинированной семантической сети}, с элементами обозначаемых ими \textit{множеств};
	\item введения целого ряда \textit{отношений}, связывающих неэлементарные фрагменты \textit{рафинированной семантической сети} с другими фрагментами, а также с сущностями других видов;
	\item введения различных классов неэлементарных фрагментов \textit{рафинированной семантической сети}.
\end{scnitemize}}
\scnaddlevel{-1};
\scnfileitem{Абсолютно все \textit{знаки}, входящие в состав \textit{рафинированной семантической сети}, являются синтаксически элементарными (атомарными) фрагментами \textit{рафинированной семантической сети}, т.е. фрагментами, "внутренняя"\ структура которых не имеет никакого значения для семантического анализа и понимания \textit{рафинированной семантической сети}. Множеству \textit{знаков}, входящих в \textit{рафинированную семантическую сеть}, как и множеству \textit{букв}, входящих в обычный \textit{текст}, ставится в соответствие \textit{алфавит}, определяющий \uline{синтаксическую типологию} таких элементарных фрагментов \textit{рафинированной семантической сети}. При этом, если \textit{алфавит} букв обычного \textit{текста} не имеет никакой семантической интерпретации, то \textit{алфавит} элементарных фрагментов \textit{рафинированной семантической сети} имеет четкую семантическую интерпретацию -- каждый элемент этого \textit{алфавита} обозначает класс знаков \textit{сущности}, \uline{синтаксический тип} которых соответствует указанному элементу \textit{алфавита} (задается этим элементом \textit{алфавита знаков}, входящих в состав \textit{рафинированной семантической сети})}
\scnaddlevel{1}
\scntext{следовательно}{Таким образом, \textit{знаки}, входящие в \textit{рафинированную семантическую сеть}, не являются \textit{именами} (терминами) и, следовательно, не привязаны ни к какому \textit{естественному языку} и не зависят от субъективных терминотворческих пристрастий различных авторов. Это значит, что при коллективной разработке \textit{рафинированных семантических сетей}, соответствующих каким-либо информационным ресурсам, терминологические споры практически исключены.}
\scntext{следовательно}{В \textit{рафинированной семантической сети} нет необходимости использовать синтаксически элементарные фрагменты, \uline{не} являющиеся знаками описываемых \textit{сущностей}, т.е. фрагменты \textit{информационной конструкции}, из которых сторятся \textit{простые знаки}, \textit{выражения}, а также различные разделители и ограничители. Более того, в \textit{рафинированной семантической сети} нет необходимости противопоставлять \textit{простые знаки} и \textit{выражения}. Как \textit{простым знакам}, так и \textit{выражениям} в \textit{рафинированной семантической сети} соответствуют элементы этой сети, имеющие аналогичные \textit{денотаты}. Но при этом \textit{выражениям} дополнительно соответствуют семантически эквивалентные неэлементарные фрагменты \textit{рафинированной семантической сети}, которые специфицируют \textit{сущности}, обозначаемые этими \textit{выражениями}.}
\scnaddlevel{-1};
\scnfileitem{Абсолютно ве описываемые \textit{связи} между описываемыми сущностями в \textit{рафинированной семантической сети} представляются \uline{явно} в виде соответствующих \textit{знаков}, обозначающих эти \textit{связи} и инцидентных знакам связываемых \textit{сущностей}. Для бинарных связей, связывающих \uline{две} описываемые сущности, \textit{знаком} связей являются \textit{ребра} или \textit{дуги} \textit{рафинированной семантической сети}.}
\scnaddlevel{1}
\scntext{следовательно}{В \textit{рафинированных семантических сетях} нет необходимости использовать такие средства, как склонения, спряжения, род (мужской, женский, средний), семантически интерпретируемая последовательность слов.}
\scnaddlevel{-1};
\scnfileitem{Все \textit{знаки}, входящие в состав \textit{рафинированной семантической сети}, входят в нее \uline{однократно}. Т.е. в рамках \textit{рафинированной семантической сети} отсутствуют пары \textit{синонимичных знаков}, т.е. \textit{знаков}, имеющих один и тот же \textit{денотат}. Таким образом, разные элементы \textit{рафинированной семантической сети} априори считаются знаками \uline{разных} сущностей. При этом эти знаки могут принадлежать одному и тому же синтаксическому типу, т.е. одному и тому же элементу алфавита соответствующего языка \textit{рафинированных семантических сетей}. Таким образом, в \textit{рафинированных семантических сетях} отсутствует синонимия не только \textit{знаков}, имеющих одинаковую синтаксическую структуру, не только знаков, имеющих одинаковый синтаксический тип, но также и просто \uline{разных} знаков.}
\scnaddlevel{1}
\scntext{следовательно}{Появление в рафинированной семантической сети синонимичных знаков превращает эту семантическую сеть в некорректную и требует отождествления (склеивания) обнаруженных синонимичных знаков.}
\scnaddlevel{-1};
\scnfileitem{В рамках \textit{рафинированной семантической сети} отсутствуют \textit{синонимичные знаки}, т.е. \textit{знаки}, которые имеют не один, а несколько \textit{денотатов}, каждому из которых соответствует свой контекст (ракурс) семантической трактовки этого \textit{знака}.}
\scnaddlevel{1}
\scnnote{Когда речь идет об омонимии знаков в привычных нам языках, имеется в виду омонимия \uline{разных} знаков, имеющих одинаковую синтаксическую структуру, т.е. омонимия разных вхождений, разных экземпляров \uline{синтаксически эквивалентных}, но семантически различных знаков. Очевидным примером такого рода омонимии являются различного вида местоимения.}
\scnaddlevel{-1};
\scnfileitem{В рамках каждой \textit{рафинированной семантической сети} отсутствует дублирование информации не только в виде многократного вхождения \textit{синонимичных знаков}, т.е. \textit{знаков} с одинаковыми денотатами, но также и в виде многократного вхождения \textit{семантически эквивалентных} \textit{рафинированных семантических сетей}. Две \textit{рафинированные семантические сети} являются \textit{семантически эквивалентными} в том и только в том случае, если:
\begin{scnitemize}
	\item они \textit{изоморфны};
	\item пары соответствия указанного \textit{изоморфизма} связывают \textit{синонимичные знаки}. 
\end{scnitemize}
Таким образом, полное исключение \textit{омонимии знаков} является необходимым и достаточным условием исключения \textit{семантически эквивалентных рафинированных семантических сетей}. Подчеркнем при этом, что запрет \textit{семантической эквивалентности} в рамках \textit{рафинированной семантической сети} не означает запрета \textit{логической эквивалентности} фрагментов \textit{рафинированной семантической сети}. Логическая эквивалентность необходима для обеспечения компактности представления некоторых знаний. Тем не менее, логической эквивалентностью хранимых в памяти знаковых конструкций увлекаться не следует, т.к. \uline{\textit{логически эквивалентные}} знаковые конструкции -- это представление одного и того же \textit{знания}, но с помощью \uline{\textit{разных наборов понятий}}. В отличие от этого \uline{\textit{семантически эквивалентные}} \textit{знаковые конструкции} -- это представление одного и того же \textit{знания} с помощь одних и тех же \textit{понятий}. Очевидно, что многообразие возможных вариантов представления одних и тех же \textit{знаний} в памяти компьютерной системы существенно усложняет решение \textit{задач}. Поэтому, полностью исключив \textit{семантическую эквивалентность} в смысловой памяти, необходимо стремиться к минимизации \textit{логической эквивалентности}. Для этого необходимо грамотное построение системы используемых \textit{понятий} в виде иерархической системы формальных \textit{онтологий}.}
\scnaddlevel{1}
\scntext{следовательно}{Интеграция (соединение, объединение) двух \textit{рафинированных семантических сетей}, в результате чего могут появиться семантически эквивалентные фрагменты, сводится к тому, чтобы результат такого соединения был приведен в соответствие с требованием отсутствия синонимии элементов и семантической эквивалентности фрагментов \textit{рафинированной семантической сети}.}
\scnaddlevel{-1};
\scnfileitem{\textit{Рафинированные семантические сети} должны быть \uline{универсальными}, т.е. должны обеспечивать представление \uline{любой} информации, в том числе, и \textit{метаинформации}, обеспечивающей описание различных связей, свойств и закономерностей самих \textit{рафинированных семантических сетей}, на множестве которых, в частности, заданно \textit{отношение} "быть подструктурой*"\, которое связывает \textit{рафинированные семантические сети} с их фрагментами (частями), т.е. с теми \textit{рафинированными семантическими сетями}, которые входят в их состав.
\newline
Каждая \textit{рафинированная семантическая сеть} трактуется как множество \textit{знаков} \uline{взаимно однозначно} соответствующих обозначаемым ими \textit{сущностям} (денотатам этих \textit{знаков}) и множество \textit{связей} между этими \textit{знаками}.
\newline
Каждая \textit{связь} между \textit{знаками} трактуется, с одной стороны, как множество \textit{знаков}, связываемых этой \textit{связью}, а, с другой стороны, как описание (отражение, модель) соответствующей \textit{связи}, которая связывает денотаты указанных \textit{знаков} или денотаты одних \textit{знаков} непосредственно с другими \textit{знаками}, или сами эти \textit{знаки}. Примером первого вида \textit{связи} между \textit{знаками} является связь между \textit{знаками} \textit{материальных сущностей}, одна из которых является частью другой. Примером второго вида \textit{связи} между \textit{знаками} является \textit{связь} между знаком, входящим в состав внутреннего смыслового представления информации, и знаком файла, являющегося электронным отражением структуры представления указанного \textit{знака} во внешних \textit{знаковых конструкциях}. Примерами третьего вида \textit{связи} между \textit{знаками} является \textit{связь} между синонимичными знаками.
\newline
Денотатами \textit{знаков} могут быть \uline{любые} описываемые сущности, причем: (1) не только конкретные (константные, фиксированные), но и произвольные (переменные, нефиксированные)  сущности, "пробегающие"\ различные множества знаков (возможных значений), (2) не только реальные (материальные), но и абстрактные сущности (например, числа, точки различных абстрактных пространств), (3) не только "внешние"\, но и "внутренние"{} сущности, являющиеся множествами знаков, входящих в состав той же самой знаковой конструкции, хранимой в памяти компьютерной системы.};
\scnfileitem{Поскольку \textit{рафинированные семантические сети} ориентированы на \textit{смысловое представление информации} в памяти \textit{компьютеров нового поколения}, необходимо, с одной стороны, использовать накопленный полезный опыт представления информации в \textit{современных компьютерах}, а, с другой стороны, обеспечить взаимодействие \textit{компьютерных систем}, построенных на \textit{современных компьютерах}, с \textit{компьютерными системами}, построенными на \textit{компьютерах нового поколения}. Для этой цели в памяти \textit{компьютеров нового поколения} можно и нужно обеспечить обработку и хранение различного вида \textit{информационных конструкций}, представленных в различных широко используемых форматах. И ничто не препятствует такие \textit{информационные конструкции}, хранимые в памяти \textit{компьютера нового поколения} и не являющиеся \textit{рафинированными семантическими сетями}, рассматривать как \textit{сущности}, описываемые \textit{рафинированной семантической сетью}, хранимой в памяти этого \textit{компьютера нового поколения}. Такой вид \textit{сущностей}, описываемых \textit{рафинированной семантической сетью} и хранимых в той же \textit{памяти}, будем называть \textit{файлами}, описываемыми соответствующуей \textit{рафинированной семантической сетью}, т.е. "электронными"{} \ образами (копиями) соответствующих \textit{информационных конструкций}. Таким образом, среди \textit{узлов рафинированной семантической сети} появляются \textit{узлы}, являющиеся знаками \textit{файлов}, т.е. \textit{узлы}, денотаты (обозначаемые \textit{сущности}) которых находятся (хранятся) в той же памяти, что и обозначающие их \textit{узлы}.}
\scnaddlevel{1}
\scntext{следовательно}{Ничто не мешает в виде \textit{файла}, описываемого \textit{рафинированной семантической сетью}, хранить \textit{имя} (термин) какой-либо \textit{сущности}, описываемой этой же семантической сетью, а также связать это \textit{имя} (точнее, узел, обозначающий это \textit{имя}) с тем элементом \textit{рафинированной семантической сети}, который обозначает ту же описываемую \textit{сущность}.}
\scnaddlevel{-1};
\scnfileitem{Следствием указанных принципов \textit{рафинированных семантических сетей} является также то, что эти принципы приводят к нелинейным \textit{знаковым конструкциям} (к \textit{графовым структурам}), что усложняет реализацию \textit{памяти компьютерных систем}, но существенно упрощает ее логическую организацию (в частности, ассоциативный доступ).
\newline
Нелинейность \textit{рафинированных семантических сетей} обусловлена тем, что:
\begin{scnitemize}
	\item каждая описываемая \textit{сущность}, т.е. \textit{сущность}, имеющая соответствующий ей \textit{знак}, может иметь неограниченное число \textit{связей} с другими описываемыми \textit{сущностями};
	\item каждая описываемая \textit{сущность} в смысловом представлении имеет единственный \textit{знак}, т.к. синонимия \textit{знаков} здесь запрещена;
	\item все \textit{связи} между описываемыми \textit{сущностями} описываются (отражаются, моделируются) \textit{связями} между \textit{знаками} этих описываемых \textit{сущностей}.
\end{scnitemize}}
\scnaddlevel{1}
\scnnote{Напомним, что нелинейность информационных конструкций характерна не только для рафинированных, но и для нерафинированных семантических сетей.}
\scnaddlevel{-1}
}

\scnsuperset{SC-код}
\scnaddlevel{1}
\scnidtf{Semantic Computer Code}
\scniselement{универсальный формальный язык}
\scniselementrole{ключевой знак}{Описание внутреннего языка ostis-сиcтем}
\scnaddlevel{1}
\scnsourcecommentpar{Раздел 0.3.1}
\scnaddlevel{-1}
\scnexplanation{В качестве \textit{стандарта} \uline{универсального} \textit{смыслового представления информации} \textit{в памяти компьютерных систем} нами предложен SC-код (Semantic Computer Code). В отличие от УСК \textit{Мартынова В.В.}, он, во-первых, носит нелинейный характер и, во-вторых, специально ориентирован на кодирование информации в памяти компьютеров \uline{нового поколения}, ориентированных на разработку семантически совместимых \textit{интеллектуальных компьютерных систем} и названных нами \textit{семантическими ассоциативными компьютерами}. Более подробно это понятие (\textit{SC-код}) рассмотрено в разделе \textit{Предметная область и онтология внутреннего языка osts-систем}. Таким образом, основым лейтмотивом предлагаемого нами \textit{смыслового представления информации} является ориентация на формальную модель памяти \textit{компьютерных} \uline{не}фон-неймановского \textit{компьютера}, предназначенного для реализации \textit{интеллектуальных систем}, использующих \textit{смысловое представление информации}. Особенностями такого представления являются следующие:
\begin{scnitemize}
	\item ассоциативность памяти;
	\item поскольку при смысловом представлении информациия содержится в конфигурации связей между знаками, переработка информации сводится к реконфигурации этих связей (к графодинамическим процессам);
	\item прозрачная семантическая интерпретируемость и, как следствие, \textit{семантическая совместимость}.
\end{scnitemize}
Подчеркнем что, неявная привязка к фон-неймановским \textit{компьютерам} присутствует во всех известных \textit{моделях представления знаний}. Одним из примеров такой зависимости, является, например, обязательность именования описываемых объектов.} 
\scnaddlevel{-1}
\scnrelfromset{достоинства}{\scnfileitem{рафинированная семантическая сеть есть \uline{объективный}, не зависящий от субъективизма и многообразия синтаксических решений, способ представления информации};
\scnfileitem{в рамках \textit{рафинированной семантической сети} существенно упрощается процедура \textit{интеграции знаний} и погружения новых знаний в \textit{базу знаний}};
\scnfileitem{существенно упрощается процедура приведения различного вида \textit{знаний} к общему виду (к согласованной системе используемых \textit{понятий})};
\scnfileitem{существенно упрощается процедура интеграции различных \textit{решателей задач} и целых \textit{компьютерных систем}};
\scnfileitem{существенно упрощается автоматизация перманентного процесса \textit{поддержки семантической совместимости} (согласованности \textit{понятий} и \textit{онтологий}) для \textit{компьютерных систем} в условиях их постоянного совершенствования};
\scnfileitem{в рамках \textit{рафинированных семантических сетей} достаточно легко осуществляется переход от информационных конструкций к информационным \uline{мета}конструкциям путем введения узлов \textit{семантической сети}, обозначающих \textit{информационные конструкции}, а также дуг, связывающих эти узлы со всеми элементами обозначаемой ими \textit{информационной конструкции}};
\scnfileitem{на основе \textit{рафинированных семантических сетей} существенно упрощается интеграция различных дисциплин в области \textit{Искуственного интеллекта}, т.е. построение \textit{Общей формальной теории интеллектуальных компьютерных систем}, так как для построения общей формальной модели \textit{интеллектуальных компьютерных систем} необходим базовый \textit{язык}, в рамках которого можно было бы легко переходить от информации (от \textit{знаний}) к \textit{метаинформации} (к метазнаниям, к спецификациям исходных \textit{знаний}). Это потверждается тем, что:
\begin{scnitemize}
	\item подавляющее число \textit{понятий} \scnbigspace \textit{Искусственного интеллекта} носит метаязыковой характер;
	\item формальное смысловое уточнение почти каждого \textit{понятия} \scnbigspace \textit{Искусственного интеллекта} требует предшествующего формального уточнения соответсвующего языка-объекта. Так, например, как можно строго говорить о \textit{языке онтологий} (т.е. \textit{языке} спецификации \textit{предметных областей}), не уточнив \textit{язык} представления самих этих \textit{предметных областей}. как можно строго говорить о \textit{языке} описания способов обработки \textit{информации}, не уточнив \textit{язык }представления самой этой обрабатываемой \textit{информации}.
\end{scnitemize}}}
\end{SCn}
\begin{SCn}
 
\scnsectionheader{\currentname}
    
\scnstartsubstruct
    
\scnheader{Предметная область речевых интерфейсов компьютерных систем}
\scniselement{предметная область}
    
\scnsdmainclasssingle{интерфейс компьютерной системы}
\scnsdclass{естественно-языковой интерфейс}
\scnsdrelation{диалоговый интерфейс}

\scnheader{автоматическое распознавание речевого собщения (Automatic Speech Recognition -- ASR)}
\scnsubset{обработка естественно языкового сообщения (Natural Language Processing -- NLP)}
\scnexplanation{процесс автоматического анализа речевого сигнала и получения данных о том, что было сказано пользователем, без определения смысловой составляющей \textit{(синтаксический уровень)}. Наиболее часто применяется для преобразования информацию из речевой в текстовую форму.}

\scnheader{понимание речевого сообщения (Spoken Language Understanding -- SPU)}
\scnsubset{понимание естественно языковго сообщения (Natural Language Understanding -- NLU)}
\scnexplanation{процесс \textit{автоматического респознавания речевого сообещения} а также выделние из сообщения данных и знаний о смысле высказывания \textit{(семантический уровень)}. Применяется для семантического анализа речевого сообщения.}

\scnheader{понимание речевого сообщения}
\filemodetrue
\scnrelfromset{ограничения существующих систем}{
Большинство современных систем понимания смысла построены на основе трехуровневой архитектуры, когда речевое сообщение последовательно проходит этапы акустического анализа речевого сигнала, лингвистического анализа, в результате которого получается текстовая форма представления исходного сообщения, а уже только потом производится его семантический анализ. Однако, из работ по психолингвистике и когнитивной психологии известно, что процессы восприятия и понимания в человеческом сознании протекают непрерывно [21], [27], и в общем случае нет необходимости в предварительном приведении речевого сообщения к текстовой форме для выполнения смыслового анали за его содержимого. Устная и письменная формы речи с равным успехом могут быть обработаны сенсорной и когнитивной системами человека [22]. Поэтому вопрос создания методов и систем в которых осуществляется непосредственный переход от обработки сообщения в речевой форме к анализу смыслового его содержимого (семантико-акустического анализа) является весьма актуальным.
;Классический трехуровневый подход (акустический, лингвистический, семантический анализ), в случае решения задачи понимания речевых сообщений, обладает рядом существенных недостатков:
\scnaddlevel{1}
;введение промежуточного этапа преобразования речевого сигнала в текст, влечет дополнительные накладные расходы, связанные с необходимостьюлингвистической обработки, увеличивая тем самым общую вычислительную сложность алгоритма;
;наличие текстового этапа обработки привносит дополнительные ошибки и искажения в следствие ограничений и неполного соответствия лингвистических моделей описываемому процессу, используемых для перехода к текстовому представлению информации на различных стадиях преобразования (фонема-морфема, морфема-лексема, лексема-словосочетание и т.д.) [14];
;при переводе речевого сигнала в текст теряется часть информации, которая может оказаться важной для понимания смысла сообщения, например, громкость, продолжительность звучания, интонация, паузы между словами, которые в тексте могут не всегда однозначно выражаться знаками препинания и др. Особенно актуальной эта проблема становится при анализе сообщений, которые не являются полными предложениями, но при этом могут быть интерпретированы слушателем. Так, например, в повседневной речи предложение, состоящее из одного только звука [a] в зависимости от громкости, интонации и продолжительности звучания может выражать боль, удивление, вопрос, выступать союзом или частицей («ааа, ну его...», «а кто это?», «а если бы сделали по-другому...») [26]
\scnaddlevel{-1}
;Перевод звукового сигнала в текст делает невозможным анализ аудиофрагментов, не являющихся речевыми сообщениями, но несущих потенциально
важную для системы информацию, например:
\scnaddlevel{1}
;условных сигналов, издаваемых объектами внешней среды, в частности, оборудованием на производстве, автомобилями на дороге и др;
;звуков, которые могут соответствовать нештатным ситуациям или сигнализировать об опасности (грохот, лязг, шипение, взрывы, т.д.)
;других звуков, которые потенциально несут информацию о состоянии окружающей среды автоматизированной системы
\scnaddhind{-1}
;Отсутствие средств анализа такого рода сигналов сильно ограничивает возможности автоматизированных систем, ориентированных на работу в постоянно меняющейся среде, в том числе - трудно предсказуемой.
}
\filemodefalse

\scnheader{семантико-акустический анализ}
\scnexplanation{процесс подразумевает первичный разбор речевого сообщения с использованием специальных техник обработки сигнала. В ходе их применения про-
изводится вычленение из потока отдельных «акустических образов» слов, которые в свою очередь будут соответствовать определенным узлам (знакам конкретных сущностей или понятий) в семантической сети.
Предполагается, что результаты этапа акустического анализа будут итерационно корректироваться с учетом информации хранящейся в базе знаний системы, в том числе за счет семантического анализа контекстно-зависимой информации.
}

\scnendstruct

\end{SCn}
    
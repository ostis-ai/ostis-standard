\begin{SCn}

\scnheader{язык смыслового представления информации}
\scnidtf{смысловой язык}
\scnidtf{семантический язык}
\scnsubdividing{язык смыслового представления информации, не являющийся языком семантических сетей;
язык семантических сетей}

\scnheader{язык семантических сетей}
\scnexplanation{Несмотря на то, что синтаксическая структура семантической сети во многом носит \uline{объективный} характер, поскольку определяется конфигурацией описываемых связей между описываемыми сущностями. Тем не менее, можно говорить о разных \textit{языках семантических сетей}, каждому из которых соответствует свой \textit{алфавит*} элементов (синтаксически атомарных фрагментов) \textit{семантических сетей}. При атом языки семантических сетей могут быть как специализированными, так и универсальными. Задача каждого из этих \textit{языков} -- обеспечить в рамках \textit{языка} полное отсутствие многообразия синтаксических форм представления одной и той же информации.}

\scnsubset{язык}
\scnaddlevel{1}
\scnidtf{множество информационных конструкций, для которого существуют, причем не обязательно в формализованном виде, (1) правила построения синтаксически корректных информационных конструкций, а также (2) правила, позволяющие установить семантическую корректность правильно построенных (синтаксически корректных) информационных конструкций}
\scnaddlevel{-1}
\scnidtf{язык, информационными конструкциями которого являются семантические сети и в рамках которого обеспечивается полное отсутствие многообразия форм представления одной и той же информации}
\scnidtf{графовый (нелинейный) язык смыслового представления информации}

\scnsubdividing{специализированный язык семантических сетей
\scnaddlevel{1}
\scnidtf{язык семантических сетей, семантическая мощность которого ограничена соответствующей предметной областью}
\scnaddlevel{-1};
универсальный язык емантических сетей
\newline
\scnaddlevel{1}
\scnnote{Человечество давно и широко использует различные специализированные языки семантических сетей -- язык принципиальных электрических схем, язык блок-схем программ, язык генеалогических деревьев и др. Но в настоящее время актуальным является создание такого \textit{универсального языка семантических сетей}
\begin{scnitemize}
	\item синтаксис и семантика которого были бы максимально просты;
	\item по отношению к которому все используемые специализированные языки были бы его подъязыками*;
	\item который был бы приспособлен к использованию в качестве внутреннего языка интеллектуальных компьютерных систем и компьютеров следующего поколения;
	\item который был бы удобной основой как для обмена информацией между интеллектуальными компьютерными системами, так и для общения интеллектуальных клмпьютерных систем с их пользователями
\end{scnitemize}
\scnaddlevel{-1}
}}
\scnsubdividing{язык нерафинированных семантических сетей;
язык рафинированных семантических сетей}

\scnheader{следует отличать*}
\scnhaselementset{язык семантических сетей
\scnaddlevel{1}
\scnidtf{язык семантических сетей, рассматриваемых как \uline{абстрактные} графовые структуры, в которых не уточняется способ их кодирования}
\scnhaselement{SC-код}
\scnaddlevel{-1};
графодинамический язык семантических сетей
\scnaddlevel{1}
\scnidtf{язык графического изображения (визуализации) семантических сетей}
\scnidtf{язык, текстами которого являются рисунки семантичеких сетей}
\scnhaselement{SCg-код}
\scnaddlevel{-1}}

\scnheader{универсальный язык семантических сетей}
\scnnote{Если ставить задачу разработки \uline{универсального}(!) языка, текстами которого являются графовые структуры, то классических графовых структур явно недостаточно. Так, например:
\begin{scnitemize}
	\item по аналогии с переходом от ребер к ребрам и гиперребрам необходим переход от дуг к ориентированным связкам, связывающим более чем два компонента и в рамках которых эти компоненты могут иметь разные роли, которые необходимо явно указывать (классическим видом таких связок являются кортежи);
	\item в семантических сетях, представляющих некоторые виды знаний, некоторые связки (ребра, дуги, гиперребра, ориентированные связки, связывающие более двух компонентов) могут быть компонентами других связок;
	\item в семантических сетях, представляющих различного вида метазнания необходимо вводить узлы, обозначающие целые фрагменты (подграфы) этих же семантических сетей, и, соответственно, вводить дуги, связывающие каждый из этих узлов со всеми элементами подграфа, обозначаемого этим узлом.
\end{scnitemize}}

\end{SCn}
\begin{SCn}
	
\scnsectionheader{\currentname}
	
\scnstartsubstruct
	
\scnheader{Подходы к генерации пользовательского интерфейса, рассматривающие различные его аспекты:}
\begin{itemize}
	\item подход на основе специализированных языков описания;
	\item контекстно-зависимый подход;
	\item подход на основе данных;
	\item онтологический подход.	

Подход на основе специализированных языков описания предполагает представление конкретного пользовательского интерфейса в платформенно независимом виде. В качестве примеров языков описания интерфейса можно привести \scncite{UIML}, \scncite{UsiXML}, \scncite{XForms} и \scncite{JavaFX FXML}. Ключевой идеей представленных языков является создание модели диалогов и форм интерфейса в независимом от используемой технологии виде, описание визуальных элементов, а также взаимосвязей между ними и их свойств для создания конкретного пользовательского интерфейса.

Контекстно-зависимый подход интегрирует использование структурного описания интерфейса на основе языков описания с поведенческой спецификацией, то есть генерация интерфейса основывается на действиях пользователя. В рамках подхода специфицируются переходы между различными видами конкретного пользовательского интерфейса. В качестве примеров реализации такого подхода можно привести \scncite{CAP3} и \scncite{MARIA}.

Подход на основе данных или моделеориентированный подход использует модель предметной области в качестве основы для создания пользовательских интерфейсов. К реализациям можно отнести \scncite{JANUS} и \scncite{Mecano}.

Существующие онтологические подходы как правило основаны на представленных ранее подходах и используют онтологии в качестве способа представления информации о конкретном пользовательском интерфейсе. Например, по аналогии с подходом на основе специализированных языков описания, был предложен фреймворк \scncite{b10}, использующий онтологию для описания пользовательского интерфейса на основе понятий, хранящихся в базе знаний. По аналогии с контекстно-зависимым подходом в рамках работы \scncite{gaulke} используется модель предметной области совместно с моделью пользовательского интерфейса, ассоциированная с онтологией действий. Проект \scncite{ActiveRaUL} совмещает UIML с моделеориентированным подходом. В рамках данного проекта онтологическая модель предметной области сопоставляется с онтологическим представлением пользовательского интерфейса. Подход, предложенный в \scncite{hitz}, совмещает данные приложения с онтологией пользовательского интерфейса для создания единого описания в базе знаний с целью последующей автоматической генерации различных вариантов интерфейса для приложений-опросников с готовыми сценариями взаимодействия с пользователем. Следует также отметить работы \scncite{vladivostok1} и \scncite{vladivostok2}, в рамках которых предложена концепция, позволяющая объединить однородную по содержанию информацию в компоненты модели интерфейса, освободить разработчика интерфейса от кодирования и формировать информацию для каждого компонента модели интерфейса с помощью редакторов, управляемых соответствующими моделями онтологий.

Принцип генерации интерфейса на основе декларативного описания лежит в основе ряда прикладных проектов. Так, проект \scncite{mermaid} позволяет автоматически генерировать диаграммы исходя из их описания, а проект \scncite{rjsf} позволяет осуществлять генерацию форм для ввода данных пользователем. Помимо этого, общий подход к генерации и отображению интерфейса на основе его описания со стороны серверной части приложения можно также встретить в промышленной разработке под терминами \textbf{Server-Driven UI} или \scncite{Backend-Driven UI}.
	
	
	
\end{itemize}
\scnrelfromset{недостатки существующих решений для генерации пользовательского интерфейса}{\scnfileitem{как правило, спецификация модели интерфейса является неполной, что приводит к сложности автоматизации процесса генерации пользовательского интерфейса};
\scnfileitem{как правило, созданные модели специфичны для конкретной платформы либо конкретной реализации пользовательского интерфейса, что препятствует их повторному использованию для других целей};
\scnfileitem{решения, которые предлагают платформенно независимое описание, позволяют генерировать лишь  простые ограниченные по функционалу пользовательские интерфейсы (приложения-опросники, диаграммы и т.д.).}}

Среди предложенных подходов онтологический подход является наиболее предпочтительным по следующим причинам:
\begin{itemize}
	\item позволяет интегрировать ранее предложенные подходы за счет единого способа представления знаний; 
	\item позволяет создать наиболее полное описание различных аспектов пользовательского интерфейса;
	\item упрощает повторное использование интерфейса.
\end{itemize}
Однако, существующие решения на основе онтологического подхода не позволяют построить полную семантическую модель интерфейса.

Предлагается подход на основе онтологического, который позволит устранить недостатки существующих решений. Ключевыми особенностями подхода являются:
\begin{itemize}
	\item фиксация описания интерфейса в виде абстракции независимо от платформы и устройства;
	\item создание полной семантической модели интерфейса. Такая модель будет содержать "лексическое" описание интерфейса (описание компонентов и их свойств, из которых формируется интерфейс), "синтаксическое" описание интерфейса (правила формирования корректного и полного интерфейса из его компонентов), но также и его семантическое описание, включающее в себя позволяющей осуществлять последующую автоматическую генерацию;
	%\item простота совершенствования и расширяемость модели интерфейса;
	\item представление спецификации средств генерации интерфейса и при необходимости самих средств в едином формате с описанием интерфейса;
	\item сокращение затрат на разработку за счет повторного использования компонентов интерфейса;
	\item сокращение затрат на разработку за счет использования иерархической структуризации модели пользовательского интерфейса, позволяющей осуществлять независимую разработку компонентов.
\end{itemize}

\scnheader{Онтологическое проектирование пользовательских интерфейсов}
 \scnrelfromset{основные принципы}{
 	\scnfileitem{пользовательский интерфейс представляет собой специализированную ostis-систему, ориентированную на решение интерфейсных задач,и имеющую в составе базу знаний и машину обработки знаний пользовательского интерфейса,что даёт возможность пользователю адресовать
пользовательскому интерфейсу различного рода вопросы};
	\scnfileitem{используется онтологический подход к проектированию пользовательского интерфейса, что
способствует чёткому разделению деятельности различных разработчиков пользовательских интерфейсов, а также унификации принципов проектирования};
	\scnfileitem{используется SC-код в качестве формального
языка внутреннего представления знаний (онтологий, предметных областей и др.), благодаря
чему обеспечивается легкость интерпретации
этих знаний и системой, и человеком - пользователем или разработчиком, а также однозначность восприятия этой информации ими};
	\scnfileitem{средствами SC-кода с помощью соответствующих онтологий описываются синтаксис и семантика всевозможных используемых внешних
языков};
	\scnfileitem{трансляции с внутреннего языка на внешний и
обратно организовываются так, чтобы механизмы трансляции не зависели от внешнего языка, для реализации нового специализированного
языка в таком случае необходимо будет только
описать его синтаксис и семантику, универсальная же модель трансляции не будет зависеть от
данного описания};
	\scnfileitem{каждый элемент управления пользовательского
интерфейса является внешним отображением
некоторого sc-элемента, хранящегося в семантической памяти (sc-памяти), что позволяет
использовать их в качестве аргументов пользовательских команд и правильно трактовать
прагматику и семантику объектов интерфейсной деятельности};
	\scnfileitem{предполагается выбор стилей визуализации,
осуществляемый в зависимости от вида отображаемых знаний (например, использование различных элементов визуализации для одних видов знаний и других - для других), что позволит пользователю быстрее обучаться новым
специализированным языкам, а также сделать
простым и понятным отображение знаний};
	\scnfileitem{модель пользовательского интерфейса строится
независимо от реализации платформы интерпретации такой модели, что позволяет легко
переносить разработанную модель на разные
платформы.}
}
Использование онтологического подхода к проектированию пользовательских интерфейсов предполагает
построение (1) онтологической модели самого пользовательского интерфейса, как специализированной ostis-системы; (2) онтологической модели процесса проектирования интерфейсов, т.е. онтологии действий разработчиков интерфейсов, построенных на основе предлагаемой модели. 

\bigskip
	
\scnendstruct \scnendcurrentsectioncomment
	
\end{SCn}
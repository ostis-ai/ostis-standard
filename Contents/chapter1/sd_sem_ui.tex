\begin{SCn}
	
\scnsectionheader{\currentname}
	
\scnstartsubstruct

\scnheader{Предметная область логико-семантических моделей интерфейсов компьютерных систем, основанных на смысловом представлении информации}
\scnsdmainclasssingle{подход к построению пользовательского интерфейса}
\scnsdclass{подход к построению пользовательского интерфейса на основе специализированных языков описания;контекстно-зависимый подход к построению пользовательского интерфейса;подход к построению пользовательского интерфейса на основе данных;онтологический подход к построению пользовательского интерфейса;онтологический подход к построению пользовательского интерфейса на основе логико-семантической модели}

\scnheader{подход к построению пользовательского интерфейса}
\scnsuperset{подход к построению пользовательского интерфейса на основе специализированных языков описания}
\scnsuperset{контекстно-зависимый подход к построению пользовательского интерфейса}
\scnsuperset{подход к построению пользовательского интерфейса на основе данных}
\scnsuperset{онтологический подход к построению пользовательского интерфейса}
\scnaddlevel{1}
\scnsuperset{онтологический подход к построению пользовательского интерфейса на основе логико-семантической модели}
\scnaddlevel{-1}

\scnheader{подход к построению пользовательского интерфейса на основе специализированных языков описания}
\scnexplanation{подход на основе специализированных языков описания предполагает представление конкретного пользовательского интерфейса в платформенно независимом виде. В качестве примеров языков описания интерфейса можно привести \scncite{UIML}, \scncite{UsiXML}, \scncite{XForms} и \scncite{JavaFX FXML}. Ключевой идеей представленных языков является создание модели диалогов и форм интерфейса в независимом от используемой технологии виде, описание визуальных элементов, а также взаимосвязей между ними и их свойств для создания конкретного пользовательского интерфейса.}
\scnrelfromset{недостатки современного состояния}{\scnfileitem{как  правило,  спецификация  модели  интерфейса является  неполной,  что  приводит  к  сложности автоматизации процесса генерации пользовательского интерфейса};
	\scnfileitem{как правило, созданные модели специфичны для конкретной платформы либо конкретной реализации пользовательского интерфейса, что препятствует их повторному использованию для других целей.};
	\scnfileitem{решения,   которые   предлагают   платформенно независимое  описание,  позволяют  генерироватьлишь  простые  ограниченные  по  функционалупользовательские интерфейсы (приложения-опросники,диаграммы и т.д.).}}

\scnheader{контекстно-зависимый подход к построению пользовательского интерфейса}
\scnexplanation{контекстно-зависимый подход интегрирует использование структурного описания интерфейса на основе языков описания с поведенческой спецификацией, то есть генерация интерфейса основывается на действиях пользователя. В рамках подхода специфицируются переходы между различными видами конкретного пользовательского интерфейса. В качестве примеров реализации такого подхода можно привести \scncite{CAP3} и \scncite{MARIA}.}

\scnheader{подход к построению пользовательского интерфейса на основе данных}
\scnexplanation{подход на основе данных или моделеориентированный подход использует модель предметной области в качестве основы для создания пользовательских интерфейсов. К реализациям можно отнести \scncite{JANUS} и \scncite{Mecano}.}

\scnheader{онтологический подход к построению пользовательского интерфейса}
\scnexplanation{cуществующие онтологические подходы как правило основаны на представленных ранее подходах и используют онтологии в качестве способа представления информации о конкретном пользовательском интерфейсе. Например, по аналогии с подходом на основе специализированных языков описания, был предложен фреймворк \scncite{UI Model-Based Approach}, использующий онтологию для описания пользовательского интерфейса на основе понятий, хранящихся в базе знаний. По аналогии с контекстно-зависимым подходом в рамках работы \scncite{gaulke} используется модель предметной области совместно с моделью пользовательского интерфейса, ассоциированная с онтологией действий. Проект \scncite{ActiveRaUL} совмещает UIML с моделеориентированным подходом. В рамках данного проекта онтологическая модель предметной области сопоставляется с онтологическим представлением пользовательского интерфейса. Подход, предложенный в \scncite{hitz}, совмещает данные приложения с онтологией пользовательского интерфейса для создания единого описания в базе знаний с целью последующей автоматической генерации различных вариантов интерфейса для приложений-опросников с готовыми сценариями взаимодействия с пользователем. Следует также отметить работы \scncite{vladivostok1} и \scncite{vladivostok2}, в рамках которых предложена концепция, позволяющая объединить однородную по содержанию информацию в компоненты модели интерфейса, освободить разработчика интерфейса от кодирования и формировать информацию для каждого компонента модели интерфейса с помощью редакторов, управляемых соответствующими моделями онтологий.}
\scnrelfromset{недостатки современного состояния}{\scnfileitem{актуальна проблема совместимости различных онтологий в рамках единой системы};
	\scnfileitem{отсутствие способности адаптироватьсяк запросам пользователя и анализировать его действия длясамостоятельного совершенствования.}}
\scnrelfromset{достоинства}{\scnfileitem{позволяет интегрировать ранее предложенные подходы за счет единого способа представления знаний.};
	\scnfileitem{позволяет создать наиболее полное описание различных аспектов пользовательского интерфейса.};
	\scnfileitem{упрощает повторное использование интерфейса.}}

\scnheader{онтологический подход к построению пользовательского интерфейса на основе логико-семантической модели}
\scnnote{для проектирования пользовательских интерфейсов предлагается использовать \textbf{\textit{онтологический подход к построению пользовательского интерфейса на основе логико-семантической модели}}, обладающий рядом важных достоинств.}
\scnrelfromset{достоинства}{\scnfileitem{возможность переноса пользовательских интерфейсов с одной платформы реализации на другую.};
	\scnfileitem{наличие общих    принципов построения пользовательских интерфейсов, что позволяет повторно использовать уже разработанные компоненты   и  снижает сроки  обучения  пользователя  новым  для  него пользовательским интерфейсам.};
	\scnfileitem{возможность модификации пользовательского интерфейса в процессе работы.};
	\scnfileitem{возможность гибкой адаптации пользовательского интерфейса под нужды конкретного пользователя.};}
\scnexplanation{подход предполагает создание полной семантической модели интерфейса, содержащей  "лексическое"{} описание  интерфейса(описание компонентов, из которых формируется интерфейс), "синтаксическое"{} описание интерфейса(правила  формирования  корректного  и  полного интерфейса из его компонентов), но также и его семантическое описание (знание о том, знаком какой сущности является отображаемый компонент). При этом семантическое описание также включает всебя назначение, область применения компонентов интерфейса, описание интерфейсной деятельности пользователя.}
 \scnrelfromset{основные принципы}{
	\scnfileitem{пользовательский интерфейс представляет собой специализированную ostis-систему, ориентированную на решение интерфейсных задач,и имеющую в составе базу знаний и машину обработки знаний пользовательского интерфейса,что даёт возможность пользователю адресовать пользовательскому интерфейсу различного рода вопросы};
	\scnfileitem{используется онтологический подход к проектированию пользовательского интерфейса, что
		способствует чёткому разделению деятельности различных разработчиков пользовательских интерфейсов, а также унификации принципов проектирования};
	\scnfileitem{используется SC-код в качестве формального
		языка внутреннего представления знаний (онтологий, предметных областей и др.), благодаря
		чему обеспечивается легкость интерпретации
		этих знаний и системой, и человеком - пользователем или разработчиком, а также однозначность восприятия этой информации ими};
	\scnfileitem{средствами SC-кода с помощью соответствующих онтологий описываются синтаксис и семантика всевозможных используемых внешних
		языков};
	\scnfileitem{трансляции с внутреннего языка на внешний и
		обратно организовываются так, чтобы механизмы трансляции не зависели от внешнего языка, для реализации нового специализированного
		языка в таком случае необходимо будет только
		описать его синтаксис и семантику, универсальная же модель трансляции не будет зависеть от
		данного описания};
	\scnfileitem{каждый элемент управления пользовательского
		интерфейса является внешним отображением
		некоторого sc-элемента, хранящегося в семантической памяти (sc-памяти), что позволяет
		использовать их в качестве аргументов пользовательских команд и правильно трактовать
		прагматику и семантику объектов интерфейсной деятельности};
	\scnfileitem{предполагается выбор стилей визуализации,
		осуществляемый в зависимости от вида отображаемых знаний (например, использование различных элементов визуализации для одних видов знаний и других - для других), что позволит пользователю быстрее обучаться новым
		специализированным языкам, а также сделать
		простым и понятным отображение знаний};
	\scnfileitem{модель пользовательского интерфейса строится
		независимо от реализации платформы интерпретации такой модели, что позволяет легко
		переносить разработанную модель на разные
		платформы.}
}

\bigskip
	
\scnendstruct \scnendcurrentsectioncomment
	
\end{SCn}
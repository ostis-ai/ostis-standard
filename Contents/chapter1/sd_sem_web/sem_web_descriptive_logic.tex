\scseparatedfragment{Представление знаний}
\scnheader{Представление знаний}

\scnsegmentheader{Дескриптивная логика}

\scnstartsubstruct

\scnheader{Дескриптивная логика}
\scnidtf{дескрипционная логика}
\scnidtf{descriptive logic}
\scnidtf{DL}
\scnidtfdef{семейство\textit{ формальных языков представления знаний}.} 

\scnsubdividing{DL общего вида; темпоральные DL; пространственные DL; пространственно-темпоральные DL; нечеткие DL}

\scnsuperset{концепт}
\scnaddlevel{1}
\scnidtf{одноместный предикат}
\scnidtf{множество}
\scnidtf{класс}
\scnaddlevel{-1}

\scnsuperset{роль}
\scnaddlevel{1}
\scnidtf{двухместный предикат}
\scnidtf{бинарное отношение}
\scnaddlevel{-1}


\scnnote{Большинство \textit{DL} более выразительны, чем \textit{пропозициональные логики}, но менее выразительны, чем\textit{ логика предикатов первого порядка}. В отличие от последних, \textit{DL} обычно разрешимы, для них определены эффективные \textit{процедуры вывода}. В целом в \textit{DL} обычно соблюдается баланс между выразительной мощностью и сложностью организации вывода. Современное название семейство \textit{DL} получило в 1980-е годы, тогда они изучались как расширения \textit{теорий фреймовых структур} и \textit{семантических сетей} механизмами \textit{формальной логики}. В 2000-е годы \textit{дескрипционные логики} получили применение в рамках концепции \textit{Semantic Web паутины}, где их предлагалось использовать при построении \textit{онтологий}. На основе \textit{DL} построены подъязыки \textit{OWL} такие как \textit{OWL-DL} и \textit{OWL-Lite}.}


\bigskip

\scnendstruct \scnendsegmentcomment{Дескриптивная логика}
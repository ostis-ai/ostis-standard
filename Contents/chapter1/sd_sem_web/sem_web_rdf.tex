\scnheader{RDF}
\scnidtf{Resource Description Framework}
\scnidtf{разработанная консорциумом Всемирной паутины модель для представления данных, в особенности — метаданных. \textit{RDF} представляет утверждения о ресурсах в виде, пригодном для машинной обработки}
\scntext{особенность}{Ресурсом в \textit{RDF} может быть любая сущность — как информационная (например, веб-сайт или изображение), так и неинформационная (например, человек, город или некое абстрактное понятие). Утверждение, высказываемое о ресурсе, имеет вид «субъект — предикат — объект» и называется триплетом}
\scntext{особенность}{Для обозначения субъектов, отношений и объектов в \textit{RDF} используются \textit{URI}}
\scnnote{\textit{RDF} сам по себе является не форматом файла, а только лишь абстрактной моделью данных, то есть описывает предлагаемую структуру, способы обработки и интерпретации данных. Для хранения и передачи информации, уложенной в модель \textit{RDF}, существует целый ряд форматов записи}
\scnrelfromlist{формат записи}{RDF/XML\\
\scnaddlevel{1}
	\scnidtf{запись в виде XML-документа}
\scnaddlevel{-1}
;RDF/JSON
\scnaddlevel{1}
	\scnidtf{запись в виде JSON-данных}
\scnaddlevel{-1}
;RDFa
\scnaddlevel{1}
	\scnidtf{запись внутри атрибутов произвольного HTML- или XHTML-документа}
\scnaddlevel{-1}
;N-Triples
\scnaddlevel{1}
	\scnidtf{компактная форма записи утверждений}
\scnaddlevel{-1}
;Turtle
\scnaddlevel{1}
	\scnidtf{компактная форма записи утверждений}
\scnaddlevel{-1}
;N3
\scnaddlevel{1}
	\scnidtf{компактная форма записи утверждений}
\scnaddlevel{-1}}
\scnnote{\textit{RDF} предоставляет средства для построения информационных моделей, но не касается семантики описываемого. Взятый в отдельности граф \textit{RDF} можно понимать только как граф. Толкование значения основывается на способности пользователей \textit{RDF} интерпретировать отдельные \textit{URI}, строковые литералы и структуру графа, и по ним интерпретировать остальные \textit{URI} и семантику данных}
\scntext{особенность}{Для выражения семантики требуются \textit{словари} (англ. vocabularies), \textit{таксономии} (англ. taxonomies) и \textit{онтологии} (англ. ontologies) и наличие в рассматриваемом графе связей с ними}
\scniselement{Словарь}
\scnaddlevel{1}
	\scnidtf{представляет собой собрание терминов, имеющих во всех контекстах использования этого словаря одинаковый смысл}
\scnaddlevel{-1}
\scniselement{Таксономия}
\scnaddlevel{1}
	\scnidtf{словарь иерархически организованных терминов}
\scnaddlevel{-1}
\scniselement{Онтология}
\scnaddlevel{1}
	\scnidtf{использует предопределённый зарезервированный словарь терминов для определения концепций и отношений между ними для конкретной предметной области}
	\scnnote{Онтологии можно использовать для выражения семантики терминов словаря, их взаимоотношений и контекстов использования}
\scnaddlevel{-1}
\scntext{особенность}{Большинство словарей для описываемых субъектов не только содержит предикаты и объекты, но и подразумевает для них ту или иную семантическую нагрузку, не укладывающуюся как правило в собственно RDF-представление словаря. Это могут быть способы использования тех или иных конкретных субъектов, правила, ограничения, рекомендации, обоснования необходимости использования именно их, и т. п. Как правило, это описывается в сопроводительной документации к словарю}
\scnnote{Словари PDF также называются RDF-схемами (не путать с RDFS). В настоящее существует довольно большое количество открытых словарей, некоторые из которых используются сообществом на уровне де-факто и де-юре стандартов}
\scnrelto{ключевой знак}{\scncite{RDF}}
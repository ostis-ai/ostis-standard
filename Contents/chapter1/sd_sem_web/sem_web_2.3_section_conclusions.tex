\scntext{выводы к разделу}{
	\begin{scnitemize}
		\item RDF сам по себе является моделью данных, но не затрагивает семантику описываемых данных и конкретный формат хранения текстов, представленных в такой модели. Для этого существуют довольно много отдельных стандартов;
		\item Несмотря на то, что модель RDF является нелинейной графовой моделью, в ней присутствуют ограничения, связанные с самой структурой триплета (субъект-отношение-объект):
		\scnaddlevel{1}
			\begin{scnitemizeii}
				\item Возникает неоднозначность между некоторым отношением (предикатом) вообще и парой (связкой), принадлежащей данному отношению и связывающей конкретный объект и конкретный субъект. Таким образом, нет возможности описать свойства не отношения вообще, а конкретной связи (об этой проблеме см. раздел реификация). В то же время синтаксис N3 позволяет оперировать триплетом или более сложным выражением как аргументом для других выражений;
				\item Отсутствует возможность сформировать триплет, в котором компоненты равноправны, например при описании отношения “быть родственником” и других симметричных отношениях;
			\end{scnitemizeii}
		\scnaddlevel{-1}
		\item Представленные проекты в буквальном смысле представляют собой словари терминов (vocabulary) и не являются формальными онтологиями (некоторые исследователи считают словарь терминов или тезаурус разновидностью онтологии, но неформальной), то есть не содержат формальных определений описываемых понятий, каких-либо строгих закономерностей, кроме простой транзитивности отношений класс-подкласс и явно задаваемых ограничений на значения свойств;
		\item Составители большинства стандартов при их составлении руководствовались “здравым смыслом” и исходили из потребности описать ресурсы, которые уже есть или потенциально могут появиться в глобальной сети, не пытаясь положить в основу создаваемых стандартов какую-либо математическую модель. Исходя из этого, возникает большое число трудностей:
		\scnaddlevel{1}
		\begin{scnitemizeii}
			\item Невозможность дать достаточно строгие определения для понятий, что в свою очередь, приводят к трудностям определения тех классов, к которым нужно отнести тот или иной экземпляр или подкласс;
			\item Сама по себе иерархия классов строится на интуитивном уровне и не всегда понятно, почему на каждом уровне иерархии выбрано именно это множество классов, насколько оно полное, пересекаются ли выделенные классы;
			\item Не всегда понятно, как разделить абсолютные и относительные понятия (хотя смешивать их нельзя, понятие свойства и ресурса четко разделены), и не совсем понятно, почему принципиально разделена иерархия класс-подкласс и свойство-подсвойство;
		\end{scnitemizeii}
		\scnaddlevel{-1}
		\item При записи триплета явно не вводится знак связи (дуги). При необходимости этот знак может быть введен явно, но тогда конструкция изменится, появятся две других связи (между знаком дуги и ее началом и концом). Само собой, такое изменение придется учесть при обработке.
	\end{scnitemize}
}
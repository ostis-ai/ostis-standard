\begin{SCn}
\scnheader{следует отличать*}
\scnhaselementset{предельно омонимичный класс синтаксически эквивалентных знаков
\scnaddlevel{1}
\scnidtf{класс синтаксически эквивалентных знаков, все экземпляры (все вхождения) которого являются знаками \underline{разных} сущностей}
\scnaddlevel{1}
\scntext{следовательно}{В рамках рассматриваемого класса знаков синонимия знаков отсутствует}
\scnaddlevel{-1}
\scnidtf{максимальный класс синтаксически эквивалентных знаков, среди которых отсутствуют синонимичные знаки}
\scnaddlevel{-1};
частично омонимичный класс синтаксически эквивалентных знаков
\scnaddlevel{1}
\scnidtf{класс синтаксически эквивалентных знаков, среди экземпляров которого встречаются как синонимичные знаки, так и знаки \underline{разных} сущностей}
\scnaddlevel{-1};
неомонимичный класс синтаксически эквивалентных знаков
\scnaddlevel{1}
\scnidtf{класс синтаксически эквивалентных знаков, \underline{все} экземпляры которого являются знаками одной и той же сущности}
\scnidtf{класс синтаксически эквивалентных знаков, синтаксическая структура которых однозначно идентифицирует (соответствует) обозначаемую ими сущность}
\scnaddlevel{-1};
множество особенностей (характеристик), которыми обладает сущность, обозначаемая заданным знаком*
}
\scnheader{следует отличать*}
\scnhaselementset{смысловое представление информации*
\newline
\scnaddlevel{1}
\scnidtftext{часто используемый sc-идентификатор}{смысл*}
\scnaddlevel{-1};
смысловое представление информации
\newline
\scnaddlevel{1}
\scnrelto{второй домен}{смысловое представление информации*}
\scnaddlevel{-1};
синтаксическая структура информационной конструкции*;
синтаксическая структура информационной конструкции
\newline
\scnaddlevel{1}
\scnrelto{второй домен}{синтаксическая структура информационной конструкции*}
\scnaddlevel{-1};
денотациооная семантика информационной конструкции*
\scnaddlevel{1}
\scnidtf{\textit{соответствие} (морфизм) между синтаксической структурой заданной информационной конструкции и ее \textit{смысловым представлением*}}
\scnnote{\textit{соответствие} между знаками входящими в соста \textit{рафинированной семантической сети} и их \textit{денотатами*} (обозначаемыми сущностями) являются \underline{взаимно однозначными}
\scnaddlevel{-1}}
}
\scnheader{следует отличать*}
\scnhaselementset{смысловое пространство
\newline
\scnaddlevel{-1}
\scnhaselement{SC-пространство}
\scnaddlevel{2}
\scnidtf{семантическое пространство}
\scnexplanation{объединение (соединение) всевозможных корректных абстрактных семантических сетей, принадлежащих некоторому языку абстрактных \textit{рафинированных семантических сетей}}
\scnidtf{глобальная (максимальная) абстрактная \textit{рафинированная семантическая сеть}, включающая в себя всевозможные абстрактные рафинированные семантические сети соответствующего языка}
\scnidtf{абстрактное смысловое пространство}
\scnaddlevel{-1};
абстрактная смысловая память
\scnaddlevel{1}
\scnidtf{абстрактная семантическая память}
\scnidtf{среда, обепечивающая хранение абстрактных рафинированных семантических сетей, а также редактирование этих семантических сетей и при этом абстрагирующаяся от деталей этих процессов}
\scnidtf{абстрактная графодинамическая память, обеспечивающая хранение и редактирование абстрактных рафинированных семантических сетей}
\scnaddlevel{-1};
реальная смысловая память
\scnaddlevel{1}
\scnidtf{физическая реализация абстрактной смысловой памяти}
\scnrelboth{следует отличать}{программная реализация \underline{модели} абстрактной смысловой памяти на современных компьютерах}
\scnaddlevel{-1}}
\end{SCn}
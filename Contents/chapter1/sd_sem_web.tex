\begin{SCn}
	
	\scnsectionheader{\currentname}
	
	\scnstartsubstruct
	
	\scnheader{Предметная область и онтология семантических технологий}
	
	\newpage
	\scseparatedfragment{Введение в Semantic Web}
\scnheader{Введение в Semantic Web}

	\scneqfile{В марте 1989 года Тим Бернерс-Ли предложил концепцию распределенной информационной системы с целью "объединения знаний человечества", которую он назвал \textit{"Всемирной паутиной"} (\textit{World Wide Web} - \textit{WWW}). Для её создания он объединил две существующие технологии - \textit{технологию применения IP-протоколов для передачи данных} и \textit{технологию гипертекста} (\textit{Hypertext Technology}).
	Основными элементами технологии \textit{WWW} являются :
	\begin{scnitemize}
		\item \textit{язык гипертекстовой разметки документов} (\textit{Hyper Text Markup Language} - \textit{HTML});
		\item \textit{протокол обмена гипертекстовой информацией} (\textit{Hyper Text Transfer Protocol} - \textit{HTTP});
		\item \textit{универсальный способ адресации ресурсов в сети} (\textit{Universal Resource Identifier} - \textit{URI}, и \textit{Universal Resource Locator} - \textit{URL});
		\item \textit{система доменных имен} (\textit{Domain Name System} - \textit{DNS});
		\item \textit{универсальный интерфейс шлюзов} (\textit{Common Gateway Interface} - \textit{CGI}), добавленный позже сотрудниками \textit{Национального Центра Суперкомпьютерных Приложений} (\textit{National Center for Supercomputing Applications} - \textit{NCSA});
		\item \textit{расширяемый язык разметки} (\textit{eXtensible Markup Language} - \textit{XML}), рекомендованный \textit{Консорциумом Всемирной паутины}.
	\end{scnitemize}
	
	Основная цель \textbf{\textit{Semantic Web}} создать надстройку над существующей \textit{Всемирной паутиной}, позволяющую сделать размещаемую в \textit{Интернете} информацию пригодной для машинной обработки, а именно для автоматического анализа, синтеза выводов и преобразования как самих данных, так и сделанных на их основе заключений в различные представления, полезные на практике. 
	
	В частности, такой подход позволит повысить релевантность предоставляемой пользователю информации, обеспечить предоставление только необходимой части информации, облегчить поиск информации и повысить эффективность ее использования.
	
	Важно отметить, что концепции \textit{Web 2.0} и \textit{Web 3.0}, описывающие эволюцию интернет-технологий, не связаны непосредственно с \textit{\textbf{Semantic Web}}, более того, автор концепции \textit{Web 2.0} Тим О’Рейли неоднократно выступал с критикой отождествления \textit{\textbf{Semantic Web}} и \textit{Web 3.0}.}
\bigskip
\scnendcurrentsectioncomment


	
	\newpage
	\scseparatedfragment{Представление знаний}
\scnheader{Представление знаний}

\scnsegmentheader{Дескриптивная логика}

\scnstartsubstruct

\scnheader{Дескриптивная логика}
\scnidtf{дескрипционная логика}
\scnidtf{descriptive logic}
\scnidtf{DL}
\scnidtfdef{семейство\textit{ формальных языков представления знаний}.} 

\scnsubdividing{DL общего вида; темпоральные DL; пространственные DL; пространственно-темпоральные DL; нечеткие DL}

\scnsuperset{концепт}
\scnaddlevel{1}
\scnidtf{одноместный предикат}
\scnidtf{множество}
\scnidtf{класс}
\scnaddlevel{-1}

\scnsuperset{роль}
\scnaddlevel{1}
\scnidtf{двухместный предикат}
\scnidtf{бинарное отношение}
\scnaddlevel{-1}


\scnnote{Большинство \textit{DL} более выразительны, чем \textit{пропозициональные логики}, но менее выразительны, чем\textit{ логика предикатов первого порядка}. В отличие от последних, \textit{DL} обычно разрешимы, для них определены эффективные \textit{процедуры вывода}. В целом в \textit{DL} обычно соблюдается баланс между выразительной мощностью и сложностью организации вывода. Современное название семейство \textit{DL} получило в 1980-е годы, тогда они изучались как расширения \textit{теорий фреймовых структур} и \textit{семантических сетей} механизмами \textit{формальной логики}. В 2000-е годы \textit{дескрипционные логики} получили применение в рамках концепции \textit{Semantic Web паутины}, где их предлагалось использовать при построении \textit{онтологий}. На основе \textit{DL} построены подъязыки \textit{OWL} такие как \textit{OWL-DL} и \textit{OWL-Lite}.}


\bigskip

\scnendstruct \scnendsegmentcomment{Дескриптивная логика}
	
	\newpage
	\scseparatedfragment{Обработка информации}
\scnheader{Обработка информации}

\scnstartsubstruct
\scnnote{Ключевыми \textit{задачами} \textit{обработки информации}, представленной при помощи \textit{средств} \textit{\textbf{Semantic Web}}, являются:
	\begin{scnitemize}
		\item Выполнение \textit{поисковых запросов} (\textit{Query});
		\item \textit{Логический вывод} на \textit{онтологиях} (порождения новой \textit{информации} на основе имеющейся).
	\end{scnitemize}}

\scnsegmentheader{Языки запросов}

\scnstartsubstruct

\scnexplanation{Для выполнения \textit{запросов} к \textit{хранилищу}, содержащему \textit{информацию}, записанную в \textit{RDF}, используются \textit{языки запросов}(\textit{Query languages})}

\scnheader{SPARQL}
\scnidtf{SPARQL Protocol and RDF Query Language}
\scnidtfdef{\textit{язык запросов}, признанный в качестве \textit{стандарта} W3C в 2008 году}

\scnrelfrom{поддерживаемые операции}{операции SQL}
	\scnaddlevel{1}
	\scnsubdividing{объединение (JOIN);сортировка (SORT)}
	\scnaddlevel{-1}

\scnnote{\textit{Запрос} \textit{SPARQL} по сути представляет собой \textit{шаблон} некоторого\textit{ RDF-графа}, в котором неизвестные \textit{сущности} задаются \textit{переменными}.
\textit{SPARQL} позволяет получить \textit{результат запроса} в нескольких формах:
\begin{scnitemize}
	\item \textit{Оператор} \textit{SELECT} позволяет указать, значения каких \textit{переменных} мы хотели бы получить, \textit{результат} возвращается в виде \textit{таблицы};
	\item \textit{Оператор CONSTRUCT} позволяет получить \textit{ответ} в виде \textit{RDF-графа};
	\item \textit{Оператор ASK} позволяет получить \textit{ответ} вида да/нет (существует ли вообще такая \textit{конструкция});
	\item \textit{Оператор CONSTRUCT} позволяет получить \textit{ответ} в виде \textit{RDF-графа}, \textit{содержание} которого может дополнительно уточняться при \textit{запросе};
\end{scnitemize}


Например:

\textit{PREFIX foaf: <http://xmlns.com/foaf/0.1/>\newline
SELECT ?name\newline
?email\newline
WHERE\newline
{\newline
	?person  a   foaf:Person .\newline
	?person  foaf:name  ?name .\newline
	?person  foaf:mbox  ?email .\newline
}}\newline

Данный запрос позволяет получить \textit{имена} и \textit{e-mail} всех \textit{персон}, известных \textit{системе}.

\textit{PREFIX ex: <http://example.com/exampleOntology#>\newline
SELECT ?capital\newline
?country\newline
WHERE\newline
\{\newline
	?x  ex:cityname   	?capital   ;\newline
	ex:isCapitalOf	?y     	.\newline
	?y  ex:countryname	?country   ;\newline
	ex:isInContinent  ex:Africa  .\newline
\}}\newline

Данный запрос позволяет получить названия всех столичных городов в Африке.\\

Ссылка:
\scncite{https://www.w3.org/TR/sparql11-query/}}

\scnheader{GraphQL}
\scnidtfdef{открытый \textit{язык} для различного рода \textit{графовых хранилищ} и \textit{СУБД}}
\scnrelfrom{синтаксическая схожесть}{JSON}
\scnrelfromlist{оперирование понятиями}{поле; значение поля}
\scnrelto{разработчик}{\textit{\textbf{Facebook}}}
\scnnote{\textit{Язык} \textit{GraphQL} был предложен компанией Facebook в 2012 году. Существует большое число реализаций, поддерживающих \textit{GraphQL}.
	
Ссылка:
\scncite{https://graphql.org/} }

\scnheader{Gremlin}
\scnidtfdef{популярный \textit{язык обхода графа} (\textit{graph traversal language})}
\scnrelto{разработчик}{\textit{\textbf{Apache}}}
\scnnote{Gremlin позволяет описывать \textit{запросы} к \textit{хранилищу графов} как в виде \textit{шаблонов} с \textit{переменными}, так и более простым способом, указывая необходимые \textit{свойства} искомых \textit{компонентов графа}. Поддержка \textit{Gremlin} реализована в большинстве современных \textit{хранилищ}.
	
Пример:
\textit{g.V().hasLabel('movie').values('year').min()}
 
Данный \textit{запрос} позволяет найти все \textit{вершины графа}, имеющие \textit{метку} “фильм” и выбрать минимальный по году выпуска.}

\scnheader{Cypher}
\scniselement{открытый язык}
\scnidtfdef{\textit{язык запросов} для \textit{графовой СУБД Neo4j}}
\scnrelfromlist{результат запроса}{граф; таблица}
\scnaddlevel{2}
\scnexplanation{В случае, когда граф построить нельзя, например запрашиваются только значения каких-либо атрибута для заданного множества сущностей}
\scnaddlevel{-2}
\scnnote{Cypher позволяет в достаточно удобной форме описывать шаблоны, на основе которых осуществляется в поиск в графе. Как и в модели СУБД Neo4j в Cypher различаются отношения между сущностями (аналог ObjectProperty в OWL) и свойства сущностей (атрибуты, аналог DataProperty в OWL).

Пример:\newline
\textit{MATCH (nicole:Actor {name: 'Nicole Kidman'})-[:ACTED\_IN]->(movie:Movie)\newline
WHERE movie.year < 1990\newline
RETURN movie\newline}
Данный \textit{запрос} позволяет найти все фильмы старше 1990 года, в которых сыграла Николь Кидман.}

\bigskip

\scnendstruct \scnendsegmentcomment{Языки запросов}

\scnsegmentheader{Ризонеры}

\scnstartsubstruct

\scnheader{Ризонеры}
\scnidtf{reasoners}
\scnrelfrom{область применения}{логический вывод}
\scnaddlevel{2}
\scnexplanation{\textit{Ризонеры}, как правило, реализуют \textit{прямой логический вывод} на основе \textit{отношений}, описанных в \textit{онтологии}. В простейшем случае это \textit{вывод} на основе \textit{свойства транзитивности отношений}, например \textit{SubClassOf}, в более сложных могут учитываться \textit{правила}, записанные, например в \textit{SWRL}.}
\scnaddlevel{-2}

\scnnote{Часто для \textit{ризонера} оговаривается \textit{язык}, \textit{семантика} которого полностью поддерживается соответствующим \textit{ризонером}, например \textit{OWL Lite} или \textit{OWL DL}.\newline
	
Полный список признанных сообществом \textit{ризонеров} с кратким описанием их возможностей приведен в статье:
\scncite{https://www.w3.org/2001/sw/wiki/OWL/Implementations/}}

\bigskip

\scnendstruct \scnendsegmentcomment{Ризонеры}

\bigskip

\scnrelfromlist{выводы к разделу}{
	\scnfileitem {Интересно отметить, что все рассмотренные \textit{языки запросов} (включая \textit{SPARQL}, который является \textit{стандартом W3C}), кроме \textit{GraphQL}, позволяют осуществлять только \textit{поиск} (собственно \textit{запросы}), и не дают возможности что-либо создавать, редактировать или удалять в \textit{графе}, в отличие, например, от \textit{языка} \textit{SQL}, обладающем соответсвующими \textit{классами операторов}. Это означает, что фактически эта часть никак не стандартизируется и отдается на откуп разработчикам конкретного \textit{хранилища}, которые формируют соответствующий \textit{программный интерфейс}.};
	\scnfileitem {\textit{Возможности} и \textit{принципы} \textit{логического вывода} на \textit{онтологиях} зависят на \textit{синтаксиса} и \textit{семантики} \textit{языков}, используемых для \textit{представления знаний}. В остальном \textit{обработка информации}, записанной \textit{средствами} \textit{\textbf{Semantic Web}} никак не стандартизируется - реализуются \textit{программные интерфейсы} для \textit{доступа} к \textit{хранилищам} для различных \textit{языков программирования}, после чего разработчик конкретной \textit{системы} сам определяет, как использовать полученную \textit{информацию}.}}

\bigskip
\scnendstruct 
\scnendcurrentsectioncomment
	
	\newpage
	\scseparatedfragment{Направления развития Semantic Web}
\scnheader{Направления развития Semantic Web}

\scneqfile{Несмотря на довольно очевидную выгоду от ее \textit{внедрения}, \textit{общая концепция Semantic Web} в настоящее время не реализована даже на базовом уровне. Уже в 2006 году Тим Бернерс-Ли опубликовал статью\textit{ “Semantic Web Revisited”}, в которой признал, что по ряду причин \textit{концепция} в большой степени не реализована.\newline
	Причин такой ситуации несколько:
	\begin{scnitemize}
		\item человеческий фактор и накладные расходы - описание \textit{метаданных} для какого-либо \textit{ресурса} и вообще его \textit{формализация} требует дополнительных затрат, иногда довольно существенных, но в то же время не дает очевидной прямой выгоды в ближайшей перспективе. Кроме того, многие коммерческие компании не заинтересованы в \textit{стандартизации} подобных \textit{решений} в принципе, поскольку это снизит объем выполняемой работы и востребованность таких компаний на рынке;
		\item сами по себе \textit{средства представления} и \textit{обработки знаний \textbf{Semantic Web}} достаточно сложны в использовании для неподготовленного человека;
		\item остается открытым вопрос с выбором \textit{универсальной системы понятий верхнего уровня}, которые бы гарантированно позволили описать любую \textit{информацию}; кроме того, возникает вопрос о том, возможно ли вообще выбрать такую \textit{систему понятий};
		\item добавление \textit{метаинформации} автоматически приводит к частичному дублированию уже имеющейся \textit{информации} и соответствующим расходам.
	\end{scnitemize}
}
\scnendcurrentsectioncomment

\bigskip

\scnrelfromlist{выводы к разделу}{
\scnfileitem {\textit{Стандарты} \textit{\textbf{Semantic Web}} практически не меняются на протяжении более 15 лет. Это связано с тем, что \begin{scnitemize}\item сами \textit{стандарты} все еще описаны в традиционной  \textit{гипертекстовой форме}, которая достаточно трудоемка в сопровождении;\item существует огромное количество \textit{средств}, которые жестко завязаны на определенную \textit{версию} того или иного \textit{стандарта} и сопровождаются сторонними коллективами разработчиков. Соответственно, каждое изменение \textit{стандарта} должно отразиться и в указанных \textit{средствах}, что зачастую очень сложно;\item значительная часть \textit{средств}, построенных на основе \textit{стандартов}, напрямую зависит не только от \textit{семантики языков}, но и от конкретной \textit{формы записи} того или иного \textit{языка}, что еще усугубляет проблему из предыдущего пункта.\end{scnitemize}};
\scnfileitem {Фактически все \textit{семейство стандартов} \textit{\textbf{Semantic Web}} изначально создавалось для реализации идей \textit{семантической паутины}, т.е. описания \textit{метаинформации} для различных \textit{web-ресурсов}, приведение их к согласованному виду (как с синтаксическом, так и в семантическом плане) и т.д. Указанные \textit{стандарты} изначально не претендовали на всеобщую \textit{стандартизацию принципов представления} и \textit{обработки} любого рода \textit{информации} в \textit{интеллектуальных системах} произвольного назначения. Однако, де-факто в настоящее время они часто используются именно в такой роли, поскольку не существует более мощной и удобной альтернативы, общепризнанной в мировом сообществе.}}
	
	\bigskip
	\scnendstruct \scnendfragmentcomment
	
\end{SCn}


\begin{SCn}
	
\scnsectionheader{\currentname}
	
\scnstartsubstruct

/*2.2. Принципы идентификации сущностей (ресурсов)*/
\scnheader{Принципы идентификации сущностей (ресурсов)}

\scniselement{URI}
\scnaddlevel{1}
	\scnidtf{Universal Resource Identifier}
	\scnidtf{Uniform Resource Identifier}
	\scnidtf{символьная строка, позволяющая идентифицировать какой-либо ресурс: документ, изображение, файл, службу, ящик электронной почты и т. д.}
	\scntext{структура}{URI = scheme:[//[пользователь@]хост[:порт]]путь[?запрос][#фрагмент]}
	\scnaddlevel{1}
		\scntext{пример}{https://example.com/path/resource.txt#fragment}
	\scnaddlevel{-1}
	\scnsuperset{URL}
	\scnaddlevel{1}
		\scnidtf{Universal Resource Locator}
		\scnidtf{это \textit{URI}, который, помимо идентификации ресурса, предоставляет ещё и информацию о местонахождении этого ресурса}
	\scnaddlevel{-1}
	\scnsuperset{URN}
	\scnaddlevel{1}
		\scnidtf{Universal Resource Name}
		\scnidtf{это \textit{URI}, который только идентифицирует ресурс в определённом пространстве имён (и, соответственно, в определённом контексте), но не указывает его местонахождение}
		\scntext{структура}{<URN> ::= "urn:" <ID пространства имен> ":" <Имя ресурса>}
		\scntext{пример}{urn:isbn:5170224575 (номер книги в пространстве ISBN)}
	\scnaddlevel{-1}
	\scntext{особенность}{\textit{URI} используются для обозначения субъектов, отношений и объектов в \textit{RDF}}
	\scnaddlevel{1}
		\scntext{пример}{http://localhost/Institute#Student (класс)\\
		http://localhost/Institute#hasAuthor (отношение)}
	\scnaddlevel{-1}
	\scntext{особенность}{\textit{URI} позволяют отличать сущности, имеющие одинаковое название, но возможно, разную трактовку в разных онтологиях}
	\scnrelto{ключевой знак}{\scncite{URI}}
\scnaddlevel{-1}

\scniselement{IRI}
\scnaddlevel{1}
	\scntext{особенность}{В \textit{URI} можно использовать только ограниченный набор латинских символов и знаков препинания (даже меньший, нежели в ASCII). Это входит в противоречие с принципом интернационализма, провозглашаемого W3C.\\
	Эту проблему призван решить стандарт \textit{IRI} (Internationalized Resource Identifier), разрешающий использовать любые символы Юникода
	}
	\scnrelto{ключевой знак}{\scncite{IRI}}
\scnaddlevel{-1}

\scniselement{XRI}
\scnaddlevel{1}
	\scnidtf{Extensible Resource Identifier}
	\scnidtf{расширяемый идентификатор ресурса, разработанный организацией OASIS, прежде всего, как будущая замена \textit{URL} в Интернете}
	\scntext{особенность}{\textit{XRI} — это новая, совместимая с \textit{IRI} и \textit{URI} схема (протокол) для создания абстрактных идентификаторов ресурсов. Такие идентификаторы не зависят от платформы, домена, путей и программ — они полностью абстрактны и поэтому могут быть едины для всех доменов и каталогов}
	\scnrelfromset{слои идентификатора}{I-Numbers\\
		\scnaddlevel{1}
			\scnidtf{постоянные сетевые адреса (похожие на IP-адреса)}
			\scntext{особенность}{Такие адреса будут регистрироваться на какой-либо ресурс (человека, организацию, приложение, файл, цифровой объект и т. д.) и никогда больше не перерегистрироваться (в отличие от IP-адресов и доменов DNS). Это означает, что идентификатор \textit{I-Number} всегда можно будет использовать как адрес для какого-либо ресурса (по крайней мере, пока этот ресурс доступен в сети). Идентификаторы \textit{I-Numbers} очень эффективны, они специально разработаны и оптимизированы для обработки сетевыми маршрутизаторами}
		\scnaddlevel{-1}
		;I-Names\\
		\scnaddlevel{1}
			\scnidtf{удобные для человека названия (похожие на домены системы DNS)}
			\scntext{особенность}{Как имена доменов DNS разрешаются DNS-серверами в IP-адреса, так и \textit{I-Names} будут разрешаться в \textit{I-Numbers}. Но, в отличие от \textit{I-Numbers}, идентификаторы \textit{I-Names} смогут быть переопределены их владельцами для идентификации других ресурсов. Например, если \textit{I-Name} ассоциирован с названием компании, а компания решила изменить своё название, то она может передать свой старый идентификатор \textit{I-Name} другой компании, хотя при этом обе компании останутся со своими старыми идентификаторами \textit{I-Numbers}}
		\scnaddlevel{-1}
	}
	\scniselement{Глобальный контекст XRI}
	\scnaddlevel{1}
		\scnrelfromset{cпецсимволы глобального контекста}{
			\scnfileitem{символ = для частных лиц}
			;\scnfileitem{символ @ для организаций и бизнеса}
			;\scnfileitem{символ + для общих понятий (например: аэропорт, дом, шкаф и т. п.)}
		}
		\scntext{пример}{=Mary.Jones\\
			@Jones.and.Company\\
			+phone.number\\
			+phone.number/(+area.code)\\
			=Mary.Jones/(+phone.number)\\
			@Jones.and.Company/(+phone.number)\\
			@Jones.and.Company/((+phone.number)/(+area.code))}
	\scnaddlevel{-1}
\scnaddlevel{-1}

/*Выводы к разделу 2.2*/\\
\scntext{выводы к разделу}{
	\begin{scnitemize}
		\item Одной из основных задач \textit{URI} и других стандартов идентификации является возможность отличать сущности, имеющие одинаковое название, но возможно, разную трактовку в разных онтологиях (пространствах имен). Данная проблема действительно важна и она решается существующими стандартами;
		\item Вопрос с решением аналогичной проблемы в \textit{OSTIS} активно не поднимался, нужно подумать, актуальна ли она для нас. У нас нет жесткой привязки к именам и в принципе нет трагедии в том, что два разных \textit{sc-элемента} будут иметь одинаковые имена. Проблема возникнет в ситуации, когда мы захотим объединить две базы знаний, где есть сущности с разной семантикой, но одинаковыми именами;
		\item Отправной точкой при разработке стандартов идентификации ресурсов являлась необходимость описания ресурсов в глобальной сети, а не построение модели мира вообще. В связи с этим терминология, используемая при определении идентификаторов, их синтаксис и т.д. “заточены” под концепцию всемирной паутины, однако их пытаются применять как основу для описания любых предметных областей, что не всегда получается понятно и логично. Например, при описании геометрии придется оперировать понятиями путь, запрос, ресурс и т.д., что довольно странно;
		\item Существуют другие недостатки, в частности, создатель \textit{URI}, Тим Бернерс-Ли, говорил, что система доменных имён, лежащая в основе \textit{URL}, — плохое решение, навязывающее ресурсам иерархическую архитектуру, мало подходящую для гипертекстового веба.
	\end{scnitemize}
}

/*2.3.1. Ранние языки представления информации*/
\scnheader{Ранние языки представления информации}
\scniselement{KL-ONE}
\scnaddlevel{1}
	\scnidtf{система языков, разработанная в 1980ых и интегрирующая идеи семантических сетей и фреймов (понятие класса, подкласса, наследования и т.д.), на основе которых были построены одни из первых средств прямого вывода на онтологиях (на основе отношения класс-подкласс)}
\scnaddlevel{-1}

\scniselement{LOOM}
\scnaddlevel{1}
	\scnidtf{язык со строгой формальной семантикой, разработанный в 1990ых на основе \textit{KL-ONE}, конструкциям которого могут быть поставлены в соответствие либо теоретико-множественные выражения, либо выражения логики предикатов первого порядка}
\scnaddlevel{-1}

\scniselement{KIF}
\scnaddlevel{1}
	\scnidtf{Knowledge Interchange Format}
	\scnidtf{язык, схожий с \textit{LOOM} и другими \textit{KL-ONE} языками, но предназначенный в первую очередь для обмена знаниями между компьютерными системами}
	\scntext{особенность}{Тексты \textit{KIF} могут быть протранслированы в \textit{RDF} и наоборот}
	\scnreltolist{ключевой знак}{\scncite{KIFa};\scncite{KIFb}}
\scnaddlevel{-1}

\scniselement{KQML}
\scnaddlevel{1}
	\scnidtf{Knowledge Query and Manipulation Language}
	\scnidtf{язык, предназначенный для обмена сообщениями между агентами, в настоящее время вытеснен стандартом ACL (Agent Communication Language) от FIPA}
\scnaddlevel{-1}

/*2.3.2. RDF*/
\scnheader{RDF}
\scnidtf{Resource Description Framework}
\scnidtf{разработанная консорциумом Всемирной паутины модель для представления данных, в особенности — метаданных. \textit{RDF} представляет утверждения о ресурсах в виде, пригодном для машинной обработки}
\scntext{особенность}{Ресурсом в \textit{RDF} может быть любая сущность — как информационная (например, веб-сайт или изображение), так и неинформационная (например, человек, город или некое абстрактное понятие). Утверждение, высказываемое о ресурсе, имеет вид «субъект — предикат — объект» и называется триплетом}
\scntext{особенность}{Для обозначения субъектов, отношений и объектов в \textit{RDF} используются \textit{URI}}
\scnnote{\textit{RDF} сам по себе является не форматом файла, а только лишь абстрактной моделью данных, то есть описывает предлагаемую структуру, способы обработки и интерпретации данных. Для хранения и передачи информации, уложенной в модель \textit{RDF}, существует целый ряд форматов записи}
\scnrelfromlist{формат записи}{RDF/XML\\
\scnaddlevel{1}
	\scnidtf{запись в виде XML-документа}
\scnaddlevel{-1}
;RDF/JSON
\scnaddlevel{1}
	\scnidtf{запись в виде JSON-данных}
\scnaddlevel{-1}
;RDFa
\scnaddlevel{1}
	\scnidtf{запись внутри атрибутов произвольного HTML- или XHTML-документа}
\scnaddlevel{-1}
;N-Triples
\scnaddlevel{1}
	\scnidtf{компактная форма записи утверждений}
\scnaddlevel{-1}
;Turtle
\scnaddlevel{1}
	\scnidtf{компактная форма записи утверждений}
\scnaddlevel{-1}
;N3
\scnaddlevel{1}
	\scnidtf{компактная форма записи утверждений}
\scnaddlevel{-1}}
\scnnote{\textit{RDF} предоставляет средства для построения информационных моделей, но не касается семантики описываемого. Взятый в отдельности граф \textit{RDF} можно понимать только как граф. Толкование значения основывается на способности пользователей \textit{RDF} интерпретировать отдельные \textit{URI}, строковые литералы и структуру графа, и по ним интерпретировать остальные \textit{URI} и семантику данных}
\scntext{особенность}{Для выражения семантики требуются \textit{словари} (англ. vocabularies), \textit{таксономии} (англ. taxonomies) и \textit{онтологии} (англ. ontologies) и наличие в рассматриваемом графе связей с ними}
\scniselement{Словарь}
\scnaddlevel{1}
	\scnidtf{представляет собой собрание терминов, имеющих во всех контекстах использования этого словаря одинаковый смысл}
\scnaddlevel{-1}
\scniselement{Таксономия}
\scnaddlevel{1}
	\scnidtf{словарь иерархически организованных терминов}
\scnaddlevel{-1}
\scniselement{Онтология}
\scnaddlevel{1}
	\scnidtf{использует предопределённый зарезервированный словарь терминов для определения концепций и отношений между ними для конкретной предметной области}
	\scnnote{Онтологии можно использовать для выражения семантики терминов словаря, их взаимоотношений и контекстов использования}
\scnaddlevel{-1}
\scntext{особенность}{Большинство словарей для описываемых субъектов не только содержит предикаты и объекты, но и подразумевает для них ту или иную семантическую нагрузку, не укладывающуюся как правило в собственно RDF-представление словаря. Это могут быть способы использования тех или иных конкретных субъектов, правила, ограничения, рекомендации, обоснования необходимости использования именно их, и т. п. Как правило, это описывается в сопроводительной документации к словарю}
\scnnote{Словари PDF также называются RDF-схемами (не путать с RDFS). В настоящее существует довольно большое количество открытых словарей, некоторые из которых используются сообществом на уровне де-факто и де-юре стандартов}
\scnrelto{ключевой знак}{\scncite{RDF}}

\scnendstruct

\end{SCn}

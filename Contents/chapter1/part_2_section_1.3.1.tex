\begin{SCn}
	
\scnheader{смысловое представление информации}
\scnidtf{смысловая форма представления информации}
\scnidtf{смысловое представление информационной конструкции}
\scnidtf{знаковая конструкция (текст), представленная в смысловой форме}
\scnidtf{смысловое представление информационной конструкции}
\scnidtftext{часто используемый sc-идентификатор}{смысл}
\scnidtf{смысловое представление}
\scnidtf{семантическое представление информации}

\scntext{основной принцип}{Как можно меньше лишнего, не имеющего отношения к смыслу представляемой информации.}
\scnidtf{такое представление информационной конструкции, которое существенно прощает соответствие между структурой самой этой информационной конструкции и описываемой (отображаемой) ею конфигурацией связей между рассматриваемыми (исследуемыми) сущностями}
\scnidtf{смысловое представление знаковой конструкции}
\scnidtf{абстрактная знаковая конструкция, являющаяся \uline{инвариантом} соответствующего максимального класса семантически эквивалентных знаковых конструкций}
\scnidtf{смысл информационной конструкции}
\scnidtf{денотационная семантика информационной конструкции}
\scnidtf{смысловое представление информационной конструкции}


\scnnote{Суть (смысл, денотационная семантика) любой информационной конструкции (информационной модели) сводится к описанию системы (конфигурации) связей между списываемыми (рассматриваемыми) сущностями. Важно, чтобы эта суть не была \uline{закамуфлирована} различными "синтаксическими"{} деталями, не имеющими никакого отношения к указанному смыслу (синтаксическая структура знаков, многократное повторение одного и того же знака, синонимия, омонимия, местоимения, предлоги, знаки препинания, разделители, ограничители, падежи и т.п.) а обусловленными \uline{формой} представления информационных конструкций, например, их линейностью.}


\scnexplanation{Смысловое представление любой информации в конечном счете сводится:
	\scnaddlevel{1}
	\begin{scnitemize}
		\item{к перечню знаков конкретных описываемых сущностей - как первичных сущностей, так и вторичных сущностей, которые сами являются информационными конструкциями (фрагментами данной конструкции)};
		\item{к явному описанию связи между знаками вторичных сущностей и самими этими сущностями (т.е. фрагментами информационной конструкции)};
		\item{к описанию других связей между описываемыми сущностями}
	\end{scnitemize}
}
\scnaddlevel{-1}

\scnauthorcomment{Дооформить ссылки}

\scnexplanation{Формализация смысла представляемой информации, т.е. строгое уточнение того, что такое \textit{смысловое представление информации}, является объективной основой для \uline{унификации} представления информации в \textit{памяти компьютерных систем} и \uline{ключом} к решению многих проблем семантической совместимости и эволюции компьютерных систем и технологий.

Согласно \textit{Мартынову В. В.} ~\scncite{Martynov}, <<фактически всякая мыслительная деятельность человека (не только научная), как полагают многие ученые, использует \uline{внутренний семантический код}, на который переводят с естественного языка и с которого переводят на естественный язык. Поразительная способность человека к идентификации огромного множества структурно различных фраз с одинаковым \textit{смыслом} и способность \uline{запомнить смысл вне этих фраз} убеждает нас в этом.>>

Приведем также слова \textit{Мельчука И. А.}~\scncite{MelchukST}:

<<Идея была следующая -- язык надо описывать следующим образом: надо уметь записывать смыслы фраз. \uline{Не фразы, а их \textit{смыслы}}, что отдельно. Плюс построить систему, которая по смыслу строит фразу. Это та область или тот поворот исследований, при котором интуиция способного лингвиста работает лучше всего: как выразить на данном языке данный смысл. Это -- то, для чего лингвистов учат..

Лингвистический \textit{смысл} научного текста -- это совсем не то, что ты, читая его, из него извлекаешь. Это, очень грубо говоря, инвариант синонимических перифраз. Ты можешь один и тот же смысл выразить очень многими способами. Когда ты говоришь, то можешь сказать по-разному: ``Сейчас я налью тебе вина'', или: ``Дай, я тебе предложу вина'', или: ``Не выпить ли нам по бокалу?'', -- все это имеет один и тот же смысл. И вот можно придумать, как записывать этот \textit{смысл}. Именно его. Не фразу, а \textit{смысл}. И работать надо от этого \textit{смысла} к реальным фразам. Синтаксис там по дороге тоже нужен, но он нужен именно по дороге, он не может быть ни конечной целью, ни начальной точкой. Это -- промежуточное дело.>>~\scncite{Melchuk}.
}

\scnnote{Грамотная унификация (стандартизация) \textit{смыслового представления информации} не должна привести к ограничению творческой свободы авторов различного вида публикуемых научно-технических знаний (и, в том числе, разработчиков \textit{баз знаний}), не должна гарантировать \textit{семантическую совместимость} различных \textit{знаний}, представленных различными авторами (разумеется, при условии соблюдения соответствующих правил построения этих \textit{знаний}). При этом любые \textit{стандарты} (в том числе и принятые стандарты \textit{смыслового представления информации}) должны постоянно эволюционировать. Текущая версия любого стандарта должна быть не догмой, а точкой опоры для дальнейшего совершенствования этого стандарта.}

\scnsuperset{УСК}
\scnaddlevel{1}
\scnidtf{Универсальный Семантический Код}
\scnrelfrom{автор}{Мартынов В. В.}
\scnnote{Разработанный Мартыновым В. В. Универсальный Семантический Код стал важнейшим этапом создания универсальных формальных средств смыслового представления знаний. Основная методологическая идея \textit{Мартынова В. В.}, касающаяся построения \textit{языка смыслового представления знаний}, заключается в том, чтобы выделить смысловые "кирпичики"{}, имеющие достаточно общий характер, а многообразие конкретных смыслов конструировать комбинаторно за счёт различных комбинаций (конфигураций) из этих "кирпичей"{}. Это можно назвать принципом минимизации типов атомарных смысловых фрагментов}

\scnauthorcomment{Дооформить библиографию}

\scnrelto{ключевой знак}{Книга УСК}


\scnheader{смысловое представление информации*}
\scnidtfexp{\textit{Бинарное ориентированное отношение}, каждая \textit{пара} которого связывает некоторую \textit{информационную конструкцию} со смысловым представлением этой \textit{информационной конструкции*}}

\scnsubset{формализация*}


\end{SCn}
\begin{SCn}

\scnsectionheader{Предметная область и онтология интерфейсов компьютерных систем}

\scnstartsubstruct

\scnheader{Предметная область интерфейсов компьютерных систем}
\scnsdmainclasssingle{интерфейс компьютерной системы}
\scnsdclass{***}
\scnsdrelation{***}

\scnheader{интерфейс компьютерной системы}
\scnsuperset{пользовательский интерфейс}
\scnaddlevel{1}
\scnsuperset{естественно-языковой интерфейс}
\scnaddlevel{1}
\scnsuperset{речевой интерфейс}
\scnaddlevel{-1}
\scnaddlevel{-1}


\scnheader{пользовательский интерфейс}
	\scnexplanation{Одним из наиболее важных компонентов компьютерной системы, обеспечивающим обмен информацией между пользователем и компьютерной системой, является пользовательский интерфейс. Он является совокупностью аппаратных и программных средств, обеспечивающих взаимодействие пользователя и компьютерной системы.}
}

\scnrelfromvector{пользовательский интерфейс}{
командный пользовательский интерфейс IMS;
графический пользовательский интерфейс\\
\scnaddlevel{1}
\scnrelto{включение}{SILK-интерфейс}
\scnaddlevel{1}
\scnrelto{включение}{естественно-языковой интерфейс}
\scnaddlevel{1}
\scnidtf{речевой интерфейс}
\scnaddlevel{-3}
}

\scnheader{командный пользовательский интерфейс IMS}
\scnexplanation{Пользовательский интерфейс, при котором обмен информацией между компьютером и пользователем осуществляется путем написания текстовых инструкций или команд.}

\scnheader{графический пользовательский интерфейс}
\scnexplanation{Пользовательский интерфейс компьютерных систем, при котором обмен информацией между копьютерной системой и пользователем осуществляеться в форме диалога, с помощью окон, меню, и других элементов управления.}

\scnheader{SILK-интерфейс}
\scnexplanation{Для диалога между пользователем и системой может использоваться человеческая речь. При таком общении компьютер распознаёт и анализирует речь пользователя, также результат выполниния команд преобразуется в понятную человеку форму,например, в речь или изображения.}

\scnheader{естественно-языковой интерфейс}
\scnexplanation{Обмен информации происходит за счёт диалога, в процессе которого компьютер и пользователь общаются с помощью речи.}

\scnheader{пользовательский интерфейс компьютерных систем}
\scntext{проблемы разработки}{Несмотря на то, что в настоящее время существует большое число вариантов пользовательских интерфейсов, есть необходимость снижения накладных расходов и сроков их разработки, также необходимо устранить невозможность их адаптации под особенности конкретного пользователя. Поскольку пользователь любой системы общается с ней посредством интерфейса, то проблемы, связанные с интерфейсом, часто формируют негативное мнение о всей системе в целом и не позволяют в полной мере использовать ее функционал. Наиболее значимые недостатки современных пользовательских интерфейсов:
\begin{scnitemize}
\item сложность интерфейса компьютерных систем различного рода;
\item велики сроки разработки и затраты на проектирование и поддержку пользовательских интерфейсов;
\item отсутствие унификации в принципах построения пользовательских интерфейсов;
\item затруднена возможность переноса пользовательских интерфейсов с одной платформы реализации на другую
\end{scnitemize}

Следствиями указанных проблем являются:
\begin{scnitemize}
\item отсутствие унификации в принципах построения пользовательских интерфейсов затрудняет
возможность распараллеливания процесса проектирования пользовательских интерфейсов, а также ограничивает возможность повторного использования уже разработанных компонентов;
\item увеличенные затраты времени на обучение использованию таких интерфейсов и изучение дополнительных материалов;
\item Из-за увелечения времени разработки сложнее становится процесс совершенствования пользовательских интерфейсов, что приводит к их быстрому моральному устареванию;
\item Из-за больших отличий между различными пользовательскими интерфейсами увеличиваются сроки переобучения пользователей на работу с новым интерфейсом
\end{scnitemize}
}

\scnendstruct


\end{SCn}
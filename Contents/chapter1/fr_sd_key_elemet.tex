
\begin{SCn}

	\scnstartsubstruct
	\scnsectionheader{Система ключевых знаков}
	\scnexplanation{Система ключевых знаков Стандарта OSTIS должна стать целостным дополнением  к оглавлению  Стандарта OSTIS}
	\scnrelfrom{}{иерархия и последовательнось  ключевыз знаков  должно  четко соответствовать  иерархии и последовательности  разделов  стандарта}
  	\scnrelfrom{}{система ключевых знаков  Стандарта OSTIS, как и его Оглавление должна восприниматься (читаться) как целостный понятный текст}
	\scnhaselement{неатомарный раздел}
	\scnhaselement{атомарный раздел}
	\scnrelfrom{ключевой знак}{\scnrelfrom{ключевая сущность}{\scnidtf{SC-код}\scnidtf{SCg-код}\scnidtf{SCs-код}\scnidtf{SCn-код}}}
	\scnhaselement{декомпозиция* \\ 
		\scnaddlevel{1}
		\scnhaselement{конкатенация*}
		\scnaddlevel{-1}
}
	\scnheader{Предметная область}
	\scnidtf{\textit{sc-модель предметной области}}
	\scnidtf{sc-текст является представлением предметной области}
	\scnhaselement{максимальный класс объекта иследоавния\\}
	
	\scnlistitem{онтология'}
	\scnlistitem{немаксимальный \textit{класс объекта иследоавния}\\ 
		\scnaddlevel{1}
		\scnidtf{подкласс максимального объекта исследования'} 
	\scnaddlevel{-1}}
	\scnlistitem{иследуемое отношение'}
	\scnlistitem{исследуемый парамерт'}
	\\ 
	\scnlistitem{алфавит*(языка)}
	\\
	\scnlistitem{сформированное множество\scnaddlevel{1} \scnidtf{конечное множество все элементы которого представленны соответствующими sc-элементами}
	\scnaddlevel{-1}}
	\scnlistitem{бинарное отношение}
	\scnlistitem{ориентированное отношение}
	\scnlistitem{первый домен*}
	\scnlistitem{второй домен*}
	\scnlistitem{\textit{пояснение*}}
	\scnlistitem{\textit{семантическая эквивалентность*}}
	\scnlistitem{\textit{следствие*}}
	\scnlistitem{\textit{примечание*}}
	\scnlistitem{\textit{опредедение*}}
	

	\scnheader{Стандарт OSTIS}
	\scnrelfrom{ключевой объект спецификации}{
		Технология OSTIS\\
		\scnrelfrom{основные создаваемые продукты}{
			ostis-система
			\scnidtf{множество всевохможных ostis-систем}
			\scnidtf{компьютерная система построенная на Технологии OSTIS }
			\scnrelfromvector{ключевые понятия,соответсвующие принципам легразим}{\\
				\scnlistitem{cмысловые предсталения информации}
				\scnlistitem{агенто-ориентированная обработка информации}
				\scnlistitem{интерфейс компьютерной системы
					\scnaddlevel{1}
					\scnidtf{интерфес компьютерной системы,построенный на основе онтологий}
					\scnidtf{ontology based interface}
					\scnaddlevel{-1}
				}
				\scnlistitem{мультимодальность}
				\scnlistitem{конвергенция}
				\scnlistitem{семантическая совместимость}
				\scnlistitem{унификация}
				\scnlistitem{мультимодальная база знаний}
				\scnlistitem{универсальный язык смыслового представления знаний}
				\scnlistitem{мультимодальный решатель задач }
				\scnlistitem{мультимодальный интерфейс компьютерной системы}			\scnlistitem{гибридная интеллектульная компьютерная система
					\scnaddlevel{1}	
					\scnidtf{мультимодальная интеллектульная компьютерная система}
				}
				\scnaddlevel{-1}	
				\scnlistitem{мультимодальный(гибридный) характер ИКС в целом}
				\scnlistitem{мультимодальный характер баз знаний ИКС(интеллектульная компьютерная система)}
				\scnlistitem{мультимодальный(мультиязычный)интерфейс ИКС}
				\scnlistitem{конвергенция ИКС семантическая совместимость ИКС(знаний, моделей, решателей задач моделей взаимодействий с внешней средой моделей общения с внешним субъектом)}
				\scnlistitem{онтологическая логико-семантическая модель ИКС}
				\scnlistitem{онтологическая модель базы знаний ИКС}
				\scnlistitem{онтологическая моддель решения задач ИКС}
				\scnlistitem{онтологическая модель интерфейса  ИКС}
			}
			\scnaddlevel{-1}	
			\scnheader{онтологическая модель}
			\scnaddlevel{1}	\scnidtf{модель основанная на иерархической системе онтологий обеспечивающей четких спецификацией всех используемые понятий}
			\scnaddlevel{-1}
		}
	}
   \scnaddlevel{2}
	\scnrelfromvector{ключевые знаки}{
			база знаний OSTIS-системы
			\scnaddlevel{1}	
			\scnidtf{база знаний, предстваленная в sc-коде}
			\scnidtf{sc-модель в базе знаний}
		\scnaddlevel{-1}	
		\scnlistitem{технология проектирования баз знаний остис системы}
		\scnlistitem{агентно-ориентированная модель обработки информации \scnaddlevel{1} \scnidtf{многоагентная модель обработки информации} \scnidtf{\textit{модель обработки информации},рассматривющая \textit{процесс обработки информации }как \textit{деятельность},выполняемую некоторым \textit{коллективом} самостоятельных \textit{информациионных агентов}(агентов обработки информации) }
		\scnaddlevel{-1}}
		\scnlistitem{
			логическая sc-модель обработки знаний
		}
		\scnlistitem{
			технология проектирования логических sc-моделей обработки знаний
		}
		\scnlistitem{
			продукционной sc-моделей обработки знаний
		}
		\scnlistitem{
			технология проектирования продукционных sc-моделей обработки знаний
		}
		\scnlistitem{
			sc-модель искусственной нейронной сети 
		}
		\scnlistitem{
			технология проектирования  sc-моделей искусственных нейронных сетей
		}
		\scnlistitem{
			sc-модель интерфейса ostis-системы
		}
		\scnlistitem{
			технология проектирования  sc-моделей интерфейсов ostis-системы
		}
		\scnlistitem{
			sc-модель \#sc-интерфейса ostis-системы
			\scnaddlevel{1}	
			\scnidtf{онтологическая модель sc-интерфейса построенная на основе sc-кода}	
		}
		\scnaddlevel{-1}	
		\scnlistitem{технология проектирования  sc-моделей \#sc-интерфейсов ostis-системы}\\
		\scnlistitem{программная плотформа реализации ostis-системы, построенная на основе систем управления графовыми базами данных
			\scnaddlevel{1}	
			\scnidtf{программная система интерпритации логик-семантических моделей ostis-систем, постровенных на основе графовой с.у.б.д.(СУБД)}\\
			\scnidtf{система управления базами знаний(с.у.б.з,СУБЗ) ostis-систем, построенная на основе с.у.б.д(СУБД)}}
		\scnaddlevel{-1}	
		\scnlistitem{
			ассоциатиный семантический компьютер для ostis-систем
			\scnaddlevel{1}	
			\scnidtf{компьютер с ассоциативной графодинамической (структурно реконфигурируемой),памятью ориентированный на реализацию ostis-систем}\\
			\scnidtf{компьютер с ассоциативной графодинамической памятью обеспечивающий интерпритацию логико-семантических моделей ostis-систем}\\
			\scnidtf{аппаратная плотформа реализации ostis-систем}
		}
		\scnaddlevel{-1}	
		\scnlistitem{Экосистема с OSTIS	\scnaddlevel{1} \scnidtf{Экосистема ostis-систем и их пользователей}\\
			\scnidtf{Вариант построения smart-общества(общества 5.0) на основе ostis-систем}
			\scnaddlevel{-1}
		}
   	}


	\bigskip
	
	\scnendstruct 
\end{SCn}

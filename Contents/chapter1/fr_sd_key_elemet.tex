
\begin{SCn}

	\scnstartsubstruct
	\scnsectionheader{Система ключевых знаков}
	\scnidtfexp{Система ключевых знаков Стандарта OSTIS должна стать целостным дополнением  к оглавлению  Стандарта OSTIS
		\begin{scnitemize}
			\scnlistitem{иерархия и последовательность ключевых знаков  должны четко соответствовать иерархии и последовательности  разделов  стандарта}
  			\scnlistitem{система ключевых знаков  Стандарта OSTIS, как и его Оглавление должна восприниматься (читаться) как целостный понятный текст}
  		\end{scnitemize}
}

\scnheader{Стандарт OSTIS}
\scnrelfromvector{ключевые знаки}{\\
	
	\scnlistitem{база знаний OSTIS-системы
		\scnaddlevel{1}	
		\scnidtf{база знаний, предстваленная в sc-коде}
		\scnidtf{sc-модель в базе знаний}
	}
\scnaddlevel{-1}
	\scnlistitem{технология проектирования баз знаний ostis-системы}
	
	\scnlistitem{
		логическая sc-модель обработки знаний
	}
	\scnlistitem{
		технология проектирования логических sc-моделей обработки знаний
	}
	\scnlistitem{
		продукционная sc-модель обработки знаний
	}
	\scnlistitem{
		технология проектирования продукционных sc-моделей обработки знаний
	}
	\scnlistitem{
		sc-модель искусственной нейронной сети 
	}
	\scnlistitem{
		технология проектирования sc-моделей искусственных нейронных сетей
	}
	\scnlistitem{
		sc-модель интерфейса ostis-системы
	}
	\scnlistitem{
		технология проектирования  sc-моделей интерфейсов ostis-систем
	}
	\scnlistitem{
		sc-модель sc-интерфейса ostis-системы
		\scnaddlevel{1}	
		\scnidtf{онтологическая модель sc-интерфейса построенная на основе sc-кода}	
	}
	\scnaddlevel{-1}	
	\scnlistitem{технология проектирования  sc-моделей sc-интерфейсов ostis-системы}\\
	\scnlistitem{программная плотформа реализации ostis-систем, построенная на основе системы управления графовыми базами данных
		\scnaddlevel{1}	
		\scnidtf{программная система интерпретации логико-семантических моделей ostis-систем, постровенная на основе графовой с.у.б.д.(СУБД)}\\
		\scnidtf{система управления базами знаний(с.у.б.з,СУБЗ) ostis-систем, построенная на основе графовых с.у.б.д(СУБД)}}
	\scnaddlevel{-1}	
	\scnlistitem{
		ассоциатиный семантический компьютер для ostis-систем
		\scnaddlevel{1}	
		\scnidtf{компьютер с ассоциативной графодинамической (структурно реконфигурируемой),памятью ориентированный на реализацию ostis-систем}\\
		\scnidtf{компьютер с ассоциативной графодинамической памятью обеспечивающий интерпретацию логико-семантических моделей ostis-систем}\\
		\scnidtf{аппаратная плотформа реализации ostis-систем}
	}
	\scnaddlevel{-1}	
	\scnlistitem{Экосистема OSTIS	\scnaddlevel{1} 
		\scnidtf{Экосистема ostis-систем и их пользователей}\\
		\scnidtf{Вариант построения smart-общества(общества 5.0) на основе ostis-систем}
		\scnaddlevel{-1}
	}
	\scnlistitem{агентно-ориентированная модель обработки информации \scnaddlevel{1} \scnidtf{многоагентная модель обработки информации} \scnidtf{\textit{модель обработки информации},рассматривющая \textit{процесс обработки информации }как \textit{деятельность},выполняемую некоторым \textit{коллективом} самостоятельных \textit{информациионных агентов}(агентов обработки информации) }
		\scnaddlevel{-1}}
}
\bigskip
	\scnrelfrom{ключевой объект спецификации}{
	Технология OSTIS\\
	\scnrelfrom{основные создаваемые продукты}{
		ostis-система
		\scnidtf{множество всевохможных ostis-систем}
		\scnidtf{компьютерная система построенная на Технологии OSTIS }
		\scnrelfromvector{ключевые понятия,соответсвующие принципам лежащими в основе}{\\
			\scnlistitem{cмысловые представление информации}
			\scnlistitem{агентно-ориентированная обработка информации}
			\scnlistitem{интерфейс компьютерной системы
				\scnaddlevel{1}
				\scnidtf{интерфес компьютерной системы,построенный на основе онтологий}
				\scnidtf{ontology based interface}
				\scnaddlevel{-1}
			}
			\scnlistitem{мультимодальность}
			\scnlistitem{конвергенция}
			\scnlistitem{семантическая совместимость}
			\scnlistitem{унификация}
			\scnlistitem{мультимодальная база знаний}
			\scnlistitem{универсальный язык смыслового представления знаний}
			\scnlistitem{мультимодальный решатель задач }
			\scnlistitem{мультимодальный интерфейс компьютерной системы}			\scnlistitem{гибридная интеллектульная компьютерная система
				\scnaddlevel{1}	
				\scnidtf{мультимодальная интеллектульная компьютерная система}
			}
			\scnaddlevel{-1}	
			\scnlistitem{мультимодальный(гибридный) характер и.к.с. в целом}
			\scnlistitem{мультимодальный характер баз знаний и.к.с.}
			\scnlistitem{мультимодальный решатель задач и.к.с.}
			\scnlistitem{мультимодальный(мультиязычный)интерфейс и.к.с.}
			\scnlistitem{конвергенция и.к.с.(знаний, моделей, решателей задач моделей взаимодействий с внешней средой моделей общения с внешним субъектом)}
			\scnlistitem{семантическая совместимость и.к.с.(знаний, моделей, решателей задач моделей взаимодействий с внешней средой моделей общения с внешним субъектом)}
			\scnlistitem{онтологическая логико-семантическая модель и.к.с.}
			\scnlistitem{онтологическая модель базы знаний и.к.с.}
			\scnlistitem{онтологическая модель решатель задач и.к.с.}
			\scnlistitem{онтологическая модель интерфейса  и.к.с.}
		}
		\scnaddlevel{-1}	
		\scnheader{онтологическая модель}
		\scnaddlevel{1}	\scnidtf{модель основанная на иерархической системе онтологий обеспечивающей четкую спецификацию всех используемых понятий}
	}
}
\scnaddlevel{1}
	\scnhaselement{неатомарный раздел}
	\scnhaselement{атомарный раздел}
	\scnhaselement{ключевой знак}{ключевая сущность*}\\
	\scnhaselement{SC-код}\scnhaselement{SCg-код}\scnhaselement{SCs-код}\scnhaselement{SCn-код}
	\scnhaselement{декомпозиция* \\ 
		\scnaddlevel{1}
		\scnhaselement{конкатенация*}
		\scnaddlevel{-1}
}
\bigskip
	\scnhaselement{Предметная область\\
		\scnaddlevel{1}
		\scndefinition{\textit{sc-модель предметной области}}
	\scnaddlevel{-1}}
	\scnhaselement{sc-текст является представлением предметной области'}
	\scnhaselement{максимальный класс объекта иследования'\\}

	\scnhaselement{онтология}
	\scnhaselement{немаксимальный \textit{класс объекта иследоавния'}\\ 
		\scnaddlevel{1}
		\scnidtf{подкласс максимального объекта исследования'} 
	\scnaddlevel{-1}}
	\scnhaselement{иследуемое отношение'}
	\scnhaselement{исследуемый парамерт'}
	\\ 
	\scnhaselement{алфавит*(языка)}
	\\
	\scnhaselement{сформированное множество\scnaddlevel{1} \scnidtf{конечное множество, все элементы которого представленны соответствующими sc-элементами}
	\scnaddlevel{-1}}
	\scnhaselement{бинарное отношение}
	\scnhaselement{ориентированное отношение}
	\scnhaselement{первый домен*}
	\scnhaselement{второй домен*}
	\\
	\scnhaselement{\textit{пояснение*}}
	\scnhaselement{\textit{семантическая эквивалентность*}}
	\scnhaselement{\textit{следствие*}}
	\scnhaselement{\textit{примечание*}}
	\scnhaselement{\textit{опредедение*}}
   
	

	\bigskip
	
	\scnendstruct 
\end{SCn}

\scsection{Предметная область и онтология кибернетических систем}
\label{intro_hs}

\begin{SCn}
	
\scnsectionheader{\currentname}

\scnstartsubstruct

\scsectionbeginningname{Начало Предметной области и онтологии кибернетических систем}

\scnstartsubstruct

\scnidtf{Иерархическая система свойств (характеристик) кибернетических систем, определяющих общий (интегральный) уровень их качества}
\scnidtf{Эволюционный подход к определению качества и, в частности, уровня интеллекта кибернетической системы}

\scntext{аннотация}{Рассмотрена иерархическая система свойств (в т.ч. способностей) кибернетических систем, определяющих их качество и позволяющих сформулировать требования, которым должна удовлетворять высокоинтеллектуальная система (идеальная интеллектуальная система).}

\scntext{предисловие}{Свойства (способности), которым должны удовлетворять \textit{интеллектуальные системы}, рассматриваются в целом ряде публикаций. Тем не менее, для \uline{практической} реализации \textit{компьютерных систем}, обладающих указанными свойствами (способностями), т.е. \textit{интеллектуальных компьютерных систем}, необходимо детализировать (уточнить) эти \textit{свойства}, пытаясь свести их к более конструктивным, прозрачным и понятным для реализации свойствам.}

\scnrelfromset{рассматриваемые вопросы}{
\scnfileitem{По каким свойствам (параметрам, характеристикам, способностям) кибернетических систем можно оценивать уровень их качества.};
\scnfileitem{Можно ли считать уровень развития какого-либо свойства (способности) кибернетической системы, т.е. значение какого-либо ее параметра (характеристики) оценкой уровня качества кибернетической системы по соответствующему аспекту.};
\scnfileitem{Может ли какое-либо свойство кибернетических систем определять (влиять на) значение сразу нескольких свойств более высокого уровня иерархии.};
\scnfileitem{Какими отношениями свойства кибернетических систем связаны со свойствами более низкого и, соответственно, более высокого уровня иерархии.};
\scnfileitem{Зачем нужна такая иерархия свойств, определяющих качество кибернетических систем и позволяющих детализировать (уточнять) то, какими свойствами определяется уровень (степень) развития каждого свойства (значение каждого свойства) за исключением свойств, которые условно можно считать элементарными, не требующими детализации (по крайнем мере, пока).};
\scnfileitem{Может ли иерархия свойств, определяющих качество кибернетических систем, быть критерием оценки и выбора того или иного подхода к построению интеллектуальных компьютерным систем.};
\scnfileitem{Какими свойствами (способностями) должна обладать кибернетическая система, имеющая высокий уровень интеллекта.};
\scnfileitem{Какими свойствами определяется уровень интеллекта многоагентной кибернетической системы.};
\scnfileitem{Как связан уровень интеллекта многоагентной системы с уровнем интеллекта агентов, входящих в ее состав.};
\scnfileitem{Почему, например, не каждый коллектив высокоинтеллектуальных людей демонстрирует высокий уровень интеллекта самого коллектива.};
\scnfileitem{Какими дополнительными свойствами кроме достаточно высокого уровня интеллекта должны обладать агенты многоагентных систем для обеспечения высокого уровня интеллекта самой многоагентной системы как самостоятельной целостной кибернетической системы.};
\scnfileitem{Как зависит уровень интеллекта многоагентной системы от организации взаимодействия между агентами, например, от использования централизованного или децентрализованного управления.}}

\scnrelfromvector{ключевые знаки}{
	кибернетическая система\\
	\scnaddlevel{1}	
	\scnsubdividing{
		естественная кибернетическая система;
		компьютерная система
		\scnaddlevel{1}	
		\scnidtf{искусственная кибернетическая система}
		\scnaddlevel{-1};
		естественно-искусственная кибернетическая система
		\scnaddlevel{1}
		\scnidtf{кибернетическая система, являющаяся симбиозом компонентов как естественного, так и искусственного происхождения}
		\scnaddlevel{-1}}
	\scnaddlevel{-1};
	качество кибернетической системы;
	физическая оболочка кибернетической системы;
	качество физической оболочки кибернетической системы;
	интеллект
	\scnaddlevel{1}
	\scnidtf{уровень интеллекта кибернетической системы}
	\scnidtf{интеллектуальность}
	\scnaddlevel{-1};
	интеллектуальная система
	\scnaddlevel{1}
	\scnidtf{интеллектуальная кибернетическая система}
	\scnsuperset{интеллектуальная компьютерная система}
	\scnaddlevel{-1};
	информация, хранимая в памяти кибернетической системы;
	качество информации, хранимой в памяти кибернетической системы;
	база знаний;
	смысловое представление информации в памяти кибернетической системы;
	решатель задач кибернетической системы;
	качество решателя задач кибернетической системы;
	память кибернетической системы;
	качество памяти кибернетической системы;
	обучаемость кибернетической системы;
	гибкость кибернетической системы;
	стратифицированность кибернетической системы;
	рефлексивность кибернетической системы
	\scnaddlevel{1}
	\scnidtf{уровень рефлексии кибернетической системы}
	\scnaddlevel{-1};
	многоагентная система;
	качество многоагентной системы;
	унифицированность агентов многоагентной системы;
	семантическая совместимость агентов многоагентной системы;
	социализация кибернетической системы
	\scnaddlevel{1}
	\scnidtf{способность кибернетической системы своей внутренней и внешней деятельностью обеспечивать высокий уровень интеллекта тех многоагентных систем, членом (агентом) которых она является}
	\scnaddlevel{-1}}

\scnauthorcomment{Поправить библиографию}

\scnrelfromvector{библиография}{
	Винер Н. Кибернетика;
	Поспелов Д.А, Гаазе-Рапопорт М. Г.  От амёбы до робота: Модели поведения;
	Финн В.К. [11] в статье Грибовской на OSTIS-2020;
	Кузнецов О.П. - 2009кн ТеореПИ-с.5-6;
	Ярушкина Н.Г. ред 2007-НечетГС-с.88-101;
	Редько В.Г.-2019кн-МоделКЭ}

\scnauthorcomment{проверить названия и порядок после всех правок}

\scnreltovector{конкатенация сегментов}{
	Иерархическая система свойств, определяющих (уточняющих) интегральный уровень качества кибернетической системы;
	Уточнение понятия кибернетической системы;
	Свойства, определяющие общий уровень качества кибернетической системы;
	Свойства, определяющие уровень интеллекта кибернетической системы;
	Свойства, определяющие качество физической оболочки кибернетической системы;
	Свойства, определяющие качество информации, хранимой в памяти кибернетической системы;
	Свойства, определяющие качество решателя задач кибернетической системы;
	Свойства, определяющие уровень обучаемости кибернетической системы;
	Свойства, определяющие уровень социализации кибернетической системы;
	Качество памяти;
	Качество многоагентной системы;
	Итоговый сегмент Начала Раздела}

\newpage


\bigskip
\scnsegmentheader{Уточнение понятия кибернетической системы}
\scnstartsubstruct

\scnheader{кибернетическая система}
\scnidtf{cистема, которая способна \uline{управлять} своими \uline{действиями}, адаптируясь к изменениям состояния внешней среды (среды своего "обитания") в целях самосохранения (сохранения своей целостности и "комфортности"{} существования путем удержания своих "жизненно"{} важных параметров в определенных рамках "комфортности") и/или в целях формирования определенных реакций (воздействий на внешнюю среду) в ответ на определенные стимулы (на определенные ситуации или события во внешней среде), а также которая способна (при соответствующем уровне развития) эволюционировать в направлении:
\begin{scnitemize}
    \item изучения своей внешней среды как минимум для предсказания последствий своих воздействий на внешнюю среду, а также для предсказания изменений внешней среды, которые не зависят от собственных воздействий;
    \item изучения самой себя и, в частности, своего взаимодействия с внешней средой;
    \item создания технологий (методов и средств), обеспечивающих изменение своей внешней среды (условий своего существования) в собственных интересах.
\end{scnitemize}
}
\scnidtf{адаптивная система}
\scnidtf{целенаправленная (целеустремленная) система}
\scnidtf{активный субъект самостоятельной деятельности}
\scnidtf{материальная сущность, способная целенаправленно (в своих интересах) воздействовать  на среду своего обитания  как минимум для сохранения своей целостности, жизнеспособности, безопасности}
\scnnote{Уровень (степень) адаптивности, целенаправленности, активности у систем, основанных на обработке информации может быть самым различным.}
\scnidtf{система, организация функционирования которой основано на обработке информации о той среде, в которой существует эта система}
\scnidtf{материальная сущность, способная к активной  целенаправленной деятельности, которая  на определенном уровне развития указанной сущности становится "осмысленной", планируемой, преднамеренной деятельностью}
\scnidtf{субъект, способный на самостоятельное выполнение некоторых "внутренних"{} и "внешних"{} действий либо порученных извне, либо инициированных самим субъектом}
\scnidtf{сущность, способная выполнять роль субъекта деятельности}
\scnidtf{естественная или искусственно созданная система, способная мониторить и анализировать свое состояние и состояние окружающей среды, а также способная достаточно активно воздействовать на собственное на собственное состояние и на состояние окружающей среды}
\scnidtf{система, способная в достаточной степени самостоятельно взаимодействовать со своей средой , решая различные задачи}
\scnidtf{система, основанная на обработке информации}
	
\scnrelto{ключевой знак}{Глушков В. М. Кибер. - 1979 ст}
\scnaddlevel{1}
	\scniselement{статья}
\scnaddlevel{-1}
\scnauthorcomment{дооформить библиографическую ссылку}

\bigskip
\scnfragmentcaption

\scnheader{Типология кибернетических систем}
\scnstartsubstruct

\scnheader{кибернетическая система}

\scnrelfrom{разбиение}{Признак естественности или искусственности кибернетических систем}
\scnaddlevel{1}
\scneqtoset{естественная кибернетическая система\\
    \scnaddlevel{1}
    \scnidtf{кибернетическая система естественного происхождения}
    \scnsuperset{человек}
    \scnaddlevel{-1}
;компьютерная система\\
    \scnaddlevel{1}
    \scnidtf{искусственная кибернетическая система}
    \scnidtf{кибернетическая система искусственного происхождения}
    \scnidtf{технически реализованная кибернетическая система}
    \scnaddlevel{-1}
;симбиоз естественных и искусственных кибернетических систем\\
    \scnaddlevel{1}
    \scnidtf{кибернетическая система, в состав которой входят компоненты как естественного, так и искусственного происхождения}
    \scnsuperset{сообщество компьютерных систем и людей}
    \scnaddlevel{-1}}
\scnaddlevel{-1}

\scnheader{искусственная сущность}
\scnidtf{артефакт}
\scnidtf{сущность, являющаяся либо результатом человеческой деятельности, либо частью самой этой деятельности}
\scnidtf{сущность искусственного происхождения}
\scnidtf{антропогенная сущность}
\scnsuperset{научно-техническое знание}
\scnaddlevel{1}
\scnidtf{знание, приобретенное в результате научно-технической деятельности человеческого общества}
\scnaddlevel{-1}
\scnsuperset{материальная искусственная сущность}
\scnaddlevel{1}
\scnsuperset{компьютерная система}
\scnaddlevel{-1}

\scnheader{компьютерная система}
\scnidtf{искусственная кибернетическая система}
\scnnote{Особенностью компьютерных систем является то, что они могут выполнять "роль"{} не только продуктов соответствующих действий по реализации этих систем, но и сами являются \textit{субъектами*}, способными выполнять (автоматизировать) широкий спектр действий. При этом интеллектуализация этих систем существенно расширяет этот спектр. \textit{См. интеллектуальная компьютерная система}.}
\scnidtf{технически реализованная кибернетическая система}
\scnidtf{искусственная кибернетическая система}
\scnsubset{кибернетическая система}
\scnsuperset{современная компьютерная система традиционного вида}
\scnsuperset{современная интеллектуальная компьютерная система}
\scnsuperset{интеллектуальная компьютерная система следующего поколения}
\scnaddlevel{1}
\scnsuperset{ostis-система}
\scnnote{Основной тенденцией эволюции компьютерных систем является повышение уровня их интеллектуальности.}
\scnrelfromset{особенность}{\scnfileitem{Ориентация на принципиально новые компьютеры};\scnfileitem{Cущественное повышение уровня интеллекта}}
\scnaddlevel{-1}
\scnrelfrom{разбиение}{Структурная классификация кибернетических систем}
\scnaddlevel{1}
\scneqtoset{простая кибернетическая система\\
;индивидуальная кибернетическая система\\
;многоагентая система\\
\scnaddlevel{1}
\scnsubdividing{
одноуровневый коллектив кибернетических систем
    \scnaddlevel{1}
    \scnidtf{многоагентная система, агентами которой не могут быть многоагентные системы}
    \scnaddlevel{-1}
;иерархический коллектив кибернетических систем
    \scnaddlevel{1}
    \scnidtf{многоагентная система, по крайней мере одним  агентом которой является многоагентная система}
    \scnaddlevel{-1}}
\scnsubdividing{коллектив из простых кибернетических систем\\
\scnaddlevel{1}
\scnnote{Такой коллектив может быть либо одноуровневым, либо иерархическим коллективом}
\scnaddlevel{-1};
коллектив из индивидуальных кибернетических систем;коллектив из индивидуальных и простых кибернетических систем}
\scnaddlevel{-1}}


\scnheader{кибернетическая система}
\scnrelfrom{разбиение}{Классификация кибернетических систем по признаку наличия надсистемы и роли в рамках этой надсистемы}
\scnaddlevel{1}
\scneqtoset{кибернетическая система, не являющаяся частью никакой другой кибернетической системы\\
\scnaddlevel{1}
\scnidtf{кибернетическая система, не имеющая надсистем}
\scnaddlevel{-1}
;кибернетическая система, встроенная в индивидуальную кибернетическую систему\\
;агент многоагентной системы\\
\scnaddlevel{1}
\scnidtf{кибернетическая система, являющаяся агентом одной или нескольких многоагентных систем}
\scnaddlevel{-1}
}
\scnaddlevel{-1}

\scnheader{простая кибернетическая система}
\scnidtf{\textit{кибернетическая система}, уровень развития которой находится ниже уровня \textit{индивидуальных кибернетических систем} и которая является специализированным средством обработки информации специализированным решателем задач, реализующим (интерпретирующим) чаще всего один \textit{метод} решения задач и, соответственно, решающим только заданный \textit{класс задач}}
\scnidtf{специализированный \textit{решатель задач}}
\scnnote{\textit{простая кибернетическая система} может быть \textit{компонентом*}, встроенным в \textit{индивидуальную кибернетическую систему}, а также может быть \textit{агентом*} \scnbigskip \textit{многоагентной системы}, являющейся коллективом из простых кибернетических систем}

\scnheader{индивидуальная кибернетическая система}
\scnidtf{условно выделенный уровень развития \textit{кибернетических систем}, в основе которого лежит переход от \textit{специализированного решателя задач к индивидуальному решателю}, обеспечивающему интерпретацию произвольного (нефиксированного) набора \textit{методов} (программ) решения задач при условии, если эти \textit{методы} введены (загружены, записаны) в \textit{память} \textit{кибернетической системы}}
\scnidtf{кибернетическая система, способная быть самостоятельной}
\scnexplanation{Признаками индивидуальных кибернетических систем являются:
\begin{scnitemize}
    \item наличие \textit{памяти}, предназначенной для хранения как минимум интерпретируемых \textit{методов} (программ)  и обеспечивающей корректировку (редактирование) хранимых \textit{методов}, а также их удаление  из \textit{памяти} и ввод (запись) в \textit{память} новых \textit{методов};
    \item легкая возможность "программировать"{} \textit{кибернетическую систему} на решение других задач, что обеспечивается наличием \textit{универсальной модели решения задач} и, соответственно, \textit{универсальным интерпретатором \uline{любых} моделей}, представленных (записанных) на соответствующем \textit{языке};
    \item наличие пусть даже простых средств коммуникации (обмена информацией) с другими \textit{кибернетическими системами} (например, с людьми);
    \item способность входить в различные \textit{коллективы кибернетических систем}.
\end{scnitemize}
}
\scnnote{класс \textit{индивидуальных кибернетических систем} — это определенный этап эволюции кибернетических систем, означающий переход к кибернетическим системам, которые способны самостоятельно "выживать"}
\scnidtf{самостоятельная автономная, целостная кибернетическая системам}
\scnidtf{субъект деятельности}
\scnnote{\textit{индивидуальная кибернетическая система} может быть агентом (членом) многоагентной системы (членом коллектива индивидуальных кибернетических систем), но некоторые многоагентные системы могут состоять из агентов , не являющихся  \textit{индивидуальными кибернетическими системами}, представляющих собой простые специализированные кибернетические системы, выполняющие достаточно простые действия (см. коллективное поведение автоматов Стефанюк теория самовоспроизводящихся автоматов Джон фон Нейман)}
\scnauthorcomment{Исправить библиографию}

\scnidtf{кибернетическая система, которая обладает достаточной самостоятельностью (целостностью), но не является коллективом таких самостоятельных  кибернетических систем}
\scnidtf{минимальная самостоятельная (самодостаточная, в известной степени автономная) кибернетическая система}
\scnidtf{индивидуальный субъект}

\scnheader{кибернетическая система, встроенная в индивидуальную кибернетическую систему}
\scnrelfrom{включение;пример}{sc-агент ostis-системы}
\scnrelfrom{включение;пример}
{решатель задач ostis-системы}
\scnaddlevel{1}
\scnidtf{коллектив всех sc-агентов ostis-системы}
\scnaddlevel{-1}

\scnheader{многоагентная система}
\scnidtf{коллектив взаимодействующих автономных кибернетических систем, имеющих общую среду обитания (жизнедеятельности)}
\scnsubdividing{одноуровневая многоагентная система;иерархическая многоагентная система}

\scnheader{одноуровневая многоагентная система}
\scnidtf{специализированное средство решения задач, реализующее либо \uline{одну} модель параллельного (распределенного) решения задач соответствующего класса, либо комбинацию \uline{фиксированного числа} разных и параллельно реализованных моделей решения задач}
\scnsubdividing{одноуровневая однородная многоагентная система;одноуровневая неоднородная многоагентная система}

\scnheader{коллектив индивидуальных кибернетических систем}
\scnsubset{многоагентная система}
\scnidtf{многоагентная система, агентами (членами) которой являются \uline{индивидуальные}(!) кибернетические системы}
\scnsubdividing{
коллектив людей\\
\scnaddlevel{1}
\scnidtf{человеческое сообщество}
\scnaddlevel{-1}
;сообщество компьютерных систем и людей
}

\scnheader{иерархический коллектив индивидуальных кибернетических систем}
\scnidtf{многоагентная система, агентами (членами) которой могут быть:
\begin{scnitemize}
    \item индивидуальные кибернетические системы;
    \item коллективы индивидуальных кибернетических систем;
    \item коллективы, состоящие из индивидуальных кибернетических систем и коллективов индивидуальных кибернетических систем и т.д.
\end{scnitemize}
}



\bigskip

\scnfragmentcaption

\scnheader{Структура кибернетической системы}

\scnstartsubstruct

\scnheader{кибернетическая система}
\scnrelfromset{обобщенная декомпозиция}{
информация, хранимая в памяти кибернетической системы;абстрактная память кибернетической системы;решатель задач кибернетической системы;физическая оболочка кибернетической системы
}

\scnheader{информация, хранимая в памяти кибернетической системы}
\scnidtf{информация, хранимая в памяти \textit{кибернетической системы} и представляющая собой информационную модель среды, в которой действует (существует, функционирует) эта \textit{кибернетическая система}
}
\scnidtf{текущее состояние памяти кибернетической системы}
\scnidtf{текущее состояние внутренней (информационной) среды кибернетической системы}
\scnrelto{второй домен}{информация, хранимая в памяти кибернетической системы*}
\scnaddlevel{1}
\scniselement{бинарное отношение}
\scniselement{ориентированное отношение}
\scnaddlevel{-1}

\scnheader{абстрактная память кибернетической системы}
\scnidtf{внутренняя абстрактная информационная среда кибернетической системы, представляющая собой динамическую информационную  конструкцию, каждое состояние которой есть не что иное, как информация , хранимая в памяти кибернетической системы в соответствующий момент времени}
\scnidtf{абстрактная динамическая модель памяти кибернетической системы}
\scnsubset{динамическая информационная конструкция}
\scnaddlevel{1}
\scnidtf{процесс преобразования информационной конструкции}
\scnaddlevel{-1}

\scnheader{решатель задач кибернетической системы}
\scnidtf{совокупность всех навыков (умений), приобретенных кибернетической системой к рассматриваемому моменту}
\scnidtf{встроенный в кибернетическую систему субъект, способный выполнять целенаправленные ("осознанные") действия во внешней среде этой кибернетической системы, а также в её внутренней среде (в абстрактной памяти)}

\scnheader{действие кибернетической системы}
\scnsubset{действие}
\scnidtf{целенаправленное ("осознанное") действие, выполняемое кибернетической системой, а точнее, её решателем задач}
\scnsubdividing{внешнее действие кибернетической системы\\
	\scnaddlevel{1}
	\scnidtf{действие, выполняемое кибернетической системой в её внешней среде}
	\scnidtf{поведенческое действие}
	\scnaddlevel{-1}
;действие кибернетической системы, выполняемое в собственной физической оболочке
;действие кибернетической системы, выполняемое в собственной абстрактной памяти
\scnaddlevel{1}
	\scnidtf{речь идёт о действиях, направленных на преобразование информации, хранимой в памяти, но никак не на преобразование физической памяти (физической оболочки абстрактной памяти)}
\scnaddlevel{-1}	
}
\scnnote{Каждое \uline{сложное} действие,выполняемое кибернетической системой вне собственный абстрактной памяти, включает в себя поддействия, выполняемые в указанной абстрактной памяти. Это означает, что все внешние действия кибернетической системы \uline{управляются} внутренними её действиями (действиями в абстрактной памяти).}

\scnheader{задача}
\scnidtf{спецификация действия}
\scnidtf{формулировка задачи с различной степенью детализации (уточнения) специфицируемого (описываемого) действия, в состав которой может входить:
	\begin{scnitemize}
		\item описание цели (целевой ситуации);
		\item указание объектов (аргументов) действия;
		\item указание типа действия (класса действий, которому принадлежит данное действие);
		\item указание субъекта действия;
		\item указание инструмента (средств) выполненного действия;
		\item и др.
	\end{scnitemize}}

\scnnote{Процесс решения задачи и действие, специфицируемое этой задачей (точнее, процесс выполнения этого действия) суть одно и то же.}

\scnheader{задача, решаемая кибернетической системой}
\scnidtf{задача, решаемая соответствующей кибернетической системой}
\scnidtf{Второй домен отношения "быть задачей, решаемой заданной кибернетической системой*"}
\scnrelboth{следует отличать}{задача, решаемая кибернетической системой*}
\scnaddlevel{1}
\scnidtf{быть задачей, решаемой заданной кибернетической системой*}
\scnaddlevel{-1}
\scnsubdividing{задача, решаемая кибернетической системой во внешней среде\\
	\scnaddlevel{1}
	\scnidtf{внешняя задача кибернетической системы}
	\scnidtf{задача, направленная на изменение состояния внешней среды соответствующей кибернетической системы, но включающая в себя (в качестве подзадач) задачи, решаемые в памяти кибернетической системы, например: 
		\begin{scnitemize}
			\item интерфейсные задачи (анализ первичный информации о текущем состоянии внешней среды),
			\item cенсо-моторную координацию выполнения сложных действий во внешней среде, состоящих из большого количества частных (более простых) действий, находящихся на разных уровнях иерархии,
			\item задачи планирования целенаправленного поведения во внешней среде,
			\item задачи принятия решений.
		\end{scnitemize}}
	\scnaddlevel{-1}
;задача, решаемая кибернетической системой в собственной физической оболочке
;задача решаемая кибернетической системой в абстрактной памяти
	\scnaddlevel{1}
	\scnidtf{задача, полностью решаемая в памяти кибернетической системы и направленная на изменение состояния информации, хранимой в памяти кибернетической системы}
	\scnidtf{внутренняя задача кибернетической системы}
	\scnaddlevel{-1}
}

\scnheader{навык}
\scnsubset{знание}
\scnexplanation{знание частного вида, содержащее (1) некоторый метод -- знание о том, как можно решать задачи, принадлежащие соответствующему множеству задач, (2) полное знание о том, как указанный метод следует интерпретировать (реализовывать), декомпозируя исходные задачи на подзадачи и, в конечном счёте на элементарные действия, выполняемые \textit{процессором кибернетической системы}}
\scnidtf{умение}
\scnidtf{методы и средства, обеспечивающие способность \textit{кибернетической системы} решать некоторое множество задач (выполнять некоторое множество действий)}

\scnheader{интерфейс кибернетической системы}
\scnidtf{условно выделяемый компонент \textit{решателя задач кибернетической системы}, обеспечивающий решение \textit{интерфейсных задач}, направленных на \uline{непосредственную} реализацию взаимодействия \textit{кибернетической системы} с её \textit{внешней средой}}
\scnidtf{решатель интерфейсных задач кибернетической системы}
\scnrelto{обобщенная часть}{решатель задач кибернетической системы}
\scnrelboth{следует отличать}{физическое обеспечение интерфейса кибернетической системы}
\scnaddlevel{1}
\scnrelto{обобщенная часть}{физическая оболочка кибернетической системы}
\scnaddlevel{-1}

\scnheader{физическая оболочка кибернетической системы}
\scnrelfromset{обобщенная декомпозиция}{память кибернетической системы\\
;процессор кибернетической системы
;физическое обеспечение интерфейса кибернетической системы
\scnaddlevel{1}
	\scnidtf{аппаратное обеспечение интерфейса кибернетической системы с её внешней средой}
	\scnrelfromset{обобщенная декомпозиция}{сенсорная подсистема физической оболочки кибернетической системы;
	эффекторная подсистема физической оболочки кибернетической системы}
\scnaddlevel{-1}
;корпус кибернетической системы
}

\scnheader{физическая оболочка кибернетической системы}
\scnidtf{часть кибернетической системы, являющаяся "посредником"{} между её внутренней средой (памятью, в которой хранится и обрабатывается информация кибернетической системы) и её внешней средой}
\scnrelto{второй домен}{физическая оболочка кибернетической системы*}
\scnaddlevel{1}
\scniselement{бинарное отношение}
\scniselement{ориентированное отношение}





\input{Contents/chapter1/intro_hs/segment4.1-section1.1}

\scnheader{не-фактор}
\scnidtf{группа семантических свойств, определяющих качество информации, хранимой в памяти кибернетической системы}
\scneqtoset{корректность/некорректность информации, хранимой в памяти кибернетической системы;
однозначность/неоднозначность информации, хранимой в памяти кибернетической системы;
целостность/нецелостность информации, хранимой в памяти кибернетической системы;
чистота/загрязненность информации, хранимой в памяти кибернетической системы;
достоверность/недостоверность информации, хранимой в памяти кибернетической системы;
точность/неточность информации, хранимой в памяти кибернетической системы;
четкость/нечеткость информации, хранимой в памяти кибернетической системы;
определенность/недоопределенность информации, хранимой в памяти кибернетической системы}
\scnexplanation{Ярешкина Н.Г. ред.2007кн-НечетГС-стр.10-28}
\scnaddlevel{1}
\scnrelto{цитата}{Ярушкина Н.Г. ред.2007кн-НечетГС}
\scnaddlevel{-1}

\scnheader{корректность/некорректность информации, хранимой в памяти кибернетической системы}
\scnidtf{уровень адекватности хранимой информации в той среде, в которой существует кибернетическая система и информационной моделью которой эта хранимая информация является}
%% Сложно понять что написано, возможно я где-то в склонениях перепутал

\scnheader{непротиворечивость/противоречивость информации, хранимой в памяти кибернетической системы}
\scnidtf{уровень присутствия в хранимой информации различного вида противоречий и, в частности, ошибок}

\scnheader{противоречие*}
\scnidtf{пара противоречащих друг другу фрагментов информации, хранимой в памяти кибернетической системы*}
\scnnote{Чаще всего противоречащями друг другу информационными фрагментами являются:
\begin{scnitemize}
\item{явно представленная в памяти некоторая закономерность (некоторое правило);}
\item{информационный фрагмент, не соответствующий (противоречащий) указанной закономерности}
\end{scnitemize}
\bigskip
В этом случае некорректность может присутствовать:
\begin{scnitemize}
\item{либо в информационном фрагменте, который противоречит указанной закономерности;}
\item{либо в самой этой закономерности;}
\item{либо и там и там.}
\end{scnitemize}
}

\scnheader{информационная ошибка}
\scnidtfdef{противоречие, заключающееся в нарушении некоторой закономерности (некоторого правила), которая не подвергается сомнению}

\scnheader{информационная ошибка}
\scnnote{Ошибки (ошибочные фрагменты) в хранимой информации могут быть синтаксическими и семантическими, противоречащими некоторым правилам (закономерностям), которые явно в памяти могут быть не представлены и считаются априори истинными}

\scnheader{полнота/неполнота информации, хранимой в памяти кибернетической системы}
\scnidtf{уровень того, насколько информация, хранимая в памяти кибернетической системы, описывает среду существования этой системы и используемые ею методы решения задач достаточно полно (достаточно детально) для того, чтобы кибернетическая система могла действительно решать все множество соответствующих ей задач}
\scnidtf{уровень соответствия хранимой информации объёму задач (действий), которые соответствующиая кибернетическоая система желает уметь решать (выполнять)}

\scnheader{полнота/неполнота информации, хранимой в памяти кибернетической системы}
\scnidtf{степень достаточности информации, хранимой в памяти кибернетической системы, для достижения целей этой системы, для выполнения своих "обязанностей"}
\scnnote{Чем полнее информация, хранимая в памяти кибернетической системы, чем полнее \uline{информационное обеспечение деятельности этой системы} это системы, тем эффективнее (качественее) сама эта деятельность}
\scnrelfromlist{свойство-предпосылка}{многообразие видов знаний, хранимых в памяти кибернетической системы;
структурированность информации, хранимой в памяти кибернетической системы}

\scnheader{однозначность/неоднозначность информации, хранимой в кибернетической системе}
\scnrelfromlist{свойство-предпосылка}{многообразие форм дублирования информации, хранимой в памяти кибернетической системы;
частота дублирования информации, хранимой в памяти кибернетической системы}

\scnheader{целостность/нецелостность информации, ъранимой в памяти кибернетической системы}
\scnidtf{уровень содержательной информативности информации, хранимой в памяти кибернетической системы}
\scnidtf{уровень того, насколько содержательно (семантически) \uline{связной} является информация, хранимая в памяти кибернетической системы, насколько полно специфицированы \uline{все} описываемые в памяти сущности (путём описания необходимого набора связей этих сущностей с другими описываемыми сущностями), насколько редко или часто в рамках хранимой информации встречаются \textit{информационные дыры}, соответствующие явной недостаточности некоторых спецификаций}

\scnheader{целостность/нецелостность информации, хранимой в памяти кибернетической системы}
\scnidtf{известность/неизвестность информации, хранимой в памяти кибернетической системы}
\scnidtf{многообразие форм и частота присутствия \textit{информационных дыр} в информации, хранимой в памяти кибернетической системы}
\bigskip
\scnheader{информационная дыра в информации, хранимой в памяти кибернетической системы}
\scnidtf{информация, отсутствие которой в памяти кибернетической системы существенно усложняет деятельность этой системы}
\scnnote{Примерами информационных дыр являются:
\begin{scnitemize}
	\item{отсутствующий метод решения часто встречающихся задач;}
	\item{отсутствующее определение используемого определяемого понятия;}
	\item{недостаточно подробная спецификация часто рассматриваемых сущностей}
\end{scnitemize}}

\scnheader{чистота/загрязненность информации, хранимой в памяти кибернетической системы}
\scnidtf{многообразие форм и общее количество информационного мусора, входящего в состав информации, хранимой в памяти кибернетической системы}

\scnheader{информационный мусор, входящий в состав информации, хранимой в памяти кибернетической системы}
\scnidtf{информационный фрагмент, входящий в состав информации, хранимой в памяти кибернетической системы, удаление которого существенно \uline{не} усложнит деятельность кибернетической системы}
\scnnote{Примерами информационного мусора являются:
\begin{scnitemize}
	\item{информация, которая нечасто востребована, но при необходимости может быть легко логически выведена}
	\item{информация, актуальность которой истекла}
\end{scnitemize}}

\scnheader{семантическая мощность языка представления информации в памяти кибернетической системы}
\scnidtf{семантическая мощность внутреннего языка кибернетической системы}
\scnrelfrom{свойство-предпосылка}{гибридность информации, хранимой в памяти кибернетической системы}
\scnnote{Универсальность внутреннего языка кибернетической системы является важнейщим фактором её интеллектуальности}

\scnheader{универсальный язык}
\scnidtf{язык, информационные конструкции которого могут представить (описать) \uline{любую} конфигурацию \uline{любых} связей между \uline{любыми} сущностями}

\scnheader{гибридность информации, хранимой в памяти кибернетической системы}
\scnrelfromlist{свойство-предпосылка}{многообразие видов знаний, хранимых в кибернетической системе;
степень конвергенции и интеграции различного вида знаний, хранимых в памяти кибернетической системы}

\scnheader{многообразие видов знаний, хранимых в памяти кибернетической системы}
\scnrelfromlist{частное свойство}{рефлексивность информации, хранимой в памяти кибернетической системы
\scnaddlevel{1}
\scnidtf{многообразие видов метаинформации (метазнаний), хранимых в памяти кибернетической системы}
\scnaddlevel{-1};
многообразие моделей решения задач, используемых кибернетической системой;
многообразие видов целей, анализируемых или синтезируемых кибернетической системой;
многообразие планов решения задач, решаемых кибернетической системой;
многообразие протоколов решения задач, решаемых кибернетической системой
}

\scnheader{объем информации, хранимой в памяти кибернетической системы}
\scnidtf{объем знаний, приобретенных кибернетической системой к текущему моменту}
\scnidtf{содержательная совокупность всех знаний, хранимых в текущий момент в памяти кибернетической системы}
\scnnote{Чем больше кибернетическая система знает, тем при прочих равных условиях выше уровень её качества}



\end{SCn}
\scsection{Предметная область и онтология кибернетических систем}
\label{intro_hs}

\begin{SCn}
	
\scnsectionheader{\currentname}

\scnstartsubstruct

\scsectionbeginningname{Начало Предметной области и онтологии кибернетических систем}

\scnstartsubstruct

\scnidtf{Иерархическая система свойств (характеристик) кибернетических систем, определяющих общий (интегральный) уровень их качества}
\scnidtf{Эволюционный подход к определению качества и, в частности, уровня интеллекта кибернетической системы}

\scntext{аннотация}{Система свойств (требований), определяющих качество кибернетических систем (и, в том числе, -- интеллектуальных компьютерных систем). Рассмотрена иерархическая система свойств (в т.ч. способностей) кибернетических систем, определяющих их качество и позволяющих сформулировать требования, которым должна удовлетворять высокоинтеллектуальная система (идеальная интеллектуальная система).}

\scntext{предисловие}{Свойства (способности), которым должны удовлетворять\textit{ интеллектуальные системы}, рассматриваются в целом ряде публикаций. Тем не менее, для практической реализации \textit{компьютерных систем}, обладающих указанным свойством (способностями), т.е. \textit{интеллектуальных компьютерных систем}, необходимо детализировать (уточнить) эти \textit{свойства}, пытаясь свести их к более конструктивным, прозрачным и понятным для реализации свойствам.}

\scnrelfromset{рассматриваемые вопросы}{
\scnfileitem{По каким свойствам (параметрам, характеристикам, способностям) кибернетических систем можно оценивать уровень их качества.};
\scnfileitem{Можно ли считать уровень развития какого-либо свойства (способности) кибернетической системы, т.е. значение какого-либо ее параметра (характеристики) оценкой уровня качества кибернетической системы по соответствующему аспекту.};
\scnfileitem{Может ли какое-либо свойство кибернетических систем определять (влиять на) значение сразу нескольких свойств более высокого уровня иерархии.};
\scnfileitem{Какими отношениями свойства кибернетических систем связаны со свойствами более низкого и, соответственно, более высокого уровня иерархии.};
\scnfileitem{Зачем нужна такая иерархия свойств, определяющих качество кибернетических систем и позволяющих детализировать (уточнять) то, какими свойствами определяется уровень (степень) развития каждого свойства (значение каждого свойства) за исключением свойств, которые условно можно считать элементарными, не требующими детализации (по крайнем мере, пока).};
\scnfileitem{Может ли иерархия свойств, определяющих качество кибернетических систем, быть критерием оценки и выбора того или иного подхода к построению интеллектуальных компьютерным систем.};
\scnfileitem{Какими свойствами (способностями) должна обладать кибернетическая система, имеющая высокий уровень интеллекта.};
\scnfileitem{Какими свойствами определяется уровень интеллекта многоагентной кибернетической системы.};
\scnfileitem{Как связан уровень интеллекта многоагентной системы с уровнем интеллекта агентов, входящих в ее состав.};
\scnfileitem{Почему, например, не каждый коллектив высокоинтеллектуальных людей демонстрирует высокий уровень интеллекта самого коллектива.};
\scnfileitem{Какими дополнительными свойствами кроме достаточно высокого уровня интеллекта должны обладать агенты многоагентных систем для обеспечения высокого уровня интеллекта самой многоагентной системы как самостоятельной целостной кибернетической системы.};
\scnfileitem{Как зависит уровень интеллекта многоагентной системы от организации взаимодействия между агентами, например, от использования централизованного или децентрализованного управления.}}

%\filemodetrue
\scnrelfromvector{\textbf{ключевые знаки}}{
	\scnfileitem{кибернетическая система};
	\scnfileitem{качество кибернетической системы};
	\scnfileitem{физическая оболочка кибернетической системы};
	\scnfileitem{качество физической оболочки кибернетической системы};
	\scnfileitem{интеллект};
	\scnfileitem{информация, хранимая в памяти кибернетической системы};
	\scnfileitem{качество информации, хранимой в памяти кибернетической системы};
	\scnfileitem{базы знаний};
	\scnfileitem{решатель задач кибернетической системы};
	\scnfileitem{качество решателя задач кибернетической системы};
	\scnfileitem{память кибернетической системы};
	\scnfileitem{качество памяти кибернетической системы};
	\scnfileitem{обучаемость кибернетической системы};
	\scnfileitem{многоагентная система};
	\scnfileitem{качество многоагентной системы};
	\scnfileitem{унифицированность агентов многоагентной системы};
	\scnfileitem{семантическая совместимость агентов многоагентной системы};
	\scnfileitem{социализация кибернетической системы}
	\scnaddlevel{1}
		\scnidtf{способность кибернетической системы своей внутренней и внешней деятельностью обеспечивать высокий уровень интеллекта тех многоагентных систем, членом (агентом) которых она является}
	\scnaddlevel{-1};
	\scnfileitem{кибернетическая система}
	\scnaddlevel{1}	
	\scnrelfromset{разбиение}{
		\scnfileitem{естественная кибернетическая система};
		\scnfileitem{компьютерная система}
		\scnaddlevel{1}	
			\scnidtf{искусственная кибернетическая система}
		\scnaddlevel{-1};
		\scnfileitem{естественно-искусственная кибернетическая система}
		\scnaddlevel{1}
			\scnidtf{кибернетическая система, являющаяся симбиозом компонентов как естественного, так и искусственного происхождения}
		\scnaddlevel{-1}}
	\scnaddlevel{-1};
	\scnfileitem{смысловое представление информации в памяти кибернетической системы};
	\scnfileitem{интеллект}
	\scnaddlevel{1}
		\scnidtf{уровень интеллекта кибернетической системы}
		\scnidtf{интеллектуальность}
	\scnaddlevel{-1};
	\scnfileitem{интеллектуальная система}
	\scnaddlevel{1}
		\scnidtf{интеллектуальная кибернетическая система}
		\scnsuperset{интеллектуальная компьютерная система}
	\scnaddlevel{-1};
	\scnfileitem{гибкость кибернетической системы};
	\scnfileitem{стратифицированность кибернетической системы};
	\scnfileitem{рефлексивность кибернетической системы}
	\scnaddlevel{1}
		\scnidtf{уровень рефлексии кибернетической системы}
	\scnaddlevel{-1}}
%\filemodefalse

\scnrelfromvector{\textbf{библиография}}{
	Финн В.К. [11] в статье Грибовской на OSTIS-2020;
	Винер Н. Кибернетика;
	Поспелов Д.А, Гаазе-Рапопорт М. Г.  От амёбы до робота: Модели поведения;
	Кузнецов О.П. - 2009;
	Ярушкина Н.Г. ред 2007;
	Редько В.Г.}

%\filemodetrue
\scnreltovector{\textbf{конкатенация сегментов}}{
	Иерархическая система свойств, определяющих (уточняющих) интегральный уровень качества кибернетической системы;
	\scnfileitem{Уточнение понятия кибернетической системы};
	\scnfileitem{Свойства, определяющие общий уровень качества кибернетической системы};
	\scnfileitem{Свойства, определяющие уровень интеллекта кибернетической системы};
	\scnfileitem{Свойства, определяющие качество физической оболочки кибернетической системы};
	\scnfileitem{Свойства, определяющие качество информации, хранимой в памяти кибернетической системы};
	\scnfileitem{Свойства, определяющие качество решателя задач кибернетической системы};
	\scnfileitem{Свойства, определяющие уровень обучаемости кибернетической системы};
	\scnfileitem{Свойства, определяющие уровень социализации кибернетической системы};
	\scnfileitem{Качество памяти};
	\scnfileitem{Качество М.А.С.};
	\scnfileitem{Итоговый сегмент Начала Раздела}}
%\filemodefalse


\bigskip
\scnsegmentheader{Комплекс свойств, определяющих уровень интеллекта кибернетической системы}
\scnstartsubstruct

\scnheader{интеллект}
\scniselement{свойство}
\scniselement{упорядоченное свойство}
\scnidtf{уровень (степень, величина) интеллекта кибернетической системы}
\scnidtf{Семейство классов \textit{кибернетических систем}, обладающих эквивалентным (одинаковым) уровнем интеллекта -- от низкого до высокого уровня интеллекта}
\scnidtf{свойство кибернетических систем, характеризующее эффективность их взаимодействия со своей средой (средой их "жизнедеятельности"{})}
\scnrelfrom{область определения}{кибернетическая система}
\scnexplanation{С формальной точки зрения интеллектуальность -- это семейство классов кибернетических систем, в каждый из которых входят кибернетические системы, эквивалентные по уровню и характеру проявления интеллектуальных свойств (в том числе способностей).\\
Таким образом, характер (вид) интеллектуальных свойств кибернетических систем и уровень их развития для разных кибернетических систем может быть разным. В соответствии с этим кибернетические системы можно сравнивать между собой.}
\scnnote{Основным свойством (характеристикой, качеством, параметром) кибернетической системы является уровень (степень) ее интеллекта, который является \uline{интегральной} характеристикой, определяющей уровень эффективности взаимодействия кибернетической системы со средой своего существования.}
\scnidtf{комплексное свойство (качество) кибернетической системы, определяющее уровень ее "выживаемости"{} во внешней среде и предполагающее возможность воздействия на эту среду и даже возможность ее преобразования}
\scnidtf{интеллектуальный потенциал кибернетической системы}
\scnidtf{спектр знаний, навыков и способностей к обучению кибернетической системы}
\scnidtf{интеллектуальность кибернетической системы}
\scnnote{Процесс эволюции \textit{кибернетических систем} следует рассматривать как процесс повышения уровня их качества по целому ряду свойств (характеристик) и, в первую очередь, как процесс повышения уровня их \textit{интеллекта}. При этом можно говорить об эволюции каждой конкретной \textit{кибернетической системы} в процессе своей "жизнедеятельности"{}, а также об эволюции целого класса \textit{кибернетических систем}, когда новые экземпляры этого класса являются более интеллектуальными, чем их предшественники. В таком аспекте, в частности, можно рассматривать эволюцию \textit{компьютерных систем} (искусственных кибернетических систем).}
\scnnote{Очень важно уточнить, какими иными свойствами \textit{кибернетических систем} определяется уровень и характер их интеллектуальности. Подчеркнем, что \uline{любая} \textit{кибернетическая система} обладает соответствующим уровнем интеллектуальности. Пусть даже и достаточно низким. Существенным является уточнение того, за счет чего уровень интеллектуальности \textit{кибернетической системы} может быть повышен. Нет смысла проводить четкую границу между \textit{интеллектуальными кибернетическими системами} и неинтеллектуальными. Но есть смысл уточнять направления повышения уровня интеллектуальности \textit{кибернетических систем.}}
\scntext{эпиграф}{Никто не может провести линию, отделяющую атмосферу от космоса, или черту, за которой начинается жизнь, или границу электронного облака. Все дело в степени проявления свойства.}
	\scnaddlevel{1}
	\scnrelfrom{автор}{Барт Коско}
	\scnaddlevel{-1}
\scnnote{Прежде, чем говорить о требованиях, предъявляемых к \textit{технологии проектирования и производства интеллектуальных компьютерных систем (искусственных кибернетических систем}, обладающих высоким уровнем \textit{интеллекта)}, необходимо уточнить (детализировать) \textit{свойства}, присущие указанным системам и являющиеся предпосылками, обеспечивающими высокий уровень \textit{интеллекта}. Подчеркнем, что указанные \textit{свойства}, уточняющие (детализирующие, обеспечивающие, определяющие) \textit{свойства} \scnbigspace \textit{интеллектуальных систем}\scnbigspace (\textit{свойства}, определяющие уровень \textit{интеллекта} этих систем) должны быть общими как для искусственных кибернетических систем (\textit{компьютерных систем}), так и для \textit{естественных кибернетических систем.}}
\scnidtf{интегральное качество информационного обеспечения и информационных процессов в кибернетической системе}
\scnidtf{интегральное качество кибернетической системы, определяемое:
	\begin{scnitemize}
		\item уровнем ее образованности -- качеством накопленных к заданному моменту знаний и умений (навыков);
		\item уровнем ее обучаемости -- способностью \uline{самостоятельно} повышать уровень свой образованности.
	\end{scnitemize}}
\scnrelfromlist{свойство-предпосылка}{образованность кибернетической системы
;обучаемость кибернетической системы
;социализация кибернетической системы\\
	\scnaddlevel{1}
	\scnnote{интеллект \textit{кибернетической системы}, как и лежащий в его основе познавательный процесс, выполняемый кибернетической системой, имеет социальный характер, поскольку наиболее эффективно формируется и развивается в форме взаимодействия \textit{кибернетической} системы с другими \textit{кибернетическими системами}}
	\scnaddlevel{-1}}

\end{SCn}
\scsection{Предметная область и онтология кибернетических систем}
\label{intro_hs}

\begin{SCn}

\scnsectionheader{\currentname}

\scnstartsubstruct

\scsectionbeginningname{Начало Предметной области и онтологии кибернетических систем}

\scnstartsubstruct

\scnidtf{Иерархическая система свойств (характеристик) кибернетических систем, определяющих общий (интегральный) уровень их качества}
\scnidtf{Эволюционный подход к определению качества и, в частности, уровня интеллекта кибернетической системы}

\scntext{аннотация}{Рассмотрена иерархическая система свойств (в т.ч. способностей) кибернетических систем, определяющих их качество и позволяющих сформулировать требования, которым должна удовлетворять высокоинтеллектуальная система (идеальная интеллектуальная система).}

\scntext{предисловие}{Свойства (способности), которым должны удовлетворять \textit{интеллектуальные системы}, рассматриваются в целом ряде публикаций. Тем не менее, для \uline{практической} реализации \textit{компьютерных систем}, обладающих указанными свойствами (способностями), т.е. \textit{интеллектуальных компьютерных систем}, необходимо детализировать (уточнить) эти \textit{свойства}, пытаясь свести их к более конструктивным, прозрачным и понятным для реализации свойствам.}

\scnrelfromset{рассматриваемые вопросы}{
\scnfileitem{По каким свойствам (параметрам, характеристикам, способностям) кибернетических систем можно оценивать уровень их качества.};
\scnfileitem{Можно ли считать уровень развития какого-либо свойства (способности) кибернетической системы, т.е. значение какого-либо ее параметра (характеристики) оценкой уровня качества кибернетической системы по соответствующему аспекту.};
\scnfileitem{Может ли какое-либо свойство кибернетических систем определять (влиять на) значение сразу нескольких свойств более высокого уровня иерархии.};
\scnfileitem{Какими отношениями свойства кибернетических систем связаны со свойствами более низкого и, соответственно, более высокого уровня иерархии.};
\scnfileitem{Зачем нужна такая иерархия свойств, определяющих качество кибернетических систем и позволяющих детализировать (уточнять) то, какими свойствами определяется уровень (степень) развития каждого свойства (значение каждого свойства) за исключением свойств, которые условно можно считать элементарными, не требующими детализации (по крайнем мере, пока).};
\scnfileitem{Может ли иерархия свойств, определяющих качество кибернетических систем, быть критерием оценки и выбора того или иного подхода к построению интеллектуальных компьютерным систем.};
\scnfileitem{Какими свойствами (способностями) должна обладать кибернетическая система, имеющая высокий уровень интеллекта.};
\scnfileitem{Какими свойствами определяется уровень интеллекта многоагентной кибернетической системы.};
\scnfileitem{Как связан уровень интеллекта многоагентной системы с уровнем интеллекта агентов, входящих в ее состав.};
\scnfileitem{Почему, например, не каждый коллектив высокоинтеллектуальных людей демонстрирует высокий уровень интеллекта самого коллектива.};
\scnfileitem{Какими дополнительными свойствами кроме достаточно высокого уровня интеллекта должны обладать агенты многоагентных систем для обеспечения высокого уровня интеллекта самой многоагентной системы как самостоятельной целостной кибернетической системы.};
\scnfileitem{Как зависит уровень интеллекта многоагентной системы от организации взаимодействия между агентами, например, от использования централизованного или децентрализованного управления.}}

\scnrelfromvector{ключевые знаки}{
	кибернетическая система\\
	\scnaddlevel{1}	
	\scnsubdividing{
		естественная кибернетическая система;
		компьютерная система
		\scnaddlevel{1}	
		\scnidtf{искусственная кибернетическая система}
		\scnaddlevel{-1};
		естественно-искусственная кибернетическая система
		\scnaddlevel{1}
		\scnidtf{кибернетическая система, являющаяся симбиозом компонентов как естественного, так и искусственного происхождения}
		\scnaddlevel{-1}}
	\scnaddlevel{-1};
	качество кибернетической системы;
	физическая оболочка кибернетической системы;
	качество физической оболочки кибернетической системы;
	интеллект
	\scnaddlevel{1}
	\scnidtf{уровень интеллекта кибернетической системы}
	\scnidtf{интеллектуальность}
	\scnaddlevel{-1};
	интеллектуальная система
	\scnaddlevel{1}
	\scnidtf{интеллектуальная кибернетическая система}
	\scnsuperset{интеллектуальная компьютерная система}
	\scnaddlevel{-1};
	информация, хранимая в памяти кибернетической системы;
	качество информации, хранимой в памяти кибернетической системы;
	база знаний;
	смысловое представление информации в памяти кибернетической системы;
	решатель задач кибернетической системы;
	качество решателя задач кибернетической системы;
	память кибернетической системы;
	качество памяти кибернетической системы;
	обучаемость кибернетической системы;
	гибкость кибернетической системы;
	стратифицированность кибернетической системы;
	рефлексивность кибернетической системы
	\scnaddlevel{1}
	\scnidtf{уровень рефлексии кибернетической системы}
	\scnaddlevel{-1};
	многоагентная система;
	качество многоагентной системы;
	унифицированность агентов многоагентной системы;
	семантическая совместимость агентов многоагентной системы;
	социализация кибернетической системы
	\scnaddlevel{1}
	\scnidtf{способность кибернетической системы своей внутренней и внешней деятельностью обеспечивать высокий уровень интеллекта тех многоагентных систем, членом (агентом) которых она является}
	\scnaddlevel{-1}}

\scnauthorcomment{Поправить библиографию}

\scnrelfromvector{библиография}{
	Винер Н. Кибернетика;
	Поспелов Д.А, Гаазе-Рапопорт М. Г.  От амёбы до робота: Модели поведения;
	Финн В.К. [11] в статье Грибовской на OSTIS-2020;
	Кузнецов О.П. - 2009кн ТеореПИ-с.5-6;
	Ярушкина Н.Г. ред 2007-НечетГС-с.88-101;
	Редько В.Г.-2019кн-МоделКЭ}

\scnauthorcomment{проверить названия и порядок после всех правок}

\scnreltovector{конкатенация сегментов}{
	Иерархическая система свойств, определяющих (уточняющих) интегральный уровень качества кибернетической системы;
	Уточнение понятия кибернетической системы;
	Свойства, определяющие общий уровень качества кибернетической системы;
	Свойства, определяющие уровень интеллекта кибернетической системы;
	Свойства, определяющие качество физической оболочки кибернетической системы;
	Свойства, определяющие качество информации, хранимой в памяти кибернетической системы;
	Свойства, определяющие качество решателя задач кибернетической системы;
	Свойства, определяющие уровень обучаемости кибернетической системы;
	Свойства, определяющие уровень социализации кибернетической системы;
	Качество памяти;
	Качество многоагентной системы;
	Итоговый сегмент Начала Раздела}

\newpage



\bigskip
\scnsegmentheader{Уточнение понятия кибернетической системы}
\scnstartsubstruct

\scnheader{кибернетическая система}
\scnidtf{cистема, которая способна \uline{управлять} своими \uline{действиями}, адаптируясь к изменениям состояния внешней среды (среды своего "обитания") в целях самосохранения (сохранения своей целостности и "комфортности"{} существования путем удержания своих "жизненно"{} важных параметров в определенных рамках "комфортности") и/или в целях формирования определенных реакций (воздействий на внешнюю среду) в ответ на определенные стимулы (на определенные ситуации или события во внешней среде), а также которая способна (при соответствующем уровне развития) эволюционировать в направлении:
\begin{scnitemize}
    \item изучения своей внешней среды как минимум для предсказания последствий своих воздействий на внешнюю среду, а также для предсказания изменений внешней среды, которые не зависят от собственных воздействий;
    \item изучения самой себя и, в частности, своего взаимодействия с внешней средой;
    \item создания технологий (методов и средств), обеспечивающих изменение своей внешней среды (условий своего существования) в собственных интересах.
\end{scnitemize}
}
\scnidtf{адаптивная система}
\scnidtf{целенаправленная (целеустремленная) система}
\scnidtf{активный субъект самостоятельной деятельности}
\scnidtf{материальная сущность, способная целенаправленно (в своих интересах) воздействовать  на среду своего обитания  как минимум для сохранения своей целостности, жизнеспособности, безопасности}
\scnnote{Уровень (степень) адаптивности, целенаправленности, активности у систем, основанных на обработке информации может быть самым различным.}
\scnidtf{система, организация функционирования которой основано на обработке информации о той среде, в которой существует эта система}
\scnidtf{материальная сущность, способная к активной  целенаправленной деятельности, которая  на определенном уровне развития указанной сущности становится "осмысленной", планируемой, преднамеренной деятельностью}
\scnidtf{субъект, способный на самостоятельное выполнение некоторых "внутренних"{} и "внешних"{} действий либо порученных извне, либо инициированных самим субъектом}
\scnidtf{сущность, способная выполнять роль субъекта деятельности}
\scnidtf{естественная или искусственно созданная система, способная мониторить и анализировать свое состояние и состояние окружающей среды, а также способная достаточно активно воздействовать на собственное на собственное состояние и на состояние окружающей среды}
\scnidtf{система, способная в достаточной степени самостоятельно взаимодействовать со своей средой , решая различные задачи}
\scnidtf{система, основанная на обработке информации}
	
\scnrelto{ключевой знак}{Глушков В. М. Кибер. - 1979 ст}
\scnaddlevel{1}
	\scniselement{статья}
\scnaddlevel{-1}
\scnauthorcomment{дооформить библиографическую ссылку}

\bigskip
\scnfragmentcaption

\scnheader{Типология кибернетических систем}
\scnstartsubstruct

\scnheader{кибернетическая система}

\scnrelfrom{разбиение}{Признак естественности или искусственности кибернетических систем}
\scnaddlevel{1}
\scneqtoset{естественная кибернетическая система\\
    \scnaddlevel{1}
    \scnidtf{кибернетическая система естественного происхождения}
    \scnsuperset{человек}
    \scnaddlevel{-1}
;компьютерная система\\
    \scnaddlevel{1}
    \scnidtf{искусственная кибернетическая система}
    \scnidtf{кибернетическая система искусственного происхождения}
    \scnidtf{технически реализованная кибернетическая система}
    \scnaddlevel{-1}
;симбиоз естественных и искусственных кибернетических систем\\
    \scnaddlevel{1}
    \scnidtf{кибернетическая система, в состав которой входят компоненты как естественного, так и искусственного происхождения}
    \scnsuperset{сообщество компьютерных систем и людей}
    \scnaddlevel{-1}}
\scnaddlevel{-1}

\scnheader{искусственная сущность}
\scnidtf{артефакт}
\scnidtf{сущность, являющаяся либо результатом человеческой деятельности, либо частью самой этой деятельности}
\scnidtf{сущность искусственного происхождения}
\scnidtf{антропогенная сущность}
\scnsuperset{научно-техническое знание}
\scnaddlevel{1}
\scnidtf{знание, приобретенное в результате научно-технической деятельности человеческого общества}
\scnaddlevel{-1}
\scnsuperset{материальная искусственная сущность}
\scnaddlevel{1}
\scnsuperset{компьютерная система}
\scnaddlevel{-1}

\scnheader{компьютерная система}
\scnidtf{искусственная кибернетическая система}
\scnnote{Особенностью компьютерных систем является то, что они могут выполнять "роль"{} не только продуктов соответствующих действий по реализации этих систем, но и сами являются \textit{субъектами*}, способными выполнять (автоматизировать) широкий спектр действий. При этом интеллектуализация этих систем существенно расширяет этот спектр. \textit{См. интеллектуальная компьютерная система}.}
\scnidtf{технически реализованная кибернетическая система}
\scnidtf{искусственная кибернетическая система}
\scnsubset{кибернетическая система}
\scnsuperset{современная компьютерная система традиционного вида}
\scnsuperset{современная интеллектуальная компьютерная система}
\scnsuperset{интеллектуальная компьютерная система следующего поколения}
\scnaddlevel{1}
\scnsuperset{ostis-система}
\scnnote{Основной тенденцией эволюции компьютерных систем является повышение уровня их интеллектуальности.}
\scnrelfromset{особенность}{\scnfileitem{Ориентация на принципиально новые компьютеры};\scnfileitem{Cущественное повышение уровня интеллекта}}
\scnaddlevel{-1}
\scnrelfrom{разбиение}{Структурная классификация кибернетических систем}
\scnaddlevel{1}
\scneqtoset{простая кибернетическая система\\
;индивидуальная кибернетическая система\\
;многоагентая система\\
\scnaddlevel{1}
\scnsubdividing{
одноуровневый коллектив кибернетических систем
    \scnaddlevel{1}
    \scnidtf{многоагентная система, агентами которой не могут быть многоагентные системы}
    \scnaddlevel{-1}
;иерархический коллектив кибернетических систем
    \scnaddlevel{1}
    \scnidtf{многоагентная система, по крайней мере одним  агентом которой является многоагентная система}
    \scnaddlevel{-1}}
\scnsubdividing{коллектив из простых кибернетических систем\\
\scnaddlevel{1}
\scnnote{Такой коллектив может быть либо одноуровневым, либо иерархическим коллективом}
\scnaddlevel{-1};
коллектив из индивидуальных кибернетических систем;коллектив из индивидуальных и простых кибернетических систем}
\scnaddlevel{-1}}


\scnheader{кибернетическая система}
\scnrelfrom{разбиение}{Классификация кибернетических систем по признаку наличия надсистемы и роли в рамках этой надсистемы}
\scnaddlevel{1}
\scneqtoset{кибернетическая система, не являющаяся частью никакой другой кибернетической системы\\
\scnaddlevel{1}
\scnidtf{кибернетическая система, не имеющая надсистем}
\scnaddlevel{-1}
;кибернетическая система, встроенная в индивидуальную кибернетическую систему\\
;агент многоагентной системы\\
\scnaddlevel{1}
\scnidtf{кибернетическая система, являющаяся агентом одной или нескольких многоагентных систем}
\scnaddlevel{-1}
}
\scnaddlevel{-1}

\scnheader{простая кибернетическая система}
\scnidtf{\textit{кибернетическая система}, уровень развития которой находится ниже уровня \textit{индивидуальных кибернетических систем} и которая является специализированным средством обработки информации специализированным решателем задач, реализующим (интерпретирующим) чаще всего один \textit{метод} решения задач и, соответственно, решающим только заданный \textit{класс задач}}
\scnidtf{специализированный \textit{решатель задач}}
\scnnote{\textit{простая кибернетическая система} может быть \textit{компонентом*}, встроенным в \textit{индивидуальную кибернетическую систему}, а также может быть \textit{агентом*} \scnbigskip \textit{многоагентной системы}, являющейся коллективом из простых кибернетических систем}

\scnheader{индивидуальная кибернетическая система}
\scnidtf{условно выделенный уровень развития \textit{кибернетических систем}, в основе которого лежит переход от \textit{специализированного решателя задач к индивидуальному решателю}, обеспечивающему интерпретацию произвольного (нефиксированного) набора \textit{методов} (программ) решения задач при условии, если эти \textit{методы} введены (загружены, записаны) в \textit{память} \textit{кибернетической системы}}
\scnidtf{кибернетическая система, способная быть самостоятельной}
\scnexplanation{Признаками индивидуальных кибернетических систем являются:
\begin{scnitemize}
    \item наличие \textit{памяти}, предназначенной для хранения как минимум интерпретируемых \textit{методов} (программ)  и обеспечивающей корректировку (редактирование) хранимых \textit{методов}, а также их удаление  из \textit{памяти} и ввод (запись) в \textit{память} новых \textit{методов};
    \item легкая возможность "программировать"{} \textit{кибернетическую систему} на решение других задач, что обеспечивается наличием \textit{универсальной модели решения задач} и, соответственно, \textit{универсальным интерпретатором \uline{любых} моделей}, представленных (записанных) на соответствующем \textit{языке};
    \item наличие пусть даже простых средств коммуникации (обмена информацией) с другими \textit{кибернетическими системами} (например, с людьми);
    \item способность входить в различные \textit{коллективы кибернетических систем}.
\end{scnitemize}
}
\scnnote{класс \textit{индивидуальных кибернетических систем} — это определенный этап эволюции кибернетических систем, означающий переход к кибернетическим системам, которые способны самостоятельно "выживать"}
\scnidtf{самостоятельная автономная, целостная кибернетическая системам}
\scnidtf{субъект деятельности}
\scnnote{\textit{индивидуальная кибернетическая система} может быть агентом (членом) многоагентной системы (членом коллектива индивидуальных кибернетических систем), но некоторые многоагентные системы могут состоять из агентов , не являющихся  \textit{индивидуальными кибернетическими системами}, представляющих собой простые специализированные кибернетические системы, выполняющие достаточно простые действия (см. коллективное поведение автоматов Стефанюк теория самовоспроизводящихся автоматов Джон фон Нейман)}
\scnauthorcomment{Исправить библиографию}

\scnidtf{кибернетическая система, которая обладает достаточной самостоятельностью (целостностью), но не является коллективом таких самостоятельных  кибернетических систем}
\scnidtf{минимальная самостоятельная (самодостаточная, в известной степени автономная) кибернетическая система}
\scnidtf{индивидуальный субъект}

\scnheader{кибернетическая система, встроенная в индивидуальную кибернетическую систему}
\scnrelfrom{включение;пример}{sc-агент ostis-системы}
\scnrelfrom{включение;пример}
{решатель задач ostis-системы}
\scnaddlevel{1}
\scnidtf{коллектив всех sc-агентов ostis-системы}
\scnaddlevel{-1}

\scnheader{многоагентная система}
\scnidtf{коллектив взаимодействующих автономных кибернетических систем, имеющих общую среду обитания (жизнедеятельности)}
\scnsubdividing{одноуровневая многоагентная система;иерархическая многоагентная система}

\scnheader{одноуровневая многоагентная система}
\scnidtf{специализированное средство решения задач, реализующее либо \uline{одну} модель параллельного (распределенного) решения задач соответствующего класса, либо комбинацию \uline{фиксированного числа} разных и параллельно реализованных моделей решения задач}
\scnsubdividing{одноуровневая однородная многоагентная система;одноуровневая неоднородная многоагентная система}

\scnheader{коллектив индивидуальных кибернетических систем}
\scnsubset{многоагентная система}
\scnidtf{многоагентная система, агентами (членами) которой являются \uline{индивидуальные}(!) кибернетические системы}
\scnsubdividing{
коллектив людей\\
\scnaddlevel{1}
\scnidtf{человеческое сообщество}
\scnaddlevel{-1}
;сообщество компьютерных систем и людей
}

\scnheader{иерархический коллектив индивидуальных кибернетических систем}
\scnidtf{многоагентная система, агентами (членами) которой могут быть:
\begin{scnitemize}
    \item индивидуальные кибернетические системы;
    \item коллективы индивидуальных кибернетических систем;
    \item коллективы, состоящие из индивидуальных кибернетических систем и коллективов индивидуальных кибернетических систем и т.д.
\end{scnitemize}
}



\bigskip

\scnfragmentcaption

\scnheader{Структура кибернетической системы}

\scnstartsubstruct

\scnheader{кибернетическая система}
\scnrelfromset{обобщенная декомпозиция}{
информация, хранимая в памяти кибернетической системы;абстрактная память кибернетической системы;решатель задач кибернетической системы;физическая оболочка кибернетической системы
}

\scnheader{информация, хранимая в памяти кибернетической системы}
\scnidtf{информация, хранимая в памяти \textit{кибернетической системы} и представляющая собой информационную модель среды, в которой действует (существует, функционирует) эта \textit{кибернетическая система}
}
\scnidtf{текущее состояние памяти кибернетической системы}
\scnidtf{текущее состояние внутренней (информационной) среды кибернетической системы}
\scnrelto{второй домен}{информация, хранимая в памяти кибернетической системы*}
\scnaddlevel{1}
\scniselement{бинарное отношение}
\scniselement{ориентированное отношение}
\scnaddlevel{-1}

\scnheader{абстрактная память кибернетической системы}
\scnidtf{внутренняя абстрактная информационная среда кибернетической системы, представляющая собой динамическую информационную  конструкцию, каждое состояние которой есть не что иное, как информация , хранимая в памяти кибернетической системы в соответствующий момент времени}
\scnidtf{абстрактная динамическая модель памяти кибернетической системы}
\scnsubset{динамическая информационная конструкция}
\scnaddlevel{1}
\scnidtf{процесс преобразования информационной конструкции}
\scnaddlevel{-1}

\scnheader{решатель задач кибернетической системы}
\scnidtf{совокупность всех навыков (умений), приобретенных кибернетической системой к рассматриваемому моменту}
\scnidtf{встроенный в кибернетическую систему субъект, способный выполнять целенаправленные ("осознанные") действия во внешней среде этой кибернетической системы, а также в её внутренней среде (в абстрактной памяти)}

\scnheader{действие кибернетической системы}
\scnsubset{действие}
\scnidtf{целенаправленное ("осознанное") действие, выполняемое кибернетической системой, а точнее, её решателем задач}
\scnsubdividing{внешнее действие кибернетической системы\\
	\scnaddlevel{1}
	\scnidtf{действие, выполняемое кибернетической системой в её внешней среде}
	\scnidtf{поведенческое действие}
	\scnaddlevel{-1}
;действие кибернетической системы, выполняемое в собственной физической оболочке
;действие кибернетической системы, выполняемое в собственной абстрактной памяти
\scnaddlevel{1}
	\scnidtf{речь идёт о действиях, направленных на преобразование информации, хранимой в памяти, но никак не на преобразование физической памяти (физической оболочки абстрактной памяти)}
\scnaddlevel{-1}	
}
\scnnote{Каждое \uline{сложное} действие,выполняемое кибернетической системой вне собственный абстрактной памяти, включает в себя поддействия, выполняемые в указанной абстрактной памяти. Это означает, что все внешние действия кибернетической системы \uline{управляются} внутренними её действиями (действиями в абстрактной памяти).}

\scnheader{задача}
\scnidtf{спецификация действия}
\scnidtf{формулировка задачи с различной степенью детализации (уточнения) специфицируемого (описываемого) действия, в состав которой может входить:
	\begin{scnitemize}
		\item описание цели (целевой ситуации);
		\item указание объектов (аргументов) действия;
		\item указание типа действия (класса действий, которому принадлежит данное действие);
		\item указание субъекта действия;
		\item указание инструмента (средств) выполненного действия;
		\item и др.
	\end{scnitemize}}

\scnnote{Процесс решения задачи и действие, специфицируемое этой задачей (точнее, процесс выполнения этого действия) суть одно и то же.}

\scnheader{задача, решаемая кибернетической системой}
\scnidtf{задача, решаемая соответствующей кибернетической системой}
\scnidtf{Второй домен отношения "быть задачей, решаемой заданной кибернетической системой*"}
\scnrelboth{следует отличать}{задача, решаемая кибернетической системой*}
\scnaddlevel{1}
\scnidtf{быть задачей, решаемой заданной кибернетической системой*}
\scnaddlevel{-1}
\scnsubdividing{задача, решаемая кибернетической системой во внешней среде\\
	\scnaddlevel{1}
	\scnidtf{внешняя задача кибернетической системы}
	\scnidtf{задача, направленная на изменение состояния внешней среды соответствующей кибернетической системы, но включающая в себя (в качестве подзадач) задачи, решаемые в памяти кибернетической системы, например: 
		\begin{scnitemize}
			\item интерфейсные задачи (анализ первичный информации о текущем состоянии внешней среды),
			\item cенсо-моторную координацию выполнения сложных действий во внешней среде, состоящих из большого количества частных (более простых) действий, находящихся на разных уровнях иерархии,
			\item задачи планирования целенаправленного поведения во внешней среде,
			\item задачи принятия решений.
		\end{scnitemize}}
	\scnaddlevel{-1}
;задача, решаемая кибернетической системой в собственной физической оболочке
;задача решаемая кибернетической системой в абстрактной памяти
	\scnaddlevel{1}
	\scnidtf{задача, полностью решаемая в памяти кибернетической системы и направленная на изменение состояния информации, хранимой в памяти кибернетической системы}
	\scnidtf{внутренняя задача кибернетической системы}
	\scnaddlevel{-1}
}

\scnheader{навык}
\scnsubset{знание}
\scnexplanation{знание частного вида, содержащее (1) некоторый метод -- знание о том, как можно решать задачи, принадлежащие соответствующему множеству задач, (2) полное знание о том, как указанный метод следует интерпретировать (реализовывать), декомпозируя исходные задачи на подзадачи и, в конечном счёте на элементарные действия, выполняемые \textit{процессором кибернетической системы}}
\scnidtf{умение}
\scnidtf{методы и средства, обеспечивающие способность \textit{кибернетической системы} решать некоторое множество задач (выполнять некоторое множество действий)}

\scnheader{интерфейс кибернетической системы}
\scnidtf{условно выделяемый компонент \textit{решателя задач кибернетической системы}, обеспечивающий решение \textit{интерфейсных задач}, направленных на \uline{непосредственную} реализацию взаимодействия \textit{кибернетической системы} с её \textit{внешней средой}}
\scnidtf{решатель интерфейсных задач кибернетической системы}
\scnrelto{обобщенная часть}{решатель задач кибернетической системы}
\scnrelboth{следует отличать}{физическое обеспечение интерфейса кибернетической системы}
\scnaddlevel{1}
\scnrelto{обобщенная часть}{физическая оболочка кибернетической системы}
\scnaddlevel{-1}

\scnheader{физическая оболочка кибернетической системы}
\scnrelfromset{обобщенная декомпозиция}{память кибернетической системы\\
;процессор кибернетической системы
;физическое обеспечение интерфейса кибернетической системы
\scnaddlevel{1}
	\scnidtf{аппаратное обеспечение интерфейса кибернетической системы с её внешней средой}
	\scnrelfromset{обобщенная декомпозиция}{сенсорная подсистема физической оболочки кибернетической системы;
	эффекторная подсистема физической оболочки кибернетической системы}
\scnaddlevel{-1}
;корпус кибернетической системы
}

\scnheader{физическая оболочка кибернетической системы}
\scnidtf{часть кибернетической системы, являющаяся "посредником"{} между её внутренней средой (памятью, в которой хранится и обрабатывается информация кибернетической системы) и её внешней средой}
\scnrelto{второй домен}{физическая оболочка кибернетической системы*}
\scnaddlevel{1}
\scniselement{бинарное отношение}
\scniselement{ориентированное отношение}





\bigskip

\scnheader{память кибернетической системы}

\scnidtf{физическая оболочка реализация абстрактной памяти кибернетической системы внутренней среды кибернетической системы, в рамках которой кибернетическая система формирует и использует (обрабатывает) информационную модель своей внешней среды} 
\scnnote{Не каждая кибернетическая система имеет память. В кибернетических системах, которые не имеют памяти, обработка информации сводится к обмену сигналами между компонентами этих систем. Появление в кибернетических системах памяти как среды для "централизованного" хранения и обработки информации является важнейшим этапом их эволюции. Дальнейшее эволюция кибернетических систем во многом определяется:
	\begin{scnitemize}

	\item качеством памяти как среды для хранения и обработки информации;
	\item качеством информации (информационные модели), хранимой в памяти кибернетической системы;
	 \end{scnitemize}}
\scnidtf{компонент кибернетической системы, в рамках которого теперь кибернетическая система осуществляет отображение (формировании информационной модели) среды своего существования, а также использование этой информационной модели для управления собственным поведением в указанной среде}
	 
\scnidtf{физическая оболочка для хранения информации, которую кибернетическая система приобретает и обрабатывает (т.е. меняет состояния этой информации)}
\scnidtf{физическая (аппаратная) реализация внутренней среды кибернетическая система, каковой является среда "существования" информации, накапливаемой и непосредственно используемой решателем задач этой кибернетической системы}

\scnnote{Сам факт появления в кибернетической системе памяти, которая (1) обеспечивает представление различного виды информации о среде, в рамках которой кибернетическая система решает различные задачи (выполняет различные действия), (2) обеспечивает хранение достаточно полной информационной модели указанной среды (достаточно полной для реализации своей деятельности), (3) обеспечивает высокую степень гибкости указанной хранимой в памяти информационной модели среды жизнедеятельности (т.е. лёгкость внесения изменений в эту информационную модель), существенно повышает уровень адаптивности кибернетической системы к различным изменениям своей среды}
\scnnote{"появление" \textit{памяти} кибернетических системах является основным признаком перехода от "простых" автоматов к компьютерным системам, от роботов 1-го поколение к роботам следующих поколений}
\scnidtf{физическая реализация хранилища информации, которую приобрела (накопила) к текущему моменту соответствующая кибернетическая система}
\scnidtf{физическая оболочка внутренней абстрактной информационной среды кибернетической системы}
\scnidtf{среда хранения и обработки информации}
\scnidtf{запоминающая среда}
\scnidtf{среда хранения и обработки информационных конструкций}
\scnnote{Принципы организации памяти кибернетической системы могут быть разными(ассоциативная, адресная, структурно фиксированная/структурно перестраиваемая, нелинейная/линейная). От организации памяти во многом зависит её качество}
\filemodefalse

\scntext{уровни эволюции}{
Уровни структурной эволюции кибернетических систем}
\scneqtovector{
простая кибернетическая система, не имеющая память;
простая кибернетическая система, имеющая память;
одноуровневый коллектив, не имеющий общей памяти и  одноуровневый коллектив, не имеющий общей памяти и состоящий из простых кибернетических систем, имеющих память;
иерархический коллектив,  имеющий общую памяти и состоящий из простых кибернетических систем;\\
индивидуальная кибернетическая система\\
\scnaddlevel{1}
\scnnote{Каждая индивидуальная кибернетическая система содержит память, имеющую достаточно высокий уровень качества
одноуровневый коллектив индивидуальных кибернетических систем, не имеющий общей памяти}
 \scnaddlevel{-1}
;одноуровневый коллектив индивидуальных кибернетическая систем, имеющий общую память 
;иерархический коллектив из индивидуальных кибернетических систем, не имеющий общей памяти 
;иерархический коллектив из индивидуальных кибернетических систем, имеющий общую память}\\
\scnheader{процессор кибернетической системы}\\
\scnidtf{физически (аппаратно реализованный) интерпретатор хранимых в памяти кибернетической системы методов (программ), соответствующих базовой (для данной кибернетической системы) модели решения задач, то есть такой модели решения задач, которая для данной кибернетической системы является моделью решения задач самого нижнего уровня и, следовательно, не может быть интерпретирована с помощью другой модели решения задач, используемой этой же кибернетической системой, а может быть проинтерпретирована либо путем аппаратной реализации такого интерпретатора, путём его программной реализации, например, на современных компьютерах, но в последнем случае, кроме собственного интерпретатора, необходимо также построить модель памяти реализуемой кибернетической системы}

\scnidtf{"физически" реализованные средства, обеспечивающее выполнение "элементарных" действий, направленных на изменение состояния памяти кибернетической системы (на изменение информации, хранимой в этой памяти)}
\scnidtf{"движок"("мотор") кибернетической системы}
\scnrelto{второй домен}{\textit{процессор кибернетической системы*}\\
\scnidtfexp{бинарное ориентированное отношения, каждая пара которого связывает знак кибернетической системы со знаком её процессора}
\scniselement{бинарное отношение}
\scniselement{ориентированное отношение}
}
\scnheader{компьютер}\\
\scnsubset{физическая оболочка кибернетической системы}
\scnidtf{физическая оболочка искусственной кибернетической системы} \scnidtf{аппаратное обеспечение компьютерной системы}
\scnidtf{hardware of computer system}

\scnsuperset{компьютер для интеллектуальных систем}
\scnaddlevel{1}
\scnidtf{компьютер, ориентированный на реализацию интеллектуальных компьютерных систем}
\scnnote{Развитие рынка интеллектуальных компьютерных систем существенно сдерживается неприспособленностью современного поколения компьютеров к реализации на их основе интеллектуальных компьютерных систем.Попытки создания компьютеров, приспособленных к реализации интеллектуальных компьютерных систем, не привели к успеху, т.к. эти проекты были направлены на выполнение отдельных (частных) требований, предъявляемых к физическому (аппаратному) уровню интеллектуальных систем, что неминуемо приводило к приспособленности создаваемых компьютеров к реализации не всего многообразия интеллектуальных компьютерных систем, а только некоторых подмножеств таких систем. Указанные подмножества интеллектуальных компьютерных систем в основном определялись
Ориентацией на конкретные используемые модели решения интеллектуальных задач, тогда, как важнейшим фактором, определяющим уровень интеллекта кибернетических систем (в том числе, и компьютерных систем), является их универсальность в плане многообразие используемых моделей решения задач. Следовательно, компьютер для интеллектуальных компьютерных систем должен быть эффективным аппаратным интерпретатором любых моделей решения задач (как интеллектуальных задач, так и достаточно простых задач, т.к. интеллектуальная система должна уметь решать любые задачи).} 
\scnidtf{компьютер, приспособленный к реализации интеллектуальных компьютерных систем}
\scnidtf{универсальный компьютер для интеллектуальных систем}
\scnidtf{компьютер, обеспечивающий интерпретацию любых моделей решения задач}
\scnaddlevel{-1}
\scnheader{ \textit{Семейство отношений, заданных на множестве кибернетических систем}}\\
 \scnstartsubstruct
\\
 отношений, заданное на множестве кибернетических систем\\
\scnhaselement{память кибернетической системы*}
\scnhaselement{процессор кибернетической системы*}
\scnhaselement{член коллектива}
\scnhaselement{внешняя среда кибернетической системы*}
\scnhaselement{сенсор кибернетической системы*}
\scnhaselement{эффектор кибернетической системы*}
\scnhaselement{физическая оболочка кибернетической системы*}
\scnhaselement{информация, хранимая в памяти кибернетической системы*} \scnhaselement{абстрактная память кибернетической системы*}
\scnhaselement{часть*}
\scnaddlevel{1}
\scnsuperset{встроенная кибернетическая система}
\scnaddlevel{-1}
 информация, хранимая в памяти кибернетической системы*\\ 
\scnidtf{информационная модель среды*, в которой существует (осуществляет деятельность) соответствующая кибернетическая система} 
\scnnote{От того, насколько полна, адекватна (корректна) и систематизирована (структурирована) внутренняя среда кибернетической системы, зависит уровень интеллектуальности и эффективность соответствующей кибернетической системы.}
\textit{следует отличать*}\\
\scnhaselementset{решатель задач кибернетической системы*;
решатель задач кибернетической системы\\
\scnaddlevel{1}
\scnidtf{иерархическая система моделей решения задач}
\scntext{обобщённая часть}{\textit{процессор кибернетической системы}}
\scnaddlevel{1}
\scnexplanation{Это реализация модели решения задач, обеспечивающей интерпретацию всех используемых моделей решения задач верхнего уровня} 
}\\
\scnaddlevel{-2}
\scnheader{задача, решаемые кибернетической системой} 
\scnidtf{быть задачей, решаемой заданной кибернетической системой*} \scnsuperset{задача, решаемая в памяти кибернетической системы*}
\scnaddlevel{1}
\scnidtf{внутренняя задача кибернетической системы задача*}
\scnaddlevel{-1}
\scnsuperset{задача, решаемая во внешней среде кибернетической системы*}\\
\scnheader{\textit{внешняя среда кибернетической системы}}
\scnidtf{внешняя среда} 
\scnnote{Понятие внешней среды кибернетической системы является понятием относительным, т.к. (1) разные кибернетические системы имеют в общем случае разную внешнюю среду и (2) одна кибернетическая система может входить в состав внешней среды другой кибернетической системы}
\scnidtf{быть внешней средой для заданной кибернетической системы*} \scniselement{бинарное отношение}
\scniselement{ориентированное отношение} 
\scntext{первый домен}{кибернетическая система}
\scnsuperset{внешняя информационная среда кибернетической системы*}
\scnaddlevel{1} \scnidtf{совокупность всевозможных информационных, к которым данная кибернетическая система имеет доступ и которые представлены самым различным образом (в том числе, и в памяти тех кибернетических систем (субъектов), с которыми данная система взаимодействует)}
\scnaddlevel{-1}
\scnheader{\textit{среда кибернетической системы*}}\\
\scnidtf{быть средой существования (жизнедеятельности) заданной (указанной, соответствующей) кибернетической системы}
\scnnote{В общем случае среда жизнедеятельности кибернетической системы включает в себя (1) \textit{внешнюю среду*} этой системы, (2) \textit{физическую оболочку*} этой системы и (3) её абстрактную память, т.е. внутреннюю среду*, которая является хранилищем информационной модели всей среды} 
\scnsubdividing{внешняя среда*; физическая оболочка*; абстрактная память*}


\bigskip
\scnsegmentheader{Комплекс свойств, определяющий общий уровень качества кибернетической системы}

\scnstartsubstruct

\scnheader{качество кибернетической системы}
\scnidtf{интегральный уровень качества кибернетической системы в заданный момент}
\scnidtf{комплексная оценка (характеристика) уровня качества кибернетической системы}
\scntext{пояснение}{Для того, чтобы уточнить (детализировать) понятие \textit{качества кибернетической системы}, необходимо
	\begin{scnitemize}
		\item задать метрику \textit{качества кибернетических систем} и
		\item построить иерархическую систему свойств (параметров, признаков), определяющих \textit{качество кибернетической системы}.
	\end{scnitemize}
}
\scniselement{упорядоченное свойство}
\scnidtf{эволюционный уровень кибернетической системы}
\scnidtf{интегральная (комплексная) оценка уровня развития (совершенства) кибернетической системы}
\scntext{пояснение}{\textit{Качество кибернетической системы} -- это такое свойство (характеристика) \textit{кибернетических систем}, такой признак их классификации, который позволяет разместить эти системы по "ступенькам"{} некоторой условной "эволюционной лестницы"{}. На каждую такую "ступеньку"{} попадают \textit{кибернетические системы}, имеющие одинаковый уровень развития, каждому их которых соответствует свой набор значений дополнительных свойств \textit{кибернетических систем}, которые уточняют (детализируют, специализируют) соответствующий уровень развития \textit{кибернетических систем}. Такой эволюционный подход к рассмотрению \textit{кибернетических систем} даёт возможность, во-первых, детализировать направления эволюции \textit{кибернетических систем} и, во-вторых, уточнить то место этой эволюции, где и благодаря чему осуществляется переход от неинтеллектуальных \textit{кибернетических систем} к интеллектуальным. Фактически речь идёт об эволюционной теории качества \textit{кибернетических систем}, рассматривающей эволюцию \textit{кибернетических систем} как в рамках жизненного цикла каждой из них, так и в рамках эволюции целой "популяции"{} при переходе от одного поколения \textit{кибернетических систем} к другому поколению (в частности, от одного поколения \textit{компьютерных систем} к другому).\\
В основе эволюционного подхода к рассмотрению многообразия \textit{кибернетических систем} лежит положение о том, что идеальных \textit{кибернетических систем} не существует, но существует постоянное стремление к идеалу, к большему совершенству. При этом важно уточнить, что конкретно в каждой \textit{кибернетической системе} следует изменить, чтобы привести эту систему к более совершенному виду.\\
Эволюционный подход к рассмотрению \textit{кибернетических систем} имеет важное практическое значение для развития (совершенствования) каждой конкретной \textit{компьютерной системы} (искуственной \textit{кибернетической системы}), а также для развития \textit{технологий} разработки \textit{компьютерных систем}. Так, например, развитие технологий разработки \textit{компьютерных систем} должно быть направлено на переход к таким новым архитектурным и функциональным принципам, лежащим в основе \textit{компьютерных систем}, которые
\begin{scnitemize}
	\item обеспечивают существенное снижение трудоемкости их разработки и сокращение сроков разработки, а также
	\item обеспечивают существенное повышение уровня \textit{интеллекта} и, в частности, уровня \textit{обучаемости} разрабатываемых \textit{компьютерных систем}, например, путём перехода от поддержки обучения с учителем к реализации эффективного самообучения (к автоматизации организации самостоятельного обучения).
	\end{scnitemize}
}}
\scnnote{В эволюции \textit{кибернетических систем} (и, в частности, \textit{компьютерных систем}) можно выделить целый ряд этапов:
\begin{scnitemize}
	\item переход от стимульно-реактивного поведения к поведению, предполагающему учёт постоянно накапливаемого собственного опыта, означает переход от протопамяти, которая просто фиксирует связи между стимулами и соответствующими реакциями и которая не предполагает изменения этих связей, к \textit{памяти}, которая становится средой "существования"{} информации, отражающей  собственный опыт \textit{кибернетической системы} (а в перспективе и многое другое) и которая обеспечивает высокую степень \textit{гибкости} хранимой \textit{информации}, т.е. широкие возможности изменения (корректировки) этой \textit{информации} в процессе функционирования \textit{кибернетической системы}. Таким образом, \textit{память кибернетической системы} вместе с хранимой в ней \textit{информацией} становится управляемым самой этой \textit{кибернетической системой} гибким "коммутатором"{} между её стимулами и реакциями, учитывающим не только накапливаемый собственный опыт, но и контекст (дополнительные обстоятельства) выполняемых \textit{действий} (реакций), рассматривающий выполняемые \textit{действия} с самых разных аспектов;
	\item включение в состав \textit{информации, хранимой в памяти компьютерной системы}, \textit{программ}, описывающих различные \textit{методы} обработки этой \textit{информации} и интерпретируемых \textit{процессором} указанной \textit{компьютерной системы};
	\item переход от указанной выше "коммутационной"{} трактовки \textit{информации, хранимой в памяти кибернетической системы} к её трактовке как мощной и постоянно совершенствуемой информационной модели внешней среды, в которой существует указанная \textit{кибернетическая система}. Это означает
	\begin{scnitemizeii}
		\item переход \textit{информации, хранимой в памяти кибернетической системы} на уровень \textit{базы знаний}, которой ставится в \textit{соответствие} достаточно чёткая \textit{денотационная семантика}, и
		\item переход \textit{программ}, хранимых в \textit{памяти кибернетической системы}, на уровень \textit{программ}, которые ориентированы на обработку \textit{базы знаний} и которые сами являются частью обрабатываемой \textit{базы знаний};
	\end{scnitemizeii}
	\item существенное расширение \textit{семантической мощности баз знаний} и многообразия используемых \textit{моделей решения задач}, в том числе, моделей, способных работать в условиях неполноты (недостаточности), нечеткости и недостоверности обрабатываемых \textit{знаний}.
\end{scnitemize}
}
\scnnote{Повышение качества искусственных\textit{ кибернетических систем} (\textit{компьютерных систем}) потребует формирования таких свойств (характеристик, способностей) \textit{компьютерных систем}, которые аналогичны психическим свойствам людей. Таким образом, дальнейшее развитие \textit{Искусственного интеллекта} (теории и практики создания \textit{интеллектуальных компьютерных систем} -- интеллектуальных искусственных \textit{кибернетических систем}) настоятельно потребует обобщения современной психологии (психологии биологических индивидов и их коллективов -- \textit{психологии естественных кибернетических систем}) и создания \textit{общей психологии кибернетических систем} (как естественных, так и искусственных) основанной на высоком уровне формализации.
}

\scnendstruct


\scnheader{качество кибернетической системы}
\scnrelfromlist{cвойство-предпосылка}{качество физической оболочки кибернетической системы мы качество решателя задач кибернетической системы; 
качество решателя задач кибернетической системы
\newline
\scnaddlevel{1}
\scnrelfrom{cвойство-предпосылка}{качество информации, хранимой в памяти кибернетической системы }
\scnaddlevel{-1};
качество информации, хранимой в памяти кибернетической системы;
гибридность кибернетической системы
\scnaddlevel{1}
\scnidtf{степень многообразия (1) видов знаний, хранимых в памяти кибернетической системы, (2) используемых моделей решение задач, (3) видов сенсоров и эффекторов}
\scnrelfromlist{частное свойство}{многообразие видов знаний, хранимых в памяти кибернетической системы;
многообразие моделей решения задач;
многообразие видов сенсоров и эффекторов}
\scnaddlevel{-1};
приспособленность кибернетической системы к её совершенствованию;
производительность кибернетической системы
\scnaddlevel{1}
\scnidtf{cкорость решения задач кибернетической системы}
\scnaddlevel{-1};
надежность кибернетической системы;
социализация кибернетической системы
}

\scnheader{гибридность кибернетической системы}
\scnrelfromlist{частное свойство}{многообразие видов знаний, хранимых в памяти кибернетической системы;
многообразие моделей решения задач;
многообразие видов сенсоров и эффекторов}

\scnheader{гибридная кибернетическая система}
\scnidtf{кибернетическая система, использующая многообразие рецепторных и/или эффекторных подсистем, и/или многообразие видов обрабатываемой информации, и/или многообразие способов решения задач}
\scnsuperset{гибридная компьютерная система}
\scnaddlevel{1}
\scnidtf{\textit{компьютерная система}, способная решать \textit{комплексные задачи}, требующие использования многообразия различных видов обрабатываемой информации и различных \textit{моделей решения задач}}
\scnaddlevel{-1}

\scnheader{приспособленность кибернетической системы к её совершенствованию}
\scnidtf{приспособленность кибернетической системы к эволюции, к повышению уровня своего качества}
\scnrelfromset{комплекс свойств-предпосылок}{обучаемость кибернетической системы
\scnaddlevel{1}
\scnidtf{способность кибернетической системы самостоятельно повышать уровень своего качества}
\scnidtf{способность кибернетической системы к самоэволюции, саморазвитию, устранению своих недостатков}
\scnaddlevel{-1};
приспособленность кибернетической системы к её совершенствованию, осуществляемому извне
\scnaddlevel{1}
\scnidtf{приспособленность кибернетической системы к её совершенствованию, осуществляемому внешними субъектами}
\scnidtf{удобство совершенствования кибернетической системы для её создателей}
\scnnote{Важнейшим фактором качества каждой \textit{технологии разработки кибернетических систем} является гибкость и стратифицированность разрабатываемых кибернетических систем при их совершенствовании, осуществляемом руками разработчиков}
\scnaddlevel{-1}
}
\scnrelfromset{комплекс свойств-предпосылок}{гибкость кибернетической системы;
стратифицированность кибернетической системы
\scnaddlevel{1}
\scnidtf{уровень стратифицированности кибернетической системы}
\scnidtf{качество разделения (декомпозиции) кибернетической системы на в достаточной степени независимые части (компоненты), определенные виды изменений которых не предполагают внесения изменений в другие части системы}
\scnaddlevel{-1}
}

\scnheader{гибкость кибернетической системы}
\scnidtf{реконфигурируемость кибернетической системы}
\scnidtf{модифицируемость кибернетической системы}
\scnidtf{реформируемость кибернетической системы }
\scnidtf{трансформируемость кибернетической системы}
\scnidtf{пластичность кибернетической системы}
\scnidtf{легкость реализации различного вида изменений в кибернетической системе}
\scnidtf{степень трансформенности кибернетической системы}
\scnidtf{простота и многообразие внесения изменений в кибернетическую систему}
\scnidtf{модифицируемость кибернетической системы}
\scnidtf{трансформируемость кибернетической системы} 
\scnidtf{реконфигурируемость кибернетической системы} 
\scnidtf{приспособленность к реинжинирингу кибернетической системы}
\scnidtf{мягкость}
\scnidtf{softness}
\scnidtf{приспособленность к внесению изменений}
\scnidtf{\uline{легкость} внесения изменений}
\scnnote{Чем легче вносить изменения в кибернетическую систему, тем выше скорость ее эволюции}
\scnnote{изменения могут вноситься (1) полностью самостоятельно (без учителя) (2) с помощью учителя-тренера ("терапевта"{}) путем создания определенных условий для совершенствования системы (3) "хирургически"{} -- путем непосредственного вмешательства извне (например, вмешательства разработчика)}
\scnnote{Чем выше \textit{гибкость кибернетической системы} -- тем ниже трудоемкость и меньше сроки внесения различных изменений в систему в направлении ее совершенствования (приближения к идеалу)}
\scnidtf{простота внесения изменений в кибернетическую систему и многообразие видов возможных таких изменений}
\scnrelfromset{комплекс свойств-предпосылок}{простота внесения изменений в кибернетическую систему
\newline
\scnaddlevel{1}
\scnrelfrom{свойство-предпосылка}{стратифицированность кибернетической системы}
\scnaddlevel{-1};
многообразие, возможных изменений, вносимых в кибернетическую систему
}
\scnrelfromset{комплекс частных свойств}{гибкость информации, хранимой в памяти кибернетической системы;
гибкость решателя задач кибернетической системы;
гибкость физической оболочки кибернетической системы
\scnaddlevel{1}
\newline
\scnrelfrom{частное свойство}{гибкость памяти кибернетической системы}
\scnaddlevel{-1};
гибкость интерфейса кибернетической системы}
\scnrelfromset{комплекс частных свойств}{гибкость кибернетической системы при ее совершенствовании, осуществляемом извне;
гибкость возможных самоизменений кибернетической системы
\scnaddlevel{1}
\newline
\scnrelto{свойство-предпосылка}{обучаемость кибернетической системы}
\scnaddlevel{-1}}

\scnheader{приспособленность кибернетической системы к её совершенствованию, осуществляемому извне}
\scnidtf{приспособленность кибернетическиой системы к "хирургическим"{} методам её совершенствования, реализуемым разработчиками}
\scnidtf{насколько легко осуществлять обновление, перепроектирование, тестирование, ремонт (исправление ошибок) кибернетической системы}
\scnrelfromlist{свойство-предпосылка}{простота внесения изменений в кибернетическую систему, реализуемых извне
\newline
\scnaddlevel{1}
\scnrelfrom{свойство-предпосылка}{стратифицированность кибернетической системы}
\scnaddlevel{-1};
многообразие возможных изменений кибернетической системы, реализуемых извне}

\scnheader{производительность кибернетической системы}
\scnidtf{быстродействие кибернетической системы}
\scnidtf{интегральная оценка скорости решения задач, время реакции кибернетической системы на задачные ситуации}
\scnrelfromlist{частное свойство}{производительность базового интерпретатора логико-семантической модели компьютерной системы;
качество используемых кибернетической системой методов и моделей решения задач}

\scnheader{надежность кибернетической системы}
\scnidtf{способность кибернетической системы при соответствующих условиях ее функционирования сохранять (и, точнее, не снижать) уровень всех свойств и способностей, определяющих общее (комплексное) качество кибернетической системы}
\scnrelfromlist{свойство-предпосылка}{безотказность кибернетической системы; долговечность кибернетической системы; "ремонтопригодность"{} кибернетической системы
\newline
\scnaddlevel{1}
\scnrelfrom{основной sc-идентификатор}{"ремонтопригодность"{} кибернетических систем}
\scnaddlevel{1}
\scnnote{Здесь слово "ремонтопригодность"{} взято в кавычки, т.к. речь идет не только об искусственных (технических) кибернетических системах}
\scnaddlevel{-1}
\scnaddlevel{-1}
}


\input{Contents/chapter1/intro_hs/segment3-section1.1}


\bigskip
\scnsegmentheader{Комплекс свойств, определяющих уровень интеллекта кибернетической системы}
\scnstartsubstruct

\scnheader{интеллект}
\scniselement{свойство}
\scniselement{упорядоченное свойство}
\scnidtf{уровень (степень, величина) интеллекта кибернетической системы}
\scnidtf{Семейство классов \textit{кибернетических систем}, обладающих эквивалентным (одинаковым) уровнем интеллекта -- от низкого до высокого уровня интеллекта}
\scnidtf{свойство кибернетических систем, характеризующее эффективность их взаимодействия со своей средой (средой их "жизнедеятельности"{})}
\scnrelfrom{область определения}{кибернетическая система}
\scnexplanation{С формальной точки зрения интеллектуальность -- это семейство классов кибернетических систем, в каждый из которых входят кибернетические системы, эквивалентные по уровню и характеру проявления интеллектуальных свойств (в том числе способностей).\\
Таким образом, характер (вид) интеллектуальных свойств кибернетических систем и уровень их развития для разных кибернетических систем может быть разным. В соответствии с этим кибернетические системы можно сравнивать между собой.}
\scnnote{Основным свойством (характеристикой, качеством, параметром) кибернетической системы является уровень (степень) ее интеллекта, который является \uline{интегральной} характеристикой, определяющей уровень эффективности взаимодействия кибернетической системы со средой своего существования.}
\scnidtf{комплексное свойство (качество) кибернетической системы, определяющее уровень ее "выживаемости"{} во внешней среде и предполагающее возможность воздействия на эту среду и даже возможность ее преобразования}
\scnidtf{интеллектуальный потенциал кибернетической системы}
\scnidtf{спектр знаний, навыков и способностей к обучению кибернетической системы}
\scnidtf{интеллектуальность кибернетической системы}
\scnnote{Процесс эволюции \textit{кибернетических систем} следует рассматривать как процесс повышения уровня их качества по целому ряду свойств (характеристик) и, в первую очередь, как процесс повышения уровня их \textit{интеллекта}. При этом можно говорить об эволюции каждой конкретной \textit{кибернетической системы} в процессе своей "жизнедеятельности"{}, а также об эволюции целого класса \textit{кибернетических систем}, когда новые экземпляры этого класса являются более интеллектуальными, чем их предшественники. В таком аспекте, в частности, можно рассматривать эволюцию \textit{компьютерных систем} (искусственных кибернетических систем).}
\scnnote{Очень важно уточнить, какими иными свойствами \textit{кибернетических систем} определяется уровень и характер их интеллектуальности. Подчеркнем, что \uline{любая} \textit{кибернетическая система} обладает соответствующим уровнем интеллектуальности. Пусть даже и достаточно низким. Существенным является уточнение того, за счет чего уровень интеллектуальности \textit{кибернетической системы} может быть повышен. Нет смысла проводить четкую границу между \textit{интеллектуальными кибернетическими системами} и неинтеллектуальными. Но есть смысл уточнять направления повышения уровня интеллектуальности \textit{кибернетических систем.}}
\scntext{эпиграф}{Никто не может провести линию, отделяющую атмосферу от космоса, или черту, за которой начинается жизнь, или границу электронного облака. Все дело в степени проявления свойства.}
	\scnaddlevel{1}
	\scnrelfrom{автор}{Барт Коско}
	\scnaddlevel{-1}
\scnnote{Прежде, чем говорить о требованиях, предъявляемых к \textit{технологии проектирования и производства интеллектуальных компьютерных систем (искусственных кибернетических систем}, обладающих высоким уровнем \textit{интеллекта)}, необходимо уточнить (детализировать) \textit{свойства}, присущие указанным системам и являющиеся предпосылками, обеспечивающими высокий уровень \textit{интеллекта}. Подчеркнем, что указанные \textit{свойства}, уточняющие (детализирующие, обеспечивающие, определяющие) \textit{свойства} \scnbigspace \textit{интеллектуальных систем}\scnbigspace (\textit{свойства}, определяющие уровень \textit{интеллекта} этих систем) должны быть общими как для искусственных кибернетических систем (\textit{компьютерных систем}), так и для \textit{естественных кибернетических систем.}}
\scnidtf{интегральное качество информационного обеспечения и информационных процессов в кибернетической системе}
\scnidtf{интегральное качество кибернетической системы, определяемое:
	\begin{scnitemize}
		\item уровнем ее образованности -- качеством накопленных к заданному моменту знаний и умений (навыков);
		\item уровнем ее обучаемости -- способностью \uline{самостоятельно} повышать уровень свой образованности.
	\end{scnitemize}}
\scnrelfromlist{свойство-предпосылка}{образованность кибернетической системы
;обучаемость кибернетической системы
;социализация кибернетической системы\\
	\scnaddlevel{1}
	\scnnote{интеллект \textit{кибернетической системы}, как и лежащий в его основе познавательный процесс, выполняемый кибернетической системой, имеет социальный характер, поскольку наиболее эффективно формируется и развивается в форме взаимодействия \textit{кибернетической} системы с другими \textit{кибернетическими системами}}
	\scnaddlevel{-1}}

\scnheader{образованность кибернетической системы}
\scnidtf{уровень навыков (умений), а также иных знаний, приобретенных \textit{кибернетической системой} к заданному моменту}
\scnrelfromlist{свойство-предпосылка}{\textbf{качество навыков, приобретенных кибернетической системой}
	\scnaddlevel{1}
	\scnidtf{качество умений, которыми владеет кибернетическая система в текущий момент}
	\scnrelfrom{свойство-предпосылка}{\textbf{качество информации, хранимой в памяти кибернетической системы}}
		\scnaddlevel{1}
	\scnidtf{качество знаний, приобретенных кибернетической системой к заданному моменту}
		\scnaddlevel{-1}
	\scnaddlevel{-1}
;\textbf{качество информации, хранимой в памяти кибернетической системы}\\
	\scnaddlevel{1}
	\scnnote{Следует обратить внимание на то, что \textit{качество информации, хранимой в памяти кибернетической системы}, является фактором, обеспечивающим не только \textit{качество навыков, приобретенных кибернетической системой}, но и общий \textit{уровень качества кибернетической системы}.}
	\scnaddlevel{-1}}

\scnheader{кибернетическая система}
\scnrelto{объединение}{Признак интеллектуальности кибернетических систем}
	\scnaddlevel{1}
	\scneqtoset{неинтеллектуальная кибернетическая система
;интеллектуальная система\\	
		\scnaddlevel{1}
		\scnidtf{интеллектуальная кибернетическая система}
		 \scnreltoset{объединение}{слабоинтеллектуальная система\\
		 	\scnaddlevel{1}
		 	\scnidtf{кибернетическая система со слабым интеллектом}
		 	\scnidtf{кибернетическая система с низким уровнем интеллекта}
		 	\scnidtf{кибернетическая система с элементами интеллекта}
		 	\scnaddlevel{-1}
;высокоинтеллектуальная система\\
		 	\scnaddlevel{1}
		 	\scnidtf{идеальная интеллектуальная система}
		 	\scnidtf{кибернетическая система с сильным интеллектом}
		 	\scnidtf{кибернетическая система с высоким уровнем интеллекта}
		 	\scnidtf{действительно интеллектуальная система}
		 	\scnaddlevel{-1}}
		\scnaddlevel{-1}}
	\scnaddlevel{-1}	

\scnheader{Признак интеллектуальности кибернетических систем}
\scnnote{Данный признак классификации кибернетических систем формально является не разбиением, а покрытием множества \textit{кибернетических систем}, так как отсутствует четкая грань между неинтеллектуальными и интеллектуальными кибернетическими системами, а также между слабоинтеллектуальными и высокоинтеллектуальными кибернетическими системами.}


\scnheader{интеллектуальная система}
\scnidtf{интеллектуальная кибернетическая система}
	\scnaddlevel{1}
		\scnnote{В этом термине слово "кибернетическая"{} можно опустить, так как интеллектуальными могут быть только \textit{кибернетические системы}}
	\scnaddlevel{-1}
\scnnote{Интеллектуальные кибернетические системы могут быть \textit{естественными интеллектуальными системами}, искусственными интеллектуальными системами (которые будем называть \textit{интеллектуальными компьютерными системами}), а также естественно-искусственными интеллектуальными системами, состоящими из компонентов как естественного, так и искусственного происхождения. Важнейшим примером естественно-искусственных интеллектуальных систем являются человеко-машинные системы, представляющие собой коллективы (многоагентные системы), состоящие из \textit{интеллектуальных компьютерных систем} и людей (конечных пользователей и разработчиков этих компьютерных систем).}
\scnnote{Вводя понятие \textit{интеллектуальной системы}, важно, во-первых, уточнить понятие \textit{кибернетической системы} и определить те свойства, которые присущи \uline{всем} кибернетическим системам, и, во-вторых, локализовать ту условную \uline{грань} перехода от неинтеллектуальных \textit{кибернетических систем} к интеллектуальным, а также \uline{грань} перехода от слабоинтеллектуальных к высокоинтеллектуальным кибернетическим системам. В этом и заключается уточнение феномена \textit{интеллекта} (интеллектуальности) кибернетических систем.}
\scnnote{Все \textit{свойства} (в том числе способности и активности), присущие \textit{кибернетическим системам}, в различных \textit{кибернетических системах} могут иметь самый различный уровень (уровень развития). Более того, в некоторых \textit{кибернетических системах} некоторые из этих свойств могут вообще отсутствовать. При этом в кибернетических системах, которые условно будем называть \textit{\textbf{интеллектальными системами}}, \uline{все} указанные выше свойства должны быть представлены в достаточно развитом виде. Заметим также, что мы называем \textit{интеллектуальными системами}, иногда называют кибернетическими системами с сильным интеллектом (с высоким уровнем интеллекта), противопоставляя их кибернетическим системам со слабым интеллектом (с низким уровнем интеллекта).}
\scnsubset{образованная кибернетическая система}
	\scnaddlevel{1}
	\scnidtf{кибернетическая система, имеющая высокий уровень образованности}
	\scnidtf{кибернетическая система, обладающая высоким уровнем знаний и навыков}
	\scnsubset{кибернетическая система, основанная на знаниях}
	\scnsubset{кибернетическая система, управляемая знаниями}
	\scnsubset{целенаправленная кибернетическая система}
	\scnsubset{гибридная кибернетическая система}
	\scnsubset{потенциально универсальная кибернетическая система}
	\scnaddlevel{-1}
\scnsubset{обучаемая кибернетическая система}
	\scnaddlevel{1}
	\scnidtf{когнитивная кибернетическая система}
	\scnidtf{кибернетическая система, имеющая высокий уровень обучаемости}
	\scnsubset{кибернетическая система с высоким уровнем стратифицированности своих знаний и навыков}
	\scnsubset{рефлексивная кибернетическая система}
	\scnsubset{самообучаемая кибернетическая система}
	\scnsubset{кибернетическая система с высоким уровнем познавательной активности}
	\scnaddlevel{-1}
\scnsubset{социально ориентированная кибернетическая система}
 	\scnaddlevel{1}
 	\scnidtf{кибернетическая система, имеющая высокий уровень социализации}
 	\scnsubset{кибернетическая система, способная устанавливать и поддерживать высокий уровень семантической совместимости и взаимопонимания с другими системами}
 	\scnsubset{договороспособная кибернетическая система}
 		\scnaddlevel{1}
 		\scnidtf{кибернетическая система, способная координировать (согласовывать) свою деятельность с другими системами}
 		\scnaddlevel{-1}
 	\scnaddlevel{-1}
 	

\scnheader{кибернетическая система, основанная на знаниях}
\scnidtf{кибернетическая система, в основе которой лежит формируемая в ее памяти, постоянно совершенствуемая и структурированная информационная модель той среды, в рамках которой она существует и решает соответствующие задачи}
\scnidtf{кибернетическая система, в основе которой лежит ее база знаний -- систематизированная совокупность всех используемых ею знаний}
\scnidtf{кибернетическая система, формирующая в своей памяти систематизированную информационную модель среды своего обитания и использующая эту модель для организации своего целенаправленного поведения}


\scnheader{кибернетическая система, управляемая знаниями}
\scnidtf{кибернетическая система, в которой выполняемые ею действия инициируются соответствующими ситуациями и/или событиями, возникающими в ее базе знаний}

\scnheader{целенаправленная кибернетическая система}
\scnidtf{субъект, "осознанно"{} и целенаправленно осуществляющий свою деятельность, "ведающий"{} то, что он творит}

\scnheader{обучаемая кибернетическая система}
\scnidtf{когнитивная система}
\scnidtf{кибернетическая система, способная познавать (изучать) среду своего обитания, то есть строить и постоянно уточнять в своей памяти информационную модель (описание) этой среды, а также использовать эту модель для решения различных задач (для организации своей деятельности (поведения)) в указанной среде}
\scnidtf{кибернетическая система, способная к самосовершенствованию}


\scnheader{социально ориентированная кибернетическая система}
\scnidtf{кибернетическая система, имеющая достаточно высокий уровень интеллекта, чтобы быть полезным членом различных, в том числе, и человеко-машинных сообществ}
\scnnote{Определенный уровень социально значимых качеств является необходимым условием интеллектуальности кибернетической системы. Это, своего рода, модификация теста Тьюринга -- важна не имитация, не иллюзия "человекоподобия"{}, а \uline{реальная} польза в процессе коллективного решения сложных задач.}


\scnheader{интеллектуальная компьютерная система}
\scnidtf{искусственная интеллектуальная система}
\scnidtf{искусственная кибернетическая система, обладающая высоким уровнем интеллекта (высоким уровнем знаний и умений), а также высоким уровнем обучаемости}
\scnsubset{компьютерная система}
	\scnaddlevel{1}
	\scnsubset{кибернетическая система}
	\scnaddlevel{-1}
\scnauthorcomment{не нашла нужный макрос}
\scnidtfexp{(основной sc-идентификатор) интеллектуальная компьютерная система}
	\scnaddlevel{1}
	\scntext{сокращение}{и.к.с.}
	\scnaddlevel{-1}
\scnidtf{система искусственного интеллекта}
\scnidtf{искусственная интеллектуальная система}
\scnsubset{интеллектуальная система}
	\scnaddlevel{1}
	\scnsubset{кибернетическая система}
	\scnaddlevel{-1}


\bigskip
\scnsegmentheader{Комплекс свойств, определяющих качество информации, хранимой в памяти кибернетической системы}

\scnstartsubstruct

\scnheader{информация}
\scnidtf{информационная конструкция}
\scnidtf{информационная модель, состоящая из некоторого множества различных \textit{знаков}, обозначающих моделируемые (описываемые) \textit{сущности} любого вида и, в частности, \textit{знаков}, обозначающих различного вида \textit{связи} между \textit{знаками} описываемых \textit{сущностей} (такие \textit{связи} чаще всего являются отражениями (моделями) \textit{связей} между \textit{сущностями}, которые обозначаются связываемыми \textit{знаками})}
\scnaddlevel{1}
\scnnote{Подчеркнем, что \textit{связи} между \textit{знаками} описываемых \textit{сущностей} сами также могут быть описываемыми \textit{сущностями}, но для этого указанные \textit{связи} в рамках информационной модели должны быть представлены своими \textit{знаками}. Не все \textit{связи} между \textit{знаками} являются описываемыми \textit{сущностями}. Такими неописываемыми связями являются связи инцидентности знаков.} 
\scnaddlevel{-1}
\scnidtf{конфигурация знаков}
\scnidtf{знаковая конструкция} 
\scnidtf{текст}
\scnidtf{описание (отражение) некоторого множества (1) первичных сущностей, (2) понятий, (3) связей между ними, (4) связей между связями, (5) фрагментов данного описания, (6) связей между этими фрагментами}
\scnsuperset{дискретная информационная конструкция} 
\scnaddlevel{1}
\scnidtf{информационная конструкция, у которой все входящие в неё знаки имеют чёткие границы}

\scnsuperset{дискретная информационная конструкция, у которой входящие в неё знаки имеют \uline{условную} структуру}
\scnaddlevel{-1}  

\scnsubdividing {внутренняя информационная конструкция
\scnaddlevel{1}
\scnidtf{информационная конструкция, хранимая в памяти некоторой кибернетической \textit{системы}, и непосредственно интерпретируемая (понимаемая) решателем задач этой системы}
\scnaddlevel{-1}
;внешняя информационная конструкция   
\scnaddlevel{1}
\scnidtf{информационная конструкция, представленная на каком-либо внешнем носителе или в памяти другой кибернетической системы} 
\scnaddlevel{-1}
;файл   
\scnaddlevel{1}
\scnidtf{первичный электронный образ некоторой внешней информационной конструкции} 
\scnaddlevel{-1}}
\scnexplanation {Начало раздела \ref{intro_lang}}   
\scnidtf{информационная модель} 
\scnidtf{информационная модель (отражение, описание) некоторого множества связей между некоторым описываемыми (рассматриваемыми, исследуемыми, изучаемыми) сущностями}
\scntext{определение}{Множество всевозможных информационных конструкций (понятие информационной конструкции) представляет собой множество, на котором задано 
	\begin{scnitemize}
	\item Отношение \uline{синтаксической} эквивалентности и, соответственно, семейство классов синтаксической эквивалентности информационных конструкций
	\item Отношение \uline{семантической} эквивалентности и, ответственно, семейство классов семантической эквивалентности информационных конструкций 
	\item Отношение \uline{логической} эквивалентности и, соответственно, семейство классов логической эквивалентности информационных конструкций.
\end{scnitemize} 
При этом можно говорить об инварианте каждого класса синтаксически эквивалентных информационных конструкций, об инварианте каждого класса семантически эквивалентных информационных конструкций и об инварианте каждого класса логически эквивалентных информационных конструкций 
синтаксически эквивалентные информационные конструкции могут отличаться вариантами изображения букв (различным почерком, разными шрифтами), вариантами "разрезания"{} текста на страницы и на строчки.
Семантически эквивалентные информационные конструкции могут отличаться разными именами, обозначающими одни и те же сущности, разным порядком размещения этих имён.}
 
\scnheader{денотационная семантика информационной конструкции}\\
\scnexplanation{Каждая информационная конструкция имеет денотационную семантику, описывающую то, как связаны входящие в информационную конструкцию знаки с соответствующими им денотатами (т.е. сущностями, обозначаемыми этими знаками}

\scnheader{сенсорная информация}\\
\scnsubset{информация} \\
\scnidtf{первичная информация, приобретаемая кибернетической системы с помощью её сенсоров (рецепторов)}
\scnidtf{первичная информация}
\scnnote{Подчеркнем, что \textit{сенсорная информация} \scnbigspace \textit{кибернетической системы} с точки зрения её \textit{денотационной семантики} является простейшим видом \textit{знаковой конструкции}, в которой \textit{внешняя среда} \scnbigspace \textit{кибернетической системы} описывается
\begin{scnitemize}
	\item путём задания параметрического пространства (множество параметров, признаков, \textit{свойства}, характеристик), с помощью которого описываются состояние элементарных (атомарных) фрагментов \textit{внешней среды}, которые непосредственно являются смежными (соприкасаются с) чувствительными поверхностями \textit{сенсоров кибернетической системы}; 
	\item путём пространственной декомпозиции наблюдаемой \textit{внешней среды} с выделением указанных выше элементарных фрагментов этой среды (элементарных с "точки зрения"{} \textit{сенсоров кибернетической системы}) и с явным описанием пространственных связей между указанными элементарными фрагментами (эти связи соответствует пространственным связям между сенсорами);
	\item путём темпоральной декомпозиции наблюдаемой \textit{внешней среды}, которая предполагает фиксацию моментов времени для каждого события по изменению состояния измеряемого параметра каждого элементарного фрагмента наблюдаемой \textit{внешней среды} 
\end{scnitemize}}

\scnnote{Качество (в частности, информативность) \textit{сенсорной информации} обеспечивается:
\begin{scnitemize}
\item качеством используемого параметрического пространства 
\begin{scnitemizeii}
\item многообразием видов \textit{сенсоров}, т.е. многообразием параметров (свойств), с помощью которых описывается внешняя среда
\item информативностью каждого из указанных параметров 
\item целостностью (полнотой, достаточностью) всего набора рассматриваемых параметров 
\item отсутствием избыточности в наборе этих параметров 
\end{scnitemizeii}
\item общим количеством сенсоров и количеством сенсоров, соответствующих каждому параметру
\item способностью кибернетической системы перемещать сенсоры в пространстве 
\end{scnitemize}
}
\scnnote{\textit{сенсорная информация} обеспечивает формирование первичного описания состояния и динамики изменения не только \textit{внешней среды кибернетической системы}, но также и её физической оболочки, которую можно рассматривать как часть всей \textbf{\textit{физической среды кибернетической системы}}, противопоставляя такую \textit{физическую среду кибернетической системы} её внутренней (информационной, \uline{абстрактной}) среде, в которой хранится и обрабатывается \textit{информация}, используемая \textit{кибернетической системой}. Указанную абстрактную внутреннюю среду кибернетической системы будем называть \textbf{\textit{абстрактной памятью кибернетической системы}}.}

\scnheader{язык}  
\scnidtf{множество информационных конструкции, построенных по общим синтаксическим и семантическим правилам}
\scnsuperset{внутренний язык кибернетической системы} 
\scnaddlevel{1}
\scnidtf{язык, используемый кибернетической системой для представления информации, хранимой в её памяти}
\scnaddlevel{-1}

\scnheader{информация, хранимая в памяти кибернетической системы} 
\scnidtf{совокупность \uline{всей} информации, хранимой в памяти кибернетической системы}
\scnsubset{информация} 

\scnheader{качество информации, хранимой в памяти кибернетической системы} 
\scnidtf{качество знаний, приобретенных кибернетической системой к текущему моменту}
\scnidtf{уровень качества хранимой информации} 
\scnidtf{качество информационной модели среды кибернетической системы, хранимой в её памяти}
\scnidtf{уровень качества хранимых в памяти кибернетической системы внутренней информационной модели среды существования (жизнедеятельности) этой кибернетической системы} 
\scnidtf{интегральное качество знаний, накопленных кибернетической системой к текущему моменту} 
\scnidtf{степень приближения информации, хранимой в памяти кибернетической системы к качественной информационной модели той среды, в которой существует кибернетическая система, к систематизированной базе знаний, описывающей все свойства этой среды, необходимые для функционирования этой кибернетической системы}
\scnidtf{качество хранимой в памяти кибернетической системы информационной модели среды жизнедеятельности этой системы}
\scnnote{Качество информационной модели среды "обитания"{} кибернетической системы, в частности, определяется 
\begin{scnitemize}
	\item корректностью (адекватностью) этой модели, 
	\item полнотой -- достаточностью находящейся в ней информации для эффективного функционирования кибернетической системы;
	\item структурированностью, систематизированностью.
\end{scnitemize}
Важнейшим этапом эволюции информационной модели среды кибернетической системы является переход от недостаточно полной и несистематизированной информационные модели среды к \textit{базе знаний}. Именно поэтому важнейшим этапом повышения уровня интеллектуальности компьютерной систем является переход от традиционных компьютерных систем к компьютерным системам, основанным на знаниях.}
\scnrelfrom{комплекс свойств-предпосылок}{не-фактор} 
\scnrelfromlist{свойство-предпосылка}{семантическая мощность языка представления информации в памяти кибернетической системы; 
объём информации, в память кибернетической системы; 
степень конвергенции и интеграции различного вида знаний, хранимых в памяти кибернетической системы; 
стратифицированность информации, хранимой в памяти кибернетической системы;
простота и локальность выполнения семантически целостных операций над информацией, хранимой в памяти кибернетической системы}


\bigskip
\scnfragmentcaption

\scnheader{степень конвергенции и интеграции различного вида знаний, хранимых в памяти кибернетической системы} 
\scnidtf{уровень "бесшовной"{} интеграции различного вида знаний кибернетической системы}
\scnnote{Максимальный уровень конвергенции и интеграции знаний (в том числе,  и знаний различного вида) предполагает:
\begin{scnitemize} 
	\item использование универсального базового языка, по отношению к которому всем используемым видам знаний соответствуют специализированные языки, являющиеся подъязыками указанного базового языка
	\item построение четкой иерархии указанных специализированных языков по принципу "язык-подъязык"{}
	\item явное введение семейства отношений, заданных на множестве различных знаний и, в том числе, связывающих знания различного вида
\end{scnitemize}
}
\scnrelfromlist{свойства-предпосылка}{уровень формализованности информации, хранимой в памяти кибернетической системы}
\scnheader{уровень формализованности информации, хранимой в памяти кибернетической системы}
\scnidtf{степень приближения информации, хранимой в памяти кибернетической системы, к максимально простой и компактной форме представление информационной модели некоторого множества описываемых сущностей, которая является отражением определенной конфигурации связей между указанными сущностями}
\scnnote{Высшим уровнем формализации информации, хранимой в памяти кибернетической системы, является смысловое представление информации в форме семантических сетей. Смотрите раздел Предметная область и онтология семантических сетей, семантических языков и семантических моделей баз знаний}
\scnrelboth{следует отличать}{формализация*}
	\scnaddlevel{1}
		\scnidtf{Бинарное ориентированное отношение, каждая пара которого связывает некоторую информационную конструкцию с другой информационной конструкцией, которая семантически эквивалентна первой, но имеет более высокий уровень формализованности}
		\scnnote{Приобретение навыков формального представления информации не является простой проблемой даже для человека. По сути совокупность таких навыков -- это основа математической культуры, культуры точного изложения своих соображений. Некоторые примеры, иллюстрирующие нетривиальность проблемы смотрите в Арнольд В. И. 2012кн-ЧтоТМ-стр. 75-76}
	\scnaddlevel{-1}
\scnrelboth{следует отличать}{формализация}
	\scnaddlevel{1}
		\scnidtf{деятельность, направленная на повышение уровня формализованности представление информации}
		\scntext{метафора}{сближение синтаксиса с семантикой -- сближение синтаксической структуры информационной конструкции с её смысловой структурой}
	\scnaddlevel{-1}
\scnidtf{уровень способности кибернетической системы к формальному представлению знаний и используемых понятий, к рационализации идей}
\scnidtf{степень близости языка внутреннего представления (способа внутреннего кодирования) информации в памяти кибернетической системы к смысловому представлению информации}
\scnidtf{степень близости к изоморфизму соответствие между: (1) синтаксической структурой внутреннего представления информации в памяти кибернетической системы и (2) конфигурацией связей описываемых сущностей}
\scnrelfromlist{свойства-предпосылка}{многообразие форм дублирования информации, хранимой в памяти кибернетической системы
;относительный объём дублирования информации, хранимой в памяти кибернетической системы 
;многообразие фрагментов хранимой информации, не являющихся ни знаками, ни конфигурациями знаков 
;компактность представления представление информации, хранимой в памяти кибернетической системы}
\scnheader{смысловое представление информации}
\scnidtfexp{способ представления информации, в котором минимизируются "чисто синтаксические"{} аспекты представления информационных конструкций, не имеющие непосредственной семантической интерпретации
\scnnote{
Примерами "чисто синтаксических"{} аспектов представления информационных конструкций являются:
\begin{scnitemize}
	\item буквы, которые входят в состав слов и которые, следовательно, не являются знаками описываемых сущностей;
	\item алфавиты букв различны знаков;
	\item знаки препинания (разделители и ограничители);
	\item инцидентность (порядок, последовательность) букв и других символов, входящих в состав информационной конструкции
\end{scnitemize}
}
	\scnaddlevel{1}
		\scntext{следовательно}{Информационная конструкция, представленная на каком-либо привычном для нас языке, является достаточно громоздкой информационной конструкцией, смысл которой (т.е знаки описываемых сущностей и семантически интерпретируемых связи между знаками, отражающие соответствующие связи между обозначаемыми сущностями) сильно закамуфлирован. Это существенно усложняет обработку информации. если пытаться реализовывать “осмысленные” модели решения задач, для которых “смысловые” аспекты обрабатываемой информации являются ключевыми.}
	\scnaddlevel{-1}
}
\scnheader{смысловое представление информации}
\scnnote{Существенно подчеркнуть, что приближение внутреннего представления информации в памяти кибернетической системы к смысловому представлению информации является важнейшим фактором упрощения решателя задач кибернетической системы при реализации сложных моделей решения задач, требующих глубокого анализа смысла обрабатываемой информации. А это, в свою очередь, является важнейшим фактором качества решателя задач кибернетической системы.}
\scnheader{многообразие форм дублирования информации, хранимой в памяти кибернетической системы}
\scnidtf{многообразие видов семантической эквивалентности фрагментов информации, хранимой в памяти кибернетической системы}
\scnnote{Простейшим видом семантической эквивалентности является синонимия знаков, когда два разных фрагмента хранимой информации являются знаками, имеющими один и тот же денотат (т. е обозначающими одну и ту же сущность)}
\scnheader{относительный объем дублирования информации, хранимой в памяти кибернетической системы}
\scnidtf{частота присутствия в хранимой информации семантически эквивалентных информационных фрагментов и, в частности, синонимичных знаков}
\scnheader{многобразие фрагментов хранимой информации, не являющихся ни знаками, ни конфигурациями знаков}
\scnnote{Примерами фрагментов хранимой информации, не являющихся знаками или конфигурациями знаков являются:
\begin{scnitemize}
	\item буквы, входящие в состав слов
	\item слова, входящие в состав словосочетаний
	\item различного вида разделители, знаки препинания
	\item различного вида ограничители.
\end{scnitemize}
}
\scnheader{компактность представления информации, хранимой в памяти кибернетической системы}
\scnnote{Должно уменьшиться число элементов памяти, используемых для представления информации т.е. необходим переход к более компактным, но семантически эквивалентным информационным конструкциям}
\scnheader{стратифицированность информации,хранимой в памяти кибернетической системы}
\scnrelfrom{свойство-предпосылка}{структурированность информации, хранимой в памяти кибернетической системы}
\scnidtf{способность кибернетической системы выделять такие разделы информации, хранимой в памяти этой системы, которые бы ограничивали области действия агентов решателя задач кибернетической системы, являющиеся достаточными для решения заданных задач}
	\scnaddlevel{1}
		\scnnote{Существует правило, позволяющее каждой заданной задачи поставить в соответствие априори известный (выделенный) раздел хранимой информации, являющийся областью действия агентов решателя осуществляющих решение заданной задачи. Основными видами такого рода разделов хранимой информации являются \textit{предметные области} и \textit{онтологии}.}
	\scnaddlevel{-1}
\scnrelfrom{свойство-предпосылка}{рефлексивность информации, хранимой в памяти кибернетической системы}
\scnidtf{уровень систематизации знаний, хранимых в памяти кибернетической системы}
\scnidtf{уровень перехода от неструктурированных или слабоструктурированных данных к хорошо структурированным базам знаний}
\scnidtf{уровень перехода от первичной информации к метаинформации, метаметаинформации и т.д.}
\scnheader{рефлексивность информации,хранимой в памяти кибернетической системы}
\scnidtf{уровень применения средств самоописания (метаязыковых средств) в информации, хранимой в памяти кибернетической системы}
\scnidtf{относительный, объём и многообразие метаинформации, хранимой в памяти кибернетической системы}
\scnnote{рефлексивность информации, хранимой в памяти кибернетической системы, т.е. наличие метаязыковых средств является фактором, обеспечивающим не только структуризацию хранимой информации, но возможность описания синтаксиса и семантики самых различных языков, используемых кибернетической системой.}
\scnheader{простота и локальность выполнения семантически целостных операций над информацией, хранимой в памяти кибернетической системы}
\scnnote{Данное свойство касается не самой информации, хранимой в памяти, а язык кодирования (представления) информации в памяти кибернетической системы}
\scnidtf{гибкость выполнения семантически целостных операций над информацией, хранимой в памяти кибернетической системы}
\scnheader{база знаний}
\scnidtf{база знаний кибернетической системы}
\scnsubset{информация, хранимая в памяти кибернетической системы}
\scnidtfexp{информация, хранимая в памяти кибернетической системы и имеющая высокий уровень качества по всем показателям и, в частности, высокий уровень:
\begin{scnitemize}
	\item \textit{семантической мощности языка представления информации хранимой в памяти кибернетической системы} (в базе знаний указанный язык должен быть универсальным);
	\ite \textit{гибридности информации, хранимой в памяти кибернетической системы};
	\item \textit{многообразия видов знаний, хранимых в памяти кибернетической системы};
	\item формализованности информации, хранимой в памяти кибернетической системы;
	\item \textit{структурированности информации, хранимой в памяти кибернетической системы}
\end{scnitemize}
}
\scnnote{Переход информации, хранимой в памяти кибернетической системы на уровене качества, соответствующий базам знаний, является важнейшим этапом эволюции кибернетических систем. Подчеркнем при этом, что базы знаний по уровню своего качества могут сильно отличаться друг от друга}

\scnendstruct \scninlinesourcecommentpar{Завершили Сегмент 5 Начала Раздела \ref{intro_idtf}}




\bigskip

\scnsegmentheader{Комплекс свойств, определяющих качество решателя задач кибернетической системы}
\scnstartsubstruct

\scnheader{качество решателя задач кибернетической системы}
\scnidtf{интегральная качественная оценка множества задач (действий), которые кибернетическая система способна выполнять в заданный момент}
\scnidtf{качество навыков, приобретенных кибернетической системой}
\scnnote{Основным свойством и назначением \textit{решателя задач кибернетической системы} является способность решать \textit{задачи} на основе накапливаемых (приобретаемых) \textit{кибернетической системой} различного вида \textit{навыков} с использованием \textit{процессора кибернетической системы}, являющегося универсальным интерпретатором всевозможных накопленных \textit{навыков}. При этом качество (уровень развития, уровень совершенства) указанной способности определяется целым рядом дополнительных факторов (свойств).}
\scnidtf{интеллектуальный уровень качества решателя задач кибернетической системы}
\scnidtf{интегральное качество умений (навыков), приобретенных \textit{кибернетической системой} к текущему моменту}
\scnrelfromlist{свойство-предпосылка}{общая характеристика решателя задач кибернетической системы;качество логико-семантической организации памяти кибернетической системы;качество решения интерфейсных задач в кибернетической системе}

\scnheader{общая характеристика решателя задач кибернетической системы}
\scnrelfromlist{свойство-предпосылка}{общий объем задач, решаемых кибернетической системой;многообразие видов задач, решаемых кибернетической системой;способность кибернетической системы к анализу решаемых задач;способность кибернетической системы к решению задач, методы решения которых в текущий момент известны;способность кибернетической системы к решению задач, методы решения которых ей в текущий момент не известны;множество навыков, используемых кибернетической системой;степень конвергенции и интеграции различного вида моделей решения задач, используемых кибернетической системой;качество организации взаимодействия процессов решения задач в кибернетической системе;быстродействие решателя задач кибернетической системы;способность кибернетической системы решать задачи, предполагающие использование информации, обладающей различного рода не-факторами;многообразие и качество решения задач информационного;способность кибернетической системы генерировать ответы на вопросы различного вида в случае, если они целиком или частично отсутствуют в текущем состоянии информации, хранимой в памяти;способность кибернетической системы к рассуждениям различного вида;качество целеполагания;качество реализации планов собственных действий;способность кибернетической системы к локализации такой области информации,хранимой в ее памяти, которой достаточно для обеспечения решения заданной задачи;способность кибернетической системы к выявлению существенного в информации, хранимой в ее памяти;активность кибернетической системы}

\scnheader{общий объем задач решаемых кибернетической системой}
\scnidtf{...}





\scnheader{общий объем задач, решаемых кибернетической системой}
\scnidtf{общий объем задач, которые кибернетическая система способна решать}	
\scnidtf{общий объем (множество), задач (действий), которые кибернетическая система способна (может, умеет) решать (выполнять) в заданный (в том числе, в текущий) момент}
\scnrelfrom{свойство-предпосылка}{мощность языка представления задач, решаемых кибернетической системой}
			
\scnheader{мощность языка представления задач, решаемых кибернетической системой}
\scnidtf{мощность языка спецификации (описания) различного вида действий, выполняемых кибернетической системой}
\scnnote{мощность языка представления задач прежде всего определяется многообразием видов представляемых задач (многообразием видов описываемых действий)}
		
\scnrelto{свойство-предпосылка}{многообразие видов задач, решаемых кибернетической системой}
	
\scnrelfromlist{частное свойство}{
	мощность языка представления задач, решаемых в памяти кибернетической системы\\
	\scnaddlevel{1}
	\scnrelto{свойство-предпосылка}{многообразие видов задач, решаемых в памяти кибернетической системы}
	\scnaddlevel{-1}
	;мощность языка представления задач, решаемых во внешней среде кибернетической системы\\
	\scnaddlevel{1}
	\scnrelto{свойство-предпосылка}{многообразие видов задач, решаемых во внешней среде кибернетической системы}
	\scnaddlevel{-1}
	;мощность языка представления задач, решаемых в рамках физической оболочки кибернетической системы\\
	\scnaddlevel{1}
	\scnrelto{свойство-предпосылка}{многообразие видов задач, решаемых в рамках физической оболочки кибернетической системы}
	\scnaddlevel{-1}
}
			
\scnheader{многообразие видов задач, решаемых кибернетической системой}			
\scnidtf{многообразие видов действий, которые кибернетическая система способна выполнять}	
\scnnote{подчеркнем, что каждая задача есть спецификация соответствующего (описываемого) действия. Поэтому рассмотрение многообразия видов задач, решаемых кибернетической системой, полностью соответствует многообразию видов деятельности, осуществляемой этой системой. Важно заметить, что есть виды деятельности кибернетической системы, которые определяют качество и, в частности, уровень интеллекта кибернетической системы}
\scnrelfrom{свойство-предпосылка}{мощность языка представления задач в памяти кибернетической системы}
\scnrelfromset{комплекс частных свойств}{
многообразие видов задач, решаемых в памяти кибернетической системы;многообразие видов задач, решаемых во внешней среде кибернетической системы;многообразие видов задач, решаемых в рамках физической оболочки кибернетической системы}}
				
\scnheader{способность кибернетической системы к анализу решаемых задач}
\scnidtf{способность кибернетической системы осмысливать (ведать) то, что она творит}
\scnidtf{способность анализировать свои цели и, соответственно, решаемые задачи на предмет:
\begin{scnitemize}
	\item сложности достижения;
	\item целесообразности достижения (нужности, важности, приоритетности);
	\item соответствия цели существующим нормам (правилам) соответствующей деятельности
\end{scnitemize}}
			
\scnheader{способность кибернетической системы к решению задач, методы решения которых ей в текущий момент известны
}		
\scnnote{Указанными методами могут быть не только алгоритмы, но также и функциональные программы, продукционные системы, логические исчисления, генетические алгоритмы, искусственные нейронные сети различного вида}			
\scnrelfromlist{свойство-предпосылка}{способность кибернетической системы к поиску хранимых в своей памяти методов решения инициированных задач;способность кибернетической системы к интерпретации хранимых в своей памяти методов решения задач}
			
\scnheader{способность кибернетической системы к решению задач, методы решения которых ей в текущий момент не известны
}
\scnidtf{способность кибернетической системы к решению задач, для которых не найдены соответствующие (релевантные) им методы их решения}
\scnidtf{способность кибернетической системы строить цепочку "цель-план достижения цели-система действий"}		
\scnnote{Задачи, для которых не находятся соответствующие им методы, решаются с помощью метаметодов (стратегий) решения задач, направленных:
\begin{scnitemize}
	\item на генерацию нужных исходных данных (нужного контекста), необходимых для решения каждой задачи;
	\item на генерацию плана решения задачи, описывающего сведение исходной задачи к подзадачам (до тех подзадач, методы решения которых системы известны);
	\item на сужение области решения задачи (на сужения контекста задачи, достаточного для ее решения).
\end{scnitemize}}
				
\scnheader{множество навыков, используемых кибернетической системой}
\scnidtf{объем и многообразие навыков, приобретенных кибернетической системой к текущему моменту (с помощью учителей-разработчиков или полностью самостоятельно)}
\scnidtf{возможности, навыки, приобретенные кибернетической системой}
\scnidtf{опыт, приобретенный кибернетической системой}	
\scnnote{новые навыки могут приобретаться кибернетической системой либо полностью самостоятельно, либо с помощью учителей, которые в простейшем случае просто сообщают обучаемой системе полностью сформулированные навыки. Для компьютерных систем учителями является их разработчики.}
\scnrelfromlist{частное свойство}{множество методов решения задач, используемых кибернетической системой;множество моделей решения задач, используемых кибернетической системой	;мощность языка представления в памяти кибернетической системы методов и моделей решения задач}
				
\scnheader{множество методов решения задач, используемых кибернетической системой}
\scnidtf{множество методов решения задач, используемых кибернетической системой и хранимых в ее памяти}
\scnrelto{частное свойство}{многообразие видов знаний, хранимых в памяти кибернетической системы}
				
\scnheader{метод решения задач}
\scnexplanation{\textbf{\textit{метод решения задач}} -- это \textit{вид знаний}, хранимых в \textit{памяти кибернетической системы} и содержащих информацию, которой достаточно либо для сведения каждой \textit{задачи} из соответствующего \textit{класса задач} к \textit{полной системе подзадач*}, решение которых гарантирует решение исходной \textit{задачи}, \uline{либо} для окончательного решения этой \textit{задачи} из указанного \textit{класса задач}}
				
\scnheader{множество моделей решений задач, используемых кибернетической системой}
\scnidtf{способность кибернетической системы к использованию различных видов методов решения задач, соответствующих различным моделям решения задач}
\scnidtf{многообразие методов решения задач, используемых кибернетической системой}
\scnrelfrom{свойство-предпосылка}{мощность языка представления в памяти кибернетической системы методов и моделей решения задач}
			
\scnheader{множество моделей решения задач, используемых кибернетической системой}
\scnrelfrom{примечание}{\scnstartsetlocal

	\scnheaderlocal{следует отличать*}
	\scnhaselementset{вид задач;модель решения задач\\
	\scnaddlevel{1}
	\scnexplanation{каждая \textit{модель решения задач} задается
	\begin{scnitemize} 
		\item \textit{языком}, обеспечивающим представление в \textit{памяти кибернетической системы} некоторого класса \textit{методов решения задач}
		\item интерпретатором указанных \textit{методов}, определяющим \textit{операционную семантику} указанного \textit{языка}
	\end{scnitemize}}
	\scnaddlevel{-1}
	;метод решения задач
	;класс задач\\
	\scnaddlevel{1}
		\scnidtf{\textit{множество} всех тех и только тех \textit{задач}, которые решаются с помощью соответствующего \textit{метода}}
	\scnaddlevel{-1}}\\
	\scnendstruct
}

\scnheader{качество целеполагания}
\scnidtf{качество реализации первого этапа решения сложных задач -- этапа генерации (построения) планов решения сложных задач}
\scnidtf{качество генерации планов выполнения сложных действий:
		\begin{scnitemize}
		\item как внутренних действий (в памяти кибернетической системы), так и внешних действий (во внешней среде)
		\item как собственных действий, так и действий других субъектов
		\end{scnitemize}}
\scnidtf{качество генерации планов действий кибернетической системы и, в частности, трудоемкость процесса генерации этих планов}
\scnidtf{качество организации целенаправленной деятельности кибернетической системы}
\scnidtf{качество построения цепочек "цель-план-действие"{}}
\scnidtf{качество генерации, анализа и инициирования собственных целей (собственных задач)}
\scnidtf{способность кибернетической системы к целеполаганию}
\scnrelfromlist{свойство-предпосылка}{самостоятельность целеполагания\\
	\scnaddlevel{1}
	\scnidtf{самостоятельность генерации и инициирования целей (задач), направленных на создание условий достижения соответствующих стратегических целей (сверхзадач)}
	\scnaddlevel{-1}
;целенаправленность целеполагания\\
	\scnaddlevel{1}
	\scnidtf{степень соответствия (степень полезности) генерируемых целей (задач) для достижения соответствующих стратегических целей (сверхзадач)}
	\scnaddlevel{-1}
;сбалансированность целеполагания\\
	\scnaddlevel{1}
	\scnidtf{качество расстановки приоритетов у сгенерированных и инициированных целей (задач) для обеспечения баланса между тактическими и стратегическими целями}
	\scnaddlevel{-1}}


\scnheader{самостоятельность целеполагания}
\scnidtf{способность кибернетической системы генерировать, инициировать и решать задачи, которые не являются подзадачами, инициированными внешними (другими) субъектами, а также способность на основе анализа своих возможностей отказаться от выполнения задачи, инициированной извне, переадресовав её другой кибернетической системе, либо на основе анализа самой этой задачи обосновать её нецелесообразность или некорректность}
\scnidtf{способность к самостоятельному целеполаганию (генерации идей) и к инициированию процессов их достижения (т.е. к принятию решений), способность свободно (в определенных рамках) выбирать (ставить перед собой цели)}
\scnidtf{уровень самостоятельности}
\scnidtf{способность решать задачи в комплексе, включая создание всех необходимых условий для их решения с учетом конкретных обстоятельств}
\scnidtf{умение решать задачи в условиях сильных помех (в осложненных обстоятельствах)}
\scnnote{Повышение уровня самостоятельности существенно расширяет возможности кибернетической системы, т.е. объем тех задач, которые она может решать не только в "идеальных"{} условиях, но и в реальных (осложненных) обстоятельствах.}	
\scnidtf{степень свободы выбора целей, подлежащих достижению, а также свободы генерации целей, не являющихся подцелями извне поставленных целей}


\scnheader{целенаправленность целеполагания}
\scnidtf{целеустремленность}
\scnidtf{целенаправленность}
\scnidtf{степень целостности деятельности}
\scnidtf{степень соответствия между тактическими и стратегическими уровнями деятельности}
\scnidtf{общее соотношение между временем, затраченным на "лишние"{} (ненужные, нецелесообразные, нецеленаправленные) действия и полезные действия}
\scnidtf{целесообразность деятельности}
\scnidtf{способность адекватно расставлять приоритеты своим целям и не "распыляться"{} на достижение неприоритетных (несущественных) целей}


\scnheader{качество реализации планов собственных действий}
\scnidtf{качество реализации целенаправленной деятельности на основе построенных планов}
\scnidtf{качество реализации построенных в памяти кибернетической системы планов выполнения сложных собственных действий, которые могут предполагать участие других субъектов}


\scnheader{способность кибернетической системы к локализации такой области информации, хранимой в ее памяти, которой достаточно для обеспечения решения заданной задачи}
\scnidtf{способность кибернетической системы к сужению области решения каждой решаемой ею задачи, что существенно минимизирует затраты кибернетической системы на учет и анализ факторов, априори незначимых (несущественных) для решения каждой решаемой задачи}
\scnnote{Для реализации данной способности важное значение имеет качественная стратификация базы знаний кибернетической системы на предметные области и соответствующие им онтологии.}


\scnheader{способность кибернетической системы к выявлению существенного в информации, хранимой в ее памяти}
\scnidtf{способность к выявлению (обнаружению, выделению) таких фрагментов информации, хранимой в памяти кибернетической системы, которые существенны (важны) для достижения соответствующих целей}
\scnnote{Понятие существенного (важного) фрагмента информации, хранимой в памяти кибернетической системы, относительно и определяется соответствующей задачей. Тем не менее, есть важные перманентно (постоянно) решаемые задачи, в частности задачи анализа качества информации, хранимой в памяти кибернетической системы. Существенные фрагменты хранимой информации, выделяемые в процессе решения этих задач, являются относительными не столько по отношению к решаемой задаче, сколько по отношению к текущему состоянию хранимой информации. Примерами таких фрагментов являются:
	\begin{scnitemize}
	\item обнаруженные противоречия (ошибки) с явным указанием того, что чему противоречит;
	\item обнаруженные информационные дыры, точнее точная спецификация этих дыр;
	\item обнаруженные мусорные фрагменты, которые либо носят вспомогательный характер, либо могут быть легко восстановлены (воспроизведены).
	\end{scnitemize}}
	
	
\scnheader{следует отличать*}
\scnhaselementset{
способность кибернетической системы к выявлению существенного в информации, хранимой в ее памяти\\
\scnaddlevel{1}
\scnnote{Здесь кибернетическая система выделяет информацию, которая необходима, но не обязательно достаточна для решения соответствующей задачи.}
\scnaddlevel{-1};
способность кибернетической системы к локализации такой области информации, хранимой в ее памяти, которой достаточно для обеспечения решения заданной задачи\\
\scnaddlevel{1}
\scnnote{Здесь кибернетическая система "отбрасывает"{} (исключает) информацию, которая априори несущественна для решения соответствующей (заданной) задачи.}
\scnaddlevel{-1}} 


\scnheader{активность кибернетической системы}
\scnidtf{уровень активности кибернетической системы}
\scnidtf{уровень мотивации к деятельности в различных направлениях}
\scnidtf{уровень "желания"{} действовать}
\scnidtf{активность/пассивность кибернетической системы}
\scnidtf{уровень инициативности, пассионарности, мотивированности}
\scnnote{Уровень активности кибернетической системы может быть разным для разных решаемых задач, для разных классов выполняемых действий, для разных видов деятельности.}
\scnnote{Следует отличать уровень активности (мотивации, желания) и направленность этой активности.}
\scnnote{Чем выше активность кибернетической системы, тем (при прочих равных условиях) она больше успевает сделать, следовательно, тем выше ее качество (эффективность).}
\scnrelboth{обратное свойство}{пассивность}
	\scnaddlevel{1}
	\scnidtf{уровень бездеятельности, медлительности, вялости, ленивости}
	\scnaddlevel{-1}
\scnrelfromlist{частное свойство}{познавательная активность; социальная активность}

\bigskip
\scnfragmentcaption

\scnheader{качество логико-семантической организации памяти кибернетической системы}
\scnidtf{качество базовых семантически целостных действий в памяти кибернетической системы}
\scnidtf{качество семантически элементарных (законченных, целостных) информационных процессов, выполняемых кибернетической системой в своей памяти}
\scnidtf{интегральная оценка того, насколько способствует (насколько близка) организация памяти кибернетической системы реализации "осмысленных"{} преобразований, хранимых в памяти знаний}
\scnidtf{степень приспособленности решателя задач кибернетической системы к обработке сложноструктурированных баз знаний}
\scnidtf{степень приспособленности решателя задач кибернетической системы к обработке хранимой в её памяти информации, имеющий высокий уровень качества как по форме представления информации, так и по её содержанию -- по многообразию представляемых знаний и по уровню их конвергенции и интеграции}
\scnrelfromlist{свойство-предпосылка}{семантический уровень доступа к информации, хранимой в памяти кибернетической системы
;семантическая гибкость информации, хранимой в памяти кибернетической системы
;степень конвергенции и интеграции представления навыков, хранимых в памяти кибернетической системы, с представлением обрабатываемой информации}


\scnheader{семантический уровень доступа к информации, хранимой в памяти кибернетической системы}
\scnidtf{степень ассоциативности доступа к информации, хранимой в памяти кибернетической системы}
\scnidtf{способность кибернетической системы локализовывать (находить) требуемый (запрашиваемый) фрагмент информации, хранимой в её памяти, не на основании известного адреса запрашиваемой информации (её местоположения в памяти), а на основании:
    \begin{scnitemize}
    \item известного типа запрашиваемой информации;
    \item известных сущностей, знаки которых входят в состав запрашиваемой информации;
    \item полностью или частично известной конфигурации запрашиваемой информации (т.е. конфигурации связей между известными и искомыми сущностями)
    \end{scnitemize}}
\scnexplanation{\textit{уровень доступа к информации, хранимой в памяти кибернетической системы} определяется тем, что нам достаточно знать об искомой в памяти кибернетической системы информации (в частности, об искомом знаке некоторой интересующей нас сущности). Мы можем знать место в памяти (ячейку памяти, область памяти), где находится интересующая нас информация. Такой доступ называется \uline{адресным}. Мы можем знать имя интересующей нас сущности, но не знать, где находится информация, описывающая эту сущность. Мы можем не знать имени интересующей нас сущности, но знать, как эта сущность связана с другими известными нам сущностями.}
\scnexplanation{Пусть нам необходимо локализовать (выделить) хранимую в памяти информацию, описывающую известные нам сущности, связанные известными нам отношениями, но местонахождение этой информации в памяти нам не известно. Если организация памяти нам представляет такую возможность, то такую память будем называть ассоциативной, т.е. памятью, обеспечивающей семантический доступ к хранимой в ней информации.}
\scnnote{Для того, чтобы построить информационную модель среды, в которой действует (функционирует) кибернетическая система, необходимо, с одной стороны, "разложить"{} эту информационную модель "по полочкам"{}, превратить её в некую систему из компонентов этой информационной модели, а, с другой стороны, обеспечить быстрый поиск нужного фрагмента указанной информационной модели, не зная, на каких "полочках"{}  находятся компоненты этого искомого фрагмента, который при этом может иметь произвольную конфигурацию и произвольный размер. Это и есть высший уровень \uline{ассоциативности} доступа к информации, хранимой в памяти кибернетической системы.}
\scnexplanation{Данное свойство, данная характеристика организации информации, хранимой в памяти кибернетической системы, является важнейшей характеристикой \uline{внутреннего} языка представления информации в памяти кибернетической системы. Указанная характеристика внутреннего языка определяется \uline{простотой процедур поиска} востребованных (запрашиваемых) фрагментов хранимой информации -- например, процедуры поиска знаков всех сущностей, каждая из которых связана с заданными (известными) сущностями связями заданных (известных) типов, процедуры поиска (выделения) знаков всех сущностей, которые связаны с заданной (известной) сущностью связью неважно какого типа, процедуры поиска информационного фрагмента заданному образцу (шаблону) произвольного размера и конфигурации, в котором выделены знаки известных сущностей и условные обозначения искомых сущностей.}
\scnaddhind{1}
\scnrelfrom{свойство-предпосылка}{степень близости языка внутреннего представления информации в памяти кибернетической системы к смысловому представлению информации}

\scnheader{семантическая гибкость информации, хранимой в памяти кибернетической системы}
\scnaddhind{1}
\scnrelfrom{свойство-предпосылка}{степень близости языка внутреннего представления информации в памяти кибернетической системы к смысловому представлению информации}
\scnidtf{простота реализации базовых (элементарных), но семантически целостных (семантически значимых, осмысленных) действий (операций) преобразования (обработки) информации, хранимой в памяти кибернетической системы}

\scnheader{базовое семантически целостное действие над информацией, хранимой в памяти кибернетической системы}
\scnidtf{элементарная семантически значимая (осмысленная) операция над информацией, хранимой в памяти кибернетической системы}
\scnnote{Здесь принципиальной является семантическая целостность (осмысленность) действия над хранимой информацией. Так, например, операция адресного доступа к требуемому фрагменту хранимой информации не является семантически целостной, так как смысл искомого (запрашиваемого) фрагмента хранимой информации не уточняется.}
\scnnote{Разные кибернетические системы могут использовать разные наборы классов базовых семантически целостных действий над информацией, хранимой в их памяти.}
\scnnote{Примерами \textit{базовых семантически целостных действий над информацией, хранимой в памяти кибернетической системы}, в частности, являются:
    \begin{scnitemize}
    \item операции поиска, генерации, удаления или замены связок между знаками известных сущностей;
    \item операции поиска, генерации, удаления или замены имен, приписываемых знакам известных сущностей.
    \end{scnitemize}
Существенно подчеркнуть, что простота реализации такого рода операций (т.е. гибкость хранимой в памяти информации) во многом обеспечивается стремлением к локальности выполнения этих операций. Такая локальность означает то, что при выполнении \uline{каждой} из указанных операций меняется только обрабатываемый фрагмент хранимой информации и не требуется никакого переразмещения в памяти остальной части хранимой информации.}

\scnheader{степень конвергенции и интеграции представления навыков, хранимых в памяти кибернетической системы, с представлением обрабатываемой информации}
\scnaddhind{1}
\scnrelto{частное свойство}{степень конвергенции и интеграции различного вида знаний, хранимых в памяти кибернетической системы}
    \scnaddlevel{1}
    \scnrelfrom{свойство-предпосылка}{степень близости языка внутреннего представления информации в памяти кибернетической системы к смысловому представлению информации}
    \scnaddlevel{-1}
\scnnote{Навыки кибернетической системы являются частным видом знаний, хранимых в её памяти, поэтому степень конвергенции навыков и обрабатываемых знаний определяется "глубиной"{} и "объемом"{} \uline{общих} (одинаковых) принципов, лежащих в основе как представления навыков, так представления обрабатываемых знаний.}

\bigskip
\scnfragmentcaption

\scnheader{качество решения интерфейсных задач в кибернетической системе}
\scnrelfromlist{частное свойство}{способность кибернетической системы к пониманию сенсорной информации
;способность кибернетической системы к пониманию принимаемых сообщений
;способность кибернетической системы к самостоятельной деятельности во внешней среде\\
    \scnaddlevel{1}
    \scnidtf{способность кибернетической системы к воздействию на внешнюю среду и к управлению своим поведением во внешней среде}
    \scnaddlevel{-1}}
    

\scnheader{интерфейсная задача}
\scnsuperset{задача анализа введенной информации}
    \scnaddlevel{1}
    \scnsuperset{задача анализа сенсорной информации}
        \scnaddlevel{1}
        \scnidtf{задача анализа информации, порождаемой (генерируемой) непосредственно сенсорами кибернетической системы}
        \scnsuperset{задача синтаксического анализа сенсорной информации}
        \scnsuperset{задача семантического анализа сенсорной информации}
            \scnaddlevel{1}
            \scnidtf{задача анализа сенсорной информации, направленного на \uline{понимание} этой информации -- на выявление (распознавание) в этой информации отображения (сенсорного описания) объектов, важных для кибернетической системы (т.е. объектов, описанных в базе знаний этой системы и, соответственно, представленных в этой базе знаний своими знаками либо знаками классов, которым эти объекты принадлежат), а также важных для кибернетических связей между указанными объектами}
            \scnidtf{задача генерации фрагмента базы знаний кибернетической системы, являющегося логическим следствием заданной сенсорной информации и представляющегося собой важную для кибернетической системы информацию}
            \scnidtf{задача извлечения из сенсорной информации (первичной информации) важной для кибернетической системы вторичной информации}
            \scnaddlevel{-1}
        \scnaddlevel{-1}
        \scnsuperset{задача анализа принимаемого вербального сообщения}
            \scnaddlevel{1}
            \scnidtf{задача анализа введенных знаковых конструкций}
            \scnidtf{задача анализа сообщений, введенных в кибернетическую систему}
            \scnidtf{задача анализа внешних знаковых конструкций}
            \scnsuperset{задача синтаксического анализа принимаемого вербального сообщения}
            \scnsuperset{задача трансляции принимаемого вербального сообщения на внутренний язык кибернетической системы}
            \scnsuperset{задача погружения нового фрагмента в состав согласованной части базы знаний}
                \scnaddlevel{1}
                \scnidtf{задача интеграции (встраивания) нового фрагмента базы знаний в состав базы знаний}
                \scnidtf{задача понимания нового фрагмента базы знаний в контексте её текущего состояния, что, прежде всего, требует обеспечения семантической совместимости (согласования понятий) между базой знаний и интегрируемым новым фрагментом}
                \scnaddlevel{-1}
            \scnaddlevel{-1}
    \scnaddlevel{-1}
\scnsuperset{задача управления эффекторами кибернетической системы при выполнении сложных воздействий на внешнюю среду и/или физическую оболочку этой кибернетической системы}
    \scnaddlevel{1}
    \scnidtf{задача целенаправленной сенсорно-эффекторной (в частности, сенсомоторной) координации}
    \scnaddlevel{-1}
    
\scnheader{сенсорная информация}
\scnidtf{информация, генерируемая непосредственно некоторой группой (конфигурацией) сенсоров (рецепторов) кибернетической системы}
\scnidtf{рецепторная информация}
\scnidtf{первичная информация, получаемая (приобретаемая) кибернетической системой}
\scnidtf{первичная знаковая конструкция, которая описывает те или иные свойства текущего состояния физической окружающей среды (внешней среды и физической оболочки) кибернетической системы}

\scnheader{сенсор кибернетической системы}
\scnidtf{рецептор кибернетической системы}
\scnexplanation{компонент кибернетической системы, генерирующий в памяти этой системы информацию о текущем значении соответствующего этому компоненту свойства (характеристики, параметра) того фрагмента физической окружающей среды кибернетической системы, который непосредственно смежен (пограничен) указанному компоненту}

\scnheader{эффектор кибернетической системы}
\scnidtf{компонент кибернетической системы, который способен менять своё состояние в целях непосредственного воздействия на свою физическую оболочку и на внешнюю среду}

\scnheader{способность кибернетической системы к пониманию сенсорной информации}
\scnidtf{способность к синтаксическому и семантическому анализу информации, формируемой сенсорами кибернетической системы, а также к "погружению"{} этой информации в состав общей информационной модели внешней среды кибернетической системы (в состав общей картины внешнего мира)}
\scnidtf{способность кибернетической системы к переходу от первичной (сенсорной) информации ко вторичной информации, которая описывает связи между вторичными объектами, каждый из которых представлен (описан) в первичной информации конфигурацией знаков своих частей с дополнительным описанием свойств каждой из этих частей}

\scnheader{способность кибернетической системы к самостоятельной деятельности во внешней среде}
\scnrelfromlist{свойство-предпосылка}{уровень развития эффекторов, обеспечивающих самостоятельное перемещение кибернетической системы\\
    \scnaddlevel{1}
    \scnrelfromlist{частное свойство}{уровень развития эффекторов, обеспечивающих локальное перемещение сенсоров кибернетической системы;уровень развития эффекторов, обеспечивающих функционирование манипуляторов кибернетической системы;уровень развития эффекторов, обеспечивающих перемещение всей физической оболочки кибернетической системы}
    \scnaddlevel{-1}
    ;качество управления поведением кибернетической системы во внешней среде\\
        \scnaddlevel{1}
        \scnidtf{качество сенсорно-эффекторной координации действий кибернетической системы при выполнении сложных действий во внешней среде}
        \scnaddlevel{-1}}

\bigskip

\scnheader{ограниченность обучения кибернетической системы}
\scnexplanation{Данное свойство определяет границу между теми знаниями и навыками, которые соответствующая \textit{кибернетическая система} принципиально может приобрести, и теми знаниями и навыками, которые указанная кибернетическая система не сможет приобрести никогда. Данное свойство определяет максимальный уровень потенциальных возможностей соответствующей кибернетической системы. Очевидно, что максимальная степень отсутствия ограничений в приобретении новых знаний и навыков -- это полное отсутствие ограничений, т.е. полная универсальность возможностей соответствующих кибернетических систем, которые всё могут познать и всё могут сотворить.}
\scnidtf{максимум того, чему кибернетическая система может обучиться}
\scnidtf{максимальная перспектива обучения кибернетической системы}
\scnidtf{максимальный уровень качества, который кибернетическая система может достичь в процессе обучения}
\scnrelfromlist{частное свойство}{максимальный объём знаний, которые кибернетическая система может приобрести в процессе обучения;максимальный объём навыков, которые кибернетическая система может приобрести в процессе обучения}

\scnheader{максимальный объём знаний, которые кибернетическая система может приобрести в процессе обучения}
\scnidtf{граница приобретаемых знаний, за пределы которой кибернетическая система принципиально не может перейти в процессе своего обучения}
\scnidtf{максимум того, чему можно научить соответствующую кибернетическую систему}
\scnidtf{максимальный объём знаний, которые кибернетическая система принципиально может приобрести}
\scnrelto{свойство-предпосылка}{обучаемость}
\scnnote{чем больше \textit{максимальный объём знаний, которые кибернетическая система принципиально может приобрести}, тем выше уровень \textit{обучаемости} кибернетической системы}

\scnheader{познавательная активность кибернетической системы}
\scnidtf{познавательная мотивированность}
\scnidtf{познавательная пассионарность}
\scnidtf{любознательность}
\scnidtf{активность и самостоятельность в приобретении новых знаний и навыков}
\scnidtf{стремление, активная целевая установка к постоянному совершенствованию (повышению качества) и пополнению собственной базы знаний}
\scnnote{Следует отличать
	\begin{scnitemize}
		\item способность (возможность) приобретать новые знания и навыки и совершенствовать приобретенные знания и навыки
		\item от желания (стремления) это делать.
	\end{scnitemize}}
\scnnote{желание (целевая установка) научиться решать те или иные задачи может быть сформулировано кибернетической системой либо самостоятельно, либо извне (некоторым учителем).}

\scnrelfromlist{частное свойство}{активность в изучении внешней среды;активность в анализе качества информации, хранимой в собственной памяти;активность в анализе собственных действий и действий других кибернетических систем}
\scnrelfromlist{свойство-предпосылка}{способность кибернетической системы к синтезу познавательных целей и процедур;способность кибернетической системы к самоорганизации собственного обучения;способность кибернетической системы к экспериментальным действиям}

\scnheader{способность кибернетической системы к синтезу познавательных целей и процедур}
\scnidtf{способность планировать своё обучение и управлять процессом обучения}
\scnidtf{умение задавать вопросы или целенаправленные последовательности вопросов самому себе или другим субъектам как важнейший фактор обучаемости}
\scnidtf{способность генерировать (формулировать, задавать) вопросы, адресуемые либо самому себе, либо некоторому внешнему источнику знаний и направленные на повышение качества собственных знаний и навыков}
\scnidtf{способность генерировать четкую спецификацию своей информационной потребности}
\scnidtf{способность кибернетической системы четко формулировать то, что она не знает (в частности, не умеет), но хотела бы знать и уметь}
\scnidtf{способность к формированию спецификаций информационных баз в своих знаниях}
\scnidtf{способность кибернетической системы самостоятельно генерировать цели на приобретение знаний и навыков, обеспечивающих решение различных классов задач}

\scnheader{способность кибернетической системы к самоорганизации собственного обучения}
\scnidtf{способность осуществлять управление своим обучением}
\scnidtf{способность кибернетической системы самой выполнять роль своего учителя, организующего процесс своего обучения}

\scnheader{способность кибернетической системы к экспериментальным действиям}
\scnidtf{способность к отклонениям от составленных планов своих действий для повышения качества результата или сохранении целенаправленности этих действий}
\scnidtf{способность к экспромтам и импровизации}

\scnheader{способность кибернетической системы к самосохранению}
\scnidtf{способность кибернетической системы к выявлению и устранению угроз, направленных на снижение её качества и даже на её уничтожение, что означает полную потерю необходимого качества}
\scnidtf{уровень безопасности (защищенности) кибернетической системы}
\scnexplanation{Данное свойство кибернетических систем является необходимым фактором высокого уровня обучаемости кибернетических систем. Чем выше уровень безопасности кибернетической системы, тем выше её уровень обучаемости.}
\scnidtf{способность кибернетической системы к обеспечению собственной безопасности}
\scnrelfromlist{свойство-предпосылка}{способность кибернетической системы анализировать смысл задач, инициированных извне, и отказываться от решения вредных задач}
\scnaddlevel{1}
\scntext{эпиграф}{Прежде, чем выполнять приказ, подумай}
\scnexplanation{Примером вредной задачи для \textit{ostis-системы} является запрос всех хранимых в памяти \textit{sc-элементов}}
\scnexplanation{Подчеркнем, что в современных компьютерных системах и интеллектуальных компьютерных системах подходы к обеспечению их информационной безопасности имеют принципиальные отличия, связанные, прежде всего с тем интеллектуальные компьютерные системы обладают более мощными средствами семантического и контекстного анализа приобретаемой информации.}


\scnheader{способность кибернетической системы к устранению информационных дыр, обнаруженных в информации, хранимой в ее памяти}
\scnrelfrom{свойство-предпосылка}{способность кибернетической системы генерировать ответы на вопросы различного вида в случае, если они целиком или частично отсутствуют в текущем состоянии информации, хранимой в памяти}
\scnnote{Формальным результатом обнаружения информационной дыры является формулировка запроса на недостающую информацию, которую необходимо сгенерировать.}


\scnheader{способность кибернетической системы к удалению информационного мусора, обнаруженного в информации, хранимой в ее памяти}
\scnidtf{способность кибернетической системы к забыванию (стиранию, удалению) ненужной (лишней, "отработанной"{}) информации, которая, например, играет роль информационных "лесов"{} при решении различных задач}
\scnnote{Критериями информационного мусора может быть:
		\begin{scnitemize}
		\item завершение решения задачи, для которой данная информация является вспомогательной и востребованной только в рамках решения соответствующей задачи;
		\item истечение срока давности хранения;
		\item легкая воспроизводимость (при необходимости).
		\end{scnitemize}}
		

\scnheader{способность кибернетической системы к семантическому погружению новых знаний в состав информации, хранимой в ее памяти}
\scnnote{Новая введенная в память информационная конструкция трактуется как конструкция, у которой входящие в нее знаки являются потенциальными синонимами тем знакам, которые уже присутствуют в хранимой информации. Поэтому для всех этих знаков надо проверить наличие их синонимов. После этого синонимичные знаки должны быть отождествлены. Отождествление знаков осуществляется либо путем приписывания им одинаковых идентификаторов (имен), либо путем "физического"{} склеивания этих знаков.}
\scnnote{Новой информацией, погружаемой (вводимой) в состав информации, хранимой в памяти кибернетической системы, может быть:
		\begin{scnitemize}
		\item либо принятое сообщение, поступившее от другой кибернетической системы и переведенное на внутренний язык данной системы;		
		\item либо информация, сгенерированная в результате решения какой-либо задачи.
		\end{scnitemize}}
		
	
\scnheader{способность кибернетической системы к обнаружению сходств в знаниях, хранимых в ее памяти}
\scnnote{Сходства в знаниях могут иметь самый разнообразный вид и далеко не всегда являются очевидными.}
\scnnote{Умение "видеть"{} сходство в различном и различие в сходном является важнейшим признаком интеллекта.}


\scnheader{способность кибернетической системы к конвергенции знаний, хранимых в ее памяти}
\scnidtf{способность кибернетической системы к увеличению сходств в знаниях хранимых в ее памяти}
\scnrelfrom{свойство-предпосылка}{способность к увеличению числа общих понятий для различных фрагментов информации, хранимой в памяти кибернетической системы, без ущерба качеству этих фрагментов}
\scnidtf{способность к "сближению"{} знаний путем:
		\begin{scnitemize}
		\item увеличения числа общих понятий, используемых в "сближаемых"{} знаниях;
		\item преобразования исходных знаний к их логически эквивалентным вариантам в целях получения фрагментов как можно большего размера и как можно в большем количестве, которые были бы:
				\begin{scnitemizeii}
				\item либо синтаксически изоморфными и содержащими как можно большее число общих понятий;	
				\item либо синтаксически изоморфными и одновременно семантически эквивалентными.
				\end{scnitemizeii}
		\end{scnitemize}}
		

\scnheader{способность кибернетической системы к интеграции знаний, хранимых в ее памяти}
\scnidtf{способность объединять имеющиеся знания и формировать целостную картину различных исследуемых объектов, систем, процессов, явлений}
\scnrelfrom{свойство-предпосылка}{способность кибернетической системы к конвергенции знаний, хранимых в ее памяти}
\scnnote{Качество (глубина) интеграции знаний определяется тем, насколько качественно до этого была проведена конвергенция интегрируемых знаний.}
\scnnote{Качественная ("бесшовная"{}, глубокая) интеграция различных знаний, хранимых в памяти кибернетической системы, дает возможность существенно снизить количество хранимых в памяти методов решения задач, так как позволяет некоторые ранее различные классы задач объединить в один класс задач. При этом очевидно, что такая интеграция знаний, хранимых в памяти кибернетической системы, требует разработки \uline{общих} (базовых) синтаксических и семантических принципов представления знаний различного вида.}


\scnheader{конвергенция и интеграция знаний}
\scnnote{Мы вынуждены смотреть на окружающую нас внешнюю среду (внешний мир) через "замочную скважину"{} своих сенсоров (рецепторов), своих персональных точек зрения, мировоззрения различных научных дисциплин. Но необходимо помнить, что целостную картину внешней среды (картину мира) можно построить только путем сближения (конвергенции) и соединения (интеграции) самых различных точек зрения, самых различных научных дисциплин и направлений. Мир не делится на различные дисциплины -- он един. Для эффективного решения задач конвергенции и интеграции знаний необходимо построить искусственную ("рукотворную") среду (память), в которой было бы удобно не только хранить самые различные знания, но и осуществлять конвергенцию и интеграцию этих знаний. При этом очень важно, чтобы формируемая таким образом информационная модель окружающей нас внешней среды (информационной картины мира) была общедоступна как для просмотра (ознакомления), причем, без каких бы то ни было "замочных скважин"{}, так и для ввода новых знаний, представляющих (отражающих) точку зрения их авторов.}
\scnrelfrom{эпиграф}{Древнеиндийская притча о слоне и слепцах}

		
\scnheader{следует отличать*}
\scnhaselementset{
конвергенция\scnsupergroupsign
\scnaddlevel{1}
\scnidtf{Свойство, определяющее степень близости (уровень конвергенции) между двумя заданными сущностями и, в частности, знаниями}
\scnaddlevel{-1};
конвергенция* 
\scnaddlevel{1}
\scnidtf{Множество пар близких (аналогичных, сходных) сущностей*}
\scnaddlevel{-1};
конвергенция
\scnaddlevel{1}
\scnidtf{Множество \uline{процессов} сближения различных пар сущностей}
\scnaddlevel{-1}}		
		
\scnhaselementset{
интеграция*
\scnaddlevel{1}
\scnidtf{Квазибинарное \uline{отношение}, каждая пара которого связывает множество интегрируемых сущностей с результатом интеграции*}
\scnaddlevel{-1};
интеграция
\scnaddlevel{1}
\scnidtf{Множество \uline{процессов} интеграции множества заданных сущностей}
\scnaddlevel{-1}}


\scnheader{способность кибернетической системы к обобщениям и формированию новых понятий}	
\scnnote{Важным примером обобщения является переход от задач к классам часто решаемых задач.}


\scnheader{cпособность кибернетической системы к генерации гипотез и обнаружению закономерностей в информации, хранимой в ее памяти}
\scnnote{Данная способность кибернетической системы является важнейшим фактором эволюции информации, хранимой в памяти кибернетической системы, в направлении перехода от данных (от фактографической информации) к знаниям.}


\scnheader{cпособность кибернетической системы к обоснованию или опровержению знаний, хранимых в ее памяти}
\scnnote{Примерами знаний, подлежащих обоснованию или опровержению, являются:
\begin{scnitemize}
	\item любое введенное в кибернетическую систему сообщение (любая новая информация, поступающая от любого субъекта);
	\item формулировка какой-либо задачи, предлагаемой для решения;
	\item формулировка какого-либо гипотетического утверждения (теоремы), подлежащего доказательству.
\end{scnitemize}}	
\scnidtf{способность к объяснению (обоснованию, аргументации) корректности, важности и целесообразности использовать (обратить внимание на) указываемое знание}
\scnidtf{способность либо находить в текущем состоянии базы знаний, либо генерировать (строить) ответы на \textit{почему-вопросы}}
	

\scnheader{способность кибернетической системы к экспериментальному подтверждению или опровержению гипотез о свойствах динамических систем с помощью имитационных моделей этих систем}
\scnnote{Создание динамических информационных моделей сложных динамических систем и проведение различного рода "мысленных"{} экспериментов с такими моделями является весьма перспективным и мощным методом исследования сложных динамических систем.}	


\scnheader{способность кибернетической системы к коррекции теорий, хранимых в ее памяти}
\scnidtf{способность к адаптации накопленных знаний к различным изменениям условий и жизненных ситуаций}
\scnnote{В основе данного свойства кибернетической системы лежит:
\begin{scnitemize}
	\item постоянная готовность кибернетической системы подвергнуть сомнению любое знание, хранимое в ее памяти;
	\item постоянное уточнение степени достоверности каждого знания, хранимого в памяти кибернетической системы.
\end{scnitemize}}	

\bigskip
\scnfragmentcaption

\scnheader{способность кибернетической системы к повышению качества своего решателя задач}
\scnidtf{способность кибернетической системы повышать качество своих приобретаемых навыков}
\scnrelfromlist{свойство-предпосылка}{способность кибернетической системы к повышению качества информации, хранимой в ее памяти
;семантическая гибкость возможных самоизменений решателя задач кибернетической системы
;стратифицированность решателя задач кибернетической системы
;способность кибернетической системы к анализу качества своего решателя задач
;способность кибернетической системы к целенаправленной коррекции своей деятельности
;способность кибернетической системы к оптимизации хранимых в памяти методов решения задач
;способность кибернетической системы к генерации новых методов решения задач
;способность кибернетической системы интегрировать у себя новые приобретаемые извне методы и модели решения задач}


\scnheader{семантическая гибкость возможных самоизменений решателя задач кибернетической системы}
\scnidtf{простота реализации решателем задач кибернетической системы различного рода изменений самого себя}
\scnnote{Очевидно, что семантическая гибкость решателя задач кибернетической системы во многом определяется процессором кибернетической системы (прежде всего, его универсальностью и близостью реализуемой им модели обработки информации к смысловому уровню). Но, поскольку решатель задач кибернетической системы кроме процессора включает в себя хранимые в памяти кибернетической системы методы решения различного вида задач (в том числе, и методы интерпретации методов высокого уровня), семантическая гибкость решателя задач определяется также \textit{семантической гибкостью информации, хранимой в памяти кибернетической системы}.}
\scnrelfrom{свойство-предпосылка}{семантическая гибкость информации, хранимой в памяти кибернетической системы}


\scnheader{стратифицированность решателя задач кибернетической системы}
\scnrelfromlist{частное свойство}{стратифицированность методов и навыков решения задач, представленных в памяти кибернетической системы
;стратифицированность технологий, соответствующих различным видам деятельности
;стратифицированность различного вида действий, классов действий и видов деятельности}
	\scnaddlevel{1}
	\scnaddhind{1}
	\scnrelfrom{частное свойство}{стратифицированность различного вида информационных процессов, выполняемых в памяти кибернетической системы}
	\scnaddlevel{-1}
	
	
\scnheader{качество внутренних языковых средств кибернетической системы для описания качества собственного решателя задач}
\scnaddhind{1}
\scnrelfrom{свойство-предпосылка}{качество внутренних языковых средств кибернетической системы для описания качества собственных действий}
	
	
\scnheader{способность кибернетической системы к анализу качества своего решателя задач}
\scnidtf{способность кибернетической системы к анализу (к оценке качества) своей деятельности в собственной внутренней среде (в своей памяти), а также в своей внешней среде}
\scnnote{Анализ качества решателя задач включает в себя:
			\begin{scnitemize}
			\item анализ качества используемых методов и технологий решения задач;
			\item анализ качества используемых моделей решения задач;
			\item анализ полноты набора постоянно инициированных целей (задач), направленных на эволюцию и на борьбу с деградацией (снижением качества) кибернетической системы;
			\item анализ качества выполняемых действий (процессов решения задач).
			\end{scnitemize}}
\scnidtf{способность кибернетической системы к описанию (к построению в своей памяти информационной модели) собственных действий, выполняемых в собственной памяти, а также к анализу и оценке этих действий}
\scnidtf{способность кибернетической системы к анализу своего поведения в своей внутренней среде (в своей памяти), а также в своей внешней среде и в своей физической оболочке}
\scnrelfromlist{свойство-предпосылка}{качество внутренних языковых средств кибернетической системы для описания качества собственного решателя задач
;способность кибернетической системы к анализу собственной деятельности\\
		\scnaddlevel{1}
		\scnrelfromlist{частное свойство}{способность кибернетической системы к анализу качества информационных процессов, выполняемых в собственной памяти
			\scnaddlevel{1}
			\scnidtf{способность кибернетической системы к анализу качества своего поведения (действий, информационных процессов) в собственной внутренней среде -- своих действий, сводящихся к поиску, генерации, удалению и преобразованию информационных конструкций, хранимых в собственной памяти}
			\scnaddlevel{-1}
;способность кибернетической системы к анализу качества своего поведения во внешней среде}
		\scnaddlevel{-1}
;способность кибернетической системы к анализу качества методов, хранимых в собственной памяти\\
		\scnaddlevel{1}
		\scnrelfromlist{частное свойство}{способность кибернетической системы к анализу качества методов и технологий, используемых ею для выполнения сложных действий в собственной памяти;способность кибернетической системы к анализу качества методов и технологий, используемых ею для выполнения сложных действий во внешней среде}
		\scnaddlevel{-1}}


\scnheader{качество внутренних языковых средств кибернетической системы для описания качества собственных действий}
\scnnote{В данном свойстве кибернетической системы имеется в виду описание собственных действий, выполняемых кибернетической системой как в своей внутренней среде (в собственной памяти), так и в своей внешней среде.}


\scnheader{способность кибернетической системы к анализу качества своего поведения во внешней среде}
\scnidtf{способность кибернетической системы к анализу соответствия между тем, что планировалось сделать во внешней среде и тем, что реально получилось}
\scnnote{Поведение кибернетической системы во внешней среде рассматривается ею как "эксперимент"{}, подтверждающий или опровергающий ее представление о внешней среде.}
\scnidtf{способность кибернетической системы к анализу своего опыта взаимодействия с внешней средой и, в частности, к выявлению своих ошибок}


\scnheader{способность кибернетической системы к целенаправленной коррекции своей деятельности}
\scnidtf{способность кибернетической системы к коррекции своего поведения в целях повышения его качества (эффективности)}
\scnidtf{способность кибернетической системы учиться на ошибках своей деятельности на основе анализа этих ошибок}
\scnrelfrom{свойство-предпосылка}{способность кибернетической системы к анализу собственной деятельности}


\scnheader{способность кибернетической системы к оптимизации хранимых в памяти методов решения задач}
\scnnote{Хранимые в памяти \textit{методы} решения задач разбиваются на следующие классы:
	\begin{scnitemize}
	\item \textit{методы верхнего уровня} -- интерпретируемые методы;
	\item \textit{методы базового уровня}, представленные на базовом языке программирования, который интерпретируется непосредственно процессором кибернетической системы;
	\item \textit{метаметоды}, описывающие интерпретацию методов верхнего уровня.
	\end{scnitemize}}


\scnheader{способность кибернетической системы к генерации новых методов решения задач}
\scnnote{Целесообразность генерации нового метода решения задач возникает, когда кибернетической системе приходится часто решать эквивалентные задачи некоторого класса. Генерация соответствующего метода и последующая его оптимизация позволяет существенно сократить время решения задач.}
\scnidtf{способность кибернетической системы расширять множество используемых ею методов решения задач}
\scnnote{Если добавляемые методы соответствуют используемым моделям решения задач, то, кроме добавления самих методов, желательно, чтобы в стратифицированной кибернетической системе никакие другие изменения не потребовались. Если добавляемый метод соответствует новой (ранее не известной) модели решения задач, то желательно, чтобы в стратифицированной кибернетической системе никакие другие изменения не потребовались, кроме добавления агентов, обеспечивающих интерпретацию (описание операционной семантики) методов нового класса.}
\scnnote{Речь идет о методах решения как внутренних задач, решаемых в памяти кибернетической системы, так и внешних задач, решаемых во внешней среде путем управления деятельностью эффекторов и рецепторов кибернетической системы.}
\scnrelfrom{свойство-предпосылка}{способность расширять множество использованных моделей решения задач}


\scnheader{способность кибернетической системы интегрировать у себя новые приобретаемые извне методы и модели решения задач}
\scnnote{Для обеспечения такой способности необходима:
	\begin{scnitemize}
	\item разработка универсальной базовой модели решения задач, для которой соответствующие ей методы решения задач интерпретируются процессором кибернетической системы;
	\item разработка семейства классов методов верхнего уровня, что предполагает:
		\begin{scnitemizeii}
		\item разработку языков представления методов для каждого класса методов верхнего уровня;
		\item разработку интерпретаторов для каждого класса методов верхнего уровня на основе указанной выше базовой модели решения задач.
		\end{scnitemizeii}
	\end{scnitemize}}


\bigskip
\scnfragmentcaption

\scnheader{способность кибернетической системы к повышению качества своей физической оболочки}
\scnidtf{способность кибернетической системы к самостоятельному совершенствованию (эволюции) своей физической оболочки}
\scnnote{Данная способность кибернетической системы накладывает определенные требования к построению ее физической оболочки.}
\scnrelfromlist{свойство-предпосылка}{гибкость возможных изменений физической оболочки кибернетической системы
;стратифицированность физической оболочки кибернетической системы
;способность кибернетической системы к анализу качества своей физической оболочки\\
	\scnaddlevel{1}
	\scnrelfrom{свойство-предпосылка}{качество внутренних языковых средств кибернетической системы для описания качества собственной физической оболочки}
	\scnaddlevel{-1}
;способность кибернетической системы расширять и/или совершенствовать набор собственных сенсоров и эффекторов}


\bigskip
\scnfragmentcaption

\scnheader{комплекс свойств кибернетических систем, определяющих их обучаемость по различным формам обучения}
\scneqtoset{обучаемость с учителем;самообучаемость с экспертом;самообучаемость на основе внешних информационных источников;самообучаемость без внешних информационных источников}


\scnheader{обучаемость с учителем}
\scnidtf{уровень способности к обучению под управлением внешнего субъекта-учителя}
\scnidtf{способность кибернетической системы к эффективному обучению с помощью учителя, осуществляющего управление процессом обучения}
\scnidtf{способность заданной кибернетической системы эффективно обучаться с помощью внешней кибернетической системы (внешнего субъекта, внешнего активного учителя), осуществляющей организацию обучения заданной кибернетической системы на основе различных методик обучения, учитывающих особенности обучаемой системы и определяющих характер (в том числе последовательность) передачи знаний и новыков, а также тестирование качества их усвоения}


\scnheader{самообучаемость с экспертом}
\scnidtf{способность кибернетической системы к самообучению в диалоге с экспертом-консультантом}
\scnidtf{способность кибернетической системы не просто задавать нужные для собственного обучения вопросы (информационные цели), но и вести вопросно-ответный диалог с другими субъектами (кибернетическими системами), которые являются экспертами в соответствующей области (указанные эксперты – это своего рода "пассивные учителя"{}, которые много знают и умеют в соответствующей области, могут отвечать на вопросы, но не желают управлять процессом передачи этих знаний и умений другим кибернетическим системам)}
\scnidtf{эффективность самообразования кибернетической системы, в основе которого лежит диалог, управляемый этой обучаемой системой и осуществляемый с кибернетической системой, являющейся носителем востребованных знаний и навыков}
\scnidtf{эффективность самообучения, осуществляемого в форме консультации}
\scnidtf{способность управлять процессом самообучения путем формирования последовательности вопросов (познавательных целей), адресуемых внешним субъектам}
\scnrelfrom{свойство-предпосылка}{способность кибернетической системы к синтезу познавательных целей и процедур}
	

\scnheader{обучаемость на основе пассивных внешних информационных источников}
\scnidtf{способность кибернетической системы к извлечению информации, содержащейся во внешних информационных источниках, к поиску нужных внешних источников и к построению на этой основе систематизированной картины мира}
\scnidtf{эффективность самообучения, основанного на анализе \uline{пассивных} источников информации (документов различного вида, публикаций, текстов, которые необходимо находить в различного рода библиотеках, читать и \uline{понимать})}
	

\scnheader{самообучаемость без внешних информационных источников}
\scnidtf{способность кибернетической системы формировать систематизированную модель (картину) окружающей среды, используя для ее непосредственного восприятия и изучения только собственные сенсоры и эффекторы, а также некоторые дополнительные средства, усиливающие возможности сенсоров и эффекторов}
\scnidtf{эффективность самообучения кибернетической системы, основанного исключительно на собственном опыте, на анализе собственной деятельности и собственных ошибок}
\scnnote{Данная способность кибернетической системы является необходимым, но явно недостаточным фактором ее высокого качества. Учиться только на собственном опыте -- существенно понизить уровень своего интеллекта. В этом смысле познавательный процесс социален.}

\bigskip
\scnendstruct \scninlinesourcecommentpar{Завершили \textit{Сегмент 7 Начала Раздела \ref{sd_sys_inform}} ``\nameref{sd_sys_inform}''}

\input{Contents/chapter1/intro_hs/final_segment}

\end{SCn}


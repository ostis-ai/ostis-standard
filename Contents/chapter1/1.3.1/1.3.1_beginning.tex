\begin{SCn}

\scnsectionheader{\currentname}

\scnstartsubstruct

\scnrelto{частная предметная область и онтология}{Предметная область и онтология информационных конструкций}
\scnaddlevel{1}
\scnsourcecommentpar{Раздел 2.1.2.0}
\scnaddlevel{-1}

\scnsdmainclasssingle{смысловое представление информации}

\scnsdclass{семантическая сеть\\
	\scnaddlevel{1}
	\scnsubdividing{нерафинированная семантическая сеть;рафинированная семантическая сеть}
	\scnsubdividing{абстрактная семантическая сеть\\
		\scnaddlevel{1}
		\scnidtf{семантическая сеть, абстрагирующаяся от того, как физически представлены ее элементарные (атомарные) фрагменты, а также связи инцидентности между этими фрагментами}
		\scnaddlevel{-1}
	;графически представленная семантическая сеть\\
		\scnaddlevel{1}
		\scnidtf{нарисованная семантическая сеть}
		\scnaddlevel{-1}
	;семантическая сеть, хранимая в графодинамической памяти\\
		\scnaddlevel{1}
		\scnrelboth{следует отличать}{представление семантической сети в адресной памяти}
			\scnaddlevel{1}
			\scnnotsubset{семантическая сеть}
			\scnidtf{представление семантической сети в виде линейной информационной конструкции, которая хранится в адресной памяти и которая, строго говоря, уже не является семантической сетью, но является информационной конструкцией, семантически эквивалентной соответствующей (представляемой) семантической сети}			
			\scnaddlevel{-1}
		\scnaddlevel{-1}}
	\scnaddlevel{-1}
;язык семантических сетей\\
	\scnaddlevel{1}
	\scnidtf{язык, все тексты которого являются семантическими сетями}
	\scnsubdividing{специализированный язык семантических сетей;универсальный язык семантических сетей}
	\scnsuperset{язык рафинированных семантических сетей}
	\scnaddlevel{-1}}

\scnrelfromvector{рассматриваемые вопросы}{
\scnfileitem{Что такое семантические сети и в чем их принципиальное отличие от других вариантов представления информации}
;\scnfileitem{До какой степени можно минимизировать алфавит элементов семантических сетей}
;\scnfileitem{Можно ли все описываемые связи свести к бинарным связям и почему это целесообразно}
;\scnfileitem{Можно ли разработать \uline{универсальный} язык семантических сетей}
;\scnfileitem{До какой степени можно упростить синтаксические структуры семантических сетей до, условно говоря, рафинированного вида}
;\scnfileitem{Какими достоинствами обладает семантические сети}}

\scnrelfromlist{ссылка}{Понятие Технологии OSTIS\\
	\scnaddlevel{1}
	\scnsourcecommentpar{Сегмент 3 Раздела 0.2}
	\scntext{аннотация}{В данном сегменте \textit{Документации Технологии OSTIS} рассматриваются принципы, лежащие в основе \textit{Технологии OSTIS}, основным из которых является ориентация на использование \textit{\uline{универсального} языка рафинированных семантических сетей} в качестве внутреннего языка \textit{интеллектуальных компьютерных систем}}
	\scnaddlevel{-1}
;Описание внутреннего языка ostis-систем\\
	\scnaddlevel{1}
	\scnsourcecommentpar{Раздел 0.3.1}	
	\scntext{аннотация}{В данном разделе \textit{Документации Технологии OSTIS} рассматриваются принципы, лежащие в основе \textit{универсального языка рафинированных семантических сетей}, используемого в качестве внутреннего языка \textit{ostis-систем} -- \textit{интеллектуальных компьютерных систем} следующего поколения}
	\scnaddlevel{-1}
;Описание языка графического представления знаний ostis-систем\\
	\scnaddlevel{1}
	\scnsourcecommentpar{Раздел 0.3.3}
	\scntext{аннотация}{В данном разделе \textit{Документации Технологии OSTIS} рассматриваются принципы, лежащие в основе универсального языка графически представленных семантических сетей, используемого в \textit{пользовательском интерфейсе ostis-систем}}
	\scnaddlevel{-1}
;Бирюков Б.В. ТеориСГФ-1960ст;Гладун В.П.;Скороходько;Мартынов;Шенк;Мельчук-Жолковский Смысл-Текст;Кузнецов Игорь}
	
\scnauthorcomment{Дооформить библиографию}	

\bigskip
\scnfragmentcaption

\scnheader{знак}
\scnidtf{фрагмент информационной конструкции, обладающий свойством, \uline{обозначать} некоторую сущность (объект), которая наряду с другими сущностями описывается указанной информационной конструкцией}
\scnnote{\uline{Форма} представления знаков в известной степени условна и является результатом соглашения между носителями соответствующего языка. Знак может быть, например, представлен:
	\begin{scnitemize}
	\item  в виде фрагмента речевого сообщения (последовательностью фонем);
	\item в виде строки символов (последовательности букв) в заданном алфавите;
	\item в виде иероглифа, пиктограммы;
	\item в виде жеста.
	\end{scnitemize}}
\scniselementrole{ключевой знак}{Предметная область и онтология информационных конструкций}
	\scnaddlevel{1}
	\scnsourcecommentpar{Раздел 2.1.2.0}
	\scnhaselement{раздел Базы знаний IMS.ostis}
	\scnaddlevel{-1}
\scntext{характеристика элементов данного множества}{Знаки, используемые в различных языках, характеризуются:
	\begin{scnitemize}
	\item синтаксической структурой, по совпадению (изоморфизму) которых для разных знаокв предполагается их синонимия;
	\item денотационной семантикой, т.е. той сущностью, которая обозначается соответствующим знаком;
	\item типом (классом) обозначаемой сущности, которая может быть:
	 	\begin{scnitemizeii}
		\item материальным(физическим) элементом (точкой) абстрактного пространства, множеством, которое может быть:
			\begin{scnitemizeiii}
			\item связью;
			\item классом;
			\item структурой;
			\end{scnitemizeiii}
		\item реальной и вымышленной сущностью;
		\item константной (конкретной) и переменной (произвольной) сущностью;
		\item постоянно существующей и временно существующей сущностью (прошлой, настоящей, будущей);		
		\end{scnitemizeii}
	\item множеством тех связей, которые связывают сущность, обозначаемую данным знаком с другими сущностями, а также, если данный знак обозначает некоторую связь, множеством сущностей, которые связаны этой связью, т.е. сущностей, являющихся компонентом этой связи;
	\item текущим статусом самого знака в памяти кибернетической системы, который может быть:
		\begin{scnitemizeii}
			\item логически удаленным знаком;
			\item настоящим знаком;
			\item предлагаемым (возможно, будущим) знаком.
		\end{scnitemizeii}
	\end{scnitemize}}
	
\scnheader{денотат*}
\scnidtf{денотат заданного знака*}
\scnidtf{объект, обозначаемый заданным знаком*}
\scnidtf{денотационная семантика заданного знака*}
\scnidtf{смысл заданного знака*}
\scnidtf{Бинарное ориентированное отношение, каждая пара которого связывает:
	\begin{scnitemize}
			\item некоторый знак, представленный в той или иной форме в тексте исследуемого языка;
			\item \uline{со знаком} той сущности, которая обозначается указанным выше знаком в рамках используемого метаязыка.
		\end{scnitemize}}
\scnnote{Данное отношение используется, когда с помощью одного языка необходимо описать денотационную семантику другого языка. Фактически речь идет о переводе заданного знака, входящего в состав некоторого рассматриваемого текста, принадлежащего некоторому исследуемому языку (языку-объекту), на некоторый метаязык (в нашем случае на SC-код), денотационная семантика которого нам считается априори известной. Указанный перевод есть связь заданного знака с синонимичным ему знаком, входящим в состав текста, принадлежащего указанному метаязыку.}
\scnrelboth{обратное отношение}{внешний sc-идентификатор*}
\scnaddlevel{1}
\scnidtf{быть знаком, обозначающим заданную сущность*}
\scnaddlevel{-1}
\scniselementlist{ключевой знак}{Описание внешних идентификаторов знаков, входящих в тексты внутреннего языка ostis-систем\\
	\scnaddlevel{1}
	\scnsourcecommentpar{Раздел 0.3.2}
	\scniselement{раздел Базы знаний IMS.ostis}
	\scnaddlevel{-1}
;Предметная область и онтология знаков, входящих в тексты внутреннего языка ostis-систем\\
	\scnaddlevel{1}
	\scnsourcecommentpar{Раздел 2.1.1.2}	
	\scniselement{раздел Базы знаний IMS.ostis}
	\scnaddlevel{-1}}
	
\scnheader{информационная конструкция}
\scnidtf{информация}
\scnnote{В общем случае информационная конструкция представляет собой сложную иерархическую структуру, каждому уровню иерархии которой соответствует определенный класс информационных конструкций}
\scnsuperset{синтаксически элементарный фрагмент информационной конструкции}
	\scnaddlevel{1}
	\scnidtf{атомарный фрагмент информационной конструкции}
	\scnidtf{элемент информационной конструкции}
	\scnnote{Примерами таких элементарных фрагментов информационных конструкций являются буквы}
	\scnsuperset{буква}
	\scnaddlevel{-1}
\scnsuperset{простой знак}
	\scnaddlevel{1}
	\scnidtf{семантически элементарный фрагмент информационной конструкции}
	\scnsubset{знак}
	\scnaddlevel{-1}
\scnsuperset{выражение}
\scnaddlevel{1}
	\scnidtf{сложный (непростой) знак}
	\scnidtf{знак, являющийся одновременно некоторым знанием обозначаемой сущности (спецификацией этой сущности)}
	\scnidtf{знак, построенный как выражение вида "тот, который..."{}}
	\scnidtf{знак, в состав которого входят другие знаки}
	\scnsubset{знак}
	\scnaddlevel{-1}
\scnsuperset{простой текст}
	\scnaddlevel{1}
	\scnidtf{минимальная синтаксически целостная и корректная (правильная) информационная конструкция, включающая в себя:
	\begin{scnitemize}
	\item знак некоторой описываемой связи;
	\item минимальную спецификацию указанного знака связи (указание отношения, которому это связь принадлежит);
	\item указание \uline{всех} компонентов описываемой связи (знаков всех сущностей, связываемых этой связью, и/или всех знаков, связываемых этой связью -- описываемая связь может связывать не только "внешние"{} описываемые сущности, но и сами знаки);
	\item если описываемая связь не является бинарной, то связи с её компонентами могут потребовать явного представления знаков этих связей с дополнительным указанием роли этих компонентов.
	\end{scnitemize}}
	\scnsubset{текст}
	\scnaddlevel{-1}
\scnsuperset{сложный текст}
	\scnaddlevel{1}
	\scnidtf{информационная конструкция, являющаяся результатом соединения нескольких простых текстов}
	\scnsubset{текст}
	\scnaddlevel{-1}
\scnsuperset{простое знание}
	\scnaddlevel{1}
	\scnidtf{минимальная семантические целостная информационная конструкция}
	\scnidtf{знание, в состав которого не входят другие знания}
	\scnsubset{знание}
	\scnaddlevel{-1}	
\scnsuperset{сложное знание}
	\scnaddlevel{1}
	\scnidtf{информационная конструкция, являющаяся результатом соединения нескольких простых знаний}
	\scnidtf{знание, в состав которого не входят другие знания}
	\scnsubset{знание}
	\scnaddlevel{-1}	
\scniselementrole{ключевой знак}{Предметная область и онтология информационных конструкций}
\scnaddlevel{1}
	\scnsourcecommentpar{Раздел 2.1.2.0}
\scnaddlevel{-1}
\end{SCn}
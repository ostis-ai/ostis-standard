\begin{SCn}

\bigskip
\scnfragmentcaption

\scnheader{нерафинированная семантическая сеть}
\scnnote{Переход от смыслового представления информации, не являющегося семантической сетью, к нерафинированным семантическим сетям представляет собой переход к информационным конструкциям, имеющим более простую синтаксическую структуру и денотационную семантику.\\
\newline
К нерафинированным семантическим сетям можно отнести тексты предлагаемого нами универсального формального SCg-кода, а также используемые в Semantic Web rdf-графы}

\scnsuperset{SCg-код}
\scnaddlevel{1}
\scnhaselement{универсальный формальный язык}
\scnhaselementrole{ключевой знак}{Описание языка графического представления знаний ostis-систем}
\scnaddlevel{-1}
\scnsuperset{rdf-граф}

\scnreltovector{принципы, лежащие в основе}{\scnfileitem{Поскольку в \textit{информационной конструкции} информация содержится не в самих \textit{знаках} (если не считать \textit{знаки}, являющиеся \textit{выражениями}), а в конфигурации связей между \textit{знаками}, очень важно \uline{явно} формально представить саму эту конфигурацию \textit{знаков}. И как нельзя лучше для этого подходит понятие \textit{графовой структуры} и, соответственно, понятие \textit{семантической сети}.\\
Что касается \textit{выражений}, то каждое из них легко трансформируется в \textit{семантически эквивалентную} информационную конструкцию, не являющиюся \textit{выражением}. Заметим, что \textit{выражения} используются исключительно для минимизации числа вводимых \textit{знаков} (имен) с уникальной синтаксической структурой.}
;\scnfileitem{\uline{Все} элементы, входящие в состав нерафинированной семантической сети и представленные узлами, ребрами или дугами, являются \textit{знаками}, обозначающими соответствующие описываемые \textit{сущности}, причём \textit{знаками} второго типа, которые, обозначая соответствующую \textit{сущность}, входят в \textit{информационную конструкцию} \uline{однократно} (отсутствует многократное вхождение \textit{знаков}, обозначающих одну и ту же \textit{сущность}). Также \textit{знаки} могут иметь синтаксическую структуру, которая не является уникальной для обозначаемой \textit{сущности}, а отражает только принадлежность этой сущности к соответствующих классам.
Таким образом, в \textit{нерафинированной семантической сети} в отличие от \textit{смыслового представления информации не являющегося семантической сетью}, доминируют не \textit{знаки} первого типа, а \textit{знаки} второго типа, которыми в \textit{нерафинированной семантической сети} представлены (обозначены) \uline{все} описываемые \textit{сущности}, а в \textit{смысловом представлении информации, не являющемся семанитеской сетью}, представлены \uline{только} \textit{бинарные связи} \uline{и то не все}.}
;\scnfileitem{\uline{Все} ребра \textit{нерафинированной семантической сети} являются знаками \textit{бинарных неориентированных связей} и формально трактуются как знаки \textit{двухмощных множеств}, каждым \textit{элементом} которых являются либо знак \textit{сущности}, соединяемой указанной \textit{бинарной связью}, либо \textit{знак}, который сам является \textit{сущностью}, соединяемой этой \textit{бинарной связью}. Более того, \uline{все} \textit{двухмощные множества}, не являющиеся \textit{кортежами} (ориентированными парами) в \textit{нерафинированной семантической сети} обозначаются \textit{ребрами} этой сети.}
;\scnfileitem{\uline{Все} дуги \textit{нерафинированной семантической сети} являются знаками \textit{бинарных ориентированных связей} и формально трактуются как знаки \textit{двухмощных кортежей} (ориентированных пар), каждым \textit{элементом} которых является либо знак \textit{сущности}, соединяемой указанной \textit{бинарной связи}, либо \textit{знак}, который сам является \textit{сущностью}, соединяемой этой \textit{бинарной связью}. Более того, \uline{все} \textit{ориентированные пары} в \textit{нерафинированной семантической сети} обозначаются \textit{дугами} этой сети.}
;\scnfileitem{\uline{Каждая} небинарная связь, описываемая в нерафинированной семантической сети, трактуется как множество, мощность которого не равна двум и обозначается соответствующим узлом этой сети, который соединяется дугами, принадлежащими отношению принадлежности со всеми знаками, которые либо обозначаются сущности, связывающие рассматриваемой небинарной связью, либо сами являются такими сущностями. Для описания ориентированных небинарных связей (в частности, небинарных кортежей) выделяется несколько подмножеств отношения принадлежности, соответствующих различным ролям элементов (компонентов) ориентированных небинарных связей.}
;\scnfileitem{В рамках нерафинированной семантической сети \uline{все} рассматриваемые связи между описываемыми сущностями представляются \uline{явно} в виде знаков, обозначающих эти связи.}
;\scnfileitem{В рамках нерафинированной семантической сети не используются такие средства, как разделители, ограничители и др.}
;\scnfileitem{Узлами \textit{нерафинированной семантической сети}, которые обозначают различного вида \uline{ключевые} описываемые \textit{сущности} (прежде всего, различные \textit{понятия}) приписываются уникальные \textit{знаки} (имена) этих \textit{ключевых сущностей}. Очевидно, что каждый такой \textit{узел} и приписываемое ему \textit{имя} -- это \textit{синонимичные знаки}, обозначающие одну и ту же \textit{сущность}, но являющиеся \textit{знаками} двух разных типов -- (1) \textit{знаком}, который \uline{однократно} представлен в рамках \textit{информационной конструкции}; (2) \textit{знаком}, синтаксическая структура которого \uline{взаимно однозначно} соответствует обозначаемой им \textit{сущности}.}
;\scnfileitem{Большинству узлов, обозначающих небинарные связи, большинству ребер и дуг, а также некоторым другим узлам нерафинированной семантической сети могут быть приписаны уникальные знаки (в частности, имена) понятий (чаще всего, отношений), которым принадлежат указанные узлы, ребра и дуги.}}
\end{SCn}

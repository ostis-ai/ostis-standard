
\begin{SCn}
	
\scnheader{Структуризация баз знаний ostis-сиситем}
\scnsubset{сегмент базы знаний}
\scnidtf{Структурная типология знаний ostis-системы}
\scntext{введение}{\textit{База знаний ostis-системы} имеет достаточно развитую иерархическую структуру. База знаний делится на разделы. Разделы бывают атомарными и неатомарными. Неатомарный раздел состоит из сегментов. Атомарные разделы не имеют сегментов. Разделы \textit{базы знаний ostis-системы} могут иметь самое различное назначение. Так, например, \textit{База знаний Метасистемы IMS.ostis} включает в себя:
\begin{scnitemize}
\item \textit{раздел}, содержащий текущее состояние постоянно пополняемого и совершенствуемого \textit{Стандарта} Технологии OSTIS;
\item \textit{раздел}, посвящённый описанию \textit{конечных пользователей и разработчиков} \textit{Метасистемы IMS.ostis}; 
\item \textit{раздел}, посвящённый описанию \textit{история эксплуатации  Метасистемы IMS.ostis};
\item \textit{раздел}, посвящённый описанию \textit{истории эволюции  Метасистемы IMS.ostis} (в т.ч. истории эволюции и её \textit{база знаний}); 
\item \textit{раздел} посвящённый описанию \textit{интеллектуальных компьютерных систем}, разработанных(порождённых) с помощью \textit{Метасистемы IMS.ostis}.
\end{scnitemize}}
\scnheader{выделенный фрагмент базы знаний}
\scnidtf{фрагмент базы знаний, для которого в \textit{базе знаний} вводится знак, обозначающий этот фрагмент, т.е. являющийся знаком множества \uline{всех} знаков, входящих в состав этого фрагмента. При представлении фрагмента базы знаний на внешних языках (SCg-коде, SCn-коде) указанный знак выделенного фрагмента базы знаний представляется либо в виде sc.g-контура, либо в виде пары фигурных скобок, ограничивающих текст обозначаемого фрагмента базы знаний}
\scnidtf{выделенный фрагмент базы знаний}
\scnaddlevel{1}
\scniselement{сокращённые sc-идентификатор} 
\scnaddlevel{-1}
\scnidtf{явно структурно выделенный фрагмент базы знаний}
\scnnote{явное выделение фрагмента базы знаний осуществляется:
\begin{scnitemize}
\item в \textit{SC-коде} путем введения \textit{знака}, обозначающего \textit{множество} \uline{всех} знаков, входящих в состав \textit{выделенного фрагмента базы знаний};
\item в \textit{SCg-коде} с помощью \textit{sc.g-контура}, ограничивающего sc.g-представление \textit{выделенного фрагмента базы знаний};
\item в \textit{SCn-коде} с помощью пары фигурных скобок, ограничивающих \textit{sc.n-представление} \textit{выделенного фрагмента базы знаний}.
\end{scnitemize}}
\scnsubdividing{семейство разделов базы знаний
\scnaddlevel{1}
\scnidtf{семантически целостное множество разделов базы знаний, имеющих достаточно сильные семантические связи между собой}
\scnaddlevel{-1}
;раздел базы знаний
\scnaddlevel{1}
\scnidtf{Основной (по семантической значимости) вид выделенных фрагментов баз знаний}\\
\scnaddlevel{-1}
\scnnote{В общем случае \textit{неатомарный раздел базы знаний} может иметь неограниченное число \textit{сегментов}. \textit{Сегменты базы знаний} не могут состоять из других сегментов (подсегментов). В этом смысле \textit{сегменты базы знаний} имеют атомарный характер.}
\scnidtf{раздел базы знаний ostis-системы}
\scnidtf{модуль (блок) база знаний}
\scnnote{В общем случае многим \textit{разделам базы знаний} ставятся в соответствие такие тексты, как \uline{предисловие},  \uline{введение}, \uline{заключение}, \uline{аннотация}, \uline{оглавление}, упражнения.\\ Некоторые из этих текстов могут иметь статус разделов.}
\scnaddlevel{1}
\scnsubdividing{неатомарный раздел базы знаний\\
\scnaddlevel{1}
\scnidtf{раздел базы знаний, состоящий из сегментов, декомпозируемый на сегменты} 
\scnaddlevel{-1}
;атомарный раздел базы знаний 
\scnaddlevel{1}
\scnidtf{раздел базы знаний, не содержащий сегментов}
\scnaddlevel{-1}}
\scnaddlevel{-1}
;сегмент базы знаний
\scnaddlevel{1}
\scnidtf{сегмент раздела базы знаний}
\scnidtf{сегмент базы знаний ostis-системы}
\scnidtf{структурно выделяемое sc-знание ostis-системы, структурный уровень которого ниже уровня разделов базы знаний}
\scnnote{Сегменты базы знаний не могут иметь иерархической структуры, т.е. не могут состоять из сегментов более низкого структурного уровня.}
\scnnote{Сегменты базы знаний входят в состав неатомарных разделов базы знаний.}
\scnaddlevel{-1}
;выделенный фрагмент сегмента или атомарного раздела базы знаний\\
\scnsubdividing{
выделенный фрагмент атомарного раздела базы знаний  
;выделенный фрагмент сегмента базы знаний}}
\scnsubset{sc-знание}
\scnnote{Каждый \textit{выделенный фрагмент базы знаний} представляет собой структурно оформленное (структурно выделенное) \textit{знание}, хранимое в \textit{базе знаний} \textit{интеллектуальной компьютерной системы}, (точнее, в \textit{базе знаний ostis-системы}) и представленное в формализованном виде на \textit{внутреннем языке представления знаний} (в \textit{SC-коде}) Представленное таким образом \textit{знание} будем называть \textit{sc-знанием}.}
\scnidtfexp{
\textit{информационная конструкция}, которая:
\begin{scnitemize}
\item принадлежит \textit{SC-коду};
\item является синтактически корректный;
\item обладает семантической целостностью -- отсутствием \textit{информационных дыр} ("недомолвок"{}), препятствующих её пониманию;
\item имеет нетривиальный \textit{объём информации} -- количество описываемых сущностей (в том числе, \textit{связей} и \textit{классов});
\item имеет достаточно высокое качество по другим характеристикам, в частности, достаточно высокую ценность.
\end{scnitemize}}
\scnnote{Типология \textit{sc-знаний}, в частности, по объему представленной информации определённым образом коррелирует с типологией \textit{выделенных фрагментов баз знаний} -- объём информации, содержащейся в \textit{разделах баз знаний}, должен быть приблизительно одинаковым, объём, содержащейся в разделе базы знаний, должен быть ниже объёма информации, содержащегося в любом семействе разделов базы знаний, и должен быть выше объёма информации, содержащегося в любом \textit{сегменте базы знаний}.\\
При этом в процессе эволюции базы знаний \textit{раздел базы знаний} может преобразоваться в семейства разделов, а \textit{сегмент базы знаний} может преобразоваться в \textit{раздел база знаний}.}
\scnheader{раздел база знаний}
\scnnote{Различные \textit{предметные области}, различные \textit{онтологии}, а также предметные области, объединённые с соответствующими им онтологиями, являются важнейшими видами \textit{sc-знаний ostis-систем}, обеспечивающими логически стройную систематизацию \textit{знаний ostis-систем} и, соответственно семантическую структуризацию \textit{баз знаний}. При этом указанные типы знаний обычно представляются в виде \textit{разделов базы знаний}, иерархия которых, задаваемая Отношением \textit{частная предметная область*} или Отношением \textit{частная предметная область и онтология*}, соответствует логико-семантической иерархии \textit{предметных областей}. Но, кроме \textit{разделов базы знаний}, являющихся \textit{предметными областями}, \textit{онтологиями}, \textit{предметными областями}, объединенными с соответствующими им онтологиями, вводится и целый ряд других семантических типов \textit{разделов базы знаний}, определяемых характером соотношения разделов базы знаний с используемыми (рассматриваемыми) предметными областями и онтологиями.}
\scnsuperset{предметная область}\\
фрагмент предметной области\\
\scnsuperset{интегрированная онтология}
\scnsuperset{частная онтология}
\scnsuperset{предметная область и онтология}
\scnaddlevel{1}
\scnidtf{предметная область, объединенная (интегрированная) с соответствующей ей онтологией}
\scnidtf{предметная область вместе с онтологией, которая её специфицирует}
\scnaddlevel{-1}
\scnnote{Если каждому \textit{разделу базы знаний} ostis-системы будет четко соответствовать его семантический тип, то к "синтаксическим"{} связям между \textit{разделами базы знаний} добавится большое количество "осмысленных"{} (семантические интерпретируемых) связей, определяющих "семантическое местоположение"{} ("семантические координаты"{}) каждого \textit{раздела базы знаний} во множестве всех разделов, входящих в состав базы знаний ostis-системы.}
\scnnote{В основе представления \textit{базы знаний ostis-системы} лежат развитые средства семантической структуризации баз знаний и семантической систематизации баз знаний \textit{ostis-систем}. Можно выделить следующие уровни систематизации элементов и фрагментов смыслового пространства, построенного на основе \textit{SC-кода}:
\begin{scnitemize}
\item уровень знаков всевозможных сущностей (уровень \textit{sc-элементов});
\item уровень вводимых \textit{понятий}, обозначающих ключевые (исследуемые) в предметных областях классы сущностей;
\item уровень \textit{высказываний}, описывающих закономерности (свойства) экземпляров исследуемых классов сущностей (исследуемых понятий);
\item уровень \textit{предметных областей}, \textit{онтологий} и разделов, семантический тип которых известен.
\end{scnitemize}}
\end{SCn}

\begin{SCn}
	
\scnsegmentheader{Структуризация баз знаний ostis-систем}

\scnstartsubstruct

\scnsubset{сегмент базы знаний}
\scnidtf{Структурная типология знаний ostis-системы}
\scntext{введение}{\textit{База знаний ostis-системы} имеет достаточно развитую иерархическую структуру. База знаний делится на разделы. Разделы бывают атомарными и неатомарными. Неатомарный раздел состоит из сегментов. Атомарные разделы не имеют сегментов. Разделы \textit{базы знаний ostis-системы} могут иметь самое различное назначение. Так, например, \textit{База знаний Метасистемы IMS.ostis} включает в себя:
\begin{scnitemize}
\item \textit{раздел}, содержащий текущее состояние постоянно пополняемого и совершенствуемого \textit{Стандарта} Технологии OSTIS;
\item \textit{раздел}, посвящённый описанию \textit{конечных пользователей и разработчиков} \textit{Метасистемы IMS.ostis}; 
\item \textit{раздел}, посвящённый описанию \textit{история эксплуатации  Метасистемы IMS.ostis};
\item \textit{раздел}, посвящённый описанию \textit{истории эволюции  Метасистемы IMS.ostis} (в т.ч. истории эволюции и её \textit{база знаний}); 
\item \textit{раздел} посвящённый описанию \textit{интеллектуальных компьютерных систем}, разработанных(порождённых) с помощью \textit{Метасистемы IMS.ostis}.
\end{scnitemize}}

\scnheader{выделенный фрагмент базы знаний}
\scnidtf{фрагмент базы знаний, для которого в \textit{базе знаний} вводится знак, обозначающий этот фрагмент, т.е. являющийся знаком множества \uline{всех} знаков, входящих в состав этого фрагмента. При представлении фрагмента базы знаний на внешних языках (SCg-коде, SCn-коде) указанный знак выделенного фрагмента базы знаний представляется либо в виде sc.g-контура, либо в виде пары фигурных скобок, ограничивающих текст обозначаемого фрагмента базы знаний}
\scnidtf{выделенный фрагмент базы знаний}
\scnaddlevel{1}
\scniselement{сокращённые sc-идентификатор} 
\scnaddlevel{-1}
\scnidtf{явно структурно выделенный фрагмент базы знаний}
\scnnote{явное выделение фрагмента базы знаний осуществляется:
\begin{scnitemize}
\item в \textit{SC-коде} путем введения \textit{знака}, обозначающего \textit{множество} \uline{всех} знаков, входящих в состав \textit{выделенного фрагмента базы знаний};
\item в \textit{SCg-коде} с помощью \textit{sc.g-контура}, ограничивающего \textit{sc.g-представление} \scnbigspace \textit{выделенного фрагмента базы знаний};
\item в \textit{SCn-коде} с помощью пары фигурных скобок, ограничивающих \textit{sc.n-представление} \scnbigspace \textit{выделенного фрагмента базы знаний}.
\end{scnitemize}}
\scnsubdividing{семейство разделов базы знаний
\scnaddlevel{1}
\scnidtf{семантически целостное множество разделов базы знаний, имеющих достаточно сильные семантические связи между собой}
\scnaddlevel{-1}
;раздел базы знаний
\scnaddlevel{1}
\scnidtf{Основной (по семантической значимости) вид выделенных фрагментов баз знаний}\\
\scnidtf{раздел базы знаний ostis-системы}
\scnidtf{модуль (блок) базы знаний}
\scnnote{В общем случае многим \textit{разделам базы знаний} ставятся в соответствие такие тексты, как \uline{предисловие},  \uline{введение}, \uline{заключение}, \uline{аннотация}, \uline{оглавление}, упражнения.\\ Некоторые из этих текстов могут иметь статус разделов.}
\scnsubdividing{неатомарный раздел базы знаний\\
\scnaddlevel{1}
\scnidtf{раздел базы знаний, состоящий из сегментов, декомпозируемый на сегменты} 
\scnnote{В общем случае \textit{неатомарный раздел базы знаний} может иметь неограниченное число \textit{сегментов}. \textit{Сегменты базы знаний} не могут состоять из других сегментов (подсегментов). В этом смысле \textit{сегменты базы знаний} имеют атомарный характер.}
\scnaddlevel{-1}
;атомарный раздел базы знаний 
\scnaddlevel{1}
\scnidtf{раздел базы знаний, не содержащий сегментов}
\scnaddlevel{-1}}
\scnaddlevel{-1}
;сегмент базы знаний
\scnaddlevel{1}
\scnidtf{сегмент раздела базы знаний}
\scnidtf{сегмент базы знаний ostis-системы}
\scnidtf{структурно выделяемое sc-знание ostis-системы, структурный уровень которого ниже уровня разделов базы знаний}
\scnnote{Сегменты базы знаний не могут иметь иерархической структуры, т.е. не могут состоять из сегментов более низкого структурного уровня.}
\scnnote{Сегменты базы знаний входят в состав неатомарных разделов базы знаний.}
\scnaddlevel{-1}
;выделенный фрагмент сегмента или атомарного раздела базы знаний\\
\scnaddlevel{1}
\scnsubdividing{выделенный фрагмент атомарного раздела базы знаний;выделенный фрагмент сегмента базы знаний}
\scnsubset{sc-знание}
\scnnote{Каждый \textit{выделенный фрагмент базы знаний} представляет собой структурно оформленное (структурно выделенное) \textit{знание}, хранимое в \textit{базе знаний} \textit{интеллектуальной компьютерной системы}, (точнее, в \textit{базе знаний ostis-системы}) и представленное в формализованном виде на \textit{внутреннем языке представления знаний} (в \textit{SC-коде}) Представленное таким образом \textit{знание} будем называть \textit{sc-знанием}.}
\scnaddlevel{1}
\scnidtfexp{
	\textit{информационная конструкция}, которая:
	\begin{scnitemize}
		\item принадлежит \textit{SC-коду};
		\item является синтактически корректный;
		\item обладает семантической целостностью -- отсутствием \textit{информационных дыр} ("недомолвок"{}), препятствующих её пониманию;
		\item имеет нетривиальный \textit{объём информации} -- количество описываемых сущностей (в том числе, \textit{связей} и \textit{классов});
		\item имеет достаточно высокое качество по другим характеристикам, в частности, достаточно высокую ценность.
\end{scnitemize}}
\scnnote{Типология \textit{sc-знаний}, в частности, по объему представленной информации определённым образом коррелирует с типологией \textit{выделенных фрагментов баз знаний} -- объём информации, содержащейся в \textit{разделах баз знаний}, должен быть приблизительно одинаковым, объём, содержащейся в разделе базы знаний, должен быть ниже объёма информации, содержащегося в любом семействе разделов базы знаний, и должен быть выше объёма информации, содержащегося в любом \textit{сегменте базы знаний}.\\
	При этом в процессе эволюции базы знаний \textit{раздел базы знаний} может преобразоваться в семейства разделов, а \textit{сегмент базы знаний} может преобразоваться в \textit{раздел базы знаний}.}
\scnaddlevel{-2}}

\scnheader{раздел базы знаний}
\scnnote{Различные \textit{предметные области}, различные \textit{онтологии}, а также предметные области, объединённые с соответствующими им онтологиями, являются важнейшими видами \textit{sc-знаний ostis-систем}, обеспечивающими логически стройную систематизацию \textit{знаний ostis-систем} и, соответственно семантическую структуризацию \textit{баз знаний}. При этом указанные типы знаний обычно представляются в виде \textit{разделов базы знаний}, иерархия которых, задаваемая Отношением \textit{частная предметная область*} или Отношением \textit{частная предметная область и онтология*}, соответствует логико-семантической иерархии \textit{предметных областей}. Но, кроме \textit{разделов базы знаний}, являющихся \textit{предметными областями}, \textit{онтологиями}, \textit{предметными областями}, объединенными с соответствующими им онтологиями, вводится и целый ряд других семантических типов \textit{разделов базы знаний}, определяемых характером соотношения разделов базы знаний с используемыми (рассматриваемыми) предметными областями и онтологиями.}
\scnsuperset{предметная область}
\scnsuperset{фрагмент предметной области}
\scnsuperset{интегрированная онтология}
\scnsuperset{частная онтология}
\scnsuperset{предметная область и онтология}
\scnaddlevel{1}
\scnidtf{предметная область, объединенная (интегрированная) с соответствующей ей онтологией}
\scnidtf{предметная область вместе с онтологией, которая её специфицирует}
\scnaddlevel{-1}
\scnnote{Если каждому \textit{разделу базы знаний} ostis-системы будет четко соответствовать его семантический тип, то к "синтаксическим"{} связям между \textit{разделами базы знаний} добавится большое количество "осмысленных"{} (семантические интерпретируемых) связей, определяющих "семантическое местоположение"{} ("семантические координаты"{}) каждого \textit{раздела базы знаний} во множестве всех разделов, входящих в состав базы знаний ostis-системы.}
\scnnote{В основе представления \textit{базы знаний ostis-системы} лежат развитые средства семантической структуризации баз знаний и семантической систематизации баз знаний \textit{ostis-систем}. Можно выделить следующие уровни систематизации элементов и фрагментов смыслового пространства, построенного на основе \textit{SC-кода}:
\begin{scnitemize}
\item уровень знаков всевозможных сущностей (уровень \textit{sc-элементов});
\item уровень вводимых \textit{понятий}, обозначающих ключевые (исследуемые) в предметных областях классы сущностей;
\item уровень \textit{высказываний}, описывающих закономерности (свойства) экземпляров исследуемых классов сущностей (исследуемых понятий);
\item уровень \textit{предметных областей}, \textit{онтологий} и разделов, семантический тип которых известен.
\end{scnitemize}}

\scnidtf{раздел б.з.}
\scnaddlevel{1}
\scniselement{сокращенный sc-идентификатор}
\scnaddlevel{-1}
\scnexplanation{Множество \textit{разделов баз знаний} имеет:
	\begin{scnitemize}
		\item богатую семантическую типологию;
		\item богатый набор отношений, описывающих семантические связи между разделами.
	\end{scnitemize}
	Синтаксически \textit{разделы баз знаний} могут пересекаться (иметь общие элементы), но никогда один раздел не может включаться (полностью входить в состав) другого раздела. В этом смысле понятие подраздела (точнее, \textit{частного раздела}*) имеет не "синтаксический"{} смысл, а семантический -- глубокое наследование свойств при достаточно большой степени "независимости"{} друг от друга}
\scnnote{Важным свойством \textit{раздела базы знаний} \scnbigspace \textit{ostis-системы} является его семантическая целостность -- наличие достаточно стабильного набора классов исследуемых сущностей (\textit{объектов исследования}) и достаточно стабильного семейства \textit{отношений} и семейства \textit{параметров}, заданных на различных классах объектов исследования, а также семейства \textit{классов структур} использующих указанные выше понятия (введенные \textit{классы объектов исследования}, введенные \textit{отношения} и \textit{классы структур}). Такая целостность дает возможность развивать \textit{раздел базы знаний}, не выходя "за рамки"{} используемой системы \textit{понятий}. Это позволяет развивать каждый \textit{раздел базы знаний} в известной мере \uline{независимо} от других разделов, что существенно повышает \uline{гибкость} и \uline{стратифицированность} \textit{базы знаний}.}	
\scnexplanation{Основными свойствами \textit{раздела базы знаний} как структурно \textit{выделяемого фрагмента базы знаний} являются следующие:
	\begin{scnitemize}
		\item семантическая целостность -- наличие четкого критерия, позволяющего установить для каждого конкретного знания то, включать или не включать это знание в состав данного раздела;
		\item потенциальная возможность эволюционировать в достаточной степени независимо от других разделов, но при условии соблюдения всех требований, обеспечивающих постоянную поддержку \uline{семантической совместимости} данного раздела со всеми остальными семантически смежными разделами.
\end{scnitemize}}
\scnaddlevel{1}
\scntext{следовательно}{Грамотная декомпозиция \textit{базы знаний} на разделы, основанная на четкой стратификации процесса эволюции накапливаемой человечеством общечеловеческой объединенной \textit{базы знаний}, сутью которой является \uline{минимизация} трудоемкости усилий по согласованию и обеспечению с авторами других разделов \textit{семантической совместимости} со смежными разделами, создает предпосылки высоких темпов эволюции \textit{базы знаний} в целом.}
\scnaddlevel{-1}
\scnnote{Любой \textit{раздел базы знаний} не является структурной частью другого раздела. Каждый раздел самодостаточен и целостен. Это обеспечивается тем, что в состав \textit{титульной спецификации} каждого раздела входит \textit{семантическая окрестность},описывающая связи специфицируемого раздела \uline{со всеми} семантически близкими ему разделами.\\
	При этом разные разделы могут иметь разный семантический тип:
	\begin{scnitemize}
		\item раздел может быть предметной областью, интегрированной со всеми ее онтологиями;
		\item раздел может быть просто предметной областью;
		\item раздел может быть какой-либо онтологией чего угодно (не обязательно предметной области);
		\item и т.д.
\end{scnitemize}}
\scnnote{Если в \textit{титульную спецификацию} каждого раздела будет входить семантическая спецификация каждого раздела, включающая его семантические связи со всеми семантически близкими разделами, то последовательность (порядок) разделов в "линейном"{} исходном тексте, публикуемом в качестве очередной версии Стандарта OSTIS, может быть в достаточной степени \uline{произвольной}.}


\scnheader{семейство разделов базы знаний}
\scnidtf{множество семантически связанных друг с другом разделов базы знаний}
\scnidtf{кластер разделов базы знаний}
\scnnote{Семантические связи между разделами, входящими в состав семейства разделов, представляются в рамках \textit{титульных спецификаций разделов}, каждая из которых является специальной частью соответствующего (специфицируемого) раздела, входящего в состав семейства разделов.
	
	Для каждого \textit{раздела базы знаний} в рамках его титульной спецификации формируется \textit{семантическая окрестность} его связей со всеми семантически близкими ему разделами (и, прежде всего, с теми разделами, которые входят в состав тех семейств разделов, в которые входит заданный раздел). При этом акцентируется внимание именно на семантических связях между разделами. Так, например, вместо структурной ("синтаксической"{}) связи ``быть подразделом*''{} (т.е. быть частью заданного раздела) вводится связь ``быть \textit{дочерним разделом}*''{}. \\
	Данная связь указывает направление наследования свойств исследуемых объектов заданного раздела от разделов, исследующих более общие классы объектов.
	Порядок (последовательность) разделов в рамках \textit{семейства разделов базы знаний} при наличии \uline{явно представленных} семантических связей между разделами, входящими в семейство разделов, может быть достаточно \uline{произвольным}, что очень важно, например, при формировании оглавления очередной издаваемой версии \textit{Стандарта OSTIS}. Таким образом, трактовка \textit{Стандарта OSTIS}, а также всех издаваемых версий этого Стандарта как \textit{семейства разделов базы знаний} \scnbigskip \textit{Метасистемы IMS.ostis} обеспечивает высокий уровень гибкости \textit{Стандарта OSTIS}, а также легкость "переиздаваемости"{} его версий.}


\scnheader{сегмент или атомарный раздел базы знаний}
\scnnote{Простейшей формой \textit{сегмента} или \textit{атомарного раздела базы знаний} является просто последовательность \textit{файлов ostis-системы}. Некоторые из этих файлов могут быть идентифицированными (именованными), если на них ссылаются другие файлы, а некоторые из них могут быть связаны с другими файлами различными отношениями (в частности, один файл может быть пояснением другого). Кроме того, некоторые из этих файлов могут быть формально специфицированы (например, указаны соответствующие им ключевые \textit{sc-элементы}).\\
	В самом простом случае \textit{сегмент} или \textit{атомарный раздел базы знаний} может быть \textit{sc-структурой}, состоящей из \uline{одного} (!) \textit{sc-узла}, обозначающего \textit{файл ostis-системы} (чаще всего, \textit{ея-файл ostis-системы}). Т.е. сам \textit{файл ostis-системы} может быть \textit{знанием ostis-системы}, но не может быть структурно \textit{выделяемым} \textit{фрагментом базы знаний} ostis-системы. При этом \textit{sc-узел}, обозначающий \textit{файл ostis-системы}, являющийся \textit{знанием}, может быть единственным \textit{sc-элементом} структурно выделяемого \textit{знания ostis-системы}.}
\scnnote{Для наглядного отображения (визуализации) \textit{сегмента} или \textit{атомарного раздела базы знаний ostis-систем\textit{ы} целесообразно представить указанное \textit{sc-знание} в виде конкатенации (последовательности) таких }sc-знаний, которые, во-первых, были бы достаточно крупными и логико-семантически значимыми для соответствующего \textit{сегмента} или \textit{атомарного раздела базы знаний ostis-системы} и, во-вторых, для которых существовал бы алгоритм \uline{однозначного} (!) размещения (на экране) внешнего представления этих \textit{sc-знаний} (в \textit{SCg-коде} или в \textit{SCn-коде}).\\
	Однозначность здесь означает наличие легко усваиваемого пользователями стандартного \uline{стиля визуализации} \textit{sc-знаний} и заключается в том, что многократная визуализация одного и того же \textit{sc-знания} с помощью указанного алгоритма должна приводить к синтаксически эквивалентным, а в случае \textit{SCg-кода} и к геометрически конгруэнтным текстам. Очевидно, что для произвольных \textit{sc-знаний} большого объёма такого алгоритма не существует, но для \textit{sc-знаний}, содержащих описание собственной структуры и семантической типологии собственных фрагментов, разработка такого алгоритма вполне реальна при наличии достаточного количества указанных \textit{метазнаний} о структуре отображаемых (визуализируемых) \textit{sc-знаний}.}


\scnheader{sc-идентификатор выделенного фрагмента базы знаний}
\scnidtf{название (имя) выделенного фрагмента базы знаний}
\scnexplanation{Не следует путать объект описания (спецификации) и само описание. Поэтому в \textit{sc-идентфикаторе} фрагмента базы знаний должны присутствовать слова, указывающие на семантический или структурный тип именуемого фрагмента базы знаний (описание, спецификация, анализ, сравнительный анализ, сравнение, определение, раздел, предметная область, онтология и т.п.).
	
	Таким образом, \textit{sc-идентификатор выделенного фрагмента базы знаний} ostis-системы должен иметь \uline{явное} (!) указание на то, что он является обозначением именно фрагмента базы знаний, а не того, что описывается в этом фрагменте.}
\scnnote{Мы не будем использовать такой изменчивый для нас способ идентификации разделов \textit{Стандарта OSTIS}, как нумерацию этих разделов, поскольку, например, в разных издаваемых официальных версиях \textit{Стандарта OSTIS} одному и тому же разделу \textit{Стандарта OSTIS} могут соответствовать разные номера.}


\scnheader{выделенный фрагмент базы знаний}
\scntext{основной sc-идентификатор}{выделенный фрагмент базы знаний}
\scnaddlevel{1}
\scntext{используемая аббревиатура}{выделенный фр-нт б.з.}
\scnaddlevel{-1}
\scnsubdividing{именованный фрагмент базы знаний\\
	\scnaddlevel{1}
	\scnidtf{\textit{выделенный фрагмент базы знаний}, имеющий \textit{sc-идентификатор} (имя, название)}
	\scnnote{\textit{Именованными фрагментами баз знаний} могут быть только структурно \textit{выделенные фрагменты баз знаний}}
	\scnnote{Все \textit{семейства разделов баз знаний}, все \textit{разделы баз знаний} и все \textit{сегменты баз знаний} должны быть именованными}
	\scnsuperset{семейство разделов базы знаний}
	\scnsuperset{раздел базы знаний}
	\scnsuperset{сегмент базы знаний}
	\scnaddlevel{-1}
	;неименованный фрагмент базы знаний\\
	\scnaddlevel{1}
	\scnidtf{\textit{выделенный фрагмент базы знаний}, \uline{не} имеющий \textit{sc-идентификатора} (имени, названия)}
	\scnnote{\textit{неименованными фрагментами баз знаний} могут быть только \textit{выделенные фрагменты сегментов баз знаний} либо выделенные фрагменты таких \textit{разделов баз знаний}, которые не состоят из \textit{сегментов}}
	\scnaddlevel{-1}}

\scnheader{титульная спецификация выделенного фрагмента базы знаний}
\scnexplanation{\textit{Титульная спецификация выделенного фрагмента базы знаний} ostis-системы представляет собой \textit{sc-структуру}, описывающую свойства специфицируемого знания и включающую в себя: 
	\begin{scnitemize}
		\item связи принадлежности специфицируемого знания соответствующим классам \textit{знаний ostis-систем};
		\item связи, указывающие логически предшествующее и логически следующее \textit{знание ostis-системы};
		\item связь, описывающую декомпозицию специфицируемого знания на последовательность знаний более низкого структурного уровня (декомпозицию разделов на сегменты);
		\item различного вида связи с другими \textit{знаниями ostis-систем}, которые сами "целиком"{} входят в состав спецификации специфицируемого знания (такими знаниями могут быть аннотации, предисловия, введения, оглавления, заключения);
		\item различного вида связи с другими \textit{знаниями ostis-систем}, которые сами не входят в состав спецификации специфицируемого знания (такого рода связями могут быть связи \textit{семантической близости} специфицируемого знания с другими знаниями, связи \textit{семантической эквивалентности}, связи\textit{семантического включения}, связи \textit{противоречивости знаний});
		\item связи, указывающие различного вида \textit{ключевые sc-элементы} (ключевые знаки), соответствующие специфицируемому знанию;
		\item связи специфицируемого знания с авторским коллективом, коллективом рецензентов, с датой его последнего обновления;
		\item для каждого нового целостного фрагмента, вводимого в состав \textit{базы знаний}, в истории эволюции этой \textit{базы знаний} указываются:
		\begin{scnitemizeii}
			\item \textit{автор*} или \textit{авторы*} первой версии этого фрагмента;
			\item отметка времени появления (дата-час-минута) всех версий этого фрагмента (в том числе и окончательно утверждённой, согласованной версии, которая, собственно, и становится фрагментом, включенным в согласованную часть базы знаний);
			\item \textit{рецензии*} (замечания к доработке) всех предварительных версий разрабатываемого \textit{фрагмента базы знаний};
			\item \textit{авторы*} всех указанных рецензий;
			\item отметка времени появления всех указанных рецензий;
			\item события по одобрению, утверждению различных предварительных версий разрабатываемого \textit{фрагмента базы знаний} различными рецензентами и экспертами с указанием отметки времени появления этих событий;
			\item темпоральная последовательность предварительных версий.
		\end{scnitemizeii}
	\end{scnitemize}
}
\scnnote{Знак такой спецификации явно не вводится, а сама эта спецификация непосредственно входит в состав специфицируемого фрагмента и включает в себя аннотацию, предисловие, авторов, ключевые знаки, декомпозицию специфицируемого фрагмента базы знаний и прочее}
\scnsubdividing{титульная спецификация раздела базы знаний;
	титульная спецификация семейства разделов базы знаний;
	титульная спецификация сегмента базы знаний;
	титульная спецификация выделенного фрагмента сегмента или атомарного раздела базы знаний}
\scnexplanation{\textit{Титульная спецификация выделенного фрагмента базы знаний} содержит общую информацию об этом фрагменте, является непосредственно \uline{частью} специфицируемого фрагмента \textit{базы знаний} и при этом сама \uline{не является} явно \textit{выделенным фрагментом базы знаний}}
\scniselement{спецификация}
\scnidtf{основная \textit{метаинформация} (основное \textit{метазнание}) о \textit{выделенном фрагменте базы знаний} -- о его структуре, \textit{авторах\scnrolesign}, \textit{ключевых знаках\scnrolesign} и т.д.}
\scnexplanation{\uline{неявно} \textit{выделяемый фрагмент базы знаний}, который:
	\begin{scnitemize}
		\item не имеет "собственного"{} ограничителя ("собственного"{} контура или "собственных"{} ограничивающих фигурных скобок);
		\item является семантической спецификацией соответствующего \uline{явно} выделяемого фрагмента базы знаний;
		\item является непосредственной \uline{частью} специфицируемого фрагмента базы знаний
\end{scnitemize}}
\scnnote{В \textit{sc.n-тексте} титульная спецификация фрагмента базы знаний размещается сразу после фигурной скобки, открывающей этот фрагмент}

\scnheader{титульная спецификация раздела базы знаний}
\scnnote{\textit{титульная спецификация раздела базы знаний} должна включать в себя достаточно подробное описание семантических свойств этого раздела и, в частности, подробное описание его связей с другими семантически близкими разделами. Это необходимо для обеспечения автономности разделов баз знаний.}

\scnheader{титульная спецификация семейства разделов базы знаний}
\scnnote{Если \textit{разделы базы знаний} являются семантически \uline{ключевыми} \textit{выделенными фрагментами баз знаний}, определяющими спецификацию систем используемых понятий и направления наследования свойств, то \textit{семейства разделов баз знаний} являются \uline{ключевыми} для структуризации виртуальной \textit{Базы знаний Экосистемы OSTIS}, для обмена \textit{знаниями} между различными субъектами \textit{Экосистемы OSTIS}.\\
	Поэтому типология \textit{семейств разделов баз знаний} и качество \textit{титульной спецификации семейств разделов баз знаний} имеют большое значение.}

\scnheader{титульная спецификация выделенного фрагмента базы знаний}
\scnrelfrom{множество используемых понятий}{Множество понятий используемых в титульных спецификациях выделенных фрагментов баз знаний}
\scnaddlevel{1}
\scnsuperset{класс выделенных фрагментов}
\scnaddlevel{1}
\scnhaselement{семейство разделов базы знаний}
\scnhaselement{раздел базы знаний}
\scnhaselement{неатомарный раздел базы знаний}
\scnhaselement{атомарный раздел базы знаний}
\scnhaselement{сегмент базы знаний}
\scnhaselement{выделенный фрагмент сегмента или атомарного раздела базы знаний}
\scnhaselement{выделенный фрагмент атомарного раздела базы знаний}
\scnhaselement{выделенный фрагмент сегманта базы знаний}
\scnaddlevel{-1}
\scnsuperset{отношение, связывающее выделенные фрагменты баз знаний с персонами}
\scnaddlevel{1}
\scnhaselement{автор*}
\scnhaselement{рецензент*}
\scnhaselement{эксперт*}
\scnhaselement{технический редактор*}
\scnhaselement{консультант*}
\scnaddlevel{1}
\scnidtf{активный участник обсуждения вопросов, рассматриваемых в специфицируемом фрагменте базы знаний*}
\scnaddlevel{-2}
\scnsuperset{отношение, связывающее выделенные фрагменты баз знаний с ея-файлами}
\scnaddlevel{1}
\scnhaselement{аннотация*}
\scnhaselement{предисловие*}
\scnaddlevel{1}
\scnidtf{Бинарное ориентированное отношение, каждая пара которого связывает:
	\begin{scnitemize}
		\item знак некоторого информационного ресурса (в частности, раздела базы знаний или раздела опубликованного документа);
		\item знак информационной конструкции, описывающей цели создания указанного информационного ресурса, предысторию его создания, планируемые направления дальнейшего развития, состав авторов и др.
\end{scnitemize}}
\scnaddlevel{-1}
\scnhaselement{введение*}
\scnhaselement{эпиграф*}
\scnhaselement{заключение*}
\scnhaselement{рассматриваемый вопрос*}
\scnhaselement{основные положения*}
\scnhaselement{вопрос для самопроверки*}
\scnhaselement{упражнение*}
\scnaddlevel{1}
\scnidtf{задача*}
\scnidtf{самостоятельная (индивидуальная) работа*}
\scnaddlevel{-1}
\scnhaselement{коллективный проект*}
\scnhaselement{неосновной sc-идентификатор*}
\scnaddlevel{1}
\scnnote{неосновным sc-идентификатором, в частности, может быть альтернативное название выделенного (специфицируемого) фрагмента базы знаний}
\scnaddlevel{-1}
\scnhaselement{часто используемый sc-идентификатор*}
\scnhaselement{сокращенный sc-идентификатор*}
\scnhaselement{используемое сокращение*}
\scnaddlevel{1}
\scnidtf{сокращение, используемое в специфицируемом фрагменте базы знаний при построении sc-идентификаторов, а также при оформлении ея-файлов*}
\scnaddlevel{-1}
\scnhaselement{библиографический источник, отражающий аналогичную точку зрения*}
\scnhaselement{библиографический источник, отражающий альтернативную точку зрения*}
\scnhaselement{библиографический источник, дополняющий данную точку зрения*}
\scnhaselement{сокращение*}
\scnaddlevel{1}
\scnidtf{Бинарное ориентированное отношение, каждая пара которого связывает естественно-языковую фразу с ее сокращенной записью*}
\scnaddlevel{-2}
\scnsuperset{отношение, описывающее структурные или семантические связи и между выделенными фрагментами баз знаний}
\scnaddlevel{1}
\scnhaselement{конкатенация сегментов*}
\scnhaselement{предыдущий сегмент*}
\scnaddlevel{1}
\scnidtf{предыдущий сегмент в рамках соответствующего раздела*}
\scnaddlevel{-1}
\scnhaselement{следующий сегмент*}

\scnhaselement{дочерний фрагмент базы знаний*}
\scnaddlevel{1}
\scnsuperset{дочерний раздел базы знаний*}
\scnsuperset{дочерняя предметная область*}
\scnsuperset{дочерняя предметная область и онтология*}
\scnnote{Для фрагмента базы знаний важно указать не только дочерние по отношению к нему фрагменты базы знаний, но и те фрагменты базы знаний, по отношению к которым данный фрагмент базы знаний является дочерним}
\scnaddlevel{-2}

\scnsuperset{отношение, описывающее ролевой статус знаков, входящих в состав выделенных фрагментов баз знаний}
\scnaddlevel{1}
\scnhaselement{ключевой знак*}
\scnhaselement{ключевой знак первого плана*}
\scnhaselement{ключевой знак второго плана*}
\scnhaselement{ключевой объект исследования*}
\scnhaselement{ключевое понятие*}
\scnhaselement{ключевой класс объектов исследования*}
\scnhaselement{исследуемое отношение*}
\scnhaselement{исследуемый параметр*}
\scnhaselement{исследуемый класс структур*}
\scnaddlevel{-2}

\bigskip

\scnendstruct \scnendsegmentcomment{Структуризация баз знаний ostis-систем}

\end{SCn}

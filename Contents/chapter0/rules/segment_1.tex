
\scsection{Описание общих принципов оформления внутреннего и внешнего представления информационных конструкций в ostis-системах}
\label{intro_rules}

\begin{SCn}

\scnsectionheader{\currentname}

\scnstartsubstruct

\scnreltovector{конкатенация сегментов}{
Формальная типология информационных конструкций, хранимых в памяти ostis-системы
;***
;***
;***
;***
;***
;***}

\scnsegmentheader{Формальная типология информационных конструкций, хранимых в памяти ostis-системы}

\scnstartsubstruct
\scnheader{информационная конструкция}
\scnsubdividing{знак\\
\scnaddlevel{1}
\scnidtf{семантически и структурно элементарный фрагмент информационной конструкции}
\scnidtf{элементарные атомарная информационная конструкция}
\scnidtf{информационная конструкция, состоящая из одного знака}
\scnaddlevel{-1}
;знаковая конструкция\\
\scnaddlevel{1}
\scnidtf{информационная конструкция, состоящая из \uline{нескольких} знаков} 
\scnaddlevel{-1}
}
\scnrelfrom{покрытие}{язык}
\scnaddlevel{1}
\scnidtf{множество (класс) \textit{информационных конструкций},
\begin{scnitemize}
		\item которому ставится в соответствие
		\begin{scnitemizeii}
			\item семейство \uline{синтаксических} правил построения \textit{информационных конструкций}, принадлежащих этому \textit{множеству}, а также
			\item семейство \uline{семантических} правил построения \textit{информационных конструкций} этого \textit{множества}
		\end{scnitemizeii}
		\item и которому \textit{принадлежат} не только синтаксически и семантически правильно построенные (корректные) \textit{информационные конструкции}, но и также неправильно построенные (некорректные, ошибочные) \textit{информационные конструкции}, которые, тем не менее, дают возможность их семантически интерпретировать (понимать) и, по крайней мере, локализовать допущенные в них ошибки
\end{scnitemize}}
\scnaddlevel{1}
\scnnote{Очень важно обеспечить эффективный обмен \textit{информацией} не только с \uline{грамотными} \textit{носителями языков}, но также и с не очень грамотными, которые допускают большое количество языковых ошибок.}
\scnaddlevel{-1}
\scnnote{Поскольку теоретически некоторые \textit{информационные конструкции} могут принадлежать одновременно нескольким \textit{языкам} и поскольку понятие синтаксической и семантической корректности \textit{информационных конструкций} определяются правилами соответствующих \textit{языков} (их \textit{синтаксисом} и \textit{денотационной семантикой}), для уточнения понятий структурного уровня и корректности (правильности) \textit{информационных конструкций} вводится ряд \textit{ролевых отношений}, связывающих различные \textit{языки} с принадлежащими им \textit{информационными конструкциями}.}
\scnaddlevel{-1}

\scnheader{знак'}
\scnidtf{быть \textit{знаком}, входящим в состав \textit{информационных конструкций} \uline{заданного} \textit{языка}'}

\scnheader{синтаксически выделяемый класс знаков'}
\scnidtf{быть синтаксические выделяемым \textit{классом знаков} в рамках заданного \textit{языка}'}
\scnhaselementrole{пример}{\scnlist{\scnaddlevel{-1}(SC-код $\in$ sc-узел);\scnaddlevel{1}(SC-код $\in$ sc-коннектор)}}
\scnsuperset{класс синтаксически эквивалентных знаков'}

\scnheader{класс синонимичных знаков'}
\scnidtf{быть множеством \textit{знаков} заданного \textit{языка}, обозначающих одно это уже описываемую \textit{сущность}'}

\scnheader{знаковая конструкция'}
\scnidtf{быть \textit{знаковой конструкцией} заданного \textit{языка}'}

\scnheader{текст'}
\scnidtf{быть синтаксически корректной \textit{информационной конструкцией} заданного \textit{языка}'}

\scnheader{синтаксически некорректная информационная конструкция'}
\scnidtf{быть синтаксически ошибочной \textit{информационной конструкцией} заданного \textit{языка}'}

\scnheader{семантически корректная информационная конструкция'}
\scnidtf{\textit{информационная конструкция} заданного \textit{языка}, не имеющая семантических ошибок'}

\scnheader{семантически некорректная информационная конструкция'}
\scnidtf{\textit{информационная конструкция} заданного \textit{языка}, имеющая семантические ошибки'}

\scnheader{знание'}
\scnidtf{текст заданного \textit{языка}, не имеющий семантических ошибок и обладающий целостностью и ценностью'}

\scnheader{следует отличать*}
\scnhaselementset{текст'\\
\scnaddlevel{1}
\scniselement{ролевое отношение}
\scnaddlevel{-1}
;текст\\
\scnaddlevel{1}
\scnidtf{второй домен ролевого отношения \textit{текст}'}
\scnidtf{текст некоторого языка}
\scnaddlevel{-1}
;знание'\\
\scnaddlevel{1}
\scniselement{ролевое отношение}
\scnaddlevel{-1}
;знание\\
\scnaddlevel{1}
\scnidtf{второй домен ролевого отношения \textit{знание}'}
\scnaddlevel{-1}
;информационная конструкция\\
\scnaddlevel{1}
\scnsuperset{текст}
\scnsuperset{знание}
\scnaddlevel{-1}
}

\scnheader{sc-структура}
\scnidtf{sc-конструкция}
\scnidtf{информационная конструкция SC-кода}
\scnidtftext{часто используемые sc-идентификатор}{SC-код}
\scnidtf{Множества всевозможных sc-структур}
\scnsubset{информационная конструкция}
\scnsuperset{sc-текст} 
\scnaddlevel{1}
\scnidtf{текст \textit{SC-кода}}
\scnidtf{\textit{информационная конструкция}, удовлетворяющая синтаксическим правилам \textit{SC-кода}}
\scnidtf{синтаксически корректная для \textit{SC-кода} \textit{информационная конструкция}}
\scnidtf{связное множество \textit{sc-элементов}, удовлетворяющее синтаксическим правилам \textit{SC-кода}}
\scnidtf{синтаксически правильно построеная \textit{информационная конструкция} \textit{SC-кода}}
\scnsubset{текст}
\scnsuperset{sc-знание}
\scnaddlevel{1}
\scnidtf{знание, представленное в \textit{SC-коде}}
\scnidtf{\textit{sc-текст}, удовлетворяющий определенным семантическим правилам \textit{SC-кода} и, в частности, определённым правилам семантической полноты}
\scnidtf{семантически корректный и целостный \textit{sc-текст} }
\scnsubset{знание}
\scnaddlevel{-2}

\scnheader{sc-знание}
\scnidtf{формализованное знание, хранимое в \textit{памяти ostis-системы} и непосредственно используемое (понимаемое) её \textit{решателем задач}}
\scnidtf{\textit{sc-текст}, имеющий \uline{однозначную} семантическую интерпретацию в рамках \textit{SC-пространства} (формализованного смыслового пространства)}

\scnnote{Не каждый \textit{sc-текст} является \textit{знанием}. В отличие от \textit{sc-текстов} в знаниях важна \uline{семантическая} целостность (полнота) \textit{информационных конструкций}. Так, например, \textit{логическая формула} со свободными переменными не является \textit{знанием}. Но полный текст \textit{высказывания}, включающий в себя все компоненты всех логических связок (вплоть до \textit{атомарных логических формул}) \textit{знанием} является. Второй пример: \textit{sc-текст}, в состав которого входит \textit{sc-коннектор} или \textit{sc-связка}, но не входят все компоненты этого \textit{sc-коннектора} или \textit{sc-связки},\scnbigspace \textit{знанием} не является.}

\scnnote{семантическая целостность \textit{знания ostis-системы} означает, во-первых, то, что такое \textit{знание} представляет собой \textit{информационную конструкцию}, являющуюся высказыванием, то есть информационную конструкцию, имеющую истинностное значение, которое может быть подтверждено или опровергнуто, например, экспертом (рецензентом) \textit{базы знаний ostis-системы}. Во-вторых, \textit{знание ostis-системы} должно содержать достаточно полную и \uline{однозначную} спецификацию по возможности всех входящих в него неидентифицированных (неименованных) \textit{sc-элементов}. Это необходимо для того, чтобы внешнее представление знания ostis-системы можно было по возможности однозначно погрузить ("вставить") в \textit{базу знаний} \scnbigspace \textit{ostis-системы}.}

\scnheader{информация представленная в памяти ostis-системы}
\scnidtf{\textit{информационная конструкция}, хранимая в памяти \textit{ostis-системы}}
\scnsubdividing{sc-структура\\
\scnaddlevel{1}
\scnidtf{информация, представленная в виде конструкции \textit{SC-кода}}
\scnaddlevel{-1}
;файл ostis-системы}
\scnexplanation{В \textit{ostis-системе} используются две формы представления информации в памяти системы --
\begin{scnitemize}
	\item полностью формализованное представление, понятное для решателя задач \textit{ostis-системы} (конструкции \textit{SC-кода});
	\item инородное для \textit{SC-кода} представление, используемое исключительно для коммуникации \textit{ostis-систем} с внешними субъектами, (файлы \textit{ostis-систем}).
\end{scnitemize}}

\scnheader{файл ostis-системы}
\scnidtf{файл, хранимый в sc-памяти}
\scnidtf{инородная для \textit{SC-кода} \textit{информационная конструкция}, хранимая в памяти \textit{ostis-системы} в виде содержимого sc-узла, обозначающего эту \textit{информационную конструкцию}}
\scnidtf{хранимая в sc-памяти "электронная"{} форма представления \textit{информационной конструкции}}

\bigskip

\scnendstruct \scnendcurrentsectioncomment

\end{SCn}
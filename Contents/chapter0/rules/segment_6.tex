\begin{SCn}
	
\scnheader{титульная спецификация выделенного фрагмента базы знаний}
\scnexplanation{\textit{Титульная спецификация выделенного фрагмента базы знаний} ostis-системы представляет собой \textit{sc-структуру}, описывающую свойства специфицируемого знания и включающую в себя: 
	\begin{scnitemize}
		\item связи принадлежности специфицируемого знания соответствующим классам \textit{знаний ostis-систем};
		\item связи, указывающие логически предшествующее и логически следующее \textit{знание ostis-системы};
		\item связь, описывающую декомпозицию специфицируемого знания на последовательность знаний более низкого структурного уровня (декомпозицию разделов на сегменты);
		\item различного вида связи с другими \textit{знаниями ostis-систем}, которые сами "целиком"\ входят в состав спецификации специфицируемого знания (такими знаниями могут быть аннотации, предисловия, введения, оглавления, заключения);
		\item различного вида связи с другими \textit{знаниями ostis-систем}, которые сами не входят в состав спецификации специфицируемого знания (такого рода связями могут быть связи \textit{семантической близости} специфицируемого знания с другими знаниями, связи \textit{семантической эквивалентности}, связи\textit{семантического включения}, связи \textit{противоречивости знаний});
		\item связи, указывающие различного вида \textit{ключевые sc-элементы} (ключевые знаки), соответствующие специфицируемому знанию;
		\item связи специфицируемого знания с авторским коллективом, коллективом рецензентов, с датой его последнего обновления;
		\item для каждого нового целостного фрагмента, вводимого в состав \textit{базы знаний}, в истории эволюции этой \textit{базы знаний} указываются:
		\begin{scnitemizeii}
			\item \textit{автор*} или \textit{авторы*} первой версии этого фрагмента;
			\item отметка времени появления (дата-час-минута) всех версий этого фрагмента (в том числе и окончательно утверждённой, согласованной версии, которая, собственно, и становится фрагментом, включенным в согласованную часть базы знаний);
			\item \textit{рецензии*} (замечания к доработке) всех предварительных версий разрабатываемого \textit{фрагмента базы знаний};
			\item \textit{авторы*} всех указанных рецензий;
			\item отметка времени появления всех указанных рецензий;
			\item события по одобрению, утверждению различных предварительных версий разрабатываемого \textit{фрагмента базы знаний} различными рецензентами и экспертами с указанием отметки времени появления этих событий;
			\item темпоральная последовательность предварительных версий.
		\end{scnitemizeii}
	\end{scnitemize}
}

\scnheader{титульная спецификация выделенного фрагмента базы знаний}
\scnnote{Знак такой спецификации явно не вводится, а сама эта спецификация непосредственно входит в состав специфицируемого фрагмента и включает в себя аннотацию, предисловие, авторов, ключевые знаки, декомпозицию специфицируемого фрагмента базы знаний и прочее}

\scnsubdividing{титульная спецификация раздела базы знаний;
титульная спецификация семейства разделов базы знаний;
титульная спецификация сегмента базы знаний;
титульная спецификация выделенного фрашмента сегмента или атомарного раздела базы знаний}

\scnheader{титульная спецификация выделенного фрагмента базы знаний}
\scnexplanation{\textit{Титульная спецификация выделенного фрагмента базы знаний} содержит общую информацию об этом фрагменте, является непосредственно \uline{частью} специфицируемого фрагмента \textit{базы знаний} и при этом сама \uline{не является} явно \textit{выделенным фрагментом базы знаний}}
\scnhaselement{спецификация}
\scnidtf{основная \textit{метаинформация} (основное \textit{метазнание}) о \textit{выделенном фрагменте базы знаний} -- о его структуре, \textit{авторах\scnrolesign}, \textit{ключевых знаках\scnrolesign} и т.д.}

\scnheader{титульная спецификация выделенного фрагмента базы знаний}
\scnexplanation{\uline{неявно} \textit{выделяемый фрагмент базы знаний}, который:
\begin{scnitemize}
	\item не имеет "собственного" ограничителя ("собственного" контура или "собственных" ограничивающих фигурных скобок);
	\item является семантической спецификацией соответствующего \uline{явно} выделяемого фрагмента базы знаний;
	\item является непосредственной \uline{частью} специфицируемого фрагмента базы знаний
\end{scnitemize}}
\scnnote{В \textit{sc.n-тексте} титульная спецификация фрагмента базы знаний размещается сразу после фигурной скобки, открывающей этот фрагмент}

\scnheader{титульная спецификация раздела базы знаний}
\scnnote{\textit{тиутульная спецификация раздела базы знаний} должна включать в себя достаточно подробное описание семантических свойств этого раздела и, в частности, подробное описание его связей с другими семантически близкими разделами. Это необходимо для обеспечения автономности разделов баз знаний.}

\scnheader{титульная спецификация семейства разделов базы знаний}
\scnnote{Если \textit{разделы базы знаний} являются семантически \uline{ключевыми} \textit{выделенными фрагментами баз знаний}, определяющими спецификацию систем используемых понятий и направления наследования свойств, то \textit{семейства разделов баз знаний} являются \uline{ключевыми} для структуризации виртуальной \textit{Базы знаний Экосистемы OSTIS}, для обмена \textit{знаниями} между различными субъектами \textit{Экосистемы OSTIS}.\\
Поэтому типология \textit{семейств разделов баз знаний} и качество \textit{титульной спецификации семейств разделов баз знаний} имеют большое значение.}

\scnheader{титульная спецификация выделенного фрагмента базы знаний}
\scnrelfrom{множество используемых понятий}{Множество понятий используемых в титльных спецификациях выделенных фрагментов баз знаний}
\scnaddlevel{1}
\scnsuperset{класс выделенных фрагментов}
\scnaddlevel{1}
\scnhaselement{семейство разделов базы знаний}
\scnhaselement{раздел базы знаний}
\scnhaselement{неатомарный раздел базы знаний}
\scnhaselement{атомарный раздел базы знаний}
\scnhaselement{сегмент базы знаний}
\scnhaselement{выделенный фрагмент сегмента или атомарного раздела базы знаний}
\scnhaselement{выделенный фрагмент атомарного раздела базы знаний}
\scnhaselement{выделенный фрагмент сегманта базы знаний}
\scnaddlevel{-1}
\scnsuperset{отношение, связывающее выделенные фрагменты баз знаний с персонами}
\scnaddlevel{1}
\scnhaselement{автор*}
\scnhaselement{рецензент*}
\scnhaselement{эксперт*}
\scnhaselement{технический редактор*}
\scnhaselement{консультант*}
\scnaddlevel{1}
\scnidtf{активный участник обсуждения вопросов, рассматриваемых в специфицируемом фрагменте базы знаний*}
\scnaddlevel{-1}
\scnsuperset{отношение, связывающее выделенные фрагменты баз знаний с ея-файлами}
\scnhaselement{аннотация*}
\scnhaselement{предисловие*}
\scnaddlevel{1}
\scnidtf{Бинарное ориентированное отношение, каждая пара которого связывает:
\begin{scnitemize}
	\item знак некоторого информационного ресурса (в частности, раздела базы знаний или раздела опубликованного документа)
	\item знак информационной конструкции, описывающей цели создания указанного информационного ресурса, предысторию его создания, планируемые направления дальнейшего развития, состав авторов и др.
\end{scnitemize}}
\scnaddlevel{-1}
\scnhaselement{введение*}
\scnhaselement{эпиграф*}
\scnhaselement{заключение*}
\scnhaselement{рассматриваемый вопрос*}
\scnhaselement{основные положения*}
\scnhaselement{вопрос для самопроверки*}
\scnhaselement{упражнение*}
\scnaddlevel{1}
\scnidtf{задача*}
\scnidtf{самостоятельная (индивидуальная) работа*}
\scnaddlevel{-1}
\scnhaselement{коллективный проект}

\scnhaselement{неосновной sc-идентификатор*}
\scnaddlevel{1}
\scnnote{неосновным sc-идентификатором, в частности, может быть альтернативное название выделенного (специфицируемого) фрагмента базы знаний}
\scnaddlevel{-1}
\scnhaselement{часто используемый sc-идентификатор*}
\scnhaselement{сокращенный sc-идентификатор}
\scnhaselement{используемое сокращение*}
\scnaddlevel{1}
\scnidtf{сокращение, используемое в специфицируемом фрагменте базы знаний при построении sc-идентификаторов, а также при оформлении ея-файлов*}
\scnaddlevel{-1}
\scnhaselement{библиографический источник, отражающий аналогичную точку зрения*}
\scnhaselement{библиографический источник, отражающий альтернативную точку зрения*}
\scnhaselement{библиографический источник, дополняющий данную точку зрения*}
\scnaddlevel{-1}
\scnhaselement{сокращение*}
\scnaddlevel{1}
\scnidtf{Бинарное ориентированное отношение, каждая пара которого связывает естественно-языковую фразу с ее сокращенной записью*}
\scnaddlevel{-1}
\scnsuperset{отношение, описывающее структурные или семантические связи и между выделенными фрагментами баз знаний}
\scnaddlevel{1}
\scnhaselement{конкатенация сегментов*}
\scnhaselement{предыдущий сегмент*}
\scnaddlevel{1}
\scnidtf{предыдущий сегмент в рамках соответствующего раздела*}
\scnaddlevel{-1}
\scnhaselement{следующий сегмент*}

\scnhaselement{частный фрагмент базы знаний*}
\scnaddlevel{1}
\scnsuperset{частный раздел базы знаний*}
\scnsuperset{частная предметная область*}
\scnsuperset{частная предметная область и онтология*}
\scnnote{Для фрагмента базы знаний важно указать не только частные по отношению к нему фрагменту базы знаний, но и те фрагменты базы знаний, по отношению к которым данный фрагмент базы знаний является частным}

\scnsuperset{отношение, описывающее ролевой статус знаков, входящих в состав выделенных фрагментов баз знаний}
\scnaddlevel{1}
\scnhaselement{ключевой знак*}
\scnhaselement{ключевой знак первого плана*}
\scnhaselement{ключевой знак второго плана*}
\scnhaselement{ключевой объект исследования*}
\scnhaselement{ключевое понятие*}
\scnhaselement{ключевой класс объектов исследования*}
\scnhaselement{исследуемое отношение*}
\scnhaselement{исследуемый параметр*}
\scnhaselement{исследуемый класс структур*}
\scnaddlevel{-4}

\scnendstruct \scninlinesourcecommentpar{Завершили Сегмент ``Структуризация баз знаний ostis-систем''}


\end{SCn}


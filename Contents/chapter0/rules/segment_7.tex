\begin{SCn}

\scnsegmentheader{Описание правил оформления внешнего представления знаний ostis-систем}

\scnstartsubstruct

\scnheader{внешнее представление знаний ostis-системы}
\scnexplanation{внешнее представление некоторого фрагмента базы знаний ostis-системы, используемое для ввода новой информации в состав базы знаний ostis-системы или для вывода (отображения) запрашиваемого фрагмента базы знаний}
\scnnote{Способ представления исходных текстов баз знаний ostis-систем должен быть максимально возможным образом использован и для вывода (отображения) запрашиваемых пользователем фрагментов баз знаний, особенно, если запрашиваются достаточно большие фрагменты баз знаний, которые необходимо не только представлять, но и структурировать унифицированным образом. Очевидно, что для пользователей желательно, чтобы и для ввода информации в ostis-систему, и для ее вывода использовались одни и те же языковые средства и правила оформления}
\scnnote{Требования, предъявляемые к оформлению внешних текстов знаний ostis-систем (sc-знаний) носят достаточно противоречивый характер -- с одной стороны, речь идет о формальных текстах, легко воспринимаемых (понимаемых, транслируемых) ostis-системами, а, с другой стороны, желательно, чтобы эти же формальные тексты легко воспринимались (понимались) широким кругом людей и не требовали для этого от них длительной подготовки. Отметим при этом, что работа с формальными текстами требует от человека достаточно высокой культуры \uline{точного} мышления (математической культуры).

Отметим также, что использование формальных языков является важнейшим и необходимым этапом эволюции человеческой деятельности в любой области (в математике, в физике, в технике).

Тем не менее, проблема создания универсального языка представления исходных текстов различного вида знаний, который был бы достаточно удобен как для интеллектуальных компьютерных систем, так и для \uline{широкого} круга разработчиков баз знаний и экспертов, требует конкретного решения.}

\filemodetrue
\scnreltovector{требования}{Стиль и характер оформления внешнего представления sc-знаний должен обеспечить возможность интуитивного понимания смысла текста при отсутствии понимания различного рода синтаксических деталей. Для этого:
\begin{scnitemize}
\item формальный текст должен максимально возможным образом использовать привычную для широкого круга специалистов терминологию\char59
\item структуризация, форматирование формальных текстов также должны опираться на сформировавшиеся традиции\char59
\item внешнее представление (внешний текст) sc-знания должен включать в себя такое количество отображаемых ея-файлов, прочтения которых было бы достаточно для понимания смысла представляемого sc-знания, а также для понимания формальных средств его представления
\end{scnitemize};
Формальный текст (как внутреннего, так и внешнего представления sc-знаний) должен включать в себя средства для уточнения смысла используемых знаков и соответствующих им терминов, а также смысла некоторых фрагментов формального текста. Для этого в формальный язык вводятся естественно-языковые файлы, отображаемые в исходных текстах и поясняющие используемые термины, а также комментирующие или даже полностью переводящие на естественный язык различные фрагменты формального представления \textit{базы знаний}.;
Все используемые в базе знаний ostis-системы (в том числе, и в её ея-файлах) внешние идентификаторы \textit{sc-элементов} (термины, имена, условные обозначения) должны быть формально специфицированы средствами \textit{SC-кода}. Подчеркнем, что здесь речь идет о спецификации не самих sc-элементов, а их внешних идентификаторов (в первую очередь, простых sc.s-идентификаторов) -- их происхождение, использование, авторство и т.д.;
Все отношения, параметры и другие понятия, используемые в формальных текстах должны быть пояснены в соответствующих формальных онтологиях. Первое упоминание во внешнем тексте каждого такого понятия должно быть кратко пояснено с помощью поясняющего ея-файла, а также сделана ссылка на раздел базы знаний, в которых приведена подробная и формальная спецификация указанного понятия с дополнительным указанием номера этого раздела с помощью нетранслируемого комментария.;
Аналогичным образом в отображаемом внешнем представлении sc-знания поясняются и комментируются все \uline{первые} (в рамках этого внешнего представления) использования средств формального представления знаний со ссылками на разделы базы знаний, где указанные языковые средства подробно описываются. С самого начала внешнего представления большого структурированного фрагмента базы знаний (каковым, в частности, является Стандарт OSTIS) с помощью нетранслируемых комментариев, не входящих в состав базы знаний, либо с помощью ея-файлов ostis-систем необходимо пояснять все нюансы формализации со ссылкой на ближайший раздел и сегмент, где это будет подробнее рассмотрено.;
При описании формальных средств должны быть приведены конкретные \uline{примеры} со ссылкой на раздел или сегмент, где этот пример будет рассмотрен подробнее (например, на соответствующую предметную область и онтологию);
Все комментарии и примечания, которые можно представить средствами \textit{SCn-кода} или \textit{SCg-кода}, нужно оформлять именно так. Нетранслируемыми комментариями  \uline{не стоит увлекаться}.;
В формальных текстах и в естественно-языковых файлах, входящих в состав \textit{базы знаний ostis-системы}, для идентификации (именования) \textit{sc-элементов} должны использоваться только те термины, которые являются \uline{основными}(!) внешними идентификаторами соответствующих \textit{sc-элементов}, выделяемыми жирным и нежирным курсивом. При этом, если идентификаторы (названия, имена) разделов, сегментов базы знаний находятся в позиции \uline{заголовков} указанных фрагментов базы знаний, то они оформляются жирным курсивом с увеличенным расстоянием между символами, а заголовки разделов дополнительно выделяются увеличенным размером символов.;
В согласованной (общепризнанной) части \textit{базы знаний} противоречия трактуются как выявленные ошибки в \textit{базе знаний}, подлежащие устранению. Но в истории эволюции
базы знаний противоречия могут присутствовать как противоречия разных точек зрения разных авторов. Заметим при этом, что разные точки зрения далеко не всегда являются противоречивыми (взаимоисключающими). Они могут просто дополнять друг друга, описывать исследуемые сущности с разных "ракурсов". Умение видеть противоречия только там, где, они действительно есть, и умение локализовать эти противоречия (выделить их суть) -- это необходимые навыки для разработки \uline{практически полезных} \textit{баз знаний}.}
\filemodefalse

\scnheader{sc.n-представление знаний ostis-системы}
\filemodetrue
\scnrelfromvector{правила оформления}{В исходных текстах баз знаний \textit{интеллектуальных компьютерный систем}, построенных по \textit{Технологии OSTIS} (т.е. ostis-систем) \scnsourcecomment{в том числе, в тексте данной монографии} используются следующие языки:
	\begin{scnitemize}
	\item Различные \textit{естественные языки} (\textit{Русский язык}, \textit{Английский язык} и др.)
	\item \textit{SCg-код} (Semantic Code graphical), являющийся \textit{универсальным формальным языком} графического представления \textit{семантических сетей}
	\item \textit{SCs-код} (Semantic Code string), являющийся \textit{универсальным формальным языком} линейного представления \textit{семантических сетей} в виде \textit{строк} (цепочек символов)
	\item \textit{SCn-код} (Semantic Code Natural) являющийся \textit{универсальным формальным языком} представления \textit{семантических сетей} в виде структурированных форматированных \textit{текстов} (строк, размещенных на плоскости.
	\end{scnitemize}	
\textit{Синтаксис} и \textit{денотационная семантика} указанных \textit{формальных языков подробно рассмотрены в Разделе ***}
;Основным языком внешнего представления \textit{баз знаний ostis-систем} является \textit{SCn-код}, рассмотренный в разделе  \textit{Введение в язык структурированного представления баз знаний ostis-систем}. Но в состав текста \textit{SCn-кода} могут входить тексты и других языков (тексты \textit{SCg-кода}, тексты \textit{SCs-кода}, тексты естественных языков, тексты различных искусственнных языков), а также различного рода нетекстовые информационные конструкции (рисунки, таблицы, чертежи, графики, фотографии). Указанные "инородные"{} для \textit{SCn-кода} информационные конструкции, а также описываемые тексты самого \textit{SCn-кода} оформляются во внешнем представлении базы знаний либо как нетранслируемые, но специфицируемые файлы, либо как транслируемые инородные для \textit{SCn-кода} информационные конструкции, ограниченные, соответственно, либо \textit{sc.n-рамками} (квадратными скобками), либо \textit{sc.n-контурами} (фигурными скобками).
;Структуризация внешнего представления баз знаний ostis-систем является полным отражением структуризации внутреннего представления \textit{баз знаний ostis-систем}.
;В случае, если осуществляется внешнее представление \uline{полного} текста указываемого сложноструктурированного фрагмента базы знаний, как, например, внешнее представление семейства разделов под названием \textit{``Документация Стандарт OSTIS''} вместе со всеми его разделами, то последовательность и иерархическая структура отображения разделов и сегментов указанного сложноструктурированного фрагмента базы знаний в точности соответствует иерархической структуре представляемого (отображаемого, визуализируемого) семейства разделов базы знаний.
;База знаний каждой \textit{ostis-системы} представляет собой иерархическую систему разделов, к которым должны "привязываться"{} исходные тексты каждой новой информации, вводимой в \textit{базу знаний}, и, прежде всего, исходные тексты достаточно крупных фрагментов \textit{баз знаний}. \scnsourcecomment{К таким исходным текстам, в частности, относится и данная монография.} Очевидно при этом, что нумерация разделов баз знаний не может быть стабильна. Кроме того, при оформлении исходного текста крупного фрагмента базы знаний, обладающего достаточной целостностью и по научно-технической значимости достигшего уровня монографии или диссертации, желательно иметь собственную (свою, локальную) нумерацию разделов при сохранении их иерархической структуры. Это означает, что номера разделов имеет смысл использовать только в рамках каждого  исходного вводимого текста и не должны использоваться в самой базе знаний. Таким образом, ссылаться на разделы базы знаний следует по \uline{названию} разделов.
;Внешнее представление каждого структурно \textit{выделяемого фрагмента базы знаний} ostis-системы за исключением sc-знаний нижнего уровня начинается с \uline{заголовка} (имени, названия) этого фрагмента. Указанный заголовок есть не что иное, как простой sc-идентификатор sc-узла, обозначающего представляемый фрагмент базы знаний (представляемое sc-знание). Рассматриваемый заголовок оформляется жирным курсивом с \uline{увеличенным расстоянием между символами}. При этом заголовки \uline{внешнего представления} разделов базы знаний имеют дополнительно \uline{увеличенный размер шрифта}. Заголовок внешнего представления sc-знания размещается с первого символа строчки.
После заголовка представляемого фрагмента базы знаний \uline{с новой строчки} размещается (1) sc.s-коннектор вида ``$\supset$='' (2) следом за ним на следующей строчке левая фигурная скобка (открывающая фигурная скобка).
;Внешнее представление каждого раздела базы знаний начинается \uline{с новой страницы}. Соответственно, сегмент базы знаний может начинаться \uline{не} с новой страницы.\\
;Внешнее представление каждого неструктурируемого выделяемого фрагмента базы знаний (атомарного раздела базы знаний, сегмента базы знаний) оформляется в виде \textit{sc.n-предложения}, связывающего sc-идентификатор (имя, название) представляемого фрагмента базы знаний, оформленный в виде \uline{заголовка} внешнего текста этого фрагмента (признаком чего является жирный курсив с увеличенным расстоянием между символами), с \textit{sc.n-контуром}, который фигурными скобками ограничивает sc.n-изображение (sc.n-визуализацию) представляемого фрагмента базы знаний.\\
В самом простом случае представляемым неструктурируемым выделяемым фрагментом базы знаний является \uline{один} ея-файл ostis-системы.
;Внешний текст \textit{базы знаний} может иметь самый разный объем и может касаться только \uline{одного раздела} или сегмента, а может включать в себя материалы \uline{нескольких разделов} или сегментов. Если представляемый (отображаемый) фрагмент базы знаний является sc-знанием нижнего уровня иерархии, т.е. частью \uline{неструктурированного} фрагмента базы знаний (например, частью сегмента базы знаний), то при его представлении необходимо указать, частью какого неструктурированного фрагмента базы знаний представляемый фрагмент является.\\
Для исходного текста sc-знания нижнего уровня эта информация необходима для того, чтобы знать, в какой фрагмент базы знаний требуется включить, "погрузить"{} данное вводимое \textit{sc-знание}.
;Виды выделений во внешнем представлении знаний ostis-системы:
\begin{scnitemize}
    \item с помощью шрифта
    \begin{scnitemizeii}
        \item вид шрифта (печатный, курсив)
        \item размер шрифта (стандартный, увеличенный)
        \item расстояние между символами (стандартное, увеличенное)
    \end{scnitemizeii}
    \item подчеркиванием
    \item с помощью символьных ограничителей (скобок различного вида)
\end{scnitemize}
;Выделяемые объекты:
\begin{scnitemize}
    \item основные термины (имена, идентификаторы)
    \item специфицируемые файлы
    \begin{scnitemizeii}
        \item нетранслируемые в базу знаний файлы
        \item транслируемые в базу знаний файлы
    \end{scnitemizeii}
    \item нетранслируемые комментарии к внешнему тексту
    \item цитаты (и короткие, и длинные)
    \item ключевые фрагменты ея-текста
    \item метафорические термины
\end{scnitemize}
;На любой странице внешнего текста при распечатке делается разметка тонкими вертикальными табуляционными линиями для четкой визуализации длины отступа от левого края страницы. Особенно это важно при переходе на новую страницу.} \scnsourcecomment{Завершили перечень правил оформления внешнего представления знаний ostis-системы} 
\filemodefalse

\scnheader{sc.n-представление выделенного фрагмента базы знаний}
\scnrelfromvector{обобщенная конкатенация}{признак начала sc.n-представления выделенного фрагмента базы знаний\\
	\scnaddlevel{1}
	\scnexplanation{Данным признаком является \textit{нетранслируемый комментарий}, размещенный по всей длине \textit{строчки} и состоящий
	\begin{scnitemize}
	\item либо из слова "Раздел"{} (изображенного жирным печатным шрифтом такого же размера и с таким же расстоянием между символами, что и в \textit{заголовке sc.n-представления раздела базы знаний}) и последующих "звездочек"{} до конца \textit{строчки};
	\item либо из слова "Сегмент"{} (изображенного жирным печатным шрифтом такого же размера и с таким же расстоянием между символами, что и в \textit{заголовке sc.n-представления сегмента базы знаний}) и последующих "звездочек"{} до конца \textit{строчки};
	\item либо из одних "звездочек"{} до конца \textit{строчки}.
	\end{scnitemize}}
	\scnaddlevel{-1}
;заголовок sc.n-представления выделенного фрагмента базы знаний
;связка "$\supset$="{}\\
	\scnaddlevel{1}
	\scnidtf{связка, связывающая \textit{заголовок sc.n-представления выделенного фрагмента базы знаний} с самим \textit{sc.n-текстом} этого фрагмента}
	\scnrelfrom{размещение}{с начала новой строчки}
	\scnaddlevel{-1}
;фигурная скобка, открывающая sc.n-текст выделенного фрагмента базы знаний\\
	\scnaddlevel{1}
	\scnrelfrom{размещение}{первый символ новой строчки}
	\scnaddlevel{-1}
;собственно sc.n-текст выделенного фрагмента базы знаний\\
	\scnaddlevel{1}
	\scnnote{Данный текст может занимать большое количество \textit{строчек}}
	\scnaddlevel{-1}
;фигурная скобка, закрывающая sc.n-текст выделенного фрагмента базы знаний\\
	\scnaddlevel{1}
	\scnrelfrom{размещение}{первый символ новой строчки}
	\scnidtf{первый символ \textit{строчки}, которая следует после последней \textit{строчки sc.n-текста выделенного фрагмента базы знаний}}
	\scnaddlevel{-1}
;дополнительный признак завершения sc.n-представления выделенного фрагмента базы знаний\\
	\scnaddlevel{1}
	\scnexplanation{Данный признак используется только при представлении выделенных фрагментов сегментов баз знаний и выделенных фрагментов неструктурированных разделов баз знаний, которые не декомпозируются на сегменты. При этом рассматриваемый признак изображается как нетранслируемый комментарий, имеющий длину в половину строчки и состоящий из "звездочек"{}.}
	\scnaddlevel{-1}}
\scnnote{Отличия в sc.n-представлении различного вида \textit{выделенных фрагментов баз знаний} заключается в следующем:
	\begin{scnitemize}
	\item размер шрифта и расстояние между символами в заголовке \textit{sc.n-представления выделенного фрагмента базы знаний} различны:
	\begin{scnitemizeii}
	\item самый большой -- у разделов;
	\item поменьше -- у сегментов;
	\item самый маленький (чуть больше стандартного) -- у выделенных фрагментов сегментов и фрагментов разделов, не содержащих сегменты;
	\end{scnitemizeii}
	\item для выделенных фрагментов сегментов и фрагментов, не содержащих сегменты, заголовок sc.n-представления фрагмента базы знаний и, соответственно, связка "$\supset$="{} \uline{могут отсутствовать}, т.е. такие фрагменты баз знаний могут быть неименованными.
	\end{scnitemize}}


\scnheader{заголовок sc.n-представления выделенного фрагмента базы знаний}
\scnsubset{sc-идентификатор выделенного фрагмента базы знаний}
\scnsubset{жирный курсив возможно увеличенного размера и с увеличенным расстоянием между символами}
	\scnaddlevel{1}
	\scniselement{шрифт\scnsupergroupsign}
	\scnaddlevel{-1}
\scnrelfrom{размещение}{с начала новой строчки}
\scnidtf{с первого символа новой строчки}
\scnidtf{от первой табуляционной линии размещения sc.n-текста}


\scnheader{титульная спецификация \uline{публикации} семейства разделов базы знаний}
\scnnote{Имеется в виду \uline{публикация} (издание) \textit{семейства разделов базы знаний} в виде \textit{sc.n-представления семейства разделов базы знаний}}
\scnnote{В простейшем случае публикуемое \textit{семейство разделов базы знаний} может состоять из одного \textit{раздела}. Например, это может быть публикация \textit{раздела базы знаний} в виде статьи}

\scnheader{нетранслируемый комментарий к внешнему тексту}
\scnidtf{нетранслируемый комментарий к внешнему тексту отображаемого фрагмента базы знаний ostis-системы}
\scnexplanation{естественно-языковой текст, который ограничен слева наклонной чертой и звездочкой ``/*'' и справа -- звездочкой и наклонной чертой ``*/'' и который может находиться в \uline{любом} месте внешнего текста}
\scnnote{При загрузке и трансляции \uline{исходного} текста \textit{базы знаний ostis-системы} все входящие в него нетранслируемые комментарии игнорируются -- и в том случае, если эти комментарии входят в состав изображения какого-либо файла ostis-системы (в частности, естественно-языкового файла), и в том случае, если эти комментарии входят в состав формального текста, транслируемого в \textit{SC-код}.\\
При трансляции внутреннего текста (sc-текста) базы знаний ostis-системы во внешнее представление (например, в sc.n-текст) некоторые нетранслируемые комментарии могут автоматически генерироваться.\\
Например, нетранслируемые комментарии, предшествующие заголовкам внешнего представления структурно выделяемых фрагментов баз знаний ostis-систем (разделов, сегментов, начал неатомарных разделов, завершений неатомарных фрагментов), нетранслируемые комментарии к правым (закрывающим) фигурным и квадратным скобкам, если ограничиваемые этими скобками выражения размещаются на нескольких страницах, нетранслируемые комментарии к sc.s-идентификаторам, являющиеся ссылками на номера разделов и сегментов, где подробно специфицируется сущность, именуемая комментируемым sc-идентификатором.\\
Приведем несколько конкретных примеров:\\
\scnsourcecomment{представление данного файла будет продолжено на следующей странице}\\
\scnsourcecomment{продолжение представления файла}\\
\scnsourcecomment{представление данного sc-текста будет продолжено на следующей странице}\\
\scnsourcecomment{продолжение представления sc-текста}\\
\scnsourcecomment{sc-текст (файл), обозначаемый данным sc-узлом, представлен на следующей странице (на следующих страницах)}\\
Примерами нетранслируемых комментариев к закрывающим скобкам:
\scnsourcecomment{Завершили раздел i.j.k}\\
\scnsourcecomment{Завершили начальный сегмент раздела i.j.k}\\
\scnsourcecomment{Завершили примечание к i.j.k}}

\scnheader{комментарий к внешнему тексту базы знаний}
\scnnote{Нетранслируемые комментарии к внешнему тексту базы знаний непосредственно в состав базы не входят. Указанными комментариями \uline{не следует злоупотреблять}, они должны касаться оформления внешнего текста, но не смысла представляемой базы знаний.\\
Содержательные комментарии к базе знаний должны оформляться в виде файлов, входящих в состав базы знаний.}

\scnendstruct 
\end{SCn}
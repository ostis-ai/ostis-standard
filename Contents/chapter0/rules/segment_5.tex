\begin{SCn}

\scnheader{раздел базы знаний}
\scnidtf{раздел б.з.}
	\scnaddlevel{1}
	\scniselement{сокращенный sc-идентификатор}
	\scnaddlevel{-1}
\scnexplanation{Множество \textit{разделов баз знаний} имеет:
	\begin{scnitemize}
	\item богатую семантическую типологию;
	\item богатый набор отношений, описывающих семантические связи между разделами.
	\end{scnitemize}
	Синтаксически \textit{разделы баз знаний} могут пересекаться (иметь общие элементы), но никогда один раздел не может включаться (полностью входить в состав) другого раздела. В этом смысле понятие подраздела (точнее, \textit{частного раздела}*) имеет не "синтаксический"{} смысл, а семантический -- глубокое наследование свойств при достаточно большой степени "независимости"{} друг от друга}
\scnnote{Важным свойством \textit{раздела базы знаний} \scnbigspace \textit{ostis-системы} является его семантическая целостность -- наличие достаточно стабильного набора классов исследуемых сущностей (\textit{объектов исследования}) и достаточно стабильного семейства \textit{отношений} и семейства \textit{параметров}, заданных на различных классах объектов исследования, а также семейства \textit{классов структур} использующих указанные выше понятия (введенные \textit{классы объектов исследования}, введенные \textit{отношения} и \textit{классы структур}). Такая целостность дает возможность развивать \textit{раздел базы знаний}, не выходя "за рамки"{} используемой системы \textit{понятий}. Это позволяет развивать каждый \textit{раздел базы знаний} в известной мере \uline{независимо} от других разделов, что существенно повышает \uline{гибкость} и \uline{стратифицированность} \textit{базы знаний}.}	
\scnexplanation{Основными свойствами \textit{раздела базы знаний} как структурно \textit{выделяемого фрагмента базы знаний} являются:
	\begin{scnitemize}
	\item семантическая целостность -- наличие четкого критерия, позволяющего установить для каждого конкретного знания то, включать или не включать это знание в состав данного раздела;
	\item потенциальная возможность эволюционировать в достаточной степени независимо от других разделов, но при условии соблюдения всех требований, обеспечивающих постоянную поддержку \uline{семантической совместимости} данного раздела со всеми остальными семантически смежными разделами.
	\end{scnitemize}}
	\scnaddlevel{1}
	\scntext{следовательно}{Грамотная декомпозиция \textit{базы знаний} на разделы, основанная на четкой стратификации процесса эволюции накапливаемой человечеством общечеловеческой объединенной \textit{базы знаний}, сутью которой является \uline{минимизация} трудоемкости усилий по согласованию и обеспечению с авторами других разделов \textit{семантической совместимости} со смежными разделами, создает предпосылки высоких темпов эволюции \textit{базы знаний} в целом.}
	\scnaddlevel{-1}
\scnnote{Любой \textit{раздел базы знаний} не является структурной частью другого раздела. Каждый раздел самодостаточен и целостен. Это обеспечивается тем, что в состав \textit{титульной спецификации} каждого раздела входит \textit{семантическая окрестность},описывающая связи специфицируемого раздела \uline{со всеми} семантически близкими ему разделами.\\
При этом разные разделы могут иметь разный семантический тип:
	\begin{scnitemize}
	\item раздел может быть предметной областью, интегрированной со всеми ее онтологиями;
	\item раздел может быть просто предметной областью;
	\item раздел может быть какой-либо онтологией чего угодно (не обязательно предметной области);
	\item и т.д.
	\end{scnitemize}}
\scnnote{Если в \textit{титульную спецификацию} каждого раздела будет входить семантическая спецификация каждого раздела, включающая его семантические связи со всеми семантически близкими разделами, то последовательность (порядок) разделов в "линейном"{} исходном тексте, публикуемом в качестве очередной версии Стандарта OSTIS, может быть в достаточной степени \uline{произвольной}.}


\scnheader{семейство разделов базы знаний}
\scnidtf{множество семантически связанных друг с другом разделов базы знаний}
\scnidtf{кластер разделов базы знаний}
\scnnote{Семантические связи между разделами, входящими в состав семейства разделов, представляются в рамках \textit{титульных спецификаций разделов}, каждая из которых является специальной частью соответствующего (специфицируемого) раздела, входящего в состав семейства разделов.

Для каждого \textit{раздела базы знаний} в рамках его титульной спецификации формируется \textit{семантическая окрестность} его связей со всеми семантически близкими ему разделами (и, прежде всего, с теми разделами, которые входят в состав тех семейств разделов, в которые входит заданный раздел). При этом акцентируется внимание именно на семантических связях между разделами. Так, например, вместо структурной ("синтаксической"{}) связи "быть подразделом*"{} (т.е. быть частью заданного раздела) вводится связь "быть \textit{частным разделом}*"{}. \\
Данная связь указывает направление наследования свойств исследуемых объектов заданного раздела от разделов, исследующих более общие классы объектов.
Порядок (последовательность) разделов в рамках \textit{семейства разделов базы знаний} при наличии \uline{явно представленных} семантических связей между разделами, входящими в семейство разделов, может быть достаточно \uline{произвольным}, что очень важно, например, при формировании оглавления очередной издаваемой версии \textit{Стандарта OSTIS}. Таким образом, трактовка \textit{Стандарта OSTIS}, а также всех издаваемых версий этого Стандарта как \textit{семейства разделов базы знаний} \scnbigskip \textit{Метасистемы IMS.ostis} обеспечивает высокий уровень гибкости \textit{Стандарта OSTIS}, а также легкость "переиздаваемости"{} его версий.}


\scnheader{сегмент или атомарный раздел базы знаний}
\scnnote{Простейшей формой \textit{сегмента} или \textit{атомарного раздела базы знаний} является просто последовательность \textit{файлов ostis-системы}. Некоторые из этих файлов могут быть идентифицированными (именованными), если на них ссылаются другие файлы, а некоторые из них могут быть связаны с другими файлами различными отношениями (в частности, один файл может быть пояснением другого). Кроме того, некоторые из этих файлов могут быть формально специфицированы (например, указаны соответствующие им ключевые \textit{sc-элементы}).\\
В самом простом случае \textit{сегмент} или \textit{атомарный раздел базы знаний} может быть \textit{sc-структурой}, состоящей из \uline{одного} (!) \textit{sc-узла}, обозначающего \textit{файл ostis-системы} (чаще всего, \textit{ея-файл ostis-системы}). Т.е. сам \textit{файл ostis-системы} может быть \textit{знанием ostis-системы}, но не может быть структурно \textit{выделяемым} \textit{фрагментом базы знаний} ostis-системы. При этом \textit{sc-узел}, обозначающий \textit{файл ostis-системы}, являющийся \textit{знанием}, может быть единственным \textit{sc-элементом} структурно выделяемого \textit{знания ostis-системы}.}
\scnnote{Для наглядного отображения (визуализации) \textit{сегмента} или \textit{атомарного раздела базы знаний ostis-систем\textit{ы} целесообразно представить указанное \textit{sc-знание} в виде конкатенации (последовательности) таких }sc-знаний, которые, во-первых, были бы достаточно крупными и логико-семантически значимыми для соответствующего \textit{сегмента} или \textit{атомарного раздела базы знаний ostis-системы} и, во-вторых, для которых существовал бы алгоритм \uline{однозначного} (!) размещения (на экране) внешнего представления этих \textit{sc-знаний} (в \textit{SCg-коде} или в \textit{SCn-коде}).\\
Однозначность здесь означает наличие легко усваиваемого пользователями стандартного \uline{стиля визуализации} \textit{sc-знаний} и заключается в том, что многократная визуализация одного и того же \textit{sc-знания} с помощью указанного алгоритма должна приводить к синтаксически эквивалентным, а в случае \textit{SCg-кода} и к геометрически конгруэнтным текстам. Очевидно, что для произвольных \textit{sc-знаний} большого объёма такого алгоритма не существует, но для \textit{sc-знаний}, содержащих описание собственной структуры и семантической типологии собственных фрагментов, разработка такого алгоритма вполне реальна при наличии достаточного количества указанных \textit{метазнаний} о структуре отображаемых (визуализируемых) \textit{sc-знаний}.}


\scnheader{sc-идентификатор выделенного фрагмента базы знаний}
\scnidtf{название (имя) выделенного фрагмента базы знаний}
\scnexplanation{Не следует путать объект описания (спецификации) и само описание. Поэтому в \textit{sc-идентфикаторе} фрагмента базы знаний должны присутствовать слова, указывающие на семантический или структурный тип именуемого фрагмента базы знаний (описание, спецификация, анализ, сравнительный анализ, сравнение, определение, раздел, предметная область, онтология и т.п.).

Таким образом, \textit{sc-идентификатор выделенного фрагмента базы знаний} ostis-системы должен иметь \uline{явное} (!) указание на то, что он является обозначением именно фрагмента базы знаний, а не того, что описывается в этом фрагменте.}
\scnnote{Мы не будем использовать такой изменчивый для нас способ идентификации разделов \textit{Стандарта OSTIS}, как нумерацию этих разделов, поскольку, например, в разных издаваемых официальных версиях \textit{Стандарта OSTIS} одному и тому же разделу \textit{Стандарта OSTIS} могут соответствовать разные номера.}


\scnheader{выделенный фрагмент базы знаний}
\scnrelfrom{основной sc-идентификатор}{\scnfilelong{выделенный фрагмент базы знаний}}
	\scnaddlevel{1}
	\scnrelfrom{используемая аббревиатура}{\scnfilelong{выделенный фр-нт б.з.}}
	\scnaddlevel{-1}
\scnsubdividing{именованный фрагмент базы знаний\\
	\scnaddlevel{1}
	\scnidtf{\textit{выделенный фрагмент базы знаний}, имеющий \textit{sc-идентификатор} (имя, название)}
	\scnnote{\textit{Именованными фрагментами баз знаний} могут быть только структурно \textit{выделенные фрагменты баз знаний}}
	\scnnote{Все \textit{семейства разделов баз знаний}, все \textit{разделы баз знаний} и все \textit{сегменты баз знаний} должны быть именованными}
	\scnsuperset{семейство разделов базы знаний}
	\scnsuperset{раздел базы знаний}
	\scnsuperset{сегмент базы знаний}
	\scnaddlevel{-1}
;неименованный фрагмент базы знаний\\
	\scnaddlevel{1}
	\scnidtf{\textit{выделенный фрагмент базы знаний}, \uline{не} имеющий \textit{sc-идентификатор} (имени, названия)}
	\scnnote{\textit{неименованными фрагментами баз знаний} могут быть только \textit{выделенные фрагменты сегментов баз знаний} либовыделенные фрагменты таких \textit{разделов баз знаний}, которые не состоят из \textit{сегментов}}
	\scnaddlevel{-1}}

\end{SCn}
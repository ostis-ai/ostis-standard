\begin{SCn}
\scnsuperset{фрагмент знака, представленный файлом ostis-системы}
\scnaddlevel{1}
\scnsuperset{синтаксически элементарный фрагмент информационной конструкции, представленный файлом ostis-системы}
\scnaddlevel{-1}

\scnsuperset{знак, представленный файлом ostis-системы}
\scnaddlevel{1}
\scnsuperset{sc-идентификатор, представленный файлом ostis-системы}
\scnaddlevel{-1}

\scnsuperset{\textbf{знаковая конструкция, представленная файлом ostis-системы}}
\scnaddlevel{1}
\scnsuperset{текст, представленный файлом ostis-системы}
    \scnaddlevel{1}
    \scnsuperset{знание, представленное файлом ostis-системы}
    \scnaddlevel{-1}
\scnaddlevel{-1}

\scnheader{знаковая конструкция, представленная файлом ostis-системы}
\scnsuperset{sc.s-файл ostis-системы}
\scnaddlevel{1}
\scnidtf{конструкция SCs-кода, хранимая в памяти ostis-системы в виде содержимого некоторого SC-узла}
\scnidtf{файл ostis-системы, являющийся sc.s-конструкцией}
\scnaddlevel{-1}
\scnsuperset{sc-идентификатор, представленный файлом ostis-системы}
\scnaddlevel{1}
\scnidtf{файл ostis-системы, являющийся sc-идентификатором}
\scnidtf{(sc-идентификатор $\cap$ файл ostis-системы)}
\scnreltoset{пересечение}{sc-идентификатор;файл ostis-системы}
\scnaddlevel{-1}
\scnsuperset{sc.g-файл ostis-системы}
\scnaddlevel{1}
\scnreltoset{пересечение}{sc.g-конструкция;файл ostis-системы}
\scnaddlevel{-1}
\scnsuperset{sc.n-файл ostis-системы}
\scnaddlevel{1}
\scnreltoset{пересечение}{sc.n-конструкция;файл ostis-системы}
\scnaddlevel{-1}

\scnsuperset{ея-файл ostis-системы}
\scnaddlevel{1}
\scnreltoset{пересечение}{ея-конструкция\\
    \scnaddlevel{1}
    \scnidtf{естественно-языковая конструкция}
    \scnidtf{конструкция естественного языка}
    \scnsuperset{ея-текст}
    \scnaddlevel{-1}
;файл ostis-системы}
\scnidtf{конструкция естественного языка, представленная в виде файла ostis-системы}
\scnsubset{Русский язык}
    \scnaddlevel{1}
    \scnidtf{конструкция Русского языка}
    \scnaddlevel{-1}
\scnsubset{Английский язык}
\scnidtf{естественно-языковая конструкция, являющаяся содержимм sc-узла, обозначающего эту конструкцию}
\scnaddlevel{-1}

\scnheader{файл ostis-системы}
\scnsubdividing{файл ostis-системы, предполагающий одномерную визуализацию хранимой информации
;файл ostis-системы, предполагающий двухмерную визуализацию хранимой информации
;файл ostis-системы, предполагающий трехмерную визуализацию хранимой информации}
\scnsubdividing{файл ostis-системы, представляющий статистическую информацию
;файл ostis-системы, представляющий динамическую информацию\\
\scnaddlevel{1}
\scnsuperset{видео-файл ostis-системы}
\scnsuperset{аудио-файл ostis-системы}
\scnaddlevel{-1}}
\scnsubdividing{файл-экземпляр ostis-системы\\
\scnaddlevel{1}
\scnidtf{\textit{sc-узел}, обозначающий конкретное вхождение \textit{информационной конструкции}, структура которой представлена содержимым этого \textit{sc-узла}}
\scnaddlevel{-1}
;файл-образец ostis-системы\\
\scnaddlevel{1}
\scnidtf{класс \textit{синтаксически эквивалентных} \textit{файлов-экземпляров} \textit{ostis-системы}}
\scnidtf{множество всевозможных \textit{файлов-экземпляров ostis-системы}, которые являются \textit{синтаксически эквивалентными} копиями содержимого заданного \textit{sc-узла}}
\scnaddlevel{-1}
}
\scnsubdividing{сформированный файл ostis-системы\\
\scnaddlevel{1}
\scnidtf{\textit{файл ostis-системы}, представленный \textit{sc-узлом}, имеющим сформированное (полностью построенное) содержимое}
\scnaddlevel{-1}
;несформированный файл ostis-системы\\
\scnaddlevel{1}
\scnidtf{\textit{файл ostis-системы}, представленный \textit{sc-узлом}, содержимое которого либо полностью отсутствует, либо сформировано \uline{частично}}
\scnaddlevel{-1}}
\scnnote{\textit{файл ostis-системы} можеть быть \uline{электронной копией} и знаком (!) внешней \textit{информационной конструкции}, которая может быть:
\begin{scnitemize}
\item общедоступный в сети Internet информационным ресурсом;
\item документом, опублиеованным на бумажном носители в виде какой-либо книги или статьи.
\end{scnitemize}
Кроме того , \textit{файл ostis-системы} может быть просто \uline{обозначением} указанной внешней \textit{информационной конструкции} (т.е. может быть \textit{sc-узлом}, обозначающим \textit{внешнюю информационную конструкцию}, но не имеющим содержимого). Такой \textit{sc-узел} используется формальной спецификации (средствами \textit{SC-кода}) соответствующего обозначаемого им информационного ресурса.
}
\scnheader{ея-файл ostis-системы}
\scnidtf{естественно-языковой \textit{файл ostis-системы}}
\scnexplanation{\textit{файл ostis-системы}, представляющий собой \textit{sc-узел}, содержимым которого является "электронная"{} форма \textit{информационной конструкции} (чаще всего, \textit{текста}) одного из \textit{естественных языков}}
\scnidtf{структурно выделенный ея-текст, хранимый в памяти ostis-системы (в sc-памяти)}
\scnrelfromvector{правила оформления}{
\scnfileitem{При оформлении текстов в естественно-языковых файлах (ея-файлах) \textit{ostis-систем} используются обычные разделители (точки в аббревиатурах и между предложениями\char59 круглые скобки, запятые, пробелы\char59), а также специальный символ $\square$, используемый \textit{в ея-файлах ostis-систем} как разделитель и целый ряд  следующих ограничителей, позволяющих выделять некоторые фрагменты ея-текстов:
\begin{scnitemize}
\item подчеркивание выделяет логически важные фрагменты в предложениях\char59
\item цитатные кавычки являются ограничителем цитат\char59
\item прямые кавычки ограничивают иносказательные термины, метафоры\char59
\item жирным курсивом стандартного размера выделяются идентификаторы \textit{sc-элементов} базы знаний, являющиеся ключевыми для заданного контекста\char59
\item жирным курсивом стандартного размера выделяются также условные внешние идентификаторы (обозначения) условно вводимых \textit{sc-элементов}, например, условные обозначения \textit{sc-элементов} произвольного структурного типа ($\bm{e_i}$, $\bm{e_j}$, ...), условные обозначения \textit{sc-узлов} ($\bm{v_i}$, $\bm{v_j}$, ...), условные обозначения \textit{sc-коннекторов} ($\bm{c_i}$, $\bm{c_j}$, ...), \textit{sc-дуг}, \textit{sc-связок}, не являющихся \textit{sc-коннекторами}, \textit{sc-структур} и так далее\char59
\item нежирным курсивом стандартного размера выделяются \textit{sc-идентификаторы} \textit{sc-элементов} \textit{базы знаний}, не являющиеся ключевыми для данного ея-текста.
\end{scnitemize}}
;\scnfileitem{Если в ея-тексте идентификаторы sc-элементов базы знаний (чаще всего -- \textit{простые sc-идентификаторы}), выделенные курсивом одинаковой жирности, следуют друг за другом, то число пробелов между разными идентификаторами должно быть увеличено. Если при этом выделенный sc-идентификатор включает в себя другие sc-идентификаторы, на которые желательно отдельно сослаться, то такая ссылка оформляется в виде дполнительной фразы типа "Смотрите также ..."{}}
;\scnfileitem{Текст \textit{ея-файла ostis-системы} может иметь \uline{любые} вставки не являющиеся естественно-языковыми текстами, в том числе, и фрагменты, являющиеся формальными внешними текстами представления знаний для ostis-систем (текстами SCg-кода, SCs-кода, SCn-кода). При этом указанные формальные фрагменты (вставки) могут быть как транслируемыми на внутренний язык ostis-системы (SC-код) и погружаемыми в состав ее базы знаний (т.е. фактически являться sc-текстами), так и нетранслируемыми формальными фрагментами, которые входят в состав базы знаний ostis-системы в виде содержимого соответствующих файлов. Все указанные выше "вставки" в ея-файл ostis-системы оформляются как ссылки на соответствующие sc-тексты или файлы ostis-системы. Каждая такая ссылка представляет собой \textit{sc-идентификатор} соответствующего sc-текста или файла ostis-системы и выделяется в ея-файле жирным курсивом стандартного размера со стандартным расстоянием между символами.
Таким образом, если в естественно-языковой \textit{файл ostis-системы}, необходимо "вставить" информационную конструкцию иного рода (\textit{sc.g-текст}, рисунок, таблицу, изображение), то (1) указанная конструкция оформляется как отдельный файл (2) которому приписывается имя (название), построенное по установленным правилам, и (3) на который в указанном естественно-языковом файле делается ссылка.
В естественно-языковых \textit{файлах ostis-систем} можно делать ссылки не только на другие \textit{файлы ostis-системы}, но и на \uline{именуемые} (!) фрагменты базы знаний, которые во внешнем представлении базы знаний оформляются в виде именуемых (идентифицируемых) sc.n-контуров.
Файлы и sc-тексты, на которые делаются ссылки из ея-файла, во внешнем представлении (при визуализации) базы знаний размещаются после указанного ея-файла в порядке первого их упоминания в этом ея-файле, если, конечно, на эти файлы или sc-тексты не было ссылок из ранее представленных ея-файлов.}
;\scnfileitem{В состав \textit{ея-файла ostis-системы} могут входить ссылки на любые идентифицированные (именованные) \textit{информационные конструкции} (Internet-ресурсы, библиографические источники, различные документы). Для этого достаточно указывать соответствующие \textit{sc-идентификаторы}.}
;\scnfileitem{В ея-текстах \uline{все} sc-основные идентификаторы описываемых в базе знаний сущностей должны быть выделены жирным или нежирным курсивом и могут быть представлены в любом склонении и спряжении.}
;\scnfileitem{В ея-текстах используются только основные идентификаторы (термины). Используемые синонимы явно указываются как неосновные идентификаторы.}
;\scnfileitem{Основная часть (содержимого текста) \textit{ея-файла ostis-системы} оформляется стандартным печатным шрифтом.}
}

\scnsourcecomment{Завершили перечень правил оформления содержимого \textit{ея-файлов ostis-систем}}

\scnheader{ея-файл ostis-системы}
\scnnote{Выделенные курсивом в \textit{ея-файле ostis-системы} \textit{sc-идентификаторы sc-элементов} могут являться \uline{ар\-гу\-мен\-та\-ми} различного вида \uline{запросов} к \textit{базе знаний ostis-системы} и, в первую очередь запросов типа "Что это такое"{}, предполагающих выделение из \uline{текущего} состояния \textit{базы знаний ostis-системы} семантической окрестности (спецификации) указываемого \textit{sc-элемента}, содержащей основную информацию о сущности, обозначаемой \textit{этим sc-элементом}.}


\end{SCn}
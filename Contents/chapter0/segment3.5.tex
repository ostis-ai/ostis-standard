\bigskip
\scnfragmentcaption

\scnheader{Технология OSTIS}
\scnexplanation{\textit{Технология OSTIS} рассматривается нами как один из вариантов комплексного решения всех перечисленных выше сверхзадач, которые направлены на развитие деятельности в области искусственного интеллекта и которые, очевидно, сильно связаны друг с другом. Таким образом, \textit{Технология OSTIS} включает в себя:
\begin{scnitemize}
\item и постоянно развивающуюся общую формальную теорию интеллектуальных компьютерных систем, представленную в виде базы знаний соответствующего портала научно-технических знаний;
\item и постоянно развивающийся комплекс моделей, методов и средств, используемых при проектировании интеллектуальных компьютерных систем и оформленных в виде интеллектуальной системы информационной и инструментальной поддержки (автоматизации) проектирования семантически совместимых интеллектуальных компьютерных систем;
\item и постоянно развивающуюся глобальную экосистему, состоящую из семантически совместимых взаимодействующих интеллектуальных компьютерных систем, ориентированных на комплексную автоматизацию всевозможных видов человеческой деятельности;
\end{scnitemize}}

\scnexplanation{Целью создания \textit{Технологии OSTIS} является не только построение методики, обеспечивающей четкую организацию коллективной \textit{человеческой деятельности} по проектированию, производству, эксплуатации и реинжинирингу \textit{интеллектуальных компьютерных систем}, но и построение мощных средств автоматизации (компьютерной поддержки) этой деятельности. Здесь важно подчеркнуть то, что \textit{интеллектуальные компьютерные системы}, разрабатываемые с помощью \textit{Технологии OSTIS (ostis-системы)} могут быть использованы для автоматизации \uline{любых} видов \textit{человеческой деятельности} и, в том числе, для автоматизации коллективной \textit{человеческой деятельности} по проектированию, производству, эксплуатации и реинжинирингу \textit{ostis-систем}. В рамках \textit{Технологии OSTIS} так и происходит - автоматизация проектирования, производства, эксплуатации и реинжиниринга \textit{ostis-систем} осуществляется с помощью специально предназначенных для этого \textit{ostis-систем}, некоторые из которых (например, для поддержки эксплуатации и реинжиниринга \textit{ostis-систем}) являются \textit{ostis-системами}, встроенными (интегрированными) в те \textit{ostis-системы}, поддержку эксплуатации и реинжиниринга которых они осуществляют.}
\scnidtf{Комплексная технология, обеспечивающая автоматизацию самых различных действий (в том числе, и всевозможных видов человеческой деятельности) на основе семантически совместимых интеллектуальных компьютерных систем, способных координировать (согласовывать) свои действия как с себе подобными, так и с людьми}
\scnidtf{Сумма (интеграция) всевозможных ostis-технологий}

\scnrelfromlist{достоинство}{
\scnfileitem{\textit{Технология OSTIS} представляет собой принципиально новый уровень развития \textit{информационных технологий}, в основе которого лежит переход от (from) data science к (to) knowledge science}; 
\scnfileitem{Открытый характер \textit{Технологии OSTIS} как для тех, кто желает участвовать в её развитии, так и для пользователей \textit{Технологии OSTIS} -- для разработчиков прикладных \textit{ostis-систем}};
\scnfileitem{Низкий порог вхождения для желающих развивать и желающих использовать имеющиеся в текущий момент методы и средства \textit{Технологии OSTIS}, что обеспечивается поддержкой качественного состояния документации по текущей версии \textit{Технологии OSTIS} с дополнительным описанием эволюции (развития) Технологии OSTIS, а также плана дальнейшего  её развития};
\scnfileitem{Децентрализованный характер управления проектами разработки \textit{ostis-систем}, основанный на четком согласовании коллективом разработчиков проектных задач}; 
\scnfileitem{Ориентация на новое поколение компьютеров, без появления которых дальнейшее развитие \textit{технологий искусственного интеллекта} невозможно. При этом \textit{Технология OSTIS} позволяет достаточно конструктивно сформулировать требования, предъявляемые к таким компьютерам};
\scnfileitem{\textit{Технология OSTIS} не только обеспечивает автоматизацию широкого многообразия видов человеческой деятельности, но и существенно повышает уровень (качество) этой автоматизации, благодаря (1) широкому применению методов и средств \textit{искусственного интеллекта} и (2) создание условий для \textit{конвергенции}, семантической совместимости и \textit{глубокой интеграции} как автоматизируемых видов человеческой деятельности, так и продуктов этой деятельности. В частности, это касается и автоматизации человеческой деятельности в области \textit{искусственного интеллекта}. \textit{Технология OSTIS} рассматривается как предлагаемый подход к конвергенции и интеграции как различных видов деятельности в области искусственного интеллекта, так и результатов этой деятельности (частных теорий различных компонентов и различных видов интеллектуальных систем, частных методов и средств проектирования различных видов и различных компонентов интеллектуальных компьютерных систем)};
\scnfileitem{Ориентация на разработку компьютерных систем и коллективов таких систем, имеющих высокий уровень \textit{интеллекта}; Ориентация на разработку глобальной сети \textit{интеллектуальных компьютерных систем}, обеспечивающей комплексную автоматизацию всех видов и областей \textit{человеческой деятельности}};
\scnfileitem{Создание условий для формирования \textit{рынка знаний} на основе иерархической системы семантически совместимых \textit{порталов знаний}, соответствующих самым различным областям и \textit{видам человеческой деятельности}};
\scnfileitem{Создание условий для перехода от традиционной формы публикации статей, монографий, отчетов и прочих документов к их публикации как фрагментов \textit{баз знаний} соответствующих \textit{порталов знаний}, что полностью исключает дублирование информации в публикуемых документах и обеспечивает непосредственное использование этой информации в \textit{интеллектуальных компьютерных системах}}}

\scntext{ближайшая задача}{Обеспечить низкий порог входа в Технологию OSTIS:
\begin{scnitemize}
\item для желающих участвовать в развитии \textit{Технологии OSTIS}, т.е. в совершенствовании \textit{метасистемы IMS.ostis} (системы информационной поддержки и автоматизации проектирования \textit{ostis-систем}), которая сама также является \textit{ostis-системой};
\item для разработчиков \textit{интеллектуальных компьютерных систем}, желающих использовать для этого \textit{Технологию OSTIS} (эти разработчики являются конечными пользователями \textit{Метасистем IMS.ostis});
\item для конечных пользователей всевозможных иных \textit{ostis-систем}, т.е. компьютерных систем, разработанных по \textit{Технологии OSTIS} с непосредственным использованием в качестве инструмента \textit{Метасистемы IMS.ostis} (подчеркнем при этом, что базовы принципы организации взаимодействия \textit{Метасистемы IMS.ostis} с конечными пользователями полностью совпадают с базовыми принципами организации взаимодействия всех остальных \textit{ostis-систем}, разработанных с помощью \textit{Метасистемы IMS.ostis}, со своими конечными пользователями. Это обусловлено тем, что \textit{Метасистема IMS.ostis} сама также является \textit{ostis-системой} -- материнской \textit{ostis-системой}).
\end{scnitemize}}

\scnnote{Для решения указанной задачи необходимо создать инфраструктуру коллективного перманентного обновления (совершенствования) комплексной документации по \textit{Технологии OSTIS}, которая: 
\begin{scnitemize}
\item обеспечила бы достаточную полноту и четкость фиксации текущего состояния \textit{Технологии OSTIS} и удовлетворяла бы как разработчиков \textit{Технологии OSTIS} (т.е. разработчиков \textit{Метасистемы IMS.ostis}), так и разработчиков \textit{ostis-систем}, не являющихся \textit{Метасистемой IMS.ostis} (т.е. конечных пользователей \textit{Метасистемы IMS.ostis}), и также конечных пользователей любых \textit{ostis-систем};
\item обеспечила бы высокие темпы совершенствования данной документации на основании (1) четких правил согласования и утверждения различного рода предложений, (2) максимально возможной автоматизации процессов анализа, согласования и утверждения указанных предложений, (3) постоянного расширения числа авторов и (4) четких правил защиты авторских прав;
\item обеспечила бы четкую фиксацию границ между текущим состоянием \textit{Технологии OSTIS} и разрабатываемыми, тестируемыми фрагментами её будущих версий с обоснованием таких нововведений и с планом их включения в соответствующую версию \textit{Технологии OSTIS};
\item обеспечила бы четкую семантическую интеграцию документации той части \textit{Технологии OSTIS}, которая касается проектирования семантических моделей \textit{ostis-систем} и которая фактически сводится к проектированию \textit{баз знаний ostis-систем}, а также документации той части \textit{Технологии OSTIS}, которая описывает различные варианты программной или аппаратной реализации универсального интерпретатора логико-семантических моделей \textit{ostis-систем}. Подчеркнем при этом, что универсальность используемого в \textit{Технологии OSTIS} языка представления знаний дает возможность описывать на нем все, что угодно, в том числе и интерпретаторы семантических моделей \textit{ostis-систем}. Но делать это нужно с разумной степенью детализации.
\end{scnitemize}}

\scntext{ближайшая задача}{Осуществить конвергенцию и интеграцию всевозможных частных технологий проектирования и реализации различных видов компонентов интеллектуальных компонентов систем (в частности, баз знаний, различного вида логических моделей, искусственных нейронных сетей и т.п.)}

\scnrelfromlist{класс создаваемых продуктов}{
ostis-система\\
    \scnaddlevel{1}
    \scnidtf{индивидуальная ostis-система}
    \scnnote{Существенно подчеркнуть, что \textit{Технология OSTIS} порождает не просто множество \textit{ostis-систем}, а множество семантически совместимых и взаимодействующих \textit{ostis-систем}, образующих экосистему, которую будем называть \textit{Экосистемой OSTIS} (Экосистемой ostis-систем и их пользователей). Таким образом, можно считать что интегрированным продуктом \textit{Технологии OSTIS} является не множество ostis-систем, а  система (экосистема) \textit{ostis-систем}.}
    \scnaddlevel{-1};
коллектив ostis-систем\\
    \scnaddlevel{1}
    \scnsuperset{простой коллектив ostis-систем}
        \scnaddlevel{1}
        \scnidtf{коллектив ostis-систем членами которого являются только индивидуальные ostis-системы}
        \scnaddlevel{-1}
    \scnaddlevel{-1}
    \scnaddlevel{1}
    \scnsuperset{иерархический коллектив ostis-систем}
        \scnaddlevel{1}
        \scnidtf{коллектив ostis-систем, по крайней мере одним членом которого является коллектив ostis-систем}
        \scnaddlevel{-1}
    \scnaddlevel{-1};
ostis-сообщество\\
    \scnaddlevel{1}
    \scnsuperset{простое ostis-сообщество}
    \scnsuperset{иерархическое ostis-сообщество}
        \scnaddlevel{1}
        \scnhaselement{Экосистема OSTIS}
        \scnaddlevel{-1}
    \scnaddlevel{-1}}

\scnrelfromlist{основной продукт}{
\textit{Экосистема OSTIS}\\
    \scnaddlevel{1}
    \scnidtf{Максимальное \textit{ostis-сообщество}, направленное на автоматизацию всех видов человеческой деятельности}
    \scnaddlevel{-1};
\textit{Консорциум OSTIS}\\
    \scnaddlevel{1}
    \scnidtf{\textit{ostis-сообщество}, направленное на развитие \textit{Технологии OSTIS}}
    \scnaddlevel{-1};
\textit{Метасистема IMS.ostis}\\
    \scnaddlevel{1}
    \scnidtf{Метасистема, являющаяся
        \begin{scnitemize}
        \item \textit{корпоративной ostis-системой}, обеспечивающей организацию (координацию) деятельности \textit{Консорциума OSTIS};
        \item формой представления реализации и фиксации текущего состояния \textit{Ядра Технологии OSTIS};
        \item корпоративной \textit{ostis-системой}, взаимодействующей со всеми корпоративными ostis-системами, каждая из которых координирует развитие соответствующей \textit{специализированной ostis-технологии}.
        \end{scnitemize}}
    \scnaddlevel{-1}}

\scnheader{Экосистема OSTIS}
\scnnote{Важной особенностью и достоинством \textit{Технологии OSTIS} является то, что все остальные продукты её использования (конкретные \textit{ostis-системы}) объединяются в сеть, т.е. становятся единым целостным продуктом использования \textit{Технологии OSTIS} -- \textit{Экосистема OSTIS}}
\scnexplanation{Социально-техническая сеть, состоящая из людей и ostis-систем, которые являются
\begin{scnenumerate}
\item семантически совместимыми;
\item постоянно эволюционирующими индивидуально;
\item постоянно поддерживающими свою совместимость с другими агентами в ходе своей индивидуальной эволюции;
\item способными децентрализованно координировать свою деятельность.
\end{scnenumerate}}

\bigskip
\scnendstruct \scninlinesourcecommentpar{Завершили Сегмент "\textit{Понятие технологии OSTIS}"}
\documentclass{scndocument}

\usepackage{scn}
\usepackage[T2A]{fontenc}    
\usepackage[utf8]{inputenc}  
\usepackage[english,russian]{babel}
\usepackage{graphicx}

\begin{document}
\begin{SCn}
\ActivateBG

\scnheader{Глава 7.8.
Подсистема Экосистемы OSTIS, обеспечивающая поддержку
жизненного цикла интеллектуальных геоинформационных систем
различного назначения
}
    \begin{scnrelfromlist}{авторы}
    \scnitem{Самодумкин С.А.}
    \scnitem{Зотов Н.В.}
    \end{scnrelfromlist}
    \scntext{аннотация}{Глава посвящена частной технологии проектирования интеллектуальных геоинформационных систем, построенных по принципам Технологии OSTIS. интероперабельность геоинформационных систем, построенных по предлагаемой технологии, обеспечивается за счет перехода от карты к семантическому описанию
    элементов карты.}
    \begin{scnrelfromlist}{подраздел}
        \scnitem{§ 7.8.1. Требования, предъявляемые к интеллектуальным геоинфармационным системам}
        \scnitem{§ 7.8.2. Систематизация задач, решаемых интеллектуальными геоинформационными системами}
        \scnitem{§ 7.8.3. Основные формальные онтологии баз знаний в интеллектуальных геоинформационных ostis-системах}
        \scnitem{§ 7.8.4. Решатель задач интеллектуальной геоинформационной ostis-системы}
        \scnitem{§ 7.8.5. Картографический интерфейс интеллектуальной геоинформационной ostis-системы}
        \scnitem{§ 7.8.6. Многократно используемые компоненты интеллектуальных геоинформационных ostis-систем}
        \scnitem{§ 7.8.7. Средства автоматизации проектирования интеллектуальных геоинформационных ostis-систем}
    \end{scnrelfromlist}
    \begin{scnrelfromlist}{ключевой знак}
        \scnitem{Технология проектирования интеллектуальных геоинформационных систем}
        \scnitem{Язык карт для ostis-систем}
    \end{scnrelfromlist}
    \begin{scnrelfromlist}{ключевое понятие}
        \scnitem{геоинформационная система}
        \scnitem{интеллектуальная геоинформационная система}
        \scnitem{задача геоинформационной системы}
        \scnitem{интеллектуальная геоинформационная ostis-система}
        \scnitem{база знаний интеллектуальной геоинформационной ostis-системы}
        \scnitem{геоонтология}
        \scnitem{объект местности}
        \scnitem{геосемантическая характеристика объекта местности}
        \scnitem{пространственное отношение}
        \scnitem{решатель задач интеллектуальной геоинформационной ostis-системы}
        \scnitem{картографический интерфейс}
        \scnitem{картографический интерфейс интеллектуальной геоинформационной ostis-системы}
    \end{scnrelfromlist}
    \begin{scnrelfromlist}{ключевой параметр}
        \scnitem{класс объектов местности}
    \end{scnrelfromlist}
    \begin{scnrelfromlist}{библиографическая ссылка}
        \scnitem{Крючков   А.Н..ИнтелТвГС-2006кн}
        \scnitem{Абламейко С.В..ГеогрИССЦК-2000ст}
        \scnitem{Ивакин Я.А.МетодИПГСнОО-2009дс}
        \scnitem{Белякова М.Л.ИнтелГСдУИТ-2016кн}
        \scnitem{Губаревич А.В..ОнтолПИСвОИ-2017ст}
        \scnitem{Губаревич А.В..СтрукБЗвИСпИ-2018ст}
        \scnitem{Блискавицкий А.А.КонцеПГИСиУ-2012кн}
        \scnitem{Блискавицкий А.А.СеманГОФГиИ-2014ст}
        \scnitem{Hu Y.GeospS-2019art}
        \scnitem{Janowicz K..GeospSaLSD-2012art}
        \scnitem{ЦифроКМИО-2007ст}
        \scnitem{Баталов Р.Н..ГеопрИДЗвИК-2021ст}
        \scnitem{Березко А..ИнтелГИС-2009ст}
        \scnitem{Журавков М.А..ГИСте пДПИСКГИM-2004кн}
        \scnitem{Глотов А.А.ПримеГТдФС-2014ст}
        \scnitem{Глотов А.А.ИнтелГСПиН-2015ст}
        \scnitem{Дулин С.К..ПроблОСГиСП-2016ст}
        \scnitem{Кикоть А.В.СравнМиМИО-2020ст}
        \scnitem{Кузнецов К.А..ПублиДоООП-2012ст}
        \scnitem{Орехова Д.А.Разра иИМДП-2013дс}
        \scnitem{Самодумкин С.А..СеманТКПИ-2011ст}
        \scnitem{Самодумкин С.А.ИнтелГС-2012ст}
        \scnitem{Samodumkin S.A.SemanToIGSD-2019art}
        \scnitem{Samodumkin S.SemanToIGSD-2019art}
        \scnitem{Samodumkin S.Next-IGS-2022art}
        \scnitem{СистеООАД-2017кн}
        \scnitem{Соколов М.С.Модел иАИОиА-2012дс}
        \scnitem{Атаева О.М..СредаИПДГ-2011ст}
        \scnitem{Шпаков М.В.РазраИГнОН-2004дс}
        \scnitem{Янкелевич С.С..КонцеНВКОнЗ-2019ст}
    \end{scnrelfromlist}
    \scntext{введение}{Современные программно-технологические комплексы геоинформационных систем очень эффективны, но сложны в освоении и применении, поэтому требуют специальной профессиональной подготовки конечных пользователей. Для внедрения систем геопространственного назначения в различные области знаний и сферы применения необходимо, чтобы специалисты различного вида деятельности без особых сложностей и дополнительного обучения могли решать характерные для геоинформационных систем задачи. Для этого необходим
    переход от традиционных геоинформационных систем к геоинформационным системам нового поколения переход от традиционных геоинформационных систем к геоинформационным системам нового поколения, имеющим удобный пользовательский интерфейс. Целью данной главы является создание комплексной технологии проектирования интеллектуальных геоинформационных систем нового поколения по принципам, лежащим в основе Технологии OSTIS.}
\scnheader{§ 7.8.1. Требования, предъявляемые к интеллектуальным геоинформационным
системам нового поколения
}
\scnheader{геоинформационная система}
    \scntext{общее определение}{Программная компьютерная система,
    обеспечивающая ввод, манипулирование, анализ и вывод пространственно-соотнесенных данных (геоданных) о
    территории, социальных и природных явлениях при решении задач, связанных с инвентаризацией, анализом, моделированием, прогнозированием и управлением окружающей средой и территориальной организацией общества}
\scnheader{задача геоинформационной системы}
    \begin{scnrelfromset}{разбиение}
        \scnitem{задача анализа в геоинформационной системе}
        \scnitem{задача моделирования в геоинформационной системе}
        \scnitem{задача прогнозирования в геоинформационной системе}
        \scnitem{задача управления в геоинформационной системе}
    \end{scnrelfromset}
\scnheader{основные объекты в геоинформационных системах}
    \begin{scnrelfromset}{разбиение}
        \scnitem{знания об объектах местности}
        \scnitem{данные об объектах местности}
    \end{scnrelfromset}
\scnheader{формализация знаний об объектах местности и их представление в базах знаний в интеллектуальных систем}
    \begin{scnrelfromset}{требование}
        \scnitem{установить отношения для описания свойств и закономерностей, присущих рассматриваемой предметной области и использующей объекты местности}
        \scnitem{установить геометрические характеристики, способные осуществить привязку объектов местности}
        \scnitem{учесть темпоральный(веремнный) характер существования объектов местности, что позволяет осуществить ретроспективный анализ}
        \scnitem{задача управления в геоинформационной системе}
    \end{scnrelfromset}
    \begin{scnrelfromset}{способствует}
        \scnitem{расширить предметные области и добавить новые функциональные возможности в рамках предлагаемой
    частной Технологии проектирования интеллектуальных геоинформационных систем.
    }
    \end{scnrelfromset}
    \scntext{способствует}{расширить предметные области и добавить новые функциональные возможности в рамках предлагаемой
    частной Технологии проектирования интеллектуальных геоинформационных систем.}
\scnheader{интеллектуальная геоинформационная система}
    \scnidtf{информационная система, основным объектом исследования которой являются знания и данные об объектах местности, выступающие интеграционной основой для решения прикладных задач в различных предметных областях}
    \scnsuperset{интеллектуальная геоинформационная ostis-система}
    \begin{scnindent}
        \scnsubset{знания о пользователе}
        \scnidtf{интеллектуальная геоинформационная система, разработанная по принципам Технологии OSTIS}
        \begin{scnrelfromset}{обобщённая декомпозиция}
        \scnitem{база знаний интеллектуальной геоинформационной ostis-системы}
        \scnitem{решатель задач интеллектуальной геоинформационной ostis-системы}
        \scnitem{картографический интерфейс интеллектуальной геоинформационной ostis-системы}
        \end{scnrelfromset}
    \end{scnindent}
\scnheader{наличие технологии проектирования геоинформационных систем}
    \begin{scnrelfromset}{возможность}
        \scnitem{снизить срок разработки интеллектуальных систем}
        \scnitem{повысить интеллектуальные возможности интеллектуальных систем, использующих знания об объектах местности}
    \end{scnrelfromset}
    \scntext{примечание}{При этом Технология проектирования интеллектуальных геоинформационных систем должна быть ориентирована на многократное использование функциональных компонентов системы с целью сокращения сроков проектирования и разработки прикладных систем. Таким образом, речь идет о создании частной Технологии проектирования интеллектуальных геоинформационных систем.}
\scnheader{актуальные задачи Технологии проектирования интеллектуальных геоинформационных систем}
    \begin{scnrelfromset}{разбиение}
        \scnitem{проектирование пространственных онтологий и на основе их решение проблемы семантической совместимости знаний предметных областей}
        \scnitem{решение задачи управления метаданными и совершенствования поиска, доступа и обмена в условиях растущих объемов пространственной информации и сервисов, предоставляемых многочисленными источниками геоинформации}
        \scnitem{осуществление вывода знаний с использованием пространственной и тематической информации как составляющих знаний объектов местности с использованием Языка вопросов (см. Главу 3.4. Язык вопросов для ostis-систем)}
        \scnitem{внедрение картографического интерфейса в интеллектуальные ostis-системы как естественного для человека способа представления информации об объектах местности}
    \end{scnrelfromset}
\scnheader{решение задач с точки зрения взаимодействия, интеграции и обеспечения совместимости различных прикладных систем}
    \begin{scnrelfromset}{с помощью}
        \scnitem{интеграции предметных областей и соответствующих им онтологий (вертикальный уровень)}
        \scnitem{расширения функциональных возможностей систем при помощи повторно используемых компонентов этих систем (горизонтальный уровень), в частности, проектирование компонентов для новых территорий или в новом временном интервале}
    \end{scnrelfromset}
\scnheader{материал, предложенный для выполнения предъявленных требований}
    \scntext{примечание}{предлагает рассматривать карту как информационную конструкцию, элементами которой являются объекты местности.}
    \begin{scnrelfromset}{разбиение}
        \scnitem{Предметная область и онтология объектов местности}
        \scnitem{Спецификация Языка карт для ostis-систем}
    \end{scnrelfromset}
\scnheader{описание, на основе которого предлагается переход от карт к их смыслу}
    \begin{scnrelfromset}{разбиение}
        \scnitem{формальное описание Синтаксиса Языка карт для ostis-систем}
        \scnitem{формальное описание Денотационной семантики Языка карт для ostis-систем}
    \end{scnrelfromset}
\scnheader{семантическая совместимость геоинформационных систем и их компонентов}
        \scntext{примечание}{обеспечивается благодаря общей для них онтологии объектов местности, которая необходима для интероперабельности геоинформационных систем различного назначения и их компонентов.}
\scnheader{интероперабельность геоинформационных систем}
        \scntext{примечание}{обеспечивается за счет перехода от карты к семантическому описанию элементов карты за счет перехода от карты к семантическому описанию элементов карты, то есть объектов местности и связей (пространственных отношений) между ними}
\scnheader{наличие данных обстоятельств}
    \begin{scnrelfromset}{определяет}
        \scnitem{существование научно-технической проблемы интеллектуализации геоинформационных систем}
        \scnitem{создание Технологии проектирования интеллектуальных геоинформационных систем, в основе которых лежат принципы проектирования ostis-систем}
    \end{scnrelfromset}
    
\scnheader{§ 7.8.2. Систематизация задач, решаемых интеллектуальными геоинформационными системами}

\scnheader{интеллектуализация геоинформационных систем}
    \scntext{примечание}{одним из направлений повышения эффективности использования информационно-вычислительных средств является интеллектуализация геоинформационных систем}
    \begin{scnrelfromset}{предпологает}
        \scnitem{возможность общения конечного пользователя с системой на Языке вопросов (см. Главу 3.4. Язык вопросов для ostis-систем)}
        \scnitem{использование различных интероперабельных решателей задач с возможностью объяснения полученных решений (см. Главу 3.3. Агентно-ориентированные модели гибридных решателей задач ostis-систем)}
        \scnitem{использование картографического интерфейса для визуализации исходных данных и результатов (см. Главу 4.1. Общие принципы организации интерфейсов ostis-систем).}    
    \end{scnrelfromset}
\scnheader{Реализация возможностей интеллектуальных геоинформационных систем}
    \begin{scnrelfromset}{может быть осуществлена с помощью}
        \scnitem{систем управления базами знаний}
        \scnitem{мультимедийных баз знаний и данных по областям применения}
        \scnitem{интероперабельных решателей задач}
        \scnitem{интеллектуального картографического интерфейса}    
        \scnitem{экспертных систем в различных областях деятельности людей}    
        \scnitem{систем поддержки принятия решений}    
        \scnitem{систем интеллектуальной помощи}      
    \end{scnrelfromset}
\scnheader{Задачи, решаемые с помощью интеллектуализации геоинформационных систем}
    \begin{scnrelfromset}{разбиение}
        \scnitem{использование цифрового картографического материала и данных дистанционного зондирования Земли в проблемно-ориентированных областях (см. Абламейко С.В..ГеогрИССЦК-2000ст)}
        \scnitem{планирование действий в динамически меняющейся ситуации в условиях неполных или нечетких данных с использованием экспертных знаний (см. Ивакин Я.А.МетодИПГСнОО-2009дс)}
        \scnitem{разрешение земельных споров}
        \scnitem{анализ чрезвычайных ситуаций и подготовка материалов для принятия решений по предотвращению или ликвидации их последствий}    
        \scnitem{создание систем поддержки принятия решений для прикладных геоинформационных систем территориального планирования и управления (см. Белякова М.Л.ИнтелГСдУИТ-2016кн)}    
        \scnitem{разработка диагностических экспертных систем по геологоразведочной деятельности со средствами удаленного доступа к ним}    
        \scnitem{логистическое планирование, создание экспертных систем и программных средств управления предприятиями}     \scnitem{создание систем контроля и навигации}
        \scnitem{создание экспертных систем прогнозирования возникновения и развития на местности техногенных и природных ситуаций: наводнений, землетрясений, экстремальных погодных условий (осадки, температура), эпидемий, распространения радионуклидов, химических выбросов, метеопрогноз и так далее}  
        \scnitem{создание экспертных систем выбора участков местности для строительства различных объектов}  
        \scnitem{создание экспертных систем планирования эффективного использования сельскохозяйственных земель}  
        \scnitem{создание экспертных систем и программных средств для анализа геоданных}  
        \scnitem{создание систем распознавания образов и изображений по данным дистанционного зондирования Земли}  
        \scnitem{создание банков цифровой картографической информации со средствами удаленного доступа к ним}  
        \scnitem{обработка изображений}  
        \scnitem{ретроспективный анализ событий (см.Губаревич А.В..ОнтолПИСвОИ-2017ст,Губаревич А.В..СтрукБЗвИСпИ2018ст)}  
        \scnitem{создание информационно-поисковых систем по наукам о Земле и Геоинформатике}  
        \scnitem{разработка обучающих систем для подготовки специалистов и экспертов со средствами удаленного доступа к ним}  
        \scnitem{}  
        \scnitem{}
    \end{scnrelfromset}
    \scntext{примечание}{Полное решение поставленных выше задач требует использования стандартов открытых систем и использование онтологий объектов местности как интегрирующих элементов различных предметных областей.}

\scnheader{§ 7.8.3. Основные формальные онтологии баз знаний в интеллектуальных геоинформационных ostis-системах}

\begin{scnrelfromlist}{подраздел}
    \scnitem{Пункт 7.8.3.1. Базовая классификация объектов местности}
    \scnitem{Пункт 7.8.3.2. Геосемантические характеристики объектов местности}
    \scnitem{Пункт 7.8.3.3. Топологические пространственные отношения между объектами местности}
    \scnitem{Пункт 7.8.3.4. Стратифицированная модель информационного пространства объектов местности}
\end{scnrelfromlist}


\scnheader{разработка онтологий}
    \begin{scnrelfromset}{может быть осуществлена с помощью}
        \scnitem{систем управления базами знаний}
        \scnitem{мультимедийных баз знаний и данных по областям применения}
        \scnitem{интероперабельных решателей задач}
        \scnitem{интеллектуального картографического интерфейса}    
        \scnitem{экспертных систем в различных областях деятельности людей}    
        \scnitem{систем поддержки принятия решений}    
        \scnitem{систем интеллектуальной помощи}      
    \end{scnrelfromset}
    
    \scntext{примечание}{Основным подходом к обеспечению интероперабельности является разработка онтологий. Наиболее часто онтологии, используемые в геоинформатике, обычно рассматриваются как доменные онтологии, которые принятоназывать географическими онтологиями, или геоонтологиями (см.Блискавицкий А.А.КонцеПГИСиУ-2012кн,Блискавицкий А.А.СеманГОФГиИ-2014ст ). Одной из задач в разработке онтологий является четкое и однозначное определение семантики примитивных терминов (атомарных элементов, которые не могут быть далее разделены). Чтобы решить эту проблему исследователи предложили обосновать примитивные термины геонтоглогий на основе географических явлений (см. Hu Y.GeospS-2019art, Janowicz K..GeospSaLSD 2012art).}

\scnheader{онтология}
    \scnidtf{формализация
    некоторой области знаний на основе концептуальной схемы со структурой, содержащей классы объектов, их связи
    и правила, допускающей компьютерный анализ. Соответственно в состав онтологии предметной области входят
    экземпляры, понятия, атрибуты и отношения. предметные области, для которых целесообразна разработка геоинформационных систем, предполагают построение онтологии, которую будем называть геоонтологией.}
    \scntext{примечание}{Определение онтологии дано в Главе 2.5. Структура баз знаний ostis-систем: иерархическая система предметных областей и соответствующих им онтологий.}
\scnheader{геоонтология}
    \scnidtf{географическая онтология}
    \scnidtf{онтология предметных областей, в состав экземпляров объектов местности которых входят их геосемантические характеристики}
    \scnidtf{онтология предметных областей, экземпляры объектов которых используют пространственно-соотнесенные данные о территории, социальных и природных явлениях}
    \scnsubset{онтология}
    \scnhaselement{класс объектов местности}
    \scnhaselement{пространственное отношение}
\scnheader{класс объектов местности}
    \scnidtf{класс геопространственных понятий естественного или искусственного происхождения, природных явлений,
    имеющие общие признаки (семантические атрибуты), характерные для определенного класса объектов местности и описывающие внутренние характеристики понятия}
    \begin{scnhaselementrolelist}{пример}
    
    автомагистраль
    
    \end{scnhaselementrolelist}
    \scntext{примечание}{Класс объектов местности “автомагистраль”, обладает общим семантическим атрибутом “материал покрытия”,
    а смысловое значение данного признака принимает значение, определенное на шкале значений признаков {асфальт (асфальтобетон); цементобетон; булыжник; брусчатка; гравий; камень колотый; клинкер; шлак; щебень;
    битумоминеральная смесь; металл; твердое покрытие; грунт; лед; без покрытия}.}
\scnheader{объект местности}
    \scnidtf{определенный элемент земной поверхности естественного или искусственного происхождения, природное
    явление, реально существующие на рассматриваемый момент времени в пределах области локализации, для
    которого известно или может быть установлено местоположение, включая размеры и положение границ, и
    заданы признаки, отражающие семантические атрибуты такого элемента, характерные для определенного
    класса объектов местности, с заданными пространственными отношениями, отражающими связи с другими объектами местности}
    
\scnheader{Пункт 7.8.3.1. Базовая классификация объектов местности}

\scnheader{объект местности}
    \scnrelfrom {разбиение}{ \textbf{Типология объектов местности по тематике\^} }

\begin{scnindent}
    \begin{scneqtoset}
        \scnitem{водный объект местности (сооружение)}
        \scnitem{населенный объект местности}
        \scnitem{промышленный (сельскохозяйственный или социально-культурный) объект местности}
        \scnitem{дорожная сеть (сооружение)}
        \scnitem{растительный покров (грунт)}
    \end{scneqtoset}
\end{scnindent}

\scnheader{Рисунок. Уровни иерархии классов объектов местности}
    \scneq{\scnfileimage[40em]{images/1.png}}

    \begin{scnrelfromset}{ступени классификации в иерархии классификатора}
        \scnitem{код класса}
        \scnitem{код подкласса}
        \scnitem{код группы}
        \scnitem{код подгруппы}    
        \scnitem{код отряда}    
        \scnitem{код подотряда}    
        \scnitem{код вида}
        \scnitem{код подвида}   
    \end{scnrelfromset}
    \scntext{примечание}{Для каждого объекта местности выделены основные, присущие только ему, семантические характеристики.
    Особо отметим, что метрические характеристики таким свойством не обладают. Согласно данному классификатору
    каждый класс объектов местности имеет уникальное однозначное обозначение.Таким образом, благодаря способу кодирования уже заданы
    родовидовые связи, отражающие соотношения различных классов объектов местности, а также установлены характеристики конкретного класса объектов местности. В связи с тем, что задаются основные свойства и
    отношения не конкретных физических объектов, а их классов, то такая информация является по отношению
    к конкретным объектам местности метаинформацией, а совокупность данной метаинформации представляет
    собой онтологию объектов местности, которая в свою очередь является частью базы знаний интеллектуальной
    геоинформационной системы}
\scnheader{объект местности}
    \scnrelfrom {разбиение}{ \textbf{Типология объектов местности по локализации\^} }

\begin{scnindent}
    \begin{scneqtoset}
        \scnitem{точечный объект местности}
            
            \scnidtf{объект местности, который не выражается в масштабе карты}
            \begin{scnrelfromset}{пример*:включение}
                \scnitem{колодец}
                \scnitem{осветительная опора} 
                \scnitem{дорожный знак}
            \end{scnrelfromset}
            
        \scnitem{линейный объект местности}
            \scnidtf{объект местности, длина которого выражается в масштабе карты}
            \begin{scnrelfromset}{пример*:включение}
                \scnitem{мост}
            \end{scnrelfromset}
        \scnitem{полилинейный объект местности}
            \scnidtf{объект местности, состоящий из двух и более сегментов линейных объектов}
            \begin{scnrelfromset}{пример*:включение}
                \scnitem{река}
                \scnitem{дорога} 
                \scnitem{улица}
            \end{scnrelfromset}
        \scnitem{площадной объект местности}
            \scnidtf{объект местности, площадь которого выражается в масштабе карты}
            \begin{scnrelfromset}{пример*:включение}
                \scnitem{озеро}
                \scnitem{административный район} 
                \scnitem{государство}
            \end{scnrelfromset}
    \end{scneqtoset}
\end{scnindent}

\scnheader{Пункт 7.8.3.2. Геосемантические характеристики объектов местности}
    \scntext{примечание}{Особенностью геоонтологии является использование для формализации предметных областей специальных элементов, уточняющих пространственные характеристики объектов местности, которые назовем геосемантическими характеристиками объектов местности}
\scnheader{геосемантическая характеристика объекта местности}
    \begin{scnrelfromset}{разбиение}
            \scnitem{координатное местоположение объекта местности}
            \scnitem{пространственное отношение} 
            \scnitem{пространственное отношение главных направлений}
            \scnitem{ динамика состояния объекта местности}
    \end{scnrelfromset}
\scnheader{координатное местоположение объекта местности}
    \scnidtf{географическое положение, расположение объекта местности или явления, которое задается в системе геодезических координат}
\scnheader{геокодирование}
    \scnidtf{установление связи между объектом местности и его местоположением}   
    \scnsubset{действие}
\scnheader{пространственное отношение}
    \scnidtf{класс отношений, задающие семантические свойства объекта местности по отношению к другим объектам местности}
    \begin{scnrelfromset}{разбиение}
            \scnitem{топологическое пространственное отношение}
            \scnitem{отношение пространственной упорядоченности} 
            \scnitem{метрическое пространственное отношение}
    \end{scnrelfromset}
\scnheader{отношение пространственной упорядоченности}
    \begin{scnrelfromset}{разбиение}
            \scnitem{отношение расположения объектов местности}
            \scnitem{отношение главных направлений объектов местности} 
    \end{scnrelfromset}
\scnheader{отношение расположения объектов местности}
    \scnsubset{ориентированное отношение}
    \scnidtf{позволяет определить, какое положение занимает один объект местности по отношению к другому объекту местности}
    \scnhaselement{объект местности располагается перед другим объектом местности*}
    \scnhaselement{объект местности располагается за другим объектом местности*}
    \scnhaselement{объект местности располагается слева от другого объекта местности*}
        \begin{scnindent}
            \begin{scnhaselementrolelist}{пример}
            Водонапорная башня располагается слева от дороги
            \end{scnhaselementrolelist}   
        \end{scnindent}
        
    \scnhaselement{объект местности располагается справа от другого объекта местности*}
    \scnhaselement{объект местности располагается над другим объектом местности*}
    \scnhaselement{объект местности располагается под другим объектом местности*}
    \scnhaselement{объект местности располагается ближе другого объекта местности*}
    \scnhaselement{объект местности располагается дальше другого объекта местности*}
    
\scnheader{отношение главных направлений объектов местности}
    \scnsubset{ориентированное отношение}
    \scnidtf{позволяет определить, какое главное направление занимает один объект местности по отношению к другому объекту местности}
    \scntext{примечание}{позволяет определить, какое главное направление занимает один объект местности по отношению к другому объекту местности}
    \scnhaselement{объект местности по отношению к другому объекту местности занимает главное направление север*}
    \scnhaselement{объект местности по отношению к другому объекту местности занимает главное направление северо-восток*}
    \scnhaselement{объект местности по отношению к другому объекту местности занимает главное направление восток*}
    \scnhaselement{объект местности по отношению к другому объекту местности занимает главное направление юго-восток*}
    \scnhaselement{объект местности по отношению к другому объекту местности занимает главное направление юг*}
    \scnhaselement{объект местности по отношению к другому объекту местности занимает главное направление юг-запад*}
    \scnhaselement{объект местности по отношению к другому объекту местности занимает главное направление запад*}
    \scnhaselement{объект местности по отношению к другому объекту местности занимает главное направление северо-запад*}

\scnheader{метрическое пространственное отношение}
    \scnidtf{характеризует информацию о расстоянии между объектами местности}
    \scnrelfrom{измерение}{километр}
    \scnrelfrom{измерение}{метр}
    \scnsuperset{шкальное метрическое пространственное отношение}
        \begin{scnindent}
            \scnrelfrom{измерение}{шкала}
        \end{scnindent}

\scnheader{система геодезических координат}
    \scnidtf{система координат, используемая для определения местоположения объектов на Земле}
    \begin{scnhaselementrolelist}{пример}
        WGS84
        
        \scntext{примечание}{Всемирная система геодезических параметров Земли 1984 года, в число которых входит система геоцентрических координат. В отличие от локальных систем, является единой системой для всей планеты.}
    \end{scnhaselementrolelist}
    \begin{scnhaselementrolelist}{пример}
        CК-63
    \end{scnhaselementrolelist}
    \begin{scnhaselementrolelist}{пример}
        CК-95
    \end{scnhaselementrolelist}

\scnheader{Пункт 7.8.3.3. Топологические пространственные отношения между объектами местности}
    \begin{scnrelfromset}{топологические пространственные отношения, которые можно установить между экземплярами объектов местности}
        \scnitem{включение*}
        \scnitem{граничить*}  
        \scnitem{пересечение*}  
        \scnitem{примыкание*}  
    \end{scnrelfromset}

\scnheader{топологическое пространственное отношение}
    \scnidtf{[класс пространственных отношений, заданных над объектами местности, находящихся в отношении связности и смежности между объектами местности}
    \scnrelfrom{прмечание}{топологическое пространственное отношение инвариантно относительно перемещения, поворота и масштабирования}
    \scnhaselement{включение*}
    \begin{scnindent}
    
        \scnsuperset{включение точечного объекта местности в площадной объект местности*}
        \begin{scnindent}
            \scnrelfrom{пример}{\scnfileimage[10em]{images/2.png}}
        \end{scnindent}
        
        \scnsuperset{включение линейного (полилинейного) объекта местности в площадной объект местности*}
        \begin{scnindent}
            \scnrelfrom{пример}{\scnfileimage[10em]{images/3.png}}
        \end{scnindent}
        
        \scnsuperset{включение площадного объекта местности в площадной объект местности*}
        \begin{scnindent}
            \scnrelfrom{пример}{\scnfileimage[10em]{images/4.png}}
        \end{scnindent}
    \end{scnindent}
    
    \scnhaselement{граничить*}
        \begin{scnindent}
            \scnrelfrom{пример}{\scnfileimage[10em]{images/5.png}}
        \end{scnindent}

    \scnhaselement{пересечение*}
    \begin{scnindent}
    
        \scnsuperset{пересечение двух линейных (полинейных) объектов местности*}
        \begin{scnindent}
            \scnrelfrom{пример}{\scnfileimage[10em]{images/6.png}}
        \end{scnindent}
        
        \scnsuperset{пересечение линейного (полинейного) и площадного объектов местности*}
        \begin{scnindent}
            \scnrelfrom{пример}{\scnfileimage[10em]{images/7.png}}
        \end{scnindent}
        
    \end{scnindent}
    
    \scnhaselement{примыкание*}
        \begin{scnindent}
            \scnrelfrom{пример}{\scnfileimage[10em]{images/8.png}}
        \end{scnindent}
        
    \scntext{примечание}{Отношение “включения*” будет устанавливаться между площадным и линейным, площадным и точечным, площадными объектами местности. Отношение “пересечения*” будет устанавливаться между линейными и площадными и линейными объектами местности. Отношение “граничить*” будет устанавливаться между площадными
    объектами местности. Отношение “примыкания*” устанавливается между линейными объектами местности. Для всех картографических отношений существуют структуры для их хранения.}

\scnheader{Пункт 7.8.3.4. Стратифицированная модель информационного пространства объектов местности}
    \scntext{примечание}{С целью интеграции предметных областей с пространственными компонентами геоинформационных систем,
    соответственно повышения интероперабельности этих систем, предлагается Стратифицированная модель информационного пространства объектов местности.}
    
\scnheader{Рисунок. Способ задания Стратифицированной модели информационного пространства объектов местности }
    \scneq{\scnfileimage[40em]{images/9.png}}

\scnheader{Рисунок. Стратифицированная модель информационного пространства объектов местности }
    \scneq{\scnfileimage[40em]{images/10.png}}
    \scntext{примечание}{На Рисунок. Стратифицированная модель информационного пространства объектов местности представлена
    геометрическая интерпретация предложенной гибридной модели знаний, где показано, что слой экземпляров объектов местности является интегрирующим слоем с предметными знаниями различных предметных областей, в которых уже непосредственно используются конкретные объекты местности. При такой организации знаний возможно многократно использовать разработанную онтологию объектов местности в разных предметных областях и, соответственно, для решения различных прикладных задач.}

\scnheader{§ 7.8.4. Решатель задач интеллектуальной геоинформационной ostis-системы}

    \scntext{примечание}{ешатель задач интеллектуальной геоинформационной ostis-системы представляет собой коллектив взаимодействующих друг с другом sc-агентов, позволяющих решать геоинформационные задачи. Ниже приведена базовая декомпозиция решателя задач интеллектуальной геоинформационной системы на основные классы  sc-агентов геоинформационного назначения.}

\scnheader{решатель задач интеллектуальной геоинформационной системы}
    \begin{scnrelfromset}{декомпозиция}
        \scnitem{Абстрактный sc-агент вычисления геометрических характеристик объектов местности}
        \begin{scnindent}
            \begin{scnrelfromset}{декомпозиция}
                \scnitem{Абстрактный sc-агент обработки точечных объектов местности}
                \scnitem{Абстрактный sc-агент обработки линейных (полилинейных) объектов местности}   
                \scnitem{Абстрактный sc-агент обработки площадных объектов местности}
            \end{scnrelfromset}
        \end{scnindent}
        \scnitem{Абстрактный sc-агент определения типа локализации объекта местности}
        \scnitem{Абстрактный sc-агент сопряжения с различными картографическими системами и сервисами, системами измерений и временными интервалами} 
        \begin{scnindent}
            \begin{scnrelfromset}{декомпозиция}
                \scnitem{Абстрактный sc-агент сопряжения с картографическими системами}
                \scnitem{Абстрактный sc-агент сопряжения с единицами измерений}   
                \scnitem{Абстрактный sc-агент сопряжения с временными интервалами}
            \end{scnrelfromset}
        \end{scnindent}
        \scnitem{Абстрактный sc-агент установления топологических связей между объектами местности}   
        \scnitem{Абстрактный sc-агент верификации базы знаний объектов местности}   
        \begin{scnindent}
            \begin{scnrelfromset}{декомпозиция}
                \scnitem{Абстрактный sc-агент верификации полноты заполнения базы знаний объектов местности}
                \scnitem{Абстрактный sc-агент верификации корректности значений семантических атрибутов объектов местности}   
                \scnitem{Абстрактный sc-агент верификации корректности значений пространственных атрибутов объектов местности}
            \end{scnrelfromset}
        \end{scnindent}
    \end{scnrelfromset}

\scnheader{§ 7.8.5. Картографический интерфейс интеллектуальной геоинформационной ostis-системы}

\scnheader{картографический интерфейс}
    \scntext{общее определение}{Картографический интерфейс — это пользовательский интерфейс, предназначенний для визуального отображения объектов местности на каком-то языке карт}

\scnheader{свойства картографического интерфейса}
    \begin{scnrelfromset}{разбиение}
        \scnitem{высокий уровень согласованности понятий, используемых при визуализации информации об объектах местности}
        \scnitem{высокий уровень простоты (естественностью) и понятности для любого конечного пользователя (интерфейс должен быть снисходительным к уровню подготовки пользователя, то есть дружелюбным)}  
        \scnitem{высокий уровень привлекательности (естетичности) и легкости восприятия}
        \scnitem{и так далее}  
    \end{scnrelfromset}
    \scntext{примечание}{Частным видом картографического интерфейса является картографический интерфейс интеллектуальной геоинформационной ostis-системы.}

\scnheader{Денотационная семантика Языка карт для ostis-систем}
    \begin{scnrelfromset}{включает в себя}
        \scnitem{пространственные отношения объектов местности}  
        \scnitem{геосемантические характеристики объектов местности}
    \end{scnrelfromset}

\scnheader{Язык карт для ostis-систем}
    \scntext{примечание}{Разрабатываемый Язык карт для ostis-систем относится к семейству семантически совместимых языков — sc-языков и предназначен для формального описания объектов местности и отношений между ними в геоинформационных системах. Поэтому Синтаксис Языка карт для ostis-систем, как и синтаксис любого другого sc-языка, является Синтаксисом SC-кода.}
    \begin{scnrelfromset}{позволяет}
        \scnitem{использовать минимум средств для интерпретации заданных объектов местности на карте}
        \scnitem{использовать Язык вопросов для ostis-систем}  
        \scnitem{сводить поиск на большую часть заданных вопросов к поиску информации в текущем состоянии базы знаний ostis-системы}
    \end{scnrelfromset}
    \scnidtf{Предлагаемый нами вариант внешнего языка для представления и визуализации информации об объектах местности в картографическом интерфейсе интеллектуальной геоинформационной ostis-системы}
    \scniselement{sc-язык}
    \scnrelfrom{синтаксис языка}{Синтаксис Язык карт для ostis-систем}
    \begin{scnindent}
        \scnsubset{Синтаксис SC-кода}
    \end{scnindent}
    
    \scnrelfrom{денотационная семантика языка}{Денотационная семантика Языка карт для ostis-систем}
    \begin{scnindent}
        \scnidtf{Онтологии объектов местности, пространственных отношений между ними и их геосемантических характеристик}
        \scnsuperset{Семантическая классификация вопросов}
    \end{scnindent}
    
    \scnrelfrom{операционная семантика языка}{Операционная семантика Языка карт для ostis-систем}
    \begin{scnindent}
        \scnidtf{Коллектив sc-агентов интерпретации Языка карт для ostis-систем}
    \end{scnindent}
    
    \scntext{примечание}{Ключевые достоинства картографического интерфейса интеллектуальной геоинформационной ostis-системы и
    соответствующего ему Языка карт для ostis-систем по сравнению с другими видами пользовательских интерфейсов состоит в том, что с помощью их можно визуализировать любые классы объектов местности и информацию
    о них достаточно простым способом, понятным любому пользователю.}

\scnheader{Рисунок. Описание Реки Березина на SCg-коде}
    \scneq{\scnfileimage[40em]{images/11.png}}
    \scntext{поянение к рисунку}{На данном рисунке показан пример отображения семантической окрестности объекта местности “Река Березина” на SCg-коде}

\scnheader{Рисунок. Описание Реки Березина на Языке карт}
    \scneq{\scnfileimage[40em]{images/12.png}}
    \scntext{поянение к рисунку}{На данном рисунке показан пример отображения семантической окрестности объекта местности “Река Березина” на Языке карт}

\scnheader{Средства, при помощи которых происходит отображение информации об объекте местности}
    \begin{scnrelfromset}{разбиение}
        \scnitem{Метасистемы OSTIS (см. Главу 7.2. Метасистема OSTIS)}
        \scnitem{Технологии OpenMapStreet}   
    \end{scnrelfromset}
    
\scnheader{Этапы отображения информации об объекте местности}
    \begin{scnrelfromset}{разбиение}
        \scnitem{поиск семантической окрестности заданного объекта местности (то есть пространственных отношений между заданным объектом местности и другими объектами местности, а также геосемантических характеристик заданного объекта местности)}
        \scnitem{определение географических кодов территориальных объектов местности, в которых располагается заданный объект местности, а также других объектов местности, связанных пространственными отношениями с заданным объектом местности}   
        \scnitem{получение картографических данных о территориальных объектах местности и их соотнесение с информацией о территориальных объектах местности в базе знаний}
        \scnitem{определение класса объекта местности}
        \scnitem{визуализация семантической окрестности заданного объекта местности на Языке карт}
    \end{scnrelfromset}

\scnheader{§ 7.8.6. Многократно используемые компоненты интеллектуальных
геоинформационных ostis-систем}
    \scntext{примечание}{Одним из требованием, предъявляемым к Технологии проектирования интеллектуальных геоинформационных
    систем, является обеспечение возможности совместного использования в рамках интеллектуальных геоинформационных ostis-систем различных видов знаний и различных моделей решения задач, а также различных видов интерфейсов. Следствием данного требования является необходимость реализации компонентного подхода на всех уровнях, от простых компонентов баз знаний, решателей задач и интерфейсов до целых встраиваемых ostis-систем (см. § 5.1.3. Понятие многократно используемого компонента ostis-систем). Таким образом, база знаний интеллектуальной геоинформационной ostis-системы должна включать спецификацию объектов местности, то есть описание их семантических окрестностей (см. SCg-текст. Пример многократно используемого компонента базы знаний интеллектуальных геоинформационных ostis-систем), решатель задач интеллектуальной геоинформационной ostis-системы — спецификацию sc-агентов, решающих различные геоинформационные задачи (например, sc-агент классификации географических объектов (см. SCg-текст. Пример многократно используемого компонента решателя задач интеллектуальных геоинформационных ostisсистем), sc-агент определения пространственных отношений между объектами местности и так далее), а
    картографический интерфейс интеллектуальной геоинформационной ostis-системы — описание отношений между объектами местности на карте.}
    
\scnheader{SCg-текст. Пример многократно используемого компонента базы знаний интеллектуальных геоинформационных ostis-систем}
    \scneq{\scnfileimage[40em]{images/13.png}}  
 
\scnheader{SCg-текст. Пример многократно используемого компонента решателя задач интеллектуальных геоинформационных ostis-систем}
    \scneq{\scnfileimage[40em]{images/14.png}}  

\scnheader{§ 7.8.7. Средства автоматизации проектирования интеллектуальных геоинформационных ostis-систем}

\scnheader{Этапы проектирования интеллектуальных геоинформационных систем}
    \begin{scnrelfromset}{разбиение}
        \scnitem{первый этап}
            \begin{scnindent}
                \scntext{пояснение}{На первом этапе формируется база знаний предметной области и с этой целью анализируется электронная карта и осуществляется
                трансляция в базу знаний объектов местности с установлением геосемантических характеристик объектов местности для соответствующей территории. На данном этапе определяется, во-первых, к какому классу принадлежит исследуемый объект местности и, далее в зависимости от типа объекта, формируется понятие базы знаний, соответствующее конкретному объекту местности. Таким образом, создается множество понятий, описывающих конкретные объекты местности для каждого класса объектов местности. Следует отметить, что именно на данном этапе формирования базы знаний устанавливаются геосемантические характеристики объектов местности.}
            \end{scnindent}
        \scnitem{второй этап}   
            \begin{scnindent}
                \scntext{пояснение}{а втором этапе проектирования интеллектуальной геоинформационной системы происходит интеграция полученной на первом этапе базы знаний с внешними базами знаний. На этом этапе, помимо географических знаний, добавляться знания смежных предметных областей, тем самым становится возможным установление межпредметных связей. Наглядным примером служит интеграция с биологическими классификаторами, которые в реализации представляют собой онтологию объектов флоры и фауны. Такая интеграция расширяет функциональные и интеллектуальные возможности прикладной интеллектуальной геоинформационной системы. Отметим, что на данном этапе снимается омонимия в названиях объектов местности, принадлежащих классам населенных пунктов. Для
                населенных пунктов Республики Беларусь это достигается за счет использования системы обозначений объектов
                административно-территориального деления и населенных пунктов и осуществляется семантическое сопоставление объектов местности}
            \end{scnindent}
    \end{scnrelfromset}
    
\scnheader{Принцип, по которому осуществляется семантическое сопоставление объектов местности}
    \begin{scnrelfromset}{разбиение}
        \scnitem{определяется класс объекта местности}
        \scnitem{определяется подкласс, вид, подвид объекта местности в соответствии с классификатором объектов местности, то есть виды объектов местности в геоонтологии}   
        \scnitem{определяются атрибуты и характеристики, которые присущи данному классу объектов местности}
        \scnitem{определяются значения характеристик для данного класса объекта местности}
        \scnitem{устраняется омонимия идентификации}
        \scnitem{устанавливаются соответствующие связи между объектом местности и понятием в базе знаний с установленными геосемантическими характеристиками объектов местности}
        \scnitem{устанавливаются пространственные отношения между объектами местности, отнесенными к определенным классам}
    \end{scnrelfromset}    

\scnheader{Рисунок. Этапы проектирования баз знаний интеллектуальных геоинформационных ostis-систем}
    \scneq{\scnfileimage[40em]{images/15.png}}

\scnheader{Заключение к Главе 7.8.}

\scnheader{основные положения данной главы}
    \begin{scnrelfromset}{разбиение}
        \scnitem{развитие геоинформационных систем заключается в их интеллектуализации, благодаря чему расширяется круг прикладных задач с использованием знаний об объектах местности}
        \scnitem{предложено карту рассматривать как информационную конструкцию, элементами которой являются объекты местности, тем самым обеспечивается интероперабельность геоинформационных систем за счет перехода от карты к семантическому описанию элементов карты, то есть объектов местности и связей (пространственных отношений) между ними}   
        \scnitem{обеспечение интероперабельности достигается благодаря разработке онтологий предметных областей, а установление геосемантических характеристик объектов местности позволяет задавать пространственные характеристики объектов местности}
        \scnitem{наличие частной Технологии проектирования интеллектуальных геоинформационных систем обеспечивает процесс проектирования интеллектуальных геоинформационных систем, построенных по принципам ostis-систем}
    \end{scnrelfromset}

\newpage
\end{SCn}
\end{document}
Footer
© 2023 GitHub, Inc.
Footer navigation
Terms
Privacy
Security
Status
Docs
Contact GitHub
Pricing
API
Training
Blog
About
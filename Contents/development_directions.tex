\scseparatedfragment{Общие направления развития}

\begin{SCn}
	
\scnheader{Направления развития Cтандарта OSTIS}
\scneqtovector{
\scnfileitem{Все sc-идентификаторы, входящие в состав scg-текстов, scn-текстов и различных иллюстраций должны иметь одинаковый шрифт и размер. За исключением, возможно размера sc-идентификаторов в scg-текстах.
(См., например, страницы 443-448 стандарта-2021)};
\scnfileitem{Уточнить Алфавит SCg-кода }
}


\scnheader{Cтандарт OSTIS}
\scnrelfromset{направления развития}{
	\scnfileitem{Cтандарт OSTIS}
	\scnaddlevel{1}
		\scnrelfromvector{направления перманентного развития}{
			\scnfileitem{В рамках титульных спецификация разделов Стандарта OSTIS постоянно уточнять и детализировать семантические связи каждого раздела с другими разделами};
			\scnfileitem{Когда новая именуемая (идентифицируемая) сущность, описываемая в Стандарте OSTIS (в первую очередь, каждое понятие, должна быть подробно специфицирована в соответствующем разделе}}
	\scnaddlevel{-1}
	\scnaddlevel{1}
		\scnrelfromset{направления текущего этапа развития}{
			\scnfileitem{Доработать в текущую версию Стандарта OSTIS и ввести:
				\begin{scnitemize}
					\item Оглавление
					\item Титульную спецификацию Стандарта OSTIS
				\end{scnitemizeii}};
			\scnfileitem{Привести текущую версию Стандарта OSTIS в соответствие с новой версией оглавления Стандарта OSTIS};
			\scnfileitem{Включить расширенный материал моей статьи в Шпрингер-2021 в текущую версию Стандарта OSTIS};
			\scnfileitem{Включить в Стандарта OSTIS все методические рекомендации над развитием Стандарта OSTIS (правила построения различных фрагментов, правила идентификации sc-элементов, правила спецификации sc-элементов - точнее обозначение сущностей, направления развития)}
	}
	\scnaddlevel{-1}
}


\scnheader{Cтандарт OSTIS -- Общий орг-план}
\scnrelfromset{направления развития}{
\scnfileitem{Доработать правила оформления Стандарта OSTIS:
	\begin{scnitemize}
		\item Правила структуризации, типология компонентов, их порядок семантической связи между компонентами
		\item Правила идентификации sc-элементов (именования различных классифицированных сущностей)
		\item Правила спецификации различных классов sc-элементов
	\end{scnitemizeii}};
\scnfileitem{Существенно расширить библиографию Стандарта OSTIS и библиографии всех разделов.
	“Преобразовать” простой список библ. источников в библиографическую спецификацию соответствующего текста};
\scnfileitem{Чётко сформировать общий план доработки всего текста Стандарта OSTIS, а также конкретные планы доработки каждого раздела};
\scnfileitem{Содержание всех работ, опубликованных авторами Стандарта OSTIS по Технологии OSTIS должны быть формализованы и включены в соответствующие разделы Стандарта OSTIS. (имеются в виду отчёты по лабораторным работам студентов, расчётные работы, курсовые проекты,дипломные проекты, диссертации, статьи в материалах конференций OSTIS, в Шпрингеровских сборниках, в журнале Онтология проектирования и других изданиях). Более того, если первичной публикацией новых материалов будет их включение в состав Стандарта OSTIS, то это существенно повысит результативность работа по развитию Стандарта OSTIS};
\scnfileitem{Включить в текст монографии все мои статьи на конференции OSTIS и другие публикации (в том числе книги)
	\begin{scnitemize}
		\item Это актуально для работы над изданием Стандарта OSTIS
		(в эту версию Стандарта OSTIS  надо собрать абсолютно всё, что нами сделано)
	\end{scnitemizeii}
};
\scnfileitem{Увеличить число ссылок на библиографические источники из текста Стандарта OSTIS}
}


\scnheader{Cтандарт OSTIS}
\scnrelfromset{направления развития}{
\scnfileitem{Некоторые материалы текущего состояния ряда разделов целесообразно перенести в дочерние разделы (если имеющаяся детализация этих материалов более уместна для дочерних разделов).
	Это, например, касается некоторых сегментов раздела “Анализ методологических проблем современного состояния работ в области “Искусственного интеллекта” (в частности сегмента об Экосистеме OSTIS)};
\scnfileitem{Дополнить этот список разделов};
\scnfileitem{Совершенствовать стратификацию}
}


\scnheader{Правила организации развития исходного текста Стандарта OSTIS}
\scnidtf{Правила организации коллективной деятельности по развитию исходного текста Стандарта OSTIS}


\scnheader{Cтандарт OSTIS}
\scnrelfromvector{правила построения}{
\scnfileitem{Каждому разделу принимать одного ответственного редактора и возможно несколько соавторов (ответственный редактор является  единственным автором)};
\scnfileitem{LaTeX + макросы};
\scnfileitem{GitHub  - структуризация файлов OSTIS.ai};
\scnfileitem{Предложения /рецензирования/ включения};
\scnfileitem{Рецензии: 
	\begin{scnitemize}
		\item на уровне 
	\end{scnitemizeii}
};
\scnfileitem{Извлечение и конвертирование в pdf-файл любого раздела или группы разделов};
\scnfileitem{Просмотр pdf-файла};
\scnfileitem{Конвертирование в scs};
\scnfileitem{Загрузка в sc-память };
\scnfileitem{Просмотр базы знаний}
}


	
\end{SCn}
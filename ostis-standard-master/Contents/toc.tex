
\scseparatedfragment{Оглавление Стандарта OSTIS}

\begin{SCn}

\scnsectionheader{\currentname}
\scnidtf{Оглавление текущей версии Стандарта OSTIS}
\scnexplanation{Иерархический перечень разделов, входящих в состав \textit{Стандарта OSTIS}, с дополнительной спецификацией некоторых разделов, указывающей альтернативные названия разделов, а также их авторов и редакторов}
\scnaddlevel{1}
\scnnote{Существенно подчеркнуть, что иерархия разделов \textit{Стандарта OSTIS} как и \textit{разделов} любой другой \textit{базы знаний} не означает то, что \textit{разделы} более низкого уровня иерархии входят в состав (являются частями) соответствующих разделов более высокого уровня. Связь между \textit{разделами} разных уровней иерархии означает то, что \textit{раздел} более низкого уровня иерархии является \textit{дочерним} разделом по отношению к соответствующему \textit{разделу} более высокого уровня, т.е. \textit{разделом}, который наследует свойства указанного \textit{раздела} более высокого уровня.}
\scnaddlevel{-1}
\scnnote{Описание логико-семантических связей каждого раздела \textit{Стандарта OSTIS} с другими разделами \textit{Стандарта OSTIS} приводится в рамках \textit{титульной спецификации} каждого \textit{раздела}.}
\scnnote{Названия тех \textit{разделов}, которые планируется написать в последующих изданиях \textit{Стандарта OSTIS} или \textit{разделов}, которые \uline{в данном} издании \textit{Стандарта OSTIS} не печатаются, поскольку содержание их не изменилось по сравнению с предыдущим \uline{явно указываемым} изданием \textit{Стандарта OSTIS}, выделяются также \uline{жирным курсивом}, но для них страницы в рамках данного издания \textit{Стандарта OSTIS} не указываются}
\scneqhierstruct


\end{SCn}

\normalsize 

\begingroup
\let\clearpage\relax
\tableofcontents
\endgroup

\begin{SCn}

\scnendhierstruct \scninlinesourcecommentpar{Завершили \textit{Оглавление Стандарта OSTIS}}

\end{SCn}
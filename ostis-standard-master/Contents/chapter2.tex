
\scchapter{Предметная область и онтология знаний и баз знаний ostis-систем}
\label{sec:sd_knowledge}
\begin{SCn}

\scnsectionheader{\currentname}

\scnstartsubstruct

\scnrelfromlist{дочерний раздел}{Предметная область и онтология множеств
    \scnaddlevel{1}
    \scnidtf{Предметная область и онтология \textit{знаний о множествах}}
        \scnaddlevel{1}
        \scnnote{\textit{знания о множествах} являются \uline{частным видом} \textit{знаний} и, следовательно, общие свойства сущностей, описываемых знаниями, могут наследоваться \textit{Предметной областью и онтологией множеств}}
        \scnaddlevel{-1}
    \scnaddlevel{-1}
;Предметная область и онтология связок и отношений
;Предметная область и онтология параметров, величин и шкал
;Предметная область и онтология чисел и числовых структур
;Предметная область и онтология структур
;Предметная область и онтология темпоральных сущностей
;Предметная область и онтология темпоральных сущностей баз знаний ostis-систем
;Предметная область и онтология семантических окрестностей
;Предметная область и онтология предметных областей
;Предметная область и онтология онтологий
;Предметная область и онтология логических формул, высказываний и формальных теорий
;Предметная область и онтология внешних информационных конструкций и файлов ostis-систем
;Глобальная предметная область действий и задач и соответствующая ей онтология методов и технологий}

\scnheader{Предметная область знаний и баз знаний ostis-систем}
\scniselement{предметная область}
\scnsdmainclasssingle{знание}
\scnhaselementlist{исследуемый класс классов}{вид знаний;отношение, заданное на множестве знаний}

\scnheader{знание}
\scnidtf{синтаксически корректная (для соответствующего языка) и семантически целостная информационная конструкция}
\scnsubset{информационная конструкция}
    \scnaddlevel{1}
    \scniselementrole{класс объектов исследования}{\nameref{intro_lang}}
    \scnaddlevel{-1}
\scnaddlevel{1}
\scnrelboth{следует отличать}{данные}
\scnaddlevel{1}
\scnexplanation{
	Принципиальные различия знаний и данных:
	
	\begin{scnitemize}
		\item \textit{Интерпретация}. Хранимые данные могут быть интерпретированы только человеком или программой. Данные не несут информации. Знания содержат как данные, так и их описание (правила интерпретации)
		\item \textit{Наличие связей классификации}. Данные не имеют эффективного описания связей между различными типами данных. Знания структурированы, так как можно установить соответствие между единицами знаний.
		\item \textit{Наличие ситуационных связей}. Связи описывают множество текущих ситуаций объекта. Данные трудно поддаются анализу. Из структуры и состава знаний по ситуации возможно построение процедур анализа знаний.
	\end{scnitemize}
}
\scnaddlevel{1}
\scnrelfrom{цитата}{Helpiks2015}
\scnaddlevel{-1}
\scnaddlevel{-1}
\scnaddlevel{-1}
\scnrelfrom{покрытие}{вид знаний
    \scnidtf{Множество \uline{всевозможных} видов знаний}
    \scnnote{Тот факт, что семейство \textit{видов знаний} является \textit{покрытием} Множества всевозможных \textit{знаний}, означает то, что каждое \textit{знание} принадлежит по крайней мере одному выделенному нами \textit{виду знаний}}}

    
    
   \scnheader{вид знаний}
\scnhaselement{спецификация}
    \scnaddlevel{1}
    \scnidtf{описание заданной сущности}
    \scnsuperset{спецификация материальной сущности}
    \scnsuperset{спецификация обратной сущности, не являющейся множеством}
        \scnaddlevel{1}
        \scnsuperset{спецификация геометрической точки}
        \scnsuperset{спецификация числа}
        \scnaddlevel{-1}
    \scnsuperset{спецификация множества}
        \scnaddlevel{1}
        \scnsuperset{спецификация связи}
        \scnsuperset{спецификация структуры}
        \scnsuperset{спецификация класса}
            \scnaddlevel{1}
            \scnsuperset{спецификация класса сущностей, не являющихся множествами}
            \scnsuperset{спецификация отношения}
                \scnaddlevel{1}
                \scnidtf{спецификация класса связей (связок)}
                \scnaddlevel{-1}
            \scnsuperset{спецификация класса классов}
                \scnaddlevel{1}
                \scnsuperset{спецификация параметра}
                \scnaddlevel{-1}
            \scnsuperset{спецификация класса структур}
            \scnsuperset{спецификация понятий}
                \scnaddlevel{1}
                \scnsuperset{пояснение}
                \scnsuperset{определение}
                \scnsuperset{утверждение}
                    \scnaddlevel{1}
                    \scnidtf{утверждение, описывающее свойства экземпляров (элементов) специфицируемого понятия}
                    \scnidtf{закономерность}
                    \scnaddlevel{-1}
                \scnaddlevel{-1}
            \scnaddlevel{-1}
        \scnaddlevel{-1}
    \scnsuperset{семантическая окрестность}
    \scnsuperset{однозначная спецификация}
    \scnsuperset{сравнительный анализ}
    \scnsuperset{достоинства}
    \scnsuperset{недостатки}
    \scnsuperset{структура специфицируемой сущности}
    \scnsuperset{принципы, лежащие в основе}
    \scnsuperset{обоснование предлагаемого решения}
        \scnaddlevel{1}
        \scnidtf{аргументация предлагаемого решения}
        \scnaddlevel{-1}
    \scnaddlevel{-1}

\scnhaselement{сравнение}

\scnhaselement{высказывание}
    \scnaddlevel{1}
    \scnsuperset{фактографическое высказывание}
    \scnsuperset{закономерность}
    \scnaddlevel{-1}

\scnhaselement{формальная теория}

\scnhaselement{предметная область}

\scnhaselement{предметная область и онтология
    \scnaddlevel{1}
    \scnidtf{предметная область и её онтология}
    \scnidtf{предметная область и соответствующая ей объединенная онтология}
    \scnaddlevel{-1}} 
 
\scnhaselement{метазнание}
    \scnaddlevel{1}
    \scnidtf{спецификация знания}
    \scnsuperset{аннотация}
    \scnsuperset{введение}
    \scnsuperset{предисловие}
    \scnsuperset{заключение}
    \scnsuperset{онтология}
        \scnaddlevel{1}
        \scnsuperset{онтология предметной области}
            \scnaddlevel{1}
            \scnsuperset{структурная онтология предметной области}
            \scnsuperset{теоретико-множественная онтология предметной области}
            \scnsuperset{логическая онтология предметной области}
            \scnsuperset{терминологическая онтология предметной области}
            \scnsuperset{объединенная онтология предметной области}
            \scnaddlevel{-1}
        \scnaddlevel{-1}
    \scnaddlevel{-1}

\scnhaselement{задача}
    \scnaddlevel{1}
    \scnidtf{спецификация действия}
    \scnaddlevel{-1}
\scnhaselement{план}

\scnhaselement{протокол}

\scnhaselement{результативная часть протокола}

\scnhaselement{метод}

\scnhaselement{технология}

\scnhaselement{история использования предметной области и её онтологии по решению информационных задач}
\scnhaselement{история использования предметной области и её онтологии по решению задач во внешней среде}
\scnhaselement{история эволюции предметной области и её онтологии}

\scnhaselement{база знаний}
    \scnaddlevel{1}
    \scnidtf{совокупность знаний, хранимых в памяти интеллектуальной компьютерной системы и \uline{достаточных} для того, чтобы указанная система удовлетворяла соответствующим предъявляемым к ней требованиям (в частности, чтобы она имела соответствующий уровень интеллекта)}
    \scnidtf{систематизированная совокупность знаний, хранимая в памяти интеллектуальной компьютерной системы и достаточная для обеспечения целенаправленного (целесообразного, адекватного) функционирования (поведения) этой системы как в своей внешней среде, так и в своей внутренней среде (в собственной базе знаний)}
    \scnrelfromset{обобщенная декомпозиция}{согласованная часть базы знаний
        \scnaddlevel{1}
        \scnidtf{часть базы знаний, признанная коллективом авторов на текущий момент}
        \scnaddlevel{-1}
    ;история эксплуатации базы знаний;история эволюции базы знаний;план эволюции базы знаний
        \scnaddlevel{1}
        \scnidtf{система специфицированных и согласованных действий авторов базы знаний, направленных на повышение её качества}
        \scnaddlevel{-1}}
    \scnnote{Основным факторами, определяющими качество интеллектуальной компьютерной системы, являются:
    \begin{scnitemize}
        \item качественная структуризация (систематизация) и \uline{стратификация} базы знаний интеллектуальной компьютерной системы, а также
        \item систематизация и стратификация \uline{деятельности}, которая осуществляется интеллектуальной компьютерной системой и спецификация которой является важнейшей частью базы знаний этой системы (Смотрите Раздел \textit{Глобальная предметная область действий и задач и соответствующая ей онтология методов и технологий}).
    \end{scnitemize}}
    \scnaddlevel{-1}
\scnnote{Даже небольшой перечень \textit{видов знаний} свидетельствует об огромном многообразии \textit{видов знаний}}

\scnheader{знание}
\scnsubdividing{декларативное знание
    \scnaddlevel{1}
    \scnidtf{\textit{знание}, имеющее \uline{только} \textit{денотационную семантику}, которая представляется в виде семантической \textit{спецификации} системы \textit{понятий}, используемых в этом \textit{знании}}
    \scnaddlevel{-1}
;процедурное знание
    \scnaddlevel{1}
    \scnidtf{\textit{знание}, имеющее не только \textit{денотационную семантику}, но и \textit{операционную семантику}, которая представляется в виде семейства \textit{спецификаций агентов}, осуществляющих интерпретацию \textit{процедурного знания}, направленную на решение некоторой инициированной \textit{задачи}}
    \scnidtf{функционально интерпретируемое знание, обеспечивающее решение либо конкретной задачи, либо некоторого множества инициируемых задач}
    \scnsuperset{задача}
        \scnaddlevel{1}
        \scnidtf{формулировка конкретной задачи}
        \scnsuperset{декларативная формулировка задачи}
        \scnsuperset{процедурная формулировка задачи}
        \scnaddlevel{-1}
    \scnsuperset{план}
        \scnaddlevel{1}
        \scnidtf{план решения конкретной задачи}
        \scnidtf{контекст конкретной задачи, предоставляющий всю информацию для решения всех подзадач для указанной конкретной задачи}
        \scnidtf{описание системы подзадач некоторой задачи}
        \scnaddlevel{-1}
    \scnsuperset{метод}
        \scnaddlevel{1}
        \scnidtf{обобщенное описание плана решения любой задачи из некоторого заданного класса задач}
        \scnaddlevel{-1}
    \scnsuperset{навык}
        \scnaddlevel{1}
        \scnidtf{метод, детализированный до уровня элементарных подзадач}
        \scnaddlevel{-1}
    \scnaddlevel{-1}}
    
\scnheader{отношение, заданное на множестве знаний}
\scnhaselement{дочернее знание*}
    \scnaddlevel{1}
    \scnidtf{знание, которое от "материнского"{} знания наследует все описанные там свойства объектов исследования}
    \scnnote{Факт наследования свойств описываемых объектов от "материнского"{} знания подчеркивается использованием прилагательного "дочернее"{} в sc-идентификаторе данного отношения, заданного на множестве знаний}
    \scnsuperset{дочерний раздел*}
        \scnaddlevel{1}
        \scnidtf{частный раздел*}
        \scnaddlevel{-1}
    \scnsuperset{дочерняя предметная область и онтология*}
    \scnaddlevel{-1}
\scnhaselement{спецификация*}
    \scnaddlevel{1}
    \scnidtf{быть знанием, которое является спецификацией (описанием) заданной сущности}
    \scnnote{специфицируемой сущностью может быть сущность любого вида, в том числе, и другое знание}
    \scnaddlevel{-1}
\scnhaselement{онтология*}
    \scnaddlevel{1}
    \scnidtf{быть семантической спецификацией заданного знания*}
    \scnaddlevel{-1}
\scnhaselement{семантическая эквивалентность*}
\scnhaselement{следовательно*}
    \scnaddlevel{1}
    \scnidtf{логическое следствие*}
    \scnaddlevel{-1}
\scnhaselement{логическая эквивалентность*}   
    
\bigskip    
\scnendstruct \scnendcurrentsectioncomment

\end{SCn}

\scsection{Предметная область и онтология множеств}
\label{sec:sd_sets}
\begin{SCn}

\scnsectionheader{\currentname}

\scnstartsubstruct

\scnheader{Предметная область множеств}
\scnidtf{Теоретико-множественная предметная область}
\scnidtf{Предметная область теории множеств}
\scnidtf{Предметная область, объектами исследования которой являются множества}
\scniselement{предметная область}
\scnsdmainclasssingle{множество}
\scnsdclass{конечное множество;бесконечное множество;счетное множество;несчетное множество;множество без кратных элементов;мультимножество;кратность принадлежности;класс;класс первичных sc-элементов;класс множеств;класс структур;класс классов;нечеткое множество;четкое множество;множество первичных сущностей;семейство множеств;нерефлексивное множество;рефлексивное множество;множество первичных сущностей и множеств;сформированное множество;несформированное множество;пустое множество;синглетон;пара;пара разных элементов;пара-мультимножество;тройка;кортеж;декартово произведение;булеан;мощность}
\scnsdrelation{принадлежность*;пример\scnrolesign;включение*;строгое включение*;объединение*;разбиение*;пересечение*;пара пересекающихся множеств*;попарно пересекающиеся множества*;пересекающиеся множества*;пара непересекающихся множеств*;попарно непересекающиеся множества*;непересекающиеся множества*;разность множеств*;симметрическая разность множеств*;декартово произведение*;семейство подмножеств*;булеан*;равенство множеств*}

\scnheader{множество}
\scnidtf{множество sc-элементов}
\scnidtf{sc-множество}
\scnidtf{множество знаков}
\scnidtf{множество знаков описываемых сущностей}
\scnidtf{семантически нормализованное множество}
\scnidtf{sc-знак множества sc-элементов}
\scnidtf{sc-знак множества sc-знаков}
\scnidtf{sc-текст}
\scnidtf{текст SC-кода}
\scnidtf{SC-код}
\scnsubdividing{конечное множество;бесконечное множество}
\scnsubdividing{множество без кратных элементов;мультимножество}
\scnsubdividing{связка;класс\\
    \scnaddlevel{1}
    \scnidtf{sc-элемент, обозначающий класс sc-элементов}
    \scnidtf{sc-знак множества sc-элементов, эквивалентных в том или ином смысле}
    \scnaddlevel{-1}
    ;структура\\
    \scnaddlevel{1}
    \scnidtf{sc-знак множества sc-элементов, в состав которого входят sc-связки или sc-структуры, связывающие эти sc-элементы}
    \scnaddlevel{-1}}
\scnsubdividing{четкое множество;нечеткое множество}
\scnsubdividing{множество первичных сущностей;множество множеств;множество первичных сущностей и множеств}
\scnsubdividing{рефлексивное множество;нерефлексивное множество}
\scnsubdividing{сформированное множество;несформированное множество}
\scnsubdividing{кортеж;неориентированное множество}
\scnsuperset{пустое множество}
\scnsuperset{синглетон}
\scnsuperset{пара}
\scnsuperset{тройка}
\scnexplanation{Под \textbf{\textit{множеством}} понимается соединение в некое целое M определённых хорошо различимых предметов m нашего созерцания или нашего мышления (которые будут называться «элементами» множества M). 

\textbf{\textit{множество}} – мысленная сущность, которая связывает одну или несколько сущностей в целое.

Более формально \textbf{\textit{множество}} – это абстрактный математический объект, состоящий из элементов. Связь множеств с их элементами задается бинарным ориентированным отношением \textbf{\textit{принадлежность*}}.

\textbf{\textit{множество}} может быть полностью задано следующими тремя способами:
\begin{scnitemize}
    \item путем перечисления (явного указания) всех его элементов (очевидно, что таким способом можно задать только конечное множество)
    \item с помощью определяющего высказывания, содержащего описание общего характеристического свойства, которым обладают все те и только те объекты, которые являются элементами (т.е. принадлежат) задаваемого множества.
    \item с помощью теоретико-множественных операций, позволяющих однозначно задавать новые множества на основе уже заданных (это операции объединения, пересечения, разности множеств и др.)
\end{scnitemize}
Для любого семантически ненормализованного \textbf{\textit{множества}} существует единственное семантически нормализованное \textbf{\textit{множество}}, в котором все элементы, не являющиеся знаками множеств, заменены на знаки множеств.}

\scnheader{принадлежность*}
\scnidtf{принадлежность элемента множеству*}
\scnidtf{отношение принадлежности элемента множеству*}
\scniselement{бинарное отношение}
\scniselement{ориентированное отношение}
\scnexplanation{\textbf{\textit{принадлежность*}} – это бинарное ориентированное отношение, каждая связка которого связывает множество с одним из его элементов. Элементами отношения \textbf{\textit{принадлежность*}} по умолчанию являются \textit{позитивные sc-дуги принадлежности}.}

\scnheader{конечное множество}
\scnidtf{множество с конечным числом элементов}
\scnexplanation{\textbf{\textit{конечное множество}} – это \textit{множество}, количество элементов которого конечно, т.е. существует неотрицательное целое число \textit{k}, равное количеству элементов этого множества.}

\scnheader{бесконечное множество}
\scnidtf{множество с бесконечным числом элементов}
\scnsubdividing{счетное множество;несчетное множество}
\scnexplanation{\textbf{\textit{бесконечное множество}} – это \textit{множество}, в котором для любого натурального числа \textit{n} найдётся конечное подмножество из \textit{n} элементов.}

\scnheader{счетное множество}
\scnexplanation{\textbf{\textit{счетное множество}} – это \textit{бесконечное множество}, для которого существует \textit{взаимно-однозначное соответствие} с натуральным рядом чисел.}

\scnheader{несчетное множество}
\scnidtf{континуальное множество}
\scnexplanation{\textbf{\textit{несчетное множество}} - это \textit{бесконечное множество}, элементы которого невозможно пронумеровать натуральными числами.}

\scnheader{множество без кратных элементов}
\scnidtf{классическое множество}
\scnidtf{канторовское множество}
\scnidtf{множество, состоящее из разных элементов}
\scnidtf{множество без кратного вхождения элементов}
\scnidtf{множество, все элементы которого входят в него однократно}
\scnidtf{множество, не имеющее кратного вхождения элементов}
\scnexplanation{\textbf{\textit{множество без кратных элементов}} - это \textit{множество}, для каждого элемента которого существует только одна пара принадлежности, выходящая из знака этого множества в указанный элемент.}

\scnheader{мультимножество}
\scnidtf{множество, имеющее кратные вхождения хотя бы одного элемента}
\scnidtf{множество, по крайней мере один элемент которого входит в его состав многократно}
\scnexplanation{\textbf{\textit{мультимножество}} - это \textit{множество}, для которого существует хотя бы одна кратная пара принадлежности, выходящая из знака этого множества.}

\scnheader{кратность принадлежности}
\scnidtf{кратность принадлежности элемента}
\scnidtf{кратность вхождения элемента во множество}
\scniselement{параметр}
\scnexplanation{\textbf{\textit{кратность принадлежности}} - \textit{параметр}, значением которого являются числовые величины, показывающие сколько раз входит тот или иной элемент в рассматриваемое множество. Элементами данного параметра являются классы \textit{позитивных sc-дуг принадлежности}, связывающих данное множество с элементом, кратность вхождения которого в данное множество мы хотим задать.

Таким образом, кратное вхождение элемента в мультимножество может быть задано как явным указанием \textit{позитивных sc-дуг принадлежности} этого элемента данному \textit{множеству}, так и «склеиванием» этих дуг в одну и включением ее в некоторый класс \textbf{\textit{кратности принадлежности}}.}
\scnrelfrom{описание примера}{
\scnfilescg{figures/sd_sets/multiplicityOfMembership.png}
}

\scnheader{класс}
\scnidtf{класс sc-элементов}
\scnsubdividing{класс первичных sc-элементов;класс множеств}
\scnexplanation{\textbf{\textit{класс}} – множество элементов, обладающих какими-либо явно указываемыми общими свойствами.}

\scnheader{класс первичных sc-элементов}
\scnexplanation{\textbf{\textit{класс первичных sc-элементов}} – класс, элементами которого являются только \textit{sc-элементы}, не являющиеся знаками множеств.}

\scnheader{класс множеств}
\scnsubdividing{отношение;класс структур;класс классов}
\scnexplanation{\textbf{\textit{класс множеств}} – класс, элементами которого являются только \textit{sc-элементы}, являющиеся знаками множеств.}

\scnheader{класс структур}
\scnexplanation{\textbf{\textit{класс структур}} – класс, элементами которого являются \textit{структуры}.}

\scnheader{класс классов}
\scnexplanation{\textbf{\textit{класс классов}} – класс, элементами которого являются \textit{классы}.}

\scnheader{нечеткое множество}
\scnexplanation{\textbf{\textit{нечеткое множество}} – это \textit{множество}, которое представляет собой совокупность элементов произвольной природы, относительно которых нельзя точно утверждать – обладают ли эти элементы некоторым характеристическим свойством, которое используется для задания этого нечеткого множества. Принадлежность элементов такому множеству указывается при помощи \textit{нечетких позитивных sc-дуг принадлежности}.}

\scnheader{четкое множество}
\scnexplanation{\textbf{\textit{четкое множество}} – это \textit{множество}, принадлежность элементов которому достоверна и указывается при помощи \textit{четких позитивных sc-дуг принадлежности}.}

\scnheader{множество первичных сущностей}
\scnsuperset{класс первичных сущностей}
\scnsubset{множество}
\scnexplanation{\textbf{\textit{множество первичных сущностей}} – это \textit{множество}, элементы которого не являются знаками множеств.}

\scnheader{семейство множеств}
\scnidtf{множество множеств}
\scnsuperset{класс классов}
\scnexplanation{\textbf{\textit{семейство множеств}} – это \textit{множество}, элементами которого являются знаки множеств.}

\scnheader{нерефлексивное множество}
\scnexplanation{\textbf{\textit{нерефлексивное множеств}} – это \textit{множество}, знак которого не является элементом этого множества}

\scnheader{рефлексивное множество}
\scnexplanation{\textbf{\textit{рефлексивное множеств}} – это \textit{множество}, знак которого является элементом этого множества}

\scnheader{множество первичных сущностей и множеств}
\scnexplanation{\textbf{\textit{множество первичных сущностей и множеств}} – это \textit{множество}, элементами которого являются как знаки множеств, так и знаки сущностей, не являющихся множествами.}

\scnheader{сформированное множество}
\scniselement{ситуативное множество}
\scnexplanation{\textbf{\textit{сформированное множество }} - это \textit{множество}, все элементы которого известны и перечислены в данный момент.}

\scnheader{несформированное множество}
\scniselement{ситуативное множество}
\scnexplanation{\textbf{\textit{несформированное множество}} - это \textit{множество}, не все элементы которого известны и перечислены в данный момент.}

\scnheader{пустое множество}
\scniselement{мощность}
\scnexplanation{\textbf{\textit{пустое множество}} – это \textit{множество}, которому не принадлежит ни один элемент.}

\scnheader{синглетон}
\scniselement{мощность}
\scnidtf{множество мощности 1}
\scnidtf{одноэлементное множество}
\scnidtf{одномощное множество}
\scnidtf{множество, мощность которого равна 1}
\scnidtf{множество, имеющее мощность равную единице}
\scnidtf{синглетон из sc-элемента} 
\scnidtf{sc-синглеон}
\scnsubset{конечное множество}
\scnexplanation{\textbf{\textit{синглетон}} – это \textit{множество}, состоящее из одного элемента.

Другими словами - любое множество \textit{Si} есть \textbf{\textit{синглетон}} тогда и только тогда, когда существует принадлежность этому множеству, которая совпадает с любой принадлежностью этому множеству.}

\scnheader{пара}
\scniselement{мощность}
\scnidtf{множество мощности два}
\scnidtf{двухэлементное множество}
\scnidtf{двумощное множество}
\scnidtf{множество, мощность которого равна 2}
\scnidtf{sc-пара}
\scnidtf{пара sc-элементов}
\scnsubset{конечное множество}
\scnsubdividing{пара разных элементов;пара-мультимножество}
\scnexplanation{\textbf{\textit{пара}} – это \textit{множество}, состоящее из двух элементов.

Другими словами – любое множество есть \textbf{\textit{пара}} тогда и только тогда, когда существуют две различные принадлежности этому множеству такие, что любая принадлежность этому множеству совпадает с одной из них.}

\scnheader{пара разных элементов}
\scnidtf{канторовская пара}
\scnidtf{канторовская пара sc-элементов}
\scnidtf{канторовское двумощное множество}

\scnheader{пара-мультимножество}
\scnidtf{пара-петля}
\scnidtf{sc-петля}
\scnidtf{двумощное мультимножество}

\scnheader{тройка}
\scniselement{мощность}
\scnidtf{тройка}
\scnidtf{sc-тройка}
\scnidtf{множество мощности три}
\scnidtf{множество, мощность которого равна 3}
\scnsubset{конечное множество}
\scnexplanation{\textbf{\textit{тройка}} – это \textit{множество}, состоящее из трех элементов.

Другими словами – любое множество есть \textbf{\textit{тройка}} тогда и только тогда, когда существуют три различные принадлежности этому множеству такие, что любая принадлежность этому множеству совпадает с одной из них.}

\scnheader{кортеж}
%\scnidtf{кортеж}
\scnidtf{вектор}
\scnexplanation{\textbf{\textit{кортеж}} – это множество, представляющее собой упорядоченный набор элементов, т.е. такое множество, порядок элементов в котором имеет значение. Пары принадлежности элементов \textbf{\textit{кортежу}} могут дополнительно принадлежать каким-либо \textit{ролевым отношениям}, при этом, в рамках каждого \textbf{\textit{кортежа}} должен существовать хотя бы один элемент, роль которого дополнительно уточнена \textit{ролевым отношением}.}

\scnheader{пример\scnrolesign}
\scnidtf{типичный пример\scnrolesign}
\scnidtf{типичный экземпляр заданного класса\scnrolesign}
\scniselement{ролевое отношение}
\scnexplanation{\textbf{\textit{пример\scnrolesign}} – это \textit{ролевое отношение}, связывающее некоторое \textit{множество} с элементом, являющимся примером этого множества.}

\scnheader{включение*}
\scnidtf{включение множеств*}
\scnidtf{быть подмножеством*}
\scniselement{бинарное отношение}
\scniselement{ориентированное отношение}
\scniselement{транзитивное отношение}
\scnrelfrom{область определения}{множество}
\scnsuperset{строгое включение*}
\scntext{определение}{\textbf{\textit{включение*}} – это бинарное ориентированное отношение, каждая связка которого связывает два множества. Будем говорить, что \textit{Множество Si} \textbf{\textit{включает*}} в себя \textit{Множество Sj} в том и только том случае, если каждый элемент \textit{Множества Sj} является также и элементом \textit{Множества Si}.}
\scnrelfrom{описание примера}{
\scnfilescg{figures/sd_sets/inclusion.png}}
\scnaddlevel{1}
\scnexplanation{Множество {Sj} включается во множество \textit{Si}.}
\scnaddlevel{-1}

\scnheader{строгое включение*}
\scnidtf{строгое включение множеств*}
\scnsubset{включение*}
\scniselement{бинарное отношение}
\scniselement{ориентированное отношение}
\scnrelfrom{область определения}{множество}
\scntext{определение}{\textbf{\textit{строгое включение*}} – это \textit{бинарное ориентированное отношение}, областью определения которого является семейство всевозможных множеств. Будем говорить, что \textit{Множество Si} \textbf{\textit{строго включает*}} в себя \textit{Множество Sj} в том и только том случае, если каждый элемент \textit{Множество Sj} является также и элементом \textit{Множество Si}, при этом существует хотя бы один элемент \textit{Множество Si}, не являющийся элементом \textit{Множество Sj}.}
\scnrelfrom{описание примера}{
\scnfilescg{figures/sd_sets/strictInclusion.png}}
\scnaddlevel{1}
\scnexplanation{Множество \textit{Sj} строго включается во множество \textit{Si}.}
\scnaddlevel{-1}
\scnrelfrom{изображение}{
\scnfileimage{\includegraphics[width=0.4\linewidth]{figures/sd_sets/inclusion2.png}}}

\scnheader{объединение*}
\scnidtf{объединение множеств*}
\scniselement{квазибинарное отношение}
\scniselement{ориентированное отношение}
\scntext{определение}{\textbf{\textit{объединение*}} – это \textit{квазибинарное ориентированное отношение}, областью определения которого является семейство всевозможных множеств. Будем говорить, что \textit{Множество Si} является объединением \textit{Множество Sj} и \textit{Множество Sk} тогда и только тогда, когда любой элемент \textit{Множество Si} является элементом или \textit{Множество Sj} или \textit{Множество Sk}.}
\scnrelfrom{описание примера}{
\scnfilescg {figures/sd_sets/union.png}}
\scnaddlevel{1}
\scnexplanation{Множество \textit{Si} является объединением множеств \textit{Sj}, \textit{Sk} и \textit{Sm}.}
\scnaddlevel{-1}
\scnrelfrom{изображение}{
\scnfileimage{\includegraphics[width=0.6\linewidth]{figures/sd_sets/union2.png}}}

\scnheader{разбиение*}
\scnidtf{разбиение  множества*}
\scnidtf{объединение попарно непересекающихся множеств*}
\scnidtf{декомпозиция множества*}
\scniselement{квазибинарное отношение}
\scniselement{ориентированное отношение}
\scniselement{отношение декомпозиции}
\scntext{определение}{\textbf{\textit{разбиение*}} – это \textit{квазибинарное ориентированное отношение}, областью определения которого является семейство всевозможных множеств. В результате разбиения множества получается множество попарно непересекающихся множеств, объединение которых есть исходное множество.\\
Семейство множеств \{\textit{S1…, Sn}\} является разбиением множества \textit{Si} в том и только том случае, если:
\begin{scnitemize}
    \item семейство \{\textit{S1…, Sn}\} является семейством \textit{попарно непересекающихся множеств};
    \item семейство \{\textit{S1…, Sn}\} является покрытием множества \textit{Si} (или другими словами, множество \textit{Si} является \textit{объединением} множеств, входящих в указанное выше семейство)
\end{scnitemize}
}
\scnrelfrom{описание примера}{
\scnfilescg{figures/sd_sets/split.png}}
\scnaddlevel{1}
\scnexplanation{Множество \textit{Si} разбивается на множества \textit{Sj}, \textit{Sk} и \textit{Sm}.}
\scnaddlevel{-1}
\scnrelfrom{изображение}{
\scnfileimage{\includegraphics[width=0.5\linewidth]{figures/sd_sets/split2.png}}}

\scnheader{пересечение*}
\scnidtf{пересечение множеств*}
\scniselement{квазибинарное отношение}
\scniselement{ориентированное отношение}
\scntext{определение}{\textbf{\textit{пересечение*}} – это операция над множествами, аргументами которой являются два или большее число множеств, а результатом является множество, элементами которого являются все те и только те сущности, которые одновременно принадлежат каждому множеству, которое входит в семейство аргументов этой операции.\\
Будем говорить, что \textit{Множество Si} является пересечением \textit{Множество Sj} и \textit{Множество Sk} тогда и только тогда, когда любой элемент \textit{Множество Si} является элементом \textit{Множество Sj} и элементом \textit{Множество Sk}.}
\scnrelfrom{описание примера}{
\scnfilescg{figures/sd_sets/intersection.png}}
\scnaddlevel{1}
\scnexplanation{Множество \textit{Si} является результатом пересечения множеств \textit{Sj}, \textit{Sk} и \textit{Sm}.}
\scnaddlevel{-1}
\scnrelfrom{изображение}{
\scnfileimage{\includegraphics[width=0.5\linewidth]{figures/sd_sets/intersection2.png}}}

\scnheader{пара пересекающихся множеств*}
\scniselement{бинарное отношение}
\scniselement{неориентированное отношение}
\scnexplanation{\textbf{\textit{пара пересекающихся множеств*}} – \textit{бинарное неориентированное отношение} между двумя \textit{множествами}, имеющими непустое \textit{пересечение*}.}
\scntext{определение}{\textbf{\textit{пара пересекающихся множеств*}} – \textit{бинарное неориентированное отношение} между двумя \textit{множествами}, имеющими, по крайней мере, один общий для этих двух множеств элемент.}
\scnrelfrom{описание примера}{
\scnfilescg{figures/sd_sets/pairOfIntersectingSets.png}}
\scnaddlevel{1}
\scnexplanation{Множество \textit{Si} и множество \textit{Sj} являются парой пересекающихся множеств.}
\scnaddlevel{-1}
\scnrelfrom{изображение}{
\scnfileimage{\includegraphics[width=0.5\linewidth]{figures/sd_sets/pairOfIntersectingSets2.png}}}

\scnheader{попарно пересекающиеся множества*}
\scnidtf{семейство попарно пересекающихся множеств*}
\scnsuperset{пересекающиеся множества*}
\scniselement{отношение}
\scntext{определение}{\textbf{\textit{попарно пересекающиеся множества*}} – семейство множеств, каждая пара которых является парой пересекающихся множеств, т.е. каждая пара которых имеет хотя бы один общий элемент}
\scntext{примечание}{Не каждое \textit{семейство попарно пересекающихся множеств*} является \textit{семейством пересекающихся множеств*}, хотя обратное верно.}
\scnrelfrom{изображение}{
\scnfilescg{figures/sd_sets/pairwiseIntersectingSets.png}}
\scnaddlevel{1}
\scnexplanation{Множества \textit{Si}, \textit{Sj}, \textit{Sk} и \textit{Sl} являются попарно пересекающимися множествами.}
\scnaddlevel{-1}
\scnrelfrom{изображение}{
\scnfileimage{\includegraphics[width=0.7\linewidth]{figures/sd_sets/pairwiseIntersectingSets2.png}}}

\scnheader{пересекающиеся множества*}
\scnidtf{семейство пересекающихся множеств*}
\scnidtf{быть семейством пересекающихся множеств*}
\scnidtf{семейство множеств, имеющих по крайней мере один элемент, являющийся общим для всех этих множеств*}
\scnsuperset{попарно пересекающиеся множества*}
\scntext{определение}{\textbf{\textit{пересекающиеся множества*}} – это семейство множеств, имеющих по крайней мере один общий для всех этих множеств элемент}
\scnrelfrom{описание примера}{
\scnfilescg{figures/sd_sets/intersectingSets.png}}
\scnaddlevel{1}
\scnexplanation{Множества \textit{Si}, \textit{Sj}, \textit{Sk} и \textit{Sl} являются пересекающимися множествами.}
\scnaddlevel{-1}

\scnheader{пара непересекающихся множеств*}
\scniselement{бинарное отношение}
\scniselement{неориентированное отношение}
\scntext{определение}{\textbf{\textit{пара непересекающихся множеств*}} – это \textit{бинарное неориентированное отношение} между \textit{множествами}, результатом \textit{пересечения*} которых есть пустое множество.}
\scnrelfrom{описание примера}{
\scnfilescg{figures/sd_sets/pairOfNonIntersectingSets.png}}
\scnaddlevel{1}
\scnexplanation{Множества \textit{Si} и \textit{Sj} являются парой непересекающихся множеств.}
\scnaddlevel{-1}
\scnrelfrom{изображение}{
\scnfileimage{\includegraphics[width=0.5\linewidth]{figures/sd_sets/pairOfNonIntersectingSets2.png}}}

\scnheader{попарно непересекающиеся множества*}
\scnidtf{семейство попарно непересекающихся множеств*}
\scnsubset{непересекающиеся множества*}
\scntext{определение}{\textbf{\textit{попарно непересекающиеся множества*}} – семейство множеств, каждая пара которых является парой непересекающихся множеств, т.е. каждая пара которых не имеет ни одного общего элемента}
\scnrelfrom{изображение}{
\scnfilescg{figures/sd_sets/pairwiseNonIntersectingSets.png}}
\scnaddlevel{1}
\scnexplanation{Множества \textit{Si}, \textit{Sj}, \textit{Sk} и \textit{Sl} являются попарно непересекающимися множествами.}
\scnaddlevel{-1}

\scnheader{непересекающиеся множества*}
\scnidtf{семейство непересекающихся множеств*}
\scnidtf{быть семейством непересекающихся множеств*}
\scntext{определение}{\textbf{\textit{непересекающиеся множества*}} – это семейство множеств, не имеющих ни одного общего элемента для всех этих множеств}
\scnrelfrom{изображение}{
\scnfilescg{figures/sd_sets/nonIntersectingSets.png}
\scnexplanation{Множества \textit{Si}, \textit{Sj}, \textit{Sk} и \textit{Sl} являются непересекающимися множествами.}}

\scnheader{разность множеств*}
\scniselement{бинарное отношение}
\scniselement{ориентированное отношение}
\scntext{определение}{\textbf{\textit{разность множеств*}} – это \textit{бинарное ориентированное отношение}, связывающее между собой \textit{ориентированную пару}, первым элементом которой является уменьшаемое множество, вторым - вычитаемое множество, и множество, являющееся результатом вычитания вычитаемого из уменьшаемого, в которое входят все элементы первого множества, не входящие во второе множество.}
\scnrelfrom{описание примера}{
\scnfilescg{figures/sd_sets/setDifference.png}}
\scnaddlevel{1}
\scnexplanation{Множество \textit{Si} является результатом разности множеств \textit{Sj} и \textit{Sk}.}
\scnaddlevel{-1}
\scnrelfrom{изображение}{
\scnfileimage{
\includegraphics[width=0.5\linewidth]{figures/sd_sets/setDifference2.png}}}

\scnheader{симметрическая разность множеств*}
\scniselement{бинарное отношение}
\scniselement{ориентированное отношение}
\scntext{определение}{\textbf{\textit{симметрическая разность множеств*}} – это \textit{бинарное ориентированное отношение}, связывающее между собой \textit{пару} множеств и множество, являющееся результатом симметрической разности элементов указанной пары. Будем называть \textit{Множество Si} результатом симметрической разности \textit{Множества Sj} и \textit{Множества Sk} тогда и только тогда, когда любой элемент \textit{Множества Si} является или элементом \textit{Множества Sj} или \textit{Множества Sk}, но не является элементом обоих множеств.}
\scnrelfrom{описание примера}{
\scnfilescg{figures/sd_sets/symmetricDifferenceOfSets.png}
\scnexplanation{Множество \textit{Si} является результатом симметрической разности множеств \textit{Sj} и \textit{Sk}.}}
\scnrelfrom{изображение}{
\scnfileimage{\includegraphics[width=0.5\linewidth]{figures/sd_sets/symmetricDifferenceOfSets2.png}}}

\scnheader{декартово произведение*}
\scnidtf{декартово произведение множеств*}
\scnidtf{прямое произведение множеств*}
\scniselement{бинарное отношение}
\scniselement{ориентированное отношение}
\scntext{определение}{\textbf{\textit{декартово произведение*}} – это \textit{бинарное ориентированное отношение} между \textit{ориентированной парой} множеств и \textit{множеством}, элементами которого являются всевозможные упорядоченные пары, первыми элементами которых являются элементы первого из указанных множеств, вторыми – элементы второго из указанных множеств.}
\scnrelfrom{описание примера}{
\scnfilescg{figures/sd_sets/cartesianMultiplication.png}}
\scnaddlevel{1}
\scnexplanation{Множество \textit{Si} является результатом декартова произведения множеств \textit{Sj} и \textit{Sk}.}
\scnaddlevel{-1}

\scnheader{декартово произведение}
\scnidtf{второй домен отношения быть декартовым произведением}
\scnrelfrom{второй домен}{декартово произведение*}

\scnheader{семейство подмножеств*}
\scnidtf{семейство подмножеств заданного множества*}
\scniselement{бинарное отношение}
\scniselement{ориентированное отношение}
\scnsuperset{булеан*}
\scntext{определение}{\textbf{\textit{семейство подмножеств*}} – это \textit{бинарное ориентированное отношение} между множеством и некоторым семейством множеств, каждое из которых является подмножеством первого множества.}
\scnrelfrom{описание примера}{
\scnfilescg{figures/sd_sets/familyOfSubsets.png}
}

\scnheader{булеан*}
\scnidtf{булеан множества*}
\scnidtf{семейство всевозможных подмножеств заданного множества*}
\scniselement{бинарное отношение}
\scniselement{ориентированное отношение}
\scntext{определение}{\textbf{\textit{булеан*}} – это \textit{бинарное ориентированное отношение} между множеством и некоторым семейством множеств, каждое из которых является подмножеством первого множества.}
\scnrelfrom{описание примера}{
\scnfilescg{figures/sd_sets/boulean.png}
}

\scnheader{булеан}
\scnidtf{второй домен отношения быть булеаном}
\scnrelfrom{второй домен}{булеан*}

\scnheader{мощность}
\scnidtf{мощность множеств}
\scnidtf{кардинальное число}
\scnidtf{общее число вхождений элементов в заданное множество}
\scnidtf{класс эквивалентности, элементами которого являются знаки всех тех и только тех множеств, которые имеют одинаковую мощность}
\scnidtf{класс эквивалентности, соответствующий отношению быть парой множеств, имеющих одинаковую мощность (равномощность множеств)}
\scnidtf{величина мощности множеств}
\scnidtf{трансфинитное число}
\scnidtf{мощность по Кантору}
\scniselement{параметр}
\scnexplanation{\textbf{\textit{мощность}} – это \textit{параметр}, элементами которых являются \textit{множества}, имеющие одинаковое количество элементов. Значением данного параметра является числовая величина, задающая количество элементов, входящих в данный класс множеств, т.е. по сути, количество \textit{позитивных sc-дуг принадлежности}, выходящих из данного \textit{множества}.

Для двух множеств, имеющих одинаковую мощность, существует взаимно-однозначное соответствие между ними (между множествами вхождений элементов в эти множества – на случай мультимножеств).}
\scnrelfrom{описание примера}{
\scnfilescg{figures/sd_sets/power.png}
}

\scnheader{равенство множеств*}
\scniselement{бинарное отношение}
\scniselement{неориентированное отношение}
\scnidtf{быть равными множествами*}
\scntext{определение}{\textbf{\textit{равенство множеств}}* - бинарное неориентированное отношение, выражающее отношение равенства множеств.

Любые два множества являются равными множествами тогда и только тогда, когда первое является включением второго и второе является включением первого.}
\scnrelfrom{описание примера}{
\scnfilescg{figures/sd_sets/equalityOfSets.png}}
\scnaddlevel{1}
\scnexplanation{Множество \textit{Si} равно множеству \textit{Sj}.}
\scnaddlevel{-1}

\scnendstruct \scnendcurrentsectioncomment

\end{SCn}

\scsection{Предметная область и онтология связок и отношений}
\label{sec:sd_rels}
\begin{SCn}

\scnsectionheader{\currentname}

\scnstartsubstruct

\scnheader{Предметная область связок и отношений}
\scniselement{предметная область}
\scnsdmainclasssingle{связь}
\scnsdclass{бинарная связь;sc-коннектор;неатомарная бинарная связь;небинарная связь;неориентированная связь;ориентированная связь;отношение;класс равномощных связок;класс связок разной мощности;унарное отношение;бинарное отношение;квазибинарное отношение;тернарное отношение;небинарное отношение;ориентированное отношение;неориентированное отношение;рефлексивное отношение;антирефлексивное отношение;частично рефлексивное отношение;симметричное отношение;антисимметричное отношение;частично симметричное отношение;транзитивное отношение;антитранзитивное отношение;частично транзитивное отношение;связанное отношение;отношение порядка;отношение строгого порядка;отношение нестрогого порядка;отношение толерантности;отношение эквивалентности;ролевое отношение;числовой атрибут;неролевое отношение;неролевое бинарное отношение;арность;метаотношение;отношение декомпозиции;отношение интеграции}
\scnsdrelation{область определения*;атрибут отношения*;домен*;первый домен*;второй домен*;композиция отношений*;фактор-множество*;соответствие*;отношение соответствия*;область отправления';область прибытия’;образ';прообраз';всюду определенное соответствие*;частично определенное соответствие*;сюръективное соответствие*;несюръективное соответствие*;однозначное соответствие*;обратное соответствие*;обратимое соответствие*;неоднозначное соответствие*;инъективное соответствие*;взаимно однозначное соответствие*;множество сочетаний*;множество размещений*;множество перестановок*}

\scnheader{связь}
\scnidtf{связка sc-элементов}
\scnidtf{sc-связка}
\scnexplanation{\textbf{\textit{связь}} – множество, являющееся абстрактной моделью связи между описываемыми сущностями, которые или знаки которых являются элементами этого множества.}
\scntext{примечание}{Напомним, что все элементы множества, представленного в SC-коде, являются знаками, но описываемыми сущностями могут быть не только сущности, обозначаемые sc-элементами, но и сами эти sc-элементы.}
\scnsubdividing{бинарная связь;небинарная связь}
\scnsubdividing{неориентированная связь;ориентированная связь}

\scnheader{бинарная связь}
\scnsubdividing{sc-коннектор;неатомарная бинарная связь}
\scntext{примечание}{Данное разбиение осуществляется на основе синтаксического признака, а не семантического, поскольку каждый \textit{sc-коннектор} может быть записан в памяти при помощи семантически эквивалентной конструкции, содержащей знак самой связи и пары принадлежности, ведущие к ее элементам, уточненные, при необходимости ролевыми отношениями.}

\scnheader{sc-коннектор}
\scnidtf{атомарная бинарная связь}
\scnexplanation{Каждый \textbf{\textit{sc-коннектор}} представлен в \textit{sc-памяти} одним \textit{sc-элементом} и семантически эквивалентен конструкции, содержащей знак некоторой \textit{бинарной связи} и пары принадлежности, ведущие к элементам этой связи, уточненные, при необходимости ролевыми отношениями.

Такая конструкция может быть обозначена \textbf{\textit{sc-коннектором}} только в случае, когда роли компонентов соответствующей бинарной связи указываются только при помощи \textit{числовых атрибутов 1’} и \textit{2’} или не уточняются вообще.}

\scnheader{неатомарная бинарная связь}
\scnexplanation{\textbf{\textit{неатомарная бинарная связь}} – \textit{бинарная связь}, роли компонентов которой не могут быть заданы только при помощи \textit{ролевых отношений 1'} и \textit{2'}, или не заданы совсем, а требуют дополнительного уточнения при помощи более частных ролевых отношений.}

\scnheader{небинарная связь}
\scnexplanation{\textbf{\textit{небинарная связь}} – связь, имеющая больше двух элементов.}

\scnheader{неориентированная связь}
\scnsuperset{неориентированное множество}
\scnexplanation{\textbf{\textit{неориентированная связь}} – связь, все элементы которой имеют одинаковые роли (при этом соответствующее ролевое отношение, как правило, явно не указывается).}

\scnheader{ориентированная связь}
\scnsuperset{ориентированное множество}
\scnexplanation{\textbf{\textit{ориентированная связь}} – связь, в которой с помощью ролевых отношений, указываются роли компонентов этой связи.}

\scnheader{отношение}
\scnidtf{класс связей}
\scnidtf{класс sc-связок}
\scnidtf{множество отношений}
\scnidtf{Множество всевозможных отношений}
\scntext{определение}{\textbf{\textit{отношение}}, \textit{заданное на множестве M} – это подмножество \textit{декартового произведения} этого множества самого на себя некоторое количество раз.

В более широком смысле \textbf{\textit{отношение}} – это математическая структура, которая формально определяет свойства различных объектов и их взаимосвязи.}
\scnsubdividing{класс равномощных связок;класс связок разной мощности}
\scnsubdividing{бинарное отношение;небинарное отношение}
\scnsubdividing{ориентированное отношение;неориентированное отношение}
\scnsubdividing{ролевое отношение;неролевое отношение}

\scnheader{класс равномощных связок}
\scnidtf{класс связок фиксированной арности}
\scnidtf{отношение, обладающее свойством арности}
\scnsuperset{унарное отношение}
\scnsuperset{бинарное отношение}
\scnsuperset{тернарное отношение}
\scntext{определение}{\textbf{\textit{класс равномощных связок}} – класс связок, имеющих одинаковую мощность.}

\scnheader{класс связок разной мощности}
\scnidtf{отношение нефиксированной арности}
\scnsubset{небинарное отношение}
\scntext{определение}{\textbf{\textit{класс связок разной мощности}} – класс связок, имеющих разную мощность.}

\scnheader{унарное отношение}
\scnidtf{отношение арности один}
\scnidtf{одноместное отношение}
\scnidtf{множество синглетонов}
\scntext{определение}{\textbf{\textit{унарное отношение}} – это множество таких отношений на множестве M, являющихся любым подмножеством множества M.}

\scnheader{бинарное отношение}
\scnidtf{отношение арности два}
\scnidtf{двухместное отношение}
\scnsuperset{квазибинарное отношение}
\scnsuperset{отношение порядка}
\scnsuperset{отношение толерантности}
\scnsubdividing{рефлексивное отношение;антирефлексивное отношение;частично рефлексивное отношение}
\scnsubdividing{симметричное отношение;антисимметричное отношение;частично симметричное отношение}
\scnsubdividing{транзитивное отношение;антитранзитивное отношение;частично транзитивное отношение}
\scnsubdividing{ролевое отношение;неролевое бинарное отношение}
\scntext{определение}{\textbf{\textit{бинарное отношение}} – это множество таких отношений на множестве \textbf{\textit{M}}, являющихся подмножеством \textit{декартова произведения} множества \textbf{\textit{M}}.\\
Если \textbf{\textit{бинарное отношение R}} задано на \textit{множестве} \textbf{\textit{М}} и два элемента этого множества \textbf{\textit{a}} и \textbf{\textit{b}} связаны данным отношением, то будем обозначать такую связь как \textbf{\textit{aRb}}.}

\scnheader{квазибинарное отношение}
\scnexplanation{\textbf{\textit{квазибинарное отношение}} – множество ориентированных пар, первые компоненты которых являются связками.\\
Таким образом, \textit{sc-дуги}, принадлежащие \textbf{\textit{квазибинарным отношениям}}, всегда выходят из связок.}
\scntext{sc-утверждение}{В область определения квазибинарного отношения будем включать:
\begin{scnitemize}
    \item вторые компоненты ориентированных пар, принадлежащих этому отношению;
    \item элементы первых компонентов ориентированных пар, принадлежащих этому отношению;
    \item других элементов область определения квазибинарного отношения не содержит.
\end{scnitemize}
}

\scnheader{тернарное отношение}
\scnidtf{отношение арности три}
\scnidtf{трехместное отношение}
\scnexplanation{\textbf{\textit{тернарное отношение}} – это множество отношений, связывающих между собой три элемента.}

\scnheader{небинарное отношение}
\scnexplanation{\textbf{\textit{небинарное отношение}} – это множество отношений, хотя бы одна из связок каждого из которых имеет значение мощности больше двух.}

\scnheader{ориентированное отношение}
\scntext{определение}{\textbf{\textit{ориентированное отношение}} – это множество таких отношений, каждая связка которых является ориентированным множеством.}

\scnheader{неориентированное отношение}
\scntext{определение}{\textbf{\textit{неориентированное отношение}} – это множество таких отношений, каждая связка которых является неориентированным множеством.}

\scnheader{рефлексивное отношение}
\scntext{определение}{\textbf{\textit{рефлексивное отношение}} – это \textit{бинарное отношение}, любая пара которого есть канторовское множество.}

\scnheader{антирефлексивное отношение}
\scntext{определение}{\textbf{\textit{антирефлексивное отношение R}} на \textit{множестве} \textbf{\textit{A}} – это \textit{бинарное отношение}, в котором все элементы множества \textbf{\textit{A}} не находятся в отношении \textbf{\textit{R}} к самому себе.}

\scnheader{частично рефлексивное отношение}
\scntext{определение}{\textbf{\textit{частично рефлексивное отношение R}} на \textit{множестве} \textbf{\textit{A}} – это \textit{бинарное отношение},  в котором хотя бы один (но не все) элемент множества \textbf{\textit{A}} находится в отношении \textbf{\textit{R}} к самому себе.}

\scnheader{симметричное отношение}
\scntext{определение}{\textbf{\textit{симметричное отношение R}} на \textit{множестве} \textbf{\textit{A}} – это \textit{бинарное отношение}, в котором для каждой пары элементов \textbf{\textit{а}} и \textbf{\textit{b}} этого множества выполнение отношения \textbf{\textit{aRb}} влечёт выполнение \textbf{\textit{bRa}}.}

\scnheader{антисимметричное отношение}
\scntext{определение}{\textbf{\textit{антисимметричное отношение R}} на \textit{множестве} \textbf{\textit{A}} – это \textit{бинарное отношение}, в котором для каждой пары элементов \textbf{\textit{а}} и \textbf{\textit{b}} этого множества выполнение отношений \textbf{\textit{aRb}} и \textbf{\textit{bRa}} влечёт равенство \textbf{\textit{a}} и \textbf{\textit{b}}.}

\scnheader{частично симметричное отношение}
\scntext{определение}{\textbf{\textit{частично симметричное отношение R}} на \textit{множестве} \textbf{\textit{A}} – это \textit{бинарное отношение}, в котором для каждой пары элементов \textbf{\textit{а}} и \textbf{\textit{b}} (но не для всех таких пар) этого множества выполнение отношения \textbf{\textit{aRb}} влечёт выполнение \textbf{\textit{bRa}}.}

\scnheader{транзитивное отношение}
\scntext{определение}{\textbf{\textit{транзитивное отношение R}} на \textit{множестве} \textbf{\textit{A}} – это \textit{бинарное отношение}, в котором для любых трёх элементов этого множества \textbf{\textit{a, b, c}} выполнение отношений \textbf{\textit{aRb}} и \textbf{\textit{bRc}} влечёт выполнение отношения \textbf{\textit{aRc}}.}

\scnheader{антитранзитивное отношение}
\scntext{определение}{\textbf{\textit{антитранзитивное отношение R}} на \textit{множестве} \textbf{\textit{A}} – это \textit{бинарное отношение}, в котором для любых трёх элементов этого множества \textbf{\textit{a, b, c}} выполнение отношений \textbf{\textit{aRb}} и \textbf{\textit{bRc}} не влечёт выполнение отношения \textbf{\textit{aRc}}.}

\scnheader{частично транзитивное отношение}
\scntext{определение}{\textbf{\textit{частично транзитивное отношение R}} на \textit{множестве} \textbf{\textit{A}} – это \textit{бинарное отношение}, в котором для каждых трёх элементов этого множества \textbf{\textit{a, b, c}} (но не для всех таких троек) выполнение отношений \textbf{\textit{aRb}} и \textbf{\textit{bRc}} влечёт выполнение отношения \textbf{\textit{aRc}}.}

\scnheader{связанное отношение}
\scnsubset{бинарное отношение}
\scntext{определение}{\textbf{\textit{связанное отношение R}} на \textit{множестве} \textbf{\textit{A}} – это \textit{бинарное отношение}, в котором для каждой пары элементов \textbf{\textit{а}} и \textbf{\textit{b}} этого множества выполняется одно из двух отношений: \textbf{\textit{aRb}} или \textbf{\textit{bRa}}.}

\scnheader{отношение порядка}
\scnsubdividing{отношение строгого порядка;отношение нестрогого порядка}
\scntext{определение}{\textbf{\textit{отношение порядка}} – это \textit{бинарное отношение}, обладающее свойством транзитивности и антисимметричности.}

\scnheader{отношение строгого порядка}
\scntext{определение}{\textbf{\textit{отношение строгого порядка}} – это \textit{отношение порядка}, обладающее свойством антирефлексивности.}

\scnheader{отношение нестрогого порядка}
\scntext{определение}{\textbf{\textit{отношение нестрогого порядка}} – это \textit{отношение порядка}, обладающее свойством рефлексивности.}

\scnheader{отношение толерантности}
\scntext{определение}{\textbf{\textit{отношение толерантности}} – это \textit{бинарное отношение}, принадлежащее классам \textit{симметричное отношение} и \textit{рефлексивное отношение}.}

\scnheader{отношение эквивалентности}
\scnidtf{максимальное семейство отношений эквивалентности}
\scnsubset{отношение толерантности}
\scntext{определение}{\textbf{\textit{отношение эквивалентности}} – это \textit{отношение толерантности}, принадлежащее классу \textit{транзитивных отношений}}
\scntext{примечание}{Каждое отношение эквивалентности уточняет то, что мы считаем эквивалентными сущностями, т.е. то, на какие сходства этих сущностей мы обращаем внимание и какие их отличия мы игнорируем (не учитываем).}

\scnheader{ролевое отношение}
\scnidtf{атрибут}
\scnidtf{атрибутивное отношение}
\scnidtf{отношение, которое задает роль элементов в рамках некоторого множества}
\scnidtf{отношение, являющееся подмножеством отношения принадлежности}
\scnrelto{семейство подмножеств}{принадлежность*}
\scnsubset{бинарное отношение}
\scnsuperset{числовой атрибут}
\scnexplanation{\textbf{\textit{ролевое отношение}} – это отношение, являющееся подмножеством отношения принадлежности.}
\scntext{правило идентификации экземпляров}{В конце каждого \textit{идентификатора}, соответствующего экземплярам класса \textbf{\textit{ролевое отношение}}, не являющегося системным, ставится знак «'».

Например:\\
\textit{ключевой экземпляр’}

Из-за ограничений в разрешенном алфавите символов, в системном идентификаторе не может быть использовать знак «'», поэтому в начале каждого \textit{системного идентификатора}, соответствующего экземплярам класса \textbf{\textit{ролевое отношение}} ставится префикс «rrel\_».

Например:\\
\textit{rrel\_key\_sc\_element}}

\scnheader{числовой атрибут}
\scnidtf{порядковый номер}
\scnidtf{номер компонента ориентированной связки}
\scnhaselement{1’; 2’; 3’; 4’; 5’; 6’; 7’; 8’; 9’; 10’}
\scnexplanation{\textbf{\textit{числовой атрибут}} – \textit{ролевое отношение}, задающее порядковый номер элемента некоторой ориентированной связки, не уточняя при этом семантику такой принадлежности. Во многих случаях бывает достаточно использовать числовые атрибуты, чтобы различать компоненты связки, семантика каждого из которых дополнительно оговаривается, например, при определении отношения, которому данная связка принадлежит.}

\scnheader{неролевое отношение}
\scnsubdividing{небинарное отношение;неролевое бинарное отношение}
\scnexplanation{\textbf{\textit{неролевое отношение}} – отношение, не являющееся подмножеством отношения принадлежности.}
\scntext{правило идентификации экземпляров}{В конце каждого \textit{идентификатора}, соответствующего экземплярам класса \textbf{\textit{неролевое отношение}}, не являющегося системным, ставится знак «*».

Например:\\
\textit{включение*}

Из-за ограничений в разрешенном алфавите символов, в системном идентификаторе не может быть использовать знак «*», поэтому в начале каждого \textit{системного идентификатора}, соответствующего экземплярам класса \textbf{\textit{неролевое отношение}} ставится префикс «nrel\_».

Например:\\
\textit{nrel\_inclusion}}

\scnheader{неролевое бинарное отношение}
\scnexplanation{\textbf{\textit{неролевое бинарное отношение}} – \textit{бинарное отношение}, не являющееся \textit{ролевым отношением}.}

\scnheader{арность}
\scnidtf{арность отношения}
\scniselement{параметр}
\scnexplanation{\textbf{\textit{арность}} – это параметр, каждый элемент которого представляет собой класс \textit{отношений}, каждая связка которых имеет одинаковую \textit{мощность}. Значение данного \textit{параметра} совпадает со значением \textit{мощности} каждой из таких связок.}
\scnrelfrom{описание примера}{
\scnfilescg{figures/sd_relations/arity.png}}


\scnheader{область определения*}
\scnidtf{область определения отношения*}
\scniselement{бинарное отношение}
\scnexplanation{\textbf{\textit{область определения*}} – это \textit{бинарное отношение}, связывающее отношение со множеством, являющимся его областью определения.

Областью определения отношения будем называть результат теоретико-множественного объединения всех связок этого отношения, или, другими словами, результат теоретико-множественного объединения всех множеств, являющихся доменами данного отношения.}
\scnrelfrom{описание примера}{
\scnfilescg{figures/sd_relations/domain.png}}

\scnheader{атрибут отношения*}
\scnidtf{ролевой атрибут, используемый в связках заданного отношения*}
\scniselement{бинарное отношение}
\scnexplanation{\textbf{\textit{атрибут отношения*}} – это \textit{бинарное отношение}, связывающее заданное отношение с \textit{ролевым отношением}, используемым в данном отношении для уточнения роли того или иного элемента связок данного отношения.}
\scnrelfrom{описание примера}{
\scnfilescg{figures/sd_relations/relationshipAttribute.png}}


\scnheader{домен*}
\scnidtf{домен отношения по заданному атрибуту*}
\scniselement{бинарное отношение}
\scnexplanation{\textbf{\textit{домен*}} – это \textit{бинарное отношение}, связывающее связку отношения \textit{атрибут отношения*} со множеством, являющимся доменом заданного отношения по заданному атрибуту. Множество \textbf{\textit{di}} является доменом отношения \textbf{\textit{ri}} по атрибуту \textbf{\textit{ai}} в том и только том случае, если элементами этого множества являются все те и только те элементы связок отношения \textbf{\textit{ri}}, которые имеют в рамках этих связок атрибут \textbf{\textit{ai}}.}
\scnrelfrom{описание примера}{
\scnfilescg{figures/sd_relations/domen.png}}


\scnheader{первый домен*}
\scniselement{бинарное отношение}
\scntext{определение}{\textbf{\textit{первый домен*}} – это \textit{бинарное отношение}, связывающее отношение с множеством, являющимся доменом по атрибуту \textbf{\textit {1'}} данного отношения.}
\scnrelfrom{описание примера}{
\scnfilescg{figures/sd_relations/firstDomen.png}}

\scnheader{второй домен*}
\scniselement{бинарное отношение}
\scntext{определение}{\textbf{\textit{второй домен*}} – это \textit{бинарное отношение}, связывающее отношение с множеством, являющимся доменом по атрибуту \textbf{\textit{2'}} данного отношения.}
\scnrelfrom{описание примера}{
\scnfilescg{figures/sd_relations/secondDomen.png}}

\scnheader{композиция отношений*}
\scniselement{квазибинарное отношение}
\scntext{определение}{\textbf{\textit{композиция отношений*}} – это \textit{квазибинарное отношение}, связывающее два бинарных отношения с отношением, являющимся их композицией. Под композицией бинарных отношений \textbf{\textit{R}} и \textbf{\textit{S}} будем понимать множество $\{(x, y) | \exists z(xSz \wedge zRy)\}$}
\scnrelfrom{описание примера}{
\scnfilescg{figures/sd_relations/relationshipComposition.png}}

\scnheader{фактор-множество*}
\scnidtf{быть фактор-множеством*}
\scnidtf{множество всевозможных максимальных множеств из попарно эквивалентных элементов*}
\scnidtf{множество всевозможных классов эквивалентности для заданного отношения эквивалентности*}
\scniselement{бинарное отношение}
\scntext{определение}{\textbf{\textit{фактор множество*}} - это бинарное ориентированное отношение, каждая связка которого связывает некоторое отношение эквивалентности со множеством всех соответствующих этому отношению классов эквивалентности. Каждый такой класс представляет собой максимальное множество сущностей, каждая пара которых принадлежит указанному выше отношению эквивалентности.}
\scnrelfrom{описание примера}{
\scnfilescg{figures/sd_relations/factor_set.png}}

\scnheader{метаотношение}
\scntext{определение}{метаотношение - это \textit{отношение}, в каждой связке которого есть по крайней мере один компонент, являющийся знаком некоторого \textit{отношения}.}

\scnheader{отношение декомпозиции}
\scnhaselement{разбиение*}
\scnhaselement{декомпозиция раздела*}
\scnhaselement{декомпозиция абстрактного объекта*}

\scnheader{отношение интеграции}
\scnhaselement{объединение*}

\scnheader{соответствие*}
\scnidtf{наличие соответствия*}
\scniselement{бинарное отношение}
\scnsubdividing{соответствие между непересекающимися множествами*;соответствие между строго пересекающимися множествами*;соответствие, область отправления и область прибытия которого совпадают*}
\scnsubdividing{всюду определенное соответствие*;частично определенное соответствие*}
\scnsubdividing{сюръекция*;несюръективное соответствие*}
\scnsubdividing{однозначное соответствие*;неоднозначное соответствие*}
\scntext{определение}{\textbf{\textit{соответствие*}} – \textit{бинарное отношение}, заданное на множествах и задающее наличие отношения, в котором участвуют только элементы этих множеств.}
\scnrelfrom{описание примера}{
\scnfilescg{figures/sd_relations/conformity.png}}

\scnheader{отношение соответствия*}
\scniselement{бинарное отношение}
\scntext{определение}{\textbf{\textit{отношение соответствия*}} – \textit{бинарное отношение}, связывающее ориентированную пару множеств, на которых задано \textit{соответствие*} и некоторое подмножество \textit{декартова произведения*} этих \textit{множеств}.}
\scnrelfrom{описание примера}{
\scnfilescg{figures/sd_relations/relationshipConformity.png}}

\scnheader{область отправления'}
\scnidtf{область отправления соответствия’}
\scnidtf{область определения соответствия’}
\scnidtf{первый компонент пары в отношении соответствия’}
\scniselement{ролевое отношение}
\scntext{определение}{\textbf{\textit{область отправления'}} – \textit{ролевое отношение}, указывающее на первый компонент пары в рамках отношения \textit{соответствие*}.}
\scnrelfrom{описание примера}{
\scnfilescg{figures/sd_relations/departureArea.png}}

\scnheader{область прибытия’}
\scnidtf{область прибытия соответствия'}
\scnidtf{область значений соответствия'}
\scniselement{ролевое отношение}
\scntext{определение}{\textbf{\textit{область прибытия’}} – \textit{ролевое отношение}, указывающее на второй компонент пары в рамках отношения \textit{соответствие*}.}
\scnrelfrom{описание примера}{
\scnfilescg{figures/sd_relations/arrivalArea.png}}

\scnheader{образ'}
\scnidtf{образ соответствия’}
\scniselement{ролевое отношение}
\scntext{определение}{\textbf{\textit{образ'}} – \textit{ролевое отношение}, указывающее на второй компонент каждой пары в рамках множества пар, которое является вторым компонентом \textit{отношения соответствия*}.}
\scnrelfrom{описание примера}{
\scnfilescg{figures/sd_relations/form.png}}

\scnheader{прообраз'}
\scnidtf{прообраз соответствия’}
\scniselement{ролевое отношение}
\scntext{определение}{\textbf{\textit{прообраз'}} – \textit{ролевое отношение}, указывающее на первый компонент каждой пары в рамках множества пар, которое является первым компонентом \textit{отношения соответствия*}.}
\scnrelfrom{описание примера}{
\scnfilescg{figures/sd_relations/prototype.png}}

\scnheader{всюду определенное соответствие*}
\scnidtf{полное соответствие*}
\scnidtf{наличие всюду определенного соответствия*}
\scntext{определение}{\textbf{\textit{всюду определенное соответствие*}} – это \textit{соответствие*}, при котором существует \textit{образ’} для каждого элемента \textit{области отправления'} данного \textit{соответствия*}.}
\scnrelfrom{описание примера}{
\scnfilescg{figures/sd_relations/surjection.png}}
\scnrelfrom{изображение}{
\scnfileimage{\includegraphics[width=0.5\linewidth]{figures/sd_relations/surjection2.png}}}


\scnheader{частично определенное соответствие*}
\scnidtf{наличие частично определенного соответствия*}
\scntext{определение}{\textbf{\textit{частично определенное соответствие*}} – это \textit{соответствие*}, при котором существует \textit{образ’} для некоторых, но не всех элементов \textit{области отправления'} данного \textit{соответствия*}.}
\scnrelfrom{описание примера}{
\scnfilescg{figures/sd_relations/partiallyDefinedConformity.png}}
\scnrelfrom{изображение}{
\scnfileimage{\includegraphics[width=0.5\linewidth]{figures/sd_relations/partiallySurjection.png}}}


\scnheader{сюръективное соответствие*}
\scnidtf{наличие сюръективного соответствия*}
\scnidtf{сюръекция*}
\scntext{определение}{\textbf{\textit{сюръективное соответствие*}} – это \textit{соответствие*}, при котором существует \textit{прообраз’} для каждого элемента \textit{области прибытия'} данного \textit{соответствия*}.}
\scnrelfrom{описание примера}{
\scnfilescg{figures/sd_relations/surjectiveConformity.png}}
\scnrelfrom{изображение}{
\scnfileimage{\includegraphics[width=0.5\linewidth]{figures/sd_relations/surjectiveConformity2.png}}}

\scnheader{несюръективное соответствие*}
\scnidtf{наличие несюръективного соответствия*}
\scntext{определение}{\textbf{\textit{несюръективное соответствие*}} – это \textit{соответствие*}, при котором не для каждого элемента \textit{области прибытия'} данного \textit{соответствия*} существует \textit{прообраз’}.}
\scnrelfrom{описание примера}{
\scnfilescg{figures/sd_relations/nonSurjectiveConformity.png}}
\scnrelfrom{изображение}{
\scnfileimage{\includegraphics[width=0.5\linewidth]{figures/sd_relations/nonSurjectiveConformity2.png}}}

\scnheader{однозначное соответствие*}
\scnidtf{наличие однозначного соответствия*}
\scnidtf{функциональное соответветствие*}
\scnidtf{функция*}
\scntext{определение}{\textbf{\textit{однозначное соответствие*}} – это \textit{соответствие*}, при котором каждому элементу из \textit{области отправления'} соответствия ставится не более, чем один элемент из \textit{области прибытия’} соответствия.}
\scnrelfrom{описание примера}{
\scnfilescg{figures/sd_relations/singleConformity.png}}
\scnrelfrom{изображение}{
\scnfileimage{\includegraphics[width=0.5\linewidth]{figures/sd_relations/singleConformity2.png}}}

\scnheader{обратное соответствие*}
\scniselement{бинарное отношение}
\scnrelfrom{область определения}{соответствие*}
\scntext{определение}{\textbf{\textit{обратное соответствие*}} – \textit{бинарное отношение}, связывающее два \textit{соответствия*}, при этом выполняются следующие условия:
\begin{scnitemize}
    \item \textit{область отправления’} первого соответствия является \textit{областью прибытия'} второго;
    \item \textit{область прибытия’} первого соответствия является \textit{областью отправления'} второго;
    \item для каждой пары, входящей в состав отношения первого соответствия, существует пара, входящая в состав отношения второго соответствия, при этом \textit{образ’} и \textit{прообраз'} в рамках первой указанной пары являются соответственно \textit{прообразом'} и \textit{образом’} в рамках второй.
\end{scnitemize}
}

\scnheader{обратимое соответствие*}
\scnsubset{однозначное соответствие*}
\scntext{определение}{\textbf{\textit{обратимое соответствие*}} – такое \textit{однозначное соответствие*}, для которого \textit{обратное соответствие*} также является \textit{однозначным соответствием*}.}

\scnheader{неоднозначное соответствие*}
\scntext{определение}{\textbf{\textit{неоднозначное соответствие*}} – это \textit{соответствие*}, при котором хотя бы одному элементу из \textit{области отправления’} соответствия ставится более, чем один элемент из \textit{области прибытия'} соответствия.}
\scnrelfrom{описание примера}{
\scnfilescg{figures/sd_relations/nonSingleConformity.png}}
\scnrelfrom{изображение}{
\scnfileimage{\includegraphics[width=0.5\linewidth]{figures/sd_relations/nonSingleConformity2.png}}}

\scnheader{инъективное соответствие*}
\scnidtf{инъекция*}
\scnsubset{однозначное соответствие*}
\scntext{определение}{\textbf{\textit{инъективное соответствие*}} – это \textit{соответствие*}, при котором разным элементам из \textit{области отправления’} соответствия всегда соответствуют разные элементы из \textit{области прибытия'} соответствия и наоборот.}
\scnrelfrom{описание примера}{
\scnfilescg{figures/sd_relations/injectiveConformity.png}}
\scnrelfrom{изображение}{
\scnfileimage{\includegraphics[width=0.5\linewidth]{figures/sd_relations/injectiveConformity2.png}}}

\scnheader{взаимно однозначное соответствие*}
\scnidtf{биекция*}
\scnsubset{всюду определенное соответствие*}
\scnsubset{сюръективное соответствие*}
\scnsubset{инъективное соответствие*}
\scntext{определение}{\textbf{\textit{взаимно однозначное соответствие*}} – это \textit{инъективное соответствие*}, являющееся всюду определенным и сюръективным.}
\scnrelfrom{описание примера}{
\scnfilescg{figures/sd_relations/bijectiveConformity.png}}
\scnrelfrom{изображение}{
\scnfileimage{\includegraphics[width=0.5\linewidth]{figures/sd_relations/bijectiveConformity2.png}}}


\scnheader{множество сочетаний*}
\scnidtf{множество всевозможных сочетаний*}
\scnidtf{множество всевозможных сочетаний заданной арности из элементов заданного множества*}
\scnidtf{множество всех неориентированных связок заданной арности*}
\scnidtf{множество всех подмножеств заданной мощности*}
\scnidtf{семейство всевозможных сочетаний*}
\scntext{определение}{\textbf{\textit{множество сочетаний*}} - \textit{отношение}, связывающее некоторое множество и семейство всевозможных множеств, имеющих значение мощности, меньше либо равное мощности исходного множества и состоящих из тех же элементов, что и это множество.}
\scntext{утверждение}{Мощность \textbf{\textit{множества сочетаний*}} может быть вычислена как n!/(k!(n-k)!), где \textbf{\textit{n}} – мощность исходного множества, \textbf{\textit{k}} – мощность элементов множества сочетаний.}
\scnrelfrom{описание примера}{
\scnfilescg{figures/sd_relations/setsOfCombinations.png}
\scnexplanation{Для Множества \textbf{\textit{Si}} представлено множество сочетаний по 2 элемента.}}

\scnheader{множество размещений*}
\scntext{определение}{\textbf{\textit{множество размещений*}} - \textit{отношение}, связывающее некоторое множество и семейство всевозможных кортежей, имеющих значение мощности, меньше либо равное мощности исходного множества и состоящих из тех же элементов, что и это множество.}
\scntext{утверждение}{Мощность \textbf{\textit{множества размещений*}} может быть вычислена как n!/(n-k)!, где \textbf{\textit{n}} – мощность исходного множества, \textbf{\textit{k}} – мощность элементов множества сочетаний.}
\scnrelfrom{описание примера}{
\scnfilescg{figures/sd_relations/setsOfPlacements.png}
\scnexplanation{Для Множества \textbf{\textit{Si}} представлено множество размещений по 2 элемента.}}

\scnheader{множество перестановок*}
\scnsubset{множество размещений*}
\scntext{определение}{\textbf{\textit{множество перестановок*}} - \textit{отношение}, связывающее некоторое множество и семейство всевозможных кортежей, равномощных исходному множеству и состоящих из тех же элементов, что и это множество.}
\scntext{утверждение}{Мощность \textbf{\textit{множества перестановок*}} может быть вычислена как n!, где \textbf{\textit{n}} – мощность исходного множества.}
\scnrelfrom{описание примера}{
\scnfilescg{figures/sd_relations/setsOfPermutations.png}
\scnexplanation{Для Множества \textbf{\textit{Si}} представлено его множество перестановок.}}

\bigskip
\scnendstruct \scnendcurrentsectioncomment

\end{SCn}

\scsection{Предметная область и онтология параметров, величин и шкал}
\label{sec:sd_params}
\begin{SCn}

\scnsectionheader{Предметная область и онтология параметров, величин и измерений}

\scnstartsubstruct

\scnheader{Предметная область параметров, величин и измерений}
\scnidtf{Предметная область параметров и классов эквивалентности, являющихся их элементами (значениями, величинами)}
\scniselement{предметная область}
\scnsdmainclasssingle{параметр}
\scnsdclass{измеряемый параметр;неизмеряемый параметр;уровень класса эквивалентности;величина;точная величина;неточная величина;интервальная величина;параметрическая модель;измерение с фиксированной единицей измерения ;измерение по шкале;арифметическое выражение на величинах;арифметическая операция на величинах;действие. измерение;задача. измерение}
\scnsdrelation{область определения параметра*;эталон';измерение*;точность*;единица измерения*;нулевая отметка*;сумма величин*;произведение величин*;возведение величин в степень*;большая величина*;равенство величин*;большая или равная величина*}

\scnauthorcomment{ввести отношение, показывающее единичную отметку для измерений по шкале}

\scnheader{параметр}
\scnidtf{характеристика}
\scnidtf{свойство}
\scnidtf{признак}
\scnidtf{класс классов}
\scnidtf{измеряемое свойство}
\scnidtf{признак классификации или покрытия некоторого класса сущностей}
\scnidtf{признак разбиения или покрытия некоторого класса сущностей}
\scnidtf{семейство множеств, разбивающих или покрывающих некоторый класс сущностей}
\scnidtf{семейство классов сущностей, обладающих одинаковым соответствующим свойством}
\scnidtf{фактор-множество, соответствующее некоторому отношению эквивалентности, или аналог фактор-множества, соответствующий некоторому отношению толерантности}
\scnreltoset{разбиение}{измеряемый параметр;неизмеряемый параметр}
\scnexplanation{Каждый \textbf{\textit{параметр}} представляет собой класс, являющийся семейством всевозможных классов эквивалентности или толерантности, задаваемых либо \textit{отношением эквивалентности}, либо \textit{отношением толерантности} (симметричным, рефлексивным, но частично транзитивным). Так, например, элементами (значениями, величинами) \textbf{\textit{параметра}} \textit{длина} являются либо классы эквивалентности, задаваемые отношением эквивалентности «иметь точно одинаковую длину*», либо классы толерантности, задаваемые отношением вида «иметь приблизительно одинаковую длину с указываемой точностью*», либо интервальные классы, задаваемые бинарными отношениями вида «иметь длину, находящуюся в одном и том же указываемом интервале*» (например, от 1 метра до 2 метров).\\
Примерами параметров как отношений эквивалентности являются:
\begin{scnitemize}
    \item равновеликость геометрических фигур (по длине, площади, объему – в зависимости от размерности этих фигур);
    \item иметь одинаковый цвет (быть эквивалентными по цвету);
    \item эквивалентность, по вкусу, запаху, твердости и т.д.
\end{scnitemize}

Заметим, что среди элементов (значений, величин) параметра могут встречаться пересекающиеся множества (классы), но объединение всех элементов каждого параметра есть не что иное, как класс всевозможных сущностей, обладающих этим параметром (свойством, характеристикой). Например, класс всех сущностей, имеющих длину, класс всех сущностей, обладающих цветом.

Каждый конкретный параметр (характеристика), т.е. каждый элемент класса всевозможных параметров (характеристик) есть, по сути, признак классификации сущностей, обладающих это характеристикой, по принципу эквивалентности (одинаковости значения) этой характеристики. Например, параметр \textit{цвет} разбивает множество сущностей имеющих цвет на классы, каждый из которых включает в себя сущности, имеющие одинаковый цвет. Параметр может разбиваться на классы для уточнения некоторого свойства, например элементами параметра цвет будут классы, соответствующие конкретным цветам (синий, красный и т.д.), в свою очередь каждый конкретный цвет может включать более частные классы, уточняющие данное свойство, например, темно-синий, светло-красный и т.д.

Другими словами, каждому множеству сущностей может ставиться в соответствие набор (семейство) параметров (параметрическое пространство), которыми обладают сущности этого множества – аналог семейства отношений, определенных (заданных) на этом множестве. Часто бывает важно построить такое параметрическое пространство, «точки» которого взаимно-однозначно соответствуют параметризуемым сущностям (например, набор параметров, позволяющих однозначно идентифицировать, установить личность каждого человека). 

Таким образом, для каждого используемого элемента (значения) какого-либо параметра, необходимо явно указывать спецификацию этого значения (точное значение, неточное значение, интервальное значение, точность, интервал).}
\scnrelfrom{типичная семантическая окрестность}{
\scnfilelong{
\begin{figure}[H]
\centering
\includegraphics[width=0.8\linewidth]{figures/sd_parameters_and_quantities/parameterDescription.png}
\end{figure}
}}

\scnheader{область определения параметра*}
\scnidtf{множество всех тех и только тех сущностей, которые являются компонентами значений заданного параметра*}
\scnidtf{множество всех тех и только тех сущностей, которые обладают заданным параметром*}
\scnrelto{включение}{объединение*}

\scnheader{измеряемый параметр}
\scnidtf{количественный параметр}
\scnidtf{семейство измеряемых величин}
\scnidtf{семейство классов эквивалентности, каждому из которых может быть поставлено в соответствие числовое значение}
\scnexplanation{Каждый \textbf{\textit{измеряемый параметр}} представляет собой \textit{параметр}, значение (элемент, экземпляр) которого трактуется как \textit{величина}, которой можно поставить в соответствие ее числовое значение на основании выбранной единицы измерения и точки отсчета (нулевой отметки выбранной шкалы).}

\scnheader{неизмеряемый параметр}
\scnidtf{качественный параметр}

\scnheader{уровень класса эквивалентности}
\scnidtf{уровень параметра}
\scniselement{параметр}
\scnexplanation{Параметр \textbf{\textit{уровень класса эквивалентности}} задает уровень некоторого значения некоторого \textit{параметра} в иерархии значений этого параметра. Уровень класса эквивалентности равен 1, если значение параметра не является частным по отношению к другому значению этого параметра, равен 2, если значение параметра является частным по отношению к значению этого параметра с уровнем 1 и т.д.}
\scnrelfrom{типичная семантическая окрестность}{
\scnfilelong{
\begin{figure}[H]
\centering
\includegraphics[width=0.8\linewidth]{figures/sd_parameters_and_quantities/color.png}
\end{figure}
}}

\scnheader{величина}
\scnidtf{значение количественного параметра}
\scnidtf{значение измеряемого параметра}
\scnidtf{класс сущностей, имеющих одинаковое значение соответствующего параметра}
\scnrelfromlist{включение}{точная величина;неточная величина;интервальная величина}
\scnexplanation{Каждая \textbf{\textit{величина}} представляет собой однозначный и независящий от шкалы измерения результат измерения некоторой характеристики у некоторой сущности.

Каждой \textbf{\textit{величине}} можно поставить в соответствие ее числовое значение на основании выбранной единицы измерения и точки отсчета (нулевой отметки выбранной шкалы).

Нельзя путать значение параметра (\textbf{\textit{величину}}) и значение величины по некоторой шкале, которое может быть скалярным и векторным.}

\scnheader{точная величина}
\scnidtf{точное значение параметра}
\scnidtf{множество всех точных значений параметра}
\scnidtf{значение параметра, являющееся семейством классов эквивалентности, соответствующим некоторому отношению эквивалентности}
\scnidtf{класс эквивалентности}
\scnexplanation{Каждая \textbf{\textit{точная величина}} имеет одно фиксированное значение в некоторой единице измерения или по какой-либо шкале. При этом считается, что все элементы такого класса имеют одинаковое значение данного параметра и отклонениями можно пренебречь.

Каждой \textbf{\textit{точной величине}} можно поставить в соответствие группу \textit{неточных величин}, являющихся не разбиениями, а покрытиями того же множества, но с разной степенью точности.}
\scnrelfrom{типичная семантическая окрестность}{
\scnfilelong{
\begin{figure}[H]
\centering
\includegraphics[width=0.8\linewidth]{figures/sd_parameters_and_quantities/exactLength.png}
\end{figure}}
\scntext{комментарий}{В данном примере \textit{ki} обозначает класс сущностей, имеющих длину ровно 5 метров. Конкретный пример такой сущности - \textit{bi}.}}

\scnheader{неточная величина}
\scnidtf{множество неточных значений параметра}
\scnidtf{приблизительная величина}
\scnidtf{приблизительное значение параметра}
\scnidtf{значение параметра в интервале с нефиксированными границами}
\scnexplanation{Каждой \textbf{\textit{неточной величине}} ставится в соответствие ее значение в некоторой единице измерения или по какой-либо шкале, а также дополнительно указывается \textit{точность*}, т.е. возможное отклонение от данного значения.}
\scnrelfrom{типичная семантическая окрестность}{
\scnfilelong{
\begin{figure}[H]
\centering
\includegraphics[width=0.8\linewidth]{figures/sd_parameters_and_quantities/approximateLength.png}
\end{figure}}
\scntext{комментарий}{В данном примере \textit{ki} обозначает класс сущностей, имеющих длину примерно 25 метров, при этом максимально возможное отклонение от этого значения составляет \textit{kj}, то есть 2 метра. Конкретный пример такой сущности - \textit{bi}.}}

\scnheader{интервальная  величина}
\scnidtf{интервальное значение параметра}
\scnidtf{значение параметра в интервале с фиксированными границами}
\scnidtf{интервал значения параметра из множества пересекающихся интервалов разной длины, имеющих нефиксированные границы}
\scnexplanation{Каждая \textbf{\textit{интервальная величина}} представляет собой класс сущностей, находящихся в рамках точно заданного интервала, минимальная и максимальная точка которого являются \textit{точными величинами}. Результатом \textit{измерения*} такой величины является ориентированная пара, первым компонентом которой является левая (меньшая) граница интервала, вторым компонентом – правая (большая) граница интервала.}
\scnrelfrom{типичная семантическая окрестность}{
\scnfilelong{
\begin{figure}[H]
\centering
\includegraphics[width=0.8\linewidth]{figures/sd_parameters_and_quantities/intervalLength.png}
\end{figure}}
\scntext{комментарий}{В данном примере \textit{ki} обозначает класс сущностей, имеющих длину, которая лежит в интервале от \textit{kj} до \textit{kl}, то есть в интервале от 4 до 5 метров, а \textit{bi} – конкретный пример такой сущности.}}

\scnheader{эталон'}
\scnidtf{образец'}
\scniselement{ролевое отношение}
\scnexplanation{Ролевое отношение \textit{эталон'} указывает на тот элемент значения некоторого параметра, который в рамках данного класса эквивалентности считается эталонным, то есть он используется как образец при определении данного параметра.

\textit{эталон'} может задаваться как для измеряемых, так и для неизмеряемых параметров, например, эталон метра или эталон красоты.}

\scnheader{измерение*}
\scnidtf{значение параметра*}
\scnidtf{значение величины*}
\scnidtf{измерение как соответствие*}
\scnidtf{результат измерения заданной величины в заданной единице измерения и по заданной шкале*}
\scnidtf{бинарное ориентированное отношение, связывающее различные величины с результатами их измерения в различных единицах измерения и по различным шкалам*}
\scnexplanation{Связки отношения \textit{измерение*} связывают величину и ее значение в некоторой единице измерения (в том числе, в интервале) или по некоторой шкале. Конкретная единица измерения или шкала указывается дополнительно при помощи соответствующего отношения. Одной величине может соответствовать только одно значение в каждой возможной единице измерения или одна точка на некоторой шкале.}

\scnheader{точность*}
\scnidtf{отклонение*}
\scnidtf{степень точности неточного значения параметра*}
\scniselement{бинарное отношение}
\scnexplanation{Связки отношения \textbf{\textit{точность*}} связывают \textit{неточную величину} и \textit{точную величину} того же класса, задающую максимальное возможное отклонение указанной \textit{неточной величины} от своего значения.}

\scnheader{параметрическая модель}
\scnidtf{параметрическая спецификация}
\scnidtf{параметрическое sc-описание заданной сущности}
\scnidtf{описание сущности как точки в некотором параметрическом (признаковом) пространстве}
\scnrelto{включение}{семантическая окрестность}
\scnexplanation{Каждая \textbf{\textit{параметрическая модель}} представляет собой описание заданной сущности в некотором параметрическом пространстве количественных и качественных \textit{параметров}, т.е. указание того, какие значения заданных параметров (характеристик) соответствуют описываемой (заданной) сущности.}

\scnheader{единица измерения*}
\scniselement{бинарное отношение}
\scnexplanation{Связки отношения \textbf{\textit{единица измерения*}} связывают знак конкретного \textbf{\textit{измерения с фиксированной единицей измерения}} и некоторую \textit{точную величину}, входящую в тот же конкретный \textit{параметр}, что и первый компонент связок этого конкретного измерения, и которая используется в данном случае в качестве единицы измерения.}

\scnheader{измерение с фиксированной единицей измерения }
\scnrelto{семейство подмножеств}{измерение*}
\scnexplanation{Каждая \textbf{\textit{измерение с фиксированной единицей измерения}} представляет собой подмножество отношения \textit{измерение*} и характеризуется некоторой \textit{единицей измерения*}, которая является элементом того же параметра (семейством сущностей, имеющих значение данного параметра, совпадающее с этой единицей измерения).}

\scnheader{измерение по шкале }
\scnidtf{шкала}
\scnrelto{семейство подмножеств}{измерение*}
\scnexplanation{Каждая \textbf{\textit{измерение по шкале}} представляет собой подмножество отношения \textit{измерение*} и характеризуется не единицей измерения, а некоторой точкой отсчета для данной \textbf{\textit{шкалы}}. Результатом \textbf{\textit{измерения по шкале}} будет некоторая точка шкалы, отстоящая от точки отсчета на определенное расстояние в нужную сторону (меньшую или большую). Понятно, что это расстояние может быть измерено любыми единицами измерения, но его величина при этом останется неизменной.

Не стоит путать измерение по \textbf{\textit{измерение по шкале}}, которое зависит от \textit{нулевой отметки*}, с измерением изменения того же \textit{параметра}, которое характеризуется единицей измерения и не зависит от точки отсчета. Например, не стоит путать дату по некоторому календарю, соответствующую \textit{началу} какого-либо процесса, и \textit{длительность} этого процесса, которая не зависит от выбранного календаря.}
\scnrelfrom{типичная семантическая окрестность}{
\scnfilelong{
\begin{figure}[H]
\centering
\includegraphics[width=0.8\linewidth]{figures/sd_parameters_and_quantities/scale.png}
\end{figure}}
\scntext{комментарий}{В данном примере \textit{ki} обозначает класс сущностей, имеющих точную температуру в 330 К, а \textit{bi} – конкретный пример такой сущности.}}

\scnheader{нулевая отметка*}
\scnidtf{нуль по шкале*}
\scnidtf{начало отсчета*}
\scniselement{бинарное отношение}
\scnexplanation{Связки отношения \textbf{\textit{нулевая отметка*}} связывают знак некоторого \textit{измерения по шкале} со знаком \textit{точной величины} того же \textit{параметра}, которая в рамках данной шкалы принимается за точку отсчета.}

\scnheader{арифметическое выражение на величинах}
\scnexplanation{Каждое \textbf{\textit{арифметическое выражение на величинах}} представляет собой \textit{связку}, компонентами которой являются элементы или подмножества некоторого \textit{количественного параметра}.}

\scnheader{арифметическая операция на величинах}
\scnrelto{семейство подмножеств}{арифметическое выражение на величинах}
\scnexplanation{Каждая \textbf{\textit{арифметическая операция на величинах}} представляет собой \textit{отношение}, элементами которого являются \textit{арифметические выражения на величинах}, то есть множество \textit{арифметических выражений на величинах} какого-либо одного вида.}

\scnheader{сумма величин*}
\scnidtf{сложение величин*}
\scniselement{арифметическая операция на величинах}
\scniselement{квазибинарное отношение}
\scnexplanation{\textbf{\textit{сумма величин*}} – это \textit{арифметическая операция на величинах}, аналогичная \textit{арифметической операции сумма*} для \textit{чисел}.

Первым компонентом связки отношения \textbf{\textit{сумма величин*}} является подмножество некоторого \textit{количественного параметра} (слагаемые \textit{величины}), содержащее два или более элемента, вторым компонентом – элемент этого же \textit{количественного параметра}, значение которого в любой \textit{единице измерения*} является результатом сложения значений всех слагаемых \textit{величин} в той же \textit{единице измерения*}. При несовпадении \textit{единиц измерения} слагаемых величин необходимо воспользоваться соотношениями между \textit{единицами измерения}, которые задаются при помощи операций \textit{произведение величин*} и \textit{возведение величин в степень*}.}


\scnheader{произведение величин*}
\scnidtf{умножение величин*}
\scniselement{арифметическая операция на величинах}
\scniselement{квазибинарное отношение}
\scnexplanation{\textbf{\textit{произведение величин*}} – это \textit{арифметическая операция на величинах}, аналогичная \textit{арифметической операции произведение*} для \textit{чисел}.

Первым компонентом связки отношения \textbf{\textit{произведение величин*}} является \textit{связка}, элементами которой являются либо \textit{величины количественных параметров}, либо \textit{числа}, но при этом хотя бы один элемент должен быть \textit{величиной}. Вторым компонентов является \textit{величина количественного параметра}.

Операция \textbf{\textit{произведение величин*}} может быть использована для задания соотношения между \textit{единицами измерения*} в рамках одного \textit{количественного параметра}.}
\scnrelfrom{описание типичного экземпляра}{
\scnfilelong{
\begin{figure}[H]
\centering
\includegraphics[width=0.5\linewidth]{figures/sd_parameters_and_quantities/multiplicationOfQuantities.png}
\end{figure}}
}
\scnrelfrom{описание типичного экземпляра}{
\scnfilelong{
\begin{figure}[H]
\centering
\includegraphics[width=0.8\linewidth]{figures/sd_parameters_and_quantities/multiplicationOfQuantities2.png}
\end{figure}}
}

\scnheader{возведение величин в степень*}
\scniselement{арифметическая операция на величинах}
\scniselement{бинарное отношение}
\scnexplanation{\textbf{\textit{возведение величин в степень*}} – это \textit{арифметическая операция на величинах}, аналогичная \textit{арифметической операции возведение в степень*} для \textit{чисел}.

Первым компонентом связки отношения \textbf{\textit{возведение величин в степень*}} является ориентированная пара, первым компонентом которой является \textit{величина количественного параметра} (основание степени), вторым – \textit{число} (показатель степени); Вторым компонентом связки отношения \textbf{\textit{возведение величин в степень*}} является \textit{величина количественного параметра} (результат возведения в степень).}
\scnrelfrom{описание типичного экземпляра}{
\scnfilelong{
\begin{figure}[H]
\centering
\includegraphics[width=0.5\linewidth]{figures/sd_parameters_and_quantities/exponentiation.png}
\end{figure}}
}
\scnrelfrom{описание типичного экземпляра}{
\scnfilelong{
\begin{figure}[H]
\centering
\includegraphics[width=0.8\linewidth]{figures/sd_parameters_and_quantities/exponentiationTo2.png}
\end{figure}}
}

\scnheader{большая величина*}
\scniselement{арифметическая операция на величинах}
\scniselement{бинарное отношение}
\scniselement{отношение строгого порядка}
\scnexplanation{\textbf{\textit{большая величина*}} – это \textit{арифметическая операция на величинах}, аналогичная \textit{арифметической операции больше*} для \textit{чисел}.\\
Из двух величин большей является та, \textit{значение} которой в любой \textit{единице измерения*} \textit{больше*} значения другой \textit{величины} в той же \textit{единице измерения}.}

\scnheader{равенство величин*}
\scniselement{арифметическая операция на величинах}
\scniselement{бинарное отношение}
\scniselement{симметричное отношение}
\scniselement{рефлексивное отношение}
\scniselement{транзитивное отношение}
\scnexplanation{\textbf{\textit{равенство величин*}} – это \textit{арифметическая операция на величинах}, аналогичная \textit{арифметической операции равенство*} для \textit{чисел}.

Отношение \textbf{\textit{равенство величин*}} носит исключительно дидактический характер, и явно не указывается, поскольку связывает попарно все элементы одной и той же \textit{величины} каждого \textit{количественного параметра}.}

\scnheader{большая или равная величина*}
\scniselement{арифметическая операция на величинах}
\scniselement{бинарное отношение}
\scniselement{отношение нестрогого порядка}
\scnexplanation{\textbf{\textit{большая или равная величина*}} – это \textit{арифметическая операция на величинах}, аналогичная \textit{арифметической операции больше или равно*} для \textit{чисел}.

В рамках каждой связки данного отношения первая \textit{величина} (первый компонент связки) может быть \textit{большей величиной*} или быть для второй \textit{равной величиной*}.}

\scnheader{действие. измерение}
\scnidtf{измерение как действие}
\scnidtf{действие, направленное на установление связи, принадлежащей отношению измерение* и связывающей величину, которая принадлежит заданному параметру, и которой принадлежит заданная сущность, и соответствующее значение этой величины на некоторой шкале}
\scnidtf{действие, направленное на решение задачи измерения заданного параметра у заданной сущности}
\scnrelto{включение}{действие}

\scnheader{задача. измерение}
\scnidtf{спецификация действия измерения}
\scnidtf{спецификация действия, целью которого является измерение заданного параметра у заданной сущности}
\scnrelto{включение}{задача}

\scnendstruct

\end{SCn}

\scsection{Предметная область и онтология чисел и числовых структур}
\begin{SCn}

\scnsectionheader{\currentname}

\scnstartsubstruct

\scnheader{Предметная область чисел и числовых структур}
\scniselement{предметная область}
\scnsdmainclasssingle{число}
\scnsdclass{натуральное число;целое число;рациональное число;иррациональное число;действительное число;комплексное число;отрицательное число;положительное число;арифметическое выражение;арифметическая операция;Число Пи;Число нуль;Число один;Мнимая единица;числовая структура;система счисления;десятичная система счисления;двоичная система счисления;шестнадцатеричная система счисления; дробь; обыкновенная дробь; десятичная дробь; цифра; арабская цифра; римская цифра}
\scnsdrelation{противоположные числа*;модуль*;сумма*;произведение*;возведение в степень*;больше*;равенство*;больше или равно*}

\scnheader{число}
\scnidtf{множество чисел}
\scnsubset{абстрактная терминальная сущность}
\scnexplanation{\textbf{\textit{число}} – это основное понятие математики, используемое для количественной характеристики, сравнения, нумерации объектов и их частей. Письменными знаками для обозначения чисел служат \textit{цифры}.}

\scnheader{цифра}
\scnidtf{множество цифр}
\scnsubset{внутренний файл ostis-системы}
\scnrelfromlist{включение}{арабская цифра;римская цифра}
\scnexplanation{\textbf{\textit{цифра}} -– это множество файлов, обозначающих вхождение этой цифры во всевозможные записи чисел с помощью этой цифры.}

\scnheader{натуральное число}
\scnidtf{множество натуральных чисел}
\scnexplanation{\textbf{\textit{натуральное число}} – это подмножество множества \textit{целых чисел}, которые используются при счете предметов.}
\scnsubset{целое число}

\scnheader{целое число}
\scnidtf{множество целых чисел}
\scnexplanation{\textbf{\textit{целое число}} – это подмножество множества \textit{рациональных чисел}, получаемых объединением \textit{натуральных чисел} с множеством чисел, \textit{противоположных* натуральным} и \textit{нулём}.}
\scnsubset{рациональное число}

\scnheader{рациональное число}
\scnidtf{множество рациональных чисел}
\scnexplanation{\textbf{\textit{рациональное число}} – это число, представляемое \textit{обыкновенной дробью}, где числитель — \textit{целое число}, а знаменатель — \textit{натуральное число}.}
\scnsubset{действительное число}

\scnheader{дробь}
\scnidtf{множество дробей}
\scnrelfromlist{включение}{обыкновенная дробь; десятичная дробь}
\scnexplanation{\textbf{\textit{дробь}} — это число, состоящее из одной или нескольких равных частей (долей) единицы}

\scnheader{обыкновення дробь}
\scnidtf{множество обыкновенных дробей}
\scnidtf{множество простых дробей}
\scnexplanation{\textbf{\textit{обыкновенная дробь}} - запись \textit{рационального числа} в виде ${\displaystyle \pm {\frac {m}{n}}}$ или ${\pm m/n}$, где ${n\neq 0}$.Горизонтальная или косая черта обозначает знак деления, в результате которого получается частное. Делимое называется числителем дроби, а делитель — знаменателем.}

\scnheader{десятичная дробь}
\scnidtf{множество десятичных дробей}
\scnexplanation{\textbf{\textit{десятичная дробь}} - Десятичная дробь — разновидность дроби, которая представляет собой способ представления действительных чисел в виде ${\pm d_m \ldots d_1 d_0{,} d_{-1} d_{-2} \ldots}$, где , — десятичная запятая, служащая разделителем между целой и дробной частью числа, ${d_{k}}$m — десятичные цифры.}

\scnheader{иррациональное число}
\scnidtf{множество иррациональных чисел}
\scnexplanation{\textbf{\textit{иррациональное число}} – это \textit{вещественное число}, которое не является рациональным, то есть не может быть представлено в виде дроби, где числитель — \textit{целое число}, знаменатель — \textit{натуральное число}. Любое \textbf{\textit{иррациональное число}} может быть представлено в виде бесконечной непериодической десятичной дроби.}
\scnsubset{действительное число}

\scnheader{действительное число}
\scnidtf{вещественное число}
\scnidtf{множество вещественных чисел}
\scnreltoset{объединение}{рациональное число;иррациональное число}
\scnreltoset{разбиение}{положительное число;отрицательное число;$\{$Число нуль$\}$}
\scnexplanation{\textbf{\textit{действительное число}} – это множество чисел, получаемое в результате объединения иррациональных и \textit{рациональных чисел}.}
\scnsubset{комплексное число}

\scnheader{комплексное число}
\scnidtf{множество комплексных чисел}
\scnexplanation{\textbf{\textit{комплексное число}} – число вида \textit{z=a+b*i}, где \textit{a} и \textit{b} – \textit{вещественные числа}, \textit{i} – \textit{Мнимая единица}.}

\scnheader{отрицательное число}
\scnidtf{множество отрицательных чисел}
\scnexplanation{\textbf{\textit{отрицательное число}} – число \textit{меньше*} нуля.}

\scnheader{положительное число}
\scnidtf{множество положительных чисел}
\scnexplanation{\textbf{\textit{положительное число}} – число \textit{больше*} нуля.}

\scnheader{противоположные числа*}
\scniselement{бинарное неориентированное отношение}
\scnexplanation{\textbf{\textit{противоположные числа*}} – \textit{отношение}, связывающее два числа, одно из которых является \textit{отрицательным числом}, второе – \textit{положительным}, при этом \textit{модули*} этих чисел \textit{равны*}.}

\scnheader{модуль*}
\scnidtf{модуль числа*}
\scniselement{бинарное отношение}
\scnexplanation{Связки отношения \textbf{\textit{модуль*}} связывают некоторое \textit{число} (которое может быть как \textit{отрицательным}, так и \textit{положительным}) и другое \textit{число} (всегда \textit{положительное}), которое выражает расстояние от указанного числа до \textit{Числа нуль} в единицах.}

\scnheader{арифметическое выражение}
\scnidtf{множество арифметических выражений}
\scnexplanation{Каждое \textbf{\textit{арифметическое выражение}} представляет собой \textit{связку}, компонентами которой являются \textit{числа} или множества \textit{чисел}.}

\scnheader{арифметическая операция}
\scnidtf{множество арифметических операций}
\scnrelto{семейство подмножеств}{арифметическое выражение}
\scnexplanation{Каждая \textbf{\textit{арифметическая операция}} представляет собой \textit{отношение}, элементами которого являются \textit{арифметические выражения}, то есть множество \textit{арифметических выражений} какого-либо одного вида.}

\scnheader{сумма*}
\scnidtf{сложение*}
\scniselement{арифметическая операция}
\scniselement{квазибинарное отношение}
\scnexplanation{\textbf{\textit{сумма*}} – это арифметическая операция, в результате которой по данным числам (слагаемым) находится новое число (сумма), обозначающее столько единиц, сколько их содержится во всех слагаемых.

Первым компонентом связки отношения \textbf{\textit{сумма*}} является \textit{множество чисел} (слагаемых), содержащее два или более элемента, вторым компонентом – \textit{число}, являющееся результатом сложения.

Отдельно отметим, что каждая связка отношения \textbf{\textit{сумма*}} вида a = b+c может также трактоваться и как запись о вычитании чисел, например b = a-c, в связи с чем \textit{арифметическая операция} разности чисел отдельно не вводится.}
\scnrelfrom{описание типичного экземпляра}{
\scnfilescg{figures/sd_numbers/sum.png}}

\scnheader{произведение*}
\scnidtf{умножение*}
\scniselement{арифметическая операция}
\scniselement{квазибинарное отношение}
\scnexplanation{\textbf{\textit{произведение*}} – это \textit{арифметическая операция}, в результате которой один аргумент складывается столько раз, сколько показывает другой, затем результат складывается столько раз, сколько показывает третий и т.д.

Первым компонентом связки отношения \textbf{\textit{произведение*}} является \textit{множество чисел} (множителей), содержащее два или более элемента, вторым компонентом – \textit{число}, являющееся результатом произведения.

Отдельно отметим, что каждая связка отношения \textbf{\textit{произведение*}} вида a = b*c может также трактоваться и как запись о делении чисел, например b = a/c, в связи с чем \textit{арифметическая операция} деления чисел отдельно не вводится.}
\scnrelfrom{описание типичного экземпляра}{
\scnfilescg{figures/sd_numbers/multiplication.png}}

\scnheader{возведение в степень*}
\scniselement{арифметическая операция}
\scniselement{бинарное отношение}
\scnexplanation{\textbf{\textit{возведение в степень*}} – это \textit{арифметическая операция}, в результате которой число, называемое основанием степени, умножается само на себя столько раз, каков показатель степени.

Первым компонентом связки отношения \textbf{\textit{возведение в степень*}} является ориентированная пара, первым компонентом которой является \textit{число}, которое является основанием степени, вторым – \textit{число}, которое является показателем степени; Вторым компонентом связки отношения \textbf{\textit{возведение в степень*}} является \textit{число}, которое является результатом возведения в степень.

Отдельно отметим, что каждая связка отношения \textbf{\textit{возведение в степень*}} вида a = $b^c$ может также трактоваться и как запись об извлечении корня или взятии логарифма, в связи с чем \textit{арифметические операции} извлечения корня и взятия логарифма отдельно не вводится.}
\scnrelfrom{описание типичного экземпляра}{
\scnfilescg{figures/sd_numbers/pow.png}}

\scnheader{больше*}
\scniselement{арифметическая операция}
\scniselement{бинарное отношение}
\scniselement{отношение строгого порядка}
\scnexplanation{\textbf{\textit{больше*}} – это \textit{арифметическая операция} сравнения чисел. Из двух чисел на координатной прямой больше то, которое расположено правее. Соответственно, первым компонентом связки \textit{отношения} \textbf{\textit{больше*}} является большее из двух \textit{чисел}.}
\scnrelfrom{описание типичного экземпляра}{
\scnfilescg{figures/sd_numbers/more.png}}

\scnheader{равенство*}
\scnidtf{равенство чисел*}
\scniselement{арифметическая операция}
\scniselement{бинарное отношение}
\scniselement{симметричное отношение}
\scniselement{рефлексивное отношение}
\scniselement{транзитивное отношение}
\scnexplanation{\textbf{\textit{равенство*}} – отношение взаимной заменяемости \textit{чисел}, которые именно в силу этой заменяемости и считаются равными. Равные \textit{числа} на числовой прямой совпадают.}
\scnrelfrom{описание типичного экземпляра}{
\scnfilescg{figures/sd_numbers/equality.png}}
\scnheader{больше или равно*}
\scniselement{арифметическая операция}
\scniselement{бинарное отношение}
\scniselement{отношение нестрогого порядка}
\scnexplanation{\textbf{\textit{больше или равно*}} – это \textit{арифметическая операция} сравнения чисел, при которой первое \textit{число} (первый компонент связки) может быть \textit{больше*} второго или \textit{равняться*} ему.}
\scnrelfrom{описание типичного экземпляра}{
\scnfilescg{figures/sd_numbers/more.png}}

\scnheader{Число Пи}
\scniselement{иррациональное число}
\scnexplanation{\textbf{\textit{Число Пи}} – это  математическая константа, равная отношению длины окружности к длине её диаметра.}

\scnheader{Число нуль}
\scnidtf{0}
\scniselement{целое число}
\scnexplanation{\textbf{\textit{Число нуль}} – это \textit{целое число}, разделяющее на числовой прямой \textit{положительные числа} и \textit{отрицательные числа}.}

\scnheader{Число один}
\scnidtf{1}
\scniselement{целое число}
\scniselement{натуральное число}
\scnexplanation{\textbf{\textit{Число один}} – это наименьшее \textit{натуральное число}.}

\scnheader{Мнимая единица}
\scnidtf{i}
\scniselement{комплексное число}
\scnexplanation{\textbf{\textit{Мнимая единица}} – это \textit{число}, при возведении которого в степень 2 результатом будет число, противоположное \textit{Числу один}.}

\scnheader{числовая структура}
\scnsubset{структура}
\scnexplanation{\textbf{\textit{числовая структура}} – \textit{структура}, в состав которой входят знаки \textit{арифметических выражений}, а также знаки их элементов и связи между выражениями и их элементами.}

\scnheader{система счисления}
\scniselement{параметр}
\scnexplanation{Каждая \textbf{\textit{система счисления}} представляет собой класс синтаксически эквивалентных файлов, хранимых в sc-памяти, каждый из которых может являться идентификатором какого-либо \textit{числа}.

Каждая \textbf{\textit{система счисления}} характеризуется алфавитом, т.е. конечным множеством символов (цифр), которые допускается использовать при построении файлов принадлежащих данной \textbf{\textit{системе счисления}}.}

\scnheader{десятичная система счисления}
\scniselement{система счисления}

\scnheader{двоичная система счисления}
\scniselement{система счисления}

\scnheader{шестнадцатеричная система счисления}
\scniselement{система счисления}

\scnendstruct

\end{SCn}

\scsection{Предметная область и онтология структур}
\label{sec:sd_structures}
\begin{SCn}

\scnsectionheader{\currentname}

\scnstartsubstruct

\scnheader{Предметная область структур}
\scnsdmainclasssingle{структура}
\scnsdclass{связная структура;несвязная структура;тривиальная структура;нетривиальная структура;структура второго уровня;семантический уровень структурного элемента;количество семантических уровней элементов структуры}

\scnsdrelation{элемент структуры’;непредставленное множество’;полностью представленное множество’;частично представленное множество’;элемент структуры, не являющийся множеством';максимальное множество’;немаксимальное множество’;первичный элемент’;вторичный элемент’;элемент второго уровня’;метасвязь’;полиморфность*;полиморфизм*;гомоморфность*;гомоморфизм*;изоморфность*;изоморфизм*;автоморфность*;автоморфизм*;аналогичность структур*;сходство*;различие*;первичная синтаксическая структура sc-текста*}

\scnheader{структура}
\scnidtf{sc-структура}
\scnidtf{структура, представленная в виде текста SC-кода}
\scnsubdividing{связная структура;несвязная структура}
\scnsubdividing{тривиальная структура;нетривиальная структура}
\scnexplanation{\textbf{\textit{структура}} — множество \textit{sc-элементов}, удаление одного из которых может привести к нарушению целостности этого множества.}

\scnheader{связная структура}
\scnexplanation{\textit{Структуре}, представленной в \textit{SC-коде}, поставим в соответствие орграф, вершинами которого являются \textit{sc-элементы}, а дугами – пары инцидентности, связывающие \textit{sc-коннекторы} с инцидентными им \textit{sc-элементами}, которые являются компонентами указанных \textit{sc-коннекторов}.

Если полученный таким способом орграф является связным орграфом, то исходную структуру будем считать \textbf{\textit{связной структурой}}.}

\scnheader{несвязная структура}
\scnexplanation{\textit{Структуре}, представленной в \textit{SC-коде}, поставим в соответствие орграф, вершинами которого являются \textit{sc-элементы}, а дугами – пары инцидентности, связывающие \textit{sc-коннекторы} с инцидентными им \textit{sc-элементами}, которые являются компонентами указанных \textit{sc-коннекторов}.

Если полученный таким способом орграф не является связным орграфом, то исходную структуру будем считать \textbf{\textit{несвязной структурой}}.}

\scnheader{тривиальная структура}
\scnidtf{структура первого уровня}
\scnexplanation{\textbf{\textit{тривиальная структура}} – \textit{структура}, не содержащая в качестве элементов связок.}

\scnheader{нетривиальная структура}
\scnsuperset{структура второго уровня}
\scnexplanation{\textbf{\textit{нетривиальная структура}} – \textit{структура}, среди элементов которой есть хотя бы одна связка.}

\scnheader{элемент структуры’}
\scniselement{неосновное понятие}
\scnsubdividing{непредставленное множество';полностью представленное множество’;частично представленное множество’;элемент структуры, не являющийся множеством'}
\scnsubdividing{максимальное множество’;немаксимальное множество'}
\scnexplanation{\textbf{\textit{элемент структуры'}} — \textit{неосновное понятие}, \textit{ролевое отношение}, указывающее на все элементы каждой структуры.

В рамках заданной структуры ее элементы можно классифицировать по заданным признакам:
\begin{scnitemize}
\item насколько полно в рамках \underline{заданной \textit{структуры}} представлено множество, обозначаемое \textit{заданным sc-элементом} вместе с соответствующими дугами принадлежности;
\item существуют ли в рамках \underline{заданной \textit{структуры}} \textit{sc-элементы}, обозначающие множества, являющиеся надмножествами того множества, которое обозначается \underline{заданным \textit{sc-элементом}};
\item уровень («этаж») иерархии перехода от знаков к метазнакам для \underline{заданного \textit{sc-элемента}} в рамках заданной \textit{структуры}.
\end{scnitemize}
}

\scnheader{непредставленное множество’}
\scnidtf{множество, не представленное в рамках данной структуры’}
\scnidtf{быть знаком множества, элементы которого не являются элементами данной структуры'}
\scniselement{ролевое отношение}
\scnexplanation{\textbf{\textit{непредставленное множество’}} – \textit{ролевое отношение}, связывающее структуру со знаком множества, все элементы которого не являются элементами данной структуры.}

\scnheader{полностью представленное множество’}
\scnidtf{множество, полностью представленное в рамках данной структуры'}
\scnidtf{множество, все элементы которого являются элементами данной структуры'}
\scnidtf{полностью представленный класс'}
\scniselement{ролевое отношение}
\scnexplanation{\textbf{\textit{полностью представленное множество’}} – \textit{ролевое отношение}, связывающее \textit{структуру} со знаком множества (любого семантического типа – класса, связки или структуры), все элементы которого являются элементами данной \textit{структуры}.}

\scnheader{частично представленное множество’}
\scnidtf{множество, частично представленное в рамках данной структуры'}
\scnidtf{множество, некоторые элементы которого являются элементами данной структуры'}
\scnidtf{быть знаком множества, некоторые элементы которого являются элементами данной структуры'}
\scniselement{ролевое отношение}
\scnexplanation{\textbf{\textit{частично представленное множество’}} – ролевое отношение, связывающее структуру со знаком множества, не все элементы которого являются элементами данной структуры.}

\scnheader{элемент структуры, не являющийся множеством’}
\scniselement{ролевое отношение}

\scnheader{максимальное множество’}
\scnexplanation{\textbf{\textit{максимальное множество’}} – \textit{ролевое отношение}, связывающее \textit{структуру} со знаком множества, для которого не существует множества, которое было бы надмножеством указанного множества и знак которого был бы элементом этой же структуры.}

\scnheader{немаксимальное множество’}
\scnexplanation{\textbf{\textit{немаксимальное множество’}} – \textit{ролевое отношение}, связывающее \textit{структуру} со знаком множества, для которого в рамках данной \textit{структуры} существует множество, являющееся надмножеством указанного множества.}

\scnheader{первичный элемент’}
\scnidtf{первичный элемент данной структуры'}
\scnidtf{sc-элемент первого уровня в рамках данной структуры'}
\scniselement{ролевое отношение}
\scniselement{семантический уровень структурного элемента}
\scnsubset{элемент структуры’}
\scnexplanation{\textbf{\textit{первичный элемент’}} – ролевое отношение, указывающее на элемент \textit{структуры}, являющийся либо терминальным элементом, либо знаком множества, такого что не существует другого элемента этой же структуры, который был бы элементом множества, обозначаемого первым из указанных элементов структуры. При этом соответствующая пара принадлежности может существовать, но в состав данной структуры не входить.}

\scnheader{вторичный элемент’}
\scnidtf{вторичный элемент данной структуры’}
\scnidtf{элемент данной структуры имеющий семантический уровень более 2'}
\scnidtf{непервичный элемент'}
\scniselement{ролевое отношение}
\scnsubset{элемент структуры’}
\scnexplanation{\textbf{\textit{вторичный элемент’}} – ролевое отношение, указывающее на элемент структуры, обозначающий множество, все или некоторые элементы которого являются элементами указанной структуры.}
\scnsuperset{элемент второго уровня’}

\scnheader{элемент второго уровня’}
\scniselement{ролевое отношение}
\scniselement{семантический уровень структурного элемента}
\scnexplanation{\textbf{\textit{элементом второго уровня’}} в рамках заданной \textit{структуры} может быть связка первичных элементов, тривиальная структура из первичных элементов или класс первичных элементов.}

\scnheader{структура второго уровня’}
\scnexplanation{\textbf{\textit{структура второго уровня}} - \textit{структура}, среди элементов которой есть хотя бы один \textit{элемент второго уровня’}.}

\scnheader{семантический уровень структурного элемента}
\scniselement{параметр}
\scnexplanation{\textbf{\textit{семантический уровень структурного элемента}} представляет собой \textit{параметр}, каждый элемент которого является классом 
\textit{sc-дуг принадлежности}, связывающих некоторую \textit{структуру} с теми ее элементами, который имеют одинаковый семантический уровень в рамках данной структуры. Значением данного параметра является число, обозначающее указанный семантический уровень.

\textbf{\textit{семантический уровень структурного элемента}} вычисляется следующим образом:

\begin{scnitemize}
\item элементы структуры, входящие в нее с атрибутом \textit{первичный элемент'} имеют семантический уровень 1;
\item уровень элемента, не являющегося \textit{первичным элементом'} структуры вычисляется путем прибавления 1 к максимальному из уровней элементов этого элемента (множества), входящих в эту же структуру. Например, \textit{sc-дуга}, соединяющая два \textit{первичных элемента' структуры} будет иметь семантический уровень 2, а \textit{sc-элемент}, обозначающий отношение, которому принадлежит указанная \textit{sc-дуга} – семантический уровень 3.
\end{scnitemize}
}
\scnrelfrom{типичная семантическая окрестность}{
\scnfilescg{figures/sd_structures/sem_level_struct_elem.png}
}

\scnheader{количество семантических уровней элементов структуры}
\scniselement{параметр}
\scnexplanation{\textbf{\textit{количество семантических уровней элементов структуры}} – параметр, каждый элемент которого представляет собой класс структур, у которых совпадает максимальный среди семантических уровней элементов этих структур.


Значением данного параметра является число, совпадающее с указанным максимальным семантическим уровнем элементов.}

\scnheader{метасвязь’}
\scniselement{ролевое отношение}
\scnsubset{вторичный элемент’}
\scnexplanation{
\begin{scnenumerate}
    \item Каждая входящая в структуру связь, хотя бы одним компонентом которой является связь, входящая в эту же структуру, элементами которой являются \textit{первичные элементы’} этой структуры, является \textbf{\textit{метасвязью’}} указанной структуры;
    \item Каждая входящая в структуру связь, хотя бы одним компонентом которой является \textbf{\textit{метасвязь’}} этой структуры также является \textbf{\textit{метасвязью’}} указанной структуры;
\end{scnenumerate}
}

\scnheader{полиморфность*}
\scnsubset{соответствие*}
\scniselement{бинарное отношение}
\scnexplanation{\textbf{\textit{полиморфность*}} - это \textit{соответствие}, заданное на \textit{структурах}, при котором каждому элементу из области определения соответствия (первой \textit{структуры}) ставится в соответствие один или более элемент из области значения соответствия (второй \textit{структуры}), при этом существует хотя бы один элемент области определения соответствия, которому соответствуют два или более элемента из области значения соответствия.}

\scnheader{полиморфизм*}
\scniselement{бинарное отношение}

\scnheader{гомоморфность*}
\scnidtf{гомоморфность структур*}
\scnsubset{соответствие*}
\scniselement{бинарное отношение}
\scnexplanation{\textbf{\textit{гомоморфность*}} - это \textit{соответствие}, заданное на \textit{структурах}, при котором каждому элементу из области определения соответствия (первой \textit{структуры}) ставится в соответствие только один элемент из области значения соответствия (второй \textit{структуры}).}
\scnrelfrom{типичная семантическая окрестность}{
\scnfilescg{figures/sd_structures/homomorphism.png}
}

\scnheader{гомоморфизм*}
\scniselement{бинарное отношение}

\scnheader{изоморфность*}
\scnidtf{изоморфное соответствие*}
\scnidtf{изоморфность структур*}
\scnsubset{гомоморфность*}
\scniselement{бинарное отношение}
\scnexplanation{\textbf{\textit{изоморфность*}} - это \textit{гомоморфность*}, при которой для каждого элемента из области значения существует ровно один соответствующий элемент из области определения.}
\scnrelfrom{типичная семантическая окрестность}{
\scnfilescg{figures/sd_structures/isomorphism.png}
}

\scnheader{изоморфизм*}
\scniselement{бинарное отношение}

\scnheader{автомоморфность*}
\scnsubset{гомоморфность*}
\scniselement{бинарное отношение}
\scnexplanation{\textbf{\textit{автоморфность*}} - это \textit{изоморфность*}, у которой область определения соответствия и область значения соответствия совпадают.}
\scnrelfrom{типичная семантическая окрестность}{
\scnfilescg{figures/sd_structures/automorphism.png}}

\scnheader{автоморфизм*}
\scniselement{бинарное отношение}

\scnheader{аналогичность структур*}
\scnsubset{соответствие*}
\scniselement{бинарное отношение}
\scnexplanation{\textbf{\textit{аналогичность структур*}} - \textit{соответствие*}, задаваемое на структурах, и фиксирующее факт наличия некоторой аналогии на подструктурах (подмножествах) указанных структур. Каждой ориентированной паре, принадлежащей \textbf{\textit{аналогичности структур*}} может быть поставлено в соответствие множество пар, задающих \textit{сходства*} некоторых подструктур и \textit{различия*} некоторых подструктур исходных структур.}
\scnrelfrom{типичная семантическая окрестность}{
\scnfilescg{figures/sd_structures/analogy.png}}

\scnheader{сходство*}
\scniselement{бинарное отношение}

\scnheader{различие*}
\scniselement{бинарное отношение}

\scnheader{первичная синтаксическая структура sc-текста*}
\scniselement{бинарное отношение}
\scnexplanation{\textbf{\textit{первичная синтаксическая структура sc-текста*}} - это бинарное отношение, связывающее некоторый \textit{sc-текст} с другим \textit{sc-текстом}, формируемым по следующим правилам:
\begin{scnitemize}
    \item каждому \textit{sc-узлу} первого \textit{sc-текста} соответствует \textit{синглетон} (\textit{знак sc-узла}) в рамках второго \textit{sc-текста};
    \item каждому \textit{sc-коннектору} из первого \textit{sc-текста} в рамках второго \textit{sc-текста} соответствует \textit{синглетон}, обозначающий данный \textit{sc-коннектор} и соединенный с другими \textit{синглетонами} второго \textit{sc-текста} парами инцидентности двух типов, в зависимости от того, началом или концом данного \textit{sc-коннектора} являются обозначаемые этими \textit{синглетонами sc-элементы}. В случае, когда \textit{sc-коннектор} является \textit{sc-ребром}, то достаточно пар инцидентности первого типа.
    \item для каждого \textit{синглетона} в рамках второго \textit{sc-текста} явно указывается синтаксический тип, определяемый типом соответствующего ему элемента из первого \textit{sc-текста} (\textit{знак sc-константы}, \textit{знак sc-узла} и т.п.).
\end{scnitemize}


Стоит отметить, что подобным образом может быть задана синтаксическая структура любого текста, а не только sc-текста. В этом случае понадобятся другие отношения инцидентности другие классы синтаксических типов.}
\scnrelfrom{типичная семантическая окрестность}{
\scnfilescg{figures/sd_structures/primary_sc_syntax.png}}

\scnendstruct \scnendcurrentsectioncomment

\end{SCn}

\scsection{Предметная область и онтология темпоральных сущностей}
\label{sec:sd_temp_entities}
\begin{SCn}

\scnsectionheader{Предметная область и онтология темпоральных сущностей}

\scnstartsubstruct

\scnheader{Предметная область темпоральных сущностей}
\scnidtf{Предметная область темпоральных связей и отношений}
\scnidtf{Предметная область временных сущностей}
\scniselement{предметная область}
\scnsdmainclasssingle{временная сущность}
\scnsdclass{прошлая сущность;настоящая сущность;будущая сущность;временная связь;ситуация;процесс;процесс в sc-памяти;процесс во внешней среде ostis-системы;материальная сущность;воздействие;отношение;класс временных связей;класс временных и постоянных связей;множество;ситуативное множество;неситуативное множество;частично ситуативное множество;темпоральная связь;темпоральное отношение;начало;завершение;длительность;тысячелетие;век;год;месяц;сутки;час;минута;секунда}
\scnsdrelation{воздействующая сущность*;объект воздействия*;начальная ситуация*;причинная ситуация*;конечная ситуация*;событие*;последний добавленный sc-элемент’;темпоральное включение*;темпоральная часть*;начальный этап*;конечный этап*;промежуточный этап*;темпоральное включение без совпадения начальных и конечных моментов*;темпоральное включение с совпадением начальных моментов*;темпоральное включение с совпадением конечных моментов*;темпоральное совпадение*;темпоральное объединение*;темпоральная декомпозиция*;темпоральная смежность*;темпоральная последовательность с промежутком*;темпоральная последовательность с пересечением*;номер тысячелетия';номер века';номер года';номер месяца в году';номер суток в месяце';номер часа в дне';номер минуты в часе';номер секунды в минуте'}

\scnheader{временная сущность}
\scnidtf{временно существующая сущность}
\scnidtf{нестационарная сущность}
\scnidtf{сущность, имеющая и/или начало, и/или конец своего существования}
\scnidtf{sc-элемент, являющийся знаком некоторой временно существующей сущности}
\scnidtf{сущность, обладающая темпоральными характеристиками (длительностью, начальным моментом, конечным моментом и т.д.)}
\scnreltoset{разбиение}{прошлая сущность;настоящая сущность;будущая сущность}
\scnreltoset{разбиение}{временная связь;ситуация;процесс;материальная сущность}
\scnexplanation{Следует отличать:
\begin{scnitemize}
    \item временный характер сущности, обозначаемой \textit{sc-элементом};
    \item временный характер существования самого \textit{sc-элемента} в рамках \textit{sc-памяти};
\end{scnitemize}
В ходе обработки информации каждый \textit{sc-элемент} может быть удален из \textit{sc-памяти}.}

\scnheader{прошлая сущность}
\scnidtf{сущность, существовавшая в прошлом времени}
\scnidtf{сущность прошлого времени}
\scnidtf{сущность, завершившая свое существование}

\scnheader{настоящая сущность}
\scnidtf{сущность, существующая в текущий момент времени}
\scnidtf{сущность, существующая сейчас}
\scnidtf{сущность настоящего времени}

\scnheader{будущая сущность}
\scnidtf{возможно будущая сущность}
\scnidtf{прогнозируемая временная сущность}
\scnidtf{временная сущность, которая может существовать в будущем}
\scnidtf{сущность, которая может или должна начать свое существование в будущем времени}
\scnrelfrom{включение}{инициированное действие}
\scnexplanation{Каждой \textbf{\textit{будущей сущности}} можно поставить в соответствие вероятность ее возникновения.}

\scnheader{временная связь}
\scnidtf{нестационарная связь}
\scnidtf{временно существующая связь}
\scnexplanation{Каждая \textbf{\textit{временная связь}} представляет собой \textit{связку}, принадлежащую множеству \textit{временных сущностей}.

Понятие \textbf{\textit{временной связи}} не следует путать с понятием \textit{темпоральной связи}, которая сама является \textit{постоянной сущностью}, описывающей то, как связаны во времени некоторые \textit{временные сущности}.
}

\scnheader{ситуация}
\scnidtf{состояние}
\scnidtf{временная структура}
\scnidtf{временно существующая структура}
\scnidtf{квазистационарная структура}
\scnidtf{состояние некоторой динамической системы, описываемое с некоторой степенью детализации (подробности)}
\scnidtf{квазистационарная структура, существующая временно (в течение некоторого отрезка времени)}
\scnrelto{включение}{структура}
\scnexplanation{Под ситуацией понимается \textit{структура}, содержащая, по крайней мере, один элемент, который является \textit{временной сущностью}. Наличие в рамках ситуации нескольких \textit{временных сущностей} говорит о том, что существует момент времени (в прошлом, настоящем или будущем), в который все они существуют одновременно.}

\scnheader{процесс}
\scnidtf{процесс преобразования некоторой временной сущности из одного состояния в другое}
\scnidtf{процесс перехода от одной ситуации к другой}
\scnidtf{переходный процесс}
\scnidtf{абстрактный процесс}
\scnidtf{информационная модель некоторых преобразований}
\scnidtf{динамическая sc-модель}
\scnidtf{динамическая структура}
\scnrelfrom{включение}{воздействие}
\scnrelto{включение}{структура}
\scnexplanation{Каждый \textbf{\textit{процесс}} определяется (задается) либо возникновением или исчезновением связей, связывающих заданную \textit{временную сущность} с другими сущностями, либо возникновением или исчезновением связей, связывающих части указанной \textit{временной сущности} с другими сущностями. 

Многим \textbf{\textit{процессам}} можно поставить в соответствие \textit{ситуацию}, которая является его \textit{начальной ситуацией*} и \textit{ситуацию}, которая является его \textit{конечной ситуацией*}, то есть указать \textit{ситуации}, переход между которыми осуществляется в ходе \textbf{\textit{процесса}}.

При этом знаки тех \textit{временных сущностей}, с которыми связаны изменения, описываемые некоторым \textbf{\textit{процессом}}, входят в данный \textbf{\textit{процесс}} как элементы и, при необходимости уточняются дополнительными \textit{ролевыми отношениями}.}
\scnreltoset{разбиение}{процесс в sc-памяти;процесс во внешней среде ostis-системы}

\scnheader{процесс в sc-памяти}

\scnheader{процесс во внешней среде ostis-системы}

\scnheader{материальная сущность}
\scnexplanation{Каждой \textbf{\textit{материальной сущности}} можно поставить в соответствие различные \textit{процессы}, описывающие ее преобразование из одного состояния в другое.}

\scnheader{воздействие}
\scnidtf{процесс, осуществляющийся на основе (в результате) воздействия одной сущности на другую}
\scnrelfrom{включение}{действие}
\scnexplanation{Каждому \textbf{\textit{воздействию}} может быть поставлена в соответствие (1) некоторая \textit{воздействующая сущность*}, т.е. сущность, осуществляющая \textbf{\textit{воздействие}} (в частности, это может быть некоторое физическое поле), и (2) некоторый \textit{объект воздействия*}, т.е. сущность, на которую воздействие направлено. Если \textbf{\textit{воздействие}} связано с \textit{материальной сущностью}, то его объектом воздействия является либо сама эта \textit{материальная сущность}, либо некоторая ее пространственная часть.}

\scnheader{воздействующая сущность*}

\scnheader{объект воздействия*}

\scnheader{начальная ситуация*}
\scnidtf{начальная ситуация процесса*}
\scnidtf{исходная ситуация*}
\scniselement{бинарное отношение}
\scnexplanation{Связки отношения \textbf{\textit{начальная ситуация*}} связывают некоторый \textit{процесс} и некоторую ситуацию, являющуюся начальной для этого \textit{процесса}, и, как правило, изменяемой в течение выполнения этого \textit{процесса}.

Первым компонентом каждой связки отношения \textbf{\textit{начальная ситуация*}} является знак \textit{процесса}, вторым – знак начальной \textit{ситуации}.}

\scnheader{причинная ситуация*}
\scniselement{бинарное отношение}
\scnrelto{включение}{начальная ситуация*}
\scnexplanation{Под причинной ситуацией понимается такая \textit{начальная ситуация*}, которая обладает достаточной полнотой для однозначного задания инициируемого \textit{процесса}.}

\scnheader{конечная ситуация*}
\scnidtf{конечная ситуация процесса*}
\scnidtf{результирующая ситуация*}
\scniselement{бинарное отношение}
\scnexplanation{Связки отношения \textbf{\textit{конечная ситуация*}} связывают некоторый \textit{процесс} и некоторую \textit{ситуацию}, ставшую результатом выполнения этого \textit{процесса}, то есть его следствием.

Первым компонентом каждой связки отношения \textbf{\textit{конечная ситуация*}} является знак \textit{процесса}, вторым – знак конечной \textit{ситуации}.}

\scnheader{событие*}
\scniselement{бинарное отношение}
\scnexplanation{Связки отношения \textbf{\textit{событие*}} связывают знак процесса и ориентированную пару, первым компонентом которой является знак \textit{начальной ситуации*} данного процесса, вторым компонентом – знак \textit{конечной ситуации*} данного процесса.}
\scnrelfrom{типичная семантическая окрестность}{
\scnfilelong{
\begin{figure}[H]
\centering
\includegraphics[width=1\linewidth]{figures/sd_temp_entities/event.png}
\end{figure}
}}

\scnheader{отношение}
\scnreltoset{разбиение}{класс временных связей;класс постоянных связей;класс временных и постоянных связей}

\scnheader{класс временных связей}
\scnidtf{отношение, все связки которого являются нестационарными}
\scnexplanation{В общем случае \textbf{\textit{класс временных связей}} не является \textit{ситуативным множеством}, поскольку факт принадлежности некоторой \textit{временной связи} такому классу следует считать постоянным, а не временным, поскольку временность/постоянство связи и ее семантический тип, задаваемый классом (отношением), это принципиально разные параметры (характеристики, признаки) любой связи.}

\scnheader{класс постоянных связей}
\scnidtf{отношение, все связки которого являются стационарными}

\scnheader{класс временных и постоянных связей}
\scnidtf{отношение, некоторые (но не все) связки которого являются нестационарными}

\scnheader{множество}
\scnreltoset{разбиение}{ситуативное множество;неситуативное множество;частично ситуативное множество}

\scnheader{ситуативное множество}
\scnidtf{полностью ситуативное множество}
\scnexplanation{Под \textbf{\textit{ситуативным множеством}} понимается постоянное множество, у которого все выходящие из него связи принадлежности являются \textit{временными сущностями}.

В частности, ситуативное множество может использоваться как вспомогательная динамическая структура, которая содержит элементы некоторых структур, обрабатываемые в данный момент, например, это может быть копия некоторого множества, из которой постепенно удаляются элементы по мере их просмотра и обработки. В случае, когда такая структура содержит всего один элемент, ее можно считать \underline{указателем} на данный элемент, при этом в разные моменты времени это могут быть разные элементы.}

\scnheader{последний добавленный sc-элемент’}
\scniselement{ролевое отношение}

\scnheader{неситуативное множество}
\scnexplanation{Под \textbf{\textit{неситуативным множеством}} понимается постоянное множество, у которого все выходящие из него связи принадлежности являются \textit{постоянными сущностями}.}

\scnheader{частично ситуативное множество}
\scnexplanation{Под \textbf{\textit{частично ситуативным множеством}} понимается постоянное множество, у которого некоторые (но не все) выходящие из него связи принадлежности являются \textit{временными сущностями}.}

\scnheader{темпоральная связь}
\scnidtf{постоянная связь, описывающая связь во времени между временными сущностями}

\scnheader{темпоральное отношение}
\scnrelto{семейство подмножеств}{темпоральная связь}
\scnidtf{класс темпоральных связей}
\scnidtf{отношение, задающее темпоральные связи между временными сущностями}
\scnhaselement{темпоральное включение*}
\scnhaselement{темпоральное объединение*}
\scnhaselement{темпоральная декомпозиция*}
\scnhaselement{темпоральная смежность*}
\scnhaselement{темпоральная последовательность с промежутком*}
\scnhaselement{темпоральная последовательность с пересечением*}

\scnheader{темпоральное включение*}
\scnexplanation{Связки отношения \textbf{\textit{темпоральное включение*}} связывают две \textit{временные сущности}, период существования одной из которых полностью включается в период существования второй.\\
Первым компонентом каждой связки отношения \textbf{\textit{темпоральное включение*}} является знак \textit{временной сущности}, \textit{длительность} существования которой больше.}
\scnrelfromlist{включение}{темпоральная часть*;темпоральное включение без совпадения начальных и конечных моментов*;темпоральное совпадение*;темпоральное включение с совпадением начальных моментов*;темпоральное включение с совпадением конечных моментов*}

\scnheader{темпоральная часть*}
\scnidtf{этап (период) заданной временной сущности*}
\scnidtf{этап процесса существования временной сущности*}
\scnrelfromlist{включение}{начальный этап*;конечный этап*;промежуточный этап*}
\scnrelfrom{типичная семантическая окрестность}{
\scnfilelong{
\begin{figure}[H]
\centering
\includegraphics[width=1\linewidth]{figures/sd_temp_entities/temporal_part.png}
\end{figure}
}}
\scnrelfrom{иллюстрация}{
\scnfilelong{
\begin{figure}[H]
\centering
\includegraphics[width=1\linewidth]{figures/sd_temp_entities/img_temporal_part.png}
\end{figure}
}}
\scntext{примечание}{Связки отношения \textbf{\textit{темпоральная часть*}} связывают две \textit{временные сущности}, одна из которых является частью другой, например, действие и одно из его поддействий. Соответственно, период существования одной из этих сущностей всегда будет включаться в период существования другой (большей).

В отличие от более общего отношения \textit{темпоральное включение*}, связки которого могут связывать любые \textit{временные сущности}, связки отношения \textbf{\textit{темпоральное включение*}} связывают только \textit{временные сущности}, одна из которых является частью другой.}

\scnheader{начальный этап*}

\scnheader{конечный этап*}

\scnheader{промежуточный этап*}

\scnheader{темпоральное включение без совпадения начальных и конечных моментов*}
\scnidtf{строгое темпоральное включение*}
\scnrelfrom{типичная семантическая окрестность}{
\scnfilelong{
\begin{figure}[H]
\centering
\includegraphics[width=1\linewidth]{figures/sd_temp_entities/strict_temporal_inclusion.png}
\end{figure}
}}
\scnrelfrom{иллюстрация}{
\scnfilelong{
\begin{figure}[H]
\centering
\includegraphics[width=1\linewidth]{figures/sd_temp_entities/img_strict_temporal_inclusion.png}
\end{figure}
}}
%
% темпоральное включение без совпадения начальных и конечных %моментов
%

\scnheader{темпоральное включение с совпадением начальных моментов*}
\scnrelfrom{типичная семантическая окрестность}{
\scnfilelong{
\begin{figure}[H]
\centering
\includegraphics[width=1\linewidth]{figures/sd_temp_entities/temporal_include_with_match_start_points.png}
\end{figure}
}}
\scnrelfrom{иллюстрация}{
\scnfilelong{
\begin{figure}[H]
\centering
\includegraphics[width=1\linewidth]{figures/sd_temp_entities/img_temporal_include_with_match_start_points.png}
\end{figure}
}}

\scnheader{темпоральное включение с совпадением конечных моментов*}
\scnrelfrom{типичная семантическая окрестность}{
\scnfilelong{
\begin{figure}[H]
\centering
\includegraphics[width=1\linewidth]{figures/sd_temp_entities/temporal_include_with_terminal_point_match.png}
\end{figure}
}}
\scnrelfrom{иллюстрация}{
\scnfilelong{
\begin{figure}[H]
\centering
\includegraphics[width=1\linewidth]{figures/sd_temp_entities/img_temporal_include_with_terminal_point_match.png}
\end{figure}
}}

\scnheader{темпоральное совпадение*}
\scnidtf{совпадение начала и завершения*}

\scnheader{темпоральное объединение*}
\scnrelfrom{типичная семантическая окрестность}{
\scnfilelong{
\begin{figure}[H]
\centering
\includegraphics[width=1\linewidth]{figures/sd_temp_entities/temporal_union.png}
\end{figure}
}}
\scnrelfrom{иллюстрация}{
\scnfilelong{
\begin{figure}[H]
\centering
\includegraphics[width=1\linewidth]{figures/sd_temp_entities/img_temporal_union.png}
\end{figure}
}}

\scnheader{темпоральная декомпозиция*}
\scnrelfrom{типичная семантическая окрестность}{
\scnfilelong{
\begin{figure}[H]
\centering
\includegraphics[width=1\linewidth]{figures/sd_temp_entities/temporal_decomposition.png}
\end{figure}
}}
\scnrelfrom{иллюстрация}{
\scnfilelong{
\begin{figure}[H]
\centering
\includegraphics[width=1\linewidth]{figures/sd_temp_entities/img_temporal_decomposition.png}
\end{figure}
}}

\scnheader{темпоральная смежность*}
\scnidtf{строгая темпоральная последовательность (без темпорального промежутка)*}
\scnidtf{темпоральная последовательность без промежутка*}
\scnrelfrom{типичная семантическая окрестность}{
\scnfilelong{
\begin{figure}[H]
\centering
\includegraphics[width=1\linewidth]{figures/sd_temp_entities/temporal_adjacency.png}
\end{figure}
}}
\scnrelfrom{иллюстрация}{
\scnfilelong{
\begin{figure}[H]
\centering
\includegraphics[width=1\linewidth]{figures/sd_temp_entities/img_temporal_adjacency.png}
\end{figure}
}}

\scnheader{темпоральная последовательность с промежутком*}
\scnrelfrom{типичная семантическая окрестность}{
\scnfilelong{
\begin{figure}[H]
\centering
\includegraphics[width=1\linewidth]{figures/sd_temp_entities/temporal_sequence_with_intermediate.png}
\end{figure}
}}
\scnrelfrom{иллюстрация}{
\scnfilelong{
\begin{figure}[H]
\centering
\includegraphics[width=1\linewidth]{figures/sd_temp_entities/img_temporal_sequence_with_intermediate.png}
\end{figure}
}}

\scnheader{темпоральная последовательность с пересечением*}
\scnrelfrom{типичная семантическая окрестность}{
\scnfilelong{
\begin{figure}[H]
\centering
\includegraphics[width=1\linewidth]{figures/sd_temp_entities/temporal_sequence_with_intersection.png}
\end{figure}
}}
\scnrelfrom{иллюстрация}{
\scnfilelong{
\begin{figure}[H]
\centering
\includegraphics[width=1\linewidth]{figures/sd_temp_entities/img_temporal_cross_sequence.png}
\end{figure}
}}

\scnheader{начало}
\scnidtf{класс одновременно начавшихся сущностей}
\scniselement{параметр}
\scnexplanation{Каждый элемент множества \textbf{начало} представляет собой класс \textit{временных сущностей}, у которых совпадает момент начала их существования. Конкретное значение данного \textit{параметра} может быть как \textit{точной величиной}, так и \textit{неточной величиной} или \textit{интервальной величиной}.}
\scnrelfrom{типичная семантическая окрестность}{
\scnfilelong{
\begin{figure}[H]
\centering
\includegraphics[width=1\linewidth]{figures/sd_temp_entities/start.png}
\end{figure}
}}
\scncomment{В данном примере \textit{ki} обозначает класс сущностей, начавших свое существование 19 февраля 2015 года по григорианскому календарю. Конкретные примеры таких сущностей – \textit{bi} и \textit{bj}. \textit{ti} обозначает временную точку григорианского календаря, соответствующую 19 февраля 2015 года.}

\scnheader{завершение}
\scnidtf{конец}
\scnidtf{класс одновременно завершившихся сущностей}
\scniselement{параметр}
\scnexplanation{Каждый элемент множества \textbf{\textit{завершение}} представляет собой класс \textit{временных сущностей}, у которых совпадает конечный момент их существования (момент завершения существования). Конкретное значение данного \textit{параметра} может быть как \textit{точной величиной}, так и \textit{неточной величиной} или \textit{интервальной величиной}.}
\scnrelfrom{типичная семантическая окрестность}{
\scnfilelong{
\begin{figure}[H]
\centering
\includegraphics[width=1\linewidth]{figures/sd_temp_entities/completion.png}
\end{figure}
}}
\scncomment{В данном примере \textit{ki} обозначает класс сущностей, завершивших свое существование 21 февраля 2015 года по григорианскому календарю. Конкретные примеры таких сущностей – \textit{bi} и \textit{bj}. \textit{ti} обозначает временную точку григорианского календаря, соответствующую 21 февраля 2015 года.}

\scnheader{длительность}
\scnidtf{класс временных сущностей, имеющих одинаковую длительность}
\scniselement{параметр}
\scnhaselement{тысячелетие}
\scnhaselement{век}
\scnhaselement{год}
\scnhaselement{месяц}
\scnhaselement{день}
\scnhaselement{час}
\scnhaselement{минута}
\scnhaselement{секунда}
\scnexplanation{Каждый элемент множества \textbf{\textit{длительность}} представляет собой класс \textit{временных сущностей}, у которых совпадает длительность их существования. Конкретное значение данного \textit{параметра} может быть как \textit{точной величиной}, так и \textit{неточной величиной} или \textit{интервальной величиной}.}
\scnrelfrom{типичная семантическая окрестность}{
\scnfilelong{
\begin{figure}[H]
\centering
\includegraphics[width=1\linewidth]{figures/sd_temp_entities/duration.png}
\end{figure}
}}
\scncomment{В данном примере \textit{ki} обозначает класс сущностей, существовавших в течение 2 месяцев. Конкретный пример такой сущности – \textit{bi}.}

\scnheader{тысячелетие}

\scnheader{век}

\scnheader{год}

\scnheader{месяц}

\scnheader{сутки}

\scnheader{час}

\scnheader{минута}

\scnheader{секунда}

\scnheader{номер тысячелетия'}
\scnheader{номер века'}
\scnheader{номер года'}
\scnheader{номер месяца в году'}
\scnheader{номер суток в месяце'}
\scnheader{номер часа в дне'}
\scnheader{номер минуты в часе'}
\scnheader{номер секунды в минуте'}

\scnendstruct

\end{SCn}

\scsection{Предметная область и онтология темпоральных сущностей баз знаний ostis-систем}
\label{sd_temp_know_base}
\begin{SCn}

\scnsectionheader{\currentname}

\scnstartsubstruct

\scntext{введение}{Обработка информации в \textit{sc-памяти} (т.е. динамика базы знаний, хранимой в \textit{sc-памяти}) в конечном счете сводится:
	\begin{scnitemize}
		\item к появлению в \textit{sc-памяти} новых актуальных \textit{sc-узлов} и \textit{sc-коннекторов};
		\item к логическому удалению актуальных \textit{sc-элементов}, т.е. к переводу их в неактуальное состояние (это необходимо для хранения протокола изменения состояния базы знаний, в рамках которого могут описываться действия по удалению \textit{sc-элементов});
		\item к возврату логически удаленных \textit{sс-элементов} в статус актуальных (необходимость в этом может возникнуть при откате базы знаний к какой-нибудь ее прошлой версии);
		\item к физическому удалению \textit{sc-элементов};
		\item к изменению состояния актуальных (логически не удаленных \textit{sc-элементов}) -- \textit{sc-узел} может превратиться в \textit{sc-ребро}, \textit{sc-ребро} может превратиться в \textit{sc-дугу}, \textit{sc-дуга} может поменять направленность, \textit{sc-дуга} общего вида может превратиться в \textit{константную стационарную sc-дугу принадлежности}, и т.д.;
	\end{scnitemize}
	Подчеркнем, что временный характер самого \textit{sc-элемента} (т.к. он может появиться или исчезнуть) никак не связан с возможно временным характером сущности, обозначаемой этим \textit{sc-элементом}. Т.е. временный характер самого sc-элемента и временный характер сущности, которую он обозначает -- абсолютно разные вещи.
	
	Таким образом, следует четко отличать динамику внешнего мира, описываемого базой знаний, а динамику самой базы знаний (динамику внутреннего мира). При этом очень важно, чтобы описание динамики базы знаний также входило в состав каждой базы знаний.
	
	К числу понятий, используемых для описания динамики базы знаний относятся:
	\begin{scnitemize}
		\item логически удаленный sc-элемент;
		\item сформированное множество;
		\item вычисленное число;
		\item сформированное высказывание;
\end{scnitemize}}

\scnheader{Предметная область темпоральных сущностей базы знаний ostis-системы}
\scnidtf{Предметная область, описывающая динамику базы знаний, хранимой в sc-памяти}
\scniselement{предметная область}
\scnsdmainclasssingle{ситуация}
\scnsdclass{sc-элемент;наcтоящий sc-элемент;логически удаленный sc-элемент;число;невычисленное число;вычисленное число;понятие;основное понятие;неосновное понятие;понятие, переходящее из основного в неосновное;понятие, переходящее из неосновного в основное;специфицированная сущность;недостаточно специфицированная сущность;достаточно специфицированная сущность;средне специфицированная сущность;структура;файл;событие в sc-памяти*;элементарное событие в sc-памяти*;событие добавления sc-дуги, выходящей из заданного sc-элемента*;событие добавления sc-дуги, входящей в заданный sc-элемент*;событие добавления sc-ребра, инцидентного заданному sc-элементу*;событие удаления sc-дуги, выходящей из заданного sc-элемента*;событие удаления sc-дуги, входящей в заданный sc-элемент*;событие удаления sc-ребра, инцидентного заданному sc-элементу*;событие удаления sc-элемента*;событие изменения содержимого файла*}

\scnheader{sc-элемент}
\scnreltoset{разбиение}{наcтоящий sc-элемент;логически удаленный sc-элемент}

\scnheader{наcтоящий sc-элемент}
\scniselement{ситуативное множество}

\scnheader{логически удаленный sc-элемент}
\scniselement{ситуативное множество}

\scnheader{число}
\scnsubdividing{невычисленное число;вычисленное число}

\scnheader{невычисленное число}
\scniselement{ситуативное множество}

\scnheader{вычисленное число}

\scnheader{понятие}
\scnsubdividing{основное понятие;неосновное понятие;понятие, переходящее из основного в неосновное;понятие, переходящее из неосновного в основное}

\scnheader{основное понятие}
\scnidtf{основное понятие для данной ostis-системы}
\scniselement{ситуативное множество}
\scnexplanation{К \textbf{\textit{основным понятиям}} относятся те понятия, которые активно используются в системе и могут быть ключевыми элементами sc-агентов. К \textbf{\textit{основным понятиям}} относятся также все неопределяемые понятия.}

\scnheader{неосновное понятие}
\scnidtf{дополнительное понятие}
\scnidtf{вспомогательное понятие}
\scnidtf{неосновное понятие для данной ostis-системы}
\scniselement{ситуативное множество}
\scnexplanation{Каждое \textbf{\textit{неосновное понятие}} должно быть строго определено на основе \textit{основных понятий}. Такие \textbf{\textit{неосновные понятия}} используются только для понимания и правильного восприятия вводимой информации, в том числе, для выравнивания онтологий. Ключевым элементом \textit{sc-агентов} \textbf{\textit{неосновные понятия}} быть не могут.}
\scntext{правило идентификации экземпляров}{В случае, когда некоторое понятие полностью перешло из \textit{основных понятий} в неосновные, то есть стало \textbf{\textit{неосновным понятием}}, и соответствующее ему \textit{основное понятие} (через которое оно определяется) в рамках некоторого внешнего языка имеет одинаковый с ним основной идентификатор, то к идентификатору \textbf{\textit{неосновного понятия}} спереди добавляется знак \#. Если при этом соответствуюшее \textit{основное понятие} имеет в идентификаторе знак \$, добавленный в процессе перехода, то этот знак удаляется. Если указанные понятия имеют разные основные идентификаторы в рамках этого внешнего языка, то никаких дополнительных средств идентификации не используется.

Например:\\
\textit{\#трансляция sc-текста}\\
\textit{\#scp-программа}}

\scnheader{понятие, переходящее из основного в неосновное}
\scniselement{ситуативное множество}

\scnheader{понятие, переходящее из неосновного в основное}
\scniselement{ситуативное множество}
\scntext{правило идентификации экземпляров}{В случае, когда текущее \textit{основное понятие} и соответствующее ему \textbf{\textit{понятие, переходящее из неосновного в основное}} в рамках некоторого внешнего языка имеют одинаковый основной идентификатор, то к идентификатору понятия, переходящего из неосновного в основное спереди добавляется знак \$. Если указанные понятия имеют разные основные идентификаторы в рамках этого внешнего языка, то никаких дополнительных средств идентификации не используется.

Например:\\
\textit{\$трансляция sc-текста}\\
\textit{\$scp-программа}}

\scnheader{специфицированная сущность}
\scnsubdividing{недостаточно специфицированная сущность;достаточно специфицированная сущность;средне специфицированная сущность}

\scnheader{достаточно специфицированная сущность}
\scnexplanation{К \textbf{\textit{достаточно специфицированным сущностям}} предъявляются следующие требования:
\begin{scnitemize}
    \item если сущность не является понятием, то для нее должны быть указаны
    \begin{scnitemizeii}
    \item различные варианты обозначающих ее внешних знаков;
    \item классы, которым она принадлежит;
    \item связки, которыми она связана с другими сущностями (с указанием соответствующего отношения);
    \item значения параметров, которыми она обладает;
    \item те разделы базы знаний, в которых указанная сущность является ключевой;
    \item предметные области, в которые данная сущность входит.
    \end{scnitemizeii}
    \item если специфицированная сущность является понятием, то для нее должны быть указаны:
    \begin{scnitemizeii}
    \item различные варианты внешних обозначений этого понятия;
    \item предметные области, в которых это понятие исследуется;
    \item определение понятия;
    \item пояснения
    \item разделы базы знаний, в которых это понятие является ключевым;
    \item описание примера -- пример экземпляра понятия.
    \end{scnitemizeii}
\end{scnitemize}}

\scnheader{структура}
\scnsubdividing{сформированная структура;несформированная структура}
\scnsubdividing{недостаточно сформированная структура;достаточно сформированная структура;структура, имеющая средний уровень сформированности}

\scnheader{файл}
\scnsubdividing{недостаточно сформированный внутренний файл;достаточно сформированный внутренний файл;внутренний файл, имеющий средний уровень сформированности}

\scnheader{событие в sc-памяти}
\scnsuperset{событие}

\scnheader{элементарное событие в sc-памяти}
\scnsubset{событие в sc-памяти}
\scnexplanation{Под \textbf{\textit{элементарным событием в sc-памяти}} понимается такое \textit{событие}, в результате выполнения которого изменяется состояние только одного \textit{sc-элемента}.}
\scnsubdividing{событие добавления sc-дуги, выходящей из заданного sc-элемента
;событие добавления sc-дуги, входящей в заданный sc-элемент;событие добавления sc-ребра, инцидентного заданному sc-элементу;событие удаления sc-дуги, выходящей из заданного sc-элемента;событие удаления sc-дуги, входящей в заданный sc-элемент;событие удаления sc-ребра, инцидентного заданному sc-элементу;событие удаления sc-элемента;событие изменения содержимого файла}

\scnheader{точечный процесс}
\scnidtf{атомарный процесс}
\scnidtf{условно мгновенный процесс}
\scnidtf{процесс, длительность которого в данном контексте считается несущественной (пренебрежимо малой)}

\scnheader{элементарный процесс}
\scnidtf{процесс, детализация которого на входящие в него подпроцессы в текущем контексте не производится}

\bigskip
\scnendstruct \scnendcurrentsectioncomment

\end{SCn}

\scsection{Предметная область и онтология пространственных сущностей различных форм}
\label{sd_spatial_entities}

\scsection{Предметная область и онтология материальных сущностей}
\label{sd_material_entities}

\scsubsection{Предметная область и онтология персон}
\label{sd_person}

\scsubsection{Предметная область и онтология организаций}
\label{sd_organiztion}

\scsubsection{Предметная область и онтология географических объектов}
\label{sd_geograph_obj}

\scsection{Предметная область и онтология семантических окрестностей}
\label{sec:sd_sem_neigh}
\begin{SCn}

\scnsectionheader{\currentname}

\scnstartsubstruct

\scnheader{Предметная область семантических окрестностей}
\scniselement{предметная область}
\scnsdmainclasssingle{семантическая окрестность}
\scnsdclass{семантическая окрестность по инцидентным коннекторам;семантическая окрестность по выходящим дугам;семантическая окрестность по выходящим дугам принадлежности;семантическая окрестность по входящим дугам;семантическая окрестность по входящим дугам принадлежности;полная семантическая окрестность;базовая семантическая окрестность;специализированная семантическая окрестность;пояснение;примечание;правило идентификации экземпляров;терминологическая семантическая окрестность;теоретико-множественная семантическая окрестность;описание декомпозиции;логическая семантическая окрестность;описание типичного экземпляра;сравнительный анализ;иллюстрация}

\scnheader{семантическая окрестность}
\scnidtf{sc-окрестность}
\scnidtf{семантическая окрестность, представленная в виде sc-текста}
\scnidtf{sc-текст, являющийся семантической окрестностью некоторого sc-элемента}
\scnidtf{спецификация заданной сущности, знак которой указывается как ключевой элемент этой спецификации}
\scnidtf{описание заданной сущности, знак которой указывается как ключевой элемент этой спецификации}
\scnsubset{знание}
\scnsuperset{семантическая окрестность по инцидентным коннекторам}
\scnsuperset{полная семантическая окрестность}
\scnsuperset{базовая семантическая окрестность}
\scnsuperset{специализированная семантическая окрестность}
\scnexplanation{\textbf{\textit{семантическая окрестность}} – это знание, являющееся спецификацией (описанием) некоторой сущности, знак которой является ключевым элементом указанного знания. Заметим, что каждая семантическая окрестность в отличие от знаний других видов имеет только один ключевой элемент (ключевой знак, знак описываемой сущности). Заметим также, что многообразие видов семантических окрестностей свидетельствует о многообразии семантических видов описаний различных сущностей.}

\scnheader{семантическая окрестность по инцидентным коннекторам}
\scnsuperset{семантическая окрестность по выходящим дугам}
\scnsuperset{семантическая окрестность по входящим дугам}
\scnexplanation{\textbf{\textit{семантическая окрестность по инцидентным коннекторам}} – это вид семантической окрестности, в которую входят знаки всех коннекторов, инцидентных заданному элементу, а также знаки всех элементов, инцидентных указанным коннекторам.}

\scnheader{семантическая окрестность по выходящим дугам}
\scnsuperset{семантическая окрестность по выходящим дугам принадлежности}
\scnexplanation{\textbf{\textit{семантическая окрестность по выходящим дугам}} – это вид семантической окрестности, в которую входят знаки всех дуг, выходящих из заданного sc-элемента, а также знаки их вторых компонентов, также указывается факт принадлежности этих дуг каким-либо отношениям.}

\scnheader{семантическая окрестность по выходящим дугам принадлежности}
\scnexplanation{\textbf{\textit{семантическая окрестность по выходящим дугам принадлежности}} – это вид семантической окрестности, в которую входят знаки всех дуг принадлежности, выходящих из заданного sc-элемента, а также знаки их вторых компонентов. При необходимости может указывается факт принадлежности этих дуг каким-либо ролевым отношениям.}

\scnheader{семантическая окрестность по входящим дугам}
\scnsuperset{семантическая окрестность по входящим дугам принадлежности}
\scnexplanation{\textbf{\textit{семантическая окрестность по входящим дугам}} – это вид семантической окрестности, в которую входят знаки всех дуг, входящих в заданный sc-элемент, а также знаки их первых компонентов, также указывается факт принадлежности этих дуг каким-либо отношениям.}

\scnheader{семантическая окрестность по входящим дугам принадлежности}
\scnexplanation{\textbf{\textit{семантическая окрестность по входящим дугам принадлежности}} – это вид семантической окрестности, в которую входят знаки всех дуг принадлежности, входящих в заданный sc-элемент, а также знаки их первых компонентов. При необходимости может указывается факт принадлежности этих дуг каким-либо ролевым отношениям.}

\scnheader{полная семантическая окрестность}
\scnidtf{полная спецификация некоторой описываемой сущности}
\scnexplanation{\textbf{\textit{полная семантическая окрестность}} – это вид семантической окрестности, включающий описание всех связей описываемой сущности. 

Структура полной семантической окрестности определяется прежде всего семантической типологией описываемой сущности. 

Так, например, для понятия в полную семантическую окрестность необходимо включить следующую информацию (при наличии):
\begin{scnitemize}
    \item варианты идентификации на различных внешних языках;
    \item принадлежность некоторой предметной области с указанием роли, выполняемой в рамках этой предметной области;
    \item теоретико-множественные связи заданного понятия с другими sc-элементами;
    \item определение или пояснение;
    \item высказывания, описывающие свойства указанного понятия;
    \item задачи и их классы, в которых данное понятие является ключевым
    \item описание типичного примера использования указанного понятия;
    \item экземпляры описываемого понятия.
\end{scnitemize}
Для понятия, являющегося отношением дополнительно указываются:
\begin{scnitemize}
    \item домены;
    \item область определения;
    \item схема отношения;
    \item классы отношений, которым принадлежит описываемое отношение.
\end{scnitemize}
}

\scnheader{базовая семантическая окрестность}
\scnidtf{минимально достаточная семантическая окрестность}
\scnidtf{минимальная спецификация описываемой сущности}
\scnidtf{сокращенная спецификация описываемой сущности}
\scnidtf{основная семантическая окрестность}
\scnexplanation{\textbf{\textit{базовая семантическая окрестность}} – это вид семантической окрестности, содержащий минимальную (краткую) информацию об описываемой сущности

Структура базовой семантической окрестности определяется прежде всего семантической типологией описываемой сущности. 

Так, например, для понятия в базовую семантическую окрестность необходимо включить следующую информацию (при наличии): 
\begin{scnitemize}
    \item варианты идентификации на различных внешних языках;
    \item принадлежность некоторой предметной области с указанием роли, выполняемой в рамках этой предметной области;
    \item определение или пояснение.
\end{scnitemize}
Для понятия, являющегося отношением дополнительно указываются:
\begin{scnitemize}
    \item домены;
    \item область определения;
    \item описание типичного примера использования указанного отношения.
\end{scnitemize}
}

\scnheader{специализированная семантическая окрестность}
\scnsuperset{пояснение}
\scnsuperset{примечание}
\scnsuperset{правило идентификации экземпляров}
\scnsuperset{терминологическая семантическая окрестность}
\scnsuperset{теоретико-множественная семантическая окрестность}
\scnsuperset{логическая семантическая окрестность}
\scnsuperset{описание типичного экземпляра}
\scnsuperset{описание декомпозиции} 
\scnexplanation{\textbf{\textit{специализированная семантическая окрестность}} – это вид семантической окрестности, набор связей для которой уточняется отдельно для каждого класса такой окрестности.}

\scnheader{пояснение}
\scnidtf{sc-пояснение}
\scnexplanation{\textbf{\textit{пояснение}} – знак sc-текста, поясняющего описываемую сущность.}

\scnheader{примечание}
\scnidtf{sc-примечание}
\scnexplanation{\textbf{\textit{примечание}} – знак sc-текста, являющегося примечанием к описываемой сущности. В примечании обычно описываются особые свойства и исключения из правил для описываемой сущности.}

\scnheader{правило идентификации экземпляров}
\scnidtf{правило идентификации экземпляров заданного класса}
\scnexplanation{\textbf{\textit{правило идентификации экземпляров}} – это sc-текст являющийся описанием правил построения идентификаторов элементов заданного класса.}

\scnheader{терминологическая семантическая окрестность}
\scnexplanation{\textbf{\textit{терминологическая семантическая окрестность}}  –  семантическая окрестность, описывающая идентификацию указанной сущности}

\scnheader{теоретико-множественная семантическая окрестность}
\scnexplanation{\textbf{\textit{теоретико-множественная семантическая окрестность}}  –  описание связи описываемого понятия с другими понятиями с помощью теоретико-множественных отношений}

\scnheader{описание декомпозиции}
\scnidtf{семантическая окрестность, описывающая декомпозицию некоторой сущности}
\scnexplanation{\textbf{\textit{описание декомпозиции}}  –  семантическая окрестность, описывающая декомпозицию некоторой сущности на частные сущности}

\scnheader{логическая семантическая окрестность }
\scnexplanation{\textbf{\textit{логическая семантическая окрестность}}  –  семантическая окрестность, описывающая семейство высказываний, описывающих свойства данного понятия}

\scnheader{описание типичного экземпляра}
\scnidtf{описание типичного экземпляра заданного класса}
\scnidtf{типичная семантическая окрестность}
\scnidtf{типичная sc-окрестность}
\scnexplanation{\textbf{\textit{описание типичного экземпляра}} – это sc-текст являющийся описанием типичного примера использования рассматриваемого класса.}

\scnheader{сравнительный анализ}
\scnexplanation{\textbf{\textit{сравнительный анализ}} –  описание сравнительного анализа некоторой сущности с другими сущностями}

\scnheader{иллюстрация}
\scnsubset{специализированная семантическая окрестность}
\scnexplanation{\textbf{\textit{иллюстрация}} –  семантическая окрестность некоторой сущности (сущностей), иллюстрирующая некоторые свойства указанных сущностей, чаще всего, на некотором конкретном примере.}

\scnendstruct

\end{SCn}

\scsection{Предметная область и онтология предметных областей}
\label{sec:sd_sd}
\begin{SCn}
\scnsectionheader{\currentname}
\begin{scnsubstruct}
\begin{scnreltovector}{конкатенация сегментов}
\scnitem{Что такое предметная область}
\scnitem{Роли знаков, входящих в состав предметных областей}
\scnitem{Типология предметных областей и отношения, заданных на множестве предметных областей}
\scnitem{Что такое sc-язык}
\end{scnreltovector}
\scnheader{Предметная область предметных областей}
\scnidtf{Предметная область, объектами исследования которой являются предметные области}
\scntext{explanation}{В состав \textbf{\textit{Предметной области предметных областей}} входят структурные спецификации всех \textit{предметных областей}, входящих в состав базы знаний \textit{ostis-системы}, в том числе, самой \textbf{\textit{Предметной области предметных областей}}. Таким образом, \textbf{\textit{Предметная область предметных областей}} является, во-первых, \textit{рефлексивным множеством}, во-вторых, рефлексивной предметной областью, то есть \textit{предметной областью}, одним из объектов исследования которой является она сама.}\scniselement{рефлексивное множество}
\begin{scnhaselementrole}{класс объектов исследования}
предметная область\end{scnhaselementrole}
\begin{scnhaselementrolelist}{класс объектов исследования}
статическая предметная область;динамическая предметная область;понятие;sc-язык
\end{scnhaselementrolelist}
\begin{scnhaselementrolelist}{исследуемое отношение}
понятие предметной области\scnrolesign ;исследуемое понятие\scnrolesign ;максимальный класс объектов исследования\scnrolesign ;немаксимальный класс объектов исследования\scnrolesign ;исследуемый класс первичных элементов\scnrolesign ;исследуемое отношение\scnrolesign ;класс исследуемых структур\scnrolesign ;понятие, исследуемое в дочерней предметной области\scnrolesign ;понятие, исследуемое в материнской предметной области\scnrolesign ;вспомогательное понятие\scnrolesign ;дочерняя предметная область*;дочерняя предметная область по классу первичных элементов*;дочерняя предметная область по исследуемым отношениям*;предметная область sc-языка*
\end{scnhaselementrolelist}
\end{scnsubstruct}
\scnsegmentheader{Что такое предметная область}
\begin{scnsubstruct}
\scnheader{предметная область}
\scnidtf{sc-модель предметной области}
\scnidtf{sc-текст предметной области}
\scnidtf{sc-граф предметной области}
\scnidtf{представление предметной области в \textit{SC-коде}}
\scnsubset{знание}
\scnsubset{бесконечное множество}
\scntext{explanation}{\textbf{\textit{предметная область}} -- это результат интеграции (объединения) частичных семантических окрестностей, описывающих все исследуемые сущности заданного класса и имеющих одинаковый (общий) предмет исследования (то есть один и тот же набор отношений, которым должны принадлежать связки, входящие в состав интегрируемых семантических окрестностей).\textbf{\textit{предметная область}} -- \textit{структура}, в состав которой входят:\begin{scnitemize}
\item \textnormal{основные исследуемые (описываемые) объекты -- первичные и вторичные;}\item \textnormal{различные классы исследуемых объектов;}\item \textnormal{различные связки, компонентами которых являются исследуемые объекты (как первичные, так и вторичные), а также, возможно, другие такие связки -- то есть связки (как и объекты исследования) могут иметь различный структурный уровень;}\item \textnormal{различные классы указанных выше связок (то есть отношения);}\item \textnormal{различные классы объектов, не являющихся ни объектами исследования, ни указанными выше связками, но являющихся компонентами этих связок.}\end{scnitemize}
При этом все классы, объявленные исследуемыми понятиями, должны быть полностью представлены в рамках данной предметной области вместе со своими элементами, элементами элементов и т.д. вплоть до терминальных элементов.Можно говорить о типологии \textbf{\textit{предметных областей}} по разным структурным признакам:\begin{scnitemize}
\item наличие метасвязей;\item наличие исследуемых структур, входящих в состав предметной области;\item наличие исследуемых (смежных, дополнительных) объектов, которых исследуются в других предметных областях;\end{scnitemize}
Понятие \textbf{\textit{предметной области}} является важнейшим методологическим приемом, позволяющим выделить из всего многообразия исследуемого Мира только определенный класс исследуемых сущностей и только определенное семейство отношений, заданных на указанном классе. То есть осуществляется локализация, фокусирование внимания только на этом, абстрагируясь от всего остального исследуемого Мира.Во всем многообразии \textbf{\textit{предметных областей}} особое место занимают\begin{scnitemize}
\item \textit{Предметная область предметных областей}, объектами исследования которой являются всевозможные \textbf{\textit{предметные области}}, а предметом исследования -- всевозможные \textit{ролевые отношения}, связывающие предметные области с их элементами, отношения, связывающие предметные области между собой, отношение, связывающее предметные области с их онтологиями\item \textit{Предметная область сущностей}, являющаяся предметной областью самого высокого уровня и задающая базовую семантическую типологию \textit{sc-элементов}(знаков, входящих в тексты \textit{SC-кода})\item Семейство \textbf{\textit{предметных областей}}, каждая из которых задает семантику и синтаксис некоторого \textit{sc-языка}, обеспечивающего представление онтологий соответствующего вида (например, \textit{теоретико-множественных онтологий}, \textit{логических онтологий}, \textit{терминологических онтологий}, \textit{онтологий задач и способов их решения} и т.д.)\item Семейство \textbf{\textit{предметных областей}} верхнего уровня, в которых классами объектов исследования являются весьма крупные классы сущностей. К таким классам, в частности\begin{scnitemizeii}
\item класс всевозможных \textit{материальных сущностей},\item класс всевозможных \textit{множеств},\item класс всевозможных \textit{связей},\item класс всевозможных \textit{отношений},\item класс всевозможных \textit{структур},\item класс всевозможных \textit{временных (временно существующих, непостоянных сущностей) сущностей},\item класс всевозможных \textit{действий} (акций),\item класс всевозможных \textit{параметров} (характеристик),\item класс \textit{знаний} всевозможного вида \item и т.п.\end{scnitemizeii}
\end{scnitemize}
Каждой \textbf{\textit{предметной области}} можно поставить в соответствие:\begin{scnitemize}
\item семейство соответствующих ей \textit{онтологий} разного вида;\item некий язык (в нашем случае -- язык, построенный на основе \textit{SC-кода}), тексты которого представляют различные фрагменты соответствующей предметной области\end{scnitemize}
Указанные языки будем называть \textit{sc-языками}. Их синтаксис и семантика полностью задается \textit{SС-кодом} и \textit{онтологией} соответствующей \textbf{\textit{предметной области}}. Очевидно, что в первую очередь нас должны интересовать те \textit{sc-языки}, которые соответствуют \textbf{\textit{предметным областям}}, имеющим общий (условно говоря, предметно независимый) характер. К таким предметным областям, в частности, относятся:\begin{scnitemize}
\item \textit{Предметная область множеств}, описывающая множества и различные связи между ними\item \textit{Предметная область отношений и соответствий}\item \textit{Предметная область структур} (в частности, графовых)\item \textit{Предметная область чисел и числовых структур}\item и т.д\end{scnitemize}
Каждому типу знаний можно поставить в соответствие предметную область, которая является результатом интеграции всех знаний данного типа. Эти знания и становятся объектами исследования в рамках указанной предметной области.Понятие \textbf{\textit{предметной области}} может рассматриваться как обобщение понятия алгебраической системы. При этом семантическая структура базы знаний может рассматриваться как иерархическая система различных \textbf{\textit{предметных областей}}.}\scnidtf{система связей некоторого множества объектов исследования, \uline{ключевыми} элементами которой являются:\begin{scnitemize}
\item классы (точнее, знаки классов) объектов исследования (объектов, описываемых этой предметной областью);\item конкретные объекты исследования, обладающие особыми свойствами;\item классы связей, входящих в состав рассматриваемой системы -- отношения, заданные на множестве элементов рассматриваемой системы;\item параметры, заданные на множестве элементов рассматриваемой системы;\item классы структур, являющихся фрагментами рассматриваемой системы.\end{scnitemize}
}
\scnidtf{структура, представляющая собой множество связей (точнее, знаков связей) и соответствующее множество компонентов этих связей, к числу которых относится:\begin{scnitemize}
\item элементы (экземпляры) некоторых заданных классов \uline{объектов исследования} (первичных исследуемых сущностей);\item сами связи, входящие в состав указанной структуры;\item введенные классы объектов исследования;\item введенные отношения (классы связей);\item введенные параметры (классы классов эквивалентных сущностей);\item значения параметров (и, в частности, величины для измеряемых параметров);\item введенные структуры, являющиеся фрагментами (подструктурами) рассматриваемой структуры;\item введенные классы подструктур рассматтриваемой структуры.\end{scnitemize}
}
\scntext{note}{Выделяемые в рамках \textit{базы знаний} интеллектуальной системы \textit{предметные области} и соответствующие им \textit{онтологии} -- это, своего рода, семантические страты, кластеры, позволяющие разложить все хранимые в памяти \textit{знания} по семантическим полочкам при наличии четких критериев, позволяющих \uline{однозначно} определить то, на какой полочке должны находиться те или иные \textit{знания}}\scntext{note}{Существуют предметные области, в которых основным исследуемым понятием является множество всевозможных связей между экземплярами понятий, исследуемых в других предметных областях. Так, например, можно ввести Предметную область треугольников, Предметную область окружностей, а также Предметную область связей между треугольниками и окружностями.}
\end{scnsubstruct}
\scnsegmentheader{Роли знаков, входящих в состав предметной области}
\begin{scnsubstruct}
\scnheader{роль элемента предметной области}
\scnidtf{ролевое отношения, связывающее предметные области с их ключевыми знаками}
\scnidtf{роль ключевого элемента (знака ключевой сущностей) предметной области}
\scnidtf{роль ключевого знака предметной области}
\scnhaselement{класс объектов исследования\scnrolesign}
\scnhaselement{максимальный класс объектов исследования\scnrolesign}
\scnhaselement{ключевой объект исследования\scnrolesign}
\scnhaselement{понятие, используемое в предметной области\scnrolesign}
\scnhaselement{первичный исследуемый элемент предметной области\scnrolesign}
\scnhaselement{вторичный исследуемый элемент предметной области\scnrolesign}
\scnhaselement{неисследуемый элемент предметной области\scnrolesign}
\scnheader{класс объектов исследования\scnrolesign}
\scnidtf{быть классом \uline{первичных} (для данной предметной области) объектов исследования\scnrolesign}
\scntext{note}{Понятие \uline{первичного} объекта исследования для предметной области является понятием \uline{относительным} и абсолютно не зависит от типа и уровня сложности этого объекта. Само исследование (спецификация) таких первичных исследуемых объектов осуществляется:\begin{scnitemize}
\item путем введения различных классов объектов исследования, которым эти объекты принадлежат;\item путем введения различных связок из первичных объектов исследования и различных классов таких связок (отношений), которым принадлежат введенные связки;\item путем введения таких классов первичных объектов исследования, которые являются значениями вводимых параметров;\item путем введения различных структур, состоящих из первичных объектов исследования, из связок таких объектов, из введенных отношений и классов первичных объектов, из введенных параметров и значений этих параметров, и путем введения различных классов таких структур;\item путем введения различных связок из вторичных объектов исследования (т.е. из связок и структур) и путем введения различных классов таких связок;\item и далее можно переходить к объектам исследования более высокого уровня сложности, к параметрам, элементами значений которых являются такие объекты, а также к структурам, элементами которых являются объекты такого уровня и, соответственно, к классам таких структур.\end{scnitemize}
}\scnrelfrom{второй домен}{класс}
\scnsuperset{\begin{scnset}
множество;отношение\\\scnsubset{множество}
;параметр\\\scnsubset{класс классов}
;значение параметра\\\scnsubset{класс}
;структура\\\scnsubset{множество}
;темпоральная сущность;темпоральная сущность базы знаний ostis-системы;семантическая окрестность;предметная область;онтология;логическая формула;действие;задача;информационная конструкция;язык;sc-конструкция;кибернетическая система;интеллектуальная компьютерная система;знание;база знаний;решатель задач интеллектуальной компьютерной системы;интерфейс интеллектуальной компьютерной системы;компьютерная система, основанная на смысловом представлении информации;смысловое представление информации;многоагентная модель решения задач, основанная на смысловом представлении информации;логико-семантическая модель интерфейсов компьютерных систем, основанных на смысловом представлении информации;решатель задач ostis-системы;действие, выполняемое ostis-системой;задача, решаемая ostis-системой:план решения задачи, реализуемый ostis-системой;протокол решения задачи, реализованный ostis-системой;метод решения класса задач, реализуемый ostis-системой;sc-агент\\\scnidtf{внутренний агент ostis-системы, осуществляющий выполнение некоторого вида действий в памяти ostis-системы}
\scnsuperset{sc-агент обработки информации в памяти ostis-системы}
\scnsuperset{sc-агент управления внешними действиями ostis-системы}
;Базовый язык программирования ostis-систем\\\scnidtf{Язык SCP}
;искусственная нейронная сеть;интерфейс ostis-системы;интерфейсное действие пользователя ostis-системы;sc-агент интерфейса ostis-системы;естественный язык;базовый интерпретатор логико-семантических моделей ostis-систем;базовый интерпретатор логико-семантических моделей ostis-систем, реализованный программно на современных компьютерах;семантический ассоциативный компьютер;обучение пользователей ostis-систем;ostis-система персональной адаптивной поддержки всех видов деятельности пользователя;ostis-система управления рецептурным производством;ostis-система, реализующая интеллектуальный портал научно-технических знаний
\end{scnset}
}
\scntext{note}{Здесь приведено семейство тех \textit{классов объектов исследования}, для которых в текущей версии \textit{Стандарта OSTIS} представлены соответствующие \textit{предметные области}. Очевидно, что это семейство должно быть существенно расширено и включить в себя, например, такие \textit{классы} сущностей, как:\begin{scnitemize}
\item материальная сущность\item вещество\item физическое поле\item персона\item пространственная сущность\item юридическое лицо\item предприятие\item географический объект\item и многие другие\end{scnitemize}
}\scntext{note}{Особого внимания требуют те \textit{классы объектов исследования}, которые носят наиболее общий характер  которым соответствуют \textit{предметные области и онтологии} \uline{высокого уровня}. Здесь важна продуманная система декомпозиции всего множества окружающих нас сущностей на иерархическую систему \textit{классов объектов исследования}, которой соответствует иерархическая система \textit{предметных областей и онтологий}, определяющая направления \uline{наследования свойств} исследуемых объектов.}\scnheader{максимальный класс объектов исследования\scnrolesign}
\scnidtf{класс объектов исследования, для которого \uline{в заданной} (!) предметной области отсутствует другой класс объектов исследования, который был бы его надмножеством\scnrolesign}
\scntext{note}{В некоторых предметных областях может быть \uline{несколько} максимальных классов объектов исследования}\scnheader{ключевой объект исследования\scnrolesign}
\scnidtf{особый объект исследования\scnrolesign}
\scnidtf{быть знаком особого исследуемого объекта в рамках заданной предметной области\scnrolesign}
\scnidtf{объект исследования, обладающий особыми свойствами\scnrolesign}
\scnhaselementrole{пример}{$\langle$Предметная область чисел; Нуль$\rangle$}
\scntext{note}{Особыми свойствами Числа \textit{Нуль} являются:\begin{scnitemize}
\item Результатом сложения Числа \textbf{\textit{Нуль}} с любым числом \textbf{\textit{x}} является число \textbf{\textit{x}};\item Результатом умножения Числа \textbf{\textit{Нуль}} на любое число является Число \textbf{\textit{Нуль}}\end{scnitemize}
}\scnhaselement{$\langle$Предметная область чисел; Единица$\rangle$}
\scnhaselement{$\langle$Предметная область чисел; Число Пи$\rangle$}
\scnhaselement{$\langle$Предметная область чисел; Число Е$\rangle$}
\scnheader{ключевой элемент предметной области\scnrolesign}
\scnidtf{входящий в состав предметной области знак ключевой сущности\scnrolesign}
\begin{scnsubdividing}
\scnitem{понятие, используемое в предметной области\scnrolesign}
\scnitem{ключевой объект исследования\scnrolesign \\\scnidtf{знак ключевого объекта исследования\scnrolesign}
}
\end{scnsubdividing}
\scnheader{понятие, используемое в предметной области\scnrolesign}
\scnidtf{понятие, используемое в заданной предметной области не в качестве одного из объектов исследования, а в качестве \uline{ключевого} понятия\scnrolesign}
\scnsubset{используемое понятие\scnrolesign}
\scnidtf{понятие, используемое в sc-знании\scnrolesign}
\scnsubset{используемое понятие*}
\scnidtf{понятие, используемое в знании, которое может быть представлено не только в SC-коде*}
\scntext{note}{Уточнение характера использования понятия в предментной области осуществляется по трем признакам:\begin{scnitemize}
\item семантический тип используемого понятия;\item полнота вхождения элементов понятия в данную предметную область;\item наличие первого упоминания понятия;\item наличие определения понятия или объявления его неопределяемостис подробным пояснением и примерами;\item наличие исследования понятия.\end{scnitemize}
}\scnrelfrom{разбиение}{семантический тип используемого понятия}
\begin{scneqtoset}
\scnitem{класс объектов исследования\scnrolesign}
\scnitem{отношение, используемое в предметной области\scnrolesign}
\scnitem{параметр, используемый в предметной области\scnrolesign}
\scnitem{класс структур, используемый в предметной области\scnrolesign}
\end{scneqtoset}
\scnrelfrom{разбиение}{полнота вхождения элементов понятия в данную предметную область}
\begin{scneqtoset}
\scnitem{используемое понятие, все элементы которого входят в данную предметную область\scnrolesign \\\scntext{note}{Для каждого используемого отношения в предметную область здесь должны входить не только знаки связок, но и все связки целиком с их компонентами}}
\scnitem{используемое понятие, не все элементы которого входят в данную предметную область\scnrolesign}
\end{scneqtoset}
\scnrelfrom{разбиение}{наличие первого упоминания понятия}
\begin{scneqtoset}
\scnitem{понятие, вводимое в данной предметной области\scnrolesign}
\scnitem{понятие, которое в данной предметной области используется, но не вводится\scnrolesign}
\end{scneqtoset}
\scntext{note}{Будем считать, что понятие вводится в данной предметной области в том и только в том случае, если ни в одной предметной области более высокого уровня это понятие не используется. Т.е. речь идет о первом упоминании этого понятия в рамках последовательности предметных областей от родительских к дочерним}\scnrelfrom{разбиение}{наличие определения понятия или объявления его неопределяемости с подробным пояснением и примерами}
\begin{scneqtoset}
\scnitem{понятие, которое в данной предметной области определено или объявлено как неопределяемое}
\scnitem{понятие, которое в данной предметной области не имеет ни определения, ни указания факта его неопределяемости}
\end{scneqtoset}
\scnrelfrom{разбиение}{наличие исследования понятия}
\begin{scneqtoset}
\scnitem{понятие, исследуемое в данной предметной области\scnrolesign}
\scnitem{понятие, которое в данной предметной области испольуется, но не исследуется\scnrolesign}
\end{scneqtoset}
\scntext{note}{Понятие, используемое в базе знаний, может быть введено (впервые упомянуто) в одной предметной области, определено в другой, а исследоваться -- в третьей}\scnheader{первичный исследуемый элемент предметной области\scnrolesign}
\scnidtf{знак первичного объекта исследования в рамках заданной предметной области\scnrolesign}
\scnheader{вторичный исследуемый элемент предметной области\scnrolesign}
\scnidtf{знак вторичного объекта исследования в рамках предметной области\scnrolesign}
\scnsuperset{связка элементов предметной области\scnrolesign}
\scnsuperset{связка первичных элементов предметной области\scnrolesign}
\scnsuperset{метасвязка элементов предметной области\scnrolesign}
\scnsuperset{метасвязка, в число компонентов которой входят связки элементов предметной области\scnrolesign}
\scnsuperset{метасвязка, в число компонентов которой входят классы элементов предметной области\scnrolesign}
\scnsuperset{метасвязка, в число компонентов которой входят структуры элементов предметной области\scnrolesign}
\scnsuperset{класс элементов предметной области\scnrolesign}
\scnsuperset{класс первичных элементов предметной области\scnrolesign}
\scnsuperset{класс связок элементов предметной области\scnrolesign}
\scnsuperset{класс классов элементов предметной области\scnrolesign}
\scnsuperset{класс структур элементов предметной области\scnrolesign}
\scnsuperset{структура элементов предметной области\scnrolesign}
\scnsuperset{тривиальная структура первичных элементов предметной области\scnrolesign}
\scnsuperset{структура, в число подмножеств которой входят связки элементов предметной области вместе со своими компонентами\scnrolesign}
\scnsuperset{структура, в число подмножеств которой входят классы элементов предметной области вместе со своими знаками\scnrolesign}
\scnsuperset{структура, в число подмножеств которой входят другие структуры вместе со своими знаками\scnrolesign}
\scnheader{неисследуемый элемент предметной области\scnrolesign}
\scnidtf{вспомогательный элемент предметной области, исследуемый в другой (смежной) предметной области\scnrolesign}
\scntext{note}{С помощью неисследуемых элементов предметной области описываются и исследуются различные вида связи между элементами, исследуемыми в данной \textit{предметной области} с элементами, исследуемыми в других \textit{предметных областях}. При этом \textit{связки}, компонентами которых являются как исследуемые, так и неисследуемые элементы данной \textit{предметной области} считаются \uline{исследуемыми} связками этой \textit{предметной области}. Примерами неисследуемых элементов, напримр, геометрической \textit{предметной области} являются \textit{числа}, являющиеся \textit{значениями величин} таких \textit{параметров}, как \textit{расстояние}\scnsupergroupsign, \textit{длина}\scnsupergroupsign, \textit{площадь}\scnsupergroupsign, \textit{объем}\scnsupergroupsign, а также различные числовые \textit{отношения} (\textit{сложение}*, \textit{умножение}*, \textit{возведение в степень}*), теоретико-множественные \textit{отношения} (\textit{включение}*, \textit{объединение}*, \textit{пересечение}*, \textit{принадлежность}*)}\newpage\scnheader{понятие}
\scnidtf{концепт}
\scnidtf{класс сущностей, который входит в состав по крайней мере одной предметной области в качестве (в роли) ключевого исследуемого понятия}
\scntext{note}{Семейство всех введенных понятий -- это, своего рода, семантическая система координат, позволяющая специфицировать всевозможные сущности в смысловом пространстве.}\scnidtf{класс сущностей, который по крайней мере в одной \textit{предметной области} объявлен как \textit{понятие} (вводимое, исследуемое или вспомогательное)}
\scntext{note}{Каждому \textit{понятию} соответствует по крайней мере одна \textit{предметная область}, в которой это понятие является \textit{исследуемым понятием} и в которой рассматриваются основные характеристики этого \textit{понятия}. Если же в какой-либо \textit{предметной области} необходимо рассмотреть дополнительные связи этого \textit{понятия} с другими \textit{понятиями}, то оно объявляется как \textit{вспомогательное понятия}\scnrolesign .}\scnidtf{Второй домен Отношения \textit{используемое понятие}*}
\scnrelto{второй домен}{используемое понятие*}
\scnidtf{класс сущностей (класс связок (в т.ч. отношение), класс классов (в т.ч. параметр), класс структур), который по крайней мере в одной \textit{предметной области} является \textit{используемым понятием}\scnrolesign}
\end{scnsubstruct}
\scnsegmentheader{Типология предметных областей и отношения, заданные на множестве предметных областей}
\begin{scnsubstruct}
\scnheader{предметная область}
\begin{scnsubdividing}
\scnitem{статическая предметная область\\\scnidtf{стационарная предметная область}
\scnidtf{\textit{предметная область}, в которой связи между сущностями, входящими в ее состав, не зависят от времени (не меняются во времени), элементами \textbf{\textit{статической предметной области}} не могут быть \textit{временные сущности}}
}
\scnitem{квазистатическая предметная область\\\scnidtf{\textit{предметная область}, решение задач в которой не требует учета темпоральных свойств объектов исследования}
}
\scnitem{динамическая предметная область\\\scnidtf{нестационарная предметная область}
\scnidtf{\textit{предметная область}, которая описывает изменение состояния (в том числе внутренней структуры) объектов исследования и/или изменение конфигурации связей между объектами исследования}
\scnidtf{\textit{предметная область}, в которой некоторые связи между сущностями, входящими в ее состав, меняются со временем (то есть носят ситуационный, нестационарный характер, другими словами, являются \textit{временными сущностями})}
}
\end{scnsubdividing}
\begin{scnsubdividing}
\scnitem{первичная предметная область\\\scnidtf{\textit{предметная область}, объектами исследования которой являются \uline{внешние} сущности (обозначаемые первичными \textit{sc-элементами})}
}
\scnitem{вторичная предметная область\\\scnidtf{метапредметная область}
\scnidtf{\textit{предметная область}, объектами исследования которой являются \textit{sc-множества} (отношения, параметры, структуры, классы структур, знания, языки и др.)}
}
\end{scnsubdividing}
\scntext{note}{Во всем многообразии предметных областей \uline{особое} местро занимают:\begin{scnitemize}
\item \textbf{\textit{Предметная область предметных областей}}, объектами исследования которой являются всевозможные предметные области, а предметом исследования являются -- всевозможные ролевые отношения, связывающие предметные области с их элементами, отношения, связывающие предметные области между собой, отношение, связывающее предметные области с их онтологиями;\item \textbf{\textit{Предметная область сущностей}}, являющаяся предметной областью самого высокого уровня и задающая базовую семантическую типологию sc-элементов (знаков, входящих в тексты SC-кода);\item Семейство \textit{предметных областей}, каждая из которых задает семантику и синтаксис некоторого \textit{sc-языка}, обеспечивающего представление \textit{\uline{онтологий}} соответствующего вида (например, теоретико множественных онтологий терминологических онтологий);\item Семейство \textit{предметных областей} \uline{верхнего уровня}, в которых классами объектов исследования являются весьма крупные классы сущностей. К таким классам, в частности, относятся: \begin{scnitemizeii}
\item класс всевозможных материальных сущностей,\item класс всевозможных множеств,\item класс всевозможных связей,\item класс всевозможных отношений,\item класс всевозможных структур,\item класс всевозможных темпоральных (нестационарных) сущностей,\item класс всевозможных действий (воздествий, акций),\item класс всевозможных параметров (характеристик),\item класс знаний всевозможного вида и т.п.;\end{scnitemizeii}
\item Предметные области абстрактных пространств (в том числе предметные области метрических пространств). Примерами абстрактного пространства являются Евклидово пространство геометрических точек и фигур, пространство всевозможных множеств, числовое пространство, SC-пространство (унифицированное смысловое пространство знаков всевозможных сущностей).\end{scnitemize}
}\scnheader{отношение, заданное на множестве предметных областей}
\scnhaselement{\scnkeyword{дочерняя предметная область*}
}
\scnidtf{частная предметная область*}
\scnidtf{быть частной предметной областью*}
\scnidtf{близлежащий потомок предметной области*}
\scnidtf{сужение предметной области по классу объектов исследования*}
\scnidtf{предметная область, детализирующая описание одного из классов объектов исследования другой (более общей) предметной области*}
\scnidtf{предметная область, объединение классов объектов исследования которой является подмножеством объединения классов объектов исследования заданной предметной области*}
\scniselement{бинарное отношение}
\scniselement{ориентированное отношение}
\scniselement{неролевое отношение}
\scnsuperset{частная предметная область по классу первичных элементов*}
\scnsuperset{частная предметная область по исследуемым отношениям*}
\scntext{explanation}{\textit{дочерняя предметная область*} -- бинарное ориентированное отношение, с помощью которого задается иерархия предметных областей путем перехода от менее детального к более детальному рассмотрению соответствующих исследуемых понятий.}\scntext{note}{Для любой \textit{предметной области} все свойства ее \textit{объектов исследования} \uline{наследуются} всеми ее \textit{дочерними предметными областями*}.}\scnhaselement{\scnkeyword{интеграция предметных областей*}
}
\scnidtf{Отношение, связывающее заданное семейство предметных областей с предметной областью, которая является результатом их интеграции (это не только теоретико-множественное объединение заданных предметных областей, но и уточнение ролей ключевых понятий в интегрированной предметной области, поскольку одно и то же понятие в интегрируемых предметных областях может иметь разные роли).}
\scnhaselement{\scnkeyword{изоморфность предметных областей*}
}
\scnhaselement{\scnkeyword{гомоморфность предметных областей*}
}
\scnheader{расширение семейства исследуемых отношений*}
\scntext{explanation}{Переход от одной предметной области к предметной области с тем же максимальным классомобъектов исследования, но с расширенным семейством отношений и, возможно, с расширенным семейством явно выделенных классов объектов исследования (подклассов максимального класса).}\scnheader{переход к рассмотрению внутренней структуры объектов исследования*}
\scntext{explanation}{Переход от рассмотрения внешних связей объектов исследования к рассмотрению их внутренней структуры путем декомпозиции исследуемых объектов на части и путем включения в число исследуемых объектов тех, которые являются указанными частями.}\scnheader{переход к рассмотрению структур из объектов исследования*}
\scntext{explanation}{Переход от описаниязаданного класса исследуемых объектов к описанию класса всевозможных множеств, элементами которых являются указанные объекты (например, переход от предметной области геометрических точек к предметной области геометрических фигур).}
\end{scnsubstruct}
\end{SCn}


\scsection{Предметная область и онтология онтологий}
\label{sec:sd_ontologies}
\begin{SCn}

\scnsectionheader{\currentname}

\scnstartsubstruct

\scnheader{Предметная область онтологий}
\scnidtf{Предметная область теории онтологий}
\scnidtf{Предметная область, объектами исследования которой являются онтологии}
\scniselement{предметная область}
\scnsdmainclasssingle{онтология}
\scnsdclass{интегрированная онтология;структурная спецификация;теоретико-множественная онтология;логическая онтология;логическая иерархия понятий;логическая иерархия высказываний;терминологическая онтология;онтология задач и решений задач;онтология классов задач и способов решения задач}
\scnsdrelation{онтология*;используемые константы*;используемые утверждения*}

\scnheader{онтология}
\scnidtf{система понятий соответствующей предметной области}
\scnidtf{концептуальный каркас (скелет) описания некоторой предметной области}
\scnidtf{концептуальная (семантическая) основа различных языков, обеспечивающих описание объектов исследования, принадлежащих заданной предметной области}
\scnidtf{семантический интерфейс для интеграции знаний по заданной предметной области и для согласованного понимания различными субъектами этих знаний}
\scnidtf{онтология соответствующей предметной области}
\scnidtf{описание концептов и отношений заданной предметной области}
\scnrelto{включение}{знание}
\scnsubdividing{интегрированная онтология;структурная спецификация;теоретико-множественная онтология;логическая иерархия понятий;логическая онтология;логическая иерархия высказываний;терминологическая онтология;онтология задач и решений задач;онтология классов задач и способов решения задач}
\scnexplanation{\textbf{\textit{онтология}} — это вид знаний, каждое из которых является спецификацией (описанием свойств) соответствующей \textit{предметной области}, ориентированной на описание свойств и взаимосвязей понятий, входящих в состав указанной \textit{предметной области}.}

\scnheader{онтология*}
\scnidtf{sc-онтология*}
\scnidtf{быть онтологией предметной области*}
\scnidtf{sc-онтология, специфицирующая заданную предметную область*}
\scnrelfrom{первый домен}{предметная область}
\scnrelfrom{второй домен}{онтология}
\scnexplanation{\textbf{\textit{онтология*}} — это бинарное отношение, связывающее некоторую предметную область с ее онтологией (спецификацией).}

\scnheader{интегрированная онтология}
\scnexplanation{\textbf{\textit{интегрированная онтология}} — это \textit{онтология}, объединяющая все \textit{онтологии} различного вида некоторой \textit{предметной области}.}

\scnheader{структурная спецификация}
\scnexplanation{\textbf{\textit{структурная спецификация}} — это \textit{онтология}, в которой описываются роли понятий, входящих в состав \textit{предметной области}, а также связи специфицируемых \textit{предметных областей} с другими \textit{предметными областями}.}

\scnheader{теоретико-множественная онтология}
\scnexplanation{\textbf{\textit{теоретико-множественная онтология}} — это \textit{онтология}, описывающая теоретико-множественные связи между понятиями заданной \textit{предметной области} (включение, разбиение, объединение, пересечение, разность множеств, область определения, домен, функция).}

\scnheader{логическая онтология}
\scnexplanation{\textbf{\textit{логическая онтология}} — это \textit{онтология}, описание системы высказываний заданной \textit{предметной области}.}

\scnheader{логическая иерархия понятий}
\scnidtf{логическая иерархия понятий, основанная на их определениях}
\scnexplanation{\textbf{\textit{логическая иерархия понятий}} — это \textit{онтология}, являющаяся надстройкой над \textit{логической онтологией}, включающая описание системы определений понятий заданной \textit{предметной области} с указанием набора понятий, через которые определяется каждое определяемое понятие рассматриваемой \textit{предметной области}.}

\scnheader{используемые константы*}
\scniselement{квазибинарное отношение}
\scnrelfrom{второй домен}{понятие}
\scnexplanation{\textbf{\textit{используемые константы*}} — это \textit{отношение}, связывающее некоторое \textit{определение} со множеством понятий, на основании которых определяется соответствующее данному \textit{определению} понятие в рамках рассматриваемой \textit{предметной области}.}

\scnheader{логическая иерархия высказываний}
\scnidtf{логическая система доказательств}
\scnidtf{логическая иерархия утверждений}
\scnidtf{логическая иерархия высказываний, основанная их на базовых доказательствах}
\scnexplanation{\textbf{\textit{логическая иерархия высказываний}} — это \textit{онтология}, являющаяся надстройкой над \textit{логической онтологией} и включающая описание системы утверждений рассматриваемой \textit{предметной области} с указанием набора \textit{утверждений}, через которые доказывается каждое \textit{утверждение}.}

\scnheader{используемые утверждения*}
\scniselement{квазибинарное отношение}
\scnrelfrom{второй домен}{утверждение}
\scnexplanation{\textbf{\textit{используемые утверждения*}} — это \textit{отношение}, связывающее утверждение со множеством утверждений, на основании которых оно доказывается в рамках рассматриваемой \textit{предметной области}.}

\scnheader{терминологическая онтология}
\scnexplanation{\textbf{\textit{терминологическая онтология}} — это \textit{онтология}, описывающая систему основных и неосновных терминов (имен, внешних обозначений), соответствующих концептам и отношениям заданной \textit{предметной области}, а также описание правил построения терминов для сущностей, являющихся элементами (экземплярами) указанных концептов и \textit{отношений}.}

\scnheader{онтология задач и решений задач}
\scnexplanation{\textbf{\textit{онтология задач и решений задач}} — это \textit{онтология}, описывающая задачи и их классы, решаемые в рассматриваемой \textit{предметной области}.}

\scnheader{онтология классов задач и способов решения задач}
\scnexplanation{\textbf{\textit{онтология классов задач и способов решения задач}} — это \textit{онтология}, описывающая способы решения задач и их классов в рамках \textit{предметной области}. Является \textit{метазнанием*} по отношению к \textit{онтологии задач и классов задач}.}

\scnendstruct \scnendcurrentsectioncomment

\end{SCn}

\scsection{Предметная область и онтология логических формул, высказываний и логических sc-языков}
\label{sec:sd_logics}
\begin{SCn}

\scnsectionheader{\currentname}

\scnstartsubstruct

\scnheader{Предметная область логических формул, высказываний и формальных теорий}
\scniselement{предметная область}
\scnsdmainclasssingle{формальная теория}
\scnsdclass{высказывание;атомарное высказывание;неатомарное высказывание;фактографическое высказывание;логическая формула;атомарная логическая формула;неатомарная логическая формула;утверждение;определение;общезначимая логическая формула;противоречивая логическая формула;нейтральная логическая формула;выполнимая логическая формула;невыполнимая логическая формула;тавтология;квантор;формула существования;число значений переменной;кратность существования;единственное существование;логическая формула и единственность;открытая логическая формула;замкнутая логическая формула}
\scnsdrelation{предметная область\scnrolesign;аксиома\scnrolesign;теорема\scnrolesign;подформула*;логическая связка*;импликация*;если\scnrolesign;то\scnrolesign;эквиваленция*;конъюнкция*;дизъюнкция*;строгая дизъюнкция*;отрицание*;всеобщность*;неатомарное существование*;связываемые переменные\scnrolesign}

\scnheader{формальная теория}
\scnexplanation{\textbf{\textit{формальная теория}} — это множество высказываний, которые считаются истинными в рамках данной \textbf{\textit{формальной теории}}. Высказывания могут быть как фактографическими, так и логическими формулами. Некоторые высказывания считаются аксиомами, а другие доказываются на основе других высказываний в рамках этой же \textbf{\textit{формальной теории}}.

Каждая формальная теория интерпретируется (т.е. ее высказывания являются истинными) на какой-либо \textit{предметной области}, которая является максимальным из \textit{фактографических высказываний} (их \textit{объединением*}),  входящих в состав этой \textbf{\textit{формальной теории}}. Каждой \textbf{\textit{формальной теории}} соответствует одна \textit{предметная область}, которая входит в нее под атрибутом \textit{предметная область\scnrolesign}.

Каждая \textbf{\textit{формальная теория}} может рассматриваться как \textit{конъюнктивное высказывание}, априори истинное (с чьей-то точки зрения) при интерпретации на соответствующей \textit{предметной области}.}

\scnheader{предметная область\scnrolesign}
\scniselement{ролевое отношение}
\scnexplanation{\textbf{\textit{предметная область\scnrolesign}} -- это \textit{ролевое отношение}, связывающее \textit{формальную теорию} с \textit{предметной областью}, на которой данная \textit{формальная теория} интерпретируется (в рамках которой истинны \textit{высказывания}, входящие в состав этой \textit{формальной теории}). Другими словами, эта \textit{предметная область} является максимальным фактографическим высказыванием этой \textit{формальной теории}.}

\scnheader{аксиома\scnrolesign}
\scniselement{ролевое отношение}
\scnexplanation{\textbf{\textit{аксиома\scnrolesign}} -- это \textit{ролевое отношение}, связывающее \textit{формальную теорию} с \textit{высказыванием}, истинность которого не  требует доказательства в рамках этой \textit{формальной теории}.}

\scnheader{теорема\scnrolesign}
\scniselement{ролевое отношение}
\scnexplanation{\textbf{\textit{теорема\scnrolesign}} -- это \textit{ролевое отношение}, связывающее \textit{формальную теорию} с \textit{высказыванием}, истинность которого доказывается в рамках этой \textit{формальной теории}.}

\scnheader{высказывание}
\scnsubdividing{атомарное высказывание;неатомарное высказывание}
\scnsubdividing{фактографическое высказывание;логическая формула}
\scnexplanation{Под \textbf{\textit{высказыванием}} понимается некоторая \textit{структура} (в которую входят \textit{sc-константы} из некоторой предметной области и/или \textit{sc-переменные}) или \textit{логическая связка}, которая может трактоваться как истинная или ложная в рамках какой-либо \textit{предметной области}.}
\scnnote{Истинность \textbf{\textit{высказывания}} задается путем указания принадлежности знака этого высказывания \textit{формальной теории}, соответствующей данной \textit{предметной области}. Ложность высказывания задается путем указания принадлежности знака \textit{отрицания*} этого высказывания данной \textit{формальной теории}. Явно указанная непринадлежность \textbf{\textit{высказывания}} \textit{формальной теории} может говорить как о его ложности в рамках данной теории (если это указано рассмотренным выше образом), так и о том, что данное \textbf{\textit{высказывание}} вообще не рассматривается в данной \textit{формальной теории} (например, использует понятия, не принадлежащие данной \textit{предметной области}). 
Одно и то же \textbf{\textit{высказывание}} может быть истинно в рамках одной \textit{формальной теории} и ложно в рамках другой.}
\scnnote{Каждое высказывание может либо содержать только \textit{sc-элементы}, которые не являются знаками других \textbf{\textit{высказываний}} (быть атомарным), либо содержать знаки других \textbf{\textit{высказываний}} (быть неатомарным).}

\scnheader{высказывание формальной теории\scnrolesign}
\scniselement{неосновное понятие}
\scnsubdividing{истинное высказывание\scnrolesign\\
	\scnaddlevel{1}
		\scnidtf{высказывание, истинное в рамках данной формальной теории\scnrolesign}
		\scnidtf{высказывание, знак которого принадлежит данной формальной теории\scnrolesign}
	\scnaddlevel{-1}
	;ложное высказывание\scnrolesign\\
	\scnaddlevel{1}
		\scnidtf{высказывание, ложное в рамках данной формальной теории\scnrolesign}
		\scnidtf{высказывание, знак отрицания которого принадлежит данной формальной теории\scnrolesign}
	\scnaddlevel{-1}
	;нечеткое высказывание\scnrolesign\\
	\scnaddlevel{1}
		\scnidtf{гипотетическое высказывание\scnrolesign}
		\scnidtf{высказывание, возможно истинное или ложное в рамках данной формальной теории\scnrolesign}
		\scnidtf{высказывание, истинное или ложное в рамках данной формальной теории с некоторой вероятностью\scnrolesign}
	\scnaddlevel{-1}
	;бессмысленное высказывание\scnrolesign\\
	\scnaddlevel{1}
		\scnidtf{высказывание, бессмысленное в рамках данной формальной теории\scnrolesign}
		\scnidtf{высказывание, не рассматриваемое в рамках данной формальной теории\scnrolesign}
		\scnexplanation{Высказывание является бессмысленным в рамках заданной формальной теории, если в какое-либо \textit{атомарное высказывание} в его составе (или в само это высказывание, если оно является атомарным) входит какая-либо \textit{sc-константа}, не являющаяся элементом предметной области, описываемой указанной \textit{формальной теорией}.}
	\scnaddlevel{-1}}

\scnheader{атомарное высказывание}
\scnsubset{структура}
\scnsubdividing{атомарное фактографическое высказывание;атомарная логическая формула}
\scndefinition{\textbf{\textit{атомарное высказывание}} -- это \textit{высказывание}, которое содержит хотя бы один \textit{sc-элемент}, не являющийся знаком другого \textit{высказывания}.}
\scnheader{неатомарное высказывание}
\scndefinition{\textbf{\textit{неатомарное высказывание}} -- это \textit{высказывание}, в состав которого входят только знаки других \textit{высказываний}.}
\scnnote{Следует отметить, что мы не можем говорить об истинности либо ложности \textbf{\textit{неатомарного высказывания}} в рамках какой-либо \textit{формальной теории}, в случае, когда невозможно установить истинность либо ложность любого из его элементов в рамках этой же \textit{формальной теории}.}

\scnheader{фактографическое высказывание}
\scnsuperset{атомарное фактографическое высказывание}
\scnexplanation{Под \textit{фактографическим высказыванием} понимается:
\begin{scnitemize}
    \item \textit{атомарное высказывание}, в состав которого не входит ни одна \textit{sc-переменная};
    \item \textit{неатомарное высказывание}, все элементы которого также являются \textbf{\textit{фактографическими высказываниями}}.
\end{scnitemize}
}

\scnheader{логическая формула}
\scnexplanation{Под \textit{логической формулой} понимается:
\begin{scnitemize}
    \item \textit{атомарное высказывание}, в состав которого входит хотя бы одна \textit{sc-переменная};
    \item \textit{неатомарное высказывание}, хотя бы один элемент которого является \textbf{\textit{логической формулой}}.
\end{scnitemize}}
\scnsubdividing{атомарная логическая формула;неатомарная логическая формула}
\scnsubdividing{открытая логическая формула;замкнутая логическая формула}

\scnheader{атомарная логическая формула}
\scnidtf{обобщенная структура}
\scnidtf{атомарная формула существования}
\scnexplanation{Под \textbf{\textit{атомарной логической формулой}} понимается \textit{атомарное высказывание}, которое является \textit{логической формулой}.

По умолчанию \textbf{\textit{атомарная логическая формула}} трактуется как \textit{высказывание} о существовании, то есть наличия в памяти значений, соответствующих всем \textit{sc-переменным}, входящим в состав данной формулы и не попадающих под действие какого-либо другого \textit{квантора} (указанного явно или по умолчанию). Таким образом, на все \textit{sc-переменные}, входящие в состав \textbf{\textit{атомарной логической формулы}} и не попадающие под действие какого-либо другого \textit{квантора}, неявно накладывается квантор \textit{существования*}.}

\scnheader{неатомарная логическая формула}
\scnsubdividing{общезначимая логическая формула;противоречивая логическая формула;нейтральная логическая формула}
\scnsubdividing{выполнимая логическая формула;невыполнимая логическая формула}
\scnsuperset{тавтология}
\scnexplanation{Под \textbf{\textit{неатомарной логической формулой}} понимается \textit{неатомарное высказывание}, которое является \textit{логической формулой}.

Для того, чтобы рассмотреть типологию \textbf{\textit{неатомарных логических формул}}, будем говорить, что исследуется истинность самой \textbf{\textit{неатомарной логической формулы}} и всех ее \textit{подформул*} в рамках одной и той же \textit{формальной теории}, при этом не важно, какой именно. Также считается, что в рассматриваемой \textit{формальной теории} каждая \textit{подформула*} рассматриваемой \textbf{\textit{неатомарной логической формулы}} в рамках этой \textit{формальной теории} может однозначно трактоваться как либо истинная, либо ложная. В противном случае мы не можем говорить об истинности либо ложности исходной \textbf{\textit{неатомарной логической формулы}} в рамках этой \textit{формальной теории}.}

\scnheader{подформула*}
\scnidtf{частная формула*}
\scniselement{бинарное отношение}
\scniselement{ориентированное отношение}
\scniselement{транзитивное отношение}
\scndefinition{Будем называть \textbf{\textit{подформулой*}} \textit{неатомарной логической формулы} \textbf{\textit{fi}} любую \textit{логическую формулу} \textbf{\textit{fj}}, являющуюся элементом исходной формулы \textbf{\textit{fi}}, а также любую \textbf{\textit{подформулу*}} формулы \textbf{\textit{fj}}.}
\scnrelfrom{описание примера}{
\scnfilescg{figures/sd_logical_formulas/subformula.png}}

\scnheader{утверждение}
\scnidtf{текст логической формулы}
\scndefinition{\textbf{\textit{утверждение}} -- это \textit{семантическая окрестность} некоторой \textit{логической формулы}, в которую входит полный текст этой \textit{логической формулы}, а также факт принадлежности этой \textit{логической формулы} некоторой \textit{формальной теории}.}
\scnexplanation{Знак \textit{логической формулы}, семантическая окрестность которой представляет собой утверждение, является \textit{главным ключевым sc-элементом\scnrolesign} в рамках этого \textbf{\textit{утверждения}}. Знаки понятий соответствующей \textit{предметной области}, которые входят в состав какой-либо \textit{подформулы*} указанной \textit{логической формулы}, будут \textit{ключевыми sc-элементами\scnrolesign} в рамках этого \textbf{\textit{утверждения}}.

Полный текст некоторой \textit{логической формулы} включает в себя:
\begin{scnitemize}
    \item знак самой этой \textit{логической формулы};
    \item знаки всех ее \textit{подформул*};
    \item элементы всех \textit{логических формул}, знаки которых попали в данную структуру;
    \item все пары принадлежности, связывающие \textit{логические формулы}, знаки которых попали в данную структуру, с их компонентами.
\end{scnitemize}
Таким образом, факт принадлежности (истинности) логической формулы нескольким \textit{формальным теориям} будет порождать новое утверждение для каждой такой \textit{формальной теории}. Текст \textbf{\textit{утверждения}} входит в состав \textit{логической онтологии}, соответствующей \textit{предметной области}, на которой интерпретируется \textit{главный ключевой sc-элемент\scnrolesign} данного утверждения.}
\scntext{правило идентификации экземпляров}{\textbf{\textit{утверждения}} в рамках \textit{Русского языка} именуются по следующим правилам:
\begin{scnitemize}
    \item в начале идентификатора пишется сокращение \textbf{Утв.};
    \item далее в круглых скобках через точку с запятой перечисляются основные идентификаторы \textit{ключевых \mbox{sc-элементов}\scnrolesign} данного \textbf{\textit{утверждения}}. Порядок определяется в каждом конкретном случае в зависимости от того, свойства каких из этих \textit{понятий} описывает данное \textbf{\textit{утверждение}} в большей или меньшей степени.
\end{scnitemize}
}
\scnaddlevel{1}
\scntext{описание примера}{\textit{Утв. (треугольник; сторона*)}}
\scnnote{Могут быть исключения для \textbf{\textit{утверждений}}, названия которых закрепились исторически, например, \textit{Теорема Пифагора}, \textit{Аксиома о прямой и точке}.}
\scnaddlevel{-1}

\scnheader{определение}
\scnidtf{текст определения}
\scnsubset{утверждение}
\scndefinition{\textbf{\textit{определение}} -- это \textit{утверждение}, \textit{главным ключевым sc-элементом\scnrolesign} которого является связка \textit{эквиваленции*}, однозначно определяющая некоторое понятие на основе других понятий.}
\scnnote{Каждое определение имеет ровно один \textit{ключевой sc-элемент\scnrolesign} (не считая \textit{главного ключевого sc-элемента\scnrolesign}).}
\scnnote{Для одного и того же понятия в рамках одной \textit{формальной теории} может существовать несколько \textit{утверждений об эквиваленции*}, однозначно задающих некоторое понятие на основе других, однако только одно такое \textit{утверждение} в рамках этой \textit{формальной теории} может быть отмечено как \textbf{\textit{определение}}. Остальные \textit{утверждения об эквиваленции*} могут трактоваться как \textit{пояснения} данного понятия.}
\scntext{правило идентификации экземпляров}{\textbf{\textit{определения}} в рамках \textit{Русского языка} именуются по следующим правилам:
\begin{scnitemize}
    \item в начале идентификатора пишется сокращение \textbf{Опр.};
    \item далее в круглых скобках через точку с запятой записывается основной идентификатор  \textit{ключевого sc-элемента\scnrolesign} данного \textbf{\textit{определения}}.
\end{scnitemize}
}
\scnaddlevel{1}
\scntext{описание примера}{\textit{Опр. (связный граф)}}
\scnaddlevel{-1}

\scnheader{общезначимая логическая формула}
\scnsubset{выполнимая логическая формула}
\scnsubset{тавтология}
\scndefinition{\textbf{\textit{общезначимая логическая формула}} -- это \textit{логическая формула}, для которой не существует \textit{формальной теории}, в рамках которой она была бы ложной с учетом истинности и ложности всех ее \textit{подформул*} в рамках этой же \textit{формальной теории}.}

\scnheader{противоречивая логическая формула}
\scnsubset{невыполнимая логическая формула}
\scnsubset{тавтология}
\scndefinition{\textbf{\textit{противоречивая логическая формула}} -- это \textit{логическая формула}, для которой не существует \textit{формальной теории}, в рамках которой она была бы истинной с учетом истинности и ложности всех ее \textit{подформул*} в рамках этой же \textit{формальной теории}.}

\scnheader{нейтральная логическая формула}
\scnsubset{выполнимая логическая формула}
\scndefinition{\textbf{\textit{нейтральная логическая формула}} -- это \textit{логическая формула}, для которой существует хотя бы одна \textit{формальная теория}, в рамках которой эта формула ложна, и хотя бы одна \textit{формальная теория}, в рамках которой эта формула истинна.}

\scnheader{выполнимая логическая формула}
\scndefinition{\textbf{\textit{выполнимая логическая формула}} -- это \textit{логическая формула}, для которой существует хотя бы одна \textit{формальная теория}, в рамках которой эта формула истинна.}

\scnheader{невыполнимая логическая формула}
\scndefinition{\textbf{\textit{невыполнимая логическая формула}} -- это \textit{логическая формула}, для которой существует хотя бы одна \textit{формальная теория}, в рамках которой эта формула ложна.}

\scnheader{тавтология}
\scnidtf{тождественно истинная формула}
\scndefinition{\textbf{\textit{тавтология}} -- это \textit{логическая формула}, которая является либо только истинной, либо только ложной в рамках всех \textit{формальных теорий}, в которых можно установить ее истинность или ложность.}
\scnexplanation{\textbf{\textit{тавтология}} -- это такая \textit{логическая формула}, которая является либо \textit{общезначимой логической формулой}, либо \textit{противоречивой логической формулой}.}

\scnheader{логическая связка*}
\scnidtf{неатомарная логическая формула}
\scnidtf{логический оператор*}
\scnidtf{пропозициональная связка*}
\scniselement{класс связок разной мощности}
\scnrelto{семейство подмножеств}{неатомарное высказывание}
\scndefinition{\textbf{\textit{логическая связка*}} -- это отношение (класс связок), связками которого являются \textit{высказывания}.}
\scnexplanation{\textbf{\textit{логическая связка*}} -- это \textit{отношение}, областью определения которого является множество \textit{высказываний}, при этом само это отношение и некоторые его подмножества могут быть \textit{классами связок разной мощности}.}

\scnheader{импликация*}
\scnidtf{логическое следование*}
\scnsubset{логическая связка*}
\scniselement{бинарное отношение}
\scniselement{ориентированное отношение}
\scndefinition{\textbf{\textit{импликация*}} -- это множество импликативных \textit{неатомарных высказываний}, каждое из которых состоит из посылки (первый компонент \textit{высказывания}) и следствия (второй компонент \textit{высказывания}). Каждое импликативное \textit{высказывание} ложно в рамках некоторой \textit{формальной теории} в том случае, когда его посылка истинна, а заключение ложно в рамках этой же \textit{формальной теории}. В других случаях такое \textit{высказывание} истинно.}
\scnnote{По умолчанию на все переменные, входящие в обе части высказывания об \textbf{\textit{имликации*}} (или хотя бы одну из \textit{подформул*} каждой части) неявно накладывается квантор \textit{всеобщности*}, при условии, что эти переменные не связаны другим \textit{квантором}, указанным явно.}
\scnrelfrom{описание примера}{
\scnfilescg{figures/sd_logical_formulas/implication.png}}

\scnheader{если\scnrolesign}
\scnsubset{1\scnrolesign}
\scniselement{ролевое отношение}
\scndefinition{\textbf{\textit{если\scnrolesign}} -- это \textit{ролевое отношение}, используемое в связках \textit{импликации*} для указания посылки.}

\scnheader{то\scnrolesign}
\scnsubset{2\scnrolesign}
\scniselement{ролевое отношение}
\scndefinition{\textbf{\textit{то\scnrolesign}} -- это \textit{ролевое отношение}, используемое в связках \textit{импликации*} для указания следствия.}

\scnheader{эквиваленция*}
\scnidtf{эквивалентность*}
\scnsubset{логическая связка*}
\scniselement{бинарное отношение}
\scniselement{неориентированное отношение}
\scndefinition{\textbf{\textit{эквиваленция*}} -- это множество \textit{высказываний} об эквивалентности, каждое из которых истинно в рамках некоторой \textit{формальной теории} только в тех случаях, когда оба его компонента одновременно либо истинны в рамках этой же \textit{формальной теории}, либо ложны.}
\scnnote{По умолчанию на все переменные, входящие в обе части высказывания об \textbf{\textit{эквиваленции*}} (или хотя бы одну из \textit{подформул*} каждой части) неявно накладывается квантор \textit{всеобщности*}, при условии, что эти переменные не связаны другим \textit{квантором}, указанным явно.}
\scnrelfrom{описание примера}{
\scnfilescg{figures/sd_logical_formulas/equivalent.png}}

\scnheader{конъюнкция*}
\scnidtf{логическое и*}
\scnidtf{логическое умножение*}
\scnsubset{логическая связка*}
\scniselement{неориентированное отношение}
\scniselement{класс связок разной мощности}
\scndefinition{\textbf{\textit{конъюнкция*}} -- это множество конъюнктивных \textit{высказываний}, каждое из которых истинно в рамках некоторой \textit{формальной теории} только в том случае, когда все его компоненты истинны в рамках этой же \textit{формальной теории}.}
\scnrelfrom{описание примера}{
\scnfilescg{figures/sd_logical_formulas/conjunction.png}}

\scnheader{дизъюнкция*}
\scnidtf{логическое или*}
\scnidtf{логическое сложение*}
\scnidtf{включающее или*}
\scnsubset{логическая связка*}
\scniselement{неориентированное отношение}
\scniselement{класс связок разной мощности}
\scndefinition{\textbf{\textit{дизъюнкция*}} -- это множество дизъюнктивных \textit{высказываний}, каждое из которых истинно в рамках некоторой \textit{формальной теории} только в том случае, когда хотя бы один его компонент является истинным в рамках этой же \textit{формальной теории}.}
\scnrelfrom{описание примера}{
\scnfilescg{figures/sd_logical_formulas/disjunction.png}}

\scnheader{строгая дизъюнкция*}
\scnidtf{сложение по модулю 2*}
\scnidtf{исключающее или*}
\scnidtf{альтернатива*}
\scnsubset{логическая связка*}
\scniselement{неориентированное отношение}
\scniselement{класс связок разной мощности}
\scndefinition{\textbf{\textit{строгая дизъюнкция*}} -- это множество строго дизъюнктивных \textit{высказываний}, каждое из которых истинно в рамках некоторой \textit{формальной теории} только в том случае, когда ровно один его компонент является истинным в рамках этой же \textit{формальной теории}.}
\scnrelfrom{описание примера}{
\scnfilescg{figures/sd_logical_formulas/strictDisjunction.png}}

\scnheader{отрицание*}
\scnsubset{логическая связка*}
\scnsubset{синглетон}
\scndefinition{\textbf{\textit{отрицание*}} -- это множество \textit{высказываний} об отрицании, каждое из которых истинно в рамках некоторой \textit{формальной теории} только в том случае, когда его единственный элемент является ложным в рамках этой же \textit{формальной теории}.}
\scnrelfrom{описание примера}{
\scnfilescg{figures/sd_logical_formulas/negation.png}}

\scnheader{квантор}
\scnsubset{логическая связка*}
\scndefinition{\textbf{\textit{квантор}} -- это \textit{отношение}, каждая связка которой задает истинность множества \textit{логических формул}, входящих в ее состав, при выполнении дополнительных условий, связанных с некоторыми из переменных, входящих в состав этих \textit{логических формул}.}
\scnnote{Будем говорить, что переменные связаны \textbf{\textit{квантором}} или попадают под область действия данного \textbf{\textit{квантора}} (имея в виду конкретную связку конкретного \textbf{\textit{квантора}}). Таким образом, в состав каждой связки каждого \textbf{\textit{квантора}} входит \textit{атомарная формула}, являющаяся \textit{тривиальной структурой}, в которой перечислены переменные, связанные данным \textbf{\textit{квантором}}.}

\scnheader{всеобщность*}
\scnidtf{квантор всеобщности*}
\scnidtf{квантор общности*}
\scniselement{квантор}
\scniselement{ориентированное отношение}
\scniselement{класс связок разной мощности}
\scndefinition{\textbf{\textit{всеобщность}} -- это \textit{квантор}, для каждой связки которого, истинной в рамках некоторой \textit{формальной теории}, выполняется следующее утверждение: все формулы, входящие в состав этой связки истинны в рамках этой же \textit{формальной теории} при всех (любых) возможных значениях всех элементов множества \textit{связываемых переменных\scnrolesign} входящего в эту связку.}
\scnnote{Каждая связка \textit{квантора} \textbf{\textit{всеобщность*}} может быть представлена как \textit{конъюнкция*} (потенциально бесконечная) исходных \textit{логических формул}, входящих в состав этой связки, в каждой из которых все \textit{связанные переменные\scnrolesign} заменены на их возможные значения.}
\scnnote{Квантор \textbf{\textit{всеобщности*}} зачастую обозначается "$\forall$" \ и читается как "для всех"{}, "для каждого"{}, "для любого"{} или "все"{}, "каждый"{}, "любой".}
\scnrelfrom{описание примера}{
\scnfilescg{figures/sd_logical_formulas/universality.png}}

\scnheader{формула существования}
\scnidtf{существование*}
\scnsubdividing{атомарная логическая формула;неатомарное существование*}

\scnheader{неатомарное существование*}
\scnidtf{квантор неатомарного существования*}
\scniselement{квантор}
\scniselement{ориентированное отношение}
\scniselement{класс связок разной мощности}
\scndefinition{\textbf{\textit{неатомарное существование*}} -- это \textit{квантор}, для каждой связки которого, истинной в рамках некоторой \textit{формальной теории}, выполняется следующее утверждение: существуют значения всех элементов множества \textit{связываемых переменных\scnrolesign} входящего в эту связку, такие, что все формулы, входящие в состав этой связки истинны в рамках этой же \textit{формальной теории}.}
\scnnote{Каждая связка \textit{квантора} \textbf{\textit{неатомарное существование*}} может быть представлена как \textit{дизъюнкция*} (потенциально бесконечная) исходных \textit{логических формул}, входящих в состав этой связки, в каждой из которых все \textit{связанные переменные\scnrolesign} заменены на их возможные значения.}
\scnnote{квантор \textbf{\textit{существования*}} зачастую обозначается "$\exists$" \ и читается как "существует"{}, "для некоторого"{}, "найдется".}
\scnrelfrom{описание примера}{
\scnfilescg{figures/sd_logical_formulas/non_atomicExistence.png}}

\scnheader{число значений переменной}
\scniselement{параметр}
\scnexplanation{Каждый элемент \textit{параметра} \textbf{\textit{число значений переменной}} представляет собой класс ориентированных пар, первым компонентом которых является знак \textit{логической формулы}, вторым -- \textit{sc-переменная}, имеющая в рамках данной \textit{логической формулы} ограниченное известное число значений, при которых данная формула является истинной в рамках соответствующей \textit{формальной теории}.\\
Отметим, что в случае \textit{атомарной логической формулы} каждая такая связка связывает знак формулы и знак принадлежащей ей \textit{sc-переменной}, т.е. является, по сути, частным случаем пары принадлежности. В случае \textit{неатомарной логической формулы} указанная \textit{sc-переменная} может принадлежать любой из \textit{подформул*} исходной формулы.

Таким образом, \textit{измерением*} каждого значения параметра \textbf{\textit{число значений переменной}} является некоторое \textit{число}, задающее количество значений \textit{sc-переменных} в рамках \textit{логической формулы}.}

\scnheader{кратность существования}
\scniselement{параметр}
\scnrelfrom{область определения параметра}{формула существования}
\scnhaselement{единственное существование}
\scnexplanation{Каждый элемент \textit{параметра} \textbf{\textit{кратность существования}} представляет собой класс логических \textit{формул существования}, для которых  при интерпретации на соответствующей \textit{предметной области} существует ограниченное общее для всех таких формул число комбинаций значений переменных, при которых указанные формулы являются истинными в рамках соответствующей \textit{формальной теории}.

Таким образом, \textit{измерением*} каждого значения \textbf{\textit{кратности существования}} является некоторое \textit{число}, задающее количество таких комбинаций.}

\scnheader{единственное существование}
\scnidtf{однократное существование}
\scnidtf{формула существования и единственности}
\scnnote{\textbf{\textit{единственное существование}} зачастую обозначается "$\exists!$" \ и читается как "существует и единственный".}

\scnheader{логическая формула и единственность}
\scnsubset{логическая формула}
\scnsubset{единственное существование}
\scnexplanation{Каждый элемент множества \textbf{\textit{логическая формула и единственность}} представляет собой \textit{логическую формулу} (\textit{атомарную} или \textit{неатомарную}), для которой дополнительно уточняется, что при ее интерпретации на некоторой предметной области существует только один набор значений переменных, входящих в эту формулу (или ее \textit{подформулы*}), при котором указанная логическая формула истинна в рамках \textit{формальной теории}, в которую входит данная \textit{предметная область}.}

\scnheader{связываемые переменные\scnrolesign}
\scniselement{ролевое отношение}
\scndefinition{\textbf{\textit{связываемые переменные\scnrolesign}} -- это \textit{ролевое отношение}, которое связывает связку конкретного \textit{квантора} с множеством переменных, которые связаны этим квантором.}
%\scnrelfrom{описание примера}{
%\scnfilescg{figures/sd_logical_formulas/bindVariables.png}}

\scnheader{открытая логическая формула}
\scndefinition{\textbf{\textit{открытая логическая формула}} -- это \textit{логическая формула}, в рамках которой (и всех ее \textit{подформул*}) существует хотя бы одна переменная, не связанная никаким \textit{квантором}.}

\scnheader{замкнутая логическая формула}
\scndefinition{\textbf{\textit{замкнутая логическая формула}} -- это \textit{логическая формула}, в рамках которой (и всех ее \textit{подформул*}) не существует переменных, не связанных каким-либо \textit{квантором}.}

\scnheader{Примеры логических утверждений}
\scneqtoset{\scgfileitem{figures/sd_logical_formulas/ex_set_union.png};\scgfileitem{figures/sd_logical_formulas/ex_set_diff.png}}

\bigskip
\scnendstruct \scnendcurrentsectioncomment

\end{SCn}

\scsection{Предметная область и онтология файлов, внешних информационных конструкций и внешних языков ostis-систем}
\label{sd_file_internal_inform_struct}


\scsubsection{Предметная область и онтология библиографических источников}
\label{sd_bibliography}

\scsubsection{Предметная область и онтология естественных языков}
\label{sd_natural_lang}

\scsubsubsection{Предметная область и онтология синтаксиса естественных языков}
\label{sd_syntax_natural_lang}

\scsubsubsection{Предметная область и онтология денотационной семантики естественных языков}
\label{sd_sem_natural_lang}

\scsection{Глобальная предметная область воздействий и действий, а также соответствующая ей онтология методов и технологий}
\label{sec:sd_actions}
\begin{SCn}

\scnsectionheader{\currentname}

\scnstartsubstruct

\scnsdmainclasssingle{действие}
\scnsdclass{информационное действие;поведенческое действие;эффекторное действие;рецепторное действие;действие в sc-памяти;действие во внешней среде ostis-системы;эффекторное действие ostis-системы;рецепторное действие ostis-системы;инициированное действие;выполняемое действие;активное действие;отложенное действие;планируемое действие;выполненное действие;успешно выполненное действие;безуспешно выполненное действие;действие, выполненное с ошибкой;приоритет действия;субъект;внутренний субъект ostis-системы;внешний субъект ostis-системы, с которым осуществляется взаимодействие;внешний субъект ostis-системы, с которым взаимодействие не происходит;класс действий;атомарный класс действий;неатомарный класс действий;конъюнкция предшествующих действий;проверка условия;задача;процедурная формулировка задачи;декларативная формулировка задачи;класс задач;вопрос;команда;класс команд;класс команд без аргументов;класс команд с одним аргументом;класс команд с двумя аргументами;класс команд с произвольным числом аргументов;атомарный класс команд;неатомарный класс команд;план;программа;программа в sc-памяти;протокол;решение}
\scnsdrelation{дейcтвие с очень высоким приоритетом';дейcтвие с высоким приоритетом';дейcтвие со средним приоритетом';дейcтвие с низким приоритетом';дейcтвие с очень низким приоритетом';декомпозиция действия*;поддействие*;последовательность действий*;последовательность действий при положительном результате*;последовательность действий при отрицательном результате*;последовательность действий в случае ошибки*;результат*;исполнитель*;класс выполняемых действий*;заказчик*;инициатор*;объект*;контекст действия*;аргумент действия';первый аргумент действия’;второй аргумент действия’;третий аргумент действия’;класс аргументов*;класс первых аргументов*;класс вторых аргументов*}
\scnrelfromvector{ключевые знаки}{действие;класс действий;метод;класс методов;деятельность;вид деятельности}



\scnendstruct \scnendcurrentsectioncomment

\end{SCn}

\scsubsection{Локальные предметные области действий и соответствующие им онтологии}
\label{local_sd_actions}

\scchapter{Предметная область и онтология решателей задач ostis-систем}
\label{sec:sd_ps}
\begin{SCn}

\scnsectionheader{\currentname}

\scnstartsubstruct

\scnheader{Предметная область и онтология решателей задач ostis-систем}
\scnsdmainclasssingle{решатель задач ostis-системы}
\scnsdclass{***}
\scnsdrelation{***}

\scnheader{решатель задач ostis-системы}
\scnrelfromlist{включение}{решатель задач IMS;решатель задач вспомогательной компьютерной системы\\
    \scnaddlevel{1}
    \scnrelfromlist{включение}{решатель задач интерфейса компьютерной системы\\
        \scnaddlevel{1}
        \scnrelfromlist{включение}{решатель задач пользовательского интерфейса компьютерной системы;решатель задач интерфейса компьютерной системы с другими компьютерными системами;решатель задач интерфейса компьютерной системы с окружающей средой}
        \scnaddlevel{-1}
    ;решатель задач подсистемы поддержки проектирования компонентов определенного класса\\
        \scnaddlevel{1}
        \scnrelfromlist{включение}{решатель задач подсистемы поддержки проектирования баз знаний\\
            \scnaddlevel{1}
            \scnrelfrom{включение}{машина повышения качества базы знаний\\
                \scnaddlevel{1}
                \scnrelfromlist{включение}{машина верификации базы знаний\\
                    \scnaddlevel{1}
                    \scnrelfromlist{включение}{машина поиска и устранения некорректностей;машина поиска и устранения неполноты}
                    \scnaddlevel{-1}
                ;машина оптимизации базы знаний;машина выявления и устранения информационного мусора}
                \scnaddlevel{-1}}
            \scnaddlevel{-1}
        ;решатель задач подсистемы поддержки проектирования решателей\\
            \scnaddlevel{1}
            \scnrelfromlist{включение}{решатель задач подсистемы поддержки проектирования программ обработки знаний;решатель задач подсистемы поддержки проектирования агентов обработки знаний}
            \scnaddlevel{-1}
        \scnaddlevel{-1}}
    ;решатель задач подсистемы управления проектирования компьютерных систем и их компонентов}
    \scnaddlevel{-1}
;решатель задач самостоятельной компьютерной системы}

\scnheader{решатель задач ostis-системы}
\scnrelfromlist{включение}{машина информационного поиска\\
    \scnaddlevel{1}
    \scnrelfromlist{включение}{машина информационного поиска информации, удовлетворяющей заданной спецификации;машина информационного поиска информации, не удовлетворяющей заданной спецификации;машина, выявляющая отсутствие информации, удовлетворяющей заданной спецификации}
    \scnaddlevel{-1}
;решатель задач с использованием хранимых программ\\
    \scnaddlevel{1}
    \scnrelfromlist{включение}{интерпретатор нейросетевых моделей;интерпретатор генетических алгоритмов;интерпретатор императивных программ\\
        \scnaddlevel{1}
        \scnrelfromlist{включение}{интерпретатор процедурных программ;интерпретатор объектно-ориентированных программ}
        \scnaddlevel{-1}
    ;интерпретатор декларативных программ\\
        \scnaddlevel{1}
        \scnrelfromlist{включение}{интерпретатор логических программ;интерпретатор функциональных программ}
        \scnaddlevel{-1}}
    \scnaddlevel{-1}
;решатель задач в условиях, когда программа решения не известна\\
    \scnaddlevel{1}
    \scnrelfromlist{включение}{решатель, реализующий поиск решения задачи в глубину;решатель, реализующий поиск решения задачи в ширину;решатель, реализующий метод проб и ошибок;решатель, реализующий метод разбиения задачи на подзадачи;решатель, реализующий метод решения задач по аналогии;решатель, реализующий метод сведения условия задачи к языку логики предикатов первого порядка;машина логического вывода\\
        \scnaddlevel{1}
        \scnrelfromlist{включение}{машина дедуктивного вывода\\
            \scnaddlevel{1}
            \scnrelfromlist{включение}{машина прямого дедуктивного вывода;машина обратного дедуктивного вывода}
            \scnaddlevel{-1}
        ;машина индуктивного вывода;машина абдуктивного вывода;машина нечеткого вывода;машина вывода на основе логики умолчаний;машина темпорального логического вывода}
        \scnaddlevel{-1}}
    \scnaddlevel{-1}}

\scnheader{решатель задач ostis-системы}
\scnrelfromlist{включение}{решатель явно сформулированных задач\\
    \scnaddlevel{1}
    \scnrelfromlist{включение}{машина поиска значений заданного множества величин;машина установления истинности заданного логического высказывания в рамках заданной формальной теории;машина формирования способа решения указанной задачи\\
        \scnaddlevel{1}
        \scnrelfromlist{включение}{машина формирования доказательства заданного высказывания в рамках заданной формальной теории}
        \scnaddlevel{-1}
    ;машина верификации ответа на указанную задачу;машина верификации способа решения указанной задачи\\
        \scnaddlevel{1}
        \scnrelfromlist{включение}{машина верификации доказательства заданного высказывания в рамках заданной формальной теории}
        \scnaddlevel{-1}}
    \scnaddlevel{-1}
;машина классификации сущностей\\
    \scnaddlevel{1}
    \scnrelfromlist{включение}{машина соотнесения сущности с одним из заданного множества классов;машина разделения множества сущностей на классы по заданному множеству признаков}
    \scnaddlevel{-1}
;машина синтеза естественно-языковых текстов;машина анализа естественно-языковых текстов\\
    \scnaddlevel{1}
    \scnrelfromlist{включение}{машина распознавания естественно-языковых текстов;машина верификации естественно-языковых текстов}
    \scnaddlevel{-1}
;машина синтеза сигналов\\
    \scnaddlevel{1}
    \scnrelfrom{включение}{машина синтеза речи}
    \scnaddlevel{-1}
;машина анализа сигналов\\
    \scnaddlevel{1}
    \scnrelfrom{включение}{машина анализа речи\\
        \scnaddlevel{1}
        \scnrelfrom{включение}{машина распознавания речи}
        \scnaddlevel{-1}}
    \scnaddlevel{-1}
;машина обработки мультимедийных данных\\
    \scnaddlevel{1}
    \scnrelfrom{включение}{машина анализа изображений\\
        \scnaddlevel{1}
        \scnrelfrom{включение}{машина машина распознавания изображений}
        \scnaddlevel{-1}}
    \scnaddlevel{-1}}

\scnheader{структура}
\scnsubdividing{sc-конструкция нестандартного вида;sc-конструкция стандартного вида\\
    \scnaddlevel{1}
    \scnsubdividing{одноэлементная sc-конструкция;трехэлементная sc-конструкция;пятиэлементная sc-конструкция}
    \scnaddlevel{-1}}
    
\scnheader{sc-конструкция нестандартного вида}
\scnexplanation{Каждая \textit{sc-конструкция нестандартного вида} состоит из произвольного количества \textit{sc-элементов} произвольного типа.

\begin{figure}[H]
  \includegraphics{figures/sd_ps/pic_ps1.png}
\end{figure}}

\scnheader{sc-конструкция стандартного вида}
\scnexplanation{В свою очередь, каждый элемент \textit{\mbox{sc-конструкции} стандартного вида} имеет свою условную строго фиксированную позицию в рамках этой \mbox{sc-конструкции} (первый элемент, второй элемент и т. д.). В зависимости от указанной позиции вводятся дополнительные ограничения на тип соответствующего \textit{sc-элемента}.}

\scnheader{одноэлементная sc-конструкция}
\scnexplanation{Каждая \textit{одноэлементная sc-конструкция} состоит из одного \textit{sc-элемента} произвольного типа.

\begin{figure}[H]
\includegraphics{figures/sd_ps/pic_ps2.png}
\end{figure}}

\scnheader{трехэлементная sc-конструкция}
\scnexplanation{Каждая \textit{трехэлементная sc-конструкция} состоит из трех \textit{sc-элементов}. Второй элемент всегда является \textit{sc-коннектором}, остальные элементы могут быть произвольного типа.

\begin{figure}[H]
  \includegraphics{figures/sd_ps/pic_ps3.png}
\end{figure}}

\scnheader{пятиэлементная sc-конструкция}
\scnexplanation{Каждая \textit{пятиэлементная sc-конструкция} состоит из пяти \textit{sc-элементов}. Второй и четвертый элементы обязательно являются \textit{sc-коннекторами}, остальные элементы могут быть произвольного типа.

\begin{figure}[H]
  \includegraphics{figures/sd_ps/pic_ps4.png}
\end{figure}}

\scnendstruct \scnendcurrentsectioncomment

\end{SCn}

\scsection{Предметная область и онтология действий, задач, планов, протоколов и методов, реализуемых ostis-системой, а также внутренних агентов, выполняющих эти действия}
\label{sec:sd_agents}
\begin{SCn}

\scnsectionheader{\currentname}

\scnstartsubstruct

\scnheader{Предметная область и онтология действий,  задач, планов, протоколов и методов, реализуемых ostis-системой в ее памяти, а также внутренних агентов, выполняющих эти действия}
\scniselement{предметная область}
\scnsdmainclass{действие в sc-памяти;абстрактный sc-агент;sc-агент}
\scnsdclass{абстрактный sc-агент, не реализуемый на Языке SCP;абстрактный sc-агент, реализуемый на Языке SCP;Абстрактный программный sc-агент;неатомарный абстрактный sc-агент;атомарный абстрактный sc-агент;платформенно-независимый абстрактный sc-агент;платформенно-зависимый абстрактный sc-агент;внутренний абстрактный sc-агент;эффекторный абстрактный sc-агент;рецепторный абстрактный sc-агент;абстрактный sc-агент, не реализуемый на Языке SCP;абстрактный sc-агент, реализуемый на Языке SCP;
абстрактный sc-агент интерпретации scp-программ;абстрактный программный sc-агент;
абстрактный программный sc-агент, реализуемый на Языке SCP;абстрактный sc-метаагент;sc-агент;активный sc-агент;описание поведения sc-агента;тип блокировки;полная блокировка;блокировка на любое изменение;блокировка на удаление}
\scnsdrelation{декомпозиция абстрактного sc-агента*;ключевые sc-элементы sc-агента*;программа sc-агента*;первичное условие инициирования*;условие инициирования и результат*;блокировка*}

\scnheader{действие в sc-памяти}
\scnidtf{внутреннее действие ostis-системы}
\scnidtf{действие, выполняемое в sc-памяти}
\scnidtf{действие, выполняемое в абстрактной унифицированной семантической памяти}
\scnidtf{действие, выполняемое машиной обработки знаний ostis-системы}
\scnidtf{действие, выполняемое агентом или коллективом агентов ostis-системы}
\scnidtf{информационный процесс над базой знаний, хранимой в sc-памяти}
\scnidtf{процесс решения информационной задачи в sc-памяти}
\scnrelto{включение}{процесс в sc-памяти}
\scnexplanation{Каждое \textbf{\textit{действие в sc-памяти}} обозначает некоторое преобразование, выполняемое некоторым \textit{sc-агентом} (или коллективом \textit{sc-агентов}) и ориентированное на преобразование \textit{sc-памяти}. Спецификация действия после его выполнения может быть включена в протокол решения некоторой задачи. 

Преобразование состояния базы знаний включает, в том числе и информационный поиск, предполагающий (1) локализацию в базе знаний ответа на запрос, явное выделение структуры ответа и (2) трансляцию ответа на некоторый внешний язык.

Во множество \textbf{\textit{действий в sc-памяти}} входят знаки действий самого различного рода, семантика каждого из которых зависит от конкретного контекста, т.е. ориентации действия на какие-либо конкретные объекты и принадлежности действия какому-либо конкретному классу действий.

Следует четко отличать:
\begin{scnitemize}
\item Каждое конкретное \textbf{\textit{действие в sc-памяти}}, представляющее собой некоторый переходный процесс, переводящий sc-память из одного состояния в другое;
\item Каждый тип \textbf{\textit{действий в sc-памяти}}, представляющий собой некоторый класс однотипных (в том или ином смысле) действий;
\item sc-узел, обозначающий некоторое конкретное \textbf{\textit{действие в sc-памяти}};
\item sc-узел, обозначающий структуру, которая является описанием, спецификацией, заданием, постановкой соответствующего действия.
\end{scnitemize}
}
\scnrelfromlist{включение}{действие в sc-памяти, инициируемое вопросом;действие редактирования базы знаний ostis-системы;действие установки режима ostis-системы;действие редактирования файла, хранимого в sc-памяти;действие интерпретации программы, хранимой в sc-памяти}

\scnheader{действие в sc-памяти, инициируемое вопросом}
\scnidtf{действие, направленное на формирование ответа на поставленный вопрос}
\scnrelfromlist{включение}{действие. cформировать заданный файл;действие. cформировать заданную структуру\\
    \scnaddlevel{1}
    \scnrelfromlist{включение}{действие. верифицировать заданную структуру\\
        \scnaddlevel{1}
        \scnaddhind{-2}
        \scnrelfromlist{включение}{действие. установить истинность или ложность указываемого логического высказывания; действие. установить корректность или некорректность указываемой структуры; действие. сформировать структуру, описывающую некорректности, имеющиеся в указываемой структуре}
        \scnaddlevel{-1};
        действие. установить тип заданного sc-элемента\\
        \scnaddlevel{1}
        \scnrelfromlist{включение}{действие. установить позитивность/негативность указываемой sc-дуги принадлежности или непринадлежности}
        \scnaddlevel{-1};
        действие. сформировать семантическую окрестность\\
        \scnaddlevel{1}
        \scnrelfromlist{включение}{действие. сформировать полную семантическую окрестность указываемой сущности;действие. сформировать базовую семантическую окрестность указываемой сущности;действие. сформировать частную семантическую окрестность указываемой сущности}\scnaddlevel{-1};действие. сформировать структуру, описывающую связи между указываемыми сущностями\\
        \scnaddlevel{1}
        \scnrelfromlist{включение}{действие. сформировать структуру, описывающую сходства указываемых сущностей;действие. сформировать структуру, описывающую различия указываемых сущностей}
        \scnaddlevel{-1}
        ;действие. сформировать структуру, описывающую способ решения указываемой задачи;действие. сформировать план генерации ответа на указанный вопрос;действие. сформировать протокол выполнения указываемого действия
        ;действие. сформировать обоснование корректности указываемого решения;действие. верифицировать обоснование корректности указываемого решения;действие, одним из аргументов которого является некоторая обобщенная структура;действие, направленное на установление темпоральных характеристик указываемой сущности;действие, направленное на установление пространственных характеристик указываемой сущности}
    \scnaddlevel{-1}
}

\scnheader{действие редактирования базы знаний}
\scnrelfromlist{включение}{действие. изменить направление указанной sc-дуги;действие. исправить ошибки в заданной структуре;действие. преобразовать указанную структуру в соответствии с указанным правилом;действие. отождествить два указанных sc-элемента;действие. включить множество\\
    \scnaddlevel{1}
    \scnidtf{сделать все элементы множества si явно принадлежащими множеству sj, то есть сгенерировать соответствующие sc-дуги принадлежности}
    \scnaddlevel{-1}
    ;действие генерации sc-элементов\\
    \scnaddlevel{1}
    \scnrelfromlist{включение}{действие генерации, одним из аргументов которого является некоторая обобщенная структура\\
    \scnaddlevel{1}
    \scnrelfromlist{включение}{действие. сгенерировать структуру, изоморфную указываемому образцу}
    \scnaddlevel{-1}
    ;действие. сгенерировать sc-элемент указанного типа\\
    \scnaddlevel{1}
    \scnrelfromlist{включение}{действие. сгенерировать sc-коннектор указанного типа;действие. сгенерировать sc-узел указанного типа}
    \scnaddlevel{-1}
    ;действие. сгенерировать структуру, содержащую указанные sc-элементы;действие. сгенерировать файл с заданным содержимым;действие. обновить понятия\\
    \scnaddlevel{1}
    \scnidtf{действие. заменить неиспользуемое понятие на его определение через используемое понятие}
    \scnaddlevel{-1}
    ;действие. установить указанный файл в качестве основного идентификатора указанного sc-элемента;действие. протранслировать содержимое указываемого файла в sc-память;действие. интегрировать указанную структуру в текущее состояние базы знаний;действие. сгенерировать структуру, описывающую историю эволюции ostis-системы;действие. сгенерировать структуру, описывающую историю эксплуатации ostis-системы}
    \scnaddlevel{-1}
    ;действие удаления sc-элементов\\
    \scnaddlevel{1}
    \scnrelfromlist{включение}{действие. удалить указанные sc-элементы\\
    \scnaddlevel{1}
    \scnrelfromlist{включение}{действие. удалить указанный sc-элемент;действие. удалить sc-элементы, входящие в состав указанной структуры и не являющиеся ключевыми узлами каких-либо sc-агентов}
    \scnaddlevel{-1}
    ;действие. исключить указанные sc-элементы из клиентской части базы знаний}
}

\scnresetlevel

\scnheader{действие. отождествить два указанных sc-элемента}
\scnidtf{действие. совместить два указанных sc-элемента}
\scnidtf{действие. склеить два указанных sc-элемента}
\scnreltoset{разбиение}{действие. физически отождествить два указанных sc-элемента;действие. логически отождествить два указанных sc-элемента}

\scnheader{действие. отождествить два указанных sc-элемента}
\scnexplanation{Каждое \textbf{\textit{действие. отождествить два указанных sc-элемента}} может быть выполнено как \textit{действие. физически отождествить два указанных sc-элемента} или \textit{действие. логически отождествить два указанных sc-элемента}. В случае логического отождествления в протоколе деятельности агентов сохраняется само действие с его спецификацией, включающей обязательное указание того, какие элементы были сгенерированы, а какие удалены. В случае физического отождествления протокол действия не сохраняется.}

\scnheader{действие. обновить понятия}
\scnidtf{действие. заменить некоторое множество понятий на другое множество понятий}
\scnexplanation{Каждое \textbf{\textit{действие. обновить понятия}} обозначает переход от какой-то группы понятий, использовавшихся ранее, к другой группе понятий, которые будут использоваться вместо первых, и станут \textit{основными понятиями}.
В общем случае \textbf{\textit{действие. обновить понятия}} состоит из следующих этапов:

\begin{scnitemize}
    \item Определить заменяемые понятия на основе заменяющих;
    \item Внести соответствующие изменения в программы sc-агентов, ключевыми узлами которых являются обновляемые понятия;
    \item Заменить все конструкции в базе знаний, содержащие заменяемые понятия, в соответствии с определениями этих понятий через заменяющие их понятия;
    \item При необходимости,\textit{ sc-элементы}, обозначающие замененные таким образом понятия, могут быть полностью выведены из текущего состояния базы знаний.
\end{scnitemize}

Первым аргументом (входящим в знак \textit{действия} под атрибутом \textit{1’}) \textbf{\textit{действия. обновить понятия}} является знак множества \textit{sc-узлов}, обозначающих заменяемые понятия, вторым (входящим в знак \textit{действия} под атрибутом \textit{2’}) - знак множества \textit{sc-узлов}, обозначающих заменяющие понятия. В общем случае любое или оба этих множества могут быть \textit{синглетонами}.}

\scnheader{действие. удалить указанные sc-элементы}
\scnreltoset{разбиение}{действие. физически удалить указанные sc-элементы;действие. логически удалить указанные sc-элементы}
\scnexplanation{Каждое \textbf{\textit{действие. удалить указанные sc-элементы}} может быть выполнено как \textit{действие. физически удалить указанные sc-элементы} или \textit{действие. логически удалить указанные sc-элементы}. В случае логического удаления в протоколе деятельности агентов сохраняется само действие с его спецификацией, включающей обязательное указание того, какие элементы были удалены, т.е. по сути, элементы просто исключаются из текущего состояния базы знаний. В случае физического удаления протокол действия не сохраняется.

В случае удаления какого-либо \textit{sc-элемента}, инцидентные ему \textit{связки}, в том числе \textit{sc-коннекторы}, также удаляются.}

\scnheader{действие. интегрировать указанную структуру в текущее состояние базы знаний}
\scnexplanation{Для того, чтобы выполнить \textbf{\textit{действие. интегрировать указанную структуру в текущее состояние базы знаний}}, необходимо склеить \textit{sc-элементы}, входящие в интегрируемую \textit{структуру} с синонимичными им \textit{sc-элементами}, входящими в текущее состояние базы знаний, заменить неиспользуемые (например, устаревшие) понятия, входящие в интегрируемую \textit{структуру}, на используемые (т.е. заменить неиспользуемые понятия на их определения через используемые), явно включить все элементы интегрируемой \textit{структуры} в число элементов утвержденной части базы знаний и явно включить все элементы интегрируемой \textit{структуры} в число элементов одного из атомарных разделов утвержденной части базы знаний.}

\scnheader{действие установки режима ostis-системы}

\scnheader{действие редактирования файла, хранимого в sc-памяти}

\scnheader{действие интерпретации программы, хранимой в sc-памяти}
\scnrelfrom{включение}{действие интерпретации scp-программы}

\scnheader{задача, решаемая в sc-памяти}
\scnrelto{включение}{задача}
\scnidtf{спецификация действия, выполняемого в sc-памяти}
\scnidtf{структура, являющая таким описанием (постановкой, заданием) соответствующего действия в sc-памяти, которое обладает достаточной полнотой для выполения указанного действия}
\scnidtf{семантическая окрестность некоторого действия в sc-памяти, обеспечивающая достаточно полное задание этого действия}

\scnsectionheader{\currentname}

\scntext{введение}{Общие принципы организации взаимодействия \textit{sc-агентов} и пользователей
\textit{ostis-системы} через общую \textit{sc-память}
\begin{scnitemize}
    \item Каждая \textit{ostis-система} представляет собой многоагентную систему, агенты которой взаимодействуют между собой \underline{только}(!) через общую для них \textit{sc-память}. При этом пользователи \textit{ostis-системы} также считают агентами этой системы. Кроме того, \textit{sc-агенты} делятся на внутренние, рецепторные и эффекторные. Взаимодействие между агентами через общую \textit{sc-память} сводится к следующим видам действий:
    \begin{scnenumerate}
        \item К использованию общедоступных для соответствующей группы агентов части хранимой базы знаний. В простейшем случае по уровню прав доступа агенты \textit{ostis-системы} разбиваются на две группы – главные администраторы базы знаний (их может быть несколько) вместе с обслуживающими их \textit{sc-агентами} и все остальные агенты;
        \item К формированию (генерации) новых фрагментов базы знаний и/или к корректировке (редактированию) каких-либо фрагментов доступной части базы знаний;
        \item К интеграции (погружению) новых и/или обновленных фрагментов в состав доступной части базы знаний;
    \end{scnenumerate}
    \item Пользователь \textit{ostis-системы} не может сам непосредственно выполнить какое-либо действие в sc-памяти, но он может средствами пользовательского интерфейса инициировать построение (генерацию, формирование в \textit{sc-памяти}) \textit{sc-текста}, являющегося спецификацией \textit{действия в sc-памяти}, выполняемого либо одним атомарным \textit{sc-агентом} за один акт, либо одним атомарным \textit{sc-агентом} за несколько актов, либо коллективом \textit{sc-агентов}. В спецификации каждого такого \textit{действия в sc-памяти}, инициированного пользователем, этот пользователь указывается как заказчик этого действия. Таким образом, пользователь \textit{ostis-системы} дает поручения (задания, команды) \textit{sc-агентам} этой системы на выполнение различных специфицируемых им действий в sc-\textit{памятью}.
    \item Каждый \textit{sc-агент}, выполняя некоторое \textit{действие в sc-памяти}, должен «помнить», что \textit{sc-память}, над которой он работает, является общим ресурсом не только для него, но и для всех остальных \textit{sc-агентов}, работающих над этой же \textit{sc-памятью}. Поэтому \textit{sc-агент} должен соблюдать определенную этику поведения в коллективе таких \textit{sc-агентов}, которая должна минимизировать помехи, которые он создает другим \textit{sc-агентам}. 
    \item Деятельность каждого агента \textit{ostis-системы} дискретна и представляет собой множество элементарных действий (актов). При этом при выполнении практически каждого акта агент выделяет некий фрагмент базы знаний, который вообще не должен быть «виден» другим агентам (только ему самому) и/или некоторый фрагмент базы знаний, который может быть «виден», но не может изменяться другими агентами. Указанная блокировка – это своего рода «забор» (ограждение) через который другим агентам перелезать запрещено. Эта блокировка устанавливается самим агентом при выполнении соответствующего акта и снимается им же на последнем этапе выполнения этого акта.
    \item Если некий \textit{sc-агент} выполняет некоторое действие в \textit{sc-памятью}, то он на время выполнения этого действия может:
        \begin{scnenumerate}
            \item Запретить другим \textit{sc-агентам} изменять состояние некоторых sc-элементов, хранимых в \textit{sc-памяти} – удалять их, изменять тип;
            \item Запретить другим \textit{sc-агентам} добавлять или удалять элементы некоторых множеств, обозначаемых соответствующими \textit{sc-узлами};
            \item Запретить другим \textit{sc-агентам} доступ на просмотр некоторых \textit{sc-элементов}, то есть эти \textit{sc-элементы} становятся полностью «невидимыми» (полностью заблокированными) для других \textit{sc-агентов} но только на время выполнения соответствующего действия.
        \end{scnenumerate}
        
    Указанные блокировки должны быть полностью сняты до завершения выполнения соответствующего действия. Подчеркнем, что число \textit{sc-элементов}, блокируемых на время выполнения некоторого действия, в основном входят атомарные и неатомарные связки, и не должны входить \textit{sc-узлы}, обозначающие бесконечные классы каких-либо сущностей, и, тем более, sc-узлы, обозначающие различные понятия (ключевые классы различных предметных областей).
    
    Этичное (неэгоистичное) поведение \textit{sc-агента}, касающееся блокировки \textit{sc-элементов} (то есть ограничения к ним доступа другим \textit{sc-агентам}) предполагает соблюдение следующих правил:
    \begin{scnenumerate}
            \item Не следует блокировать больше \textit{sc-элементов}, чем это необходимо, то есть не следует «жадничать»;
            \item Как только для какого-либо \textit{sc-элемента} необходимость его блокировки отпадает до завершения выполнения соответствующего действия, этот \textit{sc-элемент} желательно сразу деблокировать (снять блокировку);
    \end{scnenumerate}    

    Для того, чтобы \textit{sc-агент} проверил возможность работы с каким-либо произвольным \textit{sc-элементом}, он должен либо убедиться в том, что этот \textit{sc-элемент} не входит во множество «полностью заблокированный sc-элемент», либо убедиться в том, что указанный \textit{sc-элемент} входит в указанное множество, но при это связан отношением \textit{полностью заблокированный sc-элемент*} с действием, выполняемым этим \textit{sc-агентом}. Очевидно, что больших затрат времени указанная проверка не потребует.

    Особой группой полностью заблокированных \textit{sc-элементов} (на время выполнения действия \textit{sc-агентом}) являются вспомогательные \textit{sc-элементы} («леса»), создаваемые только на время выполнения этого действия. Эти sc-элементы в конце выполнения действия должны не деблокироваться, а удаляться.
    \item Если \textit{действие в sc-памяти}, выполняемое \textit{sc-агентом}, завершилось (т.е. стало прошлой сущностью), то \textit{sc-агент} оформляет результат («сухой остаток») этого \textit{действия}, указывая (1) удаленные \textit{sc-элементы} и (2) сгенерированные sc-элементы. Это необходимо, если нам придется сделать откат этого \textit{действия}, т.е возвратиться к состоянию базы знаний до выполнения указанного \textit{действия};
\end{scnitemize}
}

\scnheader{абстрактный sc-агент}
\scnexplanation{Под \textbf{\textit{абстрактным sc-агентом}} понимается некоторый класс функционально эквивалентных \textit{sc-агентов}, разные экземпляры (т.е. представители) которого могут быть реализованы по-разному.

Каждый \textbf{\textit{абстрактный sc-агент}} имеет соответствующую ему спецификацию. В спецификацию каждого \textbf{\textit{абстрактного sc-агента}} входит:
\begin{scnitemize}
    \item указание ключевых \textit{sc-элементов} этого \textit{sc-агента}, т.е. тех \textit{sc-элементов}, хранимых в \textit{sc-памяти}, которые для данного \textit{sc-агента} являются «точками опоры»;
    \item формальное описание условий инициирования данного \textit{sc-агента}, т.е. тех \textit{ситуация} в \textit{sc-памяти}, которые инициируют деятельность данного \textit{sc-агента};
    \item формальное описание первичного условия инициирования данного \textit{sc-агента}, т.е. такой ситуации в \textit{sc-памяти}, которая побуждает \textit{sc-агента} перейти в активное состояние и начать проверку наличия своего полного условия инициирования (для \textit{внутренних абстрактных sc-агентов});
    \item строгое, полное, однозначно понимаемое описание деятельности данного \textit{sc-агента}, оформленное при помощи каких-либо понятных, общепринятых средств, не требующих специального изучения, например на естественном языке.
    \item описание результатов выполнения данного \textit{sc-агента}.
\end{scnitemize}
}
\scnreltoset{разбиение}{неатомарный абстрактный sc-агент;атомарный абстрактный sc-агент}
\scnreltoset{разбиение}{внутренний абстрактный sc-агент;эффекторный абстрактный sc-агент;рецепторный абстрактный sc-агент}
\scnreltoset{разбиение}{абстрактный sc-агент, не реализуемый на Языке SCP;абстрактный sc-агент, реализуемый на Языке SCP}
\scnreltoset{разбиение}{абстрактный sc-агент интерпретации scp-программ;абстрактный программный sc-агент;абстрактный sc-метаагент}
\scnreltoset{разбиение}{платформенно-зависимый абстрактный sc-агент\\
\scnaddlevel{1}
\scnrelfrom{включение}{абстрактный sc-агент, не реализуемый на Языке SCP}
\scnaddlevel{-1}
;платформенно-независимый абстрактный sc-агент}

\scnheader{абстрактный sc-агент, не реализуемый на Языке SCP}
\scnidtf{абстрактный sc-агент, который не может быть реализован на платформенно-независимом уровне}
\scnreltoset{разбиение}{эффекторный абстрактный sc-агент;рецепторный абстрактный sc-агент
;абстрактный sc-агент интерпретации scp-программ}

\scnheader{абстрактный sc-агент, реализуемый на Языке SCP}
\scnidtf{абстрактный sc-агент, который может быть реализован на платформенно-независимом уровне}
\scnreltoset{разбиение}{абстрактный sc-метаагент;абстрактный программный sc-агент, реализуемый на Языке SCP}

\scnheader{абстрактный программный sc-агент}
\scnreltoset{разбиение}{эффекторный абстрактный sc-агент;рецепторный абстрактный sc-агент
;абстрактный программный sc-агент, реализуемый на Языке SCP}

\scnheader{неатомарный абстрактный sc-агент}
\scnexplanation{Под \textbf{\textit{неатомарным абстрактным sc-агентом}} понимается \textit{абстрактный sc-агент}, который декомпозируется на коллектив более простых \textit{абстрактных sc-агентов}, каждый из которых в свою очередь может быть как \textit{атомарным абстрактным sc-агентом}, так и \textbf{\textit{неатомарным абстрактным sc-агентом}}. При этом в каком либо варианте \textit{декомпозиции абстрактного sc-агента*} дочерний \textbf{\textit{неатомарный абстрактный sc-агент}} может стать \textit{атомарным абстрактным sc-агентом}, и реализовываться соответствующим образом.}

\scnheader{атомарный абстрактный sc-агент}
\scnexplanation{Под \textbf{\textit{атомарным абстрактным sc-агентом}} понимается \textit{абстрактный sc-агент}, для которого уточняется платформа его реализации, т.е. существует соответствующая связка отношения \textit{программа sc-агента*}.}
\scnreltoset{разбиение}{платформенно-независимый абстрактный sc-агент;платформенно-зависимый абстрактный sc-агент}

\scnheader{платформенно-независимый абстрактный sc-агент}
\scnexplanation{К \textbf{\textit{платформенно-независимым абстрактным sc-агентам}} относят \textit{атомарные абстрактные sc-агенты}, реализованные на базовом языке программирования Технологии OSTIS, т.е. на \textit{Языке SCP}.

При описании \textbf{\textit{платформенно-независимых абстрактных sc-агентов}} под платформенной независимостью понимается платформенная независимость с точки зрения Технологии OSTIS, т.е реализация на специализированном языке программирования, ориентированном на обработку семантических сетей (\textit{Языке SCP}), поскольку \textit{атомарные sc-агенты}, реализованные на указанном языке могут свободно переноситься с одной платформы интерпретации \textit{sc-моделей} на другую. При этом языки программирования, традиционно считающиеся платформенно-независимыми в данном случае не могут считаться таковыми.

Существуют \textit{sc-агенты}, которые принципиально не могут быть реализованы на платформенно-независимом уровне, например, собственно \textit{sc-агенты} интерпретации \textit{sc-моделей} или рецепторные и эффекторные \textit{sc-агенты}, обеспечивающие взаимодействие с внешней средой.}

\scnheader{платформенно-зависимый абстрактный sc-агент}
\scnexplanation{К \textbf{\textit{платформенно-зависимым абстрактным sc-агентам}} относят \textit{атомарные абстрактные sc-агенты}, реализованные ниже уровня sc-моделей, т.е. не на \textit{Языке SCP}, а на каком-либо другом языке описания программ.

Существуют \textit{sc-агенты}, которые принципиально должны быть реализованы на платформенно-зависимом уровне, например, собственно \textit{sc-агенты} интерпретации \textit{sc-моделей} или рецепторные и эффекторные \textit{sc-агенты}, обеспечивающие взаимодействие с внешней средой.}

\scnheader{внутренний абстрактный sc-агент}
\scnexplanation{Каждый \textbf{\textit{внутренний абстрактный sc-агент}} обозначает класс \textit{sc-агентов}, которые реагируют на события в \textit{sc-памяти} и осуществляют преобразования исключительно в рамках этой же \textit{sc-памяти}.}

\scnheader{эффекторный абстрактный sc-агент}
\scnexplanation{Каждый \textbf{\textit{эффекторный абстрактный sc-агент}} обозначает класс \textit{sc-агентов}, которые реагируют на события в \textit{sc-памяти} и осуществляют преобразования во внешней относительно данной \textit{ostis-системы} среде.}

\scnheader{рецепторный абстрактный sc-агент}
\scnexplanation{Каждый \textbf{\textit{рецепторный абстрактный sc-агент}} обозначает класс \textit{sc-агентов}, которые реагируют на события во внешней относительно данной \textit{ostis-системы} среде и осуществляют преобразования в памяти данной системы.}

\scnheader{абстрактный sc-агент, не реализуемый на Языке SCP}
\scnexplanation{Каждый \textbf{\textit{абстрактный sc-агент, не реализуемый на Языке SCP}} должен быть реализован на уровне платформы интерпретации sc-моделей, в том числе, аппаратной. К таким \textit{абстрактным sc-агентам} относятся абстрактные sc-агенты интерпретации scp-программ, а также эффекторные и рецепторные абстрактные sc-агенты.}

\scnheader{абстрактный sc-агент, реализуемый на Языке SCP}
\scnexplanation{Каждый \textbf{\textit{абстрактный sc-агент, реализуемый на Языке SCP}} может быть реализован на Языке SCP, то есть платформенно-независимом уровне, но при необходимости, может реализовываться и на уровне платформы, например, с целью повышения производительности.}

\scnheader{абстрактный sc-агент интерпретации scp-программ}
\scnexplanation{К \textbf{\textit{абстрактным sc-агентам интерпретации scp-программ}} относятся не реализуемые на платформенно-независимом уровне \textit{абстрактные sc-агенты}, обеспечивающие интерпретацию \textit{scp-программ} и \textit{scp-метапрограмм}, в том числе создание \textit{scp-процессов}, собственно интерпретацию \textit{scp-операторов}, а также другие вспомогательные действия. По сути, агенты данного класса обеспечивают работу sc-агентов более высоких уровней (программных sc-агентов и sc-метаагентов), реализованных на Языке SCP, в частности, обеспечивают соблюдение указанными агентами общих принципов синхронизации.}

\scnheader{абстрактный программный sc-агент}
\scnexplanation{К \textbf{\textit{абстрактным программным sc-агентам}} относятся все \textit{абстрактные sc-агенты}, обеспечивающие основной функционал системы, то есть ее возможность решать те или иные задачи. Агенты данного класса должны работать в соответствии с общими принципами синхронизации деятельности субъектов в sc-памяти.}

\scnheader{абстрактный программный sc-агент, реализуемый на Языке SCP}

\scnheader{абстрактный sc-метаагент}
\scnexplanation{Задачей \textbf{\textit{абстрактных sc-метаагентов}} является координация деятельности \textit{абстрактных программных sc-агентов}, в частности, решение проблемы взаимоблокировок. Агенты данного класса могут быть реализованы на Языке SCP, однако для синхронизации их деятельности используются другие принципы, соответственно, для реализации таких агентов требуется Язык SCP другого уровня, типология операторов которого полностью аналогична типологии scp-операторов, однако эти операторы имеют другую операционную семантику, учитывающую отличия в принципах синхронизации (работы с \textit{блокировками*}). Программы такого языка будем называть \textit{scp-метапрограммами}, соответствующие им \textit{процессы в sc-памяти} – \textit{scp-метапроцессами}, операторы – \textit{scp-метаоператорами}.}

\scnheader{декомпозиция абстрактного sc-агента*}
\scniselement{отношение декомпозиции}
\scnexplanation{Отношение \textbf{\textit{декомпозиции абстрактного sc-агента*}} трактует \textit{неатомарные абстрактные sc-агенты} как коллективы более простых \textit{абстрактных sc-агентов}, взаимодействующих через \textit{sc-память}.

Другими словами, \textbf{\textit{декомпозиция абстрактного sc-агента*}} на \textit{абстрактные sc-агенты} более низкого уровня уточняет один из возможных подходов к реализации этого \textit{абстрактного sc-агента} путем построения коллектива более простых \textit{абстрактных sc-агентов}.}

\scnheader{sc-агент}
\scnidtf{агент над sc-памятью}
\scnrelto{включение}{субъект}
\scnrelfrom{семейство подмножеств}{абстрактный sc-агент}
\scnexplanation{Под \textbf{\textit{sc-агентом}} понимается конкретный экземпляр (с теоретико-множественной точки зрения - элемент) некоторого \textit{атомарного абстрактного sc-агента}, работающий в какой-либо конкретной интеллектуальной системе.

Таким образом, каждый \textit{sc-агент} - это субъект, способный выполнять некоторый класс однотипных действий либо только над \textit{sc-памятью}, либо над sc-памятью и внешней средой (для эффекторных \textit{sc-агентов}). Каждое такое действие инициируется либо состоянием или ситуацией в sc-памяти, либо состоянием или ситуацией во внешней среде (для рецепторных sc-агентов-датчиков),  соответствующей условию инициирования \textit{атомарного абстрактного sc-агента}, экземпляром которого является заданный \textit{sc-агент}. В данном случае можно провести аналогию между принципами объектно-ориентированного программирования, рассматривая \textit{атомарный абстрактный sc-агент} как класс, а конкретный \textit{sc-агент} – как экземпляр, конкретную имплементацию этого класса.

Взаимодействие \textit{sc-агентов} осуществляется только через \textit{sc-память}. Как следствие, результатом работы любого \textit{sc-агента} является некоторое изменение состояния \textit{sc-памяти}, т.е. удаление либо генерация каких-либо \textit{sc-элементов}.

В общем случае один \textit{sc-агент} может явно передать управление другому \textit{sc-агенту}, если этот \textit{sc-агент} априори известен. Для этого каждый \textit{sc-агент} в \textit{sc-памяти} имеет обозначающий его \textit{sc-узел}, с которым можно связать конкретную ситуацию в текущем состоянии базы знаний, которую инициируемый \textit{sc-агент} должен обработать.

Однако далеко не всегда легко определить того \textit{sc-агента}, который должен принять управление от заданного \textit{sc-агента}, в связи с чем описанная выше ситуация возникает крайне редко. Более того, иногда условие инициирования \textit{sc-агента} является результатом деятельности непредсказуемой группы \textit{sc-агентов}, равно как и одна и та же конструкция может являться условием инициирования целой группы \textit{sc-агентов}.

При этом общаются через \textit{sc-память} не \textit{программы sc-агентов*}, а сами описываемые данными программами \textit{sc-агенты}.

В процессе работы \textit{sc-агент} может сам для себя порождать вспомогательные \textit{sc-элементы}, которые сам же удаляет после завершения акта своей деятельности (это вспомогательные \textit{структуры}, которые используются в качестве «информационных лесов» только в ходе выполнения соответствующего акта деятельности и после завершения этого акта удаляются).}

\scnheader{активный sc-агент}
\scnrelto{включение}{sc-агент}
\scnexplanation{Под \textbf{\textit{активным sc-агентом}} понимается \textit{sc-агент} ostis-системы, который реагирует на события, соответствующие его условию инициирования, и, как следствие, его \textit{первичному условию инициирования*}. Не входящие во множество \textbf{\textit{активных sc-агентов}} \textit{sc-агенты} не реагируют ни на какие события в \textit{sc-памяти}.}

\scnheader{ключевые sc-элементы sc-агента*}
\scnexplanation{Связки отношения \textbf{\textit{ключевые sc-элементы sc-агента*}} связывают между собой \textit{sc-узел}, обозначающий \textit{абстрактный sc-агент} и \textit{sc-узел}, обозначающий множество \textit{sc-элементов}, которые являются ключевыми для данного \textit{абстрактного sc-агента}, то данные \textit{sc-элементы} явно упоминаются в рамках программ, реализующих данный \textit{абстрактный sc-агент}.}

\scnheader{программа sc-агента*}
\scnexplanation{Связки отношения \textbf{\textit{программа sc-агента*}} связывают между собой \textit{sc-узел}, обозначающий \textit{атомарный абстрактный sc-агент} и \textit{sc-узел}, обозначающий множество программ, реализующих указанный \textit{атомарный абстрактный sc-агент}. В случае \textit{платформенно-независимого абстрактного sc-агента} каждая связка отношения \textit{программа sc-агента*} связывает \textit{sc-узел}, обозначающий указанный \textit{абстрактный sc-агент} с множеством \textit{scp-программ}, описывающих деятельность данного \textit{абстрактного sc-агента}. Данное множество содержит одну \textit{агентную scp-программу}, и произвольное количество (может быть, и ни одной) \textit{scp-программ}, которые необходимы для выполнения указанной \textit{агентной scp-программы}.

В случае \textit{платформенно-зависимого абстрактного sc-агента} каждая связка отношения \textit{программа sc-агента*} связывает \textit{sc-узел}, обозначающий указанный \textit{абстрактный sc-агент} с множеством файлов, содержащих исходные тексты программы на некотором внешнем языке программирования, реализующей деятельность данного \textit{абстрактного sc-агента}.}

\scnheader{первичное условие инициирования*}
\scnexplanation{Связки отношения \textbf{\textit{первичное условие инициирования*}} связывают между собой \textit{sc-узел}, обозначающий \textit{абстрактный sc-агент} и бинарную ориентированную пару, описывающую первичное условие инициирования данного \textit{абстрактного sc-агента}, т.е. такой спецификацию \textit{ситуации} в \textit{sc-памяти}, возникновение которой побуждает \textit{sc-агента} перейти в активное состояние и начать проверку наличия своего полного условия инициирования.

Первым компонентом данной ориентированной пары является знак некоторого класса \textit{элементарных событий в sc-памяти*}, например, \textit{событие добавления sc-дуги, выходящей из заданного sc-элемента*}.

Вторым компонентом данной ориентированной пары является произвольный в общем случае \textit{sc-элемент}, с которым непосредственно связан указанный тип события в \textit{sc-памяти}, т.е., например, \textit{sc-элемент}, из которого выходит либо в который входит генерируемая либо удаляемая \textit{sc-дуга}, либо \textit{файл}, содержимое которого было изменено.

После того, как в \textit{sc-памяти} происходит некоторое событие, активизируются все \textit{активные sc-агенты}, \textbf{\textit{первичное условие инициирования*}} которых соответствует произошедшему событию.}

\scnheader{условие инициирования и результат*}
\scnexplanation{Связки отношения \textbf{\textit{условие инициирования и результат*}} связывают между собой \textit{sc-узел}, обозначающий \textit{абстрактный sc-агент} и бинарную ориентированную пару, связывающую условие инициирования данного \textit{абстрактного sc-агента} и результаты выполнения данного экземпляров данного \textit{sc-агента} в какой-либо конкретной системе.

Указанную ориентированную пару можно рассматривать как логическую связку импликации, при этом на \textit{sc-переменные}, присутствующие в обеих частях связки, неявно накладывается квантор всеобщности, на \textit{sc-переменные}, присутствующие либо только в посылке, либо только в заключении неявно накладывается квантор существования.

Первым компонентом указанной ориентированной пары является логическая формула, описывающая условие инициирования описываемого \textit{абстрактного sc-агента}, то есть конструкции, наличие которой в \textit{sc-памяти} побуждает \textit{sc-агент} начать работу по изменению состояния \textit{sc-памяти}. Данная логическая формула может быть как атомарной, так и неатомарной, в которой допускается использование любых связок логического языка.

Вторым компонентом указанной ориентированной пары является логическая формула, описывающая возможные результаты выполнения описываемого абстрактного \textit{sc-агента}, то есть описание произведенных им изменений состояния \textit{sc-памяти}. Данная логическая формула может быть как атомарной, так и неатомарной, в которой допускается использование любых связок логического языка.}

\scnheader{описание поведения sc-агента}
\scnrelto{включение}{семантическая окрестность}
\scnexplanation{\textbf{\textit{описание поведения sc-агента}} представляет собой \textit{семантическую окрестность}, описывающую деятельность \textit{sc-агента} до какой-либо степени детализации, однако такое описание должно быть строгим, полным и однозначно понимаемым. Как любая другая \textit{семантическая окрестность}, \textbf{\textit{описание поведения sc-агента}} может быть протранслировано на какие-либо понятные, общепринятые средства, не требующие специального изучения, например на естественный язык.\\
Описываемый \textit{абстрактный sc-агент} входит в соответствующее \textbf{\textit{описание поведения sc-агента}} под атрибутом \textit{ключевой sc-элемент'}.}

\scnheader{блокировка*}
\scniselement{бинарное отношение}
\scnexplanation{Отношение \textbf{\textit{блокировка*}} связывает знаки \textit{действий в sc-памяти} со знаками \textit{структур} (ситуативных), которые содержат элементы, заблокированные на время выполнения данного действия или на какую-то часть этого периода. Каждая такая \textit{структура} принадлежит какому-либо из \textit{типов блокировки}.

Первым компонентом связок отношения \textbf{\textit{блокировка*}} является знак \textit{действия в sc-памяти}, вторым – знак заблокированной \textit{структуры}.}
\scnrelfrom{типичная семантическая окрестность}{
\scnfilelong{
\begin{figure}[H]
\centering
\includegraphics[width=1\linewidth]{figures/sd_agents/lock.png}
\end{figure}
}}

\scnheader{тип блокировки}
\scnexplanation{Множество \textbf{\textit{тип блокировки}} содержит все возможные классы блокировок, т.е. \textit{sc-структуры}, содержащие \textit{sc-элементы}, заблокированные каким-либо \textit{sc-агентом} на время выполнения им некоторого \textit{действия в sc-памяти}.}
\scnhaselement{полная блокировка}
\scnhaselement{блокировка на любое изменение}
\scnhaselement{блокировка на удаление}

\scnheader{полная блокировка}
\scnexplanation{Каждая \textit{структура}, принадлежащая множеству \textbf{\textit{полная блокировка}} содержит \textit{sc-элементы}, просмотр и изменение (удаление, добавление инцидентных \textit{sc-коннекторов}, удаление самих \textit{sc-элементов}, изменение содержимого в случае файла) которых запрещены всем \textit{sc-агентам}, кроме собственно \textit{sc-агента}, выполняющего соответствующее данной структуре \textit{действие в sc-памяти}, связанное с ней отношением \textit{блокировка*}.

Для того, чтобы исключить возможность реализации \textit{sc-агентов}, которые могут внести изменения в конструкции, описывающие блокировки других \textit{sc-агентов}, все элементы этих конструкций, в том числе, сам знак \textit{структуры}, содержащей заблокированные \textit{sc-элементы} (принадлежащей как множеству \textbf{\textit{полная блокировка}}, так и любому другому \textit{типу блокировки}) и связки отношения \textit{блокировка*}, связывающие эту \textit{структуру} и конкретное \textit{действие в sc-памяти}, добавляются в \textbf{\textit{полную блокировку}}, соответствующую данному \textit{действию в sc-памяти}. Таким образом, каждой \textbf{\textit{полной блокировке}} соответствует петля принадлежности, связывающая ее знак с самим собой.}

\scnheader{блокировка на любое изменение}
\scnexplanation{Каждая \textit{структура}, принадлежащая множеству \textbf{\textit{блокировка на любое изменение}} содержит \textit{sc-элементы}, изменение (физическое удаление, добавление инцидентных \textit{sc-коннекторов}, физическое удаление самих \textit{sc-элементов}, изменение содержимого в случае файла) которых запрещено всем \textit{sc-агентам}, кроме собственно \textit{sc-агента}, выполняющего соответствующее данной структуре \textit{действие в sc-памяти}, связанное с ней отношением \textit{блокировка*}. Однако не запрещен просмотр (чтение) этих \textit{sc-элементов} любым \textit{sc-агентом}.}

\scnheader{блокировка на удаление}
\scnexplanation{Каждая \textit{структура}, принадлежащая множеству \textbf{\textit{блокировка на удаление}} содержит \textit{sc-элементы}, удаление которых запрещено всем \textit{sc-агентам}, кроме собственно \textit{sc-агента}, выполняющего соответствующее данной структуре \textit{действие в sc-памяти}, связанное с ней отношением \textit{блокировка*}. Однако не запрещен просмотр (чтение) этих \textit{sc-элементов} любым \textit{sc-агентом}, добавление инцидентных sc-коннекторов.}

\scnauthorcomment{в диссертации Шункевича есть больше примеров и правил про блокировки, не знаю, насколько это нужно сейчас}

\scnendstruct

\end{SCn}

\scsection{Предметная область и онтология Базового языка программирования ostis-систем}
\label{sec:sd_scp}
\begin{SCn}
\scnsectionheader{Предметная область и онтология Базового языка программирования ostis-систем}
\begin{scnsubstruct}
	\begin{scnrelfromlist}{дочерний раздел}
		\scnitem{\nameref{sd_scp_denote_sem}}
		\scnitem{\nameref{sd_scp_oper_sem}}
	\end{scnrelfromlist}
	
\scnheader{Предметная область Базового языка программирования ostis-систем (языка SCP --- Semantic Code Programming)}
\scnidtf{Предметная область Базового языка программирования ostis-систем}
\scnidtf{Предметная область Языка SCP}
\scntext{примечание}{В данную предметную область включаются все тексты программ Языка SCP. В ней исследуется типология операторов этих программ и заданные на них отношения.}
\scniselement{предметная область}
\begin{scnhaselementrole}{максимальный класс объектов исследования}
	{scp-программа}
\end{scnhaselementrole}
\begin{scnhaselementrolelist}{класс объектов исследования}
	%TODO: check by human--->
	\scnitem{агентная scp-программа}
	\scnitem{scp-процесс}
	\scnitem{scp-оператор}
	\scnitem{атомарный тип scp-оператора}
	%<---TODO: check by human
\end{scnhaselementrolelist}
\begin{scnhaselementrolelist}{исследуемое отношение}
	%TODO: check by human--->
	\scnitem{начальный оператор\scnrolesign}
	\scnitem{параметр scp-программы\scnrolesign}
	\scnitem{in-параметр\scnrolesign}
	\scnitem{out-параметр\scnrolesign}
	\scnitem{scp-операнд\scnrolesign}
	%<---TODO: check by human
\end{scnhaselementrolelist}

\scnheader{Язык SCP}
\scnidtftext{часто используемый sc-идентификатор}{scp-программа}
\scntext{пояснение}{В качестве базового языка для описания программ обработки текстов\textit{SC-кода} предлагается \textit{Язык SCP}.\\
	\textit{Язык SCP} --- это графовый язык процедурного программирования, предназначенный для эффективной обработки \textit{sc-текстов}. \textit{Язык SCP} является языком параллельного асинхронного программирования.\\
	Языком представления данных для текстов \textit{Языка SCP} (\textit{scp-программ}) является \textit{SC-код} и, соответственно, любые варианты его внешнего представления. \textit{Язык SCP} сам построен на основе \textit{SC-кода}, в следствие чего \textit{scp-программы} сами по себе могут входить в состав обрабатываемых данных для \textit{scp-программ}, в т.ч. по отношению к самим себе. Таким образом, \textit{язык SCP} предоставляет возможность построения реконфигурируемых программ. Однако для обеспечения возможности реконфигурирования программы непосредственно в процессе ее интерпретации необходимо на уровне интерпретатора \textit{Языка SCP (Aбстрактной scp-машины)} обеспечить уникальность каждой исполняемой копии исходной программы. Такую исполняемую копию, сгенерированную на основе \textit{scp-программы}, будем называть \textit{scp-процессом}. Включение знака некоторого \textit{действия в sc-памяти} во множество \textit{scp-процессов} гарантирует тот факт, что в декомпозиции данного действия будут присутствовать только знаки элементарных действий (\textit{scp-операторов}), которые может интерпретировать реализация \textit{Aбстрактной scp-машины} (интерпретатора scp-программ).\\
	\textit{Язык SCP} рассматривается как ассемблер для семантического компьютера.}

\scnheader{Базовая модель обработки sc-текстов}
\begin{scnreltoset}{объединение}
	%TODO: check by human--->
	\scnitem{Предметная область Базового языка программирования ostis-систем}
	\scnitem{Модель Абстрактной scp-машины}
	%<---TODO: check by human
\end{scnreltoset}
\begin{scnrelfromset}{особенности}
	%TODO: check by human--->
	\scnfileitem{Тексты программ \textit{Языка SCP} записываются при помощи тех же унифицированных семантических сетей, что и обрабатываемая информация, таким образом, можно сказать, что \textit{Синтаксис языка SCP} на базовом уровне совпадает с \textit{Синтаксисом SC-кода}.}
	\scnfileitem{Подход к интерпретации \textit{scp-программ} предполагает создание при каждом вызове \textit{scp-программы} уникального \textit{scp-процесса}.}
	%<---TODO: check by human
\end{scnrelfromset}
\begin{scnrelfromset}{достоинства}
	%TODO: check by human--->
	\scnfileitem{Одновременно в общей памяти могут выполняться несколько независимых\textit{sc-агентов}, при этом разные копии \textit{sc-агентов} могут выполняться на разных серверах, за счет распределенной реализации интерпретатора sc-моделей (\textit{платформы реализации sc-моделей компьютерных систем}). Более того, \textit{Язык SCP} позволяет осуществлять параллельные асинхронные вызовы подпрограмм с последующей синхронизацией, и даже параллельно выполнять операторы в рамках одной \textit{scp-программы}.}
	\scnfileitem{Перенос \textit{sc-агента} из одной системы в другую заключается в простом переносе фрагмента базы знаний, без каких-либо дополнительных операций, зависящих от платформы интерпретации.}
	\scnfileitem{Тот факт, что спецификации \textit{sc-агентов} и их программы могут быть записаны на том же языке, что и обрабатываемые знания, существенно сокращает перечень специализированных средств, предназначенных для проектирования машин обработки знаний, и упрощает их разработку за счет использования более универсальных компонентов.}
	\scnfileitem{Тот факт, что для интерпретации \textit{scp-программы} создается соответствующий ей уникальный \textit{\mbox{scp-процесс}}, позволяет по возможности оптимизировать план выполнения перед его реализацией и даже непосредственно в процессе выполнения без потенциальной опасности испортить общий универсальный алгоритм всей программы. Более того, такой подход к проектированию и интерпретации программ позволяет говорить о возможности создания самореконфигурируемых программ.}
	%<---TODO: check by human
\end{scnrelfromset}

\scnheader{Абстрактная scp-машина}
\scnrelfrom{модель}{Модель Абстрактной scp-машины}
\scntext{примечание}{\textit{Абстрактная scp-машина} представляет собой интерпретатор \textit{scp-программ}, который должен являться частью \textit{платформы интерпретации sc-моделей компьютерных систем} (хотя в общем случае могут существовать варианты платформы, не содержащие такого интерпретатора, что, однако, не позволит использовать достоинства предлагаемой базовой модели}

\scnheader{scp-программа}
\scnsubset{программа в sc-памяти}
\scntext{пояснение}{Каждая \textbf{\textit{scp-программа}} представляет собой \textit{обобщенную структуру}, описывающую один из вариантов декомпозиции действий некоторого класса, выполняемых в sc-памяти. Знак \textit{sc-переменной}, соответствующей конкретному декомпозируемому действию является в рамках \textbf{\textit{scp-программы}} \textit{ключевым sc-элементом\scnrolesign}. Также явно указывается принадлежность данного знака множеству \textit{scp-процессов}.\\
	Принадлежность некоторого действия множеству \textit{scp-процессов} гарантирует тот факт, что в декомпозиции данного действия будут присутствовать только знаки элементарных действий (\textit{scp-операторов}), которые может интерпретировать реализация абстрактной scp-машины.\\
	Таким образом, каждая \textbf{\textit{scp-программа}} описывает в обобщенном виде декомпозицию некоторого \textit{\mbox{scp-процесса}} на взаимосвязанные \textit{scp-операторы}, с указанием, при их наличии, аргументов для данного \textit{scp-процесса}.\\
	По сути каждая \textbf{\textit{scp-программа}} представляет собой описание последовательности элементарных операций, которые необходимо выполнить над семантической сетью, чтобы выполнить более сложное действие некоторого класса.}
\scnrelfrom{описание примера}{\scnfileimage[20em]{Contents/part_ps/images/sd_scp/program_example.png}}
	\begin{scnindent}
		\scntext{пояснение}{В приведенном примере показана \textit{scp-программа}, состоящая из трех \textit{scp-операторов}. Данная программа проверяет, содержится ли в заданном множестве (первый параметр) заданный элемент (второй параметр), и, если нет, то добавляет его в это множество.}
	\end{scnindent}

\begin{scnhaselementrolelist}{пример}
	\scnitem{\_scp\_process}
	\scnisvarelement{scp-процесс}
	\scnhasvarelementrole{1;in-параметр}{\_set1}
	\scnhasvarelementrole{2;in-параметр}{\_element1}
	\scnvarrelto{декомпозиция действия}{\_...}
	\begin{scnindent}
		\scnhasvarelementrole{1}{\_ operator1}
		\begin{scnindent}
			\scnisvarelement{searchElStr3}
			\scnhasvarelementrole{1; scp-операнд с заданным значением; scp-константа}{\_set1}
			\scnhasvarelementrole{2; scp-операнд со свободным значением; scp-переменная; sc-дуга основного вида}{\_arc1}
			\scnhasvarelementrole{3; scp-операнд с заданным значением; scp-константа}{\_element1}
			\scnvarrelfrom{последовательность действий при отрицательном результате}{\_operator2}
			\scnvarrelfrom{последовательность действий при положительном результате}{\_operator3}
		\end{scnindent}
		\scnhasvarelement{\_operator2}
		\begin{scnindent}
			\scnisvarelement{genElStr3}
			\scnhasvarelementrole{1:: scp-операнд с заданным значением; scp-константа}{\_set1}
			\scnhasvarelementrole{2:: scp-операнд со свободным значением; scp-переменная; sc-дуга основного вида}{\_arc1}
			\scnhasvarelementrole{3:: scp-операнд с заданным значением; scp-константа}{\_element1}
			\scnvarrelfrom{следующий оператор}{\_operator3}
		\end{scnindent}
		\scnhasvarelement{\_operator3}
		\begin{scnindent}
			\scnisvarelement{return}
		\end{scnindent}
	\end{scnindent}
\end{scnhaselementrolelist}
\end{scnsubstruct}

\scnheader{агентная scp-программа}
\scnsubset{scp-программа}
\scntext{пояснение}{\textbf{\textit{агентные scp-программы}} представляют собой частный случай \textit{scp-программ} вообще, однако заслуживают отдельного рассмотрения, поскольку используются наиболее часто. \textit{Scp-программы} данного класса представляют собой реализации программ агентов обработки знаний и имеют жестко фиксированный набор параметров. Каждая такая программа имеет ровно два \textit{in-параметра\scnrolesign}. Значение первого параметра является знаком бинарной ориентированной пары, являющейся вторым компонентом связки отношения \textit{первичное условие инициирования*} для абстрактного \textit{sc-агента}, во множество \textit{программ sc-агента*} которого входит рассматриваемая \textbf{\textit{агентная scp-программа}}, и, по сути, описывает класс событий, на которые реагирует указанный sc-агент.\\
	Значением второго параметра является \textit{sc-элемент}, с которым непосредственно связано событие, в результате возникновения которого был инициирован соответствующий \textit{sc-агент}, т.е., например, сгенерированная либо удаляемая \textit{sc-дуга} или \textit{sc-ребро}.}

\scnheader{абстрактный sc-агент, реализуемый на Языке SCP}
\begin{scnrelfromset}{принципы реализации}
	%TODO: check by human--->
	\scnfileitem{Общие принципы организации взаимодействия \textit{sc-агентов} и пользователей \textit{ostis-системы} через общую\textit{sc-память}.}
	\scnfileitem{В результате появления в sc-памяти некоторой конструкции,удовлетворяющей условию инициирования какого-либо \textit{абстрактногоsc-агента}, реализованного при помощи \textit{Языка SCP}, в \textit{sc-памяти} генерируется и инициируется \textit{scp-процесс}. В качестве шаблона для генерации используется \textit{агентная scp-программа}, указанная во множестве программ соответствующего \textit{абстрактного sc-агента}.}
	\scnfileitem{Каждый такой \textit{scp-процесс}, соответствующий некоторой \textit{агентной scp-программе}, может быть связан с набором структур, описывающих блокировки различных типов. Таким образом, синхронизация взаимодействия параллельно выполняемых \textit{scp-процесcов} осуществляется так же, как и в случае любых других \textit{действий в sc-памяти}.}
	\scnfileitem{В рамках \textit{scp-процесса} могут создаваться дочерние \textit{scp-процессы}, однако синхронизация между ними при необходимости осуществляется посредством введения дополнительных внутренних блокировок. Таким образом, каждый \textit{scp-процесс} с точки зрения \textit{процессов в sc-памяти} является атомарным и законченным актом деятельности некоторого \textit{sc-агента}.}
	\scnfileitem{Во избежание нежелательных изменений в самом теле \textit{scp-процесса}, вся конструкция, сгенерированная на основе некоторой \textit{scp-программы} (весь текст \textit{scp-процесса}), должна быть добавлена в \textit{полную блокировку}, соответствующую данному \textit{scp-процессу}.}
	\scnfileitem{Все конструкции, сгенерированные в процессе выполнения \textit{scp-процесса}, автоматически попадают в \textit{полную блокировку}, соответствующую данному \textit{scp-процессу}. Дополнительно следует отметить, что знак самой этой структуры и вся метаинформация о ней также включаются в эту структуру.}
	\scnfileitem{При необходимости можно вручную разблокировать или заблокировать некоторую конструкцию каким-либо типом блокировки, используя соответствующие \textit{scp-операторы} класса \textit{scp-оператор управления блокировками}.}
	\scnfileitem{После завершения выполнения некоторого \textit{scp-процесса} его текст как правило, удаляется из \textit{\mbox{sc-памяти}}, а все заблокированные конструкции освобождаются (разрушаются знаки структур, обозначавших блокировки).}
	\scnfileitem{Несмотря на то, что каждый \textit{scp-оператор} представляет собой атомарное \textit{действие в sc-памяти}, дополнительные блокировки, соответствующие одному оператору не вводятся, чтобы избежать громоздкости и избытка дополнительных системных конструкций, создаваемых при выполнении некоторого \textit{scp-процесса}. Вместо этого используются блокировки, общие для всего \textit{scp-процесса}. Таким образом, агенты \textit{Абстрактной scp-машины} при интерпретации \textit{scp-операторов} работают только с учетом блокировок, общих для всего интерпретируемого \textit{scp-процесса}.}
	\scnfileitem{Как правило, частный \textit{класс действий}, соответствующий конкретной \textit{scp-программе} явно не вводится, а используется более общий класс \textit{scp-процесс}, за исключением тех случаев, когда введение специального \textit{класса действий} необходимо по каким-либо другим соображениям.}
	%<---TODO: check by human
\end{scnrelfromset}

\scnheader{scp-процесс}
\scntext{пояснение}{Под \textbf{\textit{scp-процессом}} понимается некоторое \textit{действие в sc-памяти}, однозначно описывающее конкретный акт выполнения некоторой \textit{scp-программы} для заданных исходных данных. Если \textit{scp-программа} описывает алгоритм решения какой-либо задачи в общем виде, то \textit{scp-процесс} обозначает конкретное действие, реализующее данный алгоритм для заданных входных параметро
	По сути, \textbf{\textit{scp-процесс}} представляет собой уникальную копию, созданную на основе \textit{scp-программы}, в которой каждой \textit{sc-переменной}, за исключением \textit{scp-переменных\scnrolesign}, соответствует сгенерированная \textit{sc-константа}.\\
	Принадлежность некоторого действия множеству \textit{scp-процессов} гарантирует тот факт, что в декомпозиции данного действия будут присутствовать только знаки элементарных действий (\textit{scp-операторов}), которые может интерпретировать реализация \textit{Абстрактной scp-машины}.}
\begin{scnrelfromvector}{пример выполнения}
	%TODO: check by human--->
	\scnitem{\scnfileimage[20em]{Contents/part_ps/images/sd_scp/process_example.png}}
		\begin{scnindent}
			\scntext{пояснение}{Осуществляется вызов \textit{scp-программы}. Генерируется соответствующий \textit{scp-процесс}. Происходит инициирование начального оператора scp-процесса \textit{Operator1}.} 
		\end{scnindent}
	\scnitem{\scnfileimage[20em]{Contents/part_ps/images/sd_scp/process_example2.png}}
		\begin{scnindent}	
			\scntext{пояснение}{Оператор \textit{Operator1} оказался безуспешно выполненным. Производится инициирование \textit{\mbox{scp-оператора} генерации трёхэлементной конструкции} \textit{Operator2}.}
		\end{scnindent}
	\scnitem{\scnfileimage[20em]{Contents/part_ps/images/sd_scp/process_example3.png}}
		\begin{scnindent}	
			\scntext{пояснение}{Оператор \textit{Operator2} выполнился. Производится инициирование \textit{scp-оператора завершения выполнения программы} \textit{Operator3}.}
		\end{scnindent}
	\scnitem{\scnfileimage[20em]{Contents/part_ps/images/sd_scp/process_example4.png}}
		\begin{scnindent}
			\scntext{пояснение}{Оператор \textit{Operator3} выполнился. Выполнение \textit{scp-процесса} завершается.}
		\end{scnindent}
%<---TODO: check by human

\end{scnrelfromvector}
\end{SCn}


\scsubsection{Предметная область и онтология денотационной семантики Базового языка программирования ostis-систем}
\label{sec:sd_scp_denote_sem}
\begin{SCn}

\scnsectionheader{\currentname}

\scnstartsubstruct

\scnheader{Предметная область денотационной семантики языка SCP}
\scniselement{предметная область}
\scnsdmainclasssingle{***}
\scnsdclass{***}

\scnauthorcomment{По scp-операторам есть подробное описание буквально по каждому классу с примерами (некоторые устарели, но можно подправить), там методичка около 100 страниц, нужно это сюда и в какой степени}

\scnheader{scp-оператор}
\scnrelto{включение}{действие в sc-памяти}
\scnrelto{семейство подмножеств}{атомарный тип scp-оператора}
\scnreltoset{разбиение}{scp-оператор генерации конструкций\\
    \scnaddlevel{1}
    \scnreltoset{разбиение}{scp-оператор генерации конструкции по произвольному образцу;scp-оператор генерации пятиэлементной конструкции;scp-оператор генерации трехэлементной конструкции;scp-оператор генерации одноэлементной конструкции}
    \scnaddlevel{-1}
;scp-оператор ассоциативного поиска конструкций\\
    \scnaddlevel{1}
    \scnreltoset{разбиение}{scp-оператор поиска конструкции по произвольному образцу;scp-оператор поиска пятиэлементной конструкции с формированием множеств;scp-оператор поиска трехэлементной конструкции с формированием множеств;scp-оператор поиска пятиэлементной конструкции;scp-оператор поиска трехэлементной конструкции}
    \scnaddlevel{-1}
;scp-оператор удаления конструкций\\
    \scnaddlevel{1}
    \scnreltoset{разбиение}{scp-оператор удаления множества элементов трехэлементной конструкции;scp-оператор удаления одноэлементной конструкции;scp-оператор удаления пятиэлементной конструкции;scp-оператор удаления трехэлементной конструкции}
    \scnaddlevel{-1}
;scp-оператор проверки условий\\
    \scnaddlevel{1}
    \scnreltoset{разбиение}{scp-оператор сравнения числовых содержимых файлов;scp-оператор проверки равенства числовых содержимых файлов;scp-оператор проверки совпадения значений операндов;scp-оператор проверки наличия содержимого у файла;scp-оператор проверки наличия значения у переменной;scp-оператор проверки типа sc-элемента}
    \scnaddlevel{-1}
;scp-оператор управления значениями операндов\\
    \scnaddlevel{1}
    \scnreltoset{разбиение}{scp-оператор удаления значения переменной;scp-оператор присваивания значения переменной}
    \scnaddlevel{-1}
;scp-оператор управления scp-процессами\\
    \scnaddlevel{1}
    \scnreltoset{разбиение}{scp-оператор удаления значения переменной;scp-оператор завершения выполнения программы;конъюнкция предшествующих scp-операторов;scp-оператор ожидания завершения выполнения множества scp-программ;scp-оператор ожидания завершения выполнения scp-программы;scp-оператор асинхронного вызова подпрограммы}
    \scnaddlevel{-1}
;scp-оператор управления событиями\\
    \scnaddlevel{1}
    \scnreltoset{разбиение}{scp-оператор ожидания события}
    \scnaddlevel{-1}
;scp-оператор обработки содержимых файлов\\
    \scnaddlevel{1}
    \scnreltoset{разбиение}{scp-оператор вычисления арксинуса числового содержимого файла;scp-оператор вычисления арккосинуса числового содержимого файла;scp-оператор деления числовых содержимых файлов;scp-оператор умножения числовых содержимых файлов;scp-оператор вычитания числовых содержимых файлов;scp-оператор сложения числовых содержимых файлов;scp-оператор вычисления тангенса числового содержимого файла;scp-оператор вычисления косинуса числового содержимого файла;scp-оператор вычисления синуса числового содержимого файла;scp-оператор вычисления логарифма числового содержимого файла;scp-оператор возведения числового содержимого файла в степень;scp-оператор удаления содержимого файла;scp-оператор копирования содержимого файла;scp-оператор нахождения остатка от деления числовых содержимых файлов;scp-оператор нахождения целой части от деления числовых содержимых файлов;scp-оператор вычисления арктангенса числового содержимого файла;scp-оператор перевода в верхний регистр строкового содержимого файла;scp-оператор перевода в верхний регистр строкового содержимого файла;scp-оператор замены определенной части строкового содержимого файла на содержимое указанной файла;scp-оператор проверки совпадения конца строкового содержимого файла со строковом содержимым другого файла;scp-оператор проверки совпадения начальной части строкового содержимого файла со строковом содержимым другого файла;scp-оператор получения части строкового содержимого файла по индексам;scp-оператор поиска строкового содержимого файла в строковом содержимом другого файла;scp-оператор вычисления длины строкового содержимого файла;scp-оператор разбиения строки на подстроки;scp-оператор лексикографического сравнения строковых содержимых файлов;scp-оператор проверки равенства строковых содержимых файлов}
    \scnaddlevel{-1}
;scp-оператор управления блокировками\\
    \scnaddlevel{1}
    \scnreltoset{разбиение}{scp-оператор снятия всех блокировок данного scp-процесса;scp-оператор снятия блокировки с sc-элемента;scp-оператор установки полной блокировки на sc-элемент;scp-оператор установки блокировки на изменение sc-элемента;scp-оператор установки блокировки на удаление sc-элемента;scp-оператор снятия блокировки со структуры;scp-оператор установки полной блокировки на структуру;scp-оператор установки блокировки на изменение структуры;scp-оператор установки блокировки на удаление структуры}}

\scnresetlevel

\scnheader{scp-операнд’}
\scnrelto{включение}{аргумент действия'}
\scniselement{неосновное понятие}
\scniselement{ролевое отношение}
\scnreltoset{разбиение}{scp-константа';scp-переменная'}
\scnreltoset{разбиение}{scp-операнд с заданным значением';scp-операнд со свободным значением'}
\scnreltoset{разбиение}{константный sc-элемент';переменный sc-элемент'}
\scnrelfromlist{включение}{формируемое множество'\\
    \scnaddlevel{1}
    \scnreltoset{разбиение}{формируемое множество 1';формируемое множество 2';формируемое множество 3';формируемое множество 4';формируемое множество 5'}
    \scnaddlevel{-1}
;удаляемый sc-элемент';тип sc-элемента'\\
    \scnaddlevel{1}
    \scnreltoset{разбиение}{sc-узел'\\
        \scnaddlevel{1}
        \scnreltoset{разбиение}{структура';отношение'\\
            \scnaddlevel{1}
            \scnrelfrom{включение}{ролевое отношение'}
            \scnaddlevel{-1}
        ;класс'}
        \scnaddlevel{-1}
    ;sc-дуга'\\
        \scnaddlevel{1}
        \scnreltoset{разбиение}{sc-дуга общего вида';sc-дуга принадлежности'\\
            \scnaddlevel{1}
            \scnrelfrom{включение}{sc-дуга основного вида'}
            \scnreltoset{разбиение}{позитивная sc-дуга принадлежности';негативная sc-дуга принадлежности';нечеткая sc-дуга принадлежности'}
            \scnreltoset{разбиение}{временная sc-дуга принадлежности';постоянная sc-дуга принадлежности'}
            \scnaddlevel{-1}}
        \scnaddlevel{-1}
    ;sc-ребро';файл'}
    \scnaddlevel{-1}}
\scnexplanation{Ролевое отношение \textit{scp-операнд’} является неосновным понятием и указывает на принадлежность аргументов \textit{scp-оператору}. Помимо указания какого-либо класса \textit{scp-операндов’} порядок аргументов \textit{scp-оператора} дополнительно уточняется \textit{ролевыми отношениями 1'}, \textit{2'} и т. д.}

\scnheader{scp-константа'}
\scnexplanation{В рамках \textit{scp-программы} \textit{scp-константы'} явно участвуют в \textit{\mbox{scp-операторах}} в качестве элементов (в теоретико-множественном смысле) и напрямую обрабатываются при интерпретации \textit{scp-программы}. Константами в рамках \textit{scp-программы} могут быть \textit{sc-элементы} любого типа, как \textit{\mbox{sc-константы}}, так и \textit{sc-переменные}. Константа в рамках \textit{scp-программы} остается неизменной в течение всего срока интерпретации. Константа \textit{\mbox{scp-программы}} может быть рассмотрена как переменная, значение которой совпадает с самой переменной в каждый момент времени, и изменено быть не может. Таким образом, далее будем считать, что \textit{scp-константа'} и ее значение это одно и то же. Каждый \textit{in-параметр'} при интерпретации каждой конкретной копии \textit{scp-программы} становится \textit{scp-константой'} в рамках всех ее операторов, хотя в исходном теле данной программы в каждом из этих операторов он является \textit{scp-переменной’}.}

\scnheader{scp-переменная'}
\scnexplanation{В рамках \textit{scp-программы} textit{scp-переменные'} не обрабатываются явно при интерпретации, обрабатываются значения переменных. Каждая переменная \textit{scp-программы} может иметь одно значение в каждый момент времени, т. е. представляет собой ситуативный \textit{синглетон}, элементом которого является текущее значение \textit{scp-переменной'}. Значение каждой \textit{scp-переменной’} может меняться в ходе интерпретации \textit{scp-программы}. При этом интерпретатор при обработке \textit{scp-оператора} работает непосредственно со значениями \textit{\mbox{scp-переменных’}}, а не самими \textit{scp-переменными’} (которые также являются узлами той же семантической сети).}

\scnheader{scp-операнд с заданным значением'}
\scnexplanation{Значение операндов, помеченных ролевым отношением \textit{scp-операнд с заданным значением'}, считается заданным в рамках текущего \textit{scp-оператора}. Данное значение учитывается при выполнении \textit{scp-оператора} и остается неизменным после окончания выполнения \textit{scp-оператора}. Каждая \textit{scp-константа'} по умолчанию рассматривается как \textit{scp-операнд с заданным значением'}, в связи с чем явное использование данного ролевого отношения в таком случае является избыточным. В таком случае в качестве значения рассматривается непосредственно сам операнд. В случае если отношением \textit{\mbox{scp-операнд} с заданным значением'} помечена \textit{scp-переменная'}, то осуществляется попытка поиска значения для данной \textit{scp-переменной'} (ее элемента). Если попытка оказалась безуспешной, то возникает ошибка времени выполнения, которая должна быть обработана соответствующим образом.

Любой \textit{scp-операнд с заданным значением'} независимо от конкретного типа \textit{scp-оператора} может быть \textit{scp-переменной'}.}

\scnheader{scp-операнд со свободным значением'}
\scnexplanation{Значение операндов, помеченных ролевым отношением \textit{scp-операнд со свободным значением'}, считается свободным (не заданным заранее) в рамках текущего \textit{scp-оператора}. В начале выполнения \textit{scp-оператора} связь между \textit{scp-переменной'}, помеченной данным ролевым отношением, и ее элементом (значением) всегда удаляется. В результате выполнения данного оператора может быть либо сгенерировано новое значение \textit{scp-переменной'}, либо не сгенерировано, тогда \textit{scp-переменная'} будет считаться не имеющей значения. Ни одна \textit{scp-константа'} не может быть помечена как \textit{scp-операнд со свободным значением'}, поскольку константа не может изменять свое значение в ходе интерпретации \textit{scp-программы}.

Таблица \ref{table_operands_roles} показывает возможные сочетания различных ролевых отношений, указывающих роль операнда в рамках scp-оператора:

\begin{table} [H]
  \caption{Роли операндов в рамках scp-оператора}\label{table_operands_roles}
\begin{tabularx}{\hsize}{| p{43mm} | X | X |}
  \hline
  \textbf{Тип значения}
  & \multirow{2}{*}{\textbf{\shortstack[l]{scp-операнд с\\ заданным значением'}}} & \multirow{2}{*}{\textbf{\shortstack[l]{scp-операнд со\\ свободным значением'}}} \\
  \cline{0-0}
  \textbf{Константность} & & \\
\hline
 
\textbf{scp-константа'} & Разрешено, может быть опущено & Запрещено \\
\hline
\textbf{scp-переменная'} & Разрешено, значение останется неизменным & Разрешено, значение переменной будет изменено  либо потеряно\\
\hline
\end{tabularx}
\end{table}}

\scnheader{тип \mbox{sc-элемента'}}
\scnexplanation{Ролевое отношение \textit{тип \mbox{sc-элемента'}} используется для уточнения типа \textit{sc-элемента}, выступающего в роли значения некоторого операнда. \textit{тип \mbox{sc-элемента'}} имеет смысл указывать только для операндов, помеченных как \textit{scp-операнд со свободным значением'}, тогда данное уточнение типа \textit{\mbox{sc-элемента}} будет использовано для сужения области поиска либо уточнения параметров генерации каких-либо конструкций. Значением \textit{scp-операндов с заданным значением'} является конкретный, известный на момент начала выполнения \textit{scp-оператора sc-элемент} с конкретным типом, не зависящим от указания \textit{типа sc-элемента'}, в связи с чем использование ролевого отношения \textit{тип sc-элемента'} в данном случае является некорректным.

Допускается использование комбинаций семантически непротиворечащих друг другу подмножеств указанного отношения. Например, допускается комбинация \textit{константный sc-элемент'} и \textit{sc-дуга общего вида'}, но не допускается комбинация \textit{sc-узел'} и \textit{sc-дуга'}.}

\scnheader{sc-дуга основного вида'}
\scneq{(константный sc-элемент' $\cap$ позитивная sc-дуга принадлежности' $\cap$ постоянная sc-дуга принадлежности')}

\scnheader{формируемое множество'}
\scnexplanation{Ролевое отношение \textit{формируемое множество'} используется для указания того факта, что в результате выполнения \textit{scp-оператора} должно быть сформировано либо дополнено некоторое множество \textit{sc-элементов}, являющееся значением одного из операндов данного \textit{scp-оператора}. При этом если данный операнд помечен как \textit{scp-операнд со свободным значением'}, то множество будет сформировано с нуля (сгенерирован новый \textit{sc-элемент}, обозначающий данное множество), в противном случае уже существующее множество может быть дополнено. Использование данного ролевого отношения предполагает, что при его отсутствии множество бы не формировалось, а значением указанного операнда стал бы произвольный \textit{sc-элемент} из данного множества. 

Ролевое отношение \textit{формируемое множество'} без уточнения порядкового номера используется только в \textit{scp-операторах обработки произвольных конструкций}. Для явного указания номера операнда, которому соответствует \textit{формируемое множество'}, используются подмножества данного ролевого отношения, аналогичные ролевым отношениям, задающим порядок элементов в ориентированном множестве (\textit{1', 2', 3'} и т. д.), например \textit{формируемое множество 1'}, \textit{формируемое множество 2'} и т. д. Указанные ролевые отношения используются только в \textit{scp-операторах поиска конструкций с формированием множеств}.}

\scnheader{удаляемый sc-элемент'}
\scnexplanation{Ролевое отношение \textit{удаляемый sc-элемент'} используется для указания тех операндов, значение которых должно быть удалено в процессе выполнения \textit{scp-операторов удаления}. Данным ролевым отношением может быть помечен как \textit{scp-операнд с заданным значением'}, так и \textit{scp-операнд со свободным значением'}. При этом удаляемым \textit{sc-элементом} может быть как \textit{scp-константа'}, так и \textit{scp-переменная'} (в случае \textit{scp-переменной'} удаляется не только связка принадлежности между этой \textit{scp-переменной'} и ее значением, но и непосредственно сам \textit{sc-элемент}, являющийся значением).}

\scnendstruct

\end{SCn}



\scsubsection{Предметная область и онтология синтаксиса Базового языка программирования ostis-систем}
\label{syntax_basic_program_lang}


\scsubsection{Предметная область и онтология операционной семантики Базового языка программирования ostis-систем}
\label{sec:sd_scp_oper_sem}
\begin{SCn}

\scnsectionheader{\currentname}

\scnstartsubstruct

\scnheader{Предметная область операционной семантики языка SCP}
\scniselement{предметная область}
\scnsdmainclasssingle{Абстрактная scp-машина}
\scnhaselementlist{ключевой объект исследования}{Абстрактный sc-агент создания scp-процессов;Абстрактный sc-агент интерпретации scp-операторов;Абстрактный sc-агент синхронизации процесса интерпретации scp-программ;Абстрактный sc-агент уничтожения scp-процессов;Абстрактный sc-агент синхронизации событий в sc-памяти и ее реализации;Абстрактный sc-агент трансляции сформированной спецификации события в sc-памяти во внутреннее представление;Абстрактный sc-агент обработки события в sc-памяти, инициирующего агентную scp-программу}

\scnheader{Абстрактная scp-машина}
\scnreltoset{декомпозиция абстрактного sc-агента}{Абстрактный sc-агент создания scp-процессов;Абстрактный sc-агент интерпретации scp-операторов;Абстрактный sc-агент синхронизации процесса интерпретации scp-программ;Абстрактный sc-агент уничтожения scp-процессов;Абстрактный sc-агент синхронизации событий в sc-памяти и ее реализации}

\scnheader{Абстрактный sc-агент создания scp-процессов}
\scnexplanation{Задачей \textit{Абстрактного} \textit{sc-агента создания scp-процессов}
является создание \textit{scp-процессов}, соответствующих заданной
\textit{scp-программе}. Данный \textit{\mbox{sc-агент}} активируется при появлении
в \textit{sc-памяти} \textit{инициированного действия}, принадлежащего
классу \textit{действие интерпретации scp-программы}.

После проверки \textit{sc-агентом} условия инициирования выполняется
создание \textit{scp-процесса} с учетов конкретных параметров
интерпретации \textit{\mbox{scp-программы}}, после чего осуществляется поиск
\textit{начального оператора' \mbox{scp-процесса}} и добавление его во множество
\textit{настоящих сущностей}.}

\scnheader{Абстрактный sc-агент интерпретации scp-операторов}
\scnexplanation{Задачей \textit{Абстрактного sc-агента интерпретации scp-операторов}
является собственно интерпретация операторов \textit{scp-программы}, то
есть выполнение в \textit{sc-памяти} действий, описываемых конкретным
\textit{\mbox{scp-оператором}}. Данный \textit{sc-агент} активируется при появлении
в \textit{sc-памяти} \textit{scp-оператора}, принадлежащего классу
\textit{настоящих сущностей}. После выполнения действия, описываемого
\textit{scp-оператором}, \textit{scp-оператор} добавляется во множество
\textit{прошлых сущностей}. В случае когда семантика действия,
описываемого \textit{\mbox{scp-оператором}}, предполагает возможность ветвления
\textit{scp-программы} после выполнения данного \textit{\mbox{scp-оператора}}, то
используется одно из подмножеств класса \textit{выполненных действий --
безуспешно выполненное действие} или \textit{успешно выполненное
действие}.}

\scnheader{Абстрактный sc-агент синхронизации процесса интерпретации scp-программ}
\scnexplanation{Задачей \textit{Абстрактного sc-агента синхронизации процесса
интерпретации scp-программ} является обеспечение переходов между
\textit{scp-операторами} в рамках одного \textit{scp-процесса}. Данный
\textit{sc-агент} активизируется при добавлении некоторого
\textit{scp-оператора} во множество \textit{прошлых сущностей}. Далее
осуществляется переход по \textit{sc-дуге}, принадлежащей отношению
\textit{последовательность действий*} (или более частным отношениям, в
случае, если \textit{\mbox{scp-оператор}} был добавлен во множество \textit{успешно
выполненных действий} или \textit{безуспешно выполненных действий}). При
этом очередной \textit{scp-оператор} становится \textit{настоящей сущностью}
(активным \textit{scp-оператором}) в том случае, если хотя бы один
\textit{scp-оператор}, связанный с ним входящими \textit{sc-дугами},
принадлежащими отношению \textit{последовательность действий*} (или более
частным отношениям), стал \textit{прошлой сущностью} (или, соответственно,
подмножеством прошлых сущностей). В случае, когда необходимо дождаться
завершения выполнения всех предыдущих операторов, для синхронизации
используется оператор класса \textit{конъюнкция предшествующих
операторов}.}

\scnheader{Абстрактный sc-агент уничтожения scp-процессов}
\scnexplanation{Задачей \textit{Абстрактного sc-агента уничтожения scp-процессов}
является уничтожение \textit{scp-процесса}, т. е. удаление из
\textit{sc-памяти} всех \textit{sc-элементов}, его составляющих. Данный
\textit{sc-агент} активируется при появлении в \textit{sc-памяти}
\textit{scp-процесса}, принадлежащего множеству \textit{прошлых сущностей}.

При этом уничтожаемый \textit{scp-процесс} необязательно должен быть
полностью сформирован. Необходимость уничтожения не до конца
сформированного \textit{scp-процесса} может возникнуть в случае, если при
создании \textit{scp-процесса} возникли проблемы, не позволяющие
продолжить создание \textit{scp-процесса} и его выполнение.}

\scnheader{Абстрактный sc-агент синхронизации событий в sc-памяти и ее реализации}
\scnexplanation{Задачей \textit{Абстрактного sc-агента синхронизации событий в
sc-памяти и ее реализации} является обеспечение работы \textit{неатомарных
sc-агентов}, реализованных на \textit{языке SCP}.}
\scnreltoset{декомпозиция абстрактного sc-агента}{Абстрактный sc-агент трансляции сформированной спецификации события в sc-памяти во внутреннее представление;Абстрактный sc-агент обработки события в sc-памяти,
инициирующего агентную scp-программу}

\scnheader{Абстрактный sc-агент трансляции сформированной спецификации события в sc-памяти во внутреннее представление}
\scnexplanation{Задачей \textit{\textbf{Абстрактного sc-агента трансляции сформированной спецификации события в sc-памяти во внутреннее представление}}
является трансляция ориентированных пар, описывающих \textit{первичное
условие инициирования*} некоторого \textit{\mbox{sc-агента}} во внутреннее
представление элементарных событий на уровне \textit{\mbox{sc-хранилища}}, при
условии, что этот \textit{sc-агент} реализован на платформенно-независимом
уровне (с использованием \textit{языка SCP}). Условием инициирования
данного \textit{sc-агента} является появление в \textit{\mbox{sc-памяти}} нового
элемента множества \textit{активных sc-агентов}, для которого будет
найдена и протранслирована соответствующая ориентированная пара.}

\scnheader{Абстрактный sc-агент обработки события в sc-памяти,
инициирующего агентную scp-программу}
\scnexplanation{Задачей \textit{Абстрактного sc-агента обработки события в sc-памяти,
инициирующего агентную \mbox{scp-программу}}, является поиск \textit{агентной
scp-программы}, входящей во множество \textit{программ sc-агента*} для
каждого \textit{sc-агента}, первичное условие инициирования которого
соответствует событию, произошедшему в \textit{sc-памяти}, а также
генерация и инициирование действия, направленного на интерпретацию этой
программы. В результате работы данного \textit{sc-агента} в
\textit{sc-памяти} появляется \textit{инициированное действие},
принадлежащее классу \textit{действие} \textit{интерпретации scp-программы.}}

\bigskip
\scnendstruct \scnendcurrentsectioncomment

\end{SCn}



\scsection{Предметная область и онтология sc-языка вопросов}
\label{sd_sc_quest_lang}

\scsubsection{Предметная область и онтология синтаксиса sc-языка вопросов}
\label{sd_syntax_sc_quest_lang}

\scsubsection{Предметная область и онтология денотационной семантики sc-языка вопросов}
\label{sd_denot_sem_sc_quest_lang}


\scsubsection{Предметная область и онтология операционной семантики sc-языка вопросов}
\label{sd_operat_sem_sc_quest_lang}


\scsection{Предметная область и онтология операционной семантики логических sc-языков}
\label{sd_operat_sem_sc_logical_lang}


\scsection{Предметная область и онтология sc-языков программирования высокого уровня}
\label{sd_sc_program_lang}


\scsection{Предметная область и онтология sc-моделей искусственных нейронных сетей}
\label{sd_sc_model_art_neural_networks}

\scsubsection{Предметная область и онтология синтаксиса sc-моделей искусственных нейронных сетей}
\label{sd_syntax_sc_model_art_neural_networks}

\scsubsection{Предметная область и онтология денотационной семантики sc-моделей искусственных нейронных сетей}
\label{sd_denot_sem_sc_model_art_neural_networks}

\scsubsection{Предметная область и онтология операционной семантики sc-моделей искусственных нейронных сетей}
\label{sd_oper_sem_sc_model_art_neural_networks}


\scchapter{Предметная область и онтология интерфейсов ostis-систем}
\label{sec:sd_interfaces}
\begin{SCn}
	
\scnsectionheader{\currentname}
	
\scnstartsubstruct

\scnrelfromlist{дочерний раздел}{Предметная область и онтологий интерфейсных действий пользователей ostis-систем;Предметная область и онтология естественных языков}

\scnheader{Предметная область интерфейсов ostis-систем}
\scniselement{предметная область}
\scnsdmainclasssingle{пользовательский интефейс}
\scnsdclass{командный пользовательский интерфейс;графический пользовательский интерфейс;WIMP-интерфейс;SILK-интерфейс;естественно-языковой интерфейс;речевой интерфейс;пользовательский интерфейс ostis-системы;компонент пользовательского интерфейса;атомарный компонент пользовательского интерфейса;неатомарный компонент пользовательского интерфейса;визуальная часть пользовательского интерфейса ostis-системы;компонент пользовательского интерфейса для представления;компонент вывода;компонент выполнения;параграф;декоративный компонент пользовательского интерфейса;контейнер;меню;строка меню;панель инструментов;панель вкладок;окно;модальное окно;немодальное окно;интерактивный компонент пользовательского интерфейса;флаговая кнопка;радиокнопка;переключатель;кнопка-счетчик;полоса прокрутки;кнопка;Стартовая страница пользовательского интерфейса Метасистемы IMS.ostis}


\scnheader{пользовательский интерфейс}
\scnsuperset{командный пользовательский интерфейс}
\scnsuperset{графический пользовательский интерфейс}
\scnaddlevel{1}
\scnsuperset{WIMP-интерфейс}
\scnaddlevel{1}
\scnsuperset{пользовательский интерфейс ostis-системы}
\scnaddlevel{1}
\scnhaselement{Пользовательский интерфейс Метасистемы IMS.ostis}
\scnhaselement{Пользовательский интерфейс ИСС по геометрии}
\scnaddlevel{1}
\scnidtf{Пользовательский интерфейс интеллектуальной справочной системы по геометрии}
\scnaddlevel{-1}
\scnhaselement{Пользовательский интерфейс ИСС по дискретной математике}
\scnaddlevel{1}
\scnidtf{Пользовательский интерфейс интеллектуальной справочной системы по дискретной математике}
\scnaddlevel{-1}
\scnhaselement{Пользовательский интерфейс ИСС по географии}
\scnaddlevel{1}
\scnidtf{Пользовательский интерфейс интеллектуальной справочной системы по географии}
\scnaddlevel{-1}
\scnhaselement{Пользовательский интерфейс ИСС по искусственным нейронным сетям}
\scnaddlevel{1}
\scnidtf{Пользовательский интерфейс интеллектуальной справочной системы по искусственным нейронным сетям}
\scnaddlevel{-1}
\scnhaselement{Пользовательский интерфейс ИСС по лингвистике}
\scnaddlevel{1}
\scnidtf{Пользовательский интерфейс интеллектуальной справочной системы по лингвистике}
\scnaddlevel{-3}
\scnsuperset{SILK-интерфейс}
\scnaddlevel{1}
\scnidtf{(Speech – речь, Image – образ, Language – язык, Knowledge – знание)}
\scnsuperset{естественно-языковой интерфейс}
\scnaddlevel{1}
\scnsuperset{речевой интерфейс}
\scnaddlevel{-2}

\scnheader{пользовательский интерфейс}
\scnexplanation{\textit{пользовательский интерфейс} -- один из наиболее важных компонентов компьютерной системы. Представляет собой совокупность аппаратных и программных средств, обеспечивающих обмен информацией между пользователем и компьютерной системой.}

\scnheader{командный пользовательский интерфейс}
\scnexplanation{\textit{командный пользовательский интерфейс} -- пользовательский интерфейс, при котором обмен информацией между компьютерной системой и пользователем осуществляется путем написания текстовых инструкций или команд.}

\scnheader{графический пользовательский интерфейс}
\scnexplanation{\textit{графический пользовательский интерфейс} -- пользовательский интерфейс, при котором обмен информацией между компьютерной системой и пользователем осуществляется при помощи графических компонентов компьютерной системы.}

\scnheader{WIMP-интерфейс}
\scnexplanation{\textit{WIMP-интерфейс} -- пользовательский интерфейс, при котором обмен информацией между компьютерной системой и пользователем осуществляется в форме диалога при помощью окон, меню и других элементов управления.}

\scnheader{SILK-интерфейс}
\scnexplanation{\textit{SILK-интерфейс} -- пользовательский интерфейс, наиболее приближенный к естественной для человека форме общения. Компьютерная система находит для себя команды, анализируя человеческую речь и находя в ней ключевые фразы. Результат выполнения команд преобразуется в понятную человеку форму, например, в естественно-языковую форму или изображение.}

\scnheader{естественно-языковой интерфейс}
\scnexplanation{\textit{естественно-языковой интерфейс} -- SILK-интерфейс, обмен информацией между компьютерной системой и пользователем в котором происходит за счёт диалога. Диалог ведётся на одном из естественных языков.}

\scnheader{речевой интерфейс}
\scnexplanation{\textit{речевой интерфейс} -- SILK-интерфейс, обмен информацией в котором происходит за счёт диалога, в процессе которого компьютерная система и пользователь общаются с помощью речи. Данный вид интерфейса наиболее приближен к естественному общению между людьми.}

\scnheader{пользовательский интерфейс ostis-системы}
\scnsubset{ostis-система}
\scnexplanation{\textit{пользовательский интерфейс ostis-системы} представляет собой специализированную \textit{ostis-систему}, ориентированную на решение интерфейсных задач, и имеющую в своем составе базу знаний и решатель задач пользовательского интерфейса ostis-системы.\\
Для решения задачи построения пользовательского интерфейса в базе знаний \textit{пользовательского интерфейса ostis-системы} необходимо наличие sc-модели \textit{компонентов пользовательского интерфейса}, \textit{интерфейсных действий пользователей}, а также классификации \textit{пользовательских интерфейсов} вцелом. При проектировании интерфейса используется компонентный подход,который предполагает представление всего интерфейса приложения в виде отдельных специфицированных компонентов, которые могут разрабатываться и совершенствоваться независимо.}

\scnheader{компонент пользовательского интерфейса}
\scnexplanation{\textit{компонент пользовательского интерфейса} -- знак фрагмента базы знаний, имеющий определённую форму внешнего представления на экране.}
\scnsubdividing{атомарный компонент пользовательского интерфейса;неатомарный компонент пользовательского интерфейса}

\scnheader{атомарный компонент пользовательского интерфейса}
\scnexplanation{\textit{атомарный компонент пользовательского интерфейса} -- компонент пользовательского интерфейса, не содержащий в своём составе других компонентов пользовательского интерфейса.}

\scnheader{неатомарный компонент пользовательского интерфейса}
\scnexplanation{\textit{неатомарный компонент пользовательского интерфейса} -- компонент пользовательского интерфейса, состоящий из других компонентов пользовательского интерфейса.}

\scnheader{визуальная часть пользовательского интерфейса ostis-системы}
\scnsubset{неатомарный компонент пользовательского интерфейса}
\scnexplanation{\textit{визуальная часть пользовательского интерфейса ostis-системы} -- часть базы знаний пользовательского интерфейса ostis-системы, содержащая необходимые для отображения пользовательского интерфейса компоненты.}
	
\scnheader{компонент пользовательского интерфейса}
\scnidtf{user interface component}
\scnsuperset{компонент пользовательского интерфейса для отображения}
\scnaddlevel{1}
\scnidtf{presentation user interface component}
\scnaddlevel{-1}
\scnaddlevel{1}
	\scnsuperset{компонент вывода}
	\scnaddlevel{1}
	\scnidtf{output}
	\scnaddlevel{-1}
	\scnaddlevel{1}
		\scnsuperset{компонент вывода изображения}
		\scnaddlevel{1}
		\scnidtf{image-output}
		\scnaddlevel{-1}
		\scnsuperset{компонент вывода графической информации}
		\scnaddlevel{1}
		\scnidtf{graphical-output}
		\scnaddlevel{-1}
			\scnaddlevel{1}
			\scnsuperset{диаграмма}
			\scnaddlevel{1}
			\scnidtf{chart}
			\scnaddlevel{-1}
			\scnsuperset{карта}
			\scnaddlevel{1}
			\scnidtf{map}
			\scnaddlevel{-1}
			\scnsuperset{индикатор выполнения} 
			\scnaddlevel{1}
			\scnidtf{progress-bar}
			\scnaddlevel{-1}
			\scnaddlevel{-1}
		\scnsuperset{компонент вывода видео}
		\scnaddlevel{1}
		\scnidtf{video-output}
		\scnaddlevel{-1}
		\scnsuperset{компонент вывода звука}
		\scnaddlevel{1}
		\scnidtf{sound-output}
		\scnaddlevel{-1}
		\scnsuperset{компонент вывода текста}
		\scnaddlevel{1}
		\scnidtf{text-output}
		\scnaddlevel{-1}
			\scnaddlevel{1}
			\scnsuperset{заголовок}
			\scnaddlevel{1}
			\scnidtf{headline}
			\scnaddlevel{-1}
			\scnsuperset{параграф}
			\scnaddlevel{1}
			\scnidtf{paragraph}
			\scnaddlevel{-1}
			\scnsuperset{сообщение}
			\scnaddlevel{1}
			\scnidtf{message}
			\scnaddlevel{-1}
			\scnaddlevel{-2}
\scnsuperset{декоративный компонент пользовательского интерфейса}
\scnaddlevel{1}
\scnidtf{decorative user interface component}
\scnaddlevel{-1}
	\scnaddlevel{1}
	\scnsuperset{разделитель}
	\scnaddlevel{1}
	\scnidtf{separator}
	\scnaddlevel{-1}
	\scnsuperset{пустое пространство}
	\scnaddlevel{1}
	\scnidtf{blank-space}
	\scnaddlevel{-2}
\scnsuperset{контейнер}
	\scnaddlevel{1}
		\scnidtf{container}
		\scnsuperset{меню}
		\scnaddlevel{1}
		\scnidtf{menu}
		\scnaddlevel{-1}
		\scnsuperset{строка меню}
		\scnaddlevel{1}
		\scnidtf{menu-bar}
		\scnaddlevel{-1}
		\scnsuperset{панель инструментов}
		\scnaddlevel{1}
		\scnidtf{tool-bar}
		\scnaddlevel{-1}
		\scnsuperset{строка состояния}
		\scnaddlevel{1}
		\scnidtf{status-bar}
		\scnaddlevel{-1}
		\scnsuperset{таблично-строковый контейнер}
		\scnaddlevel{1}
		\scnidtf{table-row-container}
		\scnaddlevel{-1}
		\scnsuperset{списковый контейнер}
		\scnaddlevel{1}
		\scnidtf{list-container}
		\scnaddlevel{-1}
		\scnsuperset{таблично-клеточный контейнер}
		\scnaddlevel{1}
		\scnidtf{table-cell-container}
		\scnaddlevel{-1}
		\scnsuperset{древовидный контейнер}
		\scnaddlevel{1}
		\scnidtf{tree-container}
		\scnaddlevel{-1}
		\scnsuperset{панель вкладок}
		\scnaddlevel{1}
		\scnidtf{tab-pane}
		\scnaddlevel{-1}
		\scnsuperset{панель вращения}
		\scnaddlevel{1}
		\scnidtf{spin-pane}
		\scnaddlevel{-1}
		\scnsuperset{узловой контейнер}
		\scnaddlevel{1}
		\scnidtf{tree-node-container}
		\scnaddlevel{-1}
		\scnsuperset{панель прокрутки}
		\scnaddlevel{1}
		\scnidtf{scroll-pane}
		\scnaddlevel{-1}
		\scnsuperset{окно}
		\scnaddlevel{1}
		\scnidtf{window}
		\scnaddlevel{-1}
			\scnaddlevel{1}
			\scnsuperset{модальное окно}
			\scnaddlevel{1}
			\scnidtf{modal-window}
			\scnaddlevel{-1}
			\scnsuperset{немодальное окно}
			\scnaddlevel{1}
			\scnidtf{non-modal-window}
			\scnaddlevel{-4}		
\scnsuperset{интерактивный компонент пользовательского интерфейса}
\scnaddlevel{1}
\scnidtf{interactive user interface component}
\scnaddlevel{-1}
	\scnaddlevel{1}
	\scnsuperset{компонент ввода данных}
	\scnaddlevel{1}
	\scnidtf{data-input-component}
	\scnaddlevel{-1}
		\scnaddlevel{1}
		\scnsuperset{компонент ввода данных с прямой ответной реакцией}
		\scnaddlevel{1}
		\scnidtf{data-input-component-with-direct-feedback}
		\scnaddlevel{-1}
			\scnaddlevel{1}
			\scnsuperset{компонент ввода текста с прямой ответной реакцией}
			\scnaddlevel{1}
			\scnidtf{text-input-component-with-direct-feedback}
			\scnaddlevel{-1}
				\scnaddlevel{1}
				\scnsuperset{многострочное текстовое поле}
				\scnaddlevel{1}
				\scnidtf{multi-line-text-field}
				\scnaddlevel{-1}
				\scnsuperset{однострочное текстовое поле}
				\scnaddlevel{1}
				\scnidtf{single-line-text-field}
				\scnaddlevel{-1}
				\scnaddlevel{-1}
			\scnsuperset{ползунок}
			\scnaddlevel{1}
			\scnidtf{slider}
			\scnaddlevel{-1}
			\scnsuperset{область рисования}
			\scnaddlevel{1}
			\scnidtf{drawing-area}
			\scnaddlevel{-1}
			\scnsuperset{компонент выбора}
			\scnaddlevel{1}
			\scnidtf{selection-component}
			\scnaddlevel{-1}
				\scnaddlevel{1}
				\scnsuperset{компонент выбора нескольких значений}
				\scnaddlevel{1}
				\scnidtf{selection-component-multiple-values}
				\scnaddlevel{-1}
				\scnsuperset{компонент выбора одного значения}
				\scnaddlevel{1}
				\scnidtf{selection-component-single-values}
				\scnaddlevel{-1}
				\scnaddlevel{-1}
			\scnsuperset{компонент выбора данных}
			\scnaddlevel{1}
			\scnidtf{selectable-data-representation}
			\scnaddlevel{-1}	
				\scnaddlevel{1}
				\scnsuperset{флаговая кнопка}
				\scnaddlevel{1}
				\scnidtf{check-box}
				\scnaddlevel{-1}
				\scnsuperset{радиокнопка}
				\scnaddlevel{1}
				\scnidtf{radio-button}
				\scnaddlevel{-1}
				\scnsuperset{переключатель}
				\scnaddlevel{1}
				\scnidtf{toggle-button}
				\scnaddlevel{-1}
				\scnsuperset{выбираемый элемент}
				\scnidtf{selectable-item}
				\scnaddlevel{-2}
		\scnsuperset{компонент ввода данных без прямой ответной реакции}
		\scnaddlevel{1}
		\scnidtf{data-input-component-without-direct-feedback}
		\scnaddlevel{-1}
			\scnaddlevel{1}
			\scnsuperset{кнопка-счётчик}
			\scnaddlevel{1}
			\scnidtf{spin-button}
			\scnaddlevel{-1}
			\scnsuperset{компонент речевого ввода}
			\scnaddlevel{1}
			\scnidtf{speech-input}
			\scnaddlevel{-1}
			\scnsuperset{компонент ввода движений}
			\scnaddlevel{1}
			\scnidtf{motion-input}
			\scnaddlevel{-1}
			\scnaddlevel{-2}
	\scnsuperset{компонент для представления и взаимодействия с пользователем}
	\scnaddlevel{1}
	\scnidtf{presentation-manipulation-component}
	\scnaddlevel{-1}
		\scnaddlevel{1}
		\scnsuperset{активирующий компонент}
		\scnaddlevel{1}
		\scnidtf{activating-component}
		\scnaddlevel{-1}
		\scnsuperset{компонент непрерывной манипуляции}
		\scnaddlevel{1}
		\scnidtf{continuous-manipulation-component}
		\scnaddlevel{-1}
			\scnaddlevel{1}
			\scnsuperset{полоса прокрутки}
			\scnaddlevel{1}
			\scnidtf{scrollbar}
			\scnaddlevel{-1}
			\scnsuperset{компонент редактирования размера}
			\scnaddlevel{1}
			\scnidtf{resizer}
			\scnaddlevel{-1}
			\scnaddlevel{-2}
	\scnsuperset{компонент запроса действий}
	\scnaddlevel{1}
	\scnidtf{operation-trigger-component}
	\scnaddlevel{-1}
		\scnaddlevel{1}
		\scnsuperset{компонент выбора команд}
		\scnaddlevel{1}
		\scnidtf{command-selection-component}
		\scnaddlevel{-1}
			\scnaddlevel{1}
			\scnsuperset{кнопка}
			\scnaddlevel{1}
			\scnidtf{button}
			\scnaddlevel{-1}
			\scnsuperset{пункт меню}
			\scnaddlevel{1}
			\scnidtf{menu-item}
			\scnaddlevel{-1}
			\scnaddlevel{-1}
		\scnsuperset{компонент ввода команд}
		\scnaddlevel{1}
		\scnidtf{command-input-component}
		\scnaddlevel{-1}
		\scnaddlevel{-2}

\scnheader{компонент пользовательского интерфейса для представления}
\scnexplanation{\textit{компонент пользовательского интерфейса для представления} -- компонент пользовательского интерфейса, не подразумевающий взаимодействия с пользователем.}

\scnheader{компонент вывода}
\scnexplanation{\textit{компонент вывода} -- компонент пользовательского интерфейса, предназначенный для представления информации.}

\scnheader{индикатор выполнения}
\scnexplanation{\textit{индикатор выполнения} -- компонент пользовательского интерфейса, предназначенный для отображения процента выполнения какой-либо задачи.}

\scnheader{параграф}
\scnexplanation{\textit{параграф} -- компонент пользовательского интерфейса, предназначенный для отображения блоков текста. Он отделяется от других блоков пустой строкой или первой строкой с отступом.}

\scnheader{декоративный компонент пользовательского интерфейса}
\scnexplanation{\textit{декоративный компонент пользовательского интерфейса} -- компонент пользовательского интерфейса, предназначенный для стилизации интерфейса.}

\scnheader{контейнер}
\scnexplanation{\textit{контейнер} -- компонент пользовательского интерфейса, задача которого состоит в размещении набора компонентов, включённых в его состав.
}

\scnheader{меню}
\scnexplanation{\textit{меню} -- компонент пользовательского интерфейса, содержащий несколько вариантов для выбора пользователем.}

\scnheader{строка меню}
\scnexplanation{\textit{строка меню} -- горизонтальная полоса , содержащая ярлыки меню. Строка меню предоставляет пользователю место в окне, где можно найти большинство основных функций программы.}

\scnheader{панель инструментов}
\scnexplanation{\textit{панель инструментов} -- компонент пользовательского интерфейса, на котором размещаются элементы ввода или вывода данных.}

\scnheader{панель вкладок}
\scnexplanation{\textit{панель вкладок} -- контейнер, который может содержать несколько вкладок (секций) внутри, которые могут быть отображены, нажав на вкладке с названием в верхней части панели. Одновременно отображается только одна вкладка.}

\scnheader{окно}
\scnexplanation{\textit{окно} -- обособленная область экрана, содержащая различные элементы пользовательского интерфейса. Окна могут располагаться поверх друг друга.}

\scnheader{модальное окно}
\scnexplanation{\textit{модальное окно} -- окно, которое блокирует работу пользователя с системой до тех пор, пока пользователь окно не закроет.}

\scnheader{немодальное окно}
\scnexplanation{\textit{немодальное окно} -- окно, которое позволяет выполнять переключение между данным окном и другим окном без необходимости закрытия окна.}

\scnheader{интерактивный компонент пользовательского интерфейса}
\scnexplanation{\textit{интерактивный компонент пользовательского интерфейса} -- компонент пользовательского интерфейса, с помощью которого осуществляется взаимодействие с пользователем.}

\scnheader{флаговая кнопка}
\scnexplanation{\textit{флаговая кнопка} -- компонент пользовательского интерфейса, позволяющий пользователю управлять параметром с двумя состояниями — включено и отключено.}

\scnheader{радиокнопка}
\scnexplanation{\textit{радиокнопка} -- компонент пользовательского интерфейса, который позволяет пользователю выбрать одну опцию из предопределенного набора.}

\scnheader{переключатель}
\scnexplanation{\textit{переключатель} -- компонент пользовательского интерфейса, который позволяет пользователю переключаться между двумя состояниями.}

\scnheader{кнопка-счетчик}
\scnexplanation{\textit{кнопка-счетчик} -- компонент пользовательского интерфейса, как правило, ориентированный вертикально, с помощью которого пользователь может изменить значение в прилегающем текстовом поле, в результате чего значение в текстовом поле увеличивается или уменьшается.}


\scnheader{полоса прокрутки}
\scnexplanation{\textit{полоса прокрутки} -- компонент пользовательского интерфейса, который используется для отображения компонентов пользовательского интерфейса, больших по размеру, чем используемый для их отображения контейнер.}

\scnheader{кнопка}
\scnexplanation{\textit{кнопка} -- компонент пользовательского интерфейса, при нажатии на который происходит программно связанное с этим нажатием действие либо событие.}

\scnheader{Стартовая страница пользовательского интерфейса Метасистемы IMS.ostis}
\scniselement{Визуальная часть пользовательского интерфейса Метасистемы IMS.ostis}
	\scnaddlevel{1}
	\scnsubset{визуальная часть пользовательского интерфейса ostis-системы}
	\scnrelto{часть}{Пользовательский интерфейс Метасистемы IMS.ostis}
	\scnaddlevel{-1}
\scniselement{окно}
\scnrelfrom{иллюстрация}{
\scnfileimage{\includegraphics[width=1\linewidth]{figures/sd_ui/startPage.png}}}
\scnaddlevel{1}
\scnaddlevel{-1}
\scnrelfromset{декомпозиция}{
Панель навигации\\
	\scnaddlevel{1}
	\scniselement{неатомарный компонент пользовательского интерфейса}
	\scnrelfromset{декомпозиция}{
	Главное меню\\	
		\scnaddlevel{1}
		\scniselement{меню}
		\scnrelfromset{декомпозиция}{
		Пункт меню для навигации по ключевым понятиям\\
			\scnaddlevel{1}
			\scniselement{пункт меню}
			\scnaddlevel{-1}
		;Пункт меню для выполнения команд просмотра базы знаний\\
			\scnaddlevel{1}
			\scniselement{пункт меню}
			\scnaddlevel{-1}			
		;Компонент перехода в экспертный режим\\
			\scnaddlevel{1}
			\scniselement{переключатель}
			\scnaddlevel{-1}
	}
	\scnaddlevel{-1}
	;Компонент выбора языка\\
		\scnaddlevel{1}
		\scniselement{компонент выбора одного значения}
		\scnaddlevel{-1}
	;Компонент авторизации\\
		\scnaddlevel{1}
		\scniselement{кнопка}
		\scnaddlevel{-1}}
	\scnaddlevel{-1}
;Блок истории запросов пользователя\\
	\scnaddlevel{1}
	\scniselement{неатомарный компонент пользовательского интерфейса}
	\scnaddlevel{-1}
;Основной блок\\
	\scnaddlevel{1}
	\scniselement{неатомарный компонент пользовательского интерфейса}
	\scnrelfromset{декомпозиция}{
	Главное окно\\
		\scnaddlevel{1}
		\scniselement{окно}
		\scnaddlevel{-1}
	;Панель инструментов\\
		\scnaddlevel{1}
		\scniselement{неатомарный компонент пользовательского интерфейса}
		\scnrelfromset{декомпозиция}{Кнопка отправки содержимого главного окна на печать\\
			\scnaddlevel{1}
			\scniselement{кнопка}
			\scnaddlevel{-1}
		;Кнопка управления видимостью блока истории запросов пользователя\\
			\scnaddlevel{1}
			\scniselement{кнопка}
			\scnaddlevel{-1}
		;Кнопка отображения ссылки на текущий запрос пользователя\\
			\scnaddlevel{1}
			\scniselement{кнопка}
			\scnaddlevel{-1}
		;Поле поиска\\	
			\scnaddlevel{1}
			\scniselement{однострочное текстовое поле}
			\scnaddlevel{-1}
		}
	\scnaddlevel{-1}
	}
\scnaddlevel{-1}
;Панель отображения информации об авторских правах\\
	\scnaddlevel{1}
	\scniselement{неатомарный компонент пользовательского интерфейса}
	\scnaddlevel{-1}
}
\scnendstruct \scnendcurrentsectioncomment
\end{SCn}

\scsection{Предметная область и онтология интерфейсных действий пользователей ostis-системы}
\label{sd_user_interface_actions}
\begin{SCn}

\scnsectionheader{\currentname}

\scnstartsubstruct

\scnheader{Предметная область интерфейсных действий пользователей}
\scniselement{предметная область}
\scnsdmainclass{интерфейсное действие пользователя}
\scnsdclass{действие мышью;прокрутка мышью;прокрутка мышью вверх;прокрутка мышью вниз;наведение мышью;отпускание мышью;нажатие мыши;одиночное нажатие мыши;двойное нажатие мыши;жест мышью;отведение мышью;перетаскивание мышью;действие голосом;действие клавиатурой;нажатие функциональной клавиши;нажатие клавиши набора текста;действие осязанием;действие сенсором;нажатие сенсора;одиночное нажатие сенсора;двойное нажатие сенсора;жест по сенсору;жест по сенсору одним пальцем;жест по сенсору несколькими пальцами;отпускание сенсором;перетаскивание сенсором;действие пером;нажатие функциональной клавиши пером;рисование пером;написание текста пером}
\scnsdrelation{инициируемое пользовательским интерфейсом действие*}
\scnrelfrom{частная предметная область}{
	Предметная область интерфейсных действий пользователей ostis-системы
}

\scnheader{интерфейсное действие пользователя}
\scnaddlevel{1}
\scnidtf{user interface action}
\scnaddlevel{-1}
\scnexplanation{Действие, выполняемое пользователем над некоторым \textit{компонентом пользовательского интерфейса}. Для связи данного действия с \textit{компонентом пользовательского интерфейса} и необходимым к выполнению \textit{внутренним действием системы} используется отношение \textit{инициируемое пользовательским интерфейсом действие*}}
	\scnsuperset{действие мышью}
	\scnaddlevel{1}
	\scnidtf{mouse-action}
	\scnaddlevel{-1}
		\scnaddlevel{1}
		\scnsuperset{прокрутка мышью}
		\scnaddlevel{1}
		\scnidtf{mouse-scroll}
		\scnaddlevel{-1}
			\scnaddlevel{1}
			\scnsuperset{прокрутка мышью вверх}
			\scnaddlevel{1}
			\scnidtf{mouse-scroll-up}
			\scnaddlevel{-1}
			\scnsuperset{прокрутка мышью вниз}
			\scnaddlevel{1}
			\scnidtf{mouse-scroll-down}
			\scnaddlevel{-1}
			\scnaddlevel{-1}
		\scnsuperset{наведение мышью}
		\scnaddlevel{1}
		\scnidtf{mouse-hover}
		\scnaddlevel{-1}
		\scnsuperset{отпускание мышью}
		\scnaddlevel{1}
		\scnidtf{mouse-drop}
		\scnaddlevel{-1}
		\scnsuperset{нажатие мыши}
		\scnaddlevel{1}
		\scnidtf{mouse-click}
		\scnaddlevel{-1}
			\scnaddlevel{1}
			\scnsuperset{одиночное нажатие мыши}
			\scnaddlevel{1}
			\scnidtf{mouse-single-click}
			\scnaddlevel{-1}
			\scnsuperset{двойное нажатие мыши}
			\scnaddlevel{1}
			\scnidtf{mouse-double-click}
			\scnaddlevel{-1}			
			\scnaddlevel{-1}
		\scnsuperset{жест мышью}
		\scnaddlevel{1}
		\scnidtf{mouse-gesture}
		\scnaddlevel{-1}
		\scnsuperset{отведение мышью}	
		\scnaddlevel{1}
		\scnidtf{mouse-unhover}
		\scnaddlevel{-1}	
		\scnsuperset{перетаскивание мышью}
		\scnaddlevel{1}
		\scnidtf{mouse-drag}
		\scnaddlevel{-1}
		\scnaddlevel{-1}		
	\scnsuperset{действие голосом}
	\scnaddlevel{1}
	\scnidtf{speech-action}
	\scnaddlevel{-1}
	\scnsuperset{действие клавиатурой}
	\scnaddlevel{1}
	\scnidtf{keyboard-action}
	\scnaddlevel{-1}
			\scnaddlevel{1}
			\scnsuperset{нажатие функциональной клавиши}
			\scnaddlevel{1}
			\scnidtf{press-function-key}
			\scnaddlevel{-1}
			\scnsuperset{нажатие клавиши набора текста}
			\scnaddlevel{1}
			\scnidtf{type-text}
			\scnaddlevel{-1}
			\scnaddlevel{-1}	
	\scnsuperset{действие осязанием}
	\scnaddlevel{1}
	\scnidtf{tangible-action}
	\scnaddlevel{-1}	
	\scnsuperset{действие сенсором}	
	\scnaddlevel{1}
	\scnidtf{touch-action}
	\scnaddlevel{-1}
		\scnaddlevel{1}
		\scnsuperset{нажатие сенсора}
		\scnaddlevel{1}
		\scnidtf{touch-click}
		\scnaddlevel{-1}
			\scnaddlevel{1}
			\scnsuperset{одиночное нажатие сенсора}
			\scnaddlevel{1}
			\scnidtf{touch-single-click}
			\scnaddlevel{-1}
			\scnsuperset{двойное нажатие сенсора}
			\scnaddlevel{1}
			\scnidtf{touch-double-click}
			\scnaddlevel{-1}
			\scnaddlevel{-1}
		\scnsuperset{жест по сенсору}
		\scnaddlevel{1}
		\scnidtf{touch-gesture}
		\scnaddlevel{-1}
			\scnaddlevel{1}
			\scnsuperset{жест по сенсору одним пальцем}
			\scnaddlevel{1}
			\scnidtf{one-fingure-gesture}
			\scnaddlevel{-1}
			\scnsuperset{жест по сенсору несколькими пальцами}
			\scnaddlevel{1}
			\scnidtf{multiple-finger-gesture}
			\scnaddlevel{-1}
			\scnaddlevel{-1}
		\scnsuperset{отпускание сенсором}
		\scnaddlevel{1}
		\scnidtf{touch-drop}
		\scnaddlevel{-1}
		\scnsuperset{перетаскивание сенсором}
		\scnaddlevel{1}
		\scnidtf{touch-drag}
		\scnaddlevel{-1}
		\scnaddlevel{-1}
	\scnsuperset{действие пером}
	\scnaddlevel{1}
	\scnidtf{pen-base-action}
	\scnaddlevel{-1}	
		\scnaddlevel{1}
		\scnsuperset{нажатие функциональной клавиши пером}
		\scnaddlevel{1}
		\scnidtf{touch-function-key}
		\scnaddlevel{-1}
		\scnsuperset{рисование пером}
		\scnaddlevel{1}
		\scnidtf{draw}
		\scnaddlevel{-1}
		\scnsuperset{написание текста пером}
		\scnaddlevel{1}
		\scnidtf{write-text}
		\scnaddlevel{-1}
		\scnaddlevel{-1}
		
		
\scnheader{прокрутка мышью}
\scnexplanation{\textit{прокрутка мышью} -- интерфейсное действие пользователя, соответствующее прокрутке содержимого некоторого компонента пользовательского интерфейса при помощи мыши.}

\scnheader{наведение мышью}
\scnexplanation{\textit{наведение мышью} -- интерфейсное действие пользователя, соответствующее появлению курсора мыши в рамках компонента пользовательского интерфейса.}

\scnheader{отпускание мышью}
\scnexplanation{\textit{отпускание мышью} -- интерфейсное действие пользователя, соответствующее отпусканию некоторого компонента пользовательского интерфейса в рамках другого компонента пользовательского интерфейса при помощи мыши.}

\scnheader{нажатие мыши}
\scnexplanation{\textit{нажатие мыши} -- интерфейсное действие пользователя, соответствующее выполнению нажатия мыши в рамках некоторого компонента пользовательского интерфейса.}

\scnheader{отведение мышью}
\scnexplanation{\textit{отведение мышью} -- интерфейсное действие пользователя, соответствующее выходу курсора мыши за рамки компонента пользовательского интерфейса.}

\scnheader{перетаскивание мышью}
\scnexplanation{\textit{перетаскивание мышью} -- интерфейсное действие пользователя, соответствующее перетаскиванию компонента пользовательского интерфейса при помощи мыши.}

\scnheader{нажатие сенсора}
\scnexplanation{\textit{нажатие сенсора} -- интерфейсное действие пользователя, соответствующее выполнению нажатия сенсора в рамках некоторого компонента пользовательского интерфейса.}

\scnheader{жест по сенсору}
\scnexplanation{\textit{жест по сенсору} -- интерфейсное действие пользователя, соответствующее выполнению некоторого жеста, выполняемого при помощи движения пальцев на экране сенсора.}

\scnheader{отпускание сенсором}
\scnexplanation{\textit{отпускание сенсором} -- интерфейсное действие пользователя, соответствующее отпусканию некоторого компонента пользовательского интерфейса в рамках другого компонента пользовательского интерфейса при помощи сенсора.}

\scnheader{перетаскивание сенсором}
\scnexplanation{\textit{перетаскивание сенсором} -- интерфейсное действие пользователя, соответствующее перетаскиванию компонента пользовательского интерфейса при помощи сенсора.}

\scnheader{действие пером}
\scnexplanation{\textit{действие пером} -- интерфейсное действие пользователя, осуществляемое при помощи пера на графическом планшете.}

\scnheader{класс интерфейсных действий пользователя}
\scnexplanation{\textit{класс интерфейсных действий пользователя} -- множество, элементами которого являются классы \textit{интерфейсных действий пользователя}.}

\scnheader{инициируемое пользовательским интерфейсом действие*}
\scnexplanation{Используется для задания инициируемого при взаимодействии с пользовательским интерфейсом действия. Первым компонентом связки отношения \textit{инициируемое пользовательским интерфейсом действие*} является связка, элементами которой являются элемент множества компонентов пользовательского интерфейса и элемент множества \textit{класс интерфейсных действий пользователя}. Вторым компонентом является элемент множества \textit{класс внутренних действий системы}.}
\scniselement{квазибинарное отношение}
\scniselement{ориентированное отношение}
\scnrelfrom{первый домен}{компонент пользовательского интерфейса $\cup$ класс интерфейсных действий пользователя}
\scnrelfrom{второй домен}{класс внутренних действий системы}
\scnrelfrom{иллюстрация}{
	\scnfilescg{figures/sd_ui/ui_initiated_action.png}}}
\scnendstruct \scnendcurrentsectioncomment

\end{SCn}

\scsection{Предметная область и онтология сообщений, входящих в ostis-систему и выходящих из неё}
\label{sd_messages}

\scsection{Предметная область и онтология действий и внутренних агентов пользовательского интерфейса ostis-системы}
\label{sd_actions_and_internal_agent}

\scsection{Предметная область и онтология естественно-языковых интерфейсов ostis-систем}
\label{sd_natural_lang_interface}

\scsubsection{Предметная область и онтология процессов синтаксического анализа и понимания естественно-языковых сообщений, входящих в ostis-систему}
\label{sd_process_syntax_message_analysis}

\scsubsection{Предметная область и онтология процессов синтеза естественно-языковых сообщений  ostis-системы}
\label{sd_process_message_synthesis}

\scchapter{Предметная область и онтология действий и методик проектирования ostis-систем}
\label{sd_actions_methodology_design}

\scsection{Предметная область и онтология действий и методик проектирования баз знаний ostis-систем}
\label{sd_actions_methodology_know_base_design}

\scsection{Предметная область и онтология действий и методик проектирования решателей задач ostis-систем}
\label{sd_actions_methodology_problem_solver_design}

\scsection{Предметная область и онтология действий и методик проектирования интерфейсов ostis-систем}
\label{sd_actions_methodology_interface_design}

\scsection{Предметная область и онтология действий и методик \uline{обучения} проектированию ostis-систем}
\label{sd_actions_methodology_learning_design}

\scchapter{Предметная область и онтология средств проектирования ostis-систем}
\label{sd_fund_design}

\scsection{Предметная область и онтология библиотек компонентов ostis-систем, многократно используемых в разных ostis-системах}
\label{sd_biblio_component}

\scsubsection{Предметная область и онтология многократно используемых компонентов баз знаний ostis-систем}
\label{sd_know_base_component}

\scsubsection{Предметная область и онтология многократно используемых методов, хранимых в памяти ostis-систем и интерпретируемых их внутренними агентами}
\label{sd_method_agent}

\scsubsection{Предметная область и онтология многократно используемых внутренниз агентов ostis-систем}
\label{sd_internal_agent}

\scsubsection{Предметная область и онтология многократно используемых компонентов интерфейсов ostis-систем}
\label{sd_component_interface}

\scsubsection{Предметная область и онтология многократно используемых встраиваемых ostis-систем}
\label{sd_embed_sys}

\scsection{Предметная область и онтология ostis-систем автоматизации проектирования различных видов ostis-систем}
\label{sd_type_automation_design}

\scsubsection{Логико-семантическая модель встраиваемой ostis-системы автоматизации проектирования баз знаний ostis-систем}
\label{logical_model_embed_automation_design}

\scsubsubsection{Логико-семантическая модель средств редактирования, сборки и ввода исходных текстов различных компонентов проектируемой базы знаний в память ostis-системы}
\label{edit_assem_logical_model}

\scsubsubsection{Логико-семантическая модель средств редактирования проектируемой базы знаний ostis-системы на уровне её внутреннего представления}
\label{edit_tools_proj_logical_model}

\scsubsubsection{Логико-семантическая модель средств редактирования обнаружения и анализа ошибок и противоречий в базе знаний ostis-системы}
\label{detec_error_logical_model}

\scsubsubsection{Логико-семантическая модель средств обнаружения и анализа информационных дыр в базе знаний ostis-системы}
\label{detec_hole_logical_model}

\scsubsubsection{Логико-семантическая модель средств автоматизации управления взаимодействием менеджеров, авторов, рецензентов, экспертов и редакторов в процессе проектирования базы знаний ostis-системы}
\label{author_logical_model}

\scsubsection{Логико-семантическая модель ostis-системы автоматизации проектирования решателей задач ostis-систем}
\label{problem_solver_logical_model}

\scsubsubsection{Логико-семантическая ostis-системы автоматизации проектирования программ Базового языка программирования ostis-систем и, в частности, проектирования внутренних агентов ostis-систем, а также коллективов таких агентов}
\label{design_problem_solver_logical_model}

\scsubsubsection{Логико-семантическая ostis-системы автоматизации проектирования искусственных нейронных сетей, семантически совместимых с базами знаний ostis-систем}
\label{autom_design_neural_network_logical_model}


\scsubsection{Логико-семантическая модель ostis-системы автоматизации проектирования интерфейсов ostis-систем}
\label{autom_design_interface_logical_model}

\scsection{Предметная область и онтология ostis-систем автоматизации проектирования различных классов ostis-систем}
\label{sd_autom_design_class}

\scsubsection{Предметная область и онтология ostis-систем обучения проектированию ostis-систем и их компонентов}
\label{sd_learn_design_component}

\scchapter{Предметная область и онтология методов и средств производства ostis-систем}
\label{sd_method_prod_sys}

\scsection{Предметная область и онтология базовых интерпретаторов когнитивно-семантических моделей ostis-систем}
\label{sd_basic_interp_semantic_sys}


\scsectionfamily{Часть 6 Стандарта OSTIS. Платформы реализации интеллектуальных компьютерных систем нового поколения}
\label{part_platforms}

\scsection[\scneditor{Шункевич Д.В.}\protect\scnmonographychapter{Глава 6.1. Универсальная модель интерпретатора внутренних агентов решателя задач интеллектуальной компьютерной системы нового поколения}]{Предметная область и онтология методов и средств производства ostis-систем}
\label{sd_method_prod_sys}

\scsubsection[\scneditor{Шункевич Д.В.}\protect\scnmonographychapter{Глава 6.1. Универсальная модель интерпретатора внутренних агентов решателя задач интеллектуальной компьютерной системы нового поколения}]{Предметная область и онтология базовых интерпретаторов логико-семантических моделей ostis-систем}
\label{sd_interpreters}
\begin{SCn}

\scnsectionheader{\currentname}
\scnrelfromlist{подраздел}{Предметная область и онтология программных вариантов реализации базового интерпретатора семантических моделей ostis-систем на современных компьютерах;Предметная область и онтология семантических ассоциативных компьютеров для ostis-систем}

\scnstartsubstruct

\scnheader{Предметная область базовых интерпретаторов семантических моделей ostis-систем}
\scnsdmainclasssingle{***}
\scnsdclass{***}
\scnsdrelation{***}

\scnheader{универсальный интерпретатор sc-моделей компьютерных систем}
\scnsuperset{встроенная ostis-система}
\scnidtf{встроенная пустая ostis-система}
\scnidtf{универсальный интерпретатор sc-моделей ostis-систем}
\scnidtf{универсальная базовая ostis-система, обеспечивающая имитацию любой ostis-системы путем интерпретации sc-модели имитируемой ostis-системы}
\scnaddlevel{1}
\begin{adjustwidth}{0.4em}{0em}
\scnnote{соотношение между имитируемой и универсальной ostis-системой в известной мере аналогично соотношению между машиной Тьюринга и универсальной машиной Тьюринга}
\end{adjustwidth}
\scnresetlevel
\scnidtf{интерпретатор программ языка SCP -- Semantic Code Programming}

\scnidtf{scp-машина}

\scntext{реализация}{Реализация \textit{универсального интерпретатора sc-моделей компьютерных систем} может иметь большое число вариантов -- как программно, так и аппаратно реализованных. Логическая архитектура \textit{универсального интерпретатора sc-моделей компьютерных систем} обеспечивает независимость проектируемых компьютерных систем от многообразия вариантов реализации интерпретатора их моделей и включает в себя:

\begin{scnitemize}
    \item \textit{смысловую графовую ассоциативную память} (sc-память, sc-хранилище знаковых конструкций, представленных SC-коде);
    \item \textit{интерпретатор языка SCP} - базового процедурного языка программирования, ориентированного на обработку текстов SC-кода, хранимых в смысловой графовой ассоциативной памяти.
\end{scnitemize}}

\scnheader{платформа интерпретации sc-моделей компьютерных систем}
\scnauthorcomment{Интегрировать с предыдущим}
\scnidtf{Библиотека платформ реализации sc-моделей компьютерных систем}
\scnidtf{Семейство платформ интерпретации sc-моделей компьютерных систем}
\scnidtf{Библиотека интерпретаторов унифицированных логико-семантических моделей компьютерных систем}
\scnidtf{интерпретатор унифицированных логико-семантических моделей компьютерных систем}
\scnidtf{платформа реализации sc-моделей компьютерных систем}
\scnexplanation{Под \textbf{\textit{платформой интерпретации sc-моделей компьютерных систем}} понимается реализация платформы интерпретации sc-моделей, которая в общем случае включает в себя:

\begin{scnitemize}
    \item хранилище \textit{sc-текстов} (\textit{sc-хранилище});
    \item файловую память \textit{sc-машины};
    \item средства, обеспечивающие взаимодействие \textit{sc-агентов} над общей памятью;
    \item базовые средства интерфейса для взаимодействия системы с внешним миром (пользователем или другими системами). Указанные средства включают в себя, как минимум, редактор, транслятор (в sc-память и из нее) и визуализатор для одного из базовых универсальных вариантов представления \textit{SC-кода} (\textit{SCg-код}, \textit{SCs-код}, \textit{SCn-код}), средства, позволяющие задавать системе вопросы из некоторого универсального класса (например, запрос семантической окрестности некоторого объекта);
    \item реализацию \textit{абстрактной scp-машины}, то есть интерпретатор \textit{scp-программ}.
\end{scnitemize}
При необходимости, в \textbf{\textit{платформу интерпретации sc-моделей компьютерных систем}} могут быть заранее на платформенно-зависимом уровне включены какие-либо компоненты машин обработки знаний или баз знаний, например, с целью упрощения создания первой версии дочерней системы на основе \textit{Технологии OSTIS}.

Реализация платформы может осуществляться на основе произвольного набора существующих технологий, включая аппаратную реализацию каких-либо ее частей. С точки зрения \textit{Технологии OSTIS} любая \textbf{\textit{платформа интерпретации sc-моделей компьютерных систем}} является \textbf{\textit{платформенно-зависимым многократно используемым компонентом}}.}

\scnendstruct

\end{SCn}

\scsubsubsection[\scnmonographychapter{Глава 6.1. Универсальная модель интерпретатора внутренних агентов решателя задач интеллектуальной компьютерной системы нового поколения}]{Предметная область и онтология ассоциативных семантических компьютеров для ostis-систем}
\label{sd_sem_comp}
\begin{SCn}

\scnsectionheader{\currentname}

\scnstartsubstruct

\scnheader{Предметная область и онтология семантических ассоциативных компьютеров для ostis-систем}
\scnsdmainclasssingle{***}
\scnsdclass{***}
\scnsdrelation{***}

\scnheader{семантический ассоциативный компьютер}
\scnidtf{аппаратно реализованный интерпретатор семантических моделей (sc-моделей) компьютерных систем}
\scnidtf{семантический ассоциативный компьютер, управляемый знаниями}
\scnidtf{компьютер с нелинейной структурно перестраиваемой (графодинамической) ассоциативной памятью, переработка информации в которой сводится не к изменению состояния элементов памяти, а к изменению конфигурации связей между ними}
\scnidtf{sc-компьютер}
\scnidtf{scp-компьютер}
\scnidtf{компьютер, управляемый знаниями, представленными в SC-коде}
\scnidtf{компьютер, ориентированный на обработку текстов SC-кода}

\filemodetrue
\scnrelfromlist{принцип}{
нелинейная память — каждый элементарный фрагмент хранимого в памяти текста может быть инцидентен неограниченному числу других элементарных фрагментов этого текста;
структурно перестраиваемая (реконфигурируемая) память — процесс отработки хранимой в памяти информации сводится не только к изменению состояния элементов, но и к реконфигурации связей между ними;
в качестве внутреннего способа кодирования знаний, хранимых в памяти семантического ассоциативного компьютера, используется универсальный (!) способ нелинейного (графоподобного) смыслового представления знаний, названный нами SC-кодом (семантическим, смысловым компьютерным кодом);
обработка информации осуществляется коллективом агентов, работающих над общей памятью. Каждый из них реагирует на соответствующую ему ситуацию или событие в памяти (компьютер, управляемый хранимыми знаниями);
есть программно реализуемые агенты, поведение которых описывается хранимыми в памяти агентно-ориентированными программами, которые интерпретируются соответствующими коллективами агентов;
есть базовые агенты, которые не могут быть реализованы программно (в частности, это агенты интерпретации агентных программ, базовые рецепторные агенты-датчики, базовые эффекторные агенты);
все агенты работают над общей памятью одновременно. Более того, если для какого-либо агента в некоторый момент времени в различных частях памяти возникает сразу несколько условий его применения, разные акты указанного агента в разных частях памяти могут выполняться одновременно (акт агента — это неделимый, целостный процесс деятельности агента);
для того, чтобы акты агентов, параллельно выполняемые в общей памяти не "мешали"\ друг другу, для каждого акта в памяти фиксируется и постоянно актуализируется его текущее состояние. То есть каждый акт сообщает всем остальным о своих намерениях и пожеланиях, которым остальные агенты не должны препятствовать (например, это различного рода блокировки используемых элементов семантической памяти);
кроме того, агенты (точнее, выполняемые ими акты) должны соблюдать "этику"\,, стараясь не в ущерб себе создавать максимально благоприятные условия для других агентов (актов), например, не жадничать, быстрее возвращать, не захватывать (не блокировать) лишние элементы памяти, как можно скорее освобождать (деблокировать) заблокированные элементы памяти;
процессор и память семантического ассоциативного компьютера глубоко интегрированы и составляют единую процессоро-память. Процессор семантического ассоциативного компьютера равномерно "распределен"\ по его памяти так, что процессорные элементы одновременно являются и элементами памяти компьютера. Обработка информации в семантическом ассоциативном компьютере сводится к реконфигурации каналов связи между процессорными элементами,  следовательно память такого компьютера есть не что иное, как \uline{коммутатор} (!) указанных каналов связи. Таким образом, текущее состояние конфигурации этих каналов связи и есть текущее состояние обрабатываемой информации}
\filemodefalse

\scnendstruct

\end{SCn}

\scsubsubsection[\scneditors{Шункевич Д.В.;Корончик Д.Н.;Марковец В.С.;Зотов Н.В.;Орлов М.К.}\protect\scnmonographychapter{Глава 6.2. Программная платформа интеллектуальных компьютерных систем нового поколения}]{Предметная область и онтология программных вариантов реализации базового интерпретатора логико-семантических моделей ostis-систем на современных компьютерах}
\label{sd_program_interp}
\begin{SCn}
	
\scnsectionheader{\currentname}

\scnstartsubstruct

\scnrelfrom{соавтор}{Корончик Д.Н.}

\scnheader{Предметная область программных вариантов реализации базового интерпретатора логико-семантических моделей ostis-систем на современных компьютерах}
\scniselement{предметная область}
\scnsdmainclasssingle{программный вариант реализации платформы интерпретации sc-моделей компьютерных систем}
\scnhaselementlist{ключевой объект исследования}{Программный вариант реализации платформы интерпретации sc-моделей компьютерных систем;Программная модель sc-памяти;Реализация sc-хранилища и средств доступа к нему;Реализация sc-хранилища;sc-адрес;элемент sc-хранилища;метка синтаксического типа sc-элемента;метка уровня доступа sc-элемента;sc-итератор;sc-шаблон;контекст процесса в рамках программной модели sc-памяти;блокировка sc-элемента в рамках программной модели sc-памяти;подписка на событие в sc-памяти в рамках программной модели sc-памяти;Реализация файловой памяти ostis-системы;Реализация базового набора платформенно-зависимых sc-агентов и их компонентов;Реализация подсистемы взаимодействия с внешней средой с использованием сетевых протоколов;SCTP;sctp-команда;Реализация подсистемы взаимодействия с внешней средой с использованием протоколов SCTP;Реализация подсистемы взаимодействия с внешней средой с использованием протоколов на основе формата JSON;Реализация вспомогательных инструментальных средств для работы c sc-памятью;Реализация интерпретатора sc-моделей пользовательских интерфейсов;Реализация scp-интерпретатора}

\scnheader{программный вариант реализации платформы интерпретации sc-моделей компьютерных систем}
\scnidtf{программная реализация платформы интерпретации sc-моделей компьютерных систем }
\scnidtf{программный вариант реализации базового интерпретатора логико-семантических моделей компьютерных систем}
\scnidtf{программный вариант реализации базового интерпретатора логико-семантических моделей ostis-систем на современных компьютерах}
\scnidtf{вариант реализации базового интерпретатора логико-семантических моделей компьютерных систем на традиционных компьютерах с архитектурой фон Неймана}
\scnexplanation{Одним из путей, позволяющих осуществлять апробацию, развитие, а в ряде случаев и внедрение новых моделей и технологий вне зависимости от наличия соответствующих аппаратных средств является разработка программных моделей этих аппаратных средств, которые были бы функционально эквивалентны этим аппаратным средствам, но при этом интерпретировались на базе традиционной аппаратной архитектуры (в данной работе традиционной архитектурой будем считать архитектуру фон Неймана, как доминирующую в настоящее время). Очевидно, что производительность таких программных моделей в общем случае будет ниже, чем самих аппаратных решений, однако в большинстве случаев она оказывается достаточной для того, чтобы развивать соответствующую технологию параллельно с разработкой аппаратных средств и осуществления постепенного перевода уже работающих систем с программной модели на аппаратные средства.}
\scnsuperset{web-ориентированный вариант реализации платформы интерпретации sc-моделей компьютерных систем}
\scnaddlevel{1}
	\scnidtf{вариант реализации платформы интерпретации sc-моделей компьютерных систем предполагающий взаимодействие пользователей с системой посредством сети Интернет}
	\scnsubset{многопользовательский вариант реализации платформы интерпретации sc-моделей компьютерных систем}
	\scnhaselement{Программный вариант реализации платформы интерпретации sc-моделей компьютерных систем}
\scnaddlevel{-1}

\scnheader{Программный вариант реализации платформы интерпретации sc-моделей компьютерных систем}
\scntext{принципы реализации}{Поскольку sc-тексты представляют собой семантические сети, то есть, по сути, графовые конструкции определенного вида, то на нижнем уровне задача разработки программного варианта реализации платформы интерпретации sc-моделей сводится к разработке средств хранения и обработки таких графовых конструкций.
	
	К настоящему времени разработано большое количество простейших моделей представления графовых конструкций в линейной памяти, таких как матрицы смежности, списки смежности и другие (\scncite{Diskrete_Math}). Однако, при разработке сложных систем как правило приходится использовать более эффективные модели, как с точки зрения объема информации, требуемого для представления, так и с точки зрения эффективности обработки графовых конструкций, хранимых в той или иной форме.
	
	К наиболее распространенным программным средствам, ориентированным на хранение и обработку графовых конструкций относятся графовые СУБД (Neo4j \scncite{Neo4j}, ArangoDB \scncite{ArangoDB}, OrientDB \scncite{OrientDB}, Grakn \scncite{Grakn} и др.), а также так называемые rdf-хранилища (Virtuoso \scncite{Virtuoso}, Sesame \scncite{Sesame} и др.), предназначенные для хранения конструкций, представленных в модели RDF. Для доступа к информации, хранимой в рамках таких средств, могут использоваться как языки, реализуемые в рамках конкретного средства (например, язык Cypher в Neo4j), так и языки, являющиеся стандартами для большого числа систем такого класса (например, SPARQL для rdf-хранилищ).
	
	Популярность и развитость такого рода средств приводит к тому, что на первый взгляд целесообразным и эффективным кажется вариант реализации \textit{программного варианта реализации платформы интерпретации sc-моделей} на базе одного из таких средств. Однако, существует ряд причин, по которым было принято решение о реализации \textit{программного варианта реализации платформы интерпретации sc-моделей} с нуля. К ним относятся следующие:
	
	\begin{scnitemize}
		\item для обеспечения эффективности хранения и обработки информационных конструкций определенного вида (в данном случае -- конструкций SC-кода, sc-конструкций), должна учитываться специфика этих конструкций. В частности, описанные в работе \scncite{Koronchik2013} эксперименты показали значительный прирост эффективности собственного решения по сравнению с существующими на тот момент;
		\item в отличие от классических графовых конструкций, где дуга или ребро могут быть инцидентны только вершине графа (это справедливо и для rdf-графов) в SC-коде вполне типичной является ситуация, когда sc-коннектор инцидентен другому sc-коннектору или даже двум sc-коннекторам. В связи с этим существующие средства хранения графовых конструкций не позволяют в явном виде хранить sc-конструкции (sc-графы). Возможным решением данной проблемы является переход от sc-графа к орграфу инцидентности, пример которого описан в работе \scncite{Ivashenko2015}, однако такой вариант приводит к увеличению числа хранимых элементов в несколько раз и значительно снижает эффективность алгоритмов поиска из-за необходимости делать большое количество дополнительных итераций;
		\item в основе обработки информации в рамках Технологии OSTIS лежит многоагентный подход, в рамках которого агенты обработки информации, хранимой в sc-памяти (sc-агенты) реагируют на события, происходящие в sc-памяти и обмениваются информацией посредством спецификации выполняемых ими действий в sc-памяти \scncite{Shunkevich2018}. В связи с этим одной из важнейших задач является реализация в рамках \textit{программного варианта реализации платформы интерпретации sc-моделей} возможности подписки на события, происходящие в программной модели sc-памяти, которая на данный момент практически не поддерживается в рамках современных средств хранения и обработки графовых конструкций;
		\item SC-код позволяет описывать также внешние информационные конструкции любого рода (изображения, текстовые файла, аудио- и видеофайлы и т.д.), которые формально трактуются как содержимое \textit{sc-элементов}, являющихся знаками \textit{внешних файлов ostis-системы}. Таким образом, компонентом \textit{программного варианта реализации платформы интерпретации sc-моделей} должна быть реализация файловой памяти, которая позволяет хранить указанные конструкции в каких-либо общепринятых форматах. Реализация такого компонента в рамках современных средств хранения и обработки графовых конструкций также не всегда представляется возможной.
	\end{scnitemize}
	
	По совокупности перечисленных причин было принято решение о реализации \textit{программного варианта реализации платформы интерпретации sc-моделей} "с нуля"{} с учетом особенностей хранения и обработки информации в рамках Технологии OSTIS.}
\scnrelfromset{декомпозиция программной системы}{Программная модель sc-памяти;Реализация интерпретатора sc-моделей пользовательских интерфейсов}
\scnexplanation{Текущий \textit{Программный вариант реализации платформы интерпретации sc-моделей компьютерных систем} является web-ориентированным, то есть с точки зрения современной архитектуры каждая \mbox{ostis-система} представляет собой web-сайт доступный онлайн посредством обычного браузера. Такой вариант реализации обладает очевидным преимуществом -- доступ к системе возможен из любой точки мира, где есть Интернет, при этом для работы с системой не требуется никакого специализированного программного обеспечения. С другой стороны, такой вариант реализации обеспечивает возможность параллельной работы с системой нескольких пользователей.

В то же время, взаимодействие клиентской и серверной части организовано таким образом, что \mbox{web-интерфейс} может быть легко заменен на настольный или мобильный интерфейс, как универсальный, так и специализированный.

Данный вариант реализации распространяется под open-source лицензией, для хранения исходных текстов используется хостинг Github и коллективная учетная запись ostis-dev.

Реализация является кроссплатформенной и может быть собрана из исходных текстов в различных операционных системах.}
\scnrelfrom{иллюстрация}{\scnfileimage{\includegraphics[scale=0.95]{figures/sd_interpreters/platform-ostis-architecture.pdf}}}
\scnaddlevel{1}
	\scnexplanation{На приведенной иллюстрации видно, что ядром платформы является \textit{Программная модель sc-памяти} (sc-machine), которая одновременно может взаимодействовать как с \textit{Реализацией интерпретатора sc-моделей пользовательских интерфейсов} (sc-web \scncite{sc_web}), так и с любыми сторонними приложениями по соответствующим сетевым протоколам. С точки зрения общей архитектуры \textit{Реализация интерпретатора sc-моделей пользовательских интерфейсов} выступает как один из множества возможных внешних компонентов, взаимодействующих с \textit{Программной моделью sc-памяти} по сети.}
\scnaddlevel{-1}

\scnheader{Программная модель sc-памяти}
\scnidtf{sc-machine}
\scnidtf{Программная модель семантической памяти, реализованная на основе традиционной линейной памяти и включающая средства хранения sc-конструкций и базовые средства для обработки этих конструкций, в том числе удаленного доступа к ним посредством соответствующих сетевых протоколов}
\scnrelto{программная модель}{sc-память}
\scniselement{программная модель sc-памяти на основе линейной памяти}
\scntext{основной репозиторий исходных текстов}{https://github.com/ostis-dev/sc-machine.git}
\scnrelfromlist{компонент программной системы}{Реализация sc-хранилища и средств доступа к нему\\
	\scnaddlevel{1}
		\scnexplanation{В рамках текущей \textit{Программной модели sc-памяти} под \textit{sc-хранилищем} понимается компонент программной модели, осуществляющий хранение sc-конструкций и доступ к ним через программный интерфейс. В общем случае \textit{sc-хранилище} может быть реализовано по-разному. Кроме собственно \textit{sc-хранилища} \scnbispace \textit{Программная модель sc-памяти} включает также \textit{Реализацию файловой памяти ostis-системы}, предназначенную для хранения содержимого \textit{внутренних файлов ostis-систем}. Стоит отметить, что при переходе с \textit{Программной модели sc-памяти} на ее аппаратную реализацию файловую память ostis-системы целесообразно будет реализовывать на основе традиционной линейной памяти (во всяком случае, на первых этапах развития \textit{семантического компьютера}).}
	\scnaddlevel{-1}
	;Реализация базового набора платформенно-зависимых sc-агентов и их общих компонентов;Реализация подсистемы взаимодействия с внешней средой с использованием сетевых протоколов;Реализация вспомогательных инструментальных средств для работы с sc-памятью;Реализация scp-интерпретатора}
\scntext{программная документация}{http://ostis-dev.github.io/sc-machine/}
\scnrelfromlist{используемый язык программирования}{C;C++;Python}
\scnnote{Текущий вариант \textit{Программной модели sc-памяти} предполагает возможность сохранения состояния (слепка) памяти на жесткий диск и последующей загрузки из ранее сохраненного состояния. Такая возможность необходима для перезапуска системы, в случае возможных сбоев, а также при работе с исходными текстами базы знаний, когда сборка из исходных текстов сводится к формированию слепка состояния памяти, который затем помещается в \textit{Программную модель sc-памяти}.}

\scnheader{Реализация sc-хранилища и средств доступа к нему}
\scnrelfromlist{компонент программной системы}{Реализация sc-хранилища;Реализация файловой памяти ostis-системы}

\scnheader{Реализация sc-хранилища}
\scniselement{реализация sc-хранилища на основе линейной памяти}
\scnrelfrom{иллюстрация}{\scnfileimage{\includegraphics{figures/sd_interpreters/sc-storage.pdf}}}
\scnrelfrom{класс объектов программной системы}{сегмент sc-хранилища}
\scnaddlevel{1}
	\scnidtf{страница sc-хранилища}
	\scnexplanation{В рамках данной реализации \textit{sc-хранилища} \scnbigspace \textit{sc-память} моделируется в виде набора \textit{сегментов}, каждый из которых представляет собой фиксированного размера упорядоченную последовательность \textit{элементов sc-хранилища}, каждый из которых соответствует конкретному sc-элементу. В настоящее время каждый сегмент состоит из $2^{16}-1=65535$ \textit{элементов sc-хранилища}. Выделение \textit{сегментов sc-хранилища} позволяет, с одной стороны, упростить адресный доступ к \textit{элементам sc-хранилища}, с другой стороны -- реализовать возможность выгрузки части sc-памяти из оперативной памяти на файловую систему при необходимости. Во втором случае сегмент sc-хранилища становится минимальной (атомарной) выгружаемой частью sc-памяти. Механизм выгрузки сегментов реализуется в соответствии с существующими принципами организации виртуальной памяти в современных операционных системах.}
	\scnnote{Максимально возможное число сегментов ограничивается настройками программной реализации sc-хранилища (в настоящее время по умолчанию установлено количество $2^{16}-1=65535$ сегментов, но в общем случае оно может быть другим). Таким образом, технически максимальное количество хранимых sc-элементов в текущей реализации составляет около $4.3 \times 10^{9}$ sc-элементов.}
	\scnnote{По умолчанию все сегменты физически располагаются в оперативной памяти, если объема памяти не хватает, то предусмотрен механизм выгрузки части сегментов на жесткий диск (механизм виртуальной памяти).}
	\scnrelfrom{класс объектов программной системы}{элемент sc-хранилища}
		\scnaddlevel{1}
			\scnexplanation{Каждый сегмент состоит из набора структур данных, описывающих конкретные \textit{sc-элементы} (элементов sc-хранилища). Независимо от типа описываемого sc-элемента каждый \textit{элемент sc-хранилища} имеет фиксированный размер (в текущий момент -- 48 байт), что обеспечивает удобство их хранения. Таким образом, максимальный размер базы знаний в текущей программной модели sc-памяти может достигнуть 223 Гб (без учета содержимого \textit{внутренних файлов ostis-системы}, хранимого на внешней файловой системе).}
		\scnaddlevel{-1}
\scnaddlevel{-1}
\scnrelfrom{пример}{\scnfilescg{figures/sd_interpreters/storage_example.png}}
\scnaddlevel{1}
	\scnexplanation{Для наглядности в данном примере опущены \textit{метки уровня доступа}}
\scnaddlevel{-1}

\scnheader{sc-адрес}
\scnidtf{адрес элемента sc-хранилища, соответствующего заданному sc-элементу, в рамках текущего состояния реализации sc-хранилища в составе программной модели sc-памяти}
\scnexplanation{Каждый элемент sc-хранилища в текущей реализации может быть однозначно задан его адресом (sc-адресом), состоящим из номера сегмента и номера \textit{элемента sc-хранилища} в рамках сегмента. Таким образом, \textit{sc-адрес} служит уникальными координатами \textit{элемента sc-хранилища} в рамках \textit{Реализации sc-хранилища}.}
\scnnote{Sc-адрес никак не учитывается при обработке базы знаний на семантическом уровне и необходим только для обеспечения доступа к соответствующей структуре данных, хранящейся в линейной памяти на уровне \textit{Реализации sc-хранилища}.}
\scnnote{В общем случае sc-адрес элемента sc-хранилища, соответствующего заданному sc-элементу, может меняться, например, при пересборке базы знаний из исходных текстов и последующем перезапуске системы. При этом sc-адрес элемента sc-хранилища, соответствующего заданному sc-элементу, непосредственно в процессе работы системы в текущей реализации меняться не может.}
\scnnote{Для простоты будем говорить "sc-адрес sc-элемента"{}, имея в виду \textit{sc-адрес} \scnbigspace \textit{элемента sc-хранилища}, однозначно соответствующего данному \textit{sc-элементу}.}
\scnrelfromlist{семейство отношений, однозначно задающих структуру заданной сущности}{номер сегмента sc-хранилища*;номер элемента sc-хранилища в рамках сегмента*}

\scnheader{элемент sc-хранилища}
\scnidtf{ячейка sc-хранилища}
\scnidtf{элемент sc-хранилища, соответствующий sc-элементу}
\scnidtf{образ sc-элемента в рамках sc-хранилища}
\scnidtf{структура данных, каждый экземпляр которой соответствует одному sc-элементу в рамках sc-хранилища}
\scnexplanation{Каждый элемент sc-хранилища, соответствующий некоторому sc-элементу, описывается его синтаксическим типом (меткой), а также независимо от типа указывается sc-адрес первой входящей в данный sc-элемент sc-дуги и первой выходящей из данного sc-элемента sc-дуги (могут быть пустыми, если таких sc-дуг нет). 
	
Оставшиеся байты в зависимости от типа соответствующего sc-элемента (sc-узел или sc-дуга) могут использоваться либо для хранения содержимого внутреннего файла ostis-системы (может быть пустым, если sc-узел не является знаком файла), либо для хранения спецификации sc-дуги.}
\scnsubdividing{элемент sc-хранилища, соответствующий sc-узлу\\
	\scnaddlevel{1}
		\scnrelfromset{семейство отношений, однозначно задающих структуру заданной сущности}{метка синтаксического типа sc-элемента*;метка уровня доступа sc-элемента*;sc-адрес первой sc-дуги, выходящей из данного sc-элемента*;sc-адрес первой sc-дуги, входящей в данный sc-элемент*;содержимое элемента sc-хранилища*\\
		\scnaddlevel{1}
			\scnrelfrom{второй домен}{содержимое элемента sc-хранилища}
			\scnaddlevel{1}
				\scnidtf{содержимое элемента sc-хранилища, соответствующего внутреннему файлу ostis-системы}
			\scnaddlevel{-1}
			\scnexplanation{Каждый sc-узел в текущей реализации может иметь содержимое (может стать \textit{внутренним файлом ostis-системы}).
			В случае, если размер содержимого внутреннего файла ostis-системы не превышает 48 байт (размер \textit{спецификации sc-дуги в рамках sc-хранилища}, например небольшой \textit{строковый \mbox{sc-идентификатор}}), то это содержимое явно хранится в рамках элемента \mbox{sc-хранилища} в виде последовательности байт.
			В противном случае оно помещается в специальным образом организованную файловую память (за ее организацию отвечает отдельный модуль платформы, который в общем случае может быть устроен по-разному), а в рамках элемента sc-хранилища хранится уникальный адрес соответствующего файла, позволяющий быстро найти его на файловой системе.}
		\scnaddlevel{-1}}
		\scnaddlevel{1}
			\scnnote{\textit{sc-адрес первой sc-дуги, выходящей из данного sc-элемента*}, \textit{sc-адрес первой sc-дуги, входящей в данный sc-элемент*} и \textit{содержимое элемента sc-хранилища*} в общем случае могут отсутствовать (быть нулевыми, "пустыми"{}), но размер элемента в байтах останется тем же.}
		\scnaddlevel{-1}
	\scnaddlevel{-1}
	;элемент sc-хранилища, соответствующий sc-дуге\\
	\scnaddlevel{1}
	\scnrelfromset{семейство отношений, однозначно задающих структуру заданной сущности}{метка синтаксического типа sc-элемента*;метка уровня доступа sc-элемента*;sc-адрес первой sc-дуги, выходящей из данного sc-элемента*;sc-адрес первой sc-дуги, входящей в данный sc-элемент*;спецификация sc-дуги в рамках sc-хранилища*\\
		\scnaddlevel{1}
			\scnrelfrom{второй домен}{спецификация sc-дуги в рамках sc-хранилища}
			\scnaddlevel{1}
			\scnrelfromset{семейство отношений, однозначно задающих структуру заданной сущности}{sc-адрес начального sc-элемента sc-дуги*;sc-адрес конечного sc-элемента sc-дуги*;sc-адрес следующей sc-дуги, выходящей из того же sc-элемента*;sc-адрес следующей sc-дуги, входящей в тот же sc-элемент*;sc-адрес предыдущей sc-дуги, выходящей из того же sc-элемента*;sc-адрес предыдущей sc-дуги, входящей в тот же sc-элемент*}
			\scnaddlevel{-1}
		\scnaddlevel{-1}}
	\scnnote{sc-ребра в текущий момент хранятся так же, как sc-дуги, то есть имеют начальный и конечный sc-элементы, отличие заключается только в \textit{метке синтаксического типа sc-элемента}. Это приводит к ряду неудобств при обработке, но sc-ребра используются в настоящее время достаточно редко.}
	\scnaddlevel{-1}}
\scnaddlevel{1}
	\scnnote{С точки зрения программной реализации структура данных для хранения sc-узла и sc-дуги остается остается та же, но в ней меняется список полей (компонентов).\\
	Кроме того, как можно заметить каждый элемент sc-хранилища (в том числе, \textit{элемент sc-хранилища, соответствующий sc-дуге}) не хранит список sc-адресов связанных с ним sc-элементов, а хранит sc-адреса одной выходящей и одной входящей дуги, каждая из которых в свою очередь хранит sc-адреса следующей и предыдущей дуг в списке исходящих и входящих sc-дуг для соответствующих элементов.\\
	Все перечисленное позволяет:
	\begin{scnitemize}	
		\item сделать размер такой структуры фиксированным (в настоящее время 48 байт) и не зависящим от синтаксического типа хранимого sc-элемента;
		\item обеспечить возможность работы с sc-элементами без учета их синтаксического типа в случаях, когда это необходимо (например, при реализации поисковых запросов вида ``Какие sc-элементы являются элементами данного множества'', ``Какие sc-элементы непосредственно связаны с данным sc-элементом'' и т.д.);
		\item обеспечить возможность доступа к \textit{элементу sc-хранилища} за константное время;
		\item обеспечить возможность помещения \textit{элемента sc-хранилища} в процессорный кэш, что в свою очередь, позволяет ускорить обработку sc-конструкций;
	\end{scnitemize}}
\scnaddlevel{-1}
\scnnote{Текущая \textit{Программная модель sc-памяти} предполагает, что вся sc-память физически расположена на одном компьютере. Для реализации распределенного варианта \textit{Программной модели sc-памяти} предполагается расширить \textit{sc-адрес} указанием адреса того физического устройства, где хранится соответствующий \textit{элемент sc-хранилища}.}

\scnheader{метка синтаксического типа sc-элемента}
\scnidtf{уникальный числовой идентификатор, однозначно соответствующий заданному типу sc-элементов и приписываемый соответствующему элементу sc-хранилища на уровне реализации}
\scnnote{Очевидно, что тип (класс, вид) sc-элемента в sc-памяти может быть задан путем явного указания принадлежности данного sc-элемента соответствующему классу (sc-узел, sc-дуга и т.д.).
	
Однако, в рамках \textit{платформы интерпретации sc-моделей компьютерных систем} должен существовать какой-либо набор \textit{меток синтаксического типа sc-элемента}, которые задают тип элемента на уровне платформы и не имеют соответствующей sc-дуги принадлежности (а точнее -- базовой sc-дуги), явно хранимой в рамках sc-памяти (ее наличие подразумевается, однако она не хранится явно, поскольку это приведет к бесконечному увеличению числа sc-элементов, которые необходимо хранить в sc-памяти). Как минимум, должна существовать метка, соответствующая классу \textit{базовая sc-дуга}, поскольку явное указание принадлежности sc-дуги данному классу порождает еще одну \textit{базовую sc-дугу}.

Таким образом, \textit{базовые sc-дуги}, обозначающие принадлежность sc-элементов некоторому известному ограниченному набору классов представлены \uline{неявно}. Этот факт необходимо учитывать в ряде случаев, например, при проверке принадлежности sc-элемента некоторому классу, при поиске всех выходящих sc-дуг из заданного sc-элемента и т.д.

При необходимости некоторые из таких неявно хранимых sc-дуг могут быть представлены явно, например, в случае, когда такую sc-дугу необходимо включить в какое-либо множество, то есть провести в нее другую sc-дугу. В этом случае возникает необходимость синхронизации изменений, связанных с данной sc-дугой (например, ее удалении), в явном и неявном ее представлении. В текущей \textit{Реализации sc-хранилища} данный механизм не реализован.

Таким образом, полностью отказаться от \textit{меток синтаксического типа sc-элементов} невозможно, однако увеличение их числа хоть и повышает производительность платформы за счет упрощений некоторых операций по проверке типов sc-элемента, но приводит к увеличению числа ситуаций, в которых необходимо учитывать явное и неявное представление sc-дуг, что, в свою очередь, усложняет развитие платформы и разработку программного кода для обработки хранимых sc-конструкций.}
\scnrelto{второй домен}{метка синтаксического типа sc-элемента*}
\scnsuperset{метка sc-узла}
\scnaddlevel{1}
	\scntext{числовое выражение в шестнадцатеричной системе}{0x1}
\scnaddlevel{-1}
\scnsuperset{метка внутреннего файла ostis-системы}
\scnaddlevel{1}
\scntext{числовое выражение в шестнадцатеричной системе}{0x2}
\scnaddlevel{-1}
\scnsuperset{метка sc-ребра общего вида}
\scnaddlevel{1}
\scntext{числовое выражение в шестнадцатеричной системе}{0x4}
\scnaddlevel{-1}
\scnsuperset{метка sc-дуги общего вида}
\scnaddlevel{1}
\scntext{числовое выражение в шестнадцатеричной системе}{0x8}
\scnaddlevel{-1}
\scnsuperset{метка sc-дуги принадлежности}
\scnaddlevel{1}
\scntext{числовое выражение в шестнадцатеричной системе}{0x10}
\scnaddlevel{-1}
\scnsuperset{метка sc-константы}
\scnaddlevel{1}
\scntext{числовое выражение в шестнадцатеричной системе}{0x20}
\scnaddlevel{-1}
\scnsuperset{метка sc-переменной}
\scnaddlevel{1}
\scntext{числовое выражение в шестнадцатеричной системе}{0x40}
\scnaddlevel{-1}
\scnsuperset{метка позитивной sc-дуги принадлежности}
\scnaddlevel{1}
\scntext{числовое выражение в шестнадцатеричной системе}{0x80}
\scnaddlevel{-1}
\scnsuperset{метка негативной sc-дуги принадлежности}
\scnaddlevel{1}
\scntext{числовое выражение в шестнадцатеричной системе}{0x100}
\scnaddlevel{-1}
\scnsuperset{метка нечеткой sc-дуги принадлежности}
\scnaddlevel{1}
\scntext{числовое выражение в шестнадцатеричной системе}{0x200}
\scnaddlevel{-1}
\scnsuperset{метка постоянной sc-дуги}
\scnaddlevel{1}
\scntext{числовое выражение в шестнадцатеричной системе}{0x400}
\scnaddlevel{-1}
\scnsuperset{метка временной sc-дуги}
\scnaddlevel{1}
\scntext{числовое выражение в шестнадцатеричной системе}{0x800}
\scnaddlevel{-1}
\scnsuperset{метка небинарной sc-связки}
\scnaddlevel{1}
\scntext{числовое выражение в шестнадцатеричной системе}{0x80}
\scnaddlevel{-1}
\scnsuperset{метка sc-структуры}
\scnaddlevel{1}
\scntext{числовое выражение в шестнадцатеричной системе}{0x100}
\scnaddlevel{-1}
\scnsuperset{метка ролевого отношения}
\scnaddlevel{1}
\scntext{числовое выражение в шестнадцатеричной системе}{0x200}
\scnaddlevel{-1}
\scnsuperset{метка неролевого отношения}
\scnaddlevel{1}
\scntext{числовое выражение в шестнадцатеричной системе}{0x400}
\scnaddlevel{-1}
\scnsuperset{метка sc-класса}
\scnaddlevel{1}
\scntext{числовое выражение в шестнадцатеричной системе}{0x800}
\scnaddlevel{-1}
\scnsuperset{метка абстрактной сущности}
\scnaddlevel{1}
\scntext{числовое выражение в шестнадцатеричной системе}{0x1000}
\scnaddlevel{-1}
\scnsuperset{метка материальной сущности}
\scnaddlevel{1}
\scntext{числовое выражение в шестнадцатеричной системе}{0x2000}
\scnaddlevel{-1}
\scnsuperset{метка константной позитивной постоянной sc-дуги принадлежности}
\scnaddlevel{1}
\scnidtf{метка базовой sc-дуги}
\scnidtf{метка sc-дуги основного вида}
\scnreltoset{пересечение}{метка sc-дуги принадлежности;метка sc-константы;метка позитивной sc-дуги принадлежности;метка постоянной sc-дуги}
\scnnote{\textit{метки синтаксических типов sc-элементов} могут комбинироваться между собой для получения более частных классов меток. С точки зрения программной реализации такая комбинация выражается операцией побитового сложения значений соответствующих меток.}
\scnaddlevel{-1}
\scnsuperset{метка переменной позитивной постоянной sc-дуги принадлежности}
\scnaddlevel{1}
\scnreltoset{пересечение}{метка sc-дуги принадлежности;метка sc-переменной;метка позитивной sc-дуги принадлежности;метка постоянной sc-дуги}
\scnaddlevel{-1}
\scnnote{Числовые выражения некоторых классов меток могут совпадать. Это сделано для уменьшения размера элемента sc-хранилища за счет уменьшения максимального размера метки. Конфликт в данном случае не возникает, поскольку такие классы меток не могут комбинироваться, например \textit{метка ролевого отношения} и \textit{метка нечеткой sc-дуги принадлежности}.}
\scnnote{Важно отметить, что каждому из выделенных классов меток (кроме классов, получаемых путем комбинации других классов) однозначно соответствует порядковый номер бита в линейной памяти, что можно заметить, глядя на соответствующие числовые выражения классов меток. Это означает, что классы меток не включаются друг в друга, например, указание \textit{метки позитивной sc-дуги принадлежности} не означает автоматическое указание \textit{метки sc-дуги принадлежности}. Это позволяет сделать операции комбинирования и сравнения меток более эффективными.}
\scnreltoset{недостатки текущего состояния}{
\scnfileitem{На данный момент число \textit{меток синтаксического типа sc-элемента} достаточно велико, что приводит к возникновению достаточно большого числа ситуаций, в которых нужно учитывать явное и неявное хранение sc-дуг принадлежности соответствующим классам. С другой стороны, изменение набора меток с какой-либо целью в текущем варианте реализации представляет собой достаточно трудоемкую задачу (с точки зрения объема изменений в программном коде платформы и sc-агентов, реализованных на уровне платформы), а расширение набора меток без увеличения объема элемента sc-хранилища в байтах оказывается и вовсе невозможным.}\\
	\scnaddlevel{1}
		\scntext{вариант решения}{Решением данной проблемы является максимально возможная минимизация числа меток, например, до числа меток, соответствующих \textit{Алфавиту SC-кода}. В таком случае принадлежность sc-элементов любым другим классам будет записываться явно, а число ситуаций, в которых необходимо будет учитывать неявное хранение sc-дуг, будет минимальным.}
	\scnaddlevel{-1}
;
\scnfileitem{Некоторые метки из текущего набора \textit{меток синтаксического типа sc-элемента} используются достаточно редко (например, \textit{метка sc-ребра общего вида} или \textit{метка негативной sc-дуги принадлежности}), в свою очередь, в sc-памяти могут существовать классы, имеющие достаточно много элементов (например, \textit{бинарное отношение} или \textit{число}). Данный факт не позволяет в полной мере использовать эффективность наличия меток.}
	\scnaddlevel{1}
		\scntext{вариант решения}{Решением данной проблемы является отказ от заранее известного набора меток и переход к динамическому набору меток (при этом их число может оставаться фиксированным). В этом случае набор классов, выражаемых в виде меток будет формироваться на основании каких-либо критериев, например, числа элементов данного класса или частоты обращений к нему.}
	\scnaddlevel{-1}
}

\scnheader{метка уровня доступа sc-элемента}
\scnrelto{второй домен}{метка уровня доступа sc-элемента*}
\scnrelfromset{обобщенная структура}{метка уровня доступа sc-элемента на чтение;метка уровня доступа sc-элемента на запись}
\scnexplanation{В текущей \textit{Реализации sc-хранилища} \scnbigspace \textit{метки уровня доступа} используются для того, чтобы обеспечить возможность ограничения доутспа некоторых процессов в sc-памяти к некоторым sc-элементам, хранимым в sc-памяти.
	
Каждому элементу sc-хранилища соответствует \textit{метка уровня доступа sc-элемента на чтение} и \textit{метка уровня доступа sc-элемента на запись}, каждая из которых выражается числом от 0 до 255. 
	
В свою очередь, каждому процессу (чаще всего, соответствующему некоторому sc-агенту), который пытается получить доступ к данному элементу sc-хранилища (прочитать или изменить его) соответствует уровень доступа на чтение и запись, выраженный в том же числовом диапазоне. Указанный уровень доступа для процесса является частью \textit{контекста процесса}. Доступ на чтение или запись к элементу sc-хранилища не разрешается, если уровень доступа соответственно на чтение или запись у процесса ниже, чем у элемента sc-хранилища, к которому осуществляется доступ.

Таким образом нулевое значение \textit{метки уровня доступа sc-элемента на чтение} и \textit{метки уровня доступа sc-элемента на запись} означает, что любой процесс может получить неограниченный доступ к данному элементу sc-хранилища.}

\scnheader{sc-итератор}
\scnidtf{ScIterator}
\scnrelto{класс компонентов}{Реализация sc-хранилища}
\scnexplanation{С функциональной точки зрения \textit{sc-итераторы} как часть \textit{Реализации sc-хранилища} представляют собой базовое средство доступа к конструкциям, хранимым в sc-памяти, которое позволяет осуществить чтение (просмотр) конструкций, изоморфных простейшим шаблонам -- \textit{трехэлементным sc-конструкциям} и \textit{пятиэлементным sc-конструкциям} (см. \textit{Раздел \nameref{sec:sd_ps}}).
	
С точки зрения реализации \textit{sc-итератор} представляет собой структуру данных, которая соответствует определенному дополнительно уточняемому классу sc-конструкций и позволяет при помощи соответствующего набора функций последовательно осуществлять просмотр всех sc-конструкций данного класса, представленных в текущем состоянии sc-памяти (итерацию по sc-конструкциям).
	
Каждому классу \textit{sc-итераторов} соответствует некоторый известный класс (шаблон, образец) \mbox{sc-конструкций}. При создании sc-итератора данный шаблон уточняется, то есть некоторым (как минимум одному) элементам шаблона ставится в соответствие конкретный заранее известный \textit{sc-элемент} (отправная точка при поиске), а другим элементам шаблона (тем, которые нужно найти) ставится в соответствие некоторый тип sc-элемента из числа типов, соответствующих \textit{меткам синтаксического типа sc-элемента}. 

Далее путем вызова соответствующей функции (или метода класса в ООП) осуществляется последовательный просмотр всех sc-конструкций, соответствующих полученному шаблону (с учетом указанных типов sc-элементов и заранее заданных известных sc-элементов), то есть \textit{sc-итератор} последовательно "переключается"{} с одной конструкции на другую до тех пор, пока такие конструкции существуют. Проверка существования следующей конструкции проверяется непосредственно перед переключением. В общем случае конструкций, соответствующих указанному шаблону, может не существовать, в этом случае итерирование происходить не будет (будет 0 итераций).

На каждой итерации в sc-итератор записываются sc-адреса sc-элементов, входящих в соответствующую sc-конструкцию, таким образом найденные элементы могут быть обработаны нужным образом в зависимости от задачи.}
\scnsuperset{трехэлементный sc-итератор}
	\scnaddlevel{1}
		\scnrelfrom{класс sc-конструкций}{трехэлементная sc-конструкция}
	\scnaddlevel{-1}
\scnsuperset{пятиэлементный sc-итератор}
	\scnaddlevel{1}
		\scnrelfrom{класс sc-конструкций}{пятиэлементная sc-конструкция}
		\scnnote{В настоящее время \textit{пятиэлементный sc-итератор} реализуется на основе \textit{трехэлементных sc-итераторов} и в этом смысле не является атомарным. Однако, введение \textit{пятиэлементных sc-итераторов} целесообразно с точки зрения удобства разработчика программ обработки \mbox{sc-конструкций}.}
	\scnaddlevel{-1}

\scnheader{sc-шаблон}
\scnidtf{ScTemplate}
\scnidtf{структура данных в линейной памяти, описывающая обобщенную sc-структуру, которая в свою очередь может быть либо явно представлена sc-памяти, либо не представлена в ее текущем состоянии, но может быть представлена при необходимости}
\scnrelto{класс компонентов}{Реализация sc-хранилища}
\scnexplanation{\textit{Sc-итераторы} позволяют осуществлять поиск только sc-конструкций простейшей конфигурации. Для реализации поиска sc-конструкций более сложной конфигурации, а также генерации сложных sc-конструкций используются \textit{sc-шаблоны}, на основе которых затем осуществляется поиск или генерация конструкций. \textit{Sc-шаблон} представляет собой структуру данных, соответствующую некоторой \textit{обобщенной структуре}, т.е. \textit{структуре}, содержащей \textit{sc-переменные}. При помощи соответствующего набора функций можно осуществлять 
\begin{scnitemize}
	\item поиск в текущем состоянии sc-памяти \uline{всех} sc-конструкций, изоморфных заданному шаблону. В качестве параметров поиска можно указать значения для каких-либо из sc-переменных в составе шаблона. После осуществления поиска будет сформировано множество результатов поиска, каждый из которых представляет собой множество пар вида ``sc-переменная из шаблона -- соответствующая ей sc-константа''. Данное множество может быть пустым (в текущем состоянии sc-памяти нет конструкций, изоморфных заданному образцу) или содержать один или более элементов. Подстановка значений sc-переменных может осуществляться как по sc-адресу, так и по системному sc-идентификатору;
	\item генерацию sc-конструкции, изоморфной заданному шаблону. Параметры и результаты генерации формируются так же, как в случае поиска, за исключением того, что в случае генерации результат всегда один и множество результатов не формируется;
\end{scnitemize}

Таким образом, каждый \textit{sc-шаблон} фактически задает множество шаблонов, формируемых путем указания значений для sc-переменных, входящих в исходный шаблон.

Важно отметить, что \textit{sc-шаблон} представляет собой структуру данных в линейной памяти, соответствующую некоторой \textit{обобщенной структуре} в sc-памяти, но не саму эту \textit{обобщенную структуру}. Это означает, что sc-шаблон может быть автоматически сформирован на основе \textit{обобщенной структуры}, явно представленной в sc-памяти, а также сформирован на уровне программного кода путем вызова соответствующих функций (методов). Во втором случае \textit{sc-шаблон} будет существовать только в линейной памяти и соответствующая \textit{обобщенная структура} не будет явно представлена в sc-памяти. В этом случае подстановка значений sc-переменных будет возможна только по системному sc-идентификатору, поскольку sc-адресов у соответствующих элементов шаблона существовать не будет.}
\scnnote{При поиске sc-конструкций, изоморфных заданному шаблону, крайне важно с точки зрения производительности с какого sc-элемента начинать поиск. Как известно, в общем случае задача поиска в графе представляет собой NP-полную задачу, однако поиск в sc-графе позволяет учитывать семантику обрабатываемой информации, что, в свою очередь, позволяет существенно снизить время поиска. 
	
Одним из возможных вариантов оптимизации алгоритма поиска, реализованным на данный момент, является упорядочение трехэлементных sc-конструкций, входящих в состав sc-шаблона, по очередности поиска по этим sc-конструкциям по критерию снижения числа возможных вариантов поиска, которые порождает та или иная трехэлементная sc-конструкция, содержащая sc-переменные. Так, в первую очередь при поиске выбираются те трехэлементные sc-конструкции, которые изначально содержат две sc-константы, затем те, которые изначально содержат одну sc-константу. После выполнения шага поиска приоритет sc-конструкций изменяется с учетом результатов, полученных на предыдущем шаге.

Другой вариант оптимизации основывается на той особенности формализации в SC-коде, что в общем случае число sc-дуг, входящих в некоторый sc-элемент, как правило значительно меньше числа выходящих из него sc-дуг. Таким образом, целесообразным оказывается осуществлять поиск вначале по входящим sc-дугам.}
\scnnote{Можно предположить, что возможности, предоставляемые \textit{sc-шаблонами} позволяют полностью исключить использование \textit{sc-итераторов}. Однако это не совсем так по следующим причинам:
	\begin{scnitemize}
		\item функции поиска и генерации по шаблону реализуются на основе sc-итераторов, как базового средства поиска sc-конструкций в рамках \textit{Реализации sc-хранилища}.
		\item \textit{sc-итераторы} дают возможность более гибко организовать процесс поиска с учетом семантики конкретных sc-элементов, участвующих в поиске. Так например, можно учесть тот факт, что для некоторых sc-элементов число входящих sc-дуг значительно меньше, чем выходящих (или наоборот) таким образом, при поиске конструкций, содержащих такие sc-элементы более эффективно начать перебор с тех участков, где дуг потенциально меньше.
\end{scnitemize}}

\scnheader{контекст процесса в рамках программной модели sc-памяти}
\scnidtf{ScContext}
\scnidtf{контекст процесса, выполняемого на уровне программной модели sc-памяти}
\scnidtf{метаописание процесса в sc-памяти, выполняемого на уровне программной модели sc-памяти}
\scnidtf{структура данных, содержащая метаинформацию о процессе, выполняемом в sc-памяти на уровне платформы}
\scnrelto{класс компонентов}{Реализация sc-хранилища}
\scnexplanation{Каждому процессу, выполняемому в sc-памяти на уровне \textit{платформы интерпретации sc-моделей компьютерных систем} (и чаще всего соответствующего некоторому \textit{sc-агенту}, реализованному на уровне платформы) ставится в соответствие \textit{контекст процесса}, который является структурой данных, описывающей метаинформацию о данном процессе. На текущий момент контекст процесса содержит сведения об уровне доступа на чтение и запись для данного процесса (См. \textit{метка уровня доступа sc-элемента}).

При вызове в рамках процесса любых функций (методов), связанных с доступом к хранимым в sc-памяти конструкциям одним из параметров обязательно является \textit{контекст процесса}.}

\scnheader{блокировка sc-элемента в рамках программной модели sc-памяти}
\scnidtf{ScLock}
\scnrelto{класс компонентов}{Реализация sc-хранилища}
\scnrelfrom{смотрите}{\nameref{sec:sd_agents}}

\scnheader{подписка на событие в sc-памяти в рамках программной модели sc-памяти}
\scnidtf{ScEvent}
\scnidtf{структура данных, описывающая в рамках программной модели sc-памяти соответствие между классом событий в sc-памяти и действиями, которые должно быть совершены при возникновении в sc-памяти событий данного класса}
\scnrelto{класс компонентов}{Реализация sc-хранилища}
\scnexplanation{Для того, чтобы обеспечить возможность создания sc-агентов в рамках \textit{платформы интерпретации sc-моделей компьютерных систем} реализована возможность создать подписку на событие, принадлежащее одному из классов \textit{элементарных событий в sc-памяти*} (см. Раздел ``\textit{Предметная область и онтология темпоральных сущностей базы знаний ostis-системы}''), уточнив при этом sc-элемент, с которым должно быть связано событие данного класса (например, sc-элемент, для которого должна появиться входящая или исходящая sc-дуга). Подписка на событие представляет собой структуру данных, описывающую класс ожидаемых событий и функцию в программном коде, которая должна быть вызвана при возникновении данного события.
	
Все подписки на события регистрируются в рамках таблицы событий. При любом изменении в sc-памяти происходит просмотр данной таблицы и запуск функций, соответствующих произошедшему событию.

В текущей реализации обработка каждого события осуществляется в отдельном потоке операционной системы, при этом на уровне реализации задается параметр, описывающий число максимальных потоков, которые могут выполняться параллельно.

Таким образом оказывается возможным реализовать sc-агенты, реагирующие на события в sc-памяти, а также при выполнении некоторого процесса в sc-памяти приостановить его работу и дождаться возникновения некоторого события (например, создать подзадачу некоторому коллективу sc-агентов и дождаться ее решения).}

\scnheader{Реализация файловой памяти ostis-системы}
\scnexplanation{Для хранения содержимого внутренних файлов ostis-систем, размер которого превышает 48 байт, используются файлы, явно хранимые на файловой системе, доступ к которой осуществляется средствами операционной системы, на которой работает \textit{Программный вариант реализации платформы интерпретации sc-моделей компьютерных систем}.

В общем случае множество различных внутренних файлов ostis-системы могут иметь одинаковое содержимое. Было бы разумно не хранить содержимое одинаковых файлов дважды. Для этого при создании соответствуюещго sc-узла и указании файла на файловой системе, который является содержимым данного sc-узла, вычисляется hash-сумма содержимого с помощью алгоритма SHA256. В результате получается строка из 32 символов, которая и выступает в качестве \textit{содержимого элемента sc-хранилища*}. Само же содержимое копируется в
файл на файловой системе, путь к которому строится на основании hash-суммы. Рядом с этим файлом создается файл, в котором хранятся sc-адреса всех sc-узлов, имеющих одно и то же ранее указанное содержимое. Таким образом, для того, чтобы найти все sc-узлы, имеющие указанное содержимое, необходимо вычислить hash-сумму искомого содержимого-образца и проверить наличие файла на файловой системе по пути, вычисляемому из hash-суммы и если он существует, то вернуть список хранящихся sc-адресов.

Кроме того, для реализации быстрого поиска sc-элементов по их строковым sc-идентификаторам или их фрагментам (подстрокам) используется дополнительное хранилище вида ключ-значение, которое ставит в соответствие \textit{строковому sc-идентификатору} \scnbigspace \textit{sc-адрес} того \textit{sc-элемента}, идентификатором которого является данная строка (в случае основного и системного sc-идентификатора) или \textit{sc-элемента}, который является знаком \textit{внутреннего файла ostis-системы} (в случае неосновного sc-идентификатора).}

\scnheader{Реализация базового набора платформенно-зависимых sc-агентов и их общих компонентов}
\scnidtf{sc-kpm}
\scnrelfromlist{компонент программной системы}{Реализация базового набора поисковых sc-агентов\\
	\scnaddlevel{1}
		\scnrelfromlist{используемый язык программирования}{C}
		\scnrelfromlist{компонент программной системы}{Реализация Абстрактного sc-агента поиска семантической окрестности заданной сущности;Реализация Абстрактного sc-агента поиска всех сущностей, частных по отношению к заданной;Реализация Абстрактного sc-агента поиска всех сущностей, общих по отношению к заданной;Реализация Абстрактного sc-агента поиска всех sc-идентификаторов, соответствующих заданной сущности;Реализация Абстрактного sc-агента поиска базовых sc-дуг, инцидентных заданному sc-элементу\\
			\scnaddlevel{1}
				\scnrelfromlist{компонент программной системы}{Реализация Абстрактного sc-агента поиска базовых sc-дуг, входящих в заданный sc-элемент;Реализация Абстрактного sc-агента поиска базовых sc-дуг, выходящих из заданного sc-элемента;Реализация Абстрактного sc-агента поиска базовых sc-дуг, входящих в заданный sc-элемент, с указанием множеств, которым принадлежат эти sc-дуги;Реализация Абстрактного sc-агента поиска базовых sc-дуг, выходящих из заданного sc-элемента, с указанием множеств, которым принадлежат эти sc-дуги}
			\scnaddlevel{-1}}
	\scnaddlevel{-1}
	;Реализация базового механизма сборки информационного мусора\\
	\scnaddlevel{1}
		\scnrelfromlist{используемый язык программирования}{C}	
		\scnnote{Текущая реализация механизма сборки информационного мусора содержит один sc-агент, реагирующий на явное добавление какого-либо sc-элемента во множество ``информационный мусор'' и осуществляющий физическое удаление этого sc-элемента из sc-памяти}
	\scnaddlevel{-1}
	;Реализация базового набора интерфейсных sc-агентов\\
	\scnaddlevel{1}
	\scnrelfromlist{используемый язык программирования}{C++}	
	\scnrelfromlist{компонент программной системы}{Реализация Абстрактного sc-агента обработки команд пользовательского интерфейса;Реализация Абстрактного sc-агента трансляции из внутреннего представления знаний во промежуточный транспортный формат\\
	\scnaddlevel{1}
		\scnnote{В настоящее время используется подход, при котором независимо от формы внешнего представления информации, информация хранимая в sc-памяти вначале транслируется в промежуточный транспортный формат на базе JSON, который затем обрабатывается sc-агентами пользовательского интерфейса, входящими в состав \textit{Реализации интерпретатора sc-моделей пользовательских интерфейсов}}
	\scnaddlevel{1}
	}
	\scnaddlevel{-1}
}

\scnheader{Реализация подсистемы взаимодействия с внешней средой с использованием сетевых протоколов}
\scnrelfromlist{компонент программной системы}{Реализация подсистемы взаимодействия с внешней средой с использованием протокола SCTP;Реализация подсистемы взаимодействия с внешней средой с использованием протоколов на основе формата JSON}
\scnexplanation{Взаимодействие программной модели sc-памяти с внешними ресурсами может осуществляться посредством специализированного программного интерфейса (API), однако этот вариант неудобен в большинстве случае, поскольку:
	\begin{scnitemize}
		\item поддерживается только для очень ограниченного набора языков программирования (С, С++, Python);
		\item требует того, чтобы клиентское приложение, обращающееся к программной модели sc-памяти, фактически составляло с ней единое целое, таким образом исключается возможность построения распределенного коллектива ostis-систем;
		\item как следствие предыдущего пункта, исключается возможность параллельной работы с sc-памятью нескольких клиентских приложений.
	\end{scnitemize}
	
Для того, чтобы обеспечить возможность удаленного доступа к sc-памяти не учитывая при этом языки программирования, с помощью которых реализовано конкретное клиентское приложение, было принято решение о реализации возможности доступа к sc-памяти с использованием универсальных протоколов, не зависящих от средств реализации того или иного компонента или системы. В качестве таких протоколов были разработаны бинарный протокол SCTP и текстовый протокол на базе JSON.}

\scnheader{SCTP}
\scnidtf{Semantic Code Transfer Protocol}
\scnrelboth{аналогия}{HTTP}
\scnexplanation{SCTP представляет собой \textit{бинарный протокол}, позволяющий осуществлять операции чтения (поиска) и редактирования конструкций, хранящихся в sc-памяти, а также отслеживать события, происходящие в sc-памяти.

Взаимодействие между клиентом и сервером на протоколе SCTP осуществляется путем обмена \textit{\mbox{sctp-командами}}, каждая из которых представляет собой набор байт, предназначенный для машинной обработки (но не восприятия человеком).
}

\scnheader{следует отличать*}
\scnhaselementset{SCTP\\
	\scnaddlevel{1}
	\scnidtf{Semantic Code Transfer Protocol}
	\scnaddlevel{-1};
	Stream Control Transmission Protocol\\
	\scnaddlevel{1}
	\scnidtf{Протокол передачи с управлением потоком}
	\scnnote{Протокол транспортного уровня в компьютерных сетях, разработанный в 2000 году.}
	\scnaddlevel{-1}}

\scnheader{sctp-команда}
\scnrelfromset{обобщенная декомпозиция}{заголовок sctp-команды\\
	\scnaddlevel{1}
		\scnidtf{часть sctp-команды, в которой указан её тип и некоторая дополнительная информация о ней}
	\scnaddlevel{-1}
	;аргументы sctp-команды\\
	\scnaddlevel{1}
		\scnidtf{часть sctp-команды, которая содержит её аргументы и размер которой может быть разным в зависимости от типа команды.}
	\scnaddlevel{-1}}
\scnrelfromlist{включение;пример}{sctp-команда удаления sc-элемента с указанным sc-адресом;sctp-команда создания нового sc-узла указанного типа;sctp-команда получения начального и конечного элемента sc-дуги}
\scnnote{Выполнение каждой sctp-команды предполагает наличие sctp-результата, однозначно соответствующего данной команде.}

\scnheader{SCTP}
\scntext{программная документация}{http://ostis-dev.github.io/sc-machine/net/sctp/}
\scnrelfromlist{недостаток}{\scnfileitem{Команды протокола SCTP являются низкоуровневыми (ориентированы на работу с единичными sc-элементами или простейшими sc-конструкциями из 3 или 5 элементов). Это приводит к тому, что выполнение даже несложного преобразования в базе знаний или ассоциативный поиск по набору взаимосвязанных конструкций выражаются в виде достаточно большого набора sctp-команд. С учетом того, что для каждой команды существует sctp-результат, также пересылаемый по сети, это излишне нагружает сеть и сильно ухудшает производительность системы в целом. Кроме того, производительность системы начинает сильно зависеть от пропускной способности сети.};
\scnfileitem{Протокол SCTP не предназначен для восприятия человеком}}
\scnrelfromlist{достоинство}{\scnfileitem{Протокол SCTP является кросс-платформенным};\scnfileitem{Протокол SCTP может быть достаточно просто реализован практически на любом языке программирования}}
\scnrelfromlist{обобщенная реализация}{sctp-сервер\\
	\scnaddlevel{1}
		\scnexplanation{Sctp-сервер обрабатывает sctp-команды, приходящие от разных sctp-клиентов, и обеспечивает их интерпретацию в sc-памяти.}
	\scnaddlevel{-1}
	;sctp-клиент\\
	\scnaddlevel{1}
		\scnexplanation{Sctp-клиенты в общем случае могут быть реализованы на разных языках программирования и иметь разный программный интерфейс. По сути задачей sctp-клиента является преобразование высокоуровневых команд представленных в форме, удобной программисту, в одну или более низкоуровневых sctp-команд, отправка их на сервер, ожидание sctp-результата и его интерпретация.}
	\scnaddlevel{-1}}

\scnheader{Реализация подсистемы взаимодействия с внешней средой с использованием протокола SCTP}
\scnrelfromlist{компонент программной системы}{Реализация sctp-сервера;Реализация sctp-клиента\\
	\scnaddlevel{1}
	\scnnote{\textit{Реализация подсистемы взаимодействия с внешней средой с использованием протокола SCTP} включает в себя \textit{Реализацию sctp-клиента} на языке C++, в то же время есть другие реализации \textit{sctp-клиентов} в рамках того же программного варианта реализации платформы, например, в рамках \textit{Реализации интерпретатора sc-моделей пользовательских интерфейсов}.}
	\scnaddlevel{-1}}

\scnheader{Реализация подсистемы взаимодействия с внешней средой с использованием протоколов на основе формата JSON}
\scnexplanation{В связи с большим числом недостатков протокола SCTP было принято решение о разработке другого протокола на основе какого-либо общепринятого текстового транспортного формата. В качестве такого формата был выбран формат JSON.}
\scnrelto{реализация}{Протокол взаимодействия с sc-памятью на основе JSON}
\scnaddlevel{1}
\scnnote{Данный протокол пока не имеет собственного названия}
\scntext{программная документация}{http://ostis-dev.github.io/sc-machine/http/websocket/}
\scnexplanation{В рамках \textit{Протокола взаимодействия с sc-памятью на основе JSON} каждая команда представляет собой json-объект, в котором указываются идентификатор команда, тип команды и ее аргументы. В свою очередь ответ на команду также представляет собой json-объект, в котором указываются идентификатор команды, ее статус (выполнена успешно/безуспешно) и результаты. Структура аргументов и результатов команды определяется типом команды.}
\scnrelfromlist{достоинство}{\scnfileitem{JSON является общепринятым открытым форматом, для работы с которым существует большое количество библиотек для популярных языков программирования. Это, в свою очередь, упрощает реализацию клиента и сервера для протокола, построенного на базе JSON.};
\scnfileitem{Реализация протокола на базе JSON не накладывает принципиальных ограничений на объем (длину) каждой команды, в отличие от бинарного протокола. Таким образом, появляется возможность использования неатомарных команд, позволяющих, например, за один акт пересылки такой команды по сети создать сразу несколько sc-элементов. Важными примерами таких команд являются  \textit{Команда генерации по произвольному образцу} и \textit{Команда поиска по произвольному образцу}.}}
\scnnote{Можно сказать, что протокол на базе JSON является следующим шагом на пути к созданию мощного и универсального языка запросов, аналогичного языку SQL для реляционных баз данных и предназначенному для работы с sc-памятью. Следующий шагом станет реализация такого протокола на основе одного из стандартов внешнего отображения sc-конструкций, например, \textit{SCs-кода}, что, в свою очередь, позволит передавать в качестве команд целые программы обработки sc-конструкций, например на языке SCP.}
\scnaddlevel{-1}

\scnheader{Реализация вспомогательных инструментальных средств для работы с sc-памятью}
\scnrelfrom{компонент программной системы}{Реализация сборщика базы знаний из исходных текстов, записанных в SCs-коде}
\scnaddlevel{1}
\scnidtf{sc-builder}
\scnrelfrom{используемый язык}{SCs-код}
\scnexplanation{Сборщик базы знаний из исходных текстов позволяет осуществить сборку базы знаний из набора исходных текстов, записанных в SCs-коде с ограничениями (см. \textit{Раздел **про исходные тексты**}) в бинарный формат, воспринимаемый \textit{Программной моделью sc-памяти}. При этом возможна как сборка "с нуля"{} (с уничтожением ранее созданного слепка памяти), так и аддитивная сборка, когда информация, содержащаяся в заданном множестве файлов, добавляется к уже имеющемуся слепку состояния памяти.

В текущей реализации сборщик осуществляет "склеивание"{} ("слияние"{}) sc-элементов, имеющих на уровне исходных текстов одинаковые \textit{системные sc-идентификаторы}.}
\scnaddlevel{-1}

\scnheader{Реализация интерпретатора sc-моделей пользовательских интерфейсов}
\scnidtf{sc-web}
\scnexplanation{Наряду с реализацией \textit{Программной модели sc-памяти} важной частью \textit{Программного варианта реализации платформы интерпретации sc-моделей компьютерных систем} является \textit{Реализация интерпретатора sc-моделей пользовательских интерфейсов}, которая предоставляет базовые средства просмотра и редактирования базы знаний пользователем, средства для навигации по базе знаний (задания вопросов к базе знаний) и может дополняться новыми компонентами в зависимости от задач, решаемых каждой конкретной ostis-системой.}
\scnrelfromlist{используемый язык программирования}{JavaScript;TypeScript;Python}
\scnrelfrom{иллюстрация}{\scnfileimage{\includegraphics{figures/sd_interpreters/sc-web-new-arch.pdf}}}
\scnaddlevel{1}
	\scnexplanation{На данной иллюстрации показан планируемый вариант архитектуры \textit{Реализация интерпретатора sc-моделей пользовательских интерфейсов}, важным принципом которой является простота и однотипность подключения любых компонентов пользовательского интерфейса (редакторов, визуализаторов, переключателей, команд меню и т.д.). Для этого реализуется программная прослойка Sandbox, в рамках которой реализуются низкоуровневые операции взаимодействия с серверной частью и которая обеспечивает более удобный программный интерфейс для разработчиков компонентов.}
\scnaddlevel{-1}
\scnrelfromset{недостатки текущей реализации}{\scnfileitem{Отсутствие единого унифицированного механизма клиент-серверного взаимодействия. Часть компонентов (визуализатор sc-текстов в SCn-коде, команды меню и др.) работают по протоколу HTTP, часть по протоколу SCTP с использованием технологии WebSocket, это приводит к значительным трудностям при развитии платформы.};
\scnfileitem{Протокол HTTP предполагает четкое разделение активного клиента и пассивного сервера, который отвечает на запросы клиентов. Таким образом, сервер (в данном случае -- sc-память) практически не имеет возможности по своей инициативе отправить сообщение клиенту, что повышает безопасность системы, но значительно снижает ее интерактивность. Кроме того, такой вариант реализации затрудняет реализацию принятого в Технологии OSTIS многоагентного подхода, в частности, затрудняет реализацию sc-агентов на стороне клиента. Указанные проблемы могут быть решены путем постоянного мониторинга определенных событий со стороны клиента, однако такой вариант неэффективен.
Кроме того, часть интерфейса фактически работает напрямую с sc-памятью с использованием технологии WebSocket, а часть -- через прослойку на базе библиотеки tornado для языка программирования Python, что приводит к дополнительным зависимостям от сторонних библиотек.};
\scnfileitem{Часть компонентов (например, поле поиска по идентификатору) реализована сторонними средствами и практически никак не связана с sc-памятью. Это затрудняет развитие платформы.};
\scnfileitem{Текущая \textit{Реализация интерпретатора sc-моделей пользовательских интерфейсов} ориентирована только на ведение диалога с пользователем (в стиле вопрос пользователя -- ответ системы). Не поддерживаются такие очевидно необходимые ситуации, как выполнение команды, не предполагающей ответа\char59~возникновение ошибки или отсутствие ответа\char59~необходимость задания вопроса системой пользователю и т.д.};
\scnfileitem{Ограничена возможность взаимодействия пользователя с системой без использования специальных элементов управления. Например, можно задать вопрос системе, нарисовав его в SCg-коде, но ответ пользователь не увидит, хотя в памяти он будет сформирован соответствующим агентом.;
Большая часть технологий, использованных при реализации платформы, к настоящему моменту устарела, что затрудняет развитие платформы.};
\scnfileitem{Идея платформенной независимости пользовательского интерфейса (построения sc-модели пользовательского интерфейса) реализована не в полной мере. Полностью описать sc-модель пользовательского интерфейса (включая точное размещение, размеры, дизайн компонентов, их поведение и др.) в настоящее время скорее всего окажется затруднительно из-за ограничений производительности, однако вполне возможно реализовать возможность задания вопросов ко всем компонентам интерфейса, изменить их расположение и т.д., однако эти возможности нельзя реализовать в текущей версии реализации платформы.};
\scnfileitem{Интерфейсная часть работает медленно из-за недостатков  протокола SCTP и некоторых недостатков реализации серверной части на языке Python.};
\scnfileitem{Не реализован механизм наследования при добавлении новых внешних языков. Например, добавление нового языка даже очень близкого к SCg-коду требует физического копирования кода компонента и внесение соответствующих изменений, при этом получаются два никак не связанных между собой компонента, которые начинают развиваться независимо друг от друга.};
\scnfileitem{Слабый уровень задокументированности текущей \textit{Реализации интерпретатора sc-моделей пользовательских интерфейсов}.}}
\scnrelfromset{требования к будущей реализации}{\scnfileitem{Унифицировать принципы взаимодействия всех компонентов интерфейса с \textit{Программной моделью sc-памяти}, независимо от того, к какому типу относится компонент. Например, список команд меню должен формироваться через тот же механизм, что и ответ на запрос пользователя, и команда редактирования, сформированная пользователем, и команда добавления нового фрагмента в базу знаний и т.д.};
\scnfileitem{Унифицировать принципы взаимодействия пользователей с системой независимо от способа взаимодействия и внешнего языка. Например, должна быть возможность задания вопросов и выполнения других команд прямо через SCg/SCn интерфейс. При этом необходимо учитывать принципы редактирования базы знаний, чтобы пользователя не мог под видом задания вопроса внести новую информацию в согласованную часть базы знаний.};
\scnfileitem{Унифицировать принципы обработки событий, происходящих при взаимодействии пользователя с компонентами интерфейса -- поведение кнопок и других интерактивных компонентов должно задаваться не статически сторонними средствами, а реализовываться в виде агента, который, тем не менее, может быть реализован произвольным образом (не обязательно на платформенно-независимом уровне). Любое действие, совершаемое пользователем, на логическом уровне должно трактоваться и обрабатываться как инициирование агента.};
\scnfileitem{Обеспечить возможность выполнять команды (в частности, задавать вопросы) с произвольным количеством аргументов, в том числе -- без аргументов.};
\scnfileitem{Обеспечить возможность отображения ответа на вопрос по частям, если ответ очень большой и для отображения требуется много времени.};
\scnfileitem{Каждый отображаемый компонент интерфейса должен трактоваться как изображение некоторого sc-узла, описанного в базе знаний. Таким образом, пользователь должен иметь возможность задания произвольных вопросов к любым компонентам интерфейса.};
\scnfileitem{Максимально упростить и задокументировать механизм добавления новых компонентов.};
\scnfileitem{Обеспечить возможность добавления новых компонентов на основе имеющихся без создания независимых копий. Например, должна быть возможность создать компонент для языка, расширяющего язык SCg новыми примитивами, переопределять принципы размещения sc-текстов и т.д.};
\scnfileitem{Свести к минимуму зависимость от сторонних библиотек.};
\scnfileitem{Свести к минимуму использование протокола HTTP (начальная загрузка общей структуры интерфейса), обеспечить возможность равноправного двустороннего взаимодействия серверной и клиентской части.};
\scnfileitem{Полностью отказаться от протокола SCTP, перейти на протокол на базе JSON, задокументировать его.}}
\scnaddlevel{1}
	\scnnote{Очевидно, что реализация большинства из приведенных требований связана не только с собственно вариантом реализации платформы, но и требует развития теории логико-семантических моделей пользовательских интерфейсов и уточнения в рамках нее общих принципов организации пользовательских интерфейсов ostis-систем. Однако, принципиальная возможность реализации таких моделей должна быть учтена в рамках реализации платформы.}
\scnaddlevel{-1}
\scnrelfromlist{компонент программной системы}{Панель меню команд пользовательского интерфейса\\
	\scnaddlevel{1}
		\scnexplanation{\textit{Панель меню команд пользовательского интерфейса} содержит изображения классов команд (как атомарных, так и неатомарных), имеющихся на данный момент в базе знаний и входящих в декомпозицию \textit{Главного меню пользовательского интерфейса} (имеется в виду полная декомпозиция, которая в общем случае может включать несколько уровней неатомарных классов команд).\\
		Взаимодействие с изображением неатомарного класса команд инициирует команду изображения классов команд, входящих в декомпозицию данного неатомарного класса команд.\\
		Взаимодействие с изображением атомарного класса команд инициирует генерацию команды данного класса с ранее выбранными аргументами на основе соответствующей \textit{обобщенной формулировки класса команд} (шаблона класса команд).}
	\scnaddlevel{-1}
	;Компонент переключения языка идентификации отображаемых sc-элементов\\
	\scnaddlevel{1}
		\scnexplanation{\textit{Компонент переключения языка идентификации отображаемых sc-элементов} является изображением множества имеющихся в системе естественных языков. Взаимодействие пользователя с данным компонентом переключает пользовательский интерфейс в режим общения с конкретным пользователем с использованием \textit{основных sc-идентификаторов}, принадлежащих данному \textit{естественному языку}. Это значит, что при изображении sc-идентификаторов sc-элементов на каком-либо языке, например, SCg-коде или SCn-коде будут использоваться \textit{основные sc-идентификаторы}, принадлежащие данному \textit{естественному языку}. Это касается как sc-элементов, отображаемых в рамках \textit{Панели визуализации и редактирования знаний}, так и любых других sc-элементов, например, классов команд и даже самих \textit{естественных языков}, изображаемых в рамках самого \textit{Компонента переключения языка идентификации отображаемых sc-элементов}.}
	\scnaddlevel{-1}	
	;Компонент переключения внешнего языка визуализации знаний\\
	\scnaddlevel{1}
		\scnexplanation{\textit{Компонент переключения внешнего языка визуализации знаний} служит для переключения языка визуализации знаний в текущем окне, отображаемом на \textit{Панели визуализации и редактирования знаний}. В текущей реализации в качестве таких языков по умолчанию поддерживаются SCg-код и SCn-код, а также любые другие языки, входящие во множество \textit{внешних языков визуализации SC-кода}.}
	\scnaddlevel{-1}
	;Поле поиска sc-элементов по идентификатору\\
	\scnaddlevel{1}
		\scnexplanation{\textit{Поле поиска sc-элементов по идентификатору} позволяет осуществлять поиск \mbox{sc-иден|-ти|-фи|-ка|-то|-ров}, содержащих подстроку, введенную в данное поле (с учетом регистра). В результате поиска отображается список sc-идентификаторов, содержащих указанную подстроку, при взаимодействии с которыми осуществляется автоматическое задание вопроса ``Что это такое?'', аргументом которого является либо для сам sc-элемент, имеющий данный sc-идентификатор (в случае, если указанный sc-идентификатор является основным или системным, и, таким образом, указанный sc-элемент может быть определен однозначно), либо для самого внутреннего файла ostis-системы, являющегося sc-идентификатором (в случае, если данный sc-идентификатор является неосновным).}
	\scnaddlevel{-1}
	;Панель отображения диалога пользователя с ostis-системой\\
	\scnaddlevel{1}
		\scnexplanation{\textit{Панель отображения диалога пользователя с ostis-системой} отображает упорядоченный по времени список sc-элементов, являющихся знаками действий, которые инициировал пользователь в рамках диалога с ostis-системой путем взаимодействия с изображениями соответствующих классов команд (то есть, если действие было инициировано другим способом, например, путем его явного инициирования через создание дуги принадлежности множеству \textit{инициированных действий} в sc.g-редакторе, то на данной панели оно отображено не будет). При взаимодействии пользователя с любым из изображенных знаков действий на \textit{Панели визуализации и редактирования знаний} отображается окно, содержащее результат выполнения данного \textit{действия} на том языке визуализации, на котором он был отображен, когда пользователь просматривал его в последний (предыдущий) раз. Таким образом, в текущей реализации данная панель может работать только в том случае, если инициированное пользователем действие предполагает явно представленный в памяти результат данного действия. В свою очередь, из этого следует, что в настоящее время данная панель, как и в целом \textit{Реализация интерпретатора sc-моделей пользовательских интерфейсов}, позволяет работать с системой только в режиме диалога ''вопрос-ответ''.}
	\scnaddlevel{-1}
	;Панель визуализации и редактирования знаний\\
	\scnaddlevel{1}
		\scnexplanation{\textit{Панель визуализации и редактирования знаний} отображает окна, содержащие sc-текст, представленный на некотором языке из множества \textit{внешних языков визуализации SC-кода} и, как правило, являющийся результатом некоторого действия, инициированного пользователем. Если соответствующий визуализатор поддерживает возможность редактирования текстов соответствующего естественного языка, то он одновременно является также и редактором.}
		\scnrelfromlist{компонент программной системы}{Визуализатор sc.n-текстов;Визуализатор и редактор sc.g-текстов}
		\scnnote{При необходимости пользовательский интерфейс каждой конкретной ostis-системы может быть дополнен визуализаторами и редакторами различных внешних языков, которые в текущей версии \textit{Реализации интерпретатора sc-моделей пользовательских интерфейсов} будут также располагаться на \textit{Панели визуализации и редактирования знаний}.}
	\scnaddlevel{-1}}

\scnheader{Реализация scp-интерпретатора}
\scnrelto{программная реализация}{Абстрактная scp-машина}
\scnnote{Важнейшей особенностью Языка SCP является тот факт, что его программы записываются таким же образом, что и обрабатываемые ими знания, то есть в SC-коде. Это, с одной стороны, дает возможность сделать ostis-системы платформенно-независимыми (четко разделить \textit{sc-модель компьютерной системы} и платформу интерпретации таких моделей), а с другой стороны требует наличия в рамках платформы \textit{Реализации scp-интерпретатора}, то есть интерпретатора программ Языка SCP.}
\scnrelfromlist{используемый язык программирования}{C++}
\scnrelfromlist{компонент программной системы}{Реализация Абстрактного sc-агента создания scp-процессов;Реализация Абстрактного sc-агента интерпретации scp-операторов\\
	\scnaddlevel{1}
	\scnrelfromlist{компонент программной системы}{Реализация Абстрактного sc-агента интерпретации scp-операторов генерации конструкций;Реализация Абстрактного sc-агента интерпретации scp-операторов ассоциативного поиска конструкций;Реализация Абстрактного sc-агента интерпретации scp-операторов удаления конструкций;Реализация Абстрактного sc-агента интерпретации scp-операторов проверки условий;Реализация Абстрактного sc-агента интерпретации scp-операторов управления значениями операндов;Реализация Абстрактного sc-агента интерпретации scp-операторов управления scp-процессами;Реализация Абстрактного sc-агента интерпретации scp-операторов управления событиями;Реализация Абстрактного sc-агента интерпретации scp-операторов обработки содержимых числовых файлов;Реализация Абстрактного sc-агента интерпретации scp-операторов обработки содержимых строковых файлов}
	\scnaddlevel{-1}
	;Реализация Абстрактного sc-агента синхронизации процесса интерпретации scp-программ;Реализация Абстрактного sc-агента уничтожения scp-процессов;Реализация Абстрактного sc-агента синхронизации событий в sc-памяти и ее реализации}
\scnnote{Текущая \textit{Реализация scp-интерпретатора} не включает в себя специализированных средств для работы с блокировками, поскольку механизм блокировок элементов sc-памяти реализован на более низком уровне в рамках \textit{Реализация sc-хранилища и механизма доступа к нему}}

\bigskip
\scnendstruct \scnendcurrentsectioncomment

\end{SCn}

\scsectionfamily{Часть 7 Стандарта OSTIS. Методы и средства реинжиниринга и эксплуатации интеллектуальных компьютерных систем нового поколения}
\label{part_reengineering}

\scsection{Предметная область и онтология методов и средств поддержки жизненного цикла ostis-систем}
\label{sd_method_means_operation}

\scsubsection{Предметная область и онтология методов и средств реинжиниринга ostis-систем в ходе эксплуатации}
\label{sd_method_means_operation_reengineering}

\scsubsection{Предметная область и онтология встроенных ostis-систем поддержки использования ostis-систем конечными пользователями}
\label{sd_embedded_sys_support_operation}

\scsectionfamily{Часть 8 Стандарта OSTIS. Экосистема интеллектуальных компьютерных систем нового поколения и их пользователей}
\label{part_ecosystem}

\scsection[\scneditor{Загорский А.Г.}\protect\scnmonographychapter{Глава 7.2. Экосистема интеллектуальных компьютерных систем нового поколения (Экосистема OSTIS) и реализация рынка знаний на ее основе}]{Предметная область и онтология Экосистемы OSTIS}
\label{sd_ostis_ecosystem}
\begin{SCn}

\scnsectionheader{\currentname}

\scnstartsubstruct

\scnrelfromlist{дочерний раздел}{\nameref{sd_learning};\nameref{sd_assistants};\nameref{sd_portals};\nameref{sd_ecosys_enterprise}}

\scnheader{Предметная область Экосистемы OSTIS}
\scniselement{предметная область}
\scnsdmainclasssingle{Экосистема OSTIS}
\scnsdclass{ostis-система;самостоятельная ostis-система;поддержка совместимости между компьютерными системами и их пользователями в Экосистеме OSTIS;Экосистемa}
\scnrelfromlist{библиографический источник}{\scncite{DeNicola2021};\scncite{Alrehaili2021};\scncite{Alrehaili2017};\scncite{Shahzad2021};\scncite{Masaharu2018a}}
%\scnsdrelation{***}

\scnheader{Экосистема OSTIS}
\scnidtf{Социотехническая экосистема, представляющая собой коллектив взаимодействующих семантических компьютерных систем и осуществляющая перманентную поддержку эволюции и семантической совместимости всех входящих в нее систем, на протяжении всего их жизненного цикла}
\scnidtf{Неограниченно расширяемый коллектив постоянно эволюционируемых семантических компьютерных систем, которые взаимодействуют между собой и с пользователями для корпоративного решения сложных задач и для постоянной поддержки высокого уровня совместимости и взаимопонимания во взаимодействии как между собой, так и с пользователями}

\scnexplanation{Поскольку \textit{Технология OSTIS} ориентирована на разработку \textit{семантических компьютерных систем}, обладающих высоким уровнем \textit{обучаемости} и, в частности, высоким уровнем семантической \textit{совместимости}, и поскольку обучаемость и совместимость есть только \uline{способность} к обучению (т.е. к высоким темпам расширения и совершенствования своих знаний и навыков), а также \uline{способность} к обеспечению высокого уровня взаимопонимания (согласованности), необходима некая среда, социотехническая инфраструктура, в рамках которой были бы созданы максимально комфортные условия для реализации указанных выше способностей. Такая среда названа нами \textit{\textbf{Экосистемой OSTIS}}, которая представляет собой коллектив взаимодействующих (через сеть Интернет):

\begin{scnitemize}
\item самих \textit{ostis-систем};
\item пользователей указанных \textit{ostis-систем} (как конечных пользователей, так и разработчиков);
\item некоторых компьютерных систем, не являющихся \textit{ostis-системами}, но рассматриваемых ими в качестве дополнительных информационных ресурсов или сервисов.
\end{scnitemize}
}
\scntext{основная задача}{Обеспечить постоянную поддержку совместимости компьютерных систем, входящих в \textit{Экосистему OSTIS} как на этапе их разработки, так и в ходе их эксплуатации. Проблема здесь заключается в том, что в ходе эксплуатации систем, входящих в \textit{Экосистему OSTIS}, они могут изменяться из-за чего совместимость может нарушаться.

Задачами \textit{Экосистемы OSTIS} являются:
\begin{scnitemize}
\item оперативное внедрение всех согласованных изменений стандарта \textit{ostis-систем} (в том числе, и изменений систем используемых понятий и соответствующих им терминов);
\item перманентная поддержка высокого уровня взаимопонимания всех систем, входящих в \textit{Экосистему OSTIS}, и всех их пользователей; 
\item корпоративное решение различных сложных задач, требующих координации деятельности нескольких (чаще всего, априори неизвестных) \textit{ostis-систем}, а также, возможно, некоторых пользователей.
\end{scnitemize}
}
\scnnote{\textit{Экосистема OSTIS} -- это переход от самостоятельных (автономных, отдельных, целостных) \textit{ostis-систем} к коллективам самостоятельных \textit{ostis-систе}м, т.е. к распределенным \textit{ostis-системам}}

\scnheader{ostis-система}
\scnsubdividing{атомарная встроенная ostis-система\\
	\scnaddlevel{1}
		\scnidtf{ostis-система, интегрированная в состав самостоятельной ostis-системы, но не в состав другой встроенной ostis-системы}
	\scnaddlevel{-1};
	неатомарная встроенная ostis-система\\
	\scnaddlevel{1}
		\scnidtf{ostis-система, которая интегрирована в состав самостоятельной ostis-системы, и включает в себя некоторые другие встроенные ostis-системы}
		\scnsuperset{интерфейс ostis-системы}
	\scnaddlevel{-1};
	cамостоятельная ostis-система\\
	\scnaddlevel{1}
		\scnidtf{целостная ostis-система, которая должна самостоятельно решать соответствующее множество задач и, в частности, взаимодействовать с внешней средой (вербально -- с пользователями и другими компьютерными системами, так и невербально)}	
	\scnaddlevel{-1};
	коллектив ostis-систем\\
	\scnaddlevel{1}
		\scnidtf{группа общающихся ostis-систем, в состав которой могут входить не только самостоятельные ostis-системы, но и коллективы ostis-систем}
		\scnidtf{распределенная ostis-система}
	\scnaddlevel{-1};
	Экосистема OSTIS\\
	\scnaddlevel{1}
	\scniselement{максимальный коллектив ostis-систем}
	\scniselement{коллектив ostis-систем, не являющийся частью другого коллектива ostis-систем}
	\scnaddlevel{-1}
	}

\scnheader{cамостоятельная ostis-система}
\scnexplanation{Подчеркнем, что к \textit{\textbf{самостоятельным ostis-системам}}, входящим в состав \textit{Экосистемы OSTIS}, предъявляются особые требования:
\begin{scnitemize}
    \item они должны обладать всеми необходимыми знаниями и навыками для обмена сообщениями и целенаправленной организации взаимодействия с другими \textit{ostis-системам}и, входящими в \textit{Экосистему OSTIS};
    \item в условиях постоянного изменения и эволюции \textit{ostis-систем}, входящих в \textit{Экосистему OSTIS}, каждая из них должна \uline{сама следить за состоянием своей совместимости} (согласованности) со всеми остальными \textit{ostis-системами},  т.е. должна самостоятельно поддерживать эту совместимость, согласовывая с другими ostis-системами все требующие согласования изменения, происходящие у себя и в других системах.
    \item каждая система, входящая в состав \textit{Экосистемы OSTIS}, должна:
    \begin{scnitemizeii}
        \item интенсивно, активно и целенаправленно обучаться ( как с помощью  учителей-разработчиков, так и самостоятельно);
        \item сообщать всем другим системам о предлагаемых или окончательно утвержденных изменениях в \textit{онтологиях} и, в частности, в наборе используемых \textit{понятий};
        \item принимать от других \textit{ostis-систем} предложения об изменениях в \textit{онтологиях} ( в том числе в наборе используемых понятий) для согласования или утверждения этих предложений;
        \item реализовывать утвержденные изменения в \textit{онтологиях}, хранимых в ее базе знаний;
        \item способствовать поддержанию высокого уровня семантической совместимости не только с другими \textit{ostis-системами}, входящими в \textit{Экосистему OSTIS}, но и со своими \textit{пользователями} ( т.е. обучать их, информировать их об изменениях в онтологиях).
    \end{scnitemizeii}
\end{scnitemize}}

\scnheader{Экосистема OSTIS}
\scnexplanation{\textit{Экосистема OSTIS} является формой реализации, совершенствования и применения \textit{Технологии OSTIS} и, следовательно, является формой создания, развития, самоорганизации рынка семантически совместимых компьютерных систем  и включает в себя все необходимые для этого ресурсы --  информационные, технологические, кадровые, организационные, инфраструктурные. 

\textit{Экосистеме OSTIS} ставится в соответствие ее \textit{\textbf{объединенная база знаний}}, которая представляет собой \textbf{виртуальное объединение} \textit{баз знаний} всех \textit{ostis-систем}, входящих в состав \textit{Экосистемы OSTIS}. Качество этой \textit{базы знаний} (полнота, непротиворечивость, чистота) является постоянной заботой всех самостоятельных \textit{ostis-систем}, входящих в состав \textit{Экосистемы OSTIS}. Соответственно этому каждой указанной \textit{ostis-системе} ставится в соответствие своя \textit{база знаний} и своя иерархическая система \textit{sc-агентов}.

По назначению \textit{ostis-системы}, входящие в \textit{Экосистему OSTIS}, могут быть:
\begin{scnitemize}
    \item ассистентами конкретных пользователей или конкретных пользовательских коллективов;
    \item типовыми встраиваемыми подсистемами \textit{ostis-систем};
    \item системами информационной и инструментальной поддержки проектирования различных компонентов и различных классов \textit{ostis-систем};
    \item системами информационной и инструментальной поддержки проектирования или производства различных классов технических и других искусственно создаваемых систем;
    \item порталами знаний по самым различным научным дисциплинам; 
    \item системами автоматизации управления различными сложными объектами (производственными предприятиями, учебными заведениями, кафедрами вузов, конкретными обучаемыми);
    \item интеллектуальными справочными и help-системами;
    \item интеллектуальными обучающими системами, семантическими электронными учебными пособиями;
    \item интеллектуальными робототехническими системами.
\end{scnitemize}
}

\scnresetlevel

\scnheader{поддержка совместимости между компьютерными системами и их пользователями в Экосистеме OSTIS}
\scnexplanation{Есть три аспекта поддержки совместимости и взаимопонимания в \textit{Экосистеме OSTIS}

\begin{scnitemize}
\item поддержка совместимости между самими \textit{ostis-системами}, входящими в \textit{Экосистему OSTIS} в процессе их эволюции;
\item поддержка совместимости между каждой ostis-системой и текущим состоянием Технологии OSTIS в процессе эволюции этой технологии;
\item поддержка совместимости и взаимопонимания между \textit{ostis-системами}, входящими в \textit{Экосистему OSTIS}, и их пользователями при активном стимулировании со стороны \textit{Экосистемы OSTIS} того, чтобы каждый пользователь \textit{Экосистемы OSTIS} был одновременно не только активным ее конечным пользователем, но и активным ее разработчиком.
\end{scnitemize}

Таким образом, для обеспечения высокой эффективности эксплуатации и высоких темпов эволюции  \textit{Экосистемы OSTIS}, необходимо постоянно повышать уровень информационной совместимости (уровень взаимопонимания) не только между компьютерными системами, входящими в состав \textit{Экосистемы OSTIS}, но также между этими системами и их пользователями. Одним из направлений обеспечения такой совместимости является стремление к тому, чтобы \textit{база знаний} (картина мира) каждого пользователя стала частью (фрагментом) \textbf{\textit{Объединенной базы знаний Экосистемы OSTIS}}.  Это значит, что каждый пользователь должен знать, как устроена структура каждой научно-технической дисциплины (объекты исследования, предметы исследования, определения, закономерности и т.д.), как могут быть связаны между собой различные дисциплины.

Формирование таких навыков системного построения картины Мира необходимо начинать со средней школы. Для этой цели необходимо создать комплекс совместимых интеллектуальных обучающих систем по всем дисциплинам среднего образования с четко описанными междисциплинарными связями (\scncite{Bashmakov}, \scncite{Taranchuk2015}). Благодаря этому можно предотвратить формирование у пользователей "мозаичной"{} картины Мира как множества слабо связанных между собой дисциплин. А это, в свою очередь, означает существенное повышение качества образования, которое абсолютно необходимо для качественной эксплуатации компьютерных систем следующего поколения -- \textit{семантических компьютерных систем}.

Пользователи и, первую очередь, разработчики \textit{Экосистемы OSTIS}  должны иметь высокий уровень:
\begin{scnitemize}
\item математической культуры (культуры формализации) при построении формальной модели среды, в которой функционирует интеллектуальная система, формальных моделей решаемых ею задач и формальных моделей различных используемых ею способов решения задач;
\item системной культуры, позволяющей адекватно оценивать качество разрабатываемых систем с точки зрения общей теории систем и, в частности, оценивать общий уровень автоматизации, реализуемый с помощью этих систем. Системная культура предполагает стремление и умение избегать эклектики, стремление и умение обеспечить качественную стратифицированность, гибкость, рефлексивность, а также качественное сопровождение, высокий уровень обучаемости и комфортный пользовательский интерфейс разрабатываемых систем;
\item технологической культуры, обеспечивающей совместимость разрабатываемых систем и их компонентов, а также постоянное расширение библиотеки многократно используемых компонентов создаваемых систем и предполагающей высокий уровень проектной дисциплины;
\item умения работать в команде разработчиков наукоемких систем, что предполагает высокий уровень умения работать на междисциплинарных стыках, высокий уровень коммуникабельности и \uline{договороспособности}, т.е. способности не столько отстаивать свою точку зрения, сколько согласовывать ее  с точками зрения других разработчиков в интересах развития \textit{Экосистемы OSTIS};
\item активности и ответственности за общий результат -- высокие темпы эволюции \textit{Экосистемы OSTIS} в целом.
\end{scnitemize}

Таким образом высокие темпы эволюции \textit{Экосистемы OSTIS} обеспечиваются не только профессиональной квалификацией пользователей (знаниями о \textit{Технологии OSTIS}, о текущем состоянии и проблемах \textit{Экосистемы OSTIS} и навыками использования \textit{Технологии OSTIS} и интеллектуальных систем, входящих в \textit{Экосистему OSTIS}), но и соответствующими человеческими качествами. Очевидно, что современный уровень \uline{договороспособности, активности и ответственности} не может быть основой для эволюции таких систем, как \textit{Экосистема OSTIS}.

Поддержка совместимости \textit{Экосистемы OSTIS} с ее пользователями осуществляется следующим образом:


\begin{scnitemize}
\item в каждую \textit{ostis-систему} включаются встроенные ostis-системы, ориентированные
  
  \begin{scnitemizeii}
        \item на перманентный мониторинг деятельности конечных пользователей и разработчиков этой \textit{\mbox{ostis-системы}},
        \item на анализ качества и, в первую очередь, корректности этой деятельности,
        \item на перманентное ненавязчивое персонифицированное обучение, направленное на повышение качества деятельности пользователей, т.е. на повышение их квалификации;
        
  \end{scnitemizeii}
        
\item в состав \textit{Экосистемы OSTIS} включаются \textit{ostis-системы}, специально предназначенные для обучения пользователей \textit{Экосистемы OSTIS} базовым общепризнанным знаниям и навыкам решения соответствующих классов задач. Сюда входят и знания, соответствующие уровню среднего образования, и знания соответствующие базовым дисциплинам высшего образования в области информатики (и, в том числе, в области искусственного интеллекта), и базовые знания по \textit{Технологии OSTIS} и об \textit{Экосистеме OSTIS}.

\end{scnitemize}}

\scnheader{Экосистема OSTIS}
\scntext{обоснование}{Проблема создания рынка совместимых компьютерных систем --  \textbf{вызов современной науке и технике}.  От ученых, работающих в области искусственного интеллекта требуется умение коллективно работать над решением междисциплинарных проблем и доводить эти решения до общей интегрированной теории интеллектуальных систем, предполагающей интеграцию всех направлений искусственного интеллекта, и до технологий, доступных широкому кругу инженеров. От инженеров интеллектуальных систем требуется активное участие в развитии соответствующих технологий и существенное повышение уровня математической, системный, технологической и организационно-психологической культуры.

Но главной задачей здесь является снижение барьера между научными исследованиями в области искусственного интеллекта и инженерией в области разработки интеллектуальных систем. Для этого наука должна стать конструктивной и ориентированной на интеграцию своих результатов в форме комплексной технологии разработки интеллектуальных систем, а инженерия, осознав наукоемкость своей деятельности, должна активно участвовать в разработке технологий.

Особый акцент в \textit{Экосистеме OSTIS}  делается на постоянный процесс согласования \textit{онтологий} (и, в первую очередь, на согласование семейства всех используемых понятий и терминов, соответствующих этим понятиям) между \uline{всеми} (!) активными субъектами \textit{Экосистемы OSTIS} -- между всеми \textit{ostis-системами} и всеми пользователями.

При наличии \textit{ostis-систем}, являющихся персональными ассистентами пользователей во взаимодействии с \textit{Экосистемой OSTIS}, вся эта Экосистема будет восприниматься пользователями как единая интеллектуальная система, объединяющая все имеющиеся в \textit{Экосистеме OSTIS} информационные ресурсы и сервисы.

Принципы организации \textit{Экосистемы OSTIS} создают все необходимые условия для привлечения к разработке и совершенствованию \textit{Технологии OSTIS} научные, организационные и финансовые ресурсы, которые будут направлены на развитие методов и средств искусственного интеллекта и на формирование рынка семантически совместимых интеллектуальных систем.}

\scnheader{Экосистемa}
\scndefinition{Экосистема определяется как «биологическая система, состоящая из всех организмов, обнаруженных в определенной физической среде,
	взаимодействующих с ней и друг с другом.}

\scnexplanation{Существенное значение концепции \textit{экосистемы} заключается в пяти пунктах:
	
	\begin{scnitemize}
		\item Во-первых, экосистемная концепция анализирует органические сети, основываясь не только на их положительных сторонах, 
		но и на их негативных и конкурентных аспектах: конкуренции на уровне экосистемы, хищничества, паразитизма и разрушения всей системы;
		\item Во-вторых, каждое действующее лицо имеет разные атрибуты, принципы принятия решений и цели. Эти различия могут привести к
		непреднамеренным результатам на уровне экосистемы, хотя принятие решений и поведение каждого субъекта является
		рациональным в данный момент времени;
		\item В-третьих, аналитической границей экосистемы является система продукт/услуга; она не ограничивается национальными границами, 
		региональными кластерами, договорными отношениями или взаимодополняющими поставщиками. 
		В экосистему включены не только бизнес-субъекты, но и некоммерческие субъекты;
		\item В-четвертых, экосистемный анализ требует продольного наблюдения за динамической эволюцией системы товар/услуга;
		\item В-пятых, цели экосистемных исследований заключаются в поиске закономерностей принятия решений и поведенческих цепочек,
		которые сильно влияют на рост и упадок экосистемы при определенных граничных условиях.
	\end{scnitemize}
}

\scntext{основная задача}{Целями экосистемных исследований являются поиск принципов принятия решений и поведенческих цепочек, которые сильно влияют на рост и упадок экосистемы при определенных граничных условиях.
}

\scnsubdividing{перспектива промышленной экологии\\
	\scnaddlevel{1}
	\scnidtf{Перспектива отражена в исследованиях, основанных на концепции промышленной экосистемы, которая была введена \textbf{Фрошем и Галлопулосом} в 1989 году. 
		Авторы использовали понятие природной экосистемы в качестве аналогии для понимания и трансформации индустриальной системы. 
		В статье говорилось, что «традиционная модель промышленной деятельности, в которой отдельные производственные процессы берут 
		сырье и генерируют продукты для продажи, а также отходы, подлежащие утилизации, должна быть преобразована в более интегрированную модель: 
		промышленную экосистему»}
	\scnidtf{Исследователи промышленной экологии внесли свой вклад в реализацию устойчивых промышленных систем в реальном мире. Исследователи, которые использовали термин 		       «промышленная экосистема»,
		обычно сосредотачивались на применении концепции или модели \textbf{IE(промышленной экологии)} к обществу. Таким образом, концепция промышленной экосистемы применялась к реальному обществу в течение значительного времени.
		Исследователи \textbf{IE} предприняли оптимизацию энергетических, материальных и денежных сетей. Таким образом, 
		промышленная экосистема — это не просто концепция, модель или имитационный анализ.}
	\scnidtf{Что касается методологии, то в большинстве существующих исследований промышленных экосистем используется анализ энергетических или материальных потоков.
		Некоторые исследователи создали концептуальные модели промышленных экосистем .
		Расширяя концептуальное моделирование, исследователи применили системную динамику или методы химической инженерии для оптимизации симбиоза, 
		стабильности и устойчивости промышленной экосистемы}
	\scnaddlevel{-1};
	Перспектива бизнес-экосистемы\\
	\scnaddlevel{1}
	\scnidtf{Исследователи, использующие перспективу бизнес-экосистемы \textbf{(BEP)}, концентрируются на бизнес-контексте и устанавливают захват ценности или создание ценности в качестве центральных переменных. Целью исследований в этом потоке является выявление динамики и закономерностей экосистем и организационного поведения. Теоретической основой \textbf{BEP} является теория организационных границ в рамках общей теории стратегического управления. \textbf{Сантос и Эйзенхардт} (2005) разработали четыре концепции границ \textbf{(эффективность, власть, компетентность и идентичность)}.}
	\scnidtf{Анализировались пять различных типов экосистем, основное внимание уделялось конкретным концептуальным отношениям.
		Пять различных типов экосистем следующие: \textbf{1)} цифровые экосистемы; \textbf{2)} взаимодополняющие
		(подотраслевые) экосистемы; \textbf{3)} экосистемы поставщиков; \textbf{4)} экосистемы бизнес-групп (M&A); и 
		\textbf{5)} глобальные профессиональные экосистемы человеческих сетей. 
		Диапазон используемых концепций и определений экосистем является широким:
		\begin{scnitemize}
			\item Самая большая группа в рамках \textbf{BEP} состоит из восьми работ, касающихся анализа цифровых экосистем. Исследования
			в этом кластере сосредоточены на ИТ-индустрии. Нет единого мнения по определению цифровой экосистемы. Тем не менее,
			почти во всех исследованиях цифровых экосистем анализировались сложные отношения на основе ИТ между фирмами в ИТ-отраслях,
			таких как сектор программного обеспечения \textbf{(Iyer et al., 2006)}, рынок приложений \textbf{(Selander et al., 2013)} и операторы
			мобильных сетей \textbf{(Aaltonen and Tempini, 2014)}. Исследования в этом кластере могут быть классифицированы под
			следующей перспективой (перспектива управления платформой); однако эти исследования были сосредоточены не на платформе,
			а на сложных отношениях фирм в цифровой экосистеме, например, экосистемах бизнес-процессов \textbf{(Vidgen and Wang, 2006)}
			и стратегических гиперссылках \textbf{(Dellarocas et al., 2013)}.;
			\item Вторая группа, равная по размеру первой группе, сосредоточилась на комплементарной (подотраслевой) экосистеме. 
			\textbf{Аднер и Капур} описали общую схему экосистемы, состоящей из поставщика, фокусной фирмы, комплементара и клиента. Используя данные отрасли полупроводникового литографического оборудования,
			проверили свою гипотезу о динамике экосистем. \textbf{Капур и Ли} (2013) классифицировали отношения между фокусной организацией и 
			комплементаром и проверили значимость корреляций между этими классификациями и технологическими инвестициями. 
			Эти два исследования положили начало анализу бизнес-экосистемы и создали концептуальную основу для этого подхода.;
			\item Третья группа представляет экосистемный подход поставщиков. Исследователи в этой группе исследовали проблему выбора поставщика
			\textbf{(Viswanadham and Samvedi, 2013)} и создание кооперативных и разнообразных сетей поставщиков \textbf{(Hong and Snell, 2013)}.
			\item Исследование в четвертой группе рассматривает экосистему как агломерированную компанию, связанную слияниями и поглощениями.
			Исследователи проанализировали динамические изменения в бизнес-группах и взаимосвязь с экономическим ростом с использованием 
			данных о слияниях и поглощениях и патентах \textbf{(Gomez-Uranga et al., 2014, Li, 2009)}. Уникальная статья посвящена глобальной сети
			талантов человека STEM (наука, технология, инженерия и математика) как источнику глобальной инновационной экосистемы \textbf{(Lewin and Zhong, 2013)}.
		\end{scnitemize}
	}
	\scnidtf{Что касается методологии, то основными используемыми подходами являются качественные тематические исследования
		и анализ нескольких случаев с использованием оригинального обследования или базы данных. Есть предложения, в которых 
		используется качественная методология или процесс исследования. Визуализация и анализ сети, выполненные с помощью
		специального компьютерного программного обеспечения, были применены в трех работах (Basole, 2009, Battistella et al., 2013, Li, 2009).
		\textbf{Battistella et al.} (2013) создали теоретическое предложение по методологии сетевого анализа бизнес-экосистем (MOBENA).
		Методологически MOBENA имеет потенциальную применимость к другим случаям. \textbf{Hung et al.} (2013) объединили методологию исследования Delphi
		с экосистемным анализом). Дельфийский подход эффективен в достижении тщательного анализа на основе признания реальности.}
	\scnaddlevel{-1};
    Перспектива управления платформой\\
	\scnaddlevel{1}
	\scnidtf{Эта перспектива была представлена \textbf{Кусумано и Гавером} (2002) в «Элементах лидерства платформы».
		Первоначально авторы использовали термин «отраслевая экосистема» в качестве ключевого слова. Согласно недавнему обзору исследований
		платформы, существует множество работ, в которых анализируется механизм платформы. \textbf{Thomas et al.} (2014)
		классифицировали исследование платформы по четырем потокам. Среди них четвертый поток касается платформенных экосистем, которые
		состоят из общеотраслевых сетей, основанных на сложных корреляциях между фирмами. Этот поток напоминает \textbf{PMP}. Это означает,
		что исследования \textbf{PMP(управления платформой)} перекрывают значительную часть исследований платформ и пытаются прояснить связанные с ними сложные сети.
		Кроме того, термины «промышленная экосистема» с точки зрения 
		\textbf{IE(промышленной экологии)} и «отраслевая экосистема» с точки зрения \textbf{PMP(управление платформой)} похожи, но представляют
		собой концепции, которые отличаются определенным образом}	
	\scnidtf{Что касается методологии, то преобладающим методом является интенсивное многотематическое исследование.
		В тематических исследованиях большинство ученых применяли статистические или математические тесты с использованием эмпирических данных.
		Также применялся метод сетевого анализа \textbf{(Weiss and Gangadharan, 2010)}.}	
	\scnaddlevel{-1};
	Многопользовательская сеть\\
	\scnaddlevel{1}
	\scnidtf{Перспектива многопользовательской сети \textbf{(MNP)} является четвертой выявленной перспективой.
		Эта точка зрения была расширена, чтобы включить различных участников (предпринимателей и частных инвесторов, новаторов,
		которые находятся за пределами трубопроводов компаний, пользователей / сообществ пользователей, правительственных бюрократов / политиков и консорциумов). }
	\scnidtf{Направления расширения многопользовательской сети разнообразны. Существует пять областей, в которых проводится экосистемный анализ:
		\textbf{1)} предприниматели/частные инвесторы; \textbf{2)} новаторы, которые находятся за пределами конвейеров компании;
		\textbf{3)} пользователи/сообщества пользователей; \textbf{4)} правительственные бюрократы/директивные органы; 
		и \textbf{5)} консорциумы:
		\begin{scnitemize}
			\item Первая группа – это предприниматели и частные инвесторы. \textbf{Autio et al.} (2014) представили структуру, включающую государственных и частных субъектов, и предложили дефицит контекста 
			и предпринимательских инноваций. Два исследования были сосредоточены на венчурном капитале \textbf{(Samila and Sorenson, 2010)}. Еще два исследования проанализировали региональные кластеры 
			Кремниевой долины \textbf{(Bahrami and Evans, 1995)} и Фландрии \textbf{(Clarysse et al., 2014)}, включая различных действующих лиц. Остальные две статьи являются
			тематическими исследованиями начинающих компаний Chez Panisse \textbf{(Chesbrough et al., 2014)} и Acorn-ARM \textbf{(Garnsey et al., 2008)}.;
			\item Вторая группа рассматривает новаторов, которые находятся за пределами конвейеров компании. Есть три исследования,
			касающиеся реконструкции инновационных экосистем созданными компаниями путем открытия своих инновационных процессов для других,
			с использованием таких подходов, как консорциумы или высокотехнологичные кампусы \textbf{(Leten et al., 2013, Rohrbeck et al., 2009,
			van der Borgh et al., 2012)}.;
			\item Третья группа включает пользователей в их экосистемный анализ.\textbf{Hienerth et al.} (2014)
			проанализировали экосистему Lego и обнаружили, что синергия между фирмами, ведущими пользователями и сообществами 
			пользователей влияет на создание прибыльной и устойчивой экосистемы. Остальные два документа
			были посвящены экосистемам, включая взгляд развивающихся стран со стороны спроса, такие как анализ BOP (дно пирамиды)
			\textbf{(Khavul and Bruton, 2013, Ramachandran et al., 2012)}.
			\item Четвертая группа включает правительственных бюрократов/политиков. \textbf{Watanabe} (1999) использовал экосистемную
			концепцию для анализа влияния политики на промышленные технологии Министерства международной торговли и промышленности
			Японии с 1970-х годов. \textbf{Fabrizio and Hawn} (2013) собрали данные об установках солнечной энергии в США
			и проверили влияние политики выделения солнечной энергии и наличие квалифицированных установщиков.
			Авторы обнаружили, что анализ, принимающий концепцию экосистемы в качестве краеугольного камня, может выявить неопределенные 
			механизмы, связанные с их косвенным воздействием. \textbf{Groesser} (2014) проанализировал стандартизацию строительных норм в Швейцарии и обнаружил
			эволюционно динамичный процесс между добровольными строительными нормами, юридическими
			строительными нормами и стандартами.
		\end{scnitemize}
	}
    \scnidtf{Что касается методологии, системная динамика используется в двух работах \textbf{(Daim et al., 2006, Groesser, 2014)}.
	    Основной методологией является тематическое исследование и/или статистическое тестирование с
	    использованием открытых данных и/или данных обследований.}
    \scnaddlevel{-1}
    }


\bigskip
\scnendstruct \scnendcurrentsectioncomment

\end{SCn}


\scsubsection[\scnmonographychapter{Глава 7.1. Проблемы и перспективы автоматизации различных видов и областей человеческой деятельности с помощью интеллектуальных компьютерных систем нового поколения}]{Предметная область и онтология автоматизируемых видов и областей человеческой деятельности}
\label{tech_human_activity_types}

\scsubsection[\scnidtf{{Характеристики технологий \textit{автоматизации человеческой деятельности}, определяющие качество этих \textit{технологий}}}\protect\scnmonographychapter{Глава 7.1. Проблемы и перспективы автоматизации различных видов и областей человеческой деятельности с помощью интеллектуальных компьютерных систем нового поколения}]{Предметная область и онтология технологий автоматизации различных видов и областей человеческой деятельности}
\label{tech_human_activity}

\scsubsubsection[\scnidtf{История эволюции и современное состояние \textit{технологий} проектирования, реализации, сопровождения, реинжиниринга и использования \textit{компьютерных систем} и, в том числе, \textit{интеллектуальных компьютерных систем} различного назначения. История эволюции \textit{традиционных информационных технологий} и \textit{технологий Искусственного интеллекта}}\protect\scnmonographychapter{Глава 7.1. Проблемы и перспективы автоматизации различных видов и областей человеческой деятельности с помощью интеллектуальных компьютерных систем нового поколения}]{Предметная область и онтология технологий компьютеризации различных видов и областей человеческой деятельности}
\label{trad_comp_tech}

\scsection{Методологические проблемы современного состояния работ в области Искусственного интеллекта}

\label{intro_ostis}

\begin{SCn}

\scnsectionheader{\currentname}

\scnstartsubstruct

\scnsegmentheader{Начало раздела "\currentname"}

\scnstartsubstruct

\scnheader{Методологические проблемы современного состояния работ в области Искусственного интеллекта}

\scnrelfromvector{конкатенация сегментов}
{Структура деятельности в области Искусственного интеллекта;}
\scnauthorcomment{дополнить список}

\scnrelfromset{рассматриваемые вопросы}{
\scnfileitem{Каковы основные стратегические цели (сверхзадачи) научно-технической деятельности в области \textit{Искусственного интеллекта}?};
\scnfileitem{Какие проблемы являются на сегодняшний день актуальными для дальнейшего развития различных направлений \textit{Искусственного интеллекта} и для развития \textit{Искусственного интеллекта} в целом как общей (объединённой) \textit{научно-технической дисциплины}, а также для развития различных форм деятельности в этой области (научно-исследовательской деятельности создания технологий разработки интеллектуальных компьютерных систем, образовательной деятельности, бизнеса)?};
\scnfileitem{Какие проблемы являются на сегодняшний день актуальными для развития других \textit{научно-технических дисциплин} и являются ли эти проблемы аналогичными тем, которые актуальны для развития \textit{Искусственного интеллекта}?};
\scnfileitem{Какие можно предложить подходы к решению указанных выше проблем и как для этого можно использовать создаваемый сейчас новый технологический уклад в области \textit{Искусственного интеллекта} (следующий уровень технологий искусственного интеллекта)?};
\scnfileitem{Как будет выглядеть на основе следующего уровня \textit{технологий Искусственного интеллекта} комплексная автоматизация вех \textit{видов человеческой деятельности}, а также взаимодействие различных \textit{видов человеческой деятельности}, т.е. как будет выглядеть архитектура \textit{smart-общества}?};
\scnfileitem{Устраивает ли нас уровень семантической совместимости взаимопонимания между современными виртуальными компьютерными системами и что необходимо сделать для повышения этого уровня?};
\scnfileitem{Устраивает ли нас уровень семантической совместимости взаимопонимания между современными интеллектуальными компьютерными системами их пользователями и что необходимо сделать для повышения этого уровня?}}
\scntext{аннотация}{Предлагаемое вашему вниманию рассмотрение методологических проблем современного состояния работ в области \textit{Искусственного интеллекта} состоит из следующих частей:
\begin{scnitemize}
\item Анализ актуальных проблем, препятствующих дальнейшему развитию  \textit{Искусственного интеллекта} как \textit{научно-технической дисциплины}:
\begin{scnitemizeii}
\item Проблемы развития научных исследований в области \textit{Искусственного интеллекта} 
\item Проблемы разработки технологий проектирования и реализации \textit{интеллектуальных компьютерных систем};
\item Проблемы формирования рынка \textit{интеллектуальных компьютерных систем}; 
\item Образовательные проблемы в области \textit{Искусственного интеллекта};
\item Проблемы развития бизнеса в области \textit{Искусственного интеллекта}.
\end{scnitemizeii}
\item Анализ проблем автоматизации сложных видов деятельности:
\begin{scnitemizeii}
\item научно-исследовательской деятельности в рамках различных научных дисциплин;
\item создание \textit{технологий проектирования} и производства (реализации) сложных технических систем;
\item \textit{инженерной деятельности} по разработке сложных технических систем;
\item \textit{образовательной деятельности} по наукоёмким техническим специальностям
\end{scnitemizeii}
\item Формулировка принципов, лежащих в основе \textit{Технологии OSTIS}, предназначенной для решения указанных выше проблем;
\item Рассмотрение структуры \textit{Экосистемы OSTIS}, построенной по \textit{Технологии OSTIS} и обеспечивающей комплексную автоматизацию всех видов человеческой деятельности
\end{scnitemize}}

\scnrelfromset{используемые знаки общих понятий и иных сущностей}{деятельность\\
\scnaddlevel{1}
\scnidtf{область деятельности}
\scnsuperset{человеческая деятельность}
\scnaddlevel{-1}
;вид деятельности\\
\scnaddlevel{1}
\scnhaselement{проектирование}
\scnaddlevel{1}
\scnidtf{проектная деятельность}
\scnaddlevel{-1}
\scnhaselement{производство}
\scnaddlevel{1}
\scnidtf{производственная деятельность}
\scnaddlevel{-1}
\scnhaselement{наука}
\scnaddlevel{1}
\scnidtf{научная деятельность}
\scnaddlevel{-2}
;проект\\
\scnaddlevel{1}
\scnsuperset{открытый проект}
\scnaddlevel{-1}
;консорциум
;технология\\
\scnaddlevel{1}
\scnsuperset{информационная технология}
\scnaddlevel{1}
\scnsuperset{технология искусственного интеллекта}
\scnaddlevel{-2}
;кибернетическая система\\
\scnaddlevel{1}
\scnsuperset{интеллектуальная система}
\scnaddlevel{1}
\scnsuperset{интеллектуальная компьютерная система}
\scnaddlevel{1}
\scnidtf{искусственная интеллектуальная система}
\scnaddlevel{-3}
;конвергенция\scnsupergroupsign
\scnaddlevel{1}
\scnidtf{уровень конвергенции (близости)}
\scnsuperset{конвергенция кибернетических систем\scnsupergroupsign}
\scnaddlevel{-1}
;интеграция*\\
\scnaddlevel{1}
\scnsuperset{интеграция кибернетических систем*}
\scnsuperset{эклектичная интеграция*}
\scnsuperset{глубокая интеграция*}
\scnaddlevel{-1}
;интегрированная система\\
\scnaddlevel{1}
\scnsuperset{эклектичная система}
\scnsuperset{гибридная система}
\scnaddlevel{-1}
;экосистема интеллектуальных компьютерных систем
;рынок знаний\\
\scnaddlevel{1}
\scnidtf{рыночная организация порождения эволюции и применения знаний}
\scnaddlevel{-1}
;smart-общество\\
\scnaddlevel{1}
\scnidtf{общество,в основе которого лежит экосистема интеллектуальных компьютерных систем и рынок знаний}
\scnaddlevel{-1}
}
 
\scnrelfromset{ключевые знаки}
{Искусственный интеллект\\
\scnaddlevel{1}
\scniselement{научно-техническая дисциплина}
\scnaddlevel{1}
\scnsubset{научно-техническая деятельность} 
\scnaddlevel{-2};
интеллектуальная система\\
\scnaddlevel{1}
\scnsuperset{интеллектуальная компьютерная система}
\scnaddlevel{-1};
Общая теория интеллектуальных систем;
Базовая комплексная технология проектирования интеллектуальных компьютерных систем;
Технология производства спроектированных интеллектуальных компьютерных систем;
Специализированная инженерия в области Искусственного интеллекта;
Образовательная деятельность в области Искусственного интеллекта;
Бизнес-деятельность в области Искусственного интеллекта\bigskip;
\scnkeyword{Технология OSTIS};
\scnkeyword{ostis-система};
смысловое преставление информации;
агентно-ориентированная модель обработки информации в памяти; стандартизация ostis-систем;
\scnkeyword{SC-код};
абстрактная sc-машина;
конвергенция знаний в памяти;
ostis-систем;
конвергенция моделей решения задач в  ostis-системе;
интеграция знаний в памяти  ostis-системы;
интеграция моделей решения задач в  ostis-системе;
ostis-сообщество;
ostis-технология\\
\scnaddlevel{1}
\scnsuperset{ostis-технология проектирования}
\scnsuperset{ostis-технология производства}
\scnsuperset{технология эксплуатации ostis-систем}
\scnsuperset{технология реинжиниринга ostis-систем}
\scnaddlevel{-1};
\scnkeyword{Ядро Технологии OSTIS}\bigskip;
OSTIS-портал научных знаний в области Искусственного интеллекта;
Проект IMS.ostis;
\scnkeyword{Метасистема IMS.ostis};
Проект Программной реализации универсальной абстрактной sc-машины;
Проект разработки Универсального sc-компьютера;
Специализированная инженерия, осуществляемая на основе Технологии OSTIS;
Образовательная деятельность в области Искусственного интеллекта, осуществляемая на основе технологии OSTIS;
\scnkeyword{Консорциум OSTIS}\bigskip;
\scnkeyword{Экосистема OSTIS};
человеческая деятельность;
вид человеческой деятельности;
автоматизация человеческой деятельности;
качество человеческой деятельности;
субъект Экосистемы OSTIS;
Рынок знаний, реализованный в рамках Экосистемы OSTIS;
smart-общество}

\scnheader{конвергенция в области Искусственного интеллекта}

\scnrelfrom{разбиение}{Направления конвергенции в области Искусственного интеллекта}
     \scnaddlevel{1}
\scnhaselement{конвергенция Искусственного интеллекта со смежными научными дисциплинами}
    \scnaddlevel{1}
\scntext{примечание}{Искусственный интеллект

    \scnaddlevel{-3}
\scnrelboth{смежная дисциплина}{
    $\bullet$ Логика\\
    $\bullet$ Психология человека\\
    $\bullet$ Зоопсихология\\
    $\bullet$ Нейропсихология\\
    $\bullet$ Этология\\
    $\bullet$ Кибернетика\\
    $\bullet$ Общая теория систем\\
    $\bullet$ Семиотика\\
    $\bullet$ Лингвистика\\
    $\bullet$ и др.}
}
\scnaddlevel{2}
\scnhaselement{конвергенция различных направлений Искусственного интеллекта}
\scnaddlevel{1}
\scnidtf{Конвергенция различных направлений исследований в области Искусственного интеллекта, результатом которой должна быть формализованная практически ориентированная общая теория интеллектуальных систем и, в частности, интеллектуальных компьютерных систем. Разобщенность различных направлений исследований в области искусственного интеллекта является главным препятствием создания общей комплексной технологии проектирований интеллектуальных компьютерных систем}
\scnidtf{Конвергенция между различными направлениями и продуктами научных исследований в области искусственного интеллекта. Результатом (целевым продуктом) такой конвергенции должна стать общая формальная теория интеллектуальных компьютерных систем}
\scnaddlevel{-1}
\scnhaselement{конвергенция различного вида знаний в памяти интеллектуальной компьютерной системы}
\scnaddlevel{1}
\scnidtf{Конвергенция и интеграция внутреннего представления в памяти интеллектуальной компьютерной системы различного вида знаний}
\scnaddlevel{-1}
\scnhaselement{конвергенция различных моделей решения задач в памяти интеллектуальной компьютерной системы}
\scnaddlevel{1}
\scnidtf{Конвергенция и интеграция различных моделей решения задач
\begin{itemize}
    \item логико-семантическая типология задач
    \item типология моделей решения задач (задача, класс задач,метод, класс методов, модель решения задач=иерархический метод интерпретации класса методов)
\end{itemize}}
\scnaddlevel{-1}
\scnhaselement{конвергенция интеллектуальных компьютерных систем}
\scnaddlevel{1}
\scnidtf{Обеспечение семантической совместимости (взаимпопонимания) интеллектуальных систем, согласование используемых онтологий}
\scnidtf{Конвергенция между различными прикладными компьютерными системами. Результатом (целевым продуктом) такой конвергенции должна стать экосистема, состоящая из перманентно эволюционируемых, семантически совместимых и взаимодействующих интеллектуальных компьютерных систем, а также их пользователей}
\scnexplanation{Конвергенция (семантическая совместимость) всех разрабатываемых интеллектуальных компьютерных систем(в том числе прикладных), преобразующая набор индивидуальных (самостоятельных) интеллектуальных компьютерных систем различного назначения в коллектив активно взаимодействущих интеллектуальных компьютерных систем для совместного (коллективного) решения сложных (комплексных) задач и для перманентной поддержки семантической совместимости в ходе индивидуальной эволюции каждой интеллектуальной компьютерной системы}
\scnaddlevel{-1}
\scnhaselement{конвергенция средств автоматизации проектирования различного вида компонентов интеллектуальных компьютерных систем}
\scnaddlevel{1}
\scnidtf{Конвергенция (семантическая совместимость) средств автоматизации проектирования различного вида компонентов интеллектуальных компьютерных систем, результатом которой должен быть общий комплекс средств автоматизации проектирования всех компонентов интеллектуальных компьютерных систем}
\scnidtf{Конвергенция между инструментальными средствами, обеспечивающими автоматизацию проектирования различных компонентов или различных классов интеллектуальных компьютерных систем. Результатом (целевым продуктом) такой конвергенции должен стать единый комплекс методологических и инструментальных средств, ориентированный на поддержку комплексного проектирования любых интеллектуальных компьютерных систем}
\scnaddlevel{-1}
\scnhaselement{конвергенция логико-семантических моделей интеллектуальных компьютерных систем}
\scnaddlevel{1}
\scnnote{\textit{логико-семантические модели интеллектуальных компьютерных систем} являются результатом ("сухим" остатком) \textit{проектирования} этих систем и представляют собой формальное представления исходного (начального) состояния \textit{баз знаний} разрабатываемых \textit{интеллектуальных компьютерных систем}}
\scnaddlevel{-1}
\scnhaselement{конвергенция средств интерпретации логико-семантических моделей разрабатываемых интеллектуальных компьютерных систем}
\scnaddlevel{1}
\scnhaselement{конвергенция средств интерпретации логико-семантических моделей разрабатываемых интеллектуальных компьютерных систем}
\scnaddlevel{1}
\scnidtf{Конвергенция (совместимость) средств реализации (производства) интеллектуальных компьютерных систем на основе спроектированных формальных моделей создаваемых интеллектуальной компьютерной системой (средств интерпретации спроектированных моделей интеллектуальных компьютерных систем). Такая интерпретация может осуществляться либо программным путем на современных компьютерах, либо путем создания принципиально новых компьютеров, специально ориентированных на интерпретацию формальных моделей интеллектуальных компьютерных систем, помещаемых в память указанных компьютеров}
\scnaddlevel{-1}
\scnhaselement{конвергенция между информационно-программных и аппаратным обеспечением интеллектуальных компьютерных систем}
\scnaddlevel{1}
\scnidtf{Конвергенция между Software и Hardware интеллектуальных компьютерных систем}
\scnaddlevel{-1}
\scnhaselement{Конвергенция различных форм деятельности в области Искусственного интеллекта}
\scnaddlevel{1}
\scnidtf{Конвергенция между
\begin{scnitemize}
    \item научными исследованиями по созданию общей теории интеллектуальных компьютерных систем;
    \item разработкой средств автоматизации проектирования интеллектуальных компьютерных систем;
    \item разработкой средств интерпретации спроектированныз формальных моделей интеллектуальных компьютерных систем;
    \item разработкой прикладных интеллектуальных компьютерных систем различного назначения;
    \item подготовкой и перманентным повышением квалификации кадров, способных эффективно участвовать во всех перечисленных направлениях деятельности.
    \end{scnitemize}

Глубокая конвергенция между всеми этими формами деятельности возможна только тогда, когда \uline{каждый} участник создания комплексной технологии искусственного интеллекта является участником \uline{каждой} из перечисленныз форм деятельности.    
}
\scnidtf{Конвергенция между (1) научно-исследовательской деятельностью в области искусственного интеллекта; (2) инженерно-технологической деятельностью, которая направлена на разработку комплексной технологии проектирования интеллектуальных компьютерных систем и которая имеет высокий уровень наукоемкости; (3) инженерно-прикладной деятельностью, которая направлена на разработку прикладных интеллектуальных систем и которая также имеет высокий уровень наукоемкости, обусловоенной необходимостью качественной формализации соответствующих предметных областей и, в частности, методов решения задач в этих областях; (4) образованием (образовательной деятельностью) в области искусственного интеллекта, повышение эффективности которого настоятельно требует раннего и поэтапного вовлечения студентов в реальные, а не учебные проекты - сначала в инженерно-прикладные, потом в инженерно- исследовательские проекты; (5) деятельностью, направленной на создание инфраструктуры, обеспечивающей поддержку открытого массового активного международного сотрудничества по консолидации усилий, направленных на решение современных проблем в области искусственного интеллекта; (6) бизнесом в области искусственного интеллекта, который не просто должен обеспечить финансовую поддержку перечисленных видов деятельности, но и обеспечить грамотный баланс между ними, грамотное сочетания тактических и стратегических целей}

\scnheader{Искусственный интеллект}
\scnrelfromset{методологические проблемы текущего состояния}{
\scnfileitem{Далеко не всеми учеными, работающими в области искусственного интеллекта принимается прагматичность практической направленности этой науки;}
;\scnfileitem{Не всеми принимается необходимость конвергенции различных направлений искуссвенного интеллекта и необходимость их интеграции в целях построения общей теории интеллектуальных систем;}
;\scnfileitem{Нет движения к построению общей компьютерной технологии интеллектуальных компьютерных систем;}
;\scnfileitem{Нет движения к построению экосистем интеллектуальных компьютерных систем;}
;\scnfileitem{Не всеми принимается необходимость конвергенции различных форм деятельности в области искуственного интеллекта}}
\scnnote{Современная трактовка целей и задач Искусственного интеллекта как научно-технической дисциплины требует переосмысления, так как, к сожалению, носит несогласованный, а часто и значительно более узкий характер, чем этого требует текущее положение}


\scnheader{Бизнес-деятельность в области Искусственного интеллекта}
\scntext{текущее состояние}{Острая потребность в существенном повышении уровня автоматизации в самых различных областях человеческой деятельности (в промышленности, медицине, транспорте, образовании, строительстве и во многих других), а также современные результаты в развитии \textit{технологий Искусственного интеллекта} привели к существенному расширению работ по созданию \textit{прикладных интеллектуальных компьютерных систем} и к появлению большого количества коммерческих организаций, ориентированных на разработку таких приложений.}
\scnrelfromset{проблемы текущего состояния}{
\scnfileitem{Не так просто обеспечить баланс тактических и стратегических направлений развития всех форм деятельности в области \textit{Искусственного интеллекта} (научно-исследовательской деятельности, разработки технологии проектирования и производства интеллектуальных компьютерных систем, разработки прикладных систем, образовательной деятельности), а также баланс между всеми перечисленными формами деятельности.}
;\scnfileitem{В настоящее время отсутствует глубокая конвергенция различных форм деятельности в области \textit{Искусственного интеллекта} (в первую очередь, конвергенция развития технологий \textit{Искусственного интеллекта} и разработки различных прикладных интеллектуальных компьютерных систем), что существенно затрудняет развитие каждой из этих форм.}
;\scnfileitem{Высокий уровень наукоемкости работ в области \textit{Искусственного интеллекта} предъявляет особые требования к квалификации сотрудников и к их способности работать в составе творческих коллективов.}
;\scnfileitem{Для повышения квалификации своих сотрудников и для обеспечения высокого уровня своих разработок необходимо активное сотрудничество с различными научными школами, с кафедрами, осуществляющими подготовку молодых специалистов в области \textbf{\textit{Искусственного интеллекта}}, активное участие в подготовке и проведении соответствующих конференций, семинаров, выставок.}}

\scnheader{Искусственный интеллект}
\scnrelfromset{сверхзадачи текущего состояния}{
\scnfileitem{Построение и перманентное развитие \textit{общей формальной теории интеллектуальных систем}}
\scnaddlevel{1}
\scnrelfromset{подзадачи}{
\scnfileitem{Уточнение требований, предъявляемых к интеллектуальным компьютерным системам – уточнение свойств интеллектуальных компьютерных систем, определяющих высокий уровень их интеллекта.}
;\scnfileitem{Конвергенция и интеграция всевозможных видов знаний и всевозможных моделей решения задач в рамках каждой интеллектуальной компьютерной системы.}
;\scnfileitem{Ориентация на последующую разработку унифицированных семантически совместимых формальных моделей интеллектуальных систем.}
;\scnfileitem{Ориентация на разработку различного вида универсальных интерпретаторов формальных моделей интеллектуальных систем (и в том числе компьютеров нового поколения ) и обеспечение четкой стратификации между формальными моделями интеллектуальных систем и различными вариантами построения их интерпретаторов, обеспечивающей высокую степень независимости эволюции формальных моделей интеллектуальных систем и эволюции их интерпретаторов. Это требует особой детализации формальных моделей интеллектуальных систем.}
;\scnfileitem{Обеспечение коммуникационной ("социальной"{}) совместимости (договороспособности) интеллектуальных компьютерных систем, позволяющей им самостоятельно формировать коллективы интеллектуальных компьютерных систем и их пользователей, а также самостоятельно согласовывать (координировать) деятельность в рамках этих коллективов при решении сложных задач в непредсказуемых условиях. Без этого невозможна реализация таких проектов, как "умный"{} дом, "умный"{} город, "умное"{} предприятие, "умная"{} больница и т.д.}}
\scnaddlevel{-1}
;\scnfileitem{Создание и перманентное развитие \textit{общей комплексной технологии} проектирования и производства \textit{семантически совместимых} \scnbigspace \textit{интеллектуальных компьютерных систем}, способных координировать свою деятельность с себе подобными}
\scnaddlevel{1}
\scnrelfromset{подзадачи}{
\scnfileitem{Четкое описание стандарта интеллектуальных компьютерных систем, обеспечивающего семантическую совместимость разрабатываемых систем}
;\scnfileitem{Разработка мощных библиотек семантически совместимых и многократно (повторно) используемых компонентов разрабатываемых интеллектуальных компьютерных систем}
;\scnfileitem{Обеспечение низкого порога вхождения в технологию проектирования интеллектуальных компьютерных систем как для пользователей технологии (т.е. разработчиков прикладных или специализированных интеллектуальных компьютерных систем), так и для разработчиков самой технологии}
;\scnfileitem{Обеспечение высоких темпов развития технологии за счет учета опыта разработки различных приложений путем активного привлечения авторов приложений к участию в развитии (совершенствовании) технологии}}
\scnaddlevel{-1}
;\scnfileitem{Разработка компьютеров нового поколения, ориентированных на производство высокопроизводительных \textit{интеллектуальных компьютерных систем} самого различного назначения и высокого качества}
;\scnfileitem{Создание глобальной \textit{экосистемы} взаимодействующих между собой \textit{интеллектуальных компьютерных систем}, обеспечивающих комплексную автоматизацию всех \textit{видов человеческой деятельности}}
\scnaddlevel{1}
\scntext{подзадача}{Построение формальной модели человеческой деятельности в контексте теории smart-общества}
\scnaddlevel{-1}
;\scnfileitem{Создание и перманентное развитие глобальной \textit{социотехнической экосистемы}, которая состоит из \textit{интеллектуальных компьютерных систем}, а также всех пользователей этих систем, которая обеспечивает комплексную автоматизацию всех \textit{видов человеческой деятельности}}
;\scnfileitem{Необходим переход от эклектичного построения сложных \textit{интеллектуальных компьютерных систем}, использующих различные виды \textit{знаний} и различные виды \textit{моделей решения задач}, к их глубокой \textbf{интеграции} и унификации, когда одинаковые модели представления и модели обработки знаний реализуется в разных системах и подсистемах одинаково}
;\scnfileitem{Необходимо сократить дистанцию между современным уровнем \textbf{\textit{теории интеллектуальных компьютерных систем}} и практики их разработки.}}

\scnheader{Искусственный интеллект}
\scnidtf{Деятельность в области Искусственного интеллекта (как совокупность всех форм и направлений этой деятельности)}
\scntext{проблема текущего состояния}{Эпицентром современных проблем развития деятельности в области \textit{Искусственного интеллекта} является \textit{конвергенция} и \textit{глубокая интеграция} всех форм, направлений и результатов этой деятельности. Уровень взаимосвязи, взаимодействия и \textit{конвергенции} между различными формами и направлениями деятельности в области \textit{Искусственного интеллекта} явно недостаточен. Это приводит к тому, что каждая из них развивается обособленно, независимо от других.  Речь идет о \textit{конвергенции} между такими направлениями \textit{Искусственного интеллекта}, как представление знаний, решение интеллектуальных задач, интеллектуальное поведение, понимание и др., а также между такими формами \textit{человеческой деятельности в области Искусственного интеллекта}, как научные исследования, разработка технологий, разработка приложений, образование, бизнес. 
Почему на фоне уже достаточно длительного интенсивного развития научных исследований в области \textit{Искусственного интеллекта} до сих пор не создан рынок интеллектуальных компьютерных систем и комплексная технология \textit{Искусственного интеллекта}, обеспечивающая разработку широкого спектра \textit{интеллектуальных компьютерных систем} самого различного назначения и доступной широкому контингенту инженеров. 
Потому что сочетание высокого уровня наукоемкости и прагматизма этой проблемы требует для ее решения принципиально нового подхода к организации взаимодействия \textit{\uline{ученых}}, работающих в области \textit{Искусственного интеллекта}, \textit{\uline{разработчиков}} средств автоматизации проектирования \textit{интеллектуальных компьютерных систем}, \uline{\textit{разработчиков}} средств реализации интеллектуальных компьютерных систем, включая средства аппаратной поддержки интеллектуальных компьютерных систем, \uline{\textit{разработчиков}} прикладных интеллектуальных компьютерных систем. Такое \uline{целенаправленное} взаимодействие должно осуществляться как в рамках каждой из этих форм деятельности в области \textit{Искусственного интеллекта}, так и между ними. Таким образом, основной тенденцией дальнейшего развития теоретических и практических работ в области \textit{Искусственного интеллекта} является конвергенция как самых разных видов (форм и направлений) человеческой деятельности в области \textit{Искусственного интеллекта}, так и самых разных продуктов (результатов) этой деятельности. Необходимо ликвидировать барьеры между различными видами и продуктами деятельности в области \textit{Искусственного интеллекта} в целях обеспечения их совместимости и интегрируемости.
Проблема создания быстро развивающегося рынка семантически совместимых интеллектуальных систем – это вызов, адресованный специалистам в области \textit{Искусственного интеллекта}, требующий преодоления "вавилонского столпотворения"{} во всех его проявлениях, формирование высокой культуры договороспособности и унифицированной, согласованной формы представления коллективно накапливаемых, совершенствуемых и используемых знаний.
Ученые, работающие в области \textit{Искусственного интеллекта}, должны обеспечить конвергенцию результатов различных направлений \textit{Искусственного интеллекта} и построить: (1) общую теорию интеллектуальных компьютерных систем; (2) общую технологию проектирования семантически совместимых интеллектуальных компьютерных систем, включающую соответствующие стандарты интеллектуальных компьютерных систем и их компонентов. Инженеры, разрабатывающие интеллектуальные компьютерные системы, должны сотрудничать с учеными и участвовать в развитии технологии проектирования интеллектуальных компьютерных систем.}

\newpage
\scnsegmentheader{Понятие технологии OSTIS}

\scnstartsubstruct

\scnheader{Технология OSTIS}
\scnidtf{Комплекс (семейство) технологий, обеспечивающих проектирование, производство, эксплуатацию и реинжиниринг интеллектуальных \textit{компьютерных систем (ostis-систем)}, предназначенных для автоматизации самых различных видов человеческой деятельности и в основе которых лежит смысловое представление и онтологическая систематизация знаний, а также агентно-ориентированная обработка знаний}
\scnidtf{Open Semantic Technology for Intelligent Systems}
\scnaddlevel{1}
\scntext{сокращение}{OSTIS}
\scnaddlevel{-1}
\scnidtf{Семейство (комплекс) \textit{ostis-технологий}}
\scnidtf{Комплексная открытая семантическая технология проектирования, производства, эксплуатации и реинжиниринга гибридных, семантически совместимых, активных и договороспособных \textit{интеллектуальных компьютерных систем}}
\scnheader{Технология OSTIS}
\scnrelfromset{принципы, лежащие в основе}{
\scnfileitem{Ориентация на разработку \textit{интеллектуальных компьютерных систем}, имеющих высокий уровень \textit{интеллекта} и, в частности, высокий уровень \textit{социализации}. Указанные системы, разработанные по \textit{Технологии OSTIS}, будем называть \textbf{ostis-системами}}
;\scnfileitem{Ориентация на \uline{комплексную} автоматизацию всех видов и областей \textit{человеческой деятельности} путем создания сети взаимодействующих и координирующих свою деятельность \textit{ostis-систем}. Указанную сеть \textit{ostis-систем} вместе с их пользователями будем называть \textbf{\textit{Экосистемой OSTIS}}}
;\scnfileitem{Поддержка перманентной эволюции \textit{ostis-систем} в ходе их эксплуатации.}
;\scnfileitem{\textit{Технология OSTIS} реализуется в виде сети \textit{ostis-систем}, которая является частью \textit{Экосистемы OSTIS}.
Ключевой \textit{ostis-системой} указанной сети является \textbf{Метасистема IMS.ostis} (Intelligent MetaSystem), реализующая \textbf{Ядро Технологии OSTIS}, которое включает в себя базовые (предметно независимые) методы и средства проектирования и производства \textit{ostis-систем} с интеграцией в их состав типовых встроенных подсистем поддержки эксплуатации и реинжиниринга \textit{ostis-систем}. Остальные \textit{ostis-системы}, входящие в состав рассматриваемой сети, реализуют различные специализированные \textit{ostis-технологии} проектирования различных классов \textit{ostis-систем}, обеспечивающих автоматизацию любых областей и \textit{видов человеческой деятельности}, кроме \textit{проектирования ostis-систем}.}
;\scnfileitem{Конвергенция и интеграция на основе смыслового представления знаний всевозможных научных направлений \textit{Искусственного интеллекта} (в частности, всевозможных базовых знаний и навыков решения интеллектуальных задач) в рамках \textit{Общей формальной семантической теории ostis-систем}.}
;\scnfileitem{Ориентация на разработку компьютеров нового поколения, обеспечивающих эффективную (в том числе, производительную) интерпретацию логико-семантических моделей \textit{ostis-систем}, представленных базами знаний этих систем и имеющих смысловое представление.}}

\scnsegmentheader{Понятие ostis-системы}

\scnstartsubstruct

\scnheader{ostis-система}
\scnidtf{\textit{интеллектуальная компьютерная система}, спроектированная и реализованная по требованиям и стандартам \textit{Технологии OSTIS}, которые задокументированы в \textit{Общей теории ostis-систем}}

\scnheader{ostis-система}
\scnidtf{интеллектуальная компьютерная система, построенная в соответствии с принципами и требованиями Технологии OSTIS}
\scnidtf{Множество ostis-систем различного назначения}
\scnaddlevel{1}
\scniselement{имя собственное}
\scnaddlevel{-1}
\scnidtf{Множество всевозможных интеллектуальных компьютерных систем, построенных по Технологии OSTIS}

\scnheader{ostis-система}
\scnsubset{интеллектуальная компьютерная система}
\scnidtf{\textit{интеллектуальная компьютерная система}, которая построена в соответствии с требованиями и стандартами \textit{Технологии OSTIS}, что обеспечивает существенное развитие целого ряда \textit{свойств} (способностей) этой \textit{компьютерной системы}, позволяющих значительно повысить \textit{уровень интеллекта} этой системы (и, прежде всего, ее \textit{уровень обучаемости} и \textit{уровень социализации})} 

\scnauthorcomment{Добавить классификацию из пояснения}

\scnsubdividing{индивидуальная ostis-система;коллективная ostis-система\\
\scnaddlevel{1}
    \scnsubdividing{простой коллектив ostis-систем;иерархический коллектив ostis-систем}   
\scnaddlevel{-1}
}

\scnheader{ostis-система}
\scnexplanation{интеллектуальная компьютерная система, разработанная, разрабатываемая или совершенствуемая по технологии OSTIS}
\scnnote{Когда речь идет о таком компоненте технологии OSTIS, как модели ostis-систем, фактически имеется в виду теория ostis-систем, включающая в себя строгое формальное уточнение того, как устроена ostis-система, какова ее архитектура, принципы организации памяти, принципы организации представления информации, принципы организации интерфейса с внешней средой (в том числе, с пользователями)}

\scnheader{ostis-система}
\scnrelfromset{принципы, лежащие в основе}{
\scnfileitem{Информация, хранимая в памяти \textit{ostis-системы}, имеет смысловое представление.}
;\scnfileitem{В основе организации решения задач в памяти \textit{ostis-системы} лежит \textit{агентно-ориентированная модель обработки информации}, управляемая ситуациями и событиями, возникающими в обрабатываемой информации (точнее, в обрабатываемой базе знаний).}
;\scnfileitem{Унификация базового набора (базовой системы) используемых понятий, что является основой обеспечения \textit{семантической совместимости} всех \textit{ostis-систем}.}
;\scnfileitem{В основе структуризации информации (базы знаний), хранимой в памяти \textit{ostis-системы}, лежит иерархическая система \textit{предметных областей} и соответствующих им \textit{формальных онтологий}.}
;\scnfileitem{Способность к пониманию (к семантическому погружению, к семантической интеграции) новых приобретаемых знаний (и, в том числе, новых навыков) в состав текущего состояния \textit{базы знаний}.}
;\scnfileitem{Способность к \textit{семантической конвергенции} (к обнаружению сходств) новых приобретаемых знаний (и, в частности, навыков) со знаниями, входящими в состав текущего состояния базы знаний \textit{ostis-системы}.}
;\scnfileitem{Способность \textit{ostis-системы} поддерживать высокий уровень своей \textit{семантической совместимости} (высокий уровень взаимопонимания) с другими \textit{ostis-системами}.}
;\scnfileitem{Способность ostis-системы согласовывать, координировать свою деятельность с другими \textit{ostis-системами}.}
;\scnfileitem{\scnauthorcomment{статья на OSTIS-2020}}
;\scnfileitem{\scnauthorcomment{статья на OSTIS-2020}}
;\scnfileitem{\scnauthorcomment{статья на OSTIS-2020}}
;\scnfileitem{\scnauthorcomment{статья на OSTIS-2020}}}
\scntext{следовательно}{Перечисленные свойства \textit{ostis-систем} свидетельствуют о том, что они имеют существенно более высокий \textit{уровень интеллекта} и, в частности, более высокий \textit{уровень социализации} по сравнению с современными \textit{интеллектуальными компьютерными системами}. \scnauthorcomment{См. начало Раздела 1.1}}

\scnheader{ostis-система}
\scnrelfromset{принципы, лежащие в основе}{
\scnfileitem{смысловое представление информации в памяти компьютерных систем, направленное на устранение недостатков современных компьютерных систем и технологий путем повышения уровня интеллектуальности компьютерных систем}
;\scnfileitem{децентрализация управления решателем задач
\begin{itemize}
	\item внутренняя МАС
	\item внешняя МАС
\end{itemize}}
;\scnfileitem{интеграция различных видов знаний}
;\scnfileitem{интеграция различных моделей решателей задач}
;\scnfileitem{ориентация на компьютеры нового поколения}
;\scnfileitem{обеспечение семантической совместимости компьютерных систем}
;\scnfileitem{обеспечение поддержания семантической совместимости компьютерных систем в ходе эволюции}
;\scnfileitem{способность к координации деятельности}}

\scnheader{ostis-система}
\scnrelfromset{принципы, лежащие в основе}{
\scnfileitem{Память ostis-системы является графодинамической (т.е. нелинейной (графовой) и структурно перестраиваемой). Переработка информации в памяти ostis-системы сводится не столько к изменению состояния элементов памяти (это происходит только при изменении синтаксического типа элементов и при изменении содержимого тех элементов, которые обозначают файлы), сколько к изменению \uline{конфигурации связей} между ними.}
;\scnfileitem{Хранение информации в памяти ostis-системы ориентируется на \uline{смысловое} представление информации – без синонимов, омонимов знаков и без семантической эквивалентности информационных конструкций.}
;\scnfileitem{С точки зрения архитектуры ostis-система представляет собой \uline{иерархическую} многоагентную систему с общедоступной памятью (т.е. с памятью, общедоступной \uline{всем} агентам ostis-системы). 
Заметим при этом, что общая память большинства исследуемых в настоящее время многоагентных систем является не общедоступной, а распределенной, т.е. представляет собой абстрактное (виртуальное) объединение, в состав которого входит память каждого агента многоагентной системы. Координация деятельности агентов ostis-системы при выполнении сложных \textit{действий в памяти} ostis-системы реализуется также через \textit{память ostis-системы} с помощью хранимых в памяти \textit{методов} решения различных классов задач, а также с помощью хранимых в памяти \textit{планов} решения конкретных задач.
На основании этого можно строить неограниченную иерархическую систему агентов ostis-системы – от элементарных агентов, обеспечивающих выполнение базовых действий в памяти ostis-системы, до неэлементарных агентов, представляющих собой коллективы (группы) элементарных и/или неэлементарных агентов, обеспечивающих решение различных типовых задач с помощью соответствующих методов и планов.}
;\scnfileitem{Организация выполнения \textit{ostis-системой действий во внешней среде} осуществляется следующим образом:
\begin{scnitemize}
	\item Выделяются классы \textit{элементарных действий во внешней среде}, для реализации каждого из которых вводятся \textit{эффекторные агенты} ostis-системы.
	\item Координация деятельности \textit{эффекторных агентов} ostis-системы при выполнении \textit{сложных действий во внешней среде} осуществляется через \textit{память ostis-системы} с помощью хранимых в памяти \textit{методов} и \textit{планов} решения различных задач во \textit{внешней среде}, а также с помощью \textit{рецепторных агентов} ostis-системы, обеспечивающих обратную связь и, соответственно, сенсомоторную координацию.
\end{scnitemize}}}

\scnheader{ostis-система}
\scnrelfromset{принципы, лежащие в основе}{
\scnfileitem{Способность понимать друг друга, а также любого своего пользователя
\scnaddlevel{-2}
\scnidtf {Совместимость используемых понятий (по терминам и по денотационной семантике)}
\scnidtf {Семантическая совместимость}
\scnaddlevel{2}}
;\scnfileitem{Способность поддерживать взаимопонимание в процессе индивидуальной эволюции, приводящей к расширению и/или корректировке системы используемых понятий}
;\scnfileitem{Способность координировать свою деятельность с другими системами при решении задач, которые усилиями одной (индивидуальной) интеллектуальной компьютерной системы не могут быть решены либо принципиально, либо за разумное время}}

\scnheader{ostis-система}
\scnrelfromset{принципы, лежащие в основе}{
\scnfileitem{Высокая степень индивидуальной обучаемости интеллектуальных компьютерных систем
\begin{itemize}
	\item гибкости
	\item стратифицированности
	\item рефлексивности
	\item универсальность средств представления и образования знаний
\end{itemize}}
;\scnfileitem{Высокая степень семантической совместимости и, как следствие, коллективной обучаемости интеллектуальных компьютерных систем
\begin{itemize}
	\item семантической совместимости
\end{itemize}}
;\scnfileitem{Основа для автоматизации рынка знаний}}

\scnmakeset{память*;ostis-система}
\scnrelfrom{сужение второго домена заданного отношения для заданного первого домена}{память ostis-системы}
\scnaddlevel{1}
\scnsubset{смысловая память}
\scnaddlevel{-1}


\scnmakeset{информация, хранимая в памяти кибернетической системы*;  ostis-система}
\scnrelfrom{сужение второго домена заданного отношения для заданного первого домена}{база знаний ostis-системы}
\scnaddlevel{1}
\scnsubset{смысловое представление информации}
\scnaddlevel{-1}


\scnheader{память ostis-системы}
\scnsubset{смысловая память}

\scnheader{информация, хранимая в памяти ostis-системы}
\scnsubset{смысловое представление информации}

\scnheader{решатель задач ostis-системы }
\scnsubset{агентно-ориентированная модель обработки информации в памяти}

\newpage
\scnsegmentheader{Текущее состояние и проблемы дальнейшего развития деятельности в области Искусственного интеллекта}
\scnstartsubstruct

\scntext{аннотация}{Рассмотрим в каких направлениях должна происходить эволюция повышенного качества деятельности в области \textit{Искусственного интеллекта}, а также эволюция продуктов этой деятельности}

\bigskip
\scnfragmentcaption

\scnheader{Научно-исследовательская деятельность в области Искусственного интеллекта}
\scntext{текущее состояние}{}
\scnauthorcomment{добавить из статей}

\scnheader{Научно-исследовательская деятельность в области Искусственного интеллекта}
\scnrelfromset{проблемы текущего состояния}{
\scnfileitem{Отсутствует согласованность систем \textit{понятий} в разных направлениях \textit{Искусственного интеллекта} и, как следствие, отсутствует \textit{семантическая совместимость} и \textit{конвергенция} этих направлений, в результате чего ни о каком движении в направлении построения \textit{общей теории интеллектуальных систем} с высоким уровнем формализации и речи быть не может. Существование и продолжающееся увеличение "высоты барьеров"{} между различными направлениями исследований в области \textit{Искусственного интеллекта} проявляется в том, что специалист, работающий в рамках какого-либо направления \textit{Искусственного интеллекта}, посещая заседания "не своей"{} секции на конференции по \textit{Искусственному интеллекту}, мало что там может понять и, соответственно, извлечь полезного для себя.};
\scnfileitem{Отсутствует мотивация и осознание острой необходимости в указанной \textit{конвергенции} между различными направлениями \textit{Искусственного интеллекта}.};
\scnfileitem{Отсутствует реальное движение в направлении построения \textit{Общей теории интеллектуальных систем}, поскольку отсутствует соответствующая мотивация и осознание острой практической необходимости в этом.}
}

\bigskip
\scnfragmentcaption

\scnheader{Разработка базовой комплексной технологии проектирования интеллектуальных компьютерных систем}
\scntext{текущее состояние}{Современная технология \textit{Искусственного интеллекта} представляет собой целое семейство всевозможных частных технологий, ориентированных на разработку и сопровождение различного вида компонентов \textit{интеллектуальных компьютерных систем}, реализующих самые различные модели представления и обработки информации, различные модели решения задач, ориентированных на разработку различных классов \textit{интеллектуальных компьютерных систем}.}
\scnrelfromset{проблемы текущего состояния}{
\scnfileitem{высокая трудоемкость разработки интеллектуальных компьютерных систем};
\scnfileitem{необходимая высокая квалификация разработчиков};
\scnfileitem{современные технологии \textit{Искусственного интеллекта} принципиально не обеспечивают разработки таких \textit{интеллектуальных компьютерных систем}, в которых устраняются недостатки современных \textit{интеллектуальных компьютерных систем}};
\scnfileitem{совместимость частных технологий \textit{Искусственного интеллекта} практически отсутствует и, как следствие, отсутствует \textit{семантическая совместимость} разрабатываемых \textit{интеллектуальных компьютерных систем}, поэтому их системная интеграция осуществляется \uline{вручную}.};
\scnfileitem{Разрабатываемые \textit{интеллектуальные компьютерные системы} не способны \uline{самостоятельно} координировать свою деятельность друг с другом следовательно
\begin{scnitemize}
\item{нет общей комплексной технологии проектирования интеллектуальных компьютерных систем};
\item{не обеспечивается совместимость и взаимодействие разрабатываемых систем (синтаксическая и семантическая совместимость)};
\item{нет совместимости между существующими частными технологиями проектирования различных компонентов интеллектуальных компьютерных систем (базы знаний, нейросетевые модели, интеллектуальные интерфейсы и т.д.)};
\item{есть инструментальные средства по компонентам, но "склеивать"{} (соединять, интегрировать) это надо вручную};
\item{нет системы инструментальных средств}
\end{scnitemize}
}
}

\bigskip
\scnfragmentcaption

\scnheader{Разработка технологии производства спроектированных интеллектуальных компьютерных систем}
\scntext{текущее состояние}{Был сделан целый ряд попыток разработки \textit{компьютеров} нового поколения, ориентированных на использование в \textit{интеллектуальных компьютерных системах}. Но все они оказались неудачными, так как не были ориентированы на всё многообразие моделей решения задач в \textit{интеллектуальных компьютерных системах}. В этом смысле они не были \textit{\uline{универсальными} компьютерами} для \textit{интеллектуальных компьютерных систем}.}
\scnrelfromset{проблемы текущего состояния}{
\scnfileitem{Разрабатываемые \textit{интеллектуальные компьютерные системы} могут использовать самые различные комбинации \textit{моделей решения интеллектуальных задач} (логических моделей, соответствующих различного вида логикам, нейросетевых моделей различного вида, моделей целеполагания, синтеза планов, моделей управления сложными объектами, моделей понимания и синтеза текстов естественного языка и т.д.). Современные (традиционные, фон-неймановские) \textit{компьютеры} не в состоянии достаточно производительно интерпретировать всё многообразие указанных моделей решения задач. При этом разработка специализированных \textit{компьютеров}, ориентированных на интерпретацию какой-либо одной модели решения задач (нейросетевой модели или какой-либо логической модели) проблему не решает, так как в \textit{интеллектуальной компьютерной системе} необходимо использовать сразу несколько разных моделей решения задач, причём в различных сочетаниях.}
}

\bigskip
\scnfragmentcaption

\scnheader{Специализированная инженерия в области Искусственного интеллекта}
\scnidtf{Деятельность, направленная на разработку \textit{интеллектуальных компьютерных систем} различного назначения с использованием имеющихся для этого моделей, методов и средств}
\scnidtf{Деятельность по проектированию и производству \textit{интеллектуальных компьютерных систем}}
\scnidtf{Деятельность, направленная на формирование рынка \textit{интеллектуальных компьютерных систем}}
\scnrelfrom{в перспективе}{Специализированная инженерия в области \textit{Искусственного интеллекта}, осуществляемая специальной частью Экосистемы OSTIS}
	\scnaddlevel{1}
	\scnrelfrom{продукт}{Экосистема OSTIS}
	\scnrelfrom{субъект действия}{часть Экосистемы OSTIS, осуществляющая специализированную инженерию в области \textit{Искусственного интеллекта}}
	\scnaddlevel{-1}

\scntext{текущее состояние}{}
\scnauthorcomment{добавить из статей}

\scnrelfromset{проблемы текущего состояния}{
\scnfileitem{Отсутствует четкая систематизация многообразия \textit{интеллектуальных компьютерных систем}, соответствующая систематизации автоматизируемых \textit{видов человеческой деятельности}.};
\scnfileitem{Отсутствует \textit{конвергенция} \scnbigspace \textit{интеллектуальных компьютерных систем}, обеспечивающих автоматизацию \textit{областей человеческой деятельности}, принадлежащих одному и тому же \textit{виду человеческой деятельности}.};
\scnfileitem{Отсутствует \textit{семантическая совместимость}(семантическая унификация, взаимопонимание) между \textit{интеллектуальными компьютерными системами}, основной причиной чего является отсутствие согласованной системы общих используемых \textit{понятий}.};
\scnfileitem{Семантическая недружественность \textit{пользовательского интерфейса} и отсутствие встроенной справочной системы, позволяющей запрашивать информацию об элементах интерфейса и возможностях системы, приводят к низкой эффективности эксплуатации всех возможностей \textit{интеллектуальной компьютерной системы}.};
\scnfileitem{Анализ проблем автоматизации всех \textit{видов человеческой деятельности} убеждает в том, что дальнейшая автоматизация \textit{человеческой деятельности} требует не только повышения уровня \textit{интеллекта} соответствующих \textit{интеллектуальных компьютерных систем}, но и реализации их способности
\begin{scnitemize}
\item устанавливать свою \textit{семантическую совместимость} (взаимопонимание) как с другими \textit{компьютерными системами}, так и со своими пользователями\char59
\item поддерживать эту \textit{семантическую совместимость} в процессе собственной эволюции, а также эволюции пользователей и других \textit{компьютерных систем}\char59
\item координировать свою деятельность с пользователями и другими \textit{компьютерными системами} при коллективно решении различных задач\char59
\item участвовать в распределении работ (подзадач) при коллективном решении различных задач.
\end{scnitemize}
Важно подчеркнуть то, что реализация вышеперечисленных способностей создаст возможность для существенной и даже полной автоматизации \textit{системной интеграции} \scnbigspace \textit{компьютерных систем} в комплексы взаимодействующих систем и автоматизации реинжиниринга таких комплексов. Такая автоматизация системной интеграции и её реинжиниринга:
\begin{scnitemize}
\item даст возможность комплексам кибернетических систем \uline{самостоятельно} адаптироваться к решению новых задач\char59
\item существенно повысит эффективность эксплуатации таких комплексов компьютерных систем, так как реинжиниринг системной интеграции компьютерных систем, входящих в такой комплекс, часто востребован (например, при реконструкции предприятия)\char59
\item существенно сокращает число ошибок по сравнению с "ручным"{} (неавтоматизированным) выполнением \textit{системной интеграции} и её \textit{реинжиниринга}, которые, к тому же, требует высокой квалификации.
\end{scnitemize}
Таким образом следующий этап повышения уровня автоматизации \textit{человеческой деятельности} настоятельно требует создания таких \textit{интеллектуальных компьютерных систем}, которые могли бы легко сами (без системного интегратора) объединяться для совместного решения сложных задач. 
}
}

\bigskip
\scnfragmentcaption

\scnheader{Образовательная деятельность в области искусственного интеллекта}
\scntext{текущее состояние}{Целенаправленная подготовка специалистов в области Искусственного интеллекта имеет богатую историю и осуществляется во многих ведущих университетах (Stanford University, MIT, МГУ (Москва), НИУ МЭИ (Москва), РГГУ (Москва), СПбГУ (Санкт-Петербург), ДВФУ (Владивосток), НГТУ (Новосибирск), НТУУ КПИ (Киев), БГУИР (Минск), БГУ (Минск), БрГТУ (Брест) и других).}
\scnrelfromset{проблемы текущего состояния}{
\scnfileitem{Поскольку деятельность в области \textit{Искусственного интеллекта} сочетает в себе и высокую степень наукоемкости и высокую степень сложности инженерных работ, подготовка специалистов в этой области требует одновременного формирования у них как научно-исследовательских навыков, культуры и стиля мышления, так и инженерно-практических навыков, культуры и стиля мышления. С точки зрения методики и психологии обучения сочетание фундаментальной научной и инженерно-практической подготовки специалистов является весьма сложный образовательной педагогической задачей.};
\scnfileitem{Отсутствует \textit{семантическая совместимость} между различными учебными дисциплинами, что приводит к "мозаичности"{} восприятия информации};
\scnfileitem{Отсутствует системный подход к подготовке молодых специалистов в области \textit{Искусственного интеллекта}};
\scnfileitem{Нет персонификации обучения};
\scnfileitem{Нет установки на выявление, раскрытие и развитие таланта творческого проектирования};
\scnfileitem{Отсутствует целенаправленное формирование мотивации к творчеству};
\scnfileitem{Нет формирования навыков работы в реальных коллективах разработчиков};
\scnfileitem{Отсутствует адаптация к реальной практической деятельности};
\scnfileitem{Любая современная технология (в том числе и Технология OSTIS) должна иметь высокие темпы своего развития, поскольку без этого невозможно поддерживать высокий уровень её конкурентоспособности. Но для быстро развиваемой технологии требуется:
\begin{scnitemize}
\item не просто высокая квалификация кадров, использующих и развивающих технологию,
\item но и высокие \uline{темпы} повышения уровня этой квалификации, так как без этого невозможно эффективно использовать и развивать \uline{быстро меняющуюся} технологию.
\end{scnitemize}
\bigskip
Из этого следует, что образовательная деятельность в области \textit{Искусственного интеллекта} и соответствующая ей технология должна быть не просто важной частью деятельности в области \textit{Искусственного интеллекта}, а частью, глубоко интегрированной во все остальные виды деятельности в области \textit{Искусственного интеллекта}. Так, например, каждая \textit{интеллектуальная компьютерная система} должная быть ориентирована не только на обслуживание своих конечных пользователей, не только на организацию целенаправленного взаимодействия со своими разработчиками, которые постоянно совершенствуют эту систему, и не только на обеспечение минимального "порога вхождения"{} для новых конечных пользователей и разработчиков, но и на организацию постоянного и персонифицированного повышения квалификации каждого своего конечного пользователя и разработчика в условиях постоянных изменений, вносимых в указанную \textit{интеллектуальную компьютерную систему}. Для этого эксплуатируемая \textit{интеллектуальная компьютерная система} должна "знать"{}, что в ней изменилось, на что она способна и как эти способности инициировать (содержание и форма, соответствующих пользовательских команд)
}
}

\scnendstruct

\newpage
\scnheader{смысловое представление информации}
\scnrelfromset{принципы, лежащие в основе}{
\scnfileitem{Каждый синтаксически элементарный (атомарный) фрагмент представленной информации является обозначением некоторой сущности, которая может быть реальной или абстрактной, конкретной (фиксированной, константной) или произвольной (переменной), постоянной или временной, четкой (достоверной) или нечеткой (недостоверной с возможным дополнительным уточнением степени правдоподобности).}
	\scnaddlevel{1}
\scntext{следовательно}{В состав смыслового представления информации не могут входить буквы (не являются обозначениями сущностей), слова, словосочетания (не являются элементарными фрагментами), разделители, ограничители (не являются обозначениями сущностей)}
	\scnaddlevel{-1}
;\scnfileitem{В рамках смыслового представления информации отсутствует синонимия (пары синонимичных знаков), омонимия  (омонимичные знаки), семантическая эквивалентность (пары семантически эквивалентных информационных конструкций), т.е. отсутствует любая форма дублирования информации, а также отсутствует неоднозначность соотношения между знаками и их денотатами.}}
	\scnaddlevel{1}
\scntext{следовательно}{Смысловое представление информации не может выглядеть как цепочка (строка, последовательность) синтаксически элементарных фрагментов, поскольку каждая описываемая сущность и взаимно однозначно соответствующий ей ее знак может быть связана не с двумя, а с любым количеством описываемых сущностей. Другими словами, смысловое представление информации является нелинейной (графовой) информационной конструкцией.}
		\scnaddlevel{1}
\scntext{следовательно}{Если внутреннее представление информации в памяти компьютерной системы является смысловым представлением, то обработка информации в такой памяти носит графодинамический характер и сводится не к изменению состояния элементов памяти, а к изменению конфигурации связей между ними.}
	\scnaddlevel{-2}
\scnnote{Ключевая проблема современного этапа развития общей теории интеллектуальных компьютерных систем и технологии их разработки – это проблема обеспечения \textbf{\textit{семантической совместимости}} 
\begin{scnenumerate}
	\item различных видов знаний, входящих в состав баз знаний интеллектуальных компьютерных систем;
	\item различных видов моделей решателей задач;
	\item различных интеллектуальных компьютерных систем в целом;
\end{scnenumerate}
Для решения этой проблемы очевидно необходима унификация (стандартизация) формы представления знаний в памяти интеллектуальных компьютерных систем. Предлагаемым нами подходом для такой унификации и является ориентация на \textbf{\textit{смысловое представление информации}} (знаний) в памяти интеллектуальных компьютерных систем. Основой предполагаемого нами подхода к обеспечению высокого уровня обучаемости т семантической совместимости интеллектуальных компьютерных систем, а также к разработке стандарта интеллектуальных компьютерных систем является унификация \textbf{\textit{смыслового представления информации}} (знаний) в памяти интеллектуальных компьютерных систем и построение глобального \textbf{\textit{смыслового пространства}} знаний.}
\scnaddlevel{1}
\scnnote{Информация в знаковой конструкции в основном содержится не в самих знаках (в их структуре), а в связях между знаками. При этом существенно, чтобы эти связи (синтаксические связи) имели четкую смысловую (семантическую) интерпретацию. 
Если структура знаков содержит информацию об обозначаемой сущности всегда можно заменить на "бесструктурные"{} знаки, которые имеют семантическую окрестность} 

\scnheader{семантическая сеть}
\scnsubset{смысловое представление информации}
\scnexplanation{Семантическая сеть нами рассматривается не как красивая метафора сложноструктурированных знаковых конструкций, а как формальное уточнение понятия смыслового представления информации, как принцип представления информации, лежащей в основе нового поколения компьютерных языков и самих компьютерных систем -- графовых языков и графовых компьютеров.}
\scnsubset{знаковая конструкция}
\scnexplanation{Семантическая сеть -- это знаковая конструкция, обладающая следующими свойствами:
	\begin{scnitemize}
		\item "внутренюю"{} структуру (строение) знаков, входящих в семантическую сеть не требуется учитывать при ее семантическом анализе (понимании)
		\item Смысл семантической сети определяется денотационной семантикой всех входящих в нее знаков и конфигурацией связей инцидентности этих знаков
		\item Из двух инцидентных знаков, входящих в семантическую сеть, один является знаком связи
		\item Отсутствие синонимии, омонимии
	\end{scnitemize}
}
\scnrelfrom{предлагаемый подход}{\scnkeyword{SC-код}}
	\scnaddlevel{1}
	\scnidtf{Предлагаемое в рамках \textit{Технологии OSTIS} уточнение понятия \textit{семантической сети}}
	\scnaddlevel{-1}
\scnsuperset{SC-код}
	\scnaddlevel{1}
	\scnidtf{Semantic Computer Code}
	 \scnrelfromlist{смотрите}{Раздел \ref{intro_sc_code}; Раздел \ref{sd_sc_code}}
	 
\scnheader{многоагентная система}
\scnsubset{кибернетическая система}
\scnexplanation{Кибернетическая система, представляющая собой множество кибернетических систем, способных коммуницировать, т.е. обмениваться информацией друг с другом (причем не обязательно каждый с каждым)}

\scnheader{агент*}
\scnidtf{агент многоагентной системы*}

\scnheader{внешняя среда*}
\scnidtf{внешняя среда кибернетической системы}

\scnheader{память*}
\scnidtf{внутренняя (информационная) среда кибернетической системы}
\scnnote{Не каждая кибернетическая система (в том числе многоагентная система) имеет явно выделенную память, являющуюся хранилищем накапливаемой информации, накапливаемого опыта.}

\scnheader{многоагентная система}
\scnsubdividing{многоагентная система без общей памяти;многоагентная система с общей памятью}
\scnsubdividing{многоагентная система, в которой управление агентами осуществляется только путем обмена сообщениями между ними;многоагентная система, в которой управление агентами осуществляется через общую для них память}
\scnsubdividing{многоагентная система с централизованным управлением агентами;многоагентная система с децентрализованным управлением агентами}
\scnsubdividing{многоагентная система, в которой областью деятельности всех ее агентов является только внешняя среда этой системы;многоагентная система, в которой областью деятельности ее агентов является как внешняя среда, так и память этой системы\\
	\scnaddlevel{1}	
\scnnote{некоторые агенты такой системы могут работать только в памяти}\scnaddlevel{-1}}

	
\scnheader{агентно-ориентированная модель обработки информации в памяти}
\scnidtf{агентно-ориентированная модель решения задач}	
\scnidtf{агентно-ориентированная архитектура решателя задач, представляющая собой многоагентную систему, в которой управление ее агентами осуществляется общей для них памятью и областью деятельности агентов является та же самая общая для них память}
	\scnaddlevel{1}
	\scntext{следовательно}{условием инициирования каждого указанного агента является возникновение в указанной памяти соответствующего вида ситуации или события}
	\scnaddlevel{-1}
\scnreltoset{пересечение}{многоагентная система, в которой управление агентами осуществляется через общую для них память;
многоагентная система с децентрализованным управлением агентами;
многоагентная система, в которой областью деятельности ее агентов является как внешняя среда, так и память этой системы}

\scnheader{агентно-ориентированная модель обработки информации в памяти}
\scnrelfromset{принципы, лежащие в основе}{
\scnfileitem{Распределение целенаправленной деятельности между агентами, выполняющими различные действия в памяти, осуществляется на основе генерируемой в базе знаний иерархической системы, описывающей связь (сведение) инициированных целей (задач) с подцелями (подзадачами).}
;\scnfileitem{Условием инициирования агента является появление в базе знаний формулировки той цели (задачи), которая, во-первых, инициирована, а, во-вторых, либо может быть полностью достигнута (решена) этим агентом, либо может быть этим агентом достигнута (решена) частично.}
;\scnfileitem{В результате частичного достижения (решения) некоторой цели (задачи) агент может сгенерировать новые подцели (подзадачи).}
;\scnfileitem{Таким образом, условием инициирования агента обработки информации (базы знаний) является появление соответствующей этому агенту ситуации или соответствия.}}
\scnrelfrom{предлагаемый подход}{\scnkeyword{абстрактная sc-машина}}
	\scnaddlevel{1}
	\scnidtf{Предлагаемое в рамках технологии OSTIS уточнение понятия агентно-ориентированной модели обработки информации в памяти}
	\scnaddlevel{-1}
\scnsuperset{абстрактная sc-машина}

\scnheader{агентно-ориентированная модель обработки информации в памяти}
\scnnote{Децентрализованное (агентно-ориентированное) управление процессом решения задач в ostis-системах реализуется как на внутреннем уровне (на уровне решателя задач ostis-системы), так и на внешнем уровне (на уровне взаимодействия между ostis-системами)}

\scnheader{стандартизация ostis-систем}
\scnidtf{унификация ostis-систем}
\scnexplanation{Стандартизация ostis-систем включает в себя:
	\begin{scnitemize}
		\item cтандартизацию языка внутреннего представления информации в памяти ostis-систем;
		\item cтандартизацию принципов децентрализованного управления обработкой информации в памяти ostis-систем;
		\item cтандартизацию языка описания ситуаций и событий (в памяти ostis-систем), которые являются условиями инициирования различных информационных процессов в памяти ostis-систем;
		\item стандартизацию базового языка спецификации (описания, программирования) агентов, выполняющих соответствующие информационные процессы в памяти ostis-систем;
		\item стандартизацию базовых языков ввода/вывода информации в/из памяти ostis-систем.
	\end{scnitemize}}

\scnheader{SC-код}
\scnidtf{Стандарт \textit{смыслового представления} информации в памяти \textit{ostis-системы}, а, точнее, \textit{стандарт семантических сетей}}

\scnheader{абстрактная sc-машина}
\scnidtf{Стандарт \textit{агентно-ориентированной модели обработки информации в памяти ostis-системы}}

\scnheader{стандартизация}
\scnidtf{унификация}
\scnrelfromset{проблемы текущего состояния}{
\scnfileitem{Разработка и совершенствование стандартов происходит очень медленно}
;\scnfileitem{В разработке и совершенствовании стандартов принимает участие явно недостаточное число профессионалов -- не учитываются все мнения}
;\scnfileitem{В разработке и совершенствовании стандарта отсутствует четкая методика формирования консенсуса}
;\scnfileitem{При введении новой версии стандарта отсутствует четкая методика перевода на новую версию стандарта всех систем, разработанных по предыдущей версии}}
\scntext{предлагаемый подход}{Стандарт -- это перманентно совершенствуемая база знаний, поддержку эволюции которой осуществляет соответствующий портал}

\scnheader{конвергенция знаний в памяти ostis-системы}
\scnrelfromset{принципы, лежащие в основе}{
\scnfileitem{Вводится \uline{универсальный}  базовый язык внутреннего \uline{смыслового} представления знаний в памяти ostis-систем (\textit{SC-код}), по строению к которому все внутренние языки, ориентированные на представление знаний различного вида (логические языки, языки представления методов решения задач (в частности, программ), язык формулировки задач, онтологические языки и многие другие) являются подъязыками \textit{SC-кода}, синтаксис которых полностью совпадает с синтаксисом \textit{SC-кода}.}
;\scnfileitem{Конвергенция различных знаний сводится к согласованию систем понятий, используемых для представления знаний различного вида. Такое согласование направлено на увеличение числа общих понятий, используемых при представлении различных знаний.}}

\scnheader{конвергенция моделей решения задач в ostis-системе}
\scnrelfromset{принципы, лежащие в основе}{
\scnfileitem{Синтаксис языка представления соответствующего класса методов решения задач в памяти -- синтаксис SC-кода}
;\scnfileitem{Денотационная семантика описывается в виде соответствующей онтологии и представляется в виде текста SC-кода}
;\scnfileitem{Операционная семантика каждой модели решения задач -- коллектив \uline{агентов}. Он может быть иерархическим на основе различных моделей решателей, но есть базовая модель интерпретации \uline{любых} методов -- 
	\begin{scnitemize}
	\item Язык SCP
		\begin{scnitemizeii}
		\item cинтаксис совпадает с синтаксисом SC-кода
		\item денотационная семантика -- процедурный язык программирования в графодинамической памяти
		\item операционная семантика реализуется на уровне программной или аппаратной платформы
		\end{scnitemizeii}
	\item sc-агенты работают в общей среде -- (sc-памяти) параллельно, асинхронно на основе ряда правил, позволяющих им не "мешать"{} друг другу
	\end{scnitemize}}}
	
\scnheader{интеграция знаний в памяти ostis-системы*}
\scnexplanation{Интеграция знаний в памяти ostis-систем сводится к склеиванию (отождествлению) синонимичных знаков}

\scnheader{интеграция моделей решения задач в ostis-системе*}
\scnexplanation{Поскольку модель решения задач, используемая ostis-системой, представлена в памяти ostis-системы как соответствующий вид знаний, интеграция различных моделей решения задач может происходить в ostis-системе точно так же, как и интеграция любых других видов знаний. Кроме того, когда речь идет об интеграции различных моделей решения задач, имеется в виду возможность одновременного использования различных моделей решения задач при обработке одних и тех же знаний и, в частности, при решении одной и той же задачи. Такая возможность в ostis-системе обеспечивается \textit{агентно-ориентированной моделью обработки информации} в памяти ostis-системы. Таким образом, такого рода интеграция различных моделей решения задач для ostis-систем является тривиальной.}

\scnheader{ostis-система}
\scnrelfromset{достоинства}{
\scnfileitem{Высокий уровень способности \textit{ostis-системы} осуществлять семантическую интеграцию знаний в своей памяти (в частности, при погружении новых знаний в текущее состояние базы знаний) \uline{обеспечивается} смысловым характером внутреннего кодирования информации,  хранимой в памяти ostis-системы и, в частности, тем, что во внутреннем коде базы знаний \textit{ostis-системы} запрещены омонимичные знаки и пары синонимичных знаков.}
;\scnfileitem{Высокий уровень способности интегрировать различные виды знаний в \textit{ostis-системах} \uline{обеспечивается} тем, что каждый язык, ориентированный на представление знаний соответствующего вида является \uline{подъязыком} одного и того же базового языка \textit{SC-кода}.}\\
\scnaddlevel{1}
\scnnote{Кроме того можно говорить об иерархии sc-языков}
\scnaddlevel{-1}
;\scnfileitem{Высокий уровень способности интегрировать различные модели решения задач в \textit{ostis-системах} \uline{обеспечивается}:
	\begin{scnitemize}
	\item тем, что все эти модели ориентированы на обработку информации, представленной в \textit{SC-коде}
	\item один и тот же фрагмент базы знаний ostis-системы (т.е. одна и та же конструкция SC-кода) может одновременно обрабатываться несколькими \uline{разными} моделями решения задач
	\item все модели решения задач в ostis-системах интегрируются с помощью одной и той же базовой модели решения задач -- \textit{scp-модели решения задач} \scnauthorcomment{(пояснить)}
	\end{scnitemize}}
;\scnfileitem{Высокий уровень обучаемости \textit{ostis-систем} \uline{обеспечивается}:
	\begin{scnitemize}
	\item высоким уровнем семантической гибкости информации, хранимой в памяти ostis-системы, поскольку каждое удаление или добавление синтаксически элементарного фрагмента хранимой информации, а также удаление или добавление каждой связи инцидентности между такими элементами имеет четкую семантическую интерпретацию;
	\item высоким уровнем стратифицированности хранимой информации, что обеспечивается онтологически ориентированной структуризацией базы знаний ostis-системы; 
	\item высоким уровнем рефлексии ostis-системы, что обеспечивается мощными метаязыковыми возможностями языка внутреннего представления информации (знаний) в памяти \textit{ostis-систем}.
	\end{scnitemize}}
;\scnfileitem{Каждая \textit{ostis-система} имеет высокий \textit{уровень обучаемости} (способности к быстрому расширению своих \textit{знаний} и \textit{навыков}) и высокий \textit{уровень социализации} (способности к эффективному участию в деятельности различных коллективов – коллективов, состоящих из \textit{ostis-систем}, и сообществ, состоящих из \textit{ostis-систем} и людей}
	\scnaddlevel{1}
\scnrelfromset{детализация достоинства}{
\scnfileitem{Существуют четкие формальные критерии, определяющие \textit{уровень семантической совместимости} (уровень семантической конвергенции) различных знаний, навыков, целых \textit{ostis-систем} (точнее, баз знаний этих систем). Очевидно, что \textit{уровень семантической совместимости} прежде всего определяется количеством "точек соприкосновения"{} в сравниваемых \textit{знаниях}, \textit{навыках} и \textit{базах знаний} – это \textit{знаки}, присутствующие \uline{в разных} сравниваемых объектах, но имеющие одинаковые денотаты (т.е. обозначающие одинаковые сущности). При этом среди таких знаков, обозначающих одинаковые сущности и присутствующих в разных сравниваемых объектах особенно важны знаки, обозначающие \textit{понятия}.
Количество таких общих понятий в сравниваемых знаниях, навыках, базах знаний определяет уровень семантической совместимости (уровень согласованности) систем используемых понятий в сравниваемых указанных объектах. Увеличение количества знаков, обозначающих одинаковые сущности и присутствующих в разных сравниваемых объектах, может привести к тому, что в разных указанных сравниваемых объектах будут присутствовать не только семантически эквивалентные знаки, но и семантически эквивалентные целые фрагменты (целые информационные конструкции).
Существенно при этом подчеркнуть, что семантически эквивалентные знаковые конструкции, представленные на внутреннем языке ostis-систем (в SC-коде), в памяти разных ostis-систем всегда являются синтаксически изоморфными графовыми конструкциями, в которых соответствие изоморфизма связывает знаки, хранимые в памяти разных ostis-систем, но обозначающие одинаковые сущности (точнее, одну и ту же сущность). Заметим также, что в рамках памяти каждой индивидуальной \textit{ostis-системы} синонимия знаков и, соответственно, семантическая эквивалентность знаковых конструкций запрещены.}
;\scnfileitem{Благодаря постоянно развиваемым семантическим стандартам \textit{Технологии OSTIS} , которые представлены системой формальных онтологий для самых различных предметных областей, разрабатываемые \textit{ostis-системы} \uline{изначально} имеют достаточно высокий \textit{уровень семантической совместимости} со всеми остальными \textit{ostis-системами}. Более того, в \textit{Технологии OSTIS} выделяется целое ядро всех ostis-систем, содержащее фундаментальные базовые знания и базовые навыки, одинаковые для всех ostis-систем и позволяющее каждой копии этого ядра развиваться (общаться, специализироваться) в любом направлении.}
;\scnfileitem{Каждая ostis-система, взаимодействуя с людьми (пользователями) или с другими ostis-системами, обладает способностью повышать уровень семантической совместимости (взаимопонимания) с ними, а также поддерживать (сохранять) высокий уровень такой совместимости в условиях (1) собственной эволюции, (2) эволюции других ostis-систем и пользователей, (3) эволюции семантических стандартов Технологии OSTIS. Указанное взаимодействие, в основном, направлено на согласование изменений в системе используемых понятий, т.е. корректировки соответствующих фрагментов онтологий.}
;\scnfileitem{Благодаря высокому уровню семантической совместимости ostis-систем и смысловому представлению знаний в памяти ostis-систем существенно снижается сложность и повышается качество семантического анализа и понимания информации, поступающей (сообщаемой, передаваемой) ostis-системе от других ostis-систем или пользователей.}
;\scnfileitem{Каждая ostis-система способна:
	\begin{scnitemize}
	\item самостоятельно или по приглашению войти в состав ostis-коллектива (коллектива ostis-систем) или в состав ostis-сообщества, состоящего из ostis-систем и людей. Такие коллективы и сообщества создаются на временной (разовой) или постоянной основе для коллективного решения сложных задач;
	\item участвовать в распределении (в т.ч. в согласовании распределения) задач -- как "разовых"{} задач, так и долгосрочных задач (обязанностей);
	\item мониторить состояние всего процесса коллективной деятельности и координировать свою деятельность с деятельностью других членов коллектива при возможных непредсказуемых изменениях условий (состояния) соответствующей среды.
	\end{scnitemize}}}
	\scnaddlevel{-1}
;\scnfileitem{Высокий уровень интеллекта ostis-систем и, соответственно, высокий уровень их самостоятельности и целенаправленности позволяет ostis-системам быть полноправными членами самых различных сообществ, в рамках которых ostis-системы получают права самостоятельно инициировать (на основе детального анализа текущего положения дел и, в том числе, текущего состояния плана действий сообщества) широкий спектр действий (задач), выполняемых другими членами сообщества, и тем самым участвовать в согласовании и координации деятельности членов сообщества.}
;\scnfileitem{Способность ostis­системы согласовывать свою деятельность с другими ostis-системами, а также корректировать деятельность всего коллектива ostis-систем, адаптируясь к различного вида изменениям среды (условий), в которой эта деятельность осуществляется, позволяет существенно автоматизировать деятельность системного интегратора как на этапе сборки коллектива ostis-систем, так и на этапе его обновления (реинжиниринга).}}
\scnnote{Достоинства ostis-систем обеспечиваются:
	\begin{scnitemize}
	\item достоинствами SC-кода -- языка внутреннего кодирования информации, хранимой в памяти ostis-систем;
	\item достоинствами организации sc-памяти -- памяти ostis-систем;
	\item достоинствами sc-моделей баз знаний ostis\textit{–}систем – средствами структуризации таких баз знаний;
	\item достоинствами sc-моделей решения задач -- агентно-ориентированных моделей решения задач, используемых в ostis-системах.
	\end{scnitemize}}
	
\scnendstruct \scninlinesourcecommentpar{Завершили рассмотрение понятия ostis-системы}

\scnsegmentheader{Понятие ostis-технологии}
\scnstartsubstruct

\scnheader{ostis-технология}
\scnreltoset{объединение}{
ostis-технология проектирования\\
\scnaddlevel{1}
	\scnrelfromset{разбиение}{
		ostis-технология проектирования ostis-систем соответствующего класса\\
		\scnaddlevel{1}
			\scnhaselement{Базовая ostis-технология проектирования ostis-систем}
		\scnaddlevel{-1}
		;ostis-технология проектирования соответствующего класса компонентов ostis-систем\\
		\scnaddlevel{1}
			\scnhaselement{Базовая ostis-технология проектирования баз знаний ostis-систем}
			\scnhaselement{Базовая ostis-технология проектирования решателей задач ostis-систем}
			\scnhaselement{Базовая ostis-технология проектирования интерфейсов ostis-систем}
		\scnaddlevel{-1}
		;ostis-технология проектирования объектов заданного класса, не являющихся ostis-системами\\
	}
\scnaddlevel{-1}
;ostis-технология производства\\
\scnaddlevel{1}
	\scnsuperset{технология производства спроектированных ostis-систем}
	\scnsuperset{ostis-технология управления производством спроектированных продуктов заданного класса, не являющихся ostis-системами}
\scnaddlevel{-1}
;технология эксплуатации ostis-систем\\
\scnaddlevel{1}
	\scnhaselement{Базовая технология эксплуатации ostis-систем}
	\scnsuperset{технология эксплуатации ostis-систем соответствующего класса}
	\scnaddlevel{1}
		\scnsuperset{ostis-технология управления производством спроектированных продуктов заданного класса, не являющихся ostis-системами}
		\scnaddlevel{1}
			\scnidtf{технология эксплуатации ostis-систем управления производством спроектированных продуктов заданного класса, не являющихся ostis-системами}
			\scnaddlevel{-1}
	\scnaddlevel{-1}
\scnaddlevel{-1}
;технология реинжиниринга ostis-систем\\
\scnaddlevel{1}
	\scnhaselement{Базовая технология реинжиниринга ostis-систем}
	\scnsuperset{технология реинжиниринга ostis-систем соответствующего класса}
\scnaddlevel{-1}
}

\scnheader{ostis-технология}
\scnidtf{компонент Технологии OSTIS}
\scnhaselement{Ядро Технологии OSTIS}
\scnaddlevel{1}
	\scnidtf{Базовая ostis-технология}
\scnaddlevel{-1}
\scnsuperset{частная ostis-технология}
\scnaddlevel{1}
	\scnsuperset{ostis-технология проектирования соответствующего класса компонентов ostis-систем}
	\scnaddlevel{1}
		\scnhaselement{Технология проектирования баз знаний ostis-систем}
		\scnhaselement{Технология проектирования решателей задач ostis-систем}
		\scnhaselement{Технология проектирования невербальных интерфейсов ostis-систем с внешней средой}
		\scnhaselement{Технология проектирования интерфейсов ostis-систем с другими техническими системами}
		\scnhaselement{Технология проектирования пользовательских интерфейсов ostis-систем}
	\scnaddlevel{-1}
\scnaddlevel{-1}
\scnsuperset{специализированная ostis-технология проектирования \scnkeyword{ostis-систем соответствующего класса}}
\scnaddlevel{1}
	\scnhaselement{Технология проектирования ostis-систем управления предприятиями рецептурного производства}
	\scnhaselement{Технология проектирования ostis-систем управления предприятиями производства молочной продукции}
	\scnhaselement{Технология проектирования интеллектуальных обучающих ostis-систем}
	\scnhaselement{Технология проектирования интеллектуальных обучающих ostis-систем для школьников}
	\scnhaselement{Технология проектирования интеллектуальных обучающих ostis-систем для подготовки специалистов в области Математики}
	\scnhaselement{Технология проектирования интеллектуальных обучающих ostis-систем для подготовки специалистов в области Искуственного интеллекта}
\scnaddlevel{-1}

\scnheader{ostis-технология проектирования}
\scnnote{Каждой ostis-технологии проектирования соответсвует своя ostis-система автоматизации проектирования соответствующего класса объектов}
\scnrelfrom{соответствующее семейство средств автоматизации}{ostis-система автоматизации проектирования}
\scnrelfrom{соответствующее семейство классов проектируемых объектов}{
	{\normalfont ( } ostis-система автоматизации проектирования ostis-систем	$\cup$ ostis-система автоматизации проектирования объектов, не являющихся ostis-системами {\normalfont ) }
}
\scnsuperset{ostis-технология проектирования ostis-систем соответствующего класса}

\scnheader{ostis-технология проектирования ostis-систем соответствующего класса}
\scnidtf{технология проектирования \textit{ostis-систем} соответствующего ( заданного) класса, который, в свою очередь, соответствует определенному \textit{виду человеческой деятельности}, подвиды которого автоматизируются с помощью указанных выше проектируемых \textit{ostis-систем}}

\scnheader{ostis-технология}
\scnrelfromlist{отношение, заданное на данном множестве}{частная технология*; специализированная технология*; комплекс специализированных технологий*}
\scnidtf{Базовая частная или специализированная технология, входящая в состав комплексной \textit{Технологии OSTIS}, которая:
\begin{scnitemize}
	\item направлена на автоматизацию конкретного вида человеческой деятельности;
	\item ориентирована на использование ostis-систем (как индивидуальных, так и коллективных) в качестве самостоятельных субъектов или активных интеллектуальных инструментов, либо на использование человеко-машинных ostis-сообществ при решении:
	\begin{scnitemizeii}
		\item как задач, выполняемых в памяти ostis-систем (в т.ч. в памяти коллективов ostis-систем);
		\item так и задач, выполняемых во внешней среде ostis-систем, в процессе решения которых субъектами соответствующих действий либо ostis-системы (индивидуальные или коллективные), либо конкретные персоны, либо ostis-сообщества.
	\end{scnitemizeii}
\end{scnitemize}
}
\scnidtf{Множество всевозможных технологий, соответствующих стандартам технологии OSTIS и направленных на автоматизацию различных конкретных видов человеческой деятельности}
\scnrelboth{следует отличать}{Технология OSTIS}
\scnaddlevel{1}
	\scnnote{\textit{Технология OSTIS} в отличие от понятия \textit{ostis-технологии} представляет собой не множество технологий, а комплекс взаимосвязанных между собой самых различных технологий, превращающий указанное множество технологий в единую объединенную технологию, в сумму взаимосвязанных глубоко интегрированных технологий. В этом смысле Технология OSTIS является максимальной ostis-технологией, в состав которой входят все ostis-технологии.}
\scnaddlevel{-1}
\scnsuperset{пример*:}
\scnaddlevel{1}
	\scnlistitem{ostis-технология проектирования и перепроектирования} 	\scnlistitem{ostis-технология производства}
	\scnlistitem{ostis-технология публикации и согласования результатов научно-технической деятельности (в широком смысле)}
	\scnlistitem{ostis-технология образования}
\scnaddlevel{-1}
\scnhaselements{пример':}
\scnaddlevel{1}
	\scnlistitem{Технолония OSTIS}
	\scnlistitem{OSTIS-технология проектирования, реализации и реинжиниринга ostis-систем}
	\scnlistitem{OSTIS-технология разработки стандартов Технологии OSTIS}
\scnaddlevel{-1}

\scnheader{ostis-технология коллективной разработки информационных ресурсов (различного вида)}
\scnsuperset{ostis-технология коллективного проектирования}
\scnsuperset{ostis-технология коллективной разработки планов}
\scnsuperset{ostis-технология публикации и согласования результатов научно-технической деятельности}
\scnsubset{ostis-технология}

\scnheader{ostis-технология эксплуатации ostis-систем}
\scnidtf{Общие методы и средства (языковые и интерфейсные) организации взаимодействия ostis-систем со своими конечными пользователями}
\scnsubset{ostis-технология}
\scnnote{Поскольку в рамках Экосистемы OSTIS каждому человеку придется взаимодействовать с больщим числом ostis-систем разного назначения, принципы организации взаимодействия всех ostis-систем со своими пользователями должны быть абсолютно одинаковыми. Удобство (usability) пользовательских интерфейсов должно быть направлено не только на синтаксическую красоту, но и на простую семантическую интерпретацию (понятность).}

\scnheader{ostis-технология проектирования ostis-систем}
\scnidtf{Технология построения (разработки) логико-семантических моделей (sc-моделей) ostis-систем}
\scniselement{ostis-технология}
\scnnote{Продуктом каждого завершенного (целостного) коллективного проекта, реализованного в рамках этой технологии, является полная \textit{логико-семантическая модель ostis-системы}.}
\scnrelfrom{класс продуктов}{логико-семантическая модель ostis-системы}
\scnrelfrom{средство}{Метасистема IMS OSTIS}
\scnrelfrom{класс субъектов}{коллектив разработчиков ostis-системы}
\scnrelfrom{класс исходных данных}{исходная спецификация ostis-системы}

\scnheader{ostis-технология реализации ostis-систем}
\scnidtf{Технология сборки и установки ostis-систем}
\scniselement{ostis-технология}
\scnrelfrom{исходная информация}{логико-семантическая модель ostis-системы}
\scnrelfrom{комплектация}{универсальный интерпретатор логико-семантических моделей ostis-систем}
\scnaddlevel{1}
	\scnnote{Это, своего рода, "мотор"{}, "движок"{} ostis-систем}
\scnaddlevel{-1}
\scnrelfrom{методы}{Методика реализации ostis-систем}
\scnrelfrom{активный инструмент}{Метасистема IMS OSTIS}
\scnrelfrom{продукты}{ostis-система}

\scnheader{ostis-технология обновления ostis-систем}
\scnidtf{Технология реинжирования (перепроектирования) ostis-систем в ходе их эксплуатации}
\scniselement{ostis-технология}
\scnheader{следует отличать*}
\scnhaselementset{Технология обновления ostis-систем; Технология проектирования ostis-систем}
\scnaddlevel{1}
	\scnnote{Эти технологии сходны. Их методы и средства совпадают. Не совпадают только исходные данные и результаты, которыми в Технологии обновления ostis-систем являются предшествующие и последующие состояния ostis-систем. В Технологии проектирования ostis-систем исходными данными являются исходные спецификации (замыслы) проектируемых ostis-систем, и результатами -- полные логико-семантические модели этих систем}
\scnaddlevel{-1}

\scnheader{Технология OSTIS}
\scnidtf{Совокупность (интеграция, объединение) всех ostis-технологий}
\scnrelto{интеграция}{ostis-технология}
\scnidtf{Комплекс (множество) семантически совместимых \textit{технологий}, в состав которого входит \textit{Ядро Технологии OSTIS} и иерархическая система \textit{ostis-технологий}, каждая из которых ориентирована на \textit{проектирование}, \textit{производство}, \textit{эксплуатацию} или \textit{реинжиринг} соответствующего \textit{класса ostis-систем}, обеспечивающих автоматизацию соответствующего \textit{вида человеческой деятельности}. При этом каждая такая проектируемая \textit{ostis-система} автоматизирует либо область, либо \textit{вид человеческой деятельности}, которая (который) является соответственно либо экземпляром (элементом), либо подвидом (подклассом) указанного выше \textit{вида человеческой деятельности}, соответствующего используемой \textit{специализированной ostis-технологии}.}

%стр 197 уточнить ключевой элемент (неролевое или ролевое?)
\scnheader{Ядро Технологии OSTIS}
\scnidtf{Универсальная базовая \textit{ostis-технология}}
\scnidtf{Универсальный компонент Технологии OSTIS}

\scniselementrole{ключевой элемент}{\itshape ostis-технология}
\scnrelto{ядро}{Технология OSTIS}
\scnhaselement{технология}
\scnrelfrom{вид деятельности, выполняемой с помощью технологии}{проектирование, производство, эксплуатация и реинжиринг ostis-системы}
\scnaddlevel{1}
	\scnreltoset{объединение}{
		проектирование ostis-системы\\
		\scnaddlevel{1}
			\scnidtf{построение логико-семантической модели ostis-системы}
		\scnaddlevel{-1}
		;производство ostis-системы\\
		\scnaddlevel{1}
			\scnidtf{сборка логико-семантической модели ostis-системы и загрузка этой модели в память универсального интерпретатора таких моделей}
		\scnaddlevel{-1}
		;эксплуатация ostis-системы\\
		\scnaddlevel{1}
			\scnidtf{базовый (предметно-независимый) уровень организации деятельности конечного пользователя ostis-системы с помощью соответствующих методов	и средств}
		\scnaddlevel{-1}
		;реинжиринг ostis-системы\\
		\scnaddlevel{1}
			\scnidtf{совершенствование ostis-системы в процессе её эксплуатации}
		\scnaddlevel{-1}
	}
	\scnrelfrom{создаваемые продукты}{
		ostis-система\\
		\scnidtf{\textit{интеллектуальная компьютерная система}, построенная в соответствии со стандартом \textit{Технологии OSTIS}, предъявляемым к продуктам, создаваемым с помощью этой технологии}
		\scnaddlevel{1}
			\scnnote{Указанный стандарт продуктов, создаваемых с помощью технологии OSTIS есть не что иное, как \textit{общая формальная семантическая теория интеллектуальных компьютерных систем}}
		\scnaddlevel{-1}
	}
\scnaddlevel{-1}
\scnrelfromlist{частная технология}{
	Базовая Технология Проектирования ostis-систем\\
	\scnaddlevel{1}
		\scnrelfromlist{частная технология}{
			Технология проектирования баз знаний ostis-систем\\
			;Технология проектирования решателей задач ostis-систем\\
			;Технология проектирования интерфейсов ostis-систем\\
			\scnrelfromlist{частная технология}{
				Технология проектирования невербальных интерфейсов ostis-систем с внешней средой\\
				;Технология проектирования интерфейсов ostis-систем с другими техническими системами\\
				;Технология проектирования пользовательских интерфейсов ostis-систем
			}
		}
		\scnrelfrom{реализация}{Метасистема IMS.ostis}
		\scnaddlevel{1}
			\scnidtf{Intelligent MetaSystem for ostis-systems design}
			\scnidtf{OSTIS-система автоматизации проектирования ostis-систем}
		\scnaddlevel{-1}
	\scnaddlevel{-1}
	;Технология производства ostis-систем\\
	\scnaddlevel{1}
		\scnexplanation{Основным компонентом, точнее, инструментальным средством \textit{технологии производства ostis-систем} является \textit{универсальный интерпретатор логико-семантических моделей ostis-систем}. Указанные \textit{логико-семантические модели ostis-систем} являются результатом \textit{проектирования ostis-систем} и представляют собой начальные (исходные) состояния \textit{баз знаний} разрабатываемых \textit{ostis-систем}. В отличие от \textit{инструмента производства ostis-систем}, методика их производства весьма проста и сводится к сборке разработанных логико-семантических моделей (начального состояния \textit{баз знаний}) разрабатываемых \textit{ostis-систем} и загрузке этих моделей в память \textit{универсального интерпретатора логико-семантических моделей ostis-систем}.}
		\scnrelfrom{реализация}{универсальный интерпретатор логико-семантических моделей ostis-систем}
		\scnaddlevel{1}
			\scnexplanation{Такой интерпретатор логико-семантических моделей ostis-систем может быть реализован либо программно на \textit{современных компьютерах}, либо аппаратно в виде компьютеров нового поколения, ориентированных на реализацию интеллектуальных компьютерных систем.}
			\scnexplanation{С формальной точки зрения универсальный интерпретатор логико-семантических моделей ostis-систем является "пустой"{} остис системой, способной только на приём формализованной информации и её запись в свою память.}
		\scnaddlevel{-1}
	\scnaddlevel{-1}
	;Базовая технология эксплуатации ostis-систем\\
	\scnaddlevel{1}
		\scnidtf{Общая технология эксплуатации ostis-систем, включающая в себя общие методы и средства, используемые в процессе эксплуатации любых ostis-систем}
		\scnrelfrom{реализация}{встраиваемая ostis-система поддержки эксплуатации ostis-систем}
		\scnaddlevel{1}
			\scnexplanation{Данная ostis-система входит (интегрирована) в состав каждой ostis-системы.}
		\scnaddlevel{-1}
	\scnaddlevel{-1}
	;Базовая технология реинжиниринга ostis-систем\\
	\scnaddlevel{1}
		\scnrelfrom{реализация}{встраиваемая ostis-система поддержки реинжиниринга ostis-систем}
		\scnaddlevel{1}
			\scnexplanation{Данная ostis-система входит (интегрирована) в состав каждой ostis-системы и обеспечивает внесение изменений "руками"{} инженеров, сопровождающих эксплуатацию ostis-системы, или авторов базы знаний этой ostis-системы в текущее состояние базы знаний ostis-системы в ходе её экспуатации}
		\scnaddlevel{-1}
	\scnaddlevel{-1}
}
\scnrelfromlist{специализированная технология}{
	Общая технология проектирования ostis-систем автоматизации проектирования\\
	\scnaddlevel{1}
		\scnrelfromlist{специализированная технология}{
		Технология проектирования ostis-систем автоматизации проектирования строительных объектов\\
		;Технология проектирования ostis-систем автоматизации проектирования автомобилей\\
		;Технология проектирования ostis-систем автоматизации проектирования интегральных микросхем
		}
	\scnaddlevel{-1}
	;Технология проектирования ostis-систем управления производством\\
	\scnaddlevel{1}
		\scnrelfromlist{специализированная технология}{
		Технология проектирования ostis-систем управления строительством различных объектов\\
		;Технология проектирования ostis-систем управления производством автомобилей\\
		;Технология проектирования ostis-систем управления производством микросхем\\
		;Технология проектирования ostis-систем управления предприятиями рецептурного производства\\
		\scnaddlevel{1}
			\scnrelfrom{специализированная технология}{				Технология проектирования ostis-систем управления предприятиями производства молочной продукции}
		\scnaddlevel{-1}
		}
	\scnaddlevel{-1}
	;Технология проектирования интеллектуальных обучающих ostis-систем\\
	\scnaddlevel{1}
		\scnrelfromset{комплекс специализированных технологий}{
		Технология проектирования интеллектуальных обучающих ostis-систем для школьников\\
		;Технология проектирования интеллектуальных обучающих ostis-систем для студентов по общеобразовательным дисциплинам\\
		;Технология проектирования интеллектуальных обучающих ostis-систем для студентов по профильным дисциплинам\\
		;Технология проектирования интеллектуальных обучающих ostis-систем для магистрантов
		}
		\scnrelfromset{комплекс специализированных технологий}{
		Технология проектирования интеллектуальных обучающих ostis-систем по Математике\\
		;Технология проектирования интеллектуальных обучающих ostis-систем по Искусственному интеллекту
		}
		%стр 205 уточнить многоточие
		\scnnote{При проектировании конкретной обучающей ostis-системы необходимо использовать сразу две ostis-технологии...}
	\scnaddlevel{-1}
}

\scnheader{специализированная ostis-технология}
\scnnote{Приведённый нами перечень \textit{специализированных ostis-технологий} охватывает только некоторые области (фрагменты) \textit{человеческой деятельности}, подлежащие автоматизации с помощью \textit{ostis-технологий} в рамках \textit{Экосистемы OSTIS}.}

\scnheader{Ядро Технологии OSTIS}
\scnnote{Форма реализации \textit{Ядра Технологии OSTIS} (в виде ostis-системы \textit{IMS.ostis}) позволяет:
\begin{scnitemize}
	\item использовать достоинства \textit{Технологии OSTIS} для повышения уровня автоматизации развития самой \textit{Технологии OSTIS} и для существенного повышения темпов такого развития;
	\item приобрести очень важный опыт применения \textit{Технологии OSTIS};
	\item создать центрально ядро \textit{Экосистемы OSTIS}, обеспечивающее поддержку семантической совместимости всех \textit{ostis-систем} и \textit{ostis-сообществ}, входящих в состав \textit{Экосистемы OSTIS}.
\end{scnitemize}
}

\scnendstruct
\scninlinesourcecommentpar{Завершили рассмотрение \textit{понятия ostis-технологии}}

\newpage
\begin{SCn}

\scnfragmentcaption

\scnheader{Специализированная инженерия в области Искусственного интеллекта}
\scnrelfrom{предлагаемый подход}{Специализированная инженерия, осуществляемая на основе Технологии OSTIS}
\scnaddlevel{1}
\scnrelfromset{декомпозиция}{Разработка ostis-систем автоматизации проектирования различных классов ostis-систем\\
    \scnaddlevel{1}
    \scnidtf{Разработка специализированных ostis-технологий}
    \scnrelfromlist{часть}{Разработка ostis-систем автоматизации проектирования ostis-систем автоматизации проектирования\\
        \scnaddlevel{1}
        \scnidtf{Разработка ostis-технологий проектирования}
        \scnaddlevel{-1}
    ;Разработка ostis-систем автоматизации проектирования ostis-систем автоматизации производства\\
        \scnaddlevel{1}
        \scnidtf{Разработка ostis-технологий управления производством}
        \scnaddlevel{-1}
    ;Разработка ostis-систем автоматизации проектирования ostis-систем управления транспортными системами\\
        \scnaddlevel{1}
        \scnidtf{Разработка ostis-технологий управления транспортными системами}
        \scnaddlevel{-1}
    ;Разработка ostis-систем автоматизации проектирования диагностических ostis-систем\\
        \scnaddlevel{1}
        \scnidtf{Разработка ostis-технологий диагностики (технической, медицинской)}
        \scnaddlevel{-1}
    ;Разработка ostis-систем автоматизации проектирования обучающих ostis-систем\\
        \scnaddlevel{1}
        \scnidtf{Разработка ostis-технологий обучения людей}
        \scnaddlevel{-1}
    ;Разработка ostis-систем автоматизации проектирования ostis-систем управления "умными"{} домами\\
        \scnaddlevel{1}
        \scnidtf{Разработка ostis-технологий управления "умными"{} домами}
        \scnaddlevel{-1}
    ;Разработка ostis-систем автоматизации проектирования ostis-систем управления "умными"{} больницами\\
        \scnaddlevel{1}
        \scnidtf{Разработка ostis-технологий управления "умными"{} больницами}
        \scnaddlevel{-1}
    ;Разработка ostis-систем автоматизации проектирования ostis-систем управления "умными"{} поликлиниками\\
        \scnaddlevel{1}
        \scnidtf{Разработка ostis-технологий управления "умными"{} поликлиниками}
        \scnaddlevel{-1}
    ;Разработка ostis-систем автоматизации проектирования ostis-систем управления "умными"{} городскими районами\\
        \scnaddlevel{1}
        \scnidtf{Разработка ostis-технологий управления "умными"{} городскими районами}
        \scnaddlevel{-1}
    ;Разработка ostis-систем автоматизации проектирования ostis-систем управления "умными"{} городами\\
        \scnaddlevel{1}
        \scnidtf{Разработка ostis-технологий управления "умными"{} городами}
        \scnaddlevel{-1}
    ;и т. д.}
    \scnaddlevel{-1}
;Разработка (на основе соответствующих ostis-технологий проектирования) ostis-систем автоматизации проектирования различных классов объектов, не являющихся ostis-системами\\
    \scnaddlevel{1}
    \scnrelfromlist{часть}{Разработка семейства ostis-систем автоматизации проектирования различных видов интегральных микросхем;Разработка семейства ostis-систем автоматизации проектирования различных видов автомобилей;Разработка семейства ostis-систем автоматизации проектирования различных видов строительных объектов;и т.д.}
    \scnaddlevel{-1}
;Разработка ostis-систем автоматизации производства\\
    \scnaddlevel{1}
    \scnidtf{Разработка интеллектуальных систем управления производственными предприятиями}
    \scnaddlevel{-1}
;Разработка ostis-систем управления транспортными средствами;Разработка диагностических ostis-систем;Разработка обучающих ostis-систем;Разработка ostis-систем управления "умными"{} домами;Разработка ostis-систем управления "умными"{} больницами;Разработка ostis-систем управления "умными"{} поликлиниками;Разработка ostis-систем управления "умными"{} городскими районами;Разработка ostis-систем управления "умными"{} городами;и т.д.}
\scnaddlevel{-1}

\bigskip
\scnfragmentcaption

\scnheader{Образовательная деятельность в области Искусственного интеллекта}
\scnrelfrom{предлагаемый подход}{Образовательная деятельность в области Искусственного интеллекта, осуществляемая на основе Технологии OSTIS}
    \scnaddlevel{1}
    \scniselement{образовательная деятельность}
    \scniselement{человеческая деятельность, осуществляемая на основе Технологии OSTIS}
        \scnaddlevel{1}
        \scnidtf{человеческая деятельность, комплексная автоматизация которой осуществляется либо индивидуальной \textit{ostis-системой}, либо \textit{коллективом (сетью) ostis-систем} (сетью ostis-систем)}
        \scnaddlevel{-1}
    \scnrelfrom{субъект}{OSTIS-сообщество Образовательной деятельности в области Искусственного интеллекта, осуществляемой на основе Технологии OSTIS}
		\scnaddlevel{1}    	
    	\scnidtf{глобальное (максимальное) OSTIS-сообщество, осуществляющее Образовательную деятельность в области Искусственного интеллекта и обеспечивающее активное и взаимовыгодное сотрудничество между всеми заинтересованными в этом субъектами и, в первую очередь, с соответствующими кафедрами различных вузов}
    	
    	\scnrelto{часть}{Экосистема OSTIS}
        	\scnaddlevel{1}
	        \scnidtf{глобальная сеть ostis-систем вместе с их пользователями}
    	    \scnidtf{глобальное ostis-сообщество}
	        \scnaddlevel{-1}
    	\scniselement{ostis-сообщество}
        	\scnaddlevel{1}
	        \scnidtf{локальная сеть ostis-систем вместе с их пользователями}
	        \scnaddlevel{-1}
	    \scnexplanation{Данное \textit{ostis-сообщество} включает в себя:
\begin{scnitemize}
    \item все кафедры, которые готовят молодых специалистов в области Искусственного интеллекта и которые могут входить в состав самых различных вузов;
    \item все те организации, которые разрабатывают или эксплуатируют интеллектуальные компьютерные системы и которые готовы сотрудничать с вузами для повышения квалификации поступающих к ним молодых специалистов в области Искусственного интеллекта;
    \item студентов, магистрантов и аспирантов, обучающихся в области Искусственного интеллекта в разных вузах;
    \item их преподавателей;
    \item семейство интеллектуальных обучающих ostis-систем по различным дисциплинам (направлениям) Искусственного интеллекта, которые семантически совместимы и тесно связаны с \textit{OSTIS-порталом научных знаний по Искусственному интеллекту} и с \textit{Метасистемой IMS.ostis};
    \item \textit{OSTIS-портал научных знаний по Искусственному интеллекту}, осуществляющий поддержку развития Общей теории интеллектуальных систем как естественного, так и искусственного происхождения;
    \item \textit{Метасистема IMS.ostis}, осуществляющая поддержку развития Общей теории интеллектуальных компьютерных систем (искусственных интеллектуальных систем) и поддержку развития Базовой универсальной комплексной технологии проектирования интеллектуальных компьютерных систем;
    \item семейство персональных ostis-ассистентов студентов, магистрантов и аспирантов, обучающихся в области Искусственного интеллекта;
    \item семейство персональных ostis-ассистентов преподавателей, осуществляющих подготовку молодых специалистов в области Искусственного интеллекта;
    \item семейство кафедральных корпоративных ostis-систем, осуществляющих управление учебным процессом на уровне кафедр, обеспечивающих подготовку молодых специалистов в области Искусственного интеллекта. В рамках таких корпоративных ostis-систем осуществляется:
    \begin{scnitemizeii}
        \item составление кафедрального расписания занятий на следующий семестр и его согласование с расписанием других кафедр этого же вуза;
        \item распределение учебной нагрузки на очередной семестр и учебный год;
        \item мониторинг проведения различного вида занятий (лекций, консультаций, семинаров, практических занятий, зачетов/экзаменов);
        \item мониторинг самостоятельной деятельности обучаемых (курсовых и дипломных проектов, рефератов, диссертаций, тестов и др.);
        \item фиксация текущего соответствия между учебными дисциплинами и разделами Общей теории интеллектуальных систем и Базовой универсальной комплексной технологии проектирования интеллектуальных компьютерных систем (речь идет не только о дисциплинах, непосредственно относящихся к Искусственному интеллекту, но и о различных общеобразовательных и смежных дисциплинах, таких, как теория познания, методология, иностранные языки, современные компьютерные системы и сети, компьютеры нового поколения, теория алгоритмов и программ, ориентированных на современные компьютеры, семантическая теория алгоритмов и программ, ориентированных на обработку баз знаний и др.). Принципиально важно сформировать у студентов, магистрантов и аспирантов целостную картину проблематики Искусственного интеллекта и место Искусственного интеллекта в общей Картине Мира. Барьеров между учебными дисциплинами быть не должно.
    \end{scnitemizeii}
    \item Корпоративная ostis-система OSTIS-сообщества, являющегося субъектом Образовательной деятельности в области Искусственного интеллекта. Через эту корпоративную ostis-систему осуществляется взаимодействие между всеми членами указанного ostis-сообщества и, прежде всего между кафедрами, осуществляющими подготовку молодых специалистов в области Искусственного интеллекта.
\end{scnitemize}}
        \scnaddlevel{-1}        
    \scnaddlevel{-1}

\scnrelfromvector{принципы, лежащие в основе}{\scnfileitem{Подготовка молодых специалистов в области Искусственного интеллекта должна осуществляться путем поэтапного и непосредственного их подключения к реальным коллективным проектам:
\begin{scnitemize}
    \item к развитию базы знаний по Общей теории интеллектуальных систем, хранимой в памяти соответствующего интеллектуального портала знаний{;}
    \item к развитию базы знаний по \textit{Общей теории интеллектуальных компьютерных систем}, хранимой в памяти соответствующего интеллектуального портала знаний (в памяти \textit{Метасистемы IMS.ostis}){;}
    \item к развитию базы знаний по Базовой комплексной технологии проектирования интеллектуальных компьютерных систем, хранимой в памяти интеллектуальной компьютерной системы автоматизации проектирования интеллектуальных компьютерных систем (в памяти Метасистемы IMS.ostis){;}
    \item к развитию различных методов и средств проектирования различных компонентов интеллектуальных компьютерных систем{;}
    \item к развитию различных специализированных технологий проектирования различных классов интеллектуальных компьютерных систем{;}
    \item к разработке различных прикладных интеллектуальных компьютерных систем на основе развиваемой Базовой (универсальной) комплексной технологии проектирования интеллектуальных компьютерных систем.
\end{scnitemize}}
;\scnfileitem{Каждый студент и магистрант в процессе обучения привлекается к нескольким разным формам деятельности в области Искусственного интеллекта и, в частности, обязательно и к разработке приложений, и к развитию технологий. Специалист, занимающийся автоматизацией какой-либо деятельности должен на себе прочувствовать проблемы и трудности этой автоматизируемой деятельности}
;\scnfileitem{Все студенты, магистранты и преподаватели должны активно участвовать в анализе эффективности своей образовательной деятельности и активно способствовать повышению эффективности и повышению уровня автоматизации этой деятельности с помощью развиваемой технологии проектирования и производства интеллектуальных компьютерных систем. Данный принцип можно условно назвать устранением синдрома "сапожника без сапог"{}.}
;\scnfileitem{Результаты самостоятельной работы студентов и магистрантов (лабораторных работ, практических занятий, рефератов, курсовых работ и проектов, дипломных работ и проектов, магистерских диссертаций) должны быть востребованы в тех проектах, к которым они подключены и должны быть доведены до уровня внедрения в эти проекты, т. е. должны быть по соответствующей процедуре согласованы и одобрены. При этом приветствуется и соответствующим образом поощряется любая такого рода инициатива студентов и магистрантов. Указанная востребованность (полезность) результатов самостоятельной работы студентов и магистрантов предполагает то, что отчеты по этим результатам оформляются в формализованном виде -- в виде исходных текстов соответствующих фрагментов баз знаний. При этом указанные результаты могут требовать как весьма высокой квалификации, так и не очень высокой (например, квалификации первокурсника). К таким несложным, но весьма полезным работам относятся:
\begin{scnitemize}
    \item введение в базы знаний полезных библиографических ссылок и цитат;
    \item сравнительный анализ различных положений, представленных в некоторой разрабатываемой базе знаний;
    \item различные пояснения, примечания и комментарии, вводимые в базу знаний;
    \item спецификация выявленных в разрабатываемой базе знаний ошибок,  противоречий, информационных дыр и информационного мусора;
    \item примеры, иллюстрирующие различные понятия;
    \item упражнения к различным разделам разрабатываемых баз знаний, которые особенно актуальны для интеллектуальных компьютерных систем, используемых в учебном процессе (это не только интеллектуальные обучающие системы).
\end{scnitemize}}
;\scnfileitem{Вклад каждого студента и магистранта в развитие всех проектов, в которых он принимает участие, фиксируется и при подведении итогов по каждому семестру соответствующим образом оценивается. Это своего рода предтеча будущего рынка знаний.}
;\scnfileitem{Учебным пособием по каждой учебной дисциплине должна быть база знаний или некоторый раздел базы знаний некоторой интеллектуальной компьютерной системы. Такой может быть либо интеллектуальная обучающая система, либо, например, Метасистема IMS.ostis. Условием максимально эффективного проведения лекционного занятия является предварительное прочтение студентами или магистрантами материала предстоящей лекции (соответствующего раздела базы знаний). Тогда на лекции можно акцентировать внимание не на изложение материала, опубликованного в виде базы знаний, а на обсуждение непонятных фрагментов этого материала, на обсуждение проблем, касающихся содержания (принципиальных положений) этого материала. Все это формирует культуру взаимопонимания и согласования различных точек зрения, а также способствует повышению качества базы знаний, представляющей материал соответствующей учебной дисциплины.}
;\scnfileitem{Важнейшей задачей подготовки молодых специалистов является формирование у них:
\begin{scnitemize}
	\item высокой математической культуры (культуры формализации);
	\item высокой системной культуры (понимания того, что количество далеко не всегда переходит в ожидаемое качество);
	\item высокого уровня технологической культуры, технологической дисциплины, четкого соблюдения текущих стандартов и способности участвовать в эволюции стандартов;
	\item способности работать в наукоемких проектах в составе творческих коллективов с децентрализованным управлением;
	\item способности к достижению семантической совместимости (взаимопонимания) со своими коллегами;
	\item договороспособности (способности к согласованию различных точек зрения).
\end{scnitemize}}
;\scnfileitem{Подготовку молодых специалистов в области Искусственного интеллекта можно осуществлять с ориентацией на следующие условно выделенные уровни их квалификации:
\begin{scnitemize}
	\item инженерия прикладных интеллектуальных компьютерных систем по заданной технологии{;}
	\item инженерия специализированных технологий проектирования различных классов прикладных интеллектуальных компьютерных систем (на основе базовой универсальной комплексной технологии проектирования интеллектуальных компьютерных систем){;}
	\item инженерия базовой универсальной комплексной технологии проектирования интеллектуальных компьютерных систем{;}
	\item инженерия программных и аппаратных средст, интерпретации логико-семантических моделей интеллектуальных компьютерных систем{;}
	\item инженерия комплексов интеллектуальных компьютерных систем{;}
	\item научно-исследовательская деятельность по развитию \textit{Общей формальной теории интеллектуальных компьютерных систем}.
\end{scnitemize}}}

\end{SCn}

\newpage
\scnsegmentheader{Понятие Экосистемы OSTIS}

\scnstartsubstruct

\scnidtf{Использование \textit{Технологии OSTIS} для повышения качества и, в частности, уровня автоматизации всех \textit{областей человеческой деятельности}}
\scnidtf{Понятие \textbf{Экосистемы OSTIS} как формы реализации \textit{smart-общества}, представляющего собой сеть взаимодействующих людей, интеллектуальных компьютерных систем, "умных"{} домов, "умных"{} предприятий, "умных"{} больниц, "умных"{} учебных заведений, "умных"{} городов, "умных"{} транспортных систем и т.п.}

\scnrelfromset{рассматриваемые вопросы}{
\scnfileitem{Какова архитектура \textit{Экосистемы OSTIS}};
\scnfileitem{Какова архитектура \textit{ostis-сообщества}, входящего в состав \textit{Экосистемы OSTIS}};
\scnfileitem{Как взаимодействуют между собой различные \textit{ostis-сообщества} в рамках \textit{Экосистемы OSTIS}};
\scnfileitem{Как интегрируется \textit{деятельность} различных \textit{ostis-сообществ} и результаты этой \textit{деятельности}};
\scnfileitem{Какова типология \textit{ostis-сообществ} и по каким признакам классификации можно эту типологию проводить};
\scnfileitem{Можно ли опыт автоматизации деятельности в области Искусственного интеллекта с помощью \textit{Технологии OSTIS} расширить на все многообразие областей и видов человеческой деятельности};
\scnfileitem{Как выглядит систематизация областей и видов человеческой деятельности};
\scnfileitem{Как осуществляется конвергенция и интеграция различных областей и видов человеческой деятельности};
\scnfileitem{Как взаимодействуют ostis-системы, осуществляющие автоматизацию различных областей видов человеческой деятельности};
\scnfileitem{Как может выглядеть \uline{комплексная} автоматизация всех областей и видов \textit{человеческой деятельности} с помощью \textit{Технологии OSTIS}}
}
\scnrelfromvector{план изложения}{
\scnfileitem{Что такое Экосистема OSTIS};
\scnfileitem{Структура Экосистемы OSTIS};
\scnfileitem{Что такое ostis-система, являющаяся агентом Экосистемы OSTIS};
\scnfileitem{Что такое ostis-сообщетво, являющееся агентом Экосистемы OSTIS};
\scnfileitem{Что такое Проект создания Экосистемы OSTIS};
\scnfileitem{Цель создания и основные свойства Экосистемы OSTIS};
\scnfileitem{Как структурируется человеческая деятельность};
\scnfileitem{Как выглядит рынок знаний, реализуемый в рамках Экосистемы OSTIS};
\scnfileitem{Чем определяется качество человеческой деятельности};
\scnfileitem{Что такое эффективная автоматизация человеческой деятельности};
\scnfileitem{Почему повышение эффективности человеческой деятельности невозможно без интеллектуальных компьютерных систем};
\scnfileitem{Какие достоинства имеет Экосистема OSTIS}
}

\bigskip
\scnfragmentcaption

\scnheader{Экосистема OSTIS}
\scntext{вопрос}{Какова структура Экосистемы OSTIS}
\scnheader{Экосистема OSTIS}
\scnidtf{Популяция
\begin{scnitemize}
\item семантически совместимых
\item эволюционируемых
\item активно взаимодействующих  
\item способных координировать(согласовывать) свою деятельность с другими субъектами
\end{scnitemize}
интеллектуальных компьютерных систем(\textit{ostis-систем}). При этом указанная популяция \textit{ostis-систем} поддерживает децентрализованное управление собственной деятельностью, а также деятельностью людей(пользователей \textit{ostis-систем}) и человеко-машинных сообществ(\textit{ostis-сообществ}), обеспечивая тем самым автоматизацию системной интеграции любых новых субъектов(\textit{ostis-систем}, людей, \textit{ostis-сообществ}) в состав \textit{Экосистемы OSTIS}.
}
\scnheader{Экосистема OSTIS}
\scnrelfromvector{принципы, лежащие в основе}
{
\scnfileitem{\textit{Экосистема OSTIS} представляет собой сеть \textit{ostis-сообществ}};
\scnfileitem{Каждому \textit{ostis-сообществу} взаимно однозначно соответствует \textit{корпоративная ostis-система} этого \textit{ostis-сообщества}, которая:
\begin{scnitemize}
\item обеспечивает координацию деятельности членов соответствующего \textit{ostis-сообщества};
\item является "представителем"{} этого \textit{ostis-сообщества} в других \textit{ostis-сообществах}, членом которых указанное \textit{ostis-сообщество} является.
\end{scnitemize}
};
\scnfileitem{Каждое \textit{ostis-сообщество} может входить в состав любого другого \textit{ostis-сообщества} по своей инициативе. Формально это означает, что \textit{корпоративная ostis-система} первого \textit{ostis-сообщества} является членом другого \textit{ostis-сообщества}.};
\scnfileitem{Каждому специалисту, входящему в состав Экосистемы OSTIS ставится во взаимнооднозначное соответствие его \textit{персональный ostis-ассистент}, который трактуется как \textit{корпоративная ostis-система} вырожденного \textit{ostis-сообщества}, состоящего из одного человека.}
}
\scnheader{следует отличать*}
\scnhaselementset{корпоративная ostis-система*
\scnaddlevel{1}
\scnidtf{корпоративная ostis-система данного ostis-сообщества*}
\scnaddlevel{-1};
корпоративная ostis-система
\scnaddlevel{1}\\
\scnrelto{второй домен}{корпоративная ostis-система*}
\scnaddlevel{-1};
член ostis-сообщества*;
персональный ostis-ассистент*
\scnaddlevel{1}
\scnidtf{персональный ostis-ассистент данного специалиста*}
\scnsubset{корпоративная ostis-система*}
\scnaddlevel{-1};
персональный ostis-ассистент
\scnaddlevel{1}\\
\scnrelto{второй домен}{персональный ostis-ассистент*}
\scnsubset{корпоративная ostis-система}
\scnaddlevel{-1}
}
\scnheader{есть сходства*}
\scnhaselementset{Экосистема OSTIS;
ostis-сообщество\\
\scnaddlevel{1}
\scnhaselement{Экосистема OSTIS}
\scnaddlevel{-1}
}
\scnaddlevel{1}
\scnexplanation{Экосистема OSTIS является максимальным ostis-сообществом, включающим в себя все существующее ostis-сообщества}
\scnaddlevel{-1}
\scnheader{Экосистема OSTIS}
\scnidtf{Максимальное \textit{иерархическое ostis-сообщество}, обеспечивающее комплексную автоматизацию \uline{всех} видов \textit{человеческой деятельности}}
\scnidtf{Максимальное ostis-сообщество такое ostis-сообщество, для которого не существует другого ostis-сообщества, содержащее указанное выше ostis-сообщество в качестве своего члена}
\scnidtf{Симбиоз людей и \textit{компьютерных систем}(точнее, \textit{ostis-систем}) являющийся вариантом реализации \textit{smart-общества}}
\scniselement{иерархическое ostis-сообщество}
\scnaddlevel{1}
\scnidtf{такое \textit{ostis-сообщество}, по крайней мере одним из членов которого является некоторое другое \textit{ostis-сообщество}}
\scnsubset{ostis-сообщество}
\scnaddlevel{1}
\scnrelboth{следует отличать}{коллектив ostis-систем}
\scnaddlevel{-1}
\scnaddlevel{-1}{
\scnrelto{основной продукт}{Технология OSTIS}
\scnrelto{вариант реализации}{smart-общество}
\scnheader{агент Экосистемы OSTIS}
\scnidtf{субъект Экосистемы OSTIS}
\scnidtf{субъект, входящий в состав Экосистемы OSTIS}
\scnsuperset{когнитивный агент Экосистемы OSTIS}
\scnsubdividing{индивидуальная ostis-система Экосистемы OSTIS
\scnaddlevel{1}
\scnidtf{индивидуальная ostis-система, входящая в состав Экосистемы OSTIS}
\scnaddlevel{-1};
ostis-сообщество Экосистемы OSTIS\\
\scnaddlevel{1}
\scnsubdividing{
простое ostis-сообщество Экосистемы OSTIS;
иерархическое ostis-сообщество Экосистемы OSTIS
}
\scnaddlevel{-1};
пользователь Экосистемы OSTIS
}
\scnheader{агент Экосистемы OSTIS}
\scnrelfrom{правила поведения}{ 
Правила поведения агентов Экосистемы OSTIS\\
\scneqtoset{
\scnfileitem{Согласовывать денотационную семантику всех используемых знаков(в первую очередь \uline{понятий})};
\scnfileitem{Согласовывать терминологию, соответствующую введенным знакам устранять противоречия и информационные дыры};
\scnfileitem{Постоянно бороться с синонимией и омонимией как на уровне sc-элементов(внутренних знаков), так и на уровне соответствующих им терминов и прочих внешних идентификаторов(внешних обозначений)};
\scnfileitem{Каждый агент Экосистемы OSTIS по своей инициативе может стать членом любого ostis-сообщества Экосистемы OSTIS после соответствующей регистрации}
}
}
\scnheader{Правила поведения агентов Экосистемы OSTIS}
\scnnote{
Существенно подчеркнуть, что все правила функционирования(поведения) в рамках агентов Экосистемы OSTIS должны соблюдать не только ostis-системы, являющиеся агентами(субъектами) этой Экосистемы, но и люди, которые являются её агентами. И здесь возникают очень важные проблемы, обусловленные человеческим фактором. Дело в том, что убедить человека соблюдать правила, пусть даже те которые направлены на максимальную его самореализацию и в совершенствовании которых он может реально участвовать, очень непросто, поскольку любые  подобные правила многими воспринимаются как ограничение их творческой свободы. Другими словами корректное поведение ostis-системы в роли агентов Экосистемы OSTIS значительно проще, чем корректное поведение людей в качестве таких агентов. Поведение пользователей (естественных агентов) Экосистемы OSTIS необходимо внимательно мониторить и контролировать, постоянно способствуя повышению уровня их квалификации как агентов Экосистемы OSTIS, а также повышению уровня их мотивации, целенаправленности, самореализации.}
\scnheader{следует отличать*}
\scnhaselementset{
агент Экосистемы OSTIS;
член ostis-сообщества*
}
\scnheader{Экосистема OSTIS}
\scntext{архитектура}{
В Экосистеме OSTIS можно выделить следующие уровни иерархии:
\begin{scnitemize}
\item индивидуальные компьютерные системы(\textit{индивидуальные ostis-системы} и \textit{люди}, являющиеся конечными пользователями ostis-систем);
\item иерархическая система ostis-сообществ, членами каждого из которых могут быть \textit{индивидуальные ostis-системы}, люди, а также другие \textit{ostis-сообщества};
\item \textit{Максимальное ostis-сообщество} \scnbigspace \textit{Экосистемы OSTIS}, не являющееся членом никакого другого \textit{ostis-сообщества}, входящего в состав \textit{Экосистемы OSTIS}.
\end{scnitemize}
Подчеркнем, что качество \textit{Экосистемы OSTIS} во многом определяется эффективностью взаимодействия каждой \textit{ostis-системы}(в том числе и каждого \textit{ostis-сообщества}), а также каждого \textit{человека} со своей \textit{внешней средой*}, а также качеством(чистотой), самой \textit{внешней среды*}.
Но внешняя среда каждого \textit{субъекта} каждой ostis-системы и каждого человека, входящего в \textit{Экосистему OSTIS} - это не только \textit{материальная внешняя среда*}, но и \textit{информационная внешняя среда*}, представляющая собой виртуальный распределенный информационный ресурс, являющийся интеграцией(объединением) информации, хранящейся в текущий момент в памяти всех других(остальных) \textit{субъектов}, входящих в \textit{Экосистему OSTIS}. Основной целью \textit{Экосистемы OSTIS} является повышение качества(в том числе чистоты) \textit{информационной внешней среды*} для \uline{всех} \textit{субъектов}, входящих в \textit{Экосистему OSTIS}. Фактически речь идет об \textbf{Информационной экологии человеческого общества}.
}
\scnheader{Информационная экология человеческого общества}
\scnnote{
Говоря об \textit{Информационной экологии человеческого общества} необходимо заметить следующее. Современные подходы к развитию взаимодействия с информационной средой человеческого общества можно разбить на два направления:
\begin{scnitemize} 
\item на разработку средств приспособления к недостаткам текущего состояния этой среды
\item на устранение этих недостатков путем наведения порядка в устной информационной среде и её систематизации.
\end{scnitemize}
Технология OSTIS и реализация Экосистемы OSTIS целенаправленно и в известной степени радикально ориентирована на второе направление, памятуя искусственный(рукотворный) характер происхождения этой информационной среды.}
%памятуя?
\scnendstruct



\scnendstruct
\newpage
\scnaddlevel{1}
\scnidtf{Человеко-машинная деятельность, осуществляемая в рамках \textit{Экосистемы OSTIS} и направленная на разработку и перманентное совершенствование \textit{Метасистемы 
IMS.ostis}, которая является формой представления (отображения) (1) текущего состояния \textit{Технологии OSTIS}, как комплекса методов и средств автоматизации (поддержки) разработки\textit{ostis-систем} и (2) текущего состояния самого \textit{Проекта IMS.ostis}.}
\scntext{примечание}{Принципы (правила) организации деятельности в рамках \textit{Проекта IMS.ostis} полностью совпадают с принципами (правилами) организации деятельности в рамках любого другого проекта, направленного на разработку и совершенствование любой другой ostis-системы.}
\scnrelto{ключевой подпроект}{Проект Экосистемы OSTIS}
\scnaddlevel{1}
\scnidtf{Совместная деятельность ученых, инженеров и ostis-систем, входящих в \textit{Экосистему OSTIS}, направленная на перманентное совершенствование \textit{Экосистемы OSTIS} -- на совершенствование (реинжиниринг) входящих в неё  \textit{ostis-систем} и на создание новых ostis-систем и их включение в состав \textit{Экосистемы OSTIS.}}
\scnaddlevel{-1}
\scntext{пояснение}{
ostis-система, являющаяся:
\begin{scnitemize}
	\item ostis-порталом научно-технических знаний по Технологии OSTIS, база знаний которого включает в себя:
	\begin{scnitemizeii}
		\item формальную теорию ostis-систем
		\item формальную теорию (методику) проектирования 󠇦 ostis-систем
		\item формальную спецификацию средств автоматизации проектирования ostis-систем
		\item библиотеку проектирования ostis-систем
		\item формальную спецификацию средств производства спроектированных ostis-систем
	\end{scnitemizeii}
	\item ostis-системой автоматизации (поддержки) проектирования ostis-систем
	\item ostis-системой поддержки производства (сборки, синтеза, генерации) спроектированных ostis-систем
	\item ostis-системой поддержки реинжиниринга ostis-систем в ходе их эксплуатации
\end{scnitemize}
}
\scnaddlevel{-1}

\scnheader{Метасистема IMS.ostis}
\scnidtf{Универсальная базовая (предметно-независимая) ostis-система автоматизации проектирования ostis-систем (любых ostis-систем)}
\scnrelboth{следует отличать}{специализированная ostis-система автоматизации проектирования ostis-систем}
\scniselement{ostis-система}
\scnrelto{корпоративная ostis-система}
{Консорциум OSTIS}
\scnidtf{IMS.ostis}
\scnidtf{Интеллектуальная метасистема, построенная по стандартам \textit{технологии OSTIS} и предназначенная (1) для инженеров \textit{ostis-систем} -- для поддержки проектирования. Реализации и обновления (реинжиниринга) \textit{ostis-систем} и для разработчиков \textit{Технологии OSTIS} -- для поддержки коллективной деятельности по развитию стандартов и библиотек \textit{Технологии OSTIS.}}
\scnrelto{форма реализации}{Технология OSTIS}
\scnrelto{продукт}{Проект IMS.ostis}
\scnidtf{Интеллектуальная Метасистема, являющаяся формой (вариантом) реализации (представления, оформления) \textit{Технологии OSTIS} в виде \textit{ostis-системы}}
\scntext{примечание}{Тот факт, что Технология OSTIS реализуется в виде ostis-системы, является весьма важным для эволюции Технологии OSTIS, поскольку методы и средства эволюции (перманентного совершенствования) Технологии OSTIS становятся фактически совпадающими с методами и средствами разработки любой (!) ostis-системы на всех этапах их жизненного цикла.\\
Другими словами, эволюция Технологии OSTIS осуществляется методами и средствами самой этой технологии.}
\scnidtf{Система комплексной автоматизации (информационной и инструментальной поддержки) проектирования и реализации ostis-систем, которая сама реализована также в виде ostis-системы.}
\scnidtf{Портал знаний по Технологии OSTIS, интегрированный с САПРом ostis-систем и реализованный в виде ostis-системы.}
\scniselement{портал научно-технических знаний}

\bigskip
\bigskip
\scnstartset
\scnheader{Метасистема IMS.ostis}
\scniselement{система автоматизации проектирования}
\scnaddlevel{1}
\scnidtf{CAD-система}
\scnaddlevel{1}
\scnrelto{аббревиатура}{\scnfilelong{Computer Aided Design system}}
\scnaddlevel{-2}
\scniselement{интеллектуальная обучающая система}
\scnendstruct

\scnrelboth{семантическая эквивалентность}{\scnfilelong{Метасистема IMS.ostis является одновременно и системой автоматизации проектирования ostis-систем, и интеллектуальной системой, обучающей методам  и средствам проектирования ostis-систем.}}
\scnaddlevel{1}
\scntext{следовательно}{этот факт существенно повышает качество проектирования прикладных ostis-систем, расширяет контингент разработчиков ostis-систем и интегрирует проектную (инженерную) деятельность в области искусственного интеллекта с образовательной деятельностью в этой области.}
\scnaddlevel{-1}

\scnheader{следует отличать*}
\scnhaselementset{
конвергенция
\scnaddlevel{1}
\scnidtf{Процесс сближения структурных и/или функциональных характеристик нескольких (как минимум двух) заданных сущностей}
\scnidtf{Процесс конвергенции заданных сущностей в ходе их изменения, совершенствование, эволюции}
\scnsubset{процесс}
\scnaddlevel{-1};
конвергенция\scnsupergroupsign 
\scnaddlevel{1}
\scnidtf{Степень близости (сходство) заданных сущностей}
\scniselement{свойство}
\scnaddlevel{-1}
}
\scnheader{конвергенция} 
\scnnote{ 
\textit{Конвергенция} пар конкретных искусственных сущностей (например, технических систем) есть стремление их унификацию (в частности, к стандартизации), т.е. стремление к минимизации многообразия форм решения аналогичных практических задач -- стремление к тому, чтобы все, что можно сделать одинаково, сделалось одинаково, но без ущерба требуемого качества. Последнее очень важно, так как безграмотная стандартизация может привести к существенному торможению прогресса. Ограничение многообразия форм не должно приводить к ограничению содержания, возможностей. Образно говоря, "словам должно быть тесно, а мыслям -- свободно".}
\scnnote{Методологически конвергенция искусственно создаваемых сущностей (артефактов) сводится (1) к выявлению (обнаружению) принципиальных сходств между этими сущностями, которые часто весьма закамуфлированы и их трудно "увидеть", и (2) к реализации обнаруженных сходств одинаковым образом (в одинаковой форме, в одинаковом "синтаксисе"). Образно говоря, от "семантической"{} (смысловой) эквивалентности требуется перейти и к "синтаксической" эквивалентности. Кстати, в этом как раз и заключается суть (идея) смыслового представления информации (знаний), целью которого является создание такой языковой среды (\textit{смыслового пространства}), в рамках которого (1) семантически эквивалентные информационные конструкции полностью совпадали, а (2) конвергенция информационных конструкций сводилась бы к выявлению изоморфных фрагментов этих конструкций.}
\scnnote{Очень важно уточнить, формализовать понятие конвергенции (конвергенции знаний, методов, модели решения задач, конвергенции интеллектуальных компьютерных систем в целом)}
\scnsuperset{конвергенция информационных конструкций}
\scnaddlevel{1}
\scnidtf{конвергенция синтаксических и семантических свойств информационных конструкций }
\scnaddlevel{-1}
\scnsuperset{конвергенция языков}
\scnsuperset{конвергенция научных дисциплин}
\scnaddlevel{1}
\scnidtf{конвергенция различных научных дисциплин или различных направлений одной и той же и дисциплины}
\scnaddlevel{-1}
\scnsuperset{конвергенция баз знаний}
\scnsuperset{конвергенция моделей решения задач}
\scnsuperset{конвергенция гибридных решателей задач}
\scnsuperset{конвергенция кибернетических систем}
\scnsuperset{конвергенция интеллектуальных систем}
\scnaddlevel{1}
\scnsuperset{конвергенция интеллектуальных систем, направленная на обеспечение их \uline{семантической совместимости}}
\scnaddlevel{-1}

\scnheader{конвергенция результатов научно-технической деятельности}
\scnnote{Важным препятствием для конвергенции результатов научно-технической деятельности является сформировавшийся в науке и технике акцент на выявлении не сходств, а отличий. Чтобы убедиться в этом достаточно обратить внимание на то, что уровень научных результатов оценивается научной \uline{новизной}, которая может имитироваться новизной не по существу, а по форме представления (например, с помощью новых понятий или даже новых терминов). Результаты в технике, например, в патентах также оцениваются \uline{отличиями} от предшествующих технических решений. Но для конвергенции нужны другие акценты -- ни поиск отличий, а выявление неочевидных сходств и превращения их в очевидные сходства, представленные в одинаковой \uline{форме}.}

\scnheader{совместимость\scnsupergroupsign}
\scnidtf{совместимость заданных двух или более сущностей\scnsupergroupsign}
\scnidtf{простота интеграции заданной группы сущностей\scnsupergroupsign}
\scnidtf{интегрируемость\scnsupergroupsign}
\scnnote{Степень (уровень) совместимости заданных сущностей может рассматриваться как оценка результата их конвергенции. Чем качественнее (основательнее, глубже) проведена конвергенция заданных сущностей, тем выше уровень их совместимости и, собственно, тем легче их интегрировать.}

\scnsuperset{cовместимость информационных конструкций\scnsupergroupsign}
\scnaddlevel{1}
\scnsuperset{семантическая совместимость информационных конструкций\scnsupergroupsign}
\scnaddlevel{-1}
\scnsuperset{совместимость языков\scnsupergroupsign}
\scnaddlevel{1}
\scnsuperset{семантическая совместимость языков\scnsupergroupsign}
\scnaddlevel{-1}
\scnsuperset{семантическая совместимость научных дисциплин\scnsupergroupsign}
\scnsuperset{совместимость баз знаний\scnsupergroupsign}
\scnsuperset{совместимость моделей решения задач\scnsupergroupsign}
\scnsuperset{совместимость кибернетических систем\scnsupergroupsign}
\scnaddlevel{1}
\scnsuperset{семантическая совместимость кибернетических систем\scnsupergroupsign}
\scnaddlevel{-1}
\scnsuperset{семантическая совместимость\scnsupergroupsign}

\scnheader{интеграция*}
\scnidtf{объединение нескольких разных сущностей, в результате чего возникает некоторая объединённая целостная сущность*}
\scnsuperset{эклектичная интеграция*}
\scnaddlevel{1}
\scnidtf{Интеграция разнородных (гетерогенных) сущностей, которой не предшествует конвергенция (сближение) этих сущностей*}
\scnaddlevel{-1}
\scnsuperset{глубокая интеграция*}
\scnnote{Понятие \textit{интеграции*} и особенно понятие \textit{глубокой интеграции*} имеет тесную связь с понятием \textit{конвергенции\scnsupergroupsign}. Чем выше степень конвергенции (степень сближения) интегрируемых объектов, тем выше качество результата интеграции. Особенно, если речь идёт о глубокой интеграции.}

\scnheader{глубокая интеграция*}
\scnidtf{"бесшовная"{} интеграция*}
%TODO ссылка на Грибову
\scnidtf{интеграция однородных сущностей, предполагающая глубокую взаимную "диффузию"{} (сращивание) соединяемых сущностей, которая не обязательно должна осуществляться физически}
\scnnote{Примером виртуальной глубокой интеграции является формирование коллектива \uline{семантический совместимых} индивидуальный кибернетических систем}
\scnidtf{бесшовная интеграция*}
\scnidtf{гибридизация*}
\scnidtf{интеграция, результатом которой являются гибридные объекты*}
\scnidtf{интеграция, которой предшествует высокий уровень конвергенции интегрируемых объектов*}
\scnidtf{(конвергенция + интеграция)*}
\scnidtf{"бесшовная"{} интеграция}
\scnidtf{интеграция, в результате которой возникает гибридная система*}
\scnidtf{интеграция, которой предшествует конвергенция (в частности, унификация) интегрируемых систем, приведение этих систем к максимально похожему виду (общему знаменателю)*}
%TODO сложно при чтении воспринимать, конвергенция и приведение как-то сливаются, становится не совсем понятно, к чему относится приведение к конвергенции или к интеграции, может как-то более явно указать, что конвергенция это то приведение?
\scnidtf{интеграция с "диффузией"{} , взаимопроникновением на основе унификации того, что можно сделать одинаковым*}

\scnheader{интеграция*}
\scnsuperset{интеграция информационных конструкций}
\scnsuperset{интеграция языков}
\scnsuperset{интеграция научных дисциплин}
\scnsuperset{интеграция баз знаний}
\scnsuperset{интеграция моделей решения задач}
\scnsuperset{интеграции индивидуальных кибернетических систем}
\scnaddlevel{1}
\scnsuperset{слияние индивидуальных кибернетических систем}
\scnaddlevel{1}
\scnidtf{преобразование нескольких \uline{искусственных} индивидуальных кибернетических систем в интегрированную индивидуальную кибернетическую систему, которая способна решать все задачи, каждая из которых могла бы быть решена в рамках какой-либо из интегрируемых систем}
\scnaddlevel{-1}
\scnsuperset{формирование коллектива индивидуальных кибернетических систем}
\scnaddlevel{1}
\scnidtf{формирования многоагентной системы, состоящей из индивидуальных кибернетических систем}
\scnaddlevel{-1}
\scnnote{Эффективность интеграции индивидуальных кибернетических систем определяется тем, насколько объем задач, решаемых коллективом индивидуальных кибернетических систем, превысит объединение объёмов задач, решаемых членами коллектива в отдельности.}
\scnaddlevel{-1}

\bigskip
\scnendstruct \scninlinesourcecommentpar{Завершили Сегмент "\textit{Текущее состояние и проблемы дальнейшего развития деятельности в области Искусственного интеллекта}"}

\scsection{Понятие ostis-сообщества}
\label{intro_ostis}

\begin{SCn}
\scnsegmentheader{\currentname}

\scnstartsubstruct

\scnheader{ostis-сообщество}
\scnidtf{Человеко-машинный симбиоз, представляющий собой коллектив, состоящий из людей и ostis-систем и обеспечивающий высокий уровень автоматизации определённого (соответствующего) вида человеческой деятельности.}
\filemodefalse
\scnaddlevel{1}
\scnnote{В состав каждого ostis-сообщества входит корпоративная ostis-система, которая в рамках этого ostis-сообщества выполняет: 
\begin{scnitemize}
\item роль координатора деятельности членов данного ostis-сообщества;
\item роль памяти ostis-сообщества, т.е. хранителя общих (обобществляемых, общедоступных) знаний для всех членов данного ostis-сообщества, которое несет ответственность за совершенствование этих знаний, а также для всех членов всех тех ostis-сообществ, в состав которых данное ostis-сообщество входит (указанные субъекты являются пользователями рассматриваемых общих знаний). Таким образом, корпоративная ostis-система некоторого ostis-сообщества является "официальным" представителем этого ostis-сообщества во всех ostis-сообществах, в состав которых входит, и, следовательно, является координатором деятельности даного ostis-сообщества (как единого целого) в рамках всех ostis-сообществ, в состав которых оно входит;
\end{scnitemize}
\scnaddlevel{-1}
}
\scnrelfromset{есть сходства}{ostis-сообщество; решатель задач ostis-системы}

\scnheader{есть сходства*}
\scnhaselementset{ostis-сообщество; решатель задач ostis-системы}
	\scnaddlevel{1}
	\scnexplanation{ ostis-сообщество\\
	\scnsuperset{многоагентная система,в которой управление агентами осуществляется через общую для них память}
	\scnsuperset{многоагентная система, с децентрализованным управлением агентами}
	\scnsuperset{многоагентная система, в которой областью деятельности её агентов является как внешняя среда, так и память этой системы}
	решатель задач ostis-системы\\
	\scnsuperset{многоагентная система,в которой управление агентами осуществляется через общую для них память}
	\scnsuperset{многоагентная система, с децентрализованным управлением агентами}
	\scnsuperset{многоагентная система, в которой областью деятельности её агентов является как внешняя среда, так и память этой системы}
	\scnsuperset{агентно-ориентированная модель обработки информации в памяти}}   
    \scnaddlevel{-1}

\scnrelfromlist{пример}{оркестр, играющий без дирижера или даже без композитора\\
	  \scnaddlevel{1}
	  \scntext{необходимое требование}{каждый участник оркестра должен иметь квалификацию дирижера или композитора};
	  \scnaddlevel{-1}
комплексная строительная бригада, работающая без прораба\\
	  \scnaddlevel{1}
	  \scntext{необходимое требование}{каждый участник строительной бригады должен иметь квалификацию прораба};
	  \scnaddlevel{-1}
научно-исследовательская лаборатория, работающая без заведующего и научного руководителя\\
	  \scnaddlevel{1}
	  \scntext{необходимое требование}{каждый участник научно-исследовательской лаборатории должен иметь квалификацию заведующего или научного руководителя};
	  \scnaddlevel{-1}
кафедра, работающая без заведующего и ученого секретаря\\
	  \scnaddlevel{1}
	  \scntext{необходимое требование}{каждый участник кафедры должен иметь квалификацию заведующего и ученого секретаря}
	  \scnaddlevel{-1}
}

\scnrelfromset{Специфика реализации наукоемких (непредсказуемых) проектов}{цель; мотивация общего результата; квалификция на уровне дирижера-композитора(прораба-архитектора); коммуникабельность; моральные принципы не как я красиво играю, а какая красивая музыка, которую мы вместе делаем}

\scnnote{Новый подход к наукоемкому project-менеджменту "каждый строитель" должен иметь квалификацию прораба (орекстр без дирижера, нот и указаний). Распределение ответственности, а не задач! Почему технология разработки, эксплуатации и совершенствования (реинжиниринг) нового поколения должна быть устроена на основе децентрализованного, не целенаправленного управления.}

\textbf{\textit{Необходимые факторы}}
\begin{scnenumerate}
\item \textnormal{высокая квалификация и человеческие качества (моральные) участников;}
\item \textnormal{согласованное формирование целей и подцелей;}
\item \textnormal{персонификация вкладов (самоконтроль);}
\item \textnormal{открытость.}
\end{scnenumerate}

/* Завершить рассмотрение понятия ostis-сообщества */

\end{SCn}


\scsubsection[\scneditor{Загорский А.Г.}\protect\scnmonographychapter{Глава 7.2. Экосистема интеллектуальных компьютерных систем нового поколения (Экосистема OSTIS) и реализация рынка знаний на ее основе}]{Логико-семантическая модель интеграции разнородных информационных ресурсов и сервисов в Экосистеме OSTIS в процессе ее расширения}
\label{services_integr_logical_model}

\scsubsubsection[\scneditor{Банцевич К.А.}\protect\scnmonographychapter{Глава 7.2. Экосистема интеллектуальных компьютерных систем нового поколения (Экосистема OSTIS) и реализация рынка знаний на ее основе}]{Предметная область и онтология библиографических источников и других информационных ресурсов}
\label{sd_bibliography}

\scsubsection[\scnmonographychapter{Глава 7.2. Экосистема интеллектуальных компьютерных систем нового поколения (Экосистема OSTIS) и реализация рынка знаний на ее основе}]{Предметная область и онтология семантически совместимых интеллектуальных ostis-порталов научных знаний}
\label{sd_portals}
\begin{SCn}
    \scnsectionheader{Предметная область и онтология семантически совместимых интеллектуальных ostis-порталов научных знаний}
    \begin{scnsubstruct}
    \scnrelfrom{дочерний раздел}{\nameref{ims_ostis_model}}
    \scniselement{раздел базы знаний}
    \scnhaselementrole{ключевой sc-элемент}{Предметная область семантически совместимых интеллектуальных порталов научно-технических знаний}

    \begin{scnrelfromlist}{библиографическая ссылка}
        \scnitem{\scncite{Van2005}}
        \scnitem{\scncite{Mack2001}}
    \end{scnrelfromlist}
    
    \scnheader{Предметная область семантически совместимых интеллектуальных порталов научно-технических знаний}
    \scniselement{предметная область}
    \begin{scnhaselementrolelist}{максимальный класс объектов исследования}
        \scnitem{портал научных знаний}
    \end{scnhaselementrolelist}
    \begin{scnhaselementrolelist}{класс объектов исследования}
        \scnitem{портал знаний}
        \scnitem{ostis-портал знаний}
    \end{scnhaselementrolelist}
    
    \scnheader{портал научных знаний}
    \scntext{примечание}{Понятие \textit{портала знаний} представляет собой один из способов создания централизованного доступа к информации, которая может быть необходима для решения задач, связанных с работой в организации. Такие порталы могут содержать информацию, связанную с процессами, документацией, процедурами, обучающими материалами, а также ответы на часто задаваемые вопросы.}
    \begin{scnindent}
        \begin{scnrelfromset}{смотрите}
            \scnitem{\scncite{Van2005}}
            \scnitem{\scncite{Mack2001}}
        \end{scnrelfromset}
    \end{scnindent}
    \scntext{примечание}{Одним из ключевых преимуществ \textit{порталов знаний} является их способность к сбору и хранению информации из различных источников, таких как базы данных, системы управления документами, системы управления проектами и так далее. Это позволяет пользователям получать полную и актуальную информацию в одном месте.}
    \scntext{примечание}{На основе \textit{портала знаний} обеспечивается возможность взаимодействия между пользователями путем создания форумов, обсуждений и коллективного редактирования документов. Это может способствовать обмену знаниями и опытом между сотрудниками организации и повысить эффективность их работы.}
    \scntext{примечание}{При создании \textit{порталов знаний} возникают проблемы, связанные с организацией и управлением информацией. Например, необходимо обеспечить корректное и структурированное хранение информации, ее поиск и обновление. Также необходимо учитывать потребности пользователей и обеспечить удобный и интуитивно понятный интерфейс.}
    \begin{scnrelfromlist}{цель}
        \scnfileitem{Ускорение погружения каждого человека в новые для него научные области при постоянном сохранении общей целостной картины Мира (образовательная цель).}
        \scnfileitem{Фиксация в систематизированном виде новых научных результатов так, чтобы все основные связи новых результатов с известными были четко обозначены.}
        \scnfileitem{Автоматизация координации работ по рецензированию новых результатов.}
        \scnfileitem{Автоматизация анализа текущего состояния базы знаний.}
    \end{scnrelfromlist}
    \scntext{пояснение}{Создание интеллектуальных \textbf{\textit{порталов научных знаний}}, обеспечивающих повышение темпов интеграции и согласования различных точек зрения, --- это способ существенного повышения темпов эволюции научно-технической деятельности.\\
        Совместимые \textbf{\textit{порталы научных знаний}}, реализованные в виде \textit{ostis-систем}, входящих в \textit{Экосистему OSTIS}, являются основой новых принципов организации научной деятельности, в которой
        \begin{scnitemize}
            \item результатами этой деятельности являются не статьи, монографии, отчеты и другие научно-технические документы, а фрагменты глобальной базы знаний, разработчиками которых являются свободно формируемые научные коллективы, состоящие из специалистов в соответствующих научных дисциплинах,
            \item с помощью \textbf{\textit{порталов научных знаний}} осуществляется
            \begin{scnitemizeii}
                \item координация процесса рецензирования новой научно-технической информации, поступающей от научных работников в базы знаний этих порталов,
                \item процесс согласования различных точек зрения ученых (в частности, введению и семантической корректировке понятий, а также введению и корректировке терминов, соответствующих различным сущностям).
            \end{scnitemizeii}
        \end{scnitemize}
        Реализация семейства семантически совместимых порталов научных знаний в виде совместимых \textit{\mbox{ostis-систем}}, входящих в состав \textit{Экосистемы OSTIS}, предполагает разработку иерархической системы семантически согласованных формальных онтологий, соответствующих различным научно-техническим дисциплинам, с четко заданным наследованием свойств описываемых сущностей и с четко заданными междисциплинарными связями, которые описываются связями между соответствующими формальными онтологиями и специфицируемыми ими предметными областями.\\
        Реализация \textbf{\textit{порталов научных знаний}} в виде семейства семантически совместимых \textit{ostis-систем} означает также попытку преодолеть вавилонское столпотворение\ многообразия научно-технических языков, не меняя сути научно-технических знаний, а сводя эти знания к единой универсальной форме смыслового представления знаний в памяти порталов научных знаний, т.е. к форме которая в достаточной степени понятна как \textit{ostis-системам}, так и любым потенциальным их пользователям.}
    \scntext{пример}{Примером \textbf{\textit{портала научных знаний}}, построенного в виде \textit{ostis-системы} является \textit{Метасистема OSTIS}, содержащая все известные на текущий момент знания и навыки, входящие в состав \textit{Технологии OSTIS}.}
    \begin{scnrelfromlist}{преимущества}
        \scnfileitem{Использование методов семантической обработки информации, что позволяет более точно и эффективно организовывать и искать информацию на портале знаний.}
        \scnfileitem{Высокий уровень гибкости и расширяемости, что позволяет адаптировать \textit{ostis-порталы знаний} под различные нужды и требования пользователей.}
        \scnfileitem{Автоматическая интеграция \textit{ostis-порталов знаний} с другими \textit{ostis-системами} в рамках \textit{Экосистемы OSTIS}, что позволяет создать централизованный доступ к информации из различных источников.}
        \scnfileitem{Возможность создания персонализированного \textit{ostis-портала знаний}, который учитывает интересы и потребности каждого пользователя, что позволяет более эффективно использовать знания \textit{ostis-систем}.}
        \scnfileitem{Возможность производить \textit{ostis-порталы знаний} быстро и с минимальными затратами благодаря использованию существующих компонентов и инструментов.}
    \end{scnrelfromlist}
        \scntext{примечание}{Реализация \textit{порталов знаний} на основе \textit{Технологии OSTIS} позволяет создать более эффективную и гибкую систему для хранения, организации и поиска знаний, которая может быть адаптирована под различные требования пользователей и организаций.}

    \end{scnsubstruct}
    \scnendcurrentsectioncomment
\end{SCn}


\scsubsubsection[\scneditors{Шункевич Д.В.;Банцевич К.А.;Садовский М.Е.;Бутрин С.В.}\protect\scnmonographychapter{Глава 7.3. Метасистема OSTIS и Стандарт OSTIS}]{Логико-семантическая модель Метасистемы OSTIS}
\label{ims_ostis_model}
\begin{SCn}

\scnsectionheader{Семантическая модель Метасистемы IMS.ostis}

\scnstartsubstruct

\scnheader{Метасистема IMS.ostis}
\scntext{назначение}{Эффективность любой технологии, в том числе и \textit{\textbf{Технологии OSTIS}}~\cite{IMS} определяется не только сроками создания искусственных систем соответствующего класса, но и темпами совершенствования самой технологии (темпами совершенствования средств автоматизации и темпами совершенствования системы стандартов, лежащих в основе технологии).

Для фиксации текущего состояния \textit{Технологии OSTIS}, а также для организации ее эффективного использования и ее перманентного совершенствования с участием ученых, работающих в области искусственного интеллекта, и инженеров, разрабатывающих семантические компьютерные системы различного назначения, в состав \textit{Экосистемы OSTIS} вводится \textit{Метасистема IMS.ostis}~\cite{IMS}, назначение которой делает ее \underline{ключевой} \textit{ostis-системой} в рамках \textit{Экосистемы OSTIS}.}

\scnheader{Метасистема IMS.ostis}
\scnidtf{Интеллектуальная метасистема комплексной информационной и инструментальной поддержки проектирования совместимых семантических компьютерных систем, которая является формой реализации общей теории и технологии проектирования семантических компьютерных систем и которая поддерживает высокий темп эволюции указанной теории и технологии}
\scnidtf{Intelligent MetaSystem for intelligent systems design}
\scnidtf{IMS.ostis}
\scnidtf{Фреймворк интеллектуальных систем}
\scnidtf{Интеллектуальная метасистема комплексной поддержки проектирования совместимых семантических компьютерных систем по Технологии OSTIS}
\scnidtf{Фреймворк ostis-систем}
\scnidtf{Фреймворк IMS.ostis}

\scnheader{Метасистема IMS.ostis}
\scntext{назначение}{\textit{Метасистема IMS.ostis} является в \textit{Экосистеме OSTIS} ключевой интеллектуальной системой, которая поддерживает не только проектирование новых интеллектуальных систем и не только замену устаревших компонентов в интеллектуальных системах, входящих в состав \textit{Экосистемы OSTIS}, но и включение (интеграция) в состав \textit{Экосистемы OSTIS} новых создаваемых интеллектуальных систем.

\textit{Метасистема IMS.ostis} ориентирована на разработку и практическое внедрение методов и средств \textbf{компонентного проектирования} семантически совместимых интеллектуальных систем, которая предоставляет возможность быстрого создания интеллектуальных приложений различного назначения. Подчеркнем при этом, что сферы практического применения методики компонентного проектирования семантически совместимых интеллектуальных систем ничем не ограничены.}

\scnheader{Метасистема IMS.ostis}
\scnidtf{реализация технологии проектирования семантически совместимых компьютерных систем в виде метасистемы, построенной по той же технологии и обеспечивающей комплексную информационную и инструментальную поддержку проектирования семантически совместимых компьютерных систем}
\scntext{декомпозиция}{
\begin{scnitemize}
    \item полное описание самой Технологии OSTIS;
    \item история эволюции Технологии OSTIS;
    \item описание правил использования Технологии OSTIS;
    \item описание организационной инфраструктуры, направленной на развитие Технологии OSTIS;
    \item библиотека многократно используемых и семантически совместимых компонентов ostis-систем;
    \item методы и инструментальные средства проектирования различного вида компонентов ostis-систем;
    \item технические средства координации деятельности участников проекта, направленные на постоянное совершенствование Технологии OSTIS.
\end{scnitemize}
}

\scnheader{Проект IMS.ostis}
\scntext{подзадачи}{
\begin{scnitemize}
    \item Разработать \textit{Метасистему IMS.ostis}, обеспечивающую быстрое компонентное проектирование семантически совместимых компьютерных систем различного назначения.
    \item Разработать методы и средства, обеспечивающие интенсивное развитие рынка семантически совместимых прикладных интеллектуальных систем, созданных на основе \textit{Метасистемы IMS.ostis}.
    \item Разработать методы и средства, обеспечивающие стимулирование интенсивного развития самой \textit{Метасистемы IMS.ostis}.
\end{scnitemize}
}

\scnheader{Метасистема IMS.ostis}
\scntext{новизна}{Новизна \textit{Метасистемы IMS.ostis} заключается в унификации представления различного вида информации в памяти компьютерных систем на основе смыслового (семантического) представления этой информации, что обеспечивает:
\begin{scnitemize}
    \itemустранение дублирования одной и той же информации в разных интеллектуальных системах и в разных компонентах одной и той же системы;
    \item семантическую совместимость различных компонентов интеллектуальных систем и различных интеллектуальных систем в целом;
    \item существенное расширение библиотек совместимых многократно используемых компонентов компьютерных систем за счет "крупных"\ компонентов и, в частности, типовых подсистем.
\end{scnitemize}
}

\scnheader{Метасистема IMS.ostis}
\scntext{принципы реализации}{Принципы технической реализации \textit{Метасистемы IMS.ostis} полностью совпадают с принципами технической реализации прикладных интеллектуальных систем, разрабатываемых с помощью этой метасистемы.}
\scnidtf{интеллектуальная система, предназначенная для комплексной информационной и инструментальной поддержки проектирования семантически совместимых компьютерных систем, на назначение которых не накладывается никаких ограничений}

\scnheader{База знаний Метасистемы IMS.ostis}
\scntext{декомпозиция}{
\begin{scnitemize}
    \item текущее состояние моделей и методов, используемых при разработке интеллектуальных систем с помощью \textit{Метасистемы IMS.ostis};
    \item систематизированную библиотеку многократно используемых и совместимых компонентов интеллектуальных систем;
    \item описание инструментальных средств проектирования различного вида компонентов интеллектуальных систем (фрагментов баз знаний, решателей задач, пользовательских интерфейсов);
    \item описание средств координации коллективной деятельности, направленной на постоянное развитие \textit{Метасистемы IMS.ostis};
    \item описание истории эволюции \textit{Метасистемы IMS.ostis};
    \item описание средств проектирования различных классов интеллектуальных систем.
\end{scnitemize}
}

\scnheader{Проект IMS.ostis}
\scntext{принципы организации}{Организация \textit{Проекта IMS.ostis} реализуется в форме взаимодействия \textit{Метасистемы IMS.ostis} с его пользователями и основана на следующих принципах:
\begin{scnitemize}
    \item Решатель задач и пользовательский интерфейс \textit{Метасистемы IMS.ostis} обеспечивают поддержку всего комплекса проектных задач, решаемых разработчиками прикладных интеллектуальных систем, а также разработчиками самой \textit{Метасистемы IMS.ostis}.
    \item Для стимулирования развития рынка совместимых прикладных интеллектуальных систем, разработанных с помощью \textit{Метасистемы IMS.ostis} и развития самой этой метасистемы используются технические средства анализа и оценки объекта и значимости персонального вклада каждого разработчика в специальных условных единицах.
    \item Для стимулирования развития рынка совместимых прикладных интеллектуальных систем, разработанных с помощью \textit{Метасистемы IMS.ostis}, за каждую такую интеллектуальную систему, зарегистрированную и специфицированную в рамках \textit{Метасистемы IMS.ostis}, разработчикам выделяется вознаграждение в используемых условных единицах после того, как эта прикладная система будет протестирована на предмет семантической совместимости с другими системами, разработанными с помощью \textit{Метасистемы IMS.ostis}. При этом \textit{Метасистемы IMS.ostis} становится площадкой для рекламы и распространения интеллектуальных систем, разработанных с его помощью.
    \item Стимулирование развития самой \textit{Метасистемы IMS.ostis} осуществляется следующим образом. Участие в развитии \textit{Метасистемы IMS.ostis} носит открытый характер, для чего достаточно соответствующим образом зарегистрироваться. Авторские права каждого разработчика \textit{Метасистемы IMS.ostis} защищаются и каждый его вклад в зависимости от его ценности автоматически измеряется и фиксируется в используемых условных единицах.
    \item Участие в развитии \textit{Метасистемы IMS.ostis} может иметь самые различные формы (в простейшем случае, это может быть указание на конкретные ошибки, на конкретные трудности, с которыми пользователь столкнулся, формулировка конкретных пожеланий; более сложным вкладом является добавление в базу знаний метасистемы новых знаний, новых компонентов в библиотеку многократно используемых компонентов). При этом автор нового многократно используемого компонента, включенного в библиотеку \textit{Метасистемы IMS.ostis}, может выбрать любую лицензию для его распространения и, в том числе, назначить ему любую цену.
    \item Использование \textit{Метасистемы IMS.ostis} зарегистрированными пользователями  для ознакомления с ним носит бесплатный открытый характер. При коммерческой разработке прикладных интеллектуальных систем стоимость каждого обращения к библиотекам \textit{Метасистемы IMS.ostis} вполне доступна, но существенно снижается в зависимости от степени активности пользователя в развитии \textit{Метасистемы IMS.ostis}. Это еще один механизм стимулирования участия в развитии \textit{Метасистемы IMS.ostis}.
\end{scnitemize}

Таким образом, указанные принципы организации \textit{Метасистемы IMS.ostis} обеспечивают на постоянной основе привлечение к разработке \textit{Метасистемы IMS.ostis} и к формированию рынка семантически совместимых прикладных интеллектуальных систем неограниченные научные, технические и финансовые ресурсы и, в частности, привлечение любых специалистов, желающих участвовать в этом открытом проекте.
}

\scnendstruct

\end{SCn}

\scsubsection{Предметная область и онтология семантически совместимых информационно-справочных ostis-систем и интеллектуальных help-систем}
\label{sd_help_semantic_comp_sys}

\scsubsection[\scnmonographychapter{Глава 7.2. Экосистема интеллектуальных компьютерных систем нового поколения (Экосистема OSTIS) и реализация рынка знаний на ее основе}]{Предметная область и онтология семантически совместимых интеллектуальных корпоративных ostis-систем различного назначения}
\label{sd_purpos_semantic_comp_sys}
\begin{SCn}
    \scnsectionheader{Предметная область и онтология семантически совместимых интеллектуальных корпоративных ostis-систем различного назначения}
    \begin{scnsubstruct}

        \begin{scnrelfromlist}{библиографическая ссылка}
            \scnitem{\scncite{Ameri2005}}
            \scnitem{\scncite{Gerhard2017}}
        \end{scnrelfromlist}

        \scnheader{Предметная область и онтология семантически совместимых интеллектуальный корпоративных ostis-систем различного назначения}
        \scniselement{предметная область}
        \begin{scnhaselementrole}{максимальный класс объектов исследования}
            {интеллектуальная комната данных}
        \end{scnhaselementrole}
        \begin{scnhaselementrolelist}{класс объектов исследования}
            \scnitem{корпоративная система}
            \scnitem{корпоративная ostis-система}
        \end{scnhaselementrolelist}

        \scnheader{интеллектуальная комната данных}
        \scnidtf{Intelligent Data Room}
        \scnidtf{IDR}
        \scnidtf{система, которая позволяет отслеживать, анализировать и постепенно автоматизировать все процессы обработки данных в компании}
        \begin{scnrelfromset}{принципы работы}
            \scnfileitem{Интеллектуальные подсистемы (агенты) упорядочивают структуру ваших данных таким образом, что актуальная информация всегда доступна, а устаревшая информация автоматически архивируется или удаляется в соответствии с законами о хранении и защите данных.}
            \scnfileitem{Запросы к системе выполняются в виде простых инструкций, система помогает менеджерам вводить необходимую информацию, осуществляет частичную или полную автоматизацию обновления информации из баз данных, доступных через Интернет.}
            \scnfileitem{Искусственный интеллект (ИИ) выполняет структуризацию и классификацию документов и информации для принятия быстрых и правильных решений, автоматически обрабатывает документы и доступные базы данных для отбора ключевой информации, необходимой в данный момент и в будущем.}
            \scnfileitem{Существующее системное окружение на предприятии может быть легко подключено к искусственному интеллекту через открытые интерфейсы, вся информация остается доступной. Все ключевые системы данных синхронизируются с искусственным интеллектом, данные постоянно сравниваются друг с другом, чтобы избежать потерь.}
            \scnfileitem{Вся информация доступна в базе знаний, которая является источником данных для рабочих процессов, отчетности и комплексных проверок.}
            \scnfileitem{Таким образом, предлагаемая платформа на основе искусственного интеллекта позволяет представить всю компанию единым целостным образом.}
        \end{scnrelfromset}
        \begin{scnrelfromset}{достоинства внедрения}
            \scnfileitem{\textit{IDR} помогает собирать и оценивать информацию без преднамеренных искажений или ошибок, связанных с человеческим фактором.}
            \scnfileitem{Компания с \textit{IDR} полностью контролирует свои данные.}
            \scnfileitem{Система предоставляет только высококачественные, достоверные и актуальные данные.}
            \scnfileitem{Цифровое представление всех процессов компании обеспечивает интегрированную обработку информации внутри компании, что дает полную прозрачность управления, облегчает доступ ко всей информации и ее анализ.}
            \scnfileitem{Благодаря поддержке подсистем искусственного интеллекта все необходимые данные из документов, процессов и внешних источников могут быть извлечены, структурированы и грамотно оценены.}
            \scnfileitem{Искусственный интеллект предоставляет инструмент для интеллектуальной оцифровки и интеграции знаний вашей компании и взаимодействия между всеми заинтересованными сторонами в рамках вашей компании, как следствие --- обеспечивает автоматическую поддержку соответствующих бизнес-процессов и устраняет локальные изолированные решения внутри компании, превращая ее в единую согласованную систему.}
        \end{scnrelfromset}
        \scntext{примечание}{Идея \textit{интеллектуальной комнаты данных} в общем случае может реализовываться двумя путями. Любой из них может реализовываться постепенно, с подключением к \textit{IDR} все новых и новых информационных ресурсов.}
        \scnsuperset{цифровой сотрудник}
        \scntext{пояснение}{Надстройка над уже существующими информационными ресурсами предприятия (различные базы данных, облачные и физические хранилища документов и т.д.). Для реализации \textit{цифрового сотрудника} необходимо в базе знаний \textit{IDR} в виде семейства онтологий описать метаинформацию об имеющихся информационных ресурсах (схемы баз данных, структуру и расположение документов и т.д.), а также механизмы доступа к этим ресурсам (например, их физическое расположение, языки запросов). На основе такого описания \textit{IDR} сможет автоматически построить необходимый набор запросов к нужным информационным ресурсам, интегрировать полученные ответы и выдать ответ пользователю в удобной ему форме. При этом пользователю системы не нужно знать, где именно и в какой форме хранится нужная ему информация, запрос к системе делается на языке, близком к естественному.}
        \begin{scnrelfromset}{достоинства}
            \scnfileitem{Нет необходимости в рамках базы знаний \textit{IDR} дублировать информацию, которая уже содержится в использовавшихся ранее информационных ресурсах, она может и дальше храниться в тех же местах и в той же форме. При этом данные могут быстро меняться, это никак не повлияет на работу системы.}
            \scnfileitem{Нет необходимости вносить изменения в уже налаженные процессы и используемое на предприятии ПО, резко переобучать людей и менять отлаженные схемы работы.}
            \scnfileitem{В общем случае реализация \textit{цифрового сотрудника } проще и дешевле в реализации и позволяет экспериментальным путем выявить наиболее больные места предприятия, где автоматизация информационных процессов и обеспечение их прозрачности позволит сэкономить наибольшее количество средств и минимизировать число ошибок.}
        \end{scnrelfromset}
        \begin{scnrelfromset}{недостатки}
            \scnfileitem{\textit{Цифровой сотрудник} не решает проблемы, связанные с дублированием информации, представленной в разной форме в разных местах, и только частично решает проблемы, связанные с ошибками при внесении новой или редактированием имеющейся информации в разнородные информационные ресурсы.}
            \scnfileitem{Из-за отсутствия унификации представления информации система \textit{IDR} ограничена в своих возможностях, в частности при верификации информации. Проверить корректность, непротиворечивость, полноту информации намного проще, если вся информация представлена в унифицированном виде (как по форме, так и по смыслу).}
        \end{scnrelfromset}
        \scnsuperset{полноценная комната данных}
        \scnidtf{полный цифровой двойник предприятия}
        \scntext{пояснение}{Реализация \textit{полноценной комнаты данных} предполагает полный перенос в базу знаний \textit{IDR} части или всей информации, хранящейся в электронном виде в информационных ресурсах предприятия. Для этого необходимо описывать в базе знаний как онтологии (системы понятий) предметной области предприятия, так и конкретные экземпляры и связи между ними. При этом очевидно, что если онтологии изменяются относительно редко и могут обновляться вручную, то конкретные экземпляры и их описание должно формироваться автоматически.}
        \begin{scnrelfromset}{достоинства}
            \scnfileitem{Полностью исключается необходимость дублирования информации в разных местах и в разных формах, таким образом снижается объем хранимой информации и минимизируется количество ошибок.}
            \scnfileitem{В онтологиях описывается только смысл информации, нет необходимости описывать существующую структуру информационных ресурсов предприятия и ориентироваться на нее.}
            \scnfileitem{Полностью задействуются возможности \textit{IDR} по верификации информации предприятия на предмет корректности, непротиворечивости и полноты.}
            \scnfileitem{Наращивание функциональных возможностей такой реализации значительно упрощается за счет гибкости подходов, лежащих в основе \textit{IDR}, в частности, подхода к разработке и структуризации базы знаний и многоагентного подхода к обработке информации.}
        \end{scnrelfromset}
        \begin{scnrelfromset}{недостатки}
            \scnfileitem{Требуются изменения в уже налаженных процессах на предприятии, отказ от части используемого в настоящий момент ПО, переобучение сотрудников, обслуживающих и развивающих информационные системы предприятия.}
            \scnfileitem{В общем случае \textit{полноценная комната данных} может оказаться более трудоемкой при первоначальной реализации, поскольку предполагает больший объем работ по формализации информации, обеспечению надежности ее хранения и доступа к ней, эффективности ее редактирования и дополнения.}
        \end{scnrelfromset}
        \scntext{примечание}{Оба варианта реализации интеллектуальной комнаты данных (\textit{полноценная комната данных} и \textit{цифровой сотрудник}) дают возможность описать в базе знаний \textit{IDR} не только информацию, которая уже активно используется на предприятии, но и дополнительные знания, которые могут оказаться полезными, например:
            \begin{itemize}
                \item Различные стандарты и документы, регламентирующие работу предприятия;
                \item Описание структуры предприятия, правил работы, основных контактов;
                \item Учебные материалы и руководства для новых сотрудников;
                \item и многое другое
            \end{itemize}}
    
    \scnheader{корпоративная система}
    \scntext{пояснение}{\textit{корпоративные системы} представляют собой программные решения, предназначенные для автоматизации бизнес-процессов и управления ресурсами и данными внутри организации. Они могут включать в себя различные подсистемы, такие как управление отношениями с клиентами, управление контентом, управление проектами, управление ресурсами предприятия, управление документами и многое другое.}
    \scntext{роль}{Обеспечение эффективного управления бизнес-процессами и ресурсами, повышении производительности и качества работы, а также обеспечении прозрачности и оперативности принятия решений на основе актуальных данных.}
    \begin{scnindent}
        \begin{scnrelfromset}{смотрите}
            \scnitem{\scncite{Ameri2005}}
            \scnitem{\scncite{Gerhard2017}}
        \end{scnrelfromset}
    \end{scnindent}
    \begin{scnrelfromset}{цель}
        \scnfileitem{Автоматизация многих рутинных задач, таких как обработка заказов, управление складом, учет финансовых операций и так далее. Это позволяет сократить время на выполнение задач и уменьшить количество ошибок.}
        \scnfileitem{Сбор, хранение и обработка данных о бизнес-процессах и ресурсах организации. Это позволяет увеличить точность и оперативность принятия решений, а также обеспечить прозрачность в управлении организацией.}
        \scnfileitem{Эффективное управление ресурсами организации, такими как финансы, трудовые ресурсы, материальные и технические ресурсы и так далее. Это позволяет сократить затраты на управление ресурсами и повысить эффективность их использования.}
        \scnfileitem{Управление отношениями с клиентами, автоматизация процессов продаж и обслуживания, а также анализ данных о клиентах. Это позволяет повысить удовлетворенность клиентов и увеличить объемы продаж.}
        \scnfileitem{Управление проектами, планирование и отслеживание выполнения работ, управление ресурсами и расписание проектов. Это позволяет повысить эффективность выполнения проектов, уменьшить сроки выполнения работ и снизить затраты на проекты.}
        \scnfileitem{Управление документами, контроль версиями, автоматизация процессов редактирования и утверждения документов. Это позволяет повысить эффективность работы с документами и обеспечить безопасность их хранения и передачи.}
    \end{scnrelfromset}
    \begin{scnrelfromset}{недостатки}
        \scnfileitem{Внедрение корпоративных систем может быть дорогостоящим и трудоемким процессом, который требует значительных ресурсов и экспертизы. Кроме того, многие системы могут потребовать изменения бизнес-процессов и требовать адаптации культуры организации.}
        \scnfileitem{Корпоративные системы могут столкнуться с проблемами совместимости с другими системами, используемыми в организации. Это может привести к проблемам с обменом данными и снижению эффективности работы.}
        \scnfileitem{Корпоративные системы могут стать мишенью для кибератак, поэтому важно обеспечить безопасность хранения и передачи данных, используемых в системах.}
        \scnfileitem{Корпоративные системы могут потребовать значительных затрат на обслуживание и поддержку, включая установку обновлений, устранение ошибок и техническую поддержку.}
        \scnfileitem{Внедрение новых корпоративных систем может потребовать обучения персонала, что может быть трудоемким и затратным процессом.}
        \scnfileitem{Внедрение корпоративных систем может потребовать изменения бизнес-процессов, что может быть сложным и вызвать сопротивление со стороны сотрудников.}
    \end{scnrelfromset}
    \scntext{примечание}{С точки зрения структуры Экосистемы OSTIS \textit{корпоративная ostis-система} осуществляет координации и эволюцию деятельности некоторых групп ostis-систем и их пользователей.}
    \scntext{примечание}{Для создания семантически совместимых интеллектуальных \textit{корпоративных систем} необходимо обеспечить высокую степень гибкости, масштабируемости, автоматизации и интеграции. Это позволит организациям более эффективно управлять ресурсами и данными и повысить их конкурентоспособность на рынке. Для достижения этих целей необходимо использовать современные технологии, такие как Аналитика данных, Машинное обучение, Искусственный интеллект и технологии распределенных вычислений. Кроме того, необходимо учитывать особенности организации и ее бизнес-процессов, чтобы обеспечить максимальную эффективность использования системы.}
    \scntext{примечание}{\textit{корпоративные ostis-системы} могут быть применены в различных областях: медицина и здравоохранение, образовательная деятельность широкого профиля, страховой бизнес, промышленная деятельность, административная деятельность, недвижимость, транспорт и так далее.}
    \scntext{определение}{\textit{корпоративная ostis-система} --- центральная \textit{ostis-система}, осуществляющая координацию, организацию, а также поддержку эволюции деятельности членов соответствующего \textit{ostis-сообщества}. \textit{корпоративная ostis-система} является представителем соответствующего \textit{ostis-сообщества} в других \textit{ostis-сообществах}, членом которых оно является.}
    \begin{scnindent}
        \scnrelfrom{пример}{SCg-текст. Пример корпоративной ostis-системы ostis-сообщества}
        \begin{scnindent}
            \scnidtf{SCg-текст. Пример корпоративной ostis-системы ostis-сообщества}
        \end{scnindent}
    \end{scnindent}

    \end{scnsubstruct}
    \scnendcurrentsectioncomment
\end{SCn}


\scsubsubsection{Предметная область и онтология организаций}
\label{sd_organiztion}

\scsubsection[\scnmonographychapter{Глава 7.2. Экосистема интеллектуальных компьютерных систем нового поколения (Экосистема OSTIS) и реализация рынка знаний на ее основе}]{Предметная область и онтология ostis-систем, являющихся персональными ассистентами пользователей, обеспечивающими организацию эффективного взаимодействия каждого пользователя с другими ostis-системами и пользователями, входящими в состав Экосистемы OSTIS}
\label{sd_assistants}
\begin{SCn}

\scnsectionheader{\currentname}

\scnstartsubstruct

\scnheader{Предметная область ostis-систем, являющихся персональными ассистентами пользователей в рамках Экосистемы OSTIS}
\scniselement{предметная область}
\scnsdmainclasssingle{персональный ostis-ассистент}
%\scnsdclass{***}
%\scnsdrelation{***}

\scnheader{персональный ostis-ассистент}
\scnidtf{ostis-система, являющаяся персональным ассистентом пользователя в рамках Экосистемы OSTIS}
\scnrelfromset{возможности}{\scnfileitem{Возможность анализа деятельности пользователя и формирования рекомендаций по ее оптимизации.};
\scnfileitem{Возможность адаптации под настроение пользователя, его личностные качества, общую окружающую обстановку, задачи, которые чаще всего решает пользователь.};
\scnfileitem{Перманентное обучение самого ассистента в процессе решения новых задач, при этом обучаемость потенциально не ограничена.};
\scnfileitem{Возможность вести диалог с пользователем на естественном языке, в том числе в речевой форме.};
\scnfileitem{Возможность отвечать на вопросы различных классов, при этом если системе что-то не понятно, то она сама может задавать встречные вопросы.};
\scnfileitem{Возможность автономного получения информации от всей окружающей среды, а не только от пользователя (в текстовой или речевой форме). При этом система может как анализировать доступные информационные источники (например, в интернете), так и анализировать окружающий ее физический мир, например, окружающие предметы или внешний вид пользователя.}}
\scnrelfromset{достоинства}{\scnfileitem{Пользователю нет необходимости хранить разную информацию в разной форме в разных местах, вся информация хранится в единой базе знаний компактно и без дублирований.};
\scnfileitem{Благодаря неограниченной обучаемости ассистенты могут потенциально автоматизировать практически любую деятельность, а не только самую рутинную.};
\scnfileitem{Благодаря базе знаний, ее структуризации и средствам поиска информации в базе знаний пользователь может получить более точную информацию более быстро.}}
\scnsuperset{персональный ostis-ассистент учебного назначения}
\scnaddlevel{1}
	\scnidtf{персональный ассистент-учитель}
\scnaddlevel{-1}
\scnsuperset{персональный ostis-ассистент по здоровому образу жизни и здоровому питанию}
\scnaddlevel{1}
	\scnidtf{персональный фитнесс-тренер}
\scnaddlevel{-1}
\scnsuperset{персональный ostis-ассистент для ухода за пациентом}
\scnsuperset{персональный секретарь-референт}

\bigskip
\scnendstruct \scnendcurrentsectioncomment

\end{SCn}

\scsubsubsection{Предметная область и онтология персон}
\label{sd_person}

\scsubsection[\scneditor{Гулякина Н.А.}\protect\scnmonographychapter{Глава 7.4. Интеллектуальные обучающие системы нового поколения}]{Предметная область и онтология семантически совместимых ostis-систем автоматизации образовательной деятельности}
\label{sd_learning}
\begin{SCn}
\bigskip
\scnsectionheader{\currentname}

\scnstartsubstruct

\scnheader{Предметная область и онтология методов и средств реализации целенаправленного и персонифицированного процесса обучения пользователей для каждой ostis-системы, входящей в состав Экосистемы OSTIS}
\scnsdmainclasssingle{***}
\scnsdclass{***}
\scnsdrelation{***}

\scnheader{обучаемость}
\scnidtf{способность системы приобретать новые знания и навыки}
\scnrelfrom{включение}{неограниченная обучаемость}

\scnheader{неограниченная обучаемость}
\scnidtf{степень обучаемости, при которой не
накладывается никаких ограничений на типологию приобретенных знаний и навыков}
\scnexplanation{Говоря другими словами, система,
обладающая неограниченной обучаемостью, при необходимости может с течением времени приобрести любое знание и способность решать любую задачу}
	\scnaddlevel{1}
	\scnnote{Уточним, что это не означает, что одна конкретная система будет уметь решать любую задачу, это означает, что система может приобрести способность решать нужную ей задачу, при этом нет принципиальных ограничений на класс таких задач.}
	\scnaddlevel{-1}
	
\scnheader{интеллектуальная компьютерная система}
\scnidtf{сложная техническая система, разработка и даже использование которой требует высоких профессиональных качеств}
\scnrelfromset{проблемы текущего состояния}{\scnfileitem{недостаточная эффективность использования современных интеллектуальных систем, трудоемкость их внедрения и сопровождения, которые в значительной мере определяются высоким порогом вхождения конечных пользователей в интеллектуальные системы}
;\scnfileitem{пользователь часто не использует значительную часть функций даже традиционных компьютерных
систем просто по той причине, что не знает об их наличии и не имеет простого механизма, позволяющего о них узнать. Для интеллектуальных систем данная проблема стоит еще более остро}
;\scnfileitem{высоки затраты на обучение разработчиков интеллектуальных систем, на их адаптацию под особенности устройства конкретной интеллектуальной системы}}
	\scnaddlevel{1}
	\scnnote{Перечисленные трудности связаны не только с естественной сложностью интеллектуальных компьютерных
систем по сравнению с традиционными компьютерными системами, но с низким уровнем документации
для таких систем, неудобством использования такой
документации, трудоемкостью локализации средств
и области решения той или иной задачи, как для
конечного пользователя, так и для разработчика.}
	\scnaddlevel{-1}
\scnrelfrom{предлагаемый подход}{Подход к решению проблемы обучения конечных пользователей и разработчиков интеллектуальных систем}

\scnheader{Подход к решению проблемы обучения конечных пользователей и разработчиков интеллектуальных систем}
\scnidtf{подход к решению указанных проблем, предполагающий дополнение каждой интеллектуальной системы модулем, представляющим собой интеллектуальную обучающую подсистему}
	\scnaddlevel{1}
	\scnnote{Целью данной подсистемы является обучение конечного пользователя и разработчика основной системы принципам работы с ней, принципам ее функционирования и развития.}
	\scnaddlevel{-1}
\scntext{основная идея}{Независимо от того, для решения каких задач разрабатывается интеллектуальная система, она должна обладать некоторыми функциями обучающей системы, даже если система изначально не является обучающей.}
	\scnaddlevel{1}
	\scnexplanation{Следовательно,
	\begin{scnitemize}
	\item пользователь должен иметь возможность
обучаться как принципам работы с интеллектуальной системой, так и иметь возможность получать
новые знания о той предметной области, для которой создается интеллектуальная система;
	\item разработчик интеллектуальных систем должен иметь возможность обучаться принципам внутреннего устройства системы, принципам ее функционирования,
назначению конкретных компонентов системы, иметь возможность локализовать ту часть системы, в которой он должен разобраться для внесения изменений в функциональные возможности системы.	
	\end{scnitemize}}
	\scnnote{Для реализации данной идеи интеллектуальная система должна содержать не только знания о той предметной области, для которой она разработана, но и:
\begin{scnitemize}
\item знания о самой себе, своей архитектуре, компонентах, функциях, принципах работы и т.д.;
\item знания о пользователе, его опыте, навыках, предпочтениях, интересах;
\item знания о задачах, которые решает сама система в
текущий момент и задачах которые планируются к решению в будущем;
\item знания об актуальных задачах по развитию системы и ее сопровождению. 
\end{scnitemize}}
	\scnaddlevel{-1}
\scnrelfrom{технологическая основа}{модель представления знаний в виде унифицированных семантических сетей с теоретико-множественной интерпретацией}
	\scnaddlevel{1}
	\scnrelfromset{возможности}{
	\scnfileitem{Указанная модель является универсальной, то есть позволяет представлять в виде однородных семантических сетей знания любого рода, в том числе конкретные факты, логические утверждения (аксиомы, теоремы, определения), текстовые и мультимедийные иллюстрации и комментарии, примеры конкретных задач с решениями, в том числе доказательства и т.д.}
	;\scnfileitem{Подобная модель представления знаний позволяет рассматривать базу знаний любой системы как иерархию предметных областей, то есть позволяет произвести семантическую структуризацию предлагаемого учащемуся материала, что существенно облегчает процесс обучения за счет систематизации знаний на основе именно их семантики, а не каких-либо других сторонних факторов. Кроме этого, знания в базе могут делиться на логические разделы, каждый из которых соответствует какому-либо фрагменту излагаемого материала. Представление знаний в виде семантической сети позволяет осуществлять свободную навигацию по любым ассоциативным связям, изучая таким образом материал в той последовательности, какая кажется более логичной для самого обучаемого. С другой стороны, такой подход позволяет указать рекомендуемую последовательность изучения материала. При необходимости структура предметных областей может быть легко перестроена.}
	;\scnfileitem{Модель представления является унифицированной, то есть знания из различных областей представляются в сходном виде, что позволяет говорить не о семействе не связанных между собой обучающих систем по различным предметным областям, а о глобальном смысловом пространстве, объединяющем в себе знания всего семейства разрабатываемых систем. В свою очередь, наличие такого смыслового пространства обеспечивает ряд дополнительных возможностей:
	\begin{scnitemize}
	\item каждая система при необходимости может использовать знания, относящиеся к другим системам, что позволяется задавать не только вопросы, касающиеся конкретной предметной области, но и вопросы, носящие междисциплинарный характер;
	\item в рамках глобального смыслового пространства можно выделить часть знаний, которые имеют отношение ко многим системам из всего комплекса, например базовые знания из области математики, логики и т.д. Концепция глобального смыслового пространства позволяет записывать такие фрагменты знаний только в одной из систем, а затем использовать их во всех остальных, что существенно уменьшает количество дублирований, сокращает сроки разработки систем и снижает накладные расходы.
	\end{scnitemize}}
	;\scnfileitem{Рассматриваемый подход к представлению знаний позволяет унифицировать не только модель представления знаний, но и модели обработки знаний, в том числе модели информационного поиска и решения задач. Данный факт позволяется говорить о возможности реализации универсального набора поисковых операций, а также о реализации универсального решателя задач, позволяющего решать различные задачи в рамках каждой из рассматриваемых предметных областей, что существенно сокращает количество реализуемых операций обработки знаний при сохранении функциональных возможностей каждой системы.}
	;\scnfileitem{Как уже было сказано выше, унифицированная модель представления знаний позволяет не только вносить в каждую систему примеры условий типовых для заданной предметной области задач с решениями, но и говорить об интеллектуальном решателе, который позволит системе генерировать ответы на поставленный вопрос за счет знаний уже имеющихся в базе знаний, например с использованием правил логического вывода.}
	;\scnfileitem{Унифицированное представление знаний позволяет не ограничивать номенклатуру пользовательских запросов только специально выделенными для этого командами, а задавать произвольный запрос системе с использованием универсального языка отображения знаний, что делает перечень возможных запросов зависящим только от количества и разнообразия знаний, внесенных в базу знаний системы.}
	;\scnfileitem{Унификация моделей пользовательских интерфейсов позволяет отображать знания различного рода в унифицированном виде независимо от предметной области, к которой эти знания относятся. Таким образом, все разрабатываемые системы будут обладать пользовательским интерфейсом, построенным по одним и тем же принципам, что позволит существенно сократить срок ознакомления учащегося со всем семейством систем. Данный факт не отрицает возможность и необходимость разработки отдельных компонентов интерфейса, ориентированных на конкретную предметную область, например, редактора геометрических чертежей, виртуальной лаборатории для проведения химических опытов и т.д.}
	;\scnfileitem{Каждый компонент пользовательского интерфейса также является отображением определенного элемента из базы знаний, что позволяет, во-первых, легко менять интерфейс системы даже во время ее работы, а, во-вторых, позволяет пользователю задавать системе вопросы не только касательно предметной области, которой посвящена данная система, но и касательно любого из компонентов интерфейса и других частей системы. Таким образом, пользователю достаточно научиться задавать системе несколько простейших вопросов, чтобы в дальнейшем изучить все тонкости работы системой уже в процессе общения с ней.}
	;\scnfileitem{Предлагаемые модели представления и обработки знаний позволяют физически отделить смысл хранимой информации от вариантов ее внешнего отображения, в частности, от идентификаторов тех или иных сущностей в рамках какого-либо естественного языка. Это дает возможность легко интернационализировать любую из разработанных систем, поскольку для перевода системы на какой-либо другой язык необходимо перевести только фрагменты текстов на естественном языке, явно хранимые в базе знаний, не затрагивая при этом сами семантические связи, то есть смысл представленной информации.}}
	\scnaddlevel{-1}
\scnrelfrom{модель обработки знаний}{модель, основанная на многоагентном подходе}
	\scnaddlevel{1}
	\scnrelfromset{преимущества}{
	\scnfileitem{Работа агентов осуществляется независимо друг от друга, что позволяет легко расширять функционал той или иной системы при необходимости, а также позволяет интегрировать в одной и той системе различные модели информационного поиска, решения задач и т.д.}
	;\scnfileitem{Работа агентов осуществляется параллельно, что существенно улучшает производительность всей системы в целом.}}
	\scnaddlevel{-1}
\scnrelfrom{предлагаемый фундамент}{\scnkeyword{Технология OSTIS}}
	\scnaddlevel{1}
	\scnidtf{открытая семантическая технология проектирования интеллектуальных систем}
	\scnnote{\textit{Технология OSTIS} позволяет интегрировать любые виды знаний и любые модели решения задач}
	\scnrelfromset{преимущества}{
\scnfileitem{В основе технологии лежит \textit{SC-код} -- универсальный и унифицированный язык кодирования информации в графодинамической памяти компьютерных систем. \textit{SC-код} позволяет в унифицированном (одинаковом) виде представить любую информацию, что позволит сделать предлагаемый подход универсальным и подходящим для любого класса интеллектуальных систем}
;\scnfileitem{\textit{Технология OSTIS} и, в частности, \textit{SC-код}, легко интегрируется с любыми современными технологиями, что позволит применить предлагаемый подход для большого числа уже разработанных интеллектуальных систем}
;\scnfileitem{\textit{SC-код} позволяет хранить и описывать в базе знаний ostis-системы любую внешнюю (инородную)
по отношению к \textit{SC-коду} информацию в виде внутренних файлов \textit{ostis-систем}. Таким образом, база знаний обучающей подсистемы может содержать в явном виде фрагменты уже имеющейся документации к системе, представленной в любой форме}
;\scnfileitem{В рамках \textit{Технологии OSTIS} уже разработаны модели баз знаний \textit{ostis-систем}, решателей задач \textit{ostis-систем} и пользовательских интерфейсов \textit{ostis-систем}, предполагающие полное их описание в базе знаний системы. Таким образом для \textit{ostis-систем} предлагаемый подход реализуется значительно проще и дает дополнительные преимущества, подробнее рассмотренные
в работе}
;\scnfileitem{одним из основных принципов \textit{Технологии OSTIS} является обеспечение гибкости (модифицируемости) систем, разрабатываемых на ее основе. Таким образом, использование \textit{Технологии OSTIS} обеспечит возможность эволюции самой интеллектуальной обучающей подсистемы}}
	\scnaddlevel{-1}

\scnheader{интеллектуальная обучающая система}
\scnsuperset{интеллектуальная система}
\scnnote{Такого рода системы по сравнению с традиционными системами электронного обучения (например, электронными учебниками) предоставляют обладают рядом существенных преимуществ.}
\scnnote{В случае реализации \textit{интеллектуальной обучающей системы} на основе \textit{Технологии OSTIS}, появляются дополнительные возможности, к числу которых можно отнести следующие:
	\begin{scnitemize}
	\item пользователю в явном виде представляется семантическая структура изучаемого учебного материала и изучаемой предметной области. При этом
обеспечивается наглядная визуализация любого
уровня указанной семантической структуры;
	\item пользователю становятся доступны достаточно полные сведения об изучаемой предметной области, отражены все ее аспекты, благодаря явному
помещению в базу знаний всех предметных закономерностей и взаимосвязей понятий;
	\item помимо возможности чтения текстов и иллюстративных материалов учебника предоставляется возможность навигации по семантическому пространству предметной области;
	\item пользователю предоставляется возможность задавать системе любые вопросы и задачи по изучаемой предметной области;
	\item пользователю предоставляется возможность под контролем системы тренироваться (приобретать практические навыки) в решении самых различных
задач по изучаемой предметной области.
 При этом система:
	\begin{scnitemizeii}
	\item осуществляет семантический анализ правильности решения задач как по свободно конструируемым ответам (результатам), так и по протоколам решения;
	\item локализует допущенные пользователем ошибки
в решении задач, определяет их причину и выдает соответствующие рекомендации пользователю.
	\end{scnitemizeii}
	\item при общении с системой пользователю предоставляется свобода в выборе любого из множества синонимичных терминов (идентификаторов), зарегистрированных в базе знаний системы;
	\item появляется принципиальная возможность реализации естественно-языкового интерфейса с пользователем (благодаря широким возможностям семантического анализа пользовательских сообщений и возможностям синтеза на семантическом уровне сообщений, адресуемых пользователям);
	\item пользователю предоставляется полная свобода в выборе последовательности изучения учебного материала (маршрута навигации по учебному материалу), в выборе решаемых им задач (в сборнике задач и лабораторных работ), но соответствующие рекомендации выдаются.
	\end{scnitemize}}
	\scnaddlevel{1}
	\scnnote{Часть из перечисленных возможностей (а в предельном случае и все их них) могут быть реализованы в рамках подсистемы обучения пользователей интеллектуальной системы.}
	\scnaddlevel{-1}


\scnheader{подсистема обучения пользователей интеллектуальных систем}
\scnexplanation{Для реализации взаимодействия \textit{подсистемы обучения пользователей интеллектуальных систем}, реализуемой на основе \textit{Технологии OSTIS} с основной \textit{интеллектуальной системой} предполагается разработка интерфейсного компонента, который также является частью подсистемы. Важно отметить, что для разных \textit{интеллектуальных систем} такие компоненты будут в значительной степени пересекаться, что, в свою очередь, позволит снизить затраты на интеграцию подсистемы обучения и основной \textit{интеллектуальной системы}}
\scnrelfrom{иллюстрация}{\scnfileimage{\includegraphics[width=0.7\linewidth]{figures/sd_learning/system_arch.png}}}

\scnnote{В случае, если рассматриваемая интеллектуальная система является \textit{ostis-системой}, ее интеграция с подсистемой обучения пользователей интеллектуальных систем осуществляется более глубоко и архитектуру полученной интегрированной системы можно изобразить следующим образом (см. Рис.\textit{ Архитектура подсистемы обучения пользователей интеллектуальных систем в составе другой ostis-системы}). Как видно из рисунка, компоненты \textit{подсистемы обучения пользователей интеллектуальных систем} просто дополняют уже существующие в основной ostis-системе компоненты, что позволяет максимально снизить затраты на интеграцию \textit{подсистемы обучения пользователей интеллектуальных систем} и основной \textit{ostis-системы}.}

\scnheader{Рис. Архитектура подсистемы обучения пользователей интеллектуальных систем в составе другой ostis-системы}
\scneqfile{\\\includegraphics[width=0.5\linewidth]{figures/sd_learning/subsystem_arch.png}\\}

\scnheader{Подход к разработке баз знаний в подсистеме обучения пользователей интеллектуальных систем}

\scnrelfromlist{пример}{
\scnfileitem{\includegraphics[width=0.3\linewidth]{figures/sd_learning/example_1.png}}
\scnaddlevel{1}
\scnnote{В данном примере показано, как используя средства структуризации баз знаний, разработанные в рамках Технологии OSTIS, можно описать в базе знаний различные виды информации об одной и той же сущности, в частности, текущую занятость и профессиональные навыки пользователя. Аналогичным образом можно описать любую другую информацию о пользователе.}
\scnaddlevel{-1};
\scnitem{
\scnaddlevel{1}
\textbf{\textit{семантическая модель базы знаний}}\\
\scnrelfromset{абстрактная базовая декомпозиция}{
история и текущие процессы эксплуатации компьютерной системы\\
	\scnrelfromset{абстрактная базовая декомпозиция}{
	\scnitem{история эксплуатации компьютерной системы}
	;\scnitem{текущие процессы эксплуатации компьютерной системы}}\\
;документация компьютерной системы
;контекст предметной части базы знаний в рамках Глобальной базы знаний
;предметная часть базы знаний
;история, текущие процессы и план развития компьютерной системы\\
	\scnrelfromset{абстрактная базовая декомпозиция}{
	\scnitem{текущие процессы развития компьютерной системы}
	;\scnitem{история развития компьютерной системы}
	;\scnitem{структура и организация проекта компьютерной системы}
	;\scnitem{план развития компьютерной системы}}}}
\scnnote{База знаний ostis-системы может быть структурирована по различным признакам. В данном примере наибольший интерес представляет структуризация базы знаний с точки зрения процесса ее разработки.}
\scnaddlevel{-1};
\scnfileitem{\includegraphics[width=0.3\linewidth]{figures/sd_learning/example_2.png}}
\scnaddlevel{1}
\scnnote{Выше показан пример описания информации об исполнителях некоторого проекта, выполняющих в нем различные роли. С точки зрения структуры базы знаний эта информация является частью раздела структура и организация проекта компьютерной системы.}
\scnaddlevel{-1};\scnfileitem{\includegraphics[width=0.6\linewidth]{figures/sd_learning/example_3.png}}
\scnaddlevel{1}
\scnnote{Выше показан пример описания проектных задач и их исполнителей с учетом квалификации каждого исполнителя. С точки зрения структуры базы знаний эта информация является частью раздела текущие процессы развития компьютерной системы.}
\scnaddlevel{-1};
\scnitem{
\scnaddlevel{1}
\textbf{\textit{Решатель задач системы контроля качества нанесения маркировки}}\\
\scnrelfromset{декомпозиция абстрактного sc-агента}{
Атомарный абстрактный sc-агент распознавания маркировки на основе нейронной сети
;Неатомарный абстрактный sc-агент принятия решений\\
	\scnrelfromset{декомпозиция абстрактного sc-агента}{
	Атомарный абстрактный sc-агент, реализующий концепцию пакета программ\\
	;Неатомарный абстрактный sc-агент достоверного вывода
	;Неатомарный абстрактный sc-агент правдоподобного вывода}\\
;Неатомарный абстрактный sc-агент ассоциативного поиска\\
;Неатомарный абстрактный sc-агент интерпретации программ управления роботизированной установкой\\
	\scnrelfromset{декомпозиция абстрактного sc-агента}{
	Атомарный абстрактный sc-агент интерпретации действия перемещения\\
	;Атомарный абстрактный sc-агент интерпретации действия захвата}}
\scnheader{Неатомарный абстрактный sc-агент достоверного вывода}
\scnrelfromset{декомпозиция абстрактного sc-агента}{
Атомарный абстрактный sc-агент, реализующий стратегию правдоподобного вывода\\
;Неатомарный абстрактный sc-агент интерпретации логических правил}
\scnheader{Неатомарный абстрактный sc-агент правдоподобного вывода}
\scnrelfromset{декомпозиция абстрактного sc-агента}{
Атомарный абстрактный sc-агент, реализующий стратегию правдоподобного вывода\\
;Неатомарный абстрактный sc-агент интерпретации логических правил}
\scnheader{Неатомарный абстрактный sc-агент интерпретации логических правил}
\scnrelfromset{декомпозиция абстрактного sc-агента}{
Атомарный абстрактный sc-агент применения импликативных правил\\
;Атомарный абстрактный sc-агент применения правил об эквиваленции}
}
\scnnote{Согласно предложенному в рамках Технологии OSTIS подходу к разработке решателей задач основу решателя составляет иерархическая система агентов над семантической памятью (sc-агентов). Структура решателя также может быть описана в базе знаний ostis-системе. В данном примере представлена структура решателя задач системы контроля качества нанесения маркировки для предприятия рецептурного производства}
\scnaddlevel{-1}}
\scnendstruct \scnendcurrentsectioncomment
\end{SCn}


\scsubsubsection[\scnmonographychapter{Глава 7.4. Интеллектуальные обучающие системы нового поколения}]{Предметная область и онтология дидактических знаний}
\label{sd_didactic}

\scsubsection{Предметная область и онтология семантически совместимых ostis-систем автоматизации проектирования и управления проектированием различных объектов}
\label{sd_management_semantic_comp_sys}

\scsubsection[\scnmonographychapter{Глава 7.6. Умное предприятие и интеллектуальные компьютерные системы нового поколения. Опыт автоматизации предприятия ``Савушкин продукт''}]{Предметная область и онтология семантически совместимых ostis-систем автоматизации производственной деятельности}
\label{sd_activity_semantic_comp_sys}

\scsubsubsection[\scneditors{Крощенко А.А.;Иванюк Д.С.;Пупена А.Н.;Зотов Н.В.;Орлов М.К.}\protect\scnmonographychapter{Глава 7.6. Умное предприятие и интеллектуальные компьютерные системы нового поколения. Опыт автоматизации предприятия “Савушкин продукт”}]{Предметная область и онтология семантически совместимых ostis-систем управления рецептурным производством}
\label{sd_ecosys_enterprise}
\begin{SCn}

\scnsectionheader{\currentname}
\scnstartsubstruct

\scniselement{раздел базы знаний}

\scnheader{Предметная область семантически совместимых ostis-систем управления рецептурным производством}
\scnsdmainclasssingle{***}
\scnsdclass{***}
\scnsdrelation{***}

\bigskip

\scnheader{ostis-система управления рецептурным производством}

\scnendstruct \scnendcurrentsectioncomment

\end{SCn}

\scsubsection[\scneditor{Самодумкин С.А.}\protect\scnmonographychapter{Глава 7.8. Интеллектуальные геоинформационные системы нового поколения}]{Предметная область и онтология геоинформационных ostis-систем}
\label{sd_geosystems}

\scsubsubsection[\scnmonographychapter{Глава 7.8. Интеллектуальные геоинформационные системы нового поколения}]{Предметная область и онтология географических объектов}
\label{sd_geograph_obj}

\scsubsection[\scnmonographychapter{Глава 7.9. Информационная безопасность интеллектуальных компьютерных систем нового поколения}]{Предметная область и онтология средств обеспечения информационной безопасности ostis-систем в рамках Экосистемы OSTIS}
\label{sd_inf_security}
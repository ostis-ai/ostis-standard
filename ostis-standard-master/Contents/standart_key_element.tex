
\begin{SCn}

\scnsectionheader{Система ключевых знаков Cтандарта OSTIS}	

\scnstartsubstruct

\scnexplanation{\textit{Система ключевых знаков Стандарта OSTIS} должна стать целостным дополнением  к Оглавлению Стандарта OSTIS
	\begin{scnitemize}
		\item иерархия и последовательность ключевых знаков  должны четко соответствовать иерархии и последовательности разделов стандарта;
		\item система ключевых знаков Стандарта OSTIS, как и его Оглавление, должна восприниматься (читаться) как целостный понятный текст
	\end{scnitemize}}

\scnheader{Стандарт OSTIS}
\scnrelfromvector{ключевые знаки}{база знаний ostis-системы\\
	\scnaddlevel{1}	
		\scnidtf{база знаний, представленная в SC-коде}
		\scnidtf{sc-модель базы знаний}
	\scnaddlevel{-1};
	технология проектирования баз знаний ostis-систем;
	логическая sc-модель обработки знаний;технология проектирования логических sc-моделей обработки знаний;
	продукционная sc-модель обработки знаний;
	технология проектирования продукционных sc-моделей обработки знаний;
	sc-модель искусственной нейронной сети;	
	технология проектирования sc-моделей искусственных нейронных сетей;
	sc-модель интерфейса ostis-системы;
	технология проектирования  sc-моделей интерфейсов ostis-систем;
	sc-модель интерфейса ostis-системы\\
	\scnaddlevel{1}
		\scnidtf{онтологическая модель интерфейса, построенная на основе SC-кода}	
	\scnaddlevel{-1};
	технология проектирования sc-моделей интерфейсов ostis-систем;
	программная платформа реализации ostis-систем, построенная на основе системы управления графовыми базами данных\\
	\scnaddlevel{1}	
		\scnidtf{программная система интерпретации логико-семантических моделей ostis-систем, построенная на основе графовой СУБД}
		\scnidtf{система управления базами знаний (СУБЗ) ostis-систем, построенная на основе графовых СУБД}
	\scnaddlevel{-1};
	ассоциативный семантический компьютер для ostis-систем\\
	\scnaddlevel{1}	
		\scnidtf{компьютер с ассоциативной графодинамической (структурно реконфигурируемой) памятью, ориентированный на реализацию ostis-систем}
		\scnidtf{компьютер с ассоциативной графодинамической памятью, обеспечивающий интерпретацию логико-семантических моделей ostis-систем}
		\scnidtf{аппаратная платформа реализации ostis-систем}
	\scnaddlevel{-1};
	Проект OSTIS;
	Стандарт OSTIS;
	Экосистема OSTIS\\
	\scnaddlevel{1} 
		\scnidtf{Экосистема ostis-систем и их пользователей}
		\scnidtf{Вариант построения smart-общества (общества 5.0) на основе ostis-систем}
	\scnaddlevel{-1};
	агентно-ориентированная модель обработки информации\\ \scnaddlevel{1}
		\scnidtf{многоагентная модель обработки информации} \scnidtf{\textit{модель обработки информации}, рассматривающая \textit{процесс обработки информации} как \textit{деятельность}, выполняемую некоторым \textit{коллективом} самостоятельных \textit{информационных агентов} (агентов обработки информации)}
	\scnaddlevel{-1}
	}

\scnrelfrom{ключевой объект спецификации}{Технология OSTIS}
\scnaddlevel{1}
	\scnrelfrom{основные создаваемые продукты}{ostis-система}
		\scnidtf{множество всевозможных ostis-систем}
		\scnidtf{компьютерная система, построенная по Технологии OSTIS}
		\scnrelfromvector{ключевые понятия, соответствующие принципам, лежащим в основе}{cмысловое представление информации;
			агентно-ориентированная обработка информации;
			интерфейс компьютерной системы\\
			\scnaddlevel{1}
				\scnidtf{интерфейс компьютерной системы, построенный на основе онтологий}
				\scnidtf{ontology based interface}
			\scnaddlevel{-1};
			мультимодальность;
			конвергенция;
			семантическая совместимость;
			унификация;
			мультимодальная база знаний;
			универсальный язык смыслового представления знаний;
			мультимодальный решатель задач;
			мультимодальный интерфейс компьютерной системы;
			гибридная интеллектуальная компьютерная система\\
			\scnaddlevel{1}	
				\scnidtf{мультимодальная интеллектуальная компьютерная система}
			\scnaddlevel{-1};	
			мультимодальный (гибридный) характер и.к.с. в целом;
			мультимодальный характер баз знаний и.к.с.;
			мультимодальный решатель задач и.к.с.;
			мультимодальный (мультиязычный) интерфейс и.к.с.;
			конвергенция и.к.с.(знаний, моделей, решателей задач, моделей взаимодействий с внешней средой, моделей общения с внешним субъектом);
			семантическая совместимость и.к.с.(знаний, моделей, решателей задач, моделей взаимодействий с внешней средой, моделей общения с внешним субъектом);
			онтологическая модель;
			онтологическая логико-семантическая модель и.к.с.;
			онтологическая модель базы знаний и.к.с.;
			онтологическая модель решатель задач и.к.с.;
			онтологическая модель интерфейса  и.к.с.;
			неатомарный раздел;
			атомарный раздел;
			ключевой знак*\\
			\scnaddlevel{1}
				\scnidtf{ключевая сущность*}
			\scnaddlevel{-1};
			SC-код;
			SCg-код;
			SCs-код;
			SCn-код;
			декомпозиция*;
			конкатенация*;
			предметная область\\
			\scnaddlevel{1}
				\scnidtf{sc-модель предметной области}
				\scnidtf{sc-текст, являющийся представлением предметной области}
			\scnaddlevel{-1};
			максимальный класс объекта исследования\scnrolesign;
			немаксимальный класс объекта исследования\scnrolesign\\
			\scnaddlevel{1}
				\scnidtf{подкласс максимального объекта исследования\scnrolesign}
			\scnaddlevel{-1};
			исследуемое отношение\scnrolesign;
			исследуемый параметр\scnrolesign;
			онтология;
			алфавит*(языка);
			сформированное множество\\
			\scnaddlevel{1}
				\scnidtf{конечное множество, все элементы которого представлены соответствующими sc-элементами}
			\scnaddlevel{-1};		
			бинарное отношение;
			ориентированное отношение;
			первый домен*;
			второй домен*;
			пояснение*;
			семантическая эквивалентность*;
			следствие*;
			примечание*;
			определение*
		}
		\scnaddlevel{-1}	
	
	\scnendstruct 
\end{SCn}

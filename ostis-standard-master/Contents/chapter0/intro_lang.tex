\scsection[\scneditors{Никифоров С.А.;Бобёр  Е.С.}\protect\scnmonographychapter{Глава 2.6. Языковые средства формального описания синтаксиса и денотационной семантики различных языков в интеллектуальных компьютерных системах нового поколения}]{Предметная область и онтология информационных конструкций и языков}
\label{intro_lang}

\begin{SCn}

\scnsectionheader{\currentname}
\scnstartsubstruct

\scniselement{раздел базы знаний}
\scnrelfromlist{дочерний раздел}{\nameref{intro_sc_code};\nameref{intro_idtf};\nameref{intro_scg};\nameref{intro_scs};\nameref{intro_scn}}

\scntext{аннотация}{Поскольку предлагаемая Вашему вниманию \textit{Публикация Стандарта OSTIS-2021} представляет собой внешний текст основной части \textit{базы знаний} ostis-системы (\textit{Метасистемы IMS.ostis}), необходимо сразу во вводном разделе этого Стандарта пояснить основные принципы, лежащие в основе внутреннего представления \textit{баз знаний} в памяти \textit{ostis-систем}, а также некоторые правила и условные обозначения, используемые в оформлении внешних текстов \textit{баз знаний ostis-систем}.\\
Подчеркнем, что все эти принципы, правила и условные обозначения детально рассмотрены в соответствующих разделах \textit{Стандарта OSTIS}, но некоторые из них необходимо пояснить до начала ознакомления с основными положениями \textit{Технологии OSTIS}. Фактически речь идет о кратком руководстве конечных пользователей \textit{ostis-систем}.

Прежде, чем рассматривать \textit{внутренний язык*} и \textit{внешние языки*}, используемые \textit{ostis-системами}, необходимо уточнить понятие \textit{информационной конструкции}, понятие \textit{знака}, понятие \textit{текста}, понятие \textit{языка}.}\\

\scnheader{Предметная область информационных конструкций и языков}
\scniselement{предметная область}
\scnsdmainclasssingle{информационная конструкция}
\scnsdclass{дискретная информационная конструкция; элемент дискретной информационной конструкции; соответствие, заданное на множестве дискретных информационных конструкций; знак; знаковая конструкция;язык;семантически выделяемый класс дискретных информационных конструкций;язык ostis-системы;ограничитель;идентификатор;имя;предложение;sc.g-текст;sc.s-текст;sc.n-текст}

\scnhaselementlist{исследуемый класс классов первичных объектов исследования}{отношение, заданное на множестве элементов дискретных информационных конструкций\scnsupergroupsign ;параметр, заданный на множестве дискретных информационных конструкций\scnsupergroupsign ;отношение, заданное на множестве знаков\scnsupergroupsign ;отношение, заданное на множестве знаковых конструкций\scnsupergroupsign ;класс знаковых конструкций\scnsupergroupsign ;параметр, заданный на множестве знаковых конструкций\scnsupergroupsign ;отношение, заданное на множестве языков\scnsupergroupsign ;параметр, заданный на множестве языков\scnsupergroupsign}


\scnsdrelation{операционная семантика информационной конструкции*;алфавит*;ограничители*;разделители*;внешний идентификатор*;предложения*}


\scnheader{ostis-система}
\scnidtf{компьютерная система, построенная по Технологии OSTIS}

\scnheader{информационная конструкция}
\scnidtf{информация}
\scnidtf{конструкция (структура), содержащая некоторые сведения о некоторых сущностях}
\scnnote{Форма представления ("изображения"{}, "материализации"), форма структуризации (синтаксическая структура), а также \textit{смысл*} (денотационная семантика) \textit{информационных конструкций} могут быть самыми различными.}

\scnheader{дискретная информационная конструкция}
\scnsubset{информационная конструкция}
\scnexplanation{Каждая \textit{дискретная информационная конструкция} — это \textit{информационная конструкция}, смысл которой задается (1) множеством элементов (синтаксически атомарных фрагментов) этой информационной конструкции, (2) алфавитом этих элементов — семейством классов синтаксически эквивалентных элементов информационной конструкции, (3) принадлежностью каждого элемента информационной конструкции соответствующему классу синтаксически эквивалентных элементов информационной конструкции, (4) конфигурацией связей инцидентности между элементами информационной конструкции.}
    \scnaddlevel{1}
    \scntext{следствие}{Форма представления элементов дискретной информационной конструкции для анализа её смысла не требует уточнения. Главным является (1) наличие простой процедуры выделения (сегментации) элементов дискретной информационной конструкции (2) наличие простой процедуры установления синтаксической эквивалентности разных элементов дискретной информационной конструкции и (3) наличие простой процедуры установления принадлежности каждого элемента дискретной информационной конструкции соответствующему классу синтаксически эквивалентных элементов (т.е. соответствующему элементу алфавита).}
    \scnaddlevel{-1}

\scnheader{элемент дискретной информационной конструкции}
\scnidtf{синтаксически атомарный фрагмент (символ), входящий в состав дискретной информационной конструкции}
\scnnote{Поскольку дискретные информационные конструкции могут иметь общие элементы (атомарные фрагменты) и даже некоторые из них могут быть частями других информационных конструкций, элемент дискретной информационной конструкции может входить в состав сразу нескольких информационных конструкций.}
\scnrelto{второй домен}{элемент дискретной информационной конструкции*}

\scnheader{отношение, заданное на множестве элементов дискретных информационных конструкций\scnsupergroupsign}
\scnhaselement{элемент дискретной информационной конструкции*}
	\scnaddlevel{1}
	\scnidtf{быть элементарным (синтаксически атомарным) фрагментом заданной дискретной информационной конструкции*}
	\scnidtf{быть элементом (атомарным фрагментом) заданной дискретной информационной конструкции*}
	\scnidtf{Бинарное ориентированное отношение, каждая пара которого связывает (1) знак некоторой дискретной информационной конструкции и (2) знак одного из элементов этой дискретной информационной конструкции*}
	\scnrelfrom{второй домен}{элемент дискретной информационной конструкции}
	\scnaddlevel{-1}
\scnhaselement{синтаксическая эквивалентность элементов дискретных информационных конструкций*}
	\scnaddlevel{1}
	\scnidtf{быть синтаксически эквивалентными элементами (атомарными фрагментами) одной и той же или разных дискретных информационных конструкций, т.е. элементами, принадлежащими одному и тому же классу синтаксически эквивалентных элементов дискретных информационных конструкций*}
	\scnaddlevel{-1}
\scnhaselement{инцидентность элементов дискретных информационных конструкций*}
	\scnaddlevel{1}
	\scniselement{бинарное ориентированное отношение\scnsupergroupsign}
	\bigskip
	\scnnote{Для \textit{линейных информационных конструкций} -- это последовательность элементов (символов), входящих в состав этих конструкций.\\
	Для дискретных информационных конструкций, конфигурация которых имеет нелинейный характер, отношение инцидентности их элементов может быть разбито на несколько частных отношений инцидентности, каждое из которых является \uline{подмножеством} объединенного отношения инцидентности. Например, для двухмерных дискретных информационных конструкций это (1) инцидентность элементов информационных конструкций "по горизонтали"{} и (2) инцидентность элементов информационных конструкций "по вертикали".}
	\scnaddlevel{-1}

\scnhaselement{неэлементарный фрагмент дискретной информационной конструкции*}
	\scnaddlevel{1}
	\scnidtf{быть дискретной информационной конструкцией, которая является \uline{подструктурой} для заданной дискретной информационной конструкции, в состав которой входит (1) подмножество элементов заданной информационной конструкции и, соответственно, (2) подмножество пар инцидентности элементов заданной информационной конструкции.}
	\scnaddlevel{-1}
\scnhaselement{алфавит дискретной информационной конструкции*}
	\scnaddlevel{1}
	\scnidtf{быть семейством попарно непересекающихся \uline{классов} синтаксически эквивалентных элементов заданной дискретной информационной конструкции*}
	\scnaddlevel{-1}
\scnhaselement{первичная синтаксическая структура дискретной информационной конструкции*}
	\scnaddlevel{1}
	\scnidtf{быть \textit{графовой структурой}, которая полностью описывает конфигурацию заданной \textit{дискретной информационной конструкции} и которая включает в себя: (1) знаки всех тех классов синтаксически эквивалентных элементов, которым принадлежат элементы описываемой дискретной информационной конструкции, (2) знаки всех элементов (атомарных фрагментов) описываемой информационной конструкции, (3) пары, описывающие инцидентность элементов описываемой информационной конструкции, (4) пары, описывающие принадлежность элементов описываемой информационной конструкции соответствующим классам синтаксически эквивалентных элементов этой информационной конструкции.}
	\scnaddlevel{-1}
\scnhaselement{синтаксическая эквивалентность дискретных информационных конструкций*}
	\scnaddlevel{1}
	\scndefinition{Дискретные информационные конструкции $\bm{Ti}$ и $\bm{Tj}$ являются синтаксически эквивалентными в том и только в том случае, если между конструкцией $\bm{Ti}$ и конструкцией $\bm{Tj}$ существует \uline{изоморфизм}, в рамках которого каждому элементу конструкции $\bm{Ti}$ соответствует синтаксически эквивалентный элемент конструкции $\bm{Tj}$, т.е. элемент, принадлежащий тому же классу синтаксически эквивалентных элементов дискретных информационных конструкций. И наоборот.}
	\scnaddlevel{-1}
\scnhaselement{копия дискретной информационной конструкции*}
	\scnaddlevel{1}
	\scnsubset{синтаксическая эквивалентность дискретных информационных конструкций*}
	\scnidtfdef{быть дискретной информационной конструкцией, которая не только синтаксически эквивалентна заданной, но и содержит информацию о форме представления элементов копируемой информационной конструкции*}
	\scnaddlevel{-1}
\scnhaselement{семантическая эквивалентность дискретных информационных конструкций*}
	\scnaddlevel{1}
	\scndefinition{Информационная конструкция $\bm{Ti}$ и информационная конструкция $\bm{Tj}$ являются \uline{семантически эквивалентными} в том и только в том случае, если \uline{каждая} сущность (в том числе, и каждая связь между сущностями), описываемая в информационной конструкции $\bm{Ti}$ описывается также и в информационной конструкции $\bm{Tj}$. И наоборот.}
	\scnaddlevel{-1}
\scnhaselement{семантическое расширение дискретной информационной конструкции*}
	\scnaddlevel{1}
	\scndefinition{Информационная конструкция $\bm{Tj}$ является семантическим расширением информационной конструкции $\bm{Ti}$ в том и только в том случае, если \uline{каждая} сущность, описываемая в $\bm{Ti}$, описывается также и в $\bm{Tj}$, но обратное неверно.}
	\scnaddlevel{-1}
\scnhaselement{синтаксис информационной конструкции*}
	\scnaddlevel{1}
	\scnidtf{синтаксическая структура информационной конструкции*}
	\scnidtf{быть синтаксической структурой для заданной информационной конструкции*}
	\scnidtf{Бинарное ориентированное отношение, каждая пара которого связывает знак некоторой информационной конструкции с синтаксической структурой этой конструкции*}
	\scnidtf{описание того, из каких частей состоит заданная информационная конструкция и как эти части (фрагменты) связаны между собой*}
	\scnrelfrom{первый домен}{информационная конструкция}
	\scnaddlevel{-1}
\bigskip
\scnhaselement{смысл*}
	\scnaddlevel{1}
	\scnidtf{смысл информационной конструкции*}
	\scnidtf{денотационная семантика информационной конструкции*}
	\scnidtf{семантическая (смысловая) структура информационной конструкции*}
	\scnidtf{быть семантической (смысловой) структурой для заданной информационной конструкции*}
	\scnidtf{быть смыслом заданной информационной конструкции*}
	\scnidtf{быть явным (формальным) представлением того, какие сущности описывает данная информационная конструкция и как эти сущности связаны между собой*}
	\scnidtf{Бинарное ориентированное отношение, каждая пара которого связывает некоторую информационную конструкцию с ее смыслом (смысловой структурой)*}
	\scnaddlevel{-1}
\scnhaselement{операционная семантика информационной конструкции*}
	\scnaddlevel{1}
	\scnidtf{правила трансформации заданной информационной конструкции*}
	\scnidtf{описание того, на основании каких правил можно осуществлять действия по преобразования (обработке, трансформации) заданной информационной конструкции, оставляя ее в рамках класса синтаксически и семантически правильных информационных конструкций*}
	\scnidtf{Бинарное ориентированное отношение, каждая пара которого связывает знак некоторой информационной конструкции со множеством правил ее трансформации*}
    \scnaddlevel{-1}
	
\scnheader{операционная семантика информационной конструкции*}
\scnrelfrom{второй домен}{операционная семантика информационной конструкции}
	
\scnheader{соответствие, заданное на множестве дискретных информационных конструкций}
\scnhaselement{соответствие между синтаксической структурой информационной конструкции и смыслом этой конструкции*}
\scnaddlevel{1}
	\scnidtf{Множество ориентированных пар, первым компонентом которых является ориентированная пара, состоящая из (1) знака синтаксической структуры некоторой информационной конструкции и (2) знака смысловой структуры этой конструкции, а вторым компонентом которых является множество ориентированных пар, связывающих фрагменты синтаксической структуры заданной информационной конструкции (которые описывают либо структуру фрагментов заданной конструкции, либо связи между фрагментами этой конструкции) с теми фрагментами смысловой структуры заданной информационной конструкции, которые семантически эквивалентны либо синтаксически представленным фрагментам заданной информационной конструкции, либо синтаксически представленным связям между такими фрагментами*}
	\scnsubset{соответствие*}
\scnaddlevel{-1}

\scnheader{параметр, заданный на множестве дискретных информационных конструкций\scnsupergroupsign}
\scnhaselement{размерность дискретных информационных конструкций\scnsupergroupsign}
	\scnaddlevel{1}
	\scnidtf{типология дискретных информационных конструкций, определяемая их размерностью}
	\scnexplanation{Это параметр дискретных информационных конструкций, определяющий характер \uline{инцидентности} элементов таких конструкций.}
	\scnhaselement{линейная информационная конструкция}
		\scnaddlevel{1}
		\scnidtfexp{дискретная информационная конструкция, каждый элемент которой может иметь не более двух инцидентных ему элементов (один слева, другой справа)}
		\scnaddlevel{-1}
	\scnhaselement{двухмерная информационная конструкция}
		\scnaddlevel{1}
		\scnidtfexp{дискретная информационная конструкция, каждый элемент которой может иметь не более четырех инцидентных ему элементов (слева-справа, сверху-снизу)}
		\scnaddlevel{-1}
	\scnhaselement{трехмерная информационная конструкция}
		\scnaddlevel{1}
		\scnidtfexp{дискретная информационная конструкция, каждый элемент которой может иметь не более шести инцидентных ему элементов (слева-справа, сверху-снизу, сзади-спереди)}
		\scnaddlevel{-1}
	\scnhaselement{четырехмерная информационная конструкция}
		\scnaddlevel{1}
		\scnidtfexp{дискретная информационная конструкция, каждый элемент которой может иметь не более восьми инцидентных ему элементов (например, слева-справа, сверху-снизу, сзади-спереди, раньше-позже)}
		\scnaddlevel{-1}
	\scnhaselement{графовая информационная конструкция}
		\scnaddlevel{1}
		\scnidtfexp{дискретная информационная конструкция, множество элементов которой разбивается на два подмножества — связки и узлы. При этом узлы могут иметь \uline{неограниченное} количество инцидентных им связок}
			\scnaddlevel{1}
			\scnnote{А в некоторых графовых информационных конструкциях и связки могут иметь неограниченное количество инцидентных им других связок}
			\scnaddlevel{-1}
		\scnaddlevel{-1}
	\scnaddlevel{-1}
\scnhaselement{типология дискретных информационных конструкций, определяемая их носителем\scnsupergroupsign}
    \scnaddlevel{1}
        \scnhaselement{некомпьютерная форма представления дискретных информационных конструкций}
	    \scnaddlevel{1}
	    \scnsuperset{аудио-сообщение}
    		\scnaddlevel{1}
		    \scnidtf{речевое сообщение}
		    \scnidtf{информационная конструкция, представленная в звуковой форме}
		    \scnaddlevel{-1}
	    \scnsuperset{информационная конструкция, представленная на языке жестов}
	    \scnsuperset{информационная конструкция, представленная в письменной форме}
    		\scnaddlevel{1}
		    \scnnote{Конкретный вид носителя для письменной формы представления информации может быть разным — бумага, папирус, береста, камень...}
		    \scnaddlevel{-1}
	\scnaddlevel{-1}
	\scnhaselement{файл}
		\scnaddlevel{1}
		\scnidtf{компьютерная ("электронная") форма (формат) представления и хранения информационных конструкций}
		\scnnote{Представление информационных конструкций в виде файлов ориентировано на представление \uline{дискретных} (!) информационных конструкций. Поэтому "файловое"{} представление недискретных информационных конструкций (например, различного рода сигналов) предполагает "дискретизацию"{} таких конструкций, т.е. преобразование их в дискретные. Так преобразуются аудио-сигналы (в частности, речевые сообщения), изображения, видео-сигналы и др.}
		\scnaddlevel{-1}
	\scnaddlevel{-1}
\scnhaselement{уровень унификации представления синтаксически эквивалентных элементов дискретных информационных конструкций\scnsupergroupsign}
	\scnaddlevel{1}
	\scnidtf{уровень "членораздельности"{} дискретных информационных конструкций}
	\scnnote{Чем выше уровень унификации представления элементов дискретных информационных конструкций, тем проще реализуется (1) процедура выделения (сегментации) элементов дискретной информационной конструкции, (2) процедура установления синтаксической эквивалентности этих элементов и (3) процедура их распознавания, т.е. процедура установления их принадлежности соответствующим классам синтаксически эквивалентных элементов.}
	\scnhaselement{дискретная информационная конструкция с низким уровнем унификации представления элементов}
		\scnaddlevel{1}
		\scnsuperset{аудио-сообщение}
		\scnsuperset{информационная конструкция, представленная на языке жестов}
		\scnsuperset{рукопись или её копия}
		\scnaddlevel{-1}
	\scnhaselement{дискретная информационная конструкция с высоким уровнем унификации представления элементов}
		\scnaddlevel{1}
		\scnsuperset{печатный текст}
		\scnsuperset{файл}
		\scnaddlevel{-1}
	\scnaddlevel{-1}
\scnaddlevel{-1}	

\scnheader{знак}
\scnexplanation{фрагмент информационной конструкции, который условно представляет (изображает) некоторую описываемую сущность, которую называют денотатом знака}
\scnsubdividing{знак, являющийся элементом дискретной информационной конструкции;знак, являющийся неатомарным фрагментом дискретной информационной конструкции}
\scnnote{Отсутствие знака, обозначающего некоторую сущность, не означает отсутствие самой этой сущности. Это означает только то, что мы даже не догадываемся о её существовании и, следовательно, не приступили к её исследованию.}

\newpage
\scnheader{отношение, заданное на множестве знаков\scnsupergroupsign}
\scnnote{Поскольку все знаки являются дискретными информационными конструкциями, множество знаков является областью задания всех отношений, заданных на множестве дискретных информационных конструкций. Тем не менее есть как минимум одно отношение, специфичное для множества знаков.}
\scnhaselement{синонимия знаков*}
	\scnaddlevel{1}
	\scndefinition{Знаки являются синонимичными в том и только в том случае, если они обозначают одну и ту же сущность.}
	\scnnote{Синонимичные знаки могут быть синтаксически эквивалентными, а могут и не быть таковыми.}
	\scnaddlevel{-1}

\scnheader{знаковая конструкция}
\scnsubset{дискретная информационная конструкция}
\scnidtfexp{дискретная информационная конструкция, которая в общем случае представляет собой конфигурацию знаков и специальных фрагментов информационной конструкции, обеспечивающих структуризацию конфигурации знаков — различного вида разделителей и ограничителей}
	\scnaddlevel{1}
	\scnnote{Для некоторых знаковых конструкций и даже для некоторых языков необходимость в разделителях и ограничителях может отсутствовать.}
	\scnaddlevel{-1}	
	
\scnheader{отношение, заданное на множестве знаковых конструкций\scnsupergroupsign}
\scnhaselement{знак*}
	\scnaddlevel{1}
	\scnidtf{быть знаком для заданной знаковой конструкции*}
	\scnaddlevel{-1}
\scnhaselement{разделитель знаковой конструкции*}
\scnhaselement{разделители знаковой конструкции*}
	\scnaddlevel{1}
	\scnidtf{Множество всех разделителей, входящих в состав заданной знаковой конструкции*}
	\scnaddlevel{-1}
\scnhaselement{ограничитель знаковой конструкции*}
\scnhaselement{ограничители знаковой конструкции*}
\scnhaselement{семантическая смежность знаковых конструкций*}
	\scnaddlevel{1}
	\scndefinition{Знаковые конструкции $\bm{Ti}$ и $\bm{Tj}$ являются смежными в том и только в том случае, если существуют синонимичные знаки $\bm{Ti}$ и $\bm{Tj}$, один из которых входит в состав конструкции $\bm{Ti}$, а второй — в состав конструкции $\bm{Tj}$}
	\scnaddlevel{-1}
\scnhaselement{конкатенация знаковых конструкций*}
	\scnaddlevel{1}
	\scnidtf{декомпозиция заданной знаковой конструкции на последовательность знаковых конструкций*}
	\scnaddlevel{-1}
	
\scnheader{класс знаковых конструкций\scnsupergroupsign}
\scnhaselement{семантически элементарная знаковая конструкция}
	\scnaddlevel{1}
	\scnidtf{знаковая конструкция, описывающая некоторую (одну) связь между некоторыми сущностями}
	\scnaddlevel{-1}
\scnhaselement{семантически связная знаковая конструкция}
	\scnaddlevel{1}
	\scnidtfdef{знаковая конструкция, которую можно представить в виде конкатенации семантически элементарных знаковых конструкций, каждая из которых семантически смежна предшествующей и последующей семантически элементарной знаковой конструкции}
	\scnaddlevel{-1}

\scnheader{параметр, заданный на множестве знаковых конструкций\scnsupergroupsign}
\scnhaselement{семантическая связность знаковых конструкций\scnsupergroupsign}
	\scnaddlevel{1}
	\scnhaselement{семантически связная знаковая конструкция}
	\scnhaselement{семантически несвязная знаковая конструкция}
	    \scnaddlevel{1}
    	\scnhaselementrole{пример}{\scnfilelong{В огороде бузина, а в Киеве дядька}}
    	\scnaddlevel{-1}
	\scnaddlevel{-1}
\scnhaselement{наличие разделителей и ограничителей\scnsupergroupsign}
	\scnaddlevel{1}
	\scnhaselement{знаковая конструкция, содержащая разделители и-или ограничители}
	\scnhaselement{знаковая конструкция без разделителей и ограничителей}

\newpage
\scnheader{язык}
\scnidtfexp{класс знаковых конструкций, для которого существуют (1) общие правила их построения и (2) общие правила их соотнесения с теми сущностями и конфигурациями сущностей, которые описываются (отражаются) указанными знаковыми конструкциями}
\scnidtf{класс знаковых конструкций, эквивалентных с точки зрения правил их построения и правил их семантической интерпретации}
\scnsubdividing{язык, у которого все знаки, входящие в состав его знаковых конструкций, являются элементарными фрагментами этих конструкций\\
	\scnaddlevel{1}
	\scnnote{Для языков такого типа существенно упрощаются методы обработки их текстов.}
	\scnaddlevel{-1}
;язык, у которого знаки, входящие в состав его знаковых конструкций, в общем случае не являются элементарными фрагментами этих конструкций}
\scnsubdividing{язык, знаковые конструкции которого содержат разделители и ограничители;язык, знаковые конструкции которого не содержат разделителей и ограничителей\\
	\scnaddlevel{1}
	\scntext{следствие}{Знаковые конструкции такого языка состоят \uline{только} из знаков.}
	\scnaddlevel{-1}}

\scnheader{отношение, заданное на множестве языков\scnsupergroupsign}
\scnidtf{отношение, область определения которого включает в себя множество всевозможных языков}
\scnhaselement{текст заданного языка*}
	\scnaddlevel{1}
	\scnidtf{знаковая конструкция, принадлежащая заданному языку*}
	\scnidtf{синтаксически правильная (правильно построенная) знаковая конструкция заданного языка*}
	\scnidtf{синтаксически корректная и целостная знаковая конструкция для заданного языка*}
	\scneq{{\normalfont(}синтаксически корректная знаковая конструкция для заданного языка* $\cap$ синтаксически целостная знаковая конструкция для заданного языка*{\normalfont)}}
	\scnaddlevel{-1}
\scnhaselement{синтаксически корректная знаковая конструкция для заданного языка*}
	\scnaddlevel{1}
	\scnidtf{знаковая конструкция, не содержащая синтаксических ошибок для заданного языка}
	\scnaddlevel{-1}
\scnhaselement{синтаксически целостная знаковая конструкция для заданного языка*}
\scnhaselement{синтаксически неправильная знаковая конструкция для заданного языка*}
	\scnaddlevel{1}
	\scneq{{\normalfont(}синтаксически некорректная знаковая конструкция для заданного языка* $\cup$ синтаксически нецелостная знаковая конструкция для заданного языка*{\normalfont)}}
	\scnsuperset{синтаксически некорректная знаковая конструкция для заданного языка*}
	\scnsuperset{синтаксически нецелостная знаковая конструкция для заданного языка*}
	\scnaddlevel{-1}
\scnhaselement{знание, представленное на заданном языке*}
	\scnaddlevel{1}
	\scnidtf{семантически правильный текст заданного языка*}
	\scneq{(семантически корректный текст заданного языка* $\cap$ семантически целостный текст заданного языка*)}
	\scnidtf{истинный текст заданного языка*}
	\scnidtf{истинное высказывание, представленное на заданном языке*}
	\scnaddlevel{-1}
\scnhaselement{семантически корректный текст заданного языка*}
	\scnaddlevel{1}
	\scnidtf{текст заданного языка, не содержащий семантических ошибок, противоречащих признанным закономерностям и фактам*}
	\scnaddlevel{-1}
\scnhaselement{семантически целостный текст заданного языка*}
	\scnaddlevel{1}
	\scnidtf{текст заданного языка, содержащий достаточную информацию для установления его истинности*}
	\scnaddlevel{-1}
\scnhaselement{семантически неправильный текст для заданного языка*}
	\scnaddlevel{1}
	\scneq{(семантически некорректный текст для заданного языка* $\cup$ семантически нецелостный текст для заданного языка*)}
	\scnsuperset{семантически некорректный текст для заданного языка*}
	\scnsuperset{семантически нецелостный текст для заданного языка*}
	\scnaddlevel{-1}
\scnhaselement{алфавит*}
	\scnaddlevel{1}
	\scnidtf{алфавит заданной информационной конструкции или заданного языка*}
	\scnidtf{семейство классов, синтаксически эквивалентных элементов (элементарных фрагментов) заданной информационной конструкции или информационных конструкций заданного языка*}
	\scnaddlevel{-1}
\scnhaselement{семейство классов синтаксически эквивалентных разделителей*}
	\scnaddlevel{1}
	\scnidtf{семейство классов синтаксически эквивалентных разделителей, входящих в состав заданной информационной конструкции или в состав информационных конструкций заданного языка*}
	\scnaddlevel{-1}
\scnhaselement{семейство классов синтаксически эквивалентных ограничителей*}
	\scnaddlevel{1}
	\scnidtf{семейство классов синтаксически эквивалентных ограничителей, входящих в состав заданной информационной конструкции или в состав информационных конструкций заданного языка*}
	\scnaddlevel{-1}	
\scnhaselement{синтаксис языка*}
	\scnaddlevel{1}
	\scnidtf{быть теорией правильно построенных информационных конструкций, принадлежащих заданному языку*}
	\scnidtf{определение понятия правильно построенной информационной конструкции заданного языка*}
	\scnidtf{синтаксические правила заданного языка*}
	\scnidtf{быть синтаксическими правилами заданного языка *}
	\scnidtf{Бинарное ориентированное отношение, каждая пара которого связывает знак некоторого языка с описанием синтаксически выделяемых классов фрагментов конструкций заданного языка, с описанием отношений, заданных на этих классах и с конъюнкцией кванторных высказываний, являющихся синтаксическими правилами заданного языка, то есть правилами, которым должны удовлетворять все синтаксические правильные (правильно построенные) конструкции (тексты) указанного языка*}
	\scnnote{При представлении синтаксиса (синтаксических правил) заданного языка используются все те понятия, которые вводятся для представления синтаксических структур информационных конструкций, принадлежащих указанному языку. Это и синтаксически выделяемые классы фрагментов указанных информационных конструкций, и отношения, заданные на множестве таких фрагментов.}
	\scnrelfrom{второй домен}{синтаксис языка}
	\scnaddlevel{-1}
\scnhaselement{описание синтаксических понятий языка*}
    \scnaddlevel{1}
    \scnidtf{описание синтаксически выделяемых классов фрагментов конструкций заданного языка*}
    \scnrelfrom{второй домен}{описание синтаксических понятий языка}
    \scnaddlevel{1}
        \scnrelto{обобщенное включение}{синтаксис языка}
    \scnaddlevel{-2}
\scnhaselement{синтаксические правила языка*}
    \scnaddlevel{1}
    \scnidtf{синтаксические правила заданного языка*}
    \scnrelfrom{второй домен}{синтаксические правила языка}
    \scnaddlevel{-1}
\scnhaselement{денотационная семантика языка*}
	\scnaddlevel{1}
	\scnidtf{быть теорией морфизмов, связывающих правильно построенные информационные конструкции заданного языка с описываемыми конфигурациями описываемых сущностей*}
	\scnaddlevel{-1}

\scnhaselement{денотационная семантика языка*}
	\scnaddlevel{1}
	\scnidtf{семантические правила заданного языка*}
	\scnidtf{быть семантическими правилами заданного языка *}
	\scnidtf{Бинарное ориентированное отношение, каждая пара которого связывает знак некоторого языка с описанием основных семантических понятий заданного языка и конъюнкцией кванторных высказываний, являющихся семантическими правилами заданного языка, то есть правилами, которым должны удовлетворять семантически правильные \uline{смысловые} информационные конструкции, соответствующие (семантические эквивалентные) синтаксически правильным конструкциям (текстам) заданного языка*}
	\scnnote{При формулировке семантических правил заданного языка используются понятия, введенные в рамках базовых онтологий (онтологий самого высокого уровня, в которых рассматриваются самые общие классы описываемых сущностей, самые общие отношения и параметры).}
	\scnrelfrom{второй домен}{денотационная семантика языка}
	\scnaddlevel{-1}
\scnhaselement{описание семантических понятий языка*}
	\scnaddlevel{1}
	\scnrelfrom{второй домен}{описание семантических понятий языка}
	\scnaddlevel{-1}
\scnhaselement{семантические правила языка*}
	\scnaddlevel{1}
	\scnrelfrom{второй домен}{семантические правила языка}
	\scnaddlevel{-1}
\bigskip
\scnhaselement{семантическая эквивалентность языков*}
	\scnaddlevel{1}
	\scnidtf{быть семантически эквивалентными языками*}
	\scndefinition{Язык $\bm{Li}$ и язык $\bm{Lj}$ будем считать \textit{семантически эквивалентными языками*} в том и только в том случае, если для каждого текста, принадлежащего языку $\bm{Li}$, существует \textit{семантически эквивалентный текст*}, принадлежащий языку $\bm{Lj}$, и наоборот.}
	\scnaddlevel{-1}
\scnhaselement{семантическое расширение языка*}
	\scnaddlevel{1}
	\scnrelboth{обратное отношение}{семантическое сужение языка*}
	\scndefinition{Язык $\bm{Lj}$ будем считать \textit{семантическим расширением*} языка $\bm{Li}$ в том и только в том случае, есть ли для каждого текста, принадлежащего языку $\bm{Li}$, существует \textit{семантически эквивалентный текст*}, принадлежащий языку $\bm{Lj}$, но обратное неверно.}
	\scnaddlevel{-1}
\scnhaselement{синтаксическое расширение языка*}
	\scnaddlevel{1}
	\scnidtf{быть семантически эквивалентным надмножеством заданного языка*}
	\scndefinition{Язык $\bm{Lj}$ будем считать \textit{синтаксическим расширением*} языка $\bm{Li}$ в том и только в том случае, если:
    \begin{scnitemize}
    \item $\bm{L_j} \supset \bm{Li}$ (то есть все тексты языка $\bm{Li}$ являются также и текстами языка $\bm{Lj}$, но обратное неверно);
    \item Язык $\bm{Lj}$ и язык $\bm{Li}$ являются \textit{семантически эквивалентными языками*}.
    \end{scnitemize}
    }
	\scnaddlevel{-1}
\scnhaselement{синтаксическое ядро языка*}
	\scnaddlevel{1}
	\scnidtf{быть синтаксическим ядром заданного языка*}
	\scnidtf{быть семантически эквивалентным подмножеством заданного языка, имеющим минимальную синтаксическую сложность*}
	\scnaddlevel{-1}
\scnhaselement{направление синтаксического расширения ядра заданного языка*}
	\scnaddlevel{1}
	\scnidtf{быть правилом трансформации информационных конструкций, принадлежащих заданному языку, которое описывает одно из направлений перехода от множества конструкций ядра этого языка ко множеству всех принадлежащих ему информационных конструкций*}
	\scnaddlevel{-1}
\scnhaselement{операционная семантика языка*}
	\scnaddlevel{1}
	\scnidtf{Бинарное ориентированное отношение, каждая пара которого связывает знак некоторого языка со множеством правил трансформации текстов этого языка*}
	\scnrelfrom{второй домен}{операционная семантика языка}
	\scnaddlevel{-1}
\scnhaselement{внутренний язык*}
	\scnaddlevel{1}
	\scnidtf{быть внутренним языком для заданной системы, основанной на обработке информации, или заданного множества таких систем*}
	\scnidtf{быть языком внутреннего представления информации в памяти заданной системы, основанной на обработке информации или заданного класса таких систем*}
	\scnaddlevel{-1}
\scnhaselement{внешний язык*}
	\scnaddlevel{1}
	\scnidtf{быть внешним языком для заданной системы, основанной на обработке информации, или заданного множества таких систем*}
	\scnidtf{быть языком, используемым для обмена информацией заданной системы, основанной на обработке информации, или заданного множества таких систем с другими системами, основанными на обработке информации, (в том числе, и с себе подобными)*}
	\scnaddlevel{-1}
\scnhaselement{используемый язык*}
	\scnaddlevel{1}
	\scneq{{\normalfont(}внутренний язык* $\cup$ внешний язык*{\normalfont)}}
	\scnidtf{язык, используемый заданной системой, основанной на обработке информации или заданного множества таких систем*}
	\scnidtf{язык, носителем которого является (которым владеет) данная система, основанная на обработке информации}
	\scnaddlevel{-1}
\scnhaselement{используемые языки*}

\scnheader{параметр, заданный на множестве языков\scnsupergroupsign}
\scnidtf{признак классификации языков}
\scnidtf{свойство, которым обладают языки}
\scnidtf{свойство языков, на основании которого можно осуществлять их классификацию}
\scnidtf{семейство классов эквивалентности языков, трактуемой в контексте того или иного свойства (характеристики), присущего языкам}
\scnhaselement{семантическая мощность языка\scnsupergroupsign}
	\scnaddlevel{1}
	\scnidtf{класс языков, семантически эквивалентных друг другу}
	\scnhaselement{универсальный язык}
		\scnaddlevel{1}
		\scnidtf{класс всевозможных универсальных языков}
		\scnnote{Очевидно, что все универсальные языки (если они действительно таковыми являются, а не только претендуют на это) семантически эквивалентны друг другу, т.е. имеют одинаковую семантическую мощность.}
		\scnaddlevel{-1}
	\scnaddlevel{-1}
\scnhaselement{уровень синтаксической сложности представления знаков в текстах языка\scnsupergroupsign}
	\scnaddlevel{1}
	\scnhaselement{язык, в текстах которого все знаки представлены синтаксически элементарными фрагментами}
	\scnhaselement{язык, в текстах которого знаки в общем случае представлены синтаксически неэлементарными фрагментами}
	\scnaddlevel{-1}
\scnhaselement{использование разделителей и ограничителей в текстах языка\scnsupergroupsign}
	\scnaddlevel{1}
	\scnhaselement{язык, в текстах которого не используются разделители и ограничители}
	\scnhaselement{язык, в текстах которого используются разделители и ограничители}
	\scnaddlevel{-1}
\scnhaselement{уровень сложности процедуры установления синонимии знаков в текстах языка\scnsupergroupsign}
	\scnaddlevel{1}
	\scnhaselement{язык, в рамках каждого текста которого синонимичные знаки отсутствуют}
		\scnaddlevel{1}
		\scnexplanation{В текстах такого языка знак каждой описываемой сущности входит \uline{однократно}.}
		\scnaddlevel{-1}
	\scnhaselement{язык, в рамках которого синонимичные знаки представлены синтаксически эквивалентными фрагментами текстов}
	\scnhaselement{флективный язык}
		\scnaddlevel{1}
		\scnidtf{язык, в рамках которого синонимичные знаки могут быть представлены синтаксически неэквивалентными фрагментами, но фрагментами, являющимися модификациями некоторого "ядра"{} этих фрагментов (при склонении и спряжении этих знаков).}
		\scnaddlevel{-1}
	\scnhaselement{язык, в рамках которого синонимичные знаки в общем случае могут быть представлены синтаксически неэквивалентными текстовыми фрагментами, структура которых носит непредсказуемый характер}
	\scnaddlevel{-1}
\scnhaselement{наличие омонимии в текстах языка\scnsupergroupsign}
	\scnaddlevel{1}
	\scnhaselement{язык, в текстах которого присутствует омонимия знаков}
		\scnaddlevel{1}
		\scnidtf{язык, в текстах которого присутствуют синтаксически эквивалентные, не синонимичные знаки}
		\scnaddlevel{-1}
	\scnhaselement{язык, в текстах которого омонимия знаков отсутствует}
	\scnaddlevel{-1}

\scnheader{семантически выделяемый класс дискретных информационных конструкций}
\scnhaselement{синтаксическая структура информационной конструкции}
    \scnaddlevel{1}
    \scnrelto{второй домен}{синтаксис информационной конструкции*}
    \scnsuperset{первичная синтаксическая структура информационной конструкции}
    \scnsuperset{вторичная синтаксическая структура информационной конструкции}
    \scnaddlevel{-1}
\scnhaselement{синтаксис языка}
\scnhaselement{описание синтаксических понятий языка}
\scnhaselement{синтаксические правила языка}
\scnhaselement{денотационная семантика языка}
\scnhaselement{описание семантических понятий языка}
\scnhaselement{семантические правила языка}
\scnhaselement{операционная семантика языка}
\scnhaselement{смысл}
    \scnaddlevel{1}
    \scnrelto{второй домен}{смысл*}
    \scnidtf{смысловая информационная конструкция}
    \scnidtf{смысловое представление информационной конструкции}
    \scnidtf{явное (формальное) представление описываемых сущностей и связей между ними}
    \scnidtf{смысловое представление информации}
    \scnnote{Для явного представления описываемых сущностей и связей между ними требуется существенное упрощение синтаксической структуры информационных конструкций}
    \scnaddlevel{-1}
    
\scnheader{язык ostis-системы}
\scnidtf{язык, используемый ostis-системами}
\scnidtf{язык представления информационных конструкций в ostis-системах}
\scnidtf{Множество языков, которыми "владеют"\ ostis-системы}
\scnsubset{формальный язык}
\scnsubset{универсальный язык}
\scnrelto{используемые языки}{ostis-система}

\scnhaselement{SC-код}
\scnaddlevel{1}
\scnidtf{Semantic Computer Code}
\scnrelto{внутренний язык}{ostis-система}
\scniselement{универсальный язык}
\scnaddlevel{-1}

\scnhaselement{SCg-код}
\scnaddlevel{1}
\scnidtf{Semantic Code graphical}
\scnidtf{\textit{внешний язык*} ostis-систем, тексты которого представляют собой графовые структуры общего вида с точно заданной \textit{денотационной семантикой*}}
\scnrelto{внешний язык}{ostis-система}
\scniselement{универсальный язык}
\scnaddlevel{-1}

\scnhaselement{SCs-код}
\scnaddlevel{1}
\scnidtf{Semantic Code string}
\scnidtf{\textit{внешний язык*} ostis-систем, тексты которого представляют собой строки (цепочки) символов}
\scnrelto{внешний язык}{ostis-система}
\scniselement{универсальный язык}
\scnaddlevel{-1}

\scnhaselement{SCn-код}
\scnaddlevel{1}
\scnidtf{Semantic Code natural}
\scnidtf{\textit{внешний язык*} ostis-систем, тексты которого представляют собой двухмерные матрицы символов, являющиеся результатом форматирования, двухмерной структуризации текстов SCs-кода.}
\scnrelto{внешний язык}{ostis-система}
\scniselement{универсальный язык}
\scnaddlevel{-1}
\scnexplanation{Для представления \textit{баз знаний ostis-систем} используется целый ряд как \textit{универсальных языков}, так и \textit{специализированных языков}, как \textit{формальных языков}, так и \textit{естественных языков}, как \textit{внутренних языков}, обеспечивающих представление информации в памяти \textit{ostis-систем}, так и \textit{внешних языков}, обеспечивающих представление информации, вводимой в память \textit{ostis-систем}, либо выводимой из этой памяти. \textit{Естественные языки} используются исключительно для представления \textit{файлов}, хранимых в памяти \textit{ostis-системы} и формально специфицируемых в рамках \textit{базы знаний} этой \textit{ostis-системы}. 

Для эксплуатации \textit{интеллектуальных компьютерных систем}, построенных на основе \textit{SC-кода}, кроме способа абстрактного внутреннего представления баз знаний (\textit{SC-кода}) потребуются несколько способов внешнего изображения абстрактных \textit{sc-текстов}, удобных для пользователей и используемых при оформлении исходных текстов \textit{баз знаний} указанных интеллектуальных компьютерных систем и исходных текстов фрагментов этих \textit{баз знаний}, а также используемых для отображения пользователям различных фрагментов \textit{баз знаний} по пользовательским запросам. В качестве таких способов внешнего отображения \textit{sc-текстов} и предлагаются указанные выше внешние языки ostis-систем (\textit{SCg-код}, \textit{SCs-код} и  \textit{SCn-код}).

Для описания перечисленных \textit{языков}, используемых \textit{ostis-системами}, в каждом из них мы выделим \textit{ядро языка*}, которое является \textit{семантически эквивалентным языком*} для соответствующего языка и имеет минимальную синтаксическую сложность. Описание каждого из указанных языков строится как описание нескольких направлений синтаксического расширения выделенного \textit{языка-ядра}.
}

\scnnote{Все основные внешние формальные языки, используемые ostis-системами (\textit{SCg-код}, \textit{SCs-код}, \textit{SCn-код}) являются различными вариантами внешнего представления текстов внутреннего языка ostis-систем -- SC-кода. Указанные языки являются универсальными и, следовательно, \textit{семантически эквивалентными языками*}.}
\newpage
\scnnote{Каждая ostis-система может приобрести способность использовать любой внешний язык (как универсальный, так и специализированный, как естественный, так и искусственный), если синтаксис и денотационная семантика этого языка будут описаны в памяти ostis-системы на ее внутреннем языке (SC-коде).}

\scnheader{следует отличать*}
\scnhaselementset{семантическое расширение языка*\\
\scnaddlevel{1}
\scnidtf{расширение семантической мощности языка*}
\scnaddlevel{-1}
;синтаксическое расширение языка*\\
\scnaddlevel{1}
\scnidtf{расширение синтаксиса языка при сохранении его семантической мощности*}
\scnaddlevel{-1}
}
\bigskip
\scnhaselementset{синтаксическое расширение языка*;направление синтаксического расширения ядра заданного языка*\\
\scnaddlevel{1}
\scnidtf{одно из (или группа) правил синтаксической трансформации текстов заданного языка, расширяющих множество синтаксически правильных текстов этого языка}
\scnnote{Таких направлений синтаксического расширения заданного языка может быть несколько.}
\scnaddlevel{-1}}
\bigskip
\scnhaselementset{
\scnmakesetlocal{синтаксис дискретной информационной конструкции*;синтаксическая структура дискретной информационной конструкции}\\
\scnaddlevel{1}
\scniselement{следует отличать*}
\scnaddlevel{-1};
\scnmakesetlocal{первичная синтаксическая структура дискретной информационной конструкции*;первичная синтаксическая структура дискретной информационной конструкции}\\
\scnaddlevel{1}
\scniselement{следует отличать*}
\scnaddlevel{-1};
\scnmakesetlocal{смысл*;смысл\\
\scnaddlevel{1}
\scnidtf{смысловая информационная конструкция}
\scnaddlevel{-1}}\\
\scnaddlevel{1}
\scniselement{следует отличать*}
\scnaddlevel{-1};
\scnmakesetlocal{синтаксис языка*;синтаксис языка}\\
\scnaddlevel{1}
\scniselement{следует отличать*}
\scnaddlevel{-1};
\scnmakesetlocal{денотационная семантика языка*;денотационная семантика языка}\\
\scnaddlevel{1}
\scniselement{следует отличать*}
\scnaddlevel{-1};
\scnmakesetlocal{операционная семантика языка*;операционная семантика языка}\\
\scnaddlevel{1}
\scniselement{следует отличать*}
\scnaddlevel{-1}
}\bigskip
\scnhaselementset{смысловое представление знака*\\
\scnaddlevel{1}
\scnidtf{Бинарное ориентированное отношение, каждая пара которого связывает фрагмент синтаксической структуры некоторой дискретной информационной конструкции (точнее, фрагмент смыслового представления этой синтаксической структуры) с элементом смыслового представления исходной дискретной информационной конструкции}
\scnaddlevel{-1}
;смысл*\\
\scnaddlevel{1}
\scnidtf{смысловое представление дискретной информационной конструкции*}
\scnaddlevel{-1}
;денотационная семантика языка}\bigskip
\scnhaselementset{информационная трансформация\\
\scnaddlevel{1}
\scnidtf{трансформация информационной конструкции}
\scnidtf{действие по преобразованию (обработке, трансформации) информационной конструкции}
\scnsubset{действие}
\scnaddlevel{-1}
;класс однотипных информационных конструкций;правило выполнения однотипных информационных трансформаций\\
\scnaddlevel{1}
\scnidtf{обобщенная спецификация класса однотипных информационных трансформаций, описывающая обобщенное условие выполнения произвольной трансформации указанного класса, ее результат и соответствие между условием и реультатом}
\scnaddlevel{-1}
;программа выполнения однотипных информационных трансформаций\\
\scnaddlevel{1}
\scnidtf{обобщенная декомпозиция произвольной информационной трансформации заданного класса на систему взаимосвязанных трансформаций более низкого уровня}
\scnaddlevel{-1}
;агент выполнения однотипных информационных трансформаций
}\bigskip

\scnheader{следует отличать*}
\scnhaselementset{денотационная семантика*\\
;описание денотационной семантики*
;операционная семантика*
;описание операционной семантики*}
\scnheader{следует отличать*}
\scnhaselementset{денотационная семантика знака\\
	;денотационная семантика текста
	;денотационная семантика языка
}
\scnheader{следует отличать*}
\scnhaselementset{синтаксис \\
	\scnaddlevel{1}
	\scnidtf{синтаксис языка}
	\scnaddlevel{-1}
		\scnaddlevel{1}
	\scnidtf{отношение, связывающее тексты некоторого языка и соответствующие тексты того же или другого языка, описывающие синтаксическую структуру этих текстов}
	\scnaddlevel{-1}
	;синтаксис*
	\scnaddlevel{1}
\scnidtf{отношение, связывающее язык и его синтаксис}
\scnaddlevel{-1}
	;синтаксическая конструкция текста
		\scnaddlevel{1}
	\scnidtf{информационная конструкция, описывающая синтаксическую структуру текста некоторого языка}
	\scnaddlevel{-1}
	;описание синтаксиса
			\scnaddlevel{1}
	\scnidtf{информационная конструкция, описывающая синтаксис некоторого языка и его свойства, включают правила построения синтаксически корректных конструкций данного языка
}
	\scnaddlevel{-1}
}

\scnheader{язык ostis-системы}
\scnnote{Следует отличать:
	\begin{scnitemize}
		\item саму описываемую сущность;
		\item текст, являющийся описанием некоторой сущности;
		\item текст, являющийся описанием некоторого другого текста, а возможно и самого себя (т.е. текст может быть описываемой сущностью);
		\item знак (обозначение) описываемой сущности в рамках заданного текста;
		\item обозначение описываемой сущности в sc-тексте (это всегда sc-элемент того или иного вида);
		\item коммуникативный (внешний для ostis-системы) уникальный (основной) идентификатор (чаще всего строковый идентификатор-имя), обозначающий соответствующую описываемую сущность и являющийся внешним идентификатором (именем) для соответствующего синонимичного ему sc-элемента. Такие идентификаторы взаимно однозначно соответствуют sc-элементам, которые имеют такие идентификаторы;
		\item вспомогательные (неосновные) внешние идентификаторы sc-элементов. Такие идентификаторы могут обладать и свойством омонимии (когда один идентификатор соответствует нескольким sc-элементам) и синонимии (когда разные идентификаторы соответствует одному sc-элементу);
		\item обозначение описываемой сущности в sc.g-тексте -- это всегда графически представленный sc.g-элемент, являющийся \uline{изображением} соответствующего sc-элемента;
		\item обозначение описываемой сущности в sc.s-предложении и в sc.n-предложении -- это всегда строка символов (либо \uline{омонимичное} изображение sc-коннекторов различного семантического типа, либо \uline{основной} строковый идентификатор, соответствующий некоторому sc-элементу, либо выражение, являющееся \uline{неатомарным} идентификатором, содержащим некоторую информацию о соответствующей именуемой сущности).
	\end{scnitemize}
	
	Подчеркнем, что каждое \uline{обозначение} описываемой сущности в SCg-коде, SCs-коде, SCn-коде рассматривается нами как \uline{изображение} соответствующего ему (синонимичного ему) sc-элемента, обозначающего ту же описываемую сущность. Таким образом, указанные языки (SCg-код, SCs-код; SCn-код) рассматриваются нами как различные варианты изображения текстов SCg-кода.}

\scnnote{Для формального описания различного рода языков, включая рассматриваемые нами языки (SCg-код, SCs-код, SCn-код) используется целый ряд метаязыковых понятий.
	
Перечислим некоторые из них: \textit{идентификатор}, \textit{класс синтаксически эквивалентных идентификаторов}, \textit{имя}, \textit{простое имя}, \textit{выражение}, \textit{внешний идентификатор*}, \textit{алфавит*}, \textit{разделители*}, \textit{ограничители*}, \textit{предложения*}}
\scnnote{Синтаксис \textit{языков представления знаний в ostis-системах} может быть формально описан различными способами. Так, например, можно использовать метаязык Бэкуса-Наура для описания синтаксиса SCs-кода или его расширение для описания синтаксиса SCn-кода. Однако значительно более логично и целесообразно описывать синтаксис всех форм внешнего отображения sc-текстов с помощью самого SC-кода. Такой подход позволит ostis-системам самостоятельно понимать, анализировать и генерировать тексты указанных языков на основе принципов, общих для любых форм внешнего представления информации, в том числе нелинейных.}

\scnheader{алфавит*}
\scnidtf{быть алфавитом для заданного множества текстов*}
\scnidtf{быть семейством максимальных множеств синтаксически однотипных элементарных (атомарных) фрагментов текстов, принадлежащих заданному множеству текстов*}

\scnheader{ограничители*}
\scnidtf{Отношение, связывающее заданный класс информационных конструкций с соответствующим классом их ограничителей}
\scnidtf{быть ограничителями, используемыми в заданном множестве информационных конструкций*}

\scnheader{ограничитель}
\scnsuperset{sc.g-ограничитель}
\scnsuperset{sc.s-ограничитель}
\scnsuperset{sc.n-ограничитель}
\scnsuperset{ограничитель, используемый в ея-файлах ostis-систем}

\scnheader{SCg-код}
\scnrelfrom{ограничители}{sc.g-ограничитель}
\scnaddlevel{1}
\scnidtf{Множество ограничителей, используемых в sc.g-текстах}
\scnaddlevel{-1}

\scnheader{разделители*}
\scnidtf{быть разделителями, используемыми в заданном множестве информационных конструкций*}
\scnrelfrom{второй домен}{разделитель}
\scnaddlevel{1}
\scnsuperset{sc.g-разделитель}
\scnaddlevel{1}
\scneq{sc.g-коннектор}
\scnaddlevel{-1}
\scnsuperset{sc.s-разделитель}
\scnsuperset{sc.n-разделитель}
\scnsuperset{разделитель, используемый в ея-файлах ostis-систем}
\scnaddlevel{-1}

\scnheader{идентификатор}
\scnsuperset{sc.s-идентификатор}
\scnidtf{cтруктурированный знак соответствующей (обозначаемой) сущности, который чаще всего представляет собой строку (цепочку символов), которую будем называть именем соответствующей сущности.} 
\scnnote{В формальных текстах (в том числе текстах SC-кода, SCg-кода, SCs-кода, SCn-кода) основные используемые идентификаторы не должны быть омонимичными, то есть должны \uline{однозначно} соответствовать идентифицируемым сущностям. Следовательно, каждая пара идентификаторов, имеющих \uline{одинаковую} структуру, должны обозначать одну и ту же сущность.}

\scnheader{следует отличать*}
\scnhaselementset{идентификатор\\
	\scnaddlevel{1}
	\scnidtf{Множество всевозможных конкретных \uline{экземпляров}, конкретных вхождений идентификаторов, имеющих различную структуру, во всевозможные тексты}
	\scnaddlevel{-1}
	;класс синтаксически эквивалентных идентификаторов\\
	\scnaddlevel{1}
	\scnidtf{класс идентификаторов, имеющих одинаковую структуру}
	\scnidtf{Семейство всевозможных множеств, каждое из которых является максимальным множеством синтаксически эквивалентных идентификаторов}
	\scnaddlevel{-1}
}

\scnheader{имя}
\scnsubset{идентификатор}
\scnidtf{строковый идентификатор}
\scnidtf{идентификатор, представляющий собой строку (цепочку) символов}
\scnsubdividing{простое имя\\
	\scnaddlevel{1}
	\scnidtf{атомарное имя}
	\scnidtf{имя, в состав которого другие имена не входят}
	\scnaddlevel{-1}
	;выражение\\
	\scnaddlevel{1}
	\scnidtf{неатомарное имя}
	\scnaddlevel{-1}
}
\scnsuperset{sc.s-идентификатор}

\scnheader{внешний идентификатор*}
\scniselement{отношение}
\scnidtf{Бинарное ориентированное отношение, каждая связка (sc-дуга) которого связывает некоторый элемент с файлом, содержимым которого является внешний идентификатор (чаще всего, имя), соответствующий указанному элементу}
\scnidtf{быть внешним идентификатором*}
\scnidtf{внешний идентификатор sc-элемента*}
\scnrelfrom{второй домен}{идентификатор}
\scnnote{Понятие внешнего идентификатора является понятием относительным и важным для ostis-систем, поскольку внутреннее для ostis-систем представление информации (в виде текстов SC-кода) оперирует не идентификаторами описываемых сущностей, а знаками, структура которых никакого значения не имеет}

\scnheader{следует отличать*}
\scnhaselementset{sc-элемент, обозначающий файл ostis-системы;sc.g-элемент, обозначающий файл ostis-системы;простой sc.s-идентификатор, обозначающий файл ostis-системы\\
	\scnaddlevel{1}
	\scnidtf{простое имя файла ostis-системы}
	\scnaddlevel{-1}
	;изображение файла ostis-системы, ограниченное sc.g-рамкой;изображение файла ostis-системы, ограниченное sc.n-рамкой;изображение строки символов, ограниченное квадратными скобками}
\scnheader{следует отличать*}
\scnhaselementset{файл-экземпляр;файл-класс\\
	\scnaddlevel{1}
	\scnidtf{файл, обозначающий класс файлов-экземпляров, синтаксически эквивалентных заданному образцу}
	\scnaddlevel{-1}
}

\scnheader{предложения*}
\scnidtf{быть множеством всех предложений заданного текста, не являющихся встроенными предложениями, то есть предложениями, входящими в состав других предложений*}
\scnrelfrom{второй домен}{предложение}

\scnheader{предложение}
\scnexplanation{минимальный семантически целостный фрагмент текста, представляющий собой конфигурацию знаков, входящих в этот фрагмент и связываемых между собой отношениями инцидентности (в частности, отношением непосредственной последовательности в строке), а также различного вида разделителями и ограничителями}

\scnheader{sc.g-текст}
\scnidtf{текст SCg-кода}
\scnsubdividing{семантически связный sc.g-текст\\
	\scnaddlevel{1}
	\scnidtf{sc.g-текст, который семантически эквивалентен \uline{связному} sc-тексту}
	\scnaddlevel{-1}
	;семантически несвязный sc.g-текст}
\scnsubdividing{синтаксически связный sc.g-текст;синтаксически несвязный sc.g-текст}

\scnheader{sc.s-текст}
\scnidtf{текст SCs-кода}
\scnsubdividing{семантически связный sc.s-текст\\
	\scnaddlevel{1}
	\scnidtf{sc.s-текст, который семантически эквивалентен \uline{связному} sc-тексту}
	\scnaddlevel{-1}
	;семантически несвязный sc.s-текст}
\scnsubdividing{синтаксически связный sc.s-текст;синтаксически несвязный sc.s-текст}

\scnheader{sc.n-текст}
\scnidtf{текст SCn-кода}
\scnsubdividing{семантически связный sc.n-текст\\
	\scnaddlevel{1}
	\scnidtf{sc.n-текст, который семантически эквивалентен \uline{связному} sc-тексту}
	\scnaddlevel{-1}
	;семантически несвязный sc.n-текст\\}
\scnsubdividing{синтаксически связный sc.n-текст\\
	;синтаксически несвязный sc.n-текст\\
}

\scnheader{следует отличать*}
\scnhaselementset{денотационная семантика*;описание денотационной семантики*;операционная семантика*;описание операционной семантики*}
\bigskip
\scnhaselementset{денотационная семантика знака;денотационная семантика текста;денотационная семантика языка}
\bigskip
\scnhaselementset{синтаксис
	\scnaddlevel{1}
	\scnidtf{синтаксис языка}
	\scnidtf{отношение, связывающее тексты некоторого языка и соответствующие тексты того же или другого языка, описывающие синтаксическую структуру этих текстов}
	\scnaddlevel{-1}
	;синтаксис*
	\scnaddlevel{1}
	\scnidtf{Отношение, связывающее язык и его синтаксис}
	\scnaddlevel{-1}
	;синтаксическая структура текста
	\scnaddlevel{1}
	\scnidtf{информационная конструкция, описывающая синтаксическую структуру текста некоторого языка}
	\scnaddlevel{-1}
	;описание синтаксиса
	\scnaddlevel{1}
	\scnidtf{информационная конструкция, описывающая синтаксис некоторого языка и его свойства, включая правила построения синтаксически корректных конструкций данного языка}
	\scnaddlevel{-1}
}


\bigskip
\scnendstruct \scnendcurrentsectioncomment

\end{SCn}
\bigskip

\scsubsection[\scnidtf{Предметная область и онтология языка внутреннего представления информационных конструкций в памяти ostis-систем}\protect\scnmonographychapter{Глава 2.1. Универсальный язык смыслового представления знаний и смысловое пространство}]{Предметная область и онтология внутреннего языка ostis-систем}
\label{intro_sc_code}

\begin{SCn}

\scnsectionheader{\currentname}

\scnstartsubstruct

\scnreltovector{конкатенация сегментов}{Основные положения внутреннего языка ostis-систем;Описание Ядра SC-кода;SC-код как синтаксическое расширение Ядра SC-кода;Использование SC-кода для формального описания собственного синтаксиса}

\scnsegmentheader{Основные положения внутреннего языка ostis-систем}
\scnstartsubstruct

\scnheader{SC-код}
\scnidtf{Язык унифицированного смыслового представления знаний в памяти \textit{интеллектуальных компьютерных систем}}
\scnidtf{Внутренний язык \textit{ostis-систем}}
\scnrelto{внутренний язык}{ostis-система}
\scntext{эпиграф}{Информация содержится не в самих знаках, а в конфигурации связей между ними.}
\scntext{эпиграф}{Он вскочил на коня и поскакал во все стороны.}

\scntext{основной внешний идентификатор sc-элемента}{\textbf{SC-код}}
\scnaddlevel{1}
\scniselement{имя собственное}
\scnaddlevel{-1}
\scntext{часто используемый неосновной внешний идентификатор sc-элемента}{sc-текст}
\scnaddlevel{1}
\scniselement{имя нарицательное}
\scnaddlevel{-1}
\scniselement{абстрактный язык}
\scnaddlevel{1}
    \scnidtf{Язык, для которого не уточняется способ представления символов (синтаксически элементарных фрагментов), входящих в состав текстов этого языка, а задается только \uline{алфавит*} этих символов, то есть семейство классов символов, считающихся синтаксически эквивалентными друг другу.}
    \scnaddlevel{1}
        \scnnote{Каждому абстрактному языку можно поставить в соответствие целое семейство \textit{реальных языков}, обеспечивающих \uline{изоморфное} реальное представление текстов указанного абстрактного языка путем уточнения способов представления (изображения, кодирования) символов, входящих в состав этих текстов, а также путем уточнения правил установления синтаксической эквивалентности этих символов. Очевидно, что во всём остальном синтаксис и денотационная семантика указанных реальных языков полностью совпадает с синтаксисом и денотационной семантикой соответствующего абстрактного языка.}
        \scnaddlevel{1}
            \scnnote{Для \textit{SC-кода} как абстрактного языка необходима разработка как минимум трех синтаксически и семантически эквивалентных ему реальных языков: (1) язык кодирования текстов \textit{SC-кода} в памяти традиционных компьютеров; (2) язык кодирования текстов \textit{SC-кода} в семантической ассоциативной памяти; (3) \textit{Ядро SCg-кода} -- язык, синтаксически и семантически эквивалентный \textit{SC-коду} и обеспечивающий графическое представление текстов \textit{SC-кода}.}
        \scnaddlevel{-1}
    \scnaddlevel{-1}
\scnaddlevel{-1}
\scniselement{графовый язык}
\scnaddlevel{1}
    \scnexplanation{язык, каждый текст которого 
    \begin{scnitemize}
    \item задается множеством входящих в него элементарных фрагментов (символов), которое, в свою очередь, состоит
    \begin{scnitemizeii}
    \item из множества узлов (вершин), возможно, синтаксически разного вида
    \item из множества связок, которые также могут принадлежать разным синтаксически выделяемым классам
    \end{scnitemizeii}
    \item задается в общем случае несколькими отношениями инцидентности связок с компонентами этих связок (при этом указанными компонентами в общем случае могут быть не только вершины, но и другие связки)
    \end{scnitemize}
    }
\scnaddlevel{-1}

\scnidtf{Универсальный язык, обеспечивающий внутреннее представление и структуризацию \uline{всех}(!), используемых ostis-системой в процессе своего функционирования.}
\scnidtf{Универсальный язык, являющийся результатом унификации (уточнения) синтаксиса и денотационной семантики семантических сетей.}
\scnaddlevel{1}
    \scnexplanation{Универсальность SC-кода обеспечивается и тем, что элементами текстов SC-кода могут быть знаки описываемых сущностей \uline{любого} вида, в том числе, и  знаки связей между описываемыми сущностями и/или их знаками}
    \scnaddlevel{1}
        \scntext{следствие}{Тексты SC-кода являются графовыми структурами расширенного вида, в которых знаки описываемых связей могут соединять не только вершины (узлы) графовой структуры, но и знаки других связей.}
    \scnaddlevel{-1}
\scnaddlevel{-1}
\scnidtf{Базовый универсальный язык внутреннего представления знаний в памяти ostis-систем.}
\scnidtf{Базовый внутренний язык ostis-систем.}
\scnidtf{Максимальный внутренний язык ostis-систем, по отношению к которому все остальные (специализированные) внутренние языки являются его подъязыками (подмножествами)}
\scnidtf{Множество всевозможных текстов SC-кода}
\scnaddlevel{1}
\scniselement{имя собственное}
\scnaddlevel{-1}
\scnidtf{текст SC-кода}
\scnaddlevel{1}
\scniselement{имя нарицательное}
\scnaddlevel{-1}


\filemodetrue
\scnrelfromvector{принципы, лежащие в основе}{\textit{Знаки} (обозначения) всех \textit{сущностей}, описываемых в \textit{sc-текстах} (текстах \textit{\textbf{SC-кода}}) представляются в виде синтаксически элементарных (атомарных) фрагментов \textit{sc-текстов} и, следовательно, не имеющих внутренней структуры, не состоящих из более простых фрагментов \textit{текста}, как, например, имена (термины), которые представляют \textit{знаки} описываемых \textit{сущностей} в привычных \textit{языках} и состоят из \textit{букв}.;\textit{Имена} (термины), \textit{естественно-языковые тексты} и другие информационные конструкции, не являющиеся \textit{sc-текстами}, могут входить в состав \textit{sc-текста}, но только как \textit{файлы}, описываемые (специфицируемые) \textit{sc-текстами}. Таким образом, в состав базы знаний \textit{интеллектуальной компьютерной системы}, построенной на основе \textit{\textbf{SC-кода}}, могут входить \textit{имена} (термины), обозначающие некоторые описываемые \textit{сущности} и представленные соответствующими \textit{файлами}. Каждый \mbox{\textit{sc-элемент}} будем называть внутренним обозначением некоторой \textit{сущности}, а \textit{имя} этой \textit{сущности}, представленное соответствующим файлом, будем называть \textit{внешним идентификатором} (внешним обозначением) этой \textit{сущности}. При этом каждый именуемый (идентифицируемый) \mbox{\textit{sc-элемент}} связывается дугой, принадлежащей отношению ``быть \textit{\textbf{внешним идентификатором*}}~'', с \textit{узлом}, содержимым которого является \textit{файл} идентификатора (в частности, \textit{имени}), обозначающего ту же \textit{сущность}, что и указанный выше \textit{sc-элемент}. \textit{Внешним идентификатором} может быть не только \textit{имя} (термин), но и иероглиф, пиктограмма, озвученное имя, жест. Особо отметим, что \textit{внешние идентификаторы} описываемых \textit{сущностей} в \textit{интеллектуальной компьютерной системе}, построенной на основе \textit{\textbf{SC-кода}}, используются только (1) для анализа информации, поступающей в эту систему из вне из различных источников, и ввода (понимания и погружения) этой информации в \textit{базу знаний}, а также (2) для синтеза различных \textit{сообщений}, адресуемых различным субъектам (в т.ч. пользователям).;Тексты \textit{\textbf{SC-кода}} (\textit{sc-тексты}) имеют в общем случае нелинейную (графовую) структуру, поскольку (1) \textit{знак} каждой описываемой сущности входит в состав \textit{sc-текста} однократно и (2) каждый такой \textit{знак} может быть инцидентен неограниченному числу других \textit{знаков}, поскольку каждая описываемая \textit{сущность} может быть связана неограниченным числом связей с другими описываемыми \textit{сущностями}.;
\textit{База знаний}, представленная текстом \textit{\textbf{SC-кода}}, является \textit{графовой структурой} специального вида, алфавит элементов которой включает в себя множество \textit{узлов}, множество \textit{ребер}, множество \textit{дуг}, множество \textit{базовых дуг} -- дуг специально выделенного типа, обеспечивающих структуризацию \textit{баз знаний}, а также множество специальных \textit{узлов}, каждый из которых имеет содержимое, являющееся \textit{файлом}, хранящимся в памяти \textit{интеллектуальной компьютерной системы}. Структурная особенность данной \textit{графовой структуры} заключается в том, что ее \textit{дуги} и \textit{ребра} могут связывать не только \textit{узел} с \textit{узлом}, но и \textit{узел} с \textit{ребром} или \textit{дугой}, \textit{ребро} или \textit{дугу} с другим \textit{ребром} или \textit{дугой}.;
\uline{Все элементы} (\textit{sc-элементы}) указанной выше \textit{графовой структуры} (текста \textit{\textbf{SC-кода}}), т.е. все ее узлы (\textit{sc-узлы}), ребра (\textit{sc-ребра}) и дуги (\textit{sc-дуги}) являются обозначениями различных сущностей. При этом ребро является обозначением бинарной неориентированной связки между двумя сущностями, каждая из которых либо представлена в рассматриваемой графовой структуре соответствующим знаком, либо является самим этим знаком. Дуга является обозначением бинарной ориентированной связки между двумя сущностями. Дуга специального вида (\textit{\textbf{базовая дуга}}) является знаком связи между узлом, обозначающим некоторое множество элементов рассматриваемой графовой структуры, и одним из элементов этой графовой структуры, который принадлежит указанному множеству. Узел, имеющий содержимое (узел, для которого содержимое существует, но может в текущий момент быть неизвестным) является знаком файла, который является содержимым этого узла. Узел, не являющийся знаком файла, может обозначать какой-либо материальный объект, первичный абстрактный объект(например, число, точку в некотором абстрактном пространстве), какую-либо бинарную связь, какое-либо множество (в частности, понятие, структуру, ситуацию, событие, процесс). При этом сущности, обозначаемые элементами рассматриваемой графовой структуры, могут быть постоянными (существующими всегда) и временными (сущностями, которым соответствует отрезок времени их существования). Кроме того, сущности, обозначаемые элементами рассматриваемой графовой структуры, могут быть константными (конкретными) сущностями и переменными (произвольными) сущностями. Каждому элементу рассматриваемой графовой структуры, являющемуся обозначением переменной сущности, ставится в соответствие область возможных значений этого обозначения. Область возможных значений каждого переменного ребра является подмножеством множества всевозможных константных ребер, область возможных значений каждой переменной дуги является подмножеством множества всевозможных константных дуг, область возможных значений каждого переменного узла является подмножеством множества всевозможных константных узлов.;
В рассматриваемой графовой структуре, являющейся представлением базы знаний в \textit{\textbf{SC-коде}}, могут, но не должны существовать разные элементы графовой структуры, обозначающие одну и ту же сущность. Если пара таких элементов обнаруживается, то эти элементы склеиваются (отождествляются). Таким образом, синонимия внутренних обозначений в базе знаний интеллектуальной компьютерной системы, построенной на основе \textit{\textbf{SC-кода}}, запрещена. При этом синонимия внешних обозначений считается нормальным явлением. Формально это означает, что из некоторых элементов рассматриваемой графовой структуры выходит несколько дуг, принадлежащих отношению ``быть \textit{\textbf{внешним идентификатором*}}~''. Из всех указанных дуг, принадлежащих отношению ``быть \textit{\textbf{внешним идентификатором*}}~'' и выходящих из одного элемента рассматриваемой графовой структуры, обязательно выделяется одна (очень редко две) путем включения их в число дуг, принадлежащих отношению ``быть \textit{\textbf{основным внешним идентификатором*}}~''. Это означает, что указываемый таким образом внешний идентификатор не является омонимичным, т.е. не может быть использован как внешний идентификатор, соответствующий другому элементу рассматриваемой графовой структуры.;
Кроме файлов, представляющих различные внешние обозначения (имена, иероглифы, пиктограммы), в памяти интеллектуальной компьютерной системе, построенной на основе \textit{\textbf{SC-кода}}, могут хранится файлы различных текстов (книг, статей, документов, примечаний, комментариев, пояснений, чертежей, рисунков, схем, фотографий, видео-материалов, аудио-материалов).;
\uline{Любую сущность}, требующую описания, в тексте \textit{\textbf{SC-кода}} можно обозначить в виде \textit{sc-элемента}. Это являетс яодним из факторов, обеспечивающих универсальность \textit{\textbf{SC-кода}}. Особо подчеркнем, что sc-элементы являются не просто обозначениями различных описываемых сущностей, а обозначениями, которые являются элементарными (атомарными) фрагментами знаковой конструкции, т.е. фрагментами, детализация структуры которых не требуется для "прочтения"{} и понимания этой знаковой конструкции.;
Текст \textit{\textbf{SC-кода}}, как и любая другая графовой структура, является абстрактным математическим объектом, не требующим детализации (уточнения) его кодирования в памяти компьютерной системы (например, в виде матрицы смежности, матрицы инцидентности, списковой структуры). Но такая детализация потребуется для технической реализации памяти, в которой хранятся и обрабатываются \textit{sc-тексты}.;
Важнейшим дополнительным свойством \textit{\textbf{SC-кода}} является то,что он удобен не просто для внутреннего представления знаний в памяти интеллектуальной компьютерной системы, но также и для внутреннего представления информации в памяти компьютеров, специально предназначенных для интерпретации семантических моделей интеллектуальных компьютерных систем. Т.е., \textit{\textbf{SC-код}} определяет синтаксические, семантические и функциональные принципы организации памяти компьютеров нового поколения, ориентированных на реализацию интеллектуальных компьютерных систем, -- принципы организации графодинамической ассоциативной семантической памяти.;
\textit{\textbf{SC-код}} рассматривается нами как объединение трех его подъязыков, в число которых входит \textit{\textbf{Ядро SC-кода}}, подъязык \textit{\textbf{SC-кода}}, обеспечивающий представление текстов \textit{\textbf{SC-кода}} (\textit{sc-текстов}) в форме орграфов классического вида, являющихся подразбиениями текстов \textit{\textbf{Ядра SC-кода}} и, соответственно, использующих \uline{явное} представление пар инцидентности элементов sc-текстов (sc-элементов), синтаксическое \textit{\textbf{Расширение Ядра SC-кода}}, обеспечивающее представление в памяти ostis-системы информационных конструкций инородного для \textit{\textbf{SC-кода}} вида.
}
\filemodefalse 
\scnaddlevel{1}
\scninlinesourcecommentpar{Завершили Описание принципов, лежащих в основе \textit{SC-кода}}
\scnaddlevel{-1}

\scnheader{SC-код}
\scnnote{Следует особо подчеркнуть, что  унификация и максимально возможное упрощение  \textbf{\textit{синтаксиса}} и \textbf{\textit{денотационной семантики}} внутреннего языка интеллектуальных компьютерных систем прежде всего необходимы потому, что подавляющий объем \textbf{\textit{знаний}}, хранимых в составе  базы знаний интеллектуальной компьютерной системы, представляют собой \textbf{\textit{метазнания}}, описывающие свойства других знаний. К \textit{метазнаниям}, в частности, следует отнести и различного вида логические высказывания и всевозможного вида программы, описания методов (навыков), обеспечивающих решение различных классов задач. Необходимо исключить зависимость формы представляемого знания от вида этого знания. Форма (структура) внутреннего представления знания любого вида должна зависеть \uline{только}(!) от смысла этого знания. Более того, конструктивное (формальное) развитие теории интеллектуальных компьютерных систем невозможно без уточнения (унификации, стандартизации) и обеспечения семантической совместимости различных видов знаний, хранимых в базе знаний интеллектуальной компьютерной  системы.  Очевидно, что многообразие форм представления семантически эквивалентных знаний делает разработку общей теории  интеллектуальных компьютерных систем практически невозможной.}

\scnheader{SC-пространство}
\scnnote{Понятие \textit{SC-пространства} наряду с понятием \textit{SC-кода} играет важнейшую роль для уточнения и формализации понятия смысла информационных конструкций, для унификации смыслового представления информации и для максимально возможного исключения субъективизма в трактовке понятия смысла. Смысл информационной конструкции в конечном счете определяется (1) конфигурацией смыслового представления этой конструкции и (2) и "местоположением" (контекстом) смыслового представления указанной информационной конструкции в рамках смыслового пространства, т.е. в рамках объединенного смыслового представления \uline{всевозможных} информационных конструкций, либо в рамках объединенного смыслового представления информации, накопленной к заданному моменту времени некоторым индивидуальным субъектом или коллективом субъектов. Таким объединенным смысловым представлением информации, в частности, является смысловое представление глобальной базы всех знаний, накопленных человечеством к текущему моменту.}
\scnexplanation{Объединение (вместилище) \uline{всевозможных} унифицированных семантических сетей (текстов SC-кода)}
\scnaddlevel{1}
    \scnnote{При теоретико-множественном объединении текстов \textit{SC-кода} семантически эквивалентные (синонимичные) элементы (синтаксически элементарные фрагменты) этих текстов считаются совпадающими элементами и при объединении указанных текстов "склеиваются".}
\scnaddlevel{-1}
\scnrelto{объединение}{SC-код}
\scnidtf{Унифицированное смысловое пространство}
\scntext{достоинство}{Важнейшим достоинством \textit{SC-пространства} является возможность уточнения таких понятий, как понятие аналогичности (сходства и отличия) различных описываемых "внешних" сущностей, аналогичности унифицированных семантических сетей (текстов \textit{SC-кода}), понятие семантической близости описываемых сущностей (в том числе, и текстов \textit{SC-кода}).}

\bigskip
\scnendstruct \scnendsegmentcomment{Основные положения внутреннего языка ostis-систем}

\newpage
\scnsegmentheader{Описание Ядра SC-кода}
\scnstartsubstruct

\scnstructheader{Синтаксис Ядра SC-кода}
\scnstartsubstruct
\scnheader{Синтаксис Ядра SC-кода}
\scnnote{\textit{Синтаксис Ядра SC-кода} задается: (1) \textit{Алфавитом Ядра SC-кода}, (2) Отношением \textit{инцидентности sc-коннекторов*}, (3) Отношением \textit{инцидентности входящих sc-дуг*}}
\scnrelto{синтаксис}{Ядро SC-кода}

\scnheader{Ядро SC-кода}
\scnrelfrom{множество всех элементов конструкций данного языка}{sc-элемент}
\scnaddlevel{1}
    \scnidtf{элемент конструкции \textit{Ядра SC-кода}}
    \scnidtf{синтаксически элементарный (атомарный) фрагмент дискретной информационной конструкции, принадлежащей \textit{Ядру SC-кода}}
    \scnidtf{Класс элементов конструкций \textit{Ядра SC-кода}}
    \scnidtf{Множество всех элементов всевозможных конструкций \textit{Ядра SC-кода}}
\scnaddlevel{-1}
\scnrelfrom{алфавит}{Алфавит Ядра SC-кода\scnsupergroupsign}

\scnheader{Алфавит Ядра SC-кода\scnsupergroupsign}
\scnidtf{Множество (Семейство) всех классов синтаксически эквивалентных sc-элементов Ядра SC-кода}
\scnidtf{класс синтаксически эквивалентных sc-элементов Ядра SC-кода}
\scnidtf{класс синтаксически эквивалентных элементов конструкций Ядра SC-кода}
\scnidtf{элемент Алфавита Ядра SC-кода}
\scnidtf{синтаксический тип sc-элемента Ядра SC-кода}
\scneqtoset{sc-узел общего вида;sc-ребро общего вида;sc-дуга общего вида;базовая sc-дуга}

\scnheader{sc-элемент}
\scnrelfrom{разбиение}{Алфавит Ядра SC-кода\scnsupergroupsign}
\scnaddlevel{1}
    \scnnote{\textit{Алфавит Ядра SC-кода} является одним из признаков классификации sc-элементов.}
    \scnnote{В процессе обработки текстов \textit{Ядра SC-кода} синтаксический тип \textit{sc-элементов} может меняться -- \textit{sc-узел} может трансформироваться в \textit{sc-ребро}, \textit{sc-ребро} -- в \textit{sc-дугу}, \textit{sc-дуга} общего вида -- в \textit{базовую sc-дугу}.}
\scnaddlevel{-1}

\scnheader{синтаксически выделяемый класс sc-элементов в рамках Ядра SC-кода\scnsupergroupsign}
\scnidtf{класс \textit{sc-элементов}, определяемый на основе \textit{Алфавита Ядра SC-кода}}
\scnhaselement{sc-коннектор}
\scnhaselement{sc-дуга}
\scnsuperset{Алфавит Ядра SC-кода\scnsupergroupsign}

\scnheader{sc-дуга}
\scnsubdividing{sc-дуга общего вида;базовая sc-дуга}

\scnheader{sc-коннектор}
\scnsubdividing{sc-ребро общего вида;sc-дуга общего вида}

\scnstructheader{Синтаксическая классификация sc-элементов в рамках Ядра SC-кода}
\scnstartsubstruct

\scnheaderlocal{sc-элемент}
\scnsubdividing{sc-узел общего вида;sc-коннектор\\
    \scnaddlevel{1}
    \scnsubdividing{sc-ребро общего вида;sc-дуга\\
    \scnaddlevel{1}
        \scnsubdividing{sc-дуга общего вида;базовая sc-дуга}
    \scnaddlevel{-1}
    }
    \scnaddlevel{-1}
    }
\bigskip
\scnendstruct

\scninlinesourcecommentpar{Завершили представление \textit{Синтаксической классификации sc-элементов в рамках Ядра SC-кода}}

\scnnote{Все классы \textit{sc-элементов}, входящие в состав синтаксической классификации sc-элементов являются синтаксически выделяемыми классами \textit{sc-элементов}}

\scnheader{инцидентность sc-коннекторов*}
\scnidtfdef{Бинарное ориентированное отношение, первым компонентом каждой ориентированной пары которого является некоторый sc-коннектор, а вторым компонентом является один из sc-элементов, соединяемых указанным sc-коннектором с некоторым другим sc-элементом, который указывается в другой паре инцидентности для этого же sc-коннектора}

\scnheader{инцидентность входящих sc-дуг*}
\scnidtfdef{Бинарное ориентированное отношение, первым компонентом каждой ориентированной пары которого является некоторая sc-дуга, а вторым компонентом --- sc-элемент, в который указанная sc-дуга входит, т.е. sc-элемент, который является вторым компонентом, соединяемым (связываемым) указанной sc-дугой}

\scnheader{Ядро SC-кода}
\scnrelfrom{синтаксические правила}{\scnstructidtf{Синтаксические правила Ядра SC-кода}}
\scnaddlevel{1}
\scnhassubstruct{
\scnstartsetlocal\\
\scnaddlevel{1}
\scnheaderlocal{инцидентность sc-коннекторов*}\\
\scnsuperset{инцидентность входящих sc-дуг*}
\scniselement{бинарное ориентированное отношение}
\scnendstruct
\scnaddlevel{-1}
;\scnfileitem{Для каждого sc-коннектора существует две и только две пары \textit{инцидентности sc-коннекторов*}, указанный sc-коннектор является первым связующим компонентом. При этом для каждой sc-дуги из двух указанных пар инцидентности \uline{одна} должна принадлежать отношению инцидентности \textit{входящей sc-дуги*}.}
;\scnfileitem{Пары инцидентности sc-коннекторов могут быть \uline{кратными}. То есть sc-коннектор может соединять (связывать) sc-элемент с самим собой. Такие sc-коннекторы будем называть петлевыми sc-коннекторами (петлевыми sc-ребрами и петлевыми sc-дугами).}
;\scnfileitem{Само \textit{Отношение инцидентности sc-коннекторов*} и, следовательно, \textit{Отношение инцидентности входящих sc-дуг*} не имеет кратных пар инцидентности. То есть sc-коннектор не может быть инцидентен самому себе.}
\newpage
;\scnfileitem{В область определения \textit{Отношения инцидентности sc-коннекторов*} и \textit{Отношения инцидентности входящих sc-дуг*} входят не только sc-узлы общего вида, но и sc-коннекторы. Это значит, что sc-коннектор может соединять (связывать) не только sc-узел с sc-узлом, но также sc-узел с sc-коннектором и даже sc-коннектор с sc-коннектором.}}
\scnaddlevel{1}
\scninlinesourcecommentpar{Завершили перечень синтаксических правил Ядра SC-кода}
\scnaddlevel{-2}
\bigskip
\scnendstruct \scninlinesourcecommentpar{Завершили изложение \textit{Синтаксиса Ядра SC-кода}}

\scnstructheader{Денотационная семантика Ядра SC-кода}
\scnstartsubstruct
\scnheader{Ядро SC-кода}
\scnrelfrom{денотационная семантика}{Денотационная семантика Ядра SC-кода}
\scnaddlevel{1}
    \scnidtf{Описание соответствия информационных конструкций, принадлежащих \textit{Ядру SC-кода}, и сущностей, описываемых этими конструкциями}
\scnaddlevel{-1}

\scnheader{параметр, заданный на множестве sc-элементов}
\scnhaselement{Алфавит Ядра SC-кода\scnsupergroupsign}
\scnhaselement{Алфавит SC-кода\scnsupergroupsign}
\scnhaselement{Структурная типология sc-элементов\scnsupergroupsign}
\scnhaselement{Типология sc-элементов по признаку константности\scnsupergroupsign}
\scnhaselement{Типология sc-элементов по признаку постоянства обозначаемой сущности\scnsupergroupsign}
\scnhaselement{Типология sc-элементов по признаку доступности sc-элемента в процессе эксплуатации и эволюции базы знаний\scnsupergroupsign}

\scnstructheader{Семантическая классификация sc-элементов}
\scnstartsubstruct

\scnheader{sc-элемент}
\scnidtf{обозначение описываемой сущности}
\scnrelfrom{разбиение}{\scnkeyword{Структурная типология sc-элементов\scnsupergroupsign}}
\scnaddlevel{1} 
    \scneqtoset{обозначение терминальной сущности\\
    \scnaddlevel{1} 
    \scnsubdividing{обозначение материальной сущности\\
    \scnaddlevel{1} 
        \scnnote{К материальным сущностям относятся физические тела, поля, биологические объекты, технические системы и многое другое.}
    \scnaddlevel{-1}
    ;обозначение абстрактной терминальной сущности\\
    \scnaddlevel{1} 
        \scnnote{Примерами абстрактных терминальных сущностей являются предельно малые физические тела, точки различных пространств, числа.}
    \scnaddlevel{-1}
    ;обозначение дискретной информационной конструкции, не принадлежащей SC-коду\\
    \scnaddlevel{1} 
        \scnidtf{обозначение информационной конструкции, не являющейся конструкцией \textit{SC-кода} и тем более \textit{Ядра SC-кода}}
        \scnidtf{обозначение "инородной"{} для \textit{SC-кода} информационной конструкции}
        \scnsuperset{обозначение файла}
        \scnaddlevel{1}
            \scnidtf{обозначение внешней информационной конструкции, представленной в электронной форме}
        \scnaddlevel{-1}
    \scnaddlevel{-1}
    }
    \scnaddlevel{-1}
    ;обозначение множества\\
    \scnaddlevel{1}
    \scnsubdividing{обозначение связки;обозначение класса;обозначение структуры}
    \scnaddlevel{-1}
    }
\scnaddlevel{-1}

\scnheader{обозначение связки}
\scnsubdividing{обозначение небинарной связки;обозначение sc-пары\\
\scnaddlevel{1}
    \scnsubdividing{обозначение неориентированной sc-пары;обозначение ориентированной пары неизвестной направленности;обозначение ориентированной sc-пары\\
    \scnaddlevel{1}
        \scnsuperset{обозначение sc-пары принадлежности}
        \scnaddlevel{1}
            \scnsubdividing{обозначение позитивной sc-пары принадлежности;обозначение негативной sc-пары принадлежности;обозначение нечеткой sc-пары принадлежности}
        \scnaddlevel{-1}
    \scnaddlevel{-1}
    }
\scnaddlevel{-1}
}

\scnheader{sc-элемент}
\scnrelfrom{разбиение}{\scnkeyword{Типология sc-элементов по признаку константности\scnsupergroupsign}}
\scnaddlevel{1}
    \scneqtoset{sc-константа\\
    \scnaddlevel{1}
        \scnidtf{константный sc-элемент}
        \scnidtf{обозначение конкретной (фиксированной) сущности}
    \scnaddlevel{-1}
    ;sc-переменная\\
    \scnaddlevel{1}
        \scnidtf{переменный sc-элемент}
        \scnidtf{обозначение произвольной сущности из некоторого множества сущностей}
        \scnidtf{sc-элемент, имеющий (принимающий) произвольное значение из некоторого множества sc-элементов}
    \scnaddlevel{-1}
    }
\scnaddlevel{-1}
\scnrelfrom{разбиение}{\scnkeyword{Типология sc-элементов по постоянству обозначаемых сущностей\scnsupergroupsign}}
\scnaddlevel{1}
    \scneqtoset{обозначение постоянной сущности;обозначение временной сущности\\
    \scnaddlevel{1}
        \scnidtf{обозначение нестационарной сущности, факт существования которой зависит от времени}
        \scnsubdividing{обозначение прошлой сущности\\
        \scnaddlevel{1}
            \scnidtf{обозначение сущности, существовавшей до текущего момента времени}
        \scnaddlevel{-1}
        ;обозначение настоящей сущности\\
        \scnaddlevel{1}
            \scnidtf{обозначение сущности, существующей в текущий момент времени}
        \scnaddlevel{-1};обозначение будущей сущности\\
        \scnaddlevel{1}
            \scnidtf{обозначение сущности, существование которой прогнозируется или планируется в будущем}
        \scnaddlevel{-1}}
    \scnaddlevel{-1}
    }
\scnaddlevel{-1}
\scnaddhind{1}
\scnrelfrom{разбиение}{\scnkeyword{Типология sc-элементов по признаку доступности sc-элемента в процессе эксплуатации и эволюции базы знаний\scnsupergroupsign}}
\scnaddhind{-1}
\scnaddlevel{1}
    \scneqtoset{удаленный sc-элемент\\
        \scnaddlevel{1}
            \scnidtf{sc-элемент, считающийся логически удаленным, но присутствующим в описании истории эксплуатации и эволюции базы знаний}
        \scnaddlevel{-1};настоящий sc-элемент\\
        \scnaddlevel{1}
            \scnidtf{sc-элемент, входящий в состав эксплуатируемой части базы знаний}
        \scnaddlevel{-1};будущий sc-элемент\\
        \scnaddlevel{1}
            \scnidtf{sc-элемент, планируемый для включения в состав эксплуатируемой части базы знаний}
        \scnaddlevel{-1}}
\scnaddlevel{-1}

\newpage
\scnheader{обозначение множества}
\scnidtf{обозначение множества sc-элементов}
\scnsubdividing{произвольное множество\\
    \scnaddlevel{1}
        \scnidtf{sc-переменная, обозначающая произвольное множество из некоторого семейства множеств}
        \scnidtf{переменное множество}
    \scnaddlevel{-1}
;множество\\
    \scnaddlevel{1}
        \scnidtf{конкретное (константное, фиксированное) множество sc-элементов}
    \scnaddlevel{-1}}

\scnheader{множество}
\scnidtf{множество sc-элементов}
\scnsubdividing{множество sc-констант\\
    \scnaddlevel{1}
        \scnidtf{множество, элементами которого являются только sc-константы}
    \scnaddlevel{-1}
    ;множество sc-переменных\\
    \scnaddlevel{1}
        \scnidtf{множество, элементами которого являются только sc-переменные}
        \scnsuperset{sc-переменная}
        \scnaddlevel{1}
            \scnidtf{множество, элементами которого являются всевозможные sc-переменные и только они}
            \scnsuperset{произвольное множество}
            \scnaddlevel{1}
                \scnidtf{sc-переменная, значениями которой являются всевозможные обозначения множеств и только они}
            \scnaddlevel{-1}
        \scnaddlevel{-1}
    \scnaddlevel{-1}
    ;множество sc-констант и sc-переменных\\
    \scnaddlevel{1}
        \scnidtf{множество, в число элементов которого входят как sc-константы, так и sc-переменные}
        \scnsuperset{обозначение множества}
            \scnaddlevel{1}
                \scnidtf{множество, элементами которого являются всевозможные \mbox{sc-переменные} и \mbox{sc-константы}, обозначающие множества и только они}
            \scnaddlevel{-1}
    \scnaddlevel{-1}}

\scnheader{обозначение связки}
\scnidtf{обозначение связи между sc-элементами и/или обозначаемыми ими сущностями}
\scnsubdividing{произвольная связка\\
    \scnaddlevel{1}
        \scnidtf{sc-переменная, значениями которой являются обозначения связок}
    \scnaddlevel{-1}
    ;связка\\
    \scnaddlevel{1}
        \scnidtf{конкретная связка sc-элементов}
    \scnaddlevel{-1}
    }
\scnsuperset{обозначение sc-пары}
\scnaddlevel{1}
    \scnidtf{обозначение связки двух sc-элементов либо одного sc-элемента с самим собой}
    \scnsuperset{sc-пара}
    \scnaddlevel{1}
        \scnidtf{конкретная sc-пара}
        \scnsubset{sc-константа}
        \scnsuperset{sc-коннектор}
        \scnsuperset{ориентированная sc-пара}
        \scnaddlevel{1}
            \scnsuperset{sc-пара принадлежности}
            \scnaddlevel{1}
                \scnsubdividing{позитивная sc-пара принадлежности\\
                    \scnaddlevel{1}
                        \scnsuperset{позитивная постоянная sc-пара принадлежности}
                        \scnaddlevel{1}
                        \scnsuperset{базовая sc-дуга}
                        \scnaddlevel{-1}
                    \scnaddlevel{-1}
                ;негативная sc-пара принадлежности;нечеткая sc-пара принадлежности}
            \scnaddlevel{-1}
        \scnaddlevel{-1}
    \scnaddlevel{-1}
\scnaddlevel{-1}

\scnheader{обозначение класса}
\scnidtf{обозначение множества sc-элементов, которые в соответствующем смысле эквивалентны друг другу, т.е. имеют одинаковые свойства}
\newpage
\scnsubdividing{произвольный класс\\
    \scnaddlevel{1}
        \scnsubset{sc-переменная}
        \scniselement{sc-константа}
    \scnaddlevel{-1}
    ;класс\\
    \scnaddlevel{1}
        \scnsubset{sc-константа}
    \scnaddlevel{-1}
    }
    
\scnheader{класс}
\scnsubdividing{класс терминальных сущностей;класс множеств\\
    \scnaddlevel{1}
        \scnsubdividing{класс связок\\
            \scnaddlevel{1}
                \scnsuperset{sc-отношение}
            \scnaddlevel{-1}
        ;класс классов\\
            \scnaddlevel{1}
                \scnsuperset{параметр}
            \scnaddlevel{-1}
        ;класс структур\\
            \scnaddlevel{1}
                \scnsuperset{sc-язык}
                \scnaddlevel{1}
                    \scnidtf{специализированный язык, являющийся подъязыком SC-кода, и обеспечивающий представление всевозможных знаний в рамках соответствующей предметной области, которая, в свою очередь, специфицируется соответствующей комплексной онтологией}
                \scnaddlevel{-1}
            \scnaddlevel{-1}
        }
    \scnaddlevel{-1}
    }
\scnhaselement{обозначение множества}
\scnaddlevel{1}
    \scnsuperset{множество}
\scnaddlevel{-1}
\scnhaselement{множество}
\scnhaselement{обозначение связки}
\scnaddlevel{1}
    \scnsuperset{связка}
\scnaddlevel{-1}
\scnhaselement{связка}
\scnhaselement{обозначение класса}
\scnaddlevel{1}
    \scnsuperset{класс}
\scnaddlevel{-1}
\scnhaselement{класс}
\scnhaselement{обозначение структуры}
\scnaddlevel{1}
    \scnsuperset{структура}
\scnaddlevel{-1}
\scnhaselement{структура}
\scnhaselement{обозначение дискретной информационной конструкции}
\scnaddlevel{1}
    \scnsuperset{дискретная информационная конструкция}
    \scnaddlevel{1}
        \scnsuperset{файл}
        \scnaddlevel{1}
            \scnsuperset{файл ostis-системы}
            \scnaddlevel{1}
                \scnsuperset{внутренний файл ostis-системы}
            \scnaddlevel{-1}
        \scnaddlevel{-1}
    \scnaddlevel{-1}
    \scnsuperset{обозначение структуры}
\scnaddlevel{-1}
\scnnote{Все семантически и синтаксически выделяемые классы sc-элементов, а также всевозможные подклассы этих классов являются экземплярами (элементами) \textit{класса}}

\scnheader{обозначение структуры}
\scnidtf{обозначение множества, не являющегося ни связкой, ни классом}
\scnsubdividing{произвольная структура\\
    \scnaddlevel{1}
        \scnsubset{sc-переменная}
    \scnaddlevel{-1}
;структура\\
    \scnaddlevel{1}
        \scnidtf{конкретная структура}
        \scnsubset{sc-константа}
    \scnaddlevel{-1}}

\scnendstruct \scninlinesourcecommentpar{Завершили представление \textit{Семантической классификации sc-элементов}}

\newpage
\scnstructheader{Соотношение между семантически и синтаксически выделяемыми классами sc-элементов в рамках Ядра SC-кода}
\scnstartsubstruct

\scnheader{семантически выделяемый класс sc-элементов}
\scnidtf{класс sc-элементов, определяемый сущностями, которые обозначаются этими sc-элементами, также доступностью (активностью использования) sc-элементов в процессе эксплуатации и эволюции базы знаний}
\scniselement{обозначение терминальной сущности}
\scnaddlevel{1}
\scnsuperset{\scnkeyword{sc-узел общего вида}}
\scnaddlevel{-1}
\scniselement{обозначение небинарной связки}
\scnaddlevel{1}
\scnsuperset{\scnkeyword{sc-узел общего вида}}
\scnaddlevel{-1}
\scniselement{обозначение sc-пары}
\scnaddlevel{1}
\scnrelboth{пара пересекающихся множеств}{sc-узел общего вида}
\scnsuperset{\scnkeyword{sc-коннектор}}
\scnnote{\textit{обозначение sc-пары} может быть представлено либо \textit{sc-узлом общего вида}, либо \textit{sc-коннектором}. При этом каждый \textit{sc-коннектор} представляет собой \textit{обозначение sc-пары}.}
\scnaddlevel{-1}
\scniselement{обозначение неориентированной sc-пары}
\scnaddlevel{1}
\scnidtf{обозначение бинарной неориентированной связи между sc-элементами}
\scnrelbothlist{пара пересекающихся множеств}{\scnkeyword{sc-узел общего вида};\scnkeyword{sc-ребро общего вида}}
\scnnote{\textit{обозначение неориентированной sc-пары} может быть представлено либо \textit{sc-узлом общего вида}, либо \textit{sc-ребром}. При этом не каждое \textit{sc-ребро} представляет обозначение \textit{неориентированный sc-пары}. Некоторые из них представляют \textit{обозначения ориентированных sc-пар неизвестной направленности}.}
\scnaddlevel{-1}
\scniselement{обозначение ориентированной sc-пары неизвестной направленности}
\scnaddlevel{1}
\scnrelbothlist{пара пересекающихся множеств}{\scnkeyword{sc-узел общего вида};\scnkeyword{sc-ребро общего вида}}
\scnaddlevel{-1}
\scniselement{обозначение ориентированной sc-пары}
\scnaddlevel{1}
\scnidtf{обозначение бинарной ориентированной связи между sc-элементами}
\scnrelbothlist{пара пересекающихся множеств}{\scnkeyword{sc-узел общего вида};\scnkeyword{sc-ребро общего вида}}
\scnsuperset{sc-дуга общего вида}
\scnaddlevel{-1}
\scniselement{константная постоянная позитивная sc-пара принадлежности}
\scnaddlevel{1}
\scnrelbothlist{пара пересекающихся множеств}{\scnkeyword{sc-узел общего вида};\scnkeyword{sc-ребро общего вида};\scnkeyword{sc-дуга общего вида}}
\scnsuperset{\scnkeyword{базовая sc-дуга}}
\scnreltoset{пересечение множеств}{sc-константа;обозначение постоянной сущности;обозначение sc-пары принадлежности}
\scnaddlevel{-1}
\scniselement{обозначение класса}
\scnaddlevel{1}
\scnsubset{\scnkeyword{sc-узел общего вида}}
\scnaddlevel{-1}
\scniselement{обозначение структуры}
\scnaddlevel{1}
\scnsubset{\scnkeyword{sc-узел общего вида}}
\scnaddlevel{-1}

\scnendstruct \scninlinesourcecommentpar{Завершили \textit{Описание Соотношения между семантически и синтаксически выделяемыми классами sc-элементов в рамках Ядра SC-кода}. В этом описании жирным курсивом выделены идентификаторы (имена) синтаксически выделяемых классов sc-элементов}

\newpage
\scnheader{Ядро SC-кода}
\scnrelfrom{семантические правила}{\scnstructidtf{Семантические правила Ядра SC-кода}}
\scnaddlevel{1}
\scnhassubstruct{\scnfileitem{Каждый sc-элемент является знаком (обозначением) некоторой описываемой сущности.};\scnfileitem{Любая сущность может быть обозначена sc-элементом и, соответственно, описана в виде конструкции Ядра SC-кода.};\scnfileitem{С помощью sc-элементов можно описать любые связи между sc-элементами и/или между сущностями, которые обозначаются этими sc-элементами. При этом указанные связи трактуются как множества связываемых sc-элементов и обозначаются sc-ребрами, sc-дугами, а в случае небинарных связей -- sc-узлами.};\scnfileitem{Поскольку каждый \mbox{sc-коннектор} семантически трактуется как обозначение пары \mbox{sc-элементов}, связываемых (соединяемых) этим \mbox{sc-коннектором}, каждая пара инцидентности \mbox{sc-коннектора} семантически интерпретируется как обозначение пары принадлежности, связывающей \mbox{sc-коннектор} с одним из элементов обозначаемой им пары \mbox{sc-элементов}.};\scnfileitem{\uline{Любая} описываемая сущность может быть обозначена sc-узлом общего вида, но обратное неверно, т.к. некоторые сущности могут быть обозначены sc-ребрами общего вида, sc-дугами общего вида, базовыми sc-дугами.};\scnfileitem{Каждое sc-ребро является обозначением либо бинарной неориентированной связи между sc-элементами, либо бинарной ориентированной связи неизвестной направленности между sc-элементами.};\scnfileitem{Любая бинарная неориентированная связь между sc-элементами может быть обозначена sc-ребром, но обратное неверно.}}
\scnaddlevel{-1}
\bigskip

\scnendstruct \scninlinesourcecommentpar{Завершили описание \textit{Денотационной семантики Ядра SC-кода}}

\scnheader{Правила синтаксической трансформации sc-элементов в рамках Ядра SC-кода}
\scnidtf{Правила модификации синтаксического типа sc-элементов в рамках Ядра SC-кода}
\scnhassubstruct{\scnfileitem{Если \textit{sc-узел общего вида} является \textit{обозначением sc-пары}, то он трансформируется в \textit{sc-коннектор}};\scnfileitem{Если \textit{sc-узел общего вида} является \textit{обозначением неориентированной sc-пары} или \textit{обозначением ориентированной sc-пары неизвестной направленности}, то он трансформируется в \textit{sc-ребро общего вида}};\scnfileitem{Если \textit{sc-узел общего вида} или \textit{sc-ребро общего вида} являются \textit{обозначением ориентированной sc-пары} и при этом дополнительно указана направленность этой sc-пары, то она трансформируется в \textit{sc-дугу общего вида}.};\scnfileitem{Если \textit{sc-узел общего вида} или \textit{sc-ребро общего вида} или \textit{sc-дуга общего вида} являются \textit{константными постоянными позитивными парами принадлежности}, то они трансформируются в \textit{базовую sc-дугу}.}
}

\scnheader{следует отличать*}
\scnhaselementset{синтаксически выделяемый класс sc-элементов в рамках Ядра SC-кода;синтаксически выделяемый класс sc-элементов в рамках SC-кода;семантически выделяемый класс sc-элементов}

\bigskip
\scnendstruct \scnendsegmentcomment{Описание Ядра SC-кода}

\scnsegmentheader{SC-код как синтаксическое расширение Ядра SC-кода}
\scnstartsubstruct

\scnstructheader{Сравнение SC-кода и Ядра SC-кода}
\scnstartsubstruct
\scnheader{SC-код}
\scnrelto{синтаксическое расширение языка}{Ядро SC-кода}
\scnidtf{Синтаксическое расширение Ядра SC-кода}
    \scnaddlevel{1}
    \scnnote{Синтаксическое расширение Ядра SC-кода заключается во введении дополнительного класса синтаксически эквивалентных элементарных фрагментов конструкций Ядра SC-кода -- sc-элементов, обозначающих внутренние файлы, хранимые в памяти ostis-системы}
    \scnaddlevel{-1}
\scnrelboth{семантическая эквивалентность языков}{Ядро SC-кода}
\scnnote{Семантическая эквивалентность \textit{SC-кода} и \textit{Ядра SC-кода} является следствием того, что \textit{SC-код} является \uline{синтаксическим} расширением \textit{Ядра SC-кода}}
\scnidtf{Результат введения в \textit{Ядро SC-кода} sc-узлов, имеющих содержимое и обозначающих файлы, хранимые в памяти ostis-системы, т.е. внутренние файлы ostis-системы}
\scnnote{Результатом просмотренного расширения \textit{Ядра SC-кода} является расширение \textit{Алфавита Ядра SC-кода}}
\scnnote{Все \textit{файлы}, представляющие собой электронные образы инородных для \textit{SC-кода} информационных конструкций, можно представить в \textit{SC-коде} с помощью графовых структур, в которых \textit{sc-элементы} обозначают буквы текстов или пиксели изображений. Но такой вариант кодирования внешних для \textit{ostis-системы} информационных конструкций не дает возможности непосредственно использовать накопленный человечеством арсенал электронных информационных ресурсов.}
\scnnote{Важнейшим видом внутренних \textit{файлов ostis-систем} являются внутренние файлы \textit{внешних идентификаторов sc-элементов} (в частности, имен sc-элементов), представляющих \textit{sc-элементы} в текстах внешних языков (в том числе, в текстах \textit{SCs-кода} и \textit{SCn-кода})} 

\bigskip
\scnmakeset{Ядро SC-кода;SC-код}

%\scnstartset
%\scnaddhind{1}
%\scnlistitem{Ядро SC-кода}
%\scnlistitem{SC-код}
%\scnaddhind{-1}
%\scnendstruct\\
\scntext{сравнение}{Множество всех элементов конструкций \textit{Ядра SC-кода} и Множество всех элементов конструкций \textit{SC-кода} полностью совпадают, т.к. для каждого элемента конструкции \textit{Ядра SC-кода} существует синонимичный ему элемент конструкции \textit{SC-кода} и наоборот. Из этого следует, что семантическая классификации \textit{элементов информационных конструкций}~~~\textit{SC-кода} и \textit{Ядра SC-кода} также полностью совпадают.

Семантика \textit{SC-кода} ничем не отличается от семантики \textit{Ядра SC-кода}. То есть все, что может быть обозначено и описано текстами \textit{SC-кода}, может быть обозначено и описано текстами \textit{Ядра SC-кода}. Отличие \textit{SC-кода} от \textit{Ядра SC-кода} заключается только в том, что в \textit{SC-код} добавляется новый синтаксически выделяемый класс sc-элементов -- класс sc-элементов, являющихся знаками конкретных (константных) файлов, хранимых в памяти ostis-системы. 

Такие \textit{"внутренние"{} файлы} необходимы для того, чтобы в \textit{памяти ostis-системы} можно было хранить и обрабатывать \textit{информационные конструкции}, не являющиеся текстами \textit{SC-кода}, что необходимо при вводе (восприятии) информации, поступающей извне, а также при генерации \textit{информационных конструкций}, передаваемых другим субъектам. 

Включение в \textit{SC-код} специальных \uline{синтаксически} выделяемых \textit{sc-узлов}, обозначающих хранимые в \textit{sc-памяти} электронные образы (файлы) различного вида \textit{информационных конструкций}, не являющихся конструкциями \textit{SC-кода}, дает возможность непосредственно в \textit{памяти ostis-системы}, то есть в одной и той же запоминающий среде обрабатывать не только конструкции \textit{SC-кода}, но и "инородные"{} для него конструкции, что для необходимо для реализации \textit{интерфейса ostis-системы}, обеспечивающего ее взаимодействие с \textit{внешней средой}. 

Без такой реализации \textit{интерфейса ostis-системы} невозможно реализовать синтаксический анализ, семантический анализ и понимание, а также невозможно реализовать синтез (генерацию) внешних информационных конструкций, принадлежащих заданному внешнему языку и семантически эквивалентных заданному смыслу. 

Поскольку все синтаксические и семантические свойства \textit{SC-кода} и \textit{Ядра SC-кода} являются весьма близкими, при описании \textit{SC-кода} акцентируется внимание на его отличия от \textit{Ядра SC-кода}, а также на более детальное рассмотрение семантической классификации элементов.
}

\scnendstruct \scninlinesourcecommentpar{Завершили \textit{Сравнение SC-кода и Ядра SC-кода}}

\newpage
\scnstructheader{Синтаксис SC-кода}
\scnstartsubstruct
\scnheader{SC-код}
\scnrelfrom{синтаксис}{синтаксис SC-кода}
\scnaddlevel{1}
\scntext{примечание}{\textit{Синтаксис SC-кода} отличается от \textit{Синтаксиса Ядра SC-кода} только тем, что в \textit{Алфавит \mbox{SC-кода}} дополнительно вводится класс sc-узлов, являющихся знаками \textit{файлов}, хранимых в памяти \textit{\mbox{ostis-системы}}}
\scnaddlevel{-1}
\scnrelfrom{множество всех экземпляров конструкций данного языка}{sc-элемент}
\scnaddlevel{1}
\scnidtf{элемент конструкции SC-кода}
\scntext{примечание}{Множество всех элементов конструкций SC-кода совпадает со множеством всех элементов конструкций Ядра SC-кода. Просто в конструкциях SC-кода некоторые sc-элементы, имеющие "синтаксическую метку"{} (синтаксический тип) \textit{sc-узла общего вида}, будут иметь "метку"{} sc-узла, являющегося знаком \textit{внутреннего файла},  хранимого в памяти \textit{ostis-системы}}
\scnaddlevel{-1}
\scnexplanation{\textit{Синтаксис SC-кода} задается
\begin{scnitemize}
\item типологией (алфавитом) sc-элементов (атомарных фрагментов текстов SC-кода);
\item правилами соединения (инцидентности) sc-элементов (например, sc-элементы каких типов не могут быть инцидентными друг другу);
\item типологией конфигураций sc-элементов (связки, классы, структуры), связями между конфигурациями sc-элементов (в частности, теоретико-множественными)
\end{scnitemize}
}
\scnheader{Алфавит SC-кода}
\scnrelto{алфавит}{SC-код}
\scnrelfrom{разбиение}{sc-элемент}
\scneq{{\normalfont(}Алфавит Ядра SC-кода $\cup$ \scnset{\scnkeyword{внутренний файл ostis-системы}}{\normalfont)}}
\scneq{\scnmakesetlocal{sc-узел общего вида; \textit{\scnkeyword{внутренний файл ostis-системы}}; sc-ребро общего вида; sc-дуга общего вида; базовая sc-дуга}}

\scnheader{Алфавит SC-кода}
\scnidtf{Алфавит sc-элементов в рамках SC-кода}
\scnidtf{Семейство всех максимальных множеств синтаксически эквивалентных (в рамках SC-кода) sc-элементов}
\scnidtf{Семейство классов синтаксически эквивалентных sc-элементов SC-кода}
\scnidtf{Семейство всех множеств, в каждое из которых входят все синтаксически эквивалентные друг другу (в рамках SC-кода) sc-элементы и только они}
\scnidtf{Фактор-множество отношения "синтаксическая эквивалентность sc-элементов в рамках SC-кода"{}}
\scneq{фактор-множество*{\normalfont(}синтаксическая эквивалентность sc-элементов в рамках SC-кода*{\normalfont)}}
\scnaddlevel{1}
\scniselement{сложный внешний идентификатор sc-элемента}
\scnaddlevel{-1}
\scnidtf{Семейство множеств sc-элементов, являющихся результатом разбиения максимального множества \mbox{sc-элементов} SC-кода (класса всевозможных sc-элементов) по признаку синтаксической эквивалентности sc-элементов}
\scnidtf{Признак (параметр) синтаксической эквивалентности sc-элементов}

\scnheader{внутренний файл ostis-системы}
\scnidtf{sc-узел, имеющий содержимое}
\scnidtf{sc-ссылка}
\scnidtf{множество всевозможных sc-узлов, имеющих содержимое, хранимое в памяти ostis-системы}
\scnidtf{внутренний файл, хранимый в памяти ostis-системы}
\scnidtf{внутренний файл для заданной ostis-системы (той ostis-системы, в памяти которой хранится sc-узел, обозначающий этот файл)}
\scnidtf{sc-узел, являющийся знаком конкретного файла, хранимого в той же sc-памяти (в качестве содержимого sc-узла), в которой находится и сам указанный sc-узел}
\scnidtf{файл, знак которого находится в той же sc-памяти, в которой находится и сам файл}
\scnidtf{"свой"{} файл ostis-системы} 
\scnsubset{внутренняя информационная конструкция}
\scnheader{синтаксически выделяемый класс sc-элементов в рамках SC-кода}
\scnidtf{класс sc-элементов, определяемый на основе Алфавита SC-кода}
\scnsuperset{Алфавит SC-кода}
\scnhaselement{sc-узел, не являющийся знаком внутреннего файла ostis-системы}
\scnstructheader{Синтаксическая классификация sc-элементов в рамках SC-кода}
\scnstartstruct

\scnheaderlocal{sc-элемент}
\scnsubdividing{sc-узел общего вида\\
    \scnaddlevel{1}
    \scnsubdividing{sc-узел, не являющийся знаком внутреннего файла ostis-системы;внутренний файл ostis-системы}
    \scnaddlevel{-1}
    ;sc-коннектор\\
    \scnaddlevel{1}
    \scnsubdividing{sc-ребро общего вида;sc-дуга\\
    \scnaddlevel{1}
    \scnsubdividing{sc-дуга общего вида;базовая sc-дуга}
    \scnaddlevel{-1}
    }
    \scnaddlevel{-1}
    }
\bigskip
\scnendstruct

\scnnote{Данная \textit{Синтаксическая классификация sc-элементов} от \textit{Синтаксической классификации sc-элементов Ядра SC-кода} отличается только дополнительным уточнением синтаксической типологии \textit{sc-узлов}}
\scnendstruct \scninlinesourcecommentpar{Завершили представление \textit{Синтаксиса SC-кода}}

\scnstructheader{Денотационная семантика SC-кода}
\scnstartsubstruct
\scnheader{Денотационная семантика SC-кода}
\scntext{аннотация}{\textit{Денотационную семантику SC-кода} рассмотрим как расширение и уточнение \textit{Денотационной семантики Ядра SC-кода} (смотрите предыдущий сегмент "\textit{Описание Ядра SC-кода}"{}). Изложение построим как последовательное уточнение следующих понятий:
\begin{scnitemize}
\item \textit{sc-переменная}
\item \textit{обозначение дискретной информационной конструкции}
\item\textit{дискретная информационная конструкция} (рассмотрим различные параметры и отношения, заданные на множестве дискретных информационных конструкций)
\item \textit{знание} (как частный вид дискретных информационных конструкций)
\item \textit{файл} (как \textit{sc-константа}, являющаяся \textit{обозначением файла})
\item \textit{внутренний файл ostis-системы}
\item\textit{структура} (как \textit{дискретная информационная конструкция}, принадлежащая \textit{SC-коду})
\end{scnitemize}}

\scnheader{SC-код}
\scnrelfrom{денотационная семантика}{Денотационная семантика SC-кода}
\scnaddlevel{1}
\scnexplanation{\textit{Денотационная семантика SC-кода} задается
\begin{scnitemize}
\item семантической интерпретацией sc-элементов и их конфигураций;
\item семантической интерпретацией инцидентности sc-элементов;
\item иерархической системой \textit{предметных областей};
\item структурой используемых понятий в каждой предметной области (исследуемые классы объектов, исследуемые отношения, исследуемые классы объектов отношений из смежных предметных областей, ключевые экземпляры исследуемых классов объектов);
\item \textit{онтологиями предметных областей};
\end{scnitemize}}

\scnstructheader{Классификация sc-переменных}
\scnstartsubstruct
\scnheader{sc-переменная}
\scnidtf{sc-элемент, представляющий собой обозначение произвольной (переменной) сущности из некоторого дополнительно уточняемого множества обозначений других сущностей, которые считаются возможными значениями указанной произвольной сущности}
\scnrelto{область задания}{значение переменной*}
\scnaddlevel{1}
\scnidtf{Бинарное ориентированное отношение, связывающее sc-переменные с их возможными значениями*}
\scnexplanation{Это одно из отношений, заданных на множестве sc-переменных}
\scnaddlevel{-1}
\scnrelfrom{разбиение}{\scnkeyword{Структурная типология sc-переменных}}
\scnaddlevel{1}
\scneq{\scnmakesetlocal{произвольная терминальная сущность
\scnaddlevel{1}
\scnidtf{sc-переменная, обозначающая терминальную сущность}
\scnidtf{sc-переменная, значением или значением значения и т.д. которой является терминальная сущность}
\scnidtf{sc-переменная, "конечным"{} значением которой является терминальная сущность}
\scnidtf{обозначение произвольной терминальной сущности}
\scnaddlevel{-1}
;произвольное множество sc-элементов
}}
\scnaddlevel{-1}
\scnsubdividing{sc-переменная, у которой логический уровень всех ее значений одинаков\\
\scnaddlevel{1}
\scnsuperset{первичная sc-переменная}
\scnaddlevel{1}
\scnidtf{sc-переменная, все значения которой являются sc-константами}
\scnaddlevel{-1}
\scnsuperset{вторичная sc-переменная}
\scnaddlevel{1}
\scnidtf{sc-переменная, все значения которой являются первичными sc-переменными}
\scnaddlevel{-1}
\scnsuperset{sc-переменная третьего уровня}
\scnaddlevel{1}
\scnidtf{sc-переменная, все значения которой являются вторичными sc-переменными}
\scnaddlevel{-2}
;sc-переменная, значения которой имеют различный логический уровень
}
\scnsubdividing{sc-переменная, у которой синтаксический тип всех её значений одинаков\\
\scnaddlevel{1}
\scnsuperset{переменный sc-узел}
\scnaddlevel{1}
\scnidtf{sc-переменная, все значения которой являются sc-узлами}
\scnaddlevel{-1}
\scnsuperset{переменное sc-ребро}
\scnsuperset{переменная sc-дуга}
\scnaddlevel{-1}
;sc-переменная, значения которой имеют различный синтаксический тип
}
\scnendstruct \scninlinesourcecommentpar{Завершили \textit{Классификации sc-переменных}}

\bigskip
\scnheader{обозначение дискретной информационной конструкции}
\scnsubdividing{обозначение дискретной информационной конструкции, не принадлежащей SC-коду;\scnkeyword{обозначение структуры}
\scnaddlevel{1}
\scnidtf{обозначение дискретной информационной конструкции, принадлежащей SC-коду}
\scnidtf{обозначение sc-конструкции}
\scnaddlevel{-1}}
\scnsubdividing{произвольная дискретная информационная конструкция\\
\scnaddlevel{1}
\scnidtf{sc-переменная, обозначающая дискретную информационную конструкцию}
\scnaddlevel{-1}
;\scnkeyword{дискретная информационная структура}
\scnaddlevel{1}
\scnidtf{sc-константа, обозначающая конкретную дискретную информационную конструкцию}
\scnaddlevel{-1}}

\scnstructheader{Описание параметров и отношений, заданных на дискретных информационных конструкциях}
\scnstartsubstruct

\scnheader{параметр, заданный на множестве дискретных информационных конструкций\scnsupergroupsign}
\scnhaselement{типология дискретных информационных конструкций, определяемая их носителем\scnsupergroupsign}
	\scnaddlevel{1}
	\scnhaselement{некомпьютерная форма представления дискретных информационных конструкций\scnsupergroupsign}
	\scnhaselement{файл}
	    \scnaddlevel{1}
	    \scnidtf{компьютерная форма предcтавления дискретных информационных конструкций в линейной адресной памяти}
	    \scnaddlevel{-1}
	\scnhaselement{структура}
	    \scnaddlevel{1}
	    \scnidtf{компьютерная форма представления дискретных информационных конструкций в графодинамической ассоциативной памяти}
	    \scnidtf{представление дискретных информационных конструкций в виде конструкций SC-кода в памяти ostis-систем}
	    \scnaddlevel{-1}
	\scnaddlevel{-1}

\scnhaselement{типология дискретных информационных конструкций, определяемая их соотношением с памятью ostis-систем\scnsupergroupsign}
    \scnaddlevel{1}
    \scnhaselement{внешняя дискретная информационная конструкция ostis-системы}
        \scnaddlevel{1}
        \scnidtf{дискретная информационная конструкция, которая находится вне памяти той ostis-системы, в которой находится sc-узел, обозначающий эту информационную конструкцию}
        \scnsubdividing{некомпьютерная форма представления дискретных информационных конструкций\\
            \scnaddlevel{1}
            \scnnote{Очевидно, что информационные конструкции такого вида принципиально не могут быть внутренними информационными конструкциями ostis-систем, хранимыми в их памяти.}
            \scnaddlevel{-1};
            внешний файл ostis-системы\\
            \scnaddlevel{1} 
            \scnsubdividing{файл компьютерной системы, которая не является ostis-системой\\
            \scnaddlevel{1}
            \scnidtf{файл, который не хранится в памяти данной ostis-системы, но о которой известно, какая система, не являющаяся ostis-системой, им "владеет"{} и как его "скачать"{}}
            \scnidtf{внешний файл ostis-системы, принадлежащий компьютерной системе, которая не является ostis-системой}
            \scnaddlevel{-1};
            файл другой ostis-системы\\
            \scnaddlevel{1}
            \scnidtf{файл, который не является внутренним файлом данной ostis-системы, в памяти которой находится знак этого файла, но является внутренним знаком другой ostis-системы}
            \scnidtf{внешний файл ostis-системы, принадлежащий другой ostis-системе}
            \scnaddlevel{-1}
            }\scnaddlevel{-1};
        внешняя структура ostis-системы
            \scnaddlevel{1}
            \scnidtf{структура, хранимая в памяти другой ostis-системы}
            \scnidtf{структура другой ostis-системы}
            \scnaddlevel{-1}
        }
    \scnaddlevel{-1}
    \bigskip
    \scnhaselement{внутренняя информационная конструкция ostis-системы}
        \scnaddlevel{1}
        \scnidtf{внутренняя для заданной ostis-системы информационная конструкция}
        \scnidtf{внутренняя информационная конструкция той ostis-системы, в памяти (sc-памяти) которой хранится знак (sc-узел) этой информационной конструкции}
        \scnnote{Внутренние информационные конструкции ostis-систем (т.е. конструкции, обрабатываемые в их памяти) могут быть только дискретными, хотя и не обязательно знаковыми.}
        \scnsubdividing{внутренний файл ostis-системы;внутренняя структура
        \scnaddlevel{1}
            \scnidtf{структура, которой в памяти данной ostis-системы соответствует не только знак этой структуры, но и она сама}
            \scnidtf{структура, хранимая и обрабатываемая в памяти данной ostis-системы}
            \scnidtf{внутренняя структура ostis-системы}
        \scnaddlevel{-1}}
    \scnsubdividing{сформированная внутренняя информационная конструкция ostis-системы;частично сформированная внутренняя информационная конструкция ostis-системы;внутренняя информационная конструкция ostis-системы на начальной стадии формирования}
    \scnaddlevel{-2}
    
\scnhaselement{типология дискретных информационных конструкций, определяемая правилами, которым они должны удовлетворять\scnsupergroupsign}
\scnaddlevel{1}
    \scnhaselement{информационная конструкция Русского языка}
    \scnhaselement{информационная конструкция Английского языка}
    \scnhaselement{структура}
    \scnaddlevel{1}
        \scnidtf{информационная конструкция SC-кода}
    \scnaddlevel{-1}
    \scnhaselement{sc.g-конструкция}
    \scnaddlevel{1}
        \scnidtf{информационная конструкция SCg-кода}
    \scnaddlevel{-1}
    \scnhaselement{sc.s-конструкция}
    \scnaddlevel{1}
        \scnidtf{информационная конструкция SCs-кода}
    \scnaddlevel{-1}
    \scnhaselement{sc.n-конструкция}
    \scnaddlevel{1}
        \scnidtf{информационная конструкция SCn-кода}
    \scnaddlevel{-1}
\scnaddlevel{-1}

\scnhaselement{наличие синтаксической связности\scnsupergroupsign}
\scnaddlevel{1}
    \scnhaselement{синтаксически связная дискретная информационная конструкция}
    \scnaddlevel{1}
        \scnidtfdef{дискретная информационная конструкция, у которой для каждой пары её элементов существует маршрут, соединяющий эти элементы и проходящий по связям их инцидентности}
    \scnaddlevel{-1}
    \scnhaselement{синтаксически несвязная дискретная информационная конструкция}
    \scnnote{Можно оценивать "силу"{} синтаксической связности -- наличие и число "мостов"{} в графе инцидентности элементов дискретной информационной конструкции, наличие и число точек "сочленения"{}, минимальное число элементов конструкции, удаление которых приводит к несвязности. Можно также оценивать уровень синтаксической несвязности дискретной информационной конструкции числом компонентов связности этой конструкции.}
\scnaddlevel{-1}
    
\scnhaselement{наличие семантической связности\scnsupergroupsign}
\scnaddlevel{1}
    \scnnote{Свойством семантической связности могут обладать только знаковые конструкции.}
    \scnhaselement{семантически связная дискретная информационная конструкция}
    \scnaddlevel{1}
    \scndefinition{Это конструкция, которая обладает следующим свойством: для любой ее декомпозиции на два синтаксически правильных компонента всегда найдется пара синонимичных знаков, один из которых находится в одном компоненте, а другой -- в другом.}
    \scnaddlevel{-1}
    \scnhaselement{семантически несвязная дискретная информационная конструкция}
\scnaddlevel{-1}

\scnheader{отношение, заданное на множестве дискретных информационных конструкций\scnsupergroupsign}
\scnhaselement{дискретная информационная конструкция заданного языка*}
\scnaddlevel{1}
    \scniselement{отношение, заданное на множестве языков\scnsupergroupsign}
    \scnsubdividing{синтаксически неправильная дискретная информационная конструкция заданного языка*\\
    \scnaddlevel{1}
        \scnreltoset{объединение}{синтаксически некорректная дискретная информационная конструкция заданного языка*;синтаксически нецелостная дискретная информационная конструкция заданного языка*}
    \scnaddlevel{-1}
    ;\scnkeyword{текст заданного языка*}\\
    \scnaddlevel{1}
        \scnreltoset{пересечение}{синтаксически корректная дискретная информационная конструкция заданного языка*;синтаксически целостная дискретная информационная конструкция заданного языка*}
        \scnsubdividing{семантически неправильный текст заданного языка*\\
        \scnaddlevel{1}
            \scnreltoset{объединение}{семантически некорректный текст заданного языка*;семантически нецелостный текст заданного языка*}
        \scnaddlevel{-1}
        ;знание, представленное в заданном языке*\\
        \scnaddlevel{1}
            \scnreltoset{пересечение}{семантически корректный текст заданного языка*;семантически целостный текст заданного языка*}
        \scnaddlevel{-1}
        }
    \scnaddlevel{-1}
    }
    
\scnaddlevel{-1}


\scnendstruct \scninlinesourcecommentpar{Завершили \textit{Описание параметров и отношений, заданных на дискретных информационных конструкциях}}

\scnheader{обозначение файла}
\scnsubdividing{произвольный файл\\
\scnaddlevel{1}
\scnidtf{sc-переменная, каждым значением которой является обозначение файла}
\scnidtf{обозначение произвольного файла}
\scnidtf{sc-переменнная, обозначающая файл}
\scnaddlevel{-1}
;\scnkeyword{файл}\\
\scnaddlevel{1}
\scnidtf{знак конкретного файла} 
\scnidtf{sc-константа, обозначающая конкретный файл}
\scnaddlevel{-1}
}

\scnsubdividing{обозначение внешнего файла ostis-системы\\
\scnaddlevel{1}
\scnsubdividing{произвольный внешний файл ostis-системы\\
;внешний файл ostis-системы\\
}
\scnaddlevel{-1}
;обозначение внутреннего файла ostis-системы\\
\scnaddlevel{1}
\scnsubdividing{произвольный внутренний файл ostis-системы\\
;внутренний файл ostis-системы\\
}
\scnaddlevel{-1}
}

\scnheader{файл}
\scnidtf{sc-узел, обозначающий файл}
\scnidtf{знак файла}
\scnsubdividing{ея-файл\\
\scnaddlevel{1}
\scnidtf{естественно-языковой файл}
\scnaddlevel{-1}
;файл, являющийся текстом формального языка\\
\scnaddlevel{1}
\scnsuperset{sc.g-файл}
\scnsuperset{sc.s-файл}
\scnsuperset{sc.n-файл}
\scnaddlevel{-1}
;файл, отражающий процесс изменения sc.g-текста\\
;графический файл\\
;файл-изображение\\
;видео-файл\\
;аудио-файл\\
}
\scnsubdividing{файл-экземпляр\\
\scnaddlevel{1}
\scnidtf{файл, являющийся конкретным электронным документом или электронным образом конкретной внешней информационной конструкции}
\scnaddlevel{-1}
;файл-образец\\
\scnaddlevel{1}
\scnidtf{файл-класс ostis-системы}
\scnidtf{файл, являющийся одновременно также и знаком множества всевозможных экземпляров (копий) этого файла}
\scnaddlevel{-1}
}
\scnsubdividing{внешний файл ostis-системы\\
;\scnkeyword{внутренний файл ostis-системы}\\
}

\scnheader{внутренний файл ostis-системы}
\scniselement{синтаксически выделяемый класс sc-элементов в рамках SC-кода}
\scniselement{семантически выделяемый класс sc-элементов в рамках SC-кода}
\scntext{примечание}{Данный класс sc-элементов, являющихся знаками файлов, хранимых в памяти ostis-систем, в отличие от других синтаксически выделяемых классов sc-элементов, представляет собой одновременно  синтаксически и семантически выделяемый класс sc-элементов. Это обусловлено (1) тем, что каждый экземпляр данного класса sc-элементов является знаком конкретного файла, хранимого в памяти ostis-системы, и (2) тем, что каждый файл, хранимый в памяти ostis-системы, может и должен быть обозначен только таким sc-элементом, который является экземпляром рассматриваемого класса sc-элементов.}
\scntext{примечание}{sc-узел может быть знаком файла, находящегося в памяти другой ostis-системы (не в той, в которой хранится этот sc-узел). Но в этом случае указанный sc-узел не будет принадлежать рассматриваемому классу sc-узлов.}
\scnidtf{знак файла ostis-системы, хранимого в "моей"{} памяти} 
\scntext{примечание}{Следует отличать синтаксическую эквивалентность файлов, семантическую эквивалентность файлов и совпадение файлов (когда речь идет об одном и том же файле). Т.е. копия файла и один и тот же файл -- это разные вещи.}

\scnstructheader{Классификация структур}
\scnstartsubstruct
\scnheader{структура}
\scnidtf{структура, элементами которой являются sc-элементы}
\scnidtf{множество sc-элементов (множество), не являющиеся ни связкой (связкой sc-элементов), ни классом (множеством всех sc-элементов, эквивалентных в определенном смысле)}
\scnidtf{знак конкретной (константной) структуры}
\scnidtf{Класс всех тех и только тех sc-элементов, каждый из которых является знаком конкретной структуры}
\scnidtf{Знак класса всех sc-элементов, являющихся знаками конкретных структур}
\scnidtf{Константный sc-элемент (точнее, sc-узел), являющийся знаком конкретного класса всех sc-элементов, являющихся знаками конкретных структур}
\scnidtf{sc-конструкция}
\scnidtf{информационная конструкция, принадлежащая SC-коду}
\scnsuperset{\scnkeyword{sc-текст}}
\scnaddlevel{1}
\scnidtf{структура, удовлетворяющая синтаксическим правилам SC-кода}
\newpage
\scnsuperset{\scnkeyword{знание}}
\scnaddlevel{1}
\scnidtf{семантически корректный и целостный sc-текст}
\scnaddlevel{-2}

\scnrelfrom{разбиение}{\scnkeyword{Наличие sc-переменных, входящих в состав структуры}}
\scnaddlevel{1}
\scneq{\scnmakesetlocal{структура, в составе которой sc-переменные не входят;структура, в состав которой входят sc-переменные\\
\scnaddlevel{1}
\scnnote{Такие структуры при представлении логических высказываний в SC-коде являются аналогами атомарных логических формул.}
\scnaddlevel{-1}
}\scnaddlevel{-1}}

\scnrelfrom{разбиение}{\scnkeyword{Темпоральная характеристика структур}}
\scnaddlevel{1}
\scneq{\scnmakesetlocal{ситуативная структура\\
\scnaddlevel{1}
\scnidtf{ситуация, представленная в SC-коде}
\scnidtf{ситуация}
\scnidtf{структура, в состав которой входят знаки временных сущностей и которая сама является временной сущностью (при этом время существования такой структуры совпадает с временем одновременного существования всех временных сущностей, знаки которых входят в состав этой ситуативной структуры).}
\scnsubdividing{ситуация во внешней среде;ситуация в sc-памяти}
\scnaddlevel{-1}
;структура, не содержащая знаков временных сущностей;динамическая структура\\
\scnaddlevel{1}
\scnexplanation{В отличие от ситуативной структуры конфигурация динамической структуры меняется во времени в зависимости от момента появления и момента завершения существования каждой временной сущности (в том числе временной связи), знак которой входит в состав динамической структуры. Каждой динамической структуре можно поставить в соответствие темпоральную последовательность состояний (ситуаций) и событий.}
}}
\scnaddlevel{-2}
\scnrelfrom{разбиение}{\scnkeyword{Наличие связности структур}}
\scnaddlevel{1}
\scneq{\scnmakesetlocal{связная структура\\
\scnaddlevel{1}
\scnrelfrom{разбиение}{Связность структур\scnsupergroupsign}
\scnaddlevel{1}
\scnidtf{Минимальное число sc-элементов, удаление которых преобразует связную структуру в несвязную}
\scniselement{одно-связная структура}
\scnaddlevel{1}
\scnrelto{разбиение}{Признак классификации структур по числу точек сочленения\scnsupergroupsign}
\scnrelto{разбиение}{Признак классификации структур по числу мостов\scnsupergroupsign}
\scnaddlevel{-3}
;несвязная структура\\
\scnaddlevel{1}
\scnrelto{разбиение}{Признак классификации структур по числу компонентов связности\scnsupergroupsign}
\scnsubset{тривиальная структура}
\scnaddlevel{1}
\scnidtf{структура, в состав элементов которой sc-коннекторы не входят}
\scnaddlevel{-2}}
\scnnote{Важнейшей особенностью SC-кода является то, что для конструкций SC-кода (для структур) нет необходимости противопоставлять синтаксическую и семантическую связность, то есть все синтаксически связные структуры являются также и семантически связными и наоборот.}
\scnaddlevel{-1}}

\scnrelfrom{разбиение}{\scnkeyword{Рефлексивность структур}}
\scnaddlevel{1}
\scneq{\scnmakesetlocal{рефлексивная структура\\
\scnaddlevel{1}
\scnidtf{структура, в число элементов которой входит sc-узел, обозначающий саму эту структуру}
\scnsubset{рефлексивное множество}
\scnaddlevel{-1}
;нерефлексивная структура
}}
\scnaddlevel{-1}
\scnrelfrom{разбиение}{\scnkeyword{Целостность структур по связкам}}
\scnaddlevel{1}
\scneq{\scnmakesetlocal{структура, содержащая все компоненты всех своих связок;структура, не содержащая все компоненты всех своих связок
}}
\scnaddlevel{-1}
\scnsubdividing{синтаксически неправильная структура\\
\scnaddlevel{1}
\scnidtf{синтаксически неправильно построенная структура}
\scnreltoset{объединение}{синтаксически некорректная структура\\
\scnaddlevel{1}
\scnidtf{структура, содержащая фрагменты, противоречащие \textit{Синтаксическим правилам SC-кода} (ошибочные фрагменты)}
\scnaddlevel{-1}
;синтаксически нецелостная структура\\
\scnaddlevel{1}
\scnidtf{структура, в которой имеется синтаксически выявленная недостаточность, неполнота (то есть имеется некоторое количество информационных дыр)}
\scnaddlevel{-1}}
\scnnote{Разделение \textit{Синтаксических правил SC-кода} на правила анализа синтаксической корректности и правила анализа синтаксической целостности (полноты) существенно упрощает процедуру синтаксического анализа \textit{структур}.}
\scnaddlevel{-1}
;\scnkeyword{sc-текст}
\scnaddlevel{1}
\scnidtf{синтаксически правильная структура}
\scnidtf{синтаксически правильно построенная структура}
\scnreltoset{пересечение}{синтаксически корректная структура;синтаксически целостная структура}
\scnsubdividing{семантически неправильный sc-текст\\
\scnaddlevel{1}
\scnreltoset{объединение}{семантически некорректный sc-текст;семантически нецелостный sc-текст}
\scnaddlevel{-1}
;\scnkeyword{знание}
\scnaddlevel{1}
\scnidtf{семантически правильно построенный sc-текст}
\scnreltoset{пересечение}{семантически корректный sc-текст;семантически целостный sc-текст}
\scnaddlevel{-1}}
\scnaddlevel{-1}
}
\scnendstruct \scninlinesourcecommentpar{Завершили \textit{Классификацию структур}}

\scnheader{знание}
\scnidtf{дискретная информационная конструкция, являющаяся знанием, представленная в некотором (дополнительно уточняемом) языке}
\scnrelto{второй домен}{знание, представленное в заданном языке*}
\scnsubset{знаковая конструкция}
\scntext{примечание}{Каждое знание является знаковой конструкцией, но не каждая знаковая конструкция является знанием, а только та, смысловое представление которой удовлетворяет определенным требованиям корректности и целостности.}
\scniselement{семантически выделяемый класс дискретных информационных конструкций\scnsupergroupsign}


\scnheader{следует отличать*}
\scnhaselementset{\scnmakesetlocal{дискретная информационная конструкция;текст\newpage;знание}
\scnaddlevel{1}
\scniselement{следует отличать*}
\scnaddlevel{-1}
;\scnmakesetlocal{структура;sc-текст;знание}
\scnaddlevel{1}
\scniselement{следует отличать*}
\scnaddlevel{-1}
}
\scnendstruct \scninlinesourcecommentpar{Завершили представление \textit{Денотационной семантики SC-кода}}
\bigskip
\scnendstruct \scnendsegmentcomment{SC-код как синтаксическое расширение Ядра SC-кода}


\scnsegmentheader{Использование SC-кода для формального описания собственного синтаксиса}
\scnstartsubstruct

\bigskip
\scnfilelong{В предыдущем сегменте ``\textit{SC-код как синтаксическое расширение Ядра SC-кода}'' рассмотрен \textit{Синтаксис SC-кода} путём:
\begin{scnitemize}
\item введения \textit{синтаксически выделяемых классов sc-элементов} в рамках \textit{SC-кода}\char59
\item описания \textit{теоретико-множественных связей} между указанными классами \textit{sc-элементов} (к такому описанию, в частности, относится \textit{Синтаксическая классификация sc-элементов в рамках SC-кода})\char59
\item введения двух отношений инцидентности \textit{sc-элементов} -- \textit{Отношения инцидентности sc-коннекторов*} и \textit{Отношения инцидентности входящих sc-дуг*}\char59
\item описания \textit{Синтаксических правил SC-кода}, которые, прежде всего, описывают формальные свойства указанных выше отношений инцидентности \textit{sc-элементов}.
\end{scnitemize}

Однако для того, чтобы получить возможность \uline{все} (!) \textit{Синтаксические правила SC-кода} записать средствами самого \textit{SC-кода}, необходимо иметь \uline{явное} представление \textit{пар} отношений инцидентности \textit{sc-элементов} в виде \textit{sc-дуг}, принадлежащим этим отношениям. В случае, если указанные \textit{sc-дуги} инцидентности являются \textit{sc-переменными}, логико-семантических проблем не возникнет. И этого, кстати, вполне достаточно, чтобы \textit{Синтаксические правила SC-кода}, сформулированные в виде \textit{логических высказываний}, записать средствами \textit{SC-кода}. Но, если разрешить \textit{sc-дугам} инцидентности быть \textit{sc-константами}, то, во-первых, в \textit{Синтаксические правила SC-кода} необходимо добавить Правило удаления \textit{константной sc-дуги инцидентности}, если эта инцидентность представлена неявно, а, во-вторых, в \textit{Правила синтаксической трансформации sc-элементов} необходимо добавить Правило трансформации (замены) \textit{константной sc-дуги инцидентности} на неявное представление этой инцидентности.

В теоретическом и, возможно, даже в практическом плане может быть интересна такая синтаксическая модификация (синтаксическое расширение) \textit{SC-кода}, в котором: 
\begin{scnitemize}
\item \uline{все} неявно представленные \textit{пары инцидентности sc-элементов} заменяются на \textit{константные sc-дуги инцидентности} -- неявно представленными \textit{парами инцидентности} остаются \uline{только} \textit{пары инцидентности} константных \textit{sc-дуг} инцидентности с компонентами этих \textit{sc-дуг}\char59
\item В \textit{Алфавит SC-кода} вводятся два новых \textit{синтаксически выделяемых класса sc-элементов} -- \textit{класс sc-дуг инцидентности sc-коннекторов}, а также \textit{класс sc-дуг инцидентности входящих sc-дуг}.
\end{scnitemize}

В результате такого преобразования конструкций \textit{SC-кода} конструкции \textit{SC-кода} перестают быть графовыми конструкциями нетрадиционного вида, в которых рёбра, гиперрёбра, дуги могут быть инцидентны другим рёбрам, гиперребрам и дугам, а становятся классическими графами с двумя типами дуг (с \textit{sc-дугами инцидентности sc-коннекторов} и с \textit{sc-дугами инцидентности входящих sc-дуг}) и с пятью типами вершин (с вершинами, представляющими \textit{sc-узлы общего вида}, с вершинами, представляющими \textit{sc-узлы}, являющиеся знаками \textit{внутренних файлов ostis-системы}, с вершинами, представляющими \textit{sc-рёбра общего вида}, с вершинами, представляющими \textit{sc-дуги общего вида}, с вершинами, представляющими \textit{базовые sc-дуги}).

Рассмотренное преобразование конструкций \textit{SC-кода} в теории графов называется поздразделением или подразбиением графа (Смотрите \scncite{Trudeau1993}).}

\bigskip
\scnendstruct \scnendsegmentcomment{Использование SC-кода для формального описания собственного синтаксиса}

\bigskip
\scnendstruct \scnendcurrentsectioncomment

\end{SCn}

\scsubsubsection[\scnmonographychapter{Глава 2.1. Универсальный язык смыслового представления знаний и смысловое пространство}]{Предметная область и онтология синтаксиса внутреннего языка ostis-систем}
\label{sd_sc_code_syntax}

\scsubsubsection[\scnmonographychapter{Глава 2.1. Универсальный язык смыслового представления знаний и смысловое пространство}]{Предметная область и онтология базовой денотационной семантики внутреннего языка ostis-систем}
\label{sd_sc_code_semantic}
\scsubsection[\scnmonographychapter{Глава 2.2. Семейство внешних языков интеллектуальных компьютерных систем нового поколения, близких языку внутреннего смыслового представления знаний (SCg, SCs, SCn)}]{Предметная область и онтология внешних идентификаторов знаков, входящих в информационные конструкции внутреннего языка ostis-систем}
\label{intro_idtf}

\begin{SCn}

\scnsectionheader{\currentname}

\scnstartsubstruct

\scnreltovector{конкатенация сегментов}{Понятие внешнего идентификатора sc-элемента;Понятие простого идентификатора sc-элемента;Понятие сложного идентификатора sc-элемента}

\scnheader{Предметная область внешних идентификаторов знаков, входящих в информационные конструкции внутреннего языка ostis-систем}
\scniselement{предметная область}
\scnsdmainclasssingle{sc-идентификатор}
\scnsdclass{основной sc-идентификатор;строковый sc-идентификатор;системный sc-идентификатор;нетранслируемый sc-идентификатор;простой sc-идентификатор;простой строковый sc-идентификатор;имя нарицательное;имя собственное;sc-выражение;ограничитель sc-выражений}


\scnsdrelation{sc-идентификатор*}

\scnsegmentheader{Понятие внешнего идентификатора sc-элемента}

\scnstartsubstruct

\scnheader{sc-идентификатор}
\scnidtf{строка символов или пиктограмма, взаимно однозначно представляющая соответствующий sc-элемент, хранимый в sc-памяти}
\scnidtf{внешний идентификатор sc-элемента}
\scntext{пояснение}{Внешние идентификаторы \textit{sc-элементов} (или, сокращенно \scnkeyword{sc-идентификаторы}) необходимы \mbox{\textit{ostis-системе}} исключительно для того, чтобы осуществлять обмен информацией с другими \textit{ostis-системами} или со своими пользователями. Для того чтобы представить свою \textit{базу знаний}, решать самые различные \textit{задачи}, связанные с анализом текущего состояния и эволюцией своей \textit{базы знаний}, задачи, связанные с анализом текущего состояния (текущих ситуаций) окружающей среды, принятием соответствующих решений (целей) и организацией соответствующих \textit{действий}, направленных на выполнение принятых решений (на достижение поставленных целей), \textit{ostis-системе} не нужны никакие внешние идентификаторы (в частности, имена) соответствующие \textit{sc-элементам}. Но для \uline{понимания} сообщений, принимаемых от других субъектов (что для \textit{ostis-системы} означает построение \textit{sc-текста},~~ \textit{семантически эквивалентного} принятому сообщению) и для анализа сообщений, передаваемых другим субъектам (что для \textit{ostis-системы} означает синтез \textit{внешнего текста},~~\textit{семантически эквивалентного} заданному \textit{sc-тексту} и удовлетворяющего некоторым дополнительным требованиям, например, эмоционального характера) \textit{ostis-системе} необходимо знать, как в принимаемом или передаваемом сообщении изображаются (представляются) \textit{знаки}, \uline{синонимичные sc-элементам}, которые уже хранятся или могут храниться в составе \textit{базы знаний}~~\textit{ostis-системы}. В качестве внешних идентификаторов \textit{sc-элементов} чаще всего используются имена (термины) соответствующих (обозначаемых) сущностей, представленные отдельными словами или словосочетаниями на различных естественных языках, но также могут использоваться иероглифы, условные обозначения, пиктограммы.

В общем случае \textit{sc-элементу} может соответствовать несколько синонимичных ему имен на разных \textit{естественных языках}. Более того, \textit{sc-элементу} может соответствовать несколько синонимичных ему имен на одном и том же \textit{естественном языке}. В этом случае одно из этих имен объявляется как основной внешний идентификатор для соответствующего \textit{sc-элемента} и соответствующего \textit{естественного языка}. Основное требование, предъявляемое к таким внешним идентификаторам это отсутствие как синонимов, так и омонимов в рамках множества основных внешних идентификаторов sc-элементов для каждого естественного языка. 

Каждый внешний идентификатор \textit{sc-элемента}, используемый ostis-системой, может быть описан (представлен) в её памяти в виде \textit{внутреннего файла ostis-системы}, т.е. в виде электронного образа всевозможных вхождений данного внешнего идентификатора во всевозможные внешние тексты соответствующего внешнего языка. В некоторых случаях явное представление в памяти не требуется, например, в случае \textit{нетранслируемых sc-идентификаторов}.}
\scnsubdividing{простой sc-идентификатор\\
\scnaddlevel{1}
\scnidtf{простой внешний идентификатор sc-элемента}
\scnaddlevel{-1}
;sc-выражение\\
\scnaddlevel{1}
\scnidtf{сложный внешний идентификатор sc-элемента, в состав которого входит один или несколько идентификаторов других sc-элементов} 
\scnaddlevel{-1}
}
\scnsubdividing{основной sc-идентификатор\\
\scnaddlevel{1}
\scnidtf{основной sc-идентификатор для носителей дополнительно указываемого языка общения (например, соответствующего естественного языка)}
\scntext{примечание}{\textit{основной sc-идентификатор} является уникальным (не имеет синонимов и омонимов) в рамках соответствующего естественного языка}
\scnsuperset{основной международный sc-идентификатор}
\scnaddlevel{1}
\scntext{примечание}{В качестве \textit{основных sc-идентификаторов} могут использоваться также общепринятые международные условные обозначения некоторых сущностей, например, обозначения часто используемых функций (sin, cos, tg, log, и т.д.), единиц измерения, денежных единиц и многое другое. Формально каждый основной международный sc-идентификатор считается основным sc-идентификатором также и для каждого естественного языка, несмотря на то, что символы, используемые в основных международных sc-идентификаторах, могут не соответствовать алфавиту некоторых или даже всех естественных языков.}
\scnaddlevel{-2}
;неосновной sc-идентификатор\\
\scnaddlevel{1}
\scntext{примечание}{С помощью неосновных sc-идентификаторов указываются возможные \textit{синонимы*} соответствующего \textit{основного sc-идентификатора}, которые в частности, могут пояснять или даже определять обозначаемую сущность, указывает на важные свойства этой сущности.}
\scnsuperset{{\normalfont(}неосновной sc-идентификатор $\cap$ пояснение{\normalfont)}}
\scnaddlevel{1}
\scnidtf{неосновной sc-идентификатор, являющийся одновременно и пояснением обозначаемой сущности}
\scnsuperset{{\normalfont(}неосновной sc-идентификатор $\cap$ определение{\normalfont)}}
\scnaddlevel{1}
\scnidtf{неосновной sc-идентификатор, являющийся одновременно и определением обозначаемого понятия}
\scnaddlevel{-2}
\scnsuperset{неосновной часто используемый sc-идентификатор}
\scnaddlevel{1}
\scntext{пояснение}{Для некоторых sc-элементов могут часто использоваться не только основные, но и неосновные sc-идентификаторы (особенно в неформальных текстах -- в пояснениях, примечаниях и т.п.). Явное выделение такого класса sc-идентификаторов позволяет упростить семантический анализ исходных текстов баз знаний.}
\scnaddlevel{-2}
;системный sc-идентификатор\\
\scnaddlevel{1}
\scnexplanation{\textit{cистемный sc-идентификатор} -- это \textit{sc-идентификатор}, являющийся уникальным в рамках всей базы знаний \textit{Экосистемы OSTIS} (\textit{Глобальной базы знаний}). Данный \textit{sc-идентификатор}, как правило, используется в исходных текстах базы знаний, при обмене сообщениями между \textit{ostis-системами}, а также для взаимодействия \textit{ostis-системы} с компонентами, реализованными с использованием средств, внешних с точки зрения \textit{Технологии OSTIS}, например, программ, написанных на традиционных языках программирования. Алфавит системных sc-идентификаторов максимально упрощен для того, чтобы обеспечить удобство автоматической обработки таких sc-идентификаторов с использованием современных технических средств, в частности, запрещены пробелы и различные специальные символы.}
\scnaddlevel{1}
\scntext{примечание}{В качестве указанного языка общения между ostis-системами может использоваться SCs-код.}
\scnaddlevel{-2}
}
\scntext{примечание}{Каждому sc-элементу может соответствовать целое семейство внешних идентификаторов этого sc-элемента, которые обычно являются терминами, именующими обозначаемую сущность. Среди этих внешних идентификаторов для каждого идентифицируемого sc-элемента выделяется один как основной идентификатор. А неосновные термины (имена), соответствующие этим sc-элементам (в том числе и классам), поясняют денотационную семантику указанного sc-элемента.}

\bigskip
\scnstartset
\scnheader{основной sc-идентификатор}
\scnsubset{файл-образец ostis-системы}
\scnendstruct

\scnrelboth{семантическая эквивалентность}{\scnfilelong{Все основные идентификаторы sc-элементов в памяти ostis-системы оформляются в виде копируемых фалов-образцов ostis-системы.}}
\scnaddlevel{1}
\scntext{пояснение}{Копии основных sc-идентификаторов входят в состав внешних текстов различных языков (\mbox{SCg-кода}, SCs-кода, SCn-кода), а также в различных падежах, склонения, спряжениях в состав файлов ostis-систем.}
\scnaddlevel{-1}
\scntext{примечание}{Аналогичное утверждение справедливо и для неосновных часто используемых sc-идентификаторов. Все остальные неосновные sc-идентификаторы считаются вспомогательными файлами-экземплярами.}

\scnheader{sc-идентификатор}
\scnsubdividing{строковый sc-идентификатор\\
\scnaddlevel{1}
\scnidtf{sc-идентификатор, представленный строкой символов, которая является именем обозначаемой сущности}
\scnidtf{имя сущности, обозначаемой идентифицируемым sc-элементом}
\scnidtf{имя (термин, словосочетание), синонимичное соответствующему (идентифицируемому) \mbox{sc-элементу} и представленное в соответствующем алфавите символов}
\scnaddlevel{-1}
;нестроковый sc-идентификатор\\
\scnaddlevel{1}
    \scnexplanation{В общем случае в качестве \textit{sc-идентификатора} некоторого \textit{sc-элемента} может выступать произвольный \textit{внутренний файл ostis-системы}, например, пиктограмма, условное обозначение или даже аудиофрагмент.}
\scnaddlevel{-1}}
\scntext{примечание}{Введенные нами sc-идентификаторы используются во всех внешних языках, близких SC-коду -- в SCg-коде, в SCs-коде и в SCn-коде.}

\scnheader{строковый sc-идентификатор}
\scnidtf{имя, приписываемое идентифицируемому sc-элементу}
\scnidtf{имя сущности, обозначаемой идентифицируемым sc-элементом}
\scnidtf{строка символов, синонимичная соответствующему идентифицируемому sc-элементу}
\scnsuperset{основной строковый sc-идентификатор}
\scnaddlevel{1}
\scnidtf{уникальное для каждого естественного языка внешнее имя, приписываемое идентифицируемому sc-элементу}
\scnsuperset{основной русскоязычный sc-идентификатор}
\scnsuperset{системный sc-идентификатор}
\scnsuperset{основной англоязычный sc-идентификатор}
\scnsuperset{основной германоязычный sc-идентификатор}
\scnsuperset{основной франкоязычный sc-идентификатор}
\scnsuperset{основной италоязычный sc-идентификатор}
\scnsuperset{основной китайскоязычный sc-идентификатор}
\scnaddlevel{-1}
\scnsuperset{системный sc-идентификатор}

\scnheader{sc-идентификатор}
\scntext{примечание}{Представление знаков в виде неатомарных фрагментов информационных конструкций (в частности, в виде имен обозначаемых сущностей, построенных в фиксированном алфавите), взаимно однозначно соответствующих обозначенным сущностям, необходимо только для того, чтобы иметь простую процедуру установления синонимии знаков, входящих в состав одной или разных информационных конструкций.}

\scnheader{системный sc-идентификатор}
\scnrelfromset{правила построения}{\scnfileitem{Символами, использующимися в \textit{системном sc-идентификаторе}, могут быть буквы латинского алфавита, цифры, знак нижнего подчеркивания и знак тире. Для обеспечения интернационализации рекомендуется формировать \textit{системные sc-идентификаторы} на основании основных англоязычных \textit{sc-идентификаторов}. Таким образом, наиболее целесообразно формировать \textit{системный sc-идентификатор} \textit{sc-элемента} из основного англоязычного путем замены всех символов, не входящих в описанный выше алфавит на символ нижнее подчеркивание (``\_''). Кроме того, заглавные буквы чаще всего заменяются на соответствующие строчные.}\\
\scnaddlevel{1}
    \scnrelfrom{пример применения}{\scnstartsetlocal\\
    \scnheaderlocal{Раздел. Предметная область и онтология множеств}
    \scnrelfrom{основной sc-идентификатор}{\scnfilelong{Section. Subject domain and ontology of sets}}
    \scnaddlevel{1}
        \scniselement{Английский язык}
        \scniselement{основной sc-идентификатор}
    \scnaddlevel{-1}
    \scnrelfrom{системный sc-идентификатор}{\scnfilelong{section\_subject\_domain\_and\_ontology\_of\_sets}}
    \scnaddlevel{1}
        \scniselement{системный sc-идентификатор}
    \scnaddlevel{-1}\\
    \scnendstruct}
\scnaddlevel{-1};
\scnfileitem{Для именования sc-элементов, являющихся знаками \textit{ролевых отношений}, вместо знака ``\scnrolesign'' в \textit{системном sc-идентификаторе} используется приставка ``rrel'' и далее после нижнего подчеркивания записывается имя \textit{ролевого отношения}.}\\
\scnaddlevel{1}
    \scnrelfrom{пример применения}{\scnstartsetlocal\\
    \scnheaderlocal{слагаемое\scnrolesign}
    \scnrelfrom{системный sc-идентификатор}{\scnfilelong{rrel\_summand}}
    \scnaddlevel{1}
        \scniselement{системный sc-идентификатор}
    \scnaddlevel{-1}\\
    \scnendstruct}
\scnaddlevel{-1};
\scnfileitem{Для именования sc-элементов, являющихся знаками \textit{неролевых отношений}, вместо знака ``*'' в \textit{системном sc-идентификаторе} используется приставка ``nrel'' и далее после нижнего подчеркивания записывается имя \textit{неролевого отношения}.}\\
\scnaddlevel{1}
    \scnrelfrom{пример применения}{\scnstartsetlocal\\
    \scnheaderlocal{включение*}
    \scnrelfrom{системный sc-идентификатор}{\scnfilelong{nrel\_inclusion}}
    \scnaddlevel{1}
        \scniselement{системный sc-идентификатор}
    \scnaddlevel{-1}\\
    \scnendstruct}
\scnaddlevel{-1};
\scnfileitem{Для именования sc-элементов, являющихся знаками классов \textit{понятий}~~в~~\textit{системном sc-идентификаторе} используется приставка ``concept'' и далее после нижнего подчеркивания записывается имя \textit{класса}.}\\
\scnaddlevel{1}
    \scnrelfrom{пример применения}{\scnstartsetlocal\\
    \scnheaderlocal{треугольник}
    \scnrelfrom{системный sc-идентификатор}{\scnfilelong{concept\_triangle}}
    \scnaddlevel{1}
        \scniselement{системный sc-идентификатор}
    \scnaddlevel{-1}\\
    \scnendstruct}
\scnaddlevel{-1}
\newpage;\scnfileitem{Для именования sc-элементов, являющихся знаками \textit{структур}~~в~~\textit{системном sc-идентификаторе} используется приставка ``struct'' и далее после нижнего подчеркивания записывается имя \textit{структуры}.}\\
\scnaddlevel{1}
    \scnrelfrom{пример применения}{\scnstartsetlocal\\
    \scnheaderlocal{Треугольник(ТочкаA;ТочкаB;ТочкаC)}
    \scnrelfrom{системный sc-идентификатор}{\scnfilelong{struct\_Triangle\_A\_B\_C}}
    \scnaddlevel{1}
        \scniselement{системный sc-идентификатор}
    \scnaddlevel{-1}\\
    \scnendstruct}
\scnaddlevel{-1}
;\scnfileitem{Для именования sc-элементов, являющихся знаками \textit{связок}~~в~~\textit{системном sc-идентификаторе} используется приставка ``tuple'' и далее после нижнего подчеркивания записывается имя \textit{связки}.}}

\scnheader{нетранслируемый sc-идентификатор}
\scnsubset{системный sc-идентификатор}
\scnidtf{sc-идентификатор, не представляемый в базе знаний ostis-системы}
\scnidtf{sc-идентификатор, существующий только вне базы знаний ostis-системы}
\scnexplanation{\textit{нетранслируемые sc-идентификаторы} используются только в рамках исходных текстов \textit{баз знаний} (в том числе, \textit{sc.s-текстов}) и при обмене сообщениями между \textit{ostis-системами} в тех случаях, когда необходимо в нескольких фрагментах исходного текста \textit{базы знаний} или передаваемого сообщения использовать имя одного и того же \textit{sc-элемента}, но при этом указанный \textit{sc-элемент} не имеет \textit{системного sc-идентификатора} и вводить его нецелесообразно. Использование \textit{нетранслируемых sc-идентификаторов} позволяет повысить читабельность и структурированность исходных текстов \textit{баз знаний}, а также позволяет обратиться к одному и тому же неименуемому (в рамках базы знаний) \textit{sc-элементу} в разных файлах исходных текстов \textit{баз знаний} или в разных сообщениях, передаваемых между \textit{ostis-системами}. В качестве таких \textit{sc-элементов} часто выступают знаки \textit{структур} и \textit{связок}.

Таким образом, \textit{нетранслируемые sc-идентификаторы} существуют только вне \textit{базы знаний ostis-системы} и при формировании базы знаний из исходных текстов или при погружении в базу знаний полученного сообщения соответствующий им \textit{внутренний файл ostis-системы} не создается.
}
\scntext{правило построения}{\textit{нетранслируемые sc-идентификаторы} строятся по тем же принципам, что и системные sc-идентификаторы, но в начале \textit{нетранслируемого sc-идентификатора} ставится одна или несколько точек (``.''), количество которых определяет область видимости данного \textit{нетранслируемого sc-идентификатора}.}
\scnaddlevel{1}
    \scnnote{Данное правило справедливо и для построения \textit{sc-идентификаторов}~~~\textit{sc-переменных}}
    \scnaddlevel{1}
        \scnexplanation{..\_var1}
    \scnaddlevel{-1}
\scnaddlevel{-1}
\scnsuperset{глобальный нетранслируемый sc-идентификатор}
\scnaddlevel{1}
    \scnexplanation{\textit{глобальные нетранслируемые sc-идентификаторы} считаются уникальными в рамках всего имеющегося набора исходных текстов \textit{баз знаний} и/или передаваемых между \textit{ostis-системами} сообщений. Таким образом, \textit{sc-элементы}, имеющие на уровне исходных текстов (в том числе, в разных файлах исходных текстов) одинаковые \textit{глобальные нетранслируемые sc-идентификаторы}, считаются синонимичными и в \textit{базе знаний ostis-системы} представляются одним и тем же \textit{sc-элементом}.
    
    Можно сказать, что областью видимости \textit{глобальных нетранслируемых sc-идентификаторов} является весь набор (репозиторий) исходных текстов некоторой базы знаний.
    }
    \scntext{правило построения}{В начале \textit{глобальных нетранслируемых sc-идентификаторов} ставится одна точка}
    \scnhaselementlist{пример}{\scnfileitem{.operator1};\scnfileitem{.node123}}
\scnaddlevel{-1}
\scnsuperset{локальный нетранслируемый sc-идентификатор}
\scnaddlevel{1}
    \scnexplanation{\textit{локальные нетранслируемые sc-идентификаторы} считаются уникальными в рамках \uline{конкретного} файла исходных текстов \textit{баз знаний} и/или передаваемого между \textit{ostis-системами} сообщения. Таким образом, \textit{sc-элементы}, имеющие в рамках одного файла исходных текстов одинаковые \textit{локальные нетранслируемые sc-идентификаторы}, считаются \uline{синонимичными}, в то время как \textit{sc-элементы}, имеющие в рамках \uline{разных} файлов исходных текстов одинаковые \textit{локальные нетранслируемые sc-идентификаторы} будут считаться \uline{разными} \textit{sc-элементами}.
    
    Можно сказать, что областью видимости \textit{локальных нетранслируемых sc-идентификаторов} является конкретный файл исходных текстов базы знаний.}
    \scntext{правило построения}{В начале \textit{глобальных нетранслируемых sc-идентификаторов} ставится две точки}
    \scnhaselementlist{пример}{\scnfileitem{..operator1};\scnfileitem{..node123}}
\scnaddlevel{-1}
\scnsuperset{уникальный нетранслируемый sc-идентификатор}
\scnaddlevel{1}
    \scneqfileclass{...}
    \scnexplanation{\textit{уникальные нетранслируемые sc-идентификаторы} используются для однократного обозначения конкретного неименуемого \textit{sc-элемента} в рамках исходных текстов \textit{баз знаний} и/или передаваемых между \textit{ostis-системами} сообщений.
    
    Кроме того, \textit{уникальные нетранслируемые sc-идентификаторы} могут использоваться при визуализации баз знаний в виде, например, sc.s-текстов или sc.n-текстов, в тех случаях, когда необходимо визуализировать sc-элемент, не имеющий sc-идентификатора.
    
    Каждое вхождение \textit{уникального нетранслируемого sc-идентификатора} в исходный текст \textit{базы знаний} или передаваемое сообщение считается обозначением \uline{разных} \textit{sc-элементов} (чаще всего, \textit{sc-узлов}).}
\scnaddlevel{-1}

\scnheader{sc-идентификатор*}
\scnsuperset{основной sc-идентификатор*}
\scnaddlevel{1}
\scnidtf{Бинарное ориентированное отношение, каждая пара которого связывает \textit{sc-элемент} с внутренним файлом \textit{ostis-системы}, который содержит \textit{основной sc-идентификатор} указанного \textit{sc-элемента}.}
\scnnote{Отношение, связывающее \textit{sc-элементы} с файлами, содержащими их \textit{основные sc-идентификаторы}, само имеет свой \textit{основной sc-идентификатор}, который представляет собой строку символов, имеющую вид ``основной sc-идентификатор*''. Заметим, что в состав первого домена отношения ``\textit{основной sc-идентификатор*}'' входит также и \textit{sc-узел}, обозначающий само это отношение.}
\scnaddlevel{-1}
\scnsuperset{системный sc-идентификатор*}
\scnaddlevel{1}
\scnidtf{Бинарное ориентированное отношение, каждая пара которого связывает \textit{sc-элемент} с внутренним файлом \textit{ostis-системы}, который содержит \textit{системный sc-идентификатор} указанного \textit{sc-элемента}.}
\scnaddlevel{-1}

\scnheader{следует отличать*}
\scnhaselementset{системный sc-идентификатор;основной sc-идентификатор}
\scnaddlevel{1}
\scntext{отличие}{Системные идентификаторы отличаются от основных, во-первых, требованием к уникальности в рамках всей базы знаний \textit{Экосистемы OSTIS} (а, значит, независимостью от внешнего языка), а во-вторых, более простым алфавитом, удобным для автоматической обработки.}
\scnaddlevel{-1}

\scnhaselementset{системный sc-идентификатор;нетранслируемый sc-идентификатор}
\scnaddlevel{1}
\scntext{сходство}{\textit{системные sc-идентификаторы} и \textit{нетранслируемые sc-идентификаторы} выполняют схожие функции, связанные с именованием \textit{sc-элементов} на уровне исходных текстов \textit{баз знаний} или передаваемых между \textit{ostis-системами} сообщений.}
\scntext{отличие}{Каждый \textit{системный sc-идентификатор} представляется в базе знаний в виде \textit{внутреннего файла ostis-системы} и связан с соответствующим \textit{sc-элементом} парой отношения \textit{системный \mbox{sc-идентификатор*}}. \textit{нетранслируемые sc-идентификаторы} не представляются в рамках \textit{базы знаний}, не имеют соответствующих \textit{внутренних файлов ostis-системы} и на уровне \textit{базы знаний} никак не связаны с идентифицируемыми ими \textit{sc-элементами}.}
\scnaddlevel{-1}

\bigskip
\scnendstruct \scnendsegmentcomment{Понятие внешнего идентификатора sc-элемента}

\scnsegmentheader{Понятие простого идентификатора sc-элемента}

\scnstartsubstruct
\scnheader{простой sc-идентификатор}
\scnidtf{идентификатор sc-элемента, в состав которого идентификаторы других sc-элементов не входят и который не содержит \textit{транслируемую в SC-код} информацию об обозначаемой им сущности}
\scnidtf{простой идентификатор sc-элемента}
\scnsuperset{простой строковый sc-идентификатор}
\scnaddlevel{1}
\scnidtf{простой sc-идентификатор, представляющий собой строку (цепочку) символов, которая является именем (названием) той же сущности, что и идентифицируемый sc-элемент}
\scnnote{простые строковые sc-идентификаторы являются наиболее распространненым видом идентификаторов, приписываемых sc-элементам}

\scnheader{простой строковый sc-идентификатор}
\scnsuperset{системный sc-идентификатор}
\scnsuperset{простой строковый идентификатор sc-переменной}
\scnaddlevel{1}
\scnhaselementrole{пример}{\scnfilelong{\_var1}}
\scnaddlevel{-1}
\scnsuperset{простой строковый sc-идентификатор неролевого отношения}
\scnaddlevel{1}
\scnidtf{простой строковый идентификатор sc-узла, являющегося знаком неролевого отношения}
\scnhaselementrole{пример}{\scnfilelong{включение множеств*}}
\scnaddlevel{-1}
\scnsuperset{простой строковый sc-идентификатор ролевого отношения}
\scnaddlevel{1}
\scnidtf{простой строковый идентификатор sc-узла, являющегося знаком ролевого отношения}
\scnhaselementrole{пример}{\scnfilelong{слагаемое\scnrolesign}}
\scnaddlevel{-1}
\scnsuperset{простой строковый sc-идентификатор класса классов}
\scnaddlevel{1}
\scnidtf{простой строковый идентификатор sc-узла, являющегося знаком класса классов}
\scnhaselementrole{пример}{\scnfilelong{длина\scnsupergroupsign}}
\scnaddlevel{-1}
\scnsuperset{sc-идентификатор внешнего файла ostis-системы}
\scnaddlevel{1}
\scnidtf{URL-идентификатор}
\scnexplanation{\textit{sc-идентификаторы} данного класса предназначены для описания местоположения внешних файлов ostis-систем и представляют собой строку символов, которая строится в соответствии со стандартом URL, а затем берется в двойные кавычки. Кавычки нужны для однозначности определения того, где начинается и заканчивается данный sc-идентификатор, поскольку в общем случае в URL разрешены пробелы. Целесообразность этого обусловлена тем, что sc-идентификаторы данного типа часто используются в файлах исходных текстов баз знаний ostis-систем.

\textit{sc-идентификаторы} внешних файлов ostis-систем с точки зрения Технологии OSTIS являются простыми строковыми sc-идентификаторами, хотя и могут содержать специальные символы, например "\%" или "/". Это связано с тем, что указанные идентификаторы не несут в себе семантически значимой информации о свойствах самого sc-элемента, обозначаемого таким sc-идентификатором, а только информацию о его расположении в текущем состоянии внешнего мира ostis-системы.}
\scnhaselementrole{пример}{\scnfilelong{"file:///home/user/image1.png"}}
    \scnaddlevel{1}
        \scnnote{Данный sc-идентификатор описывает абсолютный путь к файлу под названием "image1.png"}
    \scnaddlevel{-1}
\scnhaselementrole{пример}{\scnfilelong{"file://image1.png"}}
    \scnaddlevel{1}
        \scnnote{Данный sc-идентификатор описывает относительный путь к файлу под названием "image1.png"}
    \scnaddlevel{-1}
\scnhaselementrole{пример}{\scnfilelong{"https://conf.ostis.net/content/image1.png"}}
\newpage
    \scnaddlevel{1}
        \scnnote{Данный sc-идентификатор описывает путь к файлу под названием "image1.png", расположенному на удаленном сервере.}
    \scnaddlevel{-1}
\scnaddlevel{-1}

\scnheader{имя нарицательное}
\scnidtf{простой строковый sc-идентификатор, являющийся именем нарицательным}
\scnaddlevel{1}
\scniselement{имя нарицательное}
\scnaddlevel{-1}
\scnidtf{Множество всевозможных имен нарицательных}
\scnaddlevel{1}
\scniselement{имя собственное}
\scnaddlevel{-1}
\scnidtf{имя некоторого класса сущностей (а, точнее, имя класса sc-элементов, обозначающих сущности некоторого класса), которое, с одной стороны, является знаком всего указанного класса, а, с другой стороны, соответствует (может быть приписано) любому экземпляру этого класса}
\filemodetrue
\scnrelfromset{примеры}{треугольник; отношение\scnsupergroupsign{}; разбиение*; параметр\scnsupergroupsign{}; длина\scnsupergroupsign{}; язык\scnsupergroupsign{}; sc-текст; sc-константа; sc-переменная; персона; публикация; город}
\filemodefalse
\scnnote{\textit{имя нарицательное} всегда начинается с маленькой буквы}

\scnheader{имя собственное}
\scnidtf{\textit{простой строковый sc-идентификатор}, являющийся именем собственным}
\scnaddlevel{1}
\scniselement{имя нарицательное} 
\scnaddlevel{-1}
\scnidtf{Множество всевозможных имен собственных}
\scnaddlevel{1}
\scniselement{имя собственное}
\scnaddlevel{-1}
\scnsuperset{sc-идентификатор внешнего файла ostis-системы}
\scnidtfexp{имя, которое либо не является обозначением какого-либо класса сущностей, либо является обозначением (именем) некоторого класса сущностей, но построенным без использования нарицательного имени этого класса, либо является именем некоторого класса сущностей, построенным с использованием нарицательного имени этого класса либо путем преобразования имени нарицательного во множественное число, либо путем дополнительного использования в начале формулируемого имени таких слов или словосочетаний, как ``Знак класса'', ``Класс'', ``Знак множества'', ``Множество'', ``Знак множества всевозможных'', ``Множество всевозможных''}
\scnnote{\textit{имя собственное} всегда начинается с большой буквы}
\scnrelfromset{примеры}{\scnfileitem{Иванов Петр Николаевич};
\scnfileitem{Минск}\\
\scnaddlevel{1}
\scnrelboth{синонимия внешних идентификаторов}{\scnfilelong{Город/Минск}}
\scnaddlevel{-1};
\scnfileitem{Точка/А}
\newpage;\scnfileitem{SC-код}\\
\scnaddlevel{1}
\scnrelboth{синонимия внешних идентификаторов}{\scnfilelong{Знак множества всевозможных sc-текстов}}
\scnrelboth{синонимия внешних идентификаторов}{\scnfilelong{Множество всевозможных sc-текстов}}
\scnrelboth{синонимия внешних идентификаторов}{\scnfilelong{Класс sc-текстов}}
\scnrelboth{синонимия внешних идентификаторов}{\scnfilelong{sc-текст}}
\scnaddlevel{1}
\scniselement{имя нарицательное}
\scnaddlevel{-2}}

\scnheader{простой строковый sc-идентификатор}
\scnrelfrom{правила построения}{Правила построения простых строковых sc-идентификаторов}
\scnaddlevel{1}
\scnexplanation{\textit{Правила построения простых строковых sc-идентификаторов} включают в себя:
\begin{scnitemize}
    \item \textit{Алфавит символов, используемых в простых строковых sc-идентификаторах};
    \item Префиксы и суффиксы, используемые в простых строковых sc-идентификаторах;
    \item Разделители и ограничители, используемые в простых строковых sc-идентификаторах;
    \item Правила построения \textit{имен собственных} и \textit{имен нарицательных}, являющихся простыми строковыми sc-идентификаторами;
    \item Правила построения простых строковых sc-идентификаторов, определяемые различными классами идентифицируемых sc-элементов.
\end{scnitemize}}\bigskip
\scneqtovector{\scnfileitem{Общим правилом построения \textit{простых sc-идентификаторов} является стремление максимально возможным образом использовать сложившуюся терминологию. Но при этом следует подчеркнуть, что необходимость исключения омонимии в \textit{sc-идентификаторах} требует строгого формального \uline{уточнения} семантической интерпретации каждого используемого термина. Особо подчеркнем то, что в \textit{ostis-системах} процесс построения новых терминов (\textit{sc-идентификаторов}) и процесс совершенствования существующей терминологии по отношению к процессу эволюции \textit{баз знаний}, представленных в \textit{SC-коде}, с технической точки зрения абсолютно не зависят друг от друга. Кроме того, следует помнить, что \uline{далеко не все} \textit{sc-элементы}, входящие в состав базы знаний ostis-системы, должны иметь соответствующие им sc-идентификаторы (быть идентифицированными). Существенно подчеркнуть то, что далеко не все sc-элементы должны иметь \textit{простые sc-идентификаторы} по той простой причине, что для многих sc-элементов в случае необходимости можно достаточно легко построить идентифицирующее их \textit{sc-выражение} (sc-выражение). Тем не менее, для целого ряда сущностей (например, для понятий, исторических событий и т.п.) обойтись без простых sc-идентификаторов очень сложно. Очевидно, что идентифицированными (именованными) должны быть все используемые понятия, вводимые в соответствующих предметных областях и специфицируемые соответствующими онтологиями. Идентифицированными также должны быть обладающие особыми свойствами ключевые экземпляры (элементы) некоторых понятий, различные социально значимые объекты (персоны, населенные пункты, географические объекты, страны, организации, библиографические источники, исторические события и многое другое).};
\scnstartsetlocal\\
\scnaddlevel{1}
\scnheaderlocal{простой sc-идентификатор}
\scnaddhind{-1}
\scnrelfrom{алфавит}{Алфавит символов, используемых в простых строковых sc-идентификаторах}
\scnaddlevel{1}
\scnsuperset{Объединение заглавных и строчных букв алфавитов всевозможных естественных языков}
\scnsuperset{Алфавит символов, используемых в URL}
\scnsuperset{цифра}
\scnhaselement{запятая}
\scnhaselement{точка}
\scnhaselement{прямые кавычки}
\scnaddlevel{1}
    \scneqfileclass{ \dq }
    \scnidtf{ограничитель метафорических словосочетаний}
    \scnnote{указанный символ также может использоваться и во внутренних файлах ostis-систем, и в рамках нетранслируемых комментариев}
\scnaddlevel{-1}
\scnhaselement{знак подчеркивания}
\scnaddlevel{1}
\scnidtf{знак нижнего подчеркивания}
\scnnote{используется в начале sc-идентификаторов sc-переменных. При этом 
\begin{scnitemize}
 \item если в начале sc-идентификатора sc-переменной стоит один знак подчеркивания, то по умолчанию такая sc-переменная считается \textit{первичной sc-переменной}, если явно не указано иное\char59
 \item если в начале sc-идентификатора sc-переменной стоит подряд два знака подчеркивания, то по умолчанию такая sc-переменная считается \textit{вторичной sc-переменной}\char59
 \item если необходимо использовать \textit{sc-переменную третьего уровня} или \textit{sc-переменную, значения которой имеют различный логический уровень}, то в начале sc-идентификатора такой sc-переменной ставится один знак подчеркивания, а ее принадлежность соответствующему классу sc-переменных указывается явно
\end{scnitemize}}
\scnaddlevel{-1}
\scnhaselement{звездочка}
\scnaddlevel{1}
\scnidtf{надстрочная звездочка}
\scneqfileclass{*}
\scnnote{используется в конце \textit{простых sc-идентификаторов}, соответствующих \textit{неролевым отношениям}}
\scnaddlevel{-1}
\scnhaselement{штрих}
\scnaddlevel{1}
\scneqfileclass{\scnrolesign}
\scnidtf{прямой апостроф}
\scnnote{используется в конце \textit{простых sc-идентификаторов}, соответствующих \textit{ролевым отношениям}}
\scnaddlevel{-1}
\scnhaselement{циркумфлекс}
\scnaddlevel{1}
\scneqfileclass{\scnsupergroupsign}
\scnidtf{"крышка"}
\scnnote{используется в конце \textit{простых sc-идентификаторов}, соответствующих \textit{классам классов}}
\scnaddlevel{-1}
\scnhaselement{дефис}
\scnaddlevel{1}
\scneqfileclass{-}
\scnaddlevel{-1}
\scnhaselement{косая черта}
\scnaddlevel{1}
\scneqfileclass{/}
\scnidtf{слэш}
\scnnote{используется при формировании \textit{простых sc-идентификаторов} для экземпляров некоторого заданного \textit{класса} путем конкатенации \mbox{\textit{sc-идентификатора}} указанного \textit{класса} и некоторого sc-иден\-ти\-фи\-ка\-то\-ра (необязательно уникального, часто условного), соответствующего рассматриваемому экземпляру указанного класса}
\scnrelfromlist{пример применения}{\scnfileitem{\textit{Точка/А}};\scnfileitem{\textit{Множество/Si}};\scnfileitem{\textit{Город/Минск}};\scnfileitem{\textit{Озеро/Нарочь}}}
\scnaddlevel{-1}\\
\scnendstruct\\
\scnaddlevel{-2};
\scnfileitem{Первым символом каждого \textit{простого строкового sc-идентификатора}, идентифицирующего \textit{sc-переменную} (переменный \textit{sc-элемент}), является знак подчеркивания. При этом, если после указанного знака подчеркивания указывается \textit{имя нарицательное} некоторого \textit{класса sc-элементов}, то рассматриваемый \textit{простой строковый sc-идентификатор} становится уже \textit{sc-выражением}, содержащим информацию о том, что \textit{областью возможных значений*} идентифицируемой \textit{sc-переменной} является указанный \textit{класс sc-элементов}.};
\scnfileitem{Последним символом простого sc-идентификатора, идентифицирующего sc-узел, обозначающий неролевое отношение, заданное на множестве sc-элементов, является надстрочная \textit{звездочка}.};
\scnfileitem{Последним символом простого sc-идентификатора, идентифицирующего sc-узел, обозначающий заданное на множестве sc-элементов ролевое отношение (т.е. отношение, являющееся подмножеством отношения принадлежности), является \textit{штрих} (прямой апостроф)};
\scnfileitem{Последним символом простого sc-идентификатора, идентифицирующего sc-узел, обозначающий понятие, являющееся классом классов (таковыми в частности, являются различного рода параметры -- длина, площадь, объем, масса) является символ \textit{циркумфлекс} ("крышка"). Однако, если отсутствует омонимичный идентификатор без этого суффикса, то указанный символ можно не приписывать.};
\scnfileitem{Простой строковый sc-идентификатор может рассматриваться как последовательное \uline{сокращение} sc-идентификаторов одного и того же sc-элемента с преобразованием имен собственных в имена нарицательные и наоборот.}
\scnaddlevel{1}
\scnrelfrom{пример}{\scnstartsetlocal\\
\scnheaderlocal{множество}
\scnidtf{SC-узел, являющийся знаком множества всевозможных множеств, элементами которых являются sc-элементы}
\scnaddlevel{1}
\scnhaselement{имя собственное}
\scnnote{здесь можно убрать слова "SC-узел, являющийся"{}}
\scnaddlevel{-1}
\scnidtf{Знак множества всевозможных множеств sc-элементов}
\scnaddlevel{1}
\scnhaselement{имя собственное}
\scnnote{здесь можно убрать слово "Знак"{}}
\scnaddlevel{-1}
\scnidtf{Множество всевозможных множеств sc-элементов}
\scnaddlevel{1}
\scnhaselement{имя собственное}
\scnnote{здесь можно убрать слова "Множество всевозможных"{}}
\scnaddlevel{-1}
\scnidtf{множество sc-элементов}
\scnaddlevel{1}
\scnhaselement{имя нарицательное}
\scnaddlevel{-1}
\scnidtf{sc-множество}
\scnaddlevel{1}
\scnhaselement{имя нарицательное}
\scnaddlevel{-1}
\scnidtftext{основной sc-идентификатор}{множество}
\scnaddlevel{1}
\scnnote{такое сокращение возможно, т.к. \uline{любое} множество можно представить в виде sc-множества}\\
\scnaddlevel{-1}
\scnendstruct
\scnaddlevel{-1}};
\scnfileitem{При наличии синонимичных \textit{имен собственных} и \textit{имен нарицательных} при выборе \textit{основного sc-идентификатора} преимущество имеют \textit{имена нарицательные}. Так, например, основным идентификатором sc-узла, обозначающего \textit{SC-код}, является термин ``\textit{sc-текст}'', а термин ``\textit{SC-код}'', являющийся \textit{именем собственным}, считается \textit{неосновным часто используемым sc-идентификатором}. Подчеркнем при этом, что имя нарицательное -- это всегда имя некоторого класса сущностей (в частности, понятия). В конце этого имени может быть указан либо признак класса классов, либо признак неролевого отношения (класса связок), либо признак ролевого отношения (подмножества отношения принадлежности), либо не указано ничего. Последнее означает, что именуется класс, не являющийся ни классом классов, ни классом связок. А это, в свою очередь, означает, что именуемым классом является либо класс первичных (терминальных) сущностей, либо класс структур.};
\scnfileitem{Одно и тоже словосочетание, которому приписываются разные дополнительные признаки, может быть основой для внешних идентификаторов разных sc-элементов}
\scnaddlevel{1}
\scnrelfrom{пример}{\scnstartsetlocal\\
\bigskip
\scnheaderlocal{следует отличать*}
\scnhaselementset{элемент информационной конструкции\\
\scnaddlevel{1}
    \scnidtfexp{Множество всевозможных элементов всевозможных информационных конструкций}
\scnaddlevel{-1};
Элемент информационной конструкции/i\\
\scnaddlevel{1}
    \scnidtfexp{Некоторый конкретный элемент некоторой информационной конструкции}
\scnaddlevel{-1};
элемент информационной конструкции*\\
\scnaddlevel{1}
    \scnidtfexp{Отношение, связывающее информационные конструкции с множествами элементов этих информационных конструкций}
\scnaddlevel{-1};
Элемент информационной конструкции*/i\\
\scnaddlevel{1}
    \scnidtfexp{Некоторая конкретная пара (связка), принадлежащая Отношению, связывающему информационные конструкции с множествами элементов этих информационных конструкций}
    \scnidtfexp{Некоторая конкретная пара (связка), связывающая некоторую информационную конструкцию с множеством ее элементов}
\scnaddlevel{-1};
элемент информационной конструкции\scnrolesign\\
\scnaddlevel{1}
    \scnidtfexp{Ролевое отношение, связывающее информационные конструкции с конкретными элементами этих информационных конструкций}
\scnaddlevel{-1};
Элемент информационной конструкции\scnrolesign/i\\
\scnaddlevel{1}
    \scnidtfexp{Пара (связка) отношения принадлежности, связывающая некоторую конкретную информационную конструкцию с ее конкретным элементом}
\scnaddlevel{-1};
\_элемент информационной конструкции/i\\
\scnaddlevel{1}
    \scnidtfexp{\textit{sc-переменная}, значениями которой могут быть произвольные элементы произвольных информационных конструкций}
\scnaddlevel{-1};
\_элемент информационной конструкции*/i\\
\scnaddlevel{1}
    \scnidtfexp{\textit{sc-переменная}, значениями которой могут пары (связки) отношения ``элемент информационной конструкции*''}
\scnaddlevel{-1};
\_элемент информационной конструкции\scnrolesign/i\\
\scnaddlevel{1}
    \scnidtfexp{\textit{sc-переменная}, значениями которой могут пары принадлежности, принадлежащие ролевому отношению ``элемент информационной конструкции\scnrolesign''}
\scnaddlevel{-1}}\\
\scnendstruct\\
\scnaddlevel{-1}};
\scnfileitem{В рамках \textit{SC-кода} целесообразно вводить правила унифицированного построения \textit{простых sc-идентификаторов} и целого ряда других классов идентифицируемых сущностей -- \textit{персон}, \textit{библиографических источников} (публикаций), \textit{разделов баз знаний ostis-систем}, \textit{файлов ostis-систем}, самих \textit{ostis-систем}.}
}
\scnaddlevel{-1}

\scnendstruct \scnendsegmentcomment{Понятие простого идентификатора sc-элемента}

\scnsegmentheader{Понятие сложного идентификатора sc-элемента}

\scnstartsubstruct

%\vspace{3\baselineskip}
\newpage
\scnheader{sc-выражение}
\scnidtf{идентификатор, который не только обозначает соответствующую сущность, но также содержит информацию, представляющую собой по возможности однозначную спецификацию указанной сущности}
	\scnaddlevel{1}
	\scnnote{Однозначную спецификацию сущности, которая является понятием, называют \uline{определением} этого понятия.}
	\scnaddlevel{-1}
\scnidtf{имя соответствующей {\normalfont(}именуемой{\normalfont)} сущности построенное по принципу "та {\normalfont(}тот{\normalfont)}, которая {\normalfont(}который{\normalfont)} указываемым образом связана с другими указываемыми сущностями"{}}
\scnidtf{выражение, идентифицирующее sc-элемент}
\scnidtf{идентификатор sc-элемента, в состав которого входят другие идентификаторы и денотационная семантика которого точно определяется конкретным набором правил построения таких сложных (комплексных) идентификаторов, состоящих из определенным образом связанных между собой других идентификаторов}
\scnidtf{сложный идентификатор, состоящий из других идентификаторов}
\scnidtf{идентификатор, который представляет собой конструкцию, состоящую из нескольких других идентификаторов, а также из некоторых разделителей и ограничителей, и денотационная семантика которого \uline{однозначно} задается конфигурацией указанной конструкции}
\scnidtf{сложный sc-идентификатор}
\scnidtf{сложный (составной) внешний идентификатор sc-элемента}
\scnidtf{выражение, идентифицирующее sc-элемент}
\scnidtf{sc-идентификатор, в состав которого входит один или несколько простых sc-идентификаторов}
\scnsubdividing{sc-выражение неориентированного множества\\
	\scnaddlevel{1}
	\scnidtf{sc-выражение в фигурных скобках}
	\scnidtf{sc-выражение, ограниченное фигурными скобками}
	\scnsubdividing{sc-выражение множества, заданного перечислением\\
		\scnaddlevel{1}
		\scnidtf{sc-выражение, обозначающее множество, все элементы которого перечислены своими sc-идентификаторами}
		\scnexplanation{В качестве разделителя sc-идентификаторов sc-элементов, входящих в указанное множество может выступать точка с запятой (";") или круглый маркер ("$\bullet$").}
		\scnhaselementrole{пример}{\scnfilelong{\scnset{\textit{e1}; \textit{e2}; \textit{e3}}}}
		\scnaddlevel{1}
		\scnrelboth{семантическая эквивалентность}{\scnfilelong{Дано множество с элементами \textbf{\textit{e1}}, \textbf{\textit{e2}},\textbf{\textit{e3}}.}}%TODO пример в SCg
		\scnrelboth{семантическая эквивалентность}{\scnfilescg{figures/intro/idtf/sc_expr_example_set.png}}
			\scnaddlevel{1}
			\scniselement{sc.g-текст}
		\scnaddlevel{-1}
	\scnaddlevel{-1}
		\scnaddhind{-2}
		\scnaddhind{1}
		\scnsuperset{sc-выражение множества, заданного перечислением элементов с указанием их роли}
		\scnaddhind{-1}
	\scnaddlevel{1}
	\scnnote{В рамках такого sc-выражения можно не только перечислять \mbox{sc-идентификаторы} элементов множества, обозначаемого рассматриваемым sc-выражением, но и указывать роль этих элементов в рамках обозначаемого множества.}
	\scnhaselementrole{пример}{\scnfilelong{\scnset{\textit{ai}\scnrolesign : \textit{e1}; \textit{aj}\scnrolesign : \textit{e2}; \textit{e3}}}}
		\scnaddlevel{1}
		\scnrelboth{семантическая эквивалентность}{\scnfilelong{Дано множество с элементами \textbf{\textit{e1}}, \textbf{\textit{e2}}, \textbf{\textit{e3}}. При этом элемент \textbf{\textit{e1}} в рамках указанного множества имеет атрибут (роль) \textbf{\textit{ai\scnrolesign}}, а элемент \textbf{\textit{e2}} -- атрибут (роль) \textbf{\textit{aj\scnrolesign}}.}}%TODO пример в SCg
		\newpage
		\scnrelboth{семантическая эквивалентность}{\scnfilescg{figures/intro/idtf/sc_expr_example_role_set.png}}
			\scnaddlevel{1}
			\scniselement{sc.g-текст}
		\scnaddlevel{-1}
	\scnaddlevel{-1}
\scnaddlevel{-1}
	\scnaddlevel{-1}
	;sc-выражение структуры\\
		\scnaddlevel{1}
		\scnidtf{sc-выражение, обозначающее структуру, представленную на любом известном и легко определяемом языке (Русском, Английском, SCg-коде, SCs-коде, SCn-коде)}
		\scnexplanation{sc-выражение структуры обозначает структуру, содержащую sc-текст, семантически эквивалентный тому тексту (на некотором известном языке), который заключен в фигурные скобки. Чаще всего такой текст записывается на формальном языке, например, SCs-коде, и может быть автоматически однозначно интерпретирован. Возможна ситуация, когда указанный текст записан на менее неформальном языке, например, Русском, но в этом случае его автоматическая интерпретация значительно усложняется и в общем случае не всегда однозначна.}
		\scnnote{В текущей реализации средств разработки исходных текстов баз знаний в соответствии с более старой версией правил простроения sc-выражений вместо фигурных скобок \textit{sc-выражение структуры} ограничивается квадратными скобками со звездочками ("[*" и "*]")}
		\scnaddlevel{-1}
	}
\scnaddlevel{-1}
;sc-выражение ориентированного множества\\
	\scnaddlevel{1}
	\scnidtf{sc-выражение кортежа}
	\scnidtf{sc-выражение, ограничиваемое угловыми скобками и обозначающее упорядоченное (ориентированное) множество sc-элементов, порядок которых задаётся последовательностью перечисляемых их sc-идентификаторов}
	\scnhaselementrole{пример}{\scnfilelong{\scnvector{\textit{e1}; \textit{e2}; \textit{e3}; \textit{e4}}}}
		\scnaddlevel{1}
		\scnrelboth{семантическая эквивалентность}{\scnfilelong{\scnset{\textit{1\scnrolesign}: \textit{e1};  \textit{2\scnrolesign}: \textit{e2};  \textit{3\scnrolesign}: \textit{e3};  \textit{4\scnrolesign}: \textit{e4}}}}
	    \scnaddlevel{1}
			\scnnote{Здесь \textit{1\scnrolesign} -- это ролевое отношение ``быть первым компонентом кортежа'', \textit{2\scnrolesign} -- ролевое отношение ``быть вторым компонентом кортежа''.}
		\scnaddlevel{-1}
	\scnaddlevel{-1}
\scnaddlevel{-1}
;sc-выражение внутреннего файла ostis-системы\\
	\scnaddlevel{1}
	\scnidtf{sc-выражение, ограниченное квадратными скобками}
	\scnidtf{sc-выражение, обозначающее \textit{внутренний файл ostis-системы}, визуальное представление (изображение) которого ограничивается квадратными скобками}
	\scnsubset{sc-идентификатор внутреннего файла ostis-системы}
	\scnexplanation{\textit{sc-выражение внутреннего файла ostis-системы} обозначает \textit{внутренний файл ostis-системы}, содержимое которого заключено в квадратные скобки, ограничивающие данное sc-выражение.}
	\scnnote{Дополнительная спецификация \textit{внутреннего файла ostis-системы} легко осуществляется с помощью \textit{SC-кода}. Сюда входит \textit{язык}, на котором представлена \textit{информационная конструкция}, являющаяся содержимым \textit{файла}, формат кодирования, \textit{автор}* и многое другое.}
\scnaddlevel{-1};
sc-выражение, обозначающее файл-образец ostis-системы\\
    \scnaddlevel{1}
    \scnidtf{sc-выражение, ограниченное ограниченное квадратными скобками с восклицательными знаками}
    \newpage
    \scnsuperset{файл-образец}
    \scnaddlevel{1}
        \scnrelfrom{смотрите}{Принципы SC-кода}
        \scnaddlevel{1}
        \scniselement{сегмент раздела базы знаний}
    \scnaddlevel{-2}
    \scnhaselementrole{пример}{\scnfileclass{-}}
    \scnhaselementrole{пример}{\scnfileclass{$\sim$}}
    \scnaddlevel{-1}
;sc-выражение, построенное на основе бинарного отношения\\ 
	\scnaddlevel{1}
	\scnidtf{sc-выражение, в состав которого входят \uline{либо} (1) sc-идентификатор, обозначающий бинарное ориентированное отношение, и (2) в круглых скобках sc-идентификатор одного из элементов первого домена указанного бинарного ориентированного отношения, \uline{либо} (1) sc-идентификатор, обозначающий бинарное \uline{не}ориентированное отношение и (2) в круглых скобках sc-идентификатор одного из элементов области определения указанного бинарного неориентированного отношения}
	\scnidtf{sc-выражение, построенное путём указания некоторого бинарного отношения (обычно функционального) и одного из его аргументов (в круглых скобках)}
	\scnhaselementrole{пример}{\scnfilelong{\textit{центр*}(\textit{O})}}
    \scnsuperset{sc-выражение, построенное на основе квазибинарного отношения}
	\scnaddlevel{1}
	\scnidtf{sc-выражение, построенное путем указания некоторого sc-идентификатора квазибинарного отношения (обычно функционального) и (в фигурных скобках или угловых скобках) неупорядоченного или упорядоченного перечня аргументов указанного отношения}
	\scntext{примеры}{\textit{объединение}*($si$; $sj$; $sk$); \textit{пересечение}*($si$; $sj$; $sk$); \textit{разность множеств}*($si$; $sj$); \textit{сумма*}\{$x$; $z$\}; \textit{произведение*}($x$; $y$; $z$); \textit{sin*}($x$); \textit{cos*}($x$)}
\scnaddlevel{-2}
;sc-выражение, построенное на основе алгебраической операции\\
	\scnaddlevel{1}
	\scnidtf{sc-выражение, ограниченное круглыми скобками и построенное путем указания sc-идентификаторов, разделенных знаком алгебраической операции}
	\scnnote{Для каждого sc-выражения данного вида существует синонимическое ему sc-выражение, построенное на основе квазибинарного отношения}
	\scntext{примеры}{($s_i \cup s_j \cup s_k$); ($s_i \cap s_j \cap s_k$); ($s_i \backslash s_j$); ($x+y+z$); ($x \times y \times z$)}
\scnaddlevel{-1}
;sc-выражение, идентифицирующее sc-коннектор\\
\scnaddlevel{1}
    \scnidtf{sc-выражение, ограниченное круглыми скобками и идентифицирующее sc-коннектор, инцидентный двум указанным sc-элементам и имеющий тип, задаваемый путем изображения соответствующего sc.s-коннектора}
    \scntext{принцип}{Для упрощения восприятия и обработки \textit{sc-выражений, идентифицирующих sc-коннектор} вводится следующее ограничение: первым и третьим компонентом такого sc-выражения может быть только \textit{простой sc-идентификатор}. В рамках sc.s-текстов внутри \textit{sc-выражений, идентифицирующих sc-коннектор} допускается также использование sc.s-модификаторов.}
    \scnhaselementrole{пример}{\scnfilelong{(\textbf{\textit{e1}} $\in$ \textbf{\textit{s1}})}}
	\scnaddlevel{1}
	    \scnrelboth{семантическая эквивалентность}{\scnfilelong{(\textbf{\textit{e1}} <- \textbf{\textit{s1}})}}
		\scnrelboth{семантическая эквивалентность}{\scnfilelong{Дана базовая sc-дуга, связывающая sc-элемент \textbf{\textit{e1}} с sc-элементом \textbf{\textit{s1}}.}}
		\scnrelboth{семантическая эквивалентность}{\scnfilescg{figures/intro/idtf/sc_expr_example_affiliation.png}}
			\scnaddlevel{1}
			\scniselement{sc.g-текст}
		\scnaddlevel{-1}
\scnaddlevel{-2}
}
\newpage
\scnexplanation{Использование sc-выражений позволяет существенно сократить число "придумываемых"\ sc-идентификаторов, каковыми в конечном счете становятся только простые sc-идентификаторы, поскольку, зная то, как связан идентифицируемый sc-элемент с теми sc-элементами, которые уже имеют sc-идентификаторы, во многих случаях можно построить sc-выражение, идентифицирующее указанный sc-элемент. Кроме того, каждое sc-выражение, являясь внешним идентификатором, является также и \uline{транслируемым} формальным текстом, содержащим некоторую информацию об обозначаемой ею сущности.}
\scnrelfrom{правило интерпретации}{\scnfilelong{Очевидно, что некоторые sc-выражения могут интерпретироваться неоднозначно. Например, два одинаковых \textit{sc-выражения, идентифицирующих sc-коннектор}, могут изображать как один и тот же sc-коннектор, так и кратные sc-коннекторы одного и того же вида, связывающие одни и те же два sc-элемента. Аналогичная неоднозначность может возникать при использовании \textit{sc-выражений, построенных на основе бинарного отношения} (\textit{подмножество*(S)}) и ряда других \textit{sc-выражений}.

В то же время, некоторые sc-выражения являются однозначными, то есть всегда идентифицируют одну и ту же сущность в любом тексте, в состав которого входят. Например выражение "\textit{sin(x)}" является однозначным (при условии, что "x" в разных текстах обозначает одно и то же). Однако, однозначность или неоднозначность sc-выражений зависит от их семантики, таким образом установить факт однозначности на уровне внешнего текста достаточно сложно.

Для решения проблемы неоднозначности интерпретации sc-выражений такого рода будем считать, что каждое вхождение какого-либо sc-выражения в различные тексты (например, sc.s-тексты) обозначает \uline{разные} sc-элементы. После трансляции таких текстов в sc-текст может оказаться, что некоторые из указанных sc-выражений на самом деле обозначали одну и ту же сущность, в этом случае соответствующие sc-элементы будут "склеены"\, но уже на уровне sc-текста, а не на уровне внешнего текста.

Следует отметить, что факт совпадения sc-выражений в рамках некоторого внешнего текста может являться поводом для анализа идентифицируемых этими sc-выражениями sc-элементов на возможную синонимичность и явно фиксироваться, например, на этапе погружения внешнего текста в sc-память.}}
\scnaddlevel{1}
    \scnrelfrom{пояснение}{\scnfilescg{figures/intro/idtf/sc_expr_example_subsets.png}}
    \scnrelfrom{пояснение}{\scnfilescg{figures/intro/idtf/sc_expr_example_arcs.png}}
\scnaddlevel{-1}

\scnheader{ограничитель sc-выражений}
\scnidtf{ограничитель, используемый в sc-выражениях}
\scnidtf{ограничитель, ограничивающий sc-выражения или их компоненты}
\scnsubdividing{левый ограничитель sc-выражений\\
    \scnaddlevel{1}
    \scnidtf{начальный ограничитель sc-выражений}
    \scnidtf{открывающий ограничитель sc-выражений}
    \scnaddlevel{-1}
;правый ограничитель sc-выражений\\
    \scnaddlevel{1}
    \scnidtf{конечный ограничитель sc-выражений}
    \scnidtf{закрывающий ограничитель sc-выражений}
    \scnaddlevel{-1}}
\newpage
\scnrelboth{пара пересекающихся множеств}{фигурная скобка}
    \scnaddlevel{1}
        \scneq{{\normalfont(}\scnfileclass{ \{ } $\cup$ \scnfileclass{ \} }{\normalfont)}}
        \scnexplanation{Фигурные скобки ограничивают sc-выражение, обозначающее либо множество sc-элементов, задаваемое путем перечисления sc-идентификаторов его элементов (при необходимости с указанием ролей, под которыми эти элементы входят в указанное множество), либо множество sc-элементов, входящих в состав sc-текста, который является результатом трансляции в SC-код текста (например, sc.s-текста или ея-текста), ограниченного фигурными скобками.}
    \scnaddlevel{-1}
\scnrelboth{пара пересекающихся множеств}{квадратная скобка}
    \scnaddlevel{1}
        \scneq{{\normalfont(}\scnfileclass{ [ } $\cup$ \scnfileclass{ ] }{\normalfont)}}
        \scnexplanation{Квадратные скобки ограничивают sc-выражение, обозначающее файл-экземпляр ostis-системы, содержимое (тело) которого изображается и ограничивается указанными квадратными скобками.}
    \scnaddlevel{-1}
\scnrelboth{пара пересекающихся множеств}{квадратная скобка со звездочкой}
    \scnaddlevel{1}
        \scniselement{устаревающее понятие}
        \scneq{{\normalfont(}\scnfileclass{ [* } $\cup$ \scnfileclass{ *] }{\normalfont)}}
        \scnexplanation{Данный ограничитель в текущей версии средств разработки исходных текстов баз знаний используется в качестве ограничителя \textit{sc-выражений sc-структур} вместо фигурных скобок. Такой способ записи \textit{sc-выражений sc-структур} является устаревающим, однако используется в настоящее время.}
    \scnaddlevel{-1}
\scnrelboth{пара пересекающихся множеств}{квадратная скобка с восклицательным знаком}
    \scnaddlevel{1}
        \scneq{{\normalfont(}\scnfileclass{ ![ } $\cup$ \scnfileclass{ ]! }{\normalfont)}}
        \scnexplanation{Данный ограничитель sc-выражений ограничивает sc-выражение, обозначающее файл-класс \mbox{ostis-системы}, содержимое (тело) которого изображается и ограничивается указанными sc-ограничителями.}
    \scnaddlevel{-1}
\scnrelboth{пара пересекающихся множеств}{круглая скобка}
    \scnaddlevel{1}
        \scneq{{\normalfont(}\scnfileclass{ ( } $\cup$ \scnfileclass{ ) }{\normalfont)}}
        \scnexplanation{В sc-выражениях круглые скобки могут ограничивать:
        \begin{scnitemize}
         \item перечень (через точку с запятой) sc-идентификаторов, обозначающих аргументы заданного квазибинарного ориентированного отношения в рамках \textit{sc-выражений, построенных на основе бинарных отношений} и \textit{sc-выражений, построенных на основе алгебраических операций}\char59
         \item \textit{sc-выражение, идентифицирующее sc-коннектор}.
         \end{scnitemize}
        }
        \scnaddlevel{-1}
\scnnote{Очевидно, что многие о\textit{граничители sc-выражений} могут использоваться также в рамках \textit{внутренних файлов ostis-систем}, содержащих естественно-языковые тексты или математические выражения.}

\scnheader{sc-выражение}
\scnhaselementrole{пример}{\scnfilelong{(\textit{ei} \textit{ki} \textit{ej})}}
\scnaddlevel{1}
    \scniselement{sc-выражение, идентифицирующее sc-коннектор}
    \scnrelboth{семантическая эквивалентность}{\scnfilescg{figures/intro/idtf/sc_expr_example_connector.png}}
\scnaddlevel{-1}
\scnhaselementrole{пример}{\scnfilelong{\scnset{$\bullet$\textit{ei}~$\bullet$\textit{ej}~$\bullet$\textit{ek}}}}
\scnaddlevel{1}
    \scniselement{sc-выражение множества, заданного перечислением}
    \scnrelboth{семантическая эквивалентность}{\scnfilelong{\scnset{\textit{ei};~\textit{ej};~\textit{ek}}}}
    \newpage
    \scnrelboth{семантическая эквивалентность}{\scnfilescg{figures/intro/idtf/sc_expr_example_enum_set.png}}
\scnaddlevel{-1}
\scnhaselementrole{пример}{\scnfilelong{(\textit{si} $\cup$ \textit{sj} $\cup$ \textit{sk} $\cup$ ...)}}
\scnaddlevel{1}
    \scniselement{sc-выражение, построенное на основе алгебраической операции}
    \scnrelboth{семантическая эквивалентность}{\scnfilescg{figures/intro/idtf/sc_expr_example_union_set.png}}
\scnaddlevel{-1}
\scnhaselementrole{пример}{\scnfilelong{\scnset{\textit{si}; \textit{sj}; \textit{sk}; ...}}}
\scnaddlevel{1}
    \scniselement{sc-выражение множества, заданного перечислением}
    \scnrelboth{семантическая эквивалентность}{\scnfilescg{figures/intro/idtf/sc_expr_example_enumeration_set.png}}
\scnaddlevel{-1}
\scnhaselementrole{пример}{\scnfilelong{\scnset{$ei => ri: ej;;$}}}
\scnaddlevel{1}
    \scniselement{sc-выражение структуры}
    \scnrelboth{семантическая эквивалентность}{\scnfilescg{figures/intro/idtf/sc_expr_example_struct.png}}
\scnaddlevel{-1}
\scnhaselementrole{пример}{\scnfilelong{\textit{конкатенация*}(\scnvector{\textit{e1}; \textit{e2}}; \scnvector{\textit{e2}; \textit{e3}}; \scnvector{\textit{e3}; \textit{e4}})}}
\scnaddlevel{1}
    \scniselement{sc-выражение, построенное на основе квазибинарного отношения}
    \scnrelboth{семантическая эквивалентность}{\scnfilescg{figures/intro/idtf/sc_expr_example_concat.png}}
\scnaddlevel{-1}
\newpage
\scnhaselementrole{пример}{\scnfilelong{\textit{пересечение}*($si$; $sj$; $sk$)}}
\scnaddlevel{1}
    \scniselement{sc-выражение, построенное на основе квазибинарного отношения}
    \scnrelboth{семантическая эквивалентность}{\scnfilescg{figures/intro/idtf/sc_expr_example_intersection.png}}
\scnaddlevel{-1}
\scnhaselementrole{пример}{\scnfilelong{\scnvector{\textit{e1}; \textit{e2}; \textit{e3}}}}
\scnaddlevel{1}
    \scniselement{sc-выражение в угловых скобках}
    \scnrelboth{семантическая эквивалентность}{\scnfilescg{figures/intro/idtf/sc_expr_example_role.png}}
\scnaddlevel{-1}

\bigskip

\scnstartset
\scnheader{sc-выражение}
\scnhaselementrole{пример}{\scnfilelong{\textit{степень*}(\textit{x}, \textit{n})}}
\scnhaselementrole{пример}{\scnfilelong{\textit{корень*}(\textit{y}, \textit{n})}}
\scnhaselementrole{пример}{\scnfilelong{\textit{log*}(\textit{x}, \textit{y})}}
\scnendstruct\\
\scnnote{В данном примере представлены три разных, но логически эквивалентных друг другу квазибинарных отношения: \textit{степень*}, \textit{корень*} и \textit{log*}. В базе знаний соответствующая конструкция записывается с использованием отношения \textit{возведение в степень*}.}
\scnaddlevel{1}
    \scnrelfrom{пояснение}{\scnfilescg{figures/intro/idtf/sc_expr_example_log_root_degree.png}}
\scnaddlevel{-1}

\bigskip
 
\scnstartset
\scnheader{sc-выражение}
\scnhaselementrole{пример}{\scnfilelong{\textit{подмножество*}(\textit{si}) /\textit{i}}}
\bigskip
\scnhaselementrole{пример}{\scnfilelong{\textit{надмножество*}(\textit{sj}) /\textit{i}}}
\scnendstruct\\
\scnnote{Отношения \textit{подмножество*} и  \textit{надмножество*} не являются функциональными, таким образом необходимо явно уточнить при помощи символа ``/'' и некоторого уточняющего индекса (в данном случае просто ``\textit{i}''), что имеется в виду некоторое конкретное подмножество множества \textbf{\textit{si}} и некоторое конкретное надмножество множества \textbf{\textit{sj}}.}
\scnnote{В данном примере представлены два отношения (\textit{подмножество*} и  \textit{надмножество*}), которые по отношению друг к другу являются \textit{обратными отношениями*}. В базе знаний соответствующая конструкция записывается при помощи отношения \textit{включение множеств*}.}
\scnaddlevel{1}
    \scnrelfrom{пояснение}{\scnfilescg{figures/intro/idtf/sc_expr_example_subset_superset.png}}
\scnaddlevel{-1}

\scnheader{Правила построения sc-идентификаторов}
\scnnote{Правила построения sc-идентификаторов легко записываются на языке Бэкуса-Наура, который является формальным метаязыком, ориентированным на описание синтаксиса всевозможных линейных языков (языков, текстами которых являются строки символов-линейные информационные конструкции). Но, поскольку в рамках Технологии OSTIS любые виды знаний (в том числе и описания синтаксиса различных языков) представляются в виде текстов SC-кода, в рамках \textit{SC-кода} должен существовать подъязык, семантически эквивалентный язык Бэкуса-Наура.}

\scnheader{следует отличать*}
\scnhaselementset{sc-элемент;sc-идентификатор\\
    \scnaddlevel{1}
    \scnidtf{внешний идентификатор sc-элемента}
    \scnidtf{имя, соответствующее (приписываемое) sc-элементу}
    \scnaddlevel{-1}}
\scnhaselementset{простой sc-идентификатор;sc-выражение}
\scnhaselementset{sc-выражение структуры;sc-выражение внутреннего файла ostis-системы}
    \scnaddlevel{1}
    \scnnote{Для каждого \textit{sc-выражения структуры}, существует \textit{sc-выражение внутреннего файла \mbox{ostis-системы}}, отличающееся от \textit{sc-выражения структуры} только заменой фигурных скобок на квадратные.}
    \scnaddlevel{-1}
\scnhaselementset{sc-идентификатор;Правила построения sc-идентификаторов}
\scnhaselementset{
\scnfileitem{\textit{si} $\Rightarrow$ \textit{пересечение*:} \scnset{\textit{sj}; \textit{sk}}};
\scnfileitem{\textit{si}~$\bm{=}$~(\textit{sj} $\cap$ \textit{sk})};
\scnfileitem{\textit{si}~$\bm{\coloneqq}$~\scnfileshort{(\textit{sj} $\cap$ \textit{sk})}}
}
\scnaddlevel{1}
\scnexplanation{Из перечисленных трех текстов первый и второй \textit{семантически эквивалентны} друг другу и логически следуют из третьего, в котором дополнительно указывается, что у sc-элемента, имеющий основной sc-идентификатор "\textit{si}"{},  добавляется еще один, но не основной sc-идентификатор "(\textit{sj} $\cap$ \textit{sk})"{}.}
\scnaddlevel{-1}

\bigskip
\scnendstruct \scnendsegmentcomment{Понятие сложного идентификатора sc-элемента}

\scnendstruct \scnendcurrentsectioncomment

\end{SCn}
    
\begin{SCn}
    \scnsectionheader{\currentname}
    \begin{scnstruct}
        \begin{scnreltovector}{конкатенация сегментов}
            \scnitem{Понятие внешнего идентификатора sc-элемента}
            \scnitem{Понятие простого идентификатора sc-элемента}
            \scnitem{Понятие сложного идентификатора sc-элемента}
        \end{scnreltovector}

        \scnheader{Предметная область внешних идентификаторов знаков, входящих в информационные конструкции внутреннего языка ostis-систем}
        \scniselement{предметная область}
        \begin{scnhaselementrole}{класс объектов исследования}
            sc-идентификатор
        \end{scnhaselementrole}
        \begin{scnhaselementrolelist}{класс объектов исследования}
        {
            \scnitem{основной sc-идентификатор}
            \scnitem{строковый sc-идентификатор}
            \scnitem{системный sc-идентификанеосновной sc-идентификатортор}
            \scnitem{нетранслируемый sc-идентификатор}
            \scnitem{простой sc-идентификатор;простой строковый sc-идентификатор}
            \scnitem{имя нарицательное}
            \scnitem{имя собственное}
            \scnitem{sc-выражение}
            \scnitem{ограничитель sc-выражений}
        }
        \end{scnhaselementrolelist}
        \begin{scnhaselementrolelist}{исследуемое отношение}
            \scnitem{sc-идентификатор*}
        \end{scnhaselementrolelist}

        \scnsegmentheader{Понятие внешнего идентификатора sc-элемента}
        \begin{scnsubstruct}
            \scnheader{sc-идентификатор}
            \scnidtf{строка символов или пиктограмма, взаимно однозначно представляющая соответствующий sc-элемент, хранимый в sc-памяти}
            \scnidtf{внешний идентификатор sc-элемента}
            \scntext{пояснение}{Внешние идентификаторы \textit{sc-элементов} (или, сокращенно \scnkeyword{sc-идентификаторы}
            ) необходимы \mbox{\textit{ostis-системе}} исключительно для того, чтобы осуществлять обмен информацией с другими \textit{ostis-системами} или со своими пользователями. Для того чтобы представить свою \textit{базу знаний}, решать самые различные \textit{задачи}, связанные с анализом текущего состояния и эволюцией своей \textit{базы знаний}, задачи, связанные с анализом текущего состояния (текущих ситуаций) окружающей среды, принятием соответствующих решений (целей) и организацией соответствующих \textit{действий}, направленных на выполнение принятых решений (на достижение поставленных целей), \textit{ostis-системе} не нужны никакие внешние идентификаторы (в частности, имена) соответствующие \textit{sc-элементам}. Но для \uline{понимания} сообщений, принимаемых от других субъектов (что для \textit{ostis-системы} означает построение \textit{sc-текста},~~ \textit{семантически эквивалентного} принятому сообщению) и для анализа сообщений, передаваемых другим субъектам (что для \textit{ostis-системы} означает синтез \textit{внешнего текста},~~\textit{семантически эквивалентного} заданному \textit{sc-тексту} и удовлетворяющего некоторым дополнительным требованиям, например, эмоционального характера) \textit{ostis-системе} необходимо знать, как в принимаемом или передаваемом сообщении изображаются (представляются) \textit{знаки}, \uline{синонимичные sc-элементам}, которые уже хранятся или могут храниться в составе \textit{базы знаний}~~\textit{ostis-системы}. В качестве внешних идентификаторов \textit{sc-элементов} чаще всего используются имена (термины) соответствующих (обозначаемых) сущностей, представленные отдельными словами или словосочетаниями на различных естественных языках, но также могут использоваться иероглифы, условные обозначения, пиктограммы.В общем случае \textit{sc-элементу} может соответствовать несколько синонимичных ему имен на разных \textit{естественных языках}. Более того, \textit{sc-элементу} может соответствовать несколько синонимичных ему имен на одном и том же \textit{естественном языке}. В этом случае одно из этих имен объявляется как основной внешний идентификатор для соответствующего \textit{sc-элемента} и соответствующего \textit{естественного языка}. Основное требование, предъявляемое к таким внешним идентификаторам это отсутствие как синонимов, так и омонимов в рамках множества основных внешних идентификаторов sc-элементов для каждого естественного языка. Каждый внешний идентификатор \textit{sc-элемента}, используемый ostis-системой, может быть описан (представлен) в её памяти в виде \textit{внутреннего файла ostis-системы}, т.е. в виде электронного образа всевозможных вхождений данного внешнего идентификатора во всевозможные внешние тексты соответствующего внешнего языка. В некоторых случаях явное представление в памяти не требуется, например, в случае \textit{нетранслируемых sc-идентификаторов}.}
            \begin{scnsubdividing}
                \scnitem{простой sc-идентификатор\\\scnidtf{простой внешний идентификатор sc-элемента}}
                \scnitem{sc-выражение\\\scnidtf{сложный внешний идентификатор sc-элемента, в состав которого входит один или несколько идентификаторов других sc-элементов}}
            \end{scnsubdividing}
            \begin{scnsubdividing}
                \scnitem{основной sc-идентификатор\\\scnidtf{основной sc-идентификатор для носителей дополнительно указываемого языка общения (например, соответствующего естественного языка)}
                \scntext{примечание}{\textit{основной sc-идентификатор} является уникальным (не имеет синонимов и омонимов) в рамках соответствующего естественного языка}
                \scnsuperset{основной международный sc-идентификатор}
                \scntext{примечание}{В качестве \textit{основных sc-идентификаторов} могут использоваться также общепринятые международные условные обозначения некоторых сущностей, например, обозначения часто используемых функций (sin, cos, tg, log, и т.д.), единиц измерения, денежных единиц и многое другое. Формально каждый основной международный sc-идентификатор считается основным sc-идентификатором также и для каждого естественного языка, несмотря на то, что символы, используемые в основных международных sc-идентификаторах, могут не соответствовать алфавиту некоторых или даже всех естественных языков.}}
                \scnitem{неосновной sc-идентификатор\\\scntext{примечание}{С помощью неосновных sc-идентификаторов указываются возможные \textit{синонимы*} соответствующего \textit{основного sc-идентификатора}, которые в частности, могут пояснять или даже определять обозначаемую сущность, указывает на важные свойства этой сущности.}
                \scnsuperset{\normalfont(неосновной sc-идентификатор $\cap$ пояснение\normalfont)}
                \scnidtf{неосновной sc-идентификатор, являющийся одновременно и пояснением обозначаемой сущности}
                \scnsuperset{(неосновной sc-идентификатор $\cap$ определение)}
                \scnidtf{неосновной sc-идентификатор, являющийся одновременно и определением обозначаемого понятия}
                \scnsuperset{неосновной часто используемый sc-идентификатор}
                \scntext{пояснение}{Для некоторых sc-элементов могут часто использоваться не только основные, но и неосновные sc-идентификаторы (особенно в неформальных текстах -- в пояснениях, примечаниях и т.п.). Явное выделение такого класса sc-идентификаторов позволяет упростить семантический анализ исходных текстов баз знаний.}}
                \scnitem{системный sc-идентификатор\\\scntext{explanation}{\textit{cистемный sc-идентификатор} -- это \textit{sc-идентификатор}, являющийся уникальным в рамках всей базы знаний \textit{Экосистемы OSTIS} (\textit{Глобальной базы знаний}). Данный \textit{sc-идентификатор}, как правило, используется в исходных текстах базы знаний, при обмене сообщениями между \textit{ostis-системами}, а также для взаимодействия \textit{ostis-системы} с компонентами, реализованными с использованием средств, внешних с точки зрения \textit{Технологии OSTIS}, например, программ, написанных на традиционных языках программирования. Алфавит системных sc-идентификаторов максимально упрощен для того, чтобы обеспечить удобство автоматической обработки таких sc-идентификаторов с использованием современных технических средств, в частности, запрещены пробелы и различные специальные символы.}\scntext{примечание}{В качестве указанного языка общения между ostis-системами может использоваться SCs-код.}}
            \end{scnsubdividing}
            \scntext{примечание}{Каждому sc-элементу может соответствовать целое семейство внешних идентификаторов этого sc-элемента, которые обычно являются терминами, именующими обозначаемую сущность. Среди этих внешних идентификаторов для каждого идентифицируемого sc-элемента выделяется один как основной идентификатор. А неосновные термины (имена), соответствующие этим sc-элементам (в том числе и классам), поясняют денотационную семантику указанного sc-элемента.}
            \bigskip
            \begin{scnset}
                \scnheader{основной sc-идентификатор}
                \scnsubset{файл-образец ostis-системы}
            \end{scnset}
            \scnrelboth{семантическая эквивалентность}{\scnfilelong{Все основные идентификаторы sc-элементов в памяти ostis-системы оформляются в виде копируемых фалов-образцов ostis-системы.}}
            \scntext{пояснение}{Копии основных sc-идентификаторов входят в состав внешних текстов различных языков (\mbox{SCg-кода}, SCs-кода, SCn-кода), а также в различных падежах, склонения, спряжениях в состав файлов ostis-систем.}
            \scntext{примечание}{Аналогичное утверждение справедливо и для неосновных часто используемых sc-идентификаторов. Все остальные неосновные sc-идентификаторы считаются вспомогательными файлами-экземплярами.}
            \scnheader{sc-идентификатор}
            \begin{scnsubdividing}
                \scnitem{строковый sc-идентификатор\\\scnidtf{sc-идентификатор, представленный строкой символов, которая является именем обозначаемой сущности}
                \scnidtf{имя сущности, обозначаемой идентифицируемым sc-элементом}
                \scnidtf{имя (термин, словосочетание), синонимичное соответствующему (идентифицируемому) \mbox{sc-элементу} и представленное в соответствующем алфавите символов}}
                \scnitem{нестроковый sc-идентификатор\\\scntext{explanation}{В общем случае в качестве \textit{sc-идентификатора} некоторого \textit{sc-элемента} может выступать произвольный \textit{внутренний файл ostis-системы}, например, пиктограмма, условное обозначение или даже аудиофрагмент.}}
            \end{scnsubdividing}
            \scntext{примечание}{Введенные нами sc-идентификаторы используются во всех внешних языках, близких SC-коду -- в SCg-коде, в SCs-коде и в SCn-коде.}
            \scnheader{строковый sc-идентификатор}
            \scnidtf{имя, приписываемое идентифицируемому sc-элементу}
            \scnidtf{имя сущности, обозначаемой идентифицируемым sc-элементом}
            \scnidtf{строка символов, синонимичная соответствующему идентифицируемому sc-элементу}
            \scnsuperset{основной строковый sc-идентификатор}
            \scnidtf{уникальное для каждого естественного языка внешнее имя, приписываемое идентифицируемому sc-элементу}
            \scnsuperset{основной русскоязычный sc-идентификатор}
            \scnsuperset{системный sc-идентификатор}
            \scnsuperset{основной англоязычный sc-идентификатор}
            \scnsuperset{основной германоязычный sc-идентификатор}
            \scnsuperset{основной франкоязычный sc-идентификатор}
            \scnsuperset{основной италоязычный sc-идентификатор}
            \scnsuperset{основной китайскоязычный sc-идентификатор}
            \scnsuperset{системный sc-идентификатор}
            \scnheader{sc-идентификатор}
            \scntext{примечание}{Представление знаков в виде неатомарных фрагментов информационных конструкций (в частности, в виде имен обозначаемых сущностей, построенных в фиксированном алфавите), взаимно однозначно соответствующих обозначенным сущностям, необходимо только для того, чтобы иметь простую процедуру установления синонимии знаков, входящих в состав одной или разных информационных конструкций.}
            \scnheader{системный sc-идентификатор}
            \begin{scnrelfromset}{правила построения}
                \scnitem{
                \scnfileitem{Символами, использующимися в \textit{системном sc-идентификаторе}, могут быть буквы латинского алфавита, цифры, знак нижнего подчеркивания и знак тире. Для обеспечения интернационализации рекомендуется формировать \textit{системные sc-идентификаторы} на основании основных англоязычных \textit{sc-идентификаторов}. Таким образом, наиболее целесообразно формировать \textit{системный sc-идентификатор} \textit{sc-элемента} из основного англоязычного путем замены всех символов, не входящих в описанный выше алфавит на символ нижнее подчеркивание (``\_). Кроме того, заглавные буквы чаще всего заменяются на соответствующие строчные.}
                \scnrelfrom{пример применения}{
                    \scnstartsetlocal
                    \scnheaderlocal{Раздел. Предметная область и онтология множеств}
                    \scnrelfrom{основной sc-идентификатор}{\scnfilelong{Section. Subject domain and ontology of sets}}
                    \scniselement{Английский язык}
                    \scniselement{основной sc-идентификатор}
                    \scnrelfrom{системный sc-идентификатор}{\scnfilelong{section\_subject\_domain\_and\_ontology\_of\_sets}}
                    \scniselement{системный sc-идентификатор}
                }
                }
                \scnitem{
                    \scnfileitem{Для именования sc-элементов, являющихся знаками \textit{ролевых отношений}, вместо знака ``\scnrolesign в \textit{системном sc-идентификаторе} используется приставка ``rrel и далее после нижнего подчеркивания записывается имя \textit{ролевого отношения}.}
                        \scnrelfrom{пример применения}{\scnstartsetlocal\\\scnheaderlocal{слагаемое\scnrolesign}
                        \scnrelfrom{системный sc-идентификатор}{\scnfilelong{rrel\_summand}}
                        \scniselement{системный sc-идентификатор}
                    }
                }
                \scnitem{
                    \scnfileitem{Для именования sc-элементов, являющихся знаками \textit{неролевых отношений}, вместо знака ``* в \textit{системном sc-идентификаторе} используется приставка ``nrel и далее после нижнего подчеркивания записывается имя \textit{неролевого отношения}.}
                    \scnrelfrom{пример применения}{\scnstartsetlocal\\\scnheaderlocal{включение*}
                        \scnrelfrom{системный sc-идентификатор}{\scnfilelong{nrel\_inclusion}}
                        \scniselement{системный sc-идентификатор}
                    }
                }
                \scnitem{
                    \scnfileitem{Для именования sc-элементов, являющихся знаками классов \textit{понятий}~~в~~\textit{системном sc-идентификаторе} используется приставка ``concept и далее после нижнего подчеркивания записывается имя \textit{класса}.}
                    \scnrelfrom{пример применения}{
                        \scnstartsetlocal\\\scnheaderlocal{треугольник}
                        \scnrelfrom{системный sc-идентификатор}{\scnfilelong{concept\_triangle}}
                        \scniselement{системный sc-идентификатор}
                    }
                    \newpage
                }
                \scnitem{
                    \scnfileitem{Для именования sc-элементов, являющихся знаками \textit{структур}~~в~~\textit{системном sc-идентификаторе} используется приставка ``struct и далее после нижнего подчеркивания записывается имя \textit{структуры}.}
                    \scnrelfrom{пример применения}{\scnstartsetlocal\\\scnheaderlocal{Треугольник(ТочкаA}
                        \scnitem{ТочкаB}
                        \scnitem{ТочкаC)}
                        \scnrelfrom{системный sc-идентификатор}{\scnfilelong{struct\_Triangle\_A\_B\_C}}
                        \scniselement{системный sc-идентификатор}
                    }
                }
                \scnitem{
                    \scnfileitem{Для именования sc-элементов, являющихся знаками \textit{связок}~~в~~\textit{системном sc-идентификаторе} используется приставка ``tuple и далее после нижнего подчеркивания записывается имя \textit{связки}.}
                }
            \end{scnrelfromset}
            \scnheader{нетранслируемый sc-идентификатор}
            \scnsubset{системный sc-идентификатор}
            \scnidtf{sc-идентификатор, не представляемый в базе знаний ostis-системы}
            \scnidtf{sc-идентификатор, существующий только вне базы знаний ostis-системы}
            \scntext{explanation}{\textit{нетранслируемые sc-идентификаторы} используются только в рамках исходных текстов \textit{баз знаний} (в том числе, \textit{sc.s-текстов}) и при обмене сообщениями между \textit{ostis-системами} в тех случаях, когда необходимо в нескольких фрагментах исходного текста \textit{базы знаний} или передаваемого сообщения использовать имя одного и того же \textit{sc-элемента}, но при этом указанный \textit{sc-элемент} не имеет \textit{системного sc-идентификатора} и вводить его нецелесообразно. Использование \textit{нетранслируемых sc-идентификаторов} позволяет повысить читабельность и структурированность исходных текстов \textit{баз знаний}, а также позволяет обратиться к одному и тому же неименуемому (в рамках базы знаний) \textit{sc-элементу} в разных файлах исходных текстов \textit{баз знаний} или в разных сообщениях, передаваемых между \textit{ostis-системами}. В качестве таких \textit{sc-элементов} часто выступают знаки \textit{структур} и \textit{связок}.Таким образом, \textit{нетранслируемые sc-идентификаторы} существуют только вне \textit{базы знаний ostis-системы} и при формировании базы знаний из исходных текстов или при погружении в базу знаний полученного сообщения соответствующий им \textit{внутренний файл ostis-системы} не создается.}\scntext{правило построения}{\textit{нетранслируемые sc-идентификаторы} строятся по тем же принципам, что и системные sc-идентификаторы, но в начале \textit{нетранслируемого sc-идентификатора} ставится одна или несколько точек (``.), количество которых определяет область видимости данного \textit{нетранслируемого sc-идентификатора}.}
            \scntext{note}{Данное правило справедливо и для построения \textit{sc-идентификаторов}~~~\textit{sc-переменных}}\scntext{explanation}{..\_var1}\scnsuperset{глобальный нетранслируемый sc-идентификатор}
            \scntext{explanation}{\textit{глобальные нетранслируемые sc-идентификаторы} считаются уникальными в рамках всего имеющегося набора исходных текстов \textit{баз знаний} и/или передаваемых между \textit{ostis-системами} сообщений. Таким образом, \textit{sc-элементы}, имеющие на уровне исходных текстов (в том числе, в разных файлах исходных текстов) одинаковые \textit{глобальные нетранслируемые sc-идентификаторы}, считаются синонимичными и в \textit{базе знаний ostis-системы} представляются одним и тем же \textit{sc-элементом}.Можно сказать, что областью видимости \textit{глобальных нетранслируемых sc-идентификаторов} является весь набор (репозиторий) исходных текстов некоторой базы знаний.}\scntext{правило построения}{В начале \textit{глобальных нетранслируемых sc-идентификаторов} ставится одна точка}
            \begin{scnhaselementrolelist}
                \scnitem{пример}
            \end{scnhaselementrolelist}
            \scnsuperset{локальный нетранслируемый sc-идентификатор}
            \scntext{explanation}{\textit{локальные нетранслируемые sc-идентификаторы} считаются уникальными в рамках \uline{конкретного} файла исходных текстов \textit{баз знаний} и/или передаваемого между \textit{ostis-системами} сообщения. Таким образом, \textit{sc-элементы}, имеющие в рамках одного файла исходных текстов одинаковые \textit{локальные нетранслируемые sc-идентификаторы}, считаются \uline{синонимичными}, в то время как \textit{sc-элементы}, имеющие в рамках \uline{разных} файлов исходных текстов одинаковые \textit{локальные нетранслируемые sc-идентификаторы} будут считаться \uline{разными} \textit{sc-элементами}.Можно сказать, что областью видимости \textit{локальных нетранслируемых sc-идентификаторов} является конкретный файл исходных текстов базы знаний.}\scntext{правило построения}{В начале \textit{глобальных нетранслируемых sc-идентификаторов} ставится две точки}
            \begin{scnhaselementrolelist}
                \scnitem{пример}
            \end{scnhaselementrolelist}
            \scnsuperset{уникальный нетранслируемый sc-идентификатор}
            \scneqfileclass{...}
            \scntext{explanation}{\textit{уникальные нетранслируемые sc-идентификаторы} используются для однократного обозначения конкретного неименуемого \textit{sc-элемента} в рамках исходных текстов \textit{баз знаний} и/или передаваемых между \textit{ostis-системами} сообщений.Кроме того, \textit{уникальные нетранслируемые sc-идентификаторы} могут использоваться при визуализации баз знаний в виде, например, sc.s-текстов или sc.n-текстов, в тех случаях, когда необходимо визуализировать sc-элемент, не имеющий sc-идентификатора.Каждое вхождение \textit{уникального нетранслируемого sc-идентификатора} в исходный текст \textit{базы знаний} или передаваемое сообщение считается обозначением \uline{разных} \textit{sc-элементов} (чаще всего, \textit{sc-узлов}).}\scnheader{sc-идентификатор*}
            \scnsuperset{основной sc-идентификатор*}
            \scnidtf{Бинарное ориентированное отношение, каждая пара которого связывает \textit{sc-элемент} с внутренним файлом \textit{ostis-системы}, который содержит \textit{основной sc-идентификатор} указанного \textit{sc-элемента}.}
            \scntext{note}{Отношение, связывающее \textit{sc-элементы} с файлами, содержащими их \textit{основные sc-идентификаторы}, само имеет свой \textit{основной sc-идентификатор}, который представляет собой строку символов, имеющую вид ``основной sc-идентификатор*. Заметим, что в состав первого домена отношения ``\textit{основной sc-идентификатор*} входит также и \textit{sc-узел}, обозначающий само это отношение.}\scnsuperset{системный sc-идентификатор*}
            \scnidtf{Бинарное ориентированное отношение, каждая пара которого связывает \textit{sc-элемент} с внутренним файлом \textit{ostis-системы}, который содержит \textit{системный sc-идентификатор} указанного \textit{sc-элемента}.}
            \scnheader{следует отличать*}
            \begin{scnhaselementset}
                \scnitem{системный sc-идентификатор}
                \scnitem{основной sc-идентификатор}
            \end{scnhaselementset}
            \scntext{отличие}{Системные идентификаторы отличаются от основных, во-первых, требованием к уникальности в рамках всей базы знаний \textit{Экосистемы OSTIS} (а, значит, независимостью от внешнего языка), а во-вторых, более простым алфавитом, удобным для автоматической обработки.}
            \begin{scnhaselementset}
                \scnitem{системный sc-идентификатор}
                \scnitem{нетранслируемый sc-идентификатор}
            \end{scnhaselementset}
            \scntext{сходство}{\textit{системные sc-идентификаторы} и \textit{нетранслируемые sc-идентификаторы} выполняют схожие функции, связанные с именованием \textit{sc-элементов} на уровне исходных текстов \textit{баз знаний} или передаваемых между \textit{ostis-системами} сообщений.}
            \scntext{отличие}{Каждый \textit{системный sc-идентификатор} представляется в базе знаний в виде \textit{внутреннего файла ostis-системы} и связан с соответствующим \textit{sc-элементом} парой отношения \textit{системный \mbox{sc-идентификатор*}}. \textit{нетранслируемые sc-идентификаторы} не представляются в рамках \textit{базы знаний}, не имеют соответствующих \textit{внутренних файлов ostis-системы} и на уровне \textit{базы знаний} никак не связаны с идентифицируемыми ими \textit{sc-элементами}.}
            \bigskip
        \end{scnsubstruct}   
        \scnendsegmentcomment{Понятие внешнего идентификатора sc-элемента}
        
        \scnsegmentheader{Понятие простого идентификатора sc-элемента}
        \begin{scnsubstruct}
            \scnheader{простой sc-идентификатор}
            \scnidtf{идентификатор sc-элемента, в состав которого идентификаторы других sc-элементов не входят и который не содержит \textit{транслируемую в SC-код} информацию об обозначаемой им сущности}
            \scnidtf{простой идентификатор sc-элемента}
            \scnsuperset{простой строковый sc-идентификатор}
            \scnidtf{простой sc-идентификатор, представляющий собой строку (цепочку) символов, которая является именем (названием) той же сущности, что и идентифицируемый sc-элемент}
            \scntext{note}{простые строковые sc-идентификаторы являются наиболее распространненым видом идентификаторов, приписываемых sc-элементам}\scnheader{простой строковый sc-идентификатор}
            \scnsuperset{системный sc-идентификатор}
            \scnsuperset{простой строковый идентификатор sc-переменной}
            \scnhaselementrole{пример}{\scnfilelong{\_var1}}
            \scnsuperset{простой строковый sc-идентификатор неролевого отношения}
            \scnidtf{простой строковый идентификатор sc-узла, являющегося знаком неролевого отношения}
            \scnhaselementrole{пример}{\scnfilelong{включение множеств*}}
            \scnsuperset{простой строковый sc-идентификатор ролевого отношения}
            \scnidtf{простой строковый идентификатор sc-узла, являющегося знаком ролевого отношения}
            \scnhaselementrole{пример}{\scnfilelong{слагаемое\scnrolesign}}
            \scnsuperset{простой строковый sc-идентификатор класса классов}
            \scnidtf{простой строковый идентификатор sc-узла, являющегося знаком класса классов}
            \scnhaselementrole{пример}{\scnfilelong{длина\scnsupergroupsign}}
            \scnsuperset{sc-идентификатор внешнего файла ostis-системы}
            \scnidtf{URL-идентификатор}
            \scnhaselementrole{пример}{\scnfilelong{file:///home/user/image1.png}}
            \scntext{note}{Данный sc-идентификатор описывает абсолютный путь к файлу под названием image1.png}\scnhaselementrole{пример}{\scnfilelong{file://image1.png}}
            \scntext{note}{Данный sc-идентификатор описывает относительный путь к файлу под названием image1.png}\scnhaselementrole{пример}{\scnfilelong{https://conf.ostis.net/content/image1.png}}
            \newpage\scntext{note}{Данный sc-идентификатор описывает путь к файлу под названием image1.png, расположенному на удаленном сервере.}\scnheader{имя нарицательное}
            \scnidtf{простой строковый sc-идентификатор, являющийся именем нарицательным}
            \scniselement{имя нарицательное}
            \scnidtf{Множество всевозможных имен нарицательных}
            \scniselement{имя собственное}
            \scnidtf{имя некоторого класса сущностей (а, точнее, имя класса sc-элементов, обозначающих сущности некоторого класса), которое, с одной стороны, является знаком всего указанного класса, а, с другой стороны, соответствует (может быть приписано) любому экземпляру этого класса}
            \filemodetrue
            \begin{scnrelfromset}{примеры}
                \scnitem{треугольник}
                \scnitem{отношение\scnsupergroupsign}
                \scnitem{разбиение*}
                \scnitem{параметр\scnsupergroupsign}
                \scnitem{длина\scnsupergroupsign}
                \scnitem{язык\scnsupergroupsign}
                \scnitem{sc-текст}
                \scnitem{sc-константа}
                \scnitem{sc-переменная}
                \scnitem{персона}
                \scnitem{публикация}
                \scnitem{город}
            \end{scnrelfromset}
            \filemodefalse
            \scntext{note}{\textit{имя нарицательное} всегда начинается с маленькой буквы}\scnheader{имя собственное}
            \scnidtf{\textit{простой строковый sc-идентификатор}, являющийся именем собственным}
            \scniselement{имя нарицательное}
            \scnidtf{Множество всевозможных имен собственных}
            \scniselement{имя собственное}
            \scnsuperset{sc-идентификатор внешнего файла ostis-системы}
            \scnidtftext{explanation}{имя, которое либо не является обозначением какого-либо класса сущностей, либо является обозначением (именем) некоторого класса сущностей, но построенным без использования нарицательного имени этого класса, либо является именем некоторого класса сущностей, построенным с использованием нарицательного имени этого класса либо путем преобразования имени нарицательного во множественное число, либо путем дополнительного использования в начале формулируемого имени таких слов или словосочетаний, как ``Знак класса, ``Класс, ``Знак множества, ``Множество, ``Знак множества всевозможных, ``Множество всевозможных}
            \scntext{note}{\textit{имя собственное} всегда начинается с большой буквы}
            \begin{scnrelfromset}{примеры}
                \scnitem{
                    \scnfileitem{Иванов Петр Николаевич}
                }
                \scnitem{
                    \scnfileitem{Минск}
                    \scnrelboth{синонимия внешних идентификаторов}{\scnfilelong{Город/Минск}
                }
                }
                \scnitem{
                    \scnfileitem{Точка/А}
                    \newpage
                }
                \scnitem{
                    \scnfileitem{SC-код}
                    \scnrelboth{синонимия внешних идентификаторов}{\scnfilelong{Знак множества всевозможных sc-текстов}}
                    \scnrelboth{синонимия внешних идентификаторов}{\scnfilelong{Множество всевозможных sc-текстов}}
                    \scnrelboth{синонимия внешних идентификаторов}{\scnfilelong{Класс sc-текстов}}
                    \scnrelboth{синонимия внешних идентификаторов}{\scnfilelong{sc-текст}}
                    \scniselement{имя нарицательное}
                }
            \end{scnrelfromset}
            \scnheader{простой строковый sc-идентификатор}
            \scnrelfrom{правила построения}{Правила построения простых строковых sc-идентификаторов}
            \scntext{explanation}{
                \textit{Правила построения простых строковых sc-идентификаторов} включают в себя:
                \begin{scnitemize}
                    \item{\textit{Алфавит символов, используемых в простых строковых sc-идентификаторах}}
                    \item{Префиксы и суффиксы, используемые в простых строковых sc-идентификаторах}
                    \item{Разделители и ограничители, используемые в простых строковых sc-идентификаторах}
                    \item{Правила построения \textit{имен собственных} и \textit{имен нарицательных}, являющихся простыми строковыми sc-идентификаторами;}
                    \item{Правила построения простых строковых sc-идентификаторов, определяемые различными классами идентифицируемых sc-элементов.}
                \end{scnitemize}
            }
            \bigskip
            \begin{scneqtovector}
                \scnitem{
                    \scnfileitem{Общим правилом построения \textit{простых sc-идентификаторов} является стремление максимально возможным образом использовать сложившуюся терминологию. Но при этом следует подчеркнуть, что необходимость исключения омонимии в \textit{sc-идентификаторах} требует строгого формального \uline{уточнения} семантической интерпретации каждого используемого термина. Особо подчеркнем то, что в \textit{ostis-системах} процесс построения новых терминов (\textit{sc-идентификаторов}) и процесс совершенствования существующей терминологии по отношению к процессу эволюции \textit{баз знаний}, представленных в \textit{SC-коде}, с технической точки зрения абсолютно не зависят друг от друга. Кроме того, следует помнить, что \uline{далеко не все} \textit{sc-элементы}, входящие в состав базы знаний ostis-системы, должны иметь соответствующие им sc-идентификаторы (быть идентифицированными). Существенно подчеркнуть то, что далеко не все sc-элементы должны иметь \textit{простые sc-идентификаторы} по той простой причине, что для многих sc-элементов в случае необходимости можно достаточно легко построить идентифицирующее их \textit{sc-выражение} (sc-выражение). Тем не менее, для целого ряда сущностей (например, для понятий, исторических событий и т.п.) обойтись без простых sc-идентификаторов очень сложно. Очевидно, что идентифицированными (именованными) должны быть все используемые понятия, вводимые в соответствующих предметных областях и специфицируемые соответствующими онтологиями. Идентифицированными также должны быть обладающие особыми свойствами ключевые экземпляры (элементы) некоторых понятий, различные социально значимые объекты (персоны, населенные пункты, географические объекты, страны, организации, библиографические источники, исторические события и многое другое).}
                }
                \scnitem{
                    \scnstartsetlocal\\\scnheaderlocal{простой sc-идентификатор}
                    \scnrelfrom{алфавит}{Алфавит символов, используемых в простых строковых sc-идентификаторах}
                    \scnsuperset{Объединение заглавных и строчных букв алфавитов всевозможных естественных языков}
                    \scnsuperset{Алфавит символов, используемых в URL}
                    \scnsuperset{цифра}
                    \scnhaselement{запятая}
                    \scnhaselement{точка}
                    \scnhaselement{прямые кавычки}
                    \scneqfileclass{ \dq }
                    \scnidtf{ограничитель метафорических словосочетаний}
                    \scntext{note}{указанный символ также может использоваться и во внутренних файлах ostis-систем, и в рамках нетранслируемых комментариев}\scnhaselement{знак подчеркивания}
                    \scnidtf{знак нижнего подчеркивания}
                    \scntext{note}{
                        используется в начале sc-идентификаторов sc-переменных. При этом 
                        \begin{scnitemize}
                            \item{если в начале sc-идентификатора sc-переменной стоит один знак подчеркивания, то по умолчанию такая sc-переменная считается \textit{первичной sc-переменной}, если явно не указано иное\char59}
                            \item{если в начале sc-идентификатора sc-переменной стоит подряд два знака подчеркивания, то по умолчанию такая sc-переменная считается \textit{вторичной sc-переменной}\char59}
                            \item{если необходимо использовать \textit{sc-переменную третьего уровня} или \textit{sc-переменную, значения которой имеют различный логический уровень}, то в начале sc-идентификатора такой sc-переменной ставится один знак подчеркивания, а ее принадлежность соответствующему классу sc-переменных указывается явно}
                        \end{scnitemize}
                    }
                    \scnhaselement{звездочка}
                    \scnidtf{надстрочная звездочка}
                    \scneqfileclass{*}
                    \scntext{note}{используется в конце \textit{простых sc-идентификаторов}, соответствующих \textit{неролевым отношениям}}\scnhaselement{штрих}
                    \scneqfileclass{\scnrolesign}
                    \scnidtf{прямой апостроф}
                    \scntext{note}{используется в конце \textit{простых sc-идентификаторов}, соответствующих \textit{ролевым отношениям}}\scnhaselement{циркумфлекс}
                    \scneqfileclass{\scnsupergroupsign}
                    \scnidtf{крышка}
                    \scntext{note}{используется в конце \textit{простых sc-идентификаторов}, соответствующих \textit{классам классов}}\scnhaselement{дефис}
                    \scneqfileclass{-}
                    \scnhaselement{косая черта}
                    \scneqfileclass{/}
                    \scnidtf{слэш}
                    \scntext{note}{используется при формировании \textit{простых sc-идентификаторов} для экземпляров некоторого заданного \textit{класса} путем конкатенации \mbox{\textit{sc-идентификатора}} указанного \textit{класса} и некоторого sc-иден\-ти\-фи\-ка\-то\-ра (необязательно уникального, часто условного), соответствующего рассматриваемому экземпляру указанного класса}
                    \begin{scnrelfromlist}{пример применения}
                        \scnitem{\scnfileitem{\textit{Точка/А}}}
                        \scnitem{\scnfileitem{\textit{Множество/Si}}}
                        \scnitem{\scnfileitem{\textit{Город/Минск}}}
                        \scnitem{\scnfileitem{\textit{Озеро/Нарочь}}}
                    \end{scnrelfromlist}
                }
                \scnitem{
                    \scnfileitem{Первым символом каждого \textit{простого строкового sc-идентификатора}, идентифицирующего \textit{sc-переменную} (переменный \textit{sc-элемент}), является знак подчеркивания. При этом, если после указанного знака подчеркивания указывается \textit{имя нарицательное} некоторого \textit{класса sc-элементов}, то рассматриваемый \textit{простой строковый sc-идентификатор} становится уже \textit{sc-выражением}, содержащим информацию о том, что \textit{областью возможных значений*} идентифицируемой \textit{sc-переменной} является указанный \textit{класс sc-элементов}.}
                }
                \scnitem{
                    \scnfileitem{Последним символом простого sc-идентификатора, идентифицирующего sc-узел, обозначающий неролевое отношение, заданное на множестве sc-элементов, является надстрочная \textit{звездочка}.}
                }
                \scnitem{
                    \scnfileitem{Последним символом простого sc-идентификатора, идентифицирующего sc-узел, обозначающий заданное на множестве sc-элементов ролевое отношение (т.е. отношение, являющееся подмножеством отношения принадлежности), является \textit{штрих} (прямой апостроф)}
                }
                \scnitem{
                    \scnfileitem{Последним символом простого sc-идентификатора, идентифицирующего sc-узел, обозначающий понятие, являющееся классом классов (таковыми в частности, являются различного рода параметры -- длина, площадь, объем, масса) является символ \textit{циркумфлекс} (крышка). Однако, если отсутствует омонимичный идентификатор без этого суффикса, то указанный символ можно не приписывать.}
                }
                \scnitem{
                    \scnfileitem{Простой строковый sc-идентификатор может рассматриваться как последовательное \uline{сокращение} sc-идентификаторов одного и того же sc-элемента с преобразованием имен собственных в имена нарицательные и наоборот.}
                    \scnrelfrom{пример}{
                        \scnstartsetlocal\\\scnheaderlocal{множество}
                        \scnidtf{SC-узел, являющийся знаком множества всевозможных множеств, элементами которых являются sc-элементы}
                        \scnhaselement{имя собственное}
                        \scntext{note}{здесь можно убрать слова SC-узел, являющийся}\scnidtf{Знак множества всевозможных множеств sc-элементов}
                        \scnhaselement{имя собственное}
                        \scntext{note}{здесь можно убрать слово Знак}\scnidtf{Множество всевозможных множеств sc-элементов}
                        \scnhaselement{имя собственное}
                        \scntext{note}{здесь можно убрать слова Множество всевозможных}\scnidtf{множество sc-элементов}
                        \scnhaselement{имя нарицательное}
                        \scnidtf{sc-множество}
                        \scnhaselement{имя нарицательное}
                        \scnidtftext{основной sc-идентификатор}{множество}
                        \scntext{note}{такое сокращение возможно, т.к. \uline{любое} множество можно представить в виде sc-множества}
                    }
                }
                \scnitem{
                    \scnfileitem{При наличии синонимичных \textit{имен собственных} и \textit{имен нарицательных} при выборе \textit{основного sc-идентификатора} преимущество имеют \textit{имена нарицательные}. Так, например, основным идентификатором sc-узла, обозначающего \textit{SC-код}, является термин ``\textit{sc-текст}, а термин ``\textit{SC-код}, являющийся \textit{именем собственным}, считается \textit{неосновным часто используемым sc-идентификатором}. Подчеркнем при этом, что имя нарицательное -- это всегда имя некоторого класса сущностей (в частности, понятия). В конце этого имени может быть указан либо признак класса классов, либо признак неролевого отношения (класса связок), либо признак ролевого отношения (подмножества отношения принадлежности), либо не указано ничего. Последнее означает, что именуется класс, не являющийся ни классом классов, ни классом связок. А это, в свою очередь, означает, что именуемым классом является либо класс первичных (терминальных) сущностей, либо класс структур.}
                }
                \scnitem{
                    \scnfileitem{Одно и тоже словосочетание, которому приписываются разные дополнительные признаки, может быть основой для внешних идентификаторов разных sc-элементов}
                    \scnrelfrom{пример}{
                        \scnstartsetlocal\\\bigskip\scnheaderlocal{следует отличать*}
                        \begin{scnhaselementset}
                            \scnitem{
                                элемент информационной конструкции\\
                                \scnidtftext{explanation}{Множество всевозможных элементов всевозможных информационных конструкций}
                            }
                            \scnitem{
                                Элемент информационной конструкции/i\\
                                \scnidtftext{explanation}{Некоторый конкретный элемент некоторой информационной конструкции}
                            }
                            \scnitem{
                                элемент информационной конструкции*\\
                                \scnidtftext{explanation}{Отношение, связывающее информационные конструкции с множествами элементов этих информационных конструкций}
                            }
                            \scnitem{
                                Элемент информационной конструкции*/i\\
                                \scnidtftext{explanation}{Некоторая конкретная пара (связка), принадлежащая Отношению, связывающему информационные конструкции с множествами элементов этих информационных конструкций}
                                \scnidtftext{explanation}{Некоторая конкретная пара (связка), связывающая некоторую информационную конструкцию с множеством ее элементов}
                            }
                            \scnitem{
                                элемент информационной конструкции\scnrolesign\\
                                \scnidtftext{explanation}{Ролевое отношение, связывающее информационные конструкции с конкретными элементами этих информационных конструкций}
                            }
                            \scnitem{
                                Элемент информационной конструкции\scnrolesign/i\\
                                \scnidtftext{explanation}{Пара (связка) отношения принадлежности, связывающая некоторую конкретную информационную конструкцию с ее конкретным элементом}
                            }
                            \scnitem{
                                \_элемент информационной конструкции/i\\
                                \scnidtftext{explanation}{\textit{sc-переменная}, значениями которой могут быть произвольные элементы произвольных информационных конструкций}
                            }
                            \scnitem{
                                \_элемент информационной конструкции*/i\\
                                \scnidtftext{explanation}{\textit{sc-переменная}, значениями которой могут пары (связки) отношения ``элемент информационной конструкции*}
                            }
                            \scnitem{
                                \_элемент информационной конструкции\scnrolesign/i\\
                                \scnidtftext{explanation}{\textit{sc-переменная}, значениями которой могут пары принадлежности, принадлежащие ролевому отношению ``элемент информационной конструкции\scnrolesign}
                            }
                        \end{scnhaselementset}
                    }
                }
                \scnitem{
                    \scnfileitem{В рамках \textit{SC-кода} целесообразно вводить правила унифицированного построения \textit{простых sc-идентификаторов} и целого ряда других классов идентифицируемых сущностей -- \textit{персон}, \textit{библиографических источников} (публикаций), \textit{разделов баз знаний ostis-систем}, \textit{файлов ostis-систем}, самих \textit{ostis-систем}.}
                }
            \end{scneqtovector}
        \end{scnsubstruct}
        \scnendsegmentcomment{Понятие простого идентификатора sc-элемента}
        
        \scnsegmentheader{Понятие сложного идентификатора sc-элемента}
        \begin{scnsubstruct}
            \newpage\scnheader{sc-выражение}
            \scnidtf{идентификатор, который не только обозначает соответствующую сущность, но также содержит информацию, представляющую собой по возможности однозначную спецификацию указанной сущности}
            \scntext{note}{Однозначную спецификацию сущности, которая является понятием, называют \uline{определением} этого понятия.}\scnidtf{имя соответствующей именуемой сущности построенное по принципу та (тот), которая (который) указываемым образом связана с другими указываемыми сущностями}
            \scnidtf{выражение, идентифицирующее sc-элемент}
            \scnidtf{идентификатор sc-элемента, в состав которого входят другие идентификаторы и денотационная семантика которого точно определяется конкретным набором правил построения таких сложных (комплексных) идентификаторов, состоящих из определенным образом связанных между собой других идентификаторов}
            \scnidtf{сложный идентификатор, состоящий из других идентификаторов}
            \scnidtf{идентификатор, который представляет собой конструкцию, состоящую из нескольких других идентификаторов, а также из некоторых разделителей и ограничителей, и денотационная семантика которого \uline{однозначно} задается конфигурацией указанной конструкции}
            \scnidtf{сложный sc-идентификатор}
            \scnidtf{сложный (составной) внешний идентификатор sc-элемента}
            \scnidtf{выражение, идентифицирующее sc-элемент}
            \scnidtf{sc-идентификатор, в состав которого входит один или несколько простых sc-идентификаторов}
            \begin{scnsubdividing}
                \scnitem{sc-выражение неориентированного множества\\\scnidtf{sc-выражение в фигурных скобках}
                    \scnidtf{sc-выражение, ограниченное фигурными скобками}
                    \begin{scnsubdividing}
                        \scnitem{sc-выражение множества, заданного перечислением\\\scnidtf{sc-выражение, обозначающее множество, все элементы которого перечислены своими sc-идентификаторами}
                            \scntext{explanation}{В качестве разделителя sc-идентификаторов sc-элементов, входящих в указанное множество может выступать точка с запятой (}
                            \scnitem{) или круглый маркер ($\bullet$).}
                            \scnhaselementrole{пример}{
                                \scnfilelong{
                                    \begin{scnset}
                                        \scnitem{\textit{e1}}
                                        \scnitem{ \textit{e2}}
                                        \scnitem{ \textit{e3}}
                                    \end{scnset}
                                }
                            }
                            \scnrelboth{семантическая эквивалентность}{
                                \scnfilelong{Дано множество с элементами \textbf{\textit{e1}}, \textbf{\textit{e2}},\textbf{\textit{e3}}.}
                            }
                            %TODO пример в SCg\scnrelboth{семантическая эквивалентность}{\scnfileimage[20em]{figures/intro/idtf/sc_expr_example_set.png}
                            \scniselement{sc.g-текст}
                            \scnsuperset{sc-выражение множества, заданного перечислением элементов с указанием их роли}
                            \scntext{note}{В рамках такого sc-выражения можно не только перечислять \mbox{sc-идентификаторы} элементов множества, обозначаемого рассматриваемым sc-выражением, но и указывать роль этих элементов в рамках обозначаемого множества.}
                            \scnhaselementrole{пример}{
                                \scnfilelong{
                                    \begin{scnset}
                                        \scnitem{\textit{ai}\scnrolesign : \textit{e1}}
                                        \scnitem{ \textit{aj}\scnrolesign : \textit{e2}}
                                        \scnitem{ \textit{e3}}
                                    \end{scnset}
                                }
                            }
                            \scnrelboth{семантическая эквивалентность}{
                                \scnfilelong{Дано множество с элементами \textbf{\textit{e1}}, \textbf{\textit{e2}}, \textbf{\textit{e3}}. При этом элемент \textbf{\textit{e1}} в рамках указанного множества имеет атрибут (роль) \textbf{\textit{ai\scnrolesign}}, а элемент \textbf{\textit{e2}} -- атрибут (роль) \textbf{\textit{aj\scnrolesign}}.}
                            }
                            %TODO пример в SCg\newpage\scnrelboth{семантическая эквивалентность}{\scnfileimage[20em]{figures/intro/idtf/sc_expr_example_role_set.png}
                            \scniselement{sc.g-текст}
                        }
                        \scnitem{
                            sc-выражение структуры\\\scnidtf{sc-выражение, обозначающее структуру, представленную на любом известном и легко определяемом языке (Русском, Английском, SCg-коде, SCs-коде, SCn-коде)}
                            \scntext{note}{В текущей реализации средств разработки исходных текстов баз знаний в соответствии с более старой версией правил простроения sc-выражений вместо фигурных скобок \textit{sc-выражение структуры} ограничивается квадратными скобками со звездочками ("[*" и "*]")}
                            \scntext{explanation}{sc-выражение структуры обозначает структуру, содержащую sc-текст, семантически эквивалентный тому тексту (на некотором известном языке), который заключен в фигурные скобки. Чаще всего такой текст записывается на формальном языке, например, SCs-коде, и может быть автоматически однозначно интерпретирован. Возможна ситуация, когда указанный текст записан на менее неформальном языке, например, Русском, но в этом случае его автоматическая интерпретация значительно усложняется и в общем случае не всегда однозначна.}
                        }
                    \end{scnsubdividing}
                }
                \scnitem{sc-выражение ориентированного множества\\\scnidtf{sc-выражение кортежа}
                    \scnidtf{sc-выражение, ограничиваемое угловыми скобками и обозначающее упорядоченное (ориентированное) множество sc-элементов, порядок которых задаётся последовательностью перечисляемых их sc-идентификаторов}
                    \scnhaselementrole{пример}{
                        \scnfilelong{
                            \begin{scnvector}
                                \scnitem{\textit{e1}}
                                \scnitem{ \textit{e2}}
                                \scnitem{ \textit{e3}}
                                \scnitem{ \textit{e4}}
                            \end{scnvector}
                        }
                    }
                    \scnrelboth{семантическая эквивалентность}{
                        \scnfilelong{
                            \begin{scnset}
                                \scnitem{\textit{1\scnrolesign}: \textit{e1}}
                                \scnitem{  \textit{2\scnrolesign}: \textit{e2}}
                                \scnitem{  \textit{3\scnrolesign}: \textit{e3}}
                                \scnitem{  \textit{4\scnrolesign}: \textit{e4}}
                            \end{scnset}
                        }
                    }
                    \scntext{note}{Здесь \textit{1\scnrolesign} -- это ролевое отношение ``быть первым компонентом кортежа, \textit{2\scnrolesign} -- ролевое отношение ``быть вторым компонентом кортежа.}
                }
                \scnitem{
                    sc-выражение внутреннего файла ostis-системы\\\scnidtf{sc-выражение, ограниченное квадратными скобками}
                    \scnidtf{sc-выражение, обозначающее \textit{внутренний файл ostis-системы}, визуальное представление (изображение) которого ограничивается квадратными скобками}
                    \scnsubset{sc-идентификатор внутреннего файла ostis-системы}
                    \scntext{explanation}{\textit{sc-выражение внутреннего файла ostis-системы} обозначает \textit{внутренний файл ostis-системы}, содержимое которого заключено в квадратные скобки, ограничивающие данное sc-выражение.}\scntext{note}{Дополнительная спецификация \textit{внутреннего файла ostis-системы} легко осуществляется с помощью \textit{SC-кода}. Сюда входит \textit{язык}, на котором представлена \textit{информационная конструкция}, являющаяся содержимым \textit{файла}, формат кодирования, \textit{автор}* и многое другое.}
                }
                \scnitem{
                    sc-выражение, обозначающее файл-образец ostis-системы\\\scnidtf{sc-выражение, ограниченное ограниченное квадратными скобками с восклицательными знаками}
                    \newpage\scnsuperset{файл-образец}
                    \scnrelfrom{смотрите}{Принципы SC-кода}
                    \scniselement{сегмент раздела базы знаний}
                    \scnhaselementrole{пример}{\scnfileclass{-}}
                    \scnhaselementrole{пример}{\scnfileclass{$\sim$}}
                }
                \scnitem{
                    sc-выражение, построенное на основе бинарного отношения\\\scnidtf{sc-выражение, в состав которого входят \uline{либо} (1) sc-идентификатор, обозначающий бинарное ориентированное отношение, и (2) в круглых скобках sc-идентификатор одного из элементов первого домена указанного бинарного ориентированного отношения, \uline{либо} (1) sc-идентификатор, обозначающий бинарное \uline{не}ориентированное отношение и (2) в круглых скобках sc-идентификатор одного из элементов области определения указанного бинарного неориентированного отношения}
                    \scnidtf{sc-выражение, построенное путём указания некоторого бинарного отношения (обычно функционального) и одного из его аргументов (в круглых скобках)}
                    \scnhaselementrole{пример}{\scnfilelong{\textit{центр*}(\textit{O})}
                    }
                    \scnsuperset{sc-выражение, построенное на основе квазибинарного отношения}
                    \scnidtf{sc-выражение, построенное путем указания некоторого sc-идентификатора квазибинарного отношения (обычно функционального) и (в фигурных скобках или угловых скобках) неупорядоченного или упорядоченного перечня аргументов указанного отношения}
                    \scntext{примеры}{
                        \scnitem{\textit{объединение}*($s_i$$s_j$$s_k$)}
                        \scnitem{\textit{пересечение}*($s_i$$s_j$$s_k$)}
                        \scnitem{\textit{разность множеств}*($s_i$$s_j$)}
                        \scnitem{\textit{сумма}*($x$$z$)}
                        \scnitem{\textit{произведение}*($x$$y$$z$)}
                        \scnitem{\textit{sin*}($x$)}
                        \scnitem{\textit{cos*}($x$)}
                    }
                }
                \scnitem{
                    sc-выражение, построенное на основе алгебраической операции\\\scnidtf{sc-выражение, ограниченное круглыми скобками и построенное путем указания sc-идентификаторов, разделенных знаком алгебраической операции}
                    \scntext{note}{Для каждого sc-выражения данного вида существует синонимическое ему sc-выражение, построенное на основе квазибинарного отношения}
                    \scntext{примеры}{
                        \scnitem{($s_i \cup s_j \cup s_k$)}
                        \scnitem{($s_i \cap s_j \cap s_k$)}
                        \scnitem{($s_i \backslash s_j$)}
                        \scnitem{($x+y+z$)}
                        \scnitem{($x \times y \times z$)}
                    }
                }
                \scnitem{
                    sc-выражение, идентифицирующее sc-коннектор\\\scnidtf{sc-выражение, ограниченное круглыми скобками и идентифицирующее sc-коннектор, инцидентный двум указанным sc-элементам и имеющий тип, задаваемый путем изображения соответствующего sc.s-коннектора}
                    \scntext{принцип}{Для упрощения восприятия и обработки \textit{sc-выражений, идентифицирующих sc-коннектор} вводится следующее ограничение: первым и третьим компонентом такого sc-выражения может быть только \textit{простой sc-идентификатор}. В рамках sc.s-текстов внутри \textit{sc-выражений, идентифицирующих sc-коннектор} допускается также использование sc.s-модификаторов.}
                    \scnhaselementrole{пример}{\scnfilelong{(\textbf{\textit{e1}} $\in$ \textbf{\textit{s1}})}}
                    \scnrelboth{семантическая эквивалентность}{\scnfilelong{(\textbf{\textit{e1}} <- \textbf{\textit{s1}})}}
                    \scnrelboth{семантическая эквивалентность}{\scnfilelong{Дана базовая sc-дуга, связывающая sc-элемент \textbf{\textit{e1}} с sc-элементом \textbf{\textit{s1}}.}}
                    \scnrelboth{семантическая эквивалентность}{\scnfileimage[20em]{figures/intro/idtf/sc_expr_example_affiliation.png}}
                    \scniselement{sc.g-текст}
                }
            \end{scnsubdividing}
            \newpage
            \scntext{explanation}{Использование sc-выражений позволяет существенно сократить число придумываемых\ sc-идентификаторов, каковыми в конечном счете становятся только простые sc-идентификаторы, поскольку, зная то, как связан идентифицируемый sc-элемент с теми sc-элементами, которые уже имеют sc-идентификаторы, во многих случаях можно построить sc-выражение, идентифицирующее указанный sc-элемент. Кроме того, каждое sc-выражение, являясь внешним идентификатором, является также и \uline{транслируемым} формальным текстом, содержащим некоторую информацию об обозначаемой ею сущности.}\scnrelfrom{правило интерпретации}{\scnfilelong{Очевидно, что некоторые sc-выражения могут интерпретироваться неоднозначно. Например, два одинаковых \textit{sc-выражения, идентифицирующих sc-коннектор}, могут изображать как один и тот же sc-коннектор, так и кратные sc-коннекторы одного и того же вида, связывающие одни и те же два sc-элемента. Аналогичная неоднозначность может возникать при использовании \textit{sc-выражений, построенных на основе бинарного отношения} (\textit{подмножество*(S)}) и ряда других \textit{sc-выражений}.В то же время, некоторые sc-выражения являются однозначными, то есть всегда идентифицируют одну и ту же сущность в любом тексте, в состав которого входят. Например выражение \textit{sin(x)} является однозначным (при условии, что x в разных текстах обозначает одно и то же). Однако, однозначность или неоднозначность sc-выражений зависит от их семантики, таким образом установить факт однозначности на уровне внешнего текста достаточно сложно.Для решения проблемы неоднозначности интерпретации sc-выражений такого рода будем считать, что каждое вхождение какого-либо sc-выражения в различные тексты (например, sc.s-тексты) обозначает \uline{разные} sc-элементы. После трансляции таких текстов в sc-текст может оказаться, что некоторые из указанных sc-выражений на самом деле обозначали одну и ту же сущность, в этом случае соответствующие sc-элементы будут склеены\, но уже на уровне sc-текста, а не на уровне внешнего текста.Следует отметить, что факт совпадения sc-выражений в рамках некоторого внешнего текста может являться поводом для анализа идентифицируемых этими sc-выражениями sc-элементов на возможную синонимичность и явно фиксироваться, например, на этапе погружения внешнего текста в sc-память.}}
            \scnrelfrom{пояснение}{\scnfileimage[20em]{figures/intro/idtf/sc_expr_example_subsets.png}}
            \scnrelfrom{пояснение}{\scnfileimage[20em]{figures/intro/idtf/sc_expr_example_arcs.png}}
            \scnheader{ограничитель sc-выражений}
            \scnidtf{ограничитель, используемый в sc-выражениях}
            \scnidtf{ограничитель, ограничивающий sc-выражения или их компоненты}
            \begin{scnsubdividing}
                \scnitem{
                    левый ограничитель sc-выражений\\
                    \scnidtf{начальный ограничитель sc-выражений}
                    \scnidtf{открывающий ограничитель sc-выражений}
                }
                \scnitem{
                    правый ограничитель sc-выражений\\
                    \scnidtf{конечный ограничитель sc-выражений}
                    \scnidtf{закрывающий ограничитель sc-выражений}
                }
            \end{scnsubdividing}
            \newpage
            \scnrelboth{пара пересекающихся множеств}{фигурная скобка}
            \scneq{(\scnfileclass{ \{ } $\cup$ \scnfileclass{ \}})}
            \scntext{explanation}{Фигурные скобки ограничивают sc-выражение, обозначающее либо множество sc-элементов, задаваемое путем перечисления sc-идентификаторов его элементов (при необходимости с указанием ролей, под которыми эти элементы входят в указанное множество), либо множество sc-элементов, входящих в состав sc-текста, который является результатом трансляции в SC-код текста (например, sc.s-текста или ея-текста), ограниченного фигурными скобками.}\scnrelboth{пара пересекающихся множеств}{квадратная скобка}
            \scneq{{\normalfont(}\scnfileclass{ ![ } $\cup$ \scnfileclass{ ]! }{\normalfont)}}
            \scntext{explanation}{Квадратные скобки ограничивают sc-выражение, обозначающее файл-экземпляр ostis-системы, содержимое (тело) которого изображается и ограничивается указанными квадратными скобками.}\scnrelboth{пара пересекающихся множеств}{квадратная скобка со звездочкой}
            \scniselement{устаревающее понятие}
            \scneq{(\scnfileclass{ [* } $\cup$ \scnfileclass{ *] }
            \scntext{explanation}{Данный ограничитель в текущей версии средств разработки исходных текстов баз знаний используется в качестве ограничителя \textit{sc-выражений sc-структур} вместо фигурных скобок. Такой способ записи \textit{sc-выражений sc-структур} является устаревающим, однако используется в настоящее время.}\scnrelboth{пара пересекающихся множеств}{квадратная скобка с восклицательным знаком}
            \scntext{explanation}{Данный ограничитель sc-выражений ограничивает sc-выражение, обозначающее файл-класс \mbox{ostis-системы}, содержимое (тело) которого изображается и ограничивается указанными sc-ограничителями.}\scnrelboth{пара пересекающихся множеств}{круглая скобка}
            \scneq{{\normalfont(}\scnfileclass{ ( } $\cup$ \scnfileclass{ ) }{\normalfont)}}
            \scntext{explanation}{
                В sc-выражениях круглые скобки могут ограничивать:
                \begin{scnitemize}
                    \item{перечень (через точку с запятой) sc-идентификаторов, обозначающих аргументы заданного квазибинарного ориентированного отношения в рамках \textit{sc-выражений, построенных на основе бинарных отношений} и \textit{sc-выражений, построенных на основе алгебраических операций}\char59}
                    \item{\textit{sc-выражение, идентифицирующее sc-коннектор}.}
                \end{scnitemize}
            }
            \scntext{note}{Очевидно, что многие о\textit{граничители sc-выражений} могут использоваться также в рамках \textit{внутренних файлов ostis-систем}, содержащих естественно-языковые тексты или математические выражения.}\scnheader{sc-выражение}
            \scnhaselementrole{пример}{\scnfilelong{(\textit{ei} \textit{ki} \textit{ej})}}
            \scniselement{sc-выражение, идентифицирующее sc-коннектор}
            \scnrelboth{семантическая эквивалентность}{\scnfileimage[20em]{figures/intro/idtf/sc_expr_example_connector.png}}
            \scnhaselementrole{пример}{
                \scnfilelong{
                    \begin{scnset}
                        \scnitem{$\bullet$\textit{ei}~$\bullet$\textit{ej}~$\bullet$\textit{ek}}
                    \end{scnset}
                }
            }
            \scniselement{sc-выражение множества, заданного перечислением}
            \scnrelboth{семантическая эквивалентность}{
                \scnfilelong{
                    \begin{scnset}
                        \scnitem{\textit{ei}}
                        \scnitem{~\textit{ej}}
                        \scnitem{~\textit{ek}}
                    \end{scnset}
                }
            }
            \newpage\scnrelboth{семантическая эквивалентность}{\scnfileimage[20em]{figures/intro/idtf/sc_expr_example_enum_set.png}}
            \scnhaselementrole{пример}{\scnfilelong{(\textit{si} $\cup$ \textit{sj} $\cup$ \textit{sk} $\cup$ ...)}}
            \scniselement{sc-выражение, построенное на основе алгебраической операции}
            \scnrelboth{семантическая эквивалентность}{\scnfileimage[20em]{figures/intro/idtf/sc_expr_example_union_set.png}}
            \scnhaselementrole{пример}{
                \scnfilelong{
                    \begin{scnset}
                        \scnitem{\textit{si}}
                        \scnitem{ \textit{sj}}
                        \scnitem{ \textit{sk}}
                        \scnitem{ ...}
                    \end{scnset}
                }
            }
            \scniselement{sc-выражение множества, заданного перечислением}
            \scnrelboth{семантическая эквивалентность}{\scnfileimage[20em]{figures/intro/idtf/sc_expr_example_enumeration_set.png}
            }
            \scnhaselementrole{пример}{
                \scnfilelong{
                    \begin{scnset}
                        \scnitem{$ei => ri: ej}
                        \scnitem{$}
                    \end{scnset}
                }
            }
            \scniselement{sc-выражение структуры}
            \scnrelboth{семантическая эквивалентность}{\scnfileimage[20em]{figures/intro/idtf/sc_expr_example_struct.png}}
            \scnhaselementrole{пример}{
                \scnfilelong{
                    \textit{конкатенация*}(
                    \begin{scnvector}
                        \scnitem{\textit{e1}}
                        \scnitem{ \textit{e2}}
                    \end{scnvector}
                    \begin{scnvector}
                        \scnitem{\textit{e2}}
                        \scnitem{ \textit{e3}}
                    \end{scnvector}
                    \begin{scnvector}
                        \scnitem{\textit{e3}}
                        \scnitem{ \textit{e4}}
                    \end{scnvector}
                    )
                }
            }
            \scniselement{sc-выражение, построенное на основе квазибинарного отношения}
            \scnrelboth{семантическая эквивалентность}{\scnfileimage[20em]{figures/intro/idtf/sc_expr_example_concat.png}}
            \newpage\scnhaselementrole{пример}{\scnfilelong{\textit{пересечение}*($si$; $sj$; $sk$)}}
            \scniselement{sc-выражение, построенное на основе квазибинарного отношения}
            \scnrelboth{семантическая эквивалентность}{\scnfileimage[20em]{figures/intro/idtf/sc_expr_example_intersection.png}}
            \scnhaselementrole{пример}{
                \scnfilelong{
                    \begin{scnvector}
                        \scnitem{\textit{e1}}
                        \scnitem{ \textit{e2}}
                        \scnitem{ \textit{e3}}
                    \end{scnvector}
                }
            }
            \scniselement{sc-выражение в угловых скобках}
            \scnrelboth{семантическая эквивалентность}{\scnfileimage[20em]{figures/intro/idtf/sc_expr_example_role.png}}
            \bigskip
            \begin{scnset}
                \scnheader{sc-выражение}
                \scnhaselementrole{пример}{\scnfilelong{\textit{степень*}(\textit{x}, \textit{n})}}
                \scnhaselementrole{пример}{\scnfilelong{\textit{корень*}(\textit{y}, \textit{n})}}
                \scnhaselementrole{пример}{\scnfilelong{\textit{log*}(\textit{x}, \textit{y})}}
                \scntext{note}{В данном примере представлены три разных, но логически эквивалентных друг другу квазибинарных отношения: \textit{степень*}, \textit{корень*} и \textit{log*}. В базе знаний соответствующая конструкция записывается с использованием отношения \textit{возведение в степень*}.}\scnrelfrom{пояснение}{\scnfileimage[20em]{figures/intro/idtf/sc_expr_example_log_root_degree.png}}
                \bigskip
            \begin{scnset}
            \scnheader{sc-выражение}
            \scnhaselementrole{пример}{\scnfilelong{\textit{подмножество*}(\textit{si}) /\textit{i}}}
            \bigskip\scnhaselementrole{пример}{\scnfilelong{\textit{надмножество*}(\textit{sj}) /\textit{i}}}
            \scntext{note}{Отношения \textit{подмножество*} и  \textit{надмножество*} не являются функциональными, таким образом необходимо явно уточнить при помощи символа ``/ и некоторого уточняющего индекса (в данном случае просто ``\textit{i}), что имеется в виду некоторое конкретное подмножество множества \textbf{\textit{si}} и некоторое конкретное надмножество множества \textbf{\textit{sj}}.}\scntext{note}{В данном примере представлены два отношения (\textit{подмножество*} и  \textit{надмножество*}), которые по отношению друг к другу являются \textit{обратными отношениями*}. В базе знаний соответствующая конструкция записывается при помощи отношения \textit{включение множеств*}.}\scnrelfrom{пояснение}{\scnfileimage[20em]{figures/intro/idtf/sc_expr_example_subset_superset.png}}
            \scnheader{Правила построения sc-идентификаторов}
            \scntext{note}{Правила построения sc-идентификаторов легко записываются на языке Бэкуса-Наура, который является формальным метаязыком, ориентированным на описание синтаксиса всевозможных линейных языков (языков, текстами которых являются строки символов-линейные информационные конструкции). Но, поскольку в рамках Технологии OSTIS любые виды знаний (в том числе и описания синтаксиса различных языков) представляются в виде текстов SC-кода, в рамках \textit{SC-кода} должен существовать подъязык, семантически эквивалентный язык Бэкуса-Наура.}\scnheader{следует отличать*}
            \begin{scnhaselementset}
                \scnitem{sc-элемент}
                \scnitem{sc-идентификатор\\\scnidtf{внешний идентификатор sc-элемента}
                    \scnidtf{имя, соответствующее (приписываемое) sc-элементу}
                }
            \end{scnhaselementset}
            \begin{scnhaselementset}
                \scnitem{простой sc-идентификатор}
                \scnitem{sc-выражение}
            \end{scnhaselementset}
            \begin{scnhaselementset}
                \scnitem{sc-выражение структуры}
                \scnitem{sc-выражение внутреннего файла ostis-системы}
            \end{scnhaselementset}
            \scntext{note}{Для каждого \textit{sc-выражения структуры}, существует \textit{sc-выражение внутреннего файла \mbox{ostis-системы}}, отличающееся от \textit{sc-выражения структуры} только заменой фигурных скобок на квадратные.}
            \begin{scnhaselementset}
                \scnitem{sc-идентификатор}
                \scnitem{Правила построения sc-идентификаторов}
            \end{scnhaselementset}
            \begin{scnhaselementset}
                \scnitem{
                    \scnfileitem{
                        \textit{si} $\Rightarrow$ \textit{пересечение*:} 
                        \begin{scnset}
                            \scnitem{\textit{sj}}
                            \scnitem{ \textit{sk}}
                        \end{scnset}
                    }
                }
                \scnitem{
                    \scnfileitem{\textit{si}~$\bm{=}$~(\textit{sj} $\cap$ \textit{sk})}
                }
            \end{scnhaselementset}
            \scntext{explanation}{Из перечисленных трех текстов первый и второй \textit{семантически эквивалентны} друг другу и логически следуют из третьего, в котором дополнительно указывается, что у sc-элемента, имеющий основной sc-идентификатор \textit{si},  добавляется еще один, но не основной sc-идентификатор (\textit{sj} $\cap$ \textit{sk}).}
            \bigskip
        \end{scnsubstruct}
        \scnendsegmentcomment{Понятие сложного идентификатора sc-элемента}
    \end{scnstruct}
    \scnendcurrentsectioncomment
\end{SCn}
\scsubsection[\scnmonographychapter{Глава 2.2. Семейство внешних языков интеллектуальных компьютерных систем нового поколения, близких языку внутреннего смыслового представления знаний (SCg, SCs, SCn)}]{Предметная область и онтология языка внешнего линейного представления информационных конструкций внутреннего языка ostis-систем}
\label{intro_scs}

\begin{SCn}

\scnsectionheader{\currentname}

\scnstartsubstruct

\scnidtf{Описание \textit{SCs-кода}}
\scnreltovector{конкатенация сегментов}{Описание Алфавита SCs-кода;Описание sc.s-разделителей и sc.s-ограничителей;Описание sc.s-предложений;Описание Ядра SCs-кода и различных направлений его расширения}

\scnheader{SCs-код}
\scnidtf{Semantic Code string}
\scnidtf{Язык линейного представления знаний ostis-систем}
\scnidtf{Множество всевозможных текстов \textit{SCs-кода}}
\scnidtf{Тексты \textit{SCs-кода}}	
\scnaddlevel{1}
\scniselement{имя собственное}
\scnaddlevel{-1}
\scnidtf{текст \textit{SCs-кода}}	
\scnaddlevel{1}
\scniselement{имя нарицательное}
\scnaddlevel{-1}
\scnidtf{sc.s-текст}
\scniselement{линейный язык}
\scnrelfrom{алфавит}{Алфавит SCs-кода}
\scnrelfrom{разделители}{sc.s-разделитель}
\scnrelfrom{ограничители}{sc.s-ограничитель}
\scnrelfrom{предложения}{sc.s-предложение}
\scnrelfrom{неоднозначные обозначения описываемых сущностей}{неоднозначное sc.s-изображение sc-элемента}
\scnidtfexp{Множество линейных текстов (\textit{sc.s-текстов}), каждый из которых состоит из предложений (\textit{sc.s-предложений}), разделенных друг от друга двойной \textit{точкой с запятой} (разделителем \textit{sc.s-предложений}). При этом \mbox{\textit{sc.s-предложение}} представляет собой последовательность \textit{sc-идентификаторов}, являющихся именами описываемых \textit{сущностей} и разделяемых между собой различными \textit{sc.s-разделителями} и \textit{sc.s-ограничителями}}

\scnheader{неоднозначное sc.s-изображение sc-элемента}
\scnrelboth{пара пересекающихся множеств}{sc-выражение}
\scnidtf{условное обозначение неименуемой (неидентифицируемой) сущности}
\scnsuperset{sc.s-коннектор}
\scnaddlevel{1}
    \scnidtf{неоднозначное sc.s-изображение \textit{sc-коннектора}, являющееся также одновременно одним из видов \textit{sc.s-разделителей}}
    \scnsubset{sc.s-разделитель}
\scnaddlevel{-1}
\scnsuperset{неоднозначное sc.s-изображение sc-узла}
\scnaddlevel{1}
    \scnsuperset{условное обозначение неименуемого множества sc-элементов}
    \scnaddlevel{1}
        \scnexplanation{условное обозначение неименуемого множества sc-элементов в \textit{SCs-коде} представляется строкой из двух символов -- \textit{левой фигурной скобки} и \textit{правой фигурной скобки}.}
    \scnaddlevel{-1}
    \scnsuperset{условное обозначение неименуемого кортежа sc-элементов}
    \scnaddlevel{1}
        \scnexplanation{В \textit{SCs-коде} такое обозначение представляется двух-символьной \textit{строкой}, состоящей из \textit{левой угловой скобки} и \textit{правой угловой скобки}}
    \scnaddlevel{-1}
    \newpage
	\scnsuperset{условное обозначение неименуемого файла-экземпляра ostis-системы}
	\scnaddlevel{1}
		\scnexplanation{В \textit{SCs-коде} такое обозначение представляется двух-символьной \textit{строкой}, состоящей из \textit{левой квадратной скобки} и \textit{правой квадратной скобки}}
	\scnaddlevel{-1}
	\scnsuperset{условное обозначение неименуемого файла-образца ostis-системы}
	\scnaddlevel{1}
		\scnexplanation{В \textit{SCs-коде} такое обозначение представляется \textit{строкой}, состоящей из \textit{восклицательного знака}, \textit{левой квадратной скобки}, \textit{правой квадратной скобки} и еще одного \textit{восклицательного знака}}
	\scnaddlevel{-1}
\scnaddlevel{-1}

\scnsegmentheader{Описание Алфавита SCs-кода}
\scnstartsubstruct

\scnheader{Алфавит SCs-кода}
\scnidtf{Алфавит символов SCs-кода}
\scnidtf{множество символов SCs-кода}
\scnidtf{символ, используемый в текстах SCs-кода}
\scnreltoset{объединение}{Алфавит символов, используемых в sc.s-разделителях;Алфавит символов, используемых в sc.s-ограничителях;Алфавит символов, используемых в sc-идентификаторах\\
\scnaddlevel{1}
    \scnreltoset{объединение}{Алфавит символов, используемых в простых строковых sc-идентификаторах;Алфавит символов, используемых в sc-выражениях}
\scnaddlevel{-1}
;Алфавит символов, используемых в неоднозначных sc.s-изображениях sc-узлов
}
\scnrelfromlist{принципы}{\scnfileitem{Алфавит SCs-кода строится на основе современных общепринятых наборов символов, что позволяет упростить разработку средств для работы с sc.s-текстами с использованием современных технологий.};
\scnfileitem{В состав sc.s-текстов, как и в состав текстов любых других языков, являющихся вариантами внешнего отображения текстов SC-кода, могут входить различные файлы, в том числе естественно-языковые или даже файлы, содержащие другие sc.s-тексты. В общем случае в таких файлах могут использоваться самые разные символы, в связи с чем будем считать, что в Алфавит SCs-кода эти символы не включаются.}}

\scnheader{Алфавит символов, используемых в sc.s-разделителях}
\scnhaselements{\textit{пробел}; \textit{точка с запятой}; \textit{двоеточие}; \textit{круглый маркер}; \textit{знак равенства}}
\scnsuperset{Алфавит символов, используемых в sc.s-разделителях, изображающих связь инцидентности sc-элементов}
\scnaddlevel{1}
\scnhaselements{\scnfileclass{<};~\scnfileclass{>};~\scnfileclass{|};~\scnfileclass{-}}
\scnaddlevel{-1}
\scnsuperset{Алфавит символов, используемых в sc.s-коннекторах}
\scnaddlevel{1}
\scnsuperset{Расширенный алфавит символов, используемых в sc.s-коннекторах}
\scnaddlevel{1}
\scnidtf{Расширенный алфавит sc.s-коннекторов}
\scnhaselements{\scnfileclass{$\in$};~\scnfileclass{$\ni$};~\scnfileclass{$\notin$};~\scnfileclass{$\not \ni$};~\scnfileclass{$\subseteq$};~\scnfileclass{$\supseteq$};~\scnfileclass{$\subset$};~\scnfileclass{$\supset$};~\scnfileclass{$\leq$};~\scnfileclass{$\geq$};~\scnfileclass{$\Leftarrow$};~\scnfileclass{$\Rightarrow$};~\scnfileclass{$\Leftrightarrow$};~\scnfileclass{$\leftarrow$};~\scnfileclass{$\rightarrow$};~\scnfileclass{$\leftrightarrow$}}
\scnsuperset{Базовый алфавит символов, используемых в sc.s-коннекторах}
\scnaddlevel{1}
\scnidtf{Базовый алфавит sc.s-коннекторов}
\scnhaselements{\scnfileclass{$\sim$};~\textit{знак подчеркивания};~ \textit{знак равенства};~\scnfileclass{>};~ \scnfileclass{<};~\textit{двоеточие};~\scnfileclass{-};~\scnfileclass{|};~\scnfileclass{/}}
\scnaddlevel{-2}
\scnnote{Как в Базовом, так и в Расширенном Алфавитах sc.s-коннекторов используются следующие общие признаки, характеризующие тип изображаемого sc-коннектора:
\begin{scnitemize}
    \item \textit{знак подчеркивания} как признак изображений переменных sc-коннекторов (один  \textit{знак подчеркивания} для sc-коннекторов, являющихся первичными sc-переменными, два  \textit{знака подчеркивания} для sc-коннекторов, являющихся вторичными sc-переменными (sc-метапеременными));
    \item \textit{вертикальная черта} ("|") как признак изображений негативных sc-дуг принадлежности;
    \item \textit{косая черта} ("/") как признак изображений нечетких sc-дуг принадлежности;
    \item \textit{тильда} ("$\sim$") как признак изображений временных sc-дуг принадлежности
\end{scnitemize}

При необходимости комбинации указанных признаков перечисленные символы комбинируются так, как показано в сегменте "\textit{Описание sc.s-разделителей и sc.s-ограничителей}".
}
\scnaddlevel{-1}

\bigskip
\scnmakeset{Расширенный алфавит символов, используемых в sc.s-коннекторах; Базовый алфавит символов, используемых в sc.s-коннекторах}
\scnexplanation{Для упрощения процесса разработки исходных текстов баз знаний с использованием SCs-кода и создания соответствующих средств вводятся два алфавита символов. \textit{Базовый алфавит символов, используемых в sc.s-коннекторах} включает только символы, входящие в переносимый набор символов (portable character set) и имеющиеся на стандартной современной клавиатуре. Таким образом, для разработки исходных текстов баз знаний, использующих только \textit{Базовый алфавит символов, используемых в sc.s-коннекторах} достаточно обычного текстового редактора. \textit{Расширенный алфавит символов, используемых в sc.s-коннекторах} включает также дополнительные символы, которые позволяют сделать sc.s-тексты (и sc.n-тексты) более читабельными и наглядными. Для визуализации и разработки sc.s-текстов с использованием расширенного алфавита требуется наличие специализированных средств.}

\scnheader{Алфавит символов, используемых в sc.s-ограничителях}
\scnhaselements{\scnfileclass{(};~\scnfileclass{)};~\scnfileclass{*}}

\scnheader{Алфавит символов, используемых в неоднозначных sc.s-изображениях sc-узлов}
\scnhaselements{\scnfileclass{\{};~\scnfileclass{\}};~\scnfileclass{-};~\scnfileclass{!};~\scnfileclass{[};~\scnfileclass{]}}

\bigskip
\scnendstruct \scnendsegmentcomment{Описание Алфавита SCs-кода}

\scnsegmentheader{Описание sc.s-разделителей и sc.s-ограничителей}
\scnstartsubstruct

\scnheader{sc.s-разделитель}	
\scnidtf{разделитель, используемый в sc.s-текстах}
\scnsubdividing{sc.s-разделитель, используемый для структуризации sc.s-предложений\\
\scnaddlevel{1}
    \scnsuperset{sc.s-коннектор}
    \scnsuperset{sc.s-разделитель, изображающий связь инцидентности sc-элементов}
    \scnsuperset{двоеточие}
    \scnaddlevel{1}
        \scneqfileclass{$\colon$}
        \scnnote{Разделяет sc-идентификатор бинарного отношения и второй компонент одной из связок этого отношения в случае, если указанное бинарное отношение и его связка связаны \uline{константной} sc-дугой принадлежности.}
    \scnaddlevel{-1}
    \scnsuperset{двойное двоеточие}
    \scnaddlevel{1}
        \scneqfileclass{$\colon\colon$}
        \scnnote{Разделяет sc-идентификатор бинарного отношения и второй компонент одной из связок этого отношения в случае, если указанное бинарное отношение и его связка связаны \uline{переменной} sc-дугой принадлежности.}
    \scnaddlevel{-1}
\scnaddlevel{-1}
;sc.s-разделитель sc.s-предложений\\
\scnaddlevel{1}
    \scneqfileclass{;;}
    \scnidtf{двойная точка с запятой}
\scnaddlevel{-1}}

\scnheader{sc.s-ограничитель}
\scnsuperset{sc.s-ограничитель присоединенных sc.s-предложений}
\scnaddlevel{1}
    \scneq{{\normalfont(}\scnfileclass{ (* } $\cup$ \scnfileclass{ *) }{\normalfont)}}
    \newpage
        \scnexplanation{Круглые скобки со звездочкой ограничивают присоединенные sc.s-предложения, которые, в свою очередь, могут иметь в своем составе другие присоединенные sc.s-предложения.}
\scnaddlevel{-1}

\scnheader{sc.s-коннектор}
\scnidtf{изображение \textit{sc-коннектора} во внешнем тексте SCs-кода или SCn-кода}
\scnsubset{sc.s-разделитель}
\scnnote{типология sc.s-коннекторов полностью соответствует типологии sc.g-коннекторов, и, тем более, \mbox{sc-коннекторов}, т.к. она учитывает устоявшиеся традиции изображения связок целого ряда конкретных отношений}
\scnsubdividing{ориентированный sc.s-коннектор;неориентированный sc.s-коннектор}
\scnaddhind{-1}
\scnsubdividing{sc.s-коннектор, соответствующий sc.g-дуге принадлежности;sc.s-коннектор, соответствующий sc.g-коннектору, который не является sc.g-дугой принадлежности}

\scnheader{sc.s-разделитель, изображающий связь инцидентности sc-элементов}
\scnsubdividing{знак инцидентности “правого” sc-коннектора\\
\scnaddlevel{1}
\scnidtf{знак инцидентности sc-коннектора, sc-идентификатор которого находится справа}
\scneqfileclass{|-}
\scnaddlevel{-1}
;знак инцидентности “левого” sc-коннектора\\
\scnaddlevel{1}
\scnidtf{знак инцидентности sc-коннектора, sc-идентификатор которого находится слева}
\scneqfileclass{-|}
\scnaddlevel{-1}
;знак инцидентности входящей sc-дуги справа\\
\scnaddlevel{1}
\scnidtf{знак инцидентности sc-дуги, sc-идентификатор который находится справа}
\scneqfileclass{|<}
\scnaddlevel{-1}
;знак инцидентности входящей sc-дуги слева\\
\scnaddlevel{1}
\scnidtf{знак инцидентности sc-дуги, sc-идентификатор который находится слева}
\scneqfileclass{>|}
\scnaddlevel{-1}}
\scnexplanation{На множестве sc-элементов задано бинарное ориентированное отношение инцидентности sc-элементов, а так же подмножество этого отношения -- отношение инцидентности входящих sc-дуг, каждая пара которого связывает sc-дугу с тем sc-элементом, в который она входит.
В SC-коде sc-коннекторы могут соединять между собой не только sc-узел с sc-узлами, но и sc-узел с sc-коннектором и даже sc-коннектор с sc-коннектором. В последнем случае, указывая инцидентность sc-коннекторов, необходимо уточнить, какой из них является соединяемым (связываемым), а какой-соединяющим (связующим). Поэтому отношение инцидентности, заданное на множестве sc-элементов является ориентированным. Первый компонент пары этого отношения -- связующий sc-коннектор, а второй -- связуемый sc-элемент. Очевидно, что связующий sc-элемент всегда является sc-коннектором, а sc-узел может быть только связуемым.}
\scnnote{указанные sc.s-разделители с точки зрения синтаксической структуры sc.s-предложений аналогичны \mbox{sc.s-коннекторам}, но с точки зрения их денотационной семантики в отличие от sc.s-коннекторов они не являются изображениями соответствующих sc-коннекторов}

\scnheader{Описание изображения sc.s-коннекторов в Базовом и Расширенном алфавите}
\scnstartsubstruct
\scnheader{константная постоянная позитивная sc.s-дуга принадлежности}
\scnidtf{базовая sc.s-дуга}
\scnrelfromlist{изображение в базовом алфавите sc.s-коннекторов}{\scnfileclass{->};\scnfileclass{<-}}
\scnrelfromlist{изображение в расширенном алфавите sc.s-коннекторов}{\scnfileclass{$\ni$};\scnfileclass{$\in$}}
\scnrelfromlist{изображение в SCg-коде}{\scgfileitem{figures/intro/scs/sc.s-connectors/posTermOrientMembership.png};\scgfileitem{figures/intro/scs/sc.s-connectors/posTermOrientMembership2.png}}

\scnheader{константная постоянная негативная sc.s-дуга принадлежности}
\scnrelfromlist{изображение в базовом алфавите sc.s-коннекторов}{\scnfileclass{-| >};\scnfileclass{< |-}}
\scnrelfromlist{изображение в расширенном алфавите sc.s-коннекторов}{\scnfileclass{$\not \ni$};\scnfileclass{$\notin$}}
\scnrelfromlist{изображение в SCg-коде}{\scgfileitem{figures/intro/scs/sc.s-connectors/negPermOrient.png};\scgfileitem{figures/intro/scs/sc.s-connectors/negPermOrient2.png}}

\scnheader{константная постоянная нечеткая sc.s-дуга принадлежности}
\scnrelfromlist{изображение в базовом алфавите sc.s-коннекторов}{\scnfileclass{-/ >};\scnfileclass{< /-}}
\scnrelfromlist{изображение в расширенном алфавите sc.s-коннекторов}{\scnfileclass{/ $\ni$};\scnfileclass{ $\in$ /}}
\scnrelfromlist{изображение в SCg-коде}{\scgfileitem{figures/intro/scs/sc.s-connectors/fuzPermOrient.png};\scgfileitem{figures/intro/scs/sc.s-connectors/fuzPermOrient2.png}}

\scnheader{константная временная позитивная sc.s-дуга принадлежности}
\scnrelfromlist{изображение в базовом алфавите sc.s-коннекторов}{\scnfileclass{$\sim>$};\scnfileclass{$<\sim$}}
\scnrelfromlist{изображение в расширенном алфавите sc.s-коннекторов}{\scnfileclass{$\sim\ni$};\scnfileclass{ $\in\sim$}}
\scnrelfromlist{изображение в SCg-коде}{\scgfileitem{figures/intro/scs/sc.s-connectors/posTempOrient.png};\scgfileitem{figures/intro/scs/sc.s-connectors/posTempOrient2.png}}

\scnheader{константная временная негативная sc.s-дуга принадлежности}
\scnrelfromlist{изображение в базовом алфавите sc.s-коннекторов}{\scnfileclass{$\sim$ |>};\scnfileclass{<| $\sim$}}
\scnrelfromlist{изображение в расширенном алфавите sc.s-коннекторов}{\scnfileclass{$\sim \not\ni$};\scnfileclass{ $\notin \sim$}}
\scnrelfromlist{изображение в SCg-коде}{\scgfileitem{figures/intro/scs/sc.s-connectors/negTempOrient.png};\scgfileitem{figures/intro/scs/sc.s-connectors/negTempOrient2.png}}

\scnheader{константная временная нечеткая sc.s-дуга принадлежности}
\scnrelfromlist{изображение в базовом алфавите sc.s-коннекторов}{\scnfileclass{$\sim$ />};\scnfileclass{</ $\sim$}}
\scnrelfromlist{изображение в расширенном алфавите sc.s-коннекторов}{\scnfileclass{$\sim/\ni$};\scnfileclass{ $\in/\sim$}}
\scnrelfromlist{изображение в SCg-коде}{\scgfileitem{figures/intro/scs/sc.s-connectors/fuzTempOrient.png};\scgfileitem{figures/intro/scs/sc.s-connectors/fuzTempOrient2.png}}

\scnheader{переменная постоянная позитивная sc.s-дуга принадлежности}
\scnrelfromlist{изображение в базовом алфавите sc.s-коннекторов}{\scnfileclass{\textunderscore->};\scnfileclass{<-\textunderscore}}
\scnrelfromlist{изображение в расширенном алфавите sc.s-коннекторов}{\scnfileclass{\textunderscore$\ni$};\scnfileclass{\textunderscore$\in$}}
\scnrelfromlist{изображение в SCg-коде}{\scgfileitem{figures/intro/scs/sc.s-connectors/varPosPermOrient.png};\scgfileitem{figures/intro/scs/sc.s-connectors/varPosPermOrient2.png}}

\scnheader{переменная постоянная негативная sc.s-дуга принадлежности}
\scnrelfromlist{изображение в базовом алфавите sc.s-коннекторов}{\scnfileclass{\textunderscore-|>};\scnfileclass{<|-\textunderscore}}
\scnrelfromlist{изображение в расширенном алфавите sc.s-коннекторов}{\scnfileclass{\textunderscore$\not\ni$};\scnfileclass{$\notin$\textunderscore}}
\scnrelfromlist{изображение в SCg-коде}{\scgfileitem{figures/intro/scs/sc.s-connectors/varNegPermOrient.png};\scgfileitem{figures/intro/scs/sc.s-connectors/varNegPermOrient2.png}}

\scnheader{переменная постоянная нечеткая sc.s-дуга принадлежности}
\scnrelfromlist{изображение в базовом алфавите sc.s-коннекторов}{\scnfileclass{\textunderscore-/>};\scnfileclass{</-\textunderscore}}
\scnrelfromlist{изображение в расширенном алфавите sc.s-коннекторов}{\scnfileclass{\textunderscore$/\ni$};\scnfileclass{$\in/$\textunderscore}}
\scnrelfromlist{изображение в SCg-коде}{\scgfileitem{figures/intro/scs/sc.s-connectors/varFuzPermOrient.png};\scgfileitem{figures/intro/scs/sc.s-connectors/varFuzPermOrient2.png}}

\scnheader{переменная временная позитивная sc.s-дуга принадлежности}
\scnrelfromlist{изображение в базовом алфавите sc.s-коннекторов}{\scnfileclass{\textunderscore$\sim$>};\scnfileclass{<$\sim$\textunderscore}}
\scnrelfromlist{изображение в расширенном алфавите sc.s-коннекторов}{\scnfileclass{\textunderscore$\sim\ni$};\scnfileclass{$\in\sim$\textunderscore}}
\scnrelfromlist{изображение в SCg-коде}{\scgfileitem{figures/intro/scs/sc.s-connectors/varPosTempOrient.png};\scgfileitem{figures/intro/scs/sc.s-connectors/varPosTempOrient2.png}}

\scnheader{переменная временная негативная sc.s-дуга принадлежности}
\scnrelfromlist{изображение в базовом алфавите sc.s-коннекторов}{\scnfileclass{\textunderscore$\sim$|>};\scnfileclass{<|$\sim$\textunderscore}}
\scnrelfromlist{изображение в расширенном алфавите sc.s-коннекторов}{\scnfileclass{\textunderscore$\sim\not\ni$};\scnfileclass{$\notin\sim$\textunderscore}}
\scnrelfromlist{изображение в SCg-коде}{\scgfileitem{figures/intro/scs/sc.s-connectors/varNegTempOrient.png};\scgfileitem{figures/intro/scs/sc.s-connectors/varNegTempOrient2.png}}

\scnheader{переменная временная нечеткая sc.s-дуга принадлежности}
\scnrelfromlist{изображение в базовом алфавите sc.s-коннекторов}{\scnfileclass{\textunderscore$\sim$/>};\scnfileclass{</$\sim$\textunderscore}}
\scnrelfromlist{изображение в расширенном алфавите sc.s-коннекторов}{\scnfileclass{\textunderscore$\sim/\ni$};\scnfileclass{$\in/\sim$\textunderscore}}
\scnrelfromlist{изображение в SCg-коде}{\scgfileitem{figures/intro/scs/sc.s-connectors/varFuzTempOrient.png};\scgfileitem{figures/intro/scs/sc.s-connectors/varFuzTempOrient2.png}}

\scnheader{метапеременная постоянная позитивная sc.s-дуга принадлежности}
\scnrelfromlist{изображение в базовом алфавите sc.s-коннекторов}{\scnfileclass{\textunderscore\textunderscore->};\scnfileclass{<-\textunderscore\textunderscore}}
\scnrelfromlist{изображение в расширенном алфавите sc.s-коннекторов}{\scnfileclass{\textunderscore\textunderscore$\ni$};\scnfileclass{$\in$\textunderscore\textunderscore}}
\scnrelfromlist{изображение в SCg-коде}{\scgfileitem{figures/intro/scs/sc.s-connectors/metaPosPermOrient.png};\scgfileitem{figures/intro/scs/sc.s-connectors/metaPosPermOrient2.png}}

\scnheader{метапеременная постоянная негативная sc.s-дуга принадлежности}
\scnrelfromlist{изображение в базовом алфавите sc.s-коннекторов}{\scnfileclass{\textunderscore\textunderscore-|>};\scnfileclass{<|-\textunderscore\textunderscore}}
\scnrelfromlist{изображение в расширенном алфавите sc.s-коннекторов}{\scnfileclass{\textunderscore\textunderscore$\not\ni$};\scnfileclass{$\notin$\textunderscore\textunderscore}}
\scnrelfromlist{изображение в SCg-коде}{\scgfileitem{figures/intro/scs/sc.s-connectors/metaNegPermOrient.png};\scgfileitem{figures/intro/scs/sc.s-connectors/metaNegPermOrient2.png}}

\scnheader{метапеременная постоянная нечеткая sc.s-дуга принадлежности}
\scnrelfromlist{изображение в базовом алфавите sc.s-коннекторов}{\scnfileclass{\textunderscore\textunderscore-/>};\scnfileclass{</-\textunderscore\textunderscore}}
\scnrelfromlist{изображение в расширенном алфавите sc.s-коннекторов}{\scnfileclass{\textunderscore\textunderscore$/\ni$};\scnfileclass{$\in$/\textunderscore\textunderscore}}
\scnrelfromlist{изображение в SCg-коде}{\scgfileitem{figures/intro/scs/sc.s-connectors/metaFuzPermOrient.png};\scgfileitem{figures/intro/scs/sc.s-connectors/metaFuzPermOrient2.png}}

\scnheader{метапеременная временная позитивная sc.s-дуга принадлежности}
\scnrelfromlist{изображение в базовом алфавите sc.s-коннекторов}{\scnfileclass{\textunderscore\textunderscore$\sim$>};\scnfileclass{<$\sim$\textunderscore\textunderscore}}
\scnrelfromlist{изображение в расширенном алфавите sc.s-коннекторов}{\scnfileclass{\textunderscore\textunderscore$\sim\ni$};\scnfileclass{$\in\sim$\textunderscore\textunderscore}}
\scnrelfromlist{изображение в SCg-коде}{\scgfileitem{figures/intro/scs/sc.s-connectors/metaPosTempOrient.png};\scgfileitem{figures/intro/scs/sc.s-connectors/metaPosTempOrient2.png}}

\scnheader{метапеременная временная негативная sc.s-дуга принадлежности}
\scnrelfromlist{изображение в базовом алфавите sc.s-коннекторов}{\scnfileclass{\textunderscore\textunderscore$\sim$ |>};\scnfileclass{<| $\sim$\textunderscore\textunderscore}}
\scnrelfromlist{изображение в расширенном алфавите sc.s-коннекторов}{\scnfileclass{\textunderscore\textunderscore$\sim\not\ni$};\scnfileclass{$\notin\sim$\textunderscore\textunderscore}}
\scnrelfromlist{изображение в SCg-коде}{\scgfileitem{figures/intro/scs/sc.s-connectors/metaNegTempOrient.png};\scgfileitem{figures/intro/scs/sc.s-connectors/metaNegTempOrient2.png}}


\scnheader{метапеременная временная нечеткая sc.s-дуга принадлежности}
\scnrelfromlist{изображение в базовом алфавите sc.s-коннекторов}{\scnfileclass{\textunderscore\textunderscore$\sim$/>};\scnfileclass{</$\sim$\textunderscore\textunderscore}}
\scnrelfromlist{изображение в расширенном алфавите sc.s-коннекторов}{\scnfileclass{\textunderscore\textunderscore$\sim/\ni$};\scnfileclass{$\in/\sim$\textunderscore\textunderscore}}
\scnrelfromlist{изображение в SCg-коде}{\scgfileitem{figures/intro/scs/sc.s-connectors/metaFuzTempOrient.png};\scgfileitem{figures/intro/scs/sc.s-connectors/metaFuzTempOrient2.png}}

\scnheader{sc.s-ребро общего вида}
\scnrelfrom{изображение в базовом алфавите sc.s-коннекторов}{\scnfileclass{<>}}
\scnrelfrom{изображение в расширенном алфавите sc.s-коннекторов}{\scnfileclass{$\leftrightarrow$}}
\scnrelfrom{изображение в SCg-коде}{\scgfileitem{figures/intro/scs/sc.s-connectors/noorient.png}}

\scnheader{sc.s-дуга общего вида}
\scnrelfromlist{изображение в базовом алфавите sc.s-коннекторов}{\scnfileclass{>>};\scnfileclass{<<}}
\scnrelfromlist{изображение в расширенном алфавите sc.s-коннекторов}{\scnfileclass{$\rightarrow$};\scnfileclass{$\leftarrow$}}
\scnrelfromlist{изображение в SCg-коде}{\scgfileitem{figures/intro/scs/sc.s-connectors/orient.png};\scgfileitem{figures/intro/scs/sc.s-connectors/orient2.png}}

\scnheader{константное постоянное sc.s-ребро общего вида}
\scnrelfrom{изображение в базовом алфавите sc.s-коннекторов}{\scnfileclass{<=>}}
\scnrelfrom{изображение в расширенном алфавите sc.s-коннекторов}{\scnfileclass{$\Leftrightarrow$}}
\scnrelfrom{изображение в SCg-коде}{\scgfileitem{figures/intro/scs/sc.s-connectors/constPermNoorien.png}}

\scnheader{константная постоянная sc.s-дуга общего вида}
\scnrelfromlist{изображение в базовом алфавите sc.s-коннекторов}{\scnfileclass{=>};\scnfileclass{<=}}
\scnrelfromlist{изображение в расширенном алфавите sc.s-коннекторов}{\scnfileclass{$\Rightarrow$};\scnfileclass{$\Leftarrow$}}
\scnrelfromlist{изображение в SCg-коде}{\scgfileitem{figures/intro/scs/sc.s-connectors/constPermOrient.png};\scgfileitem{figures/intro/scs/sc.s-connectors/constPermOrient2.png}}

\scnheader{константное временное sc.s-ребро общего вида}
\scnrelfrom{изображение в базовом алфавите sc.s-коннекторов}{\scnfileclass{$\sim<=>$}}
\scnrelfrom{изображение в расширенном алфавите sc.s-коннекторов}{\scnfileclass{$\sim\Leftrightarrow$}}
\scnrelfrom{изображение в SCg-коде}{\scgfileitem{figures/intro/scs/sc.s-connectors/constTempNoorien.png}}

\scnheader{константная временная sc.s-дуга общего вида}
\scnrelfromlist{изображение в базовом алфавите sc.s-коннекторов}{\scnfileclass{$\sim=>$};\scnfileclass{$<=\sim$}}
\scnrelfromlist{изображение в расширенном алфавите sc.s-коннекторов}{\scnfileclass{$\sim\Rightarrow$};\scnfileclass{$\Leftarrow\sim$}}
\scnrelfromlist{изображение в SCg-коде}{\scgfileitem{figures/intro/scs/sc.s-connectors/constTempOrient.png};\scgfileitem{figures/intro/scs/sc.s-connectors/constTempOrient2.png}}

\scnheader{переменное постоянное sc.s-ребро общего вида}
\scnrelfrom{изображение в базовом алфавите sc.s-коннекторов}{\scnfileclass{$\textunderscore<=>$}}
\scnrelfrom{изображение в расширенном алфавите sc.s-коннекторов}{\scnfileclass{$\textunderscore\Leftrightarrow$}}
\scnrelfrom{изображение в SCg-коде}{\scgfileitem{figures/intro/scs/sc.s-connectors/varPermNoorien.png}}

\scnheader{переменная постоянная sc.s-дуга общего вида}
\scnrelfromlist{изображение в базовом алфавите sc.s-коннекторов}{\scnfileclass{$\textunderscore=>$};\scnfileclass{$<=\textunderscore$}}
\scnrelfromlist{изображение в расширенном алфавите sc.s-коннекторов}{\scnfileclass{$\textunderscore\Rightarrow$};\scnfileclass{$\Leftarrow\textunderscore$}}
\scnrelfromlist{изображение в SCg-коде}{\scgfileitem{figures/intro/scs/sc.s-connectors/varPermOrient.png};\scgfileitem{figures/intro/scs/sc.s-connectors/varPermOrient2.png}}

\scnheader{переменное временное sc.s-ребро общего вида}
\scnrelfrom{изображение в базовом алфавите sc.s-коннекторов}{\scnfileclass{$\textunderscore\sim<=>$}}
\scnrelfrom{изображение в расширенном алфавите sc.s-коннекторов}{\scnfileclass{$\textunderscore\sim\Leftrightarrow$}}
\scnrelfrom{изображение в SCg-коде}{\scgfileitem{figures/intro/scs/sc.s-connectors/varTempNoorien.png}}

\scnheader{переменная временная sc.s-дуга общего вида}
\scnrelfromlist{изображение в базовом алфавите sc.s-коннекторов}{\scnfileclass{$\textunderscore\sim=>$};\scnfileclass{$<=\sim\textunderscore$}}
\scnrelfromlist{изображение в расширенном алфавите sc.s-коннекторов}{\scnfileclass{$\textunderscore\sim\Rightarrow$};\scnfileclass{$\Leftarrow\sim\textunderscore$}}
\scnrelfromlist{изображение в SCg-коде}{\scgfileitem{figures/intro/scs/sc.s-connectors/varTempOrient.png};\scgfileitem{figures/intro/scs/sc.s-connectors/varTempOrient2.png}}

\scnheader{метапеременное постоянное sc.s-ребро общего вида}
\scnrelfrom{изображение в базовом алфавите sc.s-коннекторов}{\scnfileclass{$\textunderscore\textunderscore<=>$}}
\scnrelfrom{изображение в расширенном алфавите sc.s-коннекторов}{\scnfileclass{$\textunderscore\textunderscore\Leftrightarrow$}}
\scnrelfrom{изображение в SCg-коде}{\scgfileitem{figures/intro/scs/sc.s-connectors/metaPermNoorien.png}}

\scnheader{метапеременная постоянная sc.s-дуга общего вида}
\scnrelfromlist{изображение в базовом алфавите sc.s-коннекторов}{\scnfileclass{$\textunderscoree\textunderscore=>$};\scnfileclass{$<=\textunderscore\textunderscore$}}
\scnrelfromlist{изображение в расширенном алфавите sc.s-коннекторов}{\scnfileclass{$\textunderscore\textunderscore\Rightarrow$};\scnfileclass{$\Leftarrow\textunderscore\textunderscore$}}
\scnrelfromlist{изображение в SCg-коде}{\scgfileitem{figures/intro/scs/sc.s-connectors/metaPermOrient.png};\scgfileitem{figures/intro/scs/sc.s-connectors/metaPermOrient2.png}}

\scnheader{метапеременное временное sc.s-ребро общего вида}
\scnrelfrom{изображение в базовом алфавите sc.s-коннекторов}{\scnfileclass{$\textunderscore\textunderscore\sim<=>$}}
\scnrelfrom{изображение в расширенном алфавите sc.s-коннекторов}{\scnfileclass{$\textunderscore\textunderscore\sim\Leftrightarrow$}}
\scnrelfrom{изображение в SCg-коде}{\scgfileitem{figures/intro/scs/sc.s-connectors/metaTempNoorien.png}}

\scnheader{метапеременная временная sc.s-дуга общего вида}
\scnrelfromlist{изображение в базовом алфавите sc.s-коннекторов}{\scnfileclass{$\textunderscore\textunderscore\sim=>$};\scnfileclass{$<=\sim\textunderscore\textunderscore$}}
\scnrelfromlist{изображение в расширенном алфавите sc.s-коннекторов}{\scnfileclass{$\textunderscore\textunderscore\sim\Rightarrow$};\scnfileclass{$\Leftarrow\sim\textunderscore\textunderscore$}}
\scnrelfromlist{изображение в SCg-коде}{\scgfileitem{figures/intro/scs/sc.s-connectors/metaTempOrient.png};\scgfileitem{figures/intro/scs/sc.s-connectors/metaTempOrient2.png}}
\scnendstruct\\
\scnnote{Аналогичным образом может быть описано изображение всех классов sc.s-коннекторов, представленных в \textit{Таблица. Алфавит sc.s-коннекторов, соответствующих sc.g-дугам принадлежности} и \textit{Таблица. Алфавит sc.s-коннекторов, соответствующих sc.g-дугам принадлежности}}

\scnheader{Примеры синтаксической трансформации sc.s-предложений с использованием Расширенного алфавита sc.s-коннекторов}
\begin{scnstruct}
    \bigskip
    \scnfilelongheader{\textbf{\textit{si}} $\Rightarrow$ \textit{включение*}: \textbf{\textit{sj}}}
    \scnrelfrom{синтаксическая трансформация}{\scnfilelong{\textbf{\textit{si}} $\supseteq$ \textbf{\textit{sj}}}}
    \begin{scnindent}
        \scnrelboth{семантическая эквивалентность}{\scnfileimage[20em]{Contents/part_kb/images/scs/sc.s-connectors/examples/scs_transf_inclusion_const.png}}
    \end{scnindent}
    
    \bigskip
    \scnfilelongheader{\textbf{\textit{si}} $\textunderscore\Rightarrow$ \textit{включение*}:: \textbf{\textit{sj}}}
    \scnrelfrom{синтаксическая трансформация}{\scnfilelong{\textbf{\textit{si}} $\textunderscore\supseteq$ \textbf{\textit{sj}}}}
    \begin{scnindent}
        \scnrelboth{семантическая эквивалентность}{\scnfileimage[20em]{Contents/part_kb/images/scs/sc.s-connectors/examples/scs_transf_inclusion_var.png}}
    \end{scnindent}
    
    \bigskip
    \scnfilelongheader{\textbf{\textit{si}} $\textunderscore\textunderscore\Rightarrow$ \textit{включение*}::: \textbf{\textit{sj}}}
    \scnrelfrom{синтаксическая трансформация}{\scnfilelong{\textbf{\textit{si}} $\textunderscore\textunderscore\supseteq$ \textbf{\textit{sj}}}}
    \begin{scnindent}
        \scnrelboth{семантическая эквивалентность}{\scnfileimage[20em]{Contents/part_kb/images/scs/sc.s-connectors/examples/scs_transf_inclusion_meta.png}}
    \end{scnindent}
    
    \bigskip
    \scnfilelongheader{\textbf{\textit{si}} $\Rightarrow$ \textit{строгое включение*}: \textbf{\textit{sj}}}
    \scnrelfrom{синтаксическая трансформация}{\scnfilelong{\textbf{\textit{si}} $\supset$ \textbf{\textit{sj}}}}
    \begin{scnindent}
        \scnrelboth{семантическая эквивалентность}{\scnfileimage[20em]{Contents/part_kb/images/scs/sc.s-connectors/examples/scs_transf_strict_inclusion_const.png}}
    \end{scnindent}
    
    \bigskip
    \scnfilelongheader{\textbf{\textit{si}} $\textunderscore\Rightarrow$ \textit{строгое включение*}:: \textbf{\textit{sj}}}
    \scnrelfrom{синтаксическая трансформация}{\scnfilelong{\textbf{\textit{si}} $\textunderscore\supset$ \textbf{\textit{sj}}}}
    \begin{scnindent}
        \scnrelboth{семантическая эквивалентность}{\scnfileimage[20em]{Contents/part_kb/images/scs/sc.s-connectors/examples/scs_transf_strict_inclusion_var.png}}
    \end{scnindent}
    
    \bigskip
    \scnfilelongheader{\textbf{\textit{si}} $\textunderscore\textunderscore\Rightarrow$ \textit{строгое включение*}::: \textbf{\textit{sj}}}
    \scnrelfrom{синтаксическая трансформация}{\scnfilelong{\textbf{\textit{si}} $\textunderscore\textunderscore\supset$ \textbf{\textit{sj}}}}
    \begin{scnindent}
        \scnrelboth{семантическая эквивалентность}{\scnfileimage[20em]{Contents/part_kb/images/scs/sc.s-connectors/examples/scs_transf_strict_inclusion_meta.png}}
    \end{scnindent}
    
    \bigskip
    \scnfilelongheader{\textbf{\textit{si}} $\Rightarrow$ \textit{порядок величин*}: \textbf{\textit{sj}}}
    \scnrelfrom{синтаксическая трансформация}{\scnfilelong{\textbf{\textit{si}} $\geq$ \textbf{\textit{sj}}}}
    \begin{scnindent}
        \scnrelboth{семантическая эквивалентность}{\scnfileimage[20em]{Contents/part_kb/images/scs/sc.s-connectors/examples/scs_transf_value_order_const.png}}
    \end{scnindent}
    
    \bigskip
    \scnfilelongheader{\textbf{\textit{si}} $\textunderscore\Rightarrow$ \textit{порядок величин*}:: \textbf{\textit{sj}}}
    \scnrelfrom{синтаксическая трансформация}{\scnfilelong{\textbf{\textit{si}} $\textunderscore\geq$ \textbf{\textit{sj}}}}
    \begin{scnindent}
        \scnrelboth{семантическая эквивалентность}{\scnfileimage[20em]{Contents/part_kb/images/scs/sc.s-connectors/examples/scs_transf_value_order_var.png}}
    \end{scnindent}
    
    \bigskip
    \scnfilelongheader{\textbf{\textit{si}} $\textunderscore\textunderscore\Rightarrow$ \textit{порядок величин*}::: \textbf{\textit{sj}}}
    \scnrelfrom{синтаксическая трансформация}{\scnfilelong{\textbf{\textit{si}} $\textunderscore\textunderscore\geq$ \textbf{\textit{sj}}}}
    \begin{scnindent}
        \scnrelboth{семантическая эквивалентность}{\scnfileimage[20em]{Contents/part_kb/images/scs/sc.s-connectors/examples/scs_transf_value_order_meta.png}}
    \end{scnindent}
        
    \bigskip
    \scnfilelongheader{\textbf{\textit{si}} $\Rightarrow$ \textit{строгий порядок величин*}: \textbf{\textit{sj}}}
    \scnrelfrom{синтаксическая трансформация}{\scnfilelong{\textbf{\textit{si}} $>$ \textbf{\textit{sj}}}}
    \begin{scnindent}
        \scnrelboth{семантическая эквивалентность}{\scnfileimage[20em]{Contents/part_kb/images/scs/sc.s-connectors/examples/scs_transf_value_strict_order_const.png}}
    \end{scnindent}
    \scnnote{Данная синтаксическая трансформация представлена для расширенного алфавита. Обратите внимание, что в базовом алфавите таким же символом представлена sc.s-дуга общего вида.}
    
    \bigskip
    \scnfilelongheader{\textbf{\textit{si}} $\textunderscore\Rightarrow$ \textit{строгий порядок величин*}:: \textbf{\textit{sj}}}
    \scnrelfrom{синтаксическая трансформация}{\scnfilelong{\textbf{\textit{si}} $\textunderscore>$ \textbf{\textit{sj}}}}
    \begin{scnindent}
        \scnrelboth{семантическая эквивалентность}{\scnfileimage[20em]{Contents/part_kb/images/scs/sc.s-connectors/examples/scs_transf_value_strict_order_var.png}}
    \end{scnindent}

    \bigskip
    \scnfilelongheader{\textbf{\textit{si}} $\textunderscore\textunderscore\Rightarrow$ \textit{строгий порядок величин*}::: \textbf{\textit{sj}}}
    \scnrelfrom{синтаксическая трансформация}{\scnfilelong{\textbf{\textit{si}} $\textunderscore\textunderscore>$ \textbf{\textit{sj}}}}
    \begin{scnindent}
        \scnrelboth{семантическая эквивалентность}{\scnfileimage[20em]{Contents/part_kb/images/scs/sc.s-connectors/examples/scs_transf_value_strict_order_meta.png}}
    \end{scnindent}

    \bigskip
    \scnfilelongheader{\textbf{\textit{si}} $\Rightarrow$ \textit{внешний идентификатор*}: \textbf{\textit{sj}}}
    \scnrelfrom{синтаксическая трансформация}{\scnfilelong{\textbf{\textit{si}} $:=$ \textbf{\textit{sj}}}}
    \begin{scnindent}
        \scnrelboth{семантическая эквивалентность}{\scnfileimage[20em]{Contents/part_kb/images/scs/sc.s-connectors/examples/scs_transf_external_idtf_const.png}}
    \end{scnindent}

    \bigskip
    \scnfilelongheader{\textbf{\textit{si}} $\textunderscore\Rightarrow$ \textit{внешний идентификатор*}:: \textbf{\textit{sj}}}
    \scnrelfrom{синтаксическая трансформация}{\scnfilelong{\textbf{\textit{si}} $\textunderscore:=$ \textbf{\textit{sj}}}}
    \begin{scnindent}
        \scnrelboth{семантическая эквивалентность}{\scnfileimage[20em]{Contents/part_kb/images/scs/sc.s-connectors/examples/scs_transf_external_idtf_var.png}}
    \end{scnindent}
    
    \bigskip
    \scnfilelongheader{\textbf{\textit{si}} $\textunderscore\textunderscore\Rightarrow$ \textit{внешний идентификатор*}::: \textbf{\textit{sj}}}
    \scnrelfrom{синтаксическая трансформация}{\scnfilelong{\textbf{\textit{si}} $\textunderscore\textunderscore:=$ \textbf{\textit{sj}}}}
    \begin{scnindent}
        \scnrelboth{семантическая эквивалентность}{\scnfileimage[20em]{Contents/part_kb/images/scs/sc.s-connectors/examples/scs_transf_external_idtf_meta.png}}
    \end{scnindent}
    
    \bigskip
    \scnfilelongheader{\textbf{\textit{si}} $\Leftrightarrow$ \textit{синонимия*}: \textbf{\textit{sj}}}
    \scnrelfrom{синтаксическая трансформация}{\scnfilelong{\textbf{\textit{si}} $=$ \textbf{\textit{sj}}}}
    \begin{scnindent}
        \scnrelboth{семантическая эквивалентность}{\scnfileimage[20em]{Contents/part_kb/images/scs/sc.s-connectors/examples/scs_transf_synonymy_const.png}}
    \end{scnindent}
    
    \bigskip
    \scnfilelongheader{\textbf{\textit{si}} $\Rightarrow$ \textit{погружение*}: \textbf{\textit{sj}}}
    \scnrelfrom{синтаксическая трансформация}{\scnfilelong{\textbf{\textit{si}} $\supset=$ \textbf{\textit{sj}}}}
    \begin{scnindent}
        \scnrelboth{семантическая эквивалентность}{\scnfileimage[20em]{Contents/part_kb/images/scs/sc.s-connectors/examples/scs_transf_insertion_const.png}}
    \end{scnindent}
\end{scnstruct}
\scntext{примечание}{Аналогичным образом может быть описана трансформация предложений, содержащих любые классы \mbox{sc.s-коннекторов}, за исключением тех классов sc.s-коннекторов, которые соответствуют классам \mbox{sc-коннекторов}, входящим в Ядро SC-кода.}
\scntext{примечание}{В общем случае \textit{sc-элементы}, инцидентные \textit{sc-коннекторам}, классы которых описаны в данном примере, могут быть как \textit{sc-константами}, так и \textit{sc-переменными} (в том числе \textit{sc-метапеременными}). При этом как \textit{переменному sc-коннектору} может соответствовать \textit{константный sc-узел}, так и \textit{константному sc-коннектору} может соответствовать \textit{переменный sc-узел} (например, если возникает необходимость переменному sc-узлу приписать \textit{внешний идентификатор*}). Последняя ситуация встречается не очень часто и возникает в случае, когда область определения соответствующего \textit{отношения} имеет непустое пересечение с классом \textit{sc-переменных}.}

\scnheader{знак равенства}
\scneqfileclass{=}
\scnidtf{связь синонимии между sc-идентификаторами, по крайней мере один из которого является sc-выражением}
\scnnote{знак равенства является \textit{sc.s-разделителем} двух \textit{sc-идентификаторов}, которые идентифицируют (именуют) одну и ту же \textit{сущность} и, соответственно, являются \textit{sc-идентификаторами}* (внешними уникальными изображениями) одного и того же \textit{sc-элемента}. При этом из указанных двух sc-идентификаторов чаще всего один является простым sc-идентификатором, а второй -- sc-выражением. Реже оба эти sc-идентификатора являются sc-выражениями. И совсем редко оба они являются простыми sc-идентификаторами. Последнее обозначает то, что оба эти sc-идентификатора являются основными \textit{sc-идентификаторами}* одного и того же \textit{sc-элемента}. Пример:

\textit{SC-код} = \textit{sc.s-текст};;

Здесь первый \textit{sc-идентификатор} является \textit{именем собственным}, а второй -- \textit{именем нарицательным}.

При трансляции \textit{sc.s-текста} в \textit{SC-код} знаку равенства на некотором этапе может быть поставлено в соответствие \textit{sc-ребро}, принадлежащее отношению \textit{синонимии}* \textit{sc-элементов}, идентифицируемых \mbox{\textit{sc-идентификаторами}}, связанными знаком равенства. Но на последующем этапе указанное \textit{sc-ребро} \uline{удаляется}, а связанные им \textit{sc-элементы} \uline{склеиваются}. Таким образом \textit{sc-ребро}, принадлежащее отношению \textit{синонимии}* sc-элементов, имеет не только денотационную, но и операционную семантику.}

\scnheader{знак равенства с включением}
\scneq{{\normalfont(}\scnfileclass{$\supset$=} $\cup$ \scnfileclass{=$\subset$}{\normalfont)}}
\scnidtf{изображение \textit{sc-дуги}, принадлежащей отношению \textit{погружения}*, связывающей два \textit{sc-узла}, обозначающих \textit{sc-тексты}, первый из которых является погружающим, а второй (в который указанная \textit{sc-дуга} входит) является погружаемым, вводимым в состав первого \textit{sc-текста}}
\scnnote{\textit{sc-дуга}, принадлежащая отношению \textit{погружения}*, интерпретируется как команда погружения одного \textit{sc-текста} в состав другого. При выполнении этой команды (1) все \textit{sc-элементы} погружаемого \textit{sc-текста} становятся элементами, принадлежащими погружающему \textit{sc-тексту}, (2) все синонимичные \textit{sc-элементы}, оказавшиеся в составе погружающего \textit{sc-текста}, склеиваются, (3) \textit{sc-узел}, обозначающий погружаемый \textit{sc-текст}, а так же спецификация этого \textit{sc-текста} (включая перечень всех его \textit{sc-элементов}) погружается в историю эволюции \textit{базы знаний} вместе со спецификацией события погружения рассматриваемого \textit{sc-текста} в состав \textit{базы знаний}.}

\scnheader{{\normalfont(}знак равенства $\cup$ знак равенства с включением{\normalfont)}}
\scnnote{указанные sc.s-коннекторы отличаются от остальных sc.s-коннекторов тем, что они и соответствующие им sc-коннекторы (sc-ребра, принадлежащих отношению синонимии sc-элементов и sc-дуги, принадлежащие отношению погружения одного sc-текста в состав другого) имеют не только денотационную, но и операционную семантику, т.к. являются командами склеивания и командами погружения.}

\scnheader{sc.s-коннектор, соответствующий sc.g-дуге принадлежности}
\scnrelfrom{таблица}{Таблица. Алфавит sc.s-коннекторов, соответствующих sc.g-дугам принадлежности}

\scnheader{sc.s-коннектор, соответствующий sc.g-коннектору, который не является sc.g-дугой принадлежности}
\scnaddhind{1}
\scnrelfrom{таблица}{Таблица. Алфавит sc.s-коннекторов, соответствующих sc.g-коннекторам, которые не являются sc-дугами принадлежности}

\scnheader{Таблица. Алфавит sc.s-коннекторов, соответствующих sc.g-дугам принадлежности}
\scneqtable{
\begin{longtable}[l]{|m{4.6cm}|m{3.5cm}|m{2cm}|m{2cm}|m{2cm}|m{2cm}|}
	\hline
	\multicolumn{1}{|c}{\parbox[c]{4.6cm}{Класс \mbox{\textit{sc-элементов}}}} &
	\multicolumn{1}{|c}{\parbox[c]{3.5cm}{Изображение \textit{\mbox{sc.g-коннектора}}}} &
	\multicolumn{2}{|c}{\parbox[c]{4cm}{Изображение \textit{\mbox{sc.s-коннектора}} в Расширенном алфавите}} &
	\multicolumn{2}{|c|}{\parbox[c]{4cm}{Изображение \textit{\mbox{sc.s-коннектора}} в Базовом алфавите}}
	\hline
	\endhead
	
	\parbox[|m]{4.6cm}{\textit{ \\ константная постоянная позитивная sc-дуга принадлежности \\}} & \parbox[|m]{3.5cm}{\centering\includegraphics[width=0.8\linewidth]{figures/intro/scg/arcs/const/scg_const_perm_positive2.png}} & \parbox[|c]{2cm}{\centering\ni} & \parbox[|m]{2cm}{\centering\in} & \parbox[|c]{2cm}{\centering->} & \parbox[|m|]{2cm}{\centering<-} \\
	\hline
	
	\parbox[m|]{4.6cm}{\textit{\\ константная постоянная негативная sc-дуга принадлежности \\}} & \parbox[m|]{3.5cm}{\centering \includegraphics[width=0.8\linewidth]{figures/intro/scg/arcs/const/scg_const_perm_negative2.png}} & \parbox[m]{2cm}{\centering \not\ni} & \parbox[m|]{2cm}{\centering\notin} & \parbox[m|]{2cm}{\centering-|>} & \parbox[m]{2cm}{\centering<|-} \\
	\hline
	
	\parbox[m]{4.6cm}{\textit{\\ константная постоянная нечеткая sc-дуга принадлежности \\}} & \parbox[m]{3.5cm}{\centering\includegraphics[width=0.8\linewidth]{figures/intro/scg/arcs/const/scg_const_perm_fuzzy2.png}} & \parbox[m]{2cm}{\centering / \ni} & \parbox[m]{2cm}{\centering \in /} & \parbox[m]{2cm}{\centering -/>} & \parbox[m]{2cm}{\centering </-} \\
	\hline
	
	\parbox[m]{4.6cm}{\textit{\\ константная временная позитивная sc-дуга принадлежности \\}} & \parbox[m]{3.5cm}{\centering \includegraphics[width=0.8\linewidth]{figures/intro/scg/arcs/const/scg_const_temp_positive2.png}} & \parbox[m]{2cm}{\centering \sim \ni} & \parbox[m]{2cm}{\centering \in \sim} & \parbox[m]{2cm}{\centering \sim>} & \parbox[m]{2cm}{\centering <\sim} \\
	\hline
	
	\parbox[m]{4.6cm}{\textit{\\ константная временная негативная sc-дуга принадлежности \\}} & \parbox[m]{3.5cm}{\centering \includegraphics[width=0.8\linewidth]{figures/intro/scg/arcs/const/scg_const_temp_negative2.png}} & \parbox[m]{2cm}{\centering \sim \not\ni} & \parbox[m]{2cm}{\centering \notin \sim} & \parbox[m]{2cm}{\centering \sim|>} & \parbox[m]{2cm}{\centering <|\sim} \\
	\hline
	
	\parbox[m]{4.6cm}{\textit{\\ константная временная нечеткая sc-дуга принадлежности \\}} & \parbox[m]{3.5cm}{\centering \includegraphics[width=0.8\linewidth]{figures/intro/scg/arcs/const/scg_const_temp_fuzzy2.png}} & \parbox[m]{2cm}{\centering \sim / \ni}  & \parbox[m]{2cm}{\centering \in/\sim} & \parbox[m]{2cm}{\centering \sim/>} & \parbox[m]{2cm}{\centering </\sim} \\
	\hline
	
	\parbox[m]{4.6cm}{\textit{\\ переменная постоянная позитивная sc-дуга принадлежности \\}} & \parbox[m]{3.5cm}{\centering \includegraphics[width=0.8\linewidth]{figures/intro/scg/arcs/var/scg_var_perm_positive2.png}} & \parbox[m]{2cm}{\centering \textunderscore \ni} & \parbox[m]{2cm}{\centering \textunderscore \in} & \parbox[m]{2cm}{\centering \textunderscore->} & \parbox[m]{2cm}{\centering <-\textunderscore} \\
	\hline
	
	\parbox[m]{4.6cm}{\textit{\\ переменная постоянная негативная sc-дуга принадлежности \\}} & \parbox[m]{3.5cm}{\centering \includegraphics[width=0.8\linewidth]{figures/intro/scg/arcs/var/scg_var_perm_negative2.png}} & \parbox[m]{2cm}{\centering \textunderscore \not\ni} & \parbox[m]{2cm}{\centering \notin \textunderscore} & \parbox[m]{2cm}{\centering \textunderscore-|>} & \parbox[m]{2cm}{\centering <|-\textunderscore} \\
	\hline
	
	\parbox[m]{4.6cm}{\textit{\\ переменная постоянная нечеткая sc-дуга принадлежности \\}} & \parbox[m]{3.5cm}{\centering \includegraphics[width=0.8\linewidth]{figures/intro/scg/arcs/var/scg_var_perm_fuzzy2.png}} & \parbox[m]{2cm}{\centering \textunderscore /\ni} & \parbox[m]{2cm}{\centering \in/\textunderscore} & \parbox[m]{2cm}{\centering \textunderscore-/>} & \parbox[m]{2cm}{\centering </-\textunderscore} \\
	\hline
	
	\parbox[m]{4.6cm}{\textit{\\ переменная временная позитивная sc-дуга принадлежности \\}} & \parbox[m]{3.5cm}{\centering \includegraphics[width=0.8\linewidth]{figures/intro/scg/arcs/var/scg_var_temp_positive2.png}} & \parbox[m]{2cm}{\centering \textunderscore \sim \ni} & \parbox[m]{2cm}{\centering \in \sim \textunderscore} & \parbox[m]{2cm}{\centering \textunderscore \sim >} & \parbox[m]{2cm}{\centering < \sim \textunderscore} \\
	\hline
	
	\parbox[m]{4.6cm}{\textit{\\ переменная временная негативная sc-дуга принадлежности \\}} & \parbox[m]{3.5cm}{\centering \includegraphics[width=0.8\linewidth]{figures/intro/scg/arcs/var/scg_var_temp_negative2.png}} & \parbox[m]{2cm}{\centering \textunderscore \sim \not$\ni$} & \parbox[c]{2cm}{\centering \notin \sim \textunderscore} & \parbox[m]{2cm}{\centering \textunderscore \sim |>} & \parbox[m]{2cm}{\centering <| \sim \textunderscore} \\
	\hline
	
	\parbox[m]{4.6cm}{\textit{\\переменная временная нечеткая sc-дуга принадлежности\\}} & \parbox[m]{3.5cm}{\centering \includegraphics[width=0.8\linewidth]{figures/intro/scg/arcs/var/scg_var_temp_fuzzy2.png}} & \parbox[m]{2cm}{\centering \textunderscore \sim / \ni} & \parbox[m]{2cm}{\centering \in/\sim\textunderscore} & \parbox[m]{2cm}{\centering \textunderscore \sim />} & \parbox[m]{2cm}{\centering </\sim \textunderscore} \\
	\hline
	
	\parbox[m]{4.6cm}{\textit{\\метапеременная постоянная позитивная sc-дуга принадлежности\\}} & \parbox[m]{3.5cm}{\centering \includegraphics[width=0.8\linewidth]{figures/intro/scg/arcs/meta/scg_metavar_perm_positive2.png}} & \parbox[m]{2cm}{\centering\textunderscore \textunderscore\ni} & \parbox[m]{2cm}{\centering\in\textunderscore\textunderscore} & \parbox[m]{2cm}{\centering\textunderscore\textunderscore->} & \parbox[m]{2cm}{\centering <-\textunderscore\textunderscore} \\
	\hline
	
	\parbox[m]{4.6cm}{\textit{\\ метапеременная постоянная негативная sc-дуга принадлежности \\}} & \parbox[m]{3.5cm}{\centering \includegraphics[width=0.8\linewidth]{figures/intro/scg/arcs/meta/scg_metavar_perm_negative2.png}} & \parbox[m]{2cm}{\centering \textunderscore\textunderscore \not\ni} & \parbox[m]{2cm}{\centering \notin\textunderscore\textunderscore} & \parbox[m]{2cm}{\centering \textunderscore\textunderscore-|>} & \parbox[m]{2cm}{\centering <|-\textunderscore\textunderscore} \\
	\hline
	
	\parbox[m]{4.6cm}{\textit{\\ метапеременная постоянная нечеткая sc-дуга принадлежности \\}} & \parbox[m]{3.5cm}{\centering \includegraphics[width=0.8\linewidth]{figures/intro/scg/arcs/meta/scg_metavar_perm_fuzzy2.png}} & \parbox[m]{2cm}{\centering\textunderscore\textunderscore/\ni} & \parbox[m]{2cm}{\centering\in/\textunderscore\textunderscore} & \parbox[m]{2cm}{\centering\textunderscore\textunderscore-/>} & \parbox[m]{2cm}{\centering </-\textunderscore\textunderscore} \\
	\hline
	
	\parbox[m]{4.6cm}{\textit{\\ метапеременная временная позитивная sc-дуга принадлежности \\}} & \parbox[m]{3.5cm}{\centering \includegraphics[width=0.8\linewidth]{figures/intro/scg/arcs/meta/scg_metavar_temp_positive2.png}} & \parbox[m]{2cm}{\centering\textunderscore\textunderscore\sim\ni} & \parbox[m]{2cm}{\centering\in\sim\textunderscore\textunderscore} & \parbox[m]{2cm}{\centering\textunderscore\textunderscore\sim>} & \parbox[m]{2cm}{\centering <\sim\textunderscore\textunderscore} \\
	\hline
	
	\parbox[m]{4.6cm}{\textit{\\ метапеременная временная негативная sc-дуга принадлежности \\}} & \parbox[m]{3.5cm}{\centering \includegraphics[width=0.8\linewidth]{figures/intro/scg/arcs/meta/scg_metavar_temp_negative2.png}} & \parbox[m]{2cm}{\centering \textunderscore\textunderscore\sim \not\ni} & \parbox[m]{2cm}{\centering\notin\sim\textunderscore\textunderscore} & \parbox[m]{2cm}{\centering\textunderscore\textunderscore\sim|>} & \parbox[m]{2cm}{\centering <|\sim\textunderscore\textunderscore} \\
	\hline
	
	\parbox[m]{4.6cm}{\textit{\\ метапеременная временная нечеткая sc-дуга принадлежности \\}} & \parbox[m]{3.5cm}{\centering \includegraphics[width=0.8\linewidth]{figures/intro/scg/arcs/meta/scg_metavar_temp_fuzzy2.png}} & \parbox[m]{2cm}{\centering \textunderscore\textunderscore\sim/\ni} & \parbox[m]{2cm}{\centering\in/\sim\textunderscore\textunderscore} & \parbox[m]{2cm}{\centering\textunderscore\textunderscore\sim/>} & \parbox[m]{2cm}{\centering</\sim\textunderscore\textunderscore} \\
	\hline
\end{longtable}
}
%---------------------------------------------
\scnheader{Таблица. Алфавит sc.s-коннекторов, соответствующих sc.g-коннекторам, которые не являются sc.g-дугами принадлежности}
\scneqtable{
\begin{longtable}[l]{|m{6.2cm}|m{2.5cm}|m{2.5cm}|m{2.5cm}|m{2.5cm}|}
	\hline
	\multicolumn{1}{|c}{\parbox[c]{6.2cm}{Изображение \textit{\mbox{sc-коннектора}} в SCg}} &
	\multicolumn{2}{|c}{\parbox[c]{5cm}{Изображение \textit{\mbox{sc.s-коннектора}} в Расширенном алфавите}} &
	\multicolumn{2}{|c|}{\parbox[c]{5cm}{Изображение \textit{\mbox{sc.s-коннектора}} в Базовом алфавите}}
	\hline
	\endhead
	
	\parbox[|c]{6.2cm}{\\\centering\includegraphics[width=0.8\linewidth]{figures/intro/scs/sc.s-connectors/noorient.png}\\} & \multicolumn{2}{c|}{ \parbox[c]{5cm}{\centering\leftrightarrow}} & \multicolumn{2}{c|}{ \parbox[c]{5cm}{\centering<>}} \\
	\hline
	
	\parbox[c]{6.2cm}{\\\centering\includegraphics[width=0.8\linewidth]{figures/intro/scs/sc.s-connectors/orient.png} \\} &\parbox[c]{2.5cm}{\\\centering\rightarrow\\} & \parbox[c]{2.5cm}{\\\centering\leftarrow\\} & \parbox[c]{2.5cm}{\\\centering >>\\} & \parbox[c|]{2.5cm}{\\\centering <<\\} \\
	\hline
	
	\parbox[c]{6.2cm}{\\\centering\includegraphics[width=0.8\linewidth]{figures/intro/scs/sc.s-connectors/constPermNoorien.png}\\} & \multicolumn{2}{c|}{\parbox[c]{5cm}{\centering\Leftrightarrow}} & \multicolumn{2}{c|}{\parbox[c]{5cm}{\centering<=>}} \\
	\hline
	
	\parbox[c]{6.2cm}{\\\centering\includegraphics[width=0.8\linewidth]{figures/intro/scs/sc.s-connectors/constPermOrient.png}\\} & \parbox[c]{2.5cm}{\centering\Rightarrow} & \parbox[c]{2.5cm}{\centering\Leftarrow} & \parbox[c]{2.5cm}{\centering =>} & \parbox[c|]{2.5cm}{\centering <=} \\
	\hline
	
	\parbox[c]{6.2cm}{\\\centering\includegraphics[width=0.8\linewidth]{figures/intro/scs/sc.s-connectors/constTempNoorien.png}\\} & \multicolumn{2}{c|}{\parbox[c]{5cm}{\centering\sim\Leftrightarrow}} & \multicolumn{2}{c|}{\parbox[c|]{5cm}{\centering\sim<=>}} \\
	\hline
	
	\parbox[c]{6.2cm}{\\\centering \includegraphics[width=0.8\linewidth]{figures/intro/scs/sc.s-connectors/constTempOrient.png}\\} & \parbox[c]{2.5cm}{\centering\sim\Rightarrow} & \parbox[c]{2.5cm}{\centering\Leftarrow\sim} & \parbox[c]{2.5cm}{\centering\sim=>} & \parbox[c|]{2.5cm}{\centering <=\sim} \\
	\hline
	
	\parbox[c]{6.2cm}{\\\centering\includegraphics[width=0.8\linewidth]{figures/intro/scs/sc.s-connectors/varPermNoorien.png}\\} & \multicolumn{2}{c|}{\parbox[c]{5cm}{\centering\textunderscore\Leftrightarrow}} & \multicolumn{2}{c|}{\parbox[c|]{5cm}{\centering\textunderscore<=>}} \\
	\hline
	
	\parbox[c]{6.2cm}{\\\centering\includegraphics[width=0.8\linewidth]{figures/intro/scs/sc.s-connectors/varPermOrient.png}\\} & \parbox[c]{2.5cm}{\centering\textunderscore\Rightarrow} & \parbox[c]{2.5cm}{\centering\Leftarrow\textunderscore} & \parbox[c]{2.5cm}{\centering\textunderscore=>} & \parbox[c|]{2.5cm}{\centering <=\textunderscore} \\
	\hline
	
	\parbox[c]{6.2cm}{\\\centering\includegraphics[width=0.8\linewidth]{figures/intro/scs/sc.s-connectors/varTempNoorien.png}\\} & \multicolumn{2}{c|}{\parbox[c]{5cm}{\centering\textunderscore\sim\Leftrightarrow}} & \multicolumn{2}{c|}{\parbox[c]{5cm}{\centering\textunderscore\sim<=>}} \\
	\hline
	
	\parbox[c]{6.2cm}{\\\centering\includegraphics[width=0.8\linewidth]{figures/intro/scs/sc.s-connectors/varTempOrient.png}\\} & \parbox[c]{2.5cm}{\centering\textunderscore\sim\Rightarrow} & \parbox[c]{2.5cm}{\centering\Leftarrow\sim\textunderscore} & \parbox[c]{2.5cm}{\centering\textunderscore\sim=>} & \parbox[c|]{2.5cm}{\centering <=\sim\textunderscore} \\
	\hline
	
	\parbox[c]{6.2cm}{\\\centering\includegraphics[width=0.8\linewidth]{figures/intro/scs/sc.s-connectors/metaPermNoorien.png}\\} & \multicolumn{2}{c|}{\parbox[c]{5cm}{\centering\textunderscore\textunderscore\Leftrightarrow}} & \multicolumn{2}{c|}{\parbox[c|]{5cm}{\centering\textunderscore\textunderscore<=>}} \\
	\hline
	
	\parbox[c]{6.2cm}{\\\centering\includegraphics[width=0.8\linewidth]{figures/intro/scs/sc.s-connectors/metaPermOrient.png}\\} & \parbox[c]{2.5cm}{\centering\textunderscore\textunderscore\Rightarrow} & \parbox[c]{2.5cm}{\centering\Leftarrow\textunderscore\textunderscore} & \parbox[c]{2.5cm}{\centering\textunderscore\textunderscore=>} & \parbox[c|]{2.5cm}{\centering <=\textunderscore\textunderscore} \\
	\hline
	
	\parbox[c]{6.2cm}{\\\centering\includegraphics[width=0.8\linewidth]{figures/intro/scs/sc.s-connectors/metaTempNoorien.png}\\} & \multicolumn{2}{c|}{\parbox[c]{5cm}{\centering\textunderscore\textunderscore\sim\Leftrightarrow}} & \multicolumn{2}{c|}{\parbox[c]{5cm}{\centering\textunderscore\textunderscore\sim<=>}} \\
	\hline
	
	\parbox[c]{6.2cm}{\\\centering\includegraphics[width=0.8\linewidth]{figures/intro/scs/sc.s-connectors/metaTempOrient.png}\\} & \parbox[c]{2.5cm}{\centering\textunderscore\textunderscore\sim\Rightarrow} & \parbox[c]{2.5cm}{\centering\Leftarrow\sim\textunderscore\textunderscore} & \parbox[c]{2.5cm}{\centering\textunderscore\textunderscore\sim\Rightarrow} & \parbox[c|]{2.5cm}{\centering\Leftarrow\sim\textunderscore\textunderscore} \\
	\hline
	
	\parbox[c]{6.2cm}{\\\centering\includegraphics[width=0.8\linewidth]{figures/intro/scs/sc.s-connectors/examples/scs_transf_inclusion_const.png}\\} & \parbox[c]{2.5cm}{\centering\supseteq} & \parbox[c]{2.5cm}{\centering\subseteq} & \multicolumn{2}{c|}{\parbox[c]{5cm}{}} \\
	\hline
	
	\parbox[c]{6.2cm}{\\\centering\includegraphics[width=0.8\linewidth]{figures/intro/scs/sc.s-connectors/examples/scs_transf_inclusion_meta.png}\\} & \parbox[c]{2.5cm}{\centering\textunderscore\supseteq} & \parbox[c]{2.5cm}{\centering\subseteq\textunderscore} & \multicolumn{2}{c|}{\parbox[c|]{5cm}{\centering}} \\
	\hline
	
	\parbox[c]{6.2cm}{\\\centering\includegraphics[width=0.8\linewidth]{figures/intro/scs/sc.s-connectors/examples/scs_transf_strict_inclusion_const.png}\\} & \parbox[c]{2.5cm}{\centering\supset} & \parbox[c]{2.5cm}{\centering\subset} & \multicolumn{2}{c|}{\parbox[c|]{5cm}{}} \\
	\hline
	
	\parbox[c]{6.2cm}{\\\centering\includegraphics[width=0.8\linewidth]{figures/intro/scs/sc.s-connectors/examples/scs_transf_strict_inclusion_meta.png}\\} & \parbox[c]{2.5cm}{\centering\textunderscore\supset} & \parbox[c]{2.5cm}{\centering\subset\textunderscore} & \multicolumn{2}{c|}{\parbox[c|]{5cm}{}} \\
	\hline
	
	\parbox[c]{6.2cm}{\\\centering\includegraphics[width=0.8\linewidth]{figures/intro/scs/sc.s-connectors/examples/scs_transf_value_order_const.png}\\} & \parbox[c]{2.5cm}{\centering\geq} & \parbox[c]{2.5cm}{\centering\leq} & \multicolumn{2}{c|}{\parbox[c|]{5cm}{}} \\
	\hline
	
	\parbox[c]{6.2cm}{\\\centering\includegraphics[width=0.8\linewidth]{figures/intro/scs/sc.s-connectors/examples/scs_transf_value_order_var.png}\\} & \parbox[c]{2.5cm}{\centering\textunderscore\geq} & \parbox[c]{2.5cm}{\centering\textunderscore\leq} & \multicolumn{2}{c|}{\parbox[c]{5cm}{}}  \\
	\hline
	
	\parbox[c]{6.2cm}{\\\centering\includegraphics[width=0.8\linewidth]{figures/intro/scs/sc.s-connectors/examples/scs_transf_value_strict_order_const.png}\\} & \parbox[c]{2.5cm}{\centering >} & \parbox[c]{2.5cm}{\centering <} & \multicolumn{2}{c|}{\parbox[c]{5cm}{}} \\
	\hline
	
	\parbox[c]{6.2cm}{\\\centering\includegraphics[width=0.8\linewidth]{figures/intro/scs/sc.s-connectors/examples/scs_transf_value_strict_order_var.png}\\} & \parbox[c]{2.5cm}{\centering\textunderscore>} & \parbox[c]{2.5cm}{\centering <\textunderscore} & \multicolumn{2}{c|}{\parbox[c]{5cm}{}} \\
	\hline
	
	\parbox[c]{6.2cm}{\\\centering\includegraphics[width=0.8\linewidth]{figures/intro/scs/sc.s-connectors/examples/scs_transf_external_idtf_const.png}\\} & \multicolumn{2}{c|}{\parbox[c]{5cm}{\centering :=}} & \multicolumn{2}{c|}{\parbox[c]{5cm}{}} \\
	\hline
	
	\parbox[c]{6.2cm}{\\\centering\includegraphics[width=0.8\linewidth]{figures/intro/scs/sc.s-connectors/examples/scs_transf_external_idtf_var.png}\\} & \multicolumn{2}{c|}{\parbox[c]{5cm}{\centering\textunderscore:=}} & \multicolumn{2}{c|}{\parbox[c]{5cm}{}} \\
	\hline
	
	\parbox[c]{6.2cm}{\\\centering\includegraphics[width=0.8\linewidth]{figures/intro/scs/sc.s-connectors/examples/scs_transf_synonymy_const.png}\\} & \multicolumn{2}{c|}{\parbox[c]{5cm}{\centering =}} & \multicolumn{2}{c|}{\parbox[c]{5cm}{}} \\
	\hline
	
	\parbox[c]{6.2cm}{\\\centering\includegraphics[width=0.8\linewidth]{figures/intro/scs/sc.s-connectors/examples/scs_transf_insertion_const.png}\\} & \parbox[c]{2.5cm}{\centering\supset=} & \parbox[c]{2.5cm}{\centering = \subset} &  \multicolumn{2}{c|}{\parbox[c]{5cm}{}}  \\
	\hline
\end{longtable}
}

\bigskip
\scnendstruct \scnendsegmentcomment{Описание sc.s-разделителей и sc.s-ограничителей}

\scnsegmentheader{Описание sc.s-предложений}
\scnstartsubstruct

\scnheader{sc.s-предложение}
\scnidtf{минимальный семантически целостный фрагмент sc.s-текста}
\scnidtf{минимальный sc.s-текст}
\scnsubset{sc.s-текст}
\scnsuperset{простое sc.s-предложение\\
    \scnaddlevel{1}
    \scnidtf{минимальное sc.s-предложение}
    \scnexplanation{\textit{sc.s-предложение}, (1) \uline{состоящее} или из двух \textit{sc-идентификаторов}, соединенных между собой \textit{\mbox{sc.s-коннектором}}, или из трех \textit{sc-идентификаторов}, разделенных \textit{sc.s-разделителями, изображающими связь инцидентности sc-элементов}, и (2) завершающееся \textit{двойной точкой с запятой}}
    \scnnote{Нетрудно заметить, что простые sc.s-предложения по сути аналогичны триплетам языка RDF (\mbox{RDF-триплетам}), за тем исключением, что \textit{простое sc.s-предложение} можно "развернуть"\ при помощи \textit{Операции конверсии sc.s-предложений*} не меняя при этом его смысл, а RDF-триплет нельзя. Это является одной из причин, по которой, в отличие от RDF-триплетов, в простых \mbox{sc.s-предложениях} \textit{\mbox{sc.s-коннекторы}} и \textit{\mbox{sc.s-разделители}, изображающие связь инцидентности \mbox{sc-элементов}} не могут быть опущены, поскольку они в том числе показывают направление изображаемой ими связи между sc-элементами.}
\scnrelfromlist{пример}{
    \scnfileitem{\textit{многоугольник} $\supset$ \textit{треугольник}}
    \scnaddlevel{1}
    \scnrelboth{семантическая эквивалентность}{\scnfilescg{figures/intro/scs/inclusion.png}}
    \scnaddlevel{-1};
    \scnfileitem{\textit{сторона*} $\ni$ (\textit{Четырехугк(ТчкА\char59ТчкВ\char59ТчкС\char59ТчкD)} $\Rightarrow$ \textit{Отр(ТчкВ\char59ТчкС)})\char59\char59}
    \scnaddlevel{1}
    \scnrelboth{семантическая эквивалентность}{\scnfilescg{figures/intro/scs/side.png}}
    \scnaddlevel{-1};
    \scnfileitem{\textit{Si} |- \textit{ai} >| \textit{ei}}
    \scnaddlevel{1}
    \scnrelboth{семантическая эквивалентность}{\scnfilescg{figures/intro/scs/inclusion_incident.png}}
    \scnaddlevel{-1}
    }
\scnaddlevel{-1}}
\scnnote{Признаком завершения любого \textit{sc.s-предложения}, т.е. последними его символами является \textit{двойная точка с запятой}, которую, следовательно, можно считать разделителем \textit{sc.s-предложений}.}
\scnrelfromlist{заданная операция}{Операция конверсии sc.s-предложения*\\
    \scnaddlevel{1}
    \scnsubset{синтаксическая трансформация*}
    \scnexplanation{Каждое \textit{sc.s-предложение} (в том числе, и \textit{простое sc.s-предложение}) можно преобразовать в семантически эквивалентное ему \textit{sc.s-предложение} путем конверсии ("разворота") цепочки компонентов \textit{sc.s-предложения}. Так, например, при конверсии ("развороте") простого \textit{\mbox{sc.s-предложения}} (1) первый его \textit{\mbox{sc-идентификатор}} (первый компонент этого \textit{\mbox{sc.s-предложения}}) становится третьим компонентом конвертированного\textit{ \mbox{sc.s-предложения}}, (2) второй его \textit{\mbox{sc-идентификатор}} (третий компонент исходного \textit{\mbox{sc.s-предложения}}) становится первым компонентом "конвертированного"\ \textit{\mbox{sc.s-предложения}} и (3) второй компонент исходного \textit{\mbox{sc.s-предложения}} (\textit{\mbox{sc.s-коннектор}} или \textit{\mbox{sc.s-разделитель}, изображающий связь инцидентности \mbox{sc-элементов}}, соединяющий указанные выше компоненты) остается вторым компонентом конвертированного \textit{\mbox{sc.s-предложения}}, но меняет направленность ("$\ni$"\ заменяется на "$\in$"\ и наоборот, "$\supset$"\ на "$\subset$"\ и наоборот, "$\Rightarrow$"\ на "$\Leftarrow$"\ и наоборот и т.д.)}
    \scnnote{Можно говорить не только о конверсии sc.s-предложения, но и о конверсии sc.s-коннектора, о конверсии sc.s-разделителя, изображающего связь инцидентности sc.s-элементов.}
    \scnrelfrom{sc.s-текст до трансформации}{\scnfilelong{\textit{треугольник $\ni$ Треуг(ТчкВ\char59ТчкС\char59ТчкD)}\char59\char59}}
    \scnrelfrom{sc.s-текст после трансформации}{\scnfilelong{\textit{Треуг(ТчкВ\char59ТчкС\char59ТчкD) $\in$ треугольник}\char59\char59}}
	\scnaddlevel{1}
    \scnrelboth{семантическая эквивалентность}{\scnfilescg{figures/intro/scs/conversion.png}}
    \scnaddlevel{-2}
;Операция присоединения sc.s-предложения*\\
    \scnaddlevel{1}
    \scnsubset{синтаксическая трансформация*}
    \scnidtf{Операция соединения двух sc.s-предложений при совпадении последнего компонента первого предложения с первым компонентом второго*}
    \scnexplanation{В результате выполнения данной операции:
    \begin{scnitemize}
        \item первый компонент второго sc.s-предложения удаляется\char59
        \item оставшаяся часть второго предложения окружается sc.s-ограничителем присоединенных предложений ("(*"\ и "*)"). Разделитель sc.s-предложений (";;") также попадает внутрь указанного ограничителя\char59
        \item полученная конструкция помещается между последним компонентом первого предложения и разделителем sc.s-предложений, которым заканчивалось первое предложение\char59
        \item второе предложение, таким образом, становится присоединенным sc.s-предложением.
    \end{scnitemize}
    Аналогичным образом к любому присоединенному sc.s-предложению могут "пристыковываться"\ другие присоединенные sc.s-предложения, в общем случае уровень такой вложенности не ограничен.
    }
    \scnaddlevel{-1}
;Операция слияния sc.s-предложений*\\
    \scnaddlevel{1}
    \scnsubset{синтаксическая трансформация*}
    \scnidtf{Операция присоединения простого sc.s-предложения к sc.s-предложению, у которого последний sc.s-коннектор совпадает с sc.s-коннектором простого sc.s-предложения, а предшествующий указанному sc.s-коннектору sc-идентификатор совпадает с первым sc-идентификатором простого sc.s-предложения*}
    \scnexplanation{В результате выполнения этой операции совпадающие sc-идентификаторы и sc.s-коннекторы соединяемых sc.s-предложений "склеиваются"\ , а последние sc-иден\-ти\-фи\-ка\-то\-ры соединяемых \textit{sc.s-предложений} становятся последними компонентами объединенного \textit{sc.s-предложения},
    разделенными \textit{точкой с запятой}. Аналогичным образом можно присоединять сколько угодно простых \textit{sc.s-предложений}.}
    \scnaddlevel{-1}
;Операция разложения sc.s-предложений на простые sc.s-предложения*\\
    \scnaddlevel{1}
    \scnsubset{синтаксическая трансформация*}
    \scnexplanation{Каждое \textit{sc.s-предложение} можно разложить на множество \textit{простых sc.s-предложений}, т.е. представить в виде последовательности \textit{простых sc.s-предложений}.}
    \scnaddlevel{-1}
;Операция разложения sc.s-предложений на простые sc.s-предложения с sc.s-разделителем, изображающим связь инцидентности sc-элементов*\\
    \scnaddlevel{1}
    \scnsubset{синтаксическая трансформация*}
    \scnexplanation{Каждое \textit{sc.s-предложение} (в том числе и \textit{простое sc.s-предложение} с \textit{sc.s-коннектором}) можно представить в виде семантически эквивалентной последовательности \textit{простых \mbox{sc.s-предложений}} с \textit{sc.s-разделителем, изображающим связь инцидентности \mbox{sc-элементов}}.}
    \scnnote{Данная операция осуществляет \uline{однозначное} (!) формирование множества \textit{простых \mbox{sc.s-предложений}} указанного вида.}
    \scnaddlevel{-1}
    }

\newpage
\scnheader{sc.s-предложение}
\scnnote{Операции, заданные на множестве \textit{sc.s-предложений} можно разделить на три группы:
    \begin{scnitemize}
        \item группа операций конверсии \textit{sc.s-предложений}, состоящая из одной операции;
        \item группа операций соединения \textit{sc.s-предложений};
        \item группа операций декомпозиции \textit{sc.s-предложений} и, в частности, операций разложения \textit{sc.s-предложений}.
    \end{scnitemize}
Очевидно, что операции соединения \textit{sc.s-предложений} и операции декомпозиции \textit{sc.s-предложений} являются обратными друг другу операциями.}
\scnaddlevel{-1}

\scnheader{Описание примеров выполнения операций, заданных на множестве sc.s-предложений}
\scnstartsubstruct

\bigskip

\scnfilelong{\textit{треугольник $\ni$ Треугк(ТчкВ\char59ТчкС\char59ТчкD)}\char59\char59}
\scnrelfrom{Операция конверсии sc.s-предложения}{\scnfilelong{\textit{Треугк(ТчкВ\char59ТчкС\char59ТчкD) $\in$ треугольник}\char59\char59}}
\scnaddlevel{1}
\scnrelboth{семантическая эквивалентность}{\scnfilescg{figures/intro/scs/conversion.png}}
\scnaddlevel{-1}

\bigskip\bigskip

{\scnfilelong{\textit{треугольник $\ni$ Треугк(ТчкВ\char59ТчкС\char59ТчкD)\char59\char59
\newline
Треугк(ТчкВ\char59ТчкС\char59ТчкD) $\Rightarrow$ сторона*:включение*: Отр(ТчкВ\char59ТчкC)\char59\char59}}}
\scnrelfrom{Операция присоединения sc.s-предложения}{\scnfilelong{\textit{треугольник $\ni$ Треугк(ТчкВ\char59ТчкС\char59ТчкD) (* $\Rightarrow$ сторона*:включение*:Отр(ТчкВ\char59ТчкС)\char59\char59*) }\char59\char59}}
\scnaddlevel{1}
\scnrelboth{семантическая эквивалентность}{\scnfilescg{figures/intro/scs/joining_sentences.png}}
\scnaddlevel{-1}

\bigskip\bigskip

\scnfilelong{\textit{сторона* $\ni$ (Треугк(ТчкВ\char59Тчк С\char59ТчкD) $\Rightarrow$ Отр(ТчкВ\char59ТчкС))\char59\char59
\newline
сторона* $\ni$ (Треугк(ТчкВ\char59Тчк С\char59ТчкD) $\Rightarrow$ Отр(ТчкC\char59ТчкD))\char59\char59}}
\scnrelfrom{Операция слияния sc.s-предложений}{\scnfilelong{\textit{сторона* $\ni$ ((Треугк(ТчкВ\char59ТчкС\char59ТчкD) $\Rightarrow$ Отр(ТчкВ\char59ТчкС))\char59(Треуг(ТчкВ\char59ТчкС\char59ТчкD) $\Rightarrow$ Отр(ТчкC\char59ТчкD)))\char59\char59}}}
\scnaddlevel{1}
\scnrelfrom{синтаксическая трансформация}{\scnfilelong{\textit{Треугк(ТчкВ\char59ТчкС\char59ТчкD)}$\Rightarrow$\textit{сторона*}: \textit{Отр(ТчкВ\char59ТчкС)}\char59\textit{Отр(ТчкС\char59ТчкD)}\char59\char59}}
\scnrelboth{семантическая эквивалентность}{\scnfilescg{figures/intro/scs/joining_sentence.png}}
\scnaddlevel{-1}

\bigskip\bigskip

\newpage
{\scnfilelong{\textit{Треугк(ТчкВ\char59ТчкС\char59ТчкD) $\Rightarrow$ сторона*:включение*:Отр(ТчкВ\char59ТчкС)\char59\char59}}}
\scnrelfrom{Операция разложения sc.s-предложений на простые sc.s-предложения}{\scnfilelong{\textit{сторона* $\ni$ (Треугк(ТчкВ\char59ТчкС\char59ТчкD) $\Rightarrow$ Отр(ТчкВ\char59ТчкС))\char59\char59
\newline включение* $\ni$ (Треугк(ТчкВ\char59ТчкС\char59ТчкD) $\Rightarrow$ Отр(ТчкВ\char59ТчкС))\char59\char59}}}
\scnaddlevel{1}
\scnrelboth{семантическая эквивалентность}{\scnfilescg{figures/intro/scs/dividing_sentences.png}}
\scnaddlevel{-1}

\bigskip\bigskip

{\scnfilelong{\textit{треугольник $\ni$ Треугк(ТчкВ\char59ТчкC\char59ТчкD)}}}
\scnrelfrom{Операция разложения sc.s-предложений на простые sc.s-предложения с sc.s-разделителем, изображающим связь инцидентности sc-элементов}{\scnfilelong{\textit{треугольник |- ai >| Треугк(ТчкВ\char59ТчкС\char59ТчкD)\char59\char59
\newline
константный постоянный sc-узел, обозначающий класс $\ni$ треугольник\char59\char59
\newline
константная постоянная позитивная sc-дуга принадлежности $\ni$ ai\char59\char59
\newline
константный постоянный sc-узел общего вида $\ni$ Треугк(ТчкВ\char59ТчкC\char59ТчкD)\char59\char59}}}
\scnaddlevel{1}
\scnrelboth{семантическая эквивалентность}{\scnfilescg{figures/intro/scs/dividing_sentences_incident.png}}
\scnaddlevel{-1}

\scnendstruct

\scnheader{присоединенное sc.s-предложение}
\scnidtf{встроенное sc.s-предложение}
\scnexplanation{Присоединенные sc.s-предложения используются для того, чтобы продолжить спецификацию какого-либо sc-элемента, sc-идентификатор которого является последним компонентом в рамках какого-либо sc.s-предложения, не начиная при этом нового sc.s-предложения и, таким образом, не дублируя указанный \mbox{sc-идентификатор}. Внутрь присоединенных sc.s-предложений также могут встраиваться другие присоединенные sc.s-предложения, в общем случае уровень вложенности таких предложений не ограничен. Таким образом присоединенные sc.s-предложения описывают "ветвление"\ sc.s-предложений, при этом точками такого "ветвления"\ выступают sc-идентификаторы, входящие в состав этих sc.s-предложений.

Благодаря введению присоединенных sc.s-предложений появляется возможность любой sc-текст изобразить в виде одного sc.s-предложения, содержащего необходимое количество присоединенных sc.s-предложений. Таким образом, SCs-код по выразительной мощности становится эквивалентным SCn-коду.}

\scnheader{sc.s-предложение}
\scntext{денотационная семантика}{С семантической точки зрения \textit{sc.s-предложение} представляет собой описание некоторого \uline{маршрута} в соответствующем sc-тексте, который является графовой структурой специального вида и структура которого описывается (изображается) с помощью \textit{sc.s-предложений}. Указанный маршрут "проводится"\ по sc-коннекторам и по связям инцидентности sc-элементов, если маршрут проходит через инцидентные sc-коннекторы. В описании указанного маршрута могут дополнительно указываться множества (чаще всего отношения), которым принадлежат sc-коннекторы, входящие в описываемый маршрут. Кроме того, указанный маршрут в начале и/или в конце может иметь разветвления, когда какой-либо sc-элемент \uline{одинаково} инцидентен нескольким \uline{однотипным} sc-коннекторам, соединяющим указанный sc-элемент с некоторыми другими sc-элементами.

Таким образом каждое указанное разветвление состоит из неограниченного числа ветвей, каждая из которых состоит из одного sc-коннектора и одного связываемого им sc-элемента.}

\scnheader{компонент sc.s-предложения*}
\scnexplanation{Каждое \textit{sc.s-предложение} представляет собой последовательность (1) \textit{sc-идентификаторов}, \mbox{(2) \textit{sc.s-коннекторов}} или \textit{sc.s-разделителей}, изображающих связь инцидентности \textit{sc-элементов}, (3) \textit{точек с запятыми}, (4) \textit{ограничителей присоединенных sc.s-предложений}, завершаемая \textit{двойной точкой с запятой}. При этом непосредственно соседствовать друг с другом не могут ни \textit{\mbox{sc-идентификаторы}}, ни \textit{\mbox{sc.s-коннекторы}}, ни, очевидно, \textit{точки с запятыми} и \textit{ограничители присоединенных sc.s-предложений}.\\
Между \textit{sc-идентификаторами} в рамках \textit{sc.s-предложения} может находиться либо \textit{точка с запятой}, либо \textit{sc.s-коннектор}, либо \textit{sc.s-разделитель}, изображающий связь инцидентности \textit{sc-элементов}. Слева и справа от \textit{sc.s-коннектора} и от \textit{sc.s-разделителя}, изображающего связь инцидентности \textit{sc-элементов}, должны находиться \textit{sc-идентификаторы}.

Указанные \textit{sc-идентификаторы}, \textit{sc.s-коннекторы} и \textit{sc.s-разделители}, изображающие связь инцидентности \textit{sc-элементов}, считаются компонентами соответствующего \textit{sc.s-предложения}. Понятие "быть компонентом sc.s-предложения"\ является относительным понятием (отношением), т.к. в состав некоторых компонентов \textit{sc.s-предложения} (в состав \textit{sc-идентификаторов}, являющихся \textit{sc.s-выражениями}, ограничиваемыми фигурными или квадратными скобками) могут входить других \textit{sc.s-предложения}, состоящие из своих компонентов.}
\scnrelfrom{первый домен}{sc.s-предложение}
\scnrelfrom{второй домен}{{\normalfont (} sc-идентификатор $\cup$ sc.s-разделитель $\cup$ sc.s-ограничитель {\normalfont )}}

\scnheader{sc.s-модификатор*}
\scnsubset{компонент sc.s-предложения*}
\scnexplanation{Это дополнительный вид компонентов \textit{sc.s-предложений}. Каждый \textit{sc.s-модификатор}, являющийся компонентом некоторого \textit{sc.s-предложения}, представляет собой \textit{sc-идентификатор}, обозначающий множество (чаще всего, отношение), которому принадлежит sc-коннектор, изображенный \textit{sc.s-коннектором}, который предшествует указанному \textit{sc-идентификатору}. Признаком \textit{sc.s-модификатора} является \textit{двоеточие} (или \textit{двойное двоеточие}), которое ставится после \textit{sc.s-модификатора} и отделяет его либо от следующего за ним другого \textit{sc.s-модификатора} для этого же \textit{sc.s-коннектора}, либо от следующего за ним \textit{sc-идентификатора}, соответствующего sc-элементу, который инцидентен sc-коннектору, изображенному \textit{sc.s-коннектором}, находящимся левее рассматриваемого \textit{sc-идентификатора} после одного или нескольких \textit{sc.s-модификаторов}. Обычное ("одинарное") \textit{двоеточие} обозначает, что sc-элемент, изображенный соответствующим \mbox{sc.s-модификатором}, связан с sc-коннектором, изображенным левее этого \mbox{sc.s-модификатора}, \textit{базовой \mbox{sc-дугой}} (\textit{константной постоянной позитивной \mbox{sc-дугой} принадлежности}), \textit{двойное двоеточие} обозначает, что указанные элементы связаны \textit{переменной постоянной позитивной \mbox{sc-дугой} принадлежности}.}
\scnrelfromlist{пример}{
    \scnfileitem{\textit{Четырехугк(ТчкА;ТчкВ;ТчкС;ТчкD)} $\Rightarrow$ \textit{сторона*} : \textit{включение*} : \textit{Отр(ТчкВ;ТчкС)};; }\\
    \scnaddlevel{1}
    \scnrelboth{семантическая эквивалентность}{\scnfilescg{figures/intro/scs/modifier.png}}
    \scnaddlevel{-1}
    \newpage;
    \scnfileitem{\textit{Треугк(ТчкА;ТчкВ;ТчкС)} $\_\Rightarrow$ \textit{сторона*} :: \textit{\_s};; }\\
    \scnaddlevel{1}
    \scnrelboth{семантическая эквивалентность}{\scnfilescg{figures/intro/scs/modifier_var.png}}
    \scnaddlevel{-1}
    }

\scnheader{sc.s-текст}
\scnidtf{конкатенация \textit{sc.s-предложений}}
\scnidtf{последовательность \textit{sc.s-предложений}, разделяемых \textit{двойными точками с запятой}}
\scnsuperset{максимальный исходный sc.s-текст}
    \scnaddlevel{1}
    \scnidtf{конкатенция \textit{sc.s-предложений}, слева и справа от которой отсутствуют какие-либо символы SCs-кода}
    \scnaddlevel{-1}
\scnsuperset{максимальный sc.s-текст, включенный в структуру}
    \scnaddlevel{1}
    \scnidtf{конкатенция всех \textit{sc.s-предложений}, входящих в состав \textit{sc.s-выражения структуры}}
    \scnsuperset{sc.s-текст, включенный в структуру}
        \scnaddlevel{1}
        \scnidtf{часть цепочки \textit{sc.s-предложений}, входящих в состав максимального sc.s-текста, включенного в структуру}
        \scnsuperset{sc.s-предложение, включенное в структуру}
    \scnaddlevel{-2}
\scnnote{\textit{sc.s-предложение} является минимальным sc.s-текстом.}
\scntext{свойство}{Смысл sc.s-текста (а также \textit{sc.s-текста, включенного в структуру} не зависит от порядка \textit{\mbox{sc.s-предложений}} в этих sc-текстах. Т.е. перестановка \textit{\mbox{sc.s-предложений}} в рамках таких \mbox{sc.s-текстов} смысла этих \mbox{sc.s-текстов} не меняет (т.е. приводит к семантически эквивалентным \mbox{sc.s-текстам}), но сильно влияет на трудоемкость человеческого восприятия (на "читабельность") этих текстов.}
\scnrelfrom{пример}{\scnfilelong{
		\textit{материальный объект} $\ni$ \textit{Земля} (* => \textit{вращаться вокруг}*: \textit{спутник}*: \textit{Луна};;*);;\\
		\textit{материальный объект} $\ni$ \textit{Луна}(* => \textit{основной идентификатор}*: [Moon] (* <- \textit{Английский язык};; *); [Луна] (* <- \textit{Русский язык};; *);; *);;\\
		\textit{материальный объект} $\ni$ \textit{Солнце} (* => \textit{вращаться вокруг}*: \textit{Земля}; \textit{Марс};; *);;\\
		\textit{материальный объект} $\ni$ \textit{Марс};;}
}    
\scnaddlevel{1}
\scnrelfrom{семантическая эквивалентность}{ \scnfilescg{figures/intro/scs/scs_text_example.png}}
\scnaddlevel{-1}
    
\bigskip
\scnendstruct \scnendsegmentcomment{Завершили Описание sc.s-предложений}

\scnsegmentheader{Описание Ядра SCs-кода и различных направлений его расширения}
\scnstartsubstruct

\scnheader{Ядро SCs-кода}
\scnidtf{Подъязык SCs-кода, который использует минимальный набор синтаксических средств, но при этом имеет семантическую мощность, эквивалентную мощности SCs-кода в целом}
\scntext{принципы, лежащие в основе}{В Ядре SCs-кода:
\begin{scnitemize}
    \item используются только \textit{простые sc-идентификаторы}, в том числе \textit{sc-идентификаторы внешних файлов ostis-систем} (sc-выражения не используются);
    \item используются только \textit{sc.s-разделители, изображающие связь инцидентности sc-элементов}, а также sc.s-коннектор, изображающий константную  постоянную позитивную пару принадлежности ("$\in$" и "$\ni$" в Расширенном алфавите и "{}->{}"\ и "{}<-{}"\ в Базовом алфавите). Другие \textit{sc.s-коннекторы} не используются;
    \item не используются \textit{sc.s-модификаторы} и, соответственно, двоеточия, являющиеся признаком завершения \textit{sc.s-модификаторов};
    \item используются только \textit{простые sc.s-предложения}, которые, как следует из вышеуказанных свойств Ядра SCs-кода, либо состоят из двух \textit{простых sc-идентификаторов}, соединяемых sc.s-коннектором, изображающим константную  постоянную позитивную пару принадлежности, либо трех \textit{простых sc-идентификаторов}, разделенных \textit{sc.s-разделителями, изображающими связь инцидентности sc-элементов}.
\end{scnitemize}

Из перечисленных свойств Ядра SCs-кода следует, что для представления (изображения) любого \mbox{sc-текста} средствами Ядра SCs-кода необходимо для \uline{всех} (!) sc-элементов этого \mbox{sc-текста} (кроме константных постоянных позитивных пар принадлежности) построить соответствующие им простые \textit{sc-идентификаторы}, т.е. необходимо проименовать все указанные sc-элементы. В свою очередь, тип каждого используемого \mbox{sc-элемента} (кроме константных постоянных позитивных пар принадлежности) задается явно путем указания принадлежности этих элементов соответствующим классам sc-элементов, в том числе классам, входящим в Ядро SC-кода.

Как видно из приведенного описания, Ядро SCs-кода соответствует Ядру SCg-кода, за исключением того, что в Ядре SCg-кода нет необходимости именовать все изображаемые sc-элементы, а также в Ядре SCg-кода присутствуют графические изображения для sc-элементов, принадлежащих соответствующим классам Ядра SC-кода и эту принадлежность нет необходимости указывать явно.
}
\scnnote{Очевидно, что широко практически применять Ядро SCs-кода для записи больших фрагментов баз знаний неудобно и неэффективно. Тем не менее, с практической точки зрения Ядро SCs-кода может использоваться, например, для обмена информацией со сторонними средствами представления графовых конструкций, рассчитанными на представление информации в виде триплетов (например, RDF-хранилищ).
Для обеспечения возможности более широкого практического использования необходимы синтаксические расширения Ядра SCs-кода в целях:
\begin{scnitemize}
    \item минимизации числа идентифицируемых (именуемых) sc-элементов путем использования \textit{sc-выражений} и ликвидации необходимости идентифицировать (именовать) \uline{все} (!) sc-элементы;
    \item сокращения текста путем минимизации числа повторений одного и того же \textit{sc-идентификатора} путем соединения \textit{sc.s-предложений};
    \item повышение уровня наглядности, "читабельности"\ sc.s-текстов.
\end{scnitemize}}
\scnhaselementrole{пример}{\scnfilelong{\textit{треугольник |- ai >| Треугк(ТчкВ\char59ТчкС\char59ТчкD)\char59\char59
\newline
Треугк(ТчкВ\char59ТчкС\char59ТчкD) |- bi >| Отр(ТчкВ\char59ТчкС)\char59\char59
\newline
сторона* |- сi >| bi\char59\char59
\newline
константный постоянный sc-узел, обозначающий класс $\ni$ треугольник\char59\char59
\newline
константный постоянный sc-узел, обозначающий отношение $\ni$ сторона*\char59\char59
\newline
константная постоянная позитивная sc-дуга принадлежности $\ni$ ai\char59\char59
\newline
константная постоянная sc-дуга $\ni$ bi\char59\char59
\newline
константная постоянная позитивная sc-дуга принадлежности $\ni$ ci\char59\char59
\newline
константный постоянный sc-узел общего вида $\ni$ Отр(ТчкВ\char59ТчкС)\char59\char59
\newline
константный постоянный sc-узел общего вида $\ni$ Треугк(ТчкВ\char59ТчкC\char59ТчкD)\char59\char59}}}
\scnaddlevel{1}
\bigskip
\scnrelboth{семантическая эквивалентность}{\scnfilescg{figures/intro/scs/kernel_incident.png}}
\scnaddlevel{-1}

\scnheader{Первое направление расширения Ядра SCs-кода}
\scnidtf{Первое направление расширения Ядра SCs-кода \uline{и всех иных его расширений}}
\scntext{принципы}{По сравнению с \textit{Ядром SCs-кода} в \textit{Первом направлении расширения Ядра SCs-кода} вместо \textit{sc-идентификато|-ров}, являющихся идентификаторами (именами), которые взаимно однозначно соответствуют синонимичным им (представляемым ими) sc-коннекторам, вводятся \textit{sc.s-коннекторы}, каждый из которых соответствует не одному конкретному sc-коннектору, а некоторому классу однотипных sc-коннекторов. Очевидно, что это ликвидирует необходимость \uline{каждому} sc-коннектору приписывать уникальный \textit{sc-идентификатор}. Кроме того, \textit{Алфавит sc.s-коннекторов} включает в себя элементы этого Алфавита (классы \uline{синтаксически} эквивалентных \textit{sc.s-коннекторов}), которые соответствуют \uline{всем} (!) элементам Алфавита sc-коннекторов, но при этом дополнительно включают в себя и другие элементы Алфавита \textit{sc.s-коннекторов}, которые соответствуют часто используемым \uline{семантически} явно выделяемым классам sc-коннекторов. К таким дополнительно вводимым классам \textit{sc.s-коннекторов} относятся \textit{константные sc.s-коннекторы} включения множеств ("$\supset$"\ или "$\subset$"), \textit{переменные sc.s-коннекторы} включения множеств ("$\_\supset$"\ или "$\subset\_$"), \textit{sc.s-коннектор} синонимии ("$=$"), \textit{sc.s-коннектор} погружения ("$=\subset$"\ или "$\supset=$") и др.

Заметим, что указанное расширение Алфавита \textit{sc.s-коннекторов} аналогично расширенному Алфавиту \textit{sc.g-коннекторов} в SCg-коде и ликвидирует необходимость (как и в SCs-коде) явно специфицировать (средствами SCs-кода) синтаксически выделяемые классы \textit{sc.s-коннекторов}.}

\scnheader{Второе направление расширения Ядра SCs-кода}
\scntext{принципы}{Во Втором направлении расширения Ядра SCs-кода вводятся модификаторы \textit{sc.s-коннекторов} (\textit{\mbox{sc.s-модификаторы}}), которые позволяют достаточно компактно дополнительно специфицировать \mbox{sc-коннекторы}, изображаемые (представляемые) соответствующими \textit{sc.s-коннекторами}. Речь идет о такой часто востребованной форме спецификации sc-коннекторов, как указание множества (возможно, нескольких множеств), которому принадлежит специфицируемый  sc-коннектор (чаще всего, таким множеством является \textit{бинарное отношение} (в частности, \textit{ролевое отношение}) или \textit{квазибинарное отношение}).}

\scnheader{sc.s-модификатор*}
\scniselement{отношение}
    \scnaddlevel{1}
    \scnidtf{относительное понятие}
    \scnaddlevel{-1}
\scnidtf{модификатор sc.s-коннектора*}
\scnexplanation{\textit{sc-идентификатор}, который (1) находится либо между \textit{sc.s-коннектором} и \textit{двоеточием}, либо между \textit{двоеточиями} и (2) обозначает множество (чаще всего, отношение), которому принадлежит sc-коннектор, изображаемый ближайшим предшествующим \textit{sc.s-коннектором}. Два подряд идущих двоеточия ("::") обозначают, что указанное множество связано с указанным sc-коннектором \textit{\uline{переменной} позитивной постоянной sc-дугой принадлежности}.

Очевидно, что, если не использовать \textit{sc.s-модификаторы}, указанного вида спецификация sc-коннекторов средствами SCs-кода будет выглядеть значительно более громоздкой.}

\scnheader{Третье направление расширения Ядра SCs-кода}
\scntext{принципы}{В \textit{Третьем направлении расширения Ядра SCs-кода} осуществляется переход от использования только \textit{простых sc-идентификаторов} к использованию как \textit{простых sc-идентификаторов}, так и \textit{sc-выражений}, а также к использованию \textit{sc.s-представлений некоторых неидентифицируемых sc-узлов}. Это существенно сокращает число придумываемых \textit{простых sc-идентификаторов}, т.к. каждое \textit{sc-выражение} в конечном счете — это комбинация \textit{простых sc-идентификаторов}, построенная по правилам, которые достаточно легко семантически интерпретируются. Если проводить аналогию с SCg-кодом, то очевидно, что \textit{\mbox{sc-выражение}}, ограничиваемое фигурными скобками, есть не что иное, как информационная конструкция, ограничиваемая \textit{sc.g-контуром}, а \textit{sc-выражение}, ограничиваемое квадратными скобками есть не что иное, как информационная конструкция, ограничиваемая \textit{sc.g-рамкой}. Отличие здесь заключается в том, что круглыми и квадратными скобками можно ограничивать только линейные информационные конструкции (цепочки символов).}

\scnheader{sc.s-представление неидентифицируемого sc-узла}
\scnidtf{изображение (представление) неидентифицируемого (неименуемого) sc-узла в sc.s-тексте}
\scnidtf{sc.s-обозначение неименуемой сущности, не являющейся парой, обозначаемой sc-коннектором}
\scnidtf{sc.s-представление sc-узла, не являющееся sc-идентификатором (именем этого sc-узла)}
\scnreltoset{разбиение}{sc.s-обозначение неименуемой структуры\\
    \scnaddlevel{1}
    \scnidtf{конкатенация левой фигурной скобки и правой фигурной скобки}
    \scnaddlevel{-1}
;sc.s-обозначение неименуемой неориентированной связки\\
    \scnaddlevel{1}
    \scnidtf{конкатенация левой фигурной скобки, дефиса и правой фигурной скобки}
    \scnaddlevel{-1}
;sc.s-обозначение неименуемого кортежа\\
    \scnaddlevel{1}
    \scnidtf{конкатенация левой угловой скобки, дефиса и правой угловой скобки}
    \scnaddlevel{-1}
;sc.s-обозначение неименуемого файла-экземпляра\\
    \scnaddlevel{1}
    \scnidtf{конкатенация левой квадратной скобки и правой квадратной скобки}
    \scnaddlevel{-1}
;sc.s-обозначение неименуемого файла-класса\\
    \scnaddlevel{1}
    \scnidtf{конкатенация восклицательного знака, левой квадратной скобки и правой квадратной скобки}
    \scnaddlevel{-1}
;sc.s-обозначение неименуемой терминальной сущности\\
    \scnaddlevel{1}
    \scnidtf{конкатенация левой круглой скобки, буквы "о"\ и правой круглой скобки}
    \scnaddlevel{-1}}
\scntext{примечание}{Если одно и то же обозначение неименуемой сущности встречается в \uline{разных} \textit{sc.s-предложениях}, то считается, что это обозначения \uline{разных} сущностей, т.е. изображения \uline{разных} sc-узлов.}

\scnheader{Четвертое направление расширения Ядра SCs-кода}
\scntext{принципы}{В \textit{Четвертом направлении расширения Ядра SCs-кода} осуществляется переход от использования только \textit{простых sc.s-предложений} к использованию также \textit{sc.s-предложений}, построенных с помощью \textit{\mbox{Операции} присоединения sc.s-предложения*}. В результате этого, благодаря "склеиванию"\ одинаковых \textit{\mbox{sc-идентификаторов}}, а также "склеиванию"\ синтаксически эквивалентных \textit{\mbox{sc.s-коннекторов}} с одинаковыми \textit{\mbox{sc.s-модификаторами}} (несмотря на то, что эти "склеиваемые"\ \textit{sc.s-коннекторы} соответствуют \uline{разным} \mbox{sc-коннекторам}), существенно сокращается число копий используемых \textit{\mbox{sc-идентификаторов}} и \textit{\mbox{sc.s-коннекторов}} с их \textit{\mbox{sc.s-модификаторами}}.}

\newpage
\scnheader{Пятое направление расширения Ядра SCs-кода}
\scntext{принципы}{В \textit{Пятом направлении расширения Ядра SCs-кода} разрешается использование \textit{присоединенных \mbox{sc.s-предложений}}. В результате этого \textit{sc.s-тексты} становятся более компактными и удобными для восприятия за счет снижения числа дублируемых \textit{sc-идентификаторов} и более широких возможностей их структуризации.}

\scnheader{следует отличать*}
\scnhaselementset{sc.s-представление неидентифицируемого sc-узла;sc.s-коннектор\\
    \scnaddlevel{1}
    \scnidtf{sc.s-представление неидентифицируемого sc-коннектора}
    \scnaddlevel{-1}}\bigskip
\scnhaselementset{sc-коннектор;sc.s-коннектор}\bigskip
\scnhaselementset{sc.s-коннектор;sc.s-модификатор*\\
    \scnaddlevel{1}
    \scnidtf{модификатор sc.s-коннектора*}
    \scniselement{отношение}
    \scnaddlevel{-1}}\bigskip
\scnhaselementset{sc.s-коннектор;Правила построения sc.s-коннекторов}\bigskip
\scnhaselementset{sc.s-предложение;Правила построения sc.s-предложений}\bigskip
\scnhaselementset{sc.s-коннектор;sc.g-коннектор}\bigskip
\scnhaselementset{sc.s-текст;sc.g-текст}

\scnheader{Примеры sc.s-текстов, трансформируемых по различным направлениям расширений SCs-кода}
\begin{scneqtoset}
    \scnfileitem{\textit{включение* $\ni$ pair1;;\\включение* $\ni$ pair2;;\\включение* $\ni$ pair3;;\\включение* $\ni$ pair4;;\\включение* $\ni$ pair5;;\\сторона* $\ni$ pair1;;\\сторона* $\ni$ pair2;;\\сторона* $\ni$ pair3;;\\сторона* $\ni$ pair4;;\\сторона* $\ni$ pair5;;\\Четырехугк(ТчкА;ТчкВ;ТчкС;ТчкD) |- pair1 >| Отр(ТчкВ;ТчкС);;\\Четырехугк(ТчкА;ТчкВ;ТчкС;ТчкD) |- pair2 >| Отр(ТчкC;ТчкD);;\\Треугк(ТчкВ;ТчкС;ТчкD) |- pair3 >| Отр(ТчкВ;ТчкС);;\\Треугк(ТчкВ;ТчкС;ТчкD) |- pair4 >| Отр(ТчкC;ТчкD);;\\Треугк(ТчкВ;ТчкС;ТчкD) |- pair5 >| Отр(ТчкB;ТчкD);;\\четырехугольник $\ni$ Четырехугк(ТчкА;ТчкВ;ТчкС;ТчкD);;\\треугольник $\ni$ Треугк(ТчкВ;ТчкС;ТчкD);;\\link1 |- pair6 >| Треугк(ТчкВ;ТчкС;ТчкD);;\\декомпозиция фигуры* $\ni$ pair6;;\\link1 $\ni$ Отр(ТчкВ;ТчкС);;\\link1 $\ni$ Отр(ТчкC;ТчкD);;\\link1 $\ni$ Отр(ТчкВ;ТчкD);;}}
    \begin{scnindent}
        \scnrelboth{семантическая эквивалентность}{\scnfileimage[20em]{Contents/part_kb/images/scs/scs_transf_example.png}}
        \scnrelfrom{синтаксическая трансформация}{\scnfilelong{\textit{сторона* $\ni$ (Четырехугк(ТчкА;ТчкВ;ТчкС;ТчкD) $\Rightarrow$ Отр(ТчкВ;ТчкС));;\\сторона* $\ni$ (Четырехугк(ТчкА;ТчкВ;ТчкС;ТчкD) $\supseteq$ Отр(ТчкC;ТчкD));;\\сторона* $\ni$ (Треугк(ТчкВ;ТчкС;ТчкD) $\supseteq$ Отр(ТчкВ;ТчкС));;\\сторона* $\ni$ (Треугк(ТчкВ;ТчкС;ТчкD) $\supseteq$ Отр(ТчкC;ТчкD));;\\сторона* $\ni$ (Треугк(ТчкВ;ТчкС;ТчкD) $\supseteq$ Отр(ТчкB;ТчкD));;\\Четырехугк(ТчкА;ТчкВ;ТчкС;ТчкD) $\supseteq$ Отр(ТчкВ;ТчкС);;\\Четырехугк(ТчкА;ТчкВ;ТчкС;ТчкD) $\supseteq$ Отр(ТчкC;ТчкD);;\\Треугк(ТчкВ;ТчкС;ТчкD) $\supseteq$ Отр(ТчкВ;ТчкС);;\\Треугк(ТчкВ;ТчкС;ТчкD) $\supseteq$ Отр(ТчкC;ТчкD);;\\Треугк(ТчкВ;ТчкС;ТчкD) $\supseteq$ Отр(ТчкB;ТчкD);;\\четырехугольник $\ni$ Четырехугк(ТчкА;ТчкВ;ТчкС;ТчкD);;\\треугольник $\ni$ Треугк(ТчкВ;ТчкС;ТчкD);;\\декомпозиция фигуры* $\ni$ (link1 $\Rightarrow$ Треугк(ТчкВ;ТчкС;ТчкD));;\\link1 $\ni$ Отр(ТчкВ;ТчкС);;\\link1 $\ni$ Отр(ТчкC;ТчкD);;\\link1 $\ni$ Отр(ТчкВ;ТчкD);;}}}
        \begin{scnindent}
            \scnrelfrom{синтаксическая трансформация}{\scnfilelong{\textit{Четырехугк(ТчкА;ТчкВ;ТчкС;ТчкD) $\supseteq$ сторона*: Отр(ТчкВ;ТчкС);;\\Четырехугк(ТчкА;ТчкВ;ТчкС;ТчкD) $\supseteq$ сторона*: Отр(ТчкC;ТчкD);;\\Треугк(ТчкВ;ТчкС;ТчкD) $\supseteq$ сторона*: Отр(ТчкВ;ТчкС);;\\Треугк(ТчкВ;ТчкС;ТчкD) $\supseteq$ сторона*: Отр(ТчкC;ТчкD);;\\Треугк(ТчкВ;ТчкС;ТчкD) $\supseteq$ сторона*: Отр(ТчкB;ТчкD);;\\четырехугольник $\ni$ Четырехугк(ТчкА;ТчкВ;ТчкС;ТчкD);;\\треугольник $\ni$ Треугк(ТчкВ;ТчкС;ТчкD);;\\link1 $\Rightarrow$декомпозиция фигуры*: Треугк(ТчкВ;ТчкС;ТчкD);;\\link1 $\ni$ Отр(ТчкВ;ТчкС);;\\link1 $\ni$ Отр(ТчкC;ТчкD);;\\link1 $\ni$ Отр(ТчкВ;ТчкD);;}}}
            \begin{scnindent}
                \scnrelfrom{синтаксическая трансформация}{\scnfilelong{\textit{Четырехугк(ТчкА;ТчкВ;ТчкС;ТчкD) $\supseteq$ сторона*: Отр(ТчкВ;ТчкС); Отр(ТчкC;ТчкD);;\\Треугк(ТчкВ;ТчкС;ТчкD) $\supseteq$ сторона*: Отр(ТчкВ;ТчкС); Отр(ТчкC;ТчкD); Отр(ТчкB;ТчкD);;\\четырехугольник $\ni$ Четырехугк(ТчкА;ТчкВ;ТчкС;ТчкD);;\\треугольник $\ni$ Треугк(ТчкВ;ТчкС;ТчкD);;\\link1 $\Rightarrow$декомпозиция фигуры*: Треугк(ТчкВ;ТчкС;ТчкD);;\\link1 $\ni$ Отр(ТчкВ;ТчкС); Отр(ТчкC;ТчкD); Отр(ТчкВ;ТчкD);;}}}
                \begin{scnindent}
                    \scnrelfrom{синтаксическая трансформация}{\scnfilelong{\textit{четырехугольник $\ni$ Четырехугк(ТчкА;ТчкВ;ТчкС;ТчкD)(* $\supseteq$ сторона*: Отр(ТчкВ;ТчкС); Отр(ТчкC;ТчкD);; *);;\\треугольник $\ni$ Треугк(ТчкВ;ТчкС;ТчкD)(* $\supseteq$ сторона*: Отр(ТчкВ;ТчкС); Отр(ТчкC;ТчкD); Отр(ТчкB;ТчкD);; *);;\\Треугк(ТчкВ;ТчкС;ТчкD) $\Leftarrow$декомпозиция фигуры*: link1(* $\ni$ Отр(ТчкВ;ТчкС); Отр(ТчкC;ТчкD); Отр(ТчкВ;ТчкD);; *);;}}}
                \end{scnindent}
            \end{scnindent}
        \end{scnindent}
    \end{scnindent}
\end{scneqtoset}

\bigskip
\scnendstruct \scnendsegmentcomment{Описание Ядра SCs-кода и различных направлений его расширения}

\bigskip
\scnendstruct \scnendcurrentsectioncomment

\end{SCn}

\scsubsubsection[\scnmonographychapter{Глава 2.2. Семейство внешних языков интеллектуальных компьютерных систем нового поколения, близких языку внутреннего смыслового представления знаний (SCg, SCs, SCn)}]{Предметная область и онтология синтаксиса языка внешнего линейного представления информационных конструкций внутреннего языка ostis-систем}
\label{intro_scs_syntax}

\scsubsubsection[\scnmonographychapter{Глава 2.2. Семейство внешних языков интеллектуальных компьютерных систем нового поколения, близких языку внутреннего смыслового представления знаний (SCg, SCs, SCn)}]{Предметная область и онтология денотационной семантики языка внешнего линейного представления информационных конструкций внутреннего языка ostis-систем}
\label{intro_scs_semantic}

\scsubsubsection[\scnmonographychapter{Глава 2.2. Семейство внешних языков интеллектуальных компьютерных систем нового поколения, близких языку внутреннего смыслового представления знаний (SCg, SCs, SCn)}]{Предметная область и онтология иерархического семейства подъязыков, семантически эквивалентных языку внешнего линейного представления информационных конструкций внутреннего языка ostis-систем}
\label{intro_scs_sublang}
\scsubsection[\scnmonographychapter{Глава 2.2. Семейство внешних языков интеллектуальных компьютерных систем нового поколения, близких языку внутреннего смыслового представления знаний (SCg, SCs, SCn)}]{Предметная область и онтология языка внешнего форматированного представления информационных конструкций внутреннего языка ostis-систем}
\label{intro_scn}

\begin{SCn}

\bigskip

\scnsectionheader{\currentname}

\scnstartsubstruct

\scnheader{SCn-код}
\scnidtf{Язык структурированного представления знаний \textit{ostis-систем}}
\scnexplanation{\textit{SCn-код} является языком структурированного внешнего представления текстов \textit{SC-кода} и представляет собой синтаксическое расширение \textit{SCs-кода}, направленное на повышение наглядности и компактности текстов \textit{SCs-кода}. 
	
SCn-код позволяет перейти от линейных текстов \uline{SCs-кода} к форматированным и фактически двухмерным текстам, в которых появляется декомпозиция исходного линейного текста \uline{SCs-кода} на \uline{строчки}, размещенные "по вертикали". При этом начало всех \uline{строчек} текста фиксировано и определяется известным и ограниченным набором правил, что дает возможность использовать это при форматировании \uline{sc.n-текста} (текста, принадлежащего SCn-коду.}
\scniselement{язык двухмерных текстов}
\scnaddlevel{1}
\scnidtf{язык, каждый \textit{текст} которого задается (1) множеством входящих в него \textit{символов}, (2) отношением порядка (последовательности) \textit{символов} по "горизонтали"{}, (3) отношением порядка(последовательности) \textit{символов} по "вертикали".}
\scnaddlevel{1}
\scnexplanation{Символ, входящий в состав \textit{двухмерного текста}, в общем случае может иметь четыре "соседних"{}\textit{символа}: (1) \textit{символ}, находящийся от него \uline{слева} в рамках той же \textit{строчки}, (2) \textit{символ}, находящийся от него \uline{справа} в рамках этой же \textit{строчки}, (3) \textit{символ}, находящийся строго \uline{над} ним в предыдущей \textit{строчке} и (4) \textit{символ}, находящийся строго \uline{под ним} в следующей \textit{строчке} текста.}
\scnaddlevel{-1}
\scnaddlevel{-1}
\scntext{сравнительный анализ}{Благодаря тому, что в состав sc.n-текстов могут входить и sc.s-тексты, и sc.g-тексты (ограниченные sc.n-контуром), SCn-код можно считать интегратором различных внешних языков представления знаний.  Это дает возможность при визуализации и разработке базы знаний ostis-системы недостатки одного из предлагаемых вариантов внешнего представления sc-текстов (SCg-кода, SCs-кода, SCn-кода) компенсировать достоинствами других вариантов.}

\bigskip

\scnmakeset{SCn-код;SCs-код}
\scnrelfrom{описание связи}{\scnstartsetlocal\\
\scnheaderlocal{SCn-код}
\scnrelfrom{синтаксическое расширение;синтаксическое ядро языка}{SCs-код}
\scnendstruct
}
\scntext{отличие}{Переход от линейности sc.s-текстов к двухмерности sc.n-текстов.}
\scnaddlevel{1}
    \scntext{уточнение}{Важной особенностью SCn-кода является "двухмерный"{} характер его текстов. Это проявляется в том, что для каждого фрагмента текста SCn-кода важное значение имеет величина отступа от левого края \textit{строчки}.
    
    В тексте \textit{SCn-кода} в отличие от текста \textit{SCs-кода} для каждого фрагмента текста важное значение имеет не только то, как этот фрагмент связан с другими фрагментами "по горизонтали"{}(какой фрагмент находится \uline{левее} и какой \uline{правее} на одной и той же \textit{строчке}), но и то, как он связан с другими фрагментами "по вертикали"{}(какой фрагмент находится \uline{выше} на предыдущей \textit{строчке} и какой находится \uline{ниже} на следующей \textit{строчке}). Так, например, если в тексте \textit{SCn-кода} некоторый \textit{sc-идентификатор}(внешний идентификатор \textit{sc-элемента}) размещен сразу после вертикальной табуляционной линии и точно \uline{под ним} размещен некоторый \textit{sc.s-коннектор}, то это означает, что указанный \textit{sc-элемент} инцидентен \textit{sc-коннектору}, изображенному указанным \textit{sc.s-коннектором}.
    
    Для того, чтобы обеспечить точное задание(формулировку) правил двухмерной инцидентности элементов(элементарных фрагментов) sc.n-текстов, вводится понятие \textit{\textbf{страницы} sc.n-текста}, понятие \textit{\textbf{строчки} sc.n-текста}, а также используется специальная \uline{разметка}, представляющая собой вертикальные табуляционные линии, расстояние между которыми примерно равняется максимальной длине sc.s-коннектора (обычно это расстояние равно ширине 4-5 символов).}
\scnaddlevel{-1}

\scnheader{sc.n-текст}
\scnidtf{текст SCn-кода}
\scnidtf{последовательность предложений SCn-кода}
\scnidtf{последовательность предложений SCn-кода, каждое из которых не является частью какого-либо другого предложения из \uline{этой} последовательности}

\scnheader{страница sc.n-текста}
\scnidtf{страница, на которой размещается sc.n-текст}
\scnnote{если sc.n-текст является частью какого-либо другого файла, разделяемого на страницы, например, публикации какой-либо части базы знаний, то sc.n-страницей считается только часть страницы, на которой изображен sc.n-текст, в то время как страница указанного файла может быть больше за счет, например, белых полей по краям страницы, необходимых для последующей распечатки.}

\scnheader{строчка sc.n-текста}
\scnnote{Максимальное количество символов в строчках sc.n-текста для каждого sc.n-текста фиксировано и определяется конкретным вариантом размещения sc.n-текста. При этом, в зависимости от отступов в рамках конкретного sc.n-предложения, строчка sc.n-текста может начинаться не с левого края sc.n-текста (но всегда с какой-то из вертикальных линий разметки) и иметь произвольную длину, ограничиваемую правой границей sc.n-страницы.}

\scnheader{линия разметки sc.n-текста}
\scnidtf{табуляционная линия sc.n-текста}
\scnidtf{вертикальная линия разметки sc.n-текста}
\scnidtf{вертикальная табуляционная линия}
\scnidtf{вертикальная линия, используемая для упрощения восприятия sc.n-текстов и показывающая уровень отступа для компонентов sc.n-предложений}
\scnexplanation{1-я линия разметки ограничивает левый край sc.n-страницы, 2-я линия разметки располагается примерно между 5 и 6 символами строчки и т.д. Расстояние между линиями разметки может меняться в зависимости от размера шрифта, однако в рамках одного sc.n-текста всегда остается одинаковым. Общее количество линий разметки ограничивается максимально возможной шириной sc.n-страницы в конкретном файле ostis-системы, содержащем данный sc.n-текст.}

\scnheader{следует отличать*}
\scnhaselementset{страница sc.n-текста;строчка sc.n-текста;строка}

\bigskip

\scnmakeset{SCn-код;SCs-код}
\scnrelfrom{сходство}{\scnstartsetlocal\\
\scnheaderlocal{Алфавит SCs-кода}
\scnreltolist{алфавит}{SCs-код;SCn-код}
\scnendstruct
}
\scnaddlevel{1}
\newpage
	\scnrelboth{семантически эквивалентная информационная конструкция}			{\scnfilelong{Алфавит символов \textit{SCs-кода} является также алфавитом символов и \textit{SCn-кода}, т.е. \textit{алфавиты}* этих языков совпадают.}}
\scnaddlevel{-1}
\scntext{сходство}{Все компоненты sc.s-текстов используются также и в sc.n-текстах:
\begin{scnitemize}
\item sc-идентификаторы
\item sc.s-коннекторы
\item модификаторы sc.s-коннекторов с соответствующими разделителями (двоеточиями)
\item разделители, используемые в sc-выражениях, обозначающих sc-множества, заданные перечислением элементов с соответствующими разделителями (\textit{точкой с запятой} или \textit{круглым маркером})
\item \textit{круглые маркеры} в перечислениях идентификаторов \mbox{sc-элементов}, связанных однотипными sc-коннекторами с однотипными модификаторами с заданным sc-элементом
\item разделители предложений (двойные точки с запятой) (опускаются при преобразовании \mbox{sc.s-предложений} в \mbox{sc.n-предложения})
\item ограничители присоединенных sc.s-предложений (опускаются при преобразовании sc.s-предложений в sc.n-предложения)
\end{scnitemize}
}
\scntext{отличие}{В отличие от sc.s-текстов в sc.n-текстах:
\begin{scnitemize}
\item добавляются новые виды sc-выражений (а именно -- sc-выражений, имеющих двухмерный характер);
\item добавляется новый вид разделителей предложений -- пустая строчка;
\item меняется размещение предложений с учетом двухмерного характера такого размещения.
\end{scnitemize}
}
\scntext{отличие}{В \textit{SCn-коде} по сравнению с \textit{SCs-кодом} добавляются новые виды \textit{sc-выражений}:
\begin{scnitemize}
\item \textit{sc-выражение}, представляющее собой двухмерный \textit{\mbox{sc.n-текст}}, ограниченный \textit{sc.n-контуром} или \textit{sc.n-рамкой}. Каждый \textit{sc.n-контур} изображается условно в виде \textit{открывающей фигурной скобки} и расположенной строго \uline{под} ней через несколько строчек \textit{закрывающей фигурной скобки}. Внутри указанных скобок (начиная от линии вертикальной разметки, на которой расположены сами скобки, и до правого края \textit{страницы}) размещается sc.n-текст. Полученный sc.n-контур является изображением структуры, являющейся результатом трансляции указанного sc.n-текста в SC-код. Каждая \textit{sc.n-рамка} изображается аналогичным образом, только вместо \textit{фигурных скобок} в ней используются \textit{квадратные скобки}, либо \textit{квадратные скобки} с \textit{восклицательным знаком} (в случае файла-образца);
\item \textit{sc-выражение}, представляющее собой двухмерный \textit{sc.g-текст}, ограниченный \textit{\mbox{sc.n-контуром}} или \textit{\mbox{sc.n-рамкой}}.
\item \textit{sc-выражение}, представляющее собой ограниченное \textit{sc.n-рамкой} двухмерное графическое изображение \textit{информационной конструкции}, закодированной в некотором \textit{файле ostis-системы}. Такой \textit{информационной конструкцией} может быть таблица, рисунок, фотография, диаграмма, график и многое другое.
\end{scnitemize}
}
\scnaddlevel{1}
	\scnnote{Нетрудно заметить, что \textit{sc.n-контур} является, по сути, двухмерным эквивалентом \textit{sc-выражения структуры}, а \textit{sc.n-рамка} -- двухмерным эквивалентом \textit{sc-выражения внутреннего файла \mbox{ostis-системы}} или \textit{sc-выражения, обозначающего файл-образец ostis-системы}.}
\scnaddlevel{-1}

\scnheader{sc.n-рамка}
\scnidtf{ограничитель изображения файла \uline{ostis-системы}, используемый в \uline{sc.n-предложениях}}

\scnhaselementrole{пример}{\scnsourcecomment{начало sc.n-рамки}\vspace{\baselineskip}\scnfileimage{\bigskip}
	\scnsourcecommentpar{конец sc.n-рамки}}
\newpage
\scnnote{С формальной точки зрения \textit{sc.n-рамка} всегда представляет собой \uline{одну} \textit{строчку sc.n-текста}. Это означает, что \textit{sc.n-рамка} не может быть синтаксически разделена на части в рамках того \textit{sc.n-текста}, в котором она используется, и внутрь нее не могут вставляться, например, \textit{присоединенные sc.n-предложения} или какой-либо другой текст (за исключением случаев, когда \textit{sc.n-рамка} содержит \textit{sc.n-текст}, но в этом случае указанный \textit{sc.n-текст} все равно будет рассматриваться как целостный внешний файл, а не как фрагмент окружающего его \textit{sc.n-текста}).}

\scnheader{sc.n-контур}
\scnidtf{используемый в \uline{sc.n-предложениях} ограничитель, являющийся изображением структуры}
\scnhaselementrole{пример}{\scnsourcecomment{начало sc.n-контура}\\\settab\scnstartsetlocal \bigskip\\ \scnendstruct \scnsourcecommentpar{конец sc.n-контура}}

\bigskip

\scnmakeset{sc.s-предложение;sc.n-предложение}
\scntext{сходство}{Понятие \textit{sc.n-предложения} является естественным обобщением понятия \textit{sc.s-предложения}. Более того, \uline{аналогичным} для \textit{sc.s-предложений} образом вводятся понятия:
\begin{scnitemize}
%\item \textit{тривиального sc.n-предложения}
\item \textit{простого sc.n-предложения}
\item \textit{сложного sc.n-предложения}
\item \textit{sc.n-предложения, содержащего присоединенные sc.n-предложения}
\item \textit{sc.n-предложения, не содержащего присоединенные sc.n-предложения}
\item \textit{присоединенного sc.n-предложения}
\item \textit{неприсоединенного sc.n-предложения}
\end{scnitemize}}
\filemodetrue
\scnrelfromlist{отличие}{\uline{Если} каждое \textit{неприсоединенное sc.s-предложение} \uline{либо} являетcя первым предложением \textit{sc.s-текста}, \uline{либо} начинается после \textit{разделителя sc.s-предложений} (\textit{двойной точки с запятой}), \uline{то} каждое \textit{неприсоединенное sc.n-предложение} начинается с начала новой строчки;
\uline{Если} каждое \textit{присоединенное sc.s-предложение} начинается либо после открывающего ограничителя присоединенных sc.s-предложений (открывающей круглой скобки со звездочкой), \uline{либо} после \textit{разделителя sc.s-предложений}, \uline{то} каждое \textit{присоединенное sc.n-предложение} начинается с новой строчки под sc-идентификатором, которым завершается то sc.n-предложение (и соответственно, sc.s-предложение), в которое встраивается данное \textit{присоединенное sc.n-предложение}; 
Первый \textit{sc-идентификатор}, входящий в состав \textit{sc.n-предложения} до \textit{sc.s-коннектора} выделяется \uline{жирным} курсивом;
В \textit{sc.n-предложениях двойная точка с запятой} не используется в качестве признака завершения этих предложений и, соответственно, не используется в качестве разделителя \textit{sc.n-предложений}. Таким разделителем является \textit{пустая строчка}.}

\scntext{отличие}{Благодаря двухмерности SCn-кода появляются более широкие возможности (степени свободы) для наглядного и компактного размещения sc.n-предложений.}
\scnaddlevel{1}
\scnrelfromlist{уточнение}{При оформлении sc.n-предложения осуществляется четкая \uline{табуляция} всех присоединенных к нему sc.n-предложений, присоединяемых к исходному "по вертикали". Вертикальная линия табуляции задает левую границу исходного (максимального) sc.n-предложения или левую границу присоединенного sc.n-предложения, присоединяемого "по вертикали". Левая граница sc.n-предложения задает начало первого sc-идентификатора, входящего в состав этого \mbox{sc.n-предложения}, а также начало sc.s-коннектора, инцидентного указанному \mbox{sc-идентификатору} и размещаемого \uline{строго под} этим sc-идентификатором. Расстояние между вертикальными табуляционными линиями фиксировано и примерно равно максимальной длине sc.s-коннектора;
В отличие от sc.s-текстов: в sc.n-текстах sc.s-коннектор может быть инцидентен предшествующему sc-идентификатору (как простому, так и sc-выражению) не только "по горизонтали"{}, но и "по вертикали". Для этого sc.s-коннектор размещается строго \uline{под} предшествующим ему sc-идентификатором;
Кроме того "по вертикали"\ sc-идентификатор может быть инцидентен не одному, а \uline{нескольким} sc.s-коннекторам, которые последовательно "по вертикали"\ размещаются \uline{под} указанным sc-идентификатором. Это позволяет в рамках одного sc.n-предложения представлять произвольное число "ответвлений"\ от каждого sc-идентификатора, т.е. произвольное число sc.s-коннекторов, инцидентных этому sc-идентификатору;
Каждый sc-идентификатор, включая sc-выражение, ограничиваемого фигурными или квадратными скобками, должен размещаться сразу правее вертикальной разметочной линии, если \uline{под ним} размещается sc.s-коннектор;
Каждый sc.s-коннектор выделяется жирным некурсивным шрифтом и, если он находится \uline{под} инцидентным ему sc-идентификатором, размещается строго между двумя вертикальными разметочными линиями, прижимаясь при этом к левой из этих двух разметочных линий.}
\scnaddlevel{-1}
\filemodefalse

\scnheader{SCn-код}
\scntext{правило синтаксической трансформации}{Поскольку по отношению к SCn-коду SCs-код является \textit{синтаксическим ядром языка*}, SCn-код можно рассматривать как результат интеграции нескольких направлений расширения SCs-кода, в основе которых лежат правила синтаксической трансформации sc.s-текстов и sc.n-текстов, ориентированные на повышение эффективности использования тех возможностей обеспечения наглядности и компактности sc.n-текстов, которые открываются при переходе от линейности sc.s-текстов к двухмерности sc.g-текстов}

\scnheader{sc.n-предложение}
\scnrelfromlist{заданная операция}{Операция преобразования sc.s-предложения в sc.n-предложение*\\
	\scnaddlevel{1}
	\scnsubset{синтаксическая трансформация*}
	\scnexplanation{Каждое \textit{sc.s-предложение}, записываемое линейно ("горизонтально") может быть преобразовано в соответствующее двухмерное \textit{sc.n-предложение}. 
	Перечислим основные правила трансформации sc.s-предложений в sc.n-предложения:
	\begin{scnitemize}
		\item sc.s-коннектор может размещаться на следующей строчке под предшествующим \mbox{sc-идентификатором}, начиная с того же символа следующей строчки, что и указанный sc-идентификатор\char59
		\item если sc-идентификатор переносится на следующую строчку, то его продолжение на следующей строчке осуществляется с таким же отступом от начала строчки, с каким указанный sc-идентификатор начинается\char59
		\item перечисление sc-идентификаторов, разделенных точкой с запятой, может осуществляться не "в строчку"\ , а "в столбик"\ при размещении каждого следующего sc-идентификатора строго под предшествующим. При этом, разделительная точка с запятой может быть заменена \textit{круглым маркером}, который помещается \uline{перед} каждым перечисляемым \mbox{sc-идентификатором}\char59
		\item закрывающая фигурная или квадратная скобка может быть размещена строго \uline{под} соответствующей открывающей скобкой\char59
		\item sc-идентификатор в sc.n-предложении может быть связан с другими sc-идентификаторами через несколько разных sc.s-коннекторов. При этом, каждый из этих sc.s-коннекторов размещается строго под предшествующим, но только после того, когда будет завершена запись всей, в общем случае разветвленной, цепочки sc.s-коннекторов и sc-идентификаторов, которая начинается с предшествующего sc.s-коннектора. В SCs-коде аналога таким предложениям с неограниченной возможностью описания “разветвленных” связей sc-идентификаторов нет. Следовательно, если в sc.s-тексте sc-идентификатор может быть инцидентен не более, чем двум sc.s-коннекторам (слева и справа от него), то в sc.n-тексте sc-идентификатор может быть дополнительно инцидентен неограниченному числу (причем, не обязательно одинаковых) sc.s-коннекторов, которые размещаются “по вертикали” строго под ним.
	\end{scnitemize}}
	\scnaddlevel{-1}
	;Операция присоединения sc.n-предложения*\\
		\scnaddlevel{1}
	\scnsubset{синтаксическая трансформация*}
	\scnexplanation{Некоторое sc.n-предложение может быть присоединено к другому sc.n-предложению, если в этом другом sc.n-предложении есть sc-идентификатор (но не sc.s-модификатор), с которого начинается первое (присоединяемое) sc.n-предложение.
	Присоединение в происходит следующим образом:
	\begin{scnitemize}
		\item начальный sc-идентификатор присоединяемого предложения опускается\char59
		\item оставшаяся часть sc.n-предложения, начиная от sc.s-коннектора, записывается под таким же sc-идентификатором, но входящим в состав того sc.n-предложения, к которому присоединяется данное sc.n-предложение. С учетом этого смещаются все отступы в присоединяемом sc.n-предложении.
	\end{scnitemize}
	Таким образом может формироваться произвольное число любых разветвлений.}
	\scnaddlevel{-1}}
\scnnote{По сути, семантика sc.n-предложения -- множество маршрутов в sc-тексте, возможно пересекающихся и исходящих из заданного sc-элемента}

\scnheader{Описание примеров выполнения операций, заданных на множестве sc.n-предложений}
\scnstartsubstruct

\bigskip

\scnfilelong{\textit{Треугк(ТчкВ;ТчкС;ТчкD)} $\Rightarrow$ \textit{сторона*}: \textit{Отр(ТчкВ;ТчкС)} (* $\in$ \textit{отрезок};; *);;}\\
\scnrelfrom{Операция преобразования sc.s-предложения в sc.n-предложен ие}{\scnfilescn{
		\scnheader{Треугк(ТчкВ;ТчкС;ТчкD)}
		\scnrelfrom{сторона}{Отр(ТчкВ;ТчкС)}
		\scnaddlevel{1}
			\scniselement{отрезок}
		\scnaddlevel{-1}
		}}
\scnresetlevel

\bigskip
\bigskip

\scnfilescn{
	\scnheader{Треугк(ТчкВ;ТчкС;ТчкD)}
	\scnrelfrom{сторона}{Отр(ТчкВ;ТчкС)}

	\scnheader{Отр(ТчкВ;ТчкС)}
	\scniselement{отрезок}
}
\scnrelfrom{Операция присоединения sc.n-предложения}{\scnfilescn{
		\scnheader{Треугк(ТчкВ;ТчкС;ТчкD)}
		\scnrelfrom{сторона}{Отр(ТчкВ;ТчкС)}
		\scnaddlevel{1}
		\scniselement{отрезок}
		\scnaddlevel{-1}
}}
\scnresetlevel

\bigskip
\scnendstruct               
\bigskip

\scnendstruct \scnendcurrentsectioncomment

\end{SCn}

\scsubsubsection[\scnmonographychapter{Глава 2.2. Семейство внешних языков интеллектуальных компьютерных систем нового поколения, близких языку внутреннего смыслового представления знаний (SCg, SCs, SCn)}]{Предметная область и онтология синтаксиса языка внешнего форматированного представления информационных конструкций внутреннего языка ostis-систем}
\label{intro_scn_syntax}

\scsubsubsection[\scnmonographychapter{Глава 2.2. Семейство внешних языков интеллектуальных компьютерных систем нового поколения, близких языку внутреннего смыслового представления знаний (SCg, SCs, SCn)}]{Предметная область и онтология денотационной семантики языка внешнего форматированного представления информационных конструкций внутреннего языка ostis-систем}
\label{intro_scn_semantic}

\scsubsubsection[\scnmonographychapter{Глава 2.2. Семейство внешних языков интеллектуальных компьютерных систем нового поколения, близких языку внутреннего смыслового представления знаний (SCg, SCs, SCn)}]{Предметная область и онтология иерархического семейства подъязыков, семантически эквивалентных языку внешнего форматированного представления информационных конструкций внутреннего языка ostis-систем}
\label{intro_scn_sublang}

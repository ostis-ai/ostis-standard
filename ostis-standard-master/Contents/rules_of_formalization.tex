\begin{SCn}
	
\scnsectionheader{Правила оформления Стандарта OSTIS}
	
\scnheader{Стандарт OSTIS}
\scnrelfrom{общие правила построения}{\scnkeyword{Общие правила построения Стандарта OSTIS}}
	\scnaddlevel{1}
	\scnidtf{принципы, лежащие в основе структуризации и оформления Стандарта OSTIS}
	\scneq{
	\scnmakevectorlocal{\scnfileitem{Основной формой представления \textit{Стандарта OSTIS} как полной документации текущего состояния \textit{Технологии OSTIS} является \textit{текущее состояние} основной части \textit{базы знаний} специальной интеллектуальной компьютерной \textit{Метасистемы IMS.ostis}, обеспечивающей использование и эволюцию (перманентное совершенствование) \textit{Технологии OSTIS}. Такое представление \textit{Стандарта OSTIS} обеспечивает эффективную семантическую навигацию по содержанию \textit{Стандарта OSTIS} и возможность задавать \textit{Метасистеме IMS.ostis} широкий спектр нетривиальных вопросов о самых различных деталях и тонкостях \textit{Технологии OSTIS}};
	\scnfileitem{Непосредственно сам \textit{Стандарт OSTIS} представляет собой внутреннее \textit{смысловое представление} основной части базы знаний \textit{Метасистемы IMS.ostis} на внутреннем смысловом языке \textit{ostis-систем} (этот язык назван нами \textit{SC-кодом} - Semantic Computer Code)};   
	\scnfileitem{С семантической точки зрения \textit{Стандарт OSTIS} представляет собой иерархическую систему формальных моделей \textit{предметных областей} и соответствующих им \textit{формальных онтологий}};
	\scnfileitem{С семантической точки зрения \textit{Стандарт OSTIS} представляет собой большую \textit{рафинированную семантическую сеть}, которая, соответственно, имеет нелинейный характер и которая включает в себя знаки любых видов описываемых сущностей(материальных сущностей, абстрактных сущностей, понятий, связей, структур) и, соответственно этому, содержит связи между всеми этими видами сущностей(в частности, связи между связями, связи между структурами)};
	\scnfileitem{В состав \textit{Стандарта OSTIS} входят также файлы информационных конструкций, не являющихся конструкциями \textit{SC-кода} (в том числе и sc-текстов, принадлежащих различным естественным языкам). Такие файлы позволяют формально описывать в базе знаний синтаксис и семантику различных внешних языков, а также позволяют включать в состав базы знаний различного рода пояснения, примечания, адресуемые непосредственно пользователям и помогающие им в понимании формального текста базы знаний};
	\scnfileitem{Кроме представления \textit{Стандарта OSTIS} на внутреннем \textit{языке представления знаний} используется также внешняя форма представления \textit{Стандарта OSTIS} на \textit{внешнем языке представления знаний}. При этом указанное внешнее представление \textit{Стандарта OSTIS} должно быть структурировано и оформлено так,чтобы читатель мог достаточно легко "вручную"{} найти в этом тексте практически любую интересующую его \textit{информацию}. В качестве \textit{формального языка} внешнего представления \textit{Стандарта OSTIS} используется \textit{SCn-код}, описание которого приведено в \textit{Стандарте OSTIS} в разделе "\nameref{intro_scn}{}"};
	\scnfileitem{Предлагаемое Вам издание \textit{Стандарта OSTIS} представляется на формальном языке \textit{SCn-код}, который является языком внешнего представления больших текстов \textit{SC-кода}, в которых большое значение имеет наглядная структуризация таких текстов.};
	\scnfileitem{\textit{Стандарт OSTIS} имеет онтологическую структуризацию, т.е. представляет собой иерархическую систему связанных между собой \textit{формальных предметных областей} и соответствующих им \textit{формальных онтологий}.		
	Благодаря этому обеспечивается высокий уровень стратифицированности \textit{Стандарта OSTIS}};
	\scnfileitem{Каждому \textit{понятию}, используемому в \textit{Стандарте OSTIS}, соответствует свое место в рамках этого Стандарта, своя \textit{предметная область} и соответствующая ей \textit{онтология}, где это \textit{понятие} подробно рассматривается (исследуется), где концентрируется вся основная информация об этом \textit{понятии}, различные его свойства};
	\scnfileitem{Кроме \textit{Общих правил построения Стандарта OSTIS} в \textit{Стандарте OSTIS} приводятся описания различных частных (специализированных) правил построения (оформления) различных видов фрагментов \textit{Стандарта OSTIS}.	
	К таким видам фрагментов относятся следующие:
		\begin{scnitemize}
			\item\textit{sc-идентификатор}
			\scnaddlevel{-3}	
				\scnidtf{внешний идентификатор внутреннего знака (\textit{sc-элемента}) входящего в состав \textit{базы знаний ostis-системы}}	
				\scnidtf{\textit{информационная конструкция} (чаще всего это строка символов), обеспечивающая однозначную идентификацию соответствующей сущности, описываемой в \textit{базах знаний ostis-систем}, и являющаяся, чаще всего, именем (термином), соответствующим описываемой сущности, именем, обозначающим эту сущность во внешних текстах \textit{ostis-систем}}
			\scnaddlevel{3}
			\item\textit{sc-спецификация}
			\scnaddlevel{-3}
				\scnidtf{семантическая окрестность}
				\scnidtf{семантическая окрестность соответствующего \textit{sc-элемента} (внутреннего знака, хранимого в памяти \textit{ostis-системы} в составе её \textit{базы знаний}, представленной на внутреннем языке \textit{ostis-систем.})}
				\scnidtf{семантическая окрестность некоторого \textit{sc-элемента}, хранимого в \textit{sc-памяти}, в рамках текущего состояния этой \textit{sc-памяти}}
			\scnaddlevel{3}
			\item(\textit{sc-конструкция} $\setminus$ \textit{sc-спецификация})
			\scnaddlevel{-3}
				\scnidtf{\textit{sc-конструкция} (конструкция \textit{SC-кода} -- внутреннего языка \textit{ostis-систем}), не являющаяся \textit{sc-спецификацией}}
				\scnaddlevel{1}
				\scntext{сокращение}{\textit{sc-конструкция}, не являющаяся \textit{sc-спецификацией}}
				\scnaddlevel{-1}
			\scnaddlevel{3}
			\item(\textit{файл ostis-системы $\setminus$ sc-идентификатор})
			\scnaddlevel{-3}
				\scnidtf{\textit{файл ostis-системы}, не являющийся sc-идентификатором}
			\scnaddlevel{3}
		\end{scnitemize}};
	\scnfileitem{Правила построения \textit{sc-идентификаторов} будем также называть \textit{Правилами внешней идентификации sc-элементов}.\\
	\textit{Общие правила построения sc-идентификаторов} смотрите в разделе ``\nameref{intro_idtf}''.
	В состав этих правил входят правила внешней идентификации \textit{sc-констант, sc-переменных}, \textit{отношений, параметров},\textit{sc-конструкций, файлов ostis-систем.}};
	\scnfileitem{К числу частных правил построения sc-идентификаторов относятся:
		\begin{scnitemize}
			\item\textit{Правила построения sc-идентификаторов персон} -- смотрите раздел ``\nameref{sd_person}''.
			\item\textit{Правила построения sc-идентификаторов библиографических источников} -- смотрите раздел ``\nameref{sd_bibliography}''.
		\end{scnitemize}};
	\scnfileitem{Общие правила построения \textit{sc-конструкций} (конструкций \textit{SC-кода} -- внутреннего языка ostis-систем) смотрите в разделе ``\nameref{intro_sc_code}'', а также в разделе ``\nameref{sd_sc_code_syntax}'' и в разделе ``\nameref{sd_sc_code_semantic}''};
	\scnfileitem{К числу частных правил построения \textit{sc-конструкций} относятся:
		\begin{scnitemize}
			\item \textit{Правила построения баз знаний ostis-систем} -- смотрите раздел ``\nameref{sd_knowledge}''
			Эти правила направлены на обеспечение целостности баз знаний ostis-систем, на обеспечение (1) востребованности (нужности) знаний, входящих в состав каждой базы знаний, и (2) целостности самой базы знаний, т.е. достаточности знаний, входящих в состав каждой базы знаний для эффективного функционирования соответствующей ostis-системы
			\item \textit{Правила построения разделов и сегментов баз знаний ostis-систем}
			\item \textit{Правила представления логических формул и высказываний в базах знаний ostis-систем}
			\item \textit{Правила представления формальных предметных областей в базах знаний ostis-систем}
			\item \textit{Правила представления формальных логических онтологий в базах знаний ostis-систем}
		\end{scnitemize}};
	\scnfileitem{Формальное описание синтаксиса и семантики \textit{SCn-кода} приведено в разделе ``\nameref{intro_scn}''}}}
\scnauthorcomment{Вычитать то, что дальше}
\scnfilelong{\textit{сослаться на:}
\begin{scnitemize}
	\item правила построения sc-идентификаторов для различных классов sc-элементов
	\item правила построения различного вида sc-текстов
	\begin{itemize}
		\item разделов базы знаний
		\item баз знаний	
	\end{itemize}	
\end{scnitemize}
\begin{scnitemize}
	\item правила построения sc-спецификаций сущностей
	\begin{itemize}
		\item разделов базы знаний
		\item предметная область
		\item сегментов базы знаний
		\item библ.- источников	
	\end{itemize}	
\end{scnitemize}
\begin{scnitemize}
	\item типичные опции
	\begin{itemize}
		\item баз знаний	
	\end{itemize}	
\end{scnitemize}
\begin{scnitemize}
	\item общие направления развития
	\begin{itemize}
		\item раздел базы знаний	
	\end{itemize}	
\end{scnitemize}
\begin{scnitemize}
	\item направления развития*:
	\begin{itemize}
		\item Стандарта OSTIS
		\item Введение в язык OSTIS-Cn	
		\item Предметная область и онтология библиографии
		\item Библиография OSTIS
	\end{itemize}	
\end{scnitemize}
\begin{scnitemize}
	\item направления и правила	
	\begin{itemize}
		\item деятельность	
		\item Редколлегия Стандарта OSTIS
		\item соавтор Стандарта OSTIS
	\end{itemize}
\end{scnitemize}}

\scnheader{sc-идентификатор файла ostis-системы}
\scnrelfrom{правила построения}{Правила идентификации файлов ostis-систем}
\scnheader{sc-файл ostis-системы}
\scnrelfrom{правила построения}{Правила построения sc-файлов ostis-систем}
\scnheader{спецификация файла ostis-системы}
\scnrelfrom{правила построения}{Правила спецификации файлов ostis-систем}
	
\end{SCn}
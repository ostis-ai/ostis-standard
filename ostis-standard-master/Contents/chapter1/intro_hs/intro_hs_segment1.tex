\scnsegmentheader{Уточнение понятия кибернетической системы}
\scnstartsubstruct

\scnheader{кибернетическая система}
\scnidtf{cистема, которая способна \uline{управлять} своими \uline{действиями}, адаптируясь к изменениям состояния внешней среды (среды своего "обитания") в целях самосохранения (сохранения своей целостности и "комфортности"{} существования путем удержания своих "жизненно"{} важных параметров в определенных рамках "комфортности") и/или в целях формирования определенных реакций (воздействий на внешнюю среду) в ответ на определенные стимулы (на определенные ситуации или события во внешней среде), а также которая способна (при соответствующем уровне развития) эволюционировать в направлении:
\begin{scnitemize}
    \item изучения своей внешней среды как минимум для предсказания последствий своих воздействий на внешнюю среду, а также для предсказания изменений внешней среды, которые не зависят от собственных воздействий;
    \item изучения самой себя и, в частности, своего взаимодействия с внешней средой;
    \item создания технологий (методов и средств), обеспечивающих изменение своей внешней среды (условий своего существования) в собственных интересах.
\end{scnitemize}
}
\scnidtf{адаптивная система}
\scnidtf{целенаправленная (целеустремленная) система}
\scnidtf{активный субъект самостоятельной деятельности}
\scnidtf{материальная сущность, способная целенаправленно (в своих интересах) воздействовать  на среду своего обитания  как минимум для сохранения своей целостности, жизнеспособности, безопасности}
\scnnote{Уровень (степень) адаптивности, целенаправленности, активности у систем, основанных на обработке информации может быть самым различным.}
\scnidtf{система, организация функционирования которой основано на обработке информации о той среде, в которой существует эта система}
\scnidtf{материальная сущность, способная к активной  целенаправленной деятельности, которая  на определенном уровне развития указанной сущности становится "осмысленной", планируемой, преднамеренной деятельностью}
\scnidtf{субъект, способный на самостоятельное выполнение некоторых "внутренних"{} и "внешних"{} действий либо порученных извне, либо инициированных самим субъектом}
\scnidtf{сущность, способная выполнять роль субъекта деятельности}
\scnidtf{естественная или искусственно созданная система, способная мониторить и анализировать свое состояние и состояние окружающей среды, а также способная достаточно активно воздействовать на собственное на собственное состояние и на состояние окружающей среды}
\scnidtf{система, способная в достаточной степени самостоятельно взаимодействовать со своей средой , решая различные задачи}
\scnidtf{система, основанная на обработке информации}

\newpage	
\scnrelto{ключевой знак}{\scncite{Glushkov1979}}
\scnaddlevel{1}
	\scniselement{статья}
\scnaddlevel{-1}


\bigskip

\scnfragmentheader{Типология кибернетических систем}
\scnstartsubstruct

\scnheader{кибернетическая система}

\scnrelfrom{разбиение}{Признак естественности или искусственности кибернетических систем}
\scnaddlevel{1}
\scneqtoset{естественная кибернетическая система\\
    \scnaddlevel{1}
    \scnidtf{кибернетическая система естественного происхождения}
    \scnsuperset{человек}
    \scnaddlevel{-1}
;компьютерная система\\
    \scnaddlevel{1}
    \scnidtf{искусственная кибернетическая система}
    \scnidtf{кибернетическая система искусственного происхождения}
    \scnidtf{технически реализованная кибернетическая система}
    \scnaddlevel{-1}
;симбиоз естественных и искусственных кибернетических систем\\
    \scnaddlevel{1}
    \scnidtf{кибернетическая система, в состав которой входят компоненты как естественного, так и искусственного происхождения}
    \scnsuperset{сообщество компьютерных систем и людей}
    \scnaddlevel{-1}}
\scnaddlevel{-1}

\scnheader{искусственная сущность}
\scnidtf{артефакт}
\scnidtf{сущность, являющаяся либо результатом человеческой деятельности, либо частью самой этой деятельности}
\scnidtf{сущность искусственного происхождения}
\scnidtf{антропогенная сущность}
\scnsuperset{научно-техническое знание}
\scnaddlevel{1}
\scnidtf{знание, приобретенное в результате научно-технической деятельности человеческого общества}
\scnaddlevel{-1}
\scnsuperset{материальная искусственная сущность}
\scnaddlevel{1}
\scnsuperset{компьютерная система}
\scnaddlevel{-1}

\scnheader{компьютерная система}
\scnidtf{искусственная кибернетическая система}
\scnnote{Особенностью компьютерных систем является то, что они могут выполнять "роль"{} не только продуктов соответствующих действий по реализации этих систем, но и сами являются \textit{субъектами*}, способными выполнять (автоматизировать) широкий спектр действий. При этом интеллектуализация этих систем существенно расширяет этот спектр. \textit{См. интеллектуальная компьютерная система}.}
\scnidtf{технически реализованная кибернетическая система}
\scnidtf{искусственная кибернетическая система}
\scnsubset{кибернетическая система}
\scnsuperset{современная компьютерная система традиционного вида}
\scnsuperset{современная интеллектуальная компьютерная система}
\scnsuperset{интеллектуальная компьютерная система следующего поколения}
\scnaddlevel{1}
\scnsuperset{ostis-система}
\scnnote{Основной тенденцией эволюции компьютерных систем является повышение уровня их интеллектуальности.}
\scnrelfromset{особенность}{\scnfileitem{Ориентация на принципиально новые компьютеры};\scnfileitem{Cущественное повышение уровня интеллекта}}
\scnaddlevel{-1}
\newpage
\scnrelfrom{разбиение}{Структурная классификация кибернетических систем}
\scnaddlevel{1}
\scneqtoset{простая кибернетическая система\\
;индивидуальная кибернетическая система\\
;многоагентая система\\
\scnaddlevel{1}
\scnsubdividing{
одноуровневый коллектив кибернетических систем
    \scnaddlevel{1}
    \scnidtf{многоагентная система, агентами которой не могут быть многоагентные системы}
    \scnaddlevel{-1}
;иерархический коллектив кибернетических систем
    \scnaddlevel{1}
    \scnidtf{многоагентная система, по крайней мере одним  агентом которой является многоагентная система}
    \scnaddlevel{-1}}
\scnsubdividing{коллектив из простых кибернетических систем\\
\scnaddlevel{1}
\scnnote{Такой коллектив может быть либо одноуровневым, либо иерархическим коллективом}
\scnaddlevel{-1};
коллектив из индивидуальных кибернетических систем;коллектив из индивидуальных и простых кибернетических систем}
\scnaddlevel{-1}}


\scnheader{кибернетическая система}
\scnrelfrom{разбиение}{Классификация кибернетических систем по признаку наличия надсистемы и роли в рамках этой надсистемы}
\scnaddlevel{1}
\scneqtoset{кибернетическая система, не являющаяся частью никакой другой кибернетической системы\\
\scnaddlevel{1}
\scnidtf{кибернетическая система, не имеющая надсистем}
\scnaddlevel{-1}
;кибернетическая система, встроенная в индивидуальную кибернетическую систему\\
;агент многоагентной системы\\
\scnaddlevel{1}
\scnidtf{кибернетическая система, являющаяся агентом одной или нескольких многоагентных систем}
\scnaddlevel{-1}
}
\scnaddlevel{-1}

\scnheader{простая кибернетическая система}
\scnidtf{\textit{кибернетическая система}, уровень развития которой находится ниже уровня \textit{индивидуальных кибернетических систем} и которая является специализированным средством обработки информации специализированным решателем задач, реализующим (интерпретирующим) чаще всего один \textit{метод} решения задач и, соответственно, решающим только \textit{задачи} заданного \textit{класса задач}}
\scnidtf{специализированный \textit{решатель задач}}
\scnnote{\textit{простая кибернетическая система} может быть \textit{компонентом*}, встроенным в \textit{индивидуальную кибернетическую систему}, а также может быть \textit{агентом*} \scnbigskip \textit{многоагентной системы}, являющейся коллективом из простых кибернетических систем}

\scnheader{индивидуальная кибернетическая система}
\scnidtf{условно выделенный уровень развития \textit{кибернетических систем}, в основе которого лежит переход от \textit{специализированного решателя задач к индивидуальному решателю}, обеспечивающему интерпретацию произвольного (нефиксированного) набора \textit{методов} (программ) решения задач при условии, если эти \textit{методы} введены (загружены, записаны) в \textit{память} \textit{кибернетической системы}}
\scnidtf{кибернетическая система, способная быть самостоятельной}
\scnexplanation{Признаками индивидуальных кибернетических систем являются:
\begin{scnitemize}
    \item наличие \textit{памяти}, предназначенной для хранения как минимум интерпретируемых \textit{методов} (программ)  и обеспечивающей корректировку (редактирование) хранимых \textit{методов}, а также их удаление  из \textit{памяти} и ввод (запись) в \textit{память} новых \textit{методов};
    \item легкая возможность "перепрограммировать"{} \textit{кибернетическую систему} на решение других задач, что обеспечивается наличием \textit{универсальной модели решения задач} и, соответственно, \textit{универсальным интерпретатором \uline{любых} моделей}, представленных (записанных) на соответствующем \textit{языке};
    \item наличие пусть даже простых средств коммуникации (обмена информацией) с другими \textit{кибернетическими системами} (например, с людьми);
    \item способность входить в различные \textit{коллективы кибернетических систем}.
\end{scnitemize}
}
\scnnote{класс \textit{индивидуальных кибернетических систем} — это определенный этап эволюции кибернетических систем, означающий переход к кибернетическим системам, которые способны самостоятельно "выживать"}
\scnidtf{самостоятельная автономная, целостная кибернетическая системам}
\scnidtf{субъект деятельности}
\scnnote{\textit{индивидуальная кибернетическая система} может быть агентом (членом) многоагентной системы (членом коллектива индивидуальных кибернетических систем), но некоторые многоагентные системы могут состоять из агентов , не являющихся  \textit{индивидуальными кибернетическими системами}, представляющих собой простые специализированные кибернетические системы, выполняющие достаточно простые действия (\scncite{Stefanuk}, \scnbigspace \scncite{fonNeuman})}
\scnidtf{кибернетическая система, которая обладает достаточной самостоятельностью (целостностью), но не является коллективом таких самостоятельных  кибернетических систем}
\scnidtf{минимальная самостоятельная (самодостаточная, в известной степени автономная) кибернетическая система}
\scnidtf{индивидуальный субъект}

\scnheader{кибернетическая система, встроенная в индивидуальную кибернетическую систему}
\scnrelsuperset{пример}{sc-агент ostis-системы}
\scnrelsuperset{пример}{решатель задач ostis-системы}

\scnheader{многоагентная система}
\scnidtf{коллектив взаимодействующих автономных кибернетических систем, имеющих общую среду обитания (жизнедеятельности)}
\scnsubdividing{одноуровневая многоагентная система;иерархическая многоагентная система}

\scnheader{одноуровневая многоагентная система}
\scnidtf{специализированное средство решения задач, реализующее либо \uline{одну} модель параллельного (распределенного) решения задач соответствующего класса, либо комбинацию \uline{фиксированного числа} разных и параллельно реализованных моделей решения задач}
\scnsubdividing{одноуровневая однородная многоагентная система;одноуровневая неоднородная многоагентная система}

\scnheader{коллектив индивидуальных кибернетических систем}
\scnsubset{многоагентная система}
\scnidtf{многоагентная система, агентами (членами) которой являются \uline{индивидуальные}(!) кибернетические системы}
\scnsubdividing{
коллектив людей\\
\scnaddlevel{1}
\scnidtf{человеческое сообщество}
\scnaddlevel{-1}
;сообщество компьютерных систем и людей
}

\scnheader{иерархический коллектив индивидуальных кибернетических систем}
\scnidtf{многоагентная система, агентами (членами) которой могут быть:
\begin{scnitemize}
    \item индивидуальные кибернетические системы;
    \item коллективы индивидуальных кибернетических систем;
    \item коллективы, состоящие из индивидуальных кибернетических систем и коллективов индивидуальных кибернетических систем и т.д.
\end{scnitemize}
}
\bigskip
\scnendstruct 

\scnfragmentheader{Структура кибернетической системы}

\scnstartsubstruct

\scnheader{кибернетическая система}
\scnrelfromset{обобщенная декомпозиция}{
информация, хранимая в памяти кибернетической системы;абстрактная память кибернетической системы;решатель задач кибернетической системы;физическая оболочка кибернетической системы
}

\scnheader{информация, хранимая в памяти кибернетической системы}
\scnidtf{информация, хранимая в памяти \textit{кибернетической системы} и представляющая собой информационную модель среды, в которой действует (существует, функционирует) эта \textit{кибернетическая система}
}
\scnidtf{текущее состояние памяти кибернетической системы}
\scnidtf{текущее состояние внутренней (информационной) среды кибернетической системы}
\scnrelto{второй домен}{информация, хранимая в памяти кибернетической системы*}
\scnaddlevel{1}
\scniselement{бинарное отношение}
\scniselement{ориентированное отношение}
\scnaddlevel{-1}

\scnheader{абстрактная память кибернетической системы}
\scnidtf{внутренняя абстрактная информационная среда кибернетической системы, представляющая собой динамическую информационную  конструкцию, каждое состояние которой есть не что иное, как информация , хранимая в памяти кибернетической системы в соответствующий момент времени}
\scnidtf{абстрактная динамическая модель памяти кибернетической системы}
\scnsubset{динамическая информационная конструкция}
\scnaddlevel{1}
\scnidtf{процесс преобразования информационной конструкции}
\scnaddlevel{-1}

\scnheader{решатель задач кибернетической системы}
\scnidtf{совокупность всех навыков (умений), приобретенных кибернетической системой к рассматриваемому моменту}
\scnidtf{встроенный в кибернетическую систему субъект, способный выполнять целенаправленные ("осознанные") действия во внешней среде этой кибернетической системы, а также в её внутренней среде (в абстрактной памяти)}

\scnheader{действие кибернетической системы}
\scnsubset{действие}
\scnidtf{целенаправленное ("осознанное") действие, выполняемое кибернетической системой, а точнее, её решателем задач}
\scnsubdividing{внешнее действие кибернетической системы\\
	\scnaddlevel{1}
	\scnidtf{действие, выполняемое кибернетической системой в её внешней среде}
	\scnidtf{поведенческое действие}
	\scnaddlevel{-1}
;действие кибернетической системы, выполняемое в собственной физической оболочке
;действие кибернетической системы, выполняемое в собственной абстрактной памяти
\scnaddlevel{1}
	\scnidtf{речь идёт о действиях, направленных на преобразование информации, хранимой в памяти, но никак не на преобразование физической памяти (физической оболочки абстрактной памяти)}
\scnaddlevel{-1}	
}
\newpage
\scnnote{Каждое \uline{сложное} действие,выполняемое кибернетической системой вне собственный абстрактной памяти, включает в себя поддействия, выполняемые в указанной абстрактной памяти. Это означает, что все внешние действия кибернетической системы \uline{управляются} внутренними её действиями (действиями в абстрактной памяти).}

\scnheader{задача}
\scnidtf{спецификация действия}
\scnidtf{формулировка задачи с различной степенью детализации (уточнения) специфицируемого (описываемого) действия, в состав которой может входить:
	\begin{scnitemize}
		\item описание цели (целевой ситуации);
		\item указание объектов (аргументов) действия;
		\item указание типа действия (класса действий, которому принадлежит данное действие);
		\item указание субъекта действия;
		\item указание инструмента (средств) выполненного действия;
		\item и др.
	\end{scnitemize}}

\scnnote{Процесс решения задачи и действие, специфицируемое этой задачей (точнее, процесс выполнения этого действия) суть одно и то же.}

\scnheader{задача, решаемая кибернетической системой}
\scnidtf{задача, решаемая соответствующей кибернетической системой}
\scnidtf{Второй домен отношения "быть задачей, решаемой заданной кибернетической системой*"}
\scnrelboth{следует отличать}{задача, решаемая кибернетической системой*}
\scnaddlevel{1}
\scnidtf{быть задачей, решаемой заданной кибернетической системой*}
\scnaddlevel{-1}
\scnsubdividing{задача, решаемая кибернетической системой во внешней среде\\
	\scnaddlevel{1}
	\scnidtf{внешняя задача кибернетической системы}
	\scnidtf{задача, направленная на изменение состояния внешней среды соответствующей кибернетической системы, но включающая в себя (в качестве подзадач) задачи, решаемые в памяти кибернетической системы, например: 
		\begin{scnitemize}
			\item интерфейсные задачи (анализ первичный информации о текущем состоянии внешней среды),
			\item cенсо-моторную координацию выполнения сложных действий во внешней среде, состоящих из большого количества частных (более простых) действий, находящихся на разных уровнях иерархии,
			\item задачи планирования целенаправленного поведения во внешней среде,
			\item задачи принятия решений.
		\end{scnitemize}}
	\scnaddlevel{-1}
;задача, решаемая кибернетической системой в собственной физической оболочке
;задача решаемая кибернетической системой в абстрактной памяти
	\scnaddlevel{1}
	\scnidtf{задача, полностью решаемая в памяти кибернетической системы и направленная на изменение состояния информации, хранимой в памяти кибернетической системы}
	\scnidtf{внутренняя задача кибернетической системы}
	\scnaddlevel{-1}
}

\scnheader{навык}
\scnsubset{знание}
\scnexplanation{знание частного вида, содержащее (1) некоторый метод -- знание о том, как можно решать задачи, принадлежащие соответствующему множеству задач, (2) полное знание о том, как указанный метод следует интерпретировать (реализовывать), декомпозируя исходные задачи на подзадачи и, в конечном счёте на элементарные действия, выполняемые \textit{процессором кибернетической системы}}
\scnidtf{умение}
\scnidtf{методы и средства, обеспечивающие способность \textit{кибернетической системы} решать некоторое множество задач (выполнять некоторое множество действий)}

\scnheader{интерфейс кибернетической системы}
\scnidtf{условно выделяемый компонент \textit{решателя задач кибернетической системы}, обеспечивающий решение \textit{интерфейсных задач}, направленных на \uline{непосредственную} реализацию взаимодействия \textit{кибернетической системы} с её \textit{внешней средой}}
\scnidtf{решатель интерфейсных задач кибернетической системы}
\scnrelto{обобщенная часть}{решатель задач кибернетической системы}
\scnrelboth{следует отличать}{физическое обеспечение интерфейса кибернетической системы}
\scnaddlevel{1}
\scnrelto{обобщенная часть}{физическая оболочка кибернетической системы}
\scnaddlevel{-1}

\scnheader{физическая оболочка кибернетической системы}
\scnrelfromset{обобщенная декомпозиция}{память кибернетической системы\\
;процессор кибернетической системы
;физическое обеспечение интерфейса кибернетической системы
\scnaddlevel{1}
	\scnidtf{аппаратное обеспечение интерфейса кибернетической системы с её внешней средой}
	\scnrelfromset{обобщенная декомпозиция}{сенсорная подсистема физической оболочки кибернетической системы;
	эффекторная подсистема физической оболочки кибернетической системы}
\scnaddlevel{-1}
;корпус кибернетической системы
}

\scnheader{физическая оболочка кибернетической системы}
\scnidtf{часть кибернетической системы, являющаяся "посредником"{} между её внутренней средой (памятью, в которой хранится и обрабатывается информация кибернетической системы) и её внешней средой}
\scnrelto{второй домен}{физическая оболочка кибернетической системы*}
\scnaddlevel{1}
\scniselement{бинарное отношение}
\scniselement{ориентированное отношение}
\\

\scnheader{память кибернетической системы}
\scnidtf{физическая оболочка (реализация) абстрактной \textit{памяти кибернетической системы} -- внутренней среды \textit{кибернетической системы}, в рамках которой \textit{кибернетическая система} формирует и использует (обрабатывает) информационную модель своей внешней среды} 
\scnnote{Не каждая \textit{кибернетическая система} имеет \textit{память}. В \textit{кибернетических системах}, которые не имеют \textit{памяти}, обработка информации сводится к обмену сигналами между компонентами этих систем. Появление в \textit{кибернетических системах} памяти как среды для "централизованного"{} хранения и обработки \textit{информации} является важнейшим этапом их эволюции. Дальнейшая эволюция \textit{кибернетических систем} во многом определяется:
	\begin{scnitemize}

	\item \textit{качеством памяти} как среды для хранения и обработки информации;
	\item качеством информации (информационной модели), хранимой в памяти кибернетической системы;
	 \end{scnitemize}}

\scnidtf{компонент кибернетической системы, в рамках которого \textit{кибернетическая система} осуществляет отображение (формирование информационной модели) среды своего существования, а также использование этой информационной модели для управления собственным поведением в указанной среде}
	 
\scnidtf{физическая оболочка для хранения информации, которую кибернетическая система приобретает и обрабатывает (т.е. меняет состояния этой информации)}
\scnidtf{физическая (аппаратная) реализация \uline{внутренней} среды кибернетической системы, каковой является среда "существования"{} информации, накапливаемой и непосредственно используемой решателем задач этой кибернетической системы}

\newpage
\scnnote{Сам факт появления в кибернетической системе памяти, которая (1) обеспечивает представление различного виды информации о среде, в рамках которой кибернетическая система решает различные задачи (выполняет различные действия), (2) обеспечивает хранение достаточно полной информационной модели указанной среды (достаточно полной для реализации своей деятельности), (3) обеспечивает высокую степень гибкости указанной хранимой в памяти информационной модели среды жизнедеятельности (т.е. лёгкость внесения изменений в эту информационную модель), существенно повышает уровень адаптивности кибернетической системы к различным изменениям своей среды.}
\scnnote{"появление"{} \uline{\textit{памяти}} в кибернетических системах является основным признаком перехода от "простых"{} автоматов к компьютерным системам, от роботов 1-го поколения к роботам следующих поколений}

\scnidtf{физическая реализация хранилища информации, которую приобрела (накопила) к текущему моменту соответствующая кибернетическая система}
\scnidtf{физическая оболочка внутренней абстрактной информационной среды кибернетической системы}
\scnidtf{среда хранения и обработки информации}
\scnidtf{запоминающая среда}
\scnidtf{среда хранения и обработки информационных конструкций}
\scnnote{Принципы организации памяти кибернетической системы могут быть разными (ассоциативная, адресная, структурно фиксированная/структурно перестраиваемая, нелинейная/линейная). От организации памяти во многом зависит её качество.}


\scnnote{Термины ассоциативная и адресная память используются как синонимы для обозначения класса памяти, в котором доступ к данным осуществляется на основе содержимого, а не адреса расположения данных. В отличие от памяти с произвольным доступом, поиск в памяти с адресацией по содержимому практически не зависит от количества хранимой информации.}
\scnaddlevel{1}
\scnrelto{цитата}{\scncite{Jalaleddine99}}
\scnaddlevel{-1}

\scnheader{кибернетическая система}
\scnrelfrom{уровни эволюции}{Уровни структурной эволюции кибернетических систем}
\scnaddlevel{1}
\scneqtovector{простая кибернетическая система, не имеющая памяти;
простая кибернетическая система, имеющая память;
одноуровневый коллектив, не имеющий общей памяти и состоящий из простых кибернетических систем, не имеющих памяти;
одноуровневый коллектив, не имеющий общей памяти и состоящий из простых кибернетических систем, имеющих память;
иерархический коллектив,  имеющий общую память и состоящий из простых кибернетических систем;
индивидуальная кибернетическая система\\
\scnaddlevel{1}
	\scnnote{Каждая \textit{индивидуальная кибернетическая система} содержит \textit{память}, имеющую достаточно высокий уровень качества}
\scnaddlevel{-1}
;одноуровневый коллектив индивидуальных кибернетических систем, не имеющий общей памяти
;одноуровневый коллектив индивидуальных кибернетическая систем, имеющий общую память 
;иерархический коллектив из индивидуальных кибернетических систем, не имеющий общей памяти
;иерархический коллектив из индивидуальных кибернетических систем, имеющий общую память}}
\scnaddlevel{-1}

\scnheader{процессор кибернетической системы}\\
\scnidtf{физически (аппаратно реализованный) интерпретатор хранимых в памяти кибернетической системы методов (программ), соответствующих базовой (для данной кибернетической системы) модели решения задач, т.е. такой модели решения задач, которая для данной кибернетической системы является моделью решения задач самого нижнего уровня и, следовательно, не может быть интерпретирована с помощью другой модели решения задач, используемой этой же кибернетической системой, а может быть проинтерпретирована либо путем аппаратной реализации такого интерпретатора, либо путём его программной реализации, например, на современных компьютерах, но в последнем случае, кроме собственного интерпретатора, необходимо также построить модель памяти реализуемой кибернетической системы}

\scnidtf{"физически"{} реализованные средства, обеспечивающие выполнение "элементарных"{} действий, направленных на изменение состояния памяти кибернетической системы (на изменение информации, хранимой в этой памяти)}

\scnidtf{"движок"("мотор") кибернетической системы}
\scnrelto{второй домен}{процессор кибернетической системы*}
	\scnaddlevel{1}
	\scnidtfexp{бинарное ориентированное отношения, каждая пара которого связывает знак кибернетической системы со знаком её процессора}
	\scniselement{бинарное отношение}
	\scniselement{ориентированное отношение}
	\scnaddlevel{1}

\scnheader{компьютер}\\
\scnsubset{физическая оболочка кибернетической системы}
\scnidtf{физическая оболочка искусственной кибернетической системы} 
\scnidtf{аппаратное обеспечение компьютерной системы}
\scnidtf{hardware of computer system}

\scnsuperset{компьютер для интеллектуальных систем}
\scnaddlevel{1}
\scnidtf{компьютер, ориентированный на реализацию интеллектуальных компьютерных систем}
\scnnote{Развитие рынка интеллектуальных компьютерных систем существенно сдерживается неприспособленностью современного поколения компьютеров к реализации на их основе интеллектуальных компьютерных систем. Попытки создания компьютеров, приспособленных к реализации интеллектуальных компьютерных систем, не привели к успеху, т.к. эти проекты были направлены на выполнение отдельных (частных) требований, предъявляемых к физическому (аппаратному) уровню интеллектуальных систем, что неминуемо приводило к приспособленности создаваемых компьютеров к реализации не всего многообразия интеллектуальных компьютерных систем, а только некоторых подмножеств таких систем. Указанные подмножества интеллектуальных компьютерных систем в основном определялись ориентацией на конкретные используемые модели решения интеллектуальных задач, тогда, как важнейшим фактором, определяющим уровень интеллекта кибернетических систем (в том числе, и компьютерных систем), является их универсальность в плане многообразия используемых моделей решения задач. Следовательно, компьютер для интеллектуальных компьютерных систем должен быть эффективным аппаратным интерпретатором любых моделей решения задач (как интеллектуальных задач, так и достаточно простых задач, т.к. интеллектуальная система должна уметь решать любые задачи).} 
\scnidtf{компьютер, приспособленный к реализации интеллектуальных компьютерных систем}
\scnidtf{универсальный компьютер для интеллектуальных систем}
\scnidtf{компьютер, обеспечивающий интерпретацию любых моделей решения задач}
\scnaddlevel{-1}

\bigskip
\scnendstruct 

\scnfragmentheader{Семейство отношений, заданных на множестве кибернетических систем}

\scnstartsubstruct

\scnheader{отношение, заданное на множестве кибернетических систем}
\scnhaselement{память кибернетической системы*}
\scnhaselement{процессор кибернетической системы*}
\scnhaselement{член коллектива*}
\scnhaselement{внешняя среда кибернетической системы*}
\scnhaselement{сенсор кибернетической системы*}
\scnhaselement{эффектор кибернетической системы*}
\scnhaselement{физическая оболочка кибернетической системы*}
\scnhaselement{информация, хранимая в памяти кибернетической системы*}
\scnhaselement{абстрактная память кибернетической системы*}
\scnhaselement{часть*}
\scnaddlevel{1}
\scnsuperset{встроенная кибернетическая система*}
\scnaddlevel{-1}

\scnheader{информация, хранимая в памяти кибернетической системы*} 
\scnidtf{\textit{информационная модель среды*}, в которой существует (осуществляет деятельность) соответствующая кибернетическая система*} 

\newpage
\scnnote{От того, насколько полна, адекватна (корректна) и систематизирована (структурирована) внутренняя среда кибернетической системы, зависит уровень интеллектуальности и эффективность соответствующей кибернетической системы.}

\scnheader{следует отличать*}
\scnhaselementset{решатель задач кибернетической системы*;
решатель задач кибернетической системы\\
\scnaddlevel{1}
\scnidtf{иерархическая система моделей решения задач}
\scnrelfrom{обобщённая часть}{процессор кибернетической системы}
\scnaddlevel{1}
\scnexplanation{Это реализация модели решения задач, обеспечивающей интерпретацию всех используемых моделей решения задач верхнего уровня} 
\scnaddlevel{-2}}

\scnheader{задача, решаемая кибернетической системой*}
\scnidtf{быть задачей, решаемой заданной кибернетической системой*}
\scnsuperset{задача, решаемая в памяти кибернетической системы*}
\scnaddlevel{1}
\scnidtf{внутренняя задача кибернетической системы*}
\scnaddlevel{-1}
\scnsuperset{задача, решаемая во внешней среде кибернетической системы*}

\scnheader{\textit{внешняя среда кибернетической системы*}}
\scnidtf{внешняя среда*} 
\scnnote{Понятие \textit{внешней среды кибернетической системы*} является понятием относительным, т.к. (1) разные кибернетические системы имеют в общем случае разную внешнюю среду и (2) одна кибернетическая система может входить в состав внешней среды другой кибернетической системы}
\scnidtf{быть внешней средой для заданной кибернетической системы*} 
\scniselement{бинарное отношение}
\scniselement{ориентированное отношение} 
\scntext{первый домен}{кибернетическая система}
\scnsuperset{внешняя информационная среда кибернетической системы*}
\scnaddlevel{1}
	\scnidtf{совокупность всевозможных информационных конструкций, к которым данная кибернетическая система имеет доступ и которые представлены самым различным образом (в том числе, и в памяти тех кибернетических систем (субъектов), с которыми данная система взаимодействует)*}
\scnaddlevel{-1}

\scnheader{среда кибернетической системы*}
\scnidtf{быть средой существования (жизнедеятельности) заданной (указанной, соответствующей) кибернетической системы*}
\scnnote{В общем случае среда жизнедеятельности \textit{кибернетической системы} включает в себя (1) \textit{внешнюю среду*} этой системы, (2) \textit{физическую оболочку*} этой системы и (3) её \textit{абстрактную память}, т.е. внутреннюю среду*, которая является хранилищем информационной модели всей среды} 
\scnsubdividing{внешняя среда*;физическая оболочка*;абстрактная память*}

\bigskip

\scnendstruct 

\scnendstruct \scnendsegmentcomment{Уточнение понятия кибернетической системы}
\begin{SCn}

\scnsectionheader{\nameref{sd_methods}}

\scnstartsubstruct

\scnheader{ostis-система}
\scntext{общая методика разработки}{Архитектура компьютерных систем, разрабатываемых по \textit{Технологии OSTIS} четко стратифицирована на две подсистемы:

\begin{scnitemize}
    \item \textit{базу знаний}, которая представляет собой полную семантическую модель интеллектуальной системы (которую будем называть sc-моделью интеллектуальной системы или sc-моделью базы знаний интеллектуальной системы, так как она оформляется в виде связной знаковой конструкции, принадлежащей \textit{SC-коду} – базовому языку внутреннего смыслового представления знаний в памяти \textit{ostis-систем});
    \item базовый универсальный интерпретатор семантической модели интеллектуальной системы, хранимой в ее памяти (интерпретатор sc-модели базы знаний интеллектуальной системы).
    
\end{scnitemize}

Указанные подсистемы \textit{ostis-систем} могут разрабатываться абсолютно независимо друг от друга при соблюдении четких требований, предъявляемых со стороны \textit{Технологии OSTIS} и заключающихся в согласованной одинаковой для этих подсистем трактовке синтаксиса и семантики \textit{SC-кода}, который является универсальным языком внутреннего смыслового представления знаний в памяти \textit{ostis-систем}, а также синтаксиса и семантики \textbf{\textit{Языка SCP}} (Semantic Code Programming), который является подъязыком \textit{SC-кода} и представляет собой базовый язык агентно-ориентированного программирования, ориентированный на обработку знаковых конструкций, принадлежащих \textit{SC-коду}.

Рассмотренная стратификация \textit{ostis-системы} на совместимые между собой \textit{базу знаний} и интерпретатор базы знаний, во-первых, представляет широкие возможности для самых различных вариантов реализации интерпретатора sc-моделей баз знаний (в том числе, для различных вариантов, реализации семантических компьютеров с ассоциативной графодинамической, реконструируемой памятью) и, во-вторых, дает возможность легко переносить (перезагружать) базу знаний интеллектуальной системы в память другого интерпретатора базы знаний. Последняя возможность означает платформенную независимость интеллектуальных систем, разрабатываемых по \textit{Технологии OSTIS}, поскольку различные варианты реализации интерпретаторов sc-моделей баз знаний суть не что иное, как различные варианты платформ для реализации \textit{ostis-систем}.

Таким образом, при наличии достаточно эффективного варианта реализации интерпретатора sc-моделей баз знаний разработка \textit{ostis-системы} сводится к проектированию \textit{\textbf{sc-модели ее базы знаний}}~\cite{Davydenko2018}, которая включает в себя:

\begin{scnitemize}
    \item \textit{sc-модель интегрированного решателя задач этой ostis-системы}~\cite{Shunkevich2018}, которая, в свою очередь, включает в себя:
    \begin{scnitemizeii}
        \item sc-модели классов решаемых задач (в частности, хранимые программы языков высокого уровня);
        \item scp-программы агентов обработки знаний;
    \end{scnitemizeii}
     \item  \textit{sc-модель интегрированного интерфейса ostis-системы}, который представляет собой встроенную ostis-систему, ориентированную на решение интерфейсных задач, связанных с обеспечением непосредственного взаимодействия ostis-системы с внешней средой (как невербального рецепторно-эффекторного взаимодействия, так и вербального взаимодействия с пользователями, с другими ostis-системами, с иными компьютерными системами).
\end{scnitemize}
}

\scnendstruct

\end{SCn}
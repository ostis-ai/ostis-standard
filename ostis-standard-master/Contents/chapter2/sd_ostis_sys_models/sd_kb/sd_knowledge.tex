\begin{SCn}

\scnsectionheader{\currentname}

\scnstartsubstruct

\scnrelfromlist{дочерний раздел}{Предметная область и онтология множеств
    \scnaddlevel{1}
    \scnidtf{Предметная область и онтология \textit{знаний о множествах}}
        \scnaddlevel{1}
        \scnnote{\textit{знания о множествах} являются \uline{частным видом} \textit{знаний} и, следовательно, общие свойства сущностей, описываемых знаниями, могут наследоваться \textit{Предметной областью и онтологией множеств}}
        \scnaddlevel{-1}
    \scnaddlevel{-1}
;Предметная область и онтология связок и отношений
;Предметная область и онтология параметров, величин и шкал
;Предметная область и онтология чисел и числовых структур
;Предметная область и онтология структур
;Предметная область и онтология темпоральных сущностей
;Предметная область и онтология темпоральных сущностей баз знаний ostis-систем
;Предметная область и онтология семантических окрестностей
;Предметная область и онтология предметных областей
;Предметная область и онтология онтологий
;Предметная область и онтология логических формул, высказываний и формальных теорий
;Предметная область и онтология внешних информационных конструкций и файлов ostis-систем
;Глобальная предметная область действий и задач и соответствующая ей онтология методов и технологий}

\scnheader{Предметная область знаний и баз знаний ostis-систем}
\scniselement{предметная область}
\scnsdmainclasssingle{знание}
\scnhaselementlist{исследуемый класс классов}{вид знаний;отношение, заданное на множестве знаний}

\scnheader{знание}
\scnidtf{синтаксически корректная (для соответствующего языка) и семантически целостная информационная конструкция}
\scnsubset{информационная конструкция}
    \scnaddlevel{1}
    \scniselementrole{класс объектов исследования}{\nameref{intro_lang}}
    \scnaddlevel{-1}
\scnrelfrom{покрытие}{вид знаний
    \scnidtf{Множество \uline{всевозможных} видов знаний}
    \scnnote{Тот факт, что семейство \textit{видов знаний} является \textit{покрытием} Множества всевозможных \textit{знаний}, означает то, что каждое \textit{знание} принадлежит по крайней мере одному выделенному нами \textit{виду знаний}}}

    
    
\scnheader{вид знаний}
\scnhaselement{спецификация}
    \scnaddlevel{1}
    \scnidtf{описание заданной сущности}
    \scnsuperset{спецификация материальной сущности}
    \scnsuperset{спецификация обратной сущности, не являющейся множеством}
        \scnaddlevel{1}
        \scnsuperset{спецификация геометрической точки}
        \scnsuperset{спецификация числа}
        \scnaddlevel{-1}
    \scnsuperset{спецификация множества}
        \scnaddlevel{1}
        \scnsuperset{спецификация связи}
        \scnsuperset{спецификация структуры}
        \scnsuperset{спецификация класса}
            \scnaddlevel{1}
            \scnsuperset{спецификация класса сущностей, не являющихся множествами}
            \scnsuperset{спецификация отношения}
                \scnaddlevel{1}
                \scnidtf{спецификация класса связей (связок)}
                \scnaddlevel{-1}
            \scnsuperset{спецификация класса классов}
                \scnaddlevel{1}
                \scnsuperset{спецификация параметра}
                \scnaddlevel{-1}
            \scnsuperset{спецификация класса структур}
            \scnsuperset{спецификация понятий}
                \scnaddlevel{1}
                \scnsuperset{пояснение}
                \scnsuperset{определение}
                \scnsuperset{утверждение}
                    \scnaddlevel{1}
                    \scnidtf{утверждение, описывающее свойства экземпляров (элементов) специфицируемого понятия}
                    \scnidtf{закономерность}
                    \scnaddlevel{-1}
                \scnaddlevel{-1}
            \scnaddlevel{-1}
        \scnaddlevel{-1}
    \scnsuperset{семантическая окрестность}
    \scnsuperset{однозначная спецификация}
    \scnsuperset{сравнительный анализ}
    \scnsuperset{достоинства}
    \scnsuperset{недостатки}
    \scnsuperset{структура специфицируемой сущности}
    \scnsuperset{принципы, лежащие в основе}
    \scnsuperset{обоснование предлагаемого решения}
        \scnaddlevel{1}
        \scnidtf{аргументация предлагаемого решения}
        \scnaddlevel{-1}
    \scnaddlevel{-1}

\scnhaselement{сравнение}

\scnhaselement{высказывание}
    \scnaddlevel{1}
    \scnsuperset{фактографическое высказывание}
    \scnsuperset{закономерность}
    \scnaddlevel{-1}

\scnhaselement{формальная теория}

\scnhaselement{предметная область}

\scnhaselement{предметная область и онтология
    \scnaddlevel{1}
    \scnidtf{предметная область и её онтология}
    \scnidtf{предметная область и соответствующая ей объединенная онтология}
    \scnaddlevel{-1}} 
 
\scnhaselement{метазнание}
    \scnaddlevel{1}
    \scnidtf{спецификация знания}
    \scnsuperset{аннотация}
    \scnsuperset{введение}
    \scnsuperset{предисловие}
    \scnsuperset{заключение}
    \scnsuperset{онтология}
        \scnaddlevel{1}
        \scnsuperset{онтология предметной области}
            \scnaddlevel{1}
            \scnsuperset{структурная онтология предметной области}
            \scnsuperset{теоретико-множественная онтология предметной области}
            \scnsuperset{логическая онтология предметной области}
            \scnsuperset{терминологическая онтология предметной области}
            \scnsuperset{объединенная онтология предметной области}
            \scnaddlevel{-1}
        \scnaddlevel{-1}
    \scnaddlevel{-1}

\scnhaselement{задача}
    \scnaddlevel{1}
    \scnidtf{спецификация действия}
    \scnaddlevel{-1}
\scnhaselement{план}

\scnhaselement{протокол}

\scnhaselement{результативная часть протокола}

\scnhaselement{метод}

\scnhaselement{технология}

\scnhaselement{история использования предметной области и её онтологии по решению информационных задач}
\scnhaselement{история использования предметной области и её онтологии по решению задач во внешней среде}
\scnhaselement{история эволюции предметной области и её онтологии}

\scnhaselement{база знаний}
    \scnaddlevel{1}
    \scnidtf{совокупность знаний, хранимых в памяти интеллектуальной компьютерной системы и \uline{достаточных} для того, чтобы указанная система удовлетворяла соответствующим предъявляемым к ней требованиям (в частности, чтобы она имела соответствующий уровень интеллекта)}
    \scnidtf{систематизированная совокупность знаний, хранимая в памяти интеллектуальной компьютерной системы и достаточная для обеспечения целенаправленного (целесообразного, адекватного) функционирования (поведения) этой системы как в своей внешней среде, так и в своей внутренней среде (в собственной базе знаний)}
    \scnrelfromset{обобщенная декомпозиция}{согласованная часть базы знаний
        \scnaddlevel{1}
        \scnidtf{часть базы знаний, признанная коллективом авторов на текущий момент}
        \scnaddlevel{-1}
    ;история эксплуатации базы знаний;история эволюции базы знаний;план эволюции базы знаний
        \scnaddlevel{1}
        \scnidtf{система специфицированных и согласованных действий авторов базы знаний, направленных на повышение её качества}
        \scnaddlevel{-1}}
    \scnnote{Основным факторами, определяющими качество интеллектуальной компьютерной системы, являются:
    \begin{scnitemize}
        \item качественная структуризация (систематизация) и \uline{стратификация} базы знаний интеллектуальной компьютерной системы, а также
        \item систематизация и стратификация \uline{деятельности}, которая осуществляется интеллектуальной компьютерной системой и спецификация которой является важнейшей частью базы знаний этой системы (Смотрите Раздел \textit{Глобальная предметная область действий и задач и соответствующая ей онтология методов и технологий}).
    \end{scnitemize}}
    \scnaddlevel{-1}
\scnnote{Даже небольшой перечень \textit{видов знаний} свидетельствует об огромном многообразии \textit{видов знаний}}

\scnheader{знание}
\scnsubdividing{декларативное знание
    \scnaddlevel{1}
    \scnidtf{\textit{знание}, имеющее \uline{только} \textit{денотационную семантику}, которая представляется в виде семантической \textit{спецификации} системы \textit{понятий}, используемых в этом \textit{знании}}
    \scnaddlevel{-1}
;процедурное знание
    \scnaddlevel{1}
    \scnidtf{\textit{знание}, имеющее не только \textit{денотационную семантику}, но и \textit{операционную семантику}, которая представляется в виде семейства \textit{спецификаций агентов}, осуществляющих интерпретацию \textit{процедурного знания}, направленную на решение некоторой инициированной \textit{задачи}}
    \scnidtf{функционально интерпретируемое знание, обеспечивающее решение либо конкретной задачи, либо некоторого множества инициируемых задач}
    \scnsuperset{задача}
        \scnaddlevel{1}
        \scnidtf{формулировка конкретной задачи}
        \scnsuperset{декларативная формулировка задачи}
        \scnsuperset{процедурная формулировка задачи}
        \scnaddlevel{-1}
    \scnsuperset{план}
        \scnaddlevel{1}
        \scnidtf{план решения конкретной задачи}
        \scnidtf{контекст конкретной задачи, предоставляющий всю информацию для решения всех подзадач для указанной конкретной задачи}
        \scnidtf{описание системы подзадач некоторой задачи}
        \scnaddlevel{-1}
    \scnsuperset{метод}
        \scnaddlevel{1}
        \scnidtf{обобщенное описание плана решения любой задачи из некоторого заданного класса задач}
        \scnaddlevel{-1}
    \scnsuperset{навык}
        \scnaddlevel{1}
        \scnidtf{метод, детализированный до уровня элементарных подзадач}
        \scnaddlevel{-1}
    \scnaddlevel{-1}}
    
\scnheader{отношение, заданное на множестве знаний}
\scnhaselement{дочернее знание*}
    \scnaddlevel{1}
    \scnidtf{знание, которое от "материнского"{} знания наследует все описанные там свойства объектов исследования}
    \scnnote{Факт наследования свойств описываемых объектов от "материнского"{} знания подчеркивается использованием прилагательного "дочернее"{} в sc-идентификаторе данного отношения, заданного на множестве знаний}
    \scnsuperset{дочерний раздел*}
        \scnaddlevel{1}
        \scnidtf{частный раздел*}
        \scnaddlevel{-1}
    \scnsuperset{дочерняя предметная область и онтология*}
    \scnaddlevel{-1}
\scnhaselement{спецификация*}
    \scnaddlevel{1}
    \scnidtf{быть знанием, которое является спецификацией (описанием) заданной сущности}
    \scnnote{специфицируемой сущностью может быть сущность любого вида, в том числе, и другое знание}
    \scnaddlevel{-1}
\scnhaselement{онтология*}
    \scnaddlevel{1}
    \scnidtf{быть семантической спецификацией заданного знания*}
    \scnaddlevel{-1}
\scnhaselement{семантическая эквивалентность*}
\scnhaselement{следовательно*}
    \scnaddlevel{1}
    \scnidtf{логическое следствие*}
    \scnaddlevel{-1}
\scnhaselement{логическая эквивалентность*}   
    
\bigskip    
\scnendstruct \scnendcurrentsectioncomment

\end{SCn}
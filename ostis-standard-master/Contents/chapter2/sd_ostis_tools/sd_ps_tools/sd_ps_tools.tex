\begin{SCn}

\scnsectionheader{Предметная область и онтология средств поддержки проектирования решателей задач ostis-систем}
\scnreltoset{декомпозиция раздела}{Семантическая модель средств поддержки проектирования программ Базового языка программирования ostis-систем;Семантическая модель средств поддержки проектирования коллективов внутренних агентов ostis-систем;Интеллектуальная среда проектирования искусственных нейронных сетей, семантически совместимых с базами знаний ostis-систем}

\scnstartsubstruct

\scnheader{Предметная область средств поддержки проектирования решателей задач ostis-систем}
\scnsdmainclasssingle{***}
\scnsdclass{***}
\scnsdrelation{***}

\scnheader{Пользовательский интерфейс средств поддержки проектирования решателей задач ostis-систем}
\scnexplanation{Поскольку объектами проектирования описываемой системы автоматизации являются компоненты решателей задач, в частности, агенты и программы обработки знаний, представленные в \textit{SC-коде}, то в такой системе могут использоваться базовые средства внешнего представления текстов \textit{SC-кода}, например, на языках SCn или SCg.

Для того чтобы визуально упростить процесс верификации и отладки компонентов решателя, используется подход, предполагающий, что пользователю системы в каждый момент времени отображается только минимально необходимый набор \textit{sc-элементов}. Например, при отладке \textit{scp-процесса} достаточно отображать \textit{scp-операторы} и переходы между ними. При необходимости пользователь может вручную запросить и просмотреть спецификацию нужного \textit{scp-оператора} в момент останова. Указанный подход заложен в алгоритмы работы всех агентов описываемой системы.

Таким образом, в настоящее время пользовательский интерфейс системы автоматизации процесса построения и модификации решателей задач представлен набором интерфейсных команд, позволяющих пользователю инициировать деятельность нужного агента, входящего в состав этой системы.}

\scnendstruct

\end{SCn}
%%%%%%%%%%%%%%%%%%%% book.tex %%%%%%%%%%%%%%%%%%%%%%%%%%%%%
%
% sample root file for the chapters of your "monograph"
%
% Use this file as a template for your own input.
%
%%%%%%%%%%%%%%%% Springer-Verlag %%%%%%%%%%%%%%%%%%%%%%%%%%

% RECOMMENDED %%%%%%%%%%%%%%%%%%%%%%%%%%%%%%%%%%%%%%%%%%%%%%%%%%%
% \documentclass[graybox,envcountchap,sectrefs]{svmono}

\makeatletter
\def\input@path{{scn//}}
\makeatother

\documentclass{scndocument}

% \usepackage{mathptmx}
% \usepackage{helvet}
% \usepackage{courier}
% %
% \usepackage{type1cm}         
% \usepackage{setspace}

% \usepackage{xparse} 
% \usepackage[table]{xcolor}
% \usepackage{enumitem} 
% \usepackage{mathtools}
% \usepackage{ragged2e}
% \usepackage{float}
% \usepackage{amssymb}
\usepackage[algo2e]{algorithm2e}
\usepackage{algorithm}
% \usepackage{titlesec}
% \usepackage[tracking=true]{microtype}
% \usepackage{changepage}
% \usepackage{trimspaces}
% \usepackage{bm}
% \usepackage{datetime}
% \usepackage{suffix}
\usepackage{nameref}
% \usepackage[framemethod=TikZ]{mdframed}
\usepackage[normalem]{ulem}
\usepackage{longtable}
\usepackage[utf8]{inputenc}
\usepackage[T2A]{fontenc}
% \usepackage{amsmath}

\usepackage{scn}

% \renewcommand{\ULdepth}{1.8pt}

% \usepackage{makeidx}         % allows index generation
% \usepackage{graphicx}        % standard LaTeX graphics tool
%                              % when including figure files
% \graphicspath{{Figures/}}
% \usepackage{multicol}        % used for the two-column index
% \usepackage[bottom]{footmisc}% places footnotes at page bottom

\usepackage[english,main=russian]{babel}

\babelhyphenation{иден-ти-фи-ка-то-ры}

% Курсив и жирность для кириллицы
\usepackage{substitutefont}

\substitutefont{T2A}{\familydefault}{Tempora-TLF}
\makeatletter
\input{t2atempora-tlf.fd}
\DeclareFontShape{T2A}{Tempora-TLF}{m}{sc}{
    <-> ssub * Tempora-TLF/m/n
}{}

% see the list of further useful packages
% in the Reference Guide

\makeindex	       % used for the subject index
% please use the style svind.ist with
% your makeindex program

%%%%%%%%%%%%%%%%%%%%%%%%%%%%%%%%%%%%%%%%%%%%%%%%%%%%%%%%%%%%%%%%%%%%%

\usepackage{fancyhdr}
\pagestyle{fancy}
\renewcommand\headrulewidth{0pt}
\lhead{}\chead{\hspace{14.5em}{\normalfont\thepage}}\rhead{}
\cfoot{}
%\pagenumbering{arabic}
%\fancypagestyle{plain}{}
%	\fancyfoot[C]{\thepage}}
%\pagestyle{plain}

% \usepackage{geometry}
% \geometry{
%   a4paper,
%   left=20mm,
%   right=20mm,
%   top=20mm,
%   bottom=15mm,
%   %textwidth=117mm,
%   %textheight=191mm,
%   heightrounded, % <- I recommend this
%   hratio=1:1,
%   vratio=1:1,
% }

%
%DEFINES START

\newcommand{\showfont}{encoding: \f@encoding{},
  family: \f@family{},
  series: \f@series{},
  shape: \f@shape{},
  size: \f@size{}
}

\newcommand\tabsize{3em}

\newlength{\insize}
\setlength{\insize}{0.7em}

\newlength{\insizevar}
\setlength{\insizevar}{1.2em}

\newlength{\nisize}
\setlength{\nisize}{0.7em}

\newlength{\nisizevar}
\setlength{\nisizevar}{1.2em}

\newlength{\subsetsize}
\setlength{\subsetsize}{0.8em}

\newlength{\supsetsize}
\setlength{\supsetsize}{0.8em}

\newlength{\idtfsize}
\setlength{\idtfsize}{1.1em}

\newlength{\arrowsize}
\setlength{\arrowsize}{1em}

\newlength{\arrowsizevar}
\setlength{\arrowsizevar}{1.5em}

\newlength{\bulletsize}
\setlength{\bulletsize}{0.5em}

\newlength{\bulletbracketsize}
\setlength{\bulletbracketsize}{1.15em}

\newlength{\bracketsize}
\setlength{\bracketsize}{0.4em}

\newlength{\squaresize}
\setlength{\squaresize}{0.7em}

\newlength{\doublesquaresize}
\setlength{\doublesquaresize}{1.4em}

\newlength{\commentsize}
\setlength{\commentsize}{1em}

\newlength{\eqsize}
\setlength{\eqsize}{0.8em}

\newlength{\substructsize}
\setlength{\substructsize}{1.7em}

\newlength{\toclineskip}
\setlength{\toclineskip}{\baselineskip}

\newcommand{\scnsupergroupsign}{\textasciicircum}
\newcommand{\scnrolesign}{\,$^\prime$}

\newcommand{\sethind}{\setlength{\hangindent}{\dimexpr (\tabsize)*\value{hind} \relax} }
\newcommand{\sethindfromind}[1]{\setcounter{\hind}{\numexpr (\value{ind}+#1 \relax} }
\newcommand{\calctab}{\dimexpr (\tabsize)*\value{ind} \relax}
\newcommand{\settab}{\hspace{\calctab}}
\newcommand{\calcdiff}[1]{\dimexpr \tabsize-#1  \relax}
\newcommand{\makediff}[1]{\hspace{\calcdiff{#1}}}

\newcommand{\textspaced}[1]{\textls[200]{#1}}

%DEFINES END

\usepackage{tocloft}
\makeatletter
\renewcommand{\@cftmaketoctitle}{}
\makeatother
\renewcommand{\cftchapfont}{\bfseries}
\renewcommand{\cftsecfont}{\bfseries}
\renewcommand{\cftsubsecfont}{\bfseries}

\makeatletter
\renewcommand{\cftchappresnum}{\begin{lrbox}{\@tempboxa}}
\renewcommand{\cftchapaftersnum}{\end{lrbox}}
\cftsetindents{chapter}{0pt}{0pt}
\renewcommand{\cftsecpresnum}{\begin{lrbox}{\@tempboxa}}
\renewcommand{\cftsecaftersnum}{\end{lrbox}}
\cftsetindents{section}{\tabsize}{0pt}
\renewcommand{\cftsubsecpresnum}{\begin{lrbox}{\@tempboxa}}
\renewcommand{\cftsubsecaftersnum}{\end{lrbox}}
\cftsetindents{subsection}{\tabsize+\tabsize}{0pt}
\renewcommand{\cftsubsubsecpresnum}{\begin{lrbox}{\@tempboxa}}
\renewcommand{\cftsubsubsecaftersnum}{\end{lrbox}}
\cftsetindents{subsubsection}{\tabsize+\tabsize+\tabsize}{0pt}
\renewcommand{\cftparapresnum}{\begin{lrbox}{\@tempboxa}}
\renewcommand{\cftparaaftersnum}{\end{lrbox}}
\cftsetindents{paragraph}{\tabsize+\tabsize+\tabsize+\tabsize}{0pt}
\renewcommand{\cftsubparapresnum}{\begin{lrbox}{\@tempboxa}}
\renewcommand{\cftsubparaaftersnum}{\end{lrbox}}
\cftsetindents{subparagraph}{\tabsize+\tabsize+\tabsize+\tabsize+\tabsize}{0pt}
\renewcommand\cftchapafterpnum{\vspace{\toclineskip}}
\renewcommand\cftsecafterpnum{\vspace{\toclineskip}}
\renewcommand\cftsubsecafterpnum{\vspace{\toclineskip}}
\renewcommand\cftsubsubsecafterpnum{\vspace{\toclineskip}}
\renewcommand\cftparaafterpnum{\vspace{\toclineskip}}
\renewcommand\cftsubparaafterpnum{\vspace{\toclineskip}}
\makeatother

%SCN START

\usepackage{calc}

\newif\iffilemode
\filemodefalse

\makeatletter

\newcommand*{\trim}[1]{%
  \trim@spaces@noexp{#1}%
}

\newcounter{ind}
\newcounter{hind}
\newcounter{seg}
\newcounter{list_depth}

\newenvironment{SCn}{
\setcounter{ind}{0}
\setcounter{hind}{0} 
\setcounter{list_depth}{0}
\begin{flushleft}
}
{
\end{flushleft}
}

\newcommand{\scnheader}[1]{\scnresetlevel\sethind~\vspace{\parskip}\\
\textit{\textbf{#1}}\\}

\newcommand{\scnstructheader}[1]{\scnresetlevel\sethind~\vspace{\parskip}\\
\textit{\textbf{\textspaced{#1}}}\\}

\newcommand{\scnheaderlocal}[1]{\settab\textit{\textbf{#1}}\\}

\newcommand{\scnstructheaderlocal}[1]{\settab\textit{\textbf{\textspaced{#1}}}\\}

\newcommand{\scnstructidtf}[1]{\textit{\textspaced{#1}}}

\newcommand{\scnkeyword}[1]{\textit{\textbf{#1}}}

\newcommand{\scnsectionheader}[1]{\setcounter{seg}{0} \textspaced{\scnheader{\large #1}}}

\newcommand{\scnsegmentheader}[1]{
\addtocounter{seg}{1}
\ifnum\value{seg}>1
%\newpage
\bigskip
\bigskip
\else
\bigskip
\fi
\scnsegmentcaption \textspaced{\scnheader{#1}}
}

\newcommand{\scnfragmentheader}[1]{
\bigskip
\scnfragmentcaption \textspaced{\scnheader{#1}}
}

\newcommand{\scnsegmentheaderbeginning}[1]{
	\addtocounter{seg}{1}
	\ifnum\value{seg}>1
%	\newpage
	\bigskip
	\else
	\bigskip
	\fi
	\scnsegmentcaptionbeginning \textspaced{\scnheader{#1}}
}

\newcommand{\scnfragmentcaptiontext}[1]{
\begin{adjustwidth}{\commentsize}{0.5\textwidth+\commentsize}
	\justify
	\setlength{\parskip}{0.5em}
	\hspace{-\commentsize}\bfseries /*~#1~\filldots/\normalfont	
\end{adjustwidth}}

\newcommand{\scnfragmentcaption}{\bfseries /*\filldots/\hspace{\textwidth}\normalfont\vspace{\afterseglength}}

\makeatletter
\newcommand{\scnsegmentcaption}{\normalfont \bfseries /*~\textspaced{Сегмент}~\filldots/\hspace{\textwidth}\normalfont\vspace{\afterseglength}}
\makeatother

\NewDocumentCommand\scnlist{>{\SplitList{;}}m}
   {\ProcessList{#1}{\scnlistitem}}

\NewDocumentCommand\scnrellist{>{\SplitList{;}}m}
   {\ProcessList{#1}{\scnrellistitem}}

\NewDocumentCommand\scnrolerellist{>{\SplitList{;}}m}
   {\ProcessList{#1}{\scnrolerellistitem}}
   
\NewDocumentCommand\scnvarrellist{>{\SplitList{;}}m}
   {\ProcessList{#1}{\scnvarrellistitem}}
   
\NewDocumentCommand\scnvarrolerellist{>{\SplitList{;}}m}
   {\ProcessList{#1}{\scnvarrolerellistitem}}

\NewDocumentCommand\scnfilelist{>{\SplitList{;}}m}
   {\ProcessList{#1}{\scnfilelistitem}}

\NewDocumentCommand\scnlistbrackets{>{\SplitList{;}}m}
   {\ProcessList{#1}{\scnlistitembrackets}}

\NewDocumentCommand\scnfilelistbrackets{>{\SplitList{;}}m}
   {\ProcessList{#1}{\scnfilelistitembrackets}}

\newcommand\scnlistitem[1]{
\sethind
\settab{$\bullet$\makediff{\bulletsize}\itshape#1} \\
}

\newcommand\scnvarrellistitem[1]{{\textit{#1}*{\normalfont::~}}}

\newcommand\scnvarrolerellistitem[1]{{\textit{#1}\scnrolesign{\normalfont::~}}}

\newcommand\scnrellistitem[1]{{\textit{#1}*{\normalfont:~}}}

\newcommand\scnrolerellistitem[1]{{\textit{#1}\scnrolesign{\normalfont:~}}}

\newcommand\scnlistitembrackets[1]{
\sethind
\settab\hspace{0.2em}\hspace{\bracketsize}{$\bullet$\makediff{\bulletbracketsize}\itshape#1}\\
}

\newcommand\scnfilelistitembrackets[1]{
\addtocounter{hind}{1}
\begin{adjustwidth}{\calctab+\tabsize+\bracketsize}{0em}
\justify
\setlength{\parskip}{0.5em}
\hspace{-\tabsize}\hspace{0.2em}\normalfont$\bullet$\makediff{\bulletbracketsize}[#1]
\setlength{\parskip}{\baselineskip}
\end{adjustwidth}
\addtocounter{hind}{-1}
}

\newcommand\scnfilelistitem[1]{
\addtocounter{hind}{1}
\begin{adjustwidth}{\calctab+\tabsize+\bracketsize}{0em}
\justify
\setlength{\parskip}{0.5em}
\hspace{-\tabsize}\hspace{-\bracketsize}\normalfont$\bullet$\makediff{\bulletsize}[#1]
\setlength{\parskip}{\baselineskip}
\end{adjustwidth}
\addtocounter{hind}{-1}
}

\newcommand\scnfileitem[1]{\scnaddlevel{1}\scnfilelong{#1}\scnaddlevel{-1}}

\newcommand\scgfileitem[1]{\scnaddlevel{1}\scnfilescg{#1}\scnaddlevel{-1}}

\newcommand{\scnrelfrom}[2]{
\settab$\bm{\Rightarrow}$\makediff{\arrowsize}\scnrellist{#1}\\
\addtocounter{ind}{1}
\sethindfromind{1}
\sethind
\settab{\itshape#2}\\
\addtocounter{ind}{-1}
\sethindfromind{1}
}

\newcommand{\scnrelsuperset}[2]{
\settab$\bm{\supset}$\makediff{\supsetsize}\scnrellist{#1}\\
\addtocounter{ind}{1}
\sethindfromind{1}
\sethind
\settab{\itshape#2}\\
\addtocounter{ind}{-1}
\sethindfromind{1}
}

\newcommand{\scnvarrelfrom}[2]{
	\settab\textunderscore$\bm{\Rightarrow}$\makediff{\arrowsizevar}\scnvarrellist{#1}\\
	\addtocounter{ind}{1}
	\sethindfromind{1}
	\sethind
	\settab{\itshape #2}\\
	\addtocounter{ind}{-1}
	\sethindfromind{1}
}

\newcommand{\scnrelto}[2]{
\settab$\bm{\Leftarrow}$\makediff{\arrowsize}\scnrellist{#1}\\
\addtocounter{ind}{1}
\sethindfromind{1}
\sethind 
\settab{\itshape #2}\\
\addtocounter{ind}{-1}
\sethindfromind{1}
}

\newcommand{\scnvarrelto}[2]{
\settab\textunderscore$\bm{\Leftarrow}$\makediff{\arrowsizevar}\scnvarrellist{#1}\\
\addtocounter{ind}{1}
\sethindfromind{1}
\sethind 
\settab{\itshape #2}\\
\addtocounter{ind}{-1}
\sethindfromind{1}
}

\newcommand{\scnrelfromlist}[2]{
\settab$\bm{\Rightarrow}$\makediff{\arrowsize}\scnrellist{#1}\\
\addtocounter{ind}{1}
\addtocounter{hind}{1}
\addtocounter{list_depth}{1}
\ifnum\value{list_depth}=1
	\addtocounter{hind}{1}
\fi
\sethind 

\iffilemode
\scnfilelist{#2}
\else
\scnlist{#2}
\fi

\addtocounter{ind}{-1}
\ifnum\value{list_depth}=1
	\addtocounter{hind}{-1}
\fi
\addtocounter{list_depth}{-1}
\addtocounter{hind}{-1}
}

\newcommand{\scnreltolist}[2]{
\settab$\bm{\Leftarrow}$\makediff{\arrowsize}\scnrellist{#1}\\
\addtocounter{ind}{1}
\addtocounter{hind}{1}
\addtocounter{list_depth}{1}
\ifnum\value{list_depth}=1
	\addtocounter{hind}{1}
\fi
\sethind 

\iffilemode
\scnfilelist{#2}
\else
\scnlist{#2}
\fi

\addtocounter{ind}{-1}
\ifnum\value{list_depth}=1
	\addtocounter{hind}{-1}
\fi
\addtocounter{list_depth}{-1}
\addtocounter{hind}{-1}
}

\newcommand{\scnrelboth}[2]{
\settab$\bm{\Leftrightarrow}$\makediff{\arrowsize}\scnrellist{#1}\\
\addtocounter{ind}{1}
\addtocounter{hind}{1}
\sethind 
\settab{\itshape #2} \\
\addtocounter{hind}{-1}
\addtocounter{ind}{-1}
}

\newcommand{\scnrelbothlist}[2]{
\settab$\bm{\Leftrightarrow}$\makediff{\arrowsize}\scnrellist{#1}\\
\addtocounter{ind}{1}
\addtocounter{hind}{2}
\sethind 

\iffilemode
\scnfilelist{#2}
\else
\scnlist{#2}
\fi

\addtocounter{ind}{-1}
\addtocounter{hind}{-2}
}

\newcommand{\scnrelfromcommonset}[4]{
\settab$\bm{\Rightarrow}$\makediff{\arrowsize}\scnrellist{#3}\\
\addtocounter{ind}{1}
\addtocounter{hind}{1}
\addtocounter{list_depth}{1}
\ifnum\value{list_depth}=1
	\addtocounter{hind}{1}
\fi
\sethind 

\settab#1\\
\iffilemode
\vspace{-1.55\baselineskip}
\scnfilelistbrackets{#4}
\else
\vspace{-1\baselineskip}
\scnlistbrackets{#4}
\fi
\settab#2\\

\addtocounter{ind}{-1}
\ifnum\value{list_depth}=1
	\addtocounter{hind}{-1}
\fi
\addtocounter{list_depth}{-1}
\addtocounter{hind}{-1}
}

\newcommand{\scnrelfromset}[2]{\scnrelfromcommonset{$\pmb{\{}$}{$\pmb{\}}$}{#1}{#2}}

\newcommand{\scnrelfromvector}[2]{\scnrelfromcommonset{$\pmb{\langle}$}{$\pmb{\rangle}$}{#1}{#2}}

\newcommand{\scnreltocommonset}[4]{
\settab$\bm{\Leftarrow}$\makediff{\arrowsize}\scnrellist{#3}\\
\addtocounter{ind}{1}
\addtocounter{hind}{1}
\addtocounter{list_depth}{1}
\ifnum\value{list_depth}=1
	\addtocounter{hind}{1}
\fi
\sethind 

\settab#1\\
\iffilemode
\vspace{-1.55\baselineskip}
\scnfilelistbrackets{#4}
\else
\vspace{-1\baselineskip}
\scnlistbrackets{#4}
\fi
\settab#2\\

\addtocounter{ind}{-1}
\ifnum\value{list_depth}=1
	\addtocounter{hind}{-1}
\fi
\addtocounter{list_depth}{-1}
\addtocounter{hind}{-1}
}

\newcommand{\scnreltoset}[2]{\scnreltocommonset{$\pmb{\{}$}{$\pmb{\}}$}{#1}{#2}}

\newcommand{\scnreltovector}[2]{\scnreltocommonset{$\pmb{\langle}$}{$\pmb{\rangle}$}{#1}{#2}}

\newcommand{\scneq}[1]{
\addtocounter{hind}{1}
\sethind 
\settab$\bm{=}$\makediff{\eqsize}{\itshape #1}\\
\addtocounter{hind}{-1}
}

\newcommand{\scneqfile}[1]{
\settab$\bm{=}$\\ 
\vspace{0.5\baselineskip}
\scnfilelong{#1}
}

\newcommand{\scneqimage}[1]{
\settab$\bm{=}$\\ 
\vspace{0.5\baselineskip}
\scnfileimage{#1}
}

\newcommand{\scneqtable}[1]{
\settab$\bm{=}$\\ 
\vspace{0.5\baselineskip}
\scnfiletable{#1}
}

\newcommand{\scneqscg}[1]{
\settab$\bm{=}$\\ 
\vspace{0.5\baselineskip}
\scnfilescg{#1}
}

\newcommand{\scneqfileclass}[1]{
\settab$\bm{=}$\makediff{\eqsize}\scnfileclass{#1}\\
}

\newcommand{\scneqtocommonset}[3]{
\settab$\bm{=}$\makediff{\eqsize}#1\\
\addtocounter{ind}{1}
\addtocounter{hind}{1}
\addtocounter{list_depth}{1}
\ifnum\value{list_depth}=1
\addtocounter{hind}{1}
\fi
\sethind 

\iffilemode
\vspace{-1.55\baselineskip}
\scnfilelistbrackets{#3}
\else
\vspace{-1\baselineskip}
\scnlistbrackets{#3}
\fi
\settab#2\\

\addtocounter{ind}{-1}
\ifnum\value{list_depth}=1
\addtocounter{hind}{-1}
\fi
\addtocounter{list_depth}{-1}
\addtocounter{hind}{-1}
}

\newcommand{\scneqtoset}[1]{\scneqtocommonset{$\pmb{\{}$}{$\pmb{\}}$}{#1}}

\newcommand{\scneqtovector}[1]{\scneqtocommonset{$\pmb{\langle}$}{$\pmb{\rangle}$}{#1}}

\newcommand{\scnhassubstruct}[1]{
\settab$\bm{\supset=}$\makediff{\substructsize}$\pmb{\{}$\\
\addtocounter{ind}{1}
\addtocounter{hind}{2}
\sethind 
\iffilemode
\vspace{-1.55\baselineskip}
\scnfilelistbrackets{#1}
\else
\vspace{-1\baselineskip}
\scnlistbrackets{#1}
\fi
\settab$\pmb{\}}$\\
\addtocounter{ind}{-1}
\addtocounter{hind}{-2}
}

\newcommand{\scnsubset}[1]{
\addtocounter{hind}{1}
\sethind 
\settab$\bm{\subset}$\makediff{\subsetsize}{\itshape #1}\\
\addtocounter{hind}{-1}
}

\newcommand{\scnnotsubset}[1]{
\addtocounter{hind}{1}
\sethind 
\settab$\bm{\not\subset}$\makediff{\subsetsize}{\itshape #1}\\
\addtocounter{hind}{-1}
}

\newcommand{\scnsuperset}[1]{
\addtocounter{hind}{1}
\sethind 
\settab$\bm{\supset}$\makediff{\supsetsize}{\itshape #1}\\
\addtocounter{hind}{-1}
}

\newcommand{\scniselement}[1]{
\addtocounter{hind}{1}
\sethind
\settab$\bm{\in}$\hspace{\calcdiff{\insize}}{\itshape #1}\\
\addtocounter{hind}{-1}
}

\newcommand{\scnisvarelement}[1]{
	\addtocounter{hind}{1}
	\sethind
	\settab\textunderscore$\bm{\in}$\makediff{\insizevar}{\itshape #1}\\
	\addtocounter{hind}{-1}
}

\newcommand{\scnmakecommonsetlocal}[3]{
\hspace{-\bracketsize}#1\\
\addtocounter{ind}{1}
\addtocounter{hind}{2}
\sethind 
\vspace{-1\baselineskip}
\scnlistbrackets{#3}
\settab#2\\
\addtocounter{ind}{-1}
\addtocounter{hind}{-2}
}

\newcommand{\scnmakesetlocal}[1]{\scnmakecommonsetlocal{$\pmb{\{}$}{$\pmb{\}}$}{#1}}

\newcommand{\scnmakevectorlocal}[1]{\scnmakecommonsetlocal{$\pmb{\langle}$}{$\pmb{\rangle}$}{#1}}

\newcommand{\scnmakeset}[1]{
\settab$\pmb{\{}$\\
\sethind 
\vspace{-1\baselineskip}
\scnlistbrackets{#1}
\settab$\pmb{\}}$\\
}

\newcommand{\scnhaselementset}[1]{
\settab$\bm{\ni}$\makediff{\nisize}\scnmakesetlocal{#1}
}

\newcommand{\scnhaselementvector}[1]{
\settab$\bm{\ni}$\makediff{\nisize}\scnmakevectorlocal{#1}
}

\newcommand{\scnhaselement}[1]{
\addtocounter{hind}{1}
\sethind 
\settab$\bm{\ni}$\makediff{\nisize}{\itshape #1}\\
\addtocounter{hind}{-1}
}

\newcommand{\scnhasvarelement}[1]{
\addtocounter{hind}{1}
\sethind 
\settab\textunderscore$\bm{\ni}$\makediff{\nisizevar}{\itshape #1}\\
\addtocounter{hind}{-1}
}

\newcommand{\scnhaselements}[1]{
\addtocounter{hind}{1}
\sethind 
\settab$\bm{\ni}$\makediff{\nisize}#1\\
\addtocounter{hind}{-1}
}

\newcommand{\scnsubdividing}[1]{\scnrelfromset{разбиение}{#1}}

\newcommand{\scnhaselementrole}[2]{
\settab$\bm{\ni}$\makediff{\nisize}\scnrolerellist{#1}\\
\addtocounter{ind}{1}
\addtocounter{hind}{1}
\sethind 
\settab{\itshape #2}\\
\addtocounter{ind}{-1}
\addtocounter{hind}{-1}
}

\newcommand{\scnhasvarelementrole}[2]{
	\settab\textunderscore$\bm{\ni}$\makediff{\nisizevar}\scnvarrolerellist{#1}\\
	\addtocounter{ind}{1}
	\addtocounter{hind}{1}
	\sethind 
	\settab{\itshape #2}\\
	\addtocounter{ind}{-1}
	\addtocounter{hind}{-1}
}

\newcommand{\scnhaselementlist}[2]{
\settab$\bm{\ni}$\makediff{\nisize}\scnrolerellist{#1}\\
\addtocounter{ind}{1}
\addtocounter{hind}{2}
\sethind 
\scnlist{#2}
\addtocounter{ind}{-1}
\addtocounter{hind}{-2}
}

\newcommand{\scniselementrole}[2]{
\settab$\bm{\in}$\makediff{\insize}\scnrolerellist{#1}\\
\addtocounter{ind}{1}
\addtocounter{hind}{1}
\sethind 
\settab{\itshape #2}\\
\addtocounter{ind}{-1}
\addtocounter{hind}{-1}
}

\newcommand{\scniselementlist}[2]{
\settab$\bm{\in}$\makediff{\nisize}\scnrolerellist{#1}\\
\addtocounter{ind}{1}
\addtocounter{hind}{2}
\sethind 
\scnlist{#2}
\addtocounter{ind}{-1}
\addtocounter{hind}{-2}
}

\newcommand{\scnrole}[1]{{\itshape #1\scnrolesign}:}

\newcommand{\scnset}[1]{$\pmb{\{}$#1$\pmb{\}}$}
\newcommand{\scnvector}[1]{$\pmb{\langle}$#1$\pmb{\rangle}$}

\newcommand{\scnidtf}[1]{
\addtocounter{hind}{1}
\begin{adjustwidth}{\calctab+\tabsize+\bracketsize}{0em}
\justify
\setlength{\parindent}{0em}
\setlength{\itemindent}{0em}
\setlength{\parskip}{0.5em}
\vspace{-0.4\baselineskip}
\hspace{-\tabsize}\hspace{-\bracketsize}\normalfont$\bm{\coloneqq}$\makediff{\idtfsize}[#1]
\end{adjustwidth}
\addtocounter{hind}{-1}
}

\newcommand{\scnidtftext}[2]{
\settab$\bm{\coloneqq}$\hspace{\calcdiff{\idtfsize}}\scnrellist{#1}\\
\addtocounter{ind}{1}
\settab{\scnfilelong{#2}}\\
\addtocounter{ind}{-1}
}

\newcommand{\scnidtfdef}[1]{\scnidtftext{определение}{#1}}

\newcommand{\scnidtfexp}[1]{\scnidtftext{пояснение}{#1}}

\newcommand{\scnaddlevel}[1]{
\addtocounter{ind}{#1}
\addtocounter{hind}{#1}}

\newcommand{\scnaddhind}[1]{
\addtocounter{hind}{#1}}

\newcommand{\scnresetlevel}{
\setcounter{ind}{0}
\setcounter{hind}{0}}

\newcommand{\scnfileshort}[1]{
{\normalfont [#1]}
}

\newcommand{\scnfileclass}[1]{
{\normalfont\hspace{-0.3em}![~#1~]!}
}

\newcommand{\scnfilelong}[1]{
\begin{adjustwidth}{\calctab+0.4em}{0em}
\justify
\setlength{\parindent}{0em}
\setlength{\itemindent}{0em}
\setlength{\parskip}{0.5em}
\vspace{-1.5em}
\hspace{-0.4em}\normalfont \textbf{[}#1\textbf{]}
\end{adjustwidth}
}

\newcommand{\scnexternalfile}[1]{
\begin{adjustwidth}{\calctab+0.4em}{0em}
\justify
\setlength{\parindent}{0em}
\setlength{\itemindent}{0em}
\setlength{\parskip}{0.5em}
\vspace{-1.5em}
\normalfont#1
\end{adjustwidth}
}

\usepackage{tabularx}

\newcommand{\scnfileimage}[1]{
\begin{adjustwidth}{\calctab}{0em}
\setlength{\parindent}{0em}
\setlength{\itemindent}{0em}
\setlength{\parskip}{0.5em}
\vspace{-1.5em}
\normalfont \textbf{[}
\setlength\arrayrulewidth{1pt}
\hspace{0.1em}\begin{tabularx}{\textwidth-\calctab}{X}
\cellcolor{white}\begin{center}\vspace{-1.3\baselineskip}#1\vspace{-0.7\baselineskip}\end{center}
\end{tabularx}
\textbf{]}
\end{adjustwidth}
}

\newcommand{\scnfiletable}[1]{
\begin{adjustwidth}{\calctab}{0em}
\setlength{\parindent}{0em}
\setlength{\itemindent}{0em}
\setlength{\parskip}{0.5em}
\vspace{-1.5em}
\normalfont \textbf{[}
\rowcolors{1}{white}{white}
#1
\textbf{]}
\end{adjustwidth}
}

\newcommand{\scnfilescg}[1]{
\begin{adjustwidth}{\calctab}{0em}
\setlength{\parindent}{0em}
\setlength{\itemindent}{0em}
\setlength{\parskip}{0.5em}
\vspace{-1.5em}
\normalfont \textbf{[}
\setlength\arrayrulewidth{1pt}
\hspace{0.1em}\begin{tabularx}{\textwidth-\calctab}{X}
\cellcolor{white}\begin{center}\vspace{-1.3\baselineskip}\includegraphics[scale=0.8]{#1}\vspace{-0.7\baselineskip}\end{center}
\end{tabularx}
\textbf{]}
\end{adjustwidth}
}

\newcommand{\scnfilescn}[1]{
	\begin{adjustwidth}{\calctab}{0em}
		\setlength{\parindent}{0em}
		\setlength{\itemindent}{0em}
		\setlength{\parskip}{0.5em}
		\vspace{-1.5em}
		\normalfont \textbf{[}
		#1
		\textbf{]}
	\end{adjustwidth}
}

\newcommand{\scnstartsubstruct}{
\settab$\pmb{\supset=}$\\
\settab$\pmb{\{}$
}

\newcommand{\scnstructinclusion}{
\settab$\pmb{\supset=}$\\
}

\newcommand{\scnstartstruct}{\settab$\pmb{=\{}$}
\newcommand{\scnstartset}{\settab$\pmb{\{}$}
\newcommand{\scnstartsetlocal}{$\pmb{\{}$}
\newcommand{\scnendstruct}{\settab$\pmb{\}}$}

\newcommand{\scnendstructlocal}{$\pmb{\}}$}

\newcommand{\scnstartfile}{\settab\textbf{=}\\
\settab\textbf{[}\\}

\newcommand{\scnendfile}{\settab\textbf{]}}


\newcommand{\scnfilelongbreaks}[1]
{\scnfilelong{\\#1\\}}

\newcommand{\scntext}[2]{
\scnrelfrom{#1}{\scnfilelong{#2}}
}

\newcommand{\scncomment}[1]{
\scntext{комментарий}{#1}
}

\newcommand{\scnexplanation}[1]{
\scntext{пояснение}{#1}
}

\newcommand{\scnnote}[1]{
\scntext{примечание}{#1}
}

\newcommand{\scndefinition}[1]{
\scntext{определение}{#1}
}

\newcommand{\scnevolution}[1]{
\scntext{эволюция}{#1}
}

\newcommand{\scnproblems}[1]{
\scntext{проблемы}{#1}
}

\newcommand{\scnevolutiondirections}[1]{
\scntext{направления эволюции}{#1}
}

\newcommand{\scnevolutionproblems}[1]{
\scntext{проблемы развития}{#1}
}

\newcommand{\scnmodernstate}[1]{
\scntext{современное состояние}{#1}
}

\newcommand{\scnsolutionapproach}[1]{
\scntext{предлагаемый подход к решению}{#1}
}

\newcommand{\scnadvantages}[1]{
\scntext{достоинства}{#1}
}

\newcommand{\scnprinciples}[1]{
\scntext{принципы}{#1}
}

\newcommand{\scnspheresapplication}[1]{
\scntext{сферы применения}{#1}
}

\newcommand{\scnseminclusion}[1]{
\scntext{семантическое включение}{#1}
}

\newcommand{\scnsourcecomment}[1]{{\normalfont \normalsize \textbf{/*~}#1\textbf{{}~*/}}}

\newcommand{\scnsourcecommentpar}[1]{
\begin{adjustwidth}{\calctab+1em}{0em}
\justify
\setlength{\parindent}{0em}
\setlength{\itemindent}{0em}
\hspace{-1em}\normalfont /*~#1{}~*/
\end{adjustwidth}
}

\newcommand{\scninlinesourcecommentpar}[1]{
\begin{adjustwidth}{\calctab+1em+\tabsize}{0em}
\justify
\setlength{\parindent}{0em}
\setlength{\itemindent}{0em}
\vspace{-0.9\baselineskip}
\hspace{-1em}\normalfont /*~#1{}~*/
\end{adjustwidth}
}

\makeatletter
\newcommand{\scnendcurrentsectioncomment}{
\scninlinesourcecommentpar{Завершили Раздел ``\textit{\@currentlabelname}''}
}
\makeatother

\newcommand{\scnendsegmentcomment}[1]{
	\scninlinesourcecommentpar{Завершили Сегмент ``\textit{#1}''}
}

\newcommand{\scnendfragmentcomment}{\\
{\bfseries \normalsize /*\filldots/\hspace{0.5\textwidth}}
}

\newcommand{\scnauthorcomment}[1]{\begin{flushleft}
/*/*#1*/*/\end{flushleft}}

\newenvironment{FileFrame}{
\begin{mdframed}[linewidth=0.5mm,roundcorner=0pt]
}
{
\end{mdframed}
}

\newenvironment{ContourFrame}{
\begin{mdframed}[linewidth=0.5mm,roundcorner=20pt]
}
{
\end{mdframed}
}

%Subject domains
\newcommand{\scnsdmainclass}[1]{
\scnhaselementlist{максимальный класс объектов исследования}{#1}
}

\newcommand{\scnsdmainclasssingle}[1]{
\scnhaselementrole{максимальный класс объектов исследования}{#1}
}

\newcommand{\scnsdclass}[1]{
\scnhaselementlist{класс объектов исследования}{#1}
}

\newcommand{\scnsdrelation}[1]{
\scnhaselementlist{исследуемое отношение}{#1}
}

%Developer info

\newcommand{\timestamp}{\vspace{0.5em}\scnauthorcomment{\textbf{\large Версия от \today~ \currenttime}}}

\newcommand{\addedstart}{\reversemarginpar\marginpar{\hspace{3em}НОВОЕ~[}}

\newcommand{\addedend}{\reversemarginpar\marginpar{\hspace{3em}]~НОВОЕ}}

\newcommand{\editedstart}{\reversemarginpar\marginpar{\hspace{3em}ИЗМЕНЕНО~[}}

\newcommand{\editedend}{\reversemarginpar\marginpar{\hspace{3em}]~ИЗМЕНЕНО}}

\newcommand{\scnbigspace}{~~}

\makeatother

%SCN END

%Sections START

\titleformat{\chapter}[display]
{\normalfont\bfseries}{}{0pt}{\normalsize}

\titleformat{\section}[display]
{\normalfont\bfseries}{}{0pt}{\normalsize}

\titleformat{\subsection}[display]
{\normalfont\bfseries}{}{0pt}{\normalsize}

\titleformat{\subsubsection}[display]
{\normalfont\bfseries}{}{0pt}{\normalsize}

\titleformat{\paragraph}[display]
{\normalfont\bfseries}{}{0pt}{\normalsize}

\titlespacing*{\chapter}
{0em}{0em}{0em}
\titlespacing*{\section}
{0em}{0em}{0em}
\titlespacing*{\subsection}
{0em}{0em}{0em}
\titlespacing*{\subsubsection}
{0em}{0em}{0em}
\titlespacing*{\paragraph}
{0em}{0em}{0em}

\newcommand{\filldots}{\xleaders\hbox{*}\hfill}
\newlength\afterparlength
\setlength{\afterparlength}{-2em}

\newlength\afterseglength
\setlength{\afterseglength}{-1em}

\newcommand{\scsuperchapter}{{\normalfont\bfseries\large/*\filldots/}\vspace{0.5\afterparlength}}

\newcommand{\scchapter}[2][]{\clearpage\chapter[\textit{#2}]{/*~\textspaced{\large Раздел}~\filldots/\vspace{\afterparlength}}
\ifx&#1&
\else
\addtocontents{toc}{\vspace{\dimexpr -\toclineskip-0.4\baselineskip \relax}\parbox{\textwidth}{\begin{SCn}\protect#1\end{SCn}}
\vspace{\toclineskip}}
\fi}

\newcommand{\scsection}[2][]{\clearpage
\section[\textit{#2}]{/*~\textspaced{\large Раздел}~\filldots/\vspace{\afterparlength}}
\ifx&#1&
\else
\addtocontents{toc}{\vspace{\dimexpr -\toclineskip+0.1\baselineskip \relax}\parbox{\textwidth}{\begin{adjustwidth}{\tabsize}{0em}\begin{SCn}\protect#1\end{SCn}\end{adjustwidth}}
\vspace{\toclineskip}}
\fi}

\newcommand\scsubsection[2][]{\clearpage
\subsection[\textit{#2}]{/*~\textspaced{\large Раздел}~\filldots/\vspace{\afterparlength}}
\ifx&#1&
\else
\addtocontents{toc}{\vspace{\dimexpr -\toclineskip+0.1\baselineskip \relax}\parbox{\textwidth}{\begin{adjustwidth}{\tabsize+\tabsize}{0em}\begin{SCn}\protect#1\end{SCn}\end{adjustwidth}}
\vspace{\toclineskip}}
\fi}

\newcommand{\scsubsubsection}[2][]{\clearpage
\subsubsection[\textit{#2}]{/*~\textspaced{\large Раздел}~\filldots/\vspace{\afterparlength}}
\ifx&#1&
\else
\addtocontents{toc}{\vspace{\dimexpr -\toclineskip+0.1\baselineskip \relax}\parbox{\textwidth}{\begin{adjustwidth}{\tabsize+\tabsize+\tabsize}{0em}\begin{SCn}\protect#1\end{SCn}\end{adjustwidth}}
\vspace{\toclineskip}}
\fi}

\newcommand{\scparagraph}[2][]{\clearpage\paragraph[\textit{#2}]{/*~\textspaced{\large Раздел}~\filldots/\vspace{\afterparlength}}
\ifx&#1&
\else
\addtocontents{toc}{\vspace{\dimexpr -\toclineskip+0.1\baselineskip \relax}\parbox{\textwidth}{\begin{adjustwidth}{\tabsize+\tabsize+\tabsize+\tabsize}{0em}\begin{SCn}\protect#1\end{SCn}\end{adjustwidth}}
\vspace{\toclineskip}}
\fi}

\makeatletter
\newcommand*{\currentname}{\@currentlabelname}
\makeatother

\makeatletter
\newcommand*{\currentnumber}{\@currentlabel}
\makeatother

%Sections END

% Itemize START

\newenvironment{scnitemize}{
\begin{itemize}[leftmargin=\leftmargin-1em+\bracketsize,itemsep=-0.25em,before=\vspace{-0.8em},after=\vspace{-0.5em}]
\renewcommand{\labelitemi}{\scriptsize$\square$}
\renewcommand{\labelitemii}{\scriptsize$\square~\square$}
\renewcommand{\labelitemiii}{\scriptsize$\square~\square~\square$} 
}
{
\end{itemize}
}

\newenvironment{scnitemizeii}{
\begin{itemize}[leftmargin=\leftmargin-1em+\bracketsize+\squaresize,itemsep=-0.25em,before=\vspace{-0.3em},after=\vspace{0em}]
\renewcommand{\labelitemi}{\scriptsize$\square~\square$} 
}
{
\end{itemize}
}

\newenvironment{scnitemizeiii}{
\begin{itemize}[leftmargin=\leftmargin-1em+\bracketsize+\doublesquaresize,itemsep=-0.25em,before=\vspace{-0.3em},after=\vspace{0em}]

\renewcommand{\labelitemi}{\scriptsize$\square~\square$} 
}
{
\end{itemize}
}

\newenvironment{scnenumerate}{
\begin{enumerate}[leftmargin=\leftmargin-1em+\bracketsize,itemsep=-0.25em,before=\vspace{-0.8em},after=\vspace{-0.5em}]
}
{
\end{enumerate}
}

% Itemize END


% Background START

\usepackage[pages=all]{background}
\usepackage{tikz}
\usetikzlibrary{calc} 
\usepackage{layout}

\newcommand{\drawvlines}{
\begin{tikzpicture}[remember picture,overlay]

\coordinate (NW) at ([
            xshift=0.36in,
            yshift=-1.1in-2\voffset-\topmargin-\headheight-\headsep,
        ]current page.north west);
        
\coordinate (SW) at ([
            xshift=0.36in,
            yshift=-1.1in-2\voffset-\topmargin-\headheight-\headsep-\textheight,
        ]current page.north west);

\foreach \i in {1,2,...,18}{
%RELEASE \draw[black!25] ($(NW)+(\i*\tabsize,0)$) -- ($(SW)+(\i*\tabsize,0)$);} 
\draw[black!50] ($(NW)+(\i*\tabsize,0)$) -- ($(SW)+(\i*\tabsize,0)$);}

\end{tikzpicture}
}

\newcommand\DeactivateBG{\backgroundsetup{contents={}}}
\newcommand\ActivateBG{\backgroundsetup{contents={\drawvlines}}}

\backgroundsetup{
scale=1,
color=black,
opacity=1.0,
angle=0
}

% Background END

% SCs 

\newcommand{\scsrelto}[2]{
$\bm{\Leftarrow}$~{\itshape #1*}{\normalfont :}~{\itshape #2}}

\newcommand{\scsabrreviation}[2]{\scnfileclass{#1}\scsrelto{является сокращением}{\scnfileclass{#2}}
}


% SCs END
% 
\usepackage{expl3}
\usepackage{etoolbox}
\usepackage[
style=ieee,
citestyle=authoryear,
maxnames=3
]{biblatex}


\ExplSyntaxOn

\clist_new:N \scncitenames
\prop_new:N \scncitelist

\NewDocumentCommand \addcite {m m} 
{
    \prop_if_in:NnTF \scncitelist {#1}{}{
        \prop_gput:NnV \scncitelist {#1}{#2}
        \clist_gput_right:Nn \scncitenames {#1}
    }
}
\NewDocumentCommand \printscnbiblio {} 
{
    \clist_sort:Nn \scncitenames
    {
        \ifnumcomp{\pdfstrcmp{##1}{##2}}{>}{0}
        { \sort_return_swapped: }
        { \sort_return_same: }
    }
    \clist_map_inline:Nn \scncitenames {\prop_item:Nn \scncitelist {##1}}
}
\ExplSyntaxOff

\newcommand{\scncite}[1]{
\scncitecommon{#1}{\cite{#1}}
}

\newcommand{\scncitecommon}[2]{
\hspace{-0.6em}\textit{#2}\hspace{-0.8em}
\addcite{#2}{\printscncite{#1}{#2}}
}

\newcommand{\printscncite}[2]{
\scnheader{#2}
\scntext{библиографическая ссылка}{\fullcite{#1}}
}

\newcommand{\scnciteannotation}[1]{
	\scntext{аннотация}{\citefield{#1}{annotation}}
}

\newcommand{\scnidtfdef}[1]{%
  \scnidtftext{определение}{#1}
}

\newcommand{\scnidtfexp}[1]{%
  \scnidtftext{пояснение}{#1}
}

\newcommand{\scncomment}[1]{%
  \scntext{комментарий}{#1}
}

\newcommand{\scnexplanation}[1]{%
  \scntext{пояснение}{#1}
}

\newcommand{\scnnote}[1]{%
  \scntext{примечание}{#1}
}

\newcommand{\scndefinition}[1]{%
  \scntext{определение}{#1}
}

\newcommand{\scneditor}[1]{%
  \scnrelfrom{автор}{#1}
}

\newcommand{\scneditors}[1]{%
  \scnrelfrom{автор}{#1}
}

\newcommand{\scnmonographychapter}[1]{%
  \scnrelfrom{глава в монографии}{#1}
}

\newenvironment{scnsdmainclass}{%
  \begin{scnhaselementrolelist}{максимальный класс объектов исследования}
}{%
  \end{scnhaselementrolelist}
}

\newcommand{\scnsdmainclasssingle}[1]{%
  \scnhaselementrole{максимальный класс объектов исследования}{#1}
}

\newenvironment{scnsdclass}{%
  \begin{scnhaselementrolelist}{класс объектов исследования}
}{%
  \end{scnhaselementrolelist}
}

\newenvironment{scnsdrelation}{%
  \begin{scnhaselementrolelist}{исследуемое отношение}
}{%
  \end{scnhaselementrolelist}
}

\newenvironment{scnsubdividing}{%
  \begin{scnrelfromset}{разбиение}
}{%
  \end{scnrelfromset}
}

\newlist{textitemize}{itemize}{3}
\setlist[textitemize,1]{labelsep=\tabsize-\bulletsize-0.5\bracketsize,leftmargin=\tabsize-0.5\bracketsize,before=\vspace{-0.5em},label=$\bullet$}
\setlist[textitemize,2]{labelsep=\tabsize-\bulletsize,leftmargin=\tabsize,before=\vspace{-0.5em},label=$\bullet$}
\setlist[textitemize,3]{labelsep=\tabsize-\bulletsize-0.5\bracketsize,leftmargin=\tabsize-0.5\bracketsize,before=\vspace{-0.5em},label=$\bullet$}

\setlist[enumerate]{labelsep=\tabsize-0.7em,leftmargin=\tabsize,itemsep=-0.5em,before=\vspace{-0.5em}}

\newcommand{\myuline}[1]{%
	\uline{\phantom{#1}}%
	\llap{\contour{white}{#1}}%
}
\addbibresource{Contents/biblio/biblio.bib}

\begin{document}

\pagenumbering{arabic}

\selectlanguage{russian}
\DeactivateBG
\title{\centering
    Стандарт\\
    \textls[90]{открытой ~технологии}\\ онтологического проектирования,\\
    \textls[55]{производства ~и ~эксплуатации}\\ \textls[28]{семантически
        совместимых гибридных}\\ интеллектуальных компьютерных систем\\
    \bigskip
    \bigskip
    {\LARGE Стандарт Технологии OSTIS -- Open Semantic Technology for Intelligent
        Systems}\\
    \bigskip
    \bigskip
    {\Large Издание второе, исправленное и дополненное}}
\maketitle

%\mainmatter%%%%%%%%%%%%%%%%%%%%%%%%%%%%%%%%%%%%%%%%%%%%%%%%%%%%%%
\normalsize

\setcounter{page}{3}

\ActivateBG
%\scseparatedfragment[\protect\scnidtf{Предисловие Стандарта OSTIS-2021}]{Предисловие к первому изданию Стандарта OSTIS}

\begin{SCn}

\scnsectionheader{Предисловие к первому изданию Стандарта OSTIS}
\scneqfile{Первое издание \textit{Стандарта OSTIS} занимает особое место:
		\begin{scnitemize}
		\item Во-первых, это первый опыт издания (публикации) подобного документа, в рамках которого необходимо обеспечивать, с одной стороны, строгую формальность, а, с другой стороны, интуитивное и адекватное понимание формальных текстов со стороны читателей;
		\item Во-вторых, данный текст является описанием условно выделенной первой версии \textit{Стандарта OSTIS} (\textit{Стандарта OSTIS-2021}), в рамках которого представлены далеко не все разделы \textit{Стандарта OSTIS}. Эти разделы будут представлены в последующих версиях \textit{Стандарта OSTIS} (в \textit{Стандарте OSTIS-2022}, в \textit{Стандарте OSTIS-2023} и т.д.);
		\item Особенностью \textit{публикации} (издания) Стандарта OSTIS версии OSTIS-2021, как, впрочем, и всех последующих версий, является то, что она оформлена в виде \uline{внешнего представления} основной части \textit{базы знаний} специальной \textit{ostis-системы}, которая предназначена для комплексной поддержки проектирования \uline{семантически совместимых} \textit{ostis-систем}. Эту систему мы назвали \textit{Метасистемой OSTIS} (Intelligent MetaSystem for ostis-systems). Последовательность изложения материала во внешнем представлении \textit{базы знаний} не является единственно возможным маршрутом прочтения (просмотра) \textit{базы знаний}. Каждый читатель, войдя в \textit{Метасистему OSTIS}, может выбрать любой другой маршрут навигации по этой \textit{базе знаний}, задавая указанной метасистеме те \textit{вопросы}, которые в текущий момент его интересуют. Таким образом, читая предлагаемый вашему вниманию текст и одновременно работая с \textit{Метасистемой OSTIS}, можно значительно быстрее усвоить детали \textit{Технологии OSTIS} и значительно быстрее приступить к непосредственному использованию указанной технологии. Этому также способствует большое количество примеров семантических моделей различных фрагментов \textit{интеллектуальных компьютерных систем};
		\item Основной семантический вид \textit{разделов баз знаний ostis-систем} --- это формальное представление различных \textit{предметных областей} вместе с соответствующими им \textit{онтологиями}. При этом явно указываются связи между этими \textit{предметными областями} и \textit{онтологиями}. Таким образом, \textit{база знаний} \textit{Метасистемы OSTIS}, как и любых других \textit{интеллектуальных компьютерных систем}, построенных по \textit{Технологии OSTIS}, представляет собой иерархическую систему связанных между собой формальных моделей \textit{предметных областей} и соответствующих им \textit{онтологий}. Соответственно этому структурирован и текст \textit{Стандарта OSTIS-2021};
		\item В основе \textit{Технологии OSTIS} лежит предлагаемая нами унификация \textit{интеллектуальных компьютерных систем}, основанная, в свою очередь, на \textit{смысловом представлении знаний} в \textit{памяти интеллектуальных компьютерных систем}. Таким образом, данную \textit{публикацию} \textit{Стандарта OSTIS-2021} можно рассматривать как версию \textit{стандарта} семантических моделей \textit{интеллектуальных компьютерных систем}. Последующие \textit{публикации}, посвящённые детальному описанию различных компонентов \textit{Технологии OSTIS}, будут также оформляться как внешнее представление соответствующих \textit{разделов базы знаний} \textit{Метасистемы OSTIS} и будут отражать следующие этапы развития \textit{Технологии OSTIS}, следующие версии этой технологии, и, соответственно, следующие версии \textit{Метасистемы OSTIS};
		\item Все основные положения \textit{Технологии OSTIS} рассматривались и обсуждались на ежегодных \textit{конференциях OSTIS}, которые стали важным стимулирующим фактором становления и развития \textit{Технологии OSTIS}. Мы благодарим всех активных участников этих конференций;
		\item Важной задачей \textit{Стандарта OSTIS-2021} была выработка стилистики формализованного представления научно-технической информации, которая одновременно была бы понятна как человеку, так и интеллектуальной компьютерной системе. По сути это принципиально новый подход к оформлению научно-технических результатов, позволяющий:
		\begin{scnitemizeii}
			\item существенно повысить уровень автоматизации анализа качества (корректности, целостности) научно-технической информации;
			\item интеллектуальным компьютерным системам непосредственно (без какой-либо дополнительной {} доработки) использовать информацию (знания), содержащуюся в разработанных специалистами документах;
			\item существенно упростить согласование точек зрения различных специалистов, входящих в коллектив разработчиков той или иной научно-технической документации.
		\end{scnitemizeii}
		\item Для выработки стилистики и формального представления научно-технической информации нам было важно привлечь к обсуждению и анализу материала \textit{Стандарта OSTIS-2021} как можно больше коллег, участвующих в развитии и применении \textit{Технологии OSTIS}. При этом некоторых коллег мы включили в число соавторов соответствующих разделов монографии.
		Основной целью написания \textit{Стандарта OSTIS-2021} является создание технологических и организационных предпосылок к принципиально новому  подходу к организации \textit{научно-технической деятельности} в любой области и, в частности, в области создания и перманентного развития комплексной технологии проектирования и производства семантически совместимых интеллектуальных компьютерных систем (\textit{Технологии OSTIS}). Суть указанного подхода заключается  в глубокой конвергенции и интеграции результатов деятельности всех специалистов, участвующих в создании и развитии \textit{Технологии OSTIS}, путем организации коллективной разработки \textit{базы знаний}, являющейся формальным представлением полной \textit{Документации Технологии OSTIS}, отражающей текущее состояние этой технологии.
	\end{scnitemize}
В настоящее время уровень требований, предъявляемых к комплексу \textit{технологий Искусственного интеллекта} существенно повысился --- возникла необходимость разработки \textit{интеллектуральных компьютерных систем}, которые не только обладают высоким уровнем \textit{интеллекта}, но и обладают \textit{семантической совместимостью}, взаимопониманием, способностью координировать свою деятельность с другими системами при коллективном решении сложных \scnqq{внештатных} задач. Очевидно, что эти требования предполагают существенное развитие  \textit{стандартов интеллектуальных компьютерных систем}. Важнейшая особенность \textit{стандарта интеллектуальных компьютерных систем}, обеспечивающего их \textit{семантическую совместимость} и взаимопонимание, заключается в том, что для этого \textit{интеллектуальные компьютерные системы} должны использовать:
\begin{scnitemize}
	\item   один и тот же язык внутреннего представления знаний;
	\item   один и тот же язык их коммуникации;
	\item   одну и ту же \textit{систему понятий};
	\item   обладать достаточно большим количеством  общих (одинаковых) знаний.
\end{scnitemize}

Следовательно, разработка \textit{стандарта интеллектуальных компьютерных систем} в той части этого \textit{стандарта}, которая связана с выделением и формализацией общих (одинаковых) \textit{знаний}, хранимых в \textit{памяти интеллектуальной компьютерной системы} и необходимых для обеспечения их взаимопонимания, фактически осуществляет разработку достаточно большой  одинаковой части всех \textit{интеллектуальных компьютерных систем} (и не только прикладных),  что существенно  сокращает сроки их разработки.

К этому можно добавить возможность и целесообразность  одинаковой  для всех \textit{интеллектуальных компьютерных систем} реализации целого ряда их способностей: 
\begin{scnitemize}
	\item способности \uline{понимать} информацию, которой они обмениваются между собой,
	\item способности  \uline{договариваться} и \uline{координировать} свои действия при коллективном решении сложных интеллектуальных задач в условиях нештатных (аномальных, нестандартных, непредусмотренных) ситуаций,
	\item способности \uline{принимать решения} на основе их глубокого  обоснования,
	\item способности \uline{обучаться} и многие другие способности, обеспечивающие необходимый уровень интеллекта разрабатываемых интеллектуальных компьютерных систем.
\end{scnitemize}

Данная монография является первым этапом на пути комплексного решения указанных выше проблем.
Предназначена она одновременно:
\begin{scnitemize}
	\item для студентов, магистрантов и аспирантов, обучающихся по специальности \scnqqi{\textit{Искусственный интеллект}};
	\item для разработчиков прикладных \textit{интеллектуальных компьютерных систем};
	\item для разработчиков технологий проектирования и производства \textit{интеллектуальных компьютерных систем};
	\item для научных работников, создающих новые \textit{модели} и \textit{методы} для решения \textit{интеллектуальных задач}.
\end{scnitemize}

Данная монография сочетает:
\begin{scnitemize}
	\item строгую формализацию представляемой \textit{информации} и её доступность (возможность первичного её понимания без предварительного изучения используемого \textit{\uline{формального} языка});
	\item традиционную (\scnqq{пассивную}) форму представления материала (в \scnqq{электронном} и \scnqq{бумажном} виде) с \scnqq{активной} формой в виде \textit{интеллектуальной справочной системы}, когда \textit{компьютерная система} не только обеспечивает оперативное \textit{редактирование информации}, но и помогает пользователям  быстрее и  качественнее усваивать эту \textit{информацию} (за счет возможности отвечать на широкий спектр вопросов и учитывать индивидуальные особенности, потребности и интересы \textit{пользователей}). 
\end{scnitemize}

	Область \textit{Искусственного интеллекта} сочетает в себе как  научно-исследовательский аспект и   создание \textit{технологий} разработки \textit{интеллектуальных компьютерных систем}, а так и непосредственно разработку самих \textit{интеллектуальных компьютерных систем}. Эта область развивается настолько быстрыми темпами, что за время обучения студентов и магистрантов ситуация в области \textit{Искусственного интеллекта} меняется существенно, поэтому подготовка специалистов в этой области требует особого подхода, учитывающего высокий уровень сложности этой \textit{научно-технической области}, а также быстрые темпы развития теории \textit{интеллектуальных компьютерных систем}, технологий их проектирования, а также непосредственно практики разработки конкретных прикладных \textit{интеллектуальных компьютерных систем}.
	
	Если специалист в области \textit{Искусственного интеллекта} не будет постоянно ориентироваться в тенденциях развития каждого из этих направлений развития работ в этой области, то он быстро перестанет быть конкурентноспособным. Это значит, что специалист в области \textit{Искусственного интеллекта} должен быть в достаточной степени и ученым, и создателем технологий следующего поколения, и разработчиком конкретных приложений.
	
	Таким образом, подготовку специалистов в области \textit{Искусственного интеллекта} необходимо ориентировать не на конкретное состояние науки, технологии и практики в этой области, а на перманентный процесс эволюции всех этих направлений.
	
	Сформировать у студентов и магистрантов реальные навыки в области \textit{Искусственного интеллекта} можно только путем поэтапного и непосредственного их включения в  реальную деятельность в этой области (и в \textit{научно-исследовательскую деятельность}, и в развитие \textit{технологий искусственного интеллекта}, и в разработку \textit{прикладных интеллектуальных компьютерных систем} на основе текущего состояния соответствующих технологий). Но для этого необходимо создать соответствующую научно-исследовательскую и инженерную инфраструктуру.
	
	\textit{Научно-исследовательская деятельность  в области Искусственного интеллекта} предполагает исследование феномена \textit{интеллекта} и создание принципиально новых подходов (моделей и методов) к решению \textit{интеллектуальных задач} и к разработке принципов организации соответствующих \textit{компьютерных систем}.
	
	\uline{\textit{Развитие технологий Искусственного интеллекта}} включает в себя:
	\begin{scnitemize}
		\item разработку стандарта интеллектуальных компьютерных систем, соответствующего текущему состоянию \textit{технологий искусственного интеллекта};
		\item разработку методов, средств проектирования и реализации интеллектуальных компьютерных систем.
	\end{scnitemize}
	
	\textit{Разработка прикладных интеллектуальных компьютерных} систем  предполагает грамотное применение соответствующих \textit{технологий}.
	
	
	\textit{Учебную деятельность в области Искусственного интеллекта} необходимо ориентировать не только на формирование навыков разработки \textit{прикладных интеллектуальных компьютерных систем} по заданной \textit{технологии}, но и на формирование навыков перманентного совершенствования как непосредственно самих \textit{прикладных интеллектуальных компьютерных систем}, так и \textit{технологий} их разработки, а также на изучение принципов (моделей и методов) решения \textit{интеллектуальных задач} и организации \textit{интеллектуальных систем}.



Данная монография рассматривается нами как первый выпуск целой серии коллективных монографий, которые будут представлять последующие версии \textbf{\textit{Стандарта Технологии OSTIS}} (Open Semantic Technology for Intelligent Systems --- Стандарта \textit{технологии}, ориентированной на разработку \textit{семантически совместимых интеллектуальных компьютерных систем}). При этом предполагается существенное расширение авторского коллектива и организация всей работы на развитие \textit{Стандарта Технологии OSTIS} как \textit{открытого проекта}, целью которого является коллективное совершенствование \textit{базы знаний}, посвященной детальному описанию этого \textit{стандарта}.

При этом при подготовке даже данного издания текущей версии \textit{Стандарта Технологии OSTIS} мы приобрели хороший опыт организации коллективной деятельности такого рода, привлекая к этой работе целый ряд аспирантов, магистрантов и студентов, а также сотрудников других организаций.

Вклад некоторых из них в ряд разделов монографии позволил включить их в число \textit{соавторов} этих разделов, что отражено непосредственно в тексте монографии.

Авторы выражают благодарность:

\begin{scnitemize}
	\item Cотрудникам кафедры Интеллектуальных информационных технологий Белорусского государственного университета информатики и радиоэлектроники и кафедры Интеллектуальных информационных технологий Брестского государственного технического университета, а также сотрудникам ОАО <<Савушкин продукт>>;
	\item студентам кафедры Интеллектуальных информационных технологий Белорусского государственного университета информатики и радиоэлектроники Банцевич К.А., Бутрину С.В., Василевской А.П., Меньковой Е.А., Жмырко А.В., Григорьевой И.В., Загорскому А.Г., Марковцу В.С., Киневичу Т.О. за оказание технической помощи при подготовке текста к печати;
	\item ООО <<Интелиджент семантик системс>> и его генеральному директору Т. Грюневальду за финансовую поддержку работ по развитию \textit{Технологии OSTIS}, а также финансовую поддержку издания \textit{Стандарта OSTIS};
	\item Рецензентам --- д-ру техн. наук, профессору Александру Николаевичу Курбацкому и д-ру техн. наук, профессору Александру Арсентьевичу Дудкину;
	\item Коллегам из Советской (ныне Российской) Ассоциации Искусственного интеллекта и коллегам из Белорусского объединения специалистов в области искусственного интеллекта;
	\item Членам Программного Комитета ежегодных \textit{конференций OSTIS}, а также всем участникам этих конференций за плодотворное и конструктивное обсуждение направлений развития семантических технологий и \textit{Технологии OSTIS} в частности.
\end{scnitemize}}

\newpage

\end{SCn}
%\scseparatedfragment[\protect\scnidtf{Предисловие Стандарта OSTIS-2022}]{Предисловие ко второму изданию Стандарта OSTIS}

\begin{SCn}

\scnsectionheader{Предисловие ко второму изданию Стандарта OSTIS}

\scneqfile{Текст предисловия...}

\newpage

\end{SCn}

%\input{Contents/title_part}

%\scsectionfamily{Часть 1 Стандарта OSTIS. Введение в интеллектуальные компьютерные системы нового поколения}
\label{part_intro}

\scsection[
\protect\scnidtf{Характеристики (параметры и способности) \textit{кибернетических систем}, определяющие их качество и, в частности, уровень их \textit{интеллекта}};
\protect\scnmonographychapter{Глава 1.1. Факторы, определяющие уровень интеллектуальности кибернетических систем. Этапы эволюции компьютерных систем в направлении повышения уровня их интеллектуальности}
]{Предметная область и онтология кибернетических систем}
\label{sd_sys_inform}
\begin{SCn}
	\scnsectionheader{Предметная область и онтология кибернетических систем}

	\begin{scnsubstruct}

		\scnheader{Предметная область кибернетических систем}
		\scniselement{предметная область}
		\begin{scnhaselementrolelist}{класс объектов исследования}
			\scnitem{кибернетическая система}
		\end{scnhaselementrolelist}

		\begin{scnhaselementrolelist}{класс объектов исследования}
			\scnitem{искусственная сущность}
			\scnitem{компьютерная система}
			\scnitem{простая кибернетическая система}
			\scnitem{индивидуальная кибернетическая система}
			\scnitem{кибернетическая система, встроенная в индивидуальную
				кибернетическую систему}
			\scnitem{многоагентная система}
			\scnitem{одноуровневый коллектив кибернетических систем}
			\scnitem{коллектив индивидуальных кибернетических систем}
			\scnitem{иерархический коллектив индивидуальных кибернетических систем}
			\scnitem{информация, хранимая в памяти кибернетической системы}
			\scnitem{абстрактная память кибернетической системы}
			\scnitem{решатель задач кибернетической системы}
			\scnitem{действие кибернетической системы}
			\scnitem{задача}
			\scnitem{задача, решаемая кибернетической системой}
			\scnitem{навык}
			\scnitem{интерфейс кибернетической системы}
			\scnitem{физическая оболочка кибернетической системы}
			\scnitem{память кибернетической системы}
			\scnitem{процессор кибернетической системы}
			\scnitem{компьютер}
			\scnitem{качество кибернетической системы}
			\scnitem{гибридная кибернетическая система}
			\scnitem{приспособленность кибернетической системы к её
				совершенствованию}
			\scnitem{гибкость кибернетической системы}
			\scnitem{производительность кибернетической системы}
			\scnitem{надежность кибернетической системы}
			\scnitem{качество физической оболочки кибернетической системы}
			\scnitem{качество памяти кибернетической системы}
			\scnitem{интеллект}
			\scnitem{образованность кибернетической системы}
			\scnitem{интеллектуальная система}
			\scnitem{кибернетическая система, основанная на знаниях}
			\scnitem{кибернетическая система, управляемая знаниями}
			\scnitem{целенаправленная кибернетическая система}
			\scnitem{обучаемая кибернетическая система}
			\scnitem{социально ориентированная кибернетическая система}
			\scnitem{интеллектуальная компьютерная система}
			\scnitem{информация}
			\scnitem{сенсорная информация}
			\scnitem{качество решателя задач кибернетической системы}
			\scnitem{обучаемость кибернетической системы}
			\scnitem{стратифицированность кибернетической системы}
			\scnitem{рефлексивность кибернетической системы}
			\scnitem{синергетическая кибернетическая система}
			\scnitem{интероперабельность кибернетической системы}
		\end{scnhaselementrolelist}

		\begin{scnhaselementrolelist}{исследуемое отношение}
			\scnitem{информация, хранимая в памяти кибернетической системы*}
			\scnitem{задача, решаемая кибернетической системой*}
			\scnitem{внешняя среда кибернетической системы*}
			\scnitem{среда кибернетической системы*}
			\scnitem{агент*}
		\end{scnhaselementrolelist}

		\scnidtf{Иерархическая система свойств (характеристик) кибернетических
			систем, определяющих общий (интегральный) уровень их качества}
		\scnidtf{Эволюционный подход к определению качества и, в частности,
			уровня интеллекта кибернетической системы}
		\scntext{аннотация}{Рассмотрена иерархическая система свойств (в т.ч.
			способностей) кибернетических систем, определяющих их качество и позволяющих
			сформулировать требования, которым должна удовлетворять высокоинтеллектуальная
			система (кибернетическая система с сильным интеллектом).Уровень качества
			кибернетических систем определяется достаточно большим набором свойств
			(параметров, характеристик) кибернетических систем, каждое из которых
			определяет уровень качества кибернетической системы в соответствующем аспекте
			(ракурсе), указывая (задавая) уровень развития конкретных  способностей и
			возможностей кибернетической системы. При этом важно подчеркнуть следующее:
			\begin{scnitemize}
				\item существенное значение имеет не столько сам набор свойств, а
				иерархия этих свойств, позволяющая уточнять (детализировать) направления
				проявления (реализации) каждого свойства
				\item существенное значение также имеет \uline{баланс} уровней развития
				различных свойств --- вклад разных свойств, обеспечивающих (определяющих)
				значение одного и того же свойства более высокого уровня иерархии, а значение
				этого свойства более высокого уровня может быть разным. Из этого следует, что
				не всегда следует акцентировать внимание на развитие некоторых свойств
				(характеристик). Нужен целостный, коллективный подход
				\item рассмотренная иерархия свойств кибернетических систем является
				общей как для естественных, так и для искусственных кибернетических систем
				\item приведенная иерархическая детализация свойств кибернетических
				систем (с помощью отношения \scnqqi{\textit{частное свойство*}} и отношения
				\textit{свойство-предпосылка*}), определяющих качество таких систем, (1) дает
				возможность четко определить направления совершенствования (развития)
				кибернетических систем и (2) дает ориентир (систему критериев) для обоснования
				конкретных предложений по совершенствованию компьютерных систем, а также для
				сравнения различных альтернативных предположений
				\item особое значение для развития кибернетических систем имеют такие
				их свойства, как стратифицированность, рефлексивность и социализация
				\item важное значение имеет не только совершенствование кибернетических
				систем в соответствии с иерархической системой их свойств, но и
				совершенствование (в том числе, детализация) самой этой иерархической системы
				свойств.
			\end{scnitemize}}

		\scntext{предисловие}{Свойства (способности), которым должны
			удовлетворять \textit{интеллектуальные системы}, рассматриваются в целом ряде
			публикаций. Тем не менее, для \uline{практической} реализации
			\textit{компьютерных систем}, обладающих указанными свойствами (способностями),
			т.е. \textit{интеллектуальных компьютерных систем}, необходимо детализировать
			(уточнить) эти \textit{свойства}, пытаясь свести их к более конструктивным,
			прозрачным и понятным для реализации свойствам.}

		\begin{scnrelfromset}{рассматриваемые вопросы}
			\scnfileitem{По каким свойствам (параметрам, характеристикам,
					способностям) кибернетических систем можно оценивать уровень их качества.}
			\scnfileitem{Можно ли считать уровень развития какого-либо
					свойства (способности) кибернетической системы, т.е. значение какого-либо ее
					параметра (характеристики) оценкой уровня качества кибернетической системы по
					соответствующему аспекту.}
			\scnfileitem{Может ли какое-либо свойство кибернетических
					систем определять (влиять на) значение сразу нескольких свойств более высокого
					уровня иерархии.}
			\scnfileitem{Какими отношениями свойства кибернетических
					систем связаны со свойствами более низкого и, соответственно, более высокого
					уровня иерархии.}
			\scnfileitem{Зачем нужна такая иерархия свойств, определяющих
					качество кибернетических систем и позволяющих детализировать (уточнять) то,
					какими свойствами определяется уровень (степень) развития каждого свойства
					(значение каждого свойства) за исключением свойств, которые условно можно
					считать элементарными, не требующими детализации (по крайнем мере, пока).}
			\scnfileitem{Может ли иерархия свойств, определяющих качество
					кибернетических систем, быть критерием оценки и выбора того или иного подхода к
					построению интеллектуальных компьютерным систем.}
			\scnfileitem{Какими свойствами (способностями) должна обладать
					кибернетическая система, имеющая высокий уровень интеллекта.}
			\scnfileitem{Какими свойствами определяется уровень интеллекта
					многоагентной кибернетической системы.}
			\scnfileitem{Как связан уровень интеллекта многоагентной
					системы с уровнем интеллекта агентов, входящих в ее состав.}
			\scnfileitem{Почему, например, не каждый коллектив
					высокоинтеллектуальных людей демонстрирует высокий уровень интеллекта самого
					коллектива.}
			\scnfileitem{Какими дополнительными свойствами кроме
					достаточно высокого уровня интеллекта должны обладать агенты многоагентных
					систем для обеспечения высокого уровня интеллекта самой многоагентной системы
					как самостоятельной целостной кибернетической системы.}
			\scnfileitem{Как зависит уровень интеллекта многоагентной
					системы от организации взаимодействия между агентами, например, от
					использования централизованного или децентрализованного управления.}
		\end{scnrelfromset}

		\begin{scnrelfromvector}{ключевые знаки}
			\scnitem{кибернетическая система}
			\begin{scnindent}
				\begin{scnsubdividing}
					\scnitem{естественная кибернетическая система}
					\scnitem{компьютерная система}
                    \begin{scnindent}
                        \scnidtf{искусственная кибернетическая система}
                    \end{scnindent}
					\scnitem{естественно-искусственная кибернетическая система}
                    \begin{scnindent}
                        \scnidtf{кибернетическая система, являющаяся симбиозом компонентов как
                            естественного, так и искусственного происхождения}
                    \end{scnindent}
				\end{scnsubdividing}
            \end{scnindent}
			\scnitem{качество кибернетической системы}
			\scnitem{физическая оболочка кибернетической системы}
			\scnitem{качество физической оболочки кибернетической системы}
			\scnitem{интеллект}
				\begin{scnindent}
					\scnidtf{уровень интеллекта кибернетическойсистемы}
					\scnidtf{интеллектуальность}
				\end{scnindent}
			\scnitem{интеллектуальная система}
				\begin{scnindent}
					\scnidtf{интеллектуальная кибернетическая система}
					\scnsuperset{интеллектуальная компьютерная система}
				\end{scnindent}
			\scnitem{информация, хранимая в памяти кибернетической системы}
			\scnitem{качество информации, хранимой в памяти кибернетической
				системы}
			\scnitem{база знаний}
			\scnitem{смысловое представление информации в памяти кибернетической
				системы}
			\scnitem{решатель задач кибернетической системы}
			\scnitem{качество решателя задач кибернетической системы}
			\scnitem{память кибернетической системы}
			\scnitem{качество памяти кибернетической системы}
			\scnitem{обучаемость кибернетической системы}
			\scnitem{гибкость кибернетической системы}
			\scnitem{стратифицированность кибернетической системы}
			\scnitem{рефлексивность кибернетической системы}
				\begin{scnindent}
					\scnidtf{уровень рефлексии кибернетической системы}
				\end{scnindent}
			\scnitem{многоагентная система}
			\scnitem{качество многоагентной системы}
			\scnitem{унифицированность агентов многоагентной системы}
			\scnitem{семантическая совместимость агентов многоагентной системы}
			\scnitem{интероперабельность кибернетической системы}
				\begin{scnindent}
					\scnidtf{способность кибернетической системы своей внутренней и внешней деятельностью обеспечивать
						высокий уровень интеллекта тех многоагентных систем, членом (агентом) которых
						она является}
				\end{scnindent}
		\end{scnrelfromvector}

		\begin{scnrelfromvector}{библиография}
			\scnitem{\scncite{Viner1952}}
			\scnitem{\scncite{Pospelov1989}}
			\scnitem{\scncite{Finn2008}}
			\scnitem{\scncite{YarushinaHS}}
			\scnitem{\scncite{RedkoV2019}}
			\scnitem{\scncite{Fry2002}}
			\scnitem{\scncite{Glushkov1979}}
			\scnitem{\scncite{Nilsson2005}}
			\scnitem{\scncite{Kerr2006}}
			\scnitem{\scncite{Antsyferov2013}}
			\scnitem{\scncite{Sherif1988}}
			\scnitem{\scncite{Cho2019}}
			\scnitem{\scncite{Melekhova2018}}
			\scnitem{\scncite{Gao2002}}
			\scnitem{\scncite{Laird2009}}
			\scnitem{\scncite{Zagorskiy2022b}}
			\scnitem{\scncite{Finn2021}}
			\scnitem{\scncite{Dorri2018}}
			\scnitem{\scncite{Ferber2003}}
			\scnitem{\scncite{Hadzic2009}}
			\scnitem{\scncite{Balaji2010}}
			\scnitem{\scncite{Ouksel1999}}
			\scnitem{\scncite{Lopes2022}}
			\scnitem{\scncite{Hamilton2006}}
			\scnitem{\scncite{Neiva2016}}
		\end{scnrelfromvector}

		\begin{scnreltovector}{конкатенация сегментов}
			\scnitem{Уточнение понятия кибернетической системы}
			\scnitem{Комплекс свойств, определяющий общий уровень качества
				кибернетической системы}
			\scnitem{Комплекс свойств, определяющих качество физической оболочки
				кибернетической системы}
			\scnitem{Комплекс свойств, определяющих уровень интеллекта
				кибернетической системы}
			\scnitem{Комплекс свойств, определяющий качество информации, хранимой в
				памяти кибернетической системы}
			\scnitem{Комплекс свойств, определяющих качество решателя задач
				кибернетической системы}
			\scnitem{Комплекс свойств, определяющих уровень обучаемости
				кибернетической системы}
			\scnitem{Комплекс свойств, определяющих качество многоагентной системы}
			\scnitem{Комплекс свойств, определяющих уровень интероперабельности
				кибернетической системы как фактора существенного повышения уровня ее
				обучаемости, а также фактора существенного повышения качества всех тех
				многоагентных систем, в состав которых входит данная кибернетическая система}
			\scnitem{Направления эволюции компьютерных систем}
		\end{scnreltovector}

		\newpage
		
		\scnsegmentheader{Уточнение понятия кибернетической системы}
\begin{scnsubstruct}
    \scnheader{кибернетическая система}
    \scnidtf{cистема, которая способна \uline{управлять} своими \uline{действиями},
        адаптируясь к изменениям состояния внешней среды (среды своего обитания) в
        целях самосохранения (сохранения своей целостности и комфортности
        существования путем удержания своих жизненно  важных параметров в определенных
        рамках комфортности) и/или в целях формирования определенных реакций
        (воздействий на внешнюю среду) в ответ на определенные стимулы (на определенные
        ситуации или события во внешней среде), а также которая способна (при
        соответствующем уровне развития) эволюционировать в направлении:
        \begin{scnitemize}
            \item изучения своей внешней среды как минимум для предсказания последствий
            своих воздействий на внешнюю среду, а также для предсказания изменений внешней
            среды, которые не зависят от собственных воздействий;
            \item изучения самой себя и, в частности, своего взаимодействия с внешней
            средой;
            \item создания технологий (методов и средств), обеспечивающих изменение своей
            внешней среды (условий своего существования) в собственных интересах.
        \end{scnitemize}
    }
    \scnidtf{адаптивная система}
    \scnidtf{целенаправленная (целеустремленная) система}
    \scnidtf{активный субъект самостоятельной деятельности}
    \scnidtf{материальная сущность, способная целенаправленно (в своих интересах)
        воздействовать	на среду своего обитания  как минимум для сохранения своей
        целостности, жизнеспособности, безопасности}
    \scntext{примечание}{Уровень (степень) адаптивности, целенаправленности, активности у
        систем, основанных на обработке информации может быть самым
        различным.}
    \scnidtf{система, организация функционирования которой основано на
        обработке информации о той среде, в которой существует эта система}
    \scnidtf{материальная сущность, способная к активной  целенаправленной
        деятельности, которая  на определенном уровне развития указанной сущности
        становится осмысленной, планируемой, преднамеренной деятельностью}
    \scnidtf{субъект, способный на самостоятельное выполнение некоторых внутренних
        и внешних  действий либо порученных извне, либо инициированных самим субъектом}
    \scnidtf{сущность, способная выполнять роль субъекта деятельности}
    \scnidtf{естественная или искусственно созданная система, способная мониторить
        и анализировать свое состояние и состояние окружающей среды, а также способная
        достаточно активно воздействовать на собственное на собственное состояние и на
        состояние окружающей среды}
    \scnidtf{система, способная в достаточной степени самостоятельно
        взаимодействовать со своей средой, решая различные задачи}
    \scnidtf{система, основанная на обработке информации}
    \scnrelto{ключевой знак}{\scncite{Glushkov1979}}
    \begin{scnindent}
        \scniselement{статья}
    \end{scnindent}
    \scntext{примечание}{\scnkeyword{кибернетическая система} динамически сопоставляет полученную информацию с выбранными действиями,
        относящимися к задаче, которая определяет основную цель системы.}
        \scntext{источник}{\scncite{Fry2002}}
    \bigskip

    \scnsegmentheader{Типология кибернетических систем}
    \begin{scnsubstruct}
        \scnheader{кибернетическая система}
        \scnrelfrom{разбиение}{Признак естественности или искусственности кибернетических систем}
        \begin{scnindent}
            \begin{scneqtoset}
                \scnitem{естественная кибернетическая система}
                \begin{scnindent}
                    \scnidtf{кибернетическая система естественного происхождения}
                    \scnsuperset{человек}
                \end{scnindent}
                \scnitem{компьютерная система}
                \begin{scnindent}
                    \scnidtf{искусственная кибернетическая система}
                    \scnidtf{кибернетическая система искусственного происхождения}
                    \scnidtf{технически реализованная кибернетическая система}
                \end{scnindent}
                \scnitem{симбиоз естественных и искусственных кибернетических систем}
                \begin{scnindent}
                    \scnidtf{кибернетическая система, в состав которой входят компоненты
                        как естественного, так и искусственного происхождения}
                    \scnsuperset{сообщество компьютерных систем и людей}
                \end{scnindent}
            \end{scneqtoset}
        \end{scnindent}
        
        \scnheader{искусственная сущность}
        \scnidtf{артефакт}
        \scnidtf{сущность, являющаяся либо результатом человеческой деятельности, либо
            частью самой этой деятельности}
        \scnidtf{сущность искусственного происхождения}
        \scnidtf{антропогенная сущность}
        \scnsuperset{научно-техническое знание}
        \scnidtf{знание, приобретенное в результате научно-технической деятельности
            человеческого общества}
        \scnsuperset{материальная искусственная сущность}
        \scnsuperset{компьютерная система}
        \scnheader{компьютерная система}
        \scnidtf{искусственная кибернетическая система}
        \scntext{примечание}{Особенностью компьютерных систем является то, что они могут
            выполнять роль	не только продуктов соответствующих действий по реализации этих
            систем, но и сами являются \textit{субъектами*}, способными выполнять
            (автоматизировать) широкий спектр действий. При этом интеллектуализация этих
            систем существенно расширяет этот спектр. \textit{См. интеллектуальная
                компьютерная система}.}\scnidtf{технически реализованная кибернетическая
            система}
        \scnidtf{искусственная кибернетическая система}
        \scnsubset{кибернетическая система}
        \scnsuperset{современная компьютерная система традиционного вида}
        \scnsuperset{современная интеллектуальная компьютерная система}
        \scnsuperset{интеллектуальная компьютерная система следующего поколения}
        \scnsuperset{ostis-система}
        \scntext{примечание}{Основной тенденцией эволюции компьютерных систем является
            повышение уровня их интеллектуальности.}
        \begin{scnrelfromset}{особенность}
            \scnfileitem{Ориентация на принципиально новые компьютеры}
            \scnfileitem{Cущественное повышение уровня интеллекта}
        \end{scnrelfromset}
        
        \scnheader{кибернетическая система}
        \scnrelfrom{разбиение}{Структурная классификация кибернетических
            систем}
            \begin{scnindent}
                \begin{scneqtoset}
                    \scnitem{простая кибернетическая система}
                    \scnitem{индивидуальная кибернетическая система}
                    \scnitem{многоагентая система}
                    \begin{scnindent}
                        \begin{scnsubdividing}
                            \scnitem{одноуровневый коллектив кибернетических систем}
                                \begin{scnindent}
                                    \scnidtf{многоагентная система, агентами которой не могут быть многоагентные системы}
                                \end{scnindent}
                            \scnitem{иерархический коллектив кибернетических систем}
                                \begin{scnindent}
                                    \scnidtf{многоагентная система, по крайней мере одним	агентом которой является многоагентная система}
                                \end{scnindent}
                        \end{scnsubdividing}
                        \begin{scnsubdividing}
                            \scnitem{коллектив из простых кибернетических систем}
                                \begin{scnindent}
                                    \scntext{примечание}{Такой коллектив может быть либо одноуровневым, либо иерархическим коллективом}
                                \end{scnindent}
                            \scnitem{коллектив из индивидуальных кибернетических систем}
                            \scnitem{коллектив из индивидуальных и простых кибернетических систем}
                        \end{scnsubdividing}
                    \end{scnindent}
                \end{scneqtoset}
            \end{scnindent}
        \scnrelfrom{разбиение}{Классификация кибернетических систем по признаку наличия
            надсистемы и роли в рамках этой надсистемы}
            \begin{scnindent}
                \begin{scneqtoset}
                    \scnitem{кибернетическая система, не являющаяся частью никакой другой
                        кибернетической системы}
                        \begin{scnindent}
                            \scnidtf{кибернетическая система, не имеющая надсистем}
                        \end{scnindent}
                    \scnitem{кибернетическая система, встроенная в индивидуальную кибернетическую
                        систему}
                    \scnitem{агент многоагентной системы}
                        \begin{scnindent}
                            \scnidtf{кибернетическая система, являющаяся агентом одной или нескольких многоагентных систем}
                        \end{scnindent}
                \end{scneqtoset}
            \end{scnindent}

        \scnheader{простая кибернетическая система}
        \scnidtf{\textit{кибернетическая система}, уровень развития которой находится
            ниже уровня \textit{индивидуальных кибернетических систем} и которая является
            специализированным средством обработки информации специализированным решателем
            задач, реализующим (интерпретирующим) чаще всего один \textit{метод} решения
            задач и, соответственно, решающим только \textit{задачи} заданного
            \textit{класса задач}}
        \scnidtf{специализированный \textit{решатель задач}}
        \scntext{примечание}{\textit{простая кибернетическая система} может быть
            \textit{компонентом*}, встроенным в \textit{индивидуальную кибернетическую
            систему}, а также может быть \textit{агентом*} \textit{многоагентной системы}, являющейся коллективом из простых
            кибернетических систем}
            
        \scnheader{индивидуальная кибернетическая система}
        \scnidtf{условно выделенный уровень развития \textit{кибернетических систем}, в
            основе которого лежит переход от \textit{специализированного решателя задач к
            индивидуальному решателю}, обеспечивающему интерпретацию произвольного
            (нефиксированного) набора \textit{методов} (программ) решения задач при
            условии, если эти \textit{методы} введены (загружены, записаны) в
            \textit{память} \textit{кибернетической системы}}
        \scnidtf{кибернетическая система, способная быть самостоятельной}
        \scntext{пояснение}{Признаками индивидуальных кибернетических систем
            являются:
            \begin{scnitemize}
                \item наличие \textit{памяти}, предназначенной для хранения как минимум
                интерпретируемых \textit{методов} (программ)  и обеспечивающей корректировку
                (редактирование) хранимых \textit{методов}, а также их удаление  из
                \textit{памяти} и ввод (запись) в \textit{память} новых \textit{методов};
                \item легкая возможность перепрограммировать  \textit{кибернетическую систему}
                на решение других задач, что обеспечивается наличием \textit{универсальной
                    модели решения задач} и, соответственно, \textit{универсальным интерпретатором
                    \uline{любых} моделей}, представленных (записанных) на соответствующем
                \textit{языке};
                \item наличие пусть даже простых средств коммуникации (обмена информацией) с
                другими \textit{кибернетическими системами} (например, с людьми);
                \item способность входить в различные \textit{коллективы кибернетических
                    систем}.
            \end{scnitemize}
        }\scntext{примечание}{класс \textit{индивидуальных кибернетических систем}  это
            определенный этап эволюции кибернетических систем, означающий переход к
            кибернетическим системам, которые способны самостоятельно
            выживать}\scnidtf{самостоятельная автономная, целостная кибернетическая
            системам}
        \scnidtf{субъект деятельности}
        \scntext{примечание}{\textit{индивидуальная кибернетическая система} может быть
            агентом (членом) многоагентной системы (членом коллектива индивидуальных
            кибернетических систем), но некоторые многоагентные системы могут состоять из
            агентов , не являющихся  \textit{индивидуальными кибернетическими системами},
            представляющих собой простые специализированные кибернетические системы,
            выполняющие достаточно простые действия}
            \begin{scnindent}
                \begin{scnrelfromlist}{источник}
                    \scnitem{\scncite{Stefanuk}}
                    \scnitem{\scncite{fonNeuman}}
                \end{scnrelfromlist}
            \end{scnindent}
        \scnidtf{кибернетическая система, которая обладает
            достаточной самостоятельностью (целостностью), но не является коллективом таких
            самостоятельных  кибернетических систем}
        \scnidtf{минимальная самостоятельная (самодостаточная, в известной степени
            автономная) кибернетическая система}
        \scnidtf{индивидуальный субъект}
        \scnheader{кибернетическая система, встроенная в индивидуальную кибернетическую
            систему}
        \scnrelsuperset{пример}{sc-агент ostis-системы}
        \scnrelsuperset{пример}{решатель задач ostis-системы}
        \scnheader{многоагентная система}
        \scnidtf{коллектив взаимодействующих автономных кибернетических систем, имеющих
            общую среду обитания (жизнедеятельности)}
        \begin{scnsubdividing}
            \scnitem{коллектив из простых кибернетических систем}
            \scnitem{коллектив индивидуальных кибернетических систем}
            \scnitem{коллектив индивидуальных и простых кибернетических систем}
        \end{scnsubdividing}
        \begin{scnsubdividing}
            \scnitem{одноуровневый коллектив кибернетических систем}
            \begin{scnindent}
                \scnidtf{многоагентная система, агентами которой не могут быть многоагентные системы}
            \end{scnindent}
            \scnitem{иерархический коллектив кибернетических систем}
            \begin{scnindent}
                \scnidtf{многоагентная система, по крайней мере одним агентом которой является многоагентная система}
            \end{scnindent}
        \end{scnsubdividing}
        \scnheader{одноуровневый коллектив кибернетических систем}
        \scnidtf{специализированное средство решения задач, реализующее либо
            \uline{одну} модель параллельного (распределенного) решения задач
            соответствующего класса, либо комбинацию \uline{фиксированного числа} разных и
            параллельно реализованных моделей решения задач}
        \begin{scnsubdividing}

            \scnitem{одноуровневая однородная многоагентная система}
            \scnitem{одноуровневая неоднородная многоагентная система}

        \end{scnsubdividing}
        \scnheader{коллектив индивидуальных кибернетических систем}
        \scnsubset{многоагентная система}
        \scnidtf{многоагентная система, агентами (членами) которой являются
            \uline{индивидуальные}(!) кибернетические системы}
        \begin{scnsubdividing}

            \scnitem{коллектив людей}
                \begin{scnindent}
                    \scnidtf{человеческое сообщество}
                \end{scnindent}
            \scnitem{сообщество компьютерных систем и людей}

        \end{scnsubdividing}
        \scnheader{иерархический коллектив индивидуальных кибернетических систем}
        \scnidtf{многоагентная система, агентами (членами) которой могут быть:
            \begin{scnitemize}

                \item индивидуальные кибернетические системы;
                \item коллективы индивидуальных кибернетических систем;
                \item коллективы, состоящие из индивидуальных кибернетических систем и
                коллективов индивидуальных кибернетических систем и т.д.
            \end{scnitemize}
        }
        \bigskip
    \end{scnsubstruct}

    \scnsegmentheader{Структура кибернетической системы}
    \begin{scnsubstruct}
        \scnheader{кибернетическая система}
        \begin{scnrelfromset}{обобщенная декомпозиция}

            \scnitem{информация, хранимая в памяти кибернетической системы}
            \scnitem{абстрактная память кибернетической системы}
            \scnitem{решатель задач кибернетической системы}
            \scnitem{физическая оболочка кибернетической системы}

        \end{scnrelfromset}
        \scnheader{информация, хранимая в памяти кибернетической системы}
        \scnidtf{информация, хранимая в памяти \textit{кибернетической системы} и
            представляющая собой информационную модель среды, в которой действует
            (существует, функционирует) эта \textit{кибернетическая система}}
        \scnidtf{текущее состояние памяти кибернетической системы}
        \scnidtf{текущее состояние внутренней (информационной) среды кибернетической
            системы}
        \scnrelto{второй домен}{информация, хранимая в памяти кибернетической системы*}
        \begin{scnindent}
            \scniselement{бинарное отношение}
            \scniselement{ориентированное отношение}
        \end{scnindent}
        \scnheader{абстрактная память кибернетической системы}
        \scnidtf{внутренняя абстрактная информационная среда \textit{кибернетической системы},
            представляющая собой динамическую \textit{информационную  конструкцию}, каждое состояние
            которой есть не что иное, как \textit{информация, хранимая в памяти кибернетической
            системы} в соответствующий момент времени}
        \scnidtf{абстрактная динамическая модель памяти кибернетической системы}
        \scnsubset{динамическая информационная конструкция}
        \begin{scnindent}
            \scnidtf{процесс преобразования информационной конструкции}
        \end{scnindent}
        \scnheader{решатель задач кибернетической системы}
        \scnidtf{совокупность всех навыков (умений), приобретенных кибернетической
            системой к рассматриваемому моменту}
        \scnidtf{встроенный в кибернетическую систему субъект, способный выполнять
            целенаправленные (осознанные) действия во внешней среде этой кибернетической
            системы, а также в её внутренней среде (в абстрактной памяти)}
        \scnheader{действие кибернетической системы}
        \scnsubset{действие}
        \scnidtf{целенаправленное (осознанное) действие, выполняемое кибернетической
            системой, а точнее, её решателем задач}
        \begin{scnsubdividing}

            \scnitem{внешнее действие кибернетической системы}
            \begin{scnindent}
                \scnidtf{действие, выполняемое кибернетической системой в её внешней среде}
                \scnidtf{поведенческое действие}
            \end{scnindent}
            \scnitem{действие кибернетической системы, выполняемое в собственной физической
                оболочке}
            \scnitem{действие кибернетической системы, выполняемое в собственной
                абстрактной памяти}
            \begin{scnindent}
                \scnidtf{действие, направленное на
                    преобразование информации, хранимой в памяти, но никак не на преобразование
                    физической памяти (физической оболочки абстрактной памяти)}
            \end{scnindent}
        \end{scnsubdividing}
        \scntext{примечание}{Каждое \uline{сложное} действие,выполняемое
            кибернетической системой вне собственный абстрактной памяти, включает в себя
            поддействия, выполняемые в указанной \textit{абстрактной памяти кибернетической системы}. Это означает, что все
            \textit{внешние действия кибернетической системы} \uline{управляются} внутренними её
            действиями (действиями в абстрактной памяти).}
        
        \scnheader{задача}
        \scnidtf{спецификация действия}
        \scnidtf{формулировка задачи с различной степенью детализации (уточнения)
            специфицируемого (описываемого) действия, в состав которой может входить:
            \begin{scnitemize}

                \item описание цели (целевой ситуации);
                \item указание объектов (аргументов) действия;
                \item указание типа действия (класса действий, которому принадлежит данное
                действие);
                \item указание субъекта действия;
                \item указание инструмента (средств) выполненного действия;
                \item и др.
            \end{scnitemize}
        }
        \scntext{примечание}{Процесс решения задачи и действие, специфицируемое этой задачей
            (точнее, процесс выполнения этого действия) суть одно и то
            же.}
        \scnheader{задача, решаемая кибернетической системой}
        \scnidtf{задача, решаемая соответствующей кибернетической системой}
        \scnidtf{Второй домен отношения быть задачей, решаемой заданной кибернетической
            системой*}
        \scnrelboth{следует отличать}{задача, решаемая кибернетической системой*}
        \begin{scnindent}
            \scnidtf{быть задачей, решаемой заданной кибернетической системой*}
        \end{scnindent}
        \begin{scnsubdividing}

            \scnitem{задача, решаемая кибернетической системой во внешней среде}
                \begin{scnindent}
                    \scnidtf{внешняя задача кибернетической системы}
                    \scnidtf{задача, направленная на изменение состояния внешней среды
                        соответствующей кибернетической системы, но включающая в себя (в качестве
                        подзадач) задачи, решаемые в памяти кибернетической системы, например:
                        \begin{scnitemize}

                            \item интерфейсные задачи (анализ первичный информации о текущем состоянии
                            внешней среды),
                            \item cенсо-моторную координацию выполнения сложных действий во внешней среде,
                            состоящих из большого количества частных (более простых) действий, находящихся
                            на разных уровнях иерархии,
                            \item задачи планирования целенаправленного поведения во внешней среде,
                            \item задачи принятия решений.
                        \end{scnitemize}
                    }
                \end{scnindent}
            \scnitem{задача, решаемая кибернетической системой в собственной физической
                оболочке}
            \scnitem{задача решаемая кибернетической системой в абстрактной
                памяти}
                \begin{scnindent}
                    \scnidtf{задача, полностью решаемая в памяти кибернетической системы и
                        направленная на изменение состояния информации, хранимой в памяти
                        кибернетической системы}
                    \scnidtf{внутренняя задача кибернетической системы}
                \end{scnindent}

        \end{scnsubdividing}
        \scnheader{навык}
        \scnsubset{знание}
        \scntext{пояснение}{знание частного вида, содержащее (1) некоторый метод --
            знание о том, как можно решать задачи, принадлежащие соответствующему множеству
            задач, (2) полное знание о том, как указанный метод следует интерпретировать
            (реализовывать), декомпозируя исходные задачи на подзадачи и, в конечном счёте
            на элементарные действия, выполняемые \textit{процессором кибернетической
                системы}}\scnidtf{умение}
        \scnidtf{методы и средства, обеспечивающие способность \textit{кибернетической
                системы} решать некоторое множество задач (выполнять некоторое множество
            действий)}
        \scnheader{интерфейс кибернетической системы}
        \scnidtf{условно выделяемый компонент \textit{решателя задач кибернетической
                системы}, обеспечивающий решение \textit{интерфейсных задач}, направленных на
            \uline{непосредственную} реализацию взаимодействия \textit{кибернетической
                системы} с её \textit{внешней средой}}
        \scnidtf{решатель интерфейсных задач кибернетической системы}
        \scnrelto{обобщенная часть}{решатель задач кибернетической системы}
        \scnrelboth{следует отличать}{физическое обеспечение интерфейса кибернетической
            системы}
            \begin{scnindent}
                \scnrelto{обобщенная часть}{физическая оболочка кибернетической системы}
            \end{scnindent}

        \scnheader{физическая оболочка кибернетической системы}
        \scnidtf{часть кибернетической системы, являющаяся посредником	между её
            внутренней средой (памятью, в которой хранится и обрабатывается информация
            кибернетической системы) и её внешней средой}
        \scnrelto{второй домен}{физическая оболочка кибернетической системы*}
        \begin{scnindent}
            \scniselement{бинарное отношение}
            \scniselement{ориентированное отношение}
        \end{scnindent}

        \scnheader{кибернетическая система}
        \begin{scnrelfromset}{обобщенная декомпозиция}
            \scnitem{память кибернетической системы}
            \scnitem{процессор кибернетической системы}
            \scnitem{физическое обеспечение интерфейса кибернетической системы}
            \begin{scnindent}   
                \scnidtf{аппаратное обеспечение интерфейса кибернетической системы с её
                    внешней средой}
                \begin{scnrelfromset}{обобщенная декомпозиция}
                    \scnitem{сенсорная подсистема физической оболочки кибернетической системы}
                    \scnitem{эффекторная подсистема физической оболочки кибернетической системы}
                \end{scnrelfromset}
            \end{scnindent}
            \scnitem{корпус кибернетической системы}
        \end{scnrelfromset}
        
        \scnheader{память кибернетической системы}
        \scnidtf{физическая оболочка (реализация) абстрактной \textit{памяти
                кибернетической системы} --- внутренней среды \textit{кибернетической системы},
            в рамках которой \textit{кибернетическая система} формирует и использует
            (обрабатывает) информационную модель своей внешней среды}
        \scntext{примечание}{Не каждая \textit{кибернетическая система} имеет
            \textit{память}. В \textit{кибернетических системах}, которые не имеют
            \textit{памяти}, обработка информации сводится к обмену сигналами между
            компонентами этих систем. Появление в \textit{кибернетических системах} памяти
            как среды для централизованного  хранения и обработки \textit{информации}
            является важнейшим этапом их эволюции. Дальнейшая эволюция
            \textit{кибернетических систем} во многом определяется:
            \begin{scnitemize}

                \item \textit{качеством памяти} как среды для хранения и обработки информации;
                \item качеством информации (информационной модели), хранимой в памяти
                кибернетической системы;
            \end{scnitemize}
        }\scnidtf{компонент \textit{кибернетической системы}, в рамках которого
            \textit{кибернетическая система} осуществляет отображение (формирование
            информационной модели) среды своего существования, а также использование этой
            информационной модели для управления собственным поведением в указанной среде}
        \scnidtf{физическая оболочка для хранения информации, которую кибернетическая
            система приобретает и обрабатывает (т.е. меняет состояния этой информации)}
        \scnidtf{физическая (аппаратная) реализация \uline{внутренней} среды
            кибернетической системы, каковой является среда существования  информации,
            накапливаемой и непосредственно используемой решателем задач этой
            кибернетической системы}
        \scntext{примечание}{Сам факт появления в \textit{кибернетической системе} памяти,
            которая (1) обеспечивает представление различного виды информации о среде, в
            рамках которой \textit{кибернетическая} система решает различные задачи (выполняет
            различные действия), (2) обеспечивает хранение достаточно полной информационной
            модели указанной среды (достаточно полной для реализации своей деятельности),
            (3) обеспечивает высокую степень гибкости указанной хранимой в памяти
            информационной модели среды жизнедеятельности (т.е. лёгкость внесения изменений
            в эту информационную модель), существенно повышает уровень адаптивности
            \textit{кибернетической системы} к различным изменениям своей
            среды.}
        \scntext{примечание}{появление  \uline{\textit{памяти}} в кибернетических
            системах является основным признаком перехода от простых  автоматов к
            компьютерным системам, от роботов 1-го поколения к роботам следующих
            поколений}\scnidtf{физическая реализация хранилища информации, которую
            приобрела (накопила) к текущему моменту соответствующая кибернетическая
            система}
        \scnidtf{физическая оболочка внутренней абстрактной информационной среды
            кибернетической системы}
        \scnidtf{среда хранения и обработки информации}
        \scnidtf{запоминающая среда}
        \scnidtf{среда хранения и обработки информационных конструкций}
        \scntext{примечание}{Принципы организации памяти \textit{кибернетической системы} могут быть
            разными (ассоциативная, адресная, структурно фиксированная/структурно
            перестраиваемая, нелинейная/линейная). От организации памяти \textit{кибернетической системы} во многом зависит
            её качество.}
    
        \scnheader{кибернетическая система}
        \scnrelfrom{уровни эволюции}{Уровни структурной эволюции кибернетических систем}
        \begin{scnindent}
            \begin{scneqtovector}

                \scnitem{простая кибернетическая система, не имеющая памяти}
                \scnitem{простая кибернетическая система, имеющая память}
                \scnitem{одноуровневый коллектив, не имеющий общей памяти и состоящий из
                    простых кибернетических систем, не имеющих памяти}
                \scnitem{одноуровневый коллектив, не имеющий общей памяти и состоящий из
                    простых кибернетических систем, имеющих память}
                \scnitem{иерархический коллектив,  имеющий общую память и состоящий из простых
                    кибернетических систем}
                \scnitem{индивидуальная кибернетическая система}
                    \begin{scnindent}
                        \scntext{примечание}{Каждая \textit{индивидуальная кибернетическая система} содержит \textit{память},
                                имеющую достаточно высокий уровень качества}
                    \end{scnindent}
                \scnitem{одноуровневый коллектив индивидуальных кибернетических систем, не
                    имеющий общей памяти}
                \scnitem{одноуровневый коллектив индивидуальных кибернетическая систем, имеющий
                    общую память }
                \scnitem{иерархический коллектив из индивидуальных кибернетических систем, не
                    имеющий общей памяти}
                \scnitem{иерархический коллектив из индивидуальных кибернетических систем,
                    имеющий общую память}
            \end{scneqtovector}
        \end{scnindent}

        \scnheader{процессор кибернетической системы}
        \scnidtf{физически (аппаратно реализованный) интерпретатор хранимых в \textit{памяти
            кибернетической системы методов (программ)}, соответствующих базовой (для данной
            кибернетической системы) \textit{модели решения задач}, т.е. такой модели решения задач,
            которая для данной \textit{кибернетической системы} является \textit{моделью решения задач}
            самого нижнего уровня и, следовательно, не может быть интерпретирована с
            помощью другой \textit{модели решения задач}, используемой этой же \textit{кибернетической
            системой}, а может быть проинтерпретирована либо путем аппаратной реализации
            такого интерпретатора, либо путём его программной реализации, например, на
            современных компьютерах, но в последнем случае, кроме собственного
            интерпретатора, необходимо также построить модель \textit{памяти} реализуемой
            кибернетической системы}
        \scnidtf{физически  реализованные средства, обеспечивающие выполнение
            элементарных  действий, направленных на изменение состояния памяти
            кибернетической системы (на изменение информации, хранимой в этой памяти)}
        \scnidtf{движок(мотор) кибернетической системы}
        \scnrelto{второй домен}{процессор кибернетической системы*}
        \begin{scnindent}
            \scnidtftext{пояснение}{бинарное ориентированное отношения, каждая пара
                которого связывает знак кибернетической системы со знаком её процессора}
            \scniselement{бинарное отношение}
            \scniselement{ориентированное отношение}
        \end{scnindent}

        \scnheader{компьютер}
        \scnsubset{физическая оболочка кибернетической системы}
        \scnidtf{физическая оболочка искусственной кибернетической системы}
        \scnidtf{аппаратное обеспечение компьютерной системы}
        \scnidtf{hardware of computer system}
        \scnsuperset{компьютер для интеллектуальных систем}
        \begin{scnindent}
            \scnidtf{компьютер, ориентированный на реализацию интеллектуальных компьютерных
                систем}
            \scntext{примечание}{Развитие рынка \textit{интеллектуальных компьютерных систем} существенно
                сдерживается неприспособленностью современного поколения компьютеров к
                реализации на их основе \textit{интеллектуальных компьютерных систем}. Попытки создания
                компьютеров, приспособленных к реализации \textit{интеллектуальных компьютерных систем},
                не привели к успеху, т.к. эти проекты были направлены на выполнение отдельных
                (частных) требований, предъявляемых к физическому (аппаратному) уровню
                \textit{интеллектуальных компьютерных систем}, что неминуемо приводило к приспособленности
                создаваемых компьютеров к реализации не всего многообразия \textit{интеллектуальных
                компьютерных систем}, а только некоторых подмножеств таких систем. Указанные
                подмножества \textit{интеллектуальных компьютерных систем} в основном определялись
                ориентацией на конкретные используемые \textit{модели решения интеллектуальных задач},
                тогда, как важнейшим фактором, определяющим уровень \textit{интеллекта кибернетических
                систем} (в том числе, и компьютерных систем), является их универсальность в
                плане многообразия используемых моделей решения задач. Следовательно, компьютер
                для \textit{интеллектуальных компьютерных систем} должен быть эффективным аппаратным
                интерпретатором любых моделей решения задач (как интеллектуальных задач, так и
                достаточно простых задач, т.к. интеллектуальная система должна уметь решать
                любые задачи).}\scnidtf{компьютер, приспособленный к реализации
                интеллектуальных компьютерных систем}
            \scnidtf{универсальный компьютер для интеллектуальных систем}
            \scnidtf{компьютер, обеспечивающий интерпретацию любых моделей решения задач}
        \end{scnindent}
        \bigskip
    \end{scnsubstruct}
    \scnsegmentheader{Семейство отношений, заданных на множестве кибернетических
        систем}
    \begin{scnsubstruct}
        \scnheader{отношение, заданное на множестве кибернетических систем}
        \scnhaselement{память кибернетической системы*}
        \scnhaselement{процессор кибернетической системы*}
        \scnhaselement{член коллектива*}
        \scnhaselement{внешняя среда кибернетической системы*}
        \scnhaselement{сенсор кибернетической системы*}
        \scnhaselement{эффектор кибернетической системы*}
        \scnhaselement{физическая оболочка кибернетической системы*}
        \scnhaselement{информация, хранимая в памяти кибернетической системы*}
        \scnhaselement{абстрактная память кибернетической системы*}
        \scnhaselement{часть*}
        \begin{scnindent}
            \scnsuperset{встроенная кибернетическая система*}
        \end{scnindent}

        \scnheader{информация, хранимая в памяти кибернетической системы*}
        \scnidtf{\textit{информационная модель среды*}, в которой существует
            (осуществляет деятельность) соответствующая кибернетическая система*}
        \scntext{примечание}{От того, насколько полна, адекватна (корректна) и
            систематизирована (структурирована) внутренняя среда кибернетической системы,
            зависит уровень интеллектуальности и эффективность соответствующей
            кибернетической системы.}
            
        \scnheader{следует отличать*}
        \begin{scnhaselementset}

            \scnitem{задача, решаемая кибернетической системой*}
            \scnitem{решатель задач кибернетической системы}
            \begin{scnindent}
                \scnidtf{иерархическая система моделей решения задач}
                    \scnrelfrom{обобщённая часть}{процессор кибернетической системы}
                    \begin{scnindent}
                        \scntext{пояснение}{Это реализация модели решения задач, обеспечивающей
                        интерпретацию всех используемых моделей решения задач верхнего уровня}
                    \end{scnindent}
            \end{scnindent}
        \end{scnhaselementset}

        \scnheader{задача, решаемая кибернетической системой*}
        \scnidtf{быть задачей, решаемой заданной кибернетической системой*}
        \scnsuperset{задача, решаемая в памяти кибернетической системы*}
        \begin{scnindent}
            \scnidtf{внутренняя задача кибернетической системы*}
        \end{scnindent}
        \scnsuperset{задача, решаемая во внешней среде кибернетической системы*}
        
        \scnheader{\textit{внешняя среда кибернетической системы*}}
        \scnidtf{внешняя среда*}
        \scntext{примечание}{Понятие \textit{внешней среды кибернетической системы*} является
            понятием относительным, т.к. (1) разные \textit{кибернетические системы} имеют в общем
            случае разную внешнюю среду и (2) одна \textit{кибернетическая система} может входить в
            состав внешней среды другой кибернетической системы}
        \scnidtf{быть внешней
            средой для заданной кибернетической системы*}
        \scniselement{бинарное отношение}
        \scniselement{ориентированное отношение}
        \scnrelfrom{первый домен}{кибернетическая система}
        \scnsuperset{внешняя информационная среда кибернетической системы*}
        \begin{scnindent}
            \scnidtf{совокупность всевозможных информационных конструкций, к которым данная
                кибернетическая система имеет доступ и которые представлены самым различным
                образом (в том числе, и в памяти тех кибернетических систем (субъектов), с
                которыми данная система взаимодействует)*}
        \end{scnindent}

        \scnheader{среда кибернетической системы*}
        \scnidtf{быть средой существования (жизнедеятельности) заданной (указанной,
            соответствующей) кибернетической системы*}
        \scntext{примечание}{В общем случае среда жизнедеятельности \textit{кибернетической
                системы} включает в себя (1) \textit{внешнюю среду*} этой системы, (2)
            \textit{физическую оболочку*} этой системы и (3) её \textit{абстрактную
                память}, т.е. внутреннюю среду*, которая является хранилищем информационной
            модели всей среды}\begin{scnsubdividing}

            \scnitem{внешняя среда*}
            \scnitem{физическая оболочка*}
            \scnitem{абстрактная память*}

        \end{scnsubdividing}
        \bigskip
    \end{scnsubstruct}
\end{scnsubstruct}
\scnsourcecomment{Завершили Сегмент \scnqqi{Уточнение понятия кибернетической системы}}

		\newpage
\scnsegmentheader{Комплекс свойств, определяющий общий уровень качества кибернетической системы}
\begin{scnsubstruct}
    \scnheader{качество кибернетической системы}
    \scnidtf{интегральный уровень качества кибернетической системы в заданный момент}
    \scnidtf{комплексная оценка (характеристика) уровня качества кибернетической системы}
    \scntext{пояснение}{Для того, чтобы уточнить (детализировать) понятие \textit{качества кибернетической системы}, необходимо
        \begin{scnitemize}
            \item задать метрику \textit{качества кибернетических систем} и
            \item построить иерархическую систему свойств (параметров, признаков), определяющих \textit{качество кибернетической системы}.
        \end{scnitemize}
    }
    \scniselement{упорядоченное свойство}
    \scnidtf{эволюционный уровень кибернетической системы}
    \scnidtf{интегральная (комплексная) оценка уровня развития (совершенства) кибернетической системы}
    \scntext{пояснение}{\textit{Качество кибернетической системы} --- это такое свойство (характеристика) \textit{кибернетических систем}, такой признак их классификации, который позволяет разместить эти системы по ступенькам некоторой условной эволюционной лестницы. На каждую такую ступеньку попадают \textit{кибернетические системы}, имеющие одинаковый уровень развития, каждому их которых соответствует свой набор значений дополнительных свойств \textit{кибернетических систем}, которые уточняют (детализируют, специализируют) соответствующий уровень развития \textit{кибернетических систем}. Такой эволюционный подход к рассмотрению \textit{кибернетических систем} даёт возможность, во-первых, детализировать направления эволюции \textit{кибернетических систем} и, во-вторых, уточнить то место этой эволюции, где и благодаря чему осуществляется переход от неинтеллектуальных \textit{кибернетических систем} к интеллектуальным. Фактически речь идёт об эволюционной теории качества \textit{кибернетических систем}, рассматривающей эволюцию \textit{кибернетических систем} как в рамках жизненного цикла каждой из них, так и в рамках эволюции целой популяции при переходе от одного поколения \textit{кибернетических систем} к другому поколению (в частности, от одного поколения \textit{компьютерных систем} к другому).
        В основе эволюционного подхода к рассмотрению многообразия \textit{кибернетических систем} лежит положение о том, что идеальных \textit{кибернетических систем} не существует, но существует постоянное стремление к идеалу, к большему совершенству. При этом важно уточнить, что конкретно в каждой \textit{кибернетической системе} следует изменить, чтобы привести эту систему к более совершенному виду.
        \\Эволюционный подход к рассмотрению \textit{кибернетических систем} имеет важное практическое значение для развития (совершенствования) каждой конкретной \textit{компьютерной системы} (искуственной \textit{кибернетической системы}), а также для развития \textit{технологий} разработки \textit{компьютерных систем}. Так, например, развитие технологий разработки \textit{компьютерных систем} должно быть направлено на переход к таким новым архитектурным и функциональным принципам, лежащим в основе \textit{компьютерных систем}, которые
        \begin{scnitemize}
            \item обеспечивают существенное снижение трудоемкости их разработки и сокращение сроков разработки, а также
            \item обеспечивают существенное повышение уровня \textit{интеллекта} и, в частности, уровня \textit{обучаемости} разрабатываемых \textit{компьютерных систем}, например, путём перехода от поддержки обучения с учителем к реализации эффективного самообучения (к автоматизации организации самостоятельного обучения).
        \end{scnitemize}}
    \scntext{примечание}{В эволюции \textit{кибернетических систем} (и, в частности, \textit{компьютерных систем}) можно выделить целый ряд этапов:
        \begin{scnitemize}
            \item переход от стимульно-реактивного поведения к поведению, предполагающему учёт постоянно накапливаемого собственного опыта, означает переход от протопамяти, которая просто фиксирует связи между стимулами и соответствующими реакциями и которая не предполагает изменения этих связей, к \textit{памяти}, которая становится средой существования информации, отражающей  собственный опыт \textit{кибернетической системы} (а в перспективе и многое другое) и которая обеспечивает высокую степень \textit{гибкости} хранимой \textit{информации}, т.е. широкие возможности изменения (корректировки) этой \textit{информации} в процессе функционирования \textit{кибернетической системы}. Таким образом, \textit{память кибернетической системы} вместе с хранимой в ней \textit{информацией} становится управляемым самой этой \textit{кибернетической системой} гибким коммутатором между её стимулами и реакциями, учитывающим не только накапливаемый собственный опыт, но и контекст (дополнительные обстоятельства) выполняемых \textit{действий} (реакций), рассматривающий выполняемые \textit{действия} с самых разных аспектов;
            \item включение в состав \textit{информации, хранимой в памяти компьютерной системы}, \textit{программ}, описывающих различные \textit{методы} обработки этой \textit{информации} и интерпретируемых \textit{процессором} указанной \textit{компьютерной системы};
            \item переход от указанной выше коммутационной трактовки \textit{информации, хранимой в памяти кибернетической системы} к её трактовке как мощной и постоянно совершенствуемой информационной модели внешней среды, в которой существует указанная \textit{кибернетическая система}. Это означает
                \begin{scnitemizeii}
                    \item переход \textit{информации, хранимой в памяти кибернетической системы} на уровень \textit{базы знаний}, которой ставится в \textit{соответствие} достаточно чёткая \textit{денотационная семантика}, и
                    \item переход \textit{программ}, хранимых в \textit{памяти кибернетической системы}, на уровень \textit{программ}, которые ориентированы на обработку \textit{базы знаний} и которые сами являются частью обрабатываемой \textit{базы знаний};
                \end{scnitemizeii}
            \item существенное расширение \textit{семантической мощности баз знаний} и многообразия используемых \textit{моделей решения задач}, в том числе, моделей, способных работать в условиях неполноты (недостаточности), нечеткости и недостоверности обрабатываемых \textit{знаний}.
        \end{scnitemize}}
    \scntext{примечание}{Повышение качества искусственных\textit{ кибернетических систем} (\textit{компьютерных систем}) потребует формирования таких свойств (характеристик, способностей) \textit{компьютерных систем}, которые аналогичны психическим свойствам людей. Таким образом, дальнейшее развитие \textit{Искусственного интеллекта} (теории и практики создания \textit{интеллектуальных компьютерных систем} --- интеллектуальных искусственных \textit{кибернетических систем}) настоятельно потребует обобщения современной психологии (психологии биологических индивидов и их коллективов --- \textit{психологии естественных кибернетических систем}) и создания \textit{общей психологии кибернетических систем} (как естественных, так и искусственных) основанной на высоком уровне формализации.}
    \scntext{примечание}{Проблема выделения критериев интеллектуальности компьютерных систем рассмотрена в ряде работ. Были предложены различные системные показатели для измерения качества компьютерных систем. Поскольку системы становятся все более сложными и включают множество подсистем или компонентов, измерение их качества в нескольких измерениях становится сложной задачей. Метрики качества включают в себя набор мер, которые могут описывать атрибуты системы в терминах, не зависящих от структуры, которая приводит к этим атрибутам. Эти меры должны быть выражены количественно и должны иметь значительный уровень точности и надежности.}
    \begin{scnindent}
        \begin{scnrelfromlist}{источник}
            \scnitem{\scncite{Cho2019}}
            \scnitem{\scncite{Sherif1988}}
            \scnitem{\scncite{Finn2021}}
            \scnitem{\scncite{Nilsson2005}}
            \scnitem{\scncite{Kerr2006}}
            \scnitem{\scncite{Antsyferov2013}}
        \end{scnrelfromlist}
    \end{scnindent}

    \scnheader{качество кибернетической системы}
    \begin{scnrelfromlist}{cвойство-предпосылка}

        \scnitem{качество физической оболочки кибернетической системы}
        \scnitem{качество решателя задач кибернетической системы}
        \begin{scnindent}
            \scnrelfrom{cвойство-предпосылка}{качество информации, хранимой в памяти кибернетической системы}
        \end{scnindent}
        \scnitem{качество информации, хранимой в памяти кибернетической системы}
        \scnitem{гибридность кибернетической системы}
        \begin{scnindent}
            \scnidtf{степень многообразия (1) видов знаний, хранимых в памяти кибернетической системы, (2) используемых моделей решения задач, (3) видов сенсоров и эффекторов}
            \begin{scnrelfromlist}{частное свойство}

                \scnitem{многообразие видов знаний, хранимых в памяти кибернетической системы}
                \scnitem{многообразие моделей решения задач}
                \scnitem{многообразие видов сенсоров и эффекторов}

            \end{scnrelfromlist}
        \end{scnindent}
        \scnitem{приспособленность кибернетической системы к её совершенствованию}
        \scnitem{производительность кибернетической системы}
        \begin{scnindent}
            \scnidtf{cкорость решения задач кибернетической системы}
        \end{scnindent}
        \scnitem{надежность кибернетической системы}
        \scnitem{интероперабельность кибернетической системы}

    \end{scnrelfromlist}
    \scnheader{гибридность кибернетической системы}
    \begin{scnrelfromlist}{частное свойство}

        \scnitem{многообразие видов знаний, хранимых в памяти кибернетической системы}
        \scnitem{многообразие моделей решения задач}
        \scnitem{многообразие видов сенсоров и эффекторов}

    \end{scnrelfromlist}
    \scnheader{гибридная кибернетическая система}
    \scnidtf{кибернетическая система, использующая многообразие рецепторных и/или эффекторных подсистем, и/или многообразие видов обрабатываемой информации, и/или многообразие способов решения задач}
    \scnsuperset{гибридная компьютерная система}
    \begin{scnindent}
        \scnidtf{\textit{компьютерная система}, способная решать \textit{комплексные задачи}, требующие использования многообразия различных видов обрабатываемой информации и различных \textit{моделей решения задач}}
    \end{scnindent}

    \scnheader{приспособленность кибернетической системы к её совершенствованию}
    \scnidtf{приспособленность кибернетической системы к эволюции, к повышению уровня своего качества}
    \begin{scnrelfromset}{комплекс свойств-предпосылок}

        \scnitem{обучаемость кибернетической системы}
        \begin{scnindent}
            \scnidtf{способность кибернетической системы самостоятельно повышать уровень своего качества}
            \scnidtf{способность кибернетической системы к самоэволюции, саморазвитию, устранению своих недостатков}
        \end{scnindent}
        \scnitem{приспособленность кибернетической системы к её совершенствованию, осуществляемому извне}
        \begin{scnindent}    
            \scnidtf{приспособленность кибернетической системы к её совершенствованию, осуществляемому внешними субъектами}
            \scnidtf{удобство совершенствования кибернетической системы для её создателей}
            \scntext{примечание}{Важнейшим фактором качества каждой \textit{технологии разработки кибернетических систем} является гибкость и стратифицированность разрабатываемых кибернетических систем при их совершенствовании, осуществляемом руками разработчиков}
        \end{scnindent}
    \end{scnrelfromset}
    \begin{scnrelfromset}{комплекс свойств-предпосылок}
        \scnitem{гибкость кибернетической системы}
        \scnitem{стратифицированность кибернетической системы}
        \begin{scnindent}
            \scnidtf{уровень стратифицированности кибернетической системы}
            \scnidtf{качество разделения (декомпозиции) кибернетической системы на в достаточной степени независимые части (компоненты), определенные виды изменений которых не предполагают внесения изменений в другие части системы}
        \end{scnindent}
    \end{scnrelfromset}

    \scnheader{гибкость кибернетической системы}
    \scnidtf{реконфигурируемость кибернетической системы}
    \scnidtf{модифицируемость кибернетической системы}
    \scnidtf{реформируемость кибернетической системы }
    \scnidtf{трансформируемость кибернетической системы}
    \scnidtf{пластичность кибернетической системы}
    \scnidtf{легкость реализации различного вида изменений в кибернетической системе}
    \scnidtf{степень трансформенности кибернетической системы}
    \scnidtf{простота внесения изменений в кибернетическую систему и многообразие видов возможных таких изменений}
    \scnidtf{модифицируемость кибернетической системы}
    \scnidtf{трансформируемость кибернетической системы}
    \scnidtf{реконфигурируемость кибернетической системы}
    \scnidtf{приспособленность к реинжинирингу кибернетической системы}
    \scnidtf{мягкость}
    \scnidtf{softness}
    \scnidtf{приспособленность к внесению изменений}
    \scnidtf{\uline{легкость} внесения изменений}
    \scntext{примечание}{Чем легче вносить изменения в кибернетическую систему, тем выше скорость ее эволюции}
    \scntext{примечание}{изменения могут вноситься (1) полностью самостоятельно (без учителя) (2) с помощью учителя-тренера (терапевта) путем создания определенных условий для совершенствования системы (3) хирургически --- путем непосредственного вмешательства извне (например, вмешательства разработчика)}
    \scntext{примечание}{Чем выше \textit{гибкость кибернетической системы} --- тем ниже трудоемкость и меньше сроки внесения различных изменений в систему в направлении ее совершенствования (приближения к идеалу)}
    \begin{scnrelfromset}{комплекс свойств-предпосылок}
        \scnitem{простота внесения изменений в кибернетическую систему}
        \begin{scnindent}
            \scnrelfrom{свойство-предпосылка}{стратифицированность кибернетической системы}
        \end{scnindent}
        \scnitem{многообразие возможных изменений, вносимых в кибернетическую систему}
    \end{scnrelfromset}
    \begin{scnrelfromset}{комплекс частных свойств}
        \scnitem{гибкость информации, хранимой в памяти кибернетической системы}
        \scnitem{гибкость решателя задач кибернетической системы}
        \scnitem{гибкость физической оболочки кибернетической системы}
        \begin{scnindent}
            \scnrelfrom{частное свойство}{гибкость памяти кибернетической системы}
        \end{scnindent}
        \scnitem{гибкость интерфейса кибернетической системы}
    \end{scnrelfromset}
    \begin{scnrelfromset}{комплекс частных свойств}
        \scnitem{гибкость кибернетической системы при ее совершенствовании, осуществляемом извне}
        \scnitem{гибкость возможных самоизменений кибернетической системы}
        \begin{scnindent}
            \scnrelto{свойство-предпосылка}{обучаемость кибернетической системы}
        \end{scnindent}
    \end{scnrelfromset}

    \scnheader{приспособленность кибернетической системы к её совершенствованию, осуществляемому извне}
    \scnidtf{приспособленность кибернетическиой системы к хирургическим методам её совершенствования, реализуемым разработчиками}
    \scnidtf{насколько легко осуществлять обновление, перепроектирование, тестирование, ремонт (исправление ошибок) кибернетической системы}
    \begin{scnrelfromlist}{свойство-предпосылка}

        \scnitem{простота внесения изменений в кибернетическую систему, реализуемых извне}
        \begin{scnindent}
            \scnrelfrom{свойство-предпосылка}{стратифицированность кибернетической системы}
        \end{scnindent}
        \scnitem{многообразие возможных изменений кибернетической системы, реализуемых извне}

    \end{scnrelfromlist}

    \scnheader{производительность кибернетической системы}
    \scnidtf{быстродействие кибернетической системы}
    \scnidtf{интегральная оценка скорости решения задач, время реакции кибернетической системы на задачные ситуации}
    \begin{scnrelfromlist}{частное свойство}

        \scnitem{производительность базового интерпретатора логико-семантической модели кибернетической системы}
        \scnitem{качество используемых кибернетической системой методов и моделей решения задач}

    \end{scnrelfromlist}

    \scnheader{надежность кибернетической системы}
    \scnidtf{способность кибернетической системы при соответствующих условиях ее функционирования сохранять (и, точнее, не снижать) уровень всех свойств и способностей, определяющих общее (комплексное) качество кибернетической системы}
    \begin{scnrelfromlist}{свойство-предпосылка}

        \scnitem{безотказность кибернетической системы}
        \scnitem{долговечность кибернетической системы}
        \scnitem{ремонтопригодность кибернетической системы}
        \begin{scnindent}    
        \scnrelfrom{основной sc-идентификатор}{ремонтопригодность кибернетических систем}
            \begin{scnindent}
                \scntext{примечание}{Здесь слово ремонтопригодность взято в кавычки, т.к. речь идет не только об искусственных (технических) кибернетических системах}
            \end{scnindent}
        \end{scnindent}
    \end{scnrelfromlist}
    \bigskip
\end{scnsubstruct}

		\newpage
\scnsegmentheader{Комплекс свойств, определяющих качество физической оболочки
    \textit{кибернетической системы}}
\begin{scnsubstruct}
    \scnheader{качество физической оболочки кибернетической системы}
    \scnidtf{интегральное качество аппаратной (физической) основы \textit{кибернетической
        системы}}
    \scnidtf{hardware кибернетической системы}
    \begin{scnrelfromlist}{свойство-предпосылка}

        \scnitem{качество памяти кибернетической системы}
        \scnitem{качество процессора кибернетической системы}
        \scnitem{качество сенсоров кибернетической системы}
        \scnitem{качество эффекторов кибернетической системы}
        \scnitem{приспособленность физической оболочки кибернетической системы к ее
            совершенствованию}
        \scnitem{удобство транспортировки кибернетической системы}
        \scnitem{надежность физической оболочки кибернетической системы}

    \end{scnrelfromlist}
    \scnheader{качество памяти кибернетической системы}
    \begin{scnreltolist}{свойство-предпосылка}

        \scnitem{качество информации, хранимой в памяти кибернетической системы}
        \scnitem{качество решателя задач кибернетической системы}

    \end{scnreltolist}
    \begin{scnrelfromlist}{свойство-предпосылка}

        \scnitem{способность памяти кибернетической системы обеспечить хранение
            высококачественной информации}
        \scnitem{способность памяти кибернетической системы обеспечить функционирование
            высококачественного решателя задач}
        \scnitem{объём памяти}

    \end{scnrelfromlist}
    \scnheader{память кибернетической системы}
    \scnidtf{компонент \textit{кибернетической системы}, представляющий собой
        внутреннюю среду \textit{кибернетической системы}, в которой она хранит
        (запоминает) и преобразует \textit{информационную модель} своей \textit{внешней
            среды}. При этом важно, чтобы память обеспечивала высокий уровень
        \textit{гибкости} указанной \textit{информационной модели}. Важно также, чтобы
        эта \textit{информационная модель} была моделью не только \textit{внешней
            среды} \bigskip \textit{кибернетической системы}, но также и моделью самой этой
        \textit{информационной модели} --- описанием её \textit{текущей ситуации},
        предыстории, закономерностей. Таким образом, \textit{кибернетическая система},
        имеющая \textit{память}, функционирует в двух средах --- во внешней, в которой
        существуют и преобразуются внешние(материальные) сущности, и во внутренней, в
        которой существуют и преобразуются(обрабатываются) внутренние
        \textit{информационные конструкции}.}
    \scntext{примечание}{\textit{Кибернетические системы}, находящиеся на низком уровне
        развития(качества) \textit{памяти} не имеют. Адаптационные механизмы такой
        кибернетической системы жестко запаяны в связях между блоками обработчика
        \textit{сигналов} при переходе от \textit{сигналов}, вырабатываемых
        \textit{сенсорами} к \textit{сигналам}, которые управляют
        \textit{эффекторами}.}\scnidtf{внутренняя среда кибернетической системы,
        обеспечивающая хранение и преобразование(обработку) информационной модели
        внешней среды кибернетической системы}
    \scntext{примечание}{Сам факт возникновения памяти в \textit{кибернетической системе}
        является важнейшим этапом её эволюции. Дальнейшее развитие \textit{памяти
            кибернетической системы}, обеспечивающее:\begin{scnitemize}

            \item хранение все более качественной информации, хранимой в памяти
            \item все более качественную организацию обработки этой информации, т.е.
            переход на поддержку(обеспечение) все более качественных моделей обработки
            информации\end{scnitemize}
        является важнейшим фактором эволюции \textit{кибернетических
            систем}.}

    \scnheader{способность памяти кибернетической системы обеспечить хранение высококачественной информации}
    \begin{scnrelfromlist}{свойство-предпосылка}

        \scnitem{способность системы обеспечить компактное хранение
            сложноструктурированных баз знаний}
            \begin{scnindent}
                \scntext{примечание}{Здесь имеется в виду необходимость перехода от
                    линейной организации, памяти на физическом уровне (как последовательности ячеек
                    памяти) к нелинейной, графодинамической памяти.}
            \end{scnindent}
        \scnitem{способность памяти кибернетической системы обеспечить хранение
            широкого многообразия знаний}
            \begin{scnindent}
                \scntext{примечание}{имеется в виду хранение гибридных баз знаний}
            \end{scnindent}
    \end{scnrelfromlist}

    \scnheader{способность памяти кибернетической системы обеспечить
        функционирование высококачественного решателя задач}
    \begin{scnrelfromlist}{свойство-предпосылка}

        \scnitem{качество доступа к информации, хранимой памяти кибернетической системы}
        \begin{scnindent}
            \scntext{примечание}{Здесь имеется в виду необходимость перехода от адресного к
                ассоциативному доступу, причем, с расширением многообразия видов реализуемых
                запросов, в частности, к реализации запросов фрагментов баз знаний по заданному
                образцу произвольного размера и произвольной конфигурации.}
        \end{scnindent}
        \scnitem{логико-семантическая гибкость памяти кибернетической системы}
        \scnitem{способность памяти кибернетической системы обеспечить интерпретацию
            широкого многообразия моделей решения задач}

    \end{scnrelfromlist}

    \scnheader{логико-семантическая гибкость памяти кибернетической системы}
    \scnidtf{степень близости физической организации памяти кибернетической системы
        к реализуемым ею базовым семантически целостным действиям над информацией,
        хранимой в памяти}
    \scnidtf{простота реализации базовых семантически целостных действий над
        информацией, хранимой в памяти кибернетической системы}
    \scntext{примечание}{Важен переход от мелких действий, к элементарным действиям,
        имеющим логико-семантический смысл (целостность, законченность)}
    
    \scnheader{качество процессора кибернетической системы}
    \scnrelto{свойство-предпосылка}{качество решателя задач кибернетической
        системы}
    \begin{scnrelfromlist}{свойство-предпосылка}
        \scnitem{способность процессора кибернетической системы обеспечить функционирования высококачественного решателя задач}
    \end{scnrelfromlist}
    \begin{scnrelfromlist}{свойство-предпосылка}
        \scnitem{многообразие моделей решения задач, интерпретируемых процессором
            кибернетической системы}
        \scnitem{простота и качество интерпретации процессором системы широкого
            многообразия моделей решения задач}
        \begin{scnindent}
            \scntext{примечание}{Указанная простота определяется степенью близости
                интерпретируемых моделей решения задач к физическому уровню организации
                процессора кибернетической системы.}
        \end{scnindent}
        \scnitem{обеспечение процессором кибернетической системы качественного
            управления информационными процессами в памяти}
        \begin{scnindent}
            \scntext{примечание}{Речь идет о грамотном сочетание таких аспектов управление
                процессами, как централизация и децентрализация, синхронность и асинхронность,
                последовательность и параллельность.}\scnrelfrom{свойство-предпосылка}{уровень
                параллелизма обработки информации в памяти кибернетической системы}
            \scnidtf{максимальное количество одновременно выполняемых информационных
                процессов в памяти кибернетической системы}
        \end{scnindent}
        \scnitem{быстродействие процессора кибернетической системы}
    \end{scnrelfromlist}

    \scnheader{многообразие моделей решения задач, интерпретируемых
        процессором кибернетической системы}
    \scntext{примечание}{Максимальным уровнем качества процессора кибернетической системы
        по данном параметру является его универсальность, т.е. его принципиальная
        возможность интерпретировать любую модель решения как интеллектуальных, так и
        неинтеллектуальных задач (алгоритмизацию, процедурную параллельную синхронную,
        процедруную параллельную асинхронную, продукционную, нейросетевую,
        генетическую, функциональную, целое семейство моделей). Простота определяется степенью близости интерпретируемых моделей решения задач к
        \scnqq{физическому} уровню организации процессора кибернетической системы. Качественное управление информационными
        процессами в памяти подразумевает грамотное сочетание таких аспектов управление процессами, как
        централизация и децентрализация, синхронность и асинхронность, последовательность и
        параллельность}
        \begin{scnindent}
            \begin{scnrelfromlist}{источник}
                \scnitem{\scncite{Melekhova2018}}
            \end{scnrelfromlist}
        \end{scnindent}
        
    \scnheader{качество сенсоров кибернетической системы}
    \scnrelfrom{свойство-предпосылка}{многообразие видов сенсоров кибернетической системы}
    \begin{scnindent}    
        \scnidtf{многообразие средств восприятия (отображения) информации о текущем
                состоянии внешней среды кибернетической системы и её собственной физической
                оболочки}
    \end{scnindent}
    

    \scnheader{качество эффекторов кибернетической системы}
    \scnrelfrom{свойство-предпосылка}{многообразие видов эффекторов кибернетической системы}
    \begin{scnindent}    
        \scnidtf{многообразие средств воздействия на собственную физическую оболочку
            кибернетической системы и через нее на внешнюю среду этой системы}
        \scntext{примечание}{Эффекторы кибернетической системы являются инструментами
            воздействия кибернетической системы на свою внешнюю среду.}
    \end{scnindent}

    \scnheader{приспособленность физической оболочки кибернетической системы к её
        совершенствованию}
    \scnidtf{приспособленность кибернетической системы к повышению качества её
        физической оболочки}
    \scnidtf{простота ремонта и совершенствования таких компонентов кибернетической
        системы как память, процессор, сенсоры, эффекторы}
    \scnrelfrom{частное свойство}{ремонтопригодность физической оболочки
        кибернетической системы}
    \begin{scnrelfromset}{группа свойств-предпосылок}

        \scnitem{гибкость физической оболочки кибернетической системы}
        \scnitem{стратифицированность физической оболочки кибернетической системы}
        \begin{scnindent}    
            \scnidtf{мобильность физической оболочки кибернетической системы}
            \scnidtf{легкость сохранения целостности физической оболочки кибернетической
                системы при внесении различных изменений (локализация области учета последствий
                внесения изменений, предсказуемость последствий)}
        \end{scnindent}

    \end{scnrelfromset}
    \bigskip
\end{scnsubstruct}
\scnsourcecomment{Завершили Сегмент \scnqqi{Комплекс свойств, определяющих качество физической оболочки кибернетической системы}}

		\scnsegmentheader{Комплекс свойств, определяющих уровень интеллекта
    кибернетической системы}
\begin{scnsubstruct}
    \scnheader{интеллект}
    \scniselement{свойство}
    \scniselement{упорядоченное свойство}
    \scnidtf{уровень (степень, величина) интеллекта кибернетической системы}
    \scnidtf{Семейство классов \textit{кибернетических систем}, обладающих
        эквивалентным (одинаковым) уровнем интеллекта --- от низкого до высокого уровня
        интеллекта}
    \scnidtf{свойство кибернетических систем, характеризующее эффективность их
        взаимодействия со своей средой (средой их жизнедеятельности)}
    \scnrelfrom{область определения}{кибернетическая система}
    \scntext{пояснение}{С формальной точки зрения интеллектуальность --
        это семейство классов кибернетических систем, в каждый из которых входят
        кибернетические системы, эквивалентные по уровню и характеру проявления
        интеллектуальных свойств (в том числе способностей).
        \\Таким образом, характер (вид) интеллектуальных свойств кибернетических
        систем и уровень их развития для разных кибернетических систем может быть
        разным. В соответствии с этим кибернетические системы можно сравнивать между
        собой.}\scntext{примечание}{Основным свойством (характеристикой, качеством,
        параметром) кибернетической системы является уровень (степень) ее интеллекта,
        который является \uline{интегральной} характеристикой, определяющей уровень
        эффективности взаимодействия кибернетической системы со средой своего
        существования.}
        \begin{scnindent}
            \scntext{источник}{\scncite{Zagorskiy2022b}}
        \end{scnindent}
    \scnidtf{комплексное свойство (качество) кибернетической
        системы, определяющее уровень ее выживаемости во внешней среде и предполагающее
        возможность воздействия на эту среду и даже возможность ее преобразования}
    \scnidtf{интеллектуальный потенциал кибернетической системы}
    \scnidtf{спектр знаний, навыков и способностей к обучению кибернетической
        системы}
    \scnidtf{интеллектуальность кибернетической системы}
    \scntext{примечание}{Процесс эволюции \textit{кибернетических систем} следует
        рассматривать как процесс повышения уровня их качества по целому ряду свойств
        (характеристик) и, в первую очередь, как процесс повышения уровня их
        \textit{интеллекта}. При этом можно говорить об эволюции каждой конкретной
        \textit{кибернетической системы} в процессе своей жизнедеятельности, а также об
        эволюции целого класса \textit{кибернетических систем}, когда новые экземпляры
        этого класса являются более интеллектуальными, чем их предшественники. В таком
        аспекте, в частности, можно рассматривать эволюцию \textit{компьютерных систем}
        (искусственных кибернетических систем).}\scntext{примечание}{Очень важно уточнить,
        какими иными свойствами \textit{кибернетических систем} определяется уровень и
        характер их интеллектуальности. Подчеркнем, что \uline{любая}
        \textit{кибернетическая система} обладает соответствующим уровнем
        интеллектуальности. Пусть даже и достаточно низким. Существенным является
        уточнение того, за счет чего уровень интеллектуальности \textit{кибернетической
            системы} может быть повышен. Нет смысла проводить четкую границу между
        \textit{интеллектуальными кибернетическими системами} и неинтеллектуальными. Но
        есть смысл уточнять направления повышения уровня интеллектуальности
        \textit{кибернетических систем.}}\scntext{эпиграф}{Никто не может провести
        линию, отделяющую атмосферу от космоса, или черту, за которой начинается жизнь,
        или границу электронного облака. Все дело в степени проявления свойства.}
    \begin{scnindent}
        \scnrelfrom{автор}{Барт Коско}
    \end{scnindent}
    \scntext{примечание}{Исследователи Искусственного интеллекта определяют интеллект как неотъемлемое свойство машины.
        Их целью является построение систем, которые демонстрируют весь спектр когнитивных способностей, которые мы обнаруживаем у людей}
        \begin{scnindent}
            \begin{scnrelfromlist}{источник}
                \scnitem{\scncite{Gao2002}}
                \scnitem{\scncite{Laird2009}}
            \end{scnrelfromlist}
        \end{scnindent}
    \scntext{примечание}{Прежде, чем говорить о требованиях, предъявляемых к
        \textit{технологии проектирования и производства интеллектуальных компьютерных
            систем (искусственных кибернетических систем}, обладающих высоким уровнем
        \textit{интеллекта)}, необходимо уточнить (детализировать) \textit{свойства},
        присущие указанным системам и являющиеся предпосылками, обеспечивающими высокий
        уровень \textit{интеллекта}. Подчеркнем, что указанные \textit{свойства},
        уточняющие (детализирующие, обеспечивающие, определяющие) \textit{свойства}
        %\bigspace
        \textit{интеллектуальных систем}
        %\bigspace
        (\textit{свойства}, определяющие уровень \textit{интеллекта} этих систем)
        должны быть общими как для искусственных кибернетических систем
        (\textit{компьютерных систем}), так и для \textit{естественных кибернетических
            систем.}}\scnidtf{интегральное качество информационного обеспечения и
        информационных процессов в кибернетической системе}
    \scnidtf{интегральное качество кибернетической системы,
        определяемое:\begin{scnitemize}
            \item уровнем ее образованности --- качеством накопленных к заданному моменту
            знаний и умений (навыков);
            \item уровнем ее обучаемости --- способностью \uline{самостоятельно} повышать
            уровень своей образованности.\end{scnitemize}
    }

    \scnheader{уровень интеллекта кибернетической системы}
    \begin{scnrelfromlist}{свойство-предпосылка}
        \scnitem{образованность кибернетической системы}
        \scnitem{обучаемость кибернетической системы}
        \scnitem{интероперабельность кибернетической системы}
        \begin{scnindent}    
            \scntext{примечание}{Интеллект \textit{кибернетической системы}, как и лежащий в
                его основе познавательный процесс, выполняемый кибернетической системой, имеет
                социальный характер, поскольку наиболее эффективно формируется и развивается в
                форме взаимодействия \textit{кибернетической} системы с другими
                \textit{кибернетическими системами}.}
        \end{scnindent}
    \end{scnrelfromlist}

    \scnheader{образованность кибернетической системы}
    \scnidtf{уровень навыков (умений), а также иных знаний, приобретенных \textit{кибернетической системой} к заданному моменту}
    \begin{scnrelfromlist}{свойство-предпосылка}
    \scnitem{\textbf{качество навыков, приобретенных кибернетической системой}}
        \begin{scnindent}
            \scnidtf{качество умений, которыми владеет кибернетическая система в
                текущий момент}
            \scnrelfrom{свойство-предпосылка}{\textbf{качество информации, хранимой в памяти кибернетической системы}}
                \begin{scnindent}
                    \scnidtf{качество знаний, приобретенных кибернетической системой к заданному моменту}
                \end{scnindent}
        \end{scnindent}
    \scnitem{\textbf{качество информации, хранимой в памяти кибернетической системы}}
        \begin{scnindent}
            \scntext{примечание}{Следует обратить внимание на то, что \textit{качество 
                информации, хранимой в памяти кибернетической системы}, является фактором,
                обеспечивающим не только \textit{качество навыков, приобретенных
                кибернетической системой}, но и общий \textit{уровень качества кибернетической системы}.}
        \end{scnindent}
    \end{scnrelfromlist}

    \scnheader{кибернетическая система}
    \scnrelto{объединение}{Признак интеллектуальности кибернетических систем}
    \begin{scnindent}
        \begin{scneqtoset}
            \scnitem{неинтеллектуальная кибернетическая система}
            \scnitem{интеллектуальная система}
                \begin{scnindent}
                \scnidtf{интеллектуальная кибернетическая система}
                \begin{scnreltoset}{объединение}
                    \scnitem{слабоинтеллектуальная система}
                        \begin{scnindent}
                            \scnidtf{кибернетическая система со слабым интеллектом}
                            \scnidtf{кибернетическая система с низким уровнем интеллекта}
                            \scnidtf{кибернетическая система с элементами интеллекта}
                        \end{scnindent}
                    \scnitem{высокоинтеллектуальная система}
                        \begin{scnindent}
                            \scnidtf{идеальная интеллектуальная система}
                            \scnidtf{кибернетическая система с сильным интеллектом}
                            \scnidtf{кибернетическая система с высоким уровнем интеллекта}
                            \scnidtf{действительно интеллектуальная система}
                        \end{scnindent}
                \end{scnreltoset}
            \end{scnindent}
        \end{scneqtoset}
    \end{scnindent}

    \scnheader{Признак интеллектуальности кибернетических систем}
    \scntext{примечание}{Данный признак классификации кибернетических систем формально
        является не разбиением, а покрытием множества \textit{кибернетических систем},
        так как отсутствует четкая грань между неинтеллектуальными и интеллектуальными
        кибернетическими системами, а также между слабоинтеллектуальными и
        высокоинтеллектуальными кибернетическими системами.}
        
    \scnheader{интеллектуальная система}
    \scnidtf{интеллектуальная кибернетическая система}
    \scntext{примечание}{В этом термине слово кибернетическая можно опустить, так как
        интеллектуальными могут быть только \textit{кибернетические системы}}
    \scntext{примечание}{Интеллектуальные кибернетические системы могут быть
        \textit{естественными интеллектуальными системами}, искусственными
        интеллектуальными системами (которые будем называть \textit{интеллектуальными
        компьютерными системами}), а также естественно-искусственными интеллектуальными
        системами, состоящими из компонентов как естественного, так и искусственного
        происхождения. Важнейшим примером естественно-искусственных интеллектуальных
        систем являются человеко-машинные системы, представляющие собой коллективы
        (многоагентные системы), состоящие из \textit{интеллектуальных компьютерных
        систем} и людей (конечных пользователей и разработчиков этих компьютерных систем).}
    \scntext{примечание}{Вводя понятие \textit{интеллектуальной
        системы}, важно, во-первых, уточнить понятие \textit{кибернетической системы} и
        определить те свойства, которые присущи \uline{всем} кибернетическим системам,
        и, во-вторых, локализовать ту условную \uline{грань} перехода от
        неинтеллектуальных \textit{кибернетических систем} к интеллектуальным, а также
        \uline{грань} перехода от слабоинтеллектуальных к высокоинтеллектуальным
        кибернетическим системам. В этом и заключается уточнение феномена
        \textit{интеллекта} (интеллектуальности) кибернетических систем.}
    \scntext{примечание}{Все \textit{свойства} (в том числе способности и
        активности), присущие \textit{кибернетическим системам}, в различных
        \textit{кибернетических системах} могут иметь самый различный уровень (уровень
        развития). Более того, в некоторых \textit{кибернетических системах} некоторые
        из этих свойств могут вообще отсутствовать. При этом в кибернетических
        системах, которые условно будем называть \textit{\textbf{интеллектуальными
        системами}}, \uline{все} указанные выше свойства должны быть представлены в
        достаточно развитом виде. Заметим также, что мы называем
        \textit{интеллектуальными системами}, иногда называют кибернетическими
        системами с сильным интеллектом (с высоким уровнем интеллекта),
        противопоставляя их кибернетическим системам со слабым интеллектом (с низким
        уровнем интеллекта).}
    \scnsubset{образованная кибернетическая система}
    \begin{scnindent}
        \scnidtf{кибернетическая система, имеющая высокий уровень образованности}
        \scnidtf{кибернетическая система, обладающая высоким уровнем знаний и навыков}
        \scnsubset{кибернетическая система, основанная на знаниях}
        \scnsubset{кибернетическая система, управляемая знаниями}
        \scnsubset{целенаправленная кибернетическая система}
        \scnsubset{гибридная кибернетическая система}
        \scnsubset{потенциально универсальная кибернетическая система}
    \end{scnindent}
    \scnsubset{обучаемая кибернетическая система}
    \begin{scnindent}
        \scnidtf{когнитивная кибернетическая система}
        \scnidtf{кибернетическая система, имеющая высокий уровень обучаемости}
        \scnsubset{кибернетическая система с высоким уровнем стратифицированности своих знаний и навыков}
        \scnsubset{рефлексивная кибернетическая система}
        \scnsubset{самообучаемая кибернетическая система}
        \scnsubset{кибернетическая система с высоким уровнем познавательной активности}
    \end{scnindent}
    \scnsubset{социально ориентированная кибернетическая система}
    \begin{scnindent}
        \scnidtf{кибернетическая система, имеющая высокий уровень интероперабельности}
        \scnsubset{кибернетическая система, способная устанавливать и поддерживать
            высокий уровень семантической совместимости и взаимопонимания с другими системами}
        \scnsubset{договороспособная кибернетическая система}
        \begin{scnindent}
            \scnidtf{кибернетическая система, способная координировать (согласовывать) свою
                деятельность с другими системами}
        \end{scnindent}
    \end{scnindent}

    \scnheader{кибернетическая система, основанная на знаниях}
    \scnidtf{кибернетическая система, в основе которой лежит формируемая в ее
        памяти, постоянно совершенствуемая и структурированная информационная модель
        той среды, в рамках которой она существует и решает соответствующие задачи}
    \scnidtf{кибернетическая система, в основе которой лежит ее база знаний --
        систематизированная совокупность всех используемых ею знаний}
    \scnidtf{кибернетическая система, формирующая в своей памяти
        систематизированную информационную модель среды своего обитания и использующая
        эту модель для организации своего целенаправленного поведения}

    \scnheader{кибернетическая система, управляемая знаниями}
    \scnidtf{кибернетическая система, в которой выполняемые ею действия
        инициируются соответствующими ситуациями и/или событиями, возникающими в ее
        базе знаний}

    \scnheader{целенаправленная кибернетическая система}
    \scnidtf{субъект, осознанно и целенаправленно осуществляющий свою деятельность,
        ведающий то, что он творит}

    \scnheader{обучаемая кибернетическая система}
    \scnidtf{когнитивная система}
    \scnidtf{кибернетическая система, способная познавать (изучать) среду своего
        обитания, то есть строить и постоянно уточнять в своей памяти информационную
        модель (описание) этой среды, а также использовать эту модель для решения
        различных задач (для организации своей деятельности (поведения)) в указанной
        среде}
    \scnidtf{кибернетическая система, способная к самосовершенствованию}

    \scnheader{социально ориентированная кибернетическая система}
    \scnidtf{кибернетическая система, имеющая достаточно высокий уровень
        интеллекта, чтобы быть полезным членом различных, в том числе, и
        человеко-машинных сообществ}
    \scntext{примечание}{Определенный уровень социально значимых качеств является
        необходимым условием интеллектуальности кибернетической системы. Это, своего
        рода, модификация теста Тьюринга --- важна не имитация, не иллюзия
        человекоподобия, а \uline{реальная} польза в процессе коллективного решения
        сложных задач.}
        
    \scnheader{интеллектуальная компьютерная система}
    \scnidtf{искусственная интеллектуальная система}
    \scnidtf{искусственная кибернетическая система, обладающая высоким уровнем
        интеллекта (высоким уровнем знаний и умений), а также высоким уровнем обучаемости}
    \scnsubset{компьютерная система}
    \begin{scnindent}
        \scnsubset{кибернетическая система}
    \end{scnindent}
    \scnidtftext{основной sc-идентификатор}{интеллектуальная компьютерная система}
    \begin{scnindent}
        \scntext{сокращение}{и.к.с.}
    \end{scnindent}
    \scnidtf{система искусственного интеллекта}
    \scnidtf{искусственная интеллектуальная система}
    \scntext{примечание}{Все свойства, присущие кибернетическим системам, в различных кибернетических системах могут иметь самый
    различный уровень. Более того, в некоторых кибернетических системах некоторые из этих свойств могут вообще
    отсутствовать. При этом в кибернетических системах, которые условно будем называть интеллектуальными
    системами, все указанные выше свойства должны быть представлены в достаточно развитом виде.}
    \scnsubset{интеллектуальная система}
    \begin{scnindent}
        \scnsubset{кибернетическая система}
    \end{scnindent}
    \bigskip
\end{scnsubstruct}
\scnsourcecomment{Завершили Сегмент \scnqqi{Комплекс свойств, определяющих уровень интеллекта кибернетической системы}}

		\scnsegmentheader{Комплекс свойств, определяющих качество информации, хранимой
    в памяти кибернетической системы}
\begin{scnsubstruct}
    \scnheader{информация}
    \scnidtf{информационная конструкция}
    \scnidtf{информационная модель, состоящая из некоторого множества различных
        \textit{знаков}, обозначающих моделируемые (описываемые) \textit{сущности}
        любого вида и, в частности, \textit{знаков}, обозначающих различного вида
        \textit{связи} между \textit{знаками} описываемых \textit{сущностей} (такие
        \textit{связи} чаще всего являются отражениями (моделями) \textit{связей} между
        \textit{сущностями}, которые обозначаются связываемыми \textit{знаками})}
    \begin{scnindent}
        \scntext{примечание}{Подчеркнем, что \textit{связи} между \textit{знаками}
            описываемых \textit{сущностей} сами также могут быть описываемыми
            \textit{сущностями}, но для этого указанные \textit{связи} в рамках
            информационной модели должны быть представлены своими \textit{знаками}. Не все
            \textit{связи} между \textit{знаками} являются описываемыми
            \textit{сущностями}. Такими неописываемыми связями являются связи инцидентности
            знаков.}
    \end{scnindent}
    \scnidtf{конфигурация знаков}
    \scnidtf{знаковая конструкция}
    \scnidtf{текст}
    \scnidtf{описание (отражение) некоторого множества (1) первичных сущностей, (2)
        понятий, (3) связей между ними, (4) связей между связями, (5) фрагментов
        данного описания, (6) связей между этими фрагментами}
    \scnsuperset{дискретная информационная конструкция}
    \begin{scnindent}
        \scnidtf{информационная конструкция, у которой все входящие в неё знаки имеют
            чёткие границы}
        \scnsuperset{дискретная информационная конструкция, у которой входящие в неё
            знаки имеют \uline{условную} структуру}
    \end{scnindent}
    \scnidtf{информационная модель}
    \scnidtf{информационная модель (отражение, описание) некоторого множества
        связей между некоторым описываемыми (рассматриваемыми, исследуемыми,
        изучаемыми) сущностями}
    \scntext{определение}{Множество всевозможных информационных конструкций
        (понятие информационной конструкции) представляет собой множество, на котором
        задано

        \begin{scnitemize}

            \item Отношение \uline{синтаксической} эквивалентности и, соответственно,
            семейство классов синтаксической эквивалентности информационных конструкций

            \item Отношение \uline{семантической} эквивалентности и, ответственно,
            семейство классов семантической эквивалентности информационных конструкций

            \item Отношение \uline{логической} эквивалентности и, соответственно, семейство
            классов логической эквивалентности информационных конструкций.

        \end{scnitemize}
        При этом можно говорить об инварианте каждого класса синтаксически
        эквивалентных информационных конструкций, об инварианте каждого класса
        семантически эквивалентных информационных конструкций и об инварианте каждого
        класса логически эквивалентных информационных конструкций синтаксически
        эквивалентные информационные конструкции могут отличаться вариантами
        изображения букв (различным почерком, разными шрифтами), вариантами разрезания
        текста на страницы и на строчки.Семантически эквивалентные информационные
        конструкции могут отличаться разными именами, обозначающими одни и те же
        сущности, разным порядком размещения этих имён.}

    \scnheader{денотационная семантика информационной конструкции}
    \scntext{пояснение}{Каждая информационная конструкция имеет денотационную
        семантику, описывающую то, как связаны входящие в информационную конструкцию
        знаки с соответствующими им денотатами (т.е. сущностями, обозначаемыми этими
        знаками).}
        
    \scnheader{сенсорная информация}
    \scnsubset{информация}
    \scnidtf{первичная информация, приобретаемая кибернетической системы с помощью
        её сенсоров (рецепторов)}
    \scnidtf{первичная информация}
    \scntext{примечание}{Подчеркнем, что \textit{сенсорная информация}
        %\bigspace
        \textit{кибернетической системы} с точки зрения её \textit{денотационной
            семантики} является простейшим видом \textit{знаковой конструкции}, в которой
        \textit{внешняя среда}
        %\bigspace
        \textit{кибернетической системы} описывается

        \begin{scnitemize}

            \item путём задания параметрического пространства (множество параметров,
            признаков, \textit{свойства}, характеристик), с помощью которого описываются
            состояние элементарных (атомарных) фрагментов \textit{внешней среды}, которые
            непосредственно являются смежными (соприкасаются с) чувствительными
            поверхностями \textit{сенсоров кибернетической системы};
            \item путём пространственной декомпозиции наблюдаемой \textit{внешней среды} с
            выделением указанных выше элементарных фрагментов этой среды (элементарных с
            точки зрения  \textit{сенсоров кибернетической системы}) и с явным описанием
            пространственных связей между указанными элементарными фрагментами (эти связи
            соответствует пространственным связям между сенсорами);
            \item путём темпоральной декомпозиции наблюдаемой \textit{внешней среды},
            которая предполагает фиксацию моментов времени для каждого события по изменению
            состояния измеряемого параметра каждого элементарного фрагмента наблюдаемой
            \textit{внешней среды}
        \end{scnitemize}
    }
    \scntext{примечание}{Качество (в частности, информативность) \textit{сенсорной
            информации} обеспечивается:
        \begin{scnitemize}

            \item качеством используемого параметрического пространства
            \begin{scnitemizeii}

                \item многообразием видов \textit{сенсоров}, т.е. многообразием параметров
                (свойств), с помощью которых описывается внешняя среда

                \item информативностью каждого из указанных параметров

                \item целостностью (полнотой, достаточностью) всего набора рассматриваемых
                параметров

                \item отсутствием избыточности в наборе этих параметров
            \end{scnitemizeii}

            \item общим количеством сенсоров и количеством сенсоров, соответствующих
            каждому параметру

            \item способностью кибернетической системы перемещать сенсоры в пространстве

        \end{scnitemize}
    }\scntext{примечание}{\textit{сенсорная информация} обеспечивает формирование
        первичного описания состояния и динамики изменения не только \textit{внешней
            среды кибернетической системы}, но также и её физической оболочки, которую
        можно рассматривать как часть всей \textbf{\textit{физической среды
                кибернетической системы}}, противопоставляя такую \textit{физическую среду
            кибернетической системы} её внутренней (информационной, \uline{абстрактной})
        среде, в которой хранится и обрабатывается \textit{информация}, используемая
        \textit{кибернетической системой}. Указанную абстрактную внутреннюю среду
        кибернетической системы будем называть \textbf{\textit{абстрактной памятью
                кибернетической системы}}.}
                
    \scnheader{язык}
    \scnidtf{множество информационных конструкции, построенных по общим
        синтаксическим и семантическим правилам}
    \scnsuperset{внутренний язык кибернетической системы}
    \begin{scnindent}
        \scnidtf{язык, используемый кибернетической системой для представления
            информации, хранимой в её памяти}
    \end{scnindent}

    \scnheader{информация, хранимая в памяти кибернетической системы}
    \scnidtf{совокупность \uline{всей} информации, хранимой в памяти
        кибернетической системы}
    \scnsubset{информация}

    \scnheader{качество информации, хранимой в памяти кибернетической системы}
    \scnidtf{качество знаний, приобретенных кибернетической системой к текущему
        моменту}
    \scnidtf{уровень качества хранимой информации}
    \scnidtf{качество информационной модели среды кибернетической системы, хранимой
        в её памяти}
    \scnidtf{уровень качества хранимых в памяти кибернетической системы внутренней
        информационной модели среды существования (жизнедеятельности) этой
        кибернетической системы}
    \scnidtf{интегральное качество знаний, накопленных кибернетической системой к
        текущему моменту}
    \scnidtf{степень приближения информации, хранимой в памяти кибернетической
        системы к качественной информационной модели той среды, в которой существует
        кибернетическая система, к систематизированной базе знаний, описывающей все
        свойства этой среды, необходимые для функционирования этой кибернетической
        системы}
    \scnidtf{качество хранимой в памяти кибернетической системы информационной
        модели среды жизнедеятельности этой системы}
    \scntext{примечание}{Качество информационной модели среды обитания  кибернетической
        системы, в частности, определяется
        \begin{scnitemize}

            \item корректностью этой модели (отсутствием в ней ошибок);

            \item адекватностью этой модели;

            \item полнотой --- достаточностью находящейся в ней информации для эффективного
            функционирования кибернетической системы;

            \item структурированностью, систематизированностью.

        \end{scnitemize}
        Важнейшим этапом эволюции информационной модели среды кибернетической системы
        является переход от недостаточно полной и несистематизированной информационные
        модели среды к \textit{базе знаний}. Именно поэтому важнейшим этапом повышения
        уровня интеллектуальности компьютерной систем является переход от традиционных
        компьютерных систем к компьютерным системам, основанным на знаниях.}
    \scnrelfrom{комплекс свойств-предпосылок}{не-фактор}
    \begin{scnrelfromlist}{свойство-предпосылка}

        \scnitem{семантическая мощность языка представления информации в памяти
            кибернетической системы}
        \scnitem{ объём информации, загруженной в память кибернетической системы}
        \scnitem{ степень конвергенции и интеграции различного вида знаний, хранимых в
            памяти кибернетической системы}
        \scnitem{ стратифицированность информации, хранимой в памяти кибернетической
            системы}
        \scnitem{простота и локальность выполнения семантически целостных операций над
            информацией, хранимой в памяти кибернетической системы}

    \end{scnrelfromlist}

    \scnheader{не-фактор}
    \scnidtf{группа семантических свойств, определяющих качество информации,
        хранимой в памяти кибернетической системы}
    \begin{scneqtoset}

        \scnitem{корректность/некорректность информации, хранимой в памяти
            кибернетической системы}
        \scnitem{однозначность/неоднозначность информации, хранимой в памяти
            кибернетической системы}
        \scnitem{целостность/нецелостность информации, хранимой в памяти
            кибернетической системы}
        \scnitem{чистота/загрязненность информации, хранимой в памяти кибернетической
            системы}
        \scnitem{достоверность/недостоверность информации, хранимой в памяти
            кибернетической системы}
        \scnitem{точность/неточность информации, хранимой в памяти кибернетической
            системы}
        \scnitem{четкость/нечеткость информации, хранимой в памяти кибернетической
            системы}
        \scnitem{определенность/недоопределенность информации, хранимой в памяти
            кибернетической системы}

    \end{scneqtoset}
    \scntext{пояснение}{Ярушкина.Н.Г.НечетГС-2007кн.-стр.10-28}
    \begin{scnindent}
        \scnrelto{цитата}{\cite{YarushinaHS}}
    \end{scnindent}

    \scnheader{корректность/некорректность информации, хранимой в памяти
        кибернетической системы}
    \scnidtf{уровень адекватности хранимой информации той среде, в которой
        существует кибернетическая система и информационной моделью которой эта
        хранимая информация является}

    \scnheader{непротиворечивость/противоречивость информации, хранимой в памяти
        кибернетической системы}
    \scnidtf{уровень присутствия в хранимой информации различного вида противоречий
        и, в частности, ошибок}

    \scnheader{противоречие*}
    \scnidtf{пара противоречащих друг другу фрагментов информации, хранимой в
        памяти кибернетической системы*}
    \scntext{примечание}{Чаще всего противоречащими друг другу информационными
        фрагментами являются:

        \begin{scnitemize}

            \item явно представленная в памяти некоторая закономерность (некоторое правило)

            \item информационный фрагмент, не соответствующий (противоречащий) указанной
            закономерности

        \end{scnitemize}
        \bigskip
        В этом случае некорректность может присутствовать:

        \begin{scnitemize}

            \item либо в информационном фрагменте, который противоречит указанной
            закономерности;

            \item либо в самой этой закономерности;

            \item либо и там и там.

        \end{scnitemize}
    }
    \scnheader{информационная ошибка}
    \scnidtftext{определение}{противоречие, заключающееся в нарушении некоторой
        закономерности (некоторого правила), которая не подвергается сомнению}
    
    \scnheader{информационная ошибка}
    \scntext{примечание}{Ошибки (ошибочные фрагменты) в хранимой информации могут быть
        синтаксическими и семантическими, противоречащими некоторым правилам
        (закономерностям), которые явно в памяти могут быть не представлены и считаются
        априори истинными.}
        
    \scnheader{полнота/неполнота информации, хранимой в памяти кибернетической системы}
    \scnidtf{уровень того, насколько информация, хранимая в памяти кибернетической
        системы, описывает среду существования этой системы и используемые ею методы
        решения задач достаточно полно (достаточно детально) для того, чтобы
        кибернетическая система могла действительно решать все множество
        соответствующих ей задач}
    \scnidtf{уровень соответствия хранимой информации объёму задач (действий),
        которые соответствующая кибернетическая система желает уметь решать
        (выполнять)}
    \scnidtf{степень достаточности информации, хранимой в памяти кибернетической
        системы, для достижения целей этой системы, для выполнения своих обязанностей}
    \scntext{примечание}{Чем полнее информация, хранимая в памяти кибернетической
        системы, чем полнее \uline{информационное обеспечение деятельности этой
            системы} это системы, тем эффективнее (качественнее) сама эта
        деятельность.}\begin{scnrelfromlist}{свойство-предпосылка}

        \scnitem{многообразие видов знаний, хранимых в памяти кибернетической системы}
        \scnitem{структурированность информации, хранимой в памяти кибернетической
            системы}

    \end{scnrelfromlist}

    \scnheader{однозначность/неоднозначность информации, хранимой в кибернетической
        системе}
    \begin{scnrelfromlist}{свойство-предпосылка}

        \scnitem{многообразие форм дублирования информации, хранимой в памяти
            кибернетической системы}
        \scnitem{частота дублирования информации, хранимой в памяти кибернетической
            системы}

    \end{scnrelfromlist}

    \scnheader{целостность/нецелостность информации, хранимой в памяти
        кибернетической системы}
    \scnidtf{уровень содержательной информативности информации, хранимой в памяти
        кибернетической системы}
    \scnidtf{уровень того, насколько содержательно (семантически) \uline{связной}
        является информация, хранимая в памяти кибернетической системы, насколько полно
        специфицированы \uline{все} описываемые в памяти сущности (путём описания
        необходимого набора связей этих сущностей с другими описываемыми сущностями),
        насколько редко или часто в рамках хранимой информации встречаются
        \textit{информационные дыры}, соответствующие явной недостаточности некоторых
        спецификаций}
    \scnidtf{известность/неизвестность информации, хранимой в памяти
        кибернетической системы}
    \scnidtf{многообразие форм и частота присутствия \textit{информационных дыр} в
        информации, хранимой в памяти кибернетической системы}

    \scnheader{информационная дыра в информации, хранимой в памяти кибернетической
        системы}
    \scnidtf{информация, отсутствие которой в памяти кибернетической системы
        существенно усложняет деятельность этой системы}
    \scntext{примечание}{Примерами информационных дыр являются:
        \begin{scnitemize}
            \item отсутствующий метод решения часто встречающихся задач;
            \item отсутствующее определение используемого определяемого понятия;
            \item недостаточно подробная спецификация часто рассматриваемых сущностей
        \end{scnitemize}
    }
    
    \scnheader{чистота/загрязненность информации, хранимой в памяти
        кибернетической системы}
    \scnidtf{многообразие форм и общее количество информационного мусора, входящего
        в состав информации, хранимой в памяти кибернетической системы}
    
    \scnheader{информационный мусор, входящий в состав информации, хранимой в
        памяти кибернетической системы}
    \scnidtf{информационный фрагмент, входящий в состав информации, хранимой в
        памяти кибернетической системы, удаление которого существенно \uline{не}
        усложнит деятельность кибернетической системы}
    \scntext{примечание}{Примерами информационного мусора являются:
        \begin{scnitemize}
            \item информация, которая нечасто востребована, но при необходимости может быть легко логически выведена
            \item информация, актуальность которой истекла
        \end{scnitemize}
    }
    
    \scnheader{семантическая мощность языка представления информации в памяти
        кибернетической системы}
    \scnidtf{семантическая мощность внутреннего языка кибернетической системы}
    \scnrelfrom{свойство-предпосылка}{гибридность информации, хранимой в памяти
        кибернетической системы}
    \newpage\scntext{примечание}{Универсальность внутреннего языка кибернетической
        системы является важнейшим фактором её
        интеллектуальности}
        
    \scnheader{универсальный язык}
    \scnidtf{язык, информационные конструкции которого могут представить (описать)
        \uline{любую} конфигурацию \uline{любых} связей между \uline{любыми}
        сущностями}
    
    \scnheader{гибридность информации, хранимой в памяти кибернетической системы}
    \begin{scnrelfromlist}{свойство-предпосылка}

        \scnitem{многообразие видов знаний, хранимых в памяти кибернетической системы}
        \scnitem{степень конвергенции и интеграции различного вида знаний, хранимых в
            памяти кибернетической системы}

    \end{scnrelfromlist}
    
    \scnheader{многообразие видов знаний, хранимых в памяти кибернетической системы}
    \begin{scnrelfromlist}{частное свойство}

        \scnitem{рефлексивность информации, хранимой в памяти кибернетической системы}
        \begin{scnindent}    
            \scnidtf{многообразие видов метаинформации (метазнаний), хранимых в
                    памяти кибернетической системы}
        \end{scnindent}
        
        \scnitem{многообразие моделей решения задач, используемых кибернетической
            системой}
        \scnitem{многообразие видов целей, анализируемых или синтезируемых
            кибернетической системой}
        \scnitem{многообразие планов решения задач, решаемых кибернетической системой}
        \scnitem{многообразие протоколов решения задач, решаемых кибернетической
            системой}

    \end{scnrelfromlist}

    \scnheader{объем информации, хранимой в памяти кибернетической системы}
    \scnidtf{объем знаний, приобретенных кибернетической системой к текущему
        моменту}
    \scnidtf{содержательная совокупность всех знаний, хранимых в текущий момент в
        памяти кибернетической системы}
    \scntext{примечание}{Чем больше кибернетическая система знает, тем при прочих равных
        условиях выше уровень её качества}
        
    \bigskip
    \scnheader{степень конвергенции и интеграции различного вида знаний, хранимых в памяти кибернетической системы}
    \scnidtf{уровень бесшовной  интеграции различного вида знаний кибернетической
        системы}
    \scntext{примечание}{Максимальный уровень конвергенции и интеграции знаний (в том
        числе,	и знаний различного вида) предполагает:
        \begin{scnitemize}

            \item использование универсального базового языка, по отношению к которому всем
            используемым видам знаний соответствуют специализированные языки, являющиеся
            подъязыками указанного базового языка
            \item построение четкой иерархии указанных специализированных языков по
            принципу язык-подъязык
            \item явное введение семейства отношений, заданных на множестве различных
            знаний и, в том числе, связывающих знания различного вида
        \end{scnitemize}
    }\scnrelfrom{свойство-предпосылка}{уровень формализованности информации,
        хранимой в памяти кибернетической системы}

    \scnheader{уровень формализованности информации, хранимой в памяти
        кибернетической системы}
    \scnidtf{степень приближения информации, хранимой в памяти кибернетической
        системы, к максимально простой и компактной форме представления информационной
        модели некоторого множества описываемых сущностей, которая является отражением
        определенной конфигурации связей между указанными сущностями}
    \scntext{примечание}{Высшим уровнем формализации информации, хранимой в памяти
        кибернетической системы, является смысловое представление информации в форме
        семантических сетей. Смотрите Раздел \scnqqi{\textit{Предметная область и онтология
            семантических сетей, семантических языков и семантических моделей баз
            знаний}}.}
    \scnrelboth{следует отличать}{формализация*}
    \begin{scnindent}
        \scnidtf{Бинарное ориентированное отношение, каждая пара которого связывает
            некоторую информационную конструкцию с другой информационной конструкцией,
            которая семантически эквивалентна первой, но имеет более высокий уровень
            формализованности}
        \scntext{примечание}{Приобретение навыков формального представления информации не
            является простой проблемой даже для человека. По сути совокупность таких
            навыков --- это основа математической культуры, культуры точного изложения своих
            соображений. Некоторые примеры, иллюстрирующие нетривиальность проблемы
            смотрите в \cite{Arnold2012}}
    \end{scnindent}
    \scnrelboth{следует отличать}{формализация}
    \begin{scnindent}
        \scnidtf{деятельность, направленная на повышение уровня формализованности
            представление информации}
        \scntext{метафора}{сближение синтаксиса с семантикой --- сближение
            синтаксической структуры информационной конструкции с её смысловой структурой}
    \end{scnindent}
    \scnidtf{уровень способности кибернетической системы к формальному
        представлению знаний и используемых понятий, к рационализации идей}
    \scnidtf{степень близости языка внутреннего представления (способа внутреннего
        кодирования) информации в памяти кибернетической системы к смысловому
        представлению информации}
    \scnidtf{степень близости к изоморфизму соответствия между: (1) синтаксической
        структурой внутреннего представления информации в памяти кибернетической
        системы и (2) конфигурацией связей описываемых сущностей}
    \begin{scnrelfromlist}{свойство-предпосылка}

        \scnitem{многообразие форм дублирования информации, хранимой в памяти
            кибернетической системы}
        \scnitem{относительный объём дублирования информации, хранимой в памяти
            кибернетической системы }
        \scnitem{многообразие фрагментов хранимой информации, не являющихся ни знаками,
            ни конфигурациями знаков }
        \scnitem{компактность представления информации, хранимой в памяти
            кибернетической системы}

    \end{scnrelfromlist}
    \scnheader{смысловое представление информации}
    \scnidtftext{пояснение}{способ представления информации, в котором
        минимизируются чисто синтаксические  аспекты представления информационных
        конструкций, не имеющие непосредственной семантической интерпретации}
    \scntext{примечание}{Примерами чисто синтаксических  аспектов представления
        информационных конструкций являются:
        \begin{scnitemize}

            \item буквы, которые входят в состав слов и которые, следовательно, не являются
            знаками описываемых сущностей;
            \item алфавиты букв различных языков;
            \item знаки препинания (разделители и ограничители);
            \item инцидентность (порядок, последовательность) букв и других символов,
            входящих в состав информационной конструкции.
        \end{scnitemize}
    }
    \begin{scnindent}
        \scntext{следовательно}{Информационная конструкция, представленная на
            каком-либо привычном для нас языке, является достаточно громоздкой
            информационной конструкцией, смысл которой (т.е знаки описываемых сущностей и
            семантически интерпретируемые связи между знаками, отражающие соответствующие
            связи между обозначаемыми сущностями) сильно закамуфлирован. Это существенно
            усложняет обработку информации. если пытаться реализовывать осмысленные  модели
            решения задач, для которых смысловые  аспекты обрабатываемой информации
            являются ключевыми.}
    \end{scnindent}
    \scntext{примечание}{Существенно подчеркнуть, что приближение внутреннего
        представления информации в памяти кибернетической системы к смысловому
        представлению информации является важнейшим фактором упрощения решателя задач
        кибернетической системы при реализации сложных моделей решения задач, требующих
        глубокого анализа смысла обрабатываемой информации. А это, в свою очередь,
        является важнейшим фактором качества решателя задач кибернетической
        системы.}
        
    \newpage\scnheader{многообразие форм дублирования информации, хранимой
        в памяти кибернетической системы}
    \scnidtf{многообразие видов семантической эквивалентности фрагментов
        информации, хранимой в памяти кибернетической системы}
    \scntext{примечание}{Простейшим видом семантической эквивалентности является
        синонимия знаков, когда два разных фрагмента хранимой информации являются
        знаками, имеющими один и тот же денотат (т. е обозначающими одну и ту же
        сущность).}
        
    \scnheader{относительный объем дублирования информации, хранимой в
        памяти кибернетической системы}
    \scnidtf{частота присутствия в хранимой информации семантически эквивалентных
        информационных фрагментов и, в частности, синонимичных знаков}
    
    \scnheader{многообразие фрагментов хранимой информации, не являющихся ни
        знаками, ни конфигурациями знаков}
    \scntext{примечание}{Примерами фрагментов хранимой информации, не являющихся знаками
        или конфигурациями знаков, являются:
        \begin{scnitemize}

            \item буквы, входящие в состав слов
            \item слова, входящие в состав словосочетаний
            \item различного вида разделители, знаки препинания
            \item различного вида ограничители.
        \end{scnitemize}
    }
    
    \scnheader{компактность представления информации, хранимой в памяти
        кибернетической системы}
    \scntext{примечание}{Должно уменьшаться число элементов памяти, используемых для
        представления информации, т.е. необходим переход к более компактным, но
        семантически эквивалентным информационным
        конструкциям.}
        
    \scnheader{стратифицированность информации, хранимой в памяти
        кибернетической системы}
    \scnrelfrom{свойство-предпосылка}{структурированность информации, хранимой в
        памяти кибернетической системы}
    \scnidtf{способность кибернетической системы выделять такие разделы информации,
        хранимой в памяти этой системы, которые бы ограничивали области действия
        агентов решателя задач кибернетической системы, являющиеся достаточными для
        решения заданных задач}
    \begin{scnindent}
        \scntext{примечание}{Существует правило, позволяющее каждой заданной задаче поставить
            в соответствие априори известный (выделенный) раздел хранимой информации,
            являющийся областью действия агентов решателя, осуществляющих решение заданной
            задачи. Основными видами такого рода разделов хранимой информации являются
            \textit{предметные области} и \textit{онтологии}.}
    \end{scnindent}
    \scnrelfrom{свойство-предпосылка}{рефлексивность
        информации, хранимой в памяти кибернетической системы}
    \scnidtf{уровень систематизации знаний, хранимых в памяти кибернетической
        системы}
    \scnidtf{уровень перехода от неструктурированных или слабоструктурированных
        данных к хорошо структурированным базам знаний}
    \scnidtf{уровень перехода от первичной информации к метаинформации,
        метаметаинформации и т.д.}
    
    \scnheader{рефлексивность информации,хранимой в памяти кибернетической системы}
    \scnidtf{уровень применения средств самоописания (метаязыковых средств) в
        информации, хранимой в памяти кибернетической системы}
    \scnidtf{относительный, объём и многообразие метаинформации, хранимой в памяти
        кибернетической системы}
    \scntext{примечание}{рефлексивность информации, хранимой в памяти кибернетической
        системы, т.е. наличие метаязыковых средств, является фактором, обеспечивающим
        не только структуризацию хранимой информации, но возможность описания
        синтаксиса и семантики самых различных языков, используемых кибернетической
        системой.}
        
    \newpage\scnheader{простота и локальность выполнения семантически
        целостных операций над информацией, хранимой в памяти кибернетической системы}
    \scntext{примечание}{Данное свойство касается не самой информации, хранимой в памяти,
        а язык кодирования (представления) информации в памяти кибернетической
        системы}\scnidtf{гибкость выполнения семантически целостных операций над
        информацией, хранимой в памяти кибернетической системы}
    
    \scnheader{база знаний}
    \scnidtf{база знаний кибернетической системы}
    \scnsubset{информация, хранимая в памяти кибернетической системы}
    \scnidtftext{пояснение}{информация, хранимая в памяти кибернетической системы
        и имеющая высокий уровень качества по всем показателям и, в частности, высокий
        уровень:
        \begin{scnitemize}

            \item \textit{семантической мощности языка представления информации хранимой в
                памяти кибернетической системы} (в базе знаний указанный язык должен быть
            универсальным);
            \item \textit{гибридности информации, хранимой в памяти кибернетической
                системы};
            \item \textit{многообразия видов знаний, хранимых в памяти кибернетической
                системы};
            \item формализованности информации, хранимой в памяти кибернетической системы;
            \item \textit{структурированности информации, хранимой в памяти кибернетической
                системы}
        \end{scnitemize}
    }
    \scntext{примечание}{Переход \textit{информации, хранимой в памяти кибернетической
            системы} на уровень качества, соответствующий \textit{базам знаний}, является
        важнейшим этапом эволюции \textit{кибернетических систем}. Подчеркнем при этом,
        что \textit{базы знаний} по уровню своего качества могут сильно отличаться друг
        от друга.}
    \bigskip
\end{scnsubstruct}
\scnsourcecomment{Завершили Сегмент \scnqqi{Комплекс свойств, определяющих качество информации, хранимой в памяти кибернетической системы}}
		\scnsegmentheader{Комплекс свойств, определяющих качество решателя задач
    кибернетической системы}

\begin{scnsubstruct}
    \scnheader{качество решателя задач кибернетической системы}
    \scnidtf{интегральная качественная оценка множества задач (действий), которые
        кибернетическая система способна выполнять в заданный момент}
    \scnidtf{качество навыков, приобретенных кибернетической системой}
    \scntext{примечание}{Основным свойством и назначением \textit{решателя задач
            кибернетической системы} является способность решать \textit{задачи} на основе
        накапливаемых (приобретаемых) \textit{кибернетической системой} различного вида
        \textit{навыков} с использованием \textit{процессора кибернетической системы},
        являющегося универсальным интерпретатором всевозможных накопленных
        \textit{навыков}. При этом качество (уровень развития, уровень совершенства)
        указанной способности определяется целым рядом дополнительных факторов
        (свойств).}
    \scnidtf{интеллектуальный уровень качества решателя задач
        кибернетической системы}
    \scnidtf{интегральное качество умений (навыков), приобретенных
        \textit{кибернетической системой} к текущему моменту}

    \begin{scnrelfromlist}{свойство-предпосылка}

        \scnitem{общая характеристика решателя задач кибернетической системы}
        \scnitem{качество логико-семантической организации памяти кибернетической
            системы}
        \scnitem{качество решения интерфейсных задач в кибернетической системе}

    \end{scnrelfromlist}
    
    \scnheader{общая характеристика решателя задач кибернетической системы}
    \begin{scnrelfromlist}{свойство-предпосылка}

        \scnitem{общий объем задач, решаемых кибернетической системой}
        \scnitem{многообразие видов задач, решаемых кибернетической системой}
        \scnitem{способность кибернетической системы к анализу решаемых задач}
        \scnitem{способность кибернетической системы к решению задач, методы решения
            которых в текущий момент известны}
        \scnitem{способность кибернетической системы к решению задач, методы решения
            которых ей в текущий момент не известны}
        \scnitem{множество навыков, используемых кибернетической системой}
        \scnitem{степень конвергенции и интеграции различного вида моделей решения
            задач, используемых кибернетической системой}
        \scnitem{качество организации взаимодействия процессов решения задач в
            кибернетической системе}
        \scnitem{быстродействие решателя задач кибернетической системы}
        \scnitem{способность кибернетической системы решать задачи, предполагающие
            использование информации, обладающей различного рода не-факторами}
        \scnitem{многообразие и качество решения задач информационного поиска}
        \scnitem{способность кибернетической системы генерировать ответы на вопросы
            различного вида в случае, если они целиком или частично отсутствуют в текущем
            состоянии информации, хранимой в памяти}
        \scnitem{способность кибернетической системы к рассуждениям различного вида}
        \scnitem{качество целеполагания}
        \scnitem{качество реализации планов собственных действий}
        \scnitem{способность кибернетической системы к локализации такой области
            информации,хранимой в ее памяти, которой достаточно для обеспечения решения
            заданной задачи}
        \scnitem{способность кибернетической системы к выявлению существенного в
            информации, хранимой в ее памяти}
        \scnitem{активность кибернетической системы}

    \end{scnrelfromlist}

    \scnheader{общий объем задач, решаемых кибернетической системой}
    \scnidtf{общий объем задач, которые кибернетическая система способна решать}
    \scnidtf{общий объем (множество), задач (действий), которые кибернетическая
        система способна (может, умеет) решать (выполнять) в заданный (в том числе, в
        текущий) момент}
    \scnrelfrom{свойство-предпосылка}{мощность языка представления задач, решаемых
        кибернетической системой}

    \scnheader{мощность языка представления задач, решаемых кибернетической системой}
    \scnidtf{мощность языка спецификации (описания) различного вида действий,
        выполняемых кибернетической системой}
    \scntext{примечание}{\textit{мощность языка представления задач} прежде всего
        определяется многообразием видов представляемых задач (многообразием видов
        описываемых действий).}\scnrelto{свойство-предпосылка}{многообразие видов
        задач, решаемых кибернетической системой}

    \begin{scnrelfromlist}{частное свойство}

        \scnitem{мощность языка представления задач, решаемых в памяти кибернетической системы}
        \begin{scnindent}
            \scnrelto{свойство-предпосылка}{многообразие видов задач, решаемых в
                памяти кибернетической системы}
        \end{scnindent}
        \scnitem{мощность языка представления задач, решаемых во внешней среде кибернетической системы}
        \begin{scnindent}
            \scnrelto{свойство-предпосылка}{многообразие видов
                задач, решаемых во внешней среде кибернетической системы}
        \end{scnindent}

        \scnitem{мощность языка представления задач, решаемых в рамках физической оболочки кибернетической системы}
        \begin{scnindent}
            \scnrelto{свойство-предпосылка}{многообразие
                видов задач, решаемых в рамках физической оболочки кибернетической системы}
        \end{scnindent}

    \end{scnrelfromlist}

    \scnheader{многообразие видов задач, решаемых кибернетической системой}
    \scnidtf{многообразие видов действий, которые кибернетическая система способна
        выполнять}
    \scntext{примечание}{Подчеркнем, что каждая задача есть спецификация соответствующего
        (описываемого) действия. Поэтому рассмотрение многообразия видов задач,
        решаемых кибернетической системой, полностью соответствует многообразию видов
        деятельности, осуществляемой этой системой. Важно заметить, что есть виды
        деятельности кибернетической системы, которые определяют качество и, в
        частности, уровень интеллекта кибернетической
        системы.}\scnrelfrom{свойство-предпосылка}{мощность языка представления задач в
        памяти кибернетической системы}

    \begin{scnrelfromset}{комплекс частных свойств}

        \scnitem{многообразие видов задач, решаемых в памяти кибернетической системы}
        \scnitem{многообразие видов задач, решаемых во внешней среде кибернетической
            системы}
        \scnitem{многообразие видов задач, решаемых в рамках физической оболочки
            кибернетической системы}

    \end{scnrelfromset}

    \scnheader{способность кибернетической системы к анализу решаемых задач}
    \scnidtf{способность кибернетической системы осмысливать (ведать) то, что она
        творит}
    \scnidtf{способность анализировать свои цели и, соответственно, решаемые задачи
        на предмет:
        \begin{scnitemize}

            \item сложности достижения;
            \item целесообразности достижения (нужности, важности, приоритетности);
            \item соответствия цели существующим нормам (правилам) соответствующей
            деятельности
        \end{scnitemize}
    }

    \scnheader{способность кибернетической системы к решению задач, методы решения
        которых ей в текущий момент известны}
    \scntext{примечание}{Указанными методами могут быть не только алгоритмы, но также и
        функциональные программы, продукционные системы, логические исчисления,
        генетические алгоритмы, искусственные нейронные сети различного вида.}
    \begin{scnrelfromlist}{свойство-предпосылка}

        \scnitem{способность кибернетической системы к поиску хранимых в своей памяти
            методов решения инициированных задач}
        \scnitem{способность кибернетической системы к интерпретации хранимых в своей
            памяти методов решения задач}

    \end{scnrelfromlist}

    \scnheader{способность кибернетической системы к решению задач, методы решения
        которых ей в текущий момент не известны}
    \scnidtf{способность кибернетической системы к решению задач, для которых не
        найдены соответствующие (релевантные) им методы их решения}
    \scnidtf{способность кибернетической системы строить цепочку цель-план
        достижения цели-система действий}
    \scntext{примечание}{Задачи, для которых не находятся соответствующие им методы,
        решаются с помощью метаметодов (стратегий) решения задач, направленных:
        \begin{scnitemize}

            \item на генерацию нужных исходных данных (нужного контекста), необходимых для
            решения каждой задачи;
            \item на генерацию плана решения задачи, описывающего сведение исходной задачи
            к подзадачам (до тех подзадач, методы решения которых системы известны);
            \item на сужение области решения задачи (на сужения контекста задачи,
            достаточного для ее решения).
        \end{scnitemize}
    }
    
    \scnheader{множество навыков, используемых кибернетической системой}
    \scnidtf{объем и многообразие навыков, приобретенных кибернетической системой к
        текущему моменту (с помощью учителей-разработчиков или полностью
        самостоятельно)}
    \scnidtf{возможности, навыки, приобретенные кибернетической системой}
    \scnidtf{опыт, приобретенный кибернетической системой}
    \scntext{примечание}{Новые навыки могут приобретаться кибернетической системой либо
        полностью самостоятельно, либо с помощью учителей, которые в простейшем случае
        просто сообщают обучаемой системе полностью сформулированные навыки. Для
        компьютерных систем учителями является их разработчики.}\bigskip
    \begin{scnrelfromlist}{частное свойство}

        \scnitem{множество методов решения задач, используемых кибернетической
            системой}
        \scnitem{множество моделей решения задач, используемых кибернетической
            системой}
        \scnitem{мощность языка представления в памяти кибернетической системы методов
            и моделей решения задач}

    \end{scnrelfromlist}
    
    \scnheader{множество методов решения задач, используемых кибернетической
        системой}
    \scnidtf{множество методов решения задач, используемых кибернетической системой
        и хранимых в ее памяти}
    \scnrelto{частное свойство}{многообразие видов знаний, хранимых в памяти
        кибернетической системы}
    
        \scnheader{метод решения задач}
    \scntext{пояснение}{\textbf{\textit{метод решения задач}} --- это \textit{вид
            знаний}, хранимых в \textit{памяти кибернетической системы} и содержащих
        информацию, которой достаточно либо для сведения каждой \textit{задачи} из
        соответствующего \textit{класса задач} к \textit{полной системе подзадач*},
        решение которых гарантирует решение исходной \textit{задачи}, \uline{либо} для
        окончательного решения этой \textit{задачи} из указанного \textit{класса задач}}
            
    \scnheader{множество моделей решений задач, используемых кибернетической
        системой}
    \scnidtf{способность кибернетической системы к использованию различных видов
        методов решения задач, соответствующих различным моделям решения задач}
    \scnidtf{многообразие методов решения задач, используемых кибернетической
        системой}
    \scnrelfrom{свойство-предпосылка}{мощность языка представления в памяти
        кибернетической системы методов и моделей решения задач}
    
    \scnheader{множество моделей решения задач, используемых кибернетической
        системой}
    \begin{scnrelfromset}{примечание}
        \scnitem{следует отличать*}
        \begin{scnindent}
            \begin{scnhaselementset}
                \scnitem{вид задач}
                \scnitem{модель решения задач}
                \begin{scnindent}
                    \scntext{пояснение}{каждая \textit{модель решения задач} задается
                        \begin{scnitemize}
                            \item \textit{языком}, обеспечивающим представление в \textit{памяти
                                кибернетической системы} некоторого класса \textit{методов решения задач}
                            \item интерпретатором указанных \textit{методов}, определяющим
                            \textit{операционную семантику} указанного \textit{языка}
                        \end{scnitemize}
                    }
                \end{scnindent}
                \scnitem{метод решения задач}
                \scnitem{класс задач}
                \begin{scnindent}
                    \scnidtf{\textit{множество} всех тех и только тех
                        \textit{задач}, которые решаются с помощью соответствующего \textit{метода}}
                \end{scnindent}
            \end{scnhaselementset}
        \end{scnindent}
    \end{scnrelfromset}

    \scnheader{Степень конвергенции и интеграции различного вида моделей решения
        задач, используемых кибернетической системой}
    \scntext{примечание}{Необходим переход от эклектики никак не связанных друг с другом
        \textit{моделей решения задач} к их \textit{конвергенции}, это предполагает:
        \begin{scnitemize}
            \item разработку общего (базового) для всех \textit{моделей решения задач}
            языка описания \textit{операционной семантики} языков описания методов,
            соответствующих различным \textit{моделям решения задач};
            \item включение всех языков описания \textit{методов решения задач} в общую
            систему языков, связанных между собой отношением \scnqqi{язык-подъязык*}.
        \end{scnitemize}
    }
    
    \scnheader{качество организации взаимодействия процессов решения задач в
        кибернетической системе}
    \begin{scnrelfromlist}{частное свойство}

        \scnitem{качество управления информационным процессом в памяти кибернетической системы}
        \begin{scnindent}
            \scnrelfrom{свойство-предпосылка}{обеспечение процессором
                кибернетической системы качественного управления информационными процессами в
                памяти}
        \end{scnindent}
        
        \scnitem{качество организации взаимодействия процессов решения задач во внешней
            среде или в физической оболочке кибернетической системы}
        \begin{scnindent}
            \begin{scnrelfromlist}{свойство-предпосылка}

                \scnitem{последовательность/параллельность процессов решения задач в
                    кибернетической системе}
                \scnitem{ синхронность/асинхронность процессов решения задач в кибернетической
                    системе}
                \scnitem{ централизованной/децентрализованность управления процессами решения
                    задач в кибернетической системе}

            \end{scnrelfromlist}
        \end{scnindent}

    \end{scnrelfromlist}
    \scntext{примечание}{Качество решения каждой \textit{задачи} определяется:
        \begin{scnitemize}

            \item временем её решения (чем быстрее \textit{задача} решается, тем выше
            качество её решения);
            \item полнотой и корректностью результата решения \textit{задачи};
            \item затраченными для решения \textit{задачи} ресурсами памяти (объемом
            фрагмента хранимой информации, используемой для решения задачи);
            \item затраченным для решения \textit{задачи} ресурсами решателя задач
            (количеством используемых внутренних агентов).
        \end{scnitemize}
        Таким образом, повышение качества процесса решения каждой конкретной
        \textit{задачи}, а также каждого \textit{класса задач} (путем совершенствования
        соответствующего метода, в частности, алгоритма) является важным фактором
        повышения качества \textit{решателя задач} в
        целом.}
        
    \scnheader{агентно-ориентированная модель обработки информации в памяти}
    \scnidtf{агентно-ориентированная модель управления действиями кибернетической
        системы, выполняемыми ею в своей памяти}
    \scntext{пояснение}{Перспективным вариантом построения \textit{решателя задач
            кибернетической системы} является реализация \textit{агентно-ориентированной
            модели обработки информации}, т.е. построение \textit{решателя задач} в виде
        \textit{многоагентной системы}, агенты которой осуществляют обработку
        \textit{информации, хранимой в памяти} кибернетической системы, и управляются
        этой информацией (точнее, её текущим состоянием). Особое место среди этих
        \textit{агентов} занимают сенсорные (рецепторные) и эффекторные
        \textit{агенты}, которые, соответственно, воспринимают информацию о текущем
        состоянии \textit{внешней среды} и воздействуют на \textit{внешнюю среду}, в
        частности, путем изменения состояния \textit{физической оболочки
            кибернетической системы}.
            \\Подчеркнем, что указанная агентно-ориентированная
        модель организации взаимодействия процессов решения задач в
        \textit{кибернетической системе} по сути есть не что иное, как модель
        ситуационного управления процессами решения задач, решаемых
        \textit{кибернетической системой} как в своей \uline{внешней среде}, так и в
        своей памяти.}
        
    \scnheader{модель инициирования действий кибернетической системы}
    \scnidtf{модель управления поведением кибернетической системы}

    \begin{scnsubdividing}

        \scnitem{стимульно-реактивная модель инициирования действий}
        \begin{scnindent}
            \scntext{пояснение}{от комбинации \textit{исходных сигналов},
                формируемых, например, априори известным набором сенсоров (рецепторов) к
                комбинации выходных \textit{сигналов}, управляющих, например, априори известным набором эффекторов}
        \end{scnindent}
        \scnitem{ситуационная модель инициирования действий без учета предыстории ситуаций и событий}
        \begin{scnindent}    
            \scntext{пояснение}{действие инициируется возникновением
                в памяти \textit{ситуации} априори известной конфигурации или априори
                известного события}
        \end{scnindent}
        \scnitem{ситуационная модель инициирования действий с учетом предыстории ситуаций и событий}
        \begin{scnindent}
            \scntext{пояснение}{действие инициируется не только
                текущей \textit{ситуацией} но и предшествующими \textit{ситуациями}, т.е.
                событиями перехода от одних \textit{ситуаций} к другим}
        \end{scnindent}

    \end{scnsubdividing}
    \scntext{примечание}{Речь идет о действиях, выполняемых \textit{кибернетической
            системой} как во внешней среде, так и в своей внутренней \textit{среде} (в
        своей памяти).}
        
    \scnheader{последовательность/параллельность процессов решения
        задач в кибернетической системе}
    \scnidtf{способность одновременно решать несколько разных задач, некоторые из
        которых могут быть подзадачами одной и той же задачи}
    \scnidtf{способность одновременно решать несколько разных задач, некоторые из
        которых могут быть подзадачами одной и той же задачи}

    \begin{scnrelfromlist}{свойство-предпосылка}

        \scnitem{максимально возможное количество действий, одновременно выполняемых
            кибернетической системой}
        \scnitem{способность кибернетической системы к одновременному выполнению
            взаимосвязанных действий}
        \begin{scnindent}    
            \scnidtf{способность кибернетической системы к
                одновременному выполнению действий, выполнение каждого из которых может
                помешать выполнению другого}
            \scnidtf{способность кибернетической системы к эквилибристике}
        \end{scnindent}

    \end{scnrelfromlist}

    \begin{scnrelfromset}{комплекс частных свойств}

        \scnitem{физическая последовательность/параллельность процессов решения задач в
            кибернетической системе}
        \scnitem{ логическая последовательность/параллельность процессов решения задач
            в кибернетической системе}
        \begin{scnindent}
            \scntext{пояснение}{Логическая параллельность выполняемых процессов
                (действий) предполагает возможность существования \uline{выполняемых} процессов
                в двух режимах:
                \begin{scnitemize}

                    \item в активном режиме --- в режиме непосредственного выполнения
                    \item в режиме прерывания --- в режиме ожидания	условий (событий и/или
                    ситуаций) при возникновении которых прерванный процесс переходит в режим
                    активного процесса.
                \end{scnitemize}
            }
        \end{scnindent}

    \end{scnrelfromset}

    \begin{scnrelfromset}{комплекс частных свойств}

        \scnitem{последовательность/параллельность информационных процессов в памяти
            кибернетической системы}
        \scnitem{ последовательность/параллельность процессов решения задач во внешней
            среде или в физической оболочке кибернетической системы}

    \end{scnrelfromset}
    \scntext{примечание}{Подчеркнем, что есть целый ряд задач, решаемых кибернетической
        системой, процессы решения которых носят перманентный (постоянный) характер. К
        таким задачам относятся:
        \begin{scnitemize}

            \item поддержка высокого качества базы знаний (устранение противоречий,
            информационного мусора);
            \item поддержка семантической совместимости с другими компьютерными системами;
            \item мониторинг и анализ состояния внешней среды;
            \item обеспечение собственной безопасности;
            \item самообучение.
        \end{scnitemize}
    }
    
    \scnheader{быстродействие решателя задач кибернетической системы}
    \scnidtf{скорость решения задач в кибернетической системе}
    \scnidtf{быстродействие решателя задач кибернетической системы}
    \scnidtf{скорость реакции кибернетической системы на различные задачные
        ситуации}
    \scnrelfrom{свойство-предпосылка}{быстродействие процессора кибернетической
        системы}
    
    \scnheader{способность кибернетической системы решать задачи, предполагающие
        использование информации, обладающей различного рода не-факторами}
    \scnidtf{способность кибернетических систем решать задачи, которые:
        \begin{scnitemize}

            \item либо нечетко сформулированы (делай то, не знаю что);
            \item либо решаются в условиях неполноты, неточности, противоречивости исходных
            данных;
            \item либо являются задачами, принадлежащими классам задач, для которых
            практически невозможно построить соответствующие алгоритмы.
        \end{scnitemize}
    }
    \scnidtf{способность кибернетической системы решать труднорешаемые,
        трудноформализуемые задачи}
    \scnidtf{способность решать интеллектуальные (трудноформализуемые) задачи, для
        которых характерна:
        \begin{scnitemize}

            \item неточность и недостоверность исходных данных;
            \item отсутствие критерия качества результата;
            \item невозможность или высокая трудоемкость разработки алгоритма;
            \item необходимость учета контекста задачи.
        \end{scnitemize}
    }

    \scnheader{задача, предполагающая использование информации, обладающей
        различного рода не-факторами}
    \scnidtf{трудноформализуемая задача}
    \scnsuperset{задача проектирования}
    \scnsuperset{задача распознавания}
    \scnsuperset{задача прогнозирования}
    \scnsuperset{задача целеполагания}
    \scnsuperset{задача планирования}
    
    \scnheader{многообразие и качество решения задач информационного поиска}
    \scnrelfrom{свойство-предпосылка}{семантический уровень доступа к информации,
        хранимой в памяти кибернетической системы}
    \scnrelto{частное свойство}{многообразие видов задач, решаемых кибернетической
        системой}
    \scnidtf{способность кибернетической системы качественно решать широкое
        многообразие задач информационного поиска в рамках текущего состояния хранимой
        информации}
    \scnidtf{способность кибернетической системы находить в текущем состоянии
        хранимой информации релевантные ответы на запросы (вопросы) самого различного
        вида}
    
    \scnheader{вопрос}
    \scnidtf{запрос}
    \scnsuperset{запрос изоморфных или гомоморфных фрагментов хранимой информации
        по заданному образцу с указанием знаков известных сущностей}
    \begin{scnindent}
        \scnrelfrom{класс частных вопросов}{запрос всех связок различных отношений,
            обязывающих заданную сущность с другими}
            \begin{scnindent}
                \scnrelfrom{класс частных вопросов}{запрос всех связок заданных отношений,
                    связывающих заданную сущность с другими}
            \end{scnindent}
    \end{scnindent}
    \scnsuperset{вопрос типа \scnqqi{как связаны между собой заданные две сущности}}
    \begin{scnindent}
    \scntext{пояснение}{Две сущности будем считать связанными в том и только в
        том случае, если существует маршрут, соединяющий указанные две сущности, в
        состав которого входят связки, принадлежащие в общем случаем разным
        отношениям}\scntext{примечание}{Здесь принципиально важным является учет
        \textit{семантической силы связей} между сущностями, которая определяется
        \textit{семантической силой отношений}, которым принадлежат связки, входящие в
        состав связей (маршрутов) между сущностями.}
    \scnrelto{класс частных
        вопросов}{вопрос типа \scnqqi{как связаны между собой заданные сущности}}
        \begin{scnindent}
            \scntext{примечание}{Здесь имеется в виду произвольное количество связываемых
                сущностей, а это предполагает, что ответом на данный запрос является
                \uline{связный граф}, вершинами которого являются знаки заданных сущностей.}
        \end{scnindent}
    \end{scnindent}
    \scnsuperset{вопрос типа \scnqqi{что это такое}}
    \begin{scnindent}
        \scnidtf{запрос спецификации (описания) заданной сущности}
        \scnrelfrom{класс частных вопросов}{запрос определения}
        \begin{scnindent}
            \scnidtf{запрос определения заданного понятия}
        \end{scnindent}
        \scnrelfrom{класс частных вопросов}{запрос документации заданного объекта}
    \end{scnindent}
    \scnsuperset{почему-вопрос}
    \begin{scnindent}
        \scnsuperset{запрос причины возникновения заданной ситуации или события}
        \scnsuperset{запрос логического обоснования заданного высказывания}
            \begin{scnindent}
                \scnidtf{запрос объяснения корректности заданного высказывания, которое, в
                    частности, может быть порождено (сгенерировано) в процессе решения некоторой
                    задачи с помощью некоторого метода (алгоритма, искусственной нейронной сети
                    логического исчисления и т.п.)}
                \scnsuperset{запрос доказательства заданной теоремы}
            \end{scnindent}
    \end{scnindent}
    \scnsuperset{запрос возможных последствий заданной ситуации или события}
    \scnsuperset{запрос того, что логически следует из заданного высказывания}
    \scnsuperset{запрос метода решения данной задачи}
    \scnsuperset{запрос плана решения данной задачи}
    \begin{scnindent}
        \scnidtf{запрос декомпозиции данной задачи на систему и/или подзадач}
    \end{scnindent}
    \scnsuperset{зачем-вопрос}
    \begin{scnindent}
        \scnidtf{каково назначение заданной сущности}
        \scnidtf{для решения какой задачи (для чего, достижения какой цели) нужна
            данная сущность}
    \end{scnindent}
    \scnsuperset{запрос аналогов заданной сущности}
    \scnsuperset{запрос антиподов заданной сущности}
    \scnsuperset{запрос сходств и отличий двух связанных сущностей}
    \scnsuperset{запрос сравнительного анализа заданной сущности}
    \begin{scnindent}
        \scnsuperset{запрос достоинств заданной сущности}
        \scnsuperset{запрос недостатков заданной сущности}
    \end{scnindent}
    \scnsuperset{где-вопрос}
    \begin{scnindent}
        \scnidtf{запрос информации о местоположении заданной пространственной сущности
            примечание}
        \scntext{примечание}{Здесь запрашивается любая информация о пространственных связях
            заданной сущности}
    \end{scnindent}    
    \scnsuperset{когда-вопрос}
    \begin{scnindent}
        \scnidtf{запрос информации о темпоральных свойствах и связях заданной временной
            сущности (о моменте начала, о моменте завершения, о длительности)}
    \end{scnindent}

   \scnheader{cпособность кибернетической системы генерировать ответы на вопросы
        различного вида в случае, если они целиком или частично отсутствуют в текущем
        состоянии информации, хранимой в памяти}
    \scnidtf{способность кибернетической системы генерировать (порождать, строить,
        синтезировать, выводить) ответы на самые различные вопросы и, в частности, на
        вопросы типа \scnqqi{что это такое}, на почему-вопросы, это означает способность
        кибернетической системы \uline{объяснять} (обосновывать корректность) своих
        действий}

    \begin{scnrelfromlist}{свойство-предпосылка}

        \scnitem{семантическая гибкость информации, хранимой в памяти кибернетической
            системы}
        \scnitem{ способность кибернетической системы к рассуждениям различного вида}

    \end{scnrelfromlist}

    \scnheader{способность кибернетической системы к рассуждениям различного вида}
    \scnidtf{способность кибернетической системы к целенаправленному порождению
        (генерации) новых истинных или правдоподобных знаний (следствий) на основе
        имеющихся знаний (посылок)}

    \begin{scnrelfromlist}{частное свойство}

        \scnitem{способность кибернетической системы к дедуктивному выводу}
        \scnitem{способность кибернетической системы к индуктивному выводу}
        \scnitem{способность кибернетической системы к абдуктивному выводу}

    \end{scnrelfromlist}

    \scnheader{качество целеполагания}
    \scnidtf{качество реализации первого этапа решения сложных задач --- этапа
        генерации (построения) планов решения сложных задач}
    \scnidtf{качество генерации планов выполнения сложных действий:
        \begin{scnitemize}

            \item как внутренних действий (в памяти кибернетической системы), так и внешних
            действий (во внешней среде)
            \item как собственных действий, так и действий других субъектов
        \end{scnitemize}
    }
    \scnidtf{качество генерации планов действий кибернетической системы и, в
        частности, трудоемкость процесса генерации этих планов}
    \scnidtf{качество организации целенаправленной деятельности кибернетической
        системы}
    \scnidtf{качество построения цепочек цель-план-действие }
    \scnidtf{качество генерации, анализа и инициирования собственных целей
        (собственных задач)}
    \scnidtf{способность кибернетической системы к целеполаганию}

    \begin{scnrelfromlist}{свойство-предпосылка}
        \scnitem{самостоятельность целеполагания}
        \begin{scnindent}
            \scnidtf{самостоятельность генерации
                и инициирования целей (задач), направленных на создание условий достижения
                соответствующих стратегических целей (сверхзадач)}
        \end{scnindent}
        \scnitem{целенаправленность целеполагания}
        \begin{scnindent}
            \scnidtf{степень соответствия
                (степень полезности) генерируемых целей (задач) для достижения соответствующих
                стратегических целей (сверхзадач)}
        \end{scnindent}
        \scnitem{сбалансированность целеполагания}
        \begin{scnindent}
            \scnidtf{качество расстановки
                приоритетов у сгенерированных и инициированных целей (задач) для обеспечения
                баланса между тактическими и стратегическими целями}
        \end{scnindent}
    \end{scnrelfromlist}

    \scnheader{самостоятельность целеполагания}
    \scnidtf{способность кибернетической системы генерировать, инициировать и
        решать задачи, которые не являются подзадачами, инициированными внешними
        (другими) субъектами, а также способность на основе анализа своих возможностей
        отказаться от выполнения задачи, инициированной извне, переадресовав её другой
        кибернетической системе, либо на основе анализа самой этой задачи обосновать её
        нецелесообразность или некорректность}
    \scnidtf{способность к самостоятельному целеполаганию (генерации идей) и к
        инициированию процессов их достижения (т.е. к принятию решений), способность
        свободно (в определенных рамках) выбирать (ставить перед собой цели)}
    \scnidtf{уровень самостоятельности}
    \scnidtf{способность решать задачи в комплексе, включая создание всех
        необходимых условий для их решения с учетом конкретных обстоятельств}
    \scnidtf{умение решать задачи в условиях сильных помех (в осложненных
        обстоятельствах)}
    \scntext{примечание}{Повышение уровня самостоятельности существенно расширяет
        возможности кибернетической системы, т.е. объем тех задач, которые она может
        решать не только в идеальных  условиях, но и в реальных (осложненных)
        обстоятельствах.}\scnidtf{степень свободы выбора целей, подлежащих достижению,
        а также свободы генерации целей, не являющихся подцелями извне поставленных
        целей}

    \scnheader{целенаправленность целеполагания}
    \scnidtf{целеустремленность}
    \scnidtf{целенаправленность}
    \scnidtf{степень целостности деятельности}
    \scnidtf{степень соответствия между тактическими и стратегическими уровнями
        деятельности}
    \scnidtf{общее соотношение между временем, затраченным на лишние  (ненужные,
        нецелесообразные, нецеленаправленные) действия и полезные действия}
    \scnidtf{целесообразность деятельности}
    \scnidtf{способность адекватно расставлять приоритеты своим целям и не
        распыляться  на достижение неприоритетных (несущественных) целей}
    
        \scnheader{качество реализации планов собственных действий}
    \scnidtf{качество реализации целенаправленной деятельности на основе
        построенных планов}
    \scnidtf{качество реализации построенных в памяти кибернетической системы
        планов выполнения сложных собственных действий, которые могут предполагать
        участие других субъектов}
    
    \scnheader{способность кибернетической системы к локализации такой области
        информации, хранимой в ее памяти, которой достаточно для обеспечения решения
        заданной задачи}
    \scnidtf{способность кибернетической системы к сужению области решения каждой
        решаемой ею задачи, что существенно минимизирует затраты кибернетической
        системы на учет и анализ факторов, априори незначимых (несущественных) для
        решения каждой решаемой задачи}
    \scntext{примечание}{Для реализации данной способности важное значение имеет
        качественная стратификация базы знаний кибернетической системы на предметные
        области и соответствующие им онтологии.}
        
    \newpage\scnheader{способность кибернетической системы к выявлению существенного в информации, хранимой в ее памяти}
    \scnidtf{способность к выявлению (обнаружению, выделению) таких фрагментов
        информации, хранимой в памяти кибернетической системы, которые существенны
        (важны) для достижения соответствующих целей}
    \scntext{примечание}{Понятие существенного (важного) фрагмента информации, хранимой в
        памяти кибернетической системы, относительно и определяется соответствующей
        задачей. Тем не менее, есть важные перманентно (постоянно) решаемые задачи, в
        частности задачи анализа качества информации, хранимой в памяти кибернетической
        системы. Существенные фрагменты хранимой информации, выделяемые в процессе
        решения этих задач, являются относительными не столько по отношению к решаемой
        задаче, сколько по отношению к текущему состоянию хранимой информации.
        Примерами таких фрагментов являются:
        \begin{scnitemize}
            \item обнаруженные противоречия (ошибки) с явным указанием того, что чему
            противоречит;
            \item обнаруженные информационные дыры, точнее точная спецификация этих дыр;
            \item обнаруженные мусорные фрагменты, которые либо носят вспомогательный
            характер, либо могут быть легко восстановлены (воспроизведены).
        \end{scnitemize}
    }
    
    \scnheader{следует отличать*}
    \begin{scnhaselementset}
        \scnitem{способность кибернетической системы к выявлению существенного в
            информации, хранимой в ее памяти}
        \begin{scnindent}    
            \scntext{примечание}{Здесь кибернетическая система
                выделяет информацию, которая необходима, но не обязательно достаточна для
                решения соответствующей задачи.}
        \end{scnindent}
        \scnitem{способность кибернетической системы к локализации такой области
            информации, хранимой в ее памяти, которой достаточно для обеспечения решения
            заданной задачи}
        \begin{scnindent}    
            \scntext{примечание}{Здесь кибернетическая система отбрасывает
                (исключает) информацию, которая априори несущественна для решения
                соответствующей (заданной) задачи.}
        \end{scnindent}
    \end{scnhaselementset}

    \scnheader{активность кибернетической системы}
    \scnidtf{уровень активности кибернетической системы}
    \scnidtf{уровень мотивации к деятельности в различных направлениях}
    \scnidtf{уровень желания  действовать}
    \scnidtf{активность/пассивность кибернетической системы}
    \scnidtf{уровень инициативности, пассионарности, мотивированности}
    \scntext{примечание}{Уровень активности кибернетической системы может быть разным для
        разных решаемых задач, для разных классов выполняемых действий, для разных
        видов деятельности.}\scntext{примечание}{Следует отличать уровень активности
        (мотивации, желания) и направленность этой активности.}\scntext{примечание}{Чем выше
        активность кибернетической системы, тем (при прочих равных условиях) она больше
        успевает сделать, следовательно, тем выше ее качество
        (эффективность).}\scnrelboth{обратное свойство}{пассивность}
    \begin{scnindent}
        \scnidtf{уровень бездеятельности, медлительности, вялости, ленивости}
    \end{scnindent}
    \begin{scnrelfromlist}{частное свойство}
        \scnitem{познавательная активность}
        \scnitem{ социальная активность}
    \end{scnrelfromlist}
    
    \scnheader{качество логико-семантической организации памяти
        кибернетической системы}
    \scnidtf{качество базовых семантически целостных действий в памяти
        кибернетической системы}
    \scnidtf{качество семантически элементарных (законченных, целостных)
        информационных процессов, выполняемых кибернетической системой в своей памяти}
    \scnidtf{интегральная оценка того, насколько способствует (насколько близка)
        организация памяти кибернетической системы реализации осмысленных
        преобразований, хранимых в памяти знаний}
    \scnidtf{степень приспособленности решателя задач кибернетической системы к
        обработке сложноструктурированных баз знаний}
    \scnidtf{степень приспособленности решателя задач кибернетической системы к
        обработке хранимой в её памяти информации, имеющий высокий уровень качества как
        по форме представления информации, так и по её содержанию --- по многообразию
        представляемых знаний и по уровню их конвергенции и интеграции}

    \begin{scnrelfromlist}{свойство-предпосылка}
        \scnitem{семантический уровень доступа к информации, хранимой в памяти
            кибернетической системы}
        \scnitem{семантическая гибкость информации, хранимой в памяти кибернетической
            системы}
        \scnitem{степень конвергенции и интеграции представления навыков, хранимых в
            памяти кибернетической системы, с представлением обрабатываемой информации}
    \end{scnrelfromlist}

    \scnheader{семантический уровень доступа к информации, хранимой в памяти
        кибернетической системы}
    \scnidtf{степень ассоциативности доступа к информации, хранимой в памяти
        кибернетической системы}
    \scnidtf{способность кибернетической системы локализовывать (находить)
        требуемый (запрашиваемый) фрагмент информации, хранимой в её памяти, не на
        основании известного адреса запрашиваемой информации (её местоположения в
        памяти), а на основании:
        \begin{scnitemize}
            \item известного типа запрашиваемой информации;
            \item известных сущностей, знаки которых входят в состав запрашиваемой
            информации;
            \item полностью или частично известной конфигурации запрашиваемой информации
            (т.е. конфигурации связей между известными и искомыми сущностями)
        \end{scnitemize}
    }
    \scntext{пояснение}{\textit{уровень доступа к информации, хранимой в памяти
            кибернетической системы} определяется тем, что нам достаточно знать об искомой
        в памяти кибернетической системы информации (в частности, об искомом знаке
        некоторой интересующей нас сущности). Мы можем знать место в памяти (ячейку
        памяти, область памяти), где находится интересующая нас информация. Такой
        доступ называется \uline{адресным}. Мы можем знать имя интересующей нас
        сущности, но не знать, где находится информация, описывающая эту сущность. Мы
        можем не знать имени интересующей нас сущности, но знать, как эта сущность
        связана с другими известными нам сущностями.}\scntext{пояснение}{Пусть нам
        необходимо локализовать (выделить) хранимую в памяти информацию, описывающую
        известные нам сущности, связанные известными нам отношениями, но
        местонахождение этой информации в памяти нам не известно. Если организация
        памяти нам представляет такую возможность, то такую память будем называть
        ассоциативной, т.е. памятью, обеспечивающей семантический доступ к хранимой в
        ней информации.}\scntext{примечание}{Для того, чтобы построить информационную модель
        среды, в которой действует (функционирует) кибернетическая система, необходимо,
        с одной стороны, разложить  эту информационную модель по полочкам , превратить
        её в некую систему из компонентов этой информационной модели, а, с другой
        стороны, обеспечить быстрый поиск нужного фрагмента указанной информационной
        модели, не зная, на каких полочках   находятся компоненты этого искомого
        фрагмента, который при этом может иметь произвольную конфигурацию и
        произвольный размер. Это и есть высший уровень \uline{ассоциативности} доступа
        к информации, хранимой в памяти кибернетической
        системы.}\scntext{пояснение}{Данное свойство, данная характеристика
        организации информации, хранимой в памяти кибернетической системы, является
        важнейшей характеристикой \uline{внутреннего} языка представления информации в
        памяти кибернетической системы. Указанная характеристика внутреннего языка
        определяется \uline{простотой процедур поиска} востребованных (запрашиваемых)
        фрагментов хранимой информации --- например, процедуры поиска знаков всех
        сущностей, каждая из которых связана с заданными (известными) сущностями
        связями заданных (известных) типов, процедуры поиска (выделения) знаков всех
        сущностей, которые связаны с заданной (известной) сущностью связью неважно
        какого типа, процедуры поиска информационного фрагмента заданному образцу
        (шаблону) произвольного размера и конфигурации, в котором выделены знаки
        известных сущностей и условные обозначения искомых
        сущностей.}\scnrelfrom{свойство-предпосылка}{степень близости языка внутреннего
        представления информации в памяти кибернетической системы к смысловому
        представлению информации}

    \newpage\scnheader{семантическая гибкость информации, хранимой в памяти
        кибернетической системы}
    \scnrelfrom{свойство-предпосылка}{степень близости языка внутреннего
        представления информации в памяти кибернетической системы к смысловому
        представлению информации}
    \scnidtf{простота реализации базовых (элементарных), но семантически целостных
        (семантически значимых, осмысленных) действий (операций) преобразования
        (обработки) информации, хранимой в памяти кибернетической системы}
    
    \scnheader{базовое семантически целостное действие над информацией, хранимой в
        памяти кибернетической системы}
    \scnidtf{элементарная семантически значимая (осмысленная) операция над
        информацией, хранимой в памяти кибернетической системы}
    \scntext{примечание}{Здесь принципиальной является семантическая целостность
        (осмысленность) действия над хранимой информацией. Так, например, операция
        адресного доступа к требуемому фрагменту хранимой информации не является
        семантически целостной, так как смысл искомого (запрашиваемого) фрагмента
        хранимой информации не уточняется.}\scntext{примечание}{Разные кибернетические
        системы могут использовать разные наборы классов базовых семантически целостных
        действий над информацией, хранимой в их памяти.}\scntext{примечание}{Примерами
        \textit{базовых семантически целостных действий над информацией, хранимой в
            памяти кибернетической системы}, в частности, являются:
        \begin{scnitemize}

            \item операции поиска, генерации, удаления или замены связок между знаками
            известных сущностей;
            \item операции поиска, генерации, удаления или замены имен, приписываемых
            знакам известных сущностей.
        \end{scnitemize}
        Существенно подчеркнуть, что простота реализации такого рода операций (т.е.
        гибкость хранимой в памяти информации) во многом обеспечивается стремлением к
        локальности выполнения этих операций. Такая локальность означает то, что при
        выполнении \uline{каждой} из указанных операций меняется только обрабатываемый
        фрагмент хранимой информации и не требуется никакого переразмещения в памяти
        остальной части хранимой информации.}
        
    \scnheader{степень конвергенции и интеграции представления навыков, хранимых в памяти кибернетической системы, с
        представлением обрабатываемой информации}
    \scnrelto{частное свойство}{степень конвергенции и интеграции различного вида
        знаний, хранимых в памяти кибернетической системы}
    \begin{scnindent}
        \scnrelfrom{свойство-предпосылка}{степень близости языка внутреннего
            представления информации в памяти кибернетической системы к смысловому
            представлению информации}
    \end{scnindent}
    \scntext{примечание}{Навыки кибернетической системы являются частным видом знаний,
        хранимых в её памяти, поэтому степень конвергенции навыков и обрабатываемых
        знаний определяется глубиной  и объемом  \uline{общих} (одинаковых) принципов,
        лежащих в основе как представления навыков, так представления обрабатываемых
        знаний.}
        
    \scnheader{качество решения интерфейсных задач в
        кибернетической системе}
    \begin{scnrelfromlist}{частное свойство}
        \scnitem{способность кибернетической системы к пониманию сенсорной информации}
        \scnitem{способность кибернетической системы к пониманию принимаемых сообщений}
        \scnitem{способность кибернетической системы к самостоятельной деятельности во
            внешней среде}
            \begin{scnindent}
                \scnidtf{способность кибернетической системы к воздействию на
                внешнюю среду и к управлению своим поведением во внешней среде}
            \end{scnindent}
    \end{scnrelfromlist}

    \scnheader{интерфейсная задача}
    \scnsuperset{задача анализа введенной информации}
    \scnsuperset{задача анализа сенсорной информации}
    \begin{scnindent}
        \scnidtf{задача анализа информации, порождаемой (генерируемой) непосредственно
            сенсорами кибернетической системы}
        \scnsuperset{задача синтаксического анализа сенсорной информации}
        \scnsuperset{задача семантического анализа сенсорной информации}
        \begin{scnindent}
            \scnidtf{задача анализа сенсорной информации, направленного на
                \uline{понимание} этой информации --- на выявление (распознавание) в этой
                информации отображения (сенсорного описания) объектов, важных для
                кибернетической системы (т.е. объектов, описанных в базе знаний этой системы и,
                соответственно, представленных в этой базе знаний своими знаками либо знаками
                классов, которым эти объекты принадлежат), а также важных для кибернетических
                связей между указанными объектами}
            \scnidtf{задача генерации фрагмента базы знаний кибернетической системы,
                являющегося логическим следствием заданной сенсорной информации и
                представляющегося собой важную для кибернетической системы информацию}
            \scnidtf{задача извлечения из сенсорной информации (первичной информации)
                важной для кибернетической системы вторичной информации}
            \scnsuperset{задача анализа принимаемого вербального сообщения}
            \scnidtf{задача анализа введенных знаковых конструкций}
            \scnidtf{задача анализа сообщений, введенных в кибернетическую систему}
            \scnidtf{задача анализа внешних знаковых конструкций}
            \scnsuperset{задача синтаксического анализа принимаемого вербального сообщения}
            \scnsuperset{задача трансляции принимаемого вербального сообщения на внутренний
                язык кибернетической системы}
            \scnsuperset{задача погружения нового фрагмента в состав согласованной части
                базы знаний}
            \begin{scnindent}
                \scnidtf{задача интеграции (встраивания) нового фрагмента базы знаний в состав
                    базы знаний}
                \scnidtf{задача понимания нового фрагмента базы знаний в контексте её текущего
                    состояния, что, прежде всего, требует обеспечения семантической совместимости
                    (согласования понятий) между базой знаний и интегрируемым новым фрагментом}
            \end{scnindent}
        \end{scnindent}
    \end{scnindent}
    \scnsuperset{задача управления эффекторами кибернетической системы при
        выполнении сложных воздействий на внешнюю среду и/или физическую оболочку этой
        кибернетической системы}
    \begin{scnindent}
        \scnidtf{задача целенаправленной сенсорно-эффекторной (в частности, сенсомоторной) координации}
    \end{scnindent}

    \scnheader{сенсорная информация}
    \scnidtf{информация, генерируемая непосредственно некоторой группой
        (конфигурацией) сенсоров (рецепторов) кибернетической системы}
    \scnidtf{рецепторная информация}
    \scnidtf{первичная информация, получаемая (приобретаемая) кибернетической
        системой}
    \scnidtf{первичная знаковая конструкция, которая описывает те или иные свойства
        текущего состояния физической окружающей среды (внешней среды и физической
        оболочки) кибернетической системы}

    \scnheader{сенсор кибернетической системы}
    \scnidtf{рецептор кибернетической системы}
    \scntext{пояснение}{Компонент кибернетической системы, генерирующий в памяти
        этой системы информацию о текущем значении соответствующего этому компоненту
        свойства (характеристики, параметра) того фрагмента физической окружающей среды
        кибернетической системы, который непосредственно смежен (пограничен) указанному
        компоненту.}
        
    \scnheader{эффектор кибернетической системы}
    \scnidtf{компонент кибернетической системы, который способен менять своё
        состояние в целях непосредственного воздействия на свою физическую оболочку и
        на внешнюю среду}

    \scnheader{способность кибернетической системы к пониманию сенсорной
        информации}
    \scnidtf{способность к синтаксическому и семантическому анализу информации,
        формируемой сенсорами кибернетической системы, а также к погружению  этой
        информации в состав общей информационной модели внешней среды кибернетической
        системы (в состав общей картины внешнего мира)}
    \scnidtf{способность кибернетической системы к переходу от первичной
        (сенсорной) информации ко вторичной информации, которая описывает связи между
        вторичными объектами, каждый из которых представлен (описан) в первичной
        информации конфигурацией знаков своих частей с дополнительным описанием свойств
        каждой из этих частей}

    \newpage\scnheader{способность кибернетической системы к самостоятельной
        деятельности во внешней среде}
    \begin{scnrelfromlist}{свойство-предпосылка}
        \scnitem{уровень развития эффекторов, обеспечивающих самостоятельное
            перемещение кибернетической системы}
        \begin{scnindent}
            \begin{scnrelfromlist}{частное свойство}
                \scnitem{уровень развития эффекторов, обеспечивающих локальное перемещение
                    сенсоров кибернетической системы}
                \scnitem{уровень развития эффекторов, обеспечивающих функционирование
                    манипуляторов кибернетической системы}
                \scnitem{уровень развития эффекторов, обеспечивающих перемещение всей
                    физической оболочки кибернетической системы}
            \end{scnrelfromlist}
        \end{scnindent}
        \scnitem{качество управления поведением кибернетической системы во внешней
            среде}
        \begin{scnindent}
            \scnidtf{качество сенсорно-эффекторной координации действий
                кибернетической системы при выполнении сложных действий во внешней среде}
        \end{scnindent}
    \end{scnrelfromlist}
    \bigskip

\end{scnsubstruct}
\scnsourcecomment{Завершили Сегмент \scnqqi{Комплекс свойств, определяющих качество решателя задач кибернитической системы}}
		\scnsegmentheader{Комплекс свойств, определяющих уровень обучаемости кибернетической системы}
\begin{scnsubstruct}
    \scnheader{обучаемость кибернетической системы}
    \scnidtf{способность \textit{кибернетической системы} повышать своё качество, адаптируясь к решению новых задач, \textit{качество внутренней информации модели своей среды}, \textit{качество} своего \textit{решателя задач} и даже \textit{качество} своей \textit{физической оболочки}.}
    \scnidtf{способность кибернетической системы к самосовершенствованию с различной степенью самостоятельности (с учителем, с экспертом, с внешними источниками информации, только на собственном опыте)}
    \scnrelboth{следует отличать}{приспособленность кибернетической системы к её совершенствованию, осуществляемому извне}
    \scnidtf{способность кибернетической системы к самостоятельному повышению уровня (качества) своих знаний, навыков, а также уровня своей обучаемости}
    \scnidtf{способность кибернетической системы к самостоятельному самосовершенствованию}
    \scnidtf{скорость эволюции кибернетической системы}
    \scnidtf{уровень (степень) обучаемости кибернетической системы}
    \scnidtf{способность кибернетической системы к совершенствованию (к эволюции, к повышению уровня своего качества)}
    \scntext{примечание}{Максимальный уровень обучаемости кибернетической системы --- это её способность эволюционировать (повышать уровень своего качества) максимально быстро и \uline{в любом}(!) направлении, т.е. способность быстро и без каких-либо ограничений приобретать \uline{любые}(!) новые \textit{знания} и \textit{навыки}.}
    \scnidtf{способность кибернетической системы к повышению своего качества (в том числе, путем устранения своих недостатков, выявленных в результате самоанализа (рефлексии), в частности, в результате работы над своими ошибками, разбора собственных полетов)}
    \scnidtf{способность кибернетической системы к обучению}
    \scnidtf{умение кибернетической системы учиться}
    \scnidtf{способность кибернетической системы обучаться}
    \scntext{примечание}{Реализация способности кибернетической системы обучаться, т.е. решать перманентно инициированную сверхзадачу самообучения, накладывает \uline{дополнительные требования}, предъявляемые к \textit{информации, хранимой в памяти кибернетической системы}, \textit{к решателю задач кибернетической системы}, а в перспективе также и к \textit{физической оболочке кибернетической системы}.}
    \scnidtf{способность кибернетической системы повышать уровень своего интеллекта --- (1) общий (интегральный) уровень качества информации, хранимый в собственной памяти, (2) общий уровень качества своих приобретаемых навыков, (3) уровень своей обучаемости.}
    \scnidtf{способность кибернетической системы к максимально возможной \uline{самостоятельной эволюции}, в процессе которой кибернетическая система сама постоянно заботится о своей эволюции и о повышении темпов этой эволюции}
    \scntext{примечание}{Важнейшей характеристикой \textit{кибернетической системы} является не только то, какой уровень интеллекта (интеллектуальных возможностей) \textit{кибернетическая система} имеет в текущий момент, какое множество действий (задач) она способна выполнять, но и то, насколько быстро этот уровень может повышаться.}
    
    \scnheader{следует отличать*}
    \begin{scnhaselementset}
        \scnitem{образованность кибернетической системы}
        \begin{scnindent}
            \scnidtf{навыки и другие знания, которые кибернетическая приобрела (с учителем, экспертом или самостоятельно) к заданному моменту}
            \scnidtf{результат, который кибернетическая система достигла в процессе своей эволюции к заданному моменту}
        \end{scnindent}
        \scnitem{обучаемость кибернетической системы}
        \begin{scnindent}    
            \scnidtf{скорость повышения уровня образованности кибернетической системы}
            \scnidtf{скорость эволюции кибернетической системы}
        \end{scnindent}
        \scnitem{скорость повышения уровня обучаемости кибернетической системы}
        \begin{scnindent}
            \scnidtf{ускорение повышения уровня образованности кибернетической системы}
            \scntext{примечание}{с увеличением объема и качества приобретаемых кибернетической системой новых навыков и знаний и, в первую очередь, при грамотной их систематизации скорость обучения кибернетической системы существенно возрастает.}
        \end{scnindent}
    \end{scnhaselementset}

    \scnheader{обучаемость кибернетической системы}
    \scnrelfrom{комплекс свойств-предпосылок}{Комплекс свойств, определяющих обучаемость кибернетических систем по уровню обучаемости различных их компонентов}
    \scnrelfrom{комплекс частных свойств}{Комплекс свойств кибернетических систем, определяющих их обучаемость по различным формам обучения}
    
    \scnheader{Комплекс свойств, определяющих обучаемость кибернетических систем по уровню их гибкости, стратифицированности, рефлексивности, активности}
    \begin{scneqtoset}
        \scnitem{гибкость кибернетической системы}
        \scnitem{стратифицированность кибернетической системы}
        \scnitem{рефлексивность кибернетической системы}
        \scnitem{ограниченность обучения кибернетической системы}
        \scnitem{познавательная активность кибернетической системы}
        \scnitem{способность кибернетической системы к самосохранению}
    \end{scneqtoset}

    \scnheader{гибкость возможных самоизменений кибернетической системы}
    \scnidtf{гибкость кибернетической системы при выполнении ею изменений над самой этой системой}
    \begin{scnrelfromset}{комплекс свойств-предпосылок}

        \scnitem{простота возможных самоизменений кибернетической системы}
        \scnitem{многообразие возможных самоизменений кибернетической системы }

    \end{scnrelfromset}
    \begin{scnrelfromset}{комплекс частных свойств}

        \scnitem{семантическая гибкость обработки информации, хранимой в памяти кибернетической системы}
        \scnitem{семантическая гибкость возможных самоизменений решателя задач кибернетической системы}
        \scnitem{гибкость возможных изменений физической оболочки кибернетической системы, осуществляемых самой системой}

    \end{scnrelfromset}

    \scnheader{гибкость возможных самоизменений кибернетической системы}
    \scnidtf{гибкость кибернетической системы при её самосовершенствовании}
    \scnrelto{частное свойство}{гибкость кибернетической системы}
    \begin{scnindent}
        \scntext{примечание}{Поскольку обучение всегда сводится к внесению тех или иных изменений в обучаемую кибернетическую систему, без высокого уровня гибкости этой системы не может быть высокого уровня её обучаемости.}
        \scnrelto{свойство-предпосылка}{обучаемость кибернетической системы}
    \end{scnindent}
    
    \scnheader{простота возможных самоизменений кибернетической системы}
    \scnidtf{легкость (трудоемкость) внесения различных изменений в кибернетическую систему, осуществляемых самой этой кибернетической системой}
    \scnidtf{приспособленность кибернетической системы к самостоятельному внесению различных изменений в саму себя}
    
    \scnheader{стратифицированность кибернетической системы}
    \scnidtf{иерархическая декомпозиция кибернетической системы на такие подсистемы, структура и функционирование которых минимально возможным образом связаны друг с другом, что существенным образом сужает область учета последствий различных изменений вносимых в систему, а также область поиска причин всевозможных ошибок}
    \scnidtf{модульность кибернетической системы}
    \scnidtf{возможность разделить кибернетическую систему на такие части (страты), эволюция (изменения) которых может осуществляться независимо друг от друга.}
    \scntext{примечание}{Уровень стратифицированности определяется \begin{scnitemize}
            \item количеством страт;\item степенью зависимости страт друг от друга. \end{scnitemize}
    }\scntext{примечание}{При наличии стратифицированности кибернетической системы появляется возможность четкого определения области действия различных изменений, вносимых в кибернетическую систему, т.е. возможность четкого ограничения тех частей кибернетической системы, за пределы которых нет необходимости выходить для учета последствий внесенных в систему первичных изменений, т.е. осуществлять \uline{дополнительные} изменения, являющиеся последствиями первичных изменений.}\scntext{примечание}{Стратификация кибернетической системы --- это не просто её структуризация (прежде всего, структуризация информации, хранимой в памяти кибернетической системы), а такая её структуризация, которая четко определяет границы учета возможных последствий вносимых в систему изменений различного вида.}\begin{scnrelfromlist}{частное свойство}

        \scnitem{стратифицированность информации, хранимой в памяти кибернетической системы}
        \scnitem{стратифицированность решателя задач кибернетической системы}
        \scnitem{стратифицированность физической оболочки кибернетической системы}

    \end{scnrelfromlist}

    \scnheader{рефлексивность кибернетической системы}
    \scnidtf{уровень (степень) рефлексивности кибернетической системы}
    \scnidtf{способность кибернетической системы к самоанализу (к анализу интегрального уровня своего качества и, в том числе, уровня своего интеллекта)}
    \scnidtf{способность кибернетической системы самостоятельно анализировать (оценивать) свое качество}
    \scnidtf{уровень рефлексии кибернетической системы}
    \scnidtf{способность кибернетической системы к самоанализу --- к анализу своих знаний, навыков, своих действий во внутренней и внешней среде}
    \scnidtf{способность кибернетической системы к самонаблюдению и самоанализу}
    \scnidtf{способность кибернетической системы к рефлексии}
    \scnidtf{способность кибернетической системы к анализу своего качества}
    \scnidtf{Способность кибернетической системы к самоанализу (к анализу самой себя во всевозможных аспектах).}
    \scntext{примечание}{Конструктивным результатом рефлексии кибернетической системы является генерация в её памяти спецификации различных негативных или подозрительных особенностей, которые следует учитывать для повышения качества кибернетической системы. Такими особенностями (недостатками) могут быть выявленные противоречия (ошибки), выявленные пары синонимичных знаков, омонимичные знаки, информационные дыры и многое другое.}
    \begin{scnrelfromlist}{частное свойство}
        \scnitem{способность кибернетической системы к анализу качества информации, хранимой в её памяти}
        \scnitem{способность кибернетической системы к анализу качества своего решателя задач}
        \begin{scnindent}
            \scnrelfrom{частное свойство}{способность кибернетической системы к анализу качества своего поведения во внешней среде}
        \end{scnindent}
        \scnitem{пособность кибернетической системы к анализу качества своей физической оболочки}
        \begin{scnindent}    
            \scnrelfrom{частное свойство}{способность кибернетической системы к анализу качества физического обеспечения своего интерфейса с внешней средой}
        \end{scnindent}
    \end{scnrelfromlist}

    \scnheader{ограниченность обучения кибернетической системы}
    \scntext{пояснение}{Данное свойство определяет границу между теми знаниями и навыками, которые соответствующая \textit{кибернетическая система} принципиально может приобрести, и теми знаниями и навыками, которые указанная кибернетическая система не сможет приобрести никогда. Данное свойство определяет максимальный уровень потенциальных возможностей соответствующей кибернетической системы. Очевидно, что максимальная степень отсутствия ограничений в приобретении новых знаний и навыков --- это полное отсутствие ограничений, т.е. полная универсальность возможностей соответствующих кибернетических систем, которые всё могут познать и всё могут сотворить.}\scnidtf{максимум того, чему кибернетическая система может обучиться}
    \scnidtf{максимальная перспектива обучения кибернетической системы}
    \scnidtf{максимальный уровень качества, который кибернетическая система может достичь в процессе обучения}
    \begin{scnrelfromlist}{частное свойство}
        \scnitem{максимальный объём знаний, которые кибернетическая система может приобрести в процессе обучения}
        \scnitem{максимальный объём навыков, которые кибернетическая система может приобрести в процессе обучения}
    \end{scnrelfromlist}

    \scnheader{максимальный объём знаний, которые кибернетическая система может приобрести в процессе обучения}
    \scnidtf{граница приобретаемых знаний, за пределы которой кибернетическая система принципиально не может перейти в процессе своего обучения}
    \scnidtf{максимум того, чему можно научить соответствующую кибернетическую систему}
    \scnidtf{максимальный объём знаний, которые кибернетическая система принципиально может приобрести}
    \scnrelto{свойство-предпосылка}{обучаемость}
    \scntext{примечание}{чем больше \textit{максимальный объём знаний, которые кибернетическая система принципиально может приобрести}, тем выше уровень \textit{обучаемости} кибернетической системы}
    
    \scnheader{познавательная активность кибернетической системы}
    \scnidtf{познавательная мотивированность}
    \scnidtf{познавательная пассионарность}
    \scnidtf{любознательность}
    \scnidtf{активность и самостоятельность в приобретении новых знаний и навыков}
    \scnidtf{стремление, активная целевая установка к постоянному совершенствованию (повышению качества) и пополнению собственной базы знаний}
    \scntext{примечание}{Следует отличать
        \begin{scnitemize}
            \item способность (возможность) приобретать новые знания и навыки и совершенствовать приобретенные знания и навыки
            \item от желания (стремления) это делать.
        \end{scnitemize}}
    \scntext{примечание}{желание (целевая установка) научиться решать те или иные задачи может быть сформулировано кибернетической системой либо самостоятельно, либо извне (некоторым учителем).}\begin{scnrelfromlist}{частное свойство}

        \scnitem{активность в изучении внешней среды}
        \scnitem{активность в анализе качества информации, хранимой в собственной памяти}
        \scnitem{активность в анализе собственных действий и действий других кибернетических систем}

    \end{scnrelfromlist}
    \begin{scnrelfromlist}{свойство-предпосылка}

        \scnitem{способность кибернетической системы к синтезу познавательных целей и процедур}
        \scnitem{способность кибернетической системы к самоорганизации собственного обучения}
        \scnitem{способность кибернетической системы к экспериментальным действиям}

    \end{scnrelfromlist}

    \scnheader{способность кибернетической системы к синтезу познавательных целей и процедур}
    \scnidtf{способность планировать своё обучение и управлять процессом обучения}
    \scnidtf{умение задавать вопросы или целенаправленные последовательности вопросов самому себе или другим субъектам как важнейший фактор обучаемости}
    \scnidtf{способность генерировать (формулировать, задавать) вопросы, адресуемые либо самому себе, либо некоторому внешнему источнику знаний и направленные на повышение качества собственных знаний и навыков}
    \scnidtf{способность генерировать четкую спецификацию своей информационной потребности}
    \scnidtf{способность кибернетической системы четко формулировать то, что она не знает (в частности, не умеет), но хотела бы знать и уметь}
    \scnidtf{способность к формированию спецификаций информационных баз в своих знаниях}
    \scnidtf{способность кибернетической системы самостоятельно генерировать цели на приобретение знаний и навыков, обеспечивающих решение различных классов задач}
    
    \scnheader{способность кибернетической системы к самоорганизации собственного обучения}
    \scnidtf{способность осуществлять управление своим обучением}
    \scnidtf{способность кибернетической системы самой выполнять роль своего учителя, организующего процесс своего обучения}
    
    \scnheader{способность кибернетической системы к экспериментальным действиям}
    \scnidtf{способность к отклонениям от составленных планов своих действий для повышения качества результата или сохранении целенаправленности этих действий}
    \scnidtf{способность к экспромтам и импровизации}
    
    \scnheader{способность кибернетической системы к самосохранению}
    \scnidtf{способность кибернетической системы к выявлению и устранению угроз, направленных на снижение её качества и даже на её уничтожение, что означает полную потерю необходимого качества}
    \scnidtf{уровень самообеспечения безопасности (защищенности) кибернетической системы}
    \scntext{пояснение}{Данное свойство кибернетических систем является необходимым фактором высокого уровня обучаемости кибернетических систем. Чем выше уровень безопасности кибернетической системы, тем выше её уровень обучаемости.}\scnidtf{способность кибернетической системы к обеспечению собственной безопасности}
    \begin{scnrelfromlist}{свойство-предпосылка}
        \scnitem{способность кибернетической системы анализировать смысл задач, инициированных извне, и отказываться от решения вредных задач}
    \end{scnrelfromlist}
    \begin{scnindent}
        \scntext{эпиграф}{Прежде, чем выполнять приказ, подумай}
        \scntext{пояснение}{Примером вредной задачи для \textit{ostis-системы} является запрос всех хранимых в памяти \textit{sc-элементов}}
        \scntext{пояснение}{Подчеркнем, что в современных компьютерных системах и интеллектуальных компьютерных системах подходы к обеспечению их информационной безопасности имеют принципиальные отличия, связанные, прежде всего с тем интеллектуальные компьютерные системы обладают более мощными средствами семантического и контекстного анализа приобретаемой информации.}
        \begin{scnindent}
            \scntext{детализация}{\nameref{sd_inf_security}}
        \end{scnindent}
    \end{scnindent}

    \bigskip\scnheader{Комплекс свойств, определяющих обучаемость кибернетических систем по уровню обучаемости различных их компонентов}
    \begin{scneqtoset}
        \scnitem{способность кибернетической системы к повышению качества информации хранимой в её памяти}
        \scnitem{способность кибернетической системы к повышению качества своего решателя задач}
        \scnitem{способность кибернетической системы к повышению качества своей физической оболочки}
    \end{scneqtoset}

    \scnheader{способность кибернетической системы к повышению качества информации, хранимой в её памяти}
    \scnidtf{способность кибернетической системы к постоянному пополнению и совершенствованию информации, хранимой в её памяти, по всевозможным направлениям и, в первую очередь, в направлении повышения уровня адекватности (корректности) и полноты описания своей внешней среды и своей физической оболочки}
    \begin{scnrelfromlist}{свойство-предпосылка}

        \scnitem{семантическая гибкость информации, хранимой в памяти кибернетической системы}
        \scnitem{стратифицированность информации, хранимой в памяти кибернетической системы}
        \scnitem{способность кибернетической системы к повышению уровня структуризации информации, хранимой в памяти кибернетической системы}
        \scnitem{способность кибернетической системы к анализу качества информации, хранимой в её памяти}
        \scnitem{способность кибернетической системы к устранению противоречий, обнаруженных в информации, хранимой в её памяти}
        \scnitem{способность кибернетической системы к устранению информационных дыр, обнаруженных в информации, хранимой в её памяти}
        \scnitem{способность кибернетической системы к удалению информационного мусора, обнаруженного в информации, хранимой в её памяти}
        \scnitem{способность кибернетической системы к погружению новых \textit{знаний} в состав информации, хранимой в её памяти}
        \scnitem{способность кибернетической системы к обнаружению сходств в знаниях, хранимых в её памяти}
        \scnitem{способность кибернетической системы к конвергенции знаний, хранимых в её памяти}
        \scnitem{способность кибернетической системы к интеграции знаний, хранимых в её памяти}
        \scnitem{способность кибернетической системы к обобщениям и формированию новых понятий}
        \scnitem{способность кибернетической системы к генерации гипотез и обнаружению закономерностей в информации, хранимой в её памяти}
        \scnitem{способность кибернетической системы к обоснованию или опровержению знаний, хранимых в её памяти}
        \scnitem{способность кибернетической системы к экспериментальному подтверждению или опровержению гипотез о свойствах динамических систем с помощью имитационных моделей этих систем}
        \scnitem{способность кибернетической системы к коррекции теорий, хранимых в её памяти}

    \end{scnrelfromlist}

    \scnheader{семантическая гибкость информации, хранимой в памяти кибернетической системы}
    \scnidtf{гибкость информации, хранимой в памяти кибернетической системы, при её обработке на семантическом уровне}
    \scnidtf{гибкость возможных действий (операций), выполняемых кибернетической системой над информацией, хранимой в её памяти, и осуществляемых на семантическом (осмысленном) уровне представления этой информации}
    \scnidtf{трудоёмкость содержательного редактирования информации, хранимой в памяти кибернетической системы (поиска, удаления, вставки, замены различных фрагментов информации), при соблюдении семантической целостности и корректности всей редактируемой информации}
    \scntext{примечание}{Обработка информации на семантическом уровне предполагает такие операции над хранимой информацией, как:\begin{scnitemize}
            \item замена имени некоторой сущности \item поиск связи заданного вида между знаками заданных сущностей и корректировка этой связи\item поиск семантической окрестности знака заданной сущности, то есть поиск всех известных связей, инцидентных этому знаку и, соответственно, всех смежных ему знаков\item поиск фрагмента хранимой информации, релевантного заданному семантическому образцу ---  конфигурации знаков сущностей и связей между ними\item удаление или генерация (порождение) связи между заданными знаками\end{scnitemize}
    }\scntext{примечание}{Все операции семантического уровня обработки информации рассматривают обрабатываемую информацию на абстрактном уровне знаков описываемых сущностей и знаков связей между описываемыми сущностями. При этом указанные связи рассматриваются как частный вид описываемых (и, соответственно, обозначаемых) сущностей.}\scnidtf{простота и многообразие редактирования информации, хранимой в памяти кибернетической системы}
    \scnidtf{простота и многообразие внесения изменений в информацию, хранимую в памяти кибернетической системы}
    \scntext{пояснение}{\textit{Гибкость обработки информации, хранимой в памяти кибернетической системы}, определяется не столько трудоемкостью непосредственно самой операции редактирования, сколько теме дополнительными действиями, которые являются обязательными последствиями каждой такой операции редактирования. Так, например, изменение имени какой-либо описываемой сущности требует внесения этого изменения во всех местах, где это имя упоминается, удаление какой-либо связи между известными описываемыми сущностями требует внесения этого изменения везде, где удаляемая связь упоминается.}
    
    \scnheader{стратифицированность информации, хранимой в памяти кибернетической системы}
    \scnidtf{логико-семантическая стратифицированность информации, хранимой в памяти кибернетической системы}
    \scnrelfrom{свойство-предпосылка}{структуризация информации, хранимой в памяти кибернетической системы}
    \scnrelfrom{свойство-предпосылка}{качество метаязыковых средств представления информации, хранимой в памяти кибернетической системы}
    \begin{scnindent}
        \scnidtf{уровень развития метаязыковых средств кодирования (внутреннего представления) информации, хранимой в памяти кибернетической системы}
    \end{scnindent}

    \scnheader{способность кибернетической системы к повышению уровня структуризации информации, хранимой в памяти указанной системы}
    \scnrelboth{следует отличать}{структурированность информации, хранимой в памяти кибернетической системы}
    \begin{scnindent}
        \scnidtf{уровень структуризации информации, хранимой в памяти кибернетической системы}
        \scntext{примечание}{Качественная структуризация информации, хранимой в памяти кибернетической системы, то есть качественное разложение  этой информации по семантическим полочкам  существенно упрощает и, следовательно, ускоряет повышение качества самой этой информации.}\scnrelboth{следует отличать}{структуризация информации, хранимой в памяти кибернетической системы}
        \begin{scnindent}
            \scnsubset{действие, выполняемое кибернетической системой в своей памяти}
            \begin{scnindent}
                \scnsubset{процесс}
            \end{scnindent}
        \end{scnindent}
    \end{scnindent}

    \scnheader{способность кибернетической системы к анализу качества информации, хранимой в её памяти}
    \scnidtf{способность кибернетической системы к анализу информации, хранимой в собственной памяти, для последующего повышения качества этой информации}
    \scnrelto{частное свойство}{способность кибернетической системы к рефлексии}
    \scntext{примечание}{Рефлексия кибернетической системы, то есть анализ собственного качества, включает в себя не только анализ качества информации, хранимой в её памяти, но и анализ собственной деятельности как во внешней среде, так и в собственной памяти. При этом анализ собственной деятельности сводится к анализу описания этой деятельности, представленного в собственной памяти.}
    \begin{scnrelfromlist}{свойство-предпосылка}
        \scnitem{качество метаязыковых средств описания в памяти кибернетической системы качества информации, хранимой в её памяти}
        \scnitem{способность кибернетической системы к обнаружению противоречий в информации, хранимой в её памяти}
        \begin{scnindent}
            \begin{scnrelfromlist}{частное свойство}
                \scnitem{способность кибернетической системы к обнаружению пар синонимичных знаков, входящих в состав информации, хранимой в её памяти}
                \scnitem{способность кибернетической системы к обнаружению семантически эквивалентных фрагментов, входящих в состав информации, хранимой в её памяти}
                \scnitem{способность кибернетической системой к обнаружению омонимичных знаков в информации, хранимой в её памяти}
            \end{scnrelfromlist}
        \end{scnindent}
        \scnitem{способность кибернетической системы к обнаружению информационных дыр в информации, хранимой в её памяти}
        \scnitem{способность кибернетической системой к обнаружению информационного мусора в информации, хранимой в её памяти}
    \end{scnrelfromlist}

    \scnheader{способность кибернетической системы к устранению противоречий, обнаруженных в информации, хранимой в её памяти}
    \begin{scnrelfromlist}{частное свойство}

        \scnitem{способность кибернетической системы к устранению синонимии знаков, входящих в состав информации, хранимой в памяти указанной системы}
        \scnitem{способность кибернетической системы к устранению семантической эквивалентности фрагментов, входящих в состав информации, хранимой в памяти указанной системы}
        \scnitem{способность кибернетической системы к устранению омонимичных знаков, входящих в состав информации, хранимой в памяти указанной системы}
        \scnitem{способность кибернетической системы к устранению противоречий, обнаруженных в информации, хранимой в памяти указанной системы, и не являющихся обнаруженной синонимией, семантической эквивалентностью или омонимией}

    \end{scnrelfromlist}

    \scnheader{способность кибернетической системы к устранению семантической эквивалентности фрагментов, входящих в состав информации, хранимой в памяти указанной системы}
    \scnidtf{способность кибернетической системы к устранению дублирования информации в рамках памяти указанной системы}
    
    \scnheader{следует отличать*}
    \begin{scnhaselementset}
        \scnitem{семантическая эквивалентность*}
        \begin{scnindent}    
            \scnidtf{эквивалентность информационных конструкций по смыслу (содержанию)*}
        \end{scnindent}
        \scnitem{синтаксическая эквивалентность*}
        \begin{scnindent}
            \scnidtf{эквивалентность информационных конструкций по форме*}
        \end{scnindent}
        \scnitem{логическая эквивалентность*}
        \begin{scnindent}
            \scnidtf{пары информационных конструкций, первая из которых логически следует из второй и наоборот*}
            \scntext{примечание}{Если с семантической эквивалентности в памяти кибернетической системы можно и нужно бороться, то без логической эквивалентности обойтись трудно (как минимум из-за необходимости вводить определяемые понятия и, соответственно, формулировать определения). Тем не менее, логической эквивалентностью и, в частности, расширением числа определяемых понятий увлекаться не следует. Так, например, если определение нового понятия не является громоздким (в частности, понятия, являющегося теоретико-множественным объединением или пересечением ранее введенных понятий), то явно вводить это новое понятие не следует.}
        \end{scnindent}
    \end{scnhaselementset}

    \scnheader{способность кибернетической системы к устранению информационных дыр, обнаруженных в информации, хранимой в ее памяти}
    \scnrelfrom{свойство-предпосылка}{способность кибернетической системы генерировать ответы на вопросы различного вида в случае, если они целиком или частично отсутствуют в текущем состоянии информации, хранимой в памяти}
    \scntext{примечание}{Формальным результатом обнаружения информационной дыры является формулировка запроса на недостающую информацию, которую необходимо сгенерировать.}
    
    \scnheader{способность кибернетической системы к удалению информационного мусора, обнаруженного в информации, хранимой в ее памяти}
    \scnidtf{способность кибернетической системы к забыванию (стиранию, удалению) ненужной (лишней, отработанной ) информации, которая, например, играет роль информационных лесов  при решении различных задач}
    \scntext{примечание}{Критериями информационного мусора может быть:\begin{scnitemize}
            \item завершение решения задачи, для которой данная информация является вспомогательной и востребованной только в рамках решения соответствующей задачи;\item истечение срока давности хранения;\item легкая воспроизводимость (при необходимости).\end{scnitemize}
    }
    
    \scnheader{способность кибернетической системы к семантическому погружению новых знаний в состав информации, хранимой в ее памяти}
    \scntext{примечание}{Новая введенная в память информационная конструкция трактуется как конструкция, у которой входящие в нее знаки являются потенциальными синонимами тем знакам, которые уже присутствуют в хранимой информации. Поэтому для всех этих знаков надо проверить наличие их синонимов. После этого синонимичные знаки должны быть отождествлены. Отождествление знаков осуществляется либо путем приписывания им одинаковых идентификаторов (имен), либо путем физического  склеивания этих знаков.}\scntext{примечание}{Новой информацией, погружаемой (вводимой) в состав информации, хранимой в памяти кибернетической системы, может быть:\begin{scnitemize}
            \item либо принятое сообщение, поступившее от другой кибернетической системы и переведенное на внутренний язык данной системы;\item либо информация, сгенерированная в результате решения какой-либо задачи.\end{scnitemize}
    }
    
    \scnheader{способность кибернетической системы к обнаружению сходств в знаниях, хранимых в ее памяти}
    \scntext{примечание}{Сходства в знаниях могут иметь самый разнообразный вид и далеко не всегда являются очевидными.}\scntext{примечание}{Умение видеть  сходство в различном и различие в сходном является важнейшим признаком интеллекта.}
    
    \scnheader{способность кибернетической системы к конвергенции знаний, хранимых в ее памяти}
    \scnidtf{способность кибернетической системы к увеличению сходств в знаниях хранимых в ее памяти}
    \scnrelfrom{свойство-предпосылка}{способность к увеличению числа общих понятий для различных фрагментов информации, хранимой в памяти кибернетической системы, без ущерба качеству этих фрагментов}
    \scnidtf{способность к сближению  знаний путем:\begin{scnitemize}
            \item увеличения числа общих понятий, используемых в сближаемых  знаниях;\item преобразования исходных знаний к их логически эквивалентным вариантам в целях получения фрагментов как можно большего размера и как можно в большем количестве, которые были бы:\begin{scnitemizeii}
            \item либо синтаксически изоморфными и содержащими как можно большее число общих понятий;\item либо синтаксически изоморфными и одновременно семантически эквивалентными.\end{scnitemizeii}
        \end{scnitemize}
    }

    \scnheader{способность кибернетической системы к интеграции знаний, хранимых в ее памяти}
    \scnidtf{способность объединять имеющиеся знания и формировать целостную картину различных исследуемых объектов, систем, процессов, явлений}
    \scnrelfrom{свойство-предпосылка}{способность кибернетической системы к конвергенции знаний, хранимых в ее памяти}
    \scntext{примечание}{Качество (глубина) интеграции знаний определяется тем, насколько качественно до этого была проведена конвергенция интегрируемых знаний.}\scntext{примечание}{Качественная (бесшовная , глубокая) интеграция различных знаний, хранимых в памяти кибернетической системы, дает возможность существенно снизить количество хранимых в памяти методов решения задач, так как позволяет некоторые ранее различные классы задач объединить в один класс задач. При этом очевидно, что такая интеграция знаний, хранимых в памяти кибернетической системы, требует разработки \uline{общих} (базовых) синтаксических и семантических принципов представления знаний различного вида.}
    
    \scnheader{конвергенция и интеграция знаний}
    \scntext{примечание}{Мы вынуждены смотреть на окружающую нас внешнюю среду (внешний мир) через замочную скважину  своих сенсоров (рецепторов), своих персональных точек зрения, мировоззрения различных научных дисциплин. Но необходимо помнить, что целостную картину внешней среды (картину мира) можно построить только путем сближения (конвергенции) и соединения (интеграции) самых различных точек зрения, самых различных научных дисциплин и направлений. Мир не делится на различные дисциплины --- он един. Для эффективного решения задач конвергенции и интеграции знаний необходимо построить искусственную (рукотворную) среду (память), в которой было бы удобно не только хранить самые различные знания, но и осуществлять конвергенцию и интеграцию этих знаний. При этом очень важно, чтобы формируемая таким образом информационная модель окружающей нас внешней среды (информационной картины мира) была общедоступна как для просмотра (ознакомления), причем, без каких бы то ни было замочных скважин , так и для ввода новых знаний, представляющих (отражающих) точку зрения их авторов.}\scnrelfrom{эпиграф}{Древнеиндийская притча о слоне и слепцах}
    
    \scnheader{следует отличать*}
    \begin{scnhaselementset}

        \scnitem{конвергенция\scnsupergroupsign}
        \begin{scnindent}
            \scnidtf{Свойство, определяющее степень близости (уровень конвергенции) между двумя заданными сущностями и, в частности, знаниями}
        \end{scnindent}
        \scnitem{конвергенция*}
        \begin{scnindent}
            \scnidtf{Множество пар близких (аналогичных, сходных) сущностей*}
        \end{scnindent}
        \scnitem{конвергенция}
        \begin{scnindent}
            \scnidtf{Множество \uline{процессов} сближения различных пар сущностей}
        \end{scnindent}

    \end{scnhaselementset}
    \begin{scnhaselementset}

        \scnitem{интеграция*}
        \begin{scnindent}
            \scnidtf{Квазибинарное \uline{отношение}, каждая пара которого связывает множество интегрируемых сущностей с результатом интеграции*}
        \end{scnindent}
        \scnitem{интеграция}
        \begin{scnindent}
            \scnidtf{Множество \uline{процессов} интеграции множества заданных сущностей}
        \end{scnindent}

    \end{scnhaselementset}

    \scnheader{способность кибернетической системы к обобщениям и формированию новых понятий}
    \scntext{примечание}{Важным примером обобщения является переход от задач к классам часто решаемых задач.}
    
    \scnheader{cпособность кибернетической системы к генерации гипотез и обнаружению закономерностей в информации, хранимой в ее памяти}
    \scntext{примечание}{Данная способность кибернетической системы является важнейшим фактором эволюции информации, хранимой в памяти кибернетической системы, в направлении перехода от данных (от фактографической информации) к знаниям.}
    
    \scnheader{cпособность кибернетической системы к обоснованию или опровержению знаний, хранимых в ее памяти}
    \scntext{примечание}{Примерами знаний, подлежащих обоснованию или опровержению, являются:\begin{scnitemize}
            \item любое введенное в кибернетическую систему сообщение (любая новая информация, поступающая от любого субъекта);\item формулировка какой-либо задачи, предлагаемой для решения;\item формулировка какого-либо гипотетического утверждения (теоремы), подлежащего доказательству.\end{scnitemize}
    }\scnidtf{способность к объяснению (обоснованию, аргументации) корректности, важности и целесообразности использовать (обратить внимание на) указываемое знание}
    \scnidtf{способность либо находить в текущем состоянии базы знаний, либо генерировать (строить) ответы на \textit{почему-вопросы}}
    
    \scnheader{способность кибернетической системы к экспериментальному подтверждению или опровержению гипотез о свойствах динамических систем с помощью имитационных моделей этих систем}
    \scntext{примечание}{Создание динамических информационных моделей сложных динамических систем и проведение различного рода мысленных  экспериментов с такими моделями является весьма перспективным и мощным методом исследования сложных динамических систем.}
    
    \scnheader{способность кибернетической системы к коррекции теорий, хранимых в ее памяти}
    \scnidtf{способность к адаптации накопленных знаний к различным изменениям условий и жизненных ситуаций}
    \scntext{примечание}{В основе данного свойства кибернетической системы лежит:\begin{scnitemize}
            \item постоянная готовность кибернетической системы подвергнуть сомнению любое знание, хранимое в ее памяти;\item постоянное уточнение степени достоверности каждого знания, хранимого в памяти кибернетической системы.\end{scnitemize}
    }
    \bigskip\scnheader{способность кибернетической системы к повышению качества своего решателя задач}
    \scnidtf{способность кибернетической системы повышать качество своих приобретаемых навыков}
    \begin{scnrelfromlist}{свойство-предпосылка}

        \scnitem{способность кибернетической системы к повышению качества информации, хранимой в ее памяти}
        \scnitem{семантическая гибкость возможных самоизменений решателя задач кибернетической системы}
        \scnitem{стратифицированность решателя задач кибернетической системы}
        \scnitem{способность кибернетической системы к анализу качества своего решателя задач}
        \scnitem{способность кибернетической системы к целенаправленной коррекции своей деятельности}
        \scnitem{способность кибернетической системы к оптимизации хранимых в памяти методов решения задач}
        \scnitem{способность кибернетической системы к генерации новых методов решения задач}
        \scnitem{способность кибернетической системы интегрировать у себя новые приобретаемые извне методы и модели решения задач}

    \end{scnrelfromlist}
    \scnheader{семантическая гибкость возможных самоизменений решателя задач кибернетической системы}
    \scnidtf{простота реализации решателем задач кибернетической системы различного рода изменений самого себя}
    \scntext{примечание}{Очевидно, что семантическая гибкость решателя задач кибернетической системы во многом определяется процессором кибернетической системы (прежде всего, его универсальностью и близостью реализуемой им модели обработки информации к смысловому уровню). Но, поскольку решатель задач кибернетической системы кроме процессора включает в себя хранимые в памяти кибернетической системы методы решения различного вида задач (в том числе, и методы интерпретации методов высокого уровня), семантическая гибкость решателя задач определяется также \textit{семантической гибкостью информации, хранимой в памяти кибернетической системы}.}\scnrelfrom{свойство-предпосылка}{семантическая гибкость информации, хранимой в памяти кибернетической системы}
    \scnheader{стратифицированность решателя задач кибернетической системы}
    \begin{scnrelfromlist}{частное свойство}

        \scnitem{стратифицированность методов и навыков решения задач, представленных в памяти кибернетической системы}
        \scnitem{стратифицированность технологий, соответствующих различным видам деятельности}
        \scnitem{стратифицированность различного вида действий, классов действий и видов деятельности}

    \end{scnrelfromlist}
    \begin{scnindent}
        \scnrelfrom{частное свойство}{стратифицированность различного вида информационных процессов, выполняемых в памяти кибернетической системы}
    \end{scnindent}

    \scnheader{качество внутренних языковых средств кибернетической системы для описания качества собственного решателя задач}
    \scnrelfrom{свойство-предпосылка}{качество внутренних языковых средств кибернетической системы для описания качества собственных действий}
    
    \scnheader{способность кибернетической системы к анализу качества своего решателя задач}
    \scnidtf{способность кибернетической системы к анализу (к оценке качества) своей деятельности в собственной внутренней среде (в своей памяти), а также в своей внешней среде}
    \scntext{примечание}{Анализ качества решателя задач включает в себя:\begin{scnitemize}
            \item анализ качества используемых методов и технологий решения задач;\item анализ качества используемых моделей решения задач;\item анализ полноты набора постоянно инициированных целей (задач), направленных на эволюцию и на борьбу с деградацией (снижением качества) кибернетической системы;\item анализ качества выполняемых действий (процессов решения задач).\end{scnitemize}
    }\scnidtf{способность кибернетической системы к описанию (к построению в своей памяти информационной модели) собственных действий, выполняемых в собственной памяти, а также к анализу и оценке этих действий}
    \scnidtf{способность кибернетической системы к анализу своего поведения в своей внутренней среде (в своей памяти), а также в своей внешней среде и в своей физической оболочке}
    \begin{scnrelfromlist}{свойство-предпосылка}
        \scnitem{качество внутренних языковых средств кибернетической системы для описания качества собственного решателя задач}
        \scnitem{способность кибернетической системы к анализу собственной деятельности}
        \begin{scnindent}
            \begin{scnrelfromlist}{частное свойство}
                \scnitem{способность кибернетической системы к анализу качества информационных процессов, выполняемых в собственной памяти}
                \begin{scnindent}
                    \scnidtf{способность кибернетической системы к анализу качества своего поведения (действий, информационных процессов) в собственной внутренней среде --- своих действий, сводящихся к поиску, генерации, удалению и преобразованию информационных конструкций, хранимых в собственной памяти}
                \end{scnindent}
                \scnitem{способность кибернетической системы к анализу качества своего поведения во внешней среде}
            \end{scnrelfromlist}
        \end{scnindent}
        \scnitem{способность кибернетической системы к анализу качества методов, хранимых в собственной памяти}
        \begin{scnindent}    
            \begin{scnrelfromlist}{частное свойство}
                \scnitem{способность кибернетической системы к анализу качества методов и технологий, используемых ею для выполнения сложных действий в собственной памяти}
                \scnitem{способность кибернетической системы к анализу качества методов и технологий, используемых ею для выполнения сложных действий во внешней среде}
            \end{scnrelfromlist}
        \end{scnindent}
    \end{scnrelfromlist}

    \scnheader{качество внутренних языковых средств кибернетической системы для описания качества собственных действий}
    \scntext{примечание}{В данном свойстве кибернетической системы имеется в виду описание собственных действий, выполняемых кибернетической системой как в своей внутренней среде (в собственной памяти), так и в своей внешней среде.}
    
    \scnheader{способность кибернетической системы к анализу качества своего поведения во внешней среде}
    \scnidtf{способность кибернетической системы к анализу соответствия между тем, что планировалось сделать во внешней среде и тем, что реально получилось}
    \scntext{примечание}{Поведение кибернетической системы во внешней среде рассматривается ею как эксперимент , подтверждающий или опровергающий ее представление о внешней среде.}
    \scnidtf{способность кибернетической системы к анализу своего опыта взаимодействия с внешней средой и, в частности, к выявлению своих ошибок}
   
    \scnheader{способность кибернетической системы к целенаправленной коррекции своей деятельности}
    \scnidtf{способность кибернетической системы к коррекции своего поведения в целях повышения его качества (эффективности)}
    \scnidtf{способность кибернетической системы учиться на ошибках своей деятельности на основе анализа этих ошибок}
    \scnrelfrom{свойство-предпосылка}{способность кибернетической системы к анализу собственной деятельности}
    
    \scnheader{способность кибернетической системы к оптимизации хранимых в памяти методов решения задач}
    \scntext{примечание}{Хранимые в памяти \textit{методы} решения задач разбиваются на следующие классы:
        \begin{scnitemize}
            \item \textit{методы верхнего уровня} --- интерпретируемые методы;
            \item \textit{методы базового уровня}, представленные на базовом языке программирования, который интерпретируется непосредственно процессором кибернетической системы;
            \item \textit{метаметоды}, описывающие интерпретацию методов верхнего уровня.
        \end{scnitemize}
    }
    
    \scnheader{способность кибернетической системы к генерации новых методов решения задач}
    \scntext{примечание}{Целесообразность генерации нового метода решения задач возникает, когда кибернетической системе приходится часто решать эквивалентные задачи некоторого класса. Генерация соответствующего метода и последующая его оптимизация позволяет существенно сократить время решения задач.}\scnidtf{способность кибернетической системы расширять множество используемых ею методов решения задач}
    \scntext{примечание}{Если добавляемые методы соответствуют используемым моделям решения задач, то, кроме добавления самих методов, желательно, чтобы в стратифицированной кибернетической системе никакие другие изменения не потребовались. Если добавляемый метод соответствует новой (ранее не известной) модели решения задач, то желательно, чтобы в стратифицированной кибернетической системе никакие другие изменения не потребовались, кроме добавления агентов, обеспечивающих интерпретацию (описание операционной семантики) методов нового класса.}\scntext{примечание}{Речь идет о методах решения как внутренних задач, решаемых в памяти кибернетической системы, так и внешних задач, решаемых во внешней среде путем управления деятельностью эффекторов и рецепторов кибернетической системы.}\scnrelfrom{свойство-предпосылка}{способность расширять множество использованных моделей решения задач}
    
    \scnheader{способность кибернетической системы интегрировать у себя новые приобретаемые извне методы и модели решения задач}
    \scntext{примечание}{Для обеспечения такой способности необходима:\begin{scnitemize}
            \item разработка универсальной базовой модели решения задач, для которой соответствующие ей методы решения задач интерпретируются процессором кибернетической системы;\item разработка семейства классов методов верхнего уровня, что предполагает:\begin{scnitemizeii}
                \item разработку языков представления методов для каждого класса методов верхнего уровня;\item разработку интерпретаторов для каждого класса методов верхнего уровня на основе указанной выше базовой модели решения задач.\end{scnitemizeii}
        \end{scnitemize}
    }
    \bigskip\scnheader{способность кибернетической системы к повышению качества своей физической оболочки}
    \scnidtf{способность кибернетической системы к самостоятельному совершенствованию (эволюции) своей физической оболочки}
    \scntext{примечание}{Данная способность кибернетической системы накладывает определенные требования к построению ее физической оболочки.}\begin{scnrelfromlist}{свойство-предпосылка}

        \scnitem{гибкость возможных изменений физической оболочки кибернетической системы}
        \scnitem{стратифицированность физической оболочки кибернетической системы}
        \scnitem{способность кибернетической системы к анализу качества своей физической оболочки}
        \begin{scnindent}
            \scnrelfrom{свойство-предпосылка}{качество внутренних языковых средств кибернетической системы для описания качества собственной физической оболочки}
        \end{scnindent}
        \scnitem{способность кибернетической системы расширять и/или совершенствовать набор собственных сенсоров и эффекторов}

    \end{scnrelfromlist}

    \bigskip\scnheader{комплекс свойств кибернетических систем, определяющих их обучаемость по различным формам обучения}
    \begin{scneqtoset}

        \scnitem{обучаемость с учителем}
        \scnitem{самообучаемость с экспертом}
        \scnitem{самообучаемость на основе внешних информационных источников}
        \scnitem{самообучаемость без внешних информационных источников}

    \end{scneqtoset}

    \scnheader{обучаемость с учителем}
    \scnidtf{уровень способности к обучению под управлением внешнего субъекта-учителя}
    \scnidtf{способность кибернетической системы к эффективному обучению с помощью учителя, осуществляющего управление процессом обучения}
    \scnidtf{способность заданной кибернетической системы эффективно обучаться с помощью внешней кибернетической системы (внешнего субъекта, внешнего активного учителя), осуществляющей организацию обучения заданной кибернетической системы на основе различных методик обучения, учитывающих особенности обучаемой системы и определяющих характер (в том числе последовательность) передачи знаний и новыков, а также тестирование качества их усвоения}
    
    \scnheader{самообучаемость с экспертом}
    \scnidtf{способность кибернетической системы к самообучению в диалоге с экспертом-консультантом}
    \scnidtf{способность кибернетической системы не просто задавать нужные для собственного обучения вопросы (информационные цели), но и вести вопросно-ответный диалог с другими субъектами (кибернетическими системами), которые являются экспертами в соответствующей области (указанные эксперты  это своего рода пассивные учителя , которые много знают и умеют в соответствующей области, могут отвечать на вопросы, но не желают управлять процессом передачи этих знаний и умений другим кибернетическим системам)}
    \scnidtf{эффективность самообразования кибернетической системы, в основе которого лежит диалог, управляемый этой обучаемой системой и осуществляемый с кибернетической системой, являющейся носителем востребованных знаний и навыков}
    \scnidtf{эффективность самообучения, осуществляемого в форме консультации}
    \scnidtf{способность управлять процессом самообучения путем формирования последовательности вопросов (познавательных целей), адресуемых внешним субъектам}
    \scnrelfrom{свойство-предпосылка}{способность кибернетической системы к синтезу познавательных целей и процедур}
    
    \scnheader{обучаемость на основе пассивных внешних информационных источников}
    \scnidtf{способность кибернетической системы к извлечению информации, содержащейся во внешних информационных источниках, к поиску нужных внешних источников и к построению на этой основе систематизированной картины мира}
    \scnidtf{эффективность самообучения, основанного на анализе \uline{пассивных} источников информации (документов различного вида, публикаций, текстов, которые необходимо находить в различного рода библиотеках, читать и \uline{понимать})}
    
    \scnheader{самообучаемость без внешних информационных источников}
    \scnidtf{способность кибернетической системы формировать систематизированную модель (картину) окружающей среды, используя для ее непосредственного восприятия и изучения только собственные сенсоры и эффекторы, а также некоторые дополнительные средства, усиливающие возможности сенсоров и эффекторов}
    \scnidtf{эффективность самообучения кибернетической системы, основанного исключительно на собственном опыте, на анализе собственной деятельности и собственных ошибок}
    \scntext{примечание}{Данная способность кибернетической системы является необходимым, но явно недостаточным фактором ее высокого качества. Учиться только на собственном опыте --- существенно понизить уровень своего интеллекта. В этом смысле познавательный процесс социален.}
    \bigskip
\end{scnsubstruct}
\scnsourcecomment{Завершили Сегмент \scnqqi{Комплекс свойств, определяющих уровень обучаемости кибернитической системы}}

		\scnsegmentheader{Комплекс свойств, определяющих качество многоагентной
    системы}
\begin{scnsubstruct}
    \scnheader{многоагентная система}
    \scntext{пояснение}{Переход от \textit{кибернетических систем} к коллективам
        взаимодействующих между собой \textit{кибернетических систем}, т.е. к
        социальной организации кибернетических систем, является важнейшим фактором
        эволюции \textit{кибернетических систем}.}\scnsubset{кибернетическая система}
    \begin{scnsubdividing}

        \scnitem{моногенная многоагентная система}
        \begin{scnindent}    
            \scnidtf{однородная \textit{многоагентная система}, состоящая из однотипных \textit{агентов}}
        \end{scnindent}
        \scnitem{гетерогенная многоагентная система}
        \begin{scnindent}
            \scnidtf{неоднородная \textit{многоагентная система}, состоящая из \textit{агентов} разного типа}
        \end{scnindent}

    \end{scnsubdividing}
    \begin{scnsubdividing}

        \scnitem{простая многоагентная система}
        \begin{scnindent}    
            \scnidtf{многоагентная система, \textit{агенты} которой не являются \textit{многоагентными системами}}
        \end{scnindent}
        \scnitem{иерархическая многоагентная система}
        \begin{scnindent}
            \scnidtf{многоагентная система, некоторые или все \textit{агенты} которой являются \textit{многоагентнымисистемами}}
        \end{scnindent}

    \end{scnsubdividing}
    \scntext{примечание}{Агенты \textit{многоагентной системы} могут (но вовсе не
        обязательно должны) быть \textit{интеллектуальными системами}. Так, например,
        агенты интеллектуального решателя задач, имеющего агентно-ориентированную
        архитектуру, не являются интеллектуальными системами.}
        
    \scnheader{агент*}
    \scnidtf{быть агентом данной многоагентной системы*}
    \scnidtf{быть кибернетической системой, входящей в состав данной многоагентной
        системы*}
    \scntext{примечание}{Агентом иерархической многоагентной системы может быть другая
        многоагентная система}\scnsuperset{член многоагентной системы*}
    \begin{scnindent}
        \scnidtf{агент многоагентной системы, не являющийся агентом другого агента этой системы*}
        \scnidtf{непосредственный (ближайший) агент многоагентной системы*}
    \end{scnindent}

    \scnheader{кибернетическая система}
    \begin{scnsubdividing}
        \scnitem{индивидуальная кибернетическая система}
        \begin{scnindent}
            \scnidtftext{пояснение}{минимальная целостная \textit{кибернетическая система} обладающая достаточно высоким уровнем самостоятельности и способности выживать  в своей \textit{внешней среде}}
        \end{scnindent}
        \scnitem{кибернетическая система, являющаяся минимальным компонентом
            индивидуальной кибернетической системы}
        \begin{scnindent}
            \scntext{пояснение}{Это такой компонент, в состав которого не входят \textit{кибернетические системы}}
        \end{scnindent}
        \scnitem{кибернетическая система, являющаяся комплексом компонентов
            соответствующей индивидуальной кибернетической системы}
        \scnitem{сообщество индивидуальных кибернетических систем}
        \begin{scnindent}    
        \begin{scnsubdividing}
                \scnitem{простое сообщество индивидуальных кибернетических систем}
                \scnitem{иерархическое сообщество индивидуальных кибернетических систем}
            \end{scnsubdividing}
        \end{scnindent}
    \end{scnsubdividing}

    \scnheader{многоагентная система}
    \scnidtf{коллектив взаимодействующих  кибернетических систем}
    \begin{scnsubdividing}

        \scnitem{сообщество индивидуальных кибернетических систем}
        \scnitem{индивидуальная кибернетическая система, реализованная в виде многоагентной системы}
            \begin{scnindent}
                \scnsubset{кибернетическая система, являющаяся комплексом компонентов
                    соответствующей индивидуальной кибернетической системы}
                \scntext{пояснение}{Такая внутренняя	\textit{многоагентная система} в
                    индивидуальной кибернетической системе появляется, когда на определенном этапе
                    её эволюции \textit{решатель задач} \textit{индивидуальной кибернетической системы} переходит  на
                    \textit{агентно-ориентированную модель обработки информации} в памяти \textit{индивидуальной компьютерной системы}}
            \end{scnindent}

    \end{scnsubdividing}
    \scnidtf{кибернетическая система, представляющая собой коллектив
        взаимодействующих кибернетических систем, обладающих определенной степенью
        самостоятельности (самодостаточности, свободы выбора)}

    \scnheader{многоагентная система с централизованным управлением}
    \scnidtf{многоагентная система, в которой специально выделяются агенты, которые
        принимают решения в определенной области деятельности многоагентной системы и
        обеспечивают выполнение этих решений  путем управления деятельностью остальных
        агентов, входящих в состав этой системы}
    \scnsubset{многоагентная система}
    
    \scnheader{сообщество интеллектуальных систем с децентрализованным управлением}
    \scnidtf{многоагентная система с децентрализованным управлением, агентами
        которой являются интеллектуальные системы}
    \scnidtf{многоагентная система, в которой решения принимаются коллегиально и
        автоматически  (\uline{решения} о признании новой кем-то предложенной
        информации --- в том числе, об инициировании некоторой задачи, \uline{решения} о
        коррекции (уточнении) уже ранее признанной (одобренной, согласованной)
        информации) \uline{на основе} четко продуманной и постоянно совершенствуемой
        методики, а также \uline{на основе} активного участия всех агентов в
        формировании новых предложений, подлежащих признанию (одобрению, согласованию)}
    \scnsubset{многоагентная система}
    \scntext{примечание}{В такой многоагентной системе все агенты участвуют в управлении
        этой системы}\scnhaselement{Экосистема OSTIS}
    \scntext{пояснение}{В такой многоагентной системе отсутствуют специально
        назначенные  агенты, которые обязаны  принимать решения о том, какую
        коллективно решаемую задачу надо инициировать, и о том, как распределить между
        агентами подзадачи указанной инициированной задачи.}\scnsubset{многоагентная
        система с децентрализованным управлением}
    \scnsubset{сообщество интеллектуальных систем}
    \scntext{примечание}{Примером такой системы является оркестр, способный играть без
        дирижера. При этом подчеркнем, что каждый музыкант такого
        оркестра:\begin{scnitemize}

            \item должен иметь квалификацию не только музыканта, но и дирижера и даже
            композитора
            \item должен быть договороспособным --- уметь согласовывать свои действия с
            действиями коллег\end{scnitemize}
        Аналогичным примером децентрализованной многоагентной системы является
        строительная бригада, способная построить дом без бригадира, прораба,
        архитектора.}
    
    \scnheader{синергетическая кибернетическая система}
    \scnidtf{эволюционная многоагентная система}
    \scnidtf{многоагентная система, состоящая из когнитивных агентов}
    \scnidtf{многоагентная система, обладающая высоким уровнем коллективного
        интеллекта, атомарными агентами которой являются индивидуальные
        интеллектуальные системы, имеющие высокий уровень социализации}
    \scnrelfrom{пояснение}{Ярушкина.Н.Г.НечетГС-2007кн.-стр.88-101}
    \begin{scnindent}
        \scnrelto{цитата}{\cite{YarushinaHS}}
    \end{scnindent}
    \begin{scnrelfromlist}{пояснение}

        \scnitem{\cite{Tarasov1997}}
        \scnitem{\cite{Tarasov1998}}

    \end{scnrelfromlist}
    \scntext{примечание}{Очевидным примером синергетической кибернетической системы
        является творческий коллектив, реализующий сложный наукоемкий проект. Огромная
        сложность создания таких коллективов является главной причиной медленного
        развития целого ряда весьма актуальных научно-технических проектов, таких как
        создание принципиально нового технологического уровня автоматизации
        человеческой деятельности на основе интеллектуальных семантически совместимых
        компьютерных систем, способных самостоятельно взаимодействовать друг с
        другом.}
        
    \scnheader{многоагентная система}
    \scntext{примечание}{Переход к \textit{многоагентным системам} является важнейшим
        фактором повышения \textit{качества} (и, в частности, уровня
        \textit{интеллекта}) \textit{кибернетических систем}, т.к. уровень интеллекта
        \textit{многоагентной системы} может быть значительно выше уровня интеллекта
        каждого входящего в неё агента. Но это бывает далеко не всегда, поскольку
        важнейшим фактором качества многоагентных систем является не только качество
        входящих в неё агентов, но и организация взаимодействия агентов и, в частности,
        переход от централизованного к децентрализованному управлению. Количество
        далеко не всегда переходит в новое качество.
        \\Повышение уровня интеллекта многоагентной системы
        обеспечивается\begin{scnitemize}

            \item не только повышением уровня интеллекта и, в первую очередь, уровня
            \textit{социализации} ее агентов;
            \item не только переходом от централизованного к децентрализованного управлению
            деятельности управлению деятельностью агентов;
            \item но и качеством общей базы знаний всей многоагентной
            системы.\end{scnitemize}
    }
    
    \scnheader{социализация кибернетической системы}
    \scntext{примечание}{Когда мы говорим о \textit{социализации кибернетических систем},
        речь идет только об \textit{индивидуальных кибернетических системах}, т.е. о
        \textit{кибернетических  системах}, достигших некоторого уровня целостности и
        автономности и способных входить в состав различных коллективов. Соответственно
        этому, качество \textit{индивидуальных кибернетических систем} определяется,
        кроме всего прочего тем, насколько большой вклад \textit{индивидуальная
            кибернетическая система} вносит в повышение качества тех коллективов, в состав
        которых она входит. Указанное свойство \textit{индивидуальных кибернетических
            систем} будем называть уровнем их \textit{социализации}. Прежде, чем
        детализировать это свойство, целесообразно рассмотреть то, чем определяется
        качество коллектива кибернетических систем, например, качество творческого
        сообщества компьютерных систем и людей.}
        
    \scnheader{качество сообщества компьютерных систем и людей}
    \scntext{пояснение}{Эффективность творческого коллектива (например в области
        научно-технической деятельности) определяется:\begin{scnitemize}

            \item согласованностью мотивации (целевой установки) всего коллектива и каждого
            его члена:\begin{scnitemizeii}

                \item не должно быть синдрома лебедя, рака и щуки;
                \item не должно быть противоречий между целью коллектива и творческой
                самореализацией каждого его члена;\end{scnitemizeii}

            \item эффективной организацией децентрализованного управления деятельностью
            членов сообщества;
            \item четкой, оперативной и доступной всем фиксацией документации текущего
            состояния содеянного и направлений его дальнейшего развития;
            \item уровнем трудоемкости оперативности фиксации индивидуальных результатов в
            рамках коллективно создаваемого общего результата;
            \item уровнем структурированности и, прежде всего, стратифицированности
            обобщенной документации  (базы знаний);
            \item эффективностью ассоциативного доступа к фрагментам документации;
            \item гибкостью коллективно создаваемой базы;
            \item автоматизацией анализа содеянного и управления проектом.
        \end{scnitemize}
    }
    
    \scnheader{качество многоагентной системы}
    \begin{scnrelfromlist}{свойство-предпосылка}

        \scnitem{средний уровень интеллекта членов многоагентной системы}
        \scnitem{средний уровень социализации членов многоагентной системы}
        \scnitem{минимальный уровень социализации членов многоагентной системы}
            \begin{scnindent}
                \scntext{примечание}{Члены многоагентной системы, имеющие низкий уровень
                    социализации, существенно снижают качество системы.}
            \end{scnindent}
        \scnitem{качество организации взаимодействия членов многоагентной системы}
            \begin{scnindent}
            \scntext{примечание}{Высший уровень качества организации взаимодействия агентов
                многоагентной системы обеспечивается:
                \begin{scnitemize}
                    \item введением дополнительного специального (корпоративного) агента,
                    выполняющего функцию хранителя интегратора общих (корпоративных) знаний
                    многоагентной системы
                    \item реализацией децентрализованного взаимодействия агентов, управляемого
                    текущим состоянием информации, хранимой в памяти корпоративного агента.
                \end{scnitemize}}
            \end{scnindent}

    \end{scnrelfromlist}
    \bigskip
    \begin{scnset}
        \scnheader{ostis-система}
        \begin{scnindent}
            \scnsubset{многоагентная система, управляемая общей базой знаний}
        \end{scnindent}
    \end{scnset}
    \scntext{примечание}{Агенты \textit{ostis-системы} (sc-системы) являются
        \uline{специализированными} \textit{кибернетическими системами},
        \uline{действия} каждой из которых (кроме \textit{сенсорных sc-агентов})
        инициализируются определенного вида ситуациями и/или событиями в памяти
        \textit{ostis-системы} и \uline{заключаются} (за исключением
        \textit{эффекторных sc-агентов}) в преобразовании текущего состояния
        информации, хранимой в этой памяти. Таким образом, sc-агенты не являются
        интеллектуальными системами.}
    \bigskip
\end{scnsubstruct}
    \scnsourcecomment{Завершили Сегмент \scnqqi{Комплекс свойств, определяющих качество многоагентной системы}}

		\scnsegmentheader{Комплекс свойств, определяющих уровень социализации
    кибернетической системы как фактора существенного повышения уровня ее
    обучаемости, а также фактора существенного повышения качества всех тех
    многоагентных систем, в состав которых входит данная кибернетическая система}

\begin{scnsubstruct}
    \scnidtf{Комплекс свойств \textit{кибернетической системы}, определяющих
        необходимые требования к тем \textit{кибернетическим системам}, которые могут
        входить в состав \textit{синергетических кибернетических систем}}
    \scnheader{социализация кибернетической системы}
    \scnidtf{способность кибернетической системы взаимодействовать с другими
        кибернетическими системами в целях создания коллектива кибернетических систем
        (\textit{многоагентных систем}), уровень качества и, в частности, уровень
        \textit{интеллекта} которого выше уровня качества каждой
        \textit{кибернетической системы}, входящей в состав этого коллектива)}
    \scnidtf{комплекс способностей кибернетической системы, которые определяют ее
        вклад в уровень коллективной (социальной) интеллектуальности, т.е. в уровень
        интеллектуальности того коллектива кибернетических систем, членом которого
        данная кибернетическая система является (в уровень интеллектуальности
        соответствующей многоагентной системы)}
    \scnidtf{уровень вклада \textit{кибернетической системы} в обеспечение
        \textit{интеллекта} тех многоагентных систем, в состав которых эта
        \textit{кибернетическая система} входит}
    \scnidtf{уровень социализации кибернетической системы}
    \scnidtf{социализация}
    \scntext{примечание}{Уровень \textit{интеллекта} коллектива кибернетических систем
        (\textit{многоагентной системы}) может быть значительно ниже уровня
        \textit{интеллекта} самого глупого  члена этого коллектива, но может быть и
        значительно выше уровня \textit{интеллекта} самого умного  члена указанного
        коллектива. Для того, чтобы количество \textit{интеллектуальных систем}
        переходило в существенно более интеллектуальное качество коллектива таких
        систем, все объединяемые в коллектив \textit{интеллектуальные системы} должны
        иметь высокий уровень \textit{социализации}, что накладывает
        \uline{дополнительные требования}, предъявляемые к \textit{информации, хранимой
            в памяти}, а также к \textit{решателям задач}
        %\bigspace
        \textit{интеллектуальных систем}, объединяемых в
        коллектив.}\scntext{примечание}{Коллектив \textit{кибернетических систем} может иметь
        значительно более высокий уровень качества, в том числе, уровень интеллекта,
        чем уровень качества \textit{кибернетических систем}, являющихся членами этого
        коллектива. Но так бывает не всегда. Для того, чтобы количество членов
        коллектива \textit{кибернетической системы} перешло в более высокое качество
        самого коллектива, члены коллектива должны обладать дополнительными
        способностями, которые будем называть свойствами \textit{социализации}.
        Основными такими свойствами являются способность устанавливать и поддерживать
        достаточный уровень \textit{семантической совместимости} (взаимопонимания) с
        другими кибернетическими системами и \textit{договороспособность} (способность
        согласовывать свои действия с другими).}\scntext{примечание}{Целенаправленный обмен
        информацией между \textit{кибернетическими системами} существенно ускоряет
        процесс их обучения (процесс накопления знаний и навыков). Следовательно,
        способность эффективно использовать указанный канал накопления знаний и навыков
        существенно повышает уровень \textit{обучаемости}
        %\bigspace
        \textit{кибернетических систем}. В этом смысле можно сказать, что
        познавательный процесс социален.}\scnidtf{уровень развития социально значимых
        качеств кибернетической системы}
    \scntext{примечание}{Повышение уровня \textit{социализации}
        %\bigspace
        \textit{кибернетической системы} является, с одной стороны, дополнительным
        повышением уровня \textit{интеллекта} самой этой \textit{кибернетической
            системы}, а также фактором повышения уровня \textit{интеллекта} тех
        коллективов, тех \textit{многоагентных систем}, в состав которых эта
        \textit{кибернетическая система} входит.}\scntext{примечание}{Переход к
        \textit{многоагентным системам} не только является важным фактором повышения
        качества \textit{кибернетических систем}, но также имеет и обратную сторону
        медали	-- появление целого ряда угроз, связанного с возможными
        целенаправленными вредоносными воздействиями на \textit{многоагентную систему}
        (со стороны некоторых ее \textit{агентов}), существенно снижающими уровень ее
        качества. Наличие таких \textit{вредоносных целей} у соответствующих
        \textit{агентов} свидетельствует о нижайшем уровне \textit{социализации} этих
        \textit{агентов}.}\scnidtf{умение согласовывать (синхронизировать) свою
        деятельность с деятельностью других кибернетических систем в процессе решения
        задач, требующих коллективных усилий}
    \scnidtf{умение участвовать в децентрализованном процессе распределения
        подзадач некоторой коллективно (распределенно) решаемой задачи между членами
        заданного коллектива кибернетических систем и умение участвовать в управлении
        коллективного решения указанной задачи}
    \begin{scnindent}
        \scntext{примечание}{Речь идет о децентрализованном асинхронном управлении деятельностью коллектива кибернетических систем}
    \end{scnindent}
    \scnidtf{способность и готовность кибернетической системы к координации своей деятельности в рамках
        коллектива кибернетических систем, в состав которого она входит в целях:
        \begin{scnitemize}

            \item эффективного решения тактических задач, решаемых указанным коллективом;
            \item решения главной стратегической задачи этого коллектива --- обеспечения как
            можно более высокой скорости роста уровня интеллекта указанного коллектива.
        \end{scnitemize}
    }
    \scntext{примечание}{Подчеркнем, что повышение уровня интеллекта коллектива
        кибернетической системы (многоагентной системы) имеет свои особенности:
        \begin{scnitemize}

            \item во-первых, это забота о семантической совместимости кибернетических
            систем входящих в состав коллектива;
            \item во-вторых, это переход от виртуальной распределенной базы знаний
            коллектива к реально поддерживаемым базам знаний и к порталам корпоративных
            знаний, реализованных в виде индивидуальных кибернетических систем, через
            которые осуществляются все процессы координации и согласования деятельности
            соответствующих членов коллектива
        \end{scnitemize}
    }
    \begin{scnrelfromlist}{свойство-предпосылка}

        \scnitem{договороспособность кибернетической системы}
        \begin{scnindent}
            \scnidtftext{часто используемый sc-идентификатор}{договороспособность}
        \end{scnindent}
        \scnitem{социальная ответственность кибернетической системы}
            \begin{scnindent}
                \scnidtftext{часто используемый sc-идентификатор}{социальная ответственность}
            \end{scnindent}
        \scnitem{социальная активность кибернетической системы}
            \begin{scnindent}
                \scnidtftext{часто используемый sc-идентификатор}{социальная активность}
            \end{scnindent}

    \end{scnrelfromlist}
    \bigskip\scnheader{договороспособность кибернетической системы}

    \begin{scnrelfromlist}{свойство-предпосылка}

        \scnitem{способность кибернетической системы к пониманию принимаемых сообщений}
        \scnitem{способность кибернетической системы к формированию передаваемых
            сообщений, понятных адресатам}
        \scnitem{семантическая совместимость кибернетической системы с партнёрами }
        \scnitem{способность кибернетической системы к обеспечению семантической
            совместимости с партнёрами }
        \scnitem{коммуникабельность кибернетической системы }
        \scnitem{способность кибернетической системы к обсуждению и согласованию целей
            и планов коллективной деятельности }
        \scnitem{способность кибернетической системы брать на себя выполнение
            актуальных задач в рамках согласованных планов коллективной деятельности}

    \end{scnrelfromlist}

    \scnheader{способность кибернетической системы к пониманию принимаемых
        сообщений}
    \scnidtf{способность кибернетической системы к пониманию информации,
        поступающей извне от других кибернетических систем}
    \scnidtf{способность кибернетической системы к отображению принимаемых
        сообщений в семантически эквивалентные фрагменты собственной базы знаний}

    \begin{scnrelfromset}{комплекс частных свойств}

        \scnitem{способность кибернетической системы к пониманию принимаемых вербальных
            сообщений }
        \scnitem{способность кибернетической системы к пониманию принимаемых
            невербальных сообщений}   

    \end{scnrelfromset}
    \scnrelfrom{свойство-предпосылка}{способность кибернетической системы к
        обеспечению семантической совместимости с партнёрами}
    
    \scnheader{сообщение}
    \scnidtf{информация, передаваемая (пересылаемая) от одной кибернетической
        системы к другой или к другим кибернетическим системам}
    \scntext{примечание}{Каждому \textit{сообщению} ставится в соответствие одна
        \textit{кибернетическая система}, являющаяся \textbf{\textit{источником
                сообщения*}} и одна или несколько \textit{кибернетических систем}, являющихся
        \textbf{\textit{адресатами сообщения*}}. В соответствии с этим для каждой
        \textit{кибернетической системы} те сообщения, \textit{источником*} которых она
        является, будем называть \textbf{\textit{передаваемыми сообщениями*}}, а те
        сообщения, \textit{адресатами*} которых она является, будем называть
        \textbf{\textit{принимаемыми сообщениями*}}.}
    \begin{scnsubdividing}

        \scnitem{вербальное сообщение}
        \begin{scnindent}
            \scnidtf{передаваемая словесная информация}
        \end{scnindent}
        \scnitem{невербальное сообщение}
        \begin{scnindent}    
            \scntext{примечание}{Примерами невербальных сообщений являются пересылаемые фото-документы, видео-материалы}
        \end{scnindent}

    \end{scnsubdividing}

    \begin{scnrelfromset}{обобщённая декомпозиция}

        \scnitem{спецификация сообщения}
            \begin{scnindent}
                \begin{scnrelfromset}{обобщённая декомпозиция}

                    \scnitem{указание источника специфицируемого сообщения}
                    \scnitem{указание множества адресатов специфицируемого сообщения}
                    \scnitem{отметка момента времени отправления специфицируемого сообщения}
                    \scnitem{указание прагматического типа специфицируемого сообщения}
                    \scnitem{указания запроса, ответом на который является специфицируемое сообщение}
                        \begin{scnindent}
                            \scntext{примечание}{Если специфицируемое сообщение является ответом на некоторый запрос}
                        \end{scnindent}
                    \scnitem{указание раздела баз знаний адресатов, которому соответствует специфицируемое сообщение}
                    \scnitem{указание способа представления тела сообщения}
                        \begin{scnindent}   
                            \scntext{примечание}{Для вербальных сообщений это указание используемого  внешнего языка}
                        \end{scnindent}
                \end{scnrelfromset}
            \end{scnindent}
        \scnitem{тело сообщения}
            \begin{scnindent}    
                \scnidtf{собственно само сообщение}
            \end{scnindent}

    \end{scnrelfromset}
    \scnrelfrom{разбиение}{прагматический тип сообщения}

    \begin{scneqtoset}

        \scnitem{повествовательное сообщение}
            \begin{scnindent}
                \scnsuperset{ответ на запрос}
            \end{scnindent}
        \scnitem{вопросительное сообщение}
        \scnitem{команда редактирования баз знаний адресатов}
        \scnitem{команда, инициирующая действие адресатов в их внешней среде}

    \end{scneqtoset}

    \scnheader{следует отличать*}
    \begin{scnhaselementset}
        \scnitem{вербальная информация}
        \scnitem{файл, содержащий вербальную информацию}
            \begin{scnindent}
                \scnidtf{вербальная информация, представленная в виде файла}
            \end{scnindent}
        \scnitem{вербальное сообщение}
    \end{scnhaselementset}

    \scnheader{вербальная информация}
    \scnidtf{знаковая конструкция, которая имеет в общем случае произвольную
        денотационную семантику и которая может либо поступать на вход кибернетической
        системы через соответствующие ее сенсоры (рецепторы), либо через
        соответствующие эффекторы передаваться (пересылаться) в качестве сообщения
        другим кибернетическим системам}

    \scnheader{следует отличать*}
    \begin{scnhaselementset}
        \scnitem{вербальная информация}
        \scnitem{сенсорная информация}
    \end{scnhaselementset}
    \begin{scnindent}
        \scntext{примечание}{И \textit{вербальная информация} и \textit{сенсорная информация}
            являются \textit{знаковыми конструкциями}, но, во-первых, \textit{вербальная
            информация} может быть как внешней знаковой конструкцией, так и внутренней
            знаковой конструкцией, хранимой в памяти кибернетической системы, а
            \textit{сенсорная информация} всегда является внутренней \textit{знаковой
            конструкцией} кибернетической системы и, во-вторых, \textit{сенсорная
            информация} описывает только пограничную  для \textit{кибернетической системы}
            физическую \textit{окружающую среду}, тогда, как \textit{вербальная информация}
            может описывать все, что угодно.}
    \end{scnindent}

    \scnheader{следует отличать*}
    \begin{scnhaselementset}
        \scnitem{невербальная информация}
        \scnitem{файл, содержащий невербальную информацию}
            \begin{scnindent}    
                \scnidtf{файл, содержимым которого является электронный образ некоторой невербальной информации}
            \end{scnindent}
        \scnitem{невербальное сообщение}
            \begin{scnindent}
            \scnidtf{невербальная информация, представленная в виде файла и передаваемая
                (пересылаемая) от одной кибернетической системы к другой}
            \end{scnindent}
        \scnitem{сенсорная информация}
            \begin{scnindent}
                \scnidtf{информация, формируемая сенсорами кибернетической системы}
            \end{scnindent}
    \end{scnhaselementset}\

    \scnheader{невербальная информация}
    \scnsuperset{музыкальное произведение}
    \scnsuperset{танец}
    \scnsuperset{произведение изобразительного искусства}
    \begin{scnindent}
        \scnsuperset{живопись}
        \scnsuperset{скульптура}
        \scnsuperset{графика}
    \end{scnindent}
    \scnsuperset{статическое изображение}
    \scnsuperset{динамическое изображение}

    \scnheader{способность кибернетической системы к пониманию принимаемых
        вербальных сообщений}
    \scnidtf{способность кибернетической системы к пониманию вербальной информации,
        поступающей извне из разных источников}
    \scntext{примечание}{Понимание информации, поступающей извне, включает в себя:
        \begin{scnitemize}

            \item перевод этой информации на внутренний язык кибернетической системы;
            \item локальную верификацию вводимой информации;
            \item погружение (конвергенцию, размещение) текста, являющегося результатом
            указанного перевода в состав хранимой информации (в частности, в состав базы
            знаний)
        \end{scnitemize}
    }\scntext{примечание}{Погружение вводимой информации в состав базы знаний
        кибернетической системы сводится к выявлению и устранению противоречий,
        возникающих между погружаемым текстом и текущего состояния базы знаний. Первым
        уровнем таких противоречий являются появляющиеся при интеграции погружаемого
        текста с текущим состоянием базы знаний \textit{омонимичные знаки} и пары
        \textit{синонимичных знаков}. Омонимичные знаки появляются в результате
        ошибочного отождествления знака, входящего в состав погружаемого текста, со
        знаком, входящим в состав погружаемого текста, со знаком, входящим в состав
        текущего состояния базы знаний. Появление пар синонимичных знаков, один из
        которых входит в погружаемый текст, а второй --- в текущее состояние базы
        знаний, при погружении вводимого текста является штатным  противоречием,
        устранение которого осуществляется путем отождествления (склеивания )
        синонимичных знаков.}\scntext{примечание}{Сложность проблемы понимания вводимой
        вербальной информации заключается не только в сложности непротиворечивого
        погружения вводимой информации в текущее состояние базы знаний, но и в
        сложности трансляции этой информации с внешнего языка на внутренний язык
        кибернетической системы, т. е. в сложности генерации текста внутреннего языка,
        семантически эквивалентного вводимому тексту внешнего языка. Очевидно, что для
        естественных языков указанная трансляция является сложной задачей, так как в
        настоящее время проблема формализации синтаксиса и семантики естественных
        языков не решена.}
        
    \scnheader{семантическая совместимость кибернетической системы с партнерами}
    \scnidtf{уровень взаимопонимания кибернетической системой со своими партнерами}
    \scnidtf{степень конвергенции (близости) базы знаний кибернетической системы с базами знаний своих партнеров}
    
    \scnheader{семантическая совместимость кибернетической системы с партнерами}
    \scnrelto{частное свойство}{\textit{совместимость кибернетических систем}}
    \scntext{пояснение}{\textit{семантическая совместимость кибернетических
            систем} определяется
        \begin{scnitemize}

            \item количеством знаков, которые хранятся в памяти одной заданной
            кибернетической системы и денотационная семантика которых совпадает с
            денотационной семантикой знаков, хранимых в памяти другой заданной
            кибернетической системы (другими словами, это количество сущностей, которые
            описывают как в памяти первой кибернетической системы, так и в памяти второй
            кибернетической системы),
            \item тем, согласованы ли между двумя заданными кибернетическими системами факт
            совпадения денотационной семантики указанных выше знаков сущностей, описываемых
            в памяти как первой, так и второй кибернетической системы (такое согласование
            осуществляется путем согласования уникальных внешних идентификаторов (имен),
            которые приписываются указанным знакам сущностей и которые используются
            указанными кибернетическими системами при обмене сообщениями между ними).
        \end{scnitemize}
    }\scntext{примечание}{Прежде всего семантическая совместимость двух заданных
        кибернетических систем определяется согласованностью систем понятий,
        используемых обеими взаимодействующими кибернетическими системами, (т.е.
        совпадением семантической трактовки всех этих понятий) и включением в число
        таких общих понятий всех или почти всех неопределяемых понятий, а также тех
        определяемых понятий, которые обеими кибернетическими системами часто
        используются при определении остальных определяемых
        понятий.}\scntext{примечание}{Высокий уровень семантической совместимости даже для
        кибернетических систем с высоким уровнем интеллекта (например, для людей)
        встречается значительно реже, чем хотелось бы. Очевидно, что проблема
        обеспечения перманентной поддержки семантической совместимости
        взаимодействующих кибернетических систем является необходимым условием
        обеспечения высокого уровня взаимопонимания кибернетических систем и, как
        следствие, эффективного их взаимодействия.}
        
    \scnheader{способность кибернетической системы к обеспечению семантической совместимости с партнерами}
    \scnidtf{способность кибернетической системы к обеспечению взаимопонимания со своими партнерами.}
    \begin{scnrelfromset}{комплекс частных свойств}
        \scnitem{способность кибернетической системы к обеспечению семантической
            совместимости собственной базы знаний с базами знаний своих партнеров}
        \scnitem{способность кибернетической системы к обеспечению коммуникационной совместимости со своими партнерами}
            \begin{scnindent}
                \scntext{примечание}{Речь идет о согласовании внешних языков, используемых кибернетическими системами при их общении.}
            \end{scnindent}
    \end{scnrelfromset}
    \begin{scnrelfromset}{комплекс частных свойств}
        \scnitem{уровень предварительной семантической совместимости кибернетической системы с партнерами}
            \begin{scnindent}
                \scntext{примечание}{Речь идет об обеспечении начальной (стартовой) семантической совместимости.}
            \end{scnindent}
        \scnitem{способность кибернетической системы к перманентной поддержке
            семантической совместимости с партнерами}
        \begin{scnindent}
            \scntext{примечание}{Речь идет о перманентном процессе поддержки
                необходимого уровня семантической совместимости(взаимопонимания) в условиях
                постоянной эволюции всех взаимодействующих кибернетических систем.}
        \end{scnindent}
    \end{scnrelfromset}

    \scnheader{уровень предварительной семантической совместимости кибернетической
        системы с партнерами}
    \scnidtf{унификация представления информации, хранимой в памяти всевозможных кибернетических систем}
    \scnidtf{максимально возможная конвергенция, стандартизация, согласованность
        представления информации, хранимой в памяти всевозможных кибернетических систем}
    \scntext{примечание}{речь идет об использовании всеми кибернетическими системами
        общего универсального языка внутреннего представления знаний и о согласовании используемых ими понятий}
        
    \scnheader{способность кибернетической системы к перманентной поддержке семантической совместимости с партнерами}
    \scnidtf{способность кибернетической системы к согласованию денотационной
        семантики знаков (и, в первую очередь, знаков понятий), используемых в
        собственной базе знаний с денотационной семантике тех знаков, которые входят в
        состав информации поступающей от других кибернетических систем-партнеров}
    \scnidtf{способность кибернетической системы к повышению уровня семантической
        совместимости и взаимопонимания с другими системами (в том числе, с
        компьютерными системами, с людьми) в условиях перманентного процесса
        собственной эволюции (следствием которой является появление новых знаковых
        понятий и других описываемых сущностей, а также уточнение денотационной
        семантики используемых знаков), перманентной эволюции партнерских
        кибернетических систем и перманентной эволюции коллективно согласованной
        картины мира}
    \scntext{примечание}{Рассматриваемое свойство (способность) кибернетической системы
        заключается в \uline{самостоятельной} реализацией перманентного (постоянного)
        процесса обеспечения поддержки своей семантической совместимости \uline{со
            всеми}(!) кибернетическими системами, с которыми данная кибернетическая система
        взаимодействует в текущий момент времени. Подчеркнем при этом, что условия
        поддержки семантической совместимости постоянно меняются --- меняется состав
        партнеров , меняются (эволюционируют) сами партнеры , эволюционирует и сама
        данная кибернетическая система}
        
    \scnheader{следует отличать*}
    \begin{scnhaselementset}
        \scnitem{cпособность кибернетической системы к обеспечению семантической
            совместимости с партнерами}
            \begin{scnindent}
                \scniselement{свойство}
                \scnrelfrom{область определения}{кибернетическая система}
            \end{scnindent}
        \scnitem{cемантическая совместимость кибернетической системы с партнерами}
            \begin{scnindent}    
                \scniselement{свойство}
                \scnrelfrom{область определения}{множество всевозможных неориентированных пар кибернетических систем*}
                    \begin{scnindent}
                        \scnidtf{множество всевозможных сочетаний кибернетических систем по две*}
                        \scnidtf{множество всевозможных двухмощных множеств кибернетических систем*}
                    \end{scnindent}
                \scnidtf{степень (уровень) семантической совместимости различных пар кибернетических систем}
            \end{scnindent}
    \end{scnhaselementset}

    \scnheader{коммуникабельность кибернетической системы}
    \scnidtftext{часто используемый sc-идентификатор}{коммуникабельность}
    \scnidtf{способность кибернетической системы к установлению взаимовыгодных
        контактов с другими кибернетическими системами (в том числе, с коллективами
        интеллектуальных систем) путем честного выявления взаимовыгодных общих целей
        (интересов).}
    \scnidtf{способность кибернетической системы к формированию новых партнерских
        связей с другими кибернетическими системами}
    
    \scnheader{способность кибернетической системы к обсуждению и согласованию
        целей и планов коллективной деятельности}
    \scnidtf{способность активно участвовать в коллективном (в согласовании
        каких-либо предложений) --- т.е. в подтверждении (признании) этих предложений,
        либо в их отклонении с указанием причин или предлагаемых доработок}
    
    \scnheader{способность кибернетической системы брать на себя выполнение
        актуальных задач в рамках согласованных планов коллективной деятельности}
    \scntext{примечание}{Данная способность кибернетической системы предполагает:
        \begin{scnitemize}
            \item учет приоритета актуальных задач;
            \item учет собственных возможностей;
            \item согласование распределения актуальных задач по исполнителям;
            \item публикацию момента начала и предполагаемого момента завершения выполнения
            указанной актуальной задачи
        \end{scnitemize}
    }
    
    \scnheader{социальная ответственность кибернетической системы}
    \begin{scnrelfromlist}{свойство-предпосылка}
        \scnitem{способность кибернетической системы выполнять качественно и в срок
            взятые на себя обязательства в рамках соответствующих коллективов}
        \scnitem{ способность кибернетической системы адекватно оценивать свои
            возможности при распределении коллективной деятельности}
        \scnitem{ альтруизм/эгоизм кибернетической системы}
        \scnitem{ отсутствие/наличие действий, которые по безграмотности
            кибернетической системы снижают качество коллективов, в состав которых она
            входит}
        \scnitem{ отсутствие/наличие осознанных , мотивированных действий, снижающих
            качество коллективов, в состав которых кибернетическая система входит}
    \end{scnrelfromlist}

    \scnheader{альтруизм/эгоизм кибернетической системы}
    \scntext{примечание}{уровень мотивации к повышеннию качества коллективов, в состав
        которых кибернетическая система входит}\scntext{эпиграф}{Надо любить науку, а
        не себя в науке.}
    \scntext{эпиграф}{Ты играешь и всем своим видом показываешь: \scnqqi{Смотрите, как я
        красиво играю}, а надо играть и показывать красоту самой музыки.}
    
    \scnheader{социальная активность кибернетической системы}
    \scnidtftext{часто используемый sc-идентификатор*}{социальная активность}
    \scnidtf{пассионарность}
    \begin{scnrelfromlist}{свойство-предпосылка}
        \scnitem{способность кибернетической системы к генерации предлагаемых целей и
            планов коллективной деятельности}
        \scnitem{активность кибернетической системы в экспертизе результатов других
            участников коллективной деятельности}
        \scnitem{способность кибернетической системы к анализу качества всех
            коллективов, в состав которых она входит, а также всех членов этих коллективов}
        \scnitem{способность кибернетической системы к участию в формировании новых
            коллективов}
        \scnitem{количество и качество тех коллективов, в состав которых
            кибернетическая система входит или входила}
    \end{scnrelfromlist}

    \scnheader{способность кибернетической системы к участию в формировании коллективов}
    \scnidtf{уровень способности в создании таких коллективов кибернетических
        систем, в состав которых входит данная кибернетическая система и которые
        направлены на коллективное решение соответствующего актуального класса сложных
        комплексных задач, с каждой из которых не может справиться любая из имеющихся
        кибернетических систем.}
    \scntext{примечание}{Формирование специализированного коллектива кибернетических
        систем сводится к тому, что в памяти каждой кибернетической системы, входящей в
        коллектив, генерируется спецификация этого коллектива, включающая в себя:
        \begin{scnitemize}
            \item перечень весь членов коллектива;
            \item способности (возможности) каждого из них;
            \item обязанности в рамках коллектива;
            \item спецификацию всего множества задач (вида деятельности), для решения
            (выполнения) которых сформирован данный коллектив кибернетических систем
        \end{scnitemize}
    }\scntext{примечание}{Каждая кибернетическая система может входить в состав большого
        количества коллективов, выполняя при этом в разных коллективах в общем случае
        разные должностные обязанности , разные
        бизнес-процессы...}\scntext{примечание}{Рассмотренный принцип формирования
        специализированного коллектива, состоящего из компьютерных систем и людей,
        фактически означает автоматизацию системной интеграции компьютерных систем и
        децентрализованный (горизонтальный) характер такой интеграции, это очевидно
        предполагает наличие достаточно высокого уровня интеллекта у интегрируемых
        компьютерных систем и людей.}
    \scnheader{количество и качество тех коллективов, в состав которых кибернетическая система входит или входила}
    \scntext{пояснение}{Данная характеристика кибернетической системы уточняет
        спектр ее социальной активности}\scntext{примечание}{Чем умнее (интеллектуальнее)
        многоагентные системы, членом которых является данная кибернетическая система,
        тем выше ее социальный статус  и перспективы быть умнее --- есть у кого
        учиться}\bigskip
\end{scnsubstruct}
\scnsourcecomment{Завершили Сегмент \scnqqi{Комплекс свойств, определяющих уровень социализации кибернетической системы как фактора существенного повышения уровня ее обучаемости, а также фактора существенного повышения качества всех тех многоагентных систем, в состав которых входит данная кибернетическая система}}

		\scnsegmentheader{Направления эволюции компьютерных систем}

\begin{scnsubstruct}
    \scntext{эпиграф}{From data science to knowledge science.}

    \scnheader{эволюция компьютерных систем}
    \begin{scnsubdividing}
        \scnitem{первое направление эволюции компьютерных систем}
        \scnitem{второе направление эволюции компьютерных систем}
    \end{scnsubdividing}

    \scnheader{первое направление эволюции компьютерных систем}
    \scntext{примечание}{Первое направление эволючии включает в себя следующее:
    \begin{scnitemize}
        \item{расширение множества и многообразия задач, решаемых компьютерной системой}
        \item{повышение сложности этих задач вплоть до трудно формализуемых (трудно решаемых) задач, интеллектуальных задач, решаемых в условиях неполноты, неточности, нечеткости и так далее}
        \item{повышение качества решения задач либо путем более эффективного использования известных моделей решения задач (например, путем разработки более качественных алгоритмов), либо путем использования принципиально новых моделей решения задач}
        \item{расширение многообразия используемых видов информации (знаний)}
        \item{расширение многообразия используемых моделей решения задач}
    \end{scnitemize}}

    \scntext{примечание}{Очевидно, что расширение множества решаемых задач в условиях пусть и большой, но всегда конечной памяти компьютерной системы делает все более и более актуальным переход от частных методов и моделей решения задач к их обобщениям (или, как отмечал Д. А. Поспелов, от связки \scnqq{ключей} к набору \scnqq{отмычек})}

    \scntext{примечание}{Очевидно также, что многообразие видов задач, решаемых компьютерными системами, многообразие используемых моделей решения задач приводит:
    \begin{scnitemize}
        \item{к интегрированным информационным ресурсам}
        \item{к интегрированным решателям задач}
        \item{к интегрированным компьютерным системам}
        \item{к коллективам компьютерных систем}
    \end{scnitemize}}

    \scntext{примечание}{Проблема здесь заключается не в самой интеграции, а в ее качестве. Интеграция может быть эклектичной, если не обеспечить совместимость интегрируемых компонентов, а в случае такой совместимости интеграция может привести к новому качеству, к дополнительному расширению множества решаемых задач. Это будет означать переход от эклектичности к гибридности, синергетичности.}

    \scnheader{второе направление эволюции компьютерных систем}
    \scnidtf{повышение уровня обучаемости компьютерных систем и, как следствие, темпов их эволюции}

    \scnheader{обучаемость компьютерной системы}
    \begin{scnrelfromset}{определяется}
        \scnitem{трудоемкость и темпы расширения и совершенствования знаний и навыков компьютерной системы}
        \scnitem{уровень ограничений, накладываемых на вид приобретаемых и используемых знаний и навыков}
        \begin{scnindent}
            \scnidtf{ограничения на множество всех тех задач, которые принципиально могут быть решены данной компьютерной системой}
        \end{scnindent}
    \end{scnrelfromset}

    \scnheader{трудоемкость и темпы расширения и совершенствования знаний и навыков компьютерной системы}
    \begin{scnrelfromset}{определяется}
        \scnitem {гибкость}
        \begin{scnindent}
            \scnidtf{многообразие и трудоемкость возможных изменений, вносимых в систему в процессе пополнения системы новыми знаниями и навыками и совершенствования уже приобретенных знаний и навыков}
        \end{scnindent}
        \scnitem{стратифицированность}
        \begin{scnindent}
            \scnidtf{четкое разделение системы на достаточно независящие друг от друга уровни иерархии, то есть возможность локализации фрагментов компьютерной системы, не выходя за пределы которых, априори достаточно проводить анализ последствий тех или иных вносимых в систему изменений}
        \end{scnindent}
        \scnitem{рефлексивность} 
        \begin{scnindent}
            \scnidtf{способность анализировать собственное состояние и свою деятельность}
        \end{scnindent}
        \scnitem{гибридность} 
        \begin{scnindent}
            \scnidtf{способность приобретать и использовать широкое (а в идеале — неограниченное) многообразие знаний и навыков}
        \end{scnindent}
        \scnitem{уровень самообучаемости}
        \begin{scnindent}
            \scnidtf{уровень активности, самостоятельности, целеустремленности в процессе своего обучения, то есть уровень способности к обучению без учителя, уровень автоматизации приобретения новых знаний и навыков, а также совершенствование уже приобретенных знаний и навыков}
        \end{scnindent}
        \scnitem{совместимость} 
        \begin{scnindent}
            \scnidtf{трудоемкость интеграции}
        \end{scnindent}
        \scnitem{способность к постоянному мониторингу и поддержке своей совместимости} 
        \begin{scnindent}
            \scntext{примечание}{поддержка совместимести как с другими компьютерными системами, так и со своими пользователями в условиях интенсивной эволюции этих компьютерных систем и их пользователей}
        \end{scnindent}
    \end{scnrelfromset}

    \scnheader{совместимость компьютерных систем}
    \begin{scnrelfromset}{аспекты}
        \scnitem{глубокая интеграция компьютерных систем} 
        \begin{scnindent}
            \scnidtf{преобразование нескольких компьютерных систем в одну целостную компьютерную систему путем объединения информационных и функциональных ресурсов интегрируемых компьютерных систем}
        \end{scnindent}
        \scnitem{преобразование нескольких компьютерных систем в коллектив взаимодействующих компьютерных систем, способных к совместному корпоративному решению сложных задач}
    \end{scnrelfromset}
    \begin{scnrelfromset}{определяется}
        \scnitem{совместимость различного вида информации (знаний), хранимой в памяти компьютерной системы}
        \scnitem{совместимость различных моделей решения задач}
        \scnitem{совместимость встроенных (в том числе типовых) подсистем, входящих в состав компьютерных систем}
        \scnitem{совместимость внешней информации, поступающей на вход компьютерной системе, с информацией, хранимой в памяти компьютерной системы (трудоемкость понимания внешней информации — трансляция, погружение, выравнивание понятий)}
        \scnitem{коммуникационная (в том числе семантическая) совместимость с пользователями и с другими компьютерными системами}
    \end{scnrelfromset}

    \scnheader{обучение компьютерных систем}
    \scntext{примечание}{Важнейшая форма обучения компьютерной системы это приобретение новых знаний и навыков в \scnqq{готовом} виде, то есть в виде некоторых знаковых конструкций, вводимых в память компьютерной системы, поскольку приобретение знаний и навыков из внешних достоверных источников требует существенно меньшего времени по сравнению с их приобретением собственными силами, на основе собственного опыта и собственных ошибок. Но для того, чтобы указанная форма обучения была эффективной, необходимо максимально возможным образом упростить и формализовать механизм (процедуру) погружения новых знаний в память компьютерной системы. Для решения этой задачи ключевое значение имеет создание удобного для этой цели способа кодирования различного вида информации в памяти компьютерной системы.}
    \scntext{примечание}{Поскольку основным каналом обучения компьютерных систем является приобретение ими знаний и навыков от других субъектов — от других компьютерных систем и от пользователей (от разработчиков-учителей и от конечных пользователей), важнейшим фактором обучаемости компьютерной системы является превращение компьютерной системы в коммуникативную систему, способную эффективно общаться с внешними субъектами. Следовательно, уровнь обучаемости компьютерных систем определяется также уровнем ее совместимости с самими этими внешними субъектами, с приобретаемыми ею знаниями и навыками, то есть степенью того, как компьютерная система вместе с теми субъектами, с которыми она обменивается информацией, решает проблему \scnqq{вавилонского столпотворения}.}

    \scnheader{эволюция компьютерных систем}
    \scntext{примечание}{Таким образом, этапы эволюции традиционных компьютерных систем, в основе которых лежит их интерпретация на машинах фон Неймана, направлены на повышение качества этих систем и, в частности, на повышение уровня их интеллекта.}
    \scntext{примечание}{В качестве примера рассмотрим эволюцию языков программирования компьютерных систем:
    \begin{scnitemize}
        \item{Исходная особенность языков программирования заключается в том, что язык представления обрабатываемых программами данных (его синтаксис и денотационная семантика) не задается и фактически для любой программы или для семейства программ разрабатывается свой такой язык. (Языки программирования \scnqq{хромают} на одну ногу.)}
        \item{Данные преобразуются в базы данных, которые становятся общими для программ заданного языка программирования и изменение которых не может быть обусловлено и предусмотрено каждой из этих программ. Такие языки становятся языками программирования, ориентированными на обработку баз данных, а базам данных ставится в соответствие общий язык представления баз данных (с соответствующим синтаксисом и денотационной семантикой)}
        \item{Разные языки программирования (с разной денотационной и операционной семантикой) ориентируются на обработку баз данных, которым соответствует один и тот же язык представления баз данных (т.е. языки становятся совместимыми по обрабатываемым данными).}
        \item{Языки представления баз данных становятся универсальными и \scnqq{превращаются} в универсальные языки представления баз знаний (заметим, что продукционные и фреймовые языки представления знаний не являются универсальными)}
        \item{Разные языки программирования, ориентированные на обработку баз знаний, становятся подъязыками универсального языка представления баз знаний, т.е. становятся совместимыми не только по обрабатываемым базам знаний, но также и по своему синтаксису.}
        \item{Расширяется многообразие языков программирования, реализующих различные модели решателей задач:
            \begin{scnitemizeii}
                \item{алгоритмические языки программирования низкого и высокого уровня}
                \item{последовательные и параллельные процедурные языки программирования (синхронные и асинхронные)}
                \item{функциональные языки программирования}
                \item{логические языки программирования}
                \item{продукционные языки программирования}
                \item{объектно-ориентированные языки программирования}
                \item{генетические алгоритмы}
            \end{scnitemizeii}
        }
        \item{Создаются языки семантической спецификации программ, языки формулировки задач и стратегии поиска пути решения задач на основе заданного пакета программ различных языков программирования}
    \end{scnitemize}
    }
    \scntext{примечание}{Эволюция языков программирования подробнее рассматривается в работах Ершова А.П., Капитоновой Ю.В., Летичевского А.А., Непейводы Н.Н., Мак-Карти Дж. (язык LISP), Ковальского Р. (язык Рrolog) и других.}
\end{scnsubstruct}
		\scnsegmentheader{Итоговый сегмент Раздела Предметная область и онтология
    кибернетических систем}
\begin{scnsubstruct}
    \scnheader{качество кибернетической системы}
    \scntext{резюме}{}
    \bigskip
\end{scnsubstruct}
  %  \scnsourcecomment{Завершили Раздел \scnqqi{Предметная область и онтология кибернетических систем}}


		\bigskip
	\end{scnsubstruct}
\end{SCn}

\scsubsection[\protect\scnmonographychapter{Глава 1.1. Факторы, определяющие уровень интеллектуальности кибернетических систем. Этапы эволюции компьютерных систем в направлении повышения уровня их интеллектуальности}]{Предметная область и онтология многоагентных систем}
\label{sd_mas}

\scsubsection[
    \protect\scnidtf{История эволюции \textit{компьютерных систем} и, в том числе, \textit{интеллектуальных компьютерных систем}. Недостатки современных \textit{компьютерных систем} и интеллектуальных \textit{компьютерных систем}}
    \protect\scneditor{Сердюков Р.Е.}
    \protect\scnmonographychapter{Глава 1.1. Факторы, определяющие уровень интеллектуальности кибернетических систем. Этапы эволюции компьютерных систем в направлении повышения уровня их интеллектуальности}
    ]{Предметная область и онтология компьютерных систем}
\label{sd_comp_sys}

\scsection[
    \protect\scneditor{Шункевич Д.В.}
    \protect\scnmonographychapter{Глава 1.2. Интеллектуальные компьютерные системы нового поколения}
    ]{Предметная область и онтология интеллектуальных компьютерных систем нового поколения}
\label{sem_mod_comp_sys}
\begin{SCn}
	\scnsectionheader{Предметная область и онтология интеллектуальных компьютерных систем нового поколения}
	\begin{scnsubstruct}

		\begin{scnrelfromlist}{дочерний раздел}
			\scnitem{\nameref{sd_sem_inf_rep}}
			\scnitem{\nameref{sd_agent_solvers}}
			\scnitem{\nameref{sd_sem_ui}}
		\end{scnrelfromlist}

		\scntext{аннотация}{Данный раздел и дочерние ему разделы являются
			уточнением и обоснованием наших предложений, направленных на построение
			компьютерных систем следующего поколения, основанных на смысловом представлении
			обрабатываемой информации.}
		\scntext{основной тезис}{Для \uline{любой} \textit{компьютерной
				системы} можно построить эквивалентную ей логико-семантическую модель,
			основанную на смысловом представлении обрабатываемой информации}

		\scnheader{логико-семантическая модель компьютерной системы}
		\scntext{пояснение}{Главным фактором обеспечения совместимости
			различных видов знаний, различных моделей решения задач и различных
			компьютерных систем в целом является
			\begin{scnitemize}
				\item унификация (стандартизация) представления информации в памяти
				компьютерных систем;
				\item унификация принципов организации обработки информации в памяти
				компьютерных систем.
			\end{scnitemize}
			Унификация представления информации, используемой в компьютерных
			системах, предполагает:
			\begin{scnitemize}
				\item синтаксическую унификацию используемой информации  унификацию
				формы представления (кодирования) этой информации. При этом следует отличать:
				\begin{scnitemizeii}
					\item кодирование информации в памяти компьютерной системы (внутреннее
					представление информации);
					\item внешнее представление информации, обеспечивающее однозначность
					интерпретации (понимания, трактовки) этой информации разными пользователями и
					разными компьютерными системами;
				\end{scnitemizeii}
				\item семантическую унификацию используемой информации, в основе
				которой лежит согласование и точная спецификация всех (!) используемых понятий
				(концептов) с помощью иерархической системы формальных онтологий.
			\end{scnitemize}}

		\scnheader{стандарт}
		\scnhaselement{Стандарт OSTIS}
		\begin{scnindent}
			\scnidtf{Предлагаемый нами стандарт логико-семантических моделей
				компьютерных систем,  основанных на смысловом представлении информации, и
				технологии разработки таких моделей и соответствующих компьютерных систем}
		\end{scnindent}
		\scnidtf{знания о структуре и принципах функционирования искусственных
			систем соответствующего класса}
		\scnidtf{онтология искусственных систем некоторого класса}
		\scnidtf{теория искусственных систем некоторого класса}
		\scntext{пояснение}{Важно отметить, что грамотная унификация
			(стандартизация) должна не ограничивать творческую свободу разработчика, а
			гарантировать \uline{совместимость} его результатов с результатами других
			разработчиков. Подчеркнем также, что текущая версия любого \textit{стандарта}
			-- это не догма, а только опора для дальнейшего его совершенствования.Целью
			качественного \textit{стандарта} является не только обеспечения совместимости
			технических решений, но и минимизация дублирования (повторения) таких решений.
			Один из важных критериев качества \textit{стандарта} --- ничего
			лишнего.\textit{Стандарты}, как и другие важные для человечества
			\textit{знания}, должны быть формализованы и должны постоянно
			совершенствоваться с помощью специальных \textit{интеллектуальных компьютерных
				систем}, поддерживающих процесс эволюции стандартов путем согласования
			различных точек зрения.}
			
		\scnheader{семантическая совместимость компьютерных систем}
		\scntext{пояснение}{Уровень совместимости \textit{компьютерных
				систем} определяется трудоемкостью реализации процедур интеграции (объединения,
			соединения знаний этих систем), а также трудоемкостью и глубиной интеграции
			входящих в эти системы \textit{решателей задач} (интеграции навыков и
			интерпретаторов этих навыков). Подчеркнем при этом, что интеграция интеграции
			рознь --- от эклектики до гибридности и синергетичности дистанция огромного
			размера.
			
			Совместимые \textit{компьютерные системы} --- это компьютерные системы,
			для которых существует автоматически выполняемая процедура их интеграции,
			превращающая эти системы в единую \textit{гибридную систему}, в рамках которой
			каждая интегрируемая компьютерная система в процессе своего функционирования
			может свободно использовать любые необходимые информационные ресурсы (знания и
			навыки), входящие в состав другой интегрируемой компьютерной системы.
			
			Целостную \textit{компьютерную систему} можно рассматривать как решатель задач,
			интегрировавший несколько моделей решения задач и обладающий средствами
			взаимодействия с внешней для себя средой (с другими компьютерными системами, с
			пользователями).
			
			Таким образом, для того, чтобы повысить уровень совместимости
			\textit{компьютерных систем}, необходимо преобразовать их к виду
			\textit{многоагентных систем}, работающих над общей семантической памятью.
			Такие \textit{компьютерные системы} не всегда целесообразно непосредственно
			объединять (интегрировать) в более крупные \textit{компьютерные системы}.
			Иногда целесообразнее их объединять в \textit{коллективы взаимодействующих
			компьютерных систем}. Но при создании таких коллективов компьютерных систем
			унификация и совместимость таких систем также очень важны, т.к. существенно
			упрощают обеспечение высокого уровня их взаимопонимания. Так, например,
			противоречия между компьютерными системами, входящими в коллектив, можно
			обнаруживать путем анализа на непротиворечивость \textit{виртуальной
			объединенной базы знаний} этого коллектива. Более того, непротиворечивость
			указанной виртуальной базы знаний можно считать одним из критериев
			семантической совместимости систем, входящих в соответствующий
			коллектив.}
			
		\scnheader{компьютерная система, основанная на смысловом представлении информации}
		\scntext{пояснение}{Предлагаемое нами устранение проблем современных
			информационных технологий путем перехода к \textit{смысловому представлению
				информации} в памяти компьютерных систем фактически преобразует современные
			компьютерные системы (в том числе и современные интеллектуальные компьютерные
			системы) в \textit{компьютерные системы, основанные на смысловом представлении
				информации}, которые являются не альтернативной ветвью развития
			\textit{компьютерных систем}, а естественным этапом их эволюции, направленным
			на обеспечение высокого уровня их \textit{обучаемости} и, в первую очередь,
			\textit{совместимости}.
			
			Архитектура \textit{компьютерных систем, основанных на
			смысловом представлении информации} (см. \textit{Рис. Архитектура компьютерных
			систем, основанных на смысловом представлении информации}) практически
			совпадает с архитектурой \textit{интеллектуальных компьютерных систем},
			основанных на знаниях. Отличие здесь заключаются в том, что в
			\textit{компьютерных системах, основанных на смысловом представлении
			информации}:
			\begin{scnitemize}
				\item база знаний имеет смысловое представление;
				\item интерпретатор знаний и навыков представляет собой коллектив
				\textit{агентов}, осуществляющих обработку \textit{базы знаний}.
			\end{scnitemize}
			Как следствие этого, \textit{компьютерные системы, основанная на
			смысловом представлении информации}, обладают высоким уровнем
			\textit{обучаемости}, т.е. способностью быстро приобретать новые и
			совершенствовать уже приобретенные знания и навыки и при этом не иметь никаких
			ограничений на вид приобретаемых и совершенствуемых ею знаний и навыков, а
			также на их совместное использование.
			
			Более того, при согласовании соответствующих стандартов, а также при перманентном совершенствовании этих
			стандартов и при грамотной их поддержке в условиях интенсивной эволюции как
			самих стандартов, так и \textit{компьютерных систем, основанных на смысловом
			представлении информации} (речь идет о постоянной поддержке соответствия между
			текущим состоянием компьютерных систем и текущим состоянием эволюционируемых
			стандартов), \textit{компьютерные системы, основанные на смысловом
			представлении информации} и их компоненты обладают весьма высокой степенью
			\textit{совместимости}.
			
			Это, в свою очередь, практически исключает дублирование
			инженерных решений и дает возможность существенно ускорить разработку
			\textit{компьютерных систем, основанных на смысловом представлении информации}
			с помощью постоянно расширяемой библиотеки многократно используемых и
			совместимых между собой компонентов. 
			
			Основным лейтмотивом перехода от современных компьютерных систем (в том числе интеллектуальных) к
			\textit{компьютерным системам, основанным на смысловом представлении
				информации}, хранимой в ее памяти, является создание \textbf{\textit{общей
					семантической теории компьютерных систем}}, включающей в себя:
			\begin{scnitemize}
				\item cемантическую теорию \textit{знаний} и \textit{баз знаний};
				\item семантическую теорию \textit{задач} и \textit{моделей решения
					задач};
				\item cемантическую теорию \textit{взаимодействия информационных
					процессов};
				\item cемантическую теорию пользовательских и, в том числе,
				естественно-языковых интерфейсов;
				\item cемантическую теорию невербальных (сенсорно-эффекторных)
				интерфейсов;
				\item теорию универсальных интерпретаторов \textit{логико-семантических
					моделей компьютерных систем} и, в частности, теорию семантических компьютеров.
			\end{scnitemize}
			Эпицентром следующего этапа развития информационных технологий является
			решение проблемы обеспечения \textbf{\textit{семантической совместимости}}
			\textit{компьютерных систем} и их компонентов. Для решения этой проблемы
			необходим
			\begin{scnitemize}
				\item переход от традиционных компьютерных систем и от современных
				интеллектуальных компьютерных систем к \textit{компьютерным системам,
					основанным на смысловом представлении информации};
				\item разработка \textit{стандарта компьютерных систем, основанных на
					смысловом представлении информации}.
			\end{scnitemize}
			Очевидно, что \textit{компьютерные системы, основанных на смысловом
				представлении информации} являются компьютерными системами нового поколения,
			устраняющие многие недостатки современных компьютерных систем. Но для массовой
			разработки таких систем необходима соответствующая технология, которая должна
			включать в себя
			\begin{scnitemize}
				\item теорию \textit{компьютерных систем, основанных на смысловом
					представлении информации} и комплекс всех стандартов, обеспечивающих
				совместимость разрабатываемых систем;
				\item методы и средства проектирования \textit{компьютерных систем,
					основанных на смысловом представлении информации};
				\item методы и средства перманентного совершенствования самой
				технологии.
			\end{scnitemize}}

		\scnheader{Рис. Архитектура компьютерных систем, \textit{основанных на
				смысловом представлении информации}}
		\scneqfile{\includegraphics{Contents/part_intro/src/images/arch.pdf}}
		\bigskip

		\scnsegmentheader{Предметная область и онтология требований, предъявляемых к интеллектуальным компьютерным системам нового поколения}

\begin{scnsubstruct}

    \begin{scnrelfromlist}{ключевое понятие}
    	\scnitem{интеллектуальная компьютерная система нового поколения}
    	\scnitem{интероперабельная интеллектуальная компьютерная система}
    	\scnitem{гибридная интеллектуальная компьютерная система}
    \end{scnrelfromlist}
    
    \begin{scnrelfromlist}{ключевое отношение}
    	\scnitem{соединение интеллектуальных компьютерных систем*}
    	\begin{scnindent}
    		\scnidtf{преобразование множества интеллектуальных компьютерных систем в коллектив, членами (агентами) которого являются эти системы*}
    	\end{scnindent}
    	\scnitem{глубокая интеграция интеллектуальных компьютерных систем*}
    	\begin{scnindent}
    		\scnidtf{быть результатом преобразования множества индивидуальных интеллектуальных компьютерных систем в одну интегрированную индивидуальную интеллектуальную компьютерную систему*}
    	\end{scnindent}
    \end{scnrelfromlist}
    
    \begin{scnrelfromlist}{ключевой параметр}
    	\scnitem{интероперабельность интеллектуальных компьютерных систем\scnsupergroupsign}
    	\scnitem{семантическая совместимость пар интеллектуальных компьютерных систем\scnsupergroupsign}
    \end{scnrelfromlist}
    
    \begin{scnrelfromlist}{ключевое знание}
    	\scnitem{Требования, предъявляемые к интеллектуальным компьютерным системам нового поколения}
    	\scnitem{Принципы, лежащие в основе интеллектуальных компьютерных систем нового поколения}
    	\scnitem{Отличие данных от знаний}
    \end{scnrelfromlist}

    \scnheader{уровень интероперабельности интеллектуальных компьютерных систем}
    \scntext{примечание}{Создание различных комплексов взаимодействующих интеллектуальных компьютерных систем \uline{требует} повышения качества не только самих этих систем, но также и качества их взаимодействия. Интеллектуальные компьютерные системы нового поколения должны иметь высокий уровень интероперабельности.}
    \scnidtf{уровень коммуникационной (социальной) совместимости интеллектуальных компьютерных систем, позволяющей им самостоятельно формировать коллективы интеллектуальных компьютерных систем и их пользователей, а также самостоятельно согласовывать и координировать свою деятельность в рамках этих коллективов при решении сложных задач в частично предсказуемых условиях}
    \scnidtf{уровень способности к эффективному, целенаправленному взаимодействию с себе подобными и с пользователями в процессе коллективного (распределенного) и децентрализованного решения сложных задач}
    \begin{scnindent}
        \begin{scnrelfromset}{источник}
            \scnitem{\scncite{Yaghoobirafi2022}}
            \scnitem{\scncite{Ouksel1999}}
            \scnitem{\scncite{Lanzenberger2008}}
            \scnitem{\scncite{Neiva2016}}
            \scnitem{\scncite{Pohl2004}}
            \scnitem{\scncite{Waters2009}}
        \end{scnrelfromset}
    \end{scnindent}
    \scnidtf{уровень \scnqq{социализации} интеллектуальных компьютерных систем, полезности в рамках различных априори неизвестных сообществ (коллективов) \textit{интеллектуальных систем}}
    \scntext{примечание}{Повышение уровня \textit{интероперабельности} интеллектуальных компьютерных систем определяет переход к \textbf{\textit{интеллектуальным компьютерным системам нового поколения}}, без которых невозможна реализация таких проектов, как \textit{интеллектуальное-предприятие}, \textit{интеллектуальная-больница}, \textit{интеллектуальная-школа}, \textit{интеллектуальный-университет}, \textit{интеллектуальная-кафедра}, \textit{интеллектуальный-дом}, \textit{интеллектуальный-город}, \textit{интеллектуаль\-ное-общество}.}
        \begin{scnindent}
            \begin{scnrelfromset}{источник}
                \scnitem{\scncite{Lopes2022}}
                \scnitem{\scncite{Hamilton2006}}
            \end{scnrelfromset}
        \end{scnindent}
    
    \scnheader{интеллектуальная компьютерная система}
    \scnidtf{интеллектуальная искусственная кибернетическая система}
    \begin{scnrelfromset}{разбиение}
    	\scnitem{индивидуальная интеллектуальная компьютерная система}
    	\scnitem{интеллектуальный коллектив интеллектуальных компьютерных систем}
    	\begin{scnindent}
    		\scnidtf{интеллектуальная \textit{многоагентная система}, агенты которой являются \textit{интеллектуальными компьютерными системами}}
    		\scntext{примечание}{Не каждый \textit{коллектив интеллектуальных компьютерных систем} может оказаться интеллектуальным, поскольку уровень интеллекта такого коллектива определяется не только уровнем интеллекта его членов, но также и эффективностью (качеством) \uline{их взаимодействия}.}
    		\begin{scnrelfromset}{разбиение}
    			\scnitem{интеллектуальный коллектив \uline{индивидуальных} интеллектуальных компьютерных систем}
    			\scnitem{иерархический интеллектуальный коллектив интеллектуальных компьютерных систем}
    			\begin{scnindent}
    				\scnidtf{\textit{интеллектуальный коллектив интеллектуальных компьютерных систем}, по крайней мере одним из членов которого является \textit{интеллектуальный коллектив интеллектуальных компьютерных систем}}
    			\end{scnindent}
    		\end{scnrelfromset}
    	\end{scnindent}
    \end{scnrelfromset}
    
    \scnheader{интеллектуальные компьютерные системы нового поколения}
    \begin{scnrelfromlist}{предъявляемые требования}
    	\scnitem{высокий уровень \textit{интероперабельности}}
    	\scnitem{высокий уровень \textit{обучаемости}}
    	\scnitem{высокий уровень \textit{гибридности}}
    	\scnitem{высокий уровень способности решать \textit{интеллектуальные задачи}}
        \begin{scnindent}
            {\textit{задачи}, \textit{методы} решения которых и/или требуемая для их решения исходная информация априори неизвестны}
        \end{scnindent}
        \scnitem{высокий уровень \textit{синергетичности}}
    \end{scnrelfromlist}
    
    \scnheader{интероперабельность\scnsupergroupsign}
    \scnidtf{способность к эффективному (целенаправленному) взаимодействию с другими самостоятельными субъектами}
    \scnidtf{способность к партнерскому взаимодействию в решении \textit{комплексных задач}, требующих \textit{коллективной деятельности}}
    \scnidtf{способность работать в коллективе (в команде)}
    \scnidtf{уровень социализации}
    \scnidtf{social skills}
    
    \scnheader{высокий уровень интероперабельности}
    \begin{scnrelfromlist}{обеспечивается}
    	\scnfileitem{высоким уровнем \textit{взаимопонимания}}
    	\begin{scnindent}
    		\begin{scnrelfromlist}{обеспечивается}
    			\scnfileitem{высоким уровнем \textbf{\textit{семантической совместимости}} заданного субъекта с другими субъектами заданного коллектива}
    			\scnfileitem{высоким уровнем \textit{способности понимать} сообщения и поведение партнеров}
    			\scnfileitem{высоким уровнем \textit{способности быть понятной} для партнеров:}
                \begin{scnindent}
                    \begin{scnrelfromlist}{обеспечивается}
                        \scnfileitem{способностью понятно и обоснованно формулировать свои предложения и информацию, полезную для решения текущих задач}
    			        \scnfileitem{способностью действовать и комментировать свои действия так, чтобы они и их мотивы были понятны партнерам}
                \end{scnrelfromlist}
                \end{scnindent}
    			\scnfileitem{высоким уровнем \textit{способности к повышению уровня семантической совместимости} со своими партнерами}
    		\end{scnrelfromlist}
    	\end{scnindent}
    	\scnfileitem{высоким уровнем \textit{договороспособности}, то есть способности согласовывать с партнерами свои планы и намерения в целях своевременного обеспечения высокого качества коллективного результата}
    	\scnfileitem{высоким уровнем \textit{способности к децентрализованной координации} своих действий с действиями партнеров в непредсказуемых (нештатных) обстоятельствах}
    	\scnfileitem{высоким уровнем способности разделять ответственность с партнерами}
    	\scnfileitem{высоким уровнем \textit{способности к минимизации негативных последствий конфликтных ситуаций} с другими субъектами}
    	\begin{scnindent}
    		\begin{scnrelfromlist}{обеспечивается}
    			\scnfileitem{высоким уровнем \textit{способности к предотвращению возникновения конфликтных ситуаций}}
    			\scnfileitem{\textit{соблюдением этических норм} и правил, препятствующих возникновению разрушительных последствий конфликтных ситуаций}
    			\scnfileitem{высоким уровнем \textit{способности разделять ответственность} с партнерами за своевременное и качественное достижение общей цели}
    		\end{scnrelfromlist}
    	\end{scnindent}
    \end{scnrelfromlist}
    
    \scnheader{семантическая совместимость\scnsupergroupsign}
    \scnidtf{степень согласованности (совпадения) систем \textit{понятий} и других \textit{ключевых знаков}, используемых заданными взаимодействующими субъектами}
    \scntext{примечание}{Обеспечение \textit{семантической совместимости} требует формализации \textit{смыслового представления информации}.}
    
    \scnheader{способность разделять ответственность с партнёрами}
    \scnidtf{необходимое условие децентрализованного управления коллективной деятельностью}
    \begin{scnrelfromlist}{обеспечивается}
    	\scnfileitem{\textit{способностью к мониторингу} и анализу коллективно выполняемой деятельности}
    	\scnfileitem{\textit{способностью оперативно информировать партнеров} о неблагоприятных ситуациях, событиях, тенденциях, а также инициировать соответствующие коллективные действия}
    \end{scnrelfromlist}
    
    \scnheader{высокий уровень обучаемости интеллектуальной компьютерной системы нового поколения}
    \scnexplanation{Важнейшим направлением повышения уровня автоматизации человеческой деятельности является повышение уровня автоматизации не только проектирования интеллектуальной компьютерной системы, но и комплексной поддержки всех остальных этапов жизненного цикла \textit{интеллектуальной компьютерной системы}. В частности, это касается модернизации (совершенствования, реинжиниринга) интеллектуальной компьютерной системы непосредственно в ходе их эксплуатации. Для того, чтобы обеспечить высокий уровень автоматизации такой модернизации, необходимо существенно повысить \textbf{\textit{уровень самообучаемости}} \textit{интеллектуальной компьютерной системы} для того, что они сами (самостоятельно) могли себя модернизировать (самосовершенствовать) в ходе своего целевого функционирования.}
    
    \scnheader{высокий уровень обучаемости}
    \begin{scnrelfromlist}{обеспечивается}
    	\scnfileitem{высоким уровнем \textit{гибкости информации}, хранимой в памяти интеллектуальной системы}
    	\scnfileitem{высоким уровнем \textit{качества} \textit{стратификации информации}, хранимой в памяти интеллектуальной системы (стратифицированностью \textit{базы знаний})}
    	\scnfileitem{высоким уровнем \textit{рефлексивности} интеллектуальной системы}
    	\scnfileitem{высоким уровнем \textit{способности исправлять свои ошибки} (в том числе устранять противоречия в своей \textit{базе знаний})}
    	\scnfileitem{высоким уровнем \textit{познавательной активности}}
    	\scnfileitem{низким уровнем \textit{ограничений на вид приобретаемых знаний и навыков} (отсутствие таких ограничений означает потенциальную \textit{универсальность} интеллектуальной системы и предполагает высокий уровень ее гибридности)}
    \end{scnrelfromlist}
    
    \scnheader{обучаемость\scnsupergroupsign}
    \scnidtf{способность быстро и качественно приобретать новые \textit{знания} и \textit{навыки}, а также совершенствовать уже приобретенные \textit{знания} и \textit{навыки}}
    
    \scnheader{гибридность\scnsupergroupsign}
    \scnidtf{степень многообразия используемых \textit{видов знаний} и \textit{моделей решения задач} и уровень эффективности их совместного использования}
    \scnidtf{индивидуальная способность решать \textit{комплексные задачи}, требующие использования различных \textit{видов знаний}, а также различных комбинаций различных \textit{моделей решения задач}}
    \scntext{пояснение}{\textit{Гибридность} и \textit{интероперабельность} \textit{интеллектуальных компьютерных систем нового поколения} предполагает отказ от известной парадигмы \scnqq{черных ящиков}, поскольку:
    \begin{scnitemize}
        \item все многообразие моделей решения задач \textit{гибридной интеллектуальной компьютерной системы} должно интерпретироваться на одной общей \textit{универсальной платформе};
    	\item
    	доступность информации о том, как устроен каждый используемый метод, модель решения задач, каждый субъект существенно повышает качество их \textit{координации} при \textit{совместном решении комплексных задач};
    	\item
    	появляется возможность некоторые методы, модели решения задач и целые субъекты (например, \textit{интеллектуальные компьютерные системы}) использовать для совершенствования (повышения качества) других методов, моделей и субъектов.
    \end{scnitemize}}
    
    \scnheader{высокий уровень гибридности}
    \begin{scnrelfromlist}{обеспечивается}
    	\scnfileitem{высокой степенью многообразия используемых \textit{видов знаний} и \textit{моделей решения задач}}
    	\scnfileitem{высокой степенью \textit{конвергенции} и глубокой \textit{интеграции} (степенью взаимопроникновения) различных \textit{видов знаний} и \textit{моделей решения задач}}
    	\scnfileitem{способностью неограниченно расширять уровень своей \textit{гибридности}}
    \end{scnrelfromlist}
    
    \scnheader{характеристики \textit{интеллектуальных компьютерных систем нового поколения}}
    \begin{scnhassubset}
        \scnfileitem{\textbf{\textit{Степень}} \textbf{\textit{конвергенции}}, унификации и стандартизации \textit{интеллектуальных компьютерных систем} и их компонентов и соответствующая этому \textbf{\textit{степень интеграции}} (глубина интеграции) \textit{интеллектуальных компьютерных систем} и их компонентов.}
        \scnfileitem{\textbf{\textit{Семантическая совместимость}} между \textit{интеллектуальными компьютерными системами} в целом и \textit{семантическая совместимость} между компонентами каждой \textit{интеллектуальной компьютерной системы} (в частности, совместимость между различными \textit{видами знаний} и различными \textit{моделями обработки знаний}), которые являются основными показателями степени \textbf{\textit{конвергенции}} (сближения) между \textit{интеллектуальными компьютерными системами} и их компонентами.}
    \end{scnhassubset}
    \scntext{пояснение}{Особенность указанных характеристик \textit{интеллектуальных компьютерных систем} их компонентов заключается в том, что они играют важную роль при решении всех ключевых задач современного этапа развития \textit{Искусственного интеллекта} и тесно связаны друг с другом.}
    \scntext{пояснение}{Перечисленные требования, предъявляемые к \textit{интеллектуальным компьютерным системам нового поколения}, направлены на преодоление проклятия \textit{вавилонского столпотворения} как внутри \textit{интеллектуальных компьютерных систем нового поколения} (между внутренними \textit{информационными процессами} решения различных задач), так и между взаимодействующими самостоятельными \textit{интеллектуальными компьютерными системами нового поколения} в процессе коллективного решения \textit{комплексных задач}.}
    
    \scnheader{интеллектуальная компьютерная система нового поколения}
    \scntext{примечание}{На современном этапе эволюции \textit{интеллектуальных компьютерных систем} для существенного расширения областей их применения и качественного повышения уровня автоматизации человеческой деятельности:
        \begin{scnitemize}
            \item{необходим переход к созданию \uline{семантически совместимых} \textbf{интеллектуальных компьютерных систем \uline{нового поколения}}, ориентированных не только на индивидуальное, но и на \uline{коллективное} (совместное) решение \textit{комплексных задач}, требующих скоординированной деятельности нескольких самостоятельных интеллектуальных компьютерных систем и использования различных моделей и методов в непредсказуемых комбинациях, что необходимо для существенного расширения сфер применения \textit{интеллектуальных компьютерных систем}, для перехода от автоматизации локальных видов и областей \textit{человеческой деятельности} к комплексной автоматизации более крупных (объединенных) видов и областей этой деятельности;}
            \item{необходима разработка \textbf{Общей формальной теории и стандарта интеллектуальных компьютерных систем нового поколения};}
            \item{необходима разработка \textbf{Технологии комплексной поддержки жизненного цикла интеллектуальных компьютерных систем нового поколения}, которая включает в себя поддержку \textit{проектирования} этих систем (как начального этапа их жизненного цикла) и обеспечение их \textit{совместимости} на всех этапах их жизненного цикла;}
            \item{необходима \textbf{конвергенция} и \textbf{унификация} \textit{интеллектуальных компьютерных систем нового поколения} и их компонентов;}
            \item{необходима реализация \scnqq{бесшовной}, {диффузной}, взаимопроникающей, \textbf{глубокой интеграции семантически смежных компонентов интеллектуальных компьютерных систем}, то есть интеграции, при которой отсутствуют четкие границы (\scnqq{швы}) интегрируемых (соединяемых) компонентов, и которая может осуществляться \uline{автоматически}. Это означает переход к \textbf{\uline{гибридным} интеллектуальным компьютерным системам};}
            \item{необходимо соблюдение \textbf{Принципа бритвы Оккама} — максимально возможное структурное упрощение \textit{интеллектуальных компьютерных систем нового поколения}, исключение \uline{эклектичных} решений;}
            \item{необходима ориентация на потенциально \textbf{универсальные} (то есть способные быстро приобретать \uline{любые} знания и навыки), \textbf{синергетические} \textit{интеллектуальные компьютерные системы} с \scnqq{сильным} интеллектом}
        \end{scnitemize}}
    \begin{scnrelfromlist}{принципы, лежащие в основе}
        \scnfileitem{\textit{смысловое представление знаний} в памяти \textit{интеллектуальных компьютерных систем}, предполагающее отсутствие \textit{омонимических знаков}, которые в разных контекстах обозначают разные сущности, а также отсутствие \textit{синонимии}, то есть пар синонимичных \textit{знаков}, которые обозначают одну и ту же сущность}
        \scnfileitem{смысловое представление информационной конструкции в общем случае имеет нелинейный (графовый) характер представления информации, который является \textit{рафинированной семантической сетью}}
        \scnfileitem{фрактальный характер (масштабируемое самоподобие) структуризации представляемых знаний в базах знаний}
        \scnfileitem{использование \uline{общего} для всех интеллектуальных компьютерных систем \textit{универсального языка смыслового представления знаний} в памяти \textit{интеллектуальных компьютерных систем}, обладающий максимально простым \textit{синтаксисом}, обеспечивающий представление любых \textit{видов знаний} и имеющий неограниченные возможности перехода от \textit{знаний} к \textit{метазнаниям}. Простота синтаксиса \textit{информационных конструкций} указанного \textit{языка} позволяет называть эти конструкции \textit{рафинированными семантическими сетями}}
        \scnfileitem{\textit{структурно-перестраиваемая (графодинамическая) организация памяти} интеллектуальных компьютерных систем, при которой обработка знаний сводится не столько к изменению состояния хранимых \textit{знаков}, сколько к изменению конфигурации связей между этими \textit{знаками}}
        \scnfileitem{\textit{семантически неограниченный ассоциативный доступ к информации}, хранимой в памяти \textit{интеллектуальных компьютерных систем}, по заданному образцу произвольного размера и произвольной конфигурации}
        \scnfileitem{универсальная ситуационная многоагентная модель обработки знаний, ориентированная на обработку смыслового представления информации в ассоциативной графодинамической памяти, \textit{децентрализованное ситуационное управление информационными процессами} в памяти \textit{интеллектуальных компьютерных систем}, реализованное с помощью \textit{агентно-ориентированной модели обработки баз знаний}, в котором \textit{инициирование} новых \textit{информационных процессов} осуществляется не путем передачи управления соответствующим априори известным процедурам, а в результате возникновения соответствующих \textit{ситуаций} или \textit{событий} \textit{в памяти интеллектуальной компьютерной системы}, поскольку \scnqqi{основная проблема компьютерных систем состоит не в накоплении знаний, а в умении активизировать нужные знания в процессе решения задач} (Поспелов Д.~А.). Такой многоагентный процесс обработки информации представляет собой \textit{деятельность}, выполняемую некоторым коллективом \uline{самостоятельных} \textit{информационных агентов} (агентов обработки информации), условием инициирования каждого из которых является появление в текущем состоянии \textit{базы знаний} соответствующей этому агенту \textit{ситуации} и/или \textit{события}.
            \scnqqi{Выбор многоагентных технологий объясняется тем, что в настоящее время любая сложная производственная, логистическая или другая система может быть представлена набором взаимодействий более простых систем до любого уровня детальности, что обеспечивает фрактально-рекурсивный принцип построения многоярусных систем, построенных как открытые цифровые колонии и экосистемы ИИ. В основе многоагентных технологий лежит распределенный или децентрализованный подход к решению задач, при котором динамически обновляющаяся информация в распределенной сети интеллектуальных агентов обрабатывается непосредственно у агентов вместе с локально доступной информацией от \scnqq{соседей}. При этом существенно сокращаются как ресурсные и временные затраты на коммуникации в сети, так и время на обработку и принятие решений в центре системы (если он все-таки есть).}}
        \scnfileitem{агентно-ориентированная модель обработки знаний в памяти интеллектуальной компьютерной системы, обеспечивающая высокую степень \textit{интероперабельности} между внутренними агентами индивидуальной интеллектуальной компьютерной системы, взаимодействующими через общую память (это, фактически, \scnqq{внутренняя} интероперабельность интеллектуальной компьютерной системы нового поколения)}
        \scnfileitem{Переход к \textit{семантическим} \textit{моделям решения задач}, в основе которых лежит учет не только синтаксических (структурных) аспектов обрабатываемой информации, но также и \uline{семантических} (смысловых) аспектов этой информации - \scnqqi{From data science to knowledge science}}
        \scnfileitem{\textbf{\textit{онтологическая модель баз знаний}} \textit{интеллектуальных компьютерных систем}, то есть онтологическая структуризация всей информации, хранимой в памяти \textit{интеллектуальной компьютерной системы}, предполагающая четкую \textit{стратификацию базы знаний} в виде иерархической системы \textit{предметных областей} и соответствующих им \textit{онтологий}, каждая из которых обеспечивает семантическую \textit{спецификацию} всех \textit{понятий}, являющихся ключевыми в рамках соответствующей \textit{предметной области}}
        \scnfileitem{\textbf{\textit{онтологическая локализация решения задач}} в \textit{интеллектуальных компьютерных системах}, предполагающая \uline{локализацию} \textit{области действия} каждого хранимого в памяти \textit{метода} и каждого \textit{информационного агента} в соответствии с \textit{онтологической моделью} обрабатываемой \textit{базы знаний}. Чаще всего, такой \textit{областью действия} является одна из \textit{предметных областей} либо одна из \textit{предметных областей} вместе с соответствующей ей \textit{онтологии}}
        \scnfileitem{\textbf{\textit{онтологическая модель интерфейса}} \textit{интеллектуальной компьютерной системы}}
        \begin{scnindent}
            \begin{scnrelfromlist}{входить в состав}
                \scnfileitem{онтологическое описание \textit{синтаксиса} всех языков, используемых \textit{интеллектуальной компьютерной системой} для общения с \textit{внешними субъектами}}
                \scnfileitem{онтологическое описание \textit{денотационной семантики} каждого языка, используемого \textit{интеллектуальной компьютерной системой} для \textit{общения} с внешними \textit{субъектами}}
                \scnfileitem{семейство \textit{информационных агентов}, обеспечивающих \textit{синтаксический анализ}, \textit{семантический анализ} (перевод на внутренний смысловой язык) и \textit{понимание} (погружение в \textit{базу знаний}) любого введенного \textit{сообщения}, принадлежащего любому \textit{внешнему языку}, полное онтологическое описание которого находится в базе знаний \textit{интеллектуальной компьютерной системы}}
                \scnfileitem{семейство \textit{информационных агентов}, обеспечивающих \textit{синтез сообщений}, которые (1) адресуются внешним субъектам, с которыми общается \textit{интеллектуальная компьютерная система}, (2) \textit{семантически эквивалентны} заданным \textit{фрагментам базы знаний} интеллектуальной компьютерной системы, определяющим \textit{смысл} передаваемых \textit{сообщений}, (3) принадлежат одному из \textit{внешних языков}, полное онтологическое описание которого находится в \textit{базе знаний} интеллектуальной компьютерной системы}
            \end{scnrelfromlist}
        \end{scnindent}
        \scnfileitem{\textit{семантически дружественный характер пользовательского интерфейса}, обеспечиваемый (1) формальным описание в базе знаний средства управления пользовательским интерфейсом и (2) введением в состав \textit{интеллектуальной компьютерной системы} соответствующих help-подсистем, обеспечивающих существенное снижение языкового барьера между пользователями и \textit{интеллектуальными компьютерными системами}, что существенно повысит эффективность \textit{эксплуатации интеллектуальных компьютерных систем}}
        \scnfileitem{\textit{минимизация негативного влияния человеческого фактора} на эффективность \textit{эксплуатации} \textit{интеллектуальных компьютерных систем} благодаря реализации интероперабельного (партнерского) стиля взаимодействия не только между самими \textit{интеллектуальными компьютерными системами}, но также и между \textit{интеллектуальными компьютерными системами} и их пользователями. Ответственность за качество совместной деятельности должно быть распределено между всеми партнерами}
        \scnfileitem{\textbf{\textit{мультимодальность}} (гибридный характер) \textit{интеллектуальной компьютерной системы}}
        \begin{scnindent}
            \begin{scnrelfromlist}{предполагает}
                \scnfileitem{многообразие \textit{видов знаний}, входящих в состав \textit{базы знаний} интеллектуальной компьютерной системы}
                \scnfileitem{многообразие \textit{моделей решения задач}, используемых \textit{решателем задач} интеллектуальной компьютерной системы}
                \scnfileitem{многообразие \textit{сенсорных каналов}, обеспечивающих \textit{мониторинг} состояния \textit{внешней среды} интеллектуальной компьютерной системы}
                \scnfileitem{многообразие \textit{эффекторов}, осуществляющих \textit{воздействие на внешнюю среду}}
                \scnfileitem{многообразие \textit{языков общения} с другими субъектами (с пользователями, с интеллектуальными компьютерными системами)}
            \end{scnrelfromlist}
        \end{scnindent}
        \scnfileitem{\textbf{\textit{внутренняя семантическая совместимость}} между компонентами \textit{интеллектуальной компьютерной системы} (то есть максимально возможное введение общих, совпадающих \textit{понятий} для различных фрагментов хранимой \textit{базы знаний}), являющаяся формой \textbf{\textit{конвергенции}} и \textit{глубокой интеграции} внутри \textit{интеллектуальной компьютерной системы} для различного вида \textit{знаний} и различных \textit{моделей решения задач}, что обеспечивает эффективную реализацию \textit{мультимодальности интеллектуальной компьютерной системы}}
        \scnfileitem{\textbf{\textit{внешняя семантическая совместимость}} между различными \textit{интеллектуальными компьютерными системами}, выражающаяся не только в общности используемых \textit{понятий}, но и в общности базовых \textit{знаний} и являющаяся необходимым условием обеспечения высокого уровня \textit{интероперабельности} интеллектуальных компьютерных систем}
        \scnfileitem{ориентация на использование \textit{интеллектуальных компьютерных систем} как \textit{когнитивных агентов} в составе \textbf{\textit{иерархических многоагентных систем}}}
        \scnfileitem{фрактальный характер (масштабируемое самоподобие) структуризации иерархических коллективов интеллектуальных компьютерных систем нового поколения}
        \scnfileitem{\textbf{\textit{платформенная независимость} интеллектуальных компьютерных систем}}
        \begin{scnindent}
            \begin{scnrelfromlist}{предполагает}
                \scnfileitem{четкую \textit{стратификацию} каждой \textit{интеллектуальной компьютерной системы} (1) на \textit{логико-семантическую модель}, представленную ее \textit{базой знаний}, которая содержит не только \textit{декларативные знания}, но и знания, имеющие \textit{операционную семантику}, и (2) на \textit{платформу}, обеспечивающую \textit{интерпретацию} указанной \textit{логико-семантической модели}}
                \scnfileitem{универсальность указанной \textit{платформы} интерпретации \textit{логико-семантической модели интеллектуальной компьютерной системы}, что дает возможность каждой такой \textit{платформе} обеспечивать интерпретацию любой \textit{логико-семантической модели интеллектуальной компьютерной системы}, если эта модель представлена на том же \textit{универсальном языке смыслового представления информации}}
                \scnfileitem{многообразие вариантов реализации \textit{платформ интерпретации логико-семантических моделей интеллектуальных компьютерных систем} — как вариантов, программно реализуемых на \textit{современных компьютерах}, так и вариантов, реализуемых в виде \textit{универсальных компьютеров нового поколения}, ориентированных на использование в \textit{интеллектуальных компьютерных системах нового поколения} (такие компьютеры мы назвали \textit{ассоциативными семантическими компьютерами})}
                \scnfileitem{легко реализуемую возможность переноса (переустановки) логико-семантической модели (\textit{базы знаний}) любой \textit{интеллектуальной компьютерной системы} на любую другую \textit{платформу интерпретации логико-семантических моделей}}
            \end{scnrelfromlist}
        \end{scnindent}
        \scnfileitem{изначальная ориентация \textit{интеллектуальных компьютерных систем нового поколения} на использование \textbf{\textit{универсальных ассоциативных семантических компьютеров}} (компьютеров нового поколения) в качестве \textit{платформы интерпретации логико-семантических моделей} (баз знаний) \textit{интеллектуальных компьютерных систем}}
    \end{scnrelfromlist}
    \scntext{примечание}{В настоящее время разработано большое количество различного вида моделей решения задач, моделей представления и обработки знаний различного вида. Но в разных \textit{интеллектуальных компьютерных системах} могут быть востребованы разные комбинации этих моделей. При разработке и реализации различных \textit{интеллектуальных компьютерных систем} соответствующие методы и средства должны гарантировать \textit{логико-семантическую совместимость} разрабатываемых компонентов и, в частности, их способность использовать общие \textit{информационные ресурсы}. Для этого, очевидно, необходима \textit{унификация} указанных моделей.}
    \scntext{примечание}{\uline{Многообразие} различных видов интеллектуальных компьютерных систем и, соответственно, многообразие используемых ими комбинаций моделей представления знаний и решения задач определяется:
        \begin{scnitemize}
            \item{многообразием назначения интеллектуальных компьютерных систем и вида окружающей их среды;}
            \item{многообразием различных видов хранимых знаний;}
            \item{многообразием моделей обработки знаний и решений задач;}
            \item{многообразием различных видов сенсорных и эффекторных подсистем.}
        \end{scnitemize}}

    \scnheader{аспекты \textit{совместимости} моделей представления и обработки знаний в \textit{интеллектуальных компьютерных системах}}
    \scnsuperset{синтаксический аспект}
    \scnsuperset{семантический аспект}
    \begin{scnindent}
    \scntext{примечание}{Cогласованность систем понятий, их денотационной семантики}
    \end{scnindent}
    \scnsuperset{функциональный аспект}
    \begin{scnindent}
        \scneq{операционный аспект}
    \end{scnindent}
    
    \scnheader{следует отличать*}
    \begin{scnhaselementset}
    	\scnitem{\textit{совместимость} между компонентами \textit{интеллектуальных компьютерных систем}}
        \scnitem{\textit{совместимость} между верхним логико-семантическим уровнем используемых моделей представления и обработки знаний и различными уровнями их интерпретации вплоть до аппаратного уровня}
        \scnitem{\textit{совместимость} между индивидуальными интеллектуальными компьютерными системами}
    	\scnitem{\textit{совместимость} между индивидуальными интеллектуальными компьютерными системами и их пользователями}
    	\scnitem{\textit{совместимость} между коллективами интеллектуальных компьютерных системам}
    \end{scnhaselementset}

	\scnheader{следует отличать*}
	\begin{scnhaselementset}
		\scnitem{данные}
		\begin{scnindent}
			\scnidtf{информационная конструкция, обрабатываемая с помощью программы традиционного языка программирования}
		\end{scnindent}
		\scnitem{знание}
		\begin{scnindent}
			\scnidtf{семантически целостный фрагмент базы знаний}
		\end{scnindent}
	\end{scnhaselementset}
	\begin{scnindent}
		\scntext{отличие}{Для каждого знания всегда известен язык, на котором это знание представлено и денотационная семантика которого задана. При этом указанный язык имеет достаточно большую семантическую мощность, а в идеале является универсальным языком. В отличие от этого структуризация данных для традиционных программ осуществляется в целях упрощения самих этих программ и, следовательно, для разных программ в общем случае осуществляется по-разному. Таким образом, при разработке традиционных программ представление обрабатываемых данных осуществляется в общем случае на разных языках, денотационная семантика которых нигде не документируется и известна только разработчикам программ. Другими словами, данные для разных программ имеют денотационную семантику не только разную, но еще и априори неизвестную. По сути это форма проявления \textit{вавилонского столпотворения} в традиционных языках программирования, которые образно говоря \scnqq{хромают на одну ногу}, формализуя методы обработки информации, но не формализуя семантику обрабатываемой информации.}
	\end{scnindent} 

\end{scnsubstruct}

		\scnsegmentheader{Предметная область и онтология принципов, лежащие в основе онтологических моделей 
    мультимодальных интерфейсов интеллектуальных компьютерных систем нового поколения}

\begin{scnsubstruct}
    \begin{scnrelfromlist}{ключевое понятие}
    	\scnitem{смысловая память}
    	\scnitem{графодинамическая память}
    	\scnitem{ассоциативная память с информационным доступом по образцу произвольного размера и
            конфигурации}
        \scnitem{система ситуационного децентрализованного управления информационными процессами}
    	\scnitem{многоагентная система обработки информации в общей памяти}
    	\scnitem{язык смыслового представления задач}
        \scnitem{универсальный язык смыслового представления знаний}
    	\scnitem{язык смыслового представления методов}
        \begin{scnindent}
    		\scnidtf{интегрированный язык смыслового представления различного вида программ}
    	\end{scnindent}
    	\scnitem{инсерционная программа}
    \end{scnrelfromlist}
   
    \begin{scnrelfromlist}{ключевое знание}
    	\scnitem{Принципы, лежащие в основе решателей задач индивидуальных интеллектуальных компьютерных
            систем нового поколения}
    \end{scnrelfromlist}

    \scnheader{решатель задач интеллектуальных компьютерных систем нового поколения}
    \begin{scnrelfromlist}{предъявляемые требования}
        \scnitem{решатель задач интеллектуальных компьютерных систем нового поколения должен уметь решать 
            интеллектуальные задачи}
            \begin{scnrelfromlist}{виды задач}
                \scnitem{некачественно сформулированная задача}
                \begin{scnindent}
                    \scnidtf{задача, формулировка которой содержит различные не-факторы (неполнота, нечеткость,
                        противоречивость (некорректность) и так далее)}
                \end{scnindent}
                \scnitem{задача, для решения которой, кроме самой формулировки задачи и соответствующего метода ее
                    решения необходима дополнительная, но априори неизвестно какая информация об объектах, указанных
                    в формулировке (постановке) задачи. При этом указанная дополнительная информация
                    может присутствовать, а может и отсутствовать в текущем состоянии базы знаний интеллектуальных
                    компьютерных систем. Кроме того, для некоторых задач может быть задана (указана) та область
                    базы знаний, использования которой достаточно для поиска или генерации (в частности, логического
                    вывода) указанной дополнительной требуемой информации. Такую область базы знаний будем
                    называть областью решения соответствующей задачи}
                \scnitem{задача, для которой соответствующий метод ее решения в текущий момент не известен}
                \begin{scnrelfromlist}{решение}
                    \scnitem{переформулировать задачу, то есть сгенерировать (логически вывести) логически эквивалентную
                        формулировку исходной задачи, для которой метод ее решения в текущий момент является
                        известным}
                    \scnitem{свести исходную задачу к семейству подзадач, для которых методы их решения в текущий
                        момент известны.}
                \end{scnrelfromlist}
            \end{scnrelfromlist}
        \scnitem{процесс решения задач в интеллектуальных компьютерных системах нового поколения реализуется коллективом
            информационных агентов, обрабатывающих базу знаний интеллектуальных компьютерных систем}
        \scnitem{управление информационными процессами в памяти интеллектуальных компьютерных систем нового
            поколения осуществляется децентрализованным образом по принципам ситуационного управления}
    \end{scnrelfromlist}

    \scnheader{ситуационное управление}
    \scnidtf{ситуационно-событийное управление}
    \scntext{пояснение}{управление последовательностью выполнения действий, при котором условием (scnqq{триггером}) инициирования
        указанных действий является:
        \begin{scnitemize}
            \item{возникновение некоторых ситуаций (условий, состояний);}
            \item{и/или возникновение некоторых событий.}
        \end{scnitemize}}
        
    \scnheader{ситуация}
    \scnidtf{структура, описывающая некоторую временно существующую конфигурацию связей между некоторыми
        сущностями}
    \scnidtf{описание временно существующего состояния некоторого фрагмента (некоторой части) некоторо
        динамической системы}

    \scnheader{событие}
    \scnsuperset{возникновение временной сущности}
    \begin{scnindent}
        \scnidtf{появление, рождение, начало существования некоторой временной сущности}
    \end{scnindent}
    \scnsuperset{исчезновение временной сущности}
    \begin{scnindent}
        \scnidtf{прекращение, завершение существования некоторой временной сущности}
    \end{scnindent}
    \scnsuperset{переход от одной ситуации к другой}
    \begin{scnindent}
        \scntext{примечание}{Здесь учитывается не только факт возникновения новой ситуации, но и ее предыстория — то есть та
        ситуация, которая ей непосредственно предшествует. Так, например, реагируя на аномальное значение
        какого-либо параметра, нам важно знать:
        \begin{scnitemize}
            \item{какова динамика изменения этого параметра (увеличивается он или уменьшается и с какой скоростью);}
            \item{какие меры были предприняты ранее для ликвидации этой аномалии.}
        \end{scnitemize}}
    \end{scnindent}

    \scnheader{решатель задач индивидуальной интеллектуальной компьютерной системы нового поколения}
    \begin{scnrelfromlist}{принципы, лежащие в основе}
        \scnitem{смысловое представление обрабатываемых знаний}
        \scnitem{семантически неограниченный ассоциативный доступ к различным фрагментам знаний, хранимым в
            памяти интеллектуальных компьютерных систем нового поколения (доступ по заданному образцу произвольного
            размера и произвольной конфигурации)}
        \scnitem{графодинамический характер обработки знаний в памяти, при котором обработка знаний сводится не
            только к изменению состояния атомарных фрагментов (ячеек) памяти, но и к изменению конфигурации
            связей между этими атомарными фрагментами}
        \scnitem{ситуационное децентрализованное управление процессом обработки знаний, а также процессом организации
            взаимодействия интеллектуальных компьютерных систем с внешней средой}
        \scnitem{использование семантически мощного языка задач, обеспечивающего представление формулировок самых
            различных задач, которые могут решаться либо в рамках памяти интеллектуальной компьютерной
            системы, либо во внешней среде и которые осуществляют инициирование соответствующих процессов
            решения задач}
        \scnitem{многоагентный характер реализации процессов решения инициированных задач, в основе которого лежит
            иерархическая система агентов, каждый из которых активизируются при возникновении в памяти
            интеллектуальной компьютерной системы соответствующий ситуации или соответствующего события}
    \end{scnrelfromlist}
\end{scnsubstruct}

		\scnsegmentheader{Принципы, лежащие в основе онтологических моделей мультимодальных
    интерфейсов интеллектуальных компьютерных систем нового поколения}

\begin{scnsubstruct}
    \begin{scnrelfromlist}{ключевое понятие}
    	\scnitem{мультимодальный интерфейс}
        \scnitem{вербальный интерфейс}
        \scnitem{естественно-языковой интерфейс}
        \scnitem{внешний язык}
    	\begin{scnindent}
    		\scnidtf{язык обмена сообщениями}
    	\end{scnindent}
        \scnitem{внутренний язык}
    	\begin{scnindent}
    		\scnidtf{язык представления информации в памяти кибернетической системы}
    	\end{scnindent}
        \scnitem{синтаксис внешнего языка}
        \scnitem{денотационная семантика внешнего языка}
        \scnitem{интерфейсная задача}
        \scnitem{понимание сообщения}
        \scnitem{синтез сообщения}
        \scnitem{невербальный интерфейс}
        \scnitem{сенсор}
    	\begin{scnindent}
    		\scnidtf{рецептор}
    	\end{scnindent}
        \scnitem{сенсорная подсистема}
        \scnitem{мультисенсорная подсистема}
        \scnitem{сенсорная информация}
        \scnitem{эффектор}
        \scnitem{мультиэффекторная подсистема}
        \scnitem{сенсо-моторная координация}
    \end{scnrelfromlist}
   
    \begin{scnrelfromlist}{ключевое знание}
    	\scnitem{Принципы, лежащие в основе интерфейсов интеллектуальных компьютерных систем нового
            поколения}
    \end{scnrelfromlist}

    \scnheader{интерфейс интеллектуальной компьютерной системы нового поколения}
    \begin{scnrelfromlist}{принципы, лежащие в основе}
        \scnfileitem{интерфейс \textit{интеллектуальной компьютерной системы нового поколения} рассматривается как решатель
        задач частного вида — \textit{интерфейсных задач}}
        \begin{scnrelfromlist}{основные зачачи}
            \scnfileitem{задачи понимания вербальной информации, приобретаемой интеллектуальной компьютерной системой 
                (синтаксический анализ, семантический анализ и погружение в базу знаний интеллектуальной
                компьютерной системы)}
            \scnfileitem{задачи понимания невербальной информации, воспринимаемой сенсорными подсистемами
                интеллектуальной компьютерной системы (анализ изображений, анализ аудио-сигналов, погружение 
                результатов анализа в базу знаний интеллектуальной компьютерной системы)}
            \scnfileitem{задачи синтеза сообщений, адресуемых внешним субъектам (кибернетическим системам)}
        \end{scnrelfromlist}
        \scnfileitem{тот факт, что интерфейс \textit{интеллектуальной компьютерной системы нового поколения} является 
            решателем частного вида \textit{задач интеллектуальной компьютерной системы нового поколения}, свойства,
            лежащие в основе решателей \textit{задач интеллектуальной компьютерной систем нового поколения}, наследуются
            интерфейсами \textit{интеллектуальной компьютерной систем нового поколения}}
        \begin{scnrelfromlist}{принципы, лежащие в основе}
            \scnfileitem{смысловое представление накапливаемых (приобретаемых знаний)}
            \scnfileitem{трактовка семантического анализа приобретаемой вербальной информации как процесса перевода
                этой информации на внутренний язык смыслового представления знаний с последующим погружением
                (вводом, интеграцией) результата этого перевода в состав текущего состояния базы знаний
                \textit{интеллектуальной компьютерной системы нового поколения}}
            \scnfileitem{трактовка синтеза сообщений, адресуемых внешними субъектами как процесса обратного перевода
                некоторого фрагмента базы знаний с внутреннего языка смыслового представления информации на
                внешний язык, используемый для общения с заданным субъектом}
            \scnfileitem{агентно-ориентированная организация решения интерфейсных задач, реализуемая соответствующим
                коллективов внутренних агентов \textit{интерфейса интеллектуальных компьютерных систем нового 
                поколения}, взаимодействующих через общедоступную для них базу знаний \textit{интеллектуальной
                компьютерной системы нового поколения}}
        \end{scnrelfromlist}
        \scnfileitem{интерфейс \textit{интеллектуальной компьютерной системы нового поколения} трактуется как специализированная
            встроенная \textit{интеллектуальная компьютерная система нового поколения}, входящая в состав
            указанной выше интеллектуальной компьютерной системы, база знаний которой включает в себя:
            \begin{scnitemize}
                \item{онтологию синтаксиса внутреннего языка смыслового преставления информации}
                \item{онтологию денотационной семантики внутреннего языка смыслового представления информации}
                \item{онтологию синтаксиса всех внешних языков, используемых для общения с внешними субъектами}
                \item{онтологии денотационной семантики всех внешних языков, используемых для общения с внешними субъектами (каждая такая онтология с формальной точки зрения является описанием соответствия между текстами внешних языков и семантически эквивалентными им текстами внутреннего языка смыслового представления информации)}
            \end{scnitemize}}
        \scntext{примечание}{Подчеркнем при этом, что все указанные онтологии, входящие в состав базы знаний интерфейса
            интеллектуальных компьютерных систем нового поколения, как и вся остальная информация, входящая в
            состав этой базы знаний, представляется на внутреннем языке смыслового представления информации,
            который, соответственно используется в данном случае как метаязык}
    \end{scnrelfromlist}

    \scnheader{интерфейс индивидуальной интеллектуальной компьютерной системы нового поколения}
    \begin{scnrelfromlist}{принципы, лежащие в основе}
        \scnfileitem{интерфейс индивидуальной интеллектуальной компьютерной системы нового поколения является
            специализированным компонентом решателя задач интеллектуальной компьютерной системы нового поколения,
            то есть специализированной \uline{встроенной} (в индивидуальную интеллектуальную компьютерную систему
            нового поколения) интеллектуальной компьютерной системой нового поколения, ориентированной на
            решение интерфейсных задач, к которым относятся:
            \begin{scnitemize}
                \item{понимание принятых сообщений (их перевод на язык внутреннего смыслового представления информации
                    и погружения в текущее состояние базы знаний)}
                \item{синтез передаваемых сообщений (перевод сформированного сообщения с внутреннего языка смыслового
                    представления на используемый внешний язык)}
                \item{первичный анализ приобретаемой сенсорной информации, предполагающий распознавание некоторого
                    семейства первичных образов и сцен}
                \scnfileitem{сенсомоторная координация действий, выполняемых эффекторами интеллектуальной компьютерной системы}
            \end{scnitemize}}
        \scnfileitem{мультимодальный характер интерфейса — многообразие внешних языков, видов сенсоров и эффекторов}
        \scnfileitem{формальное онтологическое описание на языке внутреннего смыслового представления информации}
        \begin{scnrelfromlist}{виды информации}
            \scnfileitem{синтаксиса и денотационной семантики всех используемых внешних языков}
            \scnfileitem{первичных образов и сцен (ситуаций), являющихся результатом первичного анализа приобретаемой
                сенсорной информации}
            \scnfileitem{методов низкого уровня, непосредственно интерпретируемых эффекторами интеллектуальной компьютерной системы}
        \end{scnrelfromlist}
    \end{scnrelfromlist}
    \scntext{примечание}{Разговоры о дружественном и, в частности, адаптивном \textit{пользовательском интерфейсе} ведутся давно, но это, чаще
        всего, касается формы (\scnqq{синтаксической} стороны) \textit{пользовательского интерфейса}, а не смыслового содержания
        взаимодействия с пользователями. В настоящее время \textit{пользовательские интерфейсы} компьютерных систем (в
        том числе и \textit{интеллектуальных компьютерных систем}) для широкого контингента пользователей не являются
        семантически (содержательно) дружественными (семантически комфортными). Организация взаимодействия
        пользователей с компьютерными системами (в том числе и с \textit{интеллектуальными компьютерными системами})
        является \scnqq{узким местом}, оказывающим существенное влияние на эффективность \textit{автоматизации человеческой
        деятельности}. В основе современной организации взаимодействия пользователя с компьютерной системой лежит
        парадигма \uline{грамотного} пользователя, который знает, чего он хочет от используемого им инструмента и несет полную
        ответственность за качество взаимодействия с этим инструментом. Эта парадигма лежит в основе деятельности
        лесоруба во взаимодействии с топором, всадника во взаимодействии с лошадью, автоводителя, летчика во взаимодействии
        с соответствующим транспортным средством, оператора атомной электростанции, железнодорожного диспетчера и так далее.}
    \scntext{примечание}{На современном этапе развития \textit{Искусственного интеллекта} для повышения эффективности взаимодействия
        необходим переход \uline{от парадигмы грамотного управления} используемым инструментом \uline{к парадигме равноправного
        сотрудничества}, партнерскому взаимодействию интеллектуальной компьютерной системы со своим пользователем.
        \textit{Интеллектуальная компьютерная система} должна повернуться \scnqq{лицом} к пользователю. Семантическая дружественность 
        пользовательского интерфейса должна заключаться в адаптивности к особенностям и квалификации пользователя, исключении 
        любых проблем для пользователя в процессе диалога с \textit{интеллектуальной компьютерной системой}, в перманентной заботе о 
        совершенствовании коммуникационных навыков пользователя.}
    \scntext{примечание}{При организации взаимодействия пользователя с \textit{Глобальной сетью} компьютерным системам необходимо перейти
        от парадигмы \scnqq{многооконного} интерфейса, в каждом \scnqq{окне} которого свои \scnqq{правила игры}, к парадигме \scnqq{одного
        окна}. Пользователь не должен знать, какое \scnqq{окно} ему надо \scnqq{открыть} (в какую систему ему надо войти) для
        удовлетворения той или иной его потребности.
        Пользователь не должен знать, какая конкретно система будет решать его задачу. Пользователь должен уметь с
        помощью \uline{универсальных} средств сформулировать свою задачу, а соответствующая компьютерная система, входящая 
        в \textit{Глобальную сеть} и способная решить эту задачу, должна сама инициироваться, реагируя на факт появления
        указанной задачи. Таким образом пользовательский интерфейс должен быть интерфейсом пользователя не с 
        конкретной компьютерной системой, а в целом со всей \textit{Глобальной сетью компьютерных систем}.}
\end{scnsubstruct}

		\scnsegmentheader{Достоинства предлагаемых принципов, лежащие в основе
интеллектуальных компьютерных систем нового поколения}

\begin{scnsubstruct}
    \scnheader{принципы, лежащих в основе интеллектуальных компьютерных систем нового поколения}
    \scntext{достоинство}{\textbf{смысловое представление информации} в памяти \textit{интеллектуальных компьютерных систем} обеспечивает
        устранение дублирования информации, хранимой в памяти \textit{интеллектуальной компьютерной системы}, то есть
        устранение многообразия форм представления одной и той же информации, запрещение появления в одной памяти 
        \textit{семантически эквивалентных информационных конструкций} и, в том числе, синонимичных \textit{знаков}. Это
        существенно снижает сложность и повышает качество:
    \begin{scnitemize}
        \item{разработки различных \textit{моделей обработки знаний} (так как нет необходимости учитывать многообразие форм
            представления одного и того же знания);}
        \item{\textit{семантического анализа} и \textit{понимания} информации, поступающей (передаваемой) от различных внешних
            субъектов (от пользователей, от разработчиков, от других \textit{интеллектуальных компьютерных систем});}
        \item{\textit{конвергенции} и \textit{интеграции} различных видов знаний в рамках каждой \textit{интеллектуальной компьютерной
            системы};}
        \item{обеспечения \textit{семантической совместимости} и \textit{взаимопонимания} между различными \textit{интеллектуальными
            компьютерными системами}, а также между \textit{интеллектуальными компьютерными системами} и их пользователями}
    \end{scnitemize}}

    \scntext{достоинство}{Понятие \textit{семантической сети} нами рассматривается не как красивая метафора сложноструктурированных
        \textit{знаковых конструкций}, а как формальное уточнение понятия \textit{смыслового представления информации}, как принцип
        представления информации, лежащей в основе нового поколения \textit{компьютерных языков} и самих \textit{компьютерных 
        систем} — \textit{графовых языков} и \textit{графовых компьютеров}. \textit{Семантическая сеть} — это нелинейная (графовая)
        \textit{знаковая конструкция}, обладающая следующими свойствами:
        \begin{scnitemize}
            \item{все элементы (то есть синтаксически элементарные фрагменты) этой \textit{графовой структуры} (узлы и связки)
                являются знаками описываемых сущностей и, в частности, \textit{знаками связей} между этими сущностями;}
            \item{все знаки, входящие в эту \textit{графовую структуру}, не имеют \textit{синонимов} в рамках этой структуры;}
            \item{\scnqq{внутреннюю} структуру (строение) \textit{знаков}, входящих в семантическую сеть не требуется учитывать при ее
                \textit{семантическом анализе} (понимании);}
            \item{смысл \textit{семантической сети} определяется денотационной семантикой всех входящих в нее знаков и конфигурацией
                \textit{связей инцидентности} этих знаков;}
            \item{из двух \textit{инцидентных знаков}, входящих в \textit{семантическую сеть}, по крайней мере один является знаком связи.}
        \end{scnitemize}}

    \scntext{достоинство}{\textit{рафинированная семантическая сеть} — это \textit{семантическая сеть}, имеющая максимально простую 
        \textit{синтаксическую структуру}, в которой, в частности,
        \begin{scnitemize}
            \item{используется \uline{конечный} \textit{алфавит} элементов \textit{семантической сети}, то есть конечное число синтаксически
                выделяемых типов (синтаксических меток), приписываемых этим элемента;}
            \item{внешние идентификаторы (в частности, имена), приписываемые элементам \textit{семантической сети} используются
                \uline{только} для ввода/вывода информации}
        \end{scnitemize}}

    \scntext{достоинство}{\textit{агентно-ориентированная модель обработки информации} в сочетании с \textit{децентрализованным ситуационным
        управлением процессом обработки информации}, а также со \textit{смысловым представлением информации} в памяти
        \textit{интеллектуальной компьютерной системы} существенно снижает сложность и повышает качество интеграции
        \begin{scnitemize}
        \item{беспечивает автоматизацию решения сложных комплексных задач, для которых требуется создание
            временных или постоянных \uline{коллективов};}
        \item{превращает \textit{интеллектуальные компьютерные системы} в \uline{самостоятельные} активные \textit{субъекты}, способные
            инициировать различные комплексные задачи и, собственно, инициировать для этого работо-
            способные коллективы, состоящие из людей и \textit{интероперабельных интеллектуальных компьютерных
            систем} требуемой квалификации}
        \end{scnitemize}}

    \scntext{достоинство}{Высокий уровень семантической гибкости информации, хранимой в памяти интеллектуальной компьютерной
        системы нового поколения, обеспечивается тем, что каждое удаление или добавление синтаксически элементарного
        фрагмента хранимой информации, а также удаление или добавление каждой связи инцидентности между такими
        элементами имеет четкую семантическую интерпретацию.}

    \scntext{достоинство}{Высокий уровень стратифицированности информации, хранимой в памяти интеллектуальной компьютерной
        системы нового поколения, обеспечивается онтологически ориентированной структуризацией базы знаний интеллектуальной
        компьютерной системы нового поколения.}

    \scntext{достоинство}{Высокий уровень индивидуальной обучаемости интеллектуальных компьютерных систем нового поколения (то
        есть их способности к быстрому расширению своих знаний и навыков) обеспечивается:
        \begin{scnitemize}
            \item{семантической гибкостью информации, хранимой в их памяти;}
            \item{стратифицированностью этой информации;}
            \item{рефлексивностью интеллектуальных компьютерных систем нового поколения.}
        \end{scnitemize}}

    \scntext{достоинство}{Высокий уровень коллективной обучаемости интеллектуальных компьютерных систем нового поколения
        обеспечивается высоким уровнем их интероперабельности (их социализации, способности к эффективному участию в
        деятельности различных коллективов, состоящих из интеллектуальных компьютерных систем нового поколения и
        людей) и, прежде всего, высоким уровнем их взаимопонимания.}

    \scntext{достоинство}{Высокий уровень интероперабельности интеллектуальных компьютерных систем нового поколения
        принципиально меняет характер взаимодействия компьютерных систем с людьми, автоматизацию деятельности которых они
        осуществляют, — от управления этими средствами автоматизации к равноправным партнерским осмысленным
        взаимоотношениям}

    \scntext{достоинство}{Каждая интеллектуальная компьютерная система нового поколения способна:
        \begin{scnitemize}
            \item{самостоятельно или по приглашению войти в состав коллектива, состоящего из интеллектуальных компьютерных 
                систем нового поколения и/или людей. Такие коллективы создаются на временной или постоянной основе
                для коллективного решения сложных задач;}
            \item{участвовать в распределении (в том числе в согласовании распределения) задач — как \scnqq{разовых} задач, так и
                долгосрочных задач (обязанностей);}
            \item{мониторить состояние всего процесса коллективной деятельности и координировать свою деятельность с
                деятельностью других членов коллектива при возможных непредсказуемых изменениях условий (состояния)
                соответствующей среды.}
        \end{scnitemize}}

    \scntext{достоинство}{Высокий уровень интеллекта интеллектуальных компьютерных систем нового поколения и, соответственно,
        высокий уровень их самостоятельности и целенаправленности позволяет им быть полноправными членами самых
        различных сообществ, в рамках которых интеллектуальные компьютерные системы нового поколения получают
        права самостоятельно инициировать (на основе детального анализа текущего положения дел и, в том числе,
        текущего состояния плана действий сообщества) широкий спектр действий (задач), выполняемых другими членами
        сообщества, и тем самым участвовать в согласовании и координации деятельности членов сообщества. Способность
        интеллектуальной компьютерной системы нового поколения согласовывать свою деятельность с другими
        подобными системами, а также корректировать деятельность всего коллектива интеллектуальных компьютерных
        систем нового поколения, адаптируясь к различного вида изменениям среды (условий), в которой эта деятельность
        осуществляется, позволяет существенно автоматизировать деятельность системного интегратора как на этапе
        создания коллектива интеллектуальных компьютерных систем нового поколения, так и на этапе его обновления
        (реинжиниринга)}
    
    \scntext{примечание}{Достоинства интеллектуальных компьютерных систем нового поколения обеспечиваются:
        \begin{scnitemize}
            \item{достоинствами языка внутреннего смыслового кодирования информации, хранимой в памяти этих систем;}
            \item{достоинствами организации графодинамической ассоциативной смысловой памяти интеллектуальных
                компьютерных систем нового поколения;}
            \item{достоинствами смыслового представления баз знаний интеллектуальных компьютерных систем нового
                поколения и средствами онтологической структуризации баз знаний этих систем;}
            \item{достоинствами агентно-ориентированных моделей решения задач, используемых в интеллектуальных
                компьютерных системах нового поколения в сочетании с децентрализованным управлением процессом обработки
                информации.}
        \end{scnitemize}}

    \begin{scnrelfromlist}{основные положения}
    \scnfileitem{основным практически значимым направлением развития современных интеллектуальных компьютерных
        систем является переход к интероперабельным интеллектуальным компьютерным системам, способным к эффективному
        взаимодействию между собой и с пользователями, что:
        \begin{scnitemize}
            \item{обеспечивает автоматизацию решения сложных комплексных задач, для которых требуется создание
                временных или постоянных \uline{коллективов;}}
            \item{превращает интеллектуальные компьютерные системы в \uline{самостоятельные} активные субъекты, способные
                инициировать различные комплексные задачи и, собственно, инициировать для этого работоспособные
                коллективы, состоящие из людей и интероперабельных интеллектуальных компьютерных систем требуе-
                мой квалификации.}
        \end{scnitemize}}
    \scnfileitem{коллективы, состоящие из самостоятельных \textit{интероперабельных интеллектуальных компьютерных систем}
        и людей, имеют хорошие перспективы стать \textit{синергетическими} системами}
    \scnfileitem{\textit{интероперабельность интеллектуальных компьютерных систем} обеспечивается:
        \begin{scnitemize}
            \item{высоким уровнем взаимопонимания и, соответственно, семантической совместимостью;}
            \item{высоким уровнем договороспособности, то есть способности предварительно согласовывать свои действия
                с действиями других субъектов;}
            \item{высоким уровнем способности оперативно координировать свои действия с действиями других субъектов
                в ходе их выполнения}
        \end{scnitemize}}
    \scnfileitem{к числу принципов, лежащих в основе построения \textit{интероперабельных интеллектуальных компьютерных
        систем}, относятся:
        \begin{scnitemize}
            \item{смысловое представление знаний в памяти \textit{интеллектуальных компьютерных систем} в виде рафинированных 
                семантических сетей;}
            \item{использование универсального языка внутреннего смыслового представления знаний;}
            \item{графодинамическая организация обработки знаний;}
            \item{агентно-ориентированные модели решения задач;}
            \item{структуризация и стратификация баз знаний в виде иерархической системы формальных онтологий;}
            \item{семантически дружественный пользовательский интерфейс.}
        \end{scnitemize}}
    \scnfileitem{для разработки большого количества интероперабельных семантически совместимых \textit{интеллектуальных
        компьютерных систем}, обеспечивающих переход на принципиально новый уровень автоматизации \textit{человеческой
        деятельности}, необходимо создание технологии, обеспечивающей массовое производство таких \textit{интеллектуальных
        компьютерных систем}, участие в котором доступно широкому контингенту разработчиков (в том
        числе разработчиков средней квалификации и начинающих разработчиков). Основными положениями такой
        технологии являются
        \begin{scnitemize}
            \item{стандартизация \textit{интероперабельных интеллектуальных компьютерных систем};}
            \item{широкое использование \textit{компонентного проектирования} на основе мощной библиотеки семантически
                совместимых многократно используемых (типовых) компонентов \textit{интероперабельных интеллектуальных
                компьютерных систем}}
        \end{scnitemize}}
    \scnfileitem{эффективная эксплуатация \textit{интероперабельных интеллектуальных компьютерных систем} требует создания
        не только \textit{технологии проектирования} таких систем, но также и семейства технологий поддержки всех
        остальных этапов их жизненного цикла. Особенно это касается технологии перманентной поддержки \textit{семантической
        совместимости} всех взаимодействующих \textit{интероперабельных интеллектуальных компьютерных систем} в ходе их эксплуатации}
    \end{scnrelfromlist}
\end{scnsubstruct}

	\end{scnsubstruct}
\end{SCn}
\scnsourcecomment{Завершили Раздел \scnqqi{Предметная область и онтология интеллектуальных компьютерных систем нового поколения}}


\scsubsection[
    \protect\scneditor{Банцевич К.А.}
    \protect\scnmonographychapter{Глава 1.2. Интеллектуальные компьютерные системы нового поколения}
    ]{Предметная область и онтология смыслового представления информации}
\label{sd_sem_inf_rep}
\begin{SCn}
	\scnsectionheader{Предметная область и онтология смыслового представления информации}

	\begin{scnsubstruct}

		\scnrelto{частная предметная область и онтология}{Предметная область и
			онтология информационных конструкций}
		\begin{scnhaselementrolelist}{класс объектов исследования}
			\scnitem{смысловое представление информации}
			\begin{scnindent}
				\scnidtf{смысл}
			\end{scnindent}
		\end{scnhaselementrolelist}

		\begin{scnhaselementrolelist}{класс объектов исследования}
			\scnitem{семантическая сеть}
				\begin{scnindent}
					\begin{scnsubdividing}
						\scnitem{нерафинированная семантическая сеть}
						\scnitem{рафинированная семантическая сеть}
					\end{scnsubdividing}
					\begin{scnsubdividing}
						\scnitem{абстрактная семантическая сеть}
							\begin{scnindent}
								\scnidtf{семантическая сеть, абстрагирующаяся от того, как
									физически представлены ее элементарные (атомарные) фрагменты, а также связи
									инцидентности между этими фрагментами}
							\end{scnindent}
						\scnitem{графически представленная семантическая сеть}
							\begin{scnindent}
								\scnidtf{нарисованная семантическая сеть}
							\end{scnindent}
						\scnitem{семантическая сеть, хранимая в графодинамической памяти}
							\begin{scnindent}
								\scnrelboth{следует отличать}{представление семантической сети
									в адресной памяти}
								\scnnotsubset{семантическая сеть}
								\scnidtf{представление семантической сети в виде линейной
									информационной конструкции, которая хранится в адресной памяти и которая,
									строго говоря, уже не является семантической сетью, но является информационной
									конструкцией, семантически эквивалентной соответствующей (представляемой)
									семантической сети}
							\end{scnindent}
					\end{scnsubdividing}
				\end{scnindent}
			\scnitem{граф знаний}
			\begin{scnindent}
				\scnidtf{представление сложноструктурированного знания в виде графовой структуры}
			\end{scnindent}
			\scnitem{язык семантических сетей}
				\begin{scnindent}
					\scnidtf{язык, все тексты которого являются семантическими
						сетями}
					\begin{scnsubdividing}
						\scnitem{специализированный язык семантических сетей}
						\scnitem{универсальный язык семантических сетей}
					\end{scnsubdividing}
					\scnsuperset{язык рафинированных семантических сетей}
				\end{scnindent}
		\end{scnhaselementrolelist}

		\begin{scnrelfromvector}{рассматриваемые вопросы}
			\scnfileitem{Что такое семантические сети и в чем их принципиальное
				отличие от других вариантов представления информации}
			\scnfileitem{До какой степени можно минимизировать алфавит элементов
				семантических сетей}
			\scnfileitem{Можно ли все описываемые связи свести к бинарным связям и
				почему это целесообразно}
			\scnfileitem{Можно ли разработать \uline{универсальный} язык
				семантических сетей}
			\scnfileitem{До какой степени можно упростить синтаксические структуры
				семантических сетей до, условно говоря, рафинированного вида}
			\scnfileitem{Какими достоинствами обладает семантические сети}
		\end{scnrelfromvector}

		\begin{scnrelfromlist}{ссылка}
			\scnitem{Понятие Технологии OSTIS}
				\begin{scnindent}
					\scntext{аннотация}{В указанном сегменте \textit{Стандарта
							OSTIS} рассматриваются принципы, лежащие в основе \textit{Технологии OSTIS},
						основным из которых является ориентация на использование
						\textit{\uline{универсального} языка рафинированных семантических сетей} в
						качестве внутреннего языка \textit{интеллектуальных компьютерных систем}}
				\end{scnindent}
			\scnitem{Описание внутреннего языка ostis-систем}
				\begin{scnindent}
					\scntext{аннотация}{В указанном разделе \textit{Стандарта
						OSTIS} рассматриваются принципы, лежащие в основе \textit{универсального языка
						рафинированных семантических сетей}, используемого в качестве внутреннего языка
						\textit{ostis-систем} --- \textit{интеллектуальных компьютерных систем}
						следующего поколения}
				\end{scnindent}
			\scnitem{Описание языка графического представления знаний ostis-систем}
				\begin{scnindent}
					\scntext{аннотация}{В указанном разделе \textit{Стандарта
						OSTIS} рассматриваются принципы, лежащие в основе универсального языка
						графически представленных семантических сетей, используемого в
						\textit{пользовательском интерфейсе ostis-систем}}
				\end{scnindent}
			\scnitem{\cite{Birukov1960}}
			\scnitem{\cite{Morris2001}}
			\scnitem{\cite{Pirs2009}}
			\scnitem{\cite{Stepanov1971}}
			\scnitem{\cite{MelchukST}}
		\end{scnrelfromlist}

		\bigskip
		\scnheader{знак}
		\scnidtf{фрагмент информационной конструкции, обладающий свойством,
			\uline{обозначать} некоторую сущность (объект), которая наряду с другими
			сущностями описывается указанной информационной конструкцией}
		\scntext{примечание}{\uline{Форма} представления знаков в известной степени
			условна и является результатом соглашения между носителями соответствующего
			языка. Знак может быть, например, представлен:
			\begin{scnitemize}
				\item  в виде фрагмента речевого сообщения (последовательностью
				фонем);
				\item в виде строки символов (последовательности букв) в
				заданном алфавите;
				\item в виде иероглифа, пиктограммы;
				\item в виде жеста.
			\end{scnitemize}}
		\scniselementrole{ключевой знак}{Предметная область и онтология
			информационных конструкций}
			\begin{scnindent}
				\scnhaselement{раздел Базы знаний IMS.ostis}
			\end{scnindent}
		\scntext{характеристика элементов данного множества}{Знаки,
			используемые в различных языках, характеризуются:
			\begin{scnitemize}
				\item синтаксической структурой, по совпадению (изоморфизму)
				которых для разных знаокв предполагается их синонимия;
				\item денотационной семантикой, т.е. той сущностью, которая
				обозначается соответствующим знаком;
				\item типом (классом) обозначаемой сущности, которая может
				быть:
				\begin{scnitemizeii}
					\item материальным(физическим) элементом (точкой)
					абстрактного пространства, множеством, которое может быть:
					\begin{scnitemizeiii}
						\item связью;
						\item классом;
						\item структурой;
					\end{scnitemizeiii}
					\item реальной и вымышленной сущностью;
					\item константной (конкретной) и переменной
					(произвольной) сущностью;
					\item постоянно существующей и временно существующей
					сущностью (прошлой, настоящей, будущей);
				\end{scnitemizeii}
				\item множеством тех связей, которые связывают сущность,
				обозначаемую данным знаком с другими сущностями, а также, если данный знак
				обозначает некоторую связь, множеством сущностей, которые связаны этой связью,
				т.е. сущностей, являющихся компонентом этой связи;
				\item текущим статусом самого знака в памяти кибернетической
				системы, который может быть:
				\begin{scnitemizeii}
					\item логически удаленным знаком;
					\item настоящим знаком;
					\item предлагаемым (возможно, будущим) знаком.
				\end{scnitemizeii}
			\end{scnitemize}}

		\bigskip
		\scnheader{денотат*}
		\scnidtf{денотат заданного знака*}
		\scnidtf{объект, обозначаемый заданным знаком*}
		\scnidtf{денотационная семантика заданного знака*}
		\scnidtf{смысл заданного знака*}
		\scnidtf{Бинарное ориентированное отношение, каждая пара которого
			связывает:
			\begin{scnitemize}
				\item некоторый знак, представленный в той или иной форме в тексте
				исследуемого языка;
				\item \uline{со знаком} той сущности, которая обозначается указанным
				выше знаком в рамках используемого метаязыка.
			\end{scnitemize}}

		\scntext{примечание}{Данное отношение используется, когда с помощью одного
			языка необходимо описать денотационную семантику другого языка. Фактически речь
			идет о переводе заданного знака, входящего в состав некоторого рассматриваемого
			текста, принадлежащего некоторому исследуемому языку (языку-объекту), на
			некоторый метаязык (в нашем случае на SC-код), денотационная семантика которого
			нам считается априори известной. Указанный перевод есть связь заданного знака с
			синонимичным ему знаком, входящим в состав текста, принадлежащего указанному
			метаязыку.}
		\scnrelboth{обратное отношение}{внешний sc-идентификатор*}
			\begin{scnindent}
				\scnidtf{быть знаком, обозначающим заданную сущность*}
			\end{scnindent}

		\scnheader{информационная конструкция}
		\scnidtf{информация}
		\scntext{примечание}{В общем случае информационная конструкция представляет
			собой сложную иерархическую структуру, каждому уровню иерархии которой
			соответствует определенный класс информационных конструкций.}
		\scnsuperset{синтаксически элементарный фрагмент информационной конструкции}
			\begin{scnindent}
				\scnidtf{атомарный фрагмент информационной конструкции}
				\scnidtf{элемент информационной конструкции}
				\scntext{примечание}{Примерами таких элементарных фрагментов информационных
					конструкций являются буквы}
				\scnsuperset{буква}
			\end{scnindent}
		\scnsuperset{простой знак}
			\begin{scnindent}
				\scnidtf{семантически элементарный фрагмент информационной конструкции}
				\scnsubset{знак}
			\end{scnindent}
		\scnsuperset{выражение}
			\begin{scnindent}
				\scnidtf{сложный (непростой) знак}
				\scnidtf{знак, являющийся одновременно некоторым знанием обозначаемой
					сущности (спецификацией этой сущности)}
				\scnidtf{знак, построенный как выражение вида тот, который... }
				\scnidtf{знак, в состав которого входят другие знаки}
				\scnsubset{знак}
			\end{scnindent}
		\scnsuperset{простой текст}
			\begin{scnindent}
				\scnidtf{минимальная синтаксически целостная и корректная (правильная)
					информационная конструкция, включающая в себя:
					\begin{scnitemize}
						\item знак некоторой описываемой связи;
						\item минимальную спецификацию указанного знака связи (указание
						отношения, которому это связь принадлежит);
						\item указание \uline{всех} компонентов описываемой связи (знаков всех
						сущностей, связываемых этой связью, и/или всех знаков, связываемых этой связью
						-- описываемая связь может связывать не только внешние	описываемые сущности,
						но и сами знаки);
						\item если описываемая связь не является бинарной, то связи с её
						компонентами могут потребовать явного представления знаков этих связей с
						дополнительным указанием роли этих компонентов.
					\end{scnitemize}}
				\scnsubset{текст}
			\end{scnindent}
		\scnsuperset{сложный текст}
			\begin{scnindent}
				\scnidtf{информационная конструкция, являющаяся результатом соединения
					нескольких простых текстов}
				\scnsubset{текст}
			\end{scnindent}
		\scnsuperset{простое знание}
			\begin{scnindent}
				\scnidtf{минимальная семантические целостная информационная конструкция}
				\scnidtf{знание, в состав которого не входят другие знания}
				\scnsubset{знание}
			\end{scnindent}
		\scnsuperset{сложное знание}
			\begin{scnindent}
				\scnidtf{информационная конструкция, являющаяся результатом соединения
					нескольких простых знаний}
				\scnidtf{знание, в состав которого не входят другие знания}
				\scnsubset{знание}
			\end{scnindent}
		\scniselementrole{ключевой знак}{Предметная область и онтология информационных конструкций}

		\scnheader{стандартизация моделей представления и обработки информации}
		\scnidtf{предлагаемый подход к решению проблем, препятствующих дальнейшей эволюции компьютерных систем и технологий}
		\scntext{примечание}{Анализ проблем эволюции компьютерных систем разного уровня сложности, разного уровня обучаемости и
		интеллектуальности, разного назначения показывает, что проклятие \scnqq{вавилонского столпотворения} и, как следствие,
		несовместимость, дублирование и субъективизм согласовываемых информационных ресурсов и моделей их обработки 
		нас преследует везде:
		\begin{scnitemize}
			\item{и в развитии традиционных компьютерных систем;}
			\item{и в развитии технологий искусственного интеллекта;}
			\item{и в развитии методов и средств информатизации научной и инженерной деятельности.}
		\end{scnitemize}}
		

		\scnheader{проблема обеспечения совместимости информационных ресурсов и моделей их обработки}
		\begin{scnrelfromlist}{аспекты решения}
			\scnitem{обеспечение совместимости между различными компонентами компьютерных систем, а также между
			целостными компьютерными системами, входящими в коллективы компьютерных систем}
			\scnitem{обеспечение совместимости, то есть высокого уровня взаимопонимания между различными компьютерными 
			системами и их пользователями}
			\scnitem{беспечение междисциплинарной совместимости, то есть конвергенции различных областей знаний}
			\scnitem{методы и средства постоянного мониторинга и восстановления совместимости в условиях интенсивной
			эволюции компьютерных систем и их пользователей, которая часто нарушает достигнутую совместимость
			(согласованность) и требует дополнительных усилий на ее восстановление}
		\end{scnrelfromlist}

		\scnheader{подход к решению проблем эволюции компьютерных систем}
		\scntext{примечание}{Суть предлагаемого нами подхода к решению проблем эволюции компьютерных систем заключается:
		\begin{scnitemize}
			\item{в объединении всех указанных выше направлений эволюции компьютерных систем (как общих направлений, так
				и частных)}
			\item{в трактовке проблемы обеспечения \textbf{совместимости} различных видов знаний, различных
				моделей решения задач, различных компьютерных систем как \textbf{ключевой проблемы} эволюции компьютерных
				систем, решение которой существенно упростит решение и многих других проблем}
		\end{scnitemize}}

		\scnheader{совместимость}
		\scntext{примечание}{Без обеспечения совместимости информационных ресурсов, используемых в разных компьютерных
		системах, а также информационных ресурсов, представляющих знания различного семантического вида невозможно:
		\begin{scnitemize}
			\item{ни создавать \textbf{коллективы компьютерных систем}, способные координировать свои действия при кооперативном
				расширении сложных задач;}
			\item{ни создавать \textbf{гибридные компьютерные системы}, которые способны при решении сложных комплексных
				задач использовать всевозможные сочетания разных видов знаний и разных моделей решения задач;}
			\item{ни использовать \textbf{компонентную методику проектирования} компьютерных систем \textbf{на всех уровнях} иерархии
				проектируемых систем.}
		\end{scnitemize}
		О какой информационной совместимости и взаимопонимании (в том числе между специалистами) можно говорить
		при наличии ужасающей понятийной и терминологической неряшливости, терминологического псевдотворчества,
		в том числе, в области информатики.}
		
		\scnheader{следует отличать*}
		\begin{scnhaselementset}
			\scnitem{совместимость как один из факторов обучаемости, как \textbf{способность} к быстрому повышению уровня
				согласованности (интеграции, взаимопонимания). Сравните обучаемость как \textbf{способность} к быстрому расширению
				знаний и навыков, но никак не характеристика объема и качества приобретенных знаний и навыков}
			\scnitem{совместимость как характеристика достигнутого уровня согласованности (интеграции, взаимопонимания)}
		\end{scnhaselementset}

		\scnheader{следует отличать*}
		\begin{scnhaselementset}
			\scnitem{интеллект компьютерной системы как \textbf{уровень} (объем и качество) приобретенных знаний и навыков}
			\scnitem{интеллект компьютерной системы как \textbf{способность} к быстрому расширению и
				совершенствованию знаний и навыков, то есть как \textbf{скорость} повышения уровня знаний и навыков}
		\end{scnhaselementset}

		\scnheader{следует говорить*}
		\begin{scnhaselementset}
			\scnitem{о \textbf{способности} к быстрому повышению уровня согласованности}
			\scnitem{о достигнутом уровне согласованности}
			\scnitem{о самом \textbf{процессе} повышения уровня согласованности}
			\scnitem{о перманентном процессе восстановления (поддержки, сохранения) достигнутого уровня согласованности,
				поскольку в ходе эволюции компьютерных систем и их пользователей (то есть в ходе расширения и повышения
				качества их знаний и навыков) уровень их согласованности может понижаться}
		\end{scnhaselementset}

		\scnheader{обеспечения совместимости различных видов знаний, различных моделей решения задач и различных компьютерных систем}
		\begin{scnrelfromlist}{главный фактор}
			\scnitem{унификация представления информации в памяти компьютерных систем}
				\begin{scnindent}
					\scnidtf{стандартизация представления информации в памяти компьютерных систем}
				\end{scnindent}
			\scnitem{унификация принципов организации обработки информации в памяти компьютерных систем}
		\end{scnrelfromlist}

		\scnheader{унификация представления информации в памяти компьютерных систем}
		\begin{scnrelfromset}{предполагает}
			\scnitem{синтаксическая унификация используемой информации}
			\begin{scnindent}
				\scnidtf{унификация формы представления (кодирования) этой информации}
			\end{scnindent}
			\scnitem{семантическая унификация используемой информации}
			\begin{scnindent}
				\scnidtf{согласование и точная спецификация всех (!) используемых понятий 
					(концептов) с помощью иерархической системы формальных онтологий}
			\end{scnindent}
		\end{scnrelfromset}
		\scntext{примечание}{Важно отметить, что грамотная унификация (стандартизация) должна не ограничивать творческую свободу 
		разработчика, а гарантировать \textbf{совместимость} его результатов с результатами других разработчиков.}

		\scnheader{следует отличать*}
		\begin{scnhaselementset}
			\scnitem{внутреннее представление информации}
			\begin{scnindent}
				\scnidtf{кодирование информации в памяти компьютерной системы}
			\end{scnindent}
			\scnitem{внешнее представление информации}
			\begin{scnindent}
				\scnidtf{обеспечение однозначности интерпретации (понимания, трактовки) этой информации
					разными пользователями и разными компьютерными системами}
			\end{scnindent}
		\end{scnhaselementset}

		\scnheader{стандарт}
		\scntext{примечание}{Подчеркнем, что текущая версия любого \textbf{стандарта} — это не догма, а только опора для дальнейшего его совершенствования.}
		\begin{scnrelfromlist}{цель}
			\scnitem{обеспечения совместимости технических решений}
			\scnitem{минимализация дублирования (повторения) решений}
		\end{scnrelfromlist}
		\scntext{критерий качества}{ничего лишнего}
		\scntext{примечание}{Стандарты, как и другие важные для человечества знания, должны быть формализованы и должны постоянно
		совершенствоваться с помощью специальных интеллектуальных компьютерных систем, поддерживающих процесс
		эволюции стандартов путем согласования различных точек зрения.}

		\bigskip\scnheader{смысловое представление информации}
		\scnidtf{смысловая форма представления информации}
		\scnidtf{смысловое представление информационной конструкции}
		\scnidtf{знаковая конструкция (текст), представленная в смысловой
			форме}
		\scnidtf{запись (представление) информационной конструкции на смысловом уровне}
		\scnidtf{информационная конструкция синтаксическая структура которой близка ее смыслу, то есть близка 
			описываемой конфигурации связей между описываемыми сущностями}
		\scnidtftext{часто используемый sc-идентификатор}{смысл}
		\scnidtf{смысловое представление}
		\scnidtf{семантическое представление информации}
		\scntext{основной принцип}{Как можно меньше лишнего, не имеющего
			отношения к смыслу представляемой информации.}
		\scnidtf{такое представление информационной конструкции, которое
			существенно упрощает соответствие между структурой самой этой информационной
			конструкции и описываемой (отображаемой) ею конфигурацией связей между
			рассматриваемыми (исследуемыми) сущностями}
		\scnidtf{смысловое представление знаковой конструкции}
		\scnidtf{абстрактная знаковая конструкция, являющаяся
			\uline{инвариантом} соответствующего максимального класса семантически
			эквивалентных знаковых конструкций}
		\scnidtf{смысл информационной конструкции}
		\scnidtf{денотационная семантика информационной конструкции}
		\scntext{примечание}{Суть (смысл, денотационная семантика) любой
			информационной конструкции (информационной модели) сводится к описанию системы
			(конфигурации) связей между списываемыми (рассматриваемыми) сущностями. Важно,
			чтобы эта суть не была \uline{закамуфлирована} различными синтаксическими
			деталями, не имеющими никакого отношения к указанному смыслу (синтаксическая
			структура знаков, многократное повторение одного и того же знака, синонимия,
			омонимия, местоимения, предлоги, знаки препинания, разделители, ограничители,
			падежи и т.п.) а обусловленными \uline{формой} представления информационных
			конструкций, например, их линейностью.}
		\scntext{пояснение}{Смысловое представление любой информации в
			конечном счете сводится:
			\begin{scnitemize}
				\item к перечню знаков конкретных описываемых сущностей - как первичных
				            сущностей, так и вторичных сущностей, которые сами являются информационными
				            конструкциями (фрагментами данной конструкции);
				\item к явному описанию связи между знаками вторичных сущностей и
				            самими этими сущностями (т.е. фрагментами информационной конструкции);
				\item к описанию других связей между описываемыми сущностями
			\end{scnitemize}
			\vspace{-0.6\baselineskip}}
		\newpage
		\scntext{пояснение}{Формализация смысла представляемой информации,
			т.е. строгое уточнение того, что такое \textit{смысловое представление
				информации}, является объективной основой для \uline{унификации} представления
			информации в \textit{памяти компьютерных систем} и \uline{ключом} к решению
			многих проблем семантической совместимости и эволюции компьютерных систем и
			технологий.
			
			Согласно \textit{Мартынову В. В.} (\cite{Martynov}), <<фактически
			всякая мыслительная деятельность человека (не только научная), как полагают
			многие ученые, использует \uline{внутренний семантический код}, на который
			переводят с естественного языка и с которого переводят на естественный язык.
			Поразительная способность человека к идентификации огромного множества
			структурно различных фраз с одинаковым \textit{смыслом} и способность
			\uline{запомнить смысл вне этих фраз} убеждает нас в этом.>>
			
			Приведем также слова \textit{Мельчука И. А.} (\cite{MelchukST}):
			
			<<Идея была следующая --- язык надо описывать следующим образом: надо уметь записывать смыслы фраз. \uline{Не
			фразы, а их \textit{смыслы}}, что отдельно. Плюс построить систему, которая по
			смыслу строит фразу. Это та область или тот поворот исследований, при котором
			интуиция способного лингвиста работает лучше всего: как выразить на данном
			языке данный смысл. Это --- то, для чего лингвистов учат.
			
			Лингвистический	\textit{смысл} научного текста --- это совсем не то, что ты, читая его, из него
			извлекаешь. Это, очень грубо говоря, инвариант синонимических перифраз. Ты
			можешь один и тот же смысл выразить очень многими способами. Когда ты говоришь,
			то можешь сказать по-разному: \scnqqi{Сейчас я налью тебе вина}, или: \scnqqi{Дай, я тебе
			предложу вина}, или: \scnqqi{Не выпить ли нам по бокалу?}, --- все это имеет один и тот
			же смысл. И вот можно придумать, как записывать этот \textit{смысл}. Именно
			его. Не фразу, а \textit{смысл}. И работать надо от этого \textit{смысла} к
			реальным фразам. Синтаксис там по дороге тоже нужен, но он нужен именно по
			дороге, он не может быть ни конечной целью, ни начальной точкой. Это --
			промежуточное дело.>> (\cite{Melchuk}).}
		\scntext{примечание}{Грамотная унификация (стандартизация) \textit{смыслового
				представления информации} не должна привести к ограничению творческой свободы
			авторов различного вида публикуемых научно-технических знаний (и, в том числе,
			разработчиков \textit{баз знаний}), не должна гарантировать
			\textit{семантическую совместимость} различных \textit{знаний}, представленных
			различными авторами (разумеется, при условии соблюдения соответствующих правил
			построения этих \textit{знаний}). При этом любые \textit{стандарты} (в том
			числе и принятые стандарты \textit{смыслового представления информации}) должны
			постоянно эволюционировать. Текущая версия любого стандарта должна быть не
			догмой, а точкой опоры для дальнейшего совершенствования этого стандарта.}
		\scnsuperset{УСК}
			\begin{scnindent}
				\scnidtf{Универсальный Семантический Код}
				\scnrelfrom{автор}{Мартынов В. В.}
				\scntext{примечание}{Разработанный Мартыновым В. В. Универсальный
					Семантический Код стал важнейшим этапом создания универсальных формальных
					средств смыслового представления знаний. Основная методологическая идея
					\textit{Мартынова В. В.}, касающаяся построения \textit{языка смыслового
					представления знаний}, заключается в том, чтобы выделить смысловые кирпичики ,
					имеющие достаточно общий характер, а многообразие конкретных смыслов
					конструировать комбинаторно за счёт различных комбинаций (конфигураций) из этих
					кирпичей. Это можно назвать принципом минимизации типов атомарных смысловых
					фрагментов.}
				\scnrelto{ключевой знак}{\cite{Martynov1984}}
			\end{scnindent}
		\scnsuperset{семантическая сеть}
			\begin{scnindent}
				\scnsuperset{рафинированная семантическая сеть}	
			\end{scnindent}
		\scntext{примечание}{Уточнение принципов \textbf{смыслового представления информации} основано, во-первых, на четком
			противопоставление \textbf{внутреннего языка компьютерной системы}, используемого для хранения информации в памяти
			компьютера, и \textbf{внешних языков компьютерной системы}, используемых для общения (обмена сообщениями) компьютерной
			системы с пользователями и другими компьютерными системами (смысловое представление используется
			исключительно для \textbf{внутреннего представления} информации в памяти компьютерной системы), и, во-вторых,
			на максимально возможном упрощении синтаксиса внутреннего языка компьютерной системы при обеспечении
			универсальности путем исключения из такого внутреннего универсального языка средств, обеспечивающих
			коммуникационную функцию языка (то есть обмен сообщениями). Так, например, для внутреннего языка компьютерной 
			системы излишними являются такие коммуникационные средства языка, как союзы, предлоги, разделители, ограничители, 
			склонения, спряжения и другие. Внешние языки компьютерной системы могут быть как близки ее внутреннему языку, так и 
			весьма далеки от него (как, например, естественные языки)}

		\scnheader{смысловое представление информации*}
		\scnidtftext{пояснение}{\textit{Бинарное ориентированное отношение},
			каждая \textit{пара} которого связывает некоторую \textit{информационную
				конструкцию} со смысловым представлением этой \textit{информационной
				конструкции*}.}
		\scnsubset{формализация*}
		\bigskip
		
		\scnheader{смысл}
		\scnidtf{\textbf{абстрактная} знаковая конструкция, принадлежащая внутреннему языку компьютерной системы,
			являющаяся \textbf{инвариантом} максимального класса семантически эквивалентных знаковых конструкций (текстов),
			принадлежащих самым разным языкам, и удовлетворяющая следующим требованиям:
			\begin{scnitemize}
				\item{\textbf{универсальность} — возможность представления любой информации;}
				\item{\textbf{отсутствие синонимии знаков} (многократного вхождения знаков с одинаковыми денотатами);}
				\item{\textbf{отсутствие дублирования информации} в виде семантически эквивалентных текстов (не путать с логической
				эквивалентностью);}
				\item{\textbf{отсутствие омонимичных знаков} (в том числе местоимений);}
				\item{\textbf{отсутствие у знаков внутренней структуры} (атомарный характер знаков);}
				\item{\textbf{отсутствие склонений, спряжений} (как следствие отсутствия у знаков внутренней структуры);}
				\item{\textbf{отсутствие фрагментов} знаковой конструкции, не являющихся знаками (разделителей, ограничителей, и
				так далее);}
				\item{\textbf{выделение знаков связей}, компонентами которых могут быть любые знаки, с которыми знаки связей
				связываются синтаксически задаваемыми отношениями инцидентности.}
			\end{scnitemize}}

		\scnheader{принципы смыслового представления информации в памяти компьютерной системы}
		\scntext{следствие}{Знаки сущностей, входящие в смысловое представление информации, \textbf{не являются именами}
			(терминами) и, следовательно, не привязаны ни к какому естественному языку и не зависят от субъективных
			терминотворческих пристрастий различных авторов. Это значит, что при коллективной разработке смыслового
			представления каких-либо информационных ресурсов терминологические споры исключены.}
		\scntext{следствие}{Эти принципы приводят к нелинейным знаковым конструкциям (к графовым структурам), что усложняет реализацию памяти
			компьютерных систем, но существенно упрощает ее логическую организацию (в частности, ассоциативный доступ).}

		\scnheader{нелинейность смыслового представления информации}
		\scntext{обусловлено}{
			\item{Каждая описываемая сущность, то есть сущность, имеющая соответствующий ей знак, может иметь неограниченное
				число связей с другими описываемыми сущностями;}
			\item{каждая описываемая сущность в смысловом представлении имеет единственный знак, так как синонимия
				знаков здесь запрещена;}
			\item{все связи между описываемыми сущностями описываются (отражаются, моделируются) связями между знаками
				этих описываемых сущностей.}
		}

		\scnheader{универсальное смысловое представления информации}
		\scntext{примечание}{Суть можно сформулировать в виде следующих положений:
		\begin{scnitemize}
			\item{Смысловая знаковая конструкция трактуется как множество знаков, взаимно-однозначно обозначающих различные
				сущности (денотаты этих знаков) и множество связей между этими знаками;}
			\item{Каждая связь между знаками трактуется, с одной стороны, как множество знаков, связываемых этой связью,
				а, с другой стороны, как описание (отражение, модель) соответствующей связи, которая связывает денотаты
				указанных знаков или денотаты одних знаков непосредственно с другими знаками, или сами эти знаки. Примером
				первого вида связи между знаками является связь между знаками материальных сущностей, одна из
				которых является частью другой. Примером второго вида связи между знаками является связь между знаком
				множества знаков и одним из знаком, принадлежащих этому множеству, а также связь между знаком и знаком
				файла, являющегося электронным отражением структуры представления указанного знака во внешних знаковых
				конструкциях. Примерами третьего вида связи между знаками является связь между синонимичными знаками;}
			\item{Денотатами знаков могут быть (1) не только конкретные (константные, фиксированные), но и произвольные
				(переменные, нефиксированные) сущности, \scnqq{пробегающие} различные множества знаков (возможных значений),
				(2) не только реальные (материальные), но и абстрактные сущности (например, числа, точки различных
				абстрактных пространств), (3) не только \scnqq{внешние}, но и \scnqq{внутренние} сущности, являющиеся множествами
				знаков, входящих в состав той же самой знаковой конструкции.}
		\end{scnitemize}}
			
		\scnheader{формализация*}
		\scniselement{бинарное ориентированное отношение}
		\scnidtf{формализация информации*}
		\scnidtf{пара, связывающая менее формализованное и более
			формализованное представление некоторой информации*}
		\scnidtf{формализация информационной модели некоторой описываемой
			(моделируемой) системы взаимосвязанных сущностей*}
		\scnidtf{Бинарное ориентированное отношение, каждая \textit{пара}
			которого, связывает два \textit{семантически эквивалентных} знания, второе из
			которых является более точным (более точно сформированным) знанием по сравнению
			с первым \textit{знанием}*.}
		\scntext{пояснение}{Повышение точности (строгости) формулировки
			знания --- минимизация (а в идеале --- исключение) \uline{неоднозначной}
			семантической интерпретации этой формулировки, т.е. несоответствия того, что
			хотел сказать  автор формулировки, и того, как его поняли. Формализация знаний
			предполагает (1) точное (строгое) описание \textit{синтаксиса и денотационной
				семантики} того \textit{языка}, на котором формулируются \textit{знания} и (2)
			максимально возможное \uline{упрощение} синтаксических и семантических
			принципов, лежащих в основе указанного \textit{языка}. Очевидно, что
			\textit{естественные языки} указанным требованиям не удовлетворяют и,
			следовательно, не могут быть основой для точной формулировки
			\textit{научно-технических знаний} и, соответственно, для представления этих
			\textit{знаний} в \textit{памяти интеллектуальных компьютерных систем}.
			Очевидно также, что разработка \textit{\uline{универсального} языка}
			формального представления научно-технических знаний является \uline{основой}
			для глубокой конвергенции различных научно-технических дисциплин, для
			расширения областей применения современной математики и даже для появления
			новых разделов математики, которые, например, изучают общие свойства
			\textit{универсального смыслового пространства} и, в частности, свойство
			семантического расстояния(семантической близости) как между различными
			\textit{знаками}, так и между различными \textit{знаковыми конструкциями}
			(конфигурациями знаков).}
			\begin{scnindent}
				\scntext{примечание}{Слово математика  означает точное знание .}
					\begin{scnindent}
						\scnrelto{цитата}{\cite{Arnold2012}}
					\end{scnindent}
			\end{scnindent}
		\scntext{примечание}{Формализация информационной модели есть не что иное как
			движение  в сторону семантического (смыслового) представления этой модель, т.е.
			переход к такому представлению этой модели, в котором мы избавляемся от всего,
			не имеющего отношения к сути моделируемой системы и касающегося только способа
			построения этой модели (т.е. её синтаксической структуры). }
		\scntext{примечание}{Нет проблемы записать любое \textit{знание} в
			компьютерную \textit{память}. Для этого надо придумать соответствующий формат
			их кодирования. Но есть проблема представить это \textit{знание} так, чтобы с
			ним было легко работать, чтобы с использованием этого \textit{знания} можно
			было достаточно удобно (без лишних накладных расходов, обусловленных выбранным
			способом представления) решать самые различные информационные \textit{задачи}
			(задачи интеграции знаний, информационного поиска по базе знаний, верификации и
			оптимизации баз знаний, логического вывода, поиска способов решения задач,
			хранимых в базе знаний и т. д.).Какими характеристиками должно обладать удобное
			представление знаний, удовлетворяющее указанным требованиям. Очевидно, что
			такое представление есть не что иное, как формальная (математическая) модель,
			семантически эквивалентная этим знаниям. Т.е. удобно представить знание --- это
			фактически построить соответствующую этому знанию \textit{математическую
				модель}.Для интеллектуальных компьютерных систем важно не просто приобрести
			знания, но и представить их в такой форме, которая была бы удобна не только для
			человека (пользователя и разработчика), но и для различных компьютерных систем,
			т.е. не требовала бы переоформление (перезаписи) этих знаний для различных
			компьютерных систем. Очевидно, что такая форма записи (представления) знаний
			должна быть абсолютно не зависящий от различных компьютерных платформ.Это и
			есть главная цель формализации знаний, обеспечивающей эффективную автоматизацию
			обработки этих знаний.}

		\scnheader{формальное представление информации}
		\scnsubset{информация}
			\begin{scnindent}
				\scnidtf{информационная конструкция}
			\end{scnindent}
		\scntext{вопрос}{Почему разработка и использование формальных моделей
			(математических моделей) представления \textit{информации} является важнейшим
			этапом развития любой научной и научно-технической дисциплины.}
		\begin{scnindent}
			\begin{scnrelfromset}{ответ}
				\scnfileitem{Формализация любой \textit{предметной области} даёт
					возможность более конструктивно накапливать, интегрировать, понимать и
					систематизировать новые \textit{знания} об этой \textit{предметной области}}
				\scnfileitem{Формализация \textit{предметной области} обеспечивает
					более строгую верификацию, обоснование (аргументацию, доказательство) и
					согласование различных точек зрения}
				\scnfileitem{Формализация \textit{предметной области} создает условия
					для разработки строгих и легко воспроизводимых (реализуемых) \textit{методов}
					решения различных \textit{классов задач}}
			\end{scnrelfromset}
		\end{scnindent}
		\scnrelto{достоинства}{формальное представление информации}
		\scnidtf{формальное (формализованное) представление информационной
			конструкции}
		\scnsubset{смысловое представления информации}
		\scntext{примечание}{Высшим уровнем качества \textit{формального
				представления информации} является смысловое представление этой информации}
		\scnidtf{формальная модель системы описываемых взаимосвязанных
			сущностей}
		\scnidtf{математическая модель системы описываемых взаимосвязанных
			сущностей}
		\scnidtf{формула}
		\scntext{примечание}{Сам термин \scnqqi{\textit{формальное представление
				информации}} свидетельствует о том, что при таком представлении
			\textit{информации} сама \uline{форма} представляемой информационной
			конструкции (т.е. синтаксическая структура этой конструкции) имеет очевидную
			аналогию с описываемой конфигурацией связей между соответствующими
			соответствующими описываемыми \textit{сущностями}.В предельном идеальном
			случае указанная аналогия между формой и смыслом информационной конструкции
			должна быть изоморфизмом.}
		\scntext{примечание}{Формализация формализации рознь и, соответственно,
			степень приближения формы представления информации к идеальному  смысловому
			представлению может быть различной. Разработка такого идеального  \textit{языка
				смыслового представления информации} должна руководствоваться следующими
			основными критериями:
			\begin{scnitemize}
				\item максимально возможное упрощения синтаксиса (как можно меньше
				синтаксических излишеств и синтаксического разнообразия);
				\item обеспечение \uline{универсальности} языка.
			\end{scnitemize}
			Подчеркнем, что обеспечение универсальности \textit{языка смыслового
				представления информации} является весьма нетривиальной задачей, т.к. сложно
			одновременно достигнуть две противоречащие друг другу цели- обеспечить простоту
			синтаксиса языка и его неограниченную семантическую мощность. Косвенным
			подтверждением этого является большое количество созданных человечеством
			специализированных \textit{формальных языков}, \textit{языков смыслового
				представления информации} и даже \textit{языков семантических сетей}, что
			свидетельствует о востребованности \textit{смыслового представления
				информации}.}
		\begin{scnsubdividing}
			\scnitem{формальное представление информации, не являющееся смысловым}
			\scnitem{смысловое представление информации, не являющееся
				семантической сетью}
			\scnitem{нерафинированная семантическая сеть}
				\begin{scnindent}
					\scnidtf{смысловое представления информации 2-го уровня}
				\end{scnindent}
			\scnitem{рафинированная семантическая сеть}
				\begin{scnindent}
					\scnidtf{смысловое представление информации 3-го уровня}
				\end{scnindent}
		\end{scnsubdividing}

		\bigskip
		\scnheader{язык смыслового представления информации}
		\begin{scnrelfromlist}{ключевое свойство}
			\scnitem{однозначность представления информации в памяти каждой компьютерной системы}
			\begin{scnindent}
				\scnidtf{отсутствие семантически эквивалентных знаковых	конструкций, принадлежащих 
					смысловому языку и хранимых в одной смысловой памяти}
				\scntext{примечание}{При этом логическая эквивалентность таких знаковых конструкций 
					допускается и используются, например, для компактного представления некоторых знаний,
					хранимых в смысловой памяти.}
			\end{scnindent}
		\end{scnrelfromlist}

		\scnheader{смысловое представление информации, не являющееся семантической сетью}
		\scntext{примечание}{Данному уровню смыслового представления информации
			соответствуют предлагаемые нами универсальные формальные языки SCs-код и SCn-код}
		\scnsuperset{SCs-код}
			\begin{scnindent}
				\scniselement{универсальный формальный язык}
				\scniselementrole{ключевой знак}{Описание языка линейного представления знаний ostis-систем}
			\end{scnindent}
		\scnsuperset{SCn-код}
		\begin{scnindent}
			\scniselement{универсальный формальный язык}
			\scniselementrole{ключевой знак}{Описание языка структурированного представления знаний ostis-систем}
		\end{scnindent}
		\begin{scnreltovector}{принципы, лежащие в основе}
			\scnfileitem{В состав \textit{смыслового представления информации, не
				являющегося семантической сетью} могут входить все уровни иерархии
				представления информационной конструкции --
				\begin{scnitemize}
					\item синтаксически элементарные фрагменты информационной конструкции,
					из которых строятся простые знаки описываемых сущностей, а также разделители и
					ограничители
					\item простые знаки
					\item выражения
					\item простые тексты
					\item сложные тексты
					\item простые знания
					\item сложные знания.
				\end{scnitemize}}

			\scnfileitem{Множество всех описываемых сущностей, \uline{не являющихся
					связями}, разбивается на два подмножества:
				\begin{scnitemize}
					\item каждой сущности, принадлежащей первому подмножеству,
					\uline{взаимно однозначно} соответствует множество \uline{синтаксически
						эквивалентных} (синтаксически одинаковых) \textit{простых знаков}, каждый из
					которых обозначает указанную сущность
					\item каждой сущности, принадлежащей второму подмножеству,
					соответствует в общем случае \uline{семейство} множеств, кажо из которых
					является максимальным множеством синтаксически эквивалентных выражений,
					обозначающих указанную сущность.
				\end{scnitemize}}
				\begin{scnindent}
					\scntext{следовательно}{Здесь синонимия \textit{простых знаков},
						имеющих \uline{разную} синтаксическую структуру, отсутствует, а вот синонимия
						\textit{выражений}, имеющих разную синтаксическую структуру, вполне возможна.
						Подчеркнем при этом, что в рамках \textit{смыслового представления информации,
							не являющегося семантической сетью},
						% \bigspace
						\textit{знаки} (как \textit{простые знаки}, так и \textit{выражения}),
						имеющие одинаковую синтаксическую структуру, считаются также и семантически
						эквивалентными, т.е. обозначающими одну и ту же сущность. Это означает
						отсутствие омонимии синтаксически эквивалентных знаков.}
					\scntext{следовательно}{В рамках \textit{смыслового представления
							информации, не являющегося семантической сетью}, простые знаки, обозначающие
						\uline{разные} сущности, имеют легко устанавливаемое отличие своих
						синтаксических структур, а простые знаки, обозначающие одну и ту же сущность
						имеют легко устанавливаемое сходство своих синтаксических структур. Таким
						образом, в рамках \textit{смыслового представления информации, не являющегося
							семантической сетью},
						% \bigspace
						\uline{дублирование знаков}, т.е. многократное вхождение
						\textit{знаков} одной и той же сущности, \uline{допускается}.}
				\end{scnindent}
			\scnfileitem{Связи как вид описываемых сущностей имеют очень важные
				особенности:
				\begin{scnitemize}
					\item каждой описываемой \textit{связи} \uline{однозначно}, а в
					подавляющем числе случаев и \uline{взаимно однозначно} соответствует
					\textit{простой текст}, являющийся контекстом (спецификацией) этой
					\textit{связи}
					\item весьма редки \textit{кратные связи}, т.е. \textit{свзяи},
					принадлежащие одному и тому же \textit{отношению} и связывающие одинаковым
					образом одни и те же \textit{сущности}
					\item довольно редко \textit{связи} являются компонентами других
					\textit{связей}.
				\end{scnitemize}}
				\begin{scnindent}
					\scntext{следовательно}{Для подавляющего числа описываемых
						\textit{связей} нет никакой необходимости вводить обозначающие их
						\textit{знаки}, если эти \textit{связи} описываются соответствующими
						\textit{простыми текстами}. Вместо таких \textit{знаков} можно ввести условные
						представления этих \textit{связей}, отражающие их вид и направленность. Такие
						условные представления (изображения) описываемых \textit{связей} можно считать
						\textit{знаками}, но \textit{знаками}, семантические свойства которых
						принципиально отличаются от тех \textit{знаков} описываемых \textit{сущностей},
						которые мы рассматривали выше. Любые данного вида разные \textit{знаки}
						описываемых \textit{связей} даже, если, они являются \textit{синтаксически
							эквивалентными}, т.е. имеют одинаковую структуру, считаются \textit{знаками}
						\uline{разных} описываемых \textit{связей}. Синонимия таких \textit{знаков}
						принципиально возможна, но только в том случае, если \textit{простые тексты},
						описывающие соответствующие \textit{связи}, будут полностью
						\uline{продублированы}.}
				\end{scnindent}
			\scnfileitem{Для описания связей между описываемыми сущностями в
				смысловом представлении информации нет необходимости использовать такие приемы
				естественных языков, как склонения, спряжения, семантическая значимость
				последовательности знаков.}
			\scnfileitem{В случае, если с помощью \textit{простых текстов}
				необходимо описать контекст (спецификацию) нескольких \uline{кратных}
				\textit{связей}, все эти \textit{связи} необходимо обозначить \textit{знаками}
				первого типа --- знаками, \textit{синтаксическая структура} каждого из которых
				\uline{уникальна.}Кроме этого, необходимо ввести знак, который обозначает
				\textit{связь инцидентности} между описываемой \textit{связью} и компонентом
				этой \textit{связи}, и который относится к числу \textit{знаков} второго типа
				-- \textit{знаков}, разные экземпляры (разные вхождения) которого считаются
				обозначениями \uline{разных} \textit{связей}}
			\scnfileitem{Для явного указания синонимии двух разных \textit{знаков}
				первого типа, имеющих разную \textit{синтаксическую структуру}, вводится
				фиктивная \textit{связь равенства}, которая сама не является описываемой
				\textit{связью}, а только указывает факт синонимии двух \textit{знаков}, по
				крайней мере один из которых должен быть \textit{выражением}.}
			\scnfileitem{Каждая описываемая \textit{сущность} должна быть
				специфицирована путем указания типа этой \textit{сущности}. Описываемая
				\textit{сущность} может быть:
				\begin{scnitemize}
					\item \textit{материальной сущностью}
					\item \textit{точкой абстрактного пространства}
					\item \textit{множеством}:
						\begin{scnitemizeii}
							\item \textit{связью}
							\item \textit{классом}
							\item \textit{структурой}
						\end{scnitemizeii}
					\item \textit{реальной сущностью}
					\item \textit{вымышленной сущностью}
					\item \textit{константой}
					\item \textit{переменной}
					\item \textit{постоянной сущностью}
					\item \textit{временной сущностью}:
						\begin{scnitemizeii}
							\item \textit{прошлой сущностью}
							\item \textit{настоящей сущностью}
							\item \textit{будующей сущностью}.
						\end{scnitemizeii}
				\end{scnitemize}
				Кроме того, сам \textit{знак} описываемой сущности может иметь
				следующий статус:
				\begin{scnitemize}
					\item \textit{логически удаленный знак}
					\item \textit{настоящий знак}
					\item \textit{будущий знак}.
				\end{scnitemize}
			}
			\scnfileitem{Возможно дублирование информации, т.е. могут
				присутствовать семантически эквивалентные информационные конструкции, входящие
				в остав одной информационной конструкции (например, в состав информации,
				хранимой в памяти одной компьютерной системы). Но при этом есть принципиальная
				возможность обнаружить такое дублирование информации.}
		\end{scnreltovector}

		\bigskip
		\scnheader{графовая структура}
		\scnidtftext{определение}{абстрактная (математическая) структура, которая задается:
			\begin{scnitemize}
				\item множеством ее элементов:
				            \begin{scnitemizeii}
					            \item множеством ее вершин (узлов);
					            \item множеством ее связок:
					                        \begin{scnitemizeiii}
						                        \item множеством ее ребер (неориентированных пар
						                        элементов графовой структуры);
						                        \item множеством ее дуг (ориентированных пар элементов
						                        графовой структуры);
						                        \item множеством ее гиперребер, каждое из которых
						                        является конечным множеством элементов графовой структуры, имеющим мощность
						                        больше двух
					                        \end{scnitemizeiii}
				            \end{scnitemizeii}
				\item бинарным ориентированным отношением инцидентности, связывающим
				каждую связку графовой структуры с каждым компонентом (элементом) этой связки.
			\end{scnitemize}
		}
		\scnheader{следует отличать*}
		\begin{scnhaselementset}
			\scnfileitem{\textit{графовую структуру} как абстрактный математический
				объект, в рамках которого не уточняется то, как выглядят (представляются,
				изображаются) элементы графовой структуры и связи их инцидентности}
			\scnfileitem{представление (изображение) \textit{графовой структуры} --
				ее рисунок, ее представление в компьютерной памяти в виде матрицы
				инцидентности, матрицы смежности, списковой структуры}
		\end{scnhaselementset}

		\scnheader{графовая структура}
		\scnidtftext{часто используемый sc-идентификатор}{дискретная
			информационная конструкция}
		\scntext{примечание}{Поскольку любая \textit{графовая структура} является
			дискретной математической моделью, которая может описывать любое множество
			\textit{сущностей}, связанных между собой заданным множеством \textit{связей},
			все \textit{графовые структуры} с полным основанием можно считать дискретными
			\textit{информационными конструкциями}. Более того, любая дискретная
			\textit{информационная конструкция} (в том числе, и обычная цепочка символов) с
			формальной точки зрения является \textit{графовой структурой}. Тот факт, что
			теория графов рассматривает синтаксические  свойства \textit{графовых структур}
			с точностью до их изоморфизма, не лишает \textit{графовые структуры}
			соответствующих семантических  свойств.}
		\scntext{пояснение}{С семантической точки зрения графовая структура
			-- это нелинейная (в общем случае) знаковая конструкция, в состав которой могут
			входить знаки \uline{любых} сущностей. При этом указанные знаки
			\uline{синтаксически} разбиваются на два класса --
			\begin{scnitemize}
				\item на \textit{знаки} сущностей, которые не являются \uline{связями}
				между сущностями --- в теории графов такие знаки называются узлами (вершинами);
				\item на знаки \uline{связей} между \textit{сущностями} --- к таким
				\textit{знакам} относятся ребра неориентированных графов, гиперребра
				гиперграфов, дуги ориентированных графов.
			\end{scnitemize}
			Кроме того, на множестве знаков \textit{сущностей}, входящих в состав
			\textit{графовой структуры}, задаются \textit{отношения инцидентности}, которые
			связывают \textit{знаки} связей, входящих  в состав \textit{графовой
			структуры}, со знаками тех \textit{сущностей} которые являются компонентами
			указанных \textit{связей}.
			
			Теория графов рассматривает только синтаксические
			аспекты \textit{графовых структур}.Семантика \textit{графовой структуры}
			задается \textit{онтологией}, специфицирующей систему понятий, экземплярами
			которых являются элементы этой графовой структуры, т.е. \textit{знаки},
			входящие в состав этой \textit{графовой структуры}.}

		\scnheader{семантическая сеть}
		\scnidtf{\textit{графовая структура}, являющаяся \uline{формальным
				уточнением} одного из видов \textit{смыслового представления информации}}
		\scnsubset{графовая структура}
		\scnsubset{смысловое представление информации}
		\begin{scnindent}
			\scnsubset{знаковая структура}
		\end{scnindent}
		\scnidtf{графовая структура, \uline{вершины} (узлы) которой трактуются
			как знаки некоторых описываемых сущностей, а \uline{связки} (ребра, дуги,
			гиперребра, гипердуги) которой трактуются как знаки связей между описываемыми
			сущностями и/или знаками этих сущностей}
		\scnidtf{\uline{абстрактная} графовая и в общем случае нелинейная
			знаковая конструкция (знаковая структура), являющаяся вариантом
			\uline{смыслового} представления соответствующей информации}
		\scnidtftext{explanation}{информационная конструкция, в которой
			\uline{явно} выделены знаки \uline{всех} описываемых сущностей, а также знаки
			связей, которые также считаются описываемыми сущностями и которые связывают
			либо сами описываемые сущности, либо описываемые сущности со знаками других
			описываемых сущностей, либо знаки описываемых сущностей}
		\scntext{примечание}{Теоретико-графовая трактовка (уточнение)
			\textit{смыслового представления информации} является вполне естественной, т.к.
			любая описываемая сущность может иметь неограниченное количество связей с
			другими описываемыми сущностями, и очень часто анализ свойств какой-либо
			описываемой сущности предполагает анализ всех представленных (описанных) связей
			этой сущности с различными другими сущностями. Более того, для любых
			описываемых сущностей существует связывающая их связь (все в Мире
			взаимосвязано). Вопрос в том, какая это связь и нужно ли ее описывать. Далеко
			не все то, что можно описывать, целесообразно описывать.}
		\begin{scnrelfromvector}{общие предпосылки}
			\scnfileitem{Информация в знаковой конструкции содержится не в самих
				знаках, а в конфигурации связей между знаками, обозначающими описываемые
				сущности}
			\scnfileitem{Конфигурация связей между описываемыми сущностями \uline{в
					общем случае} \uline{не} являются линейной}
			\scnfileitem{Идеальным \textit{смысловым представлением информации}
				следует считать такую знаковую конструкцию, синтаксическая конфигурация связей
				между знаками которой \uline{изоморфна} конфигурации связей между описываемыми
				сущностями}
		\end{scnrelfromvector}

		\scntext{примечание}{Понятие семантической сети является основным понятием
			для \textit{Технологии OSTIS}. Ранее семантические сети рассматривались не как
			основа технологии разработки интеллектуальных компьютерных систем, а как
			наглядная иллюстрация представления знаний, не имеющая практической перспективы
			из-за сложности реализации, не обладающая универсализмом.Для нас семантические
			сети --- это
			\begin{scnitemize}
				\item формальный подход к построению знаковых конструкций;
				\item формальный подход, позволяющий создавать целое \uline{семейство}
				языков и в том числе языков \uline{универсальных};
				\item основа организации памяти нового типа --- структурно
				перестраиваемой (реконфигурируемой) памяти, обработка информации в которой
				сводится к реконфигурации связей между ее элементами.
			\end{scnitemize}}
		\begin{scnrelfromlist}{достоинства}
			\scnfileitem{\textit{Семантическая сеть} наряду с системами правил
				является весьма распространенным способом представления знаний в
				интеллектуальных системах. Особое значение этот способ представления знаний
				приобретает в связи с развитием сети интернет. Кроме ряда особенностей,
				позволяющих применять семантические сети в тех случаях, когда системы правил не
				применимы, \textit{семантические сети} обладают следующим важным свойством: они
				дают возможность \uline{соединения в одном представлении синтаксиса и
					семантики} или синтаксического и семантического аспектов описаний знаний
				предметной области. Происходит это благодаря тому, что в семантических сетях
				наряду с переменными для обозначения тех или иных объектов (элементов множеств,
				некоторых конструкций из них) присутствуют и сами эти элементы и конструкции присутствуют и связи, сопоставляющие тем или иным переменным
				множества допустимых интерпретаций. Эти обстоятельства позволяют во многих
				случаях резко \uline{уменьшить реальную вычислительную сложность решаемых
					задач}.
				\newline Помимо изобразительных возможностей, семантические сети
				обладают более серьезными достоинствами. То обстоятельство, что \uline{вся
					информация об индивиде} представлена в единственном месте --- в одной вершине --
				означает, что вся эта информация непосредственно доступна в этой вершине, что,
				в свою очередь, \uline{сокращает время поиска}, в частности, при выполнении
				унификации и подстановки в задачах логического вывода.Существует еще одна,
				более тонкая особенность расширенных семантических сетей --- они позволяют
				интегрировать в одном представлении \textit{синтаксис} и \textit{семантику}
				(т.е. интерпретацию) клаузальных форм. Это позволяет в процессе вывода
				обеспечивать взаимодействие синтаксических и семантических, теоретико-модельных
				подходов, что, в свою очередь, также является фактором, зачастую делающим вывод
				практически более эффективным}
				\begin{scnindent}
				\scnrelto{цитата}{Осипов.Г.С.МетодыИИ-2015кн,с.43-54}
					\begin{scnindent}	
						\scnrelto{часть}{\cite{Osipov2015}}
					\end{scnindent}
				\end{scnindent}
			\scnfileitem{Все связи между \textit{знаками}, входящими в состав
				\textit{семантической сети} представляются с помощью специальных связующих
				элементов \textit{семантической сети} (дуг, ребер) и, следовательно, для
				описания указанных связей в \textit{семантической сети} нет необходимости
				использовать такие средства, как предлоги, союзы, падежи, склонения, спряжения,
				различные разделители и ограничители, что существенно упрощает обработку
				\textit{знаний}.}
			\scnfileitem{Соединение синтаксических и семантических аспектов в
				\textit{семантической сети} проявляется в том, что дуга или ребро,
				синтаксически  соединяющая элементы \textit{семантической сети} описывает
				наличие соответствующей \textit{связи} между \textit{сущностями}, обозначаемыми
				указанными элементами \textit{семантической сети}.}
		\end{scnrelfromlist}

		\bigskip
		\scnheader{нерафинированная семантическая сеть}
		\scntext{примечание}{Переход от смыслового представления информации, не
			являющегося семантической сетью, к нерафинированным семантическим сетям
			представляет собой переход к информационным конструкциям, имеющим более простую
			синтаксическую структуру и денотационную семантику. К нерафинированным
			семантическим сетям можно отнести тексты предлагаемого нами универсального
			формального SCg-кода, а также используемые в Semantic Web
			rdf-графы}	
		\scnsuperset{SCg-код}
			\begin{scnindent}
				\scnhaselement{универсальный формальный язык}
				\scnhaselementrole{ключевой знак}{Описание языка графического представления знаний ostis-систем}
			\end{scnindent}
		\scnsuperset{rdf-граф}
		\begin{scnreltovector}{принципы, лежащие в основе}
			\scnfileitem{Поскольку в \textit{информационной конструкции} информация
				содержится не в самих \textit{знаках} (если не считать \textit{знаки},
				являющиеся \textit{выражениями}), а в конфигурации связей между
				\textit{знаками}, очень важно \uline{явно} формально представить саму эту
				конфигурацию \textit{знаков}. И как нельзя лучше для этого подходит понятие
				\textit{графовой структуры} и, соответственно, понятие \textit{семантической
					сети}.\\Что касается \textit{выражений}, то каждое из них легко
				трансформируется в \textit{семантически эквивалентную} информационную
				конструкцию, не являющиюся \textit{выражением}. Заметим, что \textit{выражения}
				используются исключительно для минимизации числа вводимых \textit{знаков}
				(имен) с уникальной синтаксической структурой.}
			\scnfileitem{\uline{Все} элементы, входящие в состав нерафинированной
				семантической сети и представленные узлами, ребрами или дугами, являются
				\textit{знаками}, обозначающими соответствующие описываемые \textit{сущности},
				причём \textit{знаками} второго типа, которые, обозначая соответствующую
				\textit{сущность}, входят в \textit{информационную конструкцию}
				\uline{однократно} (отсутствует многократное вхождение \textit{знаков},
				обозначающих одну и ту же \textit{сущность}). Также \textit{знаки} могут иметь
				синтаксическую структуру, которая не является уникальной для обозначаемой
				\textit{сущности}, а отражает только принадлежность этой сущности к
				соответствующих классам.Таким образом, в \textit{нерафинированной семантической
					сети} в отличие от \textit{смыслового представления информации не являющегося
					семантической сетью}, доминируют не \textit{знаки} первого типа, а
				\textit{знаки} второго типа, которыми в \textit{нерафинированной семантической
					сети} представлены (обозначены) \uline{все} описываемые \textit{сущности}, а в
				\textit{смысловом представлении информации, не являющемся семанитеской сетью},
				представлены \uline{только} \textit{бинарные связи} \uline{и то не все}.}
			\scnfileitem{\uline{Все} ребра \textit{нерафинированной семантической
					сети} являются знаками \textit{бинарных неориентированных связей} и формально
				трактуются как знаки \textit{двухмощных множеств}, каждым \textit{элементом}
				которых являются либо знак \textit{сущности}, соединяемой указанной
				\textit{бинарной связью}, либо \textit{знак}, который сам является
				\textit{сущностью}, соединяемой этой \textit{бинарной связью}. Более того,
				\uline{все} \textit{двухмощные множества}, не являющиеся \textit{кортежами}
				(ориентированными парами) в \textit{нерафинированной семантической сети}
				обозначаются \textit{ребрами} этой сети.}
			\scnfileitem{\uline{Все} дуги \textit{нерафинированной семантической
					сети} являются знаками \textit{бинарных ориентированных связей} и формально
				трактуются как знаки \textit{двухмощных кортежей} (ориентированных пар), каждым
				\textit{элементом} которых является либо знак \textit{сущности}, соединяемой
				указанной \textit{бинарной связи}, либо \textit{знак}, который сам является
				\textit{сущностью}, соединяемой этой \textit{бинарной связью}. Более того,
				\uline{все} \textit{ориентированные пары} в \textit{нерафинированной
					семантической сети} обозначаются \textit{дугами} этой сети.}
			\scnfileitem{\uline{Каждая} небинарная связь, описываемая в
				нерафинированной семантической сети, трактуется как множество, мощность
				которого не равна двум и обозначается соответствующим узлом этой сети, который
				соединяется дугами, принадлежащими отношению принадлежности со всеми знаками,
				которые либо обозначаются сущности, связывающие рассматриваемой небинарной
				связью, либо сами являются такими сущностями. Для описания ориентированных
				небинарных связей (в частности, небинарных кортежей) выделяется несколько
				подмножеств отношения принадлежности, соответствующих различным ролям элементов
				(компонентов) ориентированных небинарных связей.}
			\scnfileitem{В рамках нерафинированной семантической сети \uline{все}
				рассматриваемые связи между описываемыми сущностями представляются \uline{явно}
				в виде знаков, обозначающих эти связи.}
			\scnfileitem{В рамках нерафинированной семантической сети не
				используются такие средства, как разделители, ограничители и др.}
			\scnfileitem{Узлами \textit{нерафинированной семантической сети},
				которые обозначают различного вида \uline{ключевые} описываемые
				\textit{сущности} (прежде всего, различные \textit{понятия}) приписываются
				уникальные \textit{знаки} (имена) этих \textit{ключевых сущностей}. Очевидно,
				что каждый такой \textit{узел} и приписываемое ему \textit{имя} --- это
				\textit{синонимичные знаки}, обозначающие одну и ту же \textit{сущность}, но
				являющиеся \textit{знаками} двух разных типов --- (1) \textit{знаком}, который
				\uline{однократно} представлен в рамках \textit{информационной конструкции}
				(2) \textit{знаком}, синтаксическая структура которого \uline{взаимно
					однозначно} соответствует обозначаемой им \textit{сущности}.}
			\scnfileitem{Большинству узлов, обозначающих небинарные связи,
				большинству ребер и дуг, а также некоторым другим узлам нерафинированной
				семантической сети могут быть приписаны уникальные знаки (в частности, имена)
				понятий (чаще всего, отношений), которым принадлежат указанные узлы, ребра и
				дуги.}
		\end{scnreltovector}

		\bigskip
		\scnheader{рафинированная семантическая сеть}
		\scntext{основной принцип}{Абсолютно ничего лишнего, не имеющего
			отношения к смыслу представляемой информации}
		\scnidtf{\uline{предельно} компактная (сжатая) смысловая информационная
			модель соответствующей системы рассматриваемых (описываемых, исследуемых,
			моделируемых) сущностей}
		\scntext{примечание}{Указанная система рассматриваемых сущностей представляет
			собой конфигурацию связей между этими сущностями. Подчеркнем при этом, что
			указанные связи между рассматриваемыми сущностями также входят в число
			рассматриваемых сущностей.}\scnidtf{\textit{информационная конструкция},
			являющаяся результатом максимально возможного упрощения ее
			\textit{синтаксической структуры} при обеспечении представления \uline{любой}
			\textit{информации}, что приводит к фактическому слиянию синтаксических и
			семантических аспектов представления \textit{информации}}
		\scnidtf{\textit{семантическая сеть} внутреннего\ потребления,
			используемая для \textit{смыслового представления информации} в памяти
			\textit{компьютерных систем}}
		\scnidtf{уточнение принципов \textit{смыслового представления
				информации}, которое основано, \uline{во-первых}, на четком противопоставлении
			\textit{внутреннего языка компьютерной системы}, используемого для хранения
			информации в памяти компьютера, и \textit{внешних языков компьютерной системы},
			используемых для общения (обмена сообщений) \textit{компьютерной системы} с
			пользователями и другими \textit{компьютерными системами} (рафинированная
			семантическая сеть используется исключительно для \textit{внутреннего
				представления информации} в памяти \textit{компьютерной системы}), и,
			\uline{во-вторых} на максимально возможном упрощении \textit{синтаксиса
				внутреннего языка компьютерной системы} при обеспечении \uline{универсальности}
			путем исключения из такого внутреннего универсального языка средств,
			обеспечивающих коммуникационную функцию \textit{языка} (т.е. обмен
			сообщениями).
			\newline Так, например, для \textit{внутреннего языка компьютерной
				системы} излишними являются такие коммуникационные средства \textit{языка}, как
			союзы, предлоги, разделители, ограничители, склонения, спряжения и другие.
			\newline\textit{Внешние языки компьютерной системы} могут быть как
			близки ее внутреннему языку, так и весьма далеки от него (как, например,
			\textit{естественные языки}).}
		\scnidtf{\uline{абстрактная} знаковая конструкция, принадлежащая
			\uline{универсальному} внутреннему языку компьютерных систем и являющаяся
			\uline{инвариантом} соответствующего максимального множества семантически
			эквивалентных знаковых конструкций (текстов), принадлежащих самым различным
			языкам}

		\begin{scnrelfromvector}{принципы лежащие в основе}
			\scnfileitem{Каждый фрагмент \textit{рафинированной семантической сети}
				является либо \textit{знаком} (элементарным фрагментом, представленным либо
				\textit{узлом}, либо \textit{ребром}, либо \textit{дугой}), либо множеством
				\textit{знаков}, связанных между собой отношением \textit{инцидентности}
				элементов \textit{рафинированной семантической сети}. Указанное отношение
				\textit{инцидентности} является \textit{бинарным ориентированным отношением},
				связывающим \textit{знаки} описываемых \textit{связей} (которые представляются
				\textit{ребрами}, \textit{дугами} и \textit{узлами}, если описываемая связь
				является небинарной) со \textit{знаками}, которые либо обозначают связываемые
				\textit{сущности}, либо сами являются такими сущностями}
				\begin{scnindent} 
					\scntext{следовательно}{В состав \textit{рафинированной
							семантической сети} не входят такие средства синтаксической структуризации
						знаковых конструкций, как \textit{разделители} и \textit{ограничители}. Любая
						структуризация \textit{рафинированных семантических сетей} описывается явно с
						помощью метаязыковых средств путем:
						\begin{scnitemize}
							\item введения узлов \textit{рафинированной семантической
							сети}, обозначающих различные \uline{не\-э\-ле\-мен\-тар\-ные} фрагменты этой
							семантической сети, являющиеся \textit{множествами} узлов, ребер и дуг,
							входящих в состав обозначаемого фрагмента
							\item введения \textit{дуг принадлежности}, связывающих
							введенные \textit{узлы}, обозначающие неэлементарные фрагменты
							\textit{рафинированной семантической сети}, с элементами обозначаемых ими \textit{множеств}
							\item введения целого ряда \textit{отношений}, связывающих
							неэлементарные фрагменты \textit{рафинированной семантической сети} с другими
							фрагментами, а также с сущностями других видов
							\item введения различных классов неэлементарных фрагментов
							\textit{рафинированной семантической сети}.
						\end{scnitemize}}
				\end{scnindent}
			\scnfileitem{Абсолютно все \textit{знаки}, входящие в состав
				\textit{рафинированной семантической сети}, являются синтаксически
				элементарными (атомарными) фрагментами \textit{рафинированной семантической
					сети}, т.е. фрагментами, внутренняя\ структура которых не имеет никакого
				значения для семантического анализа и понимания \textit{рафинированной
					семантической сети}. Множеству \textit{знаков}, входящих в
				\textit{рафинированную семантическую сеть}, как и множеству \textit{букв},
				входящих в обычный \textit{текст}, ставится в соответствие \textit{алфавит},
				определяющий \uline{синтаксическую типологию} таких элементарных фрагментов
				\textit{рафинированной семантической сети}. При этом, если \textit{алфавит}
				букв обычного \textit{текста} не имеет никакой семантической интерпретации, то
				\textit{алфавит} элементарных фрагментов \textit{рафинированной семантической
					сети} имеет четкую семантическую интерпретацию --- каждый элемент этого
				\textit{алфавита} обозначает класс знаков \textit{сущности},
				\uline{синтаксический тип} которых соответствует указанному элементу
				\textit{алфавита} (задается этим элементом \textit{алфавита знаков}, входящих в
				состав \textit{рафинированной семантической сети}).}
				\begin{scnindent}
					\scntext{следовательно}{Таким образом, \textit{знаки}, входящие
						в \textit{рафинированную семантическую сеть}, не являются \textit{именами}
						(терминами) и, следовательно, не привязаны ни к какому \textit{естественному
							языку} и не зависят от субъективных терминотворческих пристрастий различных
						авторов. Это значит, что при коллективной разработке \textit{рафинированных
							семантических сетей}, соответствующих каким-либо информационным ресурсам,
						терминологические споры практически исключены.}
					\scntext{следовательно}{В \textit{рафинированной семантической
							сети} нет необходимости использовать синтаксически элементарные фрагменты,
						\uline{не} являющиеся знаками описываемых \textit{сущностей}, т.е. фрагменты
						\textit{информационной конструкции}, из которых сторятся \textit{простые
							знаки}, \textit{выражения}, а также различные разделители и ограничители. Более
						того, в \textit{рафинированной семантической сети} нет необходимости
						противопоставлять \textit{простые знаки} и \textit{выражения}. Как
						\textit{простым знакам}, так и \textit{выражениям} в \textit{рафинированной
							семантической сети} соответствуют элементы этой сети, имеющие аналогичные
						\textit{денотаты}. Но при этом \textit{выражениям} дополнительно соответствуют
						семантически эквивалентные неэлементарные фрагменты \textit{рафинированной
							семантической сети}, которые специфицируют \textit{сущности}, обозначаемые
						этими \textit{выражениями}.}
				\end{scnindent}
			\scnfileitem{Абсолютно ве описываемые \textit{связи} между описываемыми
				сущностями в \textit{рафинированной семантической сети} представляются
				\uline{явно} в виде соответствующих \textit{знаков}, обозначающих эти
				\textit{связи} и инцидентных знакам связываемых \textit{сущностей}. Для
				бинарных связей, связывающих \uline{две} описываемые сущности, \textit{знаком}
				связей являются \textit{ребра} или \textit{дуги} \textit{рафинированной
					семантической сети}.}
				\begin{scnindent}
					\scntext{следовательно}{В \textit{рафинированных семантических
							сетях} нет необходимости использовать такие средства, как склонения, спряжения,
						род (мужской, женский, средний), семантически интерпретируемая
						последовательность слов.}
				\end{scnindent}
			\scnfileitem{Все \textit{знаки}, входящие в состав
				\textit{рафинированной семантической сети}, входят в нее \uline{однократно}.
				Т.е. в рамках \textit{рафинированной семантической сети} отсутствуют пары
				\textit{синонимичных знаков}, т.е. \textit{знаков}, имеющих один и тот же
				\textit{денотат}. Таким образом, разные элементы \textit{рафинированной
					семантической сети} априори считаются знаками \uline{разных} сущностей. При
				этом эти знаки могут принадлежать одному и тому же синтаксическому типу, т.е.
				одному и тому же элементу алфавита соответствующего языка
				\textit{рафинированных семантических сетей}. Таким образом, в
				\textit{рафинированных семантических сетях} отсутствует синонимия не только
				\textit{знаков}, имеющих одинаковую синтаксическую структуру, не только знаков,
				имеющих одинаковый синтаксический тип, но также и просто \uline{разных} знаков.}
				\begin{scnindent}
					\scntext{следовательно}{Появление в рафинированной
						семантической сети синонимичных знаков превращает эту семантическую сеть в
						некорректную и требует отождествления (склеивания) обнаруженных синонимичных
						знаков.}
				\end{scnindent}
			\scnfileitem{В рамках \textit{рафинированной семантической сети}
				отсутствуют \textit{синонимичные знаки}, т.е. \textit{знаки}, которые имеют не
				один, а несколько \textit{денотатов}, каждому из которых соответствует свой
				контекст (ракурс) семантической трактовки этого \textit{знака}.}
				\begin{scnindent}
					\scntext{примечание}{Когда речь идет об омонимии знаков в привычных
						нам языках, имеется в виду омонимия \uline{разных} знаков, имеющих одинаковую
						синтаксическую структуру, т.е. омонимия разных вхождений, разных экземпляров
						\uline{синтаксически эквивалентных}, но семантически различных знаков.
						Очевидным примером такого рода омонимии являются различного вида местоимения.}
				\end{scnindent}
			\scnfileitem{В рамках каждой \textit{рафинированной семантической сети}
				отсутствует дублирование информации не только в виде многократного вхождения
				\textit{синонимичных знаков}, т.е. \textit{знаков} с одинаковыми денотатами, но
				также и в виде многократного вхождения \textit{семантически эквивалентных}
				\textit{рафинированных семантических сетей}. Две \textit{рафинированные
					семантические сети} являются \textit{семантически эквивалентными} в том и
				только в том случае, если:
				\begin{scnitemize}
					\item они \textit{изоморфны}
					\item пары соответствия указанного \textit{изоморфизма}
					связывают \textit{синонимичные знаки}.
				\end{scnitemize}
				Таким образом, полное исключение \textit{омонимии знаков}
				является необходимым и достаточным условием исключения \textit{семантически
					эквивалентных рафинированных семантических сетей}. Подчеркнем при этом, что
				запрет \textit{семантической эквивалентности} в рамках \textit{рафинированной
					семантической сети} не означает запрета \textit{логической эквивалентности}
				фрагментов \textit{рафинированной семантической сети}. Логическая
				эквивалентность необходима для обеспечения компактности представления некоторых
				знаний. Тем не менее, логической эквивалентностью хранимых в памяти знаковых
				конструкций увлекаться не следует, т.к. \uline{\textit{логически
						эквивалентные}} знаковые конструкции --- это представление одного и того же
				\textit{знания}, но с помощью \uline{\textit{разных наборов понятий}}. В
				отличие от этого \uline{\textit{семантически эквивалентные}} \textit{знаковые
					конструкции} --- это представление одного и того же \textit{знания} с помощь
				одних и тех же \textit{понятий}. Очевидно, что многообразие возможных вариантов
				представления одних и тех же \textit{знаний} в памяти компьютерной системы
				существенно усложняет решение \textit{задач}. Поэтому, полностью исключив
				\textit{семантическую эквивалентность} в смысловой памяти, необходимо
				стремиться к минимизации \textit{логической эквивалентности}. Для этого
				необходимо грамотное построение системы используемых \textit{понятий} в виде
				иерархической системы формальных \textit{онтологий}.}
				\begin{scnindent}
					\scntext{следовательно}{Интеграция (соединение, объединение)
						двух \textit{рафинированных семантических сетей}, в результате чего могут
						появиться семантически эквивалентные фрагменты, сводится к тому, чтобы
						результат такого соединения был приведен в соответствие с требованием
						отсутствия синонимии элементов и семантической эквивалентности фрагментов
						\textit{рафинированной семантической сети}.}
				\end{scnindent}
			\scnfileitem{\textit{Рафинированные семантические сети} должны быть
				\uline{универсальными}, т.е. должны обеспечивать представление \uline{любой}
				информации, в том числе, и \textit{метаинформации}, обеспечивающей описание
				различных связей, свойств и закономерностей самих \textit{рафинированных
					семантических сетей}, на множестве которых, в частности, заданно
				\textit{отношение} быть подструктурой*\, которое связывает
				\textit{рафинированные семантические сети} с их фрагментами (частями), т.е. с
				теми \textit{рафинированными семантическими сетями}, которые входят в их
				состав.\newline Каждая \textit{рафинированная семантическая сеть} трактуется
				как множество \textit{знаков} \uline{взаимно однозначно} соответствующих
				обозначаемым ими \textit{сущностям} (денотатам этих \textit{знаков}) и
				множество \textit{связей} между этими \textit{знаками}.\newline Каждая
				\textit{связь} между \textit{знаками} трактуется, с одной стороны, как
				множество \textit{знаков}, связываемых этой \textit{связью}, а, с другой
				стороны, как описание (отражение, модель) соответствующей \textit{связи},
				которая связывает денотаты указанных \textit{знаков} или денотаты одних
				\textit{знаков} непосредственно с другими \textit{знаками}, или сами эти
				\textit{знаки}. Примером первого вида \textit{связи} между \textit{знаками}
				является связь между \textit{знаками} \textit{материальных сущностей}, одна из
				которых является частью другой. Примером второго вида \textit{связи} между
				\textit{знаками} является \textit{связь} между знаком, входящим в состав
				внутреннего смыслового представления информации, и знаком файла, являющегося
				электронным отражением структуры представления указанного \textit{знака} во
				внешних \textit{знаковых конструкциях}. Примерами третьего вида \textit{связи}
				между \textit{знаками} является \textit{связь} между синонимичными
				знаками.\newline Денотатами \textit{знаков} могут быть \uline{любые}
				описываемые сущности, причем: (1) не только конкретные (константные,
				фиксированные), но и произвольные (переменные, нефиксированные)  сущности,
				пробегающие\ различные множества знаков (возможных значений), (2) не только
				реальные (материальные), но и абстрактные сущности (например, числа, точки
				различных абстрактных пространств), (3) не только внешние\, но и внутренние
				сущности, являющиеся множествами знаков, входящих в состав той же самой
				знаковой конструкции, хранимой в памяти компьютерной системы.}
			\scnfileitem{Поскольку \textit{рафинированные семантические сети}
				ориентированы на \textit{смысловое представление информации} в памяти
				\textit{компьютеров нового поколения}, необходимо, с одной стороны,
				использовать накопленный полезный опыт представления информации в
				\textit{современных компьютерах}, а, с другой стороны, обеспечить
				взаимодействие \textit{компьютерных систем}, построенных на \textit{современных
					компьютерах}, с \textit{компьютерными системами}, построенными на
				\textit{компьютерах нового поколения}. Для этой цели в памяти
				\textit{компьютеров нового поколения} можно и нужно обеспечить обработку и
				хранение различного вида \textit{информационных конструкций}, представленных в
				различных широко используемых форматах. И ничто не препятствует такие
				\textit{информационные конструкции}, хранимые в памяти \textit{компьютера
					нового поколения} и не являющиеся \textit{рафинированными семантическими
					сетями}, рассматривать как \textit{сущности}, описываемые
				\textit{рафинированной семантической сетью}, хранимой в памяти этого
				\textit{компьютера нового поколения}. Такой вид \textit{сущностей}, описываемых
				\textit{рафинированной семантической сетью} и хранимых в той же
				\textit{памяти}, будем называть \textit{файлами}, описываемыми соответствующуей
				\textit{рафинированной семантической сетью}, т.е. электронными	\ образами
				(копиями) соответствующих \textit{информационных конструкций}. Таким образом,
				среди \textit{узлов рафинированной семантической сети} появляются
				\textit{узлы}, являющиеся знаками \textit{файлов}, т.е. \textit{узлы}, денотаты
				(обозначаемые \textit{сущности}) которых находятся (хранятся) в той же памяти,
				что и обозначающие их \textit{узлы}.}
				\begin{scnindent}
					\scntext{следовательно}{Ничто не мешает в виде \textit{файла},
						описываемого \textit{рафинированной семантической сетью}, хранить \textit{имя}
						(термин) какой-либо \textit{сущности}, описываемой этой же семантической сетью,
						а также связать это \textit{имя} (точнее, узел, обозначающий это \textit{имя})
						с тем элементом \textit{рафинированной семантической сети}, который обозначает
						ту же описываемую \textit{сущность}.}
				\end{scnindent}
			\scnfileitem{Следствием указанных принципов \textit{рафинированных
					семантических сетей} является также то, что эти принципы приводят к нелинейным
				\textit{знаковым конструкциям} (к \textit{графовым структурам}), что усложняет
				реализацию \textit{памяти компьютерных систем}, но существенно упрощает ее
				логическую организацию (в частности, ассоциативный доступ).
				\newline Нелинейность \textit{рафинированных семантических
					сетей} обусловлена тем, что:
				\begin{scnitemize}
					\item каждая описываемая \textit{сущность}, т.е.
					\textit{сущность}, имеющая соответствующий ей \textit{знак}, может иметь
					неограниченное число \textit{связей} с другими описываемыми \textit{сущностями}
					\item каждая описываемая \textit{сущность} в смысловом
					представлении имеет единственный \textit{знак}, т.к. синонимия \textit{знаков}
					здесь запрещена
					\item все \textit{связи} между описываемыми \textit{сущностями}
					описываются (отражаются, моделируются) \textit{связями} между \textit{знаками}
					этих описываемых \textit{сущностей}.
				\end{scnitemize}}
				\begin{scnindent}
					\scntext{примечание}{Напомним, что нелинейность информационных
						конструкций характерна не только для рафинированных, но и для нерафинированных
						семантических сетей.}
				\end{scnindent}
		\end{scnrelfromvector}

		\scnsuperset{SC-код}
		\begin{scnindent}
			\scnidtf{Semantic Computer Code}
			\scniselement{универсальный формальный язык}
			\scniselementrole{ключевой знак}{Описание внутреннего языка
				ostis-сиcтем}

			\scntext{пояснение}{В качестве \textit{стандарта}
				\uline{универсального} \textit{смыслового представления информации} \textit{в
					памяти компьютерных систем} нами предложен SC-код (Semantic Computer Code). В
				отличие от УСК \textit{Мартынова В.В.}, он, во-первых, носит нелинейный
				характер и, во-вторых, специально ориентирован на кодирование информации в
				памяти компьютеров \uline{нового поколения}, ориентированных на разработку
				семантически совместимых \textit{интеллектуальных компьютерных систем} и
				названных нами \textit{семантическими ассоциативными компьютерами}. Более
				подробно это понятие (\textit{SC-код}) рассмотрено в разделе \textit{Предметная
					область и онтология внутреннего языка osts-систем}. Таким образом, основым
				лейтмотивом предлагаемого нами \textit{смыслового представления информации}
				является ориентация на формальную модель памяти \textit{компьютерных}
				\uline{не}фон-неймановского \textit{компьютера}, предназначенного для
				реализации \textit{интеллектуальных систем}, использующих \textit{смысловое
					представление информации}. Особенностями такого представления являются
				следующие:
				\begin{scnitemize}
					\item ассоциативность памяти;
					\item поскольку при смысловом представлении информациия содержится в
					конфигурации связей между знаками, переработка информации сводится к
					реконфигурации этих связей (к графодинамическим процессам);
					\item прозрачная семантическая интерпретируемость и, как следствие,
					\textit{семантическая совместимость}.
				\end{scnitemize}
				Подчеркнем что, неявная привязка к фон-неймановским
				\textit{компьютерам} присутствует во всех известных \textit{моделях
					представления знаний}. Одним из примеров такой зависимости, является, например,
				обязательность именования описываемых объектов.}
		\end{scnindent}
			\begin{scnrelfromset}{достоинства}
				\scnfileitem{рафинированная семантическая сеть есть
					\uline{объективный}, не зависящий от субъективизма и многообразия
					синтаксических решений, способ представления информации}
				\scnfileitem{в рамках \textit{рафинированной семантической сети}
					существенно упрощается процедура \textit{интеграции знаний} и погружения новых
					знаний в \textit{базу знаний}}
				\scnfileitem{существенно упрощается процедура приведения различного
					вида \textit{знаний} к общему виду (к согласованной системе используемых
					\textit{понятий})}
				\scnfileitem{существенно упрощается процедура интеграции различных
					\textit{решателей задач} и целых \textit{компьютерных систем}}
				\scnfileitem{существенно упрощается автоматизация перманентного
					процесса \textit{поддержки семантической совместимости} (согласованности
					\textit{понятий} и \textit{онтологий}) для \textit{компьютерных систем} в
					условиях их постоянного совершенствования}
				\scnfileitem{в рамках \textit{рафинированных семантических сетей}
					достаточно легко осуществляется переход от информационных конструкций к
					информационным \uline{мета}конструкциям путем введения узлов
					\textit{семантической сети}, обозначающих \textit{информационные конструкции},
					а также дуг, связывающих эти узлы со всеми элементами обозначаемой ими
					\textit{информационной конструкции}}
				\scnfileitem{на основе \textit{рафинированных семантических сетей}
					существенно упрощается интеграция различных дисциплин в области
					\textit{Искуственного интеллекта}, т.е. построение \textit{Общей формальной
						теории интеллектуальных компьютерных систем}, так как для построения общей
					формальной модели \textit{интеллектуальных компьютерных систем} необходим
					базовый \textit{язык}, в рамках которого можно было бы легко переходить от
					информации (от \textit{знаний}) к \textit{метаинформации} (к метазнаниям, к
					спецификациям исходных \textit{знаний}). Это потверждается тем, что:
					\begin{scnitemize}
						\item подавляющее число \textit{понятий}
						            %\bigspace
						            \textit{Искусственного интеллекта} носит метаязыковой характер
						\item формальное смысловое уточнение почти каждого \textit{понятия}
						            %\bigspace
						            \textit{Искусственного интеллекта} требует предшествующего формального
						            уточнения соответсвующего языка-объекта. Так, например, как можно строго
						            говорить о \textit{языке онтологий} (т.е. \textit{языке} спецификации
						            \textit{предметных областей}), не уточнив \textit{язык} представления самих
						            этих \textit{предметных областей}. как можно строго говорить о \textit{языке}
						            описания способов обработки \textit{информации}, не уточнив \textit{язык
						            }представления самой этой обрабатываемой \textit{информации}.
					\end{scnitemize}}
			\end{scnrelfromset}

		\bigskip
		\scnheader{язык смыслового представления информации}
		\scnidtf{смысловой язык}
		\scnidtf{семантический язык}
		\begin{scnsubdividing}
			\scnitem{язык смыслового представления информации, не являющийся языком
				семантических сетей}
			\scnitem{язык семантических сетей}
		\end{scnsubdividing}

		\scnheader{язык семантических сетей}
		\scntext{пояснение}{Несмотря на то, что синтаксическая структура
			семантической сети во многом носит \uline{объективный} характер, поскольку
			определяется конфигурацией описываемых связей между описываемыми сущностями.
			Тем не менее, можно говорить о разных \textit{языках семантических сетей},
			каждому из которых соответствует свой \textit{алфавит*} элементов
			(синтаксически атомарных фрагментов) \textit{семантических сетей}. При атом
			языки семантических сетей могут быть как специализированными, так и
			универсальными. Задача каждого из этих \textit{языков} --- обеспечить в рамках
			\textit{языка} полное отсутствие многообразия синтаксических форм представления
			одной и той же информации.}
		\scnsubset{язык}
		\begin{scnindent}
			\scnidtf{множество информационных конструкций, для которого существуют,
				причем не обязательно в формализованном виде, (1) правила построения
				синтаксически корректных информационных конструкций, а также (2) правила,
				позволяющие установить семантическую корректность правильно построенных
				(синтаксически корректных) информационных конструкций}
		\end{scnindent}
		\scnidtf{язык, информационными конструкциями которого являются
			семантические сети и в рамках которого обеспечивается полное отсутствие
			многообразия форм представления одной и той же информации}
		\scnidtf{графовый (нелинейный) язык смыслового представления
			информации}
		\begin{scnsubdividing}
				\scnitem{специализированный язык семантических сетей}
				\begin{scnindent}
					\scnidtf{язык семантических сетей, семантическая мощность
						которого ограничена соответствующей предметной областью}
				\end{scnindent}
			\scnitem{универсальный язык емантических сетей}
				\begin{scnindent}
					\scntext{примечание}{Человечество давно и широко использует
						различные специализированные языки семантических сетей --- язык принципиальных
						электрических схем, язык блок-схем программ, язык генеалогических деревьев и
						др. Но в настоящее время актуальным является создание такого
						\textit{универсального языка семантических сетей}
						\begin{scnitemize}
							\item синтаксис и семантика которого были бы максимально просты
							\item по отношению к которому все используемые
							специализированные языки были бы его подъязыками*
							\item который был бы приспособлен к использованию в качестве
							внутреннего языка интеллектуальных компьютерных систем и компьютеров следующего
							поколения
							\item который был бы удобной основой как для обмена информацией
							между интеллектуальными компьютерными системами, так и для общения
							интеллектуальных компьютерных систем с их пользователями.
						\end{scnitemize}
					}
				\end{scnindent}
		\end{scnsubdividing}
		\begin{scnsubdividing}
			\scnitem{язык нерафинированных семантических сетей}
			\scnitem{язык рафинированных семантических сетей}
		\end{scnsubdividing}

		\scnheader{следует отличать*}
		\begin{scnhaselementset}
			\scnitem{язык семантических сетей}
				\begin{scnindent}
					\scnidtf{язык семантических сетей, рассматриваемых как
						\uline{абстрактные} графовые структуры, в которых не уточняется способ их кодирования}
					\scnhaselement{SC-код}
				\end{scnindent}
			\scnitem{графодинамический язык семантических сетей}
				\begin{scnindent}
					\scnidtf{язык графического изображения (визуализации) семантических сетей}
					\scnidtf{язык, текстами которого являются рисунки семантичеких сетей}
					\scnhaselement{SCg-код}
				\end{scnindent}
		\end{scnhaselementset}

		\scnheader{универсальный язык семантических сетей}
		\scntext{примечание}{Если ставить задачу разработки \uline{универсального}(!)
			языка, текстами которого являются графовые структуры, то классических графовых
			структур явно недостаточно. Так, например:
			\begin{scnitemize}
				\item по аналогии с переходом от ребер к ребрам и гиперребрам
				необходим переход от дуг к ориентированным связкам, связывающим более чем два
				компонента и в рамках которых эти компоненты могут иметь разные роли, которые
				необходимо явно указывать (классическим видом таких связок являются кортежи);
				\item в семантических сетях, представляющих некоторые виды
				знаний, некоторые связки (ребра, дуги, гиперребра, ориентированные связки,
				связывающие более двух компонентов) могут быть компонентами других связок;
				\item в семантических сетях, представляющих различного вида
				метазнания необходимо вводить узлы, обозначающие целые фрагменты (подграфы)
				этих же семантических сетей, и, соответственно, вводить дуги, связывающие
				каждый из этих узлов со всеми элементами подграфа, обозначаемого этим узлом.
			\end{scnitemize}}
		\bigskip

		\scnheader{семантическая модель базы знаний}
		\scnidtftext{пояснение}{смысловое представление всей \textit{базы
				знаний} \textit{интеллектуальной компьютерной системы} в виде
			\textit{семантической сети}, принадлежащей \textit{универсальному языку
				семантических сетей}}
		\scntext{пояснение}{Для того, чтобы семантические сети могли быть
			использованы в качестве средства представления \textit{знаний} в памяти
			\textit{интеллектуальной компьютерной системы} необходимо:
			\begin{scnitemize}
				\item рассмотреть \textit{семантические сети} как тексты,
				представляющие \uline{различного вида} \textit{знания};
				\item уточнить синтаксис и семантику \uline{универсального} (!)
				\textit{языка представления знаний}, текстами которого являются
				\textit{семантические сети} \cite{Inform2008} --- стр.195.Считается, что
				\textit{семантические сети} являются теоретической моделью
				\textit{представления знаний}, не используемой на практике. \cite{Inform2008}
				-- стр. 207. Однако, если реализовать \textit{графодинамическую память} и
				разработать \textit{языки программирования}, ориентированные на обработку
				информации в такой памяти, то уникальные достоинства \textit{семантических
					сетей} будут практически использованы в полной мере.
			\end{scnitemize}}

		\scnheader{семантическая модель базы знаний}
		\begin{scnrelfromvector}{достоинства}
			\scnfileitem{\textit{семантическая модель базы знаний}, построенная на
				основе \textit{универсального языка семантических сетей}, обеспечивает высокий
				уровень ассоциативности доступа к требуемым фрагментам \textit{базы знаний}
				благодаря широкому многообразию реализуемых видов запросов и существенному
				снижению реальной вычислительной сложности алгоритмов доступа (информационного
				поиска)}
			\scnfileitem{\textit{семантическая модель базы знаний} позволяет
				реализовать эффективную семантическую навигацию по текущему состоянию
				\textit{базы знаний} (при просмотре \textit{базы знаний}) путем отображения
				различного вида \textit{семантических окрестностей} для указываемых
				\textit{элементов семантической сети}. При этом \textit{семантическая модель
					базы знаний} позволяет реализовать \uline{наглядную} двумерную или трехмерную
				визуализацию просматриваемого фрагмента \textit{базы знаний} (просматриваемой
				\textit{семантической сети}). Таким образом, \textit{семантическая сеть}
				является средством представления \textit{знаний}, удобным как для самой
				\textit{интеллектуальной компьютерной системы}, так и для её пользователей.}
			\scnfileitem{сущностями, вписываемыми в \textit{семантической модели
					базы знаний} и, соответственно, обозначаемыми \textit{знаками} этих сущностей,
				могут быть не только \textit{части внешней среды} соответствующие
				\textit{интеллектуальной компьютерной системе}, но и \textit{части} (фрагменты)
				самой \textit{базы знаний}.  Это дает возможность \textit{базе знаний} включать
				в себя описание собственной структуры с любой степенью детализации и
				рассматривающее самые разные аспекты такой структуризации.}
			\scnfileitem{\textit{семантическая модель базы знаний} позволяет
				\uline{явно} выделить фрагменты \textit{базы знаний}, представляющие различные
				\textit{предметные области} и соответствующие им \textit{онтологии}, а также
				\uline{явно} описать иерархию выделенных \textit{предметных областей} и иные
				связи мужду ними (например, различного рода морфизмы). Такая семантическая
				структуризация \textit{базы знаний} позволяет осуществлять локализацию области
				действия каждой конкретной операции обработки \textit{базы знаний}, что
				существенно упрощает реализацию этих операций. Каждая \textit{предметная
					область} выделенная в рамках \textit{семантической модели базы знаний},
				описывающая соответствующий класс исследуемых (описываемых) \textit{сущностей}
				(объектов исследования) и соответствующих подклассов этого \textit{класса} с
				помощью соответствующего набора \textit{отношений} (в том числе,
				\textit{функций} и \textit{алгебраических операций}) и соответствующего набора
				\textit{свойств} (параметров), представляет собой результат интеграции
				(соединения) текстов соответствующего \textit{специализированного языка
					семантической сети}.}
			\scnfileitem{размещение каждой информации, хранимой в составе
				\textit{семантической модели базы знаний}, и, соответственно, доступ к этой
				информации (поиск её в \textit{базе знаний}) определяются \uline{исключительно}
				семантическими характеристиками этой информации (т.е. её смыслом) и не зависят
				от особенностей реализации памяти \textit{интеллектуальной компьютерной
					системы}. Т.е. смысл информации \uline{однозначно} определяет её местоположение
				в \textit{семантической модели базы знаний}, а, точнее, её связи с остальной
				частью этой \textit{базы знаний}}
			\scnfileitem{если объем представления \textit{информационной
					конструкции} определить как пару, состоящую (1) из количества синтаксически
				элементарных (атомарных) фрагментов этой конструкции и (2) из числа элементов
				\textit{алфавита} указанных элементарных фрагментов, и если рассмотреть
				множество всевозможных \uline{\textit{семантически эквивалентных*}}
				представлений каждой \textit{информационной конструкции}, то наиболее
				\uline{компактным} (сжатым) её представлением окажется представление в виде
				\textit{рафинированной семантической сети}. Более того, при расширении
				\textit{базы знаний} (при увеличении числа описываемых \textit{сущностей} и, в
				частности, числа описываемых \textit{связей}) компактность
				\textit{рафинированных семантических сетей} повышается, т.к. новые описываемые
				\textit{связи} далеко не всегда рассматривают связи между \uline{новыми}
				\textit{сущностями}, которые до этого не описывались.}
			\scnfileitem{\textit{семантические сети} позволяют говорить о
				принципиально ином характере соединения (конкатенации\, интеграции) двух
				текстов в один интегрированный текст. \textit{Интеграция} двух
				\textit{семантических сетей} предполагает склеивание (отождествление)
				\uline{синонимичных} элементов интегрируемых \textit{семантических сетей}.
				Такая \textit{интеграция}, в частности, происходит при вводе (погружении) новой
				информации в состав \textit{семантической модели базы знаний}.}
			\scnfileitem{семантические модели баз знаний дают возможность:
				\begin{scnitemize}
					\item конструктивно осуществлять анализ \textit{семантической
						связности базы знаний}, путем уточнения понятия семантической силы связи между
					различными элементами и фрагментами базы знаний
					\item конструктивно осуществлять кластеризацию баз знаний
					\item задавать метрику семантического расстояния между знаками,
					входящими в состав базы знаний
					\item осуществлять в рамках базы знаний описания различного
					вида соответствий (морфизмов) между различными фрагментами базы знаний
					(изоморфизмов, гомоморфизмов, аналогий и отличий различного вида). Так,
					например, большое значение имеет исследование таких соответствий между
					различными \textit{предметными областями}
					\item широко использовать мощный арсенал теоретико-графовых
					\textit{алгоритмов} для выполнения различного рода операций обработки
					\textit{баз знаний}.
				\end{scnitemize}}
		\end{scnrelfromvector}

		\bigskip
		\scnheader{следует отличать*}
		\begin{scnhaselementset}
			\scnitem{предельно омонимичный класс синтаксически эквивалентных знаков}
				\begin{scnindent}
					\scnidtf{класс синтаксически эквивалентных знаков, все экземпляры (все
						вхождения) которого являются знаками \uline{разных} сущностей}
					\scntext{следовательно}{В рамках рассматриваемого класса знаков
						синонимия знаков отсутствует}
					\scnidtf{максимальный класс синтаксически эквивалентных знаков,
						среди которых отсутствуют синонимичные знаки}
				\end{scnindent}
			\scnitem{частично омонимичный класс синтаксически эквивалентных знаков}
				\begin{scnindent}
					\scnidtf{класс синтаксически эквивалентных знаков, среди экземпляров
						которого встречаются как синонимичные знаки, так и знаки \uline{разных}
						сущностей}
				\end{scnindent}
			\scnitem{неомонимичный класс синтаксически эквивалентных знаков}
				\begin{scnindent}
					\scnidtf{класс синтаксически эквивалентных знаков, \uline{все} экземпляры
						которого являются знаками одной и той же сущности}
					\scnidtf{класс синтаксически эквивалентных знаков,
						синтаксическая структура которых однозначно идентифицирует (соответствует)
						обозначаемую ими сущность}
				\end{scnindent}
			\scnitem{множество особенностей (характеристик), которыми обладает
				сущность, обозначаемая заданным знаком*}
		\end{scnhaselementset}

		\bigskip
		\begin{scnhaselementset}
			\scnitem{смысловое представление информации*}
				\begin{scnindent}
					\scnidtftext{часто используемый	sc-идентификатор}{смысл*}
				\end{scnindent}
			\scnitem{смысловое представление информации}
				\begin{scnindent}
					\scnrelto{второй домен}{смысловое представление информации*}
				\end{scnindent}
			\scnitem{синтаксическая структура информационной конструкции*}
			\scnitem{синтаксическая структура информационной конструкции}
				\begin{scnindent}
					\scnrelto{второй домен}{синтаксическая структура информационной конструкции*}
				\end{scnindent}
			\scnitem{денотационная семантика информационной конструкции*}
				\begin{scnindent}
					\scnidtf{\textit{соответствие} (морфизм) между синтаксической
						структурой заданной информационной конструкции и ее \textit{смысловым
							представлением*}}
					\scntext{примечание}{\textit{соответствие} между знаками входящими в
						состав \textit{рафинированной семантической сети} и их \textit{денотатами*}
						(обозначаемыми сущностями) являются \uline{взаимно однозначными}}
				\end{scnindent}
		\end{scnhaselementset}

		\begin{scnhaselementset}
			\scnitem{смысловое пространство}
				\begin{scnindent}
					\scnhaselement{SC-пространство}
					\scnidtf{семантическое пространство}
					\scntext{пояснение}{объединение (соединение) всевозможных
						корректных абстрактных семантических сетей, принадлежащих некоторому языку
						абстрактных \textit{рафинированных семантических сетей}}
					\scnidtf{глобальная (максимальная) абстрактная
						\textit{рафинированная семантическая сеть}, включающая в себя всевозможные
						абстрактные рафинированные семантические сети соответствующего языка}
					\scnidtf{абстрактное смысловое пространство}
				\end{scnindent}
			\scnitem{абстрактная смысловая память}
				\begin{scnindent}
					\scnidtf{абстрактная семантическая память}
					\scnidtf{среда, обеспечивающая хранение абстрактных
						рафинированных семантических сетей, а также редактирование этих семантических
						сетей и при этом абстрагирующаяся от деталей этих процессов}
					\scnidtf{абстрактная графодинамическая память, обеспечивающая
						хранение и редактирование абстрактных рафинированных семантических сетей}
				\end{scnindent}
			\scnitem{реальная смысловая память}
				\begin{scnindent}
					\scnidtf{физическая реализация абстрактной смысловой памяти}
					\scnrelboth{следует отличать}{программная реализация \uline{модели} абстрактной смысловой памяти на современных компьютерах}
				\end{scnindent}
		\end{scnhaselementset}
		\bigskip
	\end{scnsubstruct}
\end{SCn}
%\scnsourcecomment{Завершили Раздел \scnqq{Предметная область и онтология смыслового представления информации}}


\scsubsection[
    \protect\scneditor{Шункевич Д.В.}
    \protect\scnmonographychapter{Глава 1.2. Интеллектуальные компьютерные системы нового поколения}
    ]{Предметная область и онтология многоагентных моделей решателей задач, основанных на смысловом представлении информации}
\label{sd_agent_solvers}
\begin{SCn}
	\scnsectionheader{Предметная область и онтология многоагентных моделей решателей задач, основанных на смысловом представлении информации}

	\begin{scnsubstruct}

		\scnheader{Предметная область многоагентных онтологических моделей решателей задач, основанных на смысловом представлении информации}
		\scniselement{предметная область}
		\begin{scnhaselementrolelist}{класс объектов исследования}
			\scnitem{многоагентный подход к обработке информации}
		\end{scnhaselementrolelist}

		\begin{scnhaselementrolelist}{класс объектов исследования}
			\scnitem{интеграция решателей задач}
		\end{scnhaselementrolelist}

		\begin{scnhaselementrolelist}{исследуемое отношение}
			\scnitem{совместимость моделей решения задач*}
		\end{scnhaselementrolelist}

		\scnheader{агентно-ориентированный подход к обработке информации}
		\scntext{примечание}{В качестве основы унификации принципов обработки
			информации в компьютерных системах предлагается использовать
			\textit{агентно-ориентированный подход к обработке информации}, обладающий
			рядом важных достоинств.}
		\begin{scnrelfromset}{достоинства}
			\scnfileitem{Автономность (независимость) агентов, что позволяет
				локализовать изменения, вносимые в систему при ее эволюции, и снизить
				соответствующие трудозатраты.}
			\scnfileitem{Децентрализация обработки, т.е. отсутствие единого
				контролирующего центра, что также позволяет локализовать вносимые в систему
				изменения.}
			\scnfileitem{Возможность параллельной работы разных информационных
				процессов, соответствующих как одному агенту, так и разным агентам, как
				следствие, --- возможность распределенного решения задач. Однако возможность
				параллельного выполнения информационных процессов подразумевает наличие средств
				синхронизации такого выполнения, разработка которых является отдельной
				задачей.}
			\scnfileitem{Активность агентов и многоагентной системы в целом, дающая
				возможность при общении с такой системой не указывать явно способ решения
				поставленной задачи, а формулировать задачу в \uline{декларативном ключе}.}
		\end{scnrelfromset}

		\begin{scnindent}
			\scnrelfrom{источник}{\scncite{Wooldridge2009}}
		\end{scnindent}

		\begin{scnrelfromset}{недостатки современного состояния}
			\scnfileitem{Знания агента представляются при помощи
				узкоспециализированных языков, зачастую не предназначенных для представления
				знаний в широком смысле и онтологий в частности.}
			\scnfileitem{Большинство современных многоагентных систем предполагает,
				что взаимодействие агентов осуществляется путем обмена сообщениями
				непосредственно от агента к агенту.}
			\scnfileitem{Логический уровень взаимодействия агентов жестко привязан
				к физическому уровню реализации многоагентной системы.}
			\scnfileitem{Среда, с которой взаимодействуют агенты, уточняется
				отдельно разработчиком для каждой многоагентной системы, что приводит к
				существенным накладным расходам и несовместимости таких многоагентных систем.}
		\end{scnrelfromset}
		\begin{scnindent}
			\begin{scnrelfromset}{принципы устранения}
				\scnfileitem{Коммуникацию агентов предлагается осуществлять путем
					спецификации (в общей памяти компьютерной системы) действий (процессов),
					выполняемых агентами и направленных на решение задач.}
					\begin{scnindent}
						\scntext{детализация}{Коммуникацию агентов предлагается
							осуществлять по принципу \scnqqi{доски объявлений}, однако в отличие от классического
							подхода в роли сообщений выступают спецификации в общей семантической памяти
							выполняемых агентами действий (процессов), направленных на решение каких-либо
							задач, а в роли среды коммуникации выступает сама эта семантическая память.
							Такой подход позволяет:
							\begin{scnitemize}
								\item исключить необходимость разработки специализированного
											языка для обмена сообщениями
								\item обеспечить \scnqq{обезличенность} общения, т. е. каждый из
								агентов в общем случае не знает, какие еще агенты есть в системе, кем
								сформулирован и кому адресован тот или иной запрос. Таким образом, добавление
								или удаление агентов в систему не приводит к изменениям в других агентах, что
								обеспечивает модифицируемость всей системы
								\item агентам, в том числе конечному пользователю,
											формулировать задачи в \uline{декларативном ключе}, т. е. не указывать для
											каждой задачи способ ее решения. Таким образом, агенту заранее не нужно знать,
											каким образом система решит ту или иную задачу, достаточно лишь специфицировать
											конечный результат
								\item сделать средства коммуникации агентов и синхронизации их
											деятельности более понятными разработчику и пользователю системы, не требующими
											изучения специальных низкоуровневых типов данных и форматов сообщений. Таким
											образом повышается доступность предлагаемых решений широкому кругу
											разработчиков.
							\end{scnitemize}
							Следует отметить, что такой подход позволяет при необходимости
							организовать обмен сообщениями между агентами напрямую и, таким образом, может
							являться основой для моделирования многоагентных систем, предполагающих другие
							способы взаимодействия между агентами.}
						\end{scnindent}	
				\scnfileitem{В роли внешней среды для агентов выступает та же общая
					память, в которой формулируются задачи и посредством которой осуществляется
					взаимодействие агентов. Такой подход обеспечивает унификацию среды для
					различных систем агентов, что, в свою очередь, обеспечивает их совместимость.}
				\scnfileitem{Спецификация каждого агента описывается средствами языка
					представления знаний в той же памяти, что позволяет:
					\begin{scnitemize}
						\item минимизировать число специализированных средств, необходимых для
						спецификации агентов, как языковых, так и инструментальных
						\item с одной стороны --- минимизировать необходимую в общем случае
						спецификацию агента, которая включает условие его инициирования и программу,
						описывающую алгоритм работы агента, с другой стороны --- обеспечить возможность
						произвольного расширения спецификации для каждого конкретного случая, в том
						числе возможность реализации различных современных моделей спецификации агента.
					\end{scnitemize}}
				\scnfileitem{Синхронизацию деятельности агентов предполагается
					осуществлять на уровне выполняемых ими процессов, направленных на решение тех
					или иных задач в общей семантической памяти. Таким образом, каждый агент
					трактуется как некий абстрактный процессор, способный решать задачи
					определенного класса. При таком подходе необходимо решить задачу обеспечения
					взаимодействия параллельных асинхронных процессов в общей семантической памяти,
					для решения которой можно заимствовать и адаптировать решения, применяемые в
					традиционной линейной памяти. При этом вводится дополнительный класс агентов --
					метаагенты, задачей которых является решение возникающих проблемных ситуаций,
					таких как взаимоблокировки}
				\scnfileitem{Каждый информационный процесс в любой момент времени имеет
					ассоциативный доступ к необходимым фрагментам базы знаний, хранящейся в
					семантической памяти, за исключением фрагментов, заблокированных другими
					процессами в соответствии с соответствующим механизмом синхронизации. Таким
					образом, с одной стороны, исключается необходимость хранения каждым агентом
					информации о внешней среде, с другой стороны, каждый агент, как и в
					классических многоагентных системах, обладает только частью всей информации,
					необходимой для решения задачи.\\Важно отметить, что в общем случае невозможно
					априори предсказать, какие именно знания, модели и способы решения задач
					понадобятся системе для решения конкретной задачи. В связи с этим необходимо
					обеспечить, с одной стороны, возможность доступа ко всем необходимым фрагментам
					базы знаний (в пределе --- ко всей базе знаний), с другой стороны --- иметь
					возможность локализовать область поиска пути решения задачи, например, рамками
					одной \textit{предметной области}.\\Каждый из агентов обладает набором ключевых
					элементов (как правило, понятий), которые он использует в качестве отправных
					точек при ассоциативном поиске в рамках базы знаний. Набор таких элементов для
					каждого агента уточняется на этапах проектирования многоагентной системы в
					соответствии с рассматриваемой ниже методикой. Уменьшение числа ключевых
					элементов агента делает его более универсальным, однако снижает эффективность
					его работы за счет необходимости выполнения дополнительных  операций.}
			\end{scnrelfromset}	
				\begin{scnindent}
					\scntext{примечание}{Предлагаемый подход позволяет рассматривать решатель
						задач как иерархическую систему. Некий целостный коллектив агентов, реализующий
						какую-либо подсистему решателя (например, машину дедуктивного вывода,
						подсистему верификации базы знаний и т. д.), может рассматриваться как единый
						неатомарный агент, поскольку коллективы агентов и отдельные агенты работают в
						соответствии с одними и теми же принципами.}
				\end{scnindent}
		\end{scnindent}

		\scnheader{совместимость моделей решения задач*}
		\scntext{примечание}{\textbf{\textit{совместимость моделей решения задач*}}
			-- это возможность одновременного использования разными моделями решения задач
			одних и тех же информационных ресурсов.}

		\begin{scnrelfromset}{принципы реализации}
			\scnfileitem{Вся информация, хранимая в памяти каждой ostis-системы и
				используемая \textit{\textbf{решателем задач}} (как собственно обрабатываемая
				информация, так и хранимые в памяти интерпретируемые методы, например,
				различного вида программы), записывается в форме смыслового представления этой
				информации}
			\scnfileitem{Собственно решение каждой задачи осуществляется
				коллективом агентов, работающих над общей для них смысловой (семантической)
				памятью и выполняющих интерпретацию хранимых в этой же памяти
				\textit{методов}.}
		\end{scnrelfromset}

		\scnheader{интеграция решателей задач}
		\scnsubset{процесс}
		\begin{scnrelfromvector}{алгоритм реализации}
			\scnfileitem{Объединение множества методов первого решателя и множества
				методов второго решателя}
			\scnfileitem{Интеграция множества методов первого решателя и множества
				методов второго решателя путем взаимного погружения соответствующих
				информационных конструкций друг в друга, т.е. путем склеивания синонимов, а
				также путем выравнивания используемых ими понятий.}
			\scnfileitem{Объединение множества агентов, входящих в состав первого
				решателя, со множеством агентов, входящих во второй решатель задач.}
		\end{scnrelfromvector}

		\scntext{пояснение}{Таким образом, унификация моделей решения задач
			путем приведения этих моделей к виду семантических моделей (т. е. моделей
			обработки информации, представленной в смысловой форме) повышает уровень
			совместимости этих моделей благодаря наличию прозрачной процедуры интеграции
			информации, представленной в смысловой форме, и тривиальной процедуры
			объединения множеств \textit{агентов}. Простота процедуры объединения множеств
			\textit{агентов}, соответствующих разным решателя задач, обусловлена тем, что
			непосредственного взаимодействия между этими агентами нет, а инициирование
			каждого из них определяется им самим, а также \uline{текущим состоянием}
			хранимой в памяти информации.}
		\bigskip
	\end{scnsubstruct}
\end{SCn}
%\scnsourcecomment{Завершили Раздел \scnqq{Предметная область и онтология многоагентных моделей решения задач, основанных на смысловом представлении информации}}


\scsubsection[
    \protect\scneditor{Садовский М.Е.}
    \protect\scnmonographychapter{Глава 1.2. Интеллектуальные компьютерные системы нового поколения}
    ]{Предметная область и онтология онтологических моделей интерфейсов интеллектуальных компьютерных систем, основанных на смысловом представлении информации}
\label{sd_sem_ui}
\begin{SCn}
	\scnsectionheader{Предметная область и онтология онтологических моделей интерфейсов интеллектуальных компьютерных систем, основанных на смысловом представлении информации}

	\begin{scnsubstruct}

		\scnrelfrom{соавтор}{Садовский М.Е.}
		\scnheader{Предметная область онтологических моделей интерфейсов
			интеллектуальных компьютерных систем, основанных на смысловом представлении
			информации}
		\scniselement{предметная область}

		\begin{scnhaselementrolelist}{класс объектов исследования}
			\scnitem{подход к построению пользовательского интерфейса}
		\end{scnhaselementrolelist}

		\begin{scnhaselementrolelist}{класс объектов исследования}
			\scnitem{подход к построению пользовательского интерфейса на
				основе специализированных языков описания}
			\scnitem{контекстно-зависимый подход к построению
				пользовательского интерфейса;подход к построению пользовательского интерфейса
				на основе данных}
			\scnitem{онтологический подход к построению пользовательского
				интерфейса}
			\scnitem{онтологический подход к построению пользовательского
				интерфейса на основе логико-семантической модели}
		\end{scnhaselementrolelist}

		\scnheader{подход к построению пользовательского интерфейса}
		\scnsuperset{подход к построению пользовательского интерфейса на основе
			специализированных языков описания}
		\scnsuperset{контекстно-зависимый подход к построению пользовательского
			интерфейса}
		\scnsuperset{подход к построению пользовательского интерфейса на основе
			данных}
		\scnsuperset{онтологический подход к построению пользовательского
			интерфейса}
			\begin{scnindent}
				\scnsuperset{онтологический подход к построению пользовательского
					интерфейса на основе логико-семантической модели}
			\end{scnindent}

		\scnheader{подход к построению пользовательского интерфейса на основе
			специализированных языков описания}

		\scntext{пояснение}{подход на основе специализированных языков
			описания предполагает представление конкретного пользовательского интерфейса в
			платформенно независимом виде. В качестве примеров языков описания интерфейса
			можно привести UIML (\cite{ABRAMS19991695}), UsiXML (\cite{UsiXML}), XForms
			(\cite{XForms}) и FXML (\cite{fxml}). Ключевой идеей представленных языков
			является создание модели диалогов и форм интерфейса в независимом от
			используемой технологии виде, описание визуальных элементов, а также
			взаимосвязей между ними и их свойств для создания конкретного пользовательского
			интерфейса.}
		\begin{scnrelfromset}{недостатки современного состояния}
			\scnfileitem{как  правило,  спецификация  модели  интерфейса является
				неполной,  что	приводит  к  сложности автоматизации процесса генерации
				пользовательского интерфейса}
			\scnfileitem{как правило, созданные модели специфичны для конкретной
				платформы либо конкретной реализации пользовательского интерфейса, что
				препятствует их повторному использованию для других целей.}
			\scnfileitem{решения,	которые   предлагают   платформенно независимое
				описание,  позволяют  генерировать лишь  простые  ограниченные  по
				функционалу пользовательские интерфейсы (приложения-опросники, диаграммы и
				т.д.).}
		\end{scnrelfromset}

		\scnheader{контекстно-зависимый подход к построению пользовательского
			интерфейса}
		\scntext{пояснение}{Контекстно-зависимый подход интегрирует
			использование структурного описания интерфейса на основе языков описания с
			поведенческой спецификацией, то есть генерация интерфейса основывается на
			действиях пользователя. В рамках подхода специфицируются переходы между
			различными видами конкретного пользовательского интерфейса. В качестве примеров
			языков, реализующих идеи такого подхода можно привести CAP3 (\cite{CAP3}) и
			MARIA (\cite{MARIA}).}
		\scnheader{подход к построению пользовательского интерфейса на основе
			данных}
		\scntext{пояснение}{подход на основе данных или моделеориентированный
			подход использует модель предметной области в качестве основы для создания
			пользовательских интерфейсов. Указанный подход реализуется в таких проектах,
			как JANUS (\cite{JANUS}) и Mecano (\cite{Mecano}).}
		\scnheader{онтологический подход к построению пользовательского
			интерфейса}
		\scntext{пояснение}{cуществующие онтологические подходы как правило
			основаны на представленных ранее подходах и используют онтологии в качестве
			способа представления информации о конкретном пользовательском интерфейсе.
			Например, по аналогии с подходом на основе специализированных языков описания,
			был предложен фреймворк  (\cite{ui_model-based-approach}), использующий
			онтологию для описания пользовательского интерфейса на основе понятий,
			хранящихся в базе знаний. По аналогии с контекстно-зависимым подходом в рамках
			работы \cite{gaulke} используется модель предметной области совместно с моделью
			пользовательского интерфейса, ассоциированная с онтологией действий. Проект
			ActiveRaUL (\cite{ActiveRaUL}) совмещает UIML с моделеориентированным подходом.
			В рамках данного проекта онтологическая модель предметной области
			сопоставляется с онтологическим представлением пользовательского интерфейса.
			Подход, предложенный в работе \cite{hitz}, совмещает данные приложения с
			онтологией пользовательского интерфейса для создания единого описания в базе
			знаний с целью последующей автоматической генерации различных вариантов
			интерфейса для приложений-опросников с готовыми сценариями взаимодействия с
			пользователем. Следует также отметить работы \cite{vladivostok1} и
			\cite{vladivostok2}, в рамках которых предложена концепция, позволяющая
			объединить однородную по содержанию информацию в компоненты модели интерфейса,
			освободить разработчика интерфейса от кодирования и формировать информацию для
			каждого компонента модели интерфейса с помощью редакторов, управляемых
			соответствующими моделями онтологий.}
		\begin{scnrelfromset}{недостатки современного состояния}
			\scnfileitem{актуальна проблема совместимости различных онтологий в
				рамках единой системы}
			\scnfileitem{отсутствие способности адаптироватьсяк запросам
				пользователя и анализировать его действия длясамостоятельного
				совершенствования.}
		\end{scnrelfromset}

		\begin{scnrelfromset}{достоинства}
			\scnfileitem{позволяет интегрировать ранее предложенные подходы за счет
				единого способа представления знаний.}
			\scnfileitem{позволяет создать наиболее полное описание различных
				аспектов пользовательского интерфейса.}
			\scnfileitem{упрощает повторное использование интерфейса.}
		\end{scnrelfromset}

		\scnheader{онтологический подход к построению пользовательского
			интерфейса на основе логико-семантической модели}
		\scntext{примечание}{для проектирования пользовательских интерфейсов
			предлагается использовать \textbf{\textit{онтологический подход к построению
					пользовательского интерфейса на основе логико-семантической модели}},
			обладающий рядом важных достоинств.}
		\begin{scnrelfromset}{достоинства}
			\scnfileitem{возможность переноса пользовательских интерфейсов с одной
				платформы реализации на другую.}
			\scnfileitem{наличие общих    принципов построения пользовательских
				интерфейсов, что позволяет повторно использовать уже разработанные компоненты
				и  снижает сроки  обучения  пользователя  новым  для  него пользовательским
				интерфейсам.}
			\scnfileitem{возможность модификации пользовательского интерфейса в
				процессе работы.}
			\scnfileitem{возможность гибкой адаптации пользовательского интерфейса
				под нужды конкретного пользователя.}
		\end{scnrelfromset}

		\scntext{пояснение}{подход предполагает создание полной семантической
			модели интерфейса, содержащей  лексическое описание  интерфейса(описание
			компонентов, из которых формируется интерфейс), синтаксическое	описание
			интерфейса(правила  формирования  корректного  и  полного интерфейса из его
			компонентов), но также и его семантическое описание (знание о том, знаком какой
			сущности является отображаемый компонент). При этом семантическое описание
			также включает всебя назначение, область применения компонентов интерфейса,
			описание интерфейсной деятельности пользователя.}
		\begin{scnrelfromset}{основные принципы}
			\scnfileitem{пользовательский интерфейс представляет собой
				специализированную ostis-систему, ориентированную на решение интерфейсных
				задач,и имеющую в составе базу знаний и машину обработки знаний
				пользовательского интерфейса,что даёт возможность пользователю адресовать
				пользовательскому интерфейсу различного рода вопросы}
			\scnfileitem{используется онтологический подход к проектированию
				пользовательского интерфейса, чтоспособствует чёткому разделению деятельности
				различных разработчиков пользовательских интерфейсов, а также унификации
				принципов проектирования}
			\scnfileitem{используется SC-код в качестве формальногоязыка
				внутреннего представления знаний (онтологий, предметных областей и др.),
				благодарячему обеспечивается легкость интерпретацииэтих знаний и системой, и
				человеком - пользователем или разработчиком, а также однозначность восприятия
				этой информации ими}
			\scnfileitem{средствами SC-кода с помощью соответствующих онтологий
				описываются синтаксис и семантика всевозможных используемых внешнихязыков}
			\scnfileitem{трансляции с внутреннего языка на внешний иобратно
				организовываются так, чтобы механизмы трансляции не зависели от внешнего языка,
				для реализации нового специализированногоязыка в таком случае необходимо будет
				толькоописать его синтаксис и семантику, универсальная же модель трансляции не
				будет зависеть отданного описания}
			\scnfileitem{предполагается выбор стилей визуализации,осуществляемый в
				зависимости от вида отображаемых знаний (например, использование различных
				элементов визуализации для одних видов знаний и других - для других), что
				позволит пользователю быстрее обучаться новымспециализированным языкам, а также
				сделатьпростым и понятным отображение знаний}
			\scnfileitem{модель пользовательского интерфейса строитсянезависимо от
				реализации платформы интерпретации такой модели, что позволяет легкопереносить
				разработанную модель на разныеплатформы.}
		\end{scnrelfromset}
		\bigskip
	\end{scnsubstruct}
\end{SCn}
%\scnsourcecomment{Завершили Раздел \scnqq{Предметная область и онтология логико-семантических моделей интерфейсов компьютерных систем, основанных на смысловом представлении информации}}


\scsection[
    \protect\scnidtf{Предметная область и онтология \textit{Технологии OSTIS} --- Open Semantic Technology for Intelligent Systems}
    \protect\scnidtf{Принципы, лежащие в основе \textit{Технологии OSTIS}}
    \protect\scnmonographychapter{Глава 1.3. Принципы, лежащие в основе технологии комплексной поддержки жизненного цикла интеллектуальных компьютерных систем нового поколения}
    ]{Предметная область и онтология комплексной технологии поддержки жизненного цикла интеллектуальных компьютерных систем нового поколения}

%\scsectionfamily{Часть 2 Стандарта OSTIS. Смысловое представление и онтологическая систематизация знаний в интеллектуальных компьютерных системах нового поколения}
\label{part_representation}


\scsection{Введение в описание внутреннего языка ostis-систем и близких ему внешних языков, используемых для представления исходных текстов баз знаний}

\begin{SCn}

\scnsectionheader{\currentname}

\end{SCn}


\scsubsection{Введение в описание внутреннего языка ostis-систем}
\label{intro_sc_code}

\begin{SCn}

\scnsectionheader{\currentname}

\scnstartsubstruct

\scnsegmentheader{Первый сегмент Введения описание внутреннего языка ostis-систем}
\scnstartsubstruct

\scnheader{\currentname}
\scnreltovector{конкатенация сегментов}{Первый сегмент Введения описание внутреннего языка ostis-систем;Описание Ядра SC-кода;Описание Расширения Ядра SC-кода}

\scnheader{SC-код}
\scnidtf{Внутренний язык ostis-систем}
\scnidtf{Множество sc-текстов}
\scnidtf{sc-текст}
\scnidtf{Множество sc-конструкций}
\scnidtf{Язык унифицированного смыслового представления знаний в памяти интеллектуальных компьютерных систем}
\filemodetrue
\scnrelfromvector{принципы, лежащие в основе}{Знаки (обозначения) всех сущностей, описываемых в \textit{sc-текстах} (текстах SC-кода) представляются в виде синтаксически элементарных (атомарных) фрагментов \textit{sc-текстов} и, следовательно, не имеющих внутренней структуры, не состоящих из более простых фрагментов текста, как, например, имена (термины), которые представляют знаки описываемых сущностей в привычных языках и состоят из букв.;Имена (термины), естественно-языковые тексты и другие информационные конструкции, не являющиеся \textit{sc-текстами}, могут входить в состав \textit{sc-текста}, но только как файлы, описываемые (специфицируемые) \textit{sc-текстами}. Таким образом, в состав базы знаний \textit{интеллектуальной компьютерной системы}, построенной на основе \textit{SC-кода}, могут входить имена (термины), обозначающие некоторые описываемые сущности и представленные соответствующими файлами. Каждый sc-элемент будем называть внутренним обозначением некоторой сущности, а имя этой сущности, представленное соответствующим файлом, будем называть внешним идентификатором (внешним обозначением) этой сущности. При этом каждый именуемый (идентифицируемый) \textit{sc-элемент} связывается дугой, принадлежащей отношению "\textit{\textbf{быть внешним идентификатором*}}, с узлом, содержимым которого является файл идентификатора (в частности, имени), обозначающего ту же сущность, что и указанный выше \textit{sc-элемент}. Внешним обозначением может быть не только имя (термин), но и иероглиф, пиктограмма, озвученное имя, жест. Особо отметим, что внешние обозначения описываемых сущностей в интеллектуальной компьютерной системе, построенной на основе \textit{SC-кода}, используются только (1) для анализа информации, поступающей в эту систему из вне из различных источников, и ввода (понимания и погружения) этой информации в базу знаний, а также (2) для синтеза различных сообщений, адресуемых различным субъектам (в т.ч. пользователям).;Тексты \textit{SC-кода} (sc-тексты) имеют в общем случае нелинейную (графовую) структуру, поскольку (1) знак каждой описываемой сущности в ходит в состав sc-текста однократно и (2) каждый такой знак может быть инцидентен неограниченному числу других знаков, поскольку каждая описываемая сущность может быть связана неограниченным числом связей с другими описываемыми сущностями.;
База знаний, представленная текстом \textit{SC-кода}, является графовой структурой специального вида, алфавит элементов которой включает в себя множество узлов, множество ребер, множество дуг, множество базовых дуг -- дуг специально выделенного типа, обеспечивающих структуризацию баз знаний, а также множество специальных узлов, каждый из которых имеет содержимое, являющееся файлом, хранящимся в памяти интеллектуальной компьютерной системы. Структурная особенность данной графовой структуры заключается в том, что ее дуги и ребра могут связывать не только узел с узлом, но и узел с ребром или дугой, ребро или дугу с другим ребром или дугой.;
\uline{Все элементы} указанной выше графовой структуры (текста SC-кода), т.е. все ее узлы, ребра и дуги являются обозначениями различных сущностей. При этом ребро является обозначением бинарной неориентированной связки между двумя сущностями, каждая из которых либо представлена в рассматриваемой графовой структуре соответствующим знаком, либо является самим этим знаком. Дуга является обозначением бинарной ориентированной связки между двумя сущностями. Дуга специального вида (\textit{\textbf{базовая дуга}}) является знаком связи между узлом, обозначающим некоторое множество элементов рассматриваемой графовой структуры, и одним из элементов этой графовой структуры, который принадлежит указанному множеству. Узел, имеющий содержимое (узел, для которого содержимое существует, но может в текущий момент быть неизвестным) является знаком файла, который является содержимым этого узла. Узел, не являющийся знаком файла, может обозначать какой-либо материальный объект, первичный абстрактный объект(например, число, точку в некотором абстрактном пространстве), какую-либо бинарную связь, какое-либо множество (в частности, понятие, структуру, ситуацию, событие, процесс). При этом сущности, обозначаемые элементами рассматриваемой графовой структуры, могут быть постоянными (существующими всегда) и временными (сущностями, которым соответствует отрезок времени их существования). Кроме того, сущности, обозначаемые элементами рассматриваемой графовой структуры, могут быть константными (конкретными) сущностями и переменными (произвольными) сущностями. Каждому элементу рассматриваемой графовой структуры, являющемуся обозначением переменной сущности, ставится в соответствие область возможных значений этого обозначения. Область возможных значений каждого переменного ребра является подмножеством множества всевозможных константных ребер, область возможных значений каждой переменной дуги является подмножеством множества всевозможных константных дуг, область возможных значений каждого переменного узла является подмножеством множества всевозможных константных узлов.;
В рассматриваемой графовой структуре, являющейся представлением базы знаний в SC-коде, могут, но не должны существовать разные элементы графовой структуры, обозначающие одну и ту же сущность. Если пара таких элементов обнаруживается, то эти элементы склеиваются (отождествляются). Таким образом, синонимия внутренних обозначений в базе знаний интеллектуальной компьютерной системы, построенной на основе \textit{SC-кода,} запрещена. При этом синонимия внешних обозначений считается нормальным явлением. Формально это означает, что из некоторых элементов рассматриваемой графовой структуры выходит несколько дуг, принадлежащих отношению "\textit{\textbf{быть внешним идентификатором*}}". Из всех указанных дуг, принадлежащих отношению "\textit{\textbf{быть внешним идентификатором*}}" и выходящих из одного элемента рассматриваемой графовой структуры, обязательно выделяется одна (очень редко две) путем включения их в число дуг, принадлежащих отношению "\textit{\textbf{быть основным внешним идентификатором*}}". Это означает, что указываемый таким образом внешний идентификатор не является омонимичным, т.е. не может быть использован как внешний идентификатор, соответствующий другомуэлементу рассматриваемой графовой структуры.;
Кроме файлов, представляющих различные внешние обозначения (имена, иероглифы, пиктограммы), в памяти интеллектуальной компьютерной системе, построенной на основе \textit{SC-кода,} могут хранится файлы различных текстов (книг, статей, документов, примечаний, комментариев, пояснений, чертежей, рисунков, схем, фотографий, видео-материалов, аудио-материалов).;
\uline{Любую сущность}, требующую описания, можно обозначить в виде sc-элемента. Особо подчеркнем, что sc-элементы являются не просто обозначениями различных описываемых сущностей, а обозначениями, которые являются элементарными (атомарными) фрагментами знаковой конструкции, т.е. фрагментами, детализация структуры которых не требуется для "прочтения" и понимания этой знаковой конструкции.;
Текст \textit{\textbf{SC-кода}}, как и любая другая графовой структура, является абстрактным математическим объектом, не требующим детализации (уточнения) его кодирования в памяти компьютерной системы (например, в виде матрицы смежности, матрицы инцидентности, списковой структуры). Но такая детализация потребуется для технической реализации памяти, в которой хранятся и обрабатываются sc-тексты.;
Важнейшим дополнительным свойством \textit{\textbf{SC-кода}} является то,что он удобен не просто для внутреннего представления знаний в памяти интеллектуальной компьютерной системы, но также и для внутреннего представления информации в памяти компьютеров, специально предназначенных для интерпретации семантических моделей интеллектуальных компьютерных систем. Т.е., SC-код определяет синтаксические, семантические и функциональные принципы организации памяти компьютеров нового поколения, ориентированных на реализацию интеллектуальных компьютерных систем, -- принципы организации графодинамической ассоциативной семантической памяти.;
SC-код рассматривается нами как объединение нескольких его подъязыков, в число которых входит ядро SC-кода и его расширение, обеспечивающее ввод и вывод информации для ostis-системы на всевозможных внешних языках.
}
\filemodefalse
\scnaddlevel{1}
\scnsourcecomment{Завершили описание принципов SC-кода}
\scnaddlevel{-1}

\scnendstruct

\scnsegmentheader{Описание Ядра SC-кода}
\scnstartsubstruct

\scnheader{Ядро SC-кода}
\scnrelfrom{алфавит}{Алфавит Ядра SC-кода}
\scnaddlevel{1}
\scnhaselement{sc-узел}
\scnhaselement{sc-ребро}
 \scnaddlevel{1}
 \scnidtf{обозначение бинарной неориентированной связи между sc-элементами}
 \scnaddlevel{-1}
\scnhaselement{sc-дуга}
\scnaddlevel{1}
 \scnidtf{обозначение бинарной ориентированной связи между sc-элементами}
 \scnaddlevel{-1}
\scnhaselement{базовая sc-дуга}
 \scnaddlevel{1}
 \scnidtf{sc-дуга константной позитивной стационарной принадлежности}
 \scnidtf{знак константной позитивной стационарной пары принадлежности}
 \scnaddlevel{-1}
\scnnote{Подчеркнем, что с помощью указанных типов sc-элементов можно описать любые связи между sc-элементами, трактуя эти связи как множества связываемых sc-элементов и используя некоторые sc-узлы как знаки этих множеств.}
\scnaddlevel{-1}

\scnendstruct

\scnsegmentheader{Описание Расширения Ядра SC-кода}
\scnstartsubstruct

\scnheader{SC-код}
\scnidtf{Расширение Ядра SC-кода}
\scnidtf{Результат введения в Ядро SC-кода sc-узлов, имеющих содержимое и обозначающих файлы, хранимые в памяти ostis-системы}
\scnnote{Все файлы, представляющие собой электронные образы инородных для SC-кода информационных конструкций, можно представить в SC-кода с помощью графовых структур, в которых sc-элементы обозначают буквы текстов или пиксели изображений. Но такой вариант кодирования внешних для ostis-системы информационных конструкций не дает возможности непосредственно использовать накопленный человечеством арсенал электронных информационных ресурсов.}
\scnnote{Важнейшим видом файлов ostis-систем являются внешние идентификаторы sc-элементов (в частности, имена sc-элементов), представляющие sc-элементы в текстах внешних языков (в том числе, текстах SCs-кода и SCn-кода)} 
\scnnote{Результатом просмотренного расширения \textit{Ядра SC-кода} является расширение \textit{Алфавита Ядра SC-кода}}

\scnheader{SC-код}
\scnrelfrom{алфавит}{Алфавит SC-кода}
\scnaddlevel{1}
\scnhaselement{sc-узел}
\scnhaselement{sc-ребро}
\scnhaselement{sc-дуга}
\scnhaselement{базовая sc-дуга}
\scnhaselement{файл ostis-системы}
\scnaddlevel{-1}

\scnheader{файл ostis-системы}
\scnidtf{sc-узел с содержимым}
\scnidtf{sc-узел, имеющий содержимое}
\scnidtf{sc-узел, обозначающий файл, хранимый в памяти ostis-системы}
\scnidtf{знак файла ostis-системы}
\scnreltoset{разбиение}{ея-файл ostis-системы\\
\scnaddlevel{1}
\scnidtf{естественно-языковой файл ostis-системы}
\scnaddlevel{-1};файл ostis-системы, являющийся текстом формального языка\\
\scnaddlevel{1}
\scnsuperset{sc.g-файл ostis-системы}
\scnsuperset{sc.s-файл ostis-системы}
\scnsuperset{sc.n-файл ostis-системы}
\scnaddlevel{-1};файл ostis-системы, отражающий процесс изменения sc.g-текста;графический файл ostis-системы;файл ostis-системы, являющийся изображением;видео-файл ostis-системы;аудио-файл ostis-системы}
\scnreltoset{разбиение}{файл-экземпляр ostis-системы
\scnaddlevel{1}
\scnidtf{файл, являющийся конкретным электронным документом или электронным образом конкретной внешней информационной конструкции}
\scnaddlevel{-1};файл-класс ostis-системы
\scnaddlevel{1}
\scnidtf{файл, являющийся знаком множества всевозможных экземпляров (копий) этого файла}
\scnaddlevel{-1}
}

\scnheader{SC-код}
\scnrelfrom{синтаксис}{Cинтаксис SC-кода} 
\scnaddlevel{1}
\scnexplanation{\textit{\textbf{Синтаксис}} \textit{\textbf{SC-кода}} задается
\begin{scnitemize}
\item типологией (алфавитом) sc-элементов (атомарных фрагментов текстов sc-кода);
\item правилами соединения (инцидентности) sc-элементов (например, sc-элементы каких типов не могут быть инцидентными друг другу);
\item типологией конфигураций sc-элементов (связки, классы, структуры), связями между конфигурациями sc-элементов (в частности, теоретико-множественными)
\end{scnitemize}
}
\scnaddlevel{-1}
\scnrelfrom{денотационная семантика}{Денотационная семантика SC-кода} 
\scnaddlevel{1}
\scnexplanation{\textit{\textbf{Денотационная семантика}} \textit{\textbf{SC-кода}} задается
\begin{scnitemize}
\item
 семантической интерпретацией sc-элементов и их конфигураций;
\item
 семантической интерпретацией инцидентности sc-элементов;
\item
 иерархической системой предметных областей;
\item
 структурой используемых понятий в каждой предметной области (исследуемые классы объектов, исследуемые отношения, исследуемые классы объектов отношений из смежных предметных областей, ключевые экземпляры исследуемых классов объектов);
\item
 онтологиями предметных областей.
\end{scnitemize}
}
\scnaddlevel{-1}
\scnnote{Следует особо подчеркнуть, что  унификация и максимально возможное упрощение  \textbf{\textit{синтаксиса}} и \textbf{\textit{денотационной семантики}} внутреннего языка интеллектуальных компьютерных систем необходимы потому, что подавляющий объем \textbf{\textit{знаний}}, хранимых в составе  базы знаний интеллектуальной компьютерной системы, представляют собой \textbf{\textit{метазнания}}, описывающими свойства других знаний. Более того, по указанной причине конструктивное (формальное) развитие теории интеллектуальных компьютерных систем невозможно без уточнения (унификации, стандартизации) и обеспечения семантической совместимости различных видов знаний, хранимых в базе знаний интеллектуальной компьютерной  системы.  Очевидно, что многообразие форм представления семантически эквивалентных знаний делает разработку общей теории  интеллектуальных компьютерных систем практически невозможной. К \textit{метазнаниям}, в частности, следует отнести и различного вида логические высказывания и всевозможного вида программы, описания методов (навыков). Обеспечивающих решение различных классов информационных задач.}

\scnendstruct~
\scnsourcecomment{Завершили сегмент "Описание расширения Ядра SC-кода"}

\scnendstruct~
\scnsourcecomment{Завершили раздел "\currentname"}

\end{SCn}


\scsubsection{Неформальное введение в язык визуального представления баз знаний ostis-систем}

\begin{SCn}

\scnsectionheader{\currentname}

\end{SCn}


\scsubsection{Неформальное введение в язык гипертекстового представления баз знаний ostis-систем}

\begin{SCn}

\scnsectionheader{\currentname}

\end{SCn}


\scsection[\scneditor{Банцевич К.А.}\protect\scnmonographychapter{Глава 2.3. Структура баз знаний интеллектуальных компьютерных систем нового поколения: иерархическая система предметных областей и онтологий. Онтологии верхнего уровня. Формализация понятий семантической окрестности, предметной области и онтологии в интеллектуальных компьютерных системах нового поколения}]{Предметная область и онтология знаний и баз знаний ostis-систем}
\label{sd_knowledge}
\begin{SCn}

\scnsectionheader{\currentname}

\scnstartsubstruct

\scnrelfromlist{дочерний раздел}{Предметная область и онтология множеств
    \scnaddlevel{1}
    \scnidtf{Предметная область и онтология \textit{знаний о множествах}}
        \scnaddlevel{1}
        \scnnote{\textit{знания о множествах} являются \uline{частным видом} \textit{знаний} и, следовательно, общие свойства сущностей, описываемых знаниями, могут наследоваться \textit{Предметной областью и онтологией множеств}}
        \scnaddlevel{-1}
    \scnaddlevel{-1}
;Предметная область и онтология связок и отношений
;Предметная область и онтология параметров, величин и шкал
;Предметная область и онтология чисел и числовых структур
;Предметная область и онтология структур
;Предметная область и онтология темпоральных сущностей
;Предметная область и онтология темпоральных сущностей баз знаний ostis-систем
;Предметная область и онтология семантических окрестностей
;Предметная область и онтология предметных областей
;Предметная область и онтология онтологий
;Предметная область и онтология логических формул, высказываний и формальных теорий
;Предметная область и онтология внешних информационных конструкций и файлов ostis-систем
;Глобальная предметная область действий и задач и соответствующая ей онтология методов и технологий}

\scnheader{Предметная область знаний и баз знаний ostis-систем}
\scniselement{предметная область}
\scnsdmainclasssingle{знание}
\scnhaselementlist{исследуемый класс классов}{вид знаний;отношение, заданное на множестве знаний}

\scnheader{знание}
\scnidtf{синтаксически корректная (для соответствующего языка) и семантически целостная информационная конструкция}
\scnsubset{информационная конструкция}
    \scnaddlevel{1}
    \scniselementrole{класс объектов исследования}{\nameref{intro_lang}}
    \scnaddlevel{-1}
\scnaddlevel{1}
\scnrelboth{следует отличать}{данные}
\scnaddlevel{1}
\scnexplanation{
	Принципиальные различия знаний и данных:
	
	\begin{scnitemize}
		\item \textit{Интерпретация}. Хранимые данные могут быть интерпретированы только человеком или программой. Данные не несут информации. Знания содержат как данные, так и их описание (правила интерпретации)
		\item \textit{Наличие связей классификации}. Данные не имеют эффективного описания связей между различными типами данных. Знания структурированы, так как можно установить соответствие между единицами знаний.
		\item \textit{Наличие ситуационных связей}. Связи описывают множество текущих ситуаций объекта. Данные трудно поддаются анализу. Из структуры и состава знаний по ситуации возможно построение процедур анализа знаний.
	\end{scnitemize}
}
\scnaddlevel{1}
\scnrelfrom{цитата}{Helpiks2015}
\scnaddlevel{-1}
\scnaddlevel{-1}
\scnaddlevel{-1}
\scnrelfrom{покрытие}{вид знаний
    \scnidtf{Множество \uline{всевозможных} видов знаний}
    \scnnote{Тот факт, что семейство \textit{видов знаний} является \textit{покрытием} Множества всевозможных \textit{знаний}, означает то, что каждое \textit{знание} принадлежит по крайней мере одному выделенному нами \textit{виду знаний}}}

    
    
   \scnheader{вид знаний}
\scnhaselement{спецификация}
    \scnaddlevel{1}
    \scnidtf{описание заданной сущности}
    \scnsuperset{спецификация материальной сущности}
    \scnsuperset{спецификация обратной сущности, не являющейся множеством}
        \scnaddlevel{1}
        \scnsuperset{спецификация геометрической точки}
        \scnsuperset{спецификация числа}
        \scnaddlevel{-1}
    \scnsuperset{спецификация множества}
        \scnaddlevel{1}
        \scnsuperset{спецификация связи}
        \scnsuperset{спецификация структуры}
        \scnsuperset{спецификация класса}
            \scnaddlevel{1}
            \scnsuperset{спецификация класса сущностей, не являющихся множествами}
            \scnsuperset{спецификация отношения}
                \scnaddlevel{1}
                \scnidtf{спецификация класса связей (связок)}
                \scnaddlevel{-1}
            \scnsuperset{спецификация класса классов}
                \scnaddlevel{1}
                \scnsuperset{спецификация параметра}
                \scnaddlevel{-1}
            \scnsuperset{спецификация класса структур}
            \scnsuperset{спецификация понятий}
                \scnaddlevel{1}
                \scnsuperset{пояснение}
                \scnsuperset{определение}
                \scnsuperset{утверждение}
                    \scnaddlevel{1}
                    \scnidtf{утверждение, описывающее свойства экземпляров (элементов) специфицируемого понятия}
                    \scnidtf{закономерность}
                    \scnaddlevel{-1}
                \scnaddlevel{-1}
            \scnaddlevel{-1}
        \scnaddlevel{-1}
    \scnsuperset{семантическая окрестность}
    \scnsuperset{однозначная спецификация}
    \scnsuperset{сравнительный анализ}
    \scnsuperset{достоинства}
    \scnsuperset{недостатки}
    \scnsuperset{структура специфицируемой сущности}
    \scnsuperset{принципы, лежащие в основе}
    \scnsuperset{обоснование предлагаемого решения}
        \scnaddlevel{1}
        \scnidtf{аргументация предлагаемого решения}
        \scnaddlevel{-1}
    \scnaddlevel{-1}

\scnhaselement{сравнение}

\scnhaselement{высказывание}
    \scnaddlevel{1}
    \scnsuperset{фактографическое высказывание}
    \scnsuperset{закономерность}
    \scnaddlevel{-1}

\scnhaselement{формальная теория}

\scnhaselement{предметная область}

\scnhaselement{предметная область и онтология
    \scnaddlevel{1}
    \scnidtf{предметная область и её онтология}
    \scnidtf{предметная область и соответствующая ей объединенная онтология}
    \scnaddlevel{-1}} 
 
\scnhaselement{метазнание}
    \scnaddlevel{1}
    \scnidtf{спецификация знания}
    \scnsuperset{аннотация}
    \scnsuperset{введение}
    \scnsuperset{предисловие}
    \scnsuperset{заключение}
    \scnsuperset{онтология}
        \scnaddlevel{1}
        \scnsuperset{онтология предметной области}
            \scnaddlevel{1}
            \scnsuperset{структурная онтология предметной области}
            \scnsuperset{теоретико-множественная онтология предметной области}
            \scnsuperset{логическая онтология предметной области}
            \scnsuperset{терминологическая онтология предметной области}
            \scnsuperset{объединенная онтология предметной области}
            \scnaddlevel{-1}
        \scnaddlevel{-1}
    \scnaddlevel{-1}

\scnhaselement{задача}
    \scnaddlevel{1}
    \scnidtf{спецификация действия}
    \scnaddlevel{-1}
\scnhaselement{план}

\scnhaselement{протокол}

\scnhaselement{результативная часть протокола}

\scnhaselement{метод}

\scnhaselement{технология}

\scnhaselement{история использования предметной области и её онтологии по решению информационных задач}
\scnhaselement{история использования предметной области и её онтологии по решению задач во внешней среде}
\scnhaselement{история эволюции предметной области и её онтологии}

\scnhaselement{база знаний}
    \scnaddlevel{1}
    \scnidtf{совокупность знаний, хранимых в памяти интеллектуальной компьютерной системы и \uline{достаточных} для того, чтобы указанная система удовлетворяла соответствующим предъявляемым к ней требованиям (в частности, чтобы она имела соответствующий уровень интеллекта)}
    \scnidtf{систематизированная совокупность знаний, хранимая в памяти интеллектуальной компьютерной системы и достаточная для обеспечения целенаправленного (целесообразного, адекватного) функционирования (поведения) этой системы как в своей внешней среде, так и в своей внутренней среде (в собственной базе знаний)}
    \scnrelfromset{обобщенная декомпозиция}{согласованная часть базы знаний
        \scnaddlevel{1}
        \scnidtf{часть базы знаний, признанная коллективом авторов на текущий момент}
        \scnaddlevel{-1}
    ;история эксплуатации базы знаний;история эволюции базы знаний;план эволюции базы знаний
        \scnaddlevel{1}
        \scnidtf{система специфицированных и согласованных действий авторов базы знаний, направленных на повышение её качества}
        \scnaddlevel{-1}}
    \scnnote{Основным факторами, определяющими качество интеллектуальной компьютерной системы, являются:
    \begin{scnitemize}
        \item качественная структуризация (систематизация) и \uline{стратификация} базы знаний интеллектуальной компьютерной системы, а также
        \item систематизация и стратификация \uline{деятельности}, которая осуществляется интеллектуальной компьютерной системой и спецификация которой является важнейшей частью базы знаний этой системы (Смотрите Раздел \textit{Глобальная предметная область действий и задач и соответствующая ей онтология методов и технологий}).
    \end{scnitemize}}
    \scnaddlevel{-1}
\scnnote{Даже небольшой перечень \textit{видов знаний} свидетельствует об огромном многообразии \textit{видов знаний}}

\scnheader{знание}
\scnsubdividing{декларативное знание
    \scnaddlevel{1}
    \scnidtf{\textit{знание}, имеющее \uline{только} \textit{денотационную семантику}, которая представляется в виде семантической \textit{спецификации} системы \textit{понятий}, используемых в этом \textit{знании}}
    \scnaddlevel{-1}
;процедурное знание
    \scnaddlevel{1}
    \scnidtf{\textit{знание}, имеющее не только \textit{денотационную семантику}, но и \textit{операционную семантику}, которая представляется в виде семейства \textit{спецификаций агентов}, осуществляющих интерпретацию \textit{процедурного знания}, направленную на решение некоторой инициированной \textit{задачи}}
    \scnidtf{функционально интерпретируемое знание, обеспечивающее решение либо конкретной задачи, либо некоторого множества инициируемых задач}
    \scnsuperset{задача}
        \scnaddlevel{1}
        \scnidtf{формулировка конкретной задачи}
        \scnsuperset{декларативная формулировка задачи}
        \scnsuperset{процедурная формулировка задачи}
        \scnaddlevel{-1}
    \scnsuperset{план}
        \scnaddlevel{1}
        \scnidtf{план решения конкретной задачи}
        \scnidtf{контекст конкретной задачи, предоставляющий всю информацию для решения всех подзадач для указанной конкретной задачи}
        \scnidtf{описание системы подзадач некоторой задачи}
        \scnaddlevel{-1}
    \scnsuperset{метод}
        \scnaddlevel{1}
        \scnidtf{обобщенное описание плана решения любой задачи из некоторого заданного класса задач}
        \scnaddlevel{-1}
    \scnsuperset{навык}
        \scnaddlevel{1}
        \scnidtf{метод, детализированный до уровня элементарных подзадач}
        \scnaddlevel{-1}
    \scnaddlevel{-1}}
    
\scnheader{отношение, заданное на множестве знаний}
\scnhaselement{дочернее знание*}
    \scnaddlevel{1}
    \scnidtf{знание, которое от "материнского"{} знания наследует все описанные там свойства объектов исследования}
    \scnnote{Факт наследования свойств описываемых объектов от "материнского"{} знания подчеркивается использованием прилагательного "дочернее"{} в sc-идентификаторе данного отношения, заданного на множестве знаний}
    \scnsuperset{дочерний раздел*}
        \scnaddlevel{1}
        \scnidtf{частный раздел*}
        \scnaddlevel{-1}
    \scnsuperset{дочерняя предметная область и онтология*}
    \scnaddlevel{-1}
\scnhaselement{спецификация*}
    \scnaddlevel{1}
    \scnidtf{быть знанием, которое является спецификацией (описанием) заданной сущности}
    \scnnote{специфицируемой сущностью может быть сущность любого вида, в том числе, и другое знание}
    \scnaddlevel{-1}
\scnhaselement{онтология*}
    \scnaddlevel{1}
    \scnidtf{быть семантической спецификацией заданного знания*}
    \scnaddlevel{-1}
\scnhaselement{семантическая эквивалентность*}
\scnhaselement{следовательно*}
    \scnaddlevel{1}
    \scnidtf{логическое следствие*}
    \scnaddlevel{-1}
\scnhaselement{логическая эквивалентность*}   
    
\bigskip    
\scnendstruct \scnendcurrentsectioncomment

\end{SCn}

\scsubsection[\scnidtf{Предметная область и онтология знаний о множествах}\protect\scnmonographychapter{Глава 2.4. Формальные онтологии базовых классов сущностей - множеств, связей, отношений, параметров, величин, чисел, структур, темпоральных сущностей}]{Предметная область и онтология множеств}
\label{sd_sets}
\begin{SCn}

\scnsectionheader{\currentname}

\scnstartsubstruct

\scnheader{Предметная область множеств}
\scnidtf{Теоретико-множественная предметная область}
\scnidtf{Предметная область теории множеств}
\scnidtf{Предметная область, объектами исследования которой являются множества}
\scniselement{предметная область}
\scnsdmainclasssingle{множество}
\scnsdclass{конечное множество;бесконечное множество;счетное множество;несчетное множество;множество без кратных элементов;мультимножество;кратность принадлежности;класс;класс первичных sc-элементов;класс множеств;класс структур;класс классов;нечеткое множество;четкое множество;множество первичных сущностей;семейство множеств;нерефлексивное множество;рефлексивное множество;множество первичных сущностей и множеств;сформированное множество;несформированное множество;пустое множество;синглетон;пара;пара разных элементов;пара-мультимножество;тройка;кортеж;декартово произведение;булеан;мощность}
\scnsdrelation{принадлежность*;пример\scnrolesign;включение*;строгое включение*;объединение*;разбиение*;пересечение*;пара пересекающихся множеств*;попарно пересекающиеся множества*;пересекающиеся множества*;пара непересекающихся множеств*;попарно непересекающиеся множества*;непересекающиеся множества*;разность множеств*;симметрическая разность множеств*;декартово произведение*;семейство подмножеств*;булеан*;равенство множеств*}

\scnheader{множество}
\scnidtf{множество sc-элементов}
\scnidtf{sc-множество}
\scnidtf{множество знаков}
\scnidtf{множество знаков описываемых сущностей}
\scnidtf{семантически нормализованное множество}
\scnidtf{sc-знак множества sc-элементов}
\scnidtf{sc-знак множества sc-знаков}
\scnidtf{sc-текст}
\scnidtf{текст SC-кода}
\scnidtf{SC-код}
\scnsubdividing{конечное множество;бесконечное множество}
\scnsubdividing{множество без кратных элементов;мультимножество}
\scnsubdividing{связка;класс\\
    \scnaddlevel{1}
    \scnidtf{sc-элемент, обозначающий класс sc-элементов}
    \scnidtf{sc-знак множества sc-элементов, эквивалентных в том или ином смысле}
    \scnaddlevel{-1}
    ;структура\\
    \scnaddlevel{1}
    \scnidtf{sc-знак множества sc-элементов, в состав которого входят sc-связки или sc-структуры, связывающие эти sc-элементы}
    \scnaddlevel{-1}}
\scnsubdividing{четкое множество;нечеткое множество}
\scnsubdividing{множество первичных сущностей;множество множеств;множество первичных сущностей и множеств}
\scnsubdividing{рефлексивное множество;нерефлексивное множество}
\scnsubdividing{сформированное множество;несформированное множество}
\scnsubdividing{кортеж;неориентированное множество}
\scnsuperset{пустое множество}
\scnsuperset{синглетон}
\scnsuperset{пара}
\scnsuperset{тройка}
\scnexplanation{Под \textbf{\textit{множеством}} понимается соединение в некое целое M определённых хорошо различимых предметов m нашего созерцания или нашего мышления (которые будут называться «элементами» множества M). 

\textbf{\textit{множество}} – мысленная сущность, которая связывает одну или несколько сущностей в целое.

Более формально \textbf{\textit{множество}} – это абстрактный математический объект, состоящий из элементов. Связь множеств с их элементами задается бинарным ориентированным отношением \textbf{\textit{принадлежность*}}.

\textbf{\textit{множество}} может быть полностью задано следующими тремя способами:
\begin{scnitemize}
    \item путем перечисления (явного указания) всех его элементов (очевидно, что таким способом можно задать только конечное множество)
    \item с помощью определяющего высказывания, содержащего описание общего характеристического свойства, которым обладают все те и только те объекты, которые являются элементами (т.е. принадлежат) задаваемого множества.
    \item с помощью теоретико-множественных операций, позволяющих однозначно задавать новые множества на основе уже заданных (это операции объединения, пересечения, разности множеств и др.)
\end{scnitemize}
Для любого семантически ненормализованного \textbf{\textit{множества}} существует единственное семантически нормализованное \textbf{\textit{множество}}, в котором все элементы, не являющиеся знаками множеств, заменены на знаки множеств.}

\scnheader{принадлежность*}
\scnidtf{принадлежность элемента множеству*}
\scnidtf{отношение принадлежности элемента множеству*}
\scniselement{бинарное отношение}
\scniselement{ориентированное отношение}
\scnexplanation{\textbf{\textit{принадлежность*}} – это бинарное ориентированное отношение, каждая связка которого связывает множество с одним из его элементов. Элементами отношения \textbf{\textit{принадлежность*}} по умолчанию являются \textit{позитивные sc-дуги принадлежности}.}

\scnheader{конечное множество}
\scnidtf{множество с конечным числом элементов}
\scnexplanation{\textbf{\textit{конечное множество}} – это \textit{множество}, количество элементов которого конечно, т.е. существует неотрицательное целое число \textit{k}, равное количеству элементов этого множества.}

\scnheader{бесконечное множество}
\scnidtf{множество с бесконечным числом элементов}
\scnsubdividing{счетное множество;несчетное множество}
\scnexplanation{\textbf{\textit{бесконечное множество}} – это \textit{множество}, в котором для любого натурального числа \textit{n} найдётся конечное подмножество из \textit{n} элементов.}

\scnheader{счетное множество}
\scnexplanation{\textbf{\textit{счетное множество}} – это \textit{бесконечное множество}, для которого существует \textit{взаимно-однозначное соответствие} с натуральным рядом чисел.}

\scnheader{несчетное множество}
\scnidtf{континуальное множество}
\scnexplanation{\textbf{\textit{несчетное множество}} - это \textit{бесконечное множество}, элементы которого невозможно пронумеровать натуральными числами.}

\scnheader{множество без кратных элементов}
\scnidtf{классическое множество}
\scnidtf{канторовское множество}
\scnidtf{множество, состоящее из разных элементов}
\scnidtf{множество без кратного вхождения элементов}
\scnidtf{множество, все элементы которого входят в него однократно}
\scnidtf{множество, не имеющее кратного вхождения элементов}
\scnexplanation{\textbf{\textit{множество без кратных элементов}} - это \textit{множество}, для каждого элемента которого существует только одна пара принадлежности, выходящая из знака этого множества в указанный элемент.}

\scnheader{мультимножество}
\scnidtf{множество, имеющее кратные вхождения хотя бы одного элемента}
\scnidtf{множество, по крайней мере один элемент которого входит в его состав многократно}
\scnexplanation{\textbf{\textit{мультимножество}} - это \textit{множество}, для которого существует хотя бы одна кратная пара принадлежности, выходящая из знака этого множества.}

\scnheader{кратность принадлежности}
\scnidtf{кратность принадлежности элемента}
\scnidtf{кратность вхождения элемента во множество}
\scniselement{параметр}
\scnexplanation{\textbf{\textit{кратность принадлежности}} - \textit{параметр}, значением которого являются числовые величины, показывающие сколько раз входит тот или иной элемент в рассматриваемое множество. Элементами данного параметра являются классы \textit{позитивных sc-дуг принадлежности}, связывающих данное множество с элементом, кратность вхождения которого в данное множество мы хотим задать.

Таким образом, кратное вхождение элемента в мультимножество может быть задано как явным указанием \textit{позитивных sc-дуг принадлежности} этого элемента данному \textit{множеству}, так и «склеиванием» этих дуг в одну и включением ее в некоторый класс \textbf{\textit{кратности принадлежности}}.}
\scnrelfrom{описание примера}{
\scnfilescg{figures/sd_sets/multiplicityOfMembership.png}
}

\scnheader{класс}
\scnidtf{класс sc-элементов}
\scnsubdividing{класс первичных sc-элементов;класс множеств}
\scnexplanation{\textbf{\textit{класс}} – множество элементов, обладающих какими-либо явно указываемыми общими свойствами.}

\scnheader{класс первичных sc-элементов}
\scnexplanation{\textbf{\textit{класс первичных sc-элементов}} – класс, элементами которого являются только \textit{sc-элементы}, не являющиеся знаками множеств.}

\scnheader{класс множеств}
\scnsubdividing{отношение;класс структур;класс классов}
\scnexplanation{\textbf{\textit{класс множеств}} – класс, элементами которого являются только \textit{sc-элементы}, являющиеся знаками множеств.}

\scnheader{класс структур}
\scnexplanation{\textbf{\textit{класс структур}} – класс, элементами которого являются \textit{структуры}.}

\scnheader{класс классов}
\scnexplanation{\textbf{\textit{класс классов}} – класс, элементами которого являются \textit{классы}.}

\scnheader{нечеткое множество}
\scnexplanation{\textbf{\textit{нечеткое множество}} – это \textit{множество}, которое представляет собой совокупность элементов произвольной природы, относительно которых нельзя точно утверждать – обладают ли эти элементы некоторым характеристическим свойством, которое используется для задания этого нечеткого множества. Принадлежность элементов такому множеству указывается при помощи \textit{нечетких позитивных sc-дуг принадлежности}.}

\scnheader{четкое множество}
\scnexplanation{\textbf{\textit{четкое множество}} – это \textit{множество}, принадлежность элементов которому достоверна и указывается при помощи \textit{четких позитивных sc-дуг принадлежности}.}

\scnheader{множество первичных сущностей}
\scnsuperset{класс первичных сущностей}
\scnsubset{множество}
\scnexplanation{\textbf{\textit{множество первичных сущностей}} – это \textit{множество}, элементы которого не являются знаками множеств.}

\scnheader{семейство множеств}
\scnidtf{множество множеств}
\scnsuperset{класс классов}
\scnexplanation{\textbf{\textit{семейство множеств}} – это \textit{множество}, элементами которого являются знаки множеств.}

\scnheader{нерефлексивное множество}
\scnexplanation{\textbf{\textit{нерефлексивное множеств}} – это \textit{множество}, знак которого не является элементом этого множества}

\scnheader{рефлексивное множество}
\scnexplanation{\textbf{\textit{рефлексивное множеств}} – это \textit{множество}, знак которого является элементом этого множества}

\scnheader{множество первичных сущностей и множеств}
\scnexplanation{\textbf{\textit{множество первичных сущностей и множеств}} – это \textit{множество}, элементами которого являются как знаки множеств, так и знаки сущностей, не являющихся множествами.}

\scnheader{сформированное множество}
\scniselement{ситуативное множество}
\scnexplanation{\textbf{\textit{сформированное множество }} - это \textit{множество}, все элементы которого известны и перечислены в данный момент.}

\scnheader{несформированное множество}
\scniselement{ситуативное множество}
\scnexplanation{\textbf{\textit{несформированное множество}} - это \textit{множество}, не все элементы которого известны и перечислены в данный момент.}

\scnheader{пустое множество}
\scniselement{мощность}
\scnexplanation{\textbf{\textit{пустое множество}} – это \textit{множество}, которому не принадлежит ни один элемент.}

\scnheader{синглетон}
\scniselement{мощность}
\scnidtf{множество мощности 1}
\scnidtf{одноэлементное множество}
\scnidtf{одномощное множество}
\scnidtf{множество, мощность которого равна 1}
\scnidtf{множество, имеющее мощность равную единице}
\scnidtf{синглетон из sc-элемента} 
\scnidtf{sc-синглеон}
\scnsubset{конечное множество}
\scnexplanation{\textbf{\textit{синглетон}} – это \textit{множество}, состоящее из одного элемента.

Другими словами - любое множество \textit{Si} есть \textbf{\textit{синглетон}} тогда и только тогда, когда существует принадлежность этому множеству, которая совпадает с любой принадлежностью этому множеству.}

\scnheader{пара}
\scniselement{мощность}
\scnidtf{множество мощности два}
\scnidtf{двухэлементное множество}
\scnidtf{двумощное множество}
\scnidtf{множество, мощность которого равна 2}
\scnidtf{sc-пара}
\scnidtf{пара sc-элементов}
\scnsubset{конечное множество}
\scnsubdividing{пара разных элементов;пара-мультимножество}
\scnexplanation{\textbf{\textit{пара}} – это \textit{множество}, состоящее из двух элементов.

Другими словами – любое множество есть \textbf{\textit{пара}} тогда и только тогда, когда существуют две различные принадлежности этому множеству такие, что любая принадлежность этому множеству совпадает с одной из них.}

\scnheader{пара разных элементов}
\scnidtf{канторовская пара}
\scnidtf{канторовская пара sc-элементов}
\scnidtf{канторовское двумощное множество}

\scnheader{пара-мультимножество}
\scnidtf{пара-петля}
\scnidtf{sc-петля}
\scnidtf{двумощное мультимножество}

\scnheader{тройка}
\scniselement{мощность}
\scnidtf{тройка}
\scnidtf{sc-тройка}
\scnidtf{множество мощности три}
\scnidtf{множество, мощность которого равна 3}
\scnsubset{конечное множество}
\scnexplanation{\textbf{\textit{тройка}} – это \textit{множество}, состоящее из трех элементов.

Другими словами – любое множество есть \textbf{\textit{тройка}} тогда и только тогда, когда существуют три различные принадлежности этому множеству такие, что любая принадлежность этому множеству совпадает с одной из них.}

\scnheader{кортеж}
%\scnidtf{кортеж}
\scnidtf{вектор}
\scnexplanation{\textbf{\textit{кортеж}} – это множество, представляющее собой упорядоченный набор элементов, т.е. такое множество, порядок элементов в котором имеет значение. Пары принадлежности элементов \textbf{\textit{кортежу}} могут дополнительно принадлежать каким-либо \textit{ролевым отношениям}, при этом, в рамках каждого \textbf{\textit{кортежа}} должен существовать хотя бы один элемент, роль которого дополнительно уточнена \textit{ролевым отношением}.}

\scnheader{пример\scnrolesign}
\scnidtf{типичный пример\scnrolesign}
\scnidtf{типичный экземпляр заданного класса\scnrolesign}
\scniselement{ролевое отношение}
\scnexplanation{\textbf{\textit{пример\scnrolesign}} – это \textit{ролевое отношение}, связывающее некоторое \textit{множество} с элементом, являющимся примером этого множества.}

\scnheader{включение*}
\scnidtf{включение множеств*}
\scnidtf{быть подмножеством*}
\scniselement{бинарное отношение}
\scniselement{ориентированное отношение}
\scniselement{транзитивное отношение}
\scnrelfrom{область определения}{множество}
\scnsuperset{строгое включение*}
\scntext{определение}{\textbf{\textit{включение*}} – это бинарное ориентированное отношение, каждая связка которого связывает два множества. Будем говорить, что \textit{Множество Si} \textbf{\textit{включает*}} в себя \textit{Множество Sj} в том и только том случае, если каждый элемент \textit{Множества Sj} является также и элементом \textit{Множества Si}.}
\scnrelfrom{описание примера}{
\scnfilescg{figures/sd_sets/inclusion.png}}
\scnaddlevel{1}
\scnexplanation{Множество {Sj} включается во множество \textit{Si}.}
\scnaddlevel{-1}

\scnheader{строгое включение*}
\scnidtf{строгое включение множеств*}
\scnsubset{включение*}
\scniselement{бинарное отношение}
\scniselement{ориентированное отношение}
\scnrelfrom{область определения}{множество}
\scntext{определение}{\textbf{\textit{строгое включение*}} – это \textit{бинарное ориентированное отношение}, областью определения которого является семейство всевозможных множеств. Будем говорить, что \textit{Множество Si} \textbf{\textit{строго включает*}} в себя \textit{Множество Sj} в том и только том случае, если каждый элемент \textit{Множество Sj} является также и элементом \textit{Множество Si}, при этом существует хотя бы один элемент \textit{Множество Si}, не являющийся элементом \textit{Множество Sj}.}
\scnrelfrom{описание примера}{
\scnfilescg{figures/sd_sets/strictInclusion.png}}
\scnaddlevel{1}
\scnexplanation{Множество \textit{Sj} строго включается во множество \textit{Si}.}
\scnaddlevel{-1}
\scnrelfrom{изображение}{
\scnfileimage{\includegraphics[width=0.4\linewidth]{figures/sd_sets/inclusion2.png}}}

\scnheader{объединение*}
\scnidtf{объединение множеств*}
\scniselement{квазибинарное отношение}
\scniselement{ориентированное отношение}
\scntext{определение}{\textbf{\textit{объединение*}} – это \textit{квазибинарное ориентированное отношение}, областью определения которого является семейство всевозможных множеств. Будем говорить, что \textit{Множество Si} является объединением \textit{Множество Sj} и \textit{Множество Sk} тогда и только тогда, когда любой элемент \textit{Множество Si} является элементом или \textit{Множество Sj} или \textit{Множество Sk}.}
\scnrelfrom{описание примера}{
\scnfilescg {figures/sd_sets/union.png}}
\scnaddlevel{1}
\scnexplanation{Множество \textit{Si} является объединением множеств \textit{Sj}, \textit{Sk} и \textit{Sm}.}
\scnaddlevel{-1}
\scnrelfrom{изображение}{
\scnfileimage{\includegraphics[width=0.6\linewidth]{figures/sd_sets/union2.png}}}

\scnheader{разбиение*}
\scnidtf{разбиение  множества*}
\scnidtf{объединение попарно непересекающихся множеств*}
\scnidtf{декомпозиция множества*}
\scniselement{квазибинарное отношение}
\scniselement{ориентированное отношение}
\scniselement{отношение декомпозиции}
\scntext{определение}{\textbf{\textit{разбиение*}} – это \textit{квазибинарное ориентированное отношение}, областью определения которого является семейство всевозможных множеств. В результате разбиения множества получается множество попарно непересекающихся множеств, объединение которых есть исходное множество.\\
Семейство множеств \{\textit{S1…, Sn}\} является разбиением множества \textit{Si} в том и только том случае, если:
\begin{scnitemize}
    \item семейство \{\textit{S1…, Sn}\} является семейством \textit{попарно непересекающихся множеств};
    \item семейство \{\textit{S1…, Sn}\} является покрытием множества \textit{Si} (или другими словами, множество \textit{Si} является \textit{объединением} множеств, входящих в указанное выше семейство)
\end{scnitemize}
}
\scnrelfrom{описание примера}{
\scnfilescg{figures/sd_sets/split.png}}
\scnaddlevel{1}
\scnexplanation{Множество \textit{Si} разбивается на множества \textit{Sj}, \textit{Sk} и \textit{Sm}.}
\scnaddlevel{-1}
\scnrelfrom{изображение}{
\scnfileimage{\includegraphics[width=0.5\linewidth]{figures/sd_sets/split2.png}}}

\scnheader{пересечение*}
\scnidtf{пересечение множеств*}
\scniselement{квазибинарное отношение}
\scniselement{ориентированное отношение}
\scntext{определение}{\textbf{\textit{пересечение*}} – это операция над множествами, аргументами которой являются два или большее число множеств, а результатом является множество, элементами которого являются все те и только те сущности, которые одновременно принадлежат каждому множеству, которое входит в семейство аргументов этой операции.\\
Будем говорить, что \textit{Множество Si} является пересечением \textit{Множество Sj} и \textit{Множество Sk} тогда и только тогда, когда любой элемент \textit{Множество Si} является элементом \textit{Множество Sj} и элементом \textit{Множество Sk}.}
\scnrelfrom{описание примера}{
\scnfilescg{figures/sd_sets/intersection.png}}
\scnaddlevel{1}
\scnexplanation{Множество \textit{Si} является результатом пересечения множеств \textit{Sj}, \textit{Sk} и \textit{Sm}.}
\scnaddlevel{-1}
\scnrelfrom{изображение}{
\scnfileimage{\includegraphics[width=0.5\linewidth]{figures/sd_sets/intersection2.png}}}

\scnheader{пара пересекающихся множеств*}
\scniselement{бинарное отношение}
\scniselement{неориентированное отношение}
\scnexplanation{\textbf{\textit{пара пересекающихся множеств*}} – \textit{бинарное неориентированное отношение} между двумя \textit{множествами}, имеющими непустое \textit{пересечение*}.}
\scntext{определение}{\textbf{\textit{пара пересекающихся множеств*}} – \textit{бинарное неориентированное отношение} между двумя \textit{множествами}, имеющими, по крайней мере, один общий для этих двух множеств элемент.}
\scnrelfrom{описание примера}{
\scnfilescg{figures/sd_sets/pairOfIntersectingSets.png}}
\scnaddlevel{1}
\scnexplanation{Множество \textit{Si} и множество \textit{Sj} являются парой пересекающихся множеств.}
\scnaddlevel{-1}
\scnrelfrom{изображение}{
\scnfileimage{\includegraphics[width=0.5\linewidth]{figures/sd_sets/pairOfIntersectingSets2.png}}}

\scnheader{попарно пересекающиеся множества*}
\scnidtf{семейство попарно пересекающихся множеств*}
\scnsuperset{пересекающиеся множества*}
\scniselement{отношение}
\scntext{определение}{\textbf{\textit{попарно пересекающиеся множества*}} – семейство множеств, каждая пара которых является парой пересекающихся множеств, т.е. каждая пара которых имеет хотя бы один общий элемент}
\scntext{примечание}{Не каждое \textit{семейство попарно пересекающихся множеств*} является \textit{семейством пересекающихся множеств*}, хотя обратное верно.}
\scnrelfrom{изображение}{
\scnfilescg{figures/sd_sets/pairwiseIntersectingSets.png}}
\scnaddlevel{1}
\scnexplanation{Множества \textit{Si}, \textit{Sj}, \textit{Sk} и \textit{Sl} являются попарно пересекающимися множествами.}
\scnaddlevel{-1}
\scnrelfrom{изображение}{
\scnfileimage{\includegraphics[width=0.7\linewidth]{figures/sd_sets/pairwiseIntersectingSets2.png}}}

\scnheader{пересекающиеся множества*}
\scnidtf{семейство пересекающихся множеств*}
\scnidtf{быть семейством пересекающихся множеств*}
\scnidtf{семейство множеств, имеющих по крайней мере один элемент, являющийся общим для всех этих множеств*}
\scnsuperset{попарно пересекающиеся множества*}
\scntext{определение}{\textbf{\textit{пересекающиеся множества*}} – это семейство множеств, имеющих по крайней мере один общий для всех этих множеств элемент}
\scnrelfrom{описание примера}{
\scnfilescg{figures/sd_sets/intersectingSets.png}}
\scnaddlevel{1}
\scnexplanation{Множества \textit{Si}, \textit{Sj}, \textit{Sk} и \textit{Sl} являются пересекающимися множествами.}
\scnaddlevel{-1}

\scnheader{пара непересекающихся множеств*}
\scniselement{бинарное отношение}
\scniselement{неориентированное отношение}
\scntext{определение}{\textbf{\textit{пара непересекающихся множеств*}} – это \textit{бинарное неориентированное отношение} между \textit{множествами}, результатом \textit{пересечения*} которых есть пустое множество.}
\scnrelfrom{описание примера}{
\scnfilescg{figures/sd_sets/pairOfNonIntersectingSets.png}}
\scnaddlevel{1}
\scnexplanation{Множества \textit{Si} и \textit{Sj} являются парой непересекающихся множеств.}
\scnaddlevel{-1}
\scnrelfrom{изображение}{
\scnfileimage{\includegraphics[width=0.5\linewidth]{figures/sd_sets/pairOfNonIntersectingSets2.png}}}

\scnheader{попарно непересекающиеся множества*}
\scnidtf{семейство попарно непересекающихся множеств*}
\scnsubset{непересекающиеся множества*}
\scntext{определение}{\textbf{\textit{попарно непересекающиеся множества*}} – семейство множеств, каждая пара которых является парой непересекающихся множеств, т.е. каждая пара которых не имеет ни одного общего элемента}
\scnrelfrom{изображение}{
\scnfilescg{figures/sd_sets/pairwiseNonIntersectingSets.png}}
\scnaddlevel{1}
\scnexplanation{Множества \textit{Si}, \textit{Sj}, \textit{Sk} и \textit{Sl} являются попарно непересекающимися множествами.}
\scnaddlevel{-1}

\scnheader{непересекающиеся множества*}
\scnidtf{семейство непересекающихся множеств*}
\scnidtf{быть семейством непересекающихся множеств*}
\scntext{определение}{\textbf{\textit{непересекающиеся множества*}} – это семейство множеств, не имеющих ни одного общего элемента для всех этих множеств}
\scnrelfrom{изображение}{
\scnfilescg{figures/sd_sets/nonIntersectingSets.png}
\scnexplanation{Множества \textit{Si}, \textit{Sj}, \textit{Sk} и \textit{Sl} являются непересекающимися множествами.}}

\scnheader{разность множеств*}
\scniselement{бинарное отношение}
\scniselement{ориентированное отношение}
\scntext{определение}{\textbf{\textit{разность множеств*}} – это \textit{бинарное ориентированное отношение}, связывающее между собой \textit{ориентированную пару}, первым элементом которой является уменьшаемое множество, вторым - вычитаемое множество, и множество, являющееся результатом вычитания вычитаемого из уменьшаемого, в которое входят все элементы первого множества, не входящие во второе множество.}
\scnrelfrom{описание примера}{
\scnfilescg{figures/sd_sets/setDifference.png}}
\scnaddlevel{1}
\scnexplanation{Множество \textit{Si} является результатом разности множеств \textit{Sj} и \textit{Sk}.}
\scnaddlevel{-1}
\scnrelfrom{изображение}{
\scnfileimage{
\includegraphics[width=0.5\linewidth]{figures/sd_sets/setDifference2.png}}}

\scnheader{симметрическая разность множеств*}
\scniselement{бинарное отношение}
\scniselement{ориентированное отношение}
\scntext{определение}{\textbf{\textit{симметрическая разность множеств*}} – это \textit{бинарное ориентированное отношение}, связывающее между собой \textit{пару} множеств и множество, являющееся результатом симметрической разности элементов указанной пары. Будем называть \textit{Множество Si} результатом симметрической разности \textit{Множества Sj} и \textit{Множества Sk} тогда и только тогда, когда любой элемент \textit{Множества Si} является или элементом \textit{Множества Sj} или \textit{Множества Sk}, но не является элементом обоих множеств.}
\scnrelfrom{описание примера}{
\scnfilescg{figures/sd_sets/symmetricDifferenceOfSets.png}
\scnexplanation{Множество \textit{Si} является результатом симметрической разности множеств \textit{Sj} и \textit{Sk}.}}
\scnrelfrom{изображение}{
\scnfileimage{\includegraphics[width=0.5\linewidth]{figures/sd_sets/symmetricDifferenceOfSets2.png}}}

\scnheader{декартово произведение*}
\scnidtf{декартово произведение множеств*}
\scnidtf{прямое произведение множеств*}
\scniselement{бинарное отношение}
\scniselement{ориентированное отношение}
\scntext{определение}{\textbf{\textit{декартово произведение*}} – это \textit{бинарное ориентированное отношение} между \textit{ориентированной парой} множеств и \textit{множеством}, элементами которого являются всевозможные упорядоченные пары, первыми элементами которых являются элементы первого из указанных множеств, вторыми – элементы второго из указанных множеств.}
\scnrelfrom{описание примера}{
\scnfilescg{figures/sd_sets/cartesianMultiplication.png}}
\scnaddlevel{1}
\scnexplanation{Множество \textit{Si} является результатом декартова произведения множеств \textit{Sj} и \textit{Sk}.}
\scnaddlevel{-1}

\scnheader{декартово произведение}
\scnidtf{второй домен отношения быть декартовым произведением}
\scnrelfrom{второй домен}{декартово произведение*}

\scnheader{семейство подмножеств*}
\scnidtf{семейство подмножеств заданного множества*}
\scniselement{бинарное отношение}
\scniselement{ориентированное отношение}
\scnsuperset{булеан*}
\scntext{определение}{\textbf{\textit{семейство подмножеств*}} – это \textit{бинарное ориентированное отношение} между множеством и некоторым семейством множеств, каждое из которых является подмножеством первого множества.}
\scnrelfrom{описание примера}{
\scnfilescg{figures/sd_sets/familyOfSubsets.png}
}

\scnheader{булеан*}
\scnidtf{булеан множества*}
\scnidtf{семейство всевозможных подмножеств заданного множества*}
\scniselement{бинарное отношение}
\scniselement{ориентированное отношение}
\scntext{определение}{\textbf{\textit{булеан*}} – это \textit{бинарное ориентированное отношение} между множеством и некоторым семейством множеств, каждое из которых является подмножеством первого множества.}
\scnrelfrom{описание примера}{
\scnfilescg{figures/sd_sets/boulean.png}
}

\scnheader{булеан}
\scnidtf{второй домен отношения быть булеаном}
\scnrelfrom{второй домен}{булеан*}

\scnheader{мощность}
\scnidtf{мощность множеств}
\scnidtf{кардинальное число}
\scnidtf{общее число вхождений элементов в заданное множество}
\scnidtf{класс эквивалентности, элементами которого являются знаки всех тех и только тех множеств, которые имеют одинаковую мощность}
\scnidtf{класс эквивалентности, соответствующий отношению быть парой множеств, имеющих одинаковую мощность (равномощность множеств)}
\scnidtf{величина мощности множеств}
\scnidtf{трансфинитное число}
\scnidtf{мощность по Кантору}
\scniselement{параметр}
\scnexplanation{\textbf{\textit{мощность}} – это \textit{параметр}, элементами которых являются \textit{множества}, имеющие одинаковое количество элементов. Значением данного параметра является числовая величина, задающая количество элементов, входящих в данный класс множеств, т.е. по сути, количество \textit{позитивных sc-дуг принадлежности}, выходящих из данного \textit{множества}.

Для двух множеств, имеющих одинаковую мощность, существует взаимно-однозначное соответствие между ними (между множествами вхождений элементов в эти множества – на случай мультимножеств).}
\scnrelfrom{описание примера}{
\scnfilescg{figures/sd_sets/power.png}
}

\scnheader{равенство множеств*}
\scniselement{бинарное отношение}
\scniselement{неориентированное отношение}
\scnidtf{быть равными множествами*}
\scntext{определение}{\textbf{\textit{равенство множеств}}* - бинарное неориентированное отношение, выражающее отношение равенства множеств.

Любые два множества являются равными множествами тогда и только тогда, когда первое является включением второго и второе является включением первого.}
\scnrelfrom{описание примера}{
\scnfilescg{figures/sd_sets/equalityOfSets.png}}
\scnaddlevel{1}
\scnexplanation{Множество \textit{Si} равно множеству \textit{Sj}.}
\scnaddlevel{-1}

\scnendstruct \scnendcurrentsectioncomment

\end{SCn}

\scsubsection[\scnmonographychapter{Глава 2.4. Формальные онтологии базовых классов сущностей - множеств, связей, отношений, параметров, величин, чисел, структур, темпоральных сущностей}]{Предметная область и онтология связок и отношений}
\label{sd_rels}
\begin{SCn}

\scnsectionheader{\currentname}

\scnstartsubstruct

\scnheader{Предметная область связок и отношений}
\scniselement{предметная область}
\scnsdmainclasssingle{связь}
\scnsdclass{бинарная связь;sc-коннектор;неатомарная бинарная связь;небинарная связь;неориентированная связь;ориентированная связь;отношение;класс равномощных связок;класс связок разной мощности;унарное отношение;бинарное отношение;квазибинарное отношение;тернарное отношение;небинарное отношение;ориентированное отношение;неориентированное отношение;рефлексивное отношение;антирефлексивное отношение;частично рефлексивное отношение;симметричное отношение;антисимметричное отношение;частично симметричное отношение;транзитивное отношение;антитранзитивное отношение;частично транзитивное отношение;связанное отношение;отношение порядка;отношение строгого порядка;отношение нестрогого порядка;отношение толерантности;отношение эквивалентности;ролевое отношение;числовой атрибут;неролевое отношение;неролевое бинарное отношение;арность;метаотношение;отношение декомпозиции;отношение интеграции}
\scnsdrelation{область определения*;атрибут отношения*;домен*;первый домен*;второй домен*;композиция отношений*;фактор-множество*;соответствие*;отношение соответствия*;область отправления';область прибытия’;образ';прообраз';всюду определенное соответствие*;частично определенное соответствие*;сюръективное соответствие*;несюръективное соответствие*;однозначное соответствие*;обратное соответствие*;обратимое соответствие*;неоднозначное соответствие*;инъективное соответствие*;взаимно однозначное соответствие*;множество сочетаний*;множество размещений*;множество перестановок*}

\scnheader{связь}
\scnidtf{связка sc-элементов}
\scnidtf{sc-связка}
\scnexplanation{\textbf{\textit{связь}} – множество, являющееся абстрактной моделью связи между описываемыми сущностями, которые или знаки которых являются элементами этого множества.}
\scntext{примечание}{Напомним, что все элементы множества, представленного в SC-коде, являются знаками, но описываемыми сущностями могут быть не только сущности, обозначаемые sc-элементами, но и сами эти sc-элементы.}
\scnsubdividing{бинарная связь;небинарная связь}
\scnsubdividing{неориентированная связь;ориентированная связь}

\scnheader{бинарная связь}
\scnsubdividing{sc-коннектор;неатомарная бинарная связь}
\scntext{примечание}{Данное разбиение осуществляется на основе синтаксического признака, а не семантического, поскольку каждый \textit{sc-коннектор} может быть записан в памяти при помощи семантически эквивалентной конструкции, содержащей знак самой связи и пары принадлежности, ведущие к ее элементам, уточненные, при необходимости ролевыми отношениями.}

\scnheader{sc-коннектор}
\scnidtf{атомарная бинарная связь}
\scnexplanation{Каждый \textbf{\textit{sc-коннектор}} представлен в \textit{sc-памяти} одним \textit{sc-элементом} и семантически эквивалентен конструкции, содержащей знак некоторой \textit{бинарной связи} и пары принадлежности, ведущие к элементам этой связи, уточненные, при необходимости ролевыми отношениями.

Такая конструкция может быть обозначена \textbf{\textit{sc-коннектором}} только в случае, когда роли компонентов соответствующей бинарной связи указываются только при помощи \textit{числовых атрибутов 1’} и \textit{2’} или не уточняются вообще.}

\scnheader{неатомарная бинарная связь}
\scnexplanation{\textbf{\textit{неатомарная бинарная связь}} – \textit{бинарная связь}, роли компонентов которой не могут быть заданы только при помощи \textit{ролевых отношений 1'} и \textit{2'}, или не заданы совсем, а требуют дополнительного уточнения при помощи более частных ролевых отношений.}

\scnheader{небинарная связь}
\scnexplanation{\textbf{\textit{небинарная связь}} – связь, имеющая больше двух элементов.}

\scnheader{неориентированная связь}
\scnsuperset{неориентированное множество}
\scnexplanation{\textbf{\textit{неориентированная связь}} – связь, все элементы которой имеют одинаковые роли (при этом соответствующее ролевое отношение, как правило, явно не указывается).}

\scnheader{ориентированная связь}
\scnsuperset{ориентированное множество}
\scnexplanation{\textbf{\textit{ориентированная связь}} – связь, в которой с помощью ролевых отношений, указываются роли компонентов этой связи.}

\scnheader{отношение}
\scnidtf{класс связей}
\scnidtf{класс sc-связок}
\scnidtf{множество отношений}
\scnidtf{Множество всевозможных отношений}
\scntext{определение}{\textbf{\textit{отношение}}, \textit{заданное на множестве M} – это подмножество \textit{декартового произведения} этого множества самого на себя некоторое количество раз.

В более широком смысле \textbf{\textit{отношение}} – это математическая структура, которая формально определяет свойства различных объектов и их взаимосвязи.}
\scnsubdividing{класс равномощных связок;класс связок разной мощности}
\scnsubdividing{бинарное отношение;небинарное отношение}
\scnsubdividing{ориентированное отношение;неориентированное отношение}
\scnsubdividing{ролевое отношение;неролевое отношение}

\scnheader{класс равномощных связок}
\scnidtf{класс связок фиксированной арности}
\scnidtf{отношение, обладающее свойством арности}
\scnsuperset{унарное отношение}
\scnsuperset{бинарное отношение}
\scnsuperset{тернарное отношение}
\scntext{определение}{\textbf{\textit{класс равномощных связок}} – класс связок, имеющих одинаковую мощность.}

\scnheader{класс связок разной мощности}
\scnidtf{отношение нефиксированной арности}
\scnsubset{небинарное отношение}
\scntext{определение}{\textbf{\textit{класс связок разной мощности}} – класс связок, имеющих разную мощность.}

\scnheader{унарное отношение}
\scnidtf{отношение арности один}
\scnidtf{одноместное отношение}
\scnidtf{множество синглетонов}
\scntext{определение}{\textbf{\textit{унарное отношение}} – это множество таких отношений на множестве M, являющихся любым подмножеством множества M.}

\scnheader{бинарное отношение}
\scnidtf{отношение арности два}
\scnidtf{двухместное отношение}
\scnsuperset{квазибинарное отношение}
\scnsuperset{отношение порядка}
\scnsuperset{отношение толерантности}
\scnsubdividing{рефлексивное отношение;антирефлексивное отношение;частично рефлексивное отношение}
\scnsubdividing{симметричное отношение;антисимметричное отношение;частично симметричное отношение}
\scnsubdividing{транзитивное отношение;антитранзитивное отношение;частично транзитивное отношение}
\scnsubdividing{ролевое отношение;неролевое бинарное отношение}
\scntext{определение}{\textbf{\textit{бинарное отношение}} – это множество таких отношений на множестве \textbf{\textit{M}}, являющихся подмножеством \textit{декартова произведения} множества \textbf{\textit{M}}.\\
Если \textbf{\textit{бинарное отношение R}} задано на \textit{множестве} \textbf{\textit{М}} и два элемента этого множества \textbf{\textit{a}} и \textbf{\textit{b}} связаны данным отношением, то будем обозначать такую связь как \textbf{\textit{aRb}}.}

\scnheader{квазибинарное отношение}
\scnexplanation{\textbf{\textit{квазибинарное отношение}} – множество ориентированных пар, первые компоненты которых являются связками.\\
Таким образом, \textit{sc-дуги}, принадлежащие \textbf{\textit{квазибинарным отношениям}}, всегда выходят из связок.}
\scntext{sc-утверждение}{В область определения квазибинарного отношения будем включать:
\begin{scnitemize}
    \item вторые компоненты ориентированных пар, принадлежащих этому отношению;
    \item элементы первых компонентов ориентированных пар, принадлежащих этому отношению;
    \item других элементов область определения квазибинарного отношения не содержит.
\end{scnitemize}
}

\scnheader{тернарное отношение}
\scnidtf{отношение арности три}
\scnidtf{трехместное отношение}
\scnexplanation{\textbf{\textit{тернарное отношение}} – это множество отношений, связывающих между собой три элемента.}

\scnheader{небинарное отношение}
\scnexplanation{\textbf{\textit{небинарное отношение}} – это множество отношений, хотя бы одна из связок каждого из которых имеет значение мощности больше двух.}

\scnheader{ориентированное отношение}
\scntext{определение}{\textbf{\textit{ориентированное отношение}} – это множество таких отношений, каждая связка которых является ориентированным множеством.}

\scnheader{неориентированное отношение}
\scntext{определение}{\textbf{\textit{неориентированное отношение}} – это множество таких отношений, каждая связка которых является неориентированным множеством.}

\scnheader{рефлексивное отношение}
\scntext{определение}{\textbf{\textit{рефлексивное отношение}} – это \textit{бинарное отношение}, любая пара которого есть канторовское множество.}

\scnheader{антирефлексивное отношение}
\scntext{определение}{\textbf{\textit{антирефлексивное отношение R}} на \textit{множестве} \textbf{\textit{A}} – это \textit{бинарное отношение}, в котором все элементы множества \textbf{\textit{A}} не находятся в отношении \textbf{\textit{R}} к самому себе.}

\scnheader{частично рефлексивное отношение}
\scntext{определение}{\textbf{\textit{частично рефлексивное отношение R}} на \textit{множестве} \textbf{\textit{A}} – это \textit{бинарное отношение},  в котором хотя бы один (но не все) элемент множества \textbf{\textit{A}} находится в отношении \textbf{\textit{R}} к самому себе.}

\scnheader{симметричное отношение}
\scntext{определение}{\textbf{\textit{симметричное отношение R}} на \textit{множестве} \textbf{\textit{A}} – это \textit{бинарное отношение}, в котором для каждой пары элементов \textbf{\textit{а}} и \textbf{\textit{b}} этого множества выполнение отношения \textbf{\textit{aRb}} влечёт выполнение \textbf{\textit{bRa}}.}

\scnheader{антисимметричное отношение}
\scntext{определение}{\textbf{\textit{антисимметричное отношение R}} на \textit{множестве} \textbf{\textit{A}} – это \textit{бинарное отношение}, в котором для каждой пары элементов \textbf{\textit{а}} и \textbf{\textit{b}} этого множества выполнение отношений \textbf{\textit{aRb}} и \textbf{\textit{bRa}} влечёт равенство \textbf{\textit{a}} и \textbf{\textit{b}}.}

\scnheader{частично симметричное отношение}
\scntext{определение}{\textbf{\textit{частично симметричное отношение R}} на \textit{множестве} \textbf{\textit{A}} – это \textit{бинарное отношение}, в котором для каждой пары элементов \textbf{\textit{а}} и \textbf{\textit{b}} (но не для всех таких пар) этого множества выполнение отношения \textbf{\textit{aRb}} влечёт выполнение \textbf{\textit{bRa}}.}

\scnheader{транзитивное отношение}
\scntext{определение}{\textbf{\textit{транзитивное отношение R}} на \textit{множестве} \textbf{\textit{A}} – это \textit{бинарное отношение}, в котором для любых трёх элементов этого множества \textbf{\textit{a, b, c}} выполнение отношений \textbf{\textit{aRb}} и \textbf{\textit{bRc}} влечёт выполнение отношения \textbf{\textit{aRc}}.}

\scnheader{антитранзитивное отношение}
\scntext{определение}{\textbf{\textit{антитранзитивное отношение R}} на \textit{множестве} \textbf{\textit{A}} – это \textit{бинарное отношение}, в котором для любых трёх элементов этого множества \textbf{\textit{a, b, c}} выполнение отношений \textbf{\textit{aRb}} и \textbf{\textit{bRc}} не влечёт выполнение отношения \textbf{\textit{aRc}}.}

\scnheader{частично транзитивное отношение}
\scntext{определение}{\textbf{\textit{частично транзитивное отношение R}} на \textit{множестве} \textbf{\textit{A}} – это \textit{бинарное отношение}, в котором для каждых трёх элементов этого множества \textbf{\textit{a, b, c}} (но не для всех таких троек) выполнение отношений \textbf{\textit{aRb}} и \textbf{\textit{bRc}} влечёт выполнение отношения \textbf{\textit{aRc}}.}

\scnheader{связанное отношение}
\scnsubset{бинарное отношение}
\scntext{определение}{\textbf{\textit{связанное отношение R}} на \textit{множестве} \textbf{\textit{A}} – это \textit{бинарное отношение}, в котором для каждой пары элементов \textbf{\textit{а}} и \textbf{\textit{b}} этого множества выполняется одно из двух отношений: \textbf{\textit{aRb}} или \textbf{\textit{bRa}}.}

\scnheader{отношение порядка}
\scnsubdividing{отношение строгого порядка;отношение нестрогого порядка}
\scntext{определение}{\textbf{\textit{отношение порядка}} – это \textit{бинарное отношение}, обладающее свойством транзитивности и антисимметричности.}

\scnheader{отношение строгого порядка}
\scntext{определение}{\textbf{\textit{отношение строгого порядка}} – это \textit{отношение порядка}, обладающее свойством антирефлексивности.}

\scnheader{отношение нестрогого порядка}
\scntext{определение}{\textbf{\textit{отношение нестрогого порядка}} – это \textit{отношение порядка}, обладающее свойством рефлексивности.}

\scnheader{отношение толерантности}
\scntext{определение}{\textbf{\textit{отношение толерантности}} – это \textit{бинарное отношение}, принадлежащее классам \textit{симметричное отношение} и \textit{рефлексивное отношение}.}

\scnheader{отношение эквивалентности}
\scnidtf{максимальное семейство отношений эквивалентности}
\scnsubset{отношение толерантности}
\scntext{определение}{\textbf{\textit{отношение эквивалентности}} – это \textit{отношение толерантности}, принадлежащее классу \textit{транзитивных отношений}}
\scntext{примечание}{Каждое отношение эквивалентности уточняет то, что мы считаем эквивалентными сущностями, т.е. то, на какие сходства этих сущностей мы обращаем внимание и какие их отличия мы игнорируем (не учитываем).}

\scnheader{ролевое отношение}
\scnidtf{атрибут}
\scnidtf{атрибутивное отношение}
\scnidtf{отношение, которое задает роль элементов в рамках некоторого множества}
\scnidtf{отношение, являющееся подмножеством отношения принадлежности}
\scnrelto{семейство подмножеств}{принадлежность*}
\scnsubset{бинарное отношение}
\scnsuperset{числовой атрибут}
\scnexplanation{\textbf{\textit{ролевое отношение}} – это отношение, являющееся подмножеством отношения принадлежности.}
\scntext{правило идентификации экземпляров}{В конце каждого \textit{идентификатора}, соответствующего экземплярам класса \textbf{\textit{ролевое отношение}}, не являющегося системным, ставится знак «'».

Например:\\
\textit{ключевой экземпляр’}

Из-за ограничений в разрешенном алфавите символов, в системном идентификаторе не может быть использовать знак «'», поэтому в начале каждого \textit{системного идентификатора}, соответствующего экземплярам класса \textbf{\textit{ролевое отношение}} ставится префикс «rrel\_».

Например:\\
\textit{rrel\_key\_sc\_element}}

\scnheader{числовой атрибут}
\scnidtf{порядковый номер}
\scnidtf{номер компонента ориентированной связки}
\scnhaselement{1’; 2’; 3’; 4’; 5’; 6’; 7’; 8’; 9’; 10’}
\scnexplanation{\textbf{\textit{числовой атрибут}} – \textit{ролевое отношение}, задающее порядковый номер элемента некоторой ориентированной связки, не уточняя при этом семантику такой принадлежности. Во многих случаях бывает достаточно использовать числовые атрибуты, чтобы различать компоненты связки, семантика каждого из которых дополнительно оговаривается, например, при определении отношения, которому данная связка принадлежит.}

\scnheader{неролевое отношение}
\scnsubdividing{небинарное отношение;неролевое бинарное отношение}
\scnexplanation{\textbf{\textit{неролевое отношение}} – отношение, не являющееся подмножеством отношения принадлежности.}
\scntext{правило идентификации экземпляров}{В конце каждого \textit{идентификатора}, соответствующего экземплярам класса \textbf{\textit{неролевое отношение}}, не являющегося системным, ставится знак «*».

Например:\\
\textit{включение*}

Из-за ограничений в разрешенном алфавите символов, в системном идентификаторе не может быть использовать знак «*», поэтому в начале каждого \textit{системного идентификатора}, соответствующего экземплярам класса \textbf{\textit{неролевое отношение}} ставится префикс «nrel\_».

Например:\\
\textit{nrel\_inclusion}}

\scnheader{неролевое бинарное отношение}
\scnexplanation{\textbf{\textit{неролевое бинарное отношение}} – \textit{бинарное отношение}, не являющееся \textit{ролевым отношением}.}

\scnheader{арность}
\scnidtf{арность отношения}
\scniselement{параметр}
\scnexplanation{\textbf{\textit{арность}} – это параметр, каждый элемент которого представляет собой класс \textit{отношений}, каждая связка которых имеет одинаковую \textit{мощность}. Значение данного \textit{параметра} совпадает со значением \textit{мощности} каждой из таких связок.}
\scnrelfrom{описание примера}{
\scnfilescg{figures/sd_relations/arity.png}}


\scnheader{область определения*}
\scnidtf{область определения отношения*}
\scniselement{бинарное отношение}
\scnexplanation{\textbf{\textit{область определения*}} – это \textit{бинарное отношение}, связывающее отношение со множеством, являющимся его областью определения.

Областью определения отношения будем называть результат теоретико-множественного объединения всех связок этого отношения, или, другими словами, результат теоретико-множественного объединения всех множеств, являющихся доменами данного отношения.}
\scnrelfrom{описание примера}{
\scnfilescg{figures/sd_relations/domain.png}}

\scnheader{атрибут отношения*}
\scnidtf{ролевой атрибут, используемый в связках заданного отношения*}
\scniselement{бинарное отношение}
\scnexplanation{\textbf{\textit{атрибут отношения*}} – это \textit{бинарное отношение}, связывающее заданное отношение с \textit{ролевым отношением}, используемым в данном отношении для уточнения роли того или иного элемента связок данного отношения.}
\scnrelfrom{описание примера}{
\scnfilescg{figures/sd_relations/relationshipAttribute.png}}


\scnheader{домен*}
\scnidtf{домен отношения по заданному атрибуту*}
\scniselement{бинарное отношение}
\scnexplanation{\textbf{\textit{домен*}} – это \textit{бинарное отношение}, связывающее связку отношения \textit{атрибут отношения*} со множеством, являющимся доменом заданного отношения по заданному атрибуту. Множество \textbf{\textit{di}} является доменом отношения \textbf{\textit{ri}} по атрибуту \textbf{\textit{ai}} в том и только том случае, если элементами этого множества являются все те и только те элементы связок отношения \textbf{\textit{ri}}, которые имеют в рамках этих связок атрибут \textbf{\textit{ai}}.}
\scnrelfrom{описание примера}{
\scnfilescg{figures/sd_relations/domen.png}}


\scnheader{первый домен*}
\scniselement{бинарное отношение}
\scntext{определение}{\textbf{\textit{первый домен*}} – это \textit{бинарное отношение}, связывающее отношение с множеством, являющимся доменом по атрибуту \textbf{\textit {1'}} данного отношения.}
\scnrelfrom{описание примера}{
\scnfilescg{figures/sd_relations/firstDomen.png}}

\scnheader{второй домен*}
\scniselement{бинарное отношение}
\scntext{определение}{\textbf{\textit{второй домен*}} – это \textit{бинарное отношение}, связывающее отношение с множеством, являющимся доменом по атрибуту \textbf{\textit{2'}} данного отношения.}
\scnrelfrom{описание примера}{
\scnfilescg{figures/sd_relations/secondDomen.png}}

\scnheader{композиция отношений*}
\scniselement{квазибинарное отношение}
\scntext{определение}{\textbf{\textit{композиция отношений*}} – это \textit{квазибинарное отношение}, связывающее два бинарных отношения с отношением, являющимся их композицией. Под композицией бинарных отношений \textbf{\textit{R}} и \textbf{\textit{S}} будем понимать множество $\{(x, y) | \exists z(xSz \wedge zRy)\}$}
\scnrelfrom{описание примера}{
\scnfilescg{figures/sd_relations/relationshipComposition.png}}

\scnheader{фактор-множество*}
\scnidtf{быть фактор-множеством*}
\scnidtf{множество всевозможных максимальных множеств из попарно эквивалентных элементов*}
\scnidtf{множество всевозможных классов эквивалентности для заданного отношения эквивалентности*}
\scniselement{бинарное отношение}
\scntext{определение}{\textbf{\textit{фактор множество*}} - это бинарное ориентированное отношение, каждая связка которого связывает некоторое отношение эквивалентности со множеством всех соответствующих этому отношению классов эквивалентности. Каждый такой класс представляет собой максимальное множество сущностей, каждая пара которых принадлежит указанному выше отношению эквивалентности.}
\scnrelfrom{описание примера}{
\scnfilescg{figures/sd_relations/factor_set.png}}

\scnheader{метаотношение}
\scntext{определение}{метаотношение - это \textit{отношение}, в каждой связке которого есть по крайней мере один компонент, являющийся знаком некоторого \textit{отношения}.}

\scnheader{отношение декомпозиции}
\scnhaselement{разбиение*}
\scnhaselement{декомпозиция раздела*}
\scnhaselement{декомпозиция абстрактного объекта*}

\scnheader{отношение интеграции}
\scnhaselement{объединение*}

\scnheader{соответствие*}
\scnidtf{наличие соответствия*}
\scniselement{бинарное отношение}
\scnsubdividing{соответствие между непересекающимися множествами*;соответствие между строго пересекающимися множествами*;соответствие, область отправления и область прибытия которого совпадают*}
\scnsubdividing{всюду определенное соответствие*;частично определенное соответствие*}
\scnsubdividing{сюръекция*;несюръективное соответствие*}
\scnsubdividing{однозначное соответствие*;неоднозначное соответствие*}
\scntext{определение}{\textbf{\textit{соответствие*}} – \textit{бинарное отношение}, заданное на множествах и задающее наличие отношения, в котором участвуют только элементы этих множеств.}
\scnrelfrom{описание примера}{
\scnfilescg{figures/sd_relations/conformity.png}}

\scnheader{отношение соответствия*}
\scniselement{бинарное отношение}
\scntext{определение}{\textbf{\textit{отношение соответствия*}} – \textit{бинарное отношение}, связывающее ориентированную пару множеств, на которых задано \textit{соответствие*} и некоторое подмножество \textit{декартова произведения*} этих \textit{множеств}.}
\scnrelfrom{описание примера}{
\scnfilescg{figures/sd_relations/relationshipConformity.png}}

\scnheader{область отправления'}
\scnidtf{область отправления соответствия’}
\scnidtf{область определения соответствия’}
\scnidtf{первый компонент пары в отношении соответствия’}
\scniselement{ролевое отношение}
\scntext{определение}{\textbf{\textit{область отправления'}} – \textit{ролевое отношение}, указывающее на первый компонент пары в рамках отношения \textit{соответствие*}.}
\scnrelfrom{описание примера}{
\scnfilescg{figures/sd_relations/departureArea.png}}

\scnheader{область прибытия’}
\scnidtf{область прибытия соответствия'}
\scnidtf{область значений соответствия'}
\scniselement{ролевое отношение}
\scntext{определение}{\textbf{\textit{область прибытия’}} – \textit{ролевое отношение}, указывающее на второй компонент пары в рамках отношения \textit{соответствие*}.}
\scnrelfrom{описание примера}{
\scnfilescg{figures/sd_relations/arrivalArea.png}}

\scnheader{образ'}
\scnidtf{образ соответствия’}
\scniselement{ролевое отношение}
\scntext{определение}{\textbf{\textit{образ'}} – \textit{ролевое отношение}, указывающее на второй компонент каждой пары в рамках множества пар, которое является вторым компонентом \textit{отношения соответствия*}.}
\scnrelfrom{описание примера}{
\scnfilescg{figures/sd_relations/form.png}}

\scnheader{прообраз'}
\scnidtf{прообраз соответствия’}
\scniselement{ролевое отношение}
\scntext{определение}{\textbf{\textit{прообраз'}} – \textit{ролевое отношение}, указывающее на первый компонент каждой пары в рамках множества пар, которое является первым компонентом \textit{отношения соответствия*}.}
\scnrelfrom{описание примера}{
\scnfilescg{figures/sd_relations/prototype.png}}

\scnheader{всюду определенное соответствие*}
\scnidtf{полное соответствие*}
\scnidtf{наличие всюду определенного соответствия*}
\scntext{определение}{\textbf{\textit{всюду определенное соответствие*}} – это \textit{соответствие*}, при котором существует \textit{образ’} для каждого элемента \textit{области отправления'} данного \textit{соответствия*}.}
\scnrelfrom{описание примера}{
\scnfilescg{figures/sd_relations/surjection.png}}
\scnrelfrom{изображение}{
\scnfileimage{\includegraphics[width=0.5\linewidth]{figures/sd_relations/surjection2.png}}}


\scnheader{частично определенное соответствие*}
\scnidtf{наличие частично определенного соответствия*}
\scntext{определение}{\textbf{\textit{частично определенное соответствие*}} – это \textit{соответствие*}, при котором существует \textit{образ’} для некоторых, но не всех элементов \textit{области отправления'} данного \textit{соответствия*}.}
\scnrelfrom{описание примера}{
\scnfilescg{figures/sd_relations/partiallyDefinedConformity.png}}
\scnrelfrom{изображение}{
\scnfileimage{\includegraphics[width=0.5\linewidth]{figures/sd_relations/partiallySurjection.png}}}


\scnheader{сюръективное соответствие*}
\scnidtf{наличие сюръективного соответствия*}
\scnidtf{сюръекция*}
\scntext{определение}{\textbf{\textit{сюръективное соответствие*}} – это \textit{соответствие*}, при котором существует \textit{прообраз’} для каждого элемента \textit{области прибытия'} данного \textit{соответствия*}.}
\scnrelfrom{описание примера}{
\scnfilescg{figures/sd_relations/surjectiveConformity.png}}
\scnrelfrom{изображение}{
\scnfileimage{\includegraphics[width=0.5\linewidth]{figures/sd_relations/surjectiveConformity2.png}}}

\scnheader{несюръективное соответствие*}
\scnidtf{наличие несюръективного соответствия*}
\scntext{определение}{\textbf{\textit{несюръективное соответствие*}} – это \textit{соответствие*}, при котором не для каждого элемента \textit{области прибытия'} данного \textit{соответствия*} существует \textit{прообраз’}.}
\scnrelfrom{описание примера}{
\scnfilescg{figures/sd_relations/nonSurjectiveConformity.png}}
\scnrelfrom{изображение}{
\scnfileimage{\includegraphics[width=0.5\linewidth]{figures/sd_relations/nonSurjectiveConformity2.png}}}

\scnheader{однозначное соответствие*}
\scnidtf{наличие однозначного соответствия*}
\scnidtf{функциональное соответветствие*}
\scnidtf{функция*}
\scntext{определение}{\textbf{\textit{однозначное соответствие*}} – это \textit{соответствие*}, при котором каждому элементу из \textit{области отправления'} соответствия ставится не более, чем один элемент из \textit{области прибытия’} соответствия.}
\scnrelfrom{описание примера}{
\scnfilescg{figures/sd_relations/singleConformity.png}}
\scnrelfrom{изображение}{
\scnfileimage{\includegraphics[width=0.5\linewidth]{figures/sd_relations/singleConformity2.png}}}

\scnheader{обратное соответствие*}
\scniselement{бинарное отношение}
\scnrelfrom{область определения}{соответствие*}
\scntext{определение}{\textbf{\textit{обратное соответствие*}} – \textit{бинарное отношение}, связывающее два \textit{соответствия*}, при этом выполняются следующие условия:
\begin{scnitemize}
    \item \textit{область отправления’} первого соответствия является \textit{областью прибытия'} второго;
    \item \textit{область прибытия’} первого соответствия является \textit{областью отправления'} второго;
    \item для каждой пары, входящей в состав отношения первого соответствия, существует пара, входящая в состав отношения второго соответствия, при этом \textit{образ’} и \textit{прообраз'} в рамках первой указанной пары являются соответственно \textit{прообразом'} и \textit{образом’} в рамках второй.
\end{scnitemize}
}

\scnheader{обратимое соответствие*}
\scnsubset{однозначное соответствие*}
\scntext{определение}{\textbf{\textit{обратимое соответствие*}} – такое \textit{однозначное соответствие*}, для которого \textit{обратное соответствие*} также является \textit{однозначным соответствием*}.}

\scnheader{неоднозначное соответствие*}
\scntext{определение}{\textbf{\textit{неоднозначное соответствие*}} – это \textit{соответствие*}, при котором хотя бы одному элементу из \textit{области отправления’} соответствия ставится более, чем один элемент из \textit{области прибытия'} соответствия.}
\scnrelfrom{описание примера}{
\scnfilescg{figures/sd_relations/nonSingleConformity.png}}
\scnrelfrom{изображение}{
\scnfileimage{\includegraphics[width=0.5\linewidth]{figures/sd_relations/nonSingleConformity2.png}}}

\scnheader{инъективное соответствие*}
\scnidtf{инъекция*}
\scnsubset{однозначное соответствие*}
\scntext{определение}{\textbf{\textit{инъективное соответствие*}} – это \textit{соответствие*}, при котором разным элементам из \textit{области отправления’} соответствия всегда соответствуют разные элементы из \textit{области прибытия'} соответствия и наоборот.}
\scnrelfrom{описание примера}{
\scnfilescg{figures/sd_relations/injectiveConformity.png}}
\scnrelfrom{изображение}{
\scnfileimage{\includegraphics[width=0.5\linewidth]{figures/sd_relations/injectiveConformity2.png}}}

\scnheader{взаимно однозначное соответствие*}
\scnidtf{биекция*}
\scnsubset{всюду определенное соответствие*}
\scnsubset{сюръективное соответствие*}
\scnsubset{инъективное соответствие*}
\scntext{определение}{\textbf{\textit{взаимно однозначное соответствие*}} – это \textit{инъективное соответствие*}, являющееся всюду определенным и сюръективным.}
\scnrelfrom{описание примера}{
\scnfilescg{figures/sd_relations/bijectiveConformity.png}}
\scnrelfrom{изображение}{
\scnfileimage{\includegraphics[width=0.5\linewidth]{figures/sd_relations/bijectiveConformity2.png}}}


\scnheader{множество сочетаний*}
\scnidtf{множество всевозможных сочетаний*}
\scnidtf{множество всевозможных сочетаний заданной арности из элементов заданного множества*}
\scnidtf{множество всех неориентированных связок заданной арности*}
\scnidtf{множество всех подмножеств заданной мощности*}
\scnidtf{семейство всевозможных сочетаний*}
\scntext{определение}{\textbf{\textit{множество сочетаний*}} - \textit{отношение}, связывающее некоторое множество и семейство всевозможных множеств, имеющих значение мощности, меньше либо равное мощности исходного множества и состоящих из тех же элементов, что и это множество.}
\scntext{утверждение}{Мощность \textbf{\textit{множества сочетаний*}} может быть вычислена как n!/(k!(n-k)!), где \textbf{\textit{n}} – мощность исходного множества, \textbf{\textit{k}} – мощность элементов множества сочетаний.}
\scnrelfrom{описание примера}{
\scnfilescg{figures/sd_relations/setsOfCombinations.png}
\scnexplanation{Для Множества \textbf{\textit{Si}} представлено множество сочетаний по 2 элемента.}}

\scnheader{множество размещений*}
\scntext{определение}{\textbf{\textit{множество размещений*}} - \textit{отношение}, связывающее некоторое множество и семейство всевозможных кортежей, имеющих значение мощности, меньше либо равное мощности исходного множества и состоящих из тех же элементов, что и это множество.}
\scntext{утверждение}{Мощность \textbf{\textit{множества размещений*}} может быть вычислена как n!/(n-k)!, где \textbf{\textit{n}} – мощность исходного множества, \textbf{\textit{k}} – мощность элементов множества сочетаний.}
\scnrelfrom{описание примера}{
\scnfilescg{figures/sd_relations/setsOfPlacements.png}
\scnexplanation{Для Множества \textbf{\textit{Si}} представлено множество размещений по 2 элемента.}}

\scnheader{множество перестановок*}
\scnsubset{множество размещений*}
\scntext{определение}{\textbf{\textit{множество перестановок*}} - \textit{отношение}, связывающее некоторое множество и семейство всевозможных кортежей, равномощных исходному множеству и состоящих из тех же элементов, что и это множество.}
\scntext{утверждение}{Мощность \textbf{\textit{множества перестановок*}} может быть вычислена как n!, где \textbf{\textit{n}} – мощность исходного множества.}
\scnrelfrom{описание примера}{
\scnfilescg{figures/sd_relations/setsOfPermutations.png}
\scnexplanation{Для Множества \textbf{\textit{Si}} представлено его множество перестановок.}}

\bigskip
\scnendstruct \scnendcurrentsectioncomment

\end{SCn}

\scsubsection[\scnmonographychapter{Глава 2.4. Формальные онтологии базовых классов сущностей - множеств, связей, отношений, параметров, величин, чисел, структур, темпоральных сущностей}]{Предметная область и онтология параметров, величин и шкал}
\label{sd_params}
\begin{SCn}

\scnsectionheader{Предметная область и онтология параметров, величин и измерений}

\scnstartsubstruct

\scnheader{Предметная область параметров, величин и измерений}
\scnidtf{Предметная область параметров и классов эквивалентности, являющихся их элементами (значениями, величинами)}
\scniselement{предметная область}
\scnsdmainclasssingle{параметр}
\scnsdclass{измеряемый параметр;неизмеряемый параметр;уровень класса эквивалентности;величина;точная величина;неточная величина;интервальная величина;параметрическая модель;измерение с фиксированной единицей измерения ;измерение по шкале;арифметическое выражение на величинах;арифметическая операция на величинах;действие. измерение;задача. измерение}
\scnsdrelation{область определения параметра*;эталон';измерение*;точность*;единица измерения*;нулевая отметка*;сумма величин*;произведение величин*;возведение величин в степень*;большая величина*;равенство величин*;большая или равная величина*}

\scnauthorcomment{ввести отношение, показывающее единичную отметку для измерений по шкале}

\scnheader{параметр}
\scnidtf{характеристика}
\scnidtf{свойство}
\scnidtf{признак}
\scnidtf{класс классов}
\scnidtf{измеряемое свойство}
\scnidtf{признак классификации или покрытия некоторого класса сущностей}
\scnidtf{признак разбиения или покрытия некоторого класса сущностей}
\scnidtf{семейство множеств, разбивающих или покрывающих некоторый класс сущностей}
\scnidtf{семейство классов сущностей, обладающих одинаковым соответствующим свойством}
\scnidtf{фактор-множество, соответствующее некоторому отношению эквивалентности, или аналог фактор-множества, соответствующий некоторому отношению толерантности}
\scnreltoset{разбиение}{измеряемый параметр;неизмеряемый параметр}
\scnexplanation{Каждый \textbf{\textit{параметр}} представляет собой класс, являющийся семейством всевозможных классов эквивалентности или толерантности, задаваемых либо \textit{отношением эквивалентности}, либо \textit{отношением толерантности} (симметричным, рефлексивным, но частично транзитивным). Так, например, элементами (значениями, величинами) \textbf{\textit{параметра}} \textit{длина} являются либо классы эквивалентности, задаваемые отношением эквивалентности «иметь точно одинаковую длину*», либо классы толерантности, задаваемые отношением вида «иметь приблизительно одинаковую длину с указываемой точностью*», либо интервальные классы, задаваемые бинарными отношениями вида «иметь длину, находящуюся в одном и том же указываемом интервале*» (например, от 1 метра до 2 метров).\\
Примерами параметров как отношений эквивалентности являются:
\begin{scnitemize}
    \item равновеликость геометрических фигур (по длине, площади, объему – в зависимости от размерности этих фигур);
    \item иметь одинаковый цвет (быть эквивалентными по цвету);
    \item эквивалентность, по вкусу, запаху, твердости и т.д.
\end{scnitemize}

Заметим, что среди элементов (значений, величин) параметра могут встречаться пересекающиеся множества (классы), но объединение всех элементов каждого параметра есть не что иное, как класс всевозможных сущностей, обладающих этим параметром (свойством, характеристикой). Например, класс всех сущностей, имеющих длину, класс всех сущностей, обладающих цветом.

Каждый конкретный параметр (характеристика), т.е. каждый элемент класса всевозможных параметров (характеристик) есть, по сути, признак классификации сущностей, обладающих это характеристикой, по принципу эквивалентности (одинаковости значения) этой характеристики. Например, параметр \textit{цвет} разбивает множество сущностей имеющих цвет на классы, каждый из которых включает в себя сущности, имеющие одинаковый цвет. Параметр может разбиваться на классы для уточнения некоторого свойства, например элементами параметра цвет будут классы, соответствующие конкретным цветам (синий, красный и т.д.), в свою очередь каждый конкретный цвет может включать более частные классы, уточняющие данное свойство, например, темно-синий, светло-красный и т.д.

Другими словами, каждому множеству сущностей может ставиться в соответствие набор (семейство) параметров (параметрическое пространство), которыми обладают сущности этого множества – аналог семейства отношений, определенных (заданных) на этом множестве. Часто бывает важно построить такое параметрическое пространство, «точки» которого взаимно-однозначно соответствуют параметризуемым сущностям (например, набор параметров, позволяющих однозначно идентифицировать, установить личность каждого человека). 

Таким образом, для каждого используемого элемента (значения) какого-либо параметра, необходимо явно указывать спецификацию этого значения (точное значение, неточное значение, интервальное значение, точность, интервал).}
\scnrelfrom{типичная семантическая окрестность}{
\scnfilelong{
\begin{figure}[H]
\centering
\includegraphics[width=0.8\linewidth]{figures/sd_parameters_and_quantities/parameterDescription.png}
\end{figure}
}}

\scnheader{область определения параметра*}
\scnidtf{множество всех тех и только тех сущностей, которые являются компонентами значений заданного параметра*}
\scnidtf{множество всех тех и только тех сущностей, которые обладают заданным параметром*}
\scnrelto{включение}{объединение*}

\scnheader{измеряемый параметр}
\scnidtf{количественный параметр}
\scnidtf{семейство измеряемых величин}
\scnidtf{семейство классов эквивалентности, каждому из которых может быть поставлено в соответствие числовое значение}
\scnexplanation{Каждый \textbf{\textit{измеряемый параметр}} представляет собой \textit{параметр}, значение (элемент, экземпляр) которого трактуется как \textit{величина}, которой можно поставить в соответствие ее числовое значение на основании выбранной единицы измерения и точки отсчета (нулевой отметки выбранной шкалы).}

\scnheader{неизмеряемый параметр}
\scnidtf{качественный параметр}

\scnheader{уровень класса эквивалентности}
\scnidtf{уровень параметра}
\scniselement{параметр}
\scnexplanation{Параметр \textbf{\textit{уровень класса эквивалентности}} задает уровень некоторого значения некоторого \textit{параметра} в иерархии значений этого параметра. Уровень класса эквивалентности равен 1, если значение параметра не является частным по отношению к другому значению этого параметра, равен 2, если значение параметра является частным по отношению к значению этого параметра с уровнем 1 и т.д.}
\scnrelfrom{типичная семантическая окрестность}{
\scnfilelong{
\begin{figure}[H]
\centering
\includegraphics[width=0.8\linewidth]{figures/sd_parameters_and_quantities/color.png}
\end{figure}
}}

\scnheader{величина}
\scnidtf{значение количественного параметра}
\scnidtf{значение измеряемого параметра}
\scnidtf{класс сущностей, имеющих одинаковое значение соответствующего параметра}
\scnrelfromlist{включение}{точная величина;неточная величина;интервальная величина}
\scnexplanation{Каждая \textbf{\textit{величина}} представляет собой однозначный и независящий от шкалы измерения результат измерения некоторой характеристики у некоторой сущности.

Каждой \textbf{\textit{величине}} можно поставить в соответствие ее числовое значение на основании выбранной единицы измерения и точки отсчета (нулевой отметки выбранной шкалы).

Нельзя путать значение параметра (\textbf{\textit{величину}}) и значение величины по некоторой шкале, которое может быть скалярным и векторным.}

\scnheader{точная величина}
\scnidtf{точное значение параметра}
\scnidtf{множество всех точных значений параметра}
\scnidtf{значение параметра, являющееся семейством классов эквивалентности, соответствующим некоторому отношению эквивалентности}
\scnidtf{класс эквивалентности}
\scnexplanation{Каждая \textbf{\textit{точная величина}} имеет одно фиксированное значение в некоторой единице измерения или по какой-либо шкале. При этом считается, что все элементы такого класса имеют одинаковое значение данного параметра и отклонениями можно пренебречь.

Каждой \textbf{\textit{точной величине}} можно поставить в соответствие группу \textit{неточных величин}, являющихся не разбиениями, а покрытиями того же множества, но с разной степенью точности.}
\scnrelfrom{типичная семантическая окрестность}{
\scnfilelong{
\begin{figure}[H]
\centering
\includegraphics[width=0.8\linewidth]{figures/sd_parameters_and_quantities/exactLength.png}
\end{figure}}
\scntext{комментарий}{В данном примере \textit{ki} обозначает класс сущностей, имеющих длину ровно 5 метров. Конкретный пример такой сущности - \textit{bi}.}}

\scnheader{неточная величина}
\scnidtf{множество неточных значений параметра}
\scnidtf{приблизительная величина}
\scnidtf{приблизительное значение параметра}
\scnidtf{значение параметра в интервале с нефиксированными границами}
\scnexplanation{Каждой \textbf{\textit{неточной величине}} ставится в соответствие ее значение в некоторой единице измерения или по какой-либо шкале, а также дополнительно указывается \textit{точность*}, т.е. возможное отклонение от данного значения.}
\scnrelfrom{типичная семантическая окрестность}{
\scnfilelong{
\begin{figure}[H]
\centering
\includegraphics[width=0.8\linewidth]{figures/sd_parameters_and_quantities/approximateLength.png}
\end{figure}}
\scntext{комментарий}{В данном примере \textit{ki} обозначает класс сущностей, имеющих длину примерно 25 метров, при этом максимально возможное отклонение от этого значения составляет \textit{kj}, то есть 2 метра. Конкретный пример такой сущности - \textit{bi}.}}

\scnheader{интервальная  величина}
\scnidtf{интервальное значение параметра}
\scnidtf{значение параметра в интервале с фиксированными границами}
\scnidtf{интервал значения параметра из множества пересекающихся интервалов разной длины, имеющих нефиксированные границы}
\scnexplanation{Каждая \textbf{\textit{интервальная величина}} представляет собой класс сущностей, находящихся в рамках точно заданного интервала, минимальная и максимальная точка которого являются \textit{точными величинами}. Результатом \textit{измерения*} такой величины является ориентированная пара, первым компонентом которой является левая (меньшая) граница интервала, вторым компонентом – правая (большая) граница интервала.}
\scnrelfrom{типичная семантическая окрестность}{
\scnfilelong{
\begin{figure}[H]
\centering
\includegraphics[width=0.8\linewidth]{figures/sd_parameters_and_quantities/intervalLength.png}
\end{figure}}
\scntext{комментарий}{В данном примере \textit{ki} обозначает класс сущностей, имеющих длину, которая лежит в интервале от \textit{kj} до \textit{kl}, то есть в интервале от 4 до 5 метров, а \textit{bi} – конкретный пример такой сущности.}}

\scnheader{эталон'}
\scnidtf{образец'}
\scniselement{ролевое отношение}
\scnexplanation{Ролевое отношение \textit{эталон'} указывает на тот элемент значения некоторого параметра, который в рамках данного класса эквивалентности считается эталонным, то есть он используется как образец при определении данного параметра.

\textit{эталон'} может задаваться как для измеряемых, так и для неизмеряемых параметров, например, эталон метра или эталон красоты.}

\scnheader{измерение*}
\scnidtf{значение параметра*}
\scnidtf{значение величины*}
\scnidtf{измерение как соответствие*}
\scnidtf{результат измерения заданной величины в заданной единице измерения и по заданной шкале*}
\scnidtf{бинарное ориентированное отношение, связывающее различные величины с результатами их измерения в различных единицах измерения и по различным шкалам*}
\scnexplanation{Связки отношения \textit{измерение*} связывают величину и ее значение в некоторой единице измерения (в том числе, в интервале) или по некоторой шкале. Конкретная единица измерения или шкала указывается дополнительно при помощи соответствующего отношения. Одной величине может соответствовать только одно значение в каждой возможной единице измерения или одна точка на некоторой шкале.}

\scnheader{точность*}
\scnidtf{отклонение*}
\scnidtf{степень точности неточного значения параметра*}
\scniselement{бинарное отношение}
\scnexplanation{Связки отношения \textbf{\textit{точность*}} связывают \textit{неточную величину} и \textit{точную величину} того же класса, задающую максимальное возможное отклонение указанной \textit{неточной величины} от своего значения.}

\scnheader{параметрическая модель}
\scnidtf{параметрическая спецификация}
\scnidtf{параметрическое sc-описание заданной сущности}
\scnidtf{описание сущности как точки в некотором параметрическом (признаковом) пространстве}
\scnrelto{включение}{семантическая окрестность}
\scnexplanation{Каждая \textbf{\textit{параметрическая модель}} представляет собой описание заданной сущности в некотором параметрическом пространстве количественных и качественных \textit{параметров}, т.е. указание того, какие значения заданных параметров (характеристик) соответствуют описываемой (заданной) сущности.}

\scnheader{единица измерения*}
\scniselement{бинарное отношение}
\scnexplanation{Связки отношения \textbf{\textit{единица измерения*}} связывают знак конкретного \textbf{\textit{измерения с фиксированной единицей измерения}} и некоторую \textit{точную величину}, входящую в тот же конкретный \textit{параметр}, что и первый компонент связок этого конкретного измерения, и которая используется в данном случае в качестве единицы измерения.}

\scnheader{измерение с фиксированной единицей измерения }
\scnrelto{семейство подмножеств}{измерение*}
\scnexplanation{Каждая \textbf{\textit{измерение с фиксированной единицей измерения}} представляет собой подмножество отношения \textit{измерение*} и характеризуется некоторой \textit{единицей измерения*}, которая является элементом того же параметра (семейством сущностей, имеющих значение данного параметра, совпадающее с этой единицей измерения).}

\scnheader{измерение по шкале }
\scnidtf{шкала}
\scnrelto{семейство подмножеств}{измерение*}
\scnexplanation{Каждая \textbf{\textit{измерение по шкале}} представляет собой подмножество отношения \textit{измерение*} и характеризуется не единицей измерения, а некоторой точкой отсчета для данной \textbf{\textit{шкалы}}. Результатом \textbf{\textit{измерения по шкале}} будет некоторая точка шкалы, отстоящая от точки отсчета на определенное расстояние в нужную сторону (меньшую или большую). Понятно, что это расстояние может быть измерено любыми единицами измерения, но его величина при этом останется неизменной.

Не стоит путать измерение по \textbf{\textit{измерение по шкале}}, которое зависит от \textit{нулевой отметки*}, с измерением изменения того же \textit{параметра}, которое характеризуется единицей измерения и не зависит от точки отсчета. Например, не стоит путать дату по некоторому календарю, соответствующую \textit{началу} какого-либо процесса, и \textit{длительность} этого процесса, которая не зависит от выбранного календаря.}
\scnrelfrom{типичная семантическая окрестность}{
\scnfilelong{
\begin{figure}[H]
\centering
\includegraphics[width=0.8\linewidth]{figures/sd_parameters_and_quantities/scale.png}
\end{figure}}
\scntext{комментарий}{В данном примере \textit{ki} обозначает класс сущностей, имеющих точную температуру в 330 К, а \textit{bi} – конкретный пример такой сущности.}}

\scnheader{нулевая отметка*}
\scnidtf{нуль по шкале*}
\scnidtf{начало отсчета*}
\scniselement{бинарное отношение}
\scnexplanation{Связки отношения \textbf{\textit{нулевая отметка*}} связывают знак некоторого \textit{измерения по шкале} со знаком \textit{точной величины} того же \textit{параметра}, которая в рамках данной шкалы принимается за точку отсчета.}

\scnheader{арифметическое выражение на величинах}
\scnexplanation{Каждое \textbf{\textit{арифметическое выражение на величинах}} представляет собой \textit{связку}, компонентами которой являются элементы или подмножества некоторого \textit{количественного параметра}.}

\scnheader{арифметическая операция на величинах}
\scnrelto{семейство подмножеств}{арифметическое выражение на величинах}
\scnexplanation{Каждая \textbf{\textit{арифметическая операция на величинах}} представляет собой \textit{отношение}, элементами которого являются \textit{арифметические выражения на величинах}, то есть множество \textit{арифметических выражений на величинах} какого-либо одного вида.}

\scnheader{сумма величин*}
\scnidtf{сложение величин*}
\scniselement{арифметическая операция на величинах}
\scniselement{квазибинарное отношение}
\scnexplanation{\textbf{\textit{сумма величин*}} – это \textit{арифметическая операция на величинах}, аналогичная \textit{арифметической операции сумма*} для \textit{чисел}.

Первым компонентом связки отношения \textbf{\textit{сумма величин*}} является подмножество некоторого \textit{количественного параметра} (слагаемые \textit{величины}), содержащее два или более элемента, вторым компонентом – элемент этого же \textit{количественного параметра}, значение которого в любой \textit{единице измерения*} является результатом сложения значений всех слагаемых \textit{величин} в той же \textit{единице измерения*}. При несовпадении \textit{единиц измерения} слагаемых величин необходимо воспользоваться соотношениями между \textit{единицами измерения}, которые задаются при помощи операций \textit{произведение величин*} и \textit{возведение величин в степень*}.}


\scnheader{произведение величин*}
\scnidtf{умножение величин*}
\scniselement{арифметическая операция на величинах}
\scniselement{квазибинарное отношение}
\scnexplanation{\textbf{\textit{произведение величин*}} – это \textit{арифметическая операция на величинах}, аналогичная \textit{арифметической операции произведение*} для \textit{чисел}.

Первым компонентом связки отношения \textbf{\textit{произведение величин*}} является \textit{связка}, элементами которой являются либо \textit{величины количественных параметров}, либо \textit{числа}, но при этом хотя бы один элемент должен быть \textit{величиной}. Вторым компонентов является \textit{величина количественного параметра}.

Операция \textbf{\textit{произведение величин*}} может быть использована для задания соотношения между \textit{единицами измерения*} в рамках одного \textit{количественного параметра}.}
\scnrelfrom{описание типичного экземпляра}{
\scnfilelong{
\begin{figure}[H]
\centering
\includegraphics[width=0.5\linewidth]{figures/sd_parameters_and_quantities/multiplicationOfQuantities.png}
\end{figure}}
}
\scnrelfrom{описание типичного экземпляра}{
\scnfilelong{
\begin{figure}[H]
\centering
\includegraphics[width=0.8\linewidth]{figures/sd_parameters_and_quantities/multiplicationOfQuantities2.png}
\end{figure}}
}

\scnheader{возведение величин в степень*}
\scniselement{арифметическая операция на величинах}
\scniselement{бинарное отношение}
\scnexplanation{\textbf{\textit{возведение величин в степень*}} – это \textit{арифметическая операция на величинах}, аналогичная \textit{арифметической операции возведение в степень*} для \textit{чисел}.

Первым компонентом связки отношения \textbf{\textit{возведение величин в степень*}} является ориентированная пара, первым компонентом которой является \textit{величина количественного параметра} (основание степени), вторым – \textit{число} (показатель степени); Вторым компонентом связки отношения \textbf{\textit{возведение величин в степень*}} является \textit{величина количественного параметра} (результат возведения в степень).}
\scnrelfrom{описание типичного экземпляра}{
\scnfilelong{
\begin{figure}[H]
\centering
\includegraphics[width=0.5\linewidth]{figures/sd_parameters_and_quantities/exponentiation.png}
\end{figure}}
}
\scnrelfrom{описание типичного экземпляра}{
\scnfilelong{
\begin{figure}[H]
\centering
\includegraphics[width=0.8\linewidth]{figures/sd_parameters_and_quantities/exponentiationTo2.png}
\end{figure}}
}

\scnheader{большая величина*}
\scniselement{арифметическая операция на величинах}
\scniselement{бинарное отношение}
\scniselement{отношение строгого порядка}
\scnexplanation{\textbf{\textit{большая величина*}} – это \textit{арифметическая операция на величинах}, аналогичная \textit{арифметической операции больше*} для \textit{чисел}.\\
Из двух величин большей является та, \textit{значение} которой в любой \textit{единице измерения*} \textit{больше*} значения другой \textit{величины} в той же \textit{единице измерения}.}

\scnheader{равенство величин*}
\scniselement{арифметическая операция на величинах}
\scniselement{бинарное отношение}
\scniselement{симметричное отношение}
\scniselement{рефлексивное отношение}
\scniselement{транзитивное отношение}
\scnexplanation{\textbf{\textit{равенство величин*}} – это \textit{арифметическая операция на величинах}, аналогичная \textit{арифметической операции равенство*} для \textit{чисел}.

Отношение \textbf{\textit{равенство величин*}} носит исключительно дидактический характер, и явно не указывается, поскольку связывает попарно все элементы одной и той же \textit{величины} каждого \textit{количественного параметра}.}

\scnheader{большая или равная величина*}
\scniselement{арифметическая операция на величинах}
\scniselement{бинарное отношение}
\scniselement{отношение нестрогого порядка}
\scnexplanation{\textbf{\textit{большая или равная величина*}} – это \textit{арифметическая операция на величинах}, аналогичная \textit{арифметической операции больше или равно*} для \textit{чисел}.

В рамках каждой связки данного отношения первая \textit{величина} (первый компонент связки) может быть \textit{большей величиной*} или быть для второй \textit{равной величиной*}.}

\scnheader{действие. измерение}
\scnidtf{измерение как действие}
\scnidtf{действие, направленное на установление связи, принадлежащей отношению измерение* и связывающей величину, которая принадлежит заданному параметру, и которой принадлежит заданная сущность, и соответствующее значение этой величины на некоторой шкале}
\scnidtf{действие, направленное на решение задачи измерения заданного параметра у заданной сущности}
\scnrelto{включение}{действие}

\scnheader{задача. измерение}
\scnidtf{спецификация действия измерения}
\scnidtf{спецификация действия, целью которого является измерение заданного параметра у заданной сущности}
\scnrelto{включение}{задача}

\scnendstruct

\end{SCn}

\scsubsection[\scnmonographychapter{Глава 2.4. Формальные онтологии базовых классов сущностей - множеств, связей, отношений, параметров, величин, чисел, структур, темпоральных сущностей}]{Предметная область и онтология чисел и числовых структур}
\begin{SCn}

\scnsectionheader{\currentname}

\scnstartsubstruct

\scnheader{Предметная область чисел и числовых структур}
\scniselement{предметная область}
\scnsdmainclasssingle{число}
\scnsdclass{натуральное число;целое число;рациональное число;иррациональное число;действительное число;комплексное число;отрицательное число;положительное число;арифметическое выражение;арифметическая операция;Число Пи;Число нуль;Число один;Мнимая единица;числовая структура;система счисления;десятичная система счисления;двоичная система счисления;шестнадцатеричная система счисления; дробь; обыкновенная дробь; десятичная дробь; цифра; арабская цифра; римская цифра}
\scnsdrelation{противоположные числа*;модуль*;сумма*;произведение*;возведение в степень*;больше*;равенство*;больше или равно*}

\scnheader{число}
\scnidtf{множество чисел}
\scnsubset{абстрактная терминальная сущность}
\scnexplanation{\textbf{\textit{число}} – это основное понятие математики, используемое для количественной характеристики, сравнения, нумерации объектов и их частей. Письменными знаками для обозначения чисел служат \textit{цифры}.}

\scnheader{цифра}
\scnidtf{множество цифр}
\scnsubset{внутренний файл ostis-системы}
\scnrelfromlist{включение}{арабская цифра;римская цифра}
\scnexplanation{\textbf{\textit{цифра}} -– это множество файлов, обозначающих вхождение этой цифры во всевозможные записи чисел с помощью этой цифры.}

\scnheader{натуральное число}
\scnidtf{множество натуральных чисел}
\scnexplanation{\textbf{\textit{натуральное число}} – это подмножество множества \textit{целых чисел}, которые используются при счете предметов.}
\scnsubset{целое число}

\scnheader{целое число}
\scnidtf{множество целых чисел}
\scnexplanation{\textbf{\textit{целое число}} – это подмножество множества \textit{рациональных чисел}, получаемых объединением \textit{натуральных чисел} с множеством чисел, \textit{противоположных* натуральным} и \textit{нулём}.}
\scnsubset{рациональное число}

\scnheader{рациональное число}
\scnidtf{множество рациональных чисел}
\scnexplanation{\textbf{\textit{рациональное число}} – это число, представляемое \textit{обыкновенной дробью}, где числитель — \textit{целое число}, а знаменатель — \textit{натуральное число}.}
\scnsubset{действительное число}

\scnheader{дробь}
\scnidtf{множество дробей}
\scnrelfromlist{включение}{обыкновенная дробь; десятичная дробь}
\scnexplanation{\textbf{\textit{дробь}} — это число, состоящее из одной или нескольких равных частей (долей) единицы}

\scnheader{обыкновення дробь}
\scnidtf{множество обыкновенных дробей}
\scnidtf{множество простых дробей}
\scnexplanation{\textbf{\textit{обыкновенная дробь}} - запись \textit{рационального числа} в виде ${\displaystyle \pm {\frac {m}{n}}}$ или ${\pm m/n}$, где ${n\neq 0}$.Горизонтальная или косая черта обозначает знак деления, в результате которого получается частное. Делимое называется числителем дроби, а делитель — знаменателем.}

\scnheader{десятичная дробь}
\scnidtf{множество десятичных дробей}
\scnexplanation{\textbf{\textit{десятичная дробь}} - Десятичная дробь — разновидность дроби, которая представляет собой способ представления действительных чисел в виде ${\pm d_m \ldots d_1 d_0{,} d_{-1} d_{-2} \ldots}$, где , — десятичная запятая, служащая разделителем между целой и дробной частью числа, ${d_{k}}$m — десятичные цифры.}

\scnheader{иррациональное число}
\scnidtf{множество иррациональных чисел}
\scnexplanation{\textbf{\textit{иррациональное число}} – это \textit{вещественное число}, которое не является рациональным, то есть не может быть представлено в виде дроби, где числитель — \textit{целое число}, знаменатель — \textit{натуральное число}. Любое \textbf{\textit{иррациональное число}} может быть представлено в виде бесконечной непериодической десятичной дроби.}
\scnsubset{действительное число}

\scnheader{действительное число}
\scnidtf{вещественное число}
\scnidtf{множество вещественных чисел}
\scnreltoset{объединение}{рациональное число;иррациональное число}
\scnreltoset{разбиение}{положительное число;отрицательное число;$\{$Число нуль$\}$}
\scnexplanation{\textbf{\textit{действительное число}} – это множество чисел, получаемое в результате объединения иррациональных и \textit{рациональных чисел}.}
\scnsubset{комплексное число}

\scnheader{комплексное число}
\scnidtf{множество комплексных чисел}
\scnexplanation{\textbf{\textit{комплексное число}} – число вида \textit{z=a+b*i}, где \textit{a} и \textit{b} – \textit{вещественные числа}, \textit{i} – \textit{Мнимая единица}.}

\scnheader{отрицательное число}
\scnidtf{множество отрицательных чисел}
\scnexplanation{\textbf{\textit{отрицательное число}} – число \textit{меньше*} нуля.}

\scnheader{положительное число}
\scnidtf{множество положительных чисел}
\scnexplanation{\textbf{\textit{положительное число}} – число \textit{больше*} нуля.}

\scnheader{противоположные числа*}
\scniselement{бинарное неориентированное отношение}
\scnexplanation{\textbf{\textit{противоположные числа*}} – \textit{отношение}, связывающее два числа, одно из которых является \textit{отрицательным числом}, второе – \textit{положительным}, при этом \textit{модули*} этих чисел \textit{равны*}.}

\scnheader{модуль*}
\scnidtf{модуль числа*}
\scniselement{бинарное отношение}
\scnexplanation{Связки отношения \textbf{\textit{модуль*}} связывают некоторое \textit{число} (которое может быть как \textit{отрицательным}, так и \textit{положительным}) и другое \textit{число} (всегда \textit{положительное}), которое выражает расстояние от указанного числа до \textit{Числа нуль} в единицах.}

\scnheader{арифметическое выражение}
\scnidtf{множество арифметических выражений}
\scnexplanation{Каждое \textbf{\textit{арифметическое выражение}} представляет собой \textit{связку}, компонентами которой являются \textit{числа} или множества \textit{чисел}.}

\scnheader{арифметическая операция}
\scnidtf{множество арифметических операций}
\scnrelto{семейство подмножеств}{арифметическое выражение}
\scnexplanation{Каждая \textbf{\textit{арифметическая операция}} представляет собой \textit{отношение}, элементами которого являются \textit{арифметические выражения}, то есть множество \textit{арифметических выражений} какого-либо одного вида.}

\scnheader{сумма*}
\scnidtf{сложение*}
\scniselement{арифметическая операция}
\scniselement{квазибинарное отношение}
\scnexplanation{\textbf{\textit{сумма*}} – это арифметическая операция, в результате которой по данным числам (слагаемым) находится новое число (сумма), обозначающее столько единиц, сколько их содержится во всех слагаемых.

Первым компонентом связки отношения \textbf{\textit{сумма*}} является \textit{множество чисел} (слагаемых), содержащее два или более элемента, вторым компонентом – \textit{число}, являющееся результатом сложения.

Отдельно отметим, что каждая связка отношения \textbf{\textit{сумма*}} вида a = b+c может также трактоваться и как запись о вычитании чисел, например b = a-c, в связи с чем \textit{арифметическая операция} разности чисел отдельно не вводится.}
\scnrelfrom{описание типичного экземпляра}{
\scnfilescg{figures/sd_numbers/sum.png}}

\scnheader{произведение*}
\scnidtf{умножение*}
\scniselement{арифметическая операция}
\scniselement{квазибинарное отношение}
\scnexplanation{\textbf{\textit{произведение*}} – это \textit{арифметическая операция}, в результате которой один аргумент складывается столько раз, сколько показывает другой, затем результат складывается столько раз, сколько показывает третий и т.д.

Первым компонентом связки отношения \textbf{\textit{произведение*}} является \textit{множество чисел} (множителей), содержащее два или более элемента, вторым компонентом – \textit{число}, являющееся результатом произведения.

Отдельно отметим, что каждая связка отношения \textbf{\textit{произведение*}} вида a = b*c может также трактоваться и как запись о делении чисел, например b = a/c, в связи с чем \textit{арифметическая операция} деления чисел отдельно не вводится.}
\scnrelfrom{описание типичного экземпляра}{
\scnfilescg{figures/sd_numbers/multiplication.png}}

\scnheader{возведение в степень*}
\scniselement{арифметическая операция}
\scniselement{бинарное отношение}
\scnexplanation{\textbf{\textit{возведение в степень*}} – это \textit{арифметическая операция}, в результате которой число, называемое основанием степени, умножается само на себя столько раз, каков показатель степени.

Первым компонентом связки отношения \textbf{\textit{возведение в степень*}} является ориентированная пара, первым компонентом которой является \textit{число}, которое является основанием степени, вторым – \textit{число}, которое является показателем степени; Вторым компонентом связки отношения \textbf{\textit{возведение в степень*}} является \textit{число}, которое является результатом возведения в степень.

Отдельно отметим, что каждая связка отношения \textbf{\textit{возведение в степень*}} вида a = $b^c$ может также трактоваться и как запись об извлечении корня или взятии логарифма, в связи с чем \textit{арифметические операции} извлечения корня и взятия логарифма отдельно не вводится.}
\scnrelfrom{описание типичного экземпляра}{
\scnfilescg{figures/sd_numbers/pow.png}}

\scnheader{больше*}
\scniselement{арифметическая операция}
\scniselement{бинарное отношение}
\scniselement{отношение строгого порядка}
\scnexplanation{\textbf{\textit{больше*}} – это \textit{арифметическая операция} сравнения чисел. Из двух чисел на координатной прямой больше то, которое расположено правее. Соответственно, первым компонентом связки \textit{отношения} \textbf{\textit{больше*}} является большее из двух \textit{чисел}.}
\scnrelfrom{описание типичного экземпляра}{
\scnfilescg{figures/sd_numbers/more.png}}

\scnheader{равенство*}
\scnidtf{равенство чисел*}
\scniselement{арифметическая операция}
\scniselement{бинарное отношение}
\scniselement{симметричное отношение}
\scniselement{рефлексивное отношение}
\scniselement{транзитивное отношение}
\scnexplanation{\textbf{\textit{равенство*}} – отношение взаимной заменяемости \textit{чисел}, которые именно в силу этой заменяемости и считаются равными. Равные \textit{числа} на числовой прямой совпадают.}
\scnrelfrom{описание типичного экземпляра}{
\scnfilescg{figures/sd_numbers/equality.png}}
\scnheader{больше или равно*}
\scniselement{арифметическая операция}
\scniselement{бинарное отношение}
\scniselement{отношение нестрогого порядка}
\scnexplanation{\textbf{\textit{больше или равно*}} – это \textit{арифметическая операция} сравнения чисел, при которой первое \textit{число} (первый компонент связки) может быть \textit{больше*} второго или \textit{равняться*} ему.}
\scnrelfrom{описание типичного экземпляра}{
\scnfilescg{figures/sd_numbers/more.png}}

\scnheader{Число Пи}
\scniselement{иррациональное число}
\scnexplanation{\textbf{\textit{Число Пи}} – это  математическая константа, равная отношению длины окружности к длине её диаметра.}

\scnheader{Число нуль}
\scnidtf{0}
\scniselement{целое число}
\scnexplanation{\textbf{\textit{Число нуль}} – это \textit{целое число}, разделяющее на числовой прямой \textit{положительные числа} и \textit{отрицательные числа}.}

\scnheader{Число один}
\scnidtf{1}
\scniselement{целое число}
\scniselement{натуральное число}
\scnexplanation{\textbf{\textit{Число один}} – это наименьшее \textit{натуральное число}.}

\scnheader{Мнимая единица}
\scnidtf{i}
\scniselement{комплексное число}
\scnexplanation{\textbf{\textit{Мнимая единица}} – это \textit{число}, при возведении которого в степень 2 результатом будет число, противоположное \textit{Числу один}.}

\scnheader{числовая структура}
\scnsubset{структура}
\scnexplanation{\textbf{\textit{числовая структура}} – \textit{структура}, в состав которой входят знаки \textit{арифметических выражений}, а также знаки их элементов и связи между выражениями и их элементами.}

\scnheader{система счисления}
\scniselement{параметр}
\scnexplanation{Каждая \textbf{\textit{система счисления}} представляет собой класс синтаксически эквивалентных файлов, хранимых в sc-памяти, каждый из которых может являться идентификатором какого-либо \textit{числа}.

Каждая \textbf{\textit{система счисления}} характеризуется алфавитом, т.е. конечным множеством символов (цифр), которые допускается использовать при построении файлов принадлежащих данной \textbf{\textit{системе счисления}}.}

\scnheader{десятичная система счисления}
\scniselement{система счисления}

\scnheader{двоичная система счисления}
\scniselement{система счисления}

\scnheader{шестнадцатеричная система счисления}
\scniselement{система счисления}

\scnendstruct

\end{SCn}

\scsubsection[\scnmonographychapter{Глава 2.3. Структура баз знаний интеллектуальных компьютерных систем нового поколения: иерархическая система предметных областей и онтологий. Онтологии верхнего уровня. Формализация понятий семантической окрестности, предметной области и онтологии в интеллектуальных компьютерных системах нового поколения}]{Предметная область и онтология структур}
\label{sd_structures}
\begin{SCn}

\scnsectionheader{\currentname}

\scnstartsubstruct

\scnheader{Предметная область структур}
\scnsdmainclasssingle{структура}
\scnsdclass{связная структура;несвязная структура;тривиальная структура;нетривиальная структура;структура второго уровня;семантический уровень структурного элемента;количество семантических уровней элементов структуры}

\scnsdrelation{элемент структуры’;непредставленное множество’;полностью представленное множество’;частично представленное множество’;элемент структуры, не являющийся множеством';максимальное множество’;немаксимальное множество’;первичный элемент’;вторичный элемент’;элемент второго уровня’;метасвязь’;полиморфность*;полиморфизм*;гомоморфность*;гомоморфизм*;изоморфность*;изоморфизм*;автоморфность*;автоморфизм*;аналогичность структур*;сходство*;различие*;первичная синтаксическая структура sc-текста*}

\scnheader{структура}
\scnidtf{sc-структура}
\scnidtf{структура, представленная в виде текста SC-кода}
\scnsubdividing{связная структура;несвязная структура}
\scnsubdividing{тривиальная структура;нетривиальная структура}
\scnexplanation{\textbf{\textit{структура}} — множество \textit{sc-элементов}, удаление одного из которых может привести к нарушению целостности этого множества.}

\scnheader{связная структура}
\scnexplanation{\textit{Структуре}, представленной в \textit{SC-коде}, поставим в соответствие орграф, вершинами которого являются \textit{sc-элементы}, а дугами – пары инцидентности, связывающие \textit{sc-коннекторы} с инцидентными им \textit{sc-элементами}, которые являются компонентами указанных \textit{sc-коннекторов}.

Если полученный таким способом орграф является связным орграфом, то исходную структуру будем считать \textbf{\textit{связной структурой}}.}

\scnheader{несвязная структура}
\scnexplanation{\textit{Структуре}, представленной в \textit{SC-коде}, поставим в соответствие орграф, вершинами которого являются \textit{sc-элементы}, а дугами – пары инцидентности, связывающие \textit{sc-коннекторы} с инцидентными им \textit{sc-элементами}, которые являются компонентами указанных \textit{sc-коннекторов}.

Если полученный таким способом орграф не является связным орграфом, то исходную структуру будем считать \textbf{\textit{несвязной структурой}}.}

\scnheader{тривиальная структура}
\scnidtf{структура первого уровня}
\scnexplanation{\textbf{\textit{тривиальная структура}} – \textit{структура}, не содержащая в качестве элементов связок.}

\scnheader{нетривиальная структура}
\scnsuperset{структура второго уровня}
\scnexplanation{\textbf{\textit{нетривиальная структура}} – \textit{структура}, среди элементов которой есть хотя бы одна связка.}

\scnheader{элемент структуры’}
\scniselement{неосновное понятие}
\scnsubdividing{непредставленное множество';полностью представленное множество’;частично представленное множество’;элемент структуры, не являющийся множеством'}
\scnsubdividing{максимальное множество’;немаксимальное множество'}
\scnexplanation{\textbf{\textit{элемент структуры'}} — \textit{неосновное понятие}, \textit{ролевое отношение}, указывающее на все элементы каждой структуры.

В рамках заданной структуры ее элементы можно классифицировать по заданным признакам:
\begin{scnitemize}
\item насколько полно в рамках \underline{заданной \textit{структуры}} представлено множество, обозначаемое \textit{заданным sc-элементом} вместе с соответствующими дугами принадлежности;
\item существуют ли в рамках \underline{заданной \textit{структуры}} \textit{sc-элементы}, обозначающие множества, являющиеся надмножествами того множества, которое обозначается \underline{заданным \textit{sc-элементом}};
\item уровень («этаж») иерархии перехода от знаков к метазнакам для \underline{заданного \textit{sc-элемента}} в рамках заданной \textit{структуры}.
\end{scnitemize}
}

\scnheader{непредставленное множество’}
\scnidtf{множество, не представленное в рамках данной структуры’}
\scnidtf{быть знаком множества, элементы которого не являются элементами данной структуры'}
\scniselement{ролевое отношение}
\scnexplanation{\textbf{\textit{непредставленное множество’}} – \textit{ролевое отношение}, связывающее структуру со знаком множества, все элементы которого не являются элементами данной структуры.}

\scnheader{полностью представленное множество’}
\scnidtf{множество, полностью представленное в рамках данной структуры'}
\scnidtf{множество, все элементы которого являются элементами данной структуры'}
\scnidtf{полностью представленный класс'}
\scniselement{ролевое отношение}
\scnexplanation{\textbf{\textit{полностью представленное множество’}} – \textit{ролевое отношение}, связывающее \textit{структуру} со знаком множества (любого семантического типа – класса, связки или структуры), все элементы которого являются элементами данной \textit{структуры}.}

\scnheader{частично представленное множество’}
\scnidtf{множество, частично представленное в рамках данной структуры'}
\scnidtf{множество, некоторые элементы которого являются элементами данной структуры'}
\scnidtf{быть знаком множества, некоторые элементы которого являются элементами данной структуры'}
\scniselement{ролевое отношение}
\scnexplanation{\textbf{\textit{частично представленное множество’}} – ролевое отношение, связывающее структуру со знаком множества, не все элементы которого являются элементами данной структуры.}

\scnheader{элемент структуры, не являющийся множеством’}
\scniselement{ролевое отношение}

\scnheader{максимальное множество’}
\scnexplanation{\textbf{\textit{максимальное множество’}} – \textit{ролевое отношение}, связывающее \textit{структуру} со знаком множества, для которого не существует множества, которое было бы надмножеством указанного множества и знак которого был бы элементом этой же структуры.}

\scnheader{немаксимальное множество’}
\scnexplanation{\textbf{\textit{немаксимальное множество’}} – \textit{ролевое отношение}, связывающее \textit{структуру} со знаком множества, для которого в рамках данной \textit{структуры} существует множество, являющееся надмножеством указанного множества.}

\scnheader{первичный элемент’}
\scnidtf{первичный элемент данной структуры'}
\scnidtf{sc-элемент первого уровня в рамках данной структуры'}
\scniselement{ролевое отношение}
\scniselement{семантический уровень структурного элемента}
\scnsubset{элемент структуры’}
\scnexplanation{\textbf{\textit{первичный элемент’}} – ролевое отношение, указывающее на элемент \textit{структуры}, являющийся либо терминальным элементом, либо знаком множества, такого что не существует другого элемента этой же структуры, который был бы элементом множества, обозначаемого первым из указанных элементов структуры. При этом соответствующая пара принадлежности может существовать, но в состав данной структуры не входить.}

\scnheader{вторичный элемент’}
\scnidtf{вторичный элемент данной структуры’}
\scnidtf{элемент данной структуры имеющий семантический уровень более 2'}
\scnidtf{непервичный элемент'}
\scniselement{ролевое отношение}
\scnsubset{элемент структуры’}
\scnexplanation{\textbf{\textit{вторичный элемент’}} – ролевое отношение, указывающее на элемент структуры, обозначающий множество, все или некоторые элементы которого являются элементами указанной структуры.}
\scnsuperset{элемент второго уровня’}

\scnheader{элемент второго уровня’}
\scniselement{ролевое отношение}
\scniselement{семантический уровень структурного элемента}
\scnexplanation{\textbf{\textit{элементом второго уровня’}} в рамках заданной \textit{структуры} может быть связка первичных элементов, тривиальная структура из первичных элементов или класс первичных элементов.}

\scnheader{структура второго уровня’}
\scnexplanation{\textbf{\textit{структура второго уровня}} - \textit{структура}, среди элементов которой есть хотя бы один \textit{элемент второго уровня’}.}

\scnheader{семантический уровень структурного элемента}
\scniselement{параметр}
\scnexplanation{\textbf{\textit{семантический уровень структурного элемента}} представляет собой \textit{параметр}, каждый элемент которого является классом 
\textit{sc-дуг принадлежности}, связывающих некоторую \textit{структуру} с теми ее элементами, который имеют одинаковый семантический уровень в рамках данной структуры. Значением данного параметра является число, обозначающее указанный семантический уровень.

\textbf{\textit{семантический уровень структурного элемента}} вычисляется следующим образом:

\begin{scnitemize}
\item элементы структуры, входящие в нее с атрибутом \textit{первичный элемент'} имеют семантический уровень 1;
\item уровень элемента, не являющегося \textit{первичным элементом'} структуры вычисляется путем прибавления 1 к максимальному из уровней элементов этого элемента (множества), входящих в эту же структуру. Например, \textit{sc-дуга}, соединяющая два \textit{первичных элемента' структуры} будет иметь семантический уровень 2, а \textit{sc-элемент}, обозначающий отношение, которому принадлежит указанная \textit{sc-дуга} – семантический уровень 3.
\end{scnitemize}
}
\scnrelfrom{типичная семантическая окрестность}{
\scnfilescg{figures/sd_structures/sem_level_struct_elem.png}
}

\scnheader{количество семантических уровней элементов структуры}
\scniselement{параметр}
\scnexplanation{\textbf{\textit{количество семантических уровней элементов структуры}} – параметр, каждый элемент которого представляет собой класс структур, у которых совпадает максимальный среди семантических уровней элементов этих структур.


Значением данного параметра является число, совпадающее с указанным максимальным семантическим уровнем элементов.}

\scnheader{метасвязь’}
\scniselement{ролевое отношение}
\scnsubset{вторичный элемент’}
\scnexplanation{
\begin{scnenumerate}
    \item Каждая входящая в структуру связь, хотя бы одним компонентом которой является связь, входящая в эту же структуру, элементами которой являются \textit{первичные элементы’} этой структуры, является \textbf{\textit{метасвязью’}} указанной структуры;
    \item Каждая входящая в структуру связь, хотя бы одним компонентом которой является \textbf{\textit{метасвязь’}} этой структуры также является \textbf{\textit{метасвязью’}} указанной структуры;
\end{scnenumerate}
}

\scnheader{полиморфность*}
\scnsubset{соответствие*}
\scniselement{бинарное отношение}
\scnexplanation{\textbf{\textit{полиморфность*}} - это \textit{соответствие}, заданное на \textit{структурах}, при котором каждому элементу из области определения соответствия (первой \textit{структуры}) ставится в соответствие один или более элемент из области значения соответствия (второй \textit{структуры}), при этом существует хотя бы один элемент области определения соответствия, которому соответствуют два или более элемента из области значения соответствия.}

\scnheader{полиморфизм*}
\scniselement{бинарное отношение}

\scnheader{гомоморфность*}
\scnidtf{гомоморфность структур*}
\scnsubset{соответствие*}
\scniselement{бинарное отношение}
\scnexplanation{\textbf{\textit{гомоморфность*}} - это \textit{соответствие}, заданное на \textit{структурах}, при котором каждому элементу из области определения соответствия (первой \textit{структуры}) ставится в соответствие только один элемент из области значения соответствия (второй \textit{структуры}).}
\scnrelfrom{типичная семантическая окрестность}{
\scnfilescg{figures/sd_structures/homomorphism.png}
}

\scnheader{гомоморфизм*}
\scniselement{бинарное отношение}

\scnheader{изоморфность*}
\scnidtf{изоморфное соответствие*}
\scnidtf{изоморфность структур*}
\scnsubset{гомоморфность*}
\scniselement{бинарное отношение}
\scnexplanation{\textbf{\textit{изоморфность*}} - это \textit{гомоморфность*}, при которой для каждого элемента из области значения существует ровно один соответствующий элемент из области определения.}
\scnrelfrom{типичная семантическая окрестность}{
\scnfilescg{figures/sd_structures/isomorphism.png}
}

\scnheader{изоморфизм*}
\scniselement{бинарное отношение}

\scnheader{автомоморфность*}
\scnsubset{гомоморфность*}
\scniselement{бинарное отношение}
\scnexplanation{\textbf{\textit{автоморфность*}} - это \textit{изоморфность*}, у которой область определения соответствия и область значения соответствия совпадают.}
\scnrelfrom{типичная семантическая окрестность}{
\scnfilescg{figures/sd_structures/automorphism.png}}

\scnheader{автоморфизм*}
\scniselement{бинарное отношение}

\scnheader{аналогичность структур*}
\scnsubset{соответствие*}
\scniselement{бинарное отношение}
\scnexplanation{\textbf{\textit{аналогичность структур*}} - \textit{соответствие*}, задаваемое на структурах, и фиксирующее факт наличия некоторой аналогии на подструктурах (подмножествах) указанных структур. Каждой ориентированной паре, принадлежащей \textbf{\textit{аналогичности структур*}} может быть поставлено в соответствие множество пар, задающих \textit{сходства*} некоторых подструктур и \textit{различия*} некоторых подструктур исходных структур.}
\scnrelfrom{типичная семантическая окрестность}{
\scnfilescg{figures/sd_structures/analogy.png}}

\scnheader{сходство*}
\scniselement{бинарное отношение}

\scnheader{различие*}
\scniselement{бинарное отношение}

\scnheader{первичная синтаксическая структура sc-текста*}
\scniselement{бинарное отношение}
\scnexplanation{\textbf{\textit{первичная синтаксическая структура sc-текста*}} - это бинарное отношение, связывающее некоторый \textit{sc-текст} с другим \textit{sc-текстом}, формируемым по следующим правилам:
\begin{scnitemize}
    \item каждому \textit{sc-узлу} первого \textit{sc-текста} соответствует \textit{синглетон} (\textit{знак sc-узла}) в рамках второго \textit{sc-текста};
    \item каждому \textit{sc-коннектору} из первого \textit{sc-текста} в рамках второго \textit{sc-текста} соответствует \textit{синглетон}, обозначающий данный \textit{sc-коннектор} и соединенный с другими \textit{синглетонами} второго \textit{sc-текста} парами инцидентности двух типов, в зависимости от того, началом или концом данного \textit{sc-коннектора} являются обозначаемые этими \textit{синглетонами sc-элементы}. В случае, когда \textit{sc-коннектор} является \textit{sc-ребром}, то достаточно пар инцидентности первого типа.
    \item для каждого \textit{синглетона} в рамках второго \textit{sc-текста} явно указывается синтаксический тип, определяемый типом соответствующего ему элемента из первого \textit{sc-текста} (\textit{знак sc-константы}, \textit{знак sc-узла} и т.п.).
\end{scnitemize}


Стоит отметить, что подобным образом может быть задана синтаксическая структура любого текста, а не только sc-текста. В этом случае понадобятся другие отношения инцидентности другие классы синтаксических типов.}
\scnrelfrom{типичная семантическая окрестность}{
\scnfilescg{figures/sd_structures/primary_sc_syntax.png}}

\scnendstruct \scnendcurrentsectioncomment

\end{SCn}

\scsubsection[\scnmonographychapter{Глава 2.4. Формальные онтологии базовых классов сущностей - множеств, связей, отношений, параметров, величин, чисел, структур, темпоральных сущностей}]{Предметная область и онтология темпоральных сущностей}
\label{sd_temp_entities}
\begin{SCn}

\scnsectionheader{Предметная область и онтология темпоральных сущностей}

\scnstartsubstruct

\scnheader{Предметная область темпоральных сущностей}
\scnidtf{Предметная область темпоральных связей и отношений}
\scnidtf{Предметная область временных сущностей}
\scniselement{предметная область}
\scnsdmainclasssingle{временная сущность}
\scnsdclass{прошлая сущность;настоящая сущность;будущая сущность;временная связь;ситуация;процесс;процесс в sc-памяти;процесс во внешней среде ostis-системы;материальная сущность;воздействие;отношение;класс временных связей;класс временных и постоянных связей;множество;ситуативное множество;неситуативное множество;частично ситуативное множество;темпоральная связь;темпоральное отношение;начало;завершение;длительность;тысячелетие;век;год;месяц;сутки;час;минута;секунда}
\scnsdrelation{воздействующая сущность*;объект воздействия*;начальная ситуация*;причинная ситуация*;конечная ситуация*;событие*;последний добавленный sc-элемент’;темпоральное включение*;темпоральная часть*;начальный этап*;конечный этап*;промежуточный этап*;темпоральное включение без совпадения начальных и конечных моментов*;темпоральное включение с совпадением начальных моментов*;темпоральное включение с совпадением конечных моментов*;темпоральное совпадение*;темпоральное объединение*;темпоральная декомпозиция*;темпоральная смежность*;темпоральная последовательность с промежутком*;темпоральная последовательность с пересечением*;номер тысячелетия';номер века';номер года';номер месяца в году';номер суток в месяце';номер часа в дне';номер минуты в часе';номер секунды в минуте'}

\scnheader{временная сущность}
\scnidtf{временно существующая сущность}
\scnidtf{нестационарная сущность}
\scnidtf{сущность, имеющая и/или начало, и/или конец своего существования}
\scnidtf{sc-элемент, являющийся знаком некоторой временно существующей сущности}
\scnidtf{сущность, обладающая темпоральными характеристиками (длительностью, начальным моментом, конечным моментом и т.д.)}
\scnreltoset{разбиение}{прошлая сущность;настоящая сущность;будущая сущность}
\scnreltoset{разбиение}{временная связь;ситуация;процесс;материальная сущность}
\scnexplanation{Следует отличать:
\begin{scnitemize}
    \item временный характер сущности, обозначаемой \textit{sc-элементом};
    \item временный характер существования самого \textit{sc-элемента} в рамках \textit{sc-памяти};
\end{scnitemize}
В ходе обработки информации каждый \textit{sc-элемент} может быть удален из \textit{sc-памяти}.}

\scnheader{прошлая сущность}
\scnidtf{сущность, существовавшая в прошлом времени}
\scnidtf{сущность прошлого времени}
\scnidtf{сущность, завершившая свое существование}

\scnheader{настоящая сущность}
\scnidtf{сущность, существующая в текущий момент времени}
\scnidtf{сущность, существующая сейчас}
\scnidtf{сущность настоящего времени}

\scnheader{будущая сущность}
\scnidtf{возможно будущая сущность}
\scnidtf{прогнозируемая временная сущность}
\scnidtf{временная сущность, которая может существовать в будущем}
\scnidtf{сущность, которая может или должна начать свое существование в будущем времени}
\scnrelfrom{включение}{инициированное действие}
\scnexplanation{Каждой \textbf{\textit{будущей сущности}} можно поставить в соответствие вероятность ее возникновения.}

\scnheader{временная связь}
\scnidtf{нестационарная связь}
\scnidtf{временно существующая связь}
\scnexplanation{Каждая \textbf{\textit{временная связь}} представляет собой \textit{связку}, принадлежащую множеству \textit{временных сущностей}.

Понятие \textbf{\textit{временной связи}} не следует путать с понятием \textit{темпоральной связи}, которая сама является \textit{постоянной сущностью}, описывающей то, как связаны во времени некоторые \textit{временные сущности}.
}

\scnheader{ситуация}
\scnidtf{состояние}
\scnidtf{временная структура}
\scnidtf{временно существующая структура}
\scnidtf{квазистационарная структура}
\scnidtf{состояние некоторой динамической системы, описываемое с некоторой степенью детализации (подробности)}
\scnidtf{квазистационарная структура, существующая временно (в течение некоторого отрезка времени)}
\scnrelto{включение}{структура}
\scnexplanation{Под ситуацией понимается \textit{структура}, содержащая, по крайней мере, один элемент, который является \textit{временной сущностью}. Наличие в рамках ситуации нескольких \textit{временных сущностей} говорит о том, что существует момент времени (в прошлом, настоящем или будущем), в который все они существуют одновременно.}

\scnheader{процесс}
\scnidtf{процесс преобразования некоторой временной сущности из одного состояния в другое}
\scnidtf{процесс перехода от одной ситуации к другой}
\scnidtf{переходный процесс}
\scnidtf{абстрактный процесс}
\scnidtf{информационная модель некоторых преобразований}
\scnidtf{динамическая sc-модель}
\scnidtf{динамическая структура}
\scnrelfrom{включение}{воздействие}
\scnrelto{включение}{структура}
\scnexplanation{Каждый \textbf{\textit{процесс}} определяется (задается) либо возникновением или исчезновением связей, связывающих заданную \textit{временную сущность} с другими сущностями, либо возникновением или исчезновением связей, связывающих части указанной \textit{временной сущности} с другими сущностями. 

Многим \textbf{\textit{процессам}} можно поставить в соответствие \textit{ситуацию}, которая является его \textit{начальной ситуацией*} и \textit{ситуацию}, которая является его \textit{конечной ситуацией*}, то есть указать \textit{ситуации}, переход между которыми осуществляется в ходе \textbf{\textit{процесса}}.

При этом знаки тех \textit{временных сущностей}, с которыми связаны изменения, описываемые некоторым \textbf{\textit{процессом}}, входят в данный \textbf{\textit{процесс}} как элементы и, при необходимости уточняются дополнительными \textit{ролевыми отношениями}.}
\scnreltoset{разбиение}{процесс в sc-памяти;процесс во внешней среде ostis-системы}

\scnheader{процесс в sc-памяти}

\scnheader{процесс во внешней среде ostis-системы}

\scnheader{материальная сущность}
\scnexplanation{Каждой \textbf{\textit{материальной сущности}} можно поставить в соответствие различные \textit{процессы}, описывающие ее преобразование из одного состояния в другое.}

\scnheader{воздействие}
\scnidtf{процесс, осуществляющийся на основе (в результате) воздействия одной сущности на другую}
\scnrelfrom{включение}{действие}
\scnexplanation{Каждому \textbf{\textit{воздействию}} может быть поставлена в соответствие (1) некоторая \textit{воздействующая сущность*}, т.е. сущность, осуществляющая \textbf{\textit{воздействие}} (в частности, это может быть некоторое физическое поле), и (2) некоторый \textit{объект воздействия*}, т.е. сущность, на которую воздействие направлено. Если \textbf{\textit{воздействие}} связано с \textit{материальной сущностью}, то его объектом воздействия является либо сама эта \textit{материальная сущность}, либо некоторая ее пространственная часть.}

\scnheader{воздействующая сущность*}

\scnheader{объект воздействия*}

\scnheader{начальная ситуация*}
\scnidtf{начальная ситуация процесса*}
\scnidtf{исходная ситуация*}
\scniselement{бинарное отношение}
\scnexplanation{Связки отношения \textbf{\textit{начальная ситуация*}} связывают некоторый \textit{процесс} и некоторую ситуацию, являющуюся начальной для этого \textit{процесса}, и, как правило, изменяемой в течение выполнения этого \textit{процесса}.

Первым компонентом каждой связки отношения \textbf{\textit{начальная ситуация*}} является знак \textit{процесса}, вторым – знак начальной \textit{ситуации}.}

\scnheader{причинная ситуация*}
\scniselement{бинарное отношение}
\scnrelto{включение}{начальная ситуация*}
\scnexplanation{Под причинной ситуацией понимается такая \textit{начальная ситуация*}, которая обладает достаточной полнотой для однозначного задания инициируемого \textit{процесса}.}

\scnheader{конечная ситуация*}
\scnidtf{конечная ситуация процесса*}
\scnidtf{результирующая ситуация*}
\scniselement{бинарное отношение}
\scnexplanation{Связки отношения \textbf{\textit{конечная ситуация*}} связывают некоторый \textit{процесс} и некоторую \textit{ситуацию}, ставшую результатом выполнения этого \textit{процесса}, то есть его следствием.

Первым компонентом каждой связки отношения \textbf{\textit{конечная ситуация*}} является знак \textit{процесса}, вторым – знак конечной \textit{ситуации}.}

\scnheader{событие*}
\scniselement{бинарное отношение}
\scnexplanation{Связки отношения \textbf{\textit{событие*}} связывают знак процесса и ориентированную пару, первым компонентом которой является знак \textit{начальной ситуации*} данного процесса, вторым компонентом – знак \textit{конечной ситуации*} данного процесса.}
\scnrelfrom{типичная семантическая окрестность}{
\scnfilelong{
\begin{figure}[H]
\centering
\includegraphics[width=1\linewidth]{figures/sd_temp_entities/event.png}
\end{figure}
}}

\scnheader{отношение}
\scnreltoset{разбиение}{класс временных связей;класс постоянных связей;класс временных и постоянных связей}

\scnheader{класс временных связей}
\scnidtf{отношение, все связки которого являются нестационарными}
\scnexplanation{В общем случае \textbf{\textit{класс временных связей}} не является \textit{ситуативным множеством}, поскольку факт принадлежности некоторой \textit{временной связи} такому классу следует считать постоянным, а не временным, поскольку временность/постоянство связи и ее семантический тип, задаваемый классом (отношением), это принципиально разные параметры (характеристики, признаки) любой связи.}

\scnheader{класс постоянных связей}
\scnidtf{отношение, все связки которого являются стационарными}

\scnheader{класс временных и постоянных связей}
\scnidtf{отношение, некоторые (но не все) связки которого являются нестационарными}

\scnheader{множество}
\scnreltoset{разбиение}{ситуативное множество;неситуативное множество;частично ситуативное множество}

\scnheader{ситуативное множество}
\scnidtf{полностью ситуативное множество}
\scnexplanation{Под \textbf{\textit{ситуативным множеством}} понимается постоянное множество, у которого все выходящие из него связи принадлежности являются \textit{временными сущностями}.

В частности, ситуативное множество может использоваться как вспомогательная динамическая структура, которая содержит элементы некоторых структур, обрабатываемые в данный момент, например, это может быть копия некоторого множества, из которой постепенно удаляются элементы по мере их просмотра и обработки. В случае, когда такая структура содержит всего один элемент, ее можно считать \underline{указателем} на данный элемент, при этом в разные моменты времени это могут быть разные элементы.}

\scnheader{последний добавленный sc-элемент’}
\scniselement{ролевое отношение}

\scnheader{неситуативное множество}
\scnexplanation{Под \textbf{\textit{неситуативным множеством}} понимается постоянное множество, у которого все выходящие из него связи принадлежности являются \textit{постоянными сущностями}.}

\scnheader{частично ситуативное множество}
\scnexplanation{Под \textbf{\textit{частично ситуативным множеством}} понимается постоянное множество, у которого некоторые (но не все) выходящие из него связи принадлежности являются \textit{временными сущностями}.}

\scnheader{темпоральная связь}
\scnidtf{постоянная связь, описывающая связь во времени между временными сущностями}

\scnheader{темпоральное отношение}
\scnrelto{семейство подмножеств}{темпоральная связь}
\scnidtf{класс темпоральных связей}
\scnidtf{отношение, задающее темпоральные связи между временными сущностями}
\scnhaselement{темпоральное включение*}
\scnhaselement{темпоральное объединение*}
\scnhaselement{темпоральная декомпозиция*}
\scnhaselement{темпоральная смежность*}
\scnhaselement{темпоральная последовательность с промежутком*}
\scnhaselement{темпоральная последовательность с пересечением*}

\scnheader{темпоральное включение*}
\scnexplanation{Связки отношения \textbf{\textit{темпоральное включение*}} связывают две \textit{временные сущности}, период существования одной из которых полностью включается в период существования второй.\\
Первым компонентом каждой связки отношения \textbf{\textit{темпоральное включение*}} является знак \textit{временной сущности}, \textit{длительность} существования которой больше.}
\scnrelfromlist{включение}{темпоральная часть*;темпоральное включение без совпадения начальных и конечных моментов*;темпоральное совпадение*;темпоральное включение с совпадением начальных моментов*;темпоральное включение с совпадением конечных моментов*}

\scnheader{темпоральная часть*}
\scnidtf{этап (период) заданной временной сущности*}
\scnidtf{этап процесса существования временной сущности*}
\scnrelfromlist{включение}{начальный этап*;конечный этап*;промежуточный этап*}
\scnrelfrom{типичная семантическая окрестность}{
\scnfilelong{
\begin{figure}[H]
\centering
\includegraphics[width=1\linewidth]{figures/sd_temp_entities/temporal_part.png}
\end{figure}
}}
\scnrelfrom{иллюстрация}{
\scnfilelong{
\begin{figure}[H]
\centering
\includegraphics[width=1\linewidth]{figures/sd_temp_entities/img_temporal_part.png}
\end{figure}
}}
\scntext{примечание}{Связки отношения \textbf{\textit{темпоральная часть*}} связывают две \textit{временные сущности}, одна из которых является частью другой, например, действие и одно из его поддействий. Соответственно, период существования одной из этих сущностей всегда будет включаться в период существования другой (большей).

В отличие от более общего отношения \textit{темпоральное включение*}, связки которого могут связывать любые \textit{временные сущности}, связки отношения \textbf{\textit{темпоральное включение*}} связывают только \textit{временные сущности}, одна из которых является частью другой.}

\scnheader{начальный этап*}

\scnheader{конечный этап*}

\scnheader{промежуточный этап*}

\scnheader{темпоральное включение без совпадения начальных и конечных моментов*}
\scnidtf{строгое темпоральное включение*}
\scnrelfrom{типичная семантическая окрестность}{
\scnfilelong{
\begin{figure}[H]
\centering
\includegraphics[width=1\linewidth]{figures/sd_temp_entities/strict_temporal_inclusion.png}
\end{figure}
}}
\scnrelfrom{иллюстрация}{
\scnfilelong{
\begin{figure}[H]
\centering
\includegraphics[width=1\linewidth]{figures/sd_temp_entities/img_strict_temporal_inclusion.png}
\end{figure}
}}
%
% темпоральное включение без совпадения начальных и конечных %моментов
%

\scnheader{темпоральное включение с совпадением начальных моментов*}
\scnrelfrom{типичная семантическая окрестность}{
\scnfilelong{
\begin{figure}[H]
\centering
\includegraphics[width=1\linewidth]{figures/sd_temp_entities/temporal_include_with_match_start_points.png}
\end{figure}
}}
\scnrelfrom{иллюстрация}{
\scnfilelong{
\begin{figure}[H]
\centering
\includegraphics[width=1\linewidth]{figures/sd_temp_entities/img_temporal_include_with_match_start_points.png}
\end{figure}
}}

\scnheader{темпоральное включение с совпадением конечных моментов*}
\scnrelfrom{типичная семантическая окрестность}{
\scnfilelong{
\begin{figure}[H]
\centering
\includegraphics[width=1\linewidth]{figures/sd_temp_entities/temporal_include_with_terminal_point_match.png}
\end{figure}
}}
\scnrelfrom{иллюстрация}{
\scnfilelong{
\begin{figure}[H]
\centering
\includegraphics[width=1\linewidth]{figures/sd_temp_entities/img_temporal_include_with_terminal_point_match.png}
\end{figure}
}}

\scnheader{темпоральное совпадение*}
\scnidtf{совпадение начала и завершения*}

\scnheader{темпоральное объединение*}
\scnrelfrom{типичная семантическая окрестность}{
\scnfilelong{
\begin{figure}[H]
\centering
\includegraphics[width=1\linewidth]{figures/sd_temp_entities/temporal_union.png}
\end{figure}
}}
\scnrelfrom{иллюстрация}{
\scnfilelong{
\begin{figure}[H]
\centering
\includegraphics[width=1\linewidth]{figures/sd_temp_entities/img_temporal_union.png}
\end{figure}
}}

\scnheader{темпоральная декомпозиция*}
\scnrelfrom{типичная семантическая окрестность}{
\scnfilelong{
\begin{figure}[H]
\centering
\includegraphics[width=1\linewidth]{figures/sd_temp_entities/temporal_decomposition.png}
\end{figure}
}}
\scnrelfrom{иллюстрация}{
\scnfilelong{
\begin{figure}[H]
\centering
\includegraphics[width=1\linewidth]{figures/sd_temp_entities/img_temporal_decomposition.png}
\end{figure}
}}

\scnheader{темпоральная смежность*}
\scnidtf{строгая темпоральная последовательность (без темпорального промежутка)*}
\scnidtf{темпоральная последовательность без промежутка*}
\scnrelfrom{типичная семантическая окрестность}{
\scnfilelong{
\begin{figure}[H]
\centering
\includegraphics[width=1\linewidth]{figures/sd_temp_entities/temporal_adjacency.png}
\end{figure}
}}
\scnrelfrom{иллюстрация}{
\scnfilelong{
\begin{figure}[H]
\centering
\includegraphics[width=1\linewidth]{figures/sd_temp_entities/img_temporal_adjacency.png}
\end{figure}
}}

\scnheader{темпоральная последовательность с промежутком*}
\scnrelfrom{типичная семантическая окрестность}{
\scnfilelong{
\begin{figure}[H]
\centering
\includegraphics[width=1\linewidth]{figures/sd_temp_entities/temporal_sequence_with_intermediate.png}
\end{figure}
}}
\scnrelfrom{иллюстрация}{
\scnfilelong{
\begin{figure}[H]
\centering
\includegraphics[width=1\linewidth]{figures/sd_temp_entities/img_temporal_sequence_with_intermediate.png}
\end{figure}
}}

\scnheader{темпоральная последовательность с пересечением*}
\scnrelfrom{типичная семантическая окрестность}{
\scnfilelong{
\begin{figure}[H]
\centering
\includegraphics[width=1\linewidth]{figures/sd_temp_entities/temporal_sequence_with_intersection.png}
\end{figure}
}}
\scnrelfrom{иллюстрация}{
\scnfilelong{
\begin{figure}[H]
\centering
\includegraphics[width=1\linewidth]{figures/sd_temp_entities/img_temporal_cross_sequence.png}
\end{figure}
}}

\scnheader{начало}
\scnidtf{класс одновременно начавшихся сущностей}
\scniselement{параметр}
\scnexplanation{Каждый элемент множества \textbf{начало} представляет собой класс \textit{временных сущностей}, у которых совпадает момент начала их существования. Конкретное значение данного \textit{параметра} может быть как \textit{точной величиной}, так и \textit{неточной величиной} или \textit{интервальной величиной}.}
\scnrelfrom{типичная семантическая окрестность}{
\scnfilelong{
\begin{figure}[H]
\centering
\includegraphics[width=1\linewidth]{figures/sd_temp_entities/start.png}
\end{figure}
}}
\scncomment{В данном примере \textit{ki} обозначает класс сущностей, начавших свое существование 19 февраля 2015 года по григорианскому календарю. Конкретные примеры таких сущностей – \textit{bi} и \textit{bj}. \textit{ti} обозначает временную точку григорианского календаря, соответствующую 19 февраля 2015 года.}

\scnheader{завершение}
\scnidtf{конец}
\scnidtf{класс одновременно завершившихся сущностей}
\scniselement{параметр}
\scnexplanation{Каждый элемент множества \textbf{\textit{завершение}} представляет собой класс \textit{временных сущностей}, у которых совпадает конечный момент их существования (момент завершения существования). Конкретное значение данного \textit{параметра} может быть как \textit{точной величиной}, так и \textit{неточной величиной} или \textit{интервальной величиной}.}
\scnrelfrom{типичная семантическая окрестность}{
\scnfilelong{
\begin{figure}[H]
\centering
\includegraphics[width=1\linewidth]{figures/sd_temp_entities/completion.png}
\end{figure}
}}
\scncomment{В данном примере \textit{ki} обозначает класс сущностей, завершивших свое существование 21 февраля 2015 года по григорианскому календарю. Конкретные примеры таких сущностей – \textit{bi} и \textit{bj}. \textit{ti} обозначает временную точку григорианского календаря, соответствующую 21 февраля 2015 года.}

\scnheader{длительность}
\scnidtf{класс временных сущностей, имеющих одинаковую длительность}
\scniselement{параметр}
\scnhaselement{тысячелетие}
\scnhaselement{век}
\scnhaselement{год}
\scnhaselement{месяц}
\scnhaselement{день}
\scnhaselement{час}
\scnhaselement{минута}
\scnhaselement{секунда}
\scnexplanation{Каждый элемент множества \textbf{\textit{длительность}} представляет собой класс \textit{временных сущностей}, у которых совпадает длительность их существования. Конкретное значение данного \textit{параметра} может быть как \textit{точной величиной}, так и \textit{неточной величиной} или \textit{интервальной величиной}.}
\scnrelfrom{типичная семантическая окрестность}{
\scnfilelong{
\begin{figure}[H]
\centering
\includegraphics[width=1\linewidth]{figures/sd_temp_entities/duration.png}
\end{figure}
}}
\scncomment{В данном примере \textit{ki} обозначает класс сущностей, существовавших в течение 2 месяцев. Конкретный пример такой сущности – \textit{bi}.}

\scnheader{тысячелетие}

\scnheader{век}

\scnheader{год}

\scnheader{месяц}

\scnheader{сутки}

\scnheader{час}

\scnheader{минута}

\scnheader{секунда}

\scnheader{номер тысячелетия'}
\scnheader{номер века'}
\scnheader{номер года'}
\scnheader{номер месяца в году'}
\scnheader{номер суток в месяце'}
\scnheader{номер часа в дне'}
\scnheader{номер минуты в часе'}
\scnheader{номер секунды в минуте'}

\scnendstruct

\end{SCn}

\scsubsubsection[\scnmonographychapter{Глава 2.4. Формальные онтологии базовых классов сущностей - множеств, связей, отношений, параметров, величин, чисел, структур, темпоральных сущностей}]{Предметная область и онтология ситуаций и событий, описывающих динамику баз знаний ostis-систем}
\label{sd_temp_know_base}
\begin{SCn}

\scnsectionheader{\currentname}

\scnstartsubstruct

\scntext{введение}{Обработка информации в \textit{sc-памяти} (т.е. динамика базы знаний, хранимой в \textit{sc-памяти}) в конечном счете сводится:
	\begin{scnitemize}
		\item к появлению в \textit{sc-памяти} новых актуальных \textit{sc-узлов} и \textit{sc-коннекторов};
		\item к логическому удалению актуальных \textit{sc-элементов}, т.е. к переводу их в неактуальное состояние (это необходимо для хранения протокола изменения состояния базы знаний, в рамках которого могут описываться действия по удалению \textit{sc-элементов});
		\item к возврату логически удаленных \textit{sс-элементов} в статус актуальных (необходимость в этом может возникнуть при откате базы знаний к какой-нибудь ее прошлой версии);
		\item к физическому удалению \textit{sc-элементов};
		\item к изменению состояния актуальных (логически не удаленных \textit{sc-элементов}) -- \textit{sc-узел} может превратиться в \textit{sc-ребро}, \textit{sc-ребро} может превратиться в \textit{sc-дугу}, \textit{sc-дуга} может поменять направленность, \textit{sc-дуга} общего вида может превратиться в \textit{константную стационарную sc-дугу принадлежности}, и т.д.;
	\end{scnitemize}
	Подчеркнем, что временный характер самого \textit{sc-элемента} (т.к. он может появиться или исчезнуть) никак не связан с возможно временным характером сущности, обозначаемой этим \textit{sc-элементом}. Т.е. временный характер самого sc-элемента и временный характер сущности, которую он обозначает -- абсолютно разные вещи.
	
	Таким образом, следует четко отличать динамику внешнего мира, описываемого базой знаний, а динамику самой базы знаний (динамику внутреннего мира). При этом очень важно, чтобы описание динамики базы знаний также входило в состав каждой базы знаний.
	
	К числу понятий, используемых для описания динамики базы знаний относятся:
	\begin{scnitemize}
		\item логически удаленный sc-элемент;
		\item сформированное множество;
		\item вычисленное число;
		\item сформированное высказывание;
\end{scnitemize}}

\scnheader{Предметная область темпоральных сущностей базы знаний ostis-системы}
\scnidtf{Предметная область, описывающая динамику базы знаний, хранимой в sc-памяти}
\scniselement{предметная область}
\scnsdmainclasssingle{ситуация}
\scnsdclass{sc-элемент;наcтоящий sc-элемент;логически удаленный sc-элемент;число;невычисленное число;вычисленное число;понятие;основное понятие;неосновное понятие;понятие, переходящее из основного в неосновное;понятие, переходящее из неосновного в основное;специфицированная сущность;недостаточно специфицированная сущность;достаточно специфицированная сущность;средне специфицированная сущность;структура;файл;событие в sc-памяти*;элементарное событие в sc-памяти*;событие добавления sc-дуги, выходящей из заданного sc-элемента*;событие добавления sc-дуги, входящей в заданный sc-элемент*;событие добавления sc-ребра, инцидентного заданному sc-элементу*;событие удаления sc-дуги, выходящей из заданного sc-элемента*;событие удаления sc-дуги, входящей в заданный sc-элемент*;событие удаления sc-ребра, инцидентного заданному sc-элементу*;событие удаления sc-элемента*;событие изменения содержимого файла*}

\scnheader{sc-элемент}
\scnreltoset{разбиение}{наcтоящий sc-элемент;логически удаленный sc-элемент}

\scnheader{наcтоящий sc-элемент}
\scniselement{ситуативное множество}

\scnheader{логически удаленный sc-элемент}
\scniselement{ситуативное множество}

\scnheader{число}
\scnsubdividing{невычисленное число;вычисленное число}

\scnheader{невычисленное число}
\scniselement{ситуативное множество}

\scnheader{вычисленное число}

\scnheader{понятие}
\scnsubdividing{основное понятие;неосновное понятие;понятие, переходящее из основного в неосновное;понятие, переходящее из неосновного в основное}

\scnheader{основное понятие}
\scnidtf{основное понятие для данной ostis-системы}
\scniselement{ситуативное множество}
\scnexplanation{К \textbf{\textit{основным понятиям}} относятся те понятия, которые активно используются в системе и могут быть ключевыми элементами sc-агентов. К \textbf{\textit{основным понятиям}} относятся также все неопределяемые понятия.}

\scnheader{неосновное понятие}
\scnidtf{дополнительное понятие}
\scnidtf{вспомогательное понятие}
\scnidtf{неосновное понятие для данной ostis-системы}
\scniselement{ситуативное множество}
\scnexplanation{Каждое \textbf{\textit{неосновное понятие}} должно быть строго определено на основе \textit{основных понятий}. Такие \textbf{\textit{неосновные понятия}} используются только для понимания и правильного восприятия вводимой информации, в том числе, для выравнивания онтологий. Ключевым элементом \textit{sc-агентов} \textbf{\textit{неосновные понятия}} быть не могут.}
\scntext{правило идентификации экземпляров}{В случае, когда некоторое понятие полностью перешло из \textit{основных понятий} в неосновные, то есть стало \textbf{\textit{неосновным понятием}}, и соответствующее ему \textit{основное понятие} (через которое оно определяется) в рамках некоторого внешнего языка имеет одинаковый с ним основной идентификатор, то к идентификатору \textbf{\textit{неосновного понятия}} спереди добавляется знак \#. Если при этом соответствуюшее \textit{основное понятие} имеет в идентификаторе знак \$, добавленный в процессе перехода, то этот знак удаляется. Если указанные понятия имеют разные основные идентификаторы в рамках этого внешнего языка, то никаких дополнительных средств идентификации не используется.

Например:\\
\textit{\#трансляция sc-текста}\\
\textit{\#scp-программа}}

\scnheader{понятие, переходящее из основного в неосновное}
\scniselement{ситуативное множество}

\scnheader{понятие, переходящее из неосновного в основное}
\scniselement{ситуативное множество}
\scntext{правило идентификации экземпляров}{В случае, когда текущее \textit{основное понятие} и соответствующее ему \textbf{\textit{понятие, переходящее из неосновного в основное}} в рамках некоторого внешнего языка имеют одинаковый основной идентификатор, то к идентификатору понятия, переходящего из неосновного в основное спереди добавляется знак \$. Если указанные понятия имеют разные основные идентификаторы в рамках этого внешнего языка, то никаких дополнительных средств идентификации не используется.

Например:\\
\textit{\$трансляция sc-текста}\\
\textit{\$scp-программа}}

\scnheader{специфицированная сущность}
\scnsubdividing{недостаточно специфицированная сущность;достаточно специфицированная сущность;средне специфицированная сущность}

\scnheader{достаточно специфицированная сущность}
\scnexplanation{К \textbf{\textit{достаточно специфицированным сущностям}} предъявляются следующие требования:
\begin{scnitemize}
    \item если сущность не является понятием, то для нее должны быть указаны
    \begin{scnitemizeii}
    \item различные варианты обозначающих ее внешних знаков;
    \item классы, которым она принадлежит;
    \item связки, которыми она связана с другими сущностями (с указанием соответствующего отношения);
    \item значения параметров, которыми она обладает;
    \item те разделы базы знаний, в которых указанная сущность является ключевой;
    \item предметные области, в которые данная сущность входит.
    \end{scnitemizeii}
    \item если специфицированная сущность является понятием, то для нее должны быть указаны:
    \begin{scnitemizeii}
    \item различные варианты внешних обозначений этого понятия;
    \item предметные области, в которых это понятие исследуется;
    \item определение понятия;
    \item пояснения
    \item разделы базы знаний, в которых это понятие является ключевым;
    \item описание примера -- пример экземпляра понятия.
    \end{scnitemizeii}
\end{scnitemize}}

\scnheader{структура}
\scnsubdividing{сформированная структура;несформированная структура}
\scnsubdividing{недостаточно сформированная структура;достаточно сформированная структура;структура, имеющая средний уровень сформированности}

\scnheader{файл}
\scnsubdividing{недостаточно сформированный внутренний файл;достаточно сформированный внутренний файл;внутренний файл, имеющий средний уровень сформированности}

\scnheader{событие в sc-памяти}
\scnsuperset{событие}

\scnheader{элементарное событие в sc-памяти}
\scnsubset{событие в sc-памяти}
\scnexplanation{Под \textbf{\textit{элементарным событием в sc-памяти}} понимается такое \textit{событие}, в результате выполнения которого изменяется состояние только одного \textit{sc-элемента}.}
\scnsubdividing{событие добавления sc-дуги, выходящей из заданного sc-элемента
;событие добавления sc-дуги, входящей в заданный sc-элемент;событие добавления sc-ребра, инцидентного заданному sc-элементу;событие удаления sc-дуги, выходящей из заданного sc-элемента;событие удаления sc-дуги, входящей в заданный sc-элемент;событие удаления sc-ребра, инцидентного заданному sc-элементу;событие удаления sc-элемента;событие изменения содержимого файла}

\scnheader{точечный процесс}
\scnidtf{атомарный процесс}
\scnidtf{условно мгновенный процесс}
\scnidtf{процесс, длительность которого в данном контексте считается несущественной (пренебрежимо малой)}

\scnheader{элементарный процесс}
\scnidtf{процесс, детализация которого на входящие в него подпроцессы в текущем контексте не производится}

\bigskip
\scnendstruct \scnendcurrentsectioncomment

\end{SCn}

\scsubsection[\scnmonographychapter{Глава 2.4. Формальные онтологии базовых классов сущностей - множеств, связей, отношений, параметров, величин, чисел, структур, темпоральных сущностей}]{Предметная область и онтология пространственных сущностей различных форм}
\label{sd_spatial_entities}

\scsubsection[\scnmonographychapter{Глава 2.4. Формальные онтологии базовых классов сущностей - множеств, связей, отношений, параметров, величин, чисел, структур, темпоральных сущностей}]{Предметная область и онтология материальных сущностей}
\label{sd_material_entities}

\scsubsection[\scnmonographychapter{Глава 2.3. Структура баз знаний интеллектуальных компьютерных систем нового поколения: иерархическая система предметных областей и онтологий. Онтологии верхнего уровня. Формализация понятий семантической окрестности, предметной области и онтологии в интеллектуальных компьютерных системах нового поколения}]{Предметная область и онтология семантических окрестностей}
\label{sd_sem_neigh}
\begin{SCn}

\scnsectionheader{\currentname}

\scnstartsubstruct

\scnheader{Предметная область семантических окрестностей}
\scniselement{предметная область}
\scnsdmainclasssingle{семантическая окрестность}
\scnsdclass{семантическая окрестность по инцидентным коннекторам;семантическая окрестность по выходящим дугам;семантическая окрестность по выходящим дугам принадлежности;семантическая окрестность по входящим дугам;семантическая окрестность по входящим дугам принадлежности;полная семантическая окрестность;базовая семантическая окрестность;специализированная семантическая окрестность;пояснение;примечание;правило идентификации экземпляров;терминологическая семантическая окрестность;теоретико-множественная семантическая окрестность;описание декомпозиции;логическая семантическая окрестность;описание типичного экземпляра;сравнительный анализ;иллюстрация}

\scnheader{семантическая окрестность}
\scnidtf{sc-окрестность}
\scnidtf{семантическая окрестность, представленная в виде sc-текста}
\scnidtf{sc-текст, являющийся семантической окрестностью некоторого sc-элемента}
\scnidtf{спецификация заданной сущности, знак которой указывается как ключевой элемент этой спецификации}
\scnidtf{описание заданной сущности, знак которой указывается как ключевой элемент этой спецификации}
\scnsubset{знание}
\scnsuperset{семантическая окрестность по инцидентным коннекторам}
\scnsuperset{полная семантическая окрестность}
\scnsuperset{базовая семантическая окрестность}
\scnsuperset{специализированная семантическая окрестность}
\scnexplanation{\textbf{\textit{семантическая окрестность}} – это знание, являющееся спецификацией (описанием) некоторой сущности, знак которой является ключевым элементом указанного знания. Заметим, что каждая семантическая окрестность в отличие от знаний других видов имеет только один ключевой элемент (ключевой знак, знак описываемой сущности). Заметим также, что многообразие видов семантических окрестностей свидетельствует о многообразии семантических видов описаний различных сущностей.}

\scnheader{семантическая окрестность по инцидентным коннекторам}
\scnsuperset{семантическая окрестность по выходящим дугам}
\scnsuperset{семантическая окрестность по входящим дугам}
\scnexplanation{\textbf{\textit{семантическая окрестность по инцидентным коннекторам}} – это вид семантической окрестности, в которую входят знаки всех коннекторов, инцидентных заданному элементу, а также знаки всех элементов, инцидентных указанным коннекторам.}

\scnheader{семантическая окрестность по выходящим дугам}
\scnsuperset{семантическая окрестность по выходящим дугам принадлежности}
\scnexplanation{\textbf{\textit{семантическая окрестность по выходящим дугам}} – это вид семантической окрестности, в которую входят знаки всех дуг, выходящих из заданного sc-элемента, а также знаки их вторых компонентов, также указывается факт принадлежности этих дуг каким-либо отношениям.}

\scnheader{семантическая окрестность по выходящим дугам принадлежности}
\scnexplanation{\textbf{\textit{семантическая окрестность по выходящим дугам принадлежности}} – это вид семантической окрестности, в которую входят знаки всех дуг принадлежности, выходящих из заданного sc-элемента, а также знаки их вторых компонентов. При необходимости может указывается факт принадлежности этих дуг каким-либо ролевым отношениям.}

\scnheader{семантическая окрестность по входящим дугам}
\scnsuperset{семантическая окрестность по входящим дугам принадлежности}
\scnexplanation{\textbf{\textit{семантическая окрестность по входящим дугам}} – это вид семантической окрестности, в которую входят знаки всех дуг, входящих в заданный sc-элемент, а также знаки их первых компонентов, также указывается факт принадлежности этих дуг каким-либо отношениям.}

\scnheader{семантическая окрестность по входящим дугам принадлежности}
\scnexplanation{\textbf{\textit{семантическая окрестность по входящим дугам принадлежности}} – это вид семантической окрестности, в которую входят знаки всех дуг принадлежности, входящих в заданный sc-элемент, а также знаки их первых компонентов. При необходимости может указывается факт принадлежности этих дуг каким-либо ролевым отношениям.}

\scnheader{полная семантическая окрестность}
\scnidtf{полная спецификация некоторой описываемой сущности}
\scnexplanation{\textbf{\textit{полная семантическая окрестность}} – это вид семантической окрестности, включающий описание всех связей описываемой сущности. 

Структура полной семантической окрестности определяется прежде всего семантической типологией описываемой сущности. 

Так, например, для понятия в полную семантическую окрестность необходимо включить следующую информацию (при наличии):
\begin{scnitemize}
    \item варианты идентификации на различных внешних языках;
    \item принадлежность некоторой предметной области с указанием роли, выполняемой в рамках этой предметной области;
    \item теоретико-множественные связи заданного понятия с другими sc-элементами;
    \item определение или пояснение;
    \item высказывания, описывающие свойства указанного понятия;
    \item задачи и их классы, в которых данное понятие является ключевым
    \item описание типичного примера использования указанного понятия;
    \item экземпляры описываемого понятия.
\end{scnitemize}
Для понятия, являющегося отношением дополнительно указываются:
\begin{scnitemize}
    \item домены;
    \item область определения;
    \item схема отношения;
    \item классы отношений, которым принадлежит описываемое отношение.
\end{scnitemize}
}

\scnheader{базовая семантическая окрестность}
\scnidtf{минимально достаточная семантическая окрестность}
\scnidtf{минимальная спецификация описываемой сущности}
\scnidtf{сокращенная спецификация описываемой сущности}
\scnidtf{основная семантическая окрестность}
\scnexplanation{\textbf{\textit{базовая семантическая окрестность}} – это вид семантической окрестности, содержащий минимальную (краткую) информацию об описываемой сущности

Структура базовой семантической окрестности определяется прежде всего семантической типологией описываемой сущности. 

Так, например, для понятия в базовую семантическую окрестность необходимо включить следующую информацию (при наличии): 
\begin{scnitemize}
    \item варианты идентификации на различных внешних языках;
    \item принадлежность некоторой предметной области с указанием роли, выполняемой в рамках этой предметной области;
    \item определение или пояснение.
\end{scnitemize}
Для понятия, являющегося отношением дополнительно указываются:
\begin{scnitemize}
    \item домены;
    \item область определения;
    \item описание типичного примера использования указанного отношения.
\end{scnitemize}
}

\scnheader{специализированная семантическая окрестность}
\scnsuperset{пояснение}
\scnsuperset{примечание}
\scnsuperset{правило идентификации экземпляров}
\scnsuperset{терминологическая семантическая окрестность}
\scnsuperset{теоретико-множественная семантическая окрестность}
\scnsuperset{логическая семантическая окрестность}
\scnsuperset{описание типичного экземпляра}
\scnsuperset{описание декомпозиции} 
\scnexplanation{\textbf{\textit{специализированная семантическая окрестность}} – это вид семантической окрестности, набор связей для которой уточняется отдельно для каждого класса такой окрестности.}

\scnheader{пояснение}
\scnidtf{sc-пояснение}
\scnexplanation{\textbf{\textit{пояснение}} – знак sc-текста, поясняющего описываемую сущность.}

\scnheader{примечание}
\scnidtf{sc-примечание}
\scnexplanation{\textbf{\textit{примечание}} – знак sc-текста, являющегося примечанием к описываемой сущности. В примечании обычно описываются особые свойства и исключения из правил для описываемой сущности.}

\scnheader{правило идентификации экземпляров}
\scnidtf{правило идентификации экземпляров заданного класса}
\scnexplanation{\textbf{\textit{правило идентификации экземпляров}} – это sc-текст являющийся описанием правил построения идентификаторов элементов заданного класса.}

\scnheader{терминологическая семантическая окрестность}
\scnexplanation{\textbf{\textit{терминологическая семантическая окрестность}}  –  семантическая окрестность, описывающая идентификацию указанной сущности}

\scnheader{теоретико-множественная семантическая окрестность}
\scnexplanation{\textbf{\textit{теоретико-множественная семантическая окрестность}}  –  описание связи описываемого понятия с другими понятиями с помощью теоретико-множественных отношений}

\scnheader{описание декомпозиции}
\scnidtf{семантическая окрестность, описывающая декомпозицию некоторой сущности}
\scnexplanation{\textbf{\textit{описание декомпозиции}}  –  семантическая окрестность, описывающая декомпозицию некоторой сущности на частные сущности}

\scnheader{логическая семантическая окрестность }
\scnexplanation{\textbf{\textit{логическая семантическая окрестность}}  –  семантическая окрестность, описывающая семейство высказываний, описывающих свойства данного понятия}

\scnheader{описание типичного экземпляра}
\scnidtf{описание типичного экземпляра заданного класса}
\scnidtf{типичная семантическая окрестность}
\scnidtf{типичная sc-окрестность}
\scnexplanation{\textbf{\textit{описание типичного экземпляра}} – это sc-текст являющийся описанием типичного примера использования рассматриваемого класса.}

\scnheader{сравнительный анализ}
\scnexplanation{\textbf{\textit{сравнительный анализ}} –  описание сравнительного анализа некоторой сущности с другими сущностями}

\scnheader{иллюстрация}
\scnsubset{специализированная семантическая окрестность}
\scnexplanation{\textbf{\textit{иллюстрация}} –  семантическая окрестность некоторой сущности (сущностей), иллюстрирующая некоторые свойства указанных сущностей, чаще всего, на некотором конкретном примере.}

\scnendstruct

\end{SCn}

\scsubsection[\scnmonographychapter{Глава 2.3. Структура баз знаний интеллектуальных компьютерных систем нового поколения: иерархическая система предметных областей и онтологий. Онтологии верхнего уровня. Формализация понятий семантической окрестности, предметной области и онтологии в интеллектуальных компьютерных системах нового поколения}]{Предметная область и онтология предметных областей}
\label{sd_sd}
\begin{SCn}
\scnsectionheader{\currentname}
\begin{scnsubstruct}
\begin{scnreltovector}{конкатенация сегментов}
\scnitem{Что такое предметная область}
\scnitem{Роли знаков, входящих в состав предметных областей}
\scnitem{Типология предметных областей и отношения, заданных на множестве предметных областей}
\scnitem{Что такое sc-язык}
\end{scnreltovector}
\scnheader{Предметная область предметных областей}
\scnidtf{Предметная область, объектами исследования которой являются предметные области}
\scntext{explanation}{В состав \textbf{\textit{Предметной области предметных областей}} входят структурные спецификации всех \textit{предметных областей}, входящих в состав базы знаний \textit{ostis-системы}, в том числе, самой \textbf{\textit{Предметной области предметных областей}}. Таким образом, \textbf{\textit{Предметная область предметных областей}} является, во-первых, \textit{рефлексивным множеством}, во-вторых, рефлексивной предметной областью, то есть \textit{предметной областью}, одним из объектов исследования которой является она сама.}\scniselement{рефлексивное множество}
\begin{scnhaselementrole}{класс объектов исследования}
предметная область\end{scnhaselementrole}
\begin{scnhaselementrolelist}{класс объектов исследования}
статическая предметная область;динамическая предметная область;понятие;sc-язык
\end{scnhaselementrolelist}
\begin{scnhaselementrolelist}{исследуемое отношение}
понятие предметной области\scnrolesign ;исследуемое понятие\scnrolesign ;максимальный класс объектов исследования\scnrolesign ;немаксимальный класс объектов исследования\scnrolesign ;исследуемый класс первичных элементов\scnrolesign ;исследуемое отношение\scnrolesign ;класс исследуемых структур\scnrolesign ;понятие, исследуемое в дочерней предметной области\scnrolesign ;понятие, исследуемое в материнской предметной области\scnrolesign ;вспомогательное понятие\scnrolesign ;дочерняя предметная область*;дочерняя предметная область по классу первичных элементов*;дочерняя предметная область по исследуемым отношениям*;предметная область sc-языка*
\end{scnhaselementrolelist}
\end{scnsubstruct}
\scnsegmentheader{Что такое предметная область}
\begin{scnsubstruct}
\scnheader{предметная область}
\scnidtf{sc-модель предметной области}
\scnidtf{sc-текст предметной области}
\scnidtf{sc-граф предметной области}
\scnidtf{представление предметной области в \textit{SC-коде}}
\scnsubset{знание}
\scnsubset{бесконечное множество}
\scntext{explanation}{\textbf{\textit{предметная область}} -- это результат интеграции (объединения) частичных семантических окрестностей, описывающих все исследуемые сущности заданного класса и имеющих одинаковый (общий) предмет исследования (то есть один и тот же набор отношений, которым должны принадлежать связки, входящие в состав интегрируемых семантических окрестностей).\textbf{\textit{предметная область}} -- \textit{структура}, в состав которой входят:\begin{scnitemize}
\item \textnormal{основные исследуемые (описываемые) объекты -- первичные и вторичные;}\item \textnormal{различные классы исследуемых объектов;}\item \textnormal{различные связки, компонентами которых являются исследуемые объекты (как первичные, так и вторичные), а также, возможно, другие такие связки -- то есть связки (как и объекты исследования) могут иметь различный структурный уровень;}\item \textnormal{различные классы указанных выше связок (то есть отношения);}\item \textnormal{различные классы объектов, не являющихся ни объектами исследования, ни указанными выше связками, но являющихся компонентами этих связок.}\end{scnitemize}
При этом все классы, объявленные исследуемыми понятиями, должны быть полностью представлены в рамках данной предметной области вместе со своими элементами, элементами элементов и т.д. вплоть до терминальных элементов.Можно говорить о типологии \textbf{\textit{предметных областей}} по разным структурным признакам:\begin{scnitemize}
\item наличие метасвязей;\item наличие исследуемых структур, входящих в состав предметной области;\item наличие исследуемых (смежных, дополнительных) объектов, которых исследуются в других предметных областях;\end{scnitemize}
Понятие \textbf{\textit{предметной области}} является важнейшим методологическим приемом, позволяющим выделить из всего многообразия исследуемого Мира только определенный класс исследуемых сущностей и только определенное семейство отношений, заданных на указанном классе. То есть осуществляется локализация, фокусирование внимания только на этом, абстрагируясь от всего остального исследуемого Мира.Во всем многообразии \textbf{\textit{предметных областей}} особое место занимают\begin{scnitemize}
\item \textit{Предметная область предметных областей}, объектами исследования которой являются всевозможные \textbf{\textit{предметные области}}, а предметом исследования -- всевозможные \textit{ролевые отношения}, связывающие предметные области с их элементами, отношения, связывающие предметные области между собой, отношение, связывающее предметные области с их онтологиями\item \textit{Предметная область сущностей}, являющаяся предметной областью самого высокого уровня и задающая базовую семантическую типологию \textit{sc-элементов}(знаков, входящих в тексты \textit{SC-кода})\item Семейство \textbf{\textit{предметных областей}}, каждая из которых задает семантику и синтаксис некоторого \textit{sc-языка}, обеспечивающего представление онтологий соответствующего вида (например, \textit{теоретико-множественных онтологий}, \textit{логических онтологий}, \textit{терминологических онтологий}, \textit{онтологий задач и способов их решения} и т.д.)\item Семейство \textbf{\textit{предметных областей}} верхнего уровня, в которых классами объектов исследования являются весьма крупные классы сущностей. К таким классам, в частности\begin{scnitemizeii}
\item класс всевозможных \textit{материальных сущностей},\item класс всевозможных \textit{множеств},\item класс всевозможных \textit{связей},\item класс всевозможных \textit{отношений},\item класс всевозможных \textit{структур},\item класс всевозможных \textit{временных (временно существующих, непостоянных сущностей) сущностей},\item класс всевозможных \textit{действий} (акций),\item класс всевозможных \textit{параметров} (характеристик),\item класс \textit{знаний} всевозможного вида \item и т.п.\end{scnitemizeii}
\end{scnitemize}
Каждой \textbf{\textit{предметной области}} можно поставить в соответствие:\begin{scnitemize}
\item семейство соответствующих ей \textit{онтологий} разного вида;\item некий язык (в нашем случае -- язык, построенный на основе \textit{SC-кода}), тексты которого представляют различные фрагменты соответствующей предметной области\end{scnitemize}
Указанные языки будем называть \textit{sc-языками}. Их синтаксис и семантика полностью задается \textit{SС-кодом} и \textit{онтологией} соответствующей \textbf{\textit{предметной области}}. Очевидно, что в первую очередь нас должны интересовать те \textit{sc-языки}, которые соответствуют \textbf{\textit{предметным областям}}, имеющим общий (условно говоря, предметно независимый) характер. К таким предметным областям, в частности, относятся:\begin{scnitemize}
\item \textit{Предметная область множеств}, описывающая множества и различные связи между ними\item \textit{Предметная область отношений и соответствий}\item \textit{Предметная область структур} (в частности, графовых)\item \textit{Предметная область чисел и числовых структур}\item и т.д\end{scnitemize}
Каждому типу знаний можно поставить в соответствие предметную область, которая является результатом интеграции всех знаний данного типа. Эти знания и становятся объектами исследования в рамках указанной предметной области.Понятие \textbf{\textit{предметной области}} может рассматриваться как обобщение понятия алгебраической системы. При этом семантическая структура базы знаний может рассматриваться как иерархическая система различных \textbf{\textit{предметных областей}}.}\scnidtf{система связей некоторого множества объектов исследования, \uline{ключевыми} элементами которой являются:\begin{scnitemize}
\item классы (точнее, знаки классов) объектов исследования (объектов, описываемых этой предметной областью);\item конкретные объекты исследования, обладающие особыми свойствами;\item классы связей, входящих в состав рассматриваемой системы -- отношения, заданные на множестве элементов рассматриваемой системы;\item параметры, заданные на множестве элементов рассматриваемой системы;\item классы структур, являющихся фрагментами рассматриваемой системы.\end{scnitemize}
}
\scnidtf{структура, представляющая собой множество связей (точнее, знаков связей) и соответствующее множество компонентов этих связей, к числу которых относится:\begin{scnitemize}
\item элементы (экземпляры) некоторых заданных классов \uline{объектов исследования} (первичных исследуемых сущностей);\item сами связи, входящие в состав указанной структуры;\item введенные классы объектов исследования;\item введенные отношения (классы связей);\item введенные параметры (классы классов эквивалентных сущностей);\item значения параметров (и, в частности, величины для измеряемых параметров);\item введенные структуры, являющиеся фрагментами (подструктурами) рассматриваемой структуры;\item введенные классы подструктур рассматтриваемой структуры.\end{scnitemize}
}
\scntext{note}{Выделяемые в рамках \textit{базы знаний} интеллектуальной системы \textit{предметные области} и соответствующие им \textit{онтологии} -- это, своего рода, семантические страты, кластеры, позволяющие разложить все хранимые в памяти \textit{знания} по семантическим полочкам при наличии четких критериев, позволяющих \uline{однозначно} определить то, на какой полочке должны находиться те или иные \textit{знания}}\scntext{note}{Существуют предметные области, в которых основным исследуемым понятием является множество всевозможных связей между экземплярами понятий, исследуемых в других предметных областях. Так, например, можно ввести Предметную область треугольников, Предметную область окружностей, а также Предметную область связей между треугольниками и окружностями.}
\end{scnsubstruct}
\scnsegmentheader{Роли знаков, входящих в состав предметной области}
\begin{scnsubstruct}
\scnheader{роль элемента предметной области}
\scnidtf{ролевое отношения, связывающее предметные области с их ключевыми знаками}
\scnidtf{роль ключевого элемента (знака ключевой сущностей) предметной области}
\scnidtf{роль ключевого знака предметной области}
\scnhaselement{класс объектов исследования\scnrolesign}
\scnhaselement{максимальный класс объектов исследования\scnrolesign}
\scnhaselement{ключевой объект исследования\scnrolesign}
\scnhaselement{понятие, используемое в предметной области\scnrolesign}
\scnhaselement{первичный исследуемый элемент предметной области\scnrolesign}
\scnhaselement{вторичный исследуемый элемент предметной области\scnrolesign}
\scnhaselement{неисследуемый элемент предметной области\scnrolesign}
\scnheader{класс объектов исследования\scnrolesign}
\scnidtf{быть классом \uline{первичных} (для данной предметной области) объектов исследования\scnrolesign}
\scntext{note}{Понятие \uline{первичного} объекта исследования для предметной области является понятием \uline{относительным} и абсолютно не зависит от типа и уровня сложности этого объекта. Само исследование (спецификация) таких первичных исследуемых объектов осуществляется:\begin{scnitemize}
\item путем введения различных классов объектов исследования, которым эти объекты принадлежат;\item путем введения различных связок из первичных объектов исследования и различных классов таких связок (отношений), которым принадлежат введенные связки;\item путем введения таких классов первичных объектов исследования, которые являются значениями вводимых параметров;\item путем введения различных структур, состоящих из первичных объектов исследования, из связок таких объектов, из введенных отношений и классов первичных объектов, из введенных параметров и значений этих параметров, и путем введения различных классов таких структур;\item путем введения различных связок из вторичных объектов исследования (т.е. из связок и структур) и путем введения различных классов таких связок;\item и далее можно переходить к объектам исследования более высокого уровня сложности, к параметрам, элементами значений которых являются такие объекты, а также к структурам, элементами которых являются объекты такого уровня и, соответственно, к классам таких структур.\end{scnitemize}
}\scnrelfrom{второй домен}{класс}
\scnsuperset{\begin{scnset}
множество;отношение\\\scnsubset{множество}
;параметр\\\scnsubset{класс классов}
;значение параметра\\\scnsubset{класс}
;структура\\\scnsubset{множество}
;темпоральная сущность;темпоральная сущность базы знаний ostis-системы;семантическая окрестность;предметная область;онтология;логическая формула;действие;задача;информационная конструкция;язык;sc-конструкция;кибернетическая система;интеллектуальная компьютерная система;знание;база знаний;решатель задач интеллектуальной компьютерной системы;интерфейс интеллектуальной компьютерной системы;компьютерная система, основанная на смысловом представлении информации;смысловое представление информации;многоагентная модель решения задач, основанная на смысловом представлении информации;логико-семантическая модель интерфейсов компьютерных систем, основанных на смысловом представлении информации;решатель задач ostis-системы;действие, выполняемое ostis-системой;задача, решаемая ostis-системой:план решения задачи, реализуемый ostis-системой;протокол решения задачи, реализованный ostis-системой;метод решения класса задач, реализуемый ostis-системой;sc-агент\\\scnidtf{внутренний агент ostis-системы, осуществляющий выполнение некоторого вида действий в памяти ostis-системы}
\scnsuperset{sc-агент обработки информации в памяти ostis-системы}
\scnsuperset{sc-агент управления внешними действиями ostis-системы}
;Базовый язык программирования ostis-систем\\\scnidtf{Язык SCP}
;искусственная нейронная сеть;интерфейс ostis-системы;интерфейсное действие пользователя ostis-системы;sc-агент интерфейса ostis-системы;естественный язык;базовый интерпретатор логико-семантических моделей ostis-систем;базовый интерпретатор логико-семантических моделей ostis-систем, реализованный программно на современных компьютерах;семантический ассоциативный компьютер;обучение пользователей ostis-систем;ostis-система персональной адаптивной поддержки всех видов деятельности пользователя;ostis-система управления рецептурным производством;ostis-система, реализующая интеллектуальный портал научно-технических знаний
\end{scnset}
}
\scntext{note}{Здесь приведено семейство тех \textit{классов объектов исследования}, для которых в текущей версии \textit{Стандарта OSTIS} представлены соответствующие \textit{предметные области}. Очевидно, что это семейство должно быть существенно расширено и включить в себя, например, такие \textit{классы} сущностей, как:\begin{scnitemize}
\item материальная сущность\item вещество\item физическое поле\item персона\item пространственная сущность\item юридическое лицо\item предприятие\item географический объект\item и многие другие\end{scnitemize}
}\scntext{note}{Особого внимания требуют те \textit{классы объектов исследования}, которые носят наиболее общий характер  которым соответствуют \textit{предметные области и онтологии} \uline{высокого уровня}. Здесь важна продуманная система декомпозиции всего множества окружающих нас сущностей на иерархическую систему \textit{классов объектов исследования}, которой соответствует иерархическая система \textit{предметных областей и онтологий}, определяющая направления \uline{наследования свойств} исследуемых объектов.}\scnheader{максимальный класс объектов исследования\scnrolesign}
\scnidtf{класс объектов исследования, для которого \uline{в заданной} (!) предметной области отсутствует другой класс объектов исследования, который был бы его надмножеством\scnrolesign}
\scntext{note}{В некоторых предметных областях может быть \uline{несколько} максимальных классов объектов исследования}\scnheader{ключевой объект исследования\scnrolesign}
\scnidtf{особый объект исследования\scnrolesign}
\scnidtf{быть знаком особого исследуемого объекта в рамках заданной предметной области\scnrolesign}
\scnidtf{объект исследования, обладающий особыми свойствами\scnrolesign}
\scnhaselementrole{пример}{$\langle$Предметная область чисел; Нуль$\rangle$}
\scntext{note}{Особыми свойствами Числа \textit{Нуль} являются:\begin{scnitemize}
\item Результатом сложения Числа \textbf{\textit{Нуль}} с любым числом \textbf{\textit{x}} является число \textbf{\textit{x}};\item Результатом умножения Числа \textbf{\textit{Нуль}} на любое число является Число \textbf{\textit{Нуль}}\end{scnitemize}
}\scnhaselement{$\langle$Предметная область чисел; Единица$\rangle$}
\scnhaselement{$\langle$Предметная область чисел; Число Пи$\rangle$}
\scnhaselement{$\langle$Предметная область чисел; Число Е$\rangle$}
\scnheader{ключевой элемент предметной области\scnrolesign}
\scnidtf{входящий в состав предметной области знак ключевой сущности\scnrolesign}
\begin{scnsubdividing}
\scnitem{понятие, используемое в предметной области\scnrolesign}
\scnitem{ключевой объект исследования\scnrolesign \\\scnidtf{знак ключевого объекта исследования\scnrolesign}
}
\end{scnsubdividing}
\scnheader{понятие, используемое в предметной области\scnrolesign}
\scnidtf{понятие, используемое в заданной предметной области не в качестве одного из объектов исследования, а в качестве \uline{ключевого} понятия\scnrolesign}
\scnsubset{используемое понятие\scnrolesign}
\scnidtf{понятие, используемое в sc-знании\scnrolesign}
\scnsubset{используемое понятие*}
\scnidtf{понятие, используемое в знании, которое может быть представлено не только в SC-коде*}
\scntext{note}{Уточнение характера использования понятия в предментной области осуществляется по трем признакам:\begin{scnitemize}
\item семантический тип используемого понятия;\item полнота вхождения элементов понятия в данную предметную область;\item наличие первого упоминания понятия;\item наличие определения понятия или объявления его неопределяемостис подробным пояснением и примерами;\item наличие исследования понятия.\end{scnitemize}
}\scnrelfrom{разбиение}{семантический тип используемого понятия}
\begin{scneqtoset}
\scnitem{класс объектов исследования\scnrolesign}
\scnitem{отношение, используемое в предметной области\scnrolesign}
\scnitem{параметр, используемый в предметной области\scnrolesign}
\scnitem{класс структур, используемый в предметной области\scnrolesign}
\end{scneqtoset}
\scnrelfrom{разбиение}{полнота вхождения элементов понятия в данную предметную область}
\begin{scneqtoset}
\scnitem{используемое понятие, все элементы которого входят в данную предметную область\scnrolesign \\\scntext{note}{Для каждого используемого отношения в предметную область здесь должны входить не только знаки связок, но и все связки целиком с их компонентами}}
\scnitem{используемое понятие, не все элементы которого входят в данную предметную область\scnrolesign}
\end{scneqtoset}
\scnrelfrom{разбиение}{наличие первого упоминания понятия}
\begin{scneqtoset}
\scnitem{понятие, вводимое в данной предметной области\scnrolesign}
\scnitem{понятие, которое в данной предметной области используется, но не вводится\scnrolesign}
\end{scneqtoset}
\scntext{note}{Будем считать, что понятие вводится в данной предметной области в том и только в том случае, если ни в одной предметной области более высокого уровня это понятие не используется. Т.е. речь идет о первом упоминании этого понятия в рамках последовательности предметных областей от родительских к дочерним}\scnrelfrom{разбиение}{наличие определения понятия или объявления его неопределяемости с подробным пояснением и примерами}
\begin{scneqtoset}
\scnitem{понятие, которое в данной предметной области определено или объявлено как неопределяемое}
\scnitem{понятие, которое в данной предметной области не имеет ни определения, ни указания факта его неопределяемости}
\end{scneqtoset}
\scnrelfrom{разбиение}{наличие исследования понятия}
\begin{scneqtoset}
\scnitem{понятие, исследуемое в данной предметной области\scnrolesign}
\scnitem{понятие, которое в данной предметной области испольуется, но не исследуется\scnrolesign}
\end{scneqtoset}
\scntext{note}{Понятие, используемое в базе знаний, может быть введено (впервые упомянуто) в одной предметной области, определено в другой, а исследоваться -- в третьей}\scnheader{первичный исследуемый элемент предметной области\scnrolesign}
\scnidtf{знак первичного объекта исследования в рамках заданной предметной области\scnrolesign}
\scnheader{вторичный исследуемый элемент предметной области\scnrolesign}
\scnidtf{знак вторичного объекта исследования в рамках предметной области\scnrolesign}
\scnsuperset{связка элементов предметной области\scnrolesign}
\scnsuperset{связка первичных элементов предметной области\scnrolesign}
\scnsuperset{метасвязка элементов предметной области\scnrolesign}
\scnsuperset{метасвязка, в число компонентов которой входят связки элементов предметной области\scnrolesign}
\scnsuperset{метасвязка, в число компонентов которой входят классы элементов предметной области\scnrolesign}
\scnsuperset{метасвязка, в число компонентов которой входят структуры элементов предметной области\scnrolesign}
\scnsuperset{класс элементов предметной области\scnrolesign}
\scnsuperset{класс первичных элементов предметной области\scnrolesign}
\scnsuperset{класс связок элементов предметной области\scnrolesign}
\scnsuperset{класс классов элементов предметной области\scnrolesign}
\scnsuperset{класс структур элементов предметной области\scnrolesign}
\scnsuperset{структура элементов предметной области\scnrolesign}
\scnsuperset{тривиальная структура первичных элементов предметной области\scnrolesign}
\scnsuperset{структура, в число подмножеств которой входят связки элементов предметной области вместе со своими компонентами\scnrolesign}
\scnsuperset{структура, в число подмножеств которой входят классы элементов предметной области вместе со своими знаками\scnrolesign}
\scnsuperset{структура, в число подмножеств которой входят другие структуры вместе со своими знаками\scnrolesign}
\scnheader{неисследуемый элемент предметной области\scnrolesign}
\scnidtf{вспомогательный элемент предметной области, исследуемый в другой (смежной) предметной области\scnrolesign}
\scntext{note}{С помощью неисследуемых элементов предметной области описываются и исследуются различные вида связи между элементами, исследуемыми в данной \textit{предметной области} с элементами, исследуемыми в других \textit{предметных областях}. При этом \textit{связки}, компонентами которых являются как исследуемые, так и неисследуемые элементы данной \textit{предметной области} считаются \uline{исследуемыми} связками этой \textit{предметной области}. Примерами неисследуемых элементов, напримр, геометрической \textit{предметной области} являются \textit{числа}, являющиеся \textit{значениями величин} таких \textit{параметров}, как \textit{расстояние}\scnsupergroupsign, \textit{длина}\scnsupergroupsign, \textit{площадь}\scnsupergroupsign, \textit{объем}\scnsupergroupsign, а также различные числовые \textit{отношения} (\textit{сложение}*, \textit{умножение}*, \textit{возведение в степень}*), теоретико-множественные \textit{отношения} (\textit{включение}*, \textit{объединение}*, \textit{пересечение}*, \textit{принадлежность}*)}\newpage\scnheader{понятие}
\scnidtf{концепт}
\scnidtf{класс сущностей, который входит в состав по крайней мере одной предметной области в качестве (в роли) ключевого исследуемого понятия}
\scntext{note}{Семейство всех введенных понятий -- это, своего рода, семантическая система координат, позволяющая специфицировать всевозможные сущности в смысловом пространстве.}\scnidtf{класс сущностей, который по крайней мере в одной \textit{предметной области} объявлен как \textit{понятие} (вводимое, исследуемое или вспомогательное)}
\scntext{note}{Каждому \textit{понятию} соответствует по крайней мере одна \textit{предметная область}, в которой это понятие является \textit{исследуемым понятием} и в которой рассматриваются основные характеристики этого \textit{понятия}. Если же в какой-либо \textit{предметной области} необходимо рассмотреть дополнительные связи этого \textit{понятия} с другими \textit{понятиями}, то оно объявляется как \textit{вспомогательное понятия}\scnrolesign .}\scnidtf{Второй домен Отношения \textit{используемое понятие}*}
\scnrelto{второй домен}{используемое понятие*}
\scnidtf{класс сущностей (класс связок (в т.ч. отношение), класс классов (в т.ч. параметр), класс структур), который по крайней мере в одной \textit{предметной области} является \textit{используемым понятием}\scnrolesign}
\end{scnsubstruct}
\scnsegmentheader{Типология предметных областей и отношения, заданные на множестве предметных областей}
\begin{scnsubstruct}
\scnheader{предметная область}
\begin{scnsubdividing}
\scnitem{статическая предметная область\\\scnidtf{стационарная предметная область}
\scnidtf{\textit{предметная область}, в которой связи между сущностями, входящими в ее состав, не зависят от времени (не меняются во времени), элементами \textbf{\textit{статической предметной области}} не могут быть \textit{временные сущности}}
}
\scnitem{квазистатическая предметная область\\\scnidtf{\textit{предметная область}, решение задач в которой не требует учета темпоральных свойств объектов исследования}
}
\scnitem{динамическая предметная область\\\scnidtf{нестационарная предметная область}
\scnidtf{\textit{предметная область}, которая описывает изменение состояния (в том числе внутренней структуры) объектов исследования и/или изменение конфигурации связей между объектами исследования}
\scnidtf{\textit{предметная область}, в которой некоторые связи между сущностями, входящими в ее состав, меняются со временем (то есть носят ситуационный, нестационарный характер, другими словами, являются \textit{временными сущностями})}
}
\end{scnsubdividing}
\begin{scnsubdividing}
\scnitem{первичная предметная область\\\scnidtf{\textit{предметная область}, объектами исследования которой являются \uline{внешние} сущности (обозначаемые первичными \textit{sc-элементами})}
}
\scnitem{вторичная предметная область\\\scnidtf{метапредметная область}
\scnidtf{\textit{предметная область}, объектами исследования которой являются \textit{sc-множества} (отношения, параметры, структуры, классы структур, знания, языки и др.)}
}
\end{scnsubdividing}
\scntext{note}{Во всем многообразии предметных областей \uline{особое} местро занимают:\begin{scnitemize}
\item \textbf{\textit{Предметная область предметных областей}}, объектами исследования которой являются всевозможные предметные области, а предметом исследования являются -- всевозможные ролевые отношения, связывающие предметные области с их элементами, отношения, связывающие предметные области между собой, отношение, связывающее предметные области с их онтологиями;\item \textbf{\textit{Предметная область сущностей}}, являющаяся предметной областью самого высокого уровня и задающая базовую семантическую типологию sc-элементов (знаков, входящих в тексты SC-кода);\item Семейство \textit{предметных областей}, каждая из которых задает семантику и синтаксис некоторого \textit{sc-языка}, обеспечивающего представление \textit{\uline{онтологий}} соответствующего вида (например, теоретико множественных онтологий терминологических онтологий);\item Семейство \textit{предметных областей} \uline{верхнего уровня}, в которых классами объектов исследования являются весьма крупные классы сущностей. К таким классам, в частности, относятся: \begin{scnitemizeii}
\item класс всевозможных материальных сущностей,\item класс всевозможных множеств,\item класс всевозможных связей,\item класс всевозможных отношений,\item класс всевозможных структур,\item класс всевозможных темпоральных (нестационарных) сущностей,\item класс всевозможных действий (воздествий, акций),\item класс всевозможных параметров (характеристик),\item класс знаний всевозможного вида и т.п.;\end{scnitemizeii}
\item Предметные области абстрактных пространств (в том числе предметные области метрических пространств). Примерами абстрактного пространства являются Евклидово пространство геометрических точек и фигур, пространство всевозможных множеств, числовое пространство, SC-пространство (унифицированное смысловое пространство знаков всевозможных сущностей).\end{scnitemize}
}\scnheader{отношение, заданное на множестве предметных областей}
\scnhaselement{\scnkeyword{дочерняя предметная область*}
}
\scnidtf{частная предметная область*}
\scnidtf{быть частной предметной областью*}
\scnidtf{близлежащий потомок предметной области*}
\scnidtf{сужение предметной области по классу объектов исследования*}
\scnidtf{предметная область, детализирующая описание одного из классов объектов исследования другой (более общей) предметной области*}
\scnidtf{предметная область, объединение классов объектов исследования которой является подмножеством объединения классов объектов исследования заданной предметной области*}
\scniselement{бинарное отношение}
\scniselement{ориентированное отношение}
\scniselement{неролевое отношение}
\scnsuperset{частная предметная область по классу первичных элементов*}
\scnsuperset{частная предметная область по исследуемым отношениям*}
\scntext{explanation}{\textit{дочерняя предметная область*} -- бинарное ориентированное отношение, с помощью которого задается иерархия предметных областей путем перехода от менее детального к более детальному рассмотрению соответствующих исследуемых понятий.}\scntext{note}{Для любой \textit{предметной области} все свойства ее \textit{объектов исследования} \uline{наследуются} всеми ее \textit{дочерними предметными областями*}.}\scnhaselement{\scnkeyword{интеграция предметных областей*}
}
\scnidtf{Отношение, связывающее заданное семейство предметных областей с предметной областью, которая является результатом их интеграции (это не только теоретико-множественное объединение заданных предметных областей, но и уточнение ролей ключевых понятий в интегрированной предметной области, поскольку одно и то же понятие в интегрируемых предметных областях может иметь разные роли).}
\scnhaselement{\scnkeyword{изоморфность предметных областей*}
}
\scnhaselement{\scnkeyword{гомоморфность предметных областей*}
}
\scnheader{расширение семейства исследуемых отношений*}
\scntext{explanation}{Переход от одной предметной области к предметной области с тем же максимальным классомобъектов исследования, но с расширенным семейством отношений и, возможно, с расширенным семейством явно выделенных классов объектов исследования (подклассов максимального класса).}\scnheader{переход к рассмотрению внутренней структуры объектов исследования*}
\scntext{explanation}{Переход от рассмотрения внешних связей объектов исследования к рассмотрению их внутренней структуры путем декомпозиции исследуемых объектов на части и путем включения в число исследуемых объектов тех, которые являются указанными частями.}\scnheader{переход к рассмотрению структур из объектов исследования*}
\scntext{explanation}{Переход от описаниязаданного класса исследуемых объектов к описанию класса всевозможных множеств, элементами которых являются указанные объекты (например, переход от предметной области геометрических точек к предметной области геометрических фигур).}
\end{scnsubstruct}
\end{SCn}


\scsubsection[\scnmonographychapter{Глава 2.3. Структура баз знаний интеллектуальных компьютерных систем нового поколения: иерархическая система предметных областей и онтологий. Онтологии верхнего уровня. Формализация понятий семантической окрестности, предметной области и онтологии в интеллектуальных компьютерных системах нового поколения}]{Предметная область и онтология онтологий}
\label{sd_ontologies}
\begin{SCn}

\scnsectionheader{\currentname}

\scnstartsubstruct

\scnheader{Предметная область онтологий}
\scnidtf{Предметная область теории онтологий}
\scnidtf{Предметная область, объектами исследования которой являются онтологии}
\scniselement{предметная область}
\scnsdmainclasssingle{онтология}
\scnsdclass{интегрированная онтология;структурная спецификация;теоретико-множественная онтология;логическая онтология;логическая иерархия понятий;логическая иерархия высказываний;терминологическая онтология;онтология задач и решений задач;онтология классов задач и способов решения задач}
\scnsdrelation{онтология*;используемые константы*;используемые утверждения*}

\scnheader{онтология}
\scnidtf{система понятий соответствующей предметной области}
\scnidtf{концептуальный каркас (скелет) описания некоторой предметной области}
\scnidtf{концептуальная (семантическая) основа различных языков, обеспечивающих описание объектов исследования, принадлежащих заданной предметной области}
\scnidtf{семантический интерфейс для интеграции знаний по заданной предметной области и для согласованного понимания различными субъектами этих знаний}
\scnidtf{онтология соответствующей предметной области}
\scnidtf{описание концептов и отношений заданной предметной области}
\scnrelto{включение}{знание}
\scnsubdividing{интегрированная онтология;структурная спецификация;теоретико-множественная онтология;логическая иерархия понятий;логическая онтология;логическая иерархия высказываний;терминологическая онтология;онтология задач и решений задач;онтология классов задач и способов решения задач}
\scnexplanation{\textbf{\textit{онтология}} — это вид знаний, каждое из которых является спецификацией (описанием свойств) соответствующей \textit{предметной области}, ориентированной на описание свойств и взаимосвязей понятий, входящих в состав указанной \textit{предметной области}.}

\scnheader{онтология*}
\scnidtf{sc-онтология*}
\scnidtf{быть онтологией предметной области*}
\scnidtf{sc-онтология, специфицирующая заданную предметную область*}
\scnrelfrom{первый домен}{предметная область}
\scnrelfrom{второй домен}{онтология}
\scnexplanation{\textbf{\textit{онтология*}} — это бинарное отношение, связывающее некоторую предметную область с ее онтологией (спецификацией).}

\scnheader{интегрированная онтология}
\scnexplanation{\textbf{\textit{интегрированная онтология}} — это \textit{онтология}, объединяющая все \textit{онтологии} различного вида некоторой \textit{предметной области}.}

\scnheader{структурная спецификация}
\scnexplanation{\textbf{\textit{структурная спецификация}} — это \textit{онтология}, в которой описываются роли понятий, входящих в состав \textit{предметной области}, а также связи специфицируемых \textit{предметных областей} с другими \textit{предметными областями}.}

\scnheader{теоретико-множественная онтология}
\scnexplanation{\textbf{\textit{теоретико-множественная онтология}} — это \textit{онтология}, описывающая теоретико-множественные связи между понятиями заданной \textit{предметной области} (включение, разбиение, объединение, пересечение, разность множеств, область определения, домен, функция).}

\scnheader{логическая онтология}
\scnexplanation{\textbf{\textit{логическая онтология}} — это \textit{онтология}, описание системы высказываний заданной \textit{предметной области}.}

\scnheader{логическая иерархия понятий}
\scnidtf{логическая иерархия понятий, основанная на их определениях}
\scnexplanation{\textbf{\textit{логическая иерархия понятий}} — это \textit{онтология}, являющаяся надстройкой над \textit{логической онтологией}, включающая описание системы определений понятий заданной \textit{предметной области} с указанием набора понятий, через которые определяется каждое определяемое понятие рассматриваемой \textit{предметной области}.}

\scnheader{используемые константы*}
\scniselement{квазибинарное отношение}
\scnrelfrom{второй домен}{понятие}
\scnexplanation{\textbf{\textit{используемые константы*}} — это \textit{отношение}, связывающее некоторое \textit{определение} со множеством понятий, на основании которых определяется соответствующее данному \textit{определению} понятие в рамках рассматриваемой \textit{предметной области}.}

\scnheader{логическая иерархия высказываний}
\scnidtf{логическая система доказательств}
\scnidtf{логическая иерархия утверждений}
\scnidtf{логическая иерархия высказываний, основанная их на базовых доказательствах}
\scnexplanation{\textbf{\textit{логическая иерархия высказываний}} — это \textit{онтология}, являющаяся надстройкой над \textit{логической онтологией} и включающая описание системы утверждений рассматриваемой \textit{предметной области} с указанием набора \textit{утверждений}, через которые доказывается каждое \textit{утверждение}.}

\scnheader{используемые утверждения*}
\scniselement{квазибинарное отношение}
\scnrelfrom{второй домен}{утверждение}
\scnexplanation{\textbf{\textit{используемые утверждения*}} — это \textit{отношение}, связывающее утверждение со множеством утверждений, на основании которых оно доказывается в рамках рассматриваемой \textit{предметной области}.}

\scnheader{терминологическая онтология}
\scnexplanation{\textbf{\textit{терминологическая онтология}} — это \textit{онтология}, описывающая систему основных и неосновных терминов (имен, внешних обозначений), соответствующих концептам и отношениям заданной \textit{предметной области}, а также описание правил построения терминов для сущностей, являющихся элементами (экземплярами) указанных концептов и \textit{отношений}.}

\scnheader{онтология задач и решений задач}
\scnexplanation{\textbf{\textit{онтология задач и решений задач}} — это \textit{онтология}, описывающая задачи и их классы, решаемые в рассматриваемой \textit{предметной области}.}

\scnheader{онтология классов задач и способов решения задач}
\scnexplanation{\textbf{\textit{онтология классов задач и способов решения задач}} — это \textit{онтология}, описывающая способы решения задач и их классов в рамках \textit{предметной области}. Является \textit{метазнанием*} по отношению к \textit{онтологии задач и классов задач}.}

\scnendstruct \scnendcurrentsectioncomment

\end{SCn}

\scsubsection[\scneditors{Василевская А.П.;Зотов Н.В.;Орлов М.К.}\protect\scnmonographychapter{Глава 2.5. Смысловое представление логических формул и высказываний в различного вида логиках}]{Предметная область и онтология логических формул, высказываний и логических sc-языков}
\label{sd_logics}
\begin{SCn}

\scnsectionheader{\currentname}

\scnstartsubstruct

\scnheader{Предметная область логических формул, высказываний и формальных теорий}
\scniselement{предметная область}
\scnsdmainclasssingle{формальная теория}
\scnsdclass{высказывание;атомарное высказывание;неатомарное высказывание;фактографическое высказывание;логическая формула;атомарная логическая формула;неатомарная логическая формула;утверждение;определение;общезначимая логическая формула;противоречивая логическая формула;нейтральная логическая формула;выполнимая логическая формула;невыполнимая логическая формула;тавтология;квантор;формула существования;число значений переменной;кратность существования;единственное существование;логическая формула и единственность;открытая логическая формула;замкнутая логическая формула}
\scnsdrelation{предметная область\scnrolesign;аксиома\scnrolesign;теорема\scnrolesign;подформула*;логическая связка*;импликация*;если\scnrolesign;то\scnrolesign;эквиваленция*;конъюнкция*;дизъюнкция*;строгая дизъюнкция*;отрицание*;всеобщность*;неатомарное существование*;связываемые переменные\scnrolesign}

\scnheader{формальная теория}
\scnexplanation{\textbf{\textit{формальная теория}} — это множество высказываний, которые считаются истинными в рамках данной \textbf{\textit{формальной теории}}. Высказывания могут быть как фактографическими, так и логическими формулами. Некоторые высказывания считаются аксиомами, а другие доказываются на основе других высказываний в рамках этой же \textbf{\textit{формальной теории}}.

Каждая формальная теория интерпретируется (т.е. ее высказывания являются истинными) на какой-либо \textit{предметной области}, которая является максимальным из \textit{фактографических высказываний} (их \textit{объединением*}),  входящих в состав этой \textbf{\textit{формальной теории}}. Каждой \textbf{\textit{формальной теории}} соответствует одна \textit{предметная область}, которая входит в нее под атрибутом \textit{предметная область\scnrolesign}.

Каждая \textbf{\textit{формальная теория}} может рассматриваться как \textit{конъюнктивное высказывание}, априори истинное (с чьей-то точки зрения) при интерпретации на соответствующей \textit{предметной области}.}

\scnheader{предметная область\scnrolesign}
\scniselement{ролевое отношение}
\scnexplanation{\textbf{\textit{предметная область\scnrolesign}} -- это \textit{ролевое отношение}, связывающее \textit{формальную теорию} с \textit{предметной областью}, на которой данная \textit{формальная теория} интерпретируется (в рамках которой истинны \textit{высказывания}, входящие в состав этой \textit{формальной теории}). Другими словами, эта \textit{предметная область} является максимальным фактографическим высказыванием этой \textit{формальной теории}.}

\scnheader{аксиома\scnrolesign}
\scniselement{ролевое отношение}
\scnexplanation{\textbf{\textit{аксиома\scnrolesign}} -- это \textit{ролевое отношение}, связывающее \textit{формальную теорию} с \textit{высказыванием}, истинность которого не  требует доказательства в рамках этой \textit{формальной теории}.}

\scnheader{теорема\scnrolesign}
\scniselement{ролевое отношение}
\scnexplanation{\textbf{\textit{теорема\scnrolesign}} -- это \textit{ролевое отношение}, связывающее \textit{формальную теорию} с \textit{высказыванием}, истинность которого доказывается в рамках этой \textit{формальной теории}.}

\scnheader{высказывание}
\scnsubdividing{атомарное высказывание;неатомарное высказывание}
\scnsubdividing{фактографическое высказывание;логическая формула}
\scnexplanation{Под \textbf{\textit{высказыванием}} понимается некоторая \textit{структура} (в которую входят \textit{sc-константы} из некоторой предметной области и/или \textit{sc-переменные}) или \textit{логическая связка}, которая может трактоваться как истинная или ложная в рамках какой-либо \textit{предметной области}.}
\scnnote{Истинность \textbf{\textit{высказывания}} задается путем указания принадлежности знака этого высказывания \textit{формальной теории}, соответствующей данной \textit{предметной области}. Ложность высказывания задается путем указания принадлежности знака \textit{отрицания*} этого высказывания данной \textit{формальной теории}. Явно указанная непринадлежность \textbf{\textit{высказывания}} \textit{формальной теории} может говорить как о его ложности в рамках данной теории (если это указано рассмотренным выше образом), так и о том, что данное \textbf{\textit{высказывание}} вообще не рассматривается в данной \textit{формальной теории} (например, использует понятия, не принадлежащие данной \textit{предметной области}). 
Одно и то же \textbf{\textit{высказывание}} может быть истинно в рамках одной \textit{формальной теории} и ложно в рамках другой.}
\scnnote{Каждое высказывание может либо содержать только \textit{sc-элементы}, которые не являются знаками других \textbf{\textit{высказываний}} (быть атомарным), либо содержать знаки других \textbf{\textit{высказываний}} (быть неатомарным).}

\scnheader{высказывание формальной теории\scnrolesign}
\scniselement{неосновное понятие}
\scnsubdividing{истинное высказывание\scnrolesign\\
	\scnaddlevel{1}
		\scnidtf{высказывание, истинное в рамках данной формальной теории\scnrolesign}
		\scnidtf{высказывание, знак которого принадлежит данной формальной теории\scnrolesign}
	\scnaddlevel{-1}
	;ложное высказывание\scnrolesign\\
	\scnaddlevel{1}
		\scnidtf{высказывание, ложное в рамках данной формальной теории\scnrolesign}
		\scnidtf{высказывание, знак отрицания которого принадлежит данной формальной теории\scnrolesign}
	\scnaddlevel{-1}
	;нечеткое высказывание\scnrolesign\\
	\scnaddlevel{1}
		\scnidtf{гипотетическое высказывание\scnrolesign}
		\scnidtf{высказывание, возможно истинное или ложное в рамках данной формальной теории\scnrolesign}
		\scnidtf{высказывание, истинное или ложное в рамках данной формальной теории с некоторой вероятностью\scnrolesign}
	\scnaddlevel{-1}
	;бессмысленное высказывание\scnrolesign\\
	\scnaddlevel{1}
		\scnidtf{высказывание, бессмысленное в рамках данной формальной теории\scnrolesign}
		\scnidtf{высказывание, не рассматриваемое в рамках данной формальной теории\scnrolesign}
		\scnexplanation{Высказывание является бессмысленным в рамках заданной формальной теории, если в какое-либо \textit{атомарное высказывание} в его составе (или в само это высказывание, если оно является атомарным) входит какая-либо \textit{sc-константа}, не являющаяся элементом предметной области, описываемой указанной \textit{формальной теорией}.}
	\scnaddlevel{-1}}

\scnheader{атомарное высказывание}
\scnsubset{структура}
\scnsubdividing{атомарное фактографическое высказывание;атомарная логическая формула}
\scndefinition{\textbf{\textit{атомарное высказывание}} -- это \textit{высказывание}, которое содержит хотя бы один \textit{sc-элемент}, не являющийся знаком другого \textit{высказывания}.}
\scnheader{неатомарное высказывание}
\scndefinition{\textbf{\textit{неатомарное высказывание}} -- это \textit{высказывание}, в состав которого входят только знаки других \textit{высказываний}.}
\scnnote{Следует отметить, что мы не можем говорить об истинности либо ложности \textbf{\textit{неатомарного высказывания}} в рамках какой-либо \textit{формальной теории}, в случае, когда невозможно установить истинность либо ложность любого из его элементов в рамках этой же \textit{формальной теории}.}

\scnheader{фактографическое высказывание}
\scnsuperset{атомарное фактографическое высказывание}
\scnexplanation{Под \textit{фактографическим высказыванием} понимается:
\begin{scnitemize}
    \item \textit{атомарное высказывание}, в состав которого не входит ни одна \textit{sc-переменная};
    \item \textit{неатомарное высказывание}, все элементы которого также являются \textbf{\textit{фактографическими высказываниями}}.
\end{scnitemize}
}

\scnheader{логическая формула}
\scnexplanation{Под \textit{логической формулой} понимается:
\begin{scnitemize}
    \item \textit{атомарное высказывание}, в состав которого входит хотя бы одна \textit{sc-переменная};
    \item \textit{неатомарное высказывание}, хотя бы один элемент которого является \textbf{\textit{логической формулой}}.
\end{scnitemize}}
\scnsubdividing{атомарная логическая формула;неатомарная логическая формула}
\scnsubdividing{открытая логическая формула;замкнутая логическая формула}

\scnheader{атомарная логическая формула}
\scnidtf{обобщенная структура}
\scnidtf{атомарная формула существования}
\scnexplanation{Под \textbf{\textit{атомарной логической формулой}} понимается \textit{атомарное высказывание}, которое является \textit{логической формулой}.

По умолчанию \textbf{\textit{атомарная логическая формула}} трактуется как \textit{высказывание} о существовании, то есть наличия в памяти значений, соответствующих всем \textit{sc-переменным}, входящим в состав данной формулы и не попадающих под действие какого-либо другого \textit{квантора} (указанного явно или по умолчанию). Таким образом, на все \textit{sc-переменные}, входящие в состав \textbf{\textit{атомарной логической формулы}} и не попадающие под действие какого-либо другого \textit{квантора}, неявно накладывается квантор \textit{существования*}.}

\scnheader{неатомарная логическая формула}
\scnsubdividing{общезначимая логическая формула;противоречивая логическая формула;нейтральная логическая формула}
\scnsubdividing{выполнимая логическая формула;невыполнимая логическая формула}
\scnsuperset{тавтология}
\scnexplanation{Под \textbf{\textit{неатомарной логической формулой}} понимается \textit{неатомарное высказывание}, которое является \textit{логической формулой}.

Для того, чтобы рассмотреть типологию \textbf{\textit{неатомарных логических формул}}, будем говорить, что исследуется истинность самой \textbf{\textit{неатомарной логической формулы}} и всех ее \textit{подформул*} в рамках одной и той же \textit{формальной теории}, при этом не важно, какой именно. Также считается, что в рассматриваемой \textit{формальной теории} каждая \textit{подформула*} рассматриваемой \textbf{\textit{неатомарной логической формулы}} в рамках этой \textit{формальной теории} может однозначно трактоваться как либо истинная, либо ложная. В противном случае мы не можем говорить об истинности либо ложности исходной \textbf{\textit{неатомарной логической формулы}} в рамках этой \textit{формальной теории}.}

\scnheader{подформула*}
\scnidtf{частная формула*}
\scniselement{бинарное отношение}
\scniselement{ориентированное отношение}
\scniselement{транзитивное отношение}
\scndefinition{Будем называть \textbf{\textit{подформулой*}} \textit{неатомарной логической формулы} \textbf{\textit{fi}} любую \textit{логическую формулу} \textbf{\textit{fj}}, являющуюся элементом исходной формулы \textbf{\textit{fi}}, а также любую \textbf{\textit{подформулу*}} формулы \textbf{\textit{fj}}.}
\scnrelfrom{описание примера}{
\scnfilescg{figures/sd_logical_formulas/subformula.png}}

\scnheader{утверждение}
\scnidtf{текст логической формулы}
\scndefinition{\textbf{\textit{утверждение}} -- это \textit{семантическая окрестность} некоторой \textit{логической формулы}, в которую входит полный текст этой \textit{логической формулы}, а также факт принадлежности этой \textit{логической формулы} некоторой \textit{формальной теории}.}
\scnexplanation{Знак \textit{логической формулы}, семантическая окрестность которой представляет собой утверждение, является \textit{главным ключевым sc-элементом\scnrolesign} в рамках этого \textbf{\textit{утверждения}}. Знаки понятий соответствующей \textit{предметной области}, которые входят в состав какой-либо \textit{подформулы*} указанной \textit{логической формулы}, будут \textit{ключевыми sc-элементами\scnrolesign} в рамках этого \textbf{\textit{утверждения}}.

Полный текст некоторой \textit{логической формулы} включает в себя:
\begin{scnitemize}
    \item знак самой этой \textit{логической формулы};
    \item знаки всех ее \textit{подформул*};
    \item элементы всех \textit{логических формул}, знаки которых попали в данную структуру;
    \item все пары принадлежности, связывающие \textit{логические формулы}, знаки которых попали в данную структуру, с их компонентами.
\end{scnitemize}
Таким образом, факт принадлежности (истинности) логической формулы нескольким \textit{формальным теориям} будет порождать новое утверждение для каждой такой \textit{формальной теории}. Текст \textbf{\textit{утверждения}} входит в состав \textit{логической онтологии}, соответствующей \textit{предметной области}, на которой интерпретируется \textit{главный ключевой sc-элемент\scnrolesign} данного утверждения.}
\scntext{правило идентификации экземпляров}{\textbf{\textit{утверждения}} в рамках \textit{Русского языка} именуются по следующим правилам:
\begin{scnitemize}
    \item в начале идентификатора пишется сокращение \textbf{Утв.};
    \item далее в круглых скобках через точку с запятой перечисляются основные идентификаторы \textit{ключевых \mbox{sc-элементов}\scnrolesign} данного \textbf{\textit{утверждения}}. Порядок определяется в каждом конкретном случае в зависимости от того, свойства каких из этих \textit{понятий} описывает данное \textbf{\textit{утверждение}} в большей или меньшей степени.
\end{scnitemize}
}
\scnaddlevel{1}
\scntext{описание примера}{\textit{Утв. (треугольник; сторона*)}}
\scnnote{Могут быть исключения для \textbf{\textit{утверждений}}, названия которых закрепились исторически, например, \textit{Теорема Пифагора}, \textit{Аксиома о прямой и точке}.}
\scnaddlevel{-1}

\scnheader{определение}
\scnidtf{текст определения}
\scnsubset{утверждение}
\scndefinition{\textbf{\textit{определение}} -- это \textit{утверждение}, \textit{главным ключевым sc-элементом\scnrolesign} которого является связка \textit{эквиваленции*}, однозначно определяющая некоторое понятие на основе других понятий.}
\scnnote{Каждое определение имеет ровно один \textit{ключевой sc-элемент\scnrolesign} (не считая \textit{главного ключевого sc-элемента\scnrolesign}).}
\scnnote{Для одного и того же понятия в рамках одной \textit{формальной теории} может существовать несколько \textit{утверждений об эквиваленции*}, однозначно задающих некоторое понятие на основе других, однако только одно такое \textit{утверждение} в рамках этой \textit{формальной теории} может быть отмечено как \textbf{\textit{определение}}. Остальные \textit{утверждения об эквиваленции*} могут трактоваться как \textit{пояснения} данного понятия.}
\scntext{правило идентификации экземпляров}{\textbf{\textit{определения}} в рамках \textit{Русского языка} именуются по следующим правилам:
\begin{scnitemize}
    \item в начале идентификатора пишется сокращение \textbf{Опр.};
    \item далее в круглых скобках через точку с запятой записывается основной идентификатор  \textit{ключевого sc-элемента\scnrolesign} данного \textbf{\textit{определения}}.
\end{scnitemize}
}
\scnaddlevel{1}
\scntext{описание примера}{\textit{Опр. (связный граф)}}
\scnaddlevel{-1}

\scnheader{общезначимая логическая формула}
\scnsubset{выполнимая логическая формула}
\scnsubset{тавтология}
\scndefinition{\textbf{\textit{общезначимая логическая формула}} -- это \textit{логическая формула}, для которой не существует \textit{формальной теории}, в рамках которой она была бы ложной с учетом истинности и ложности всех ее \textit{подформул*} в рамках этой же \textit{формальной теории}.}

\scnheader{противоречивая логическая формула}
\scnsubset{невыполнимая логическая формула}
\scnsubset{тавтология}
\scndefinition{\textbf{\textit{противоречивая логическая формула}} -- это \textit{логическая формула}, для которой не существует \textit{формальной теории}, в рамках которой она была бы истинной с учетом истинности и ложности всех ее \textit{подформул*} в рамках этой же \textit{формальной теории}.}

\scnheader{нейтральная логическая формула}
\scnsubset{выполнимая логическая формула}
\scndefinition{\textbf{\textit{нейтральная логическая формула}} -- это \textit{логическая формула}, для которой существует хотя бы одна \textit{формальная теория}, в рамках которой эта формула ложна, и хотя бы одна \textit{формальная теория}, в рамках которой эта формула истинна.}

\scnheader{выполнимая логическая формула}
\scndefinition{\textbf{\textit{выполнимая логическая формула}} -- это \textit{логическая формула}, для которой существует хотя бы одна \textit{формальная теория}, в рамках которой эта формула истинна.}

\scnheader{невыполнимая логическая формула}
\scndefinition{\textbf{\textit{невыполнимая логическая формула}} -- это \textit{логическая формула}, для которой существует хотя бы одна \textit{формальная теория}, в рамках которой эта формула ложна.}

\scnheader{тавтология}
\scnidtf{тождественно истинная формула}
\scndefinition{\textbf{\textit{тавтология}} -- это \textit{логическая формула}, которая является либо только истинной, либо только ложной в рамках всех \textit{формальных теорий}, в которых можно установить ее истинность или ложность.}
\scnexplanation{\textbf{\textit{тавтология}} -- это такая \textit{логическая формула}, которая является либо \textit{общезначимой логической формулой}, либо \textit{противоречивой логической формулой}.}

\scnheader{логическая связка*}
\scnidtf{неатомарная логическая формула}
\scnidtf{логический оператор*}
\scnidtf{пропозициональная связка*}
\scniselement{класс связок разной мощности}
\scnrelto{семейство подмножеств}{неатомарное высказывание}
\scndefinition{\textbf{\textit{логическая связка*}} -- это отношение (класс связок), связками которого являются \textit{высказывания}.}
\scnexplanation{\textbf{\textit{логическая связка*}} -- это \textit{отношение}, областью определения которого является множество \textit{высказываний}, при этом само это отношение и некоторые его подмножества могут быть \textit{классами связок разной мощности}.}

\scnheader{импликация*}
\scnidtf{логическое следование*}
\scnsubset{логическая связка*}
\scniselement{бинарное отношение}
\scniselement{ориентированное отношение}
\scndefinition{\textbf{\textit{импликация*}} -- это множество импликативных \textit{неатомарных высказываний}, каждое из которых состоит из посылки (первый компонент \textit{высказывания}) и следствия (второй компонент \textit{высказывания}). Каждое импликативное \textit{высказывание} ложно в рамках некоторой \textit{формальной теории} в том случае, когда его посылка истинна, а заключение ложно в рамках этой же \textit{формальной теории}. В других случаях такое \textit{высказывание} истинно.}
\scnnote{По умолчанию на все переменные, входящие в обе части высказывания об \textbf{\textit{имликации*}} (или хотя бы одну из \textit{подформул*} каждой части) неявно накладывается квантор \textit{всеобщности*}, при условии, что эти переменные не связаны другим \textit{квантором}, указанным явно.}
\scnrelfrom{описание примера}{
\scnfilescg{figures/sd_logical_formulas/implication.png}}

\scnheader{если\scnrolesign}
\scnsubset{1\scnrolesign}
\scniselement{ролевое отношение}
\scndefinition{\textbf{\textit{если\scnrolesign}} -- это \textit{ролевое отношение}, используемое в связках \textit{импликации*} для указания посылки.}

\scnheader{то\scnrolesign}
\scnsubset{2\scnrolesign}
\scniselement{ролевое отношение}
\scndefinition{\textbf{\textit{то\scnrolesign}} -- это \textit{ролевое отношение}, используемое в связках \textit{импликации*} для указания следствия.}

\scnheader{эквиваленция*}
\scnidtf{эквивалентность*}
\scnsubset{логическая связка*}
\scniselement{бинарное отношение}
\scniselement{неориентированное отношение}
\scndefinition{\textbf{\textit{эквиваленция*}} -- это множество \textit{высказываний} об эквивалентности, каждое из которых истинно в рамках некоторой \textit{формальной теории} только в тех случаях, когда оба его компонента одновременно либо истинны в рамках этой же \textit{формальной теории}, либо ложны.}
\scnnote{По умолчанию на все переменные, входящие в обе части высказывания об \textbf{\textit{эквиваленции*}} (или хотя бы одну из \textit{подформул*} каждой части) неявно накладывается квантор \textit{всеобщности*}, при условии, что эти переменные не связаны другим \textit{квантором}, указанным явно.}
\scnrelfrom{описание примера}{
\scnfilescg{figures/sd_logical_formulas/equivalent.png}}

\scnheader{конъюнкция*}
\scnidtf{логическое и*}
\scnidtf{логическое умножение*}
\scnsubset{логическая связка*}
\scniselement{неориентированное отношение}
\scniselement{класс связок разной мощности}
\scndefinition{\textbf{\textit{конъюнкция*}} -- это множество конъюнктивных \textit{высказываний}, каждое из которых истинно в рамках некоторой \textit{формальной теории} только в том случае, когда все его компоненты истинны в рамках этой же \textit{формальной теории}.}
\scnrelfrom{описание примера}{
\scnfilescg{figures/sd_logical_formulas/conjunction.png}}

\scnheader{дизъюнкция*}
\scnidtf{логическое или*}
\scnidtf{логическое сложение*}
\scnidtf{включающее или*}
\scnsubset{логическая связка*}
\scniselement{неориентированное отношение}
\scniselement{класс связок разной мощности}
\scndefinition{\textbf{\textit{дизъюнкция*}} -- это множество дизъюнктивных \textit{высказываний}, каждое из которых истинно в рамках некоторой \textit{формальной теории} только в том случае, когда хотя бы один его компонент является истинным в рамках этой же \textit{формальной теории}.}
\scnrelfrom{описание примера}{
\scnfilescg{figures/sd_logical_formulas/disjunction.png}}

\scnheader{строгая дизъюнкция*}
\scnidtf{сложение по модулю 2*}
\scnidtf{исключающее или*}
\scnidtf{альтернатива*}
\scnsubset{логическая связка*}
\scniselement{неориентированное отношение}
\scniselement{класс связок разной мощности}
\scndefinition{\textbf{\textit{строгая дизъюнкция*}} -- это множество строго дизъюнктивных \textit{высказываний}, каждое из которых истинно в рамках некоторой \textit{формальной теории} только в том случае, когда ровно один его компонент является истинным в рамках этой же \textit{формальной теории}.}
\scnrelfrom{описание примера}{
\scnfilescg{figures/sd_logical_formulas/strictDisjunction.png}}

\scnheader{отрицание*}
\scnsubset{логическая связка*}
\scnsubset{синглетон}
\scndefinition{\textbf{\textit{отрицание*}} -- это множество \textit{высказываний} об отрицании, каждое из которых истинно в рамках некоторой \textit{формальной теории} только в том случае, когда его единственный элемент является ложным в рамках этой же \textit{формальной теории}.}
\scnrelfrom{описание примера}{
\scnfilescg{figures/sd_logical_formulas/negation.png}}

\scnheader{квантор}
\scnsubset{логическая связка*}
\scndefinition{\textbf{\textit{квантор}} -- это \textit{отношение}, каждая связка которой задает истинность множества \textit{логических формул}, входящих в ее состав, при выполнении дополнительных условий, связанных с некоторыми из переменных, входящих в состав этих \textit{логических формул}.}
\scnnote{Будем говорить, что переменные связаны \textbf{\textit{квантором}} или попадают под область действия данного \textbf{\textit{квантора}} (имея в виду конкретную связку конкретного \textbf{\textit{квантора}}). Таким образом, в состав каждой связки каждого \textbf{\textit{квантора}} входит \textit{атомарная формула}, являющаяся \textit{тривиальной структурой}, в которой перечислены переменные, связанные данным \textbf{\textit{квантором}}.}

\scnheader{всеобщность*}
\scnidtf{квантор всеобщности*}
\scnidtf{квантор общности*}
\scniselement{квантор}
\scniselement{ориентированное отношение}
\scniselement{класс связок разной мощности}
\scndefinition{\textbf{\textit{всеобщность}} -- это \textit{квантор}, для каждой связки которого, истинной в рамках некоторой \textit{формальной теории}, выполняется следующее утверждение: все формулы, входящие в состав этой связки истинны в рамках этой же \textit{формальной теории} при всех (любых) возможных значениях всех элементов множества \textit{связываемых переменных\scnrolesign} входящего в эту связку.}
\scnnote{Каждая связка \textit{квантора} \textbf{\textit{всеобщность*}} может быть представлена как \textit{конъюнкция*} (потенциально бесконечная) исходных \textit{логических формул}, входящих в состав этой связки, в каждой из которых все \textit{связанные переменные\scnrolesign} заменены на их возможные значения.}
\scnnote{Квантор \textbf{\textit{всеобщности*}} зачастую обозначается "$\forall$" \ и читается как "для всех"{}, "для каждого"{}, "для любого"{} или "все"{}, "каждый"{}, "любой".}
\scnrelfrom{описание примера}{
\scnfilescg{figures/sd_logical_formulas/universality.png}}

\scnheader{формула существования}
\scnidtf{существование*}
\scnsubdividing{атомарная логическая формула;неатомарное существование*}

\scnheader{неатомарное существование*}
\scnidtf{квантор неатомарного существования*}
\scniselement{квантор}
\scniselement{ориентированное отношение}
\scniselement{класс связок разной мощности}
\scndefinition{\textbf{\textit{неатомарное существование*}} -- это \textit{квантор}, для каждой связки которого, истинной в рамках некоторой \textit{формальной теории}, выполняется следующее утверждение: существуют значения всех элементов множества \textit{связываемых переменных\scnrolesign} входящего в эту связку, такие, что все формулы, входящие в состав этой связки истинны в рамках этой же \textit{формальной теории}.}
\scnnote{Каждая связка \textit{квантора} \textbf{\textit{неатомарное существование*}} может быть представлена как \textit{дизъюнкция*} (потенциально бесконечная) исходных \textit{логических формул}, входящих в состав этой связки, в каждой из которых все \textit{связанные переменные\scnrolesign} заменены на их возможные значения.}
\scnnote{квантор \textbf{\textit{существования*}} зачастую обозначается "$\exists$" \ и читается как "существует"{}, "для некоторого"{}, "найдется".}
\scnrelfrom{описание примера}{
\scnfilescg{figures/sd_logical_formulas/non_atomicExistence.png}}

\scnheader{число значений переменной}
\scniselement{параметр}
\scnexplanation{Каждый элемент \textit{параметра} \textbf{\textit{число значений переменной}} представляет собой класс ориентированных пар, первым компонентом которых является знак \textit{логической формулы}, вторым -- \textit{sc-переменная}, имеющая в рамках данной \textit{логической формулы} ограниченное известное число значений, при которых данная формула является истинной в рамках соответствующей \textit{формальной теории}.\\
Отметим, что в случае \textit{атомарной логической формулы} каждая такая связка связывает знак формулы и знак принадлежащей ей \textit{sc-переменной}, т.е. является, по сути, частным случаем пары принадлежности. В случае \textit{неатомарной логической формулы} указанная \textit{sc-переменная} может принадлежать любой из \textit{подформул*} исходной формулы.

Таким образом, \textit{измерением*} каждого значения параметра \textbf{\textit{число значений переменной}} является некоторое \textit{число}, задающее количество значений \textit{sc-переменных} в рамках \textit{логической формулы}.}

\scnheader{кратность существования}
\scniselement{параметр}
\scnrelfrom{область определения параметра}{формула существования}
\scnhaselement{единственное существование}
\scnexplanation{Каждый элемент \textit{параметра} \textbf{\textit{кратность существования}} представляет собой класс логических \textit{формул существования}, для которых  при интерпретации на соответствующей \textit{предметной области} существует ограниченное общее для всех таких формул число комбинаций значений переменных, при которых указанные формулы являются истинными в рамках соответствующей \textit{формальной теории}.

Таким образом, \textit{измерением*} каждого значения \textbf{\textit{кратности существования}} является некоторое \textit{число}, задающее количество таких комбинаций.}

\scnheader{единственное существование}
\scnidtf{однократное существование}
\scnidtf{формула существования и единственности}
\scnnote{\textbf{\textit{единственное существование}} зачастую обозначается "$\exists!$" \ и читается как "существует и единственный".}

\scnheader{логическая формула и единственность}
\scnsubset{логическая формула}
\scnsubset{единственное существование}
\scnexplanation{Каждый элемент множества \textbf{\textit{логическая формула и единственность}} представляет собой \textit{логическую формулу} (\textit{атомарную} или \textit{неатомарную}), для которой дополнительно уточняется, что при ее интерпретации на некоторой предметной области существует только один набор значений переменных, входящих в эту формулу (или ее \textit{подформулы*}), при котором указанная логическая формула истинна в рамках \textit{формальной теории}, в которую входит данная \textit{предметная область}.}

\scnheader{связываемые переменные\scnrolesign}
\scniselement{ролевое отношение}
\scndefinition{\textbf{\textit{связываемые переменные\scnrolesign}} -- это \textit{ролевое отношение}, которое связывает связку конкретного \textit{квантора} с множеством переменных, которые связаны этим квантором.}
%\scnrelfrom{описание примера}{
%\scnfilescg{figures/sd_logical_formulas/bindVariables.png}}

\scnheader{открытая логическая формула}
\scndefinition{\textbf{\textit{открытая логическая формула}} -- это \textit{логическая формула}, в рамках которой (и всех ее \textit{подформул*}) существует хотя бы одна переменная, не связанная никаким \textit{квантором}.}

\scnheader{замкнутая логическая формула}
\scndefinition{\textbf{\textit{замкнутая логическая формула}} -- это \textit{логическая формула}, в рамках которой (и всех ее \textit{подформул*}) не существует переменных, не связанных каким-либо \textit{квантором}.}

\scnheader{Примеры логических утверждений}
\scneqtoset{\scgfileitem{figures/sd_logical_formulas/ex_set_union.png};\scgfileitem{figures/sd_logical_formulas/ex_set_diff.png}}

\bigskip
\scnendstruct \scnendcurrentsectioncomment

\end{SCn}

\scsubsection[\scneditors{Садовский М.Е.;Никифоров С.А.}\protect\scnmonographychapter{Глава 2.6. Языковые средства формального описания синтаксиса и денотационной семантики различных языков в интеллектуальных компьютерных системах нового поколения}\protect\scnmonographychapter{Глава 4.1. Структура интерфейсов интеллектуальных компьютерных систем нового поколения}]{Предметная область и онтология файлов, внешних информационных конструкций и внешних языков ostis-систем}
\label{sd_file_internal_inform_struct}

\scsubsubsection[\scneditors{Никифоров С.А.;Бобёр  Е.С.}\protect\scnmonographychapter{Глава 2.6. Языковые средства формального описания синтаксиса и денотационной семантики различных языков в интеллектуальных компьютерных системах нового поколения}]{Предметная область и онтология естественных языков}
\label{sd_natural_languages}
\begin{SCn}

\scnsectionheader{\currentname}

\scnstartsubstruct

\scnheader{язык}
\scnsubdividing{естественный язык\\
	\scnaddlevel{1}
	\scnexplanation{Естественный язык представляет собой язык, который не был создан целенаправленно}
	\scnaddlevel{-1}
;искусственный язык\\
	\scnaddlevel{1}
	\scnexplanation{Искусственный язык представляет собой язык, специально разработанный для воплощения определённых целей}
	\scnhaselement{Эсперанто}
	\scnhaselement{Python}
	\scnsuperset{сконструированный язык}
	\scnaddlevel{1}
	\scnexplanation{Сконструированный язык представляет собой искусственный язык, предназначенный для общения людей}
	\scnhaselement{Эсперанто}
	\scnaddlevel{-1}
	\scnaddlevel{-1}
}
\scnsuperset{международный язык}
	\scnaddlevel{1}
	\scnexplanation{Международный язык представляет собой естественный или искусственный язык, использующийся для общения людей разных стран}
	\scnhaselement{Английски язык}
	\scnhaselement{Русский язык}
	\scnaddlevel{-1}

\scnheader{плановый язык}
\scnreltoset{пересечение}{сконструированный язык;международный язык}

\scnheader{язык общения}
\scnreltoset{объединение}{естественный язык;сконструированный язык}
\scnhaselement{Английски язык}
\scnhaselement{Русский язык}
\scnhaselement{Эсперанто}
\scnreltoset{объединение}{корневой язык\\
	\scnaddlevel{1}
	\scnexplanation{Корневой язык представляет собой язык, для которого характерно полное отсутствие словоизменения и наличие грамматической значимости порядка слов, представленных только корнями.}
	\scnhaselement{Английски язык}
	\scnaddlevel{-1}
;агглютинативный язык\\
	\scnaddlevel{1}
	\scnexplanation{Агглютинативный язык характеризуется развитой системой употребления суффиксов, приставок, добавляемых к неизменяемой основе слова, которые используются для выражения числа, падежа, рода и др.}
	\scnhaselement{Английски язык}
	\scnaddlevel{-1}
;флективный язык\\
	\scnaddlevel{1}
	\scnexplanation{Для флективного языка характерно развитое употребление окончаний для выражения рода, числа, падежа, сложная система склонения глаголов, чередование гласных в корне. Строгое различение частей речи.}
	\scnhaselement{Русский язык}
	\scnaddlevel{-1}
;профлективный язык\\
	\scnaddlevel{1}
	\scnexplanation{Для профлективного языка для именного словоизменения характерна агглютинация, а для глагольного – флексия и чередование гласных (аблаут).}
	\scnaddlevel{-1}
}

\scnheader{словоформа}
\scnsubset{файл}
\scnexplanation{Словоформа - это слово, представленное в определенной грамматической форме.}

\scnheader{лексема}
\scnsubset{файл}
\scnexplanation{Лексема -  слово, рассматриваемое как единица словарного состава языка в совокупности всех его конкретных грамматических форм.}

\scnheader{грамматическая категория*}
\scniselement{бинарное отношение}
\scniselement{ориентированное отношение}
\scnrelfrom{первый домен}{язык}
\scnrelfrom{второй домен}{грамматическая категория}
\scnexplanation{Грамматическая категория* это бинарное ориентированное отношение, связывающее язык со  множеством взаимоисключающих и противопоставленных друг другу грамматических значений, задающих разбиение множества словоформ.}

\scnheader{парадигма*}
\scniselement{квазибинарное отношение}
\scniselement{ориентированное отношение}
\scnsubset{покрытие*}
\scnrelfrom{первый домен}{лексема}
\scnrelfrom{второй домен}{словоформа}
\scnexplanation{Парадигма* это квазибинарное отношение, связывающее лексему со множеством словоформ, принадлежащих данной лексеме и имеющих разные грамматические значения.}

\scnheader{морфема}
\scnsubset{файл}
\scnexplanation{Морфема — наименьшая единица языка, имеющая некоторый смысл.}

\scnheader{суффикс}
\scnsubset{морфема}
\scnexplanation{Суффикс – это морфема, которая стоит после корня и обычно служит для образования новых слов, хотя может использоваться и при образовании формы одного слова.}

\scnheader{корень}
\scnsubset{морфема}
\scnexplanation{Корень — морфема, несущая лексическое значение слова (или основную часть этого значения).}

\scnheader{Английский язык}
\scnrelfromset{грамматическая категория}{
	продолжительная форма';совершенная форма';простая форма'
}
\scnrelfromset{грамматическая категория}{сравнительная степень'\\
	\scnaddlevel{1}
	\scnexplanation{Сравнительную степень используется в случае, когда необходимо отметить, что предмет или человек обладает каким-то качеством в большей степени, чем другие.}
	\scnaddlevel{-1}
;положительная степень сравнения'\\
	\scnaddlevel{1}
	\scnexplanation{Положительная степень используется, чтобы указать, что предмет или человек обладает каким-то признаком или качеством.}
	\scnaddlevel{-1}
;превосходная степень сравнения'\\
	\scnaddlevel{1}
	\scnexplanation{Превосходная степень используется для указания того факта, что предмет или человек обладает каким-то качеством в наибольшей степени.}
	\scnaddlevel{-1}
}
\scnrelfromset{грамматическая категория}{общий падеж';косвенный падеж';притяжательный падеж';именительный падеж'}
\scnrelfromset{грамматическая категория}{женский род';мужской род';средний род'}
\scnrelfromset{грамматическая категория}{множественное число';единственное число'}
\scnrelfromset{грамматическая категория}{первое лицо';второе лицо';третье лицо'}
\scnrelfromset{грамматическая категория}{будущее время';прошедшее время';настоящее время'}

\scnheader{часть речи\scnsupergroupsign}
\scnrelto{семейство подмножеств}{лексема}
\scnexplanation{Часть речи представляют определенные лексико-грамматические разряды, на которые в зависимости от лексического значения от характера морфологических признаков и синтаксической функции делятся все слова языка.}

\scnhaselement{наречие}
	\scnaddlevel{1}
	\scnreltoset{объединение}{наречие места;наречие времени;наречие меры и степени;наречие образа действия}
	\scnsubdividing{простое наречие\\
		\scnaddlevel{1}
		\scnexplanation{Простые наречия не делятся на составные части.}
		\scnaddlevel{-1}
	;производное наречие\\
		\scnaddlevel{1}
		\scnexplanation{Производные наречия образованы при помощи суффиксов.}
		\scnaddlevel{-1}
	;сложное наречие\\
		\scnaddlevel{1}
		\scnexplanation{Сложные наречия образуются из нескольких корней.}
		\scnaddlevel{-1}
	;составное наречие\\
		\scnaddlevel{1}
		\scnexplanation{Составные наречия представляют собой сочетание служебного и знаменательного слова.}
		\scnaddlevel{-1}
	}
	\scnaddlevel{-1}
\scnhaselement{существительное}
	\scnaddlevel{1}
	\scnsubdividing{простое существительное\\
		\scnaddlevel{1}
		\scnexplanation{Простые существительные состоят из одного корня.}
		\scnaddlevel{-1}	
	;производное существительное\\
		\scnaddlevel{1}
		\scnexplanation{Производные существительные состоят из корня и одной или нескольких морфем (приставок или суффиксов).}
		\scnaddlevel{-1}
	;составное существительное\\
		\scnaddlevel{1}
		\scnexplanation{Составные существительные состоят по крайней мере из двух корней.}
		\scnaddlevel{-1}
	}
	\scnsubdividing{имя собственное\\
		\scnaddlevel{1}
		\scnexplanation{Имена собственные обозначают единственные в своем роде предметы или предметы, выделяемые из общего класса.}
		\scnaddlevel{-1}
	;имя нарицательное\\
		\scnaddlevel{1}
		\scnreltoset{объединение}{исчисляемое существительное\\
			\scnaddlevel{1}
			\scnsubdividing{конкретное исчисляемое существительное\\
				\scnaddlevel{1}
				\scnexplanation{Конкретные исчисляемые существительные - названия отдельных предметов и живых существ.}
				\scnaddlevel{-1}	
			;абстрактное исчисляемое существительное существительное\\
				\scnaddlevel{1}
				\scnidtf{отвлеченное исчисляемое существительное}
				\scnexplanation{Абстрактные исчисляемые существительные - названия исчисляемых понятий.}
				\scnaddlevel{-1}
			}
			\scnexplanation{Исчисляемые существительные могут быть посчитаны и имеют форму множественного числа.}
			\scnaddlevel{-1}
		;неисчисляемое существительное\\
			\scnaddlevel{1}
			\scnsubdividing{абстрактное неисчисляемое существительное\\
				\scnaddlevel{1}
				\scnidtf{отвлеченное неисчисляемое существительное}
				\scnexplanation{Абстрактные неисчисляемые существительные - названия неисчисляемых понятий.}
				\scnaddlevel{-1}	
			;вещественное неисчисляемое существительное\\
				\scnaddlevel{1}
				\scnexplanation{Вещественные неисчисляемые существительные - названия различных веществ и материалов.}
				\scnaddlevel{-1}
			}
			\scnexplanation{Неисчисляемые существительные не могут быть посчитаны и не имеют формы множественного числа.}
			\scnaddlevel{-1}
		;собирательное существительное\\
			\scnaddlevel{1}
			\scnexplanation{Собирательные существительные имеют форму единственного числа, но обозначают при этом группы лиц или понятий, рассматриваемые как одно целое.}
			\scnaddlevel{-1}
		}
		\scnexplanation{Имена нарицательные – это общие названия для всех однородных предметов.}
		\scnaddlevel{-1}
	}
	\scnaddlevel{-1}
\scnhaselement{глагол}
	\scnaddlevel{1}
	\scnsubdividing{простой глагол\\
		\scnaddlevel{1}
		\scnexplanation{Простые глаголы состоят только из одного корня.}
		\scnaddlevel{-1}
	;производный глагол\\
		\scnaddlevel{1}
		\scnexplanation{В производных глаголах, кроме корня, есть приставка и/или суффикс.}
		\scnaddlevel{-1}
	;сложный глагол\\
		\scnaddlevel{1}
		\scnexplanation{Сложные глаголы состоят из двух основ.}
		\scnaddlevel{-1}
	;составной глагол\\
		\scnaddlevel{1}
		\scnexplanation{Составные глаголы состоят из глагола и наречия или предлога.}
		\scnaddlevel{-1}
	}
	\scnsubdividing{смысловой глагол\\
		\scnaddlevel{1}
		\scnidtf{самостоятельный глагол}
		\scnsubset{простой глагол}
		\scnexplanation{Смысловые глаголы обладают собственным лексическим значением, они обозначают определенное действие или состояние.}
		\scnaddlevel{-1}
	;служебный глагол\\
		\scnaddlevel{1}
		\scnexplanation{Служебные глаголы не имеют самостоятельного значения. Они используются только для построения сложных форм глагола или составных сказуемых.}
		\scnsubdividing{глагол-связка\\
			\scnaddlevel{1}
			\scnexplanation{Глаголы-связки служат для соединения в предложении подлежащего с определенным состоянием.}
			\scnaddlevel{-1}
		;вспомогательный глагол\\
			\scnaddlevel{1}
			\scnexplanation{Вспомогательные глаголы служат для образования сложных глагольных форм.}
			\scnaddlevel{-1}
		;модальный глагол\\
			\scnaddlevel{1}
			\scnexplanation{Модальные глаголы отражают отношение говорящего к данному действию.}
			\scnaddlevel{-1}
		}
		\scnaddlevel{-1}
	}
\scnaddlevel{-1}
\scnhaselement{прилагательное}
\scnaddlevel{1}
\scnsubdividing{простое прилагательное\\
	\scnaddlevel{1}
	\scnexplanation{Простые прилагательные не имеют в своем составе суффиксов и приставок.}
	\scnaddlevel{-1}
;производное прилагательное\\
	\scnaddlevel{1}
	\scnexplanation{В составе производных прилагательных есть суффикс и/или приставка.}
	\scnaddlevel{-1}
;сложное прилагательное английского языка\\
	\scnaddlevel{1}
	\scnexplanation{Сложные прилагательные состоят из двух или более основ.}
	\scnaddlevel{-1}
}
\scnsubdividing{качественное прилагательное\\
	\scnaddlevel{1}
	\scnexplanation{Качественные прилагательные обозначают качества предмета прямо.}
	\scnaddlevel{-1}
;относительное прилагательное\\
	\scnaddlevel{1}
	\scnexplanation{Относительные прилагательные описывают качества предмета через его отношение к материалам, месту, времени.}
	\scnaddlevel{-1}
}
\scnaddlevel{-1}
\scnhaselement{местоимение}
\scnaddlevel{1}
\scnsubdividing{личное местоимение;притяжательное местоимение;указательное местоимение;возвратное местоимение;взаимное местоимение;вопросительное местоимение;относительное местоимение;неопределенное местоимение;отрицательное местоимение;разделительное местоимение;универсальное местоимение}
\scnaddlevel{-1}
\scnhaselement{предлог}
\scnaddlevel{1}
\scnsubdividing{производный предлог\\
	\scnaddlevel{1}
	\scnexplanation{Производный предлог - предлог, связанный происхождением с другими частями речи.}
	\scnaddlevel{-1}
;непроизводный предлог\\
	\scnaddlevel{1}
	\scnexplanation{Непроизводный предлог - так называемый первообразный предлог, который не может быть соотнесен по образованию с какой-либо частью речи.}
	\scnaddlevel{-1}
;сложный предлог\\
	\scnaddlevel{1}
	\scnexplanation{Сложный предлог - предлог, включающий в себя несколько компонентов.}
	\scnaddlevel{-1}
;составной предлог\\
	\scnaddlevel{1}
	\scnexplanation{Составной предлог представляет собой словосочетание. Он включают в себя слово из другой части речи и один или два предлога.}
	\scnaddlevel{-1}
}
\scnaddlevel{-1}
\scnhaselement{союз}
\scnaddlevel{1}
\scnsubdividing{сочинительный союз\\
	\scnaddlevel{1}
	\scnexplanation{Сочинительные союзы соединяют одинаковые по значимости слова, фразы, однородные члены предложения или независимые предложения в одно сложносочиненное предложение.}
	\scnaddlevel{-1}
;подчинительный союз\\
	\scnaddlevel{1}
	\scnexplanation{Подчинительные союзы соединяют придаточное предложение с основным, от которого оно зависит по смыслу, образуя сложноподчиненное предложение.}
	\scnaddlevel{-1}
;парный союз\\
	\scnaddlevel{1}
	\scnexplanation{Парные союзы служат для соединения слов, фраз или однородных, одинаковых частей одного предложения.}
	\scnaddlevel{-1}
;союзное наречие\\
	\scnaddlevel{1}
	\scnsubset{наречие}
	\scnexplanation{Союзные наречия соединяют два независимых предложения в одно сложносочиненное, или ставятся в начало предложения для его логической связи с предыдущим предложением.}
	\scnaddlevel{-1}
}
\scnsubdividing{простой союз\\
	\scnaddlevel{1}
	\scnexplanation{Простые союзы состоят из одного корня без суффиксов или префиксов.}
	\scnaddlevel{-1}
;сложный союз\\
	\scnaddlevel{1}
	\scnexplanation{Сложные союзы образованы от других частей речи, других союзов.}
	\scnaddlevel{-1}
;составной союз\\
	\scnaddlevel{1}
	\scnexplanation{Составные союзы состоят из двух и более слов, служебных и самостоятельных частей речи. К ним также относятся парные союзы.}
	\scnaddlevel{-1}
}
\scnaddlevel{-1}
\scnhaselement{числительное}
\scnaddlevel{1}
\scnsubdividing{порядковое числительное\\
	\scnaddlevel{1}
	\scnexplanation{Порядковые числительные обозначают порядок предметов.}
	\scnaddlevel{-1}
;количественное числительное английского языка\\
	\scnaddlevel{1}
	\scnexplanation{Количественные числительные обозначают количество предметов.}
	\scnaddlevel{-1}
}
\scnaddlevel{-1}

\scnsuperset{часть речи английского языка\scnsupergroupsign}
	\scnaddlevel{1}
	\scnhaselement{наречие английского языка}
		\scnaddlevel{1}
		\scnreltoset{объединение}{наречие места английского языка\\
			\scnaddlevel{1}
			\scnsubset{наречие места}
			\scnaddlevel{-1}
		;наречие времени английского языка\\
			\scnaddlevel{1}
			\scnsubset{наречие времени}
			\scnaddlevel{-1}
		;наречие меры и степени английского языка\\
			\scnaddlevel{1}
			\scnsubset{наречие меры и степени}
			\scnaddlevel{-1}
		;наречие образа действия английского языка\\
			\scnaddlevel{1}
			\scnsubset{наречие образа действия}
			\scnaddlevel{-1}
		}
		\scnsubdividing{простое наречие английского языка\\
			\scnaddlevel{1}
			\scnsubset{простое наречие}
			\scnaddlevel{-1}
		;производное наречие английского языка\\
			\scnaddlevel{1}
			\scnsubset{производное наречие}
			\scnaddlevel{-1}
		;сложное наречие английского языка\\
			\scnaddlevel{1}
			\scnsubset{сложное наречие}
			\scnaddlevel{-1}
		;составное наречие английского языка\\
			\scnaddlevel{1}
			\scnsubset{составное наречие}
			\scnaddlevel{-1}
		}
		\scnaddlevel{-1}
	\scnhaselement{существительное английского языка}
		\scnaddlevel{1}
		\scnsubdividing{простое существительное английского языка\\
			\scnaddlevel{1}
			\scnsubset{простое существительное}
			\scnaddlevel{-1}
		;производное существительное английского языка\\
			\scnaddlevel{1}
			\scnsubset{производное существительное}
			\scnaddlevel{-1}
		;составное существительное английского языка\\
			\scnaddlevel{1}
			\scnsubset{составное существительное}
			\scnaddlevel{-1}
		}
		\scnsubdividing{имя собственное английского языка\\
			\scnaddlevel{1}
			\scnsubset{имя собственное}
			\scnaddlevel{-1}
		;имя нарицательное английского языка\\
			\scnaddlevel{1}
			\scnreltoset{объединение}{исчисляемое существительное английского языка\\
				\scnaddlevel{1}
				\scnsubdividing{конкретное исчисляемое существительное английского языка\\
					\scnaddlevel{1}
					\scnsubset{конкретное исчисляемое существительное}
					\scnaddlevel{-1}
				;абстрактное исчисляемое существительное английского языка\\
					\scnaddlevel{1}
					\scnidtf{отвлеченное исчисляемое существительное английского языка}
					\scnsubset{абстрактное исчисляемое существительное}
					\scnaddlevel{-1}
				}
				\scnsubset{исчисляемое существительное}
				\scnaddlevel{-1}
			;неисчисляемое существительное английского языка\\
				\scnaddlevel{1}
				\scnsubdividing{абстрактное неисчисляемое существительное английского языка\\
					\scnaddlevel{1}
					\scnsubset{абстрактное неисчисляемое существительное}
					\scnaddlevel{-1}
					;вещественное неисчисляемое существительное английского языка\\
					\scnaddlevel{1}
					\scnsubset{вещественное неисчисляемое существительное}
					\scnaddlevel{-1}
				}
				\scnsubset{неисчисляемое существительное}
				\scnaddlevel{-1}
			;собирательное существительное английского языка\\
				\scnaddlevel{1}
				\scnsubset{собирательное существительное}
				\scnaddlevel{-1}
			}
			\scnsubset{имя нарицательное}
			\scnaddlevel{-1}
		}
		\scnaddlevel{-1}
	\scnhaselement{глагол английского языка}
		\scnaddlevel{1}
		\scnsubdividing{простой глагол английского языка\\
			\scnaddlevel{1}
			\scnsubset{простой глагол}
			\scnaddlevel{-1}
		;производный глагол английского языка\\
			\scnaddlevel{1}
			\scnsubset{производный глагол}
			\scnaddlevel{-1}
		;сложный глагол английского языка\\
			\scnaddlevel{1}
			\scnsubset{сложный глагол}
			\scnaddlevel{-1}
		;составной глагол английского языка\\
			\scnaddlevel{1}
			\scnsubset{составной глагол}
			\scnaddlevel{-1}
		}
		\scnsubdividing{смысловой глагол английского языка\\
			\scnaddlevel{1}
			\scnidtf{самостоятельный глагол английского языка}
			\scnsubset{простой глагол}
			\scnaddlevel{-1}
		;служебный глагол английского языка\\
			\scnaddlevel{1}
			\scnsubset{служебный глагол}
			\scnaddlevel{1}
			\scnsubdividing{глагол-связка английского языка\\
				\scnaddlevel{1}
				\scnsubset{глагол-связка}
				\scnaddlevel{-1}
			;вспомогательный глагол английского языка\\
				\scnaddlevel{1}
				\scnsubset{вспомогательный глагол}
				\scnaddlevel{-1}
			;модальный глагол английского языка\\
				\scnaddlevel{1}
				\scnsubset{модальный глагол}
				\scnaddlevel{-1}
			}
			\scnaddlevel{-2}
		}
		\scnaddlevel{-1}
	\scnhaselement{прилагательное английского языка}
	\scnaddlevel{1}
	\scnsubdividing{простое прилагательное английского языка\\
		\scnaddlevel{1}
		\scnsubset{простое прилагательное}
		\scnaddlevel{-1}
	;производное прилагательное английского языка\\
		\scnaddlevel{1}
		\scnsubset{производное прилагательное}
		\scnaddlevel{-1}
	;сложное прилагательное английского языка\\
		\scnaddlevel{1}
		\scnsubset{сложное прилагательное}
		\scnaddlevel{-1}
	}
	\scnsubdividing{качественное прилагательное английского языка\\
		\scnaddlevel{1}
		\scnsubset{качественное прилагательное}
		\scnaddlevel{-1}
	;относительное прилагательное английского языка\\
		\scnaddlevel{1}
		\scnsubset{относительное прилагательное}
		\scnaddlevel{-1}
	}
	\scnaddlevel{-1}
	\scnhaselement{местоимение английского языка}
		\scnaddlevel{1}
		\scnsubdividing{личное местоимение английского языка\\
			\scnaddlevel{1}
			\scnsubset{личное местоимение}
			\scnaddlevel{-1}
		;притяжательное местоимение английского языка\\
			\scnaddlevel{1}
			\scnsubset{притяжательное местоимение}
			\scnaddlevel{-1}
		;указательное местоимение английского языка\\
			\scnaddlevel{1}
			\scnsubset{указательное местоимение}
			\scnaddlevel{-1}
		;возвратное местоимение английского языка\\
			\scnaddlevel{1}
			\scnsubset{возвратное местоимение}
			\scnaddlevel{-1}
		;взаимное местоимение английского языка\\
			\scnaddlevel{1}
			\scnsubset{взаимное местоимение}
			\scnaddlevel{-1}
		;вопросительное местоимение английского языка\\
			\scnaddlevel{1}
			\scnsubset{вопросительное местоимение}
			\scnaddlevel{-1}
		;относительное местоимение английского языка\\
			\scnaddlevel{1}
			\scnsubset{относительное местоимение}
			\scnaddlevel{-1}
		;неопределенное местоимение английского языка\\
			\scnaddlevel{1}
			\scnsubset{неопределенное местоимение}
			\scnaddlevel{-1}
		;отрицательное местоимение английского языка\\
			\scnaddlevel{1}
			\scnsubset{отрицательное местоимение}
			\scnaddlevel{-1}
		;разделительное местоимение английского языка\\
			\scnaddlevel{1}
			\scnsubset{разделительное местоимение}
			\scnaddlevel{-1}
		;универсальное местоимение английского языка\\
			\scnaddlevel{1}
			\scnsubset{универсальное местоимение}
			\scnaddlevel{-1}
		}
		\scnaddlevel{-1}
	\scnhaselement{предлог английского языка}
		\scnaddlevel{1}
		\scnsubdividing{производный предлог английского языка\\
			\scnaddlevel{1}
			\scnsubset{производный предлог}
			\scnaddlevel{-1}
		;непроизводный предлог английского языка\\
			\scnaddlevel{1}
			\scnsubset{непроизводный предлог}
			\scnaddlevel{-1}
		;сложный предлог английского языка\\
			\scnaddlevel{1}
			\scnsubset{сложный предлог}
			\scnaddlevel{-1}
		;составной предлог английского языка\\
			\scnaddlevel{1}
			\scnsubset{составной предлог}
			\scnaddlevel{-1}
		}
		\scnaddlevel{-1}
	\scnhaselement{союз английского языка}
		\scnaddlevel{1}
		\scnsubdividing{сочинительный союз английского языка\\
			\scnaddlevel{1}
			\scnsubset{сочинительный союз}
			\scnaddlevel{-1}
		;подчинительный союз английского языка\\
			\scnaddlevel{1}
			\scnsubset{подчинительный союз}
			\scnaddlevel{-1}
		;парный союз английского языка\\
			\scnaddlevel{1}
			\scnsubset{парный союз}
			\scnaddlevel{-1}
		;союзное наречие английского языка\\
			\scnaddlevel{1}
			\scnsubset{наречие английского языка}
			\scnsubset{союзное наречие}
			\scnaddlevel{-1}
		}
		\scnsubdividing{простой союз английского языка\\
			\scnaddlevel{1}
			\scnsubset{простой союз}
			\scnaddlevel{-1}
		;сложный союз английского языка\\
			\scnaddlevel{1}
			\scnsubset{сложный союз}
			\scnaddlevel{-1}
		;составной союз английского языка\\
			\scnaddlevel{1}
			\scnsubset{составной союз}
			\scnaddlevel{-1}
		}
		\scnaddlevel{-1}
	\scnhaselement{числительное английского языка}
	\scnaddlevel{1}
	\scnsubdividing{порядковое числительное английского языка\\
		\scnaddlevel{1}
		\scnsubset{порядковое числительное}
		\scnaddlevel{-1}
	;количественное числительное английского языка\\
		\scnaddlevel{1}
		\scnsubset{количественное числительное}
		\scnaddlevel{-1}
	}
	\scnaddlevel{-2}

\scnendstruct \scnendcurrentsectioncomment

\end{SCn}

\scparagraph[\scneditors{Никифоров С.А.;Бобёр  Е.С.}\protect\scnmonographychapter{Глава 2.6. Языковые средства формального описания синтаксиса и денотационной семантики различных языков в интеллектуальных компьютерных системах нового поколения}]{Предметная область и онтология синтаксиса естественных языков}
\label{sd_syntax_natural_lang}

\scparagraph[\scneditors{Никифоров С.А.;Бобёр  Е.С.}\protect\scnmonographychapter{Глава 2.6. Языковые средства формального описания синтаксиса и денотационной семантики различных языков в интеллектуальных компьютерных системах нового поколения}]{Предметная область и онтология денотационной семантики естественных языков}
\label{sd_sem_natural_lang}

\scsubsection[\scneditors{Никифоров С.А.;Шункевич Д.В.}\protect\scnmonographychapter{Глава 3.1. Формализация понятий действия, задачи, метода, средства, навыка и технологии}]{Глобальная предметная область и онтология, описывающая воздействия, действия, методы, средства и технологии}
\label{sd_actions}
\begin{SCn}

\scnsectionheader{\currentname}

\scnstartsubstruct

\scnsdmainclasssingle{действие}
\scnsdclass{информационное действие;поведенческое действие;эффекторное действие;рецепторное действие;действие в sc-памяти;действие во внешней среде ostis-системы;эффекторное действие ostis-системы;рецепторное действие ostis-системы;инициированное действие;выполняемое действие;активное действие;отложенное действие;планируемое действие;выполненное действие;успешно выполненное действие;безуспешно выполненное действие;действие, выполненное с ошибкой;приоритет действия;субъект;внутренний субъект ostis-системы;внешний субъект ostis-системы, с которым осуществляется взаимодействие;внешний субъект ostis-системы, с которым взаимодействие не происходит;класс действий;атомарный класс действий;неатомарный класс действий;конъюнкция предшествующих действий;проверка условия;задача;процедурная формулировка задачи;декларативная формулировка задачи;класс задач;вопрос;команда;класс команд;класс команд без аргументов;класс команд с одним аргументом;класс команд с двумя аргументами;класс команд с произвольным числом аргументов;атомарный класс команд;неатомарный класс команд;план;программа;программа в sc-памяти;протокол;решение}
\scnsdrelation{дейcтвие с очень высоким приоритетом';дейcтвие с высоким приоритетом';дейcтвие со средним приоритетом';дейcтвие с низким приоритетом';дейcтвие с очень низким приоритетом';декомпозиция действия*;поддействие*;последовательность действий*;последовательность действий при положительном результате*;последовательность действий при отрицательном результате*;последовательность действий в случае ошибки*;результат*;исполнитель*;класс выполняемых действий*;заказчик*;инициатор*;объект*;контекст действия*;аргумент действия';первый аргумент действия’;второй аргумент действия’;третий аргумент действия’;класс аргументов*;класс первых аргументов*;класс вторых аргументов*}
\scnrelfromvector{ключевые знаки}{действие;класс действий;метод;класс методов;деятельность;вид деятельности}



\scnendstruct \scnendcurrentsectioncomment

\end{SCn}

\scsubsubsection[\scnmonographychapter{Глава 3.1. Формализация понятий действия, задачи, метода, средства, навыка и технологии}]{Предметная область и онтология локальных предметных областей и онтологий действий}
\label{local_sd_actions}

\scsubsubsection[\scnidtf{Типология неавтоматизированных ("вручную"{} выполняемых) и автоматически выполняемых \textit{действий}, направленных на управление процессами выполнения различных \textit{сложных действий}, а также система понятий, используемая для \textit{управления сложными действиями}}]{Предметная область и онтология действий по управлению деятельностью многоагентных систем}
\label{local_sd_project_management}

%\scsectionfamily{Часть 3. Многоагентные решатели задач интеллектуальных компьютерных систем нового поколения}
\label{part_solvers}

\scsection{Глава 30. Предметная область и онтология решателей задач ostis-систем}
\label{sd_ps}
\begin{SCn}
\scnsectionheader{Предметная область и онтология решателей задач ostis-систем}
\begin{scnsubstruct}
\begin{scnrelfromlist}{дочерняя предметная область и онтология}
	\scnitem{Предметная область и онтология действий, задач, планов, протоколов и методов, реализуемых ostis-системой, а также внутренних агентов, выполняющих эти действия}
	\scnitem{Предметная область и онтология Базового языка программирования ostis-систем}
	\scnitem{Предметная область и онтология искусственных нейронных сетей и соответствующая им предметная область и онтология действий по обработке искусственных нейронных сетей}
\end{scnrelfromlist}

\scnheader{Предметная область решателей задач ostis-систем}
\begin{scnrelfromlist}{автор}
	\scnitem{Шункевич Д.В.}
\end{scnrelfromlist}
\begin{scnhaselementrolelist}{ключевой знак}
	\scnitem{Язык SCP}
	\scnitem{Абстрактная scp-машина}
\end{scnhaselementrolelist}
\begin{scnhaselementrolelist}{класс объектов исследования}
	\scnitem{действие в sc-памяти}
	\scnitem{действие в sc-памяти, инициируемое вопросом}
	\scnitem{действие редактирования базы знаний}
	\scnitem{задача, решаемая в sc-памяти}
	\scnitem{класс логически атомарных действий}
	\scnitem{sc-агент}
	\scnitem{абстрактный sc-агент}
	\scnitem{атомарный абстрактный sc-агент}
	\scnitem{неатомарный абстрактный sc-агент}
	\scnitem{абстрактный sc-агент, реализуемый на Языке SCP}
	\scnitem{абстрактный sc-агент, не реализуемый на Языке SCP}	
	\scnitem{тип блокировки}
	\scnitem{транзакция в sc-памяти}	
	\scnitem{scp-оператор}
	\scnitem{решатель задач ostis-системы}
	\scnitem{машина обработки знаний}
\end{scnhaselementrolelist}
\begin{scnhaselementrolelist}{исследуемое отношение}
	\scnitem{блокировка*}
	\scnitem{планируемая блокировка*}
	\scnitem{приоритет блокировки*}
	\scnitem{удаляемые sc-элементы*}
	\scnitem{параметр scp-программы\scnrolesign}
	\scnitem{scp-операнд\scnrolesign}
\end{scnhaselementrolelist}
\begin{scnrelfromlist}{библиографическая ссылка}
	\scnitem{\scncite{Kolesnikov2001}}
	\scnitem{\scncite{Pratt2002}}
	\scnitem{\scncite{Gladkov2006}}
	\scnitem{\scncite{Emelyanov2003}}
	\scnitem{\scncite{Berkinblit1993}}
	\scnitem{\scncite{Golovko2001}}
	\scnitem{\scncite{Gorban1996}}
	\scnitem{\scncite{Vagin2008}}
	\scnitem{\scncite{Khulick2001}}
	\scnitem{\scncite{Polya1975}}
	\scnitem{\scncite{Batyrshin2001}}
	\scnitem{\scncite{Demenkov2005}}
	\scnitem{\scncite{Pospelov1989}}
	\scnitem{\scncite{Reiter1980}}
	\scnitem{\scncite{Eremeev1997}}
	\scnitem{\scncite{Kachro1988}}
	\scnitem{\scncite{Ephymov1982}}
	\scnitem{\scncite{Raghovsky2011}}
	\scnitem{\scncite{Podkholzyn2008}}
	\scnitem{\scncite{Khurbatov2016}}
	\scnitem{\scncite{Vladimirov2010}}
	\scnitem{\scncite{AIRefBookP11990}}
	\scnitem{\scncite{Jackson1998}}
	\scnitem{\scncite{W3C}}
	\scnitem{\scncite{RDF}}
	\scnitem{\scncite{OWL}}
	\scnitem{\scncite{SPARQL}}
	\scnitem{\scncite{Neo4j}}
	\scnitem{\scncite{OWLImplementations}}
	\scnitem{\scncite{Gribova2015a}}
	\scnitem{\scncite{Gribova2011}}
	\scnitem{\scncite{Phylyppov2016}}
	\scnitem{\scncite{Borisov2014}}
	\scnitem{\scncite{Dutta1993}}
	\scnitem{\scncite{Pau1990}}
	\scnitem{\scncite{Wooldridge2009}}
	\scnitem{\scncite{Weyns2007}}
	\scnitem{\scncite{ACL}}
	\scnitem{\scncite{Finin1994}}
	\scnitem{\scncite{KIF}}
	\scnitem{\scncite{Hartung2008}}
	\scnitem{\scncite{Sims2008}}
	\scnitem{\scncite{Excelente-Toledo2004}}
	\scnitem{\scncite{NagendraPrasad1999}}
	\scnitem{\scncite{Vasconcelos2009}}
	\scnitem{\scncite{Rumbell2012}}
	\scnitem{\scncite{Gorodetsky2015}}
	\scnitem{\scncite{Bordini2007}}
	\scnitem{\scncite{Castillo2014}}
	\scnitem{\scncite{EVE}}
	\scnitem{\scncite{GAMA}}
	\scnitem{\scncite{GOAL}}
	\scnitem{\scncite{Evertsz2004}}
	\scnitem{\scncite{JADE}}
	\scnitem{\scncite{Boissier2013}}
	\scnitem{\scncite{Omicini1999}}
	\scnitem{\scncite{Jagannathan1989}}
	\scnitem{\scncite{Pospelov1986}}
	\scnitem{\scncite{Dijkstra2002}}
	\scnitem{\scncite{Hoare1983}}
	\scnitem{\scncite{Chatterjee2022}}
	\scnitem{\scncite{Narinjani2004}}
	\scnitem{\scncite{Cao2010}}
	\scnitem{\scncite{Cao2014}}
	\scnitem{\scncite{Pavel2015}}
	\scnitem{\scncite{Altshuller2010}}
	\scnitem{\scncite{Shhedrovickij1995}}
	\scnitem{\scncite{Sapatyj1986}}
	\scnitem{\scncite{Moldovan1985}}
	\scnitem{\scncite{Letichevskij2003}}
	\scnitem{\scncite{Letichevskij2012}}
\end{scnrelfromlist}

\scntext{аннотация}{В предметной области формулированы актуальные проблемы текущего состояния технологий разработки гибридных решателей задач, предложен подход к их решению на основе Технологии OSTIS. Сформулированы принципы построения решателя задач как иерархической системы навыков, основанной на многоагентном подходе, приведены онтологии агентов и выполняемых ими действий. Сформулированы принципы синхронизации деятельности агентов, а также разработана онтология базового языка программирования для реализации программ агентов и модель интерпретатора такого языка.}

\begin{scnrelfromvector}{введение}
	\scnfileitem{Одним из ключевых компонентов \textit{интеллектуальной системы}, обеспечивающим возможность решать широкий круг \textit{задач}, является \textit{решатель задач}. Их особенностью по сравнению с другими современными \textit{программными системами} является необходимость решать \textit{задачи} в условиях, когда необходимые сведения не локализованы явно в \textit{базе знаний} \textit{интеллектуальной системы} и должны быть найдены в процессе решения \textit{задачи} на основании каких-либо критериев.}
	\scnfileitem{Говоря другими словами, если в традиционных системах при решении задачи всегда подразумевается, что есть некоторые локализованные исходные данные (\scnqq{дано}) и некоторое описание желаемого результата (\scnqq{что требуется}), то в \textit{интеллектуальной системе} в качестве исходных данных при решении большого числа \textit{задач} выступает вся имеющаяся на текущий момент в системе информация, то есть вся \textit{база знаний}. Кроме того, при невозможности решения задачи в текущем состоянии базы знаний интеллектуальная система должна иметь возможность понять, чего именно не хватает для продолжения процесса решения и попытаться добыть недостающие сведения во внешней среде (например, запросить у пользователя).}
	\scnfileitem{К настоящему времени в рамках различных направлений \textit{Искусственного интеллекта} разработано большое количество различных \textit{моделей решения задач}, каждая из которых позволяет решать задачи определенного класса. Расширение областей применения \textit{интеллектуальных систем} требует от них возможности решать так называемые \textit{комплексные задачи}, решение каждой из которых требует комбинирования нескольких моделей решения задач, при этом априори неизвестно, в каком порядке и сколько раз будет применяться та или иная модель. \textit{решатели задач}, в рамках которых комбинируются несколько \textit{моделей решения задач}, получили название \textit{гибридных решателей задач}, а интеллектуальные системы, в рамках которых комбинируются различные \textit{виды знаний} и различные \textit{модели решения задач} --- \textit{гибридных интеллектуальных систем}.}
	\begin{scnindent}
		\begin{scnrelfromset}{смотрите}
			\scnitem{\scncite{Kolesnikov2001}}
		\end{scnrelfromset}
	\end{scnindent}
	\scnfileitem{Повышение эффективности разработки и эксплуатации \textit{гибридных интеллектуальных систем} требует унификации моделей представления различных \textit{видов знаний} и \textit{моделей обработки знаний}, которая бы позволила легко интегрировать на ее основе компоненты, соответствующие различным моделям решения задач.}
\end{scnrelfromvector}

\scnsectionheader{Современное состояние, проблемы в области разработки гибридных решателей задач и предлагаемый подход к их решению}
\begin{scnsubstruct}
	\begin{scnreltovector}{конкатенация сегментов}
		\scnitem{Современное состояние технологий разработки решателей задач и требования, предъявляемые к гибридным решателям задач}
		\scnitem{Предлагаемый подход к разработке гибридных решателей задач ostis-систем и обработке информации в ostis-системах}
	\end{scnreltovector}

\scnsectionheader{Современное состояние технологий разработки решателей задач и требования, предъявляемые к гибридным решателям задач}
\begin{scnsubstruct}
	\scnheader{решение задач}
	\begin{scnsubdividing}
		\scnitem{решение задач с использованием хранимых программ}  
		\scnitem{решение задач в условиях, когда программа решения не известна}
	\end{scnsubdividing}
	
	\scnheader{решение задач с использованием хранимых программ}
	\scntext{определение}{\textit{решение задач с использованием хранимых программ} --- это решение задач, в котором предполагается, что в системе заранее присутствует программа решения задачи заданного класса и решение сводится к поиску такой программы и интерпретации ее на заданных входных данных.}
	\scntext{пример}{К системам, ориентированным на такой подход к решению задач, относятся в том числе системы, использующие:
		\begin{itemize}
			\item программы, написанные на языках программирования, относящихся как к императивной, так и к декларативной парадигме, в том числе логических и функциональных;
			\item реализации генетических алгоритмов;
			\item нейросетевые модели обработки знаний.
		\end{itemize}}
	\begin{scnindent}
		\begin{scnrelfromset}{смотрите}
			\scnitem{\scncite{Pratt2002}}
			\scnitem{\scncite{Emelyanov2003}}
			\scnitem{\scncite{Gladkov2006}}
			\scnitem{\scncite{Berkinblit1993}}
			\scnitem{\scncite{Gorban1996}}
			\scnitem{\scncite{Golovko2001}}
		\end{scnrelfromset}
	\end{scnindent}
	\scntext{примечание}{Следует отметить, что даже в случае использования хранимой \textit{программы} решение \textit{задачи} далеко не всегда тривиально, поскольку, во-первых, требуется найти такую хранимую \textit{программу} на основе некоторой спецификации, во-вторых, обеспечить ее интерпретацию.}
	
	\scnheader{решение задач в условиях, когда программа решения не известна}
	\scntext{определение}{\textit{решение задач в условиях, когда программа решения не известна} --- решение задач, в котором предполагается, что в системе необязательно присутствует готовая \textit{программа} решения для \textit{класса задач}, которому принадлежит некоторая сформулированная задача, подлежащая решению. В связи с этим необходимо применять дополнительные методы поиска путей решения задачи, не рассчитанные на какой-либо узкий \textit{класс задач} (например, разбиение задачи на подзадачи, методы поиска решений в глубину и ширину, метод случайного поиска решения и метод проб и ошибок, метод деления пополам и другие), а также различные модели \textit{логического вывода}: классические дедуктивные, индуктивные, абдуктивные; модели, основанные на \textit{нечетких логиках}, \textit{логике умолчаний}, \textit{темпоральной логике}, и многие другие.}
	\begin{scnindent}
		\begin{scnrelfromset}{смотрите}
			\scnitem{\scncite{Eremeev1997}}
			\scnitem{\scncite{Reiter1980}}
			\scnitem{\scncite{Batyrshin2001}}
			\scnitem{\scncite{Pospelov1989}}
			\scnitem{\scncite{Vagin2008}}
			\scnitem{\scncite{Polya1975}}
			\scnitem{\scncite{Khulick2001}}
			\scnitem{\scncite{Vagin2008}}
			\scnitem{\scncite{Demenkov2005}}
		\end{scnrelfromset}
	\end{scnindent}
	
	\scnheader{решатель задач}
	\scnhaselement{STRIPS}
	\scnhaselement{GPS}
	\scnhaselement{QA3}
	\scnhaselement{ППР}
	\scnhaselement{ПРИЗ}
	\begin{scnindent}
		\begin{scnrelfromset}{смотрите}
			\scnitem{\scncite{Kachro1988}}
		\end{scnrelfromset}
	\end{scnindent}
	\scnhaselement{Компьютерный решатель математических задач}
	\begin{scnindent}
		\begin{scnrelfromset}{смотрите}
			\scnitem{\scncite{Podkholzyn2008}}
		\end{scnrelfromset}
	\end{scnindent}
	\scnhaselement{Решатель задач по планиметрии НИЦ ЭВТ}
	\begin{scnindent}
		\begin{scnrelfromset}{смотрите}
			\scnitem{\scncite{Khurbatov2016}}
		\end{scnrelfromset}
	\end{scnindent}
	\scnhaselement{Программный комплекс \scnqq{УДАВ}}
	\begin{scnindent}
		\begin{scnrelfromset}{смотрите}
			\scnitem{\scncite{Vladimirov2010}}
		\end{scnrelfromset}
	\end{scnindent}
	\scntext{примечание}{Подробный обзор \textit{решателей задач}, разработанных в период до 1982 года, таких как \textit{GPS}, \textit{STRIPS}, \textit{QA3}, \textit{ПРИЗ}, \textit{ППР} приведен в книге \textit{Ephymov1982}. Среди современных работ, исследующих вопросы применения \textit{моделей решения задач}, не ориентированных на конкретную предметную область, можно выделить \textit{Raghovsky2011}. Среди наиболее заметных представителей класса \textit{интеллектуальных решателей задач}, разработанных в более поздний период, можно отметить \textit{Компьютерный решатель математических задач}, \textit{Решатель задач по планиметрии НИЦ ЭВТ}, \textit{Программный комплекс \scnqqi{УДАВ}}.} 
	\begin{scnindent}
		\begin{scnrelfromset}{смотрите}
			\scnitem{\scncite{Ephymov1982}}
			\scnitem{\scncite{Raghovsky2011}}
		\end{scnrelfromset}
	\end{scnindent}
	\scntext{примечание}{Отдельного внимания заслуживают популярные в настоящее время \textit{системы компьютерной алгебры}, такие как \textit{Wolfram Mathematica}, \textit{Maple}, \textit{MathCAD} и другие. Указанные программные комплексы обладают мощной функциональностью как для проведения различного рода вычислений и экспериментов, так и для построения на их основе систем различного назначения, например обучающих. Более подробно возможности применения систем данного семейства для решения \textit{задач} в рамках \textit{Экосистемы OSTIS} рассмотрены в \textit{Интеграция инструментов компьютерной алгебры в приложения OSTIS}.}
	\begin{scnindent}
		\begin{scnrelfromset}{смотрите}
			\scnitem{Интеграция инструментов компьютерной алгебры в приложения OSTIS}
		\end{scnrelfromset}
	\end{scnindent}
	\scntext{примечание}{Однако при всем многообразии решаемых рассмотренными системами \textit{задач} множество \textit{классов задач} ограничивается имеющимся в системе набором жестко заданных приемов и алгоритмов решения \textit{задач}, явно используемых при решении той или иной \textit{задачи}. В то же время построение сложных систем, например, систем комплексной автоматизации, невозможно без обеспечения согласованного использования различных \textit{видов знаний} и \textit{моделей решения задач} в рамках одной системы при решении одной и той же \textit{комплексной задачи}. Кроме того, становится актуальной \textit{задача} поддержки такой системы в состоянии, соответствующем текущему уровню развития технологий, дополнения ее более совершенными \textit{моделями} и \textit{методами решения задач}. При этом очевидно, что подобная реконфигурация системы должна осуществляться \textit{непосредственно в процессе эксплуатации системы}, а не требовать каждый раз, например, полной остановки всего производства или отдельных его частей.}

	\scnheader{гибридный решатель задач}
	\begin{scnrelfromlist}{требование}
		\scnfileitem{В каждый момент времени \textit{решатель задач} должен обеспечивать решение задач из оговоренного класса за оговоренное время, при этом результат решения задачи должен удовлетворять некоторым известным требованиям. Другими словами, как и в случае современных \textit{компьютерных систем}, корректность результатов решения задач на этапе разработки системы должна верифицироваться специальными методами, в том числе для этого могут быть использованы такие современные подходы, как \textit{unit-тестирование}, \textit{тестирование методом \scnqqi{черного ящика}} и другие.}
		\begin{scnindent}
			\begin{scnsubdividing}
				\scnfileitem{Для явно сформулированных \textit{задач} система всегда должна давать какой-либо ответ за оговоренное время, при этом ответ может быть отрицательным (система не смогла решить поставленную задачу), возможно, с объяснением причин, по которым решение в текущий момент оказалось невозможным. Одним из факторов безуспешности решения является выход за рамки установленного промежутка времени.}
				\scnfileitem{Если явно сформулированная \textit{задача} решена, то все \textit{информационные процессы}, направленные на ее решение, должны быть уничтожены. Особенно актуальным данное требование становится в ситуации, когда для решения одной и той же задачи параллельно используются сразу несколько подходов и заранее неизвестно, какой из них приведет к результату раньше других.}
				\scnfileitem{После решения задачи вся временная информация, сгенерированная в процессе решения этой \textit{задачи} и имеющая ценность только в контексте решения указанной \textit{задачи}, должна быть удалена из памяти.}
			\end{scnsubdividing}
		\end{scnindent}
		\scnfileitem{\textit{\textbf{гибридный решатель}} должен обеспечивать возможность \textbf{согласованного использования различных моделей решения задач} при решении одной и той же \textit{комплексной задачи} в случае необходимости.}
		\scnfileitem{\textit{решатель задач} должен быть легко \textbf{модифицируемым}, то есть трудоемкость внесения изменений в уже разработанный \textit{решатель задач} должна быть минимальна. Путями повышения модифицируемости \textit{решателя задач} являются обеспечение локальности вносимых изменений, в том числе --- за счет стратификации \textit{решателя задач} на независимые уровни и обеспечение максимальной независимости компонентов \textit{решателя задач} друг от друга, а также наличие готовых компонентов, которые могут быть встроены в \textit{решатель задач} при необходимости. При этом внесение изменений должно осуществляться \textit{непосредственно в процессе эксплуатации системы}.}
		\scnfileitem{Для того чтобы \textit{интеллектуальная система} имела возможность анализировать и оптимизировать имеющийся \textit{решатель задач}, интегрировать в его состав новые компоненты (в том числе самостоятельно), оценивать важность тех или иных компонентов и применимость их для решения той или иной задачи, спецификация \textit{решателя задач} должна быть описана языком, понятным системе, например, при помощи тех же средств, что и обрабатываемые \textit{знания}. Другими словами, \textit{интеллектуальная система} и, соответственно, \textit{решатель задач} должны обладать \textit{рефлексивностью}.}
	\end{scnrelfromlist}

	\scnheader{модель решения задач}
	\scntext{проблемы текущего состояния}{Несмотря на то что в настоящее время существует большое число \textit{моделей решения задач}, многие из которых реализованы и успешно используются на практике в различных системах, остается актуальной проблема низкой согласованности принципов, лежащих в основе реализации таких моделей, и отсутствия единой унифицированной основы для реализации и интеграции различных \textit{моделей решения задач}, что приводит к тому, что:
		\begin{itemize}
			\item затруднена возможность одновременного использования различных \textit{моделей решения задач} в рамках одной системы при решении одной и той же комплексной задачи; практически невозможно комбинировать различные модели с целью решения \textit{задачи}, для которой априори отсутствует \textit{алгоритм} ее решения;
			\item практически невозможно использовать технические решения, реализованные в одной системе, в других системах, то есть возможности использования компонентного подхода при построении \textit{решателей задач} сильно ограничены. Как следствие, велико количество дублирований аналогичных решений в разных системах;
			\item фактически отсутствуют комплексные методики и средства построения \textit{решателей задач}, которые бы обеспечивали возможность проектирования, реализации и отладки \textit{решателей задач} различного вида.
		\end{itemize}}
	\begin{scnindent}
		\begin{scnrelfromlist}{следствие}
			\scnfileitem{Высокая трудоемкость разработки каждого \textit{решателя задач}, увеличение сроков их разработки, а значит, и увеличение затрат на разработку и поддержку соответствующих \textit{интеллектуальных систем}.}
			\scnfileitem{Высокая трудоемкость внесения изменений в уже разработанные \textit{решатели задач}, то есть отсутствует или сильно затруднена возможность дополнения уже разработанного \textit{решателя задач} новыми компонентами и внесения изменений в уже существующие компоненты в процессе эксплуатации системы. Таким образом, высока трудоемкость поддержки разработанных \textit{решателей задач}.}
			\scnfileitem{Высокий уровень профессиональных требований к разработчикам \textit{решателей задач}, что обусловлено, в частности:
			\begin{itemize}
				\item Высокой сложностью существующих формализмов в области решения \textit{задач}, рассчитанных на их интерпретацию \textit{компьютерной системой}, а не человеком;
				\item Отсутствием возможности рассматривать разрабатываемые \textit{решатели задач} на разных уровнях детализации, выделения на каждом уровне достаточно независимых компонентов, что затрудняет процесс проектирования, тестирования и отладки таких \textit{решателей задач}, а также снижает эффективность попыток объединения разработчиков \textit{решателей задач} в коллективы по причине увеличения накладных расходов на согласование их деятельности;
				\item Низким уровнем информационной поддержки разработчиков и автоматизации их \textit{деятельности}.
			\end{itemize}}
		\end{scnrelfromlist}
		\begin{scnrelfromvector}{решение проблем}
			\scnfileitem{Для решения перечисленных проблем необходимо разработать комплекс моделей, методики и средств разработки \textit{гибридных решателей задач}, удовлетворяющих перечисленным ранее требованиям.}
			\scnfileitem{Исторически сложились два основных подхода к построению \textit{решателей задач} \textit{интеллектуальных компьютерных систем}.}
			\scnfileitem{Первый подход предполагает наличие в системе фиксированного \textit{решателя задач} (например, машины логического вывода), к которому впоследствии добавляется \textit{база знаний}, наполнение которой определяется \textit{предметной областью}, в которой должна работать система. Такие системы получили название \scnqq{пустых} \textit{экспертных систем} или \scnqq{оболочек} (expert system shells). Данный подход, как правило, использовался для разработки относительно несложных систем и в настоящее время не имеет широкого применения.}
			\begin{scnindent}
				\begin{scnrelfromset}{смотрите}
					\scnitem{\scncite{Jackson1998}}
					\scnitem{\scncite{AIRefBookP11990}}
				\end{scnrelfromset}
			\end{scnindent}
			\scnfileitem{Второй подход, широко используемый в настоящее время, предполагает наличие программных средств доступа к информации, хранящейся в некоторой базе, совместимых с различными популярными \textit{языками программирования}. Данный подход широко используется, например, в системах, построенных на основе стандартов \textit{W3C}, таких как \textit{RDF}, \textit{OWL}, \textit{SPARQL}, а также \textit{графовых с.у.б.д.}, таких как \textit{Neo4j}. Структура \textit{решателя задач}, построенного на базе таких средств, определяется разработчиком в каждом конкретном случае и не фиксируется какими-либо стандартами. Такой подход обладает большей гибкостью, но отсутствие унификации в структуре и процессе разработки \textit{решателей задач} приводит к отсутствию совместимости компонентов \textit{решателей задач}, созданных разными разработчиками, большому количеству дублирований одних и тех же решений, повышению накладных расходов в процессе разработки и поддержки \textit{решателя задач}. Также существует большое количество реализаций так называемых \textit{ризонеров} (semantic reasoners), осуществляющих \textit{логический вывод} на \textit{онтологиях}, представленных в формате \textit{OWL 2}, а также средств разработки и редактирования таких \textit{онтологий}. Полный список таких средств, признанных консорциумом \textit{W3C}, можно найти на сайте \textit{OWLImplementations}. Как видно из приведенной на нем таблицы, подавляющее большинство средств способно осуществлять только прямой \textit{логический вывод} на основе \textit{отношений}, описанных в \textit{онтологии}.}
			\begin{scnindent}
				\begin{scnrelfromset}{смотрите}
					\scnitem{\scncite{W3C}}
					\scnitem{\scncite{OWL}}
					\scnitem{\scncite{RDF}}
					\scnitem{\scncite{SPARQL}}
					\scnitem{\scncite{Neo4j}}
					\scnitem{\scncite{OWLImplementations}}
				\end{scnrelfromset}
			\end{scnindent}
			\scnfileitem{Среди комплексных подходов к построению \textit{решателей задач}, разрабатываемых русскоязычными авторами, можно выделить проект \textit{IACPaaS}, активно развивающийся в настоящее время. Целью данного проекта является разработка облачной платформы для построения на ее основе \textit{интеллектуальных сервисов} различного назначения. В данном проекте активно используются \textit{библиотеки многократно используемых компонентов интеллектуальных систем}. Конкретно для построения \textit{решателей задач}, а также \textit{пользовательских интерфейсов} таких систем используется \textit{многоагентный подход}. Несмотря на близость некоторых технологических решений, реализуемых в проекте \textit{IACPaaS} и в рамках \textit{Технологии OSTIS}, основной целью указанного проекта является предоставление пользователю большого числа разнородных сервисов, выбор которых осуществляется самим пользователем, в то время как одним из ключевых принципов \textit{Технологии OSTIS} является разработка общей формальной основы для интеграции различных \textit{моделей решения задач} с целью их комбинирования при решении одной и той же \textit{комплексной задачи}.}
			\begin{scnindent}
				\begin{scnrelfromset}{смотрите}
					\scnitem{\scncite{Gribova2015a}}
					\scnitem{\scncite{Gribova2011}}
				\end{scnrelfromset}
			\end{scnindent}
			\scnfileitem{Задачи интеграции различных подходов, в том числе связанных с решением задач, исследуются также в работе \textit{Phylyppov2016} и других работах тех же авторов.}
			\begin{scnindent}
				\begin{scnrelfromset}{смотрите}
					\scnitem{\scncite{Phylyppov2016}}
				\end{scnrelfromset}
			\end{scnindent}
			\scnfileitem{Компонентному проектированию \textit{интеллектуальных систем, основанных на знаниях}, посвящена работа \textit{Borisov2014}, в которой обосновывается необходимость накопления и повторного использования различных компонентов \textit{интеллектуальных систем}, предлагаются возможные решения данной проблемы с использованием \textit{онтологий}.}
			\begin{scnindent}
				\begin{scnrelfromset}{смотрите}
					\scnitem{\scncite{Borisov2014}}
				\end{scnrelfromset}
			\end{scnindent}
			\scnfileitem{Состояние работ англоязычных авторов, посвященных вопросам решения задач в \textit{системах, основанных на знаниях}, и актуальных на момент начала 1990-х годах, отражено в обзорных публикациях \textit{Dutta1993}, \textit{Pau1990}. Более поздние англоязычные работы в данной области в основном ориентированы на решение конкретных частных \textit{задач} в системах, построенных на основе стандартов \textit{W3C}.}
			\begin{scnindent}
				\begin{scnrelfromset}{смотрите}
					\scnitem{\scncite{Dutta1993}}
					\scnitem{\scncite{Pau1990}}
				\end{scnrelfromset}
			\end{scnindent}
			\scnfileitem{Таким образом, можно сказать, что существует ряд конкретных разработок в направлении построения \textit{комплексных технологий разработки интеллектуальных систем} различных классов, в том числе с использованием \textit{библиотек многократно используемых компонентов}, однако проблема разработки комплексной технологии построения \textit{гибридных решателей задач} в рамках рассмотренных подходов не решена. Во многом это обусловлено отсутствием унифицированной формальной основы для представления любых \textit{видов знаний}, в том числе различного рода программ, отсутствием строгих принципов, регламентирующих процесс построения \textit{решателей задач}, а также средств поддержки разработчиков таких \textit{решателей задач} и их компонентов.}
		\end{scnrelfromvector}
	\end{scnindent}
	\end{scnsubstruct}

	\scnsectionheader{Предлагаемый подход к разработке гибридных решателей задач ostis-систем и обработке информации в ostis-системах}
	\begin{scnsubstruct}

	\scnheader{гибридный решатель задач}
	\begin{scnrelfromlist}{принципы лежащие в основе}
		\scnfileitem{В качестве основы для построения модели гибридного \textit{решателя задач} предлагается использовать \textit{многоагентный подход}. Данный подход позволяет обеспечить основу для построения параллельных асинхронных систем, имеющих распределенную архитектуру, повысить модифицируемость и производительность разработанных \textit{решателей задач}.}
		\scnfileitem{Процесс решения любой \textit{задачи} предлагается декомпозировать на \textit{логически атомарные действия}, что также позволит обеспечить совместимость и модифицируемость \textit{решателей задач}.}
		\scnfileitem{\textit{решатель задач} (как объединенный, так и \textit{решатель задач} частного вида) предлагается рассматривать как иерархическую систему, состоящую из нескольких взаимосвязанных уровней. Такой подход позволяет обеспечить возможность проектирования, отладки и верификации компонентов на разных уровнях независимо от других уровней, что существенно упрощает задачу построения \textit{решателя задач} за счет снижения накладных расходов.}
		\scnfileitem{Предлагается записывать \textit{всю} информацию о решателе и решаемых им задачах при помощи \textit{SC-кода} в той же \textit{базе знаний}, что и собственно предметные \textit{знания} системы.}
		\begin{scnindent}
			\begin{scnrelfromlist}{включение}
				\scnfileitem{\textit{Спецификация агентов}, входящих в состав \textit{решателя задач}.}
				\scnfileitem{\textit{Спецификация методов}, интерпретируемых \textit{агентами} \textit{решателя задач}.}
				\scnfileitem{Спецификация всех \textit{информационных процессов}, выполняемых агентами в \textit{семантической памяти}, в том числе --- конструкции, обеспечивающие синхронизацию выполнения параллельных процессов.}
				\scnfileitem{Спецификация всех \textit{задач}, на решение которых направлены указанные \textit{информационные процессы}.}
			\end{scnrelfromlist}
			\scntext{примечание}{Описание всей указанной информации в единой семантической  памяти позволит, во-первых, обеспечить независимость разрабатываемых \textit{решателей задач} от \textit{ostis-платформы}, во-вторых, обеспечить возможность системы анализировать происходящие в ней процессы, оптимизировать и синхронизировать их выполнение, то есть обеспечить \textit{рефлексивность} проектируемых \textit{интеллектуальных систем}.}
			\begin{scnindent}
				\begin{scnrelfromset}{смотрите}
					\scnitem{Универсальная модель интерпретации логико-семантических моделей ostis-систем}
				\end{scnrelfromset}
			\end{scnindent}
		\end{scnindent}
	\end{scnrelfromlist}
	
	\scnheader{многоагентный подход к обработке информации}
	\scntext{примечание}{Ориентация на \textit{многоагентный подход} как основа для построения \textit{гибридных решателей задач} обусловлена следующими основными преимуществами и принципами такого подхода.}
	\begin{scnrelfromlist}{принципы лежащие в основе}
		\scnfileitem{Автономность (независимость) \textit{агентов} в рамках такой системы, что позволяет локализовать изменения, вносимые в \textit{решатель задач} при его эволюции, и снизить соответствующие трудозатраты, а также обеспечить устойчивость такой системы к отказам некоторых агентов.}
		\scnfileitem{Децентрализация обработки, то есть отсутствие единого контролирующего центра, что также позволяет локализовать вносимые в \textit{решатель задач} изменения.}
		\scnfileitem{Возможность параллельной работы разных \textit{информационных процессов}, соответствующих как одному \textit{агенту}, так и разным агентам, как следствие, --- возможность распределенного решения задач. Однако возможность параллельного выполнения \textit{информационных процессов} подразумевает наличие средств синхронизации такого выполнения, разработка которых является отдельной задачей и подробно рассматривается ниже.}
		\scnfileitem{Активность \textit{агентов} и \textit{многоагентной системы} в целом, дающая возможность при общении с такой системой не указывать явно способ решения поставленной \textit{задачи}, а формулировать задачу в \textbf{декларативном ключе}.}
	\end{scnrelfromlist}
	\begin{scnrelfromset}{смотрите}
		\scnitem{\scncite{Wooldridge2009}}
	\end{scnrelfromset}
	
	\scnheader{многоагентная система}
	\scnsuperset{модель агента}
	\begin{scnindent}
		\scntext{примечание}{\textit{модель агента} входит в состав системы и включает классификацию \textit{агентов} и набор понятий, характеризующих каждый агент в рамках системы. В настоящее время наиболее популярной является модель \textit{BDI} (belief-desire-intention), в рамках которой предполагается описывать на соответствующих языках \scnqq{убеждения}, \scnqq{желания} и \scnqq{намерения} каждого агента системы.}
	\end{scnindent}
	\scnsuperset{модель среды}
	\begin{scnindent}
		\scntext{определение}{\textit{модель коммуникации агентов} --- это модель, в рамках которой находятся агенты, на события в которой они реагируют и в рамках которой могут осуществлять некоторые преобразования.}
		\begin{scnrelfromset}{смотрите}
			\scnitem{\scncite{Weyns2007}}
			\begin{scnindent}
				\scntext{примечание}{Приводится обзор разновидностей сред для многоагентных систем.}
			\end{scnindent}
		\end{scnrelfromset}
	\end{scnindent}
	\scnsuperset{модель коммуникации агентов}
	\begin{scnindent}
		\scntext{определение}{\textit{модель коммуникации агентов} --- это модель, в рамках которой уточняется язык взаимодействия \textit{агентов} (структура и классификация сообщений) и способ передачи сообщений между \textit{агентами}.}
	\end{scnindent}	

	\scnheader{модель коммуникации агентов}
	\scnsuperset{принципы обмена сообщениями между агентами}
	\begin{scnindent}
		\scntext{определение}{принципы обмена сообщениями между агентами --- это принципы, описывающие то, каким образом эти сообщения передаются от \textit{агента} к \textit{агенту}.}
	\end{scnindent}
	\scnsuperset{классификация, семантика и прагматика сообщениями между агентами}
	\begin{scnindent}
		\scntext{определение}{\textit{классификация, семантика и прагматика сообщениями между агентами} --- \textit{смысл} передаваемой информации и цель такого взаимодействия.}
		\scntext{примечание}{В настоящее время стандартами, описывающими структуру передаваемых агентами сообщений, являются \textit{Agent Communication Language} (\textit{ACL}), разработанный сообществом \textit{FIPA}, язык \textit{KQML}. Указанные стандарты уточняют базовые компоненты каждого сообщения (кодировка, язык сообщения, используемую онтологию понятий, получателя, отправителя и так далее), не ограничивая при этом \textit{смысл} сообщения в целом. Также для коммуникации между агентами используется язык \textit{KIF}, предназначенный для обмена \textit{знаниями} между любыми программными компонентами.}
		\begin{scnindent}
			\begin{scnrelfromset}{смотрите}
				\scnitem{\scncite{ACL}}
				\scnitem{\scncite{KIF}}
				\scnitem{\scncite{Finin1994}}
			\end{scnrelfromset}
		\end{scnindent}
	\end{scnindent}
	\scnsuperset{принципы координации деятельности агентов}
	\begin{scnindent}
		\scntext{примечание}{В литературе рассматривается большое число вариантов координации деятельности \textit{агентов}. В работе \textit{Hartung2008} предлагается выделить агенты более высокого уровня (\textit{метаагенты}), \textit{задачей} которых является сбор информации от \textit{агентов} нижнего уровня и их координация, схожие идеи высказываются в работе \textit{Sims2008}. В работах \textit{Excelente-Toledo2004}, \textit{NagendraPrasad1999} предлагаются варианты автоматического выбора оптимального механизма координации \textit{агентов} для достижения общей цели. Предлагаются также социально-психологические модели координации деятельности \textit{агентов}, например, на основе некоторых общих \scnqq{законов} или эмоций. В работе \textit{Gorodetsky2015} предложен вариант онтологии коллективного поведения автономных \textit{агентов}.}
		\begin{scnindent}
			\begin{scnrelfromset}{смотрите}	
				\scnitem{\scncite{Gorodetsky2015}}
				\scnitem{\scncite{NagendraPrasad1999}}
				\scnitem{\scncite{Excelente-Toledo2004}}	
				\scnitem{\scncite{Sims2008}}
				\scnitem{\scncite{Hartung2008}}
				\scnitem{\scncite{Rumbell2012}}
				\scnitem{\scncite{Vasconcelos2009}}
			\end{scnrelfromset}
		\end{scnindent}
	\end{scnindent}
	
	\scnheader{многоагентная система}
	\begin{scnrelfromlist}{проблемы текущего состояния}
		\scnfileitem{Жесткая ориентация большинства средств на модель \textit{BDI} приводит к существенным накладным расходам, связанным с необходимостью выражения конкретной практической \textit{задачи} в системе понятий \textit{BDI}. В то же время ориентация на модель \textit{BDI} неявно провоцирует искусственное разделение языков, описывающих собственно компоненты \textit{BDI} и знания \textit{агента} о внешней среде, что приводит к отсутствию \textit{унификации представления} и, соответственно, дополнительным накладным расходам.}
		\scnfileitem{Большинство современных средств построения \textit{многоагентных систем} ориентированы на представление \textit{знаний} \textit{агента} при помощи узкоспециализированных языков, зачастую не предназначенных для представления \textit{знаний} в широком смысле. Речь при этом идет как о знаниях агента о себе самом, так и \textit{знаниях} о внешней среде. В некоторых подходах вначале строится онтология, для создания которой, однако, часто используются средства с низкой выразительной способностью, не предназначенные для построения \textit{онтологий}. В конечном итоге такой подход приводит к сильной ограниченности возможностей построенных \textit{многоагентных систем} и их несовместимости.}
		\begin{scnindent}
			\begin{scnrelfromset}{смотрите}
				\scnitem{\scncite{Evertsz2004}}
				\scnitem{\scncite{JADE}}
			\end{scnrelfromset}
		\end{scnindent}
		\scnfileitem{Абсолютное большинство современных средств предполагает, что взаимодействие \textit{агентов} осуществляется путем обмена сообщениями непосредственно от \textit{агента} к \textit{агенту} или посредством специальных коммуникационных центров, например, в случае взаимодействия \textit{агентов} в глобальной сети. Такой подход обладает существенным недостатком, связанным с тем, что в этом случае каждый \textit{агент} системы должен иметь достаточно полную информацию о других агентах в системе, что приводит к дополнительным затратам ресурсов, кроме того, добавление или удаление одного или нескольких \textit{агентов} приводит к необходимости оповещения об этом других \textit{агентов}. Данная проблема решается путем организации общения агентов по принципу \scnqq{доски объявлений}, предполагающему, что сообщения помещаются в некоторую общую для всех агентов область, при этом каждый \textit{агент} в общем случае может не знать, какому из агентов адресовано сообщение и от какого из \textit{агентов} получено то или иное сообщение. Кроме того, в построенной таким образом системе легче обеспечивается параллельное решение несвязанных друг с другом \textit{задач}, поскольку сообщения, относящиеся к одной \textit{задаче}, будут игнорироваться агентами, решающими другую задачу. Однако данный подход не исключает проблему, связанную с необходимостью разработки специализированного языка взаимодействия \textit{агентов}, который в общем случае не связан с языком, на котором описываются \textit{знания} \textit{агента} о решаемых \textit{задачах} и окружающей среде.}
		\begin{scnindent}
			\begin{scnrelfromset}{смотрите}
				\scnitem{\scncite{Omicini1999}}
				\scnitem{\scncite{Jagannathan1989}}
			\end{scnrelfromset}
		\end{scnindent}
		\scnfileitem{Многие средства построения \textit{многоагентных систем} построены таким образом, что логический уровень взаимодействия \textit{агентов} жестко привязан к физическому уровню реализации \textit{многоагентной системы}. Например, при передаче сообщений от агента к агенту разработчику \textit{многоагентной системы} необходимо помимо семантически значимой информации указывать ip-адрес компьютера, на котором расположен \textit{агент-получатель}, кодировку, с помощью которой закодирован текст сообщения, и другую техническую информацию, обусловленную исключительно особенностями текущей реализации средств.}
		\scnfileitem{В большинстве подходов среда, с которой взаимодействуют \textit{агенты}, уточняется отдельно разработчиком для каждой \textit{многоагентной системы}, что с одной стороны, расширяет возможности применения соответствующих средств, но, с другой стороны, приводит к существенным накладным расходам и несовместимости таких многоагентных систем. Кроме того, в ряде случаев разработчик также обязан учитывать особенности технической реализации средств разработки в плане их стыковки с предполагаемой средой, в роли которой может выступать, например, локальная или глобальная сеть.}
	\end{scnrelfromlist}
	\begin{scnrelfromset}{смотрите}
		\scnitem{\scncite{Bordini2007}}
		\scnitem{\scncite{Castillo2014}}
		\scnitem{\scncite{EVE}}
		\scnitem{\scncite{Boissier2013}}
		\scnitem{\scncite{GOAL}}
		\scnitem{\scncite{Evertsz2004}}
		\scnitem{\scncite{JADE}}
		\scnitem{\scncite{GAMA}}
	\end{scnrelfromset}
	\begin{scnrelfromlist}{принципы устранения недостатков}
		\scnfileitem{Коммуникацию агентов предлагается осуществлять по принципу \textit{\scnqq{доски объявлений}}, однако в отличие от классического подхода в роли сообщений выступают спецификации в общей семантической памяти выполняемых \textit{агентами} \textit{действий}, направленных на решение каких-либо задач, а в роли среды коммуникации выступает сама эта \textit{семантическая память}.}
		\begin{scnrelfromlist}{следствие}
			\scnfileitem{Исключить необходимость разработки специализированного языка для обмена сообщениями.}
			\scnfileitem{Обеспечить \scnqq{обезличенность} общения, то есть каждый из \textit{агентов} в общем случае не знает, какие еще агенты есть в системе, кем сформулирован и кому адресован тот или иной запрос. Таким образом, добавление или удаление агентов в систему не приводит к изменениям в других \textit{агентах}, что обеспечивает модифицируемость всей системы.}
			\scnfileitem{Агентам, в том числе конечному пользователю, формулировать задачи в \textit{декларативном ключе}, то есть не указывать для каждой задачи способ ее решения. Таким образом, агенту заранее не нужно знать, каким образом система решит ту или иную задачу, достаточно лишь специфицировать конечный результат.}
			\scnfileitem{Сделать средства коммуникации \textit{агентов} и синхронизации их деятельности более понятными разработчику и пользователю системы, не требующими изучения специальных низкоуровневых типов данных и форматов сообщений. Таким образом повышается доступность предлагаемых решений широкому кругу разработчиков.}
		\end{scnrelfromlist}
	\end{scnrelfromlist}
	
	\scnheader{принцип \scnqq{доски объявлений}}
	\scntext{описание}{Следует отметить, что такой подход позволяет при необходимости организовать обмен сообщениями между \textit{агентами} напрямую и, таким образом, может являться основой для моделирования многоагентных систем, предполагающих другие способы взаимодействия между \textit{агентами}.
		\begin{itemize}
		\item в роли внешней среды для агентов выступает та же \textit{семантическая память}, в которой формулируются задачи и посредством которой осуществляется взаимодействие \textit{агентов}. Такой подход обеспечивает унификацию среды для различных систем \textit{агентов}, что, в свою очередь, обеспечивает их совместимость;
		\item спецификация каждого агента описывается средствами \textit{SC-кода} в \textit{базе знаний}, что позволяет:
			\begin{itemize}
			\item минимизировать число специализированных средств, необходимых для спецификации агентов, как языковых, так и инструментальных;
			\item с одной стороны --- минимизировать необходимую в общем случае спецификацию агента, которая включает условие его инициирования и \textit{программу}, описывающую алгоритм работы \textit{агента}, с другой стороны --- обеспечить возможность произвольного расширения спецификации для каждого конкретного случая, в том числе возможность реализации модели \textit{BDI} и других;
			\end{itemize}
		\item синхронизацию деятельности \textit{агентов} предполагается осуществлять на уровне выполняемых ими процессов, направленных на решений тех или иных задач в \textit{семантической памяти}. Таким образом, каждый агент трактуется как некий абстрактный процессор, способный решать задачи определенного класса. При таком подходе необходимо решить задачу обеспечения взаимодействия параллельных асинхронных процессов в общей \textit{семантической памяти}, для решения которой можно заимствовать и адаптировать решения, применяемые в традиционной \textit{линейной памяти}. При этом вводится дополнительный класс агентов --- \textit{метаагенты}, задачей которых является решение возникающих проблемных ситуаций, таких как \textit{взаимоблокировки};
		\item каждый \textit{информационный процесс} в любой момент времени имеет ассоциативный доступ к необходимым фрагментам \textit{базы знаний}, хранящейся в семантической памяти, за исключением фрагментов, заблокированных другими процессами в соответствии  с рассмотренным ниже механизмом синхронизации. Таким образом, с одной стороны, исключается необходимость хранения каждым агентом информации о внешней среде, с другой стороны, каждый \textit{агент}, как и в классических \textit{многоагентных системах}, обладает только частью всей информации, необходимой для решения задачи.
		\end{itemize}}
		
	\scnheader{гибридный решатель задач}
	\scntext{примечание}{Важно отметить, что в общем случае невозможно априори предсказать, какие именно знания, модели и способы решения задач понадобятся системе для решения конкретной задачи. В связи с этим необходимо обеспечить, с одной стороны, возможность доступа ко всем необходимым фрагментам \textit{базы знаний} (в пределе --- ко всей \textit{базе знаний}), с другой стороны --- иметь возможность локализовать область поиска пути решения \textit{задачи}, например, рамками одной \textit{предметной области}.}
	\scntext{примечание}{Каждый из \textit{агентов} обладает набором ключевых элементов (как правило, понятий), которые он использует в качестве отправных точек при ассоциативном поиске в рамках \textit{базы знаний}. Набор таких элементов для каждого \textit{агента} уточняется на этапах проектирования \textit{решателя задач}. Уменьшение числа ключевых элементов \textit{агента} делает его более универсальным, однако снижает эффективность его работы за счет необходимости выполнения дополнительных операций.}
	\scntext{предлагаемый подход}{Кроме \textit{многоагентного подхода}, в основу принципов решения задачи в рамках \textit{Технологии OSTIS} предлагается положить ряд идей, связанных с концепцией \textit{ситуационного управления}, рассмотренной в работе \textit{Д.А. Поспелова}.}
	\begin{scnindent}
	\begin{scnrelfromset}{смотрите}
		\scnitem{\scncite{Pospelov1986}}
	\end{scnrelfromset}
	\scntext{проблема}{До настоящего времени попытки реализации указанной концепции, несмотря на ее актуальность и востребованность, сводились к частным решениям для конкретных \textit{классов задач} и, к сожалению, не получили широкого распространения. В значительной степени это обусловлено отсутствием универсальной унифицированной основы, которая бы позволила на ее базе создавать языки ситуационного управления в применении к конкретным предметным областям и, что еще более важно, повторно использовать фрагменты описаний на таких языках.}
	\begin{scnindent}
		\scntext{решение}{Данную проблему можно решить используя предлагаемый в рамках \textit{Технологии OSTIS} \textit{SC-код} и семейство \textit{онтологий верхнего уровня}, разработанных на его основе.}
		\begin{scnindent}
			\scnrelfrom{смотрите}{Технология OSTIS}
			\begin{scnindent}
				\begin{scnrelfromlist}{принципы лежащие в основе}
					\scnfileitem{\textit{SC-код} как базовый язык для описания любой информации в \textit{базе знаний} и, соответственно, для построения языков ситуационного управления на его основе.}
					\scnfileitem{\textit{Базовая денотационная семантика \textit{SC-кода}}, которая позволяет обеспечить возможность формального уточнения всех используемых понятий в виде формального набора \textit{онтологий}, что позволяет обеспечить совместимость разрабатываемых систем и возможность повторного использования их компонентов.}
					\scnfileitem{\textit{агентно-ориентированный подход} к обработке информации, предполагающий реакцию \textit{агентов} на возникновение в \textit{базе знаний} определенных \textit{ситуаций} и \textit{событий}.}
				\end{scnrelfromlist}
			\end{scnindent}
		\end{scnindent}	
	\end{scnindent}
	\end{scnindent}

	\scnheader{обработка знаний в ostis-системах}
	\begin{scnrelfromlist}{достоинства}
		\scnfileitem{Поскольку обработка осуществляется \textit{агентами}, которые обмениваются сообщениями только через общую память, добавление нового агента или исключение (деактивация) одного или нескольких существующих \textit{агентов}, как правило, не приводит к изменениям в других \textit{агентах}, поскольку агенты не обмениваются сообщениями напрямую.}
		\scnfileitem{Инициирование \textit{агентов} осуществляется децентрализованно и чаще всего независимо друг от друга, таким образом, даже существенное расширение числа агентов в рамках одной системы не приводит к ухудшению ее производительности.}
		\scnfileitem{Спецификации \textit{агентов} и, как будет показано ниже, их программы могут быть записаны на том же языке, что и обрабатываемые знания, что существенно сокращает перечень специализированных средств, предназначенных для проектирования таких \textit{агентов} и их коллективов, а также их анализа, верификации и оптимизации, и упрощает разработку системы за счет использования более универсальных компонентов.}
	\end{scnrelfromlist}

\end{scnsubstruct}

\end{scnsubstruct}

\scnheader{решатель задач ostis-системы}
\scnidtf{совокупность всех навыков, которыми обладает ostis-система на текущий момент времени}
\scnidtf{иерархическая система навыков, которыми обладает ostis-система}
\scnrelto{семейство подмножеств}{навык}
\scntext{примечание}{С учетом того тезиса, что существуют \textit{методы} интепретации других \textit{методов} и, следовательно, иерархия \textit{методов}, а также, соответственно, иерархия \textit{навыков}, можно уточнить и понятие решателя задач, как \uline{иерархической системы навыков}. Таким образом, определим \textit{решатель задач ostis-системы} определяется как совокупность всех \textit{навыков}, которыми обладает ostis-система на текущий момент времени.}
\scntext{примечание}{Предлагаемый в рамках \textit{Технологии OSTIS} подход к построению решателей задач позволяет обеспечить их модифицируемость, что, в свою очередь, позволяет \textit{ostis-системе} при необходимости легко приобретать новые \textit{навыки}, модифицировать (совершенствовать) уже имеющиеся и даже избавляться от некоторых навыков с целью повышения производительности системы. Таким образом, имеет смысл говорить не о жестко фиксированном решателе задач, который разрабатывается один раз при создании первой версии системы и далее не меняется, а о совокупности навыков, фиксированной в каждый текущий момент времени, но постоянно эволюционирующей.}
\scnsuperset{объединенный решатель задач ostis-системы}
\begin{scnindent}
	\scnidtf{полный решатель задач ostis-системы}
	\scnidtf{интегрированный решатель задач ostis-системы}
	\scnidtf{решатель задач ostis-системы, реализующий все ее функциональные возможности, как основные, так и вспомогательные}
	\scntext{пояснение}{В общем случае \textit{объединенный решатель задач ostis-системы} решает задачи, связанные с:
		\begin{itemize}
			\item обеспечением основных функциональных возможностей системы (например, решение явно сформулированных задач по требованию пользователя);
			\item обеспечением корректности и оптимизацией работы самой ostis-системы (перманентно на протяжении всего жизненного цикла ostis-системы);
			\item обеспечением повышения квалификации конечных пользователей и разработчиков ostis-системы;
			\item обеспечением автоматизации развития и управления развитием ostis-системы.
		\end{itemize}}
\end{scnindent}
\scnsuperset{гибридный решатель задач ostis-системы}
\begin{scnindent}
	\scnidtf{решатель задач ostis-системы, реализующий две и более модели решения задач}
\end{scnindent}

\scnheader{машина обработки знаний}
\scnsubset{sc-агент}
\scntext{пояснение}{Под \textit{машиной обработки знаний} будем понимать совокупность интерпретаторов всех \textit{навыков}, составляющих некоторый \textit{решатель задач}. С учетом многоагентного подхода к обработке информации, используемого в рамках Технологии OSTIS, \textit{машина обработки знаний} представляет собой \textit{sc-агент} (чаще всего --- \textit{неатомарный sc-агент}), в состав которого входят более простые sc-агенты, обеспечивающие интерпретацию соответствующего множества \textit{методов}. Таким образом, \textit{машина обработки знаний} в общем случае представляет собой иерархическую систему \textit{sc-агентов}.}

\scnheader{решатель задач ostis-системы}
\scnhaselement{Решатель задач Метасистемы OSTIS}
\scnsuperset{решатель задач вспомогательной ostis-системы}
\begin{scnindent}
	\scnsuperset{решатель задач интерфейса компьютерной системы}
		\begin{scnindent}
			\begin{scnsubdividing}
				\scnitem{решатель задач пользовательского интерфейса компьютерной системы}
				\scnitem{решатель задач интерфейса компьютерной системы с другими компьютерными системами}
				\scnitem{решатель задач интерфейса компьютерной системы с окружающей средой}
			\end{scnsubdividing}
		\end{scnindent}
	\scnsuperset{решатель задач ostis-подсистемы поддержки проектирования компонентов определенного класса}
	\begin{scnindent}
		\scnsuperset{решатель задач ostis-подсистемы поддержки проектирования баз знаний}
			\begin{scnindent}
				\scnsuperset{решатель задач повышения качества базы знаний}
					\begin{scnindent}
						\scnsuperset{решатель задач верификации базы знаний}
							\begin{scnindent}
								\scnsuperset{решатель задач поиска и устранения некорректностей в базе знаний}
								\scnsuperset{решатель задач поиска и устранения неполноты}
							\end{scnindent}
						\scnsuperset{решатель задач оптимизации структуры базы знаний}
						\scnsuperset{решатель задач выявления и устранения информационного мусора}
					\end{scnindent}
			\end{scnindent}
		\scnsuperset{решатель задач ostis-подсистемы поддержки проектирования решателей задач ostis-систем}
		\begin{scnindent}
			\begin{scnsubdividing}
				\scnitem{решатель задач ostis-подсистемы поддержки проектирования программ обработки знаний}
				\scnitem{решатель задач ostis-подсистемы поддержки проектирования агентов обработки знаний}
			\end{scnsubdividing}
		\end{scnindent}
	\end{scnindent}
	\scnsuperset{решатель задач подсистемы управления проектирования компьютерных систем и их компонентов}
\end{scnindent}
\scnsuperset{решатель задач самостоятельной ostis-системы}

\scnheader{Классификация решателей задач ostis-систем по типу интерпретируемой модели решения задач}
\begin{scnsubstruct}
\scnheader{решатель задач ostis-системы}
\scnsuperset{решатель задач с использованием хранимых методов}
\begin{scnindent}
	\scnidtf{решатель, способный решать задачи тех классов, для которых на данный момент времени известен соответствующий метод решения}
	\scnsuperset{решатель задач на основе нейросетевых моделей}
	\scnsuperset{решатель задач на основе генетических алгоритмов}
	\scnsuperset{решатель задач на основе императивных программ}
	\begin{scnindent}
		\scnsuperset{решатель задач на основе процедурных программ}
		\scnsuperset{решатель задач на основе объектно-ориентированных программ}
	\end{scnindent}
	\scnsuperset{решатель задач на основе декларативных программ}
	\begin{scnindent}
		\scnsuperset{решатель задач на основе логических программ}
		\scnsuperset{решатель задач на основе функциональных программ}
	\end{scnindent}
\end{scnindent}
\scnsuperset{решатель задач в условиях, когда метод решения задач данного класса в текущий момент времени не известен}
\begin{scnindent}
	\scnidtf{решатель, реализующий стратегии решения задач, позволяющие породить метод решения задачи, который в текущий момент времени не известен ostis-системе}
	\scnidtf{решатель, использующий для решения задач метаметоды, соответствующие более общим классам задач по отношению к заданной}
	\scnidtf{решатель задач, позволяющий породить метод, который является частным по отношению к какому-либо известному ostis-системе методу и интерпретируется соответствующей машиной обработки знаний}
	\scnsuperset{решатель, реализующий стратегию поиска путей решения задачи в глубину}
	\scnsuperset{решатель, реализующий стратегию поиска путей решения задачи в ширину}
	\scnsuperset{решатель, реализующий стратегию проб и ошибок}
	\scnsuperset{решатель, реализующий стратегию разбиения задачи на подзадачи}
	\scnsuperset{решатель, реализующий стратегию решения задач по аналогии}
	\scnsuperset{решатель, реализующий концепцию интеллектуального пакета программ}
\end{scnindent}
\end{scnsubstruct}

\scnheader{Классификация машин обработки знаний, которые в общем случае могут соответствовать одним и тем же фрагментам базы знаний, но при этом в совокупности с ними образовывать разные навыки и соответственно разные решатели задач}
\begin{scnsubstruct}
\scnheader{машина обработки знаний}
\scnsuperset{машина логического вывода}
\begin{scnindent}
	\scnsuperset{машина дедуктивного вывода}
	\begin{scnindent}
		\scnsuperset{машина прямого дедуктивного вывода}
		\scnsuperset{машина обратного дедуктивного вывода}
	\end{scnindent}
	\scnsuperset{машина индуктивного вывода}
	\scnsuperset{машина абдуктивного вывода}
	\scnsuperset{машина нечеткого вывода}
	\scnsuperset{машина вывода на основе логики умолчаний}
	\scnsuperset{машина логического вывода с учетом фактора времени}
\end{scnindent}
\end{scnsubstruct}

\scnheader{Классификация решателей задач ostis-систем по типу решаемой задачи (цели решения задачи)}
\begin{scnsubstruct}
\scnheader{решатель задач ostis-системы}
\scnsuperset{решатель задач информационного поиска}
\begin{scnindent}
	\begin{scnsubdividing}
		\scnitem{решатель задач поиска информации, удовлетворяющей заданным критериям}
		\scnitem{решатель задач поиска информации, не удовлетворяющей заданным критериям}
	\end{scnsubdividing}
\end{scnindent}
\scnsuperset{решатель явно сформулированных задач}
\begin{scnindent}
	\scnidtf{решатель задач, для которых явно сформулирована цель}
	\scnsuperset{решатель задач поиска или вычисления значений заданного множества величин}
	\scnsuperset{решатель задач установления истинности заданного логического высказывания в рамках заданной формальной теории}
	\scnsuperset{решатель задач формирования доказательства заданного высказывания в рамках заданной формальной теории}
	\scnsuperset{машина верификации ответа на указанную задачу}
	\scnsuperset{машина верификации решения указанной задачи}
	\begin{scnindent}
		\scnsuperset{машина верификации доказательства заданного высказывания в рамках заданной формальной теории}
	\end{scnindent}
\end{scnindent}
\scnsuperset{решатель задач классификации сущностей}
\begin{scnindent}
	\scnsuperset{машина соотнесения сущности с одним из заданного множества классов}
	\scnsuperset{машина разделения множества сущностей на классы по заданному множеству признаков}
\end{scnindent}
\scnsuperset{решатель задач синтеза информационных конструкций}
\begin{scnindent}
	\scnsuperset{решатель задач синтеза естественно-языковых текстов}
	\scnsuperset{решатель задач синтеза изображений}
	\scnsuperset{решатель задач синтеза сигналов}
	\begin{scnindent}
		\scnsuperset{решатель задач синтеза речи}
	\end{scnindent}
\end{scnindent}
\scnsuperset{решатель задач анализа информационных конструкций}
\begin{scnindent}
	\scnsuperset{решатель задач анализа естественно-языковых текстов}
	\begin{scnindent}
		\scnsuperset{решатель задач понимания естественно-языковых текстов}
		\scnsuperset{решатель задач верификации естественно-языковых текстов}
	\end{scnindent}
	\scnsuperset{решатель задач анализа изображений}
	\begin{scnindent}
		\scnsuperset{решатель задач сегментации изображений}
		\scnsuperset{решатель задач понимания изображений}
	\end{scnindent}
	\scnsuperset{решатель задач анализа сигналов}
	\begin{scnindent}
		\scnsuperset{решатель задач анализа речи}
		\begin{scnindent}
			\scnsuperset{решатель задач понимания речи}
		\end{scnindent}
	\end{scnindent}
\end{scnindent}
\end{scnsubstruct}

\scnheader{Язык SCP} 
\scntext{примечание}{\textit{Язык SCP} позволяет установить границу между логико-семантической моделью \textit{ostis-системы} и \textit{ostis-платформой}. В связи с этим будем считать платформенно-независимыми \textit{абстрактные sc-агенты}, реализованные на \textit{Языке SCP} или более высокоуровневых языках на его основе, а платформенно-зависимыми \textit{абстрактные sc-агенты}, которые реализованы на уровне \textit{ostis-платформы} (например, с целью повышения их производительности). В то же время существует ряд \textit{абстрактных sc-агентов}, которые принципиально не могут быть реализованы на \textit{Языке SCP}.}


\scnsectionheader{Актуальные проблемы и перспективы развития технологий разработки гибридных решателей задач}
\begin{scnsubstruct}
\scntext{примечание}{В предметной области был детально рассмотрен подход к построению \textit{решателей задач}, позволяющий решить ряд фундаментальных проблем в области построения \textit{решателей задач}, таких как обеспечение совместимости различных \textit{решателей задач} и их компонентов, а также обеспечение обучаемости (модифицируемости и рефлексивности) самих \textit{решателей задач}. В то же время существует существует ряд проблем, остающихся актуальными и требующих решения.}
\begin{scnsubdividing}
\scnfileitem{Первая проблема связана с отсутствием достаточно строгой формализованной классификации задач, решаемых интеллектуальными системами, отсутствием унификации описания задач и классов задач, описания целей, хода и результата решения задачи, методов решения задач, связей между классами задач и методами решения задач данного класса. Решение данной проблемы, с одной стороны, позволит обеспечить возможность глубокой интеграции всевозможных \textit{моделей решения задач} различных классов и возможность облегчить процесс интеграции новых моделей решения задач в интеллектуальную систему, а с другой стороны, станет предпосылкой для решения других проблем, описанных ниже.}
\scnfileitem{Вторая проблема заключается в том, что на настоящий момент основное внимание в области разработки \textit{гибридных решателей задач} уделено снижению трудоемкости интеграции различных компонентов решателя задач в \textit{интеллектуальную систему} и реализации возможности накопления многократно используемых компонентов \textit{решателей задач}, однако в общем случае не говорится о том, как конкретно \textit{интеллектуальная система} будет применять те или иные компоненты при решении задач конкретных классов. Таким образом, построение общего плана решения задачи, то есть выбор методов решения задач, определение порядка их применения и выбор исходных данных (аргументов) для применения того или иного метода, фактически определяется разработчиком на этапе проектирования системы или на этапе ее эволюции в процессе эксплуатации. Предпосылкой для решения данной проблемы является решение ранее рассмотренной проблемы унификации представления задач различных классов и методов их решения. Решение же рассматриваемой проблемы предполагает разработку комплекса \textit{стратегий решения задач} (или \textit{метаметодов решения задач}), которые позволят \textit{интеллектуальной системе} самостоятельно формировать план решения задачи с учетом имеющихся в системе методов решения задач и, при возможности, даже запрашивать недостающие для решения задачи компоненты в соответствующих библиотеках. Следует отметить, что попытки разработки универсальных высокоуровневых подходов к решению задач предпринимались еще на заре развития \textit{Искусственного интеллекта}, в 1950-60ые годы, однако не увенчались успехов и вскоре прекратились. Во многом это связано с отсутствием на тот момент унифицированных моделей представления и обработки знаний, которые в настоящий момент предлагаются в рамках \textit{Технологии OSTIS}.}
\scnfileitem{Еще одна актуальная проблема, тесно связанная с рассмотренными выше, заключается в том, что интеллектуальные системы часто вынуждены решать задачи в условиях так называемых не-факторов, то есть неполноты описания задачи и возможных путей ее решения, нечеткости и некорректности имеющихся знаний, отсутствия критериев для оценки оптимальности полученного решения и т.д. В особенности это актуально при решении поведенческих задач, связанных с изменением состояния объектов среды, внешней по отношению к интеллектуальной системе. Для решения задач в подобных условиях интеллектуальная система должна не только обладать достаточным набором компонентов решателя задач, реализующих модели решения задач в условиях наличия не-факторов (нечеткие логические модели, модели машинного обучения, генетические алгоритмы и так далее), но и реализовывать \textit{стратегии решения задач}, которые бы позволили принимать решения и формировать \textit{план решения задачи} в такого рода условиях.}
\begin{scnindent}
	\begin{scnrelfromset}{смотрите}
		\scnitem{\scncite{Narinjani2004}}
	\end{scnrelfromset}
\end{scnindent}
\scnfileitem{В случае же распределенного коллектива интеллектуальных систем важнейшей проблемой является не просто обеспечение возможности решения задач таким коллективом в текущий момент времени, а перманентная поддержка семантической совместимости и, как следствие, интероперабельности систем, входящих в такой коллектив на протяжении всего их жизненного цикла. Очевидно, что каждая из систем, входящих в такой коллектив, и, соответственно, ее \textit{решатель задач} может эволюционировать независимо от других систем, но при этом всегда должна сохраняться \textit{интероперабельность} между системами, в противном случае решение задач в таком коллективе станет невозможным. Решение данной проблемы предполагает разработку методов перманентного анализа \textit{семантической совместимости} распределенного коллектива взаимодействующих интеллектуальных систем, выявления и устранения проблем.}
\end{scnsubdividing}
\begin{scnindent}
	\begin{scnrelfromlist}{решение проблем}
		\scnfileitem{Необходимо разработать комплексную онтологию действий, задач и методов их решения, а также онтологию \textit{гибридных решателей задач} на основе которой уточнить понятие решателя и его архитектуру. На основе первой версии \textit{Глобальной предметной области действий и задач и соответствующей ей онтологии методов и технологий} предлагается разработать комплексную онтологию действий и задач, решаемых \textit{ostis-системами}.}
		\begin{scnindent}
			\begin{scnrelfromset}{смотрите}
				\scnitem{Формализация понятий действия, задачи, метода, средства, навыка и технологии}
			\end{scnrelfromset}
		\end{scnindent}
		\scnfileitem{Необходимо разработать комплекс унифицированных обобщенных стратегий (метаметодов) решения задач в интеллектуальных системах, позволяющий интеллектуальной системе самостоятельно формировать план решения задачи с учетом имеющихся в системе \textit{методов решения задач}. В основу разрабатываемых стратегий кроме опыта аналогичных работ предлагается внести также некоторые общеметодологические идеи, связанные с \textit{Теорией бихевиоризма} и набирающими популярность идеями ее применения в информатике, ТРИЗ, а также \textit{СМД-методологией}, предложенной школой Г. П. Щедровицкого.}
		\begin{scnindent}
			\begin{scnrelfromset}{смотрите}
				\scnitem{\scncite{Pavel2015}}
				\scnitem{\scncite{Altshuller2010}}
				\scnitem{\scncite{Cao2010}}
				\scnitem{\scncite{Cao2014}}
				\scnitem{\scncite{Shhedrovickij1995}}
			\end{scnrelfromset}
		\end{scnindent}
		\scnfileitem{Необходимо разработать онтологическую модель формирования плана решения задачи и управления процессом решения задач в гибридных решателях задач в условиях различных не-факторов и отсутствия четких критериев оценки оптимальности полученного решения. Для разработки данной модели предлагается адаптировать теорию \textit{ситуационного управления}, и реализовать ее в контексте семантической теории \textit{решателей задач}, разрабатываемой в рамках \textit{Технологии OSTIS}.}
		\begin{scnindent}
			\begin{scnrelfromset}{смотрите}
				\scnitem{\scncite{Pospelov1986}}
			\end{scnrelfromset}
		\end{scnindent}
		\scnfileitem{Необходимо разработать комплексную онтологическую модель управления информационными процессами решения задач в интеллектуальных системах, построенных на базе унифицированных семантических моделей представления и обработки информации.}
		\scnfileitem{Необходимо разработать онтологическую модель платформы интерпретации унифицированных семантических моделей представления и обработки информации (\textit{ostis-платформы}).}
		\begin{scnindent}
			\begin{scnrelfromset}{смотрите}
				\scnitem{Универсальная модель интерпретации логико-семантических моделей ostis-систем}
			\end{scnrelfromset}
		\end{scnindent}
		\scnfileitem{Необходимо разработать комплексную иерархическую модель \textit{гибридного решателя задач}, основанную на многоагентном подходе и учитывающую необходимость решения задач как в рамках одиночных интеллектуальных систем, так и в рамках \uline{распределенных} \uline{коллективов интероперабельных} \uline{интеллектуальных систем}.}
		\scnfileitem{Необходимо разработать комплекс методов анализа качества \textit{гибридных решателей задач} и их компонентов.}
		\scnfileitem{Необходимо разработать комплекс методик и средств поддержки проектирования \textit{гибридных решателей задач}.}
		\begin{scnindent}
			\begin{scnrelfromset}{смотрите}
				\scnitem{Методика и средства компонентного проектирования решателей задач ostis-систем}
			\end{scnrelfromset}
		\end{scnindent}
	\end{scnrelfromlist}
	\scntext{примечание}{Рассмотренные проблемы связаны в первую очередь с процессом решения конкретной задачи интеллектуальной системой. В то же время очевидно, что в каждый момент времени интеллектуальная система вынуждена параллельно решать несколько задач, которые могут быть связаны как с непосредственным функциональным назначением системы, так и с обеспечением жизнедеятельности и эволюции самой системы. Во втором случае имеются в виду, в частности, задачи, связанные с актуализацией имеющихся у нее сведений о внешнем мире, поиском и устранением ошибок в базе знаний, оптимизацией структуры \textit{базы знаний} и \textit{решателя задач} системы, поиском и устранением информационного мусора и многие другие. При этом разные задачи могут иметь разный приоритет, который может меняться в зависимости от ситуации даже в процессе ее решения. В то же время, в ситуации, когда априори не известно, какой из возможных способов решения задачи окажется наиболее эффективным, может оказаться целесообразным параллельное использование нескольких подходов к решению одной и той же задачи. Таким образом, актуальной является проблема организации управления информационными процессами решения задач в интеллектуальной системе и взаимодействия параллельно выполняемых информационных процессов с учетом приоритетности процессов, возможности отслеживать текущее состояние \textit{информационных процессов}, порождать, приостанавливать и уничтожать информационные процессы. Для решения данной проблемы целесообразно заимствовать решения, широко используемые в традиционных компьютерных системах, в частности, реализуемые в современных операционных системах, и адаптировать их к специфике решения задач в интеллектуальных системах. Важно отметить, что реализация модели управления информационными процессами на основе общих унифицированных моделей обработки информации, предлагаемых в рамках \textit{Технологии OSTIS}, позволит сделать одни информационные процессы объектом анализа других информационных процессов, что, в свою очередь, даст возможность анализировать ход решения задачи непосредственно в процессе решения, оценивать эффективность тех или иных методов решения задач, накапливать наиболее удачные решения для применения в дальнейшем для решения аналогичных задач и многое другое.}
	\begin{scnrelfromvector}{примечание}
		\scnfileitem{Решение перечисленных проблем позволит разработать принципиально новую иерархическую модель \textit{гибридного решателя задач}, обладающую рядом существенных преимуществ, которая, в свою очередь, должна будет интерпретироваться на каких-либо платформах. Без унификации требований к \textit{ostis-платформе} и четкого разделения платформенно-независимой модели системы (и в частности решателя) и \textit{ostis-платформы} невозможно говорить о реализации модели \textit{решателя задач}, реализующей рассмотренные выше идеи. Это приведет к необходимости дублирования одних и тех же компонентов модели для разных платформ, значительно усложнит интеграцию компонентов \textit{решателя задач}, поскольку потребует учета при такой интеграции особенностей каждой \textit{ostis-платформы}. Кроме того, четкое разделение уровня модели системы и уровня \textit{ostis-платформы} даст возможность независимо друг от друга развивать различные платформы и модели интеллектуальных систем. Таким образом, предлагается сформулировать унифицированные требования к \textit{ostis-платформе}, а также построить общую модель такой \textit{ostis-платформы}, удовлетворяющую указанным требованиям.}
		\begin{scnindent}
			\begin{scnrelfromset}{смотрите}
				\scnitem{Универсальная модель интерпретации логико-семантических моделей ostis-систем}
			\end{scnrelfromset}
		\end{scnindent}
		\scnfileitem{С другой стороны, как уже было сказано, \textit{решатель задач} представляет собой сложную систему, ориентированную на работу со знаниями, а не с данными, в отличие от современных программных систем, в которых изначально известно, где конкретно локализованы нужные данные и в какой форме они представлены. В связи с этим, применение для разработки интеллектуальных систем современных программно-аппаратных платформ, ориентированных на адресный доступ к хранящимся в памяти данным, не всегда оказывается эффективным, поскольку при разработке интеллектуальных систем фактически приходится моделировать нелинейную память на базе линейной. Повышение эффективности решения задач интеллектуальными системами требует разработки специализированных платформ, в том числе аппаратных, ориентированных на унифицированные семантические модели представления и обработки информации. В качестве основы для таких разработок предлагается использовать предложенную в рамках \textit{Технологии OSTIS} общую концепцию \textit{ассоциативного семантического компьютера}, \textit{семантической памяти} и базового языка программирования, ориентированного на обработку информации в такой памяти, и дополнить их идеями \textit{волновых языков программирования}, \textit{инсерционного программирования} и других подходов, направленными на повышение эффективности обработки знаний, в том числе на аппаратном уровне.}
		\begin{scnindent}
			\begin{scnrelfromset}{смотрите}
				\scnitem{Ассоциативные семантические компьютеры для ostis-систем}
			\end{scnrelfromset}
		\end{scnindent}
	\end{scnrelfromvector}
\end{scnindent}
\end{scnsubstruct}

\bigskip

\scnheader{Принципы решения задач распределенными коллективами ostis-систем}
\scntext{примечание}{Разработка \textit{решателей задач} \textit{интеллектуальных систем} на настоящий момент как правило рассматриваются в контексте одиночных (самостоятельных) интеллектуальных систем, функционирующих в некоторой среде (частью которой является и пользователь, если он есть). В то же время очевидна тенденция современных информационных технологий к переходу от одиночных систем к коллективам распределенных взаимодействующих компьютерных систем, в частности, к распределенному хранению данных и распределенным вычислениям.}
\begin{scnsubdividing}
	\scnfileitem{В случае интеллектуальных компьютерных систем важнейшим свойством систем, входящих в такие коллективы, становится \uline{\textit{интероперабельность}}, то есть способность системы к согласованному взаимодействию с другими подобными системами с целью решения каких-либо задач. Таким образом, особо актуальным является переход от разработки \textit{решателей задач} отдельно взятых интеллектуальных систем к решателям задач взаимодействующих \textit{интероперабельных интеллектуальных систем}, включая разработку принципов решения задач в таких распределенных коллективах с учетом решения всех обозначенных выше проблем.}
	\scnfileitem{Важно отметить, что полностью отказаться от распределенности при решении задач даже в сравнительно простых прикладных системах нельзя, поскольку часто интеллектуальные системы вынуждены использовать различные датчики и эффекторы, которые с точки зрения общей архитектуры являются некоторыми внешними модулями (внешними агентами) и, таким образом, привносят распределенность в общую архитектуру системы.
		\\Для решения данной проблемы предлагается рассмотреть такую систему взаимодействующих \textit{интеллектуальных компьютерных систем} как \textit{многоагентную систему} и уточнить принцип поведения \textit{агентов} в такой системе.}
	\scnfileitem{Таким образом, можно говорить о двух видах многоагентных систем в рамках \textit{Технологии OSTIS}:
		\begin{itemize}
			\item внутренняя система sc-агентов над общей sc-памятью в рамках некоторой ostis-системы;
			\item распределенная система ostis-систем в рамках Экосистемы OSTIS.
		\end{itemize}}
	\scnfileitem{В обоих случаях можно говорить об \uline{иерархии агентов}:
		\begin{itemize}
			\item в рамках внутренней системы sc-агентов выделяются \textit{атомарные абстрактные sc-агенты} и \textit{неатомарные абстрактные sc-агенты}, кроме того существует иерархия sc-агентов с точки зрения языка интерпретации методов;
			\item в рамках \textit{Экосистемы OSTIS} выделяются как \textit{индивидуальные ostis-системы}, так и \textit{коллективные ostis-системы}, которые в свою очередь могут состоять как из \textit{индивидуальных ostis-систем}, так и \textit{коллективных ostis-систем}.
		\end{itemize}}
	\begin{scnindent}
		\begin{scnrelfromset}{смотрите}
			\scnitem{Семантически совместимые ostis-системы}
			\scnitem{Внутренние агенты, выполняющие действия в sc-памяти}
		\end{scnrelfromset}
	\end{scnindent}
	\scnfileitem{Ключевым отличием \textit{распределенной системы ostis-систем} от \textit{внутренней системы sc-агентов} в рамках \textit{индивидуальной ostis-системы} является отсутствие общей памяти, хранящей общую для всех \textit{sc-агентов} \textit{базу знаний} и выступающей в роли среды для коммуникации \textit{sc-агентов}. В общем случае в качестве средства коммуникации между агентами в рамках выделенных систем агентов может использоваться:
		\begin{itemize}
			\item Общая нераспределенная (монолитная) память, как в случае \textit{sc-агентов} над \textit{sc-памятью};
			\item Общая распределенная память. В этом случае с логической точки зрения агенты могут считать, что по-прежнему работают над общей памятью, в рамках которой хранится вся доступная база знаний, однако реально \textit{база знаний} будет распределена между несколькими \textit{ostis-системами} и выполняемые преобразования должны будут синхронизироваться между этими ostis-системами;
			\item Специализированные каналы связи. Очевидно, что при решении задачи в распределенном коллективе \textit{ostis-систем} должны существовать языковые и технические средства, позволяющие осуществлять передачу сообщений от одной \textit{ostis-системы} к другой.
		\end{itemize}}
	\scnfileitem{Все перечисленные средства коммуникации в зависимости от класса решаемой задачи, требуемых для ее решения \textit{знаний} и \textit{навыков}, а также существующего (доступного) в данный момент набора \textit{ostis-систем} могут \uline{комбинироваться}.}
	\scnfileitem{В основу решения задач в рамках \textit{распределенного коллектива ostis-систем} предлагается положить идею максимально возможной \uline{унификации} и \uline{конвергенции} принципов решения задач в рамках \textit{индивидуальной ostis-системы} и \textit{распределенного коллектива ostis-систем}. Такой подход обладает следующим важным достоинством: если общие принципы решения задач не зависят от того, какой конкретно набор \textit{ostis-систе}м участвует в решении той или иной задачи, то становится возможным легко переходить от \textit{индивидуальной ostis-системы} к \textit{распределенному коллективу ostis-систем} при ее усложнении без необходимости существенно пересматривать коллектив \textit{агентов}, входящих в состав такой \textit{ostis-системы} и заново продумывать используемый подход к решению задач того или иного класса.}
	\scnfileitem{Для перехода от \textit{индивидуальной ostis-системы} к \textit{коллективной ostis-системе} достаточно выполнить несколько шагов.}
	\begin{scnindent}
		\begin{scnsubdividing}
			\scnfileitem{Разделить множество классов задач, решаемых данной \textit{ostis-системой}, на семейство подмножеств, каждое из которых обладает некоторой логической целостностью, критерии которой в общем случае определяются разработчиком. При этом указанные подмножества могут пересекаться, но при объединении должны давать исходное множество, таким образом необходимо построить одно из возможных \textit{покрытий*} для множества классов задач, решаемых данной \textit{ostis-системой}.}
			\scnfileitem{Для каждого из выделенных подмножеств необходимо сформировать множество \textit{знаний} и \textit{навыков}, необходимых для решения задач данного множества классов. При этом в общем случае может оказаться необходимым пересмотр иерархии навыков и соответствующих им sc-агентов, в частности, преобразование некоторых атомарных sc-агентов в неатомарные. Теоретически избежать такой ситуации невозможно, однако подобные ситуации можно практически исключить на этапе проектирования решателей задач индивидуальных ostis-систем, делая иерархию агентов достаточно глубокой и ставя в соответствие \textit{атомарным sc-агентам} такие \textit{классы задач}, разделение которых на подклассы с практической точки зрения не имеет смысла. 
				\\Аналогичная ситуация может возникнуть и при выделении фрагментов \textit{базы знаний}. В этом случае может потребоваться пересмотр иерархии \textit{предметных областей} и \textit{онтологий} и, возможно, выделение новых предметных областей. Как и в случае с \textit{решателями задач}, избежать такой ситуации на практике возможно в случае, если иерархия предметных областей будет достаточно глубокой для того, чтобы выделение более частных предметных областей было практически нецелесообразным.}
			\scnfileitem{Каждое сформированное таким образом множество знаний и навыков становится соответственно базой знаний и решателем задач новой ostis-системы, которая будет способна реализовать только часть функциональных возможностей исходной ostis-системы.}
		\end{scnsubdividing}
		\scntext{примечание}{Такое разделение может выполняться итерационно и для полученных \textit{ostis-систем} в общем случае неограниченное количество раз, создавая на каждой итерации новое \scnqq{поколение} ostis-систем, полученное путем декомпозиции исходной \textit{ostis-системы}.}
	\end{scnindent}
	\scnfileitem{Таким образом, предлагаемая идея унификации принципов решения задач в \textit{ostis-системах} любого рода позволяет 
		\begin{itemize}
			\item с практической точки зрения снять ограничение на расширение функциональных возможностей (обучение) не только \textit{индивидуальной ostis-системы}, но и \textit{коллективной ostis-системы}, позволяя, таким образом, постоянно наращивать функциональные возможности \textit{Экосистемы OSTIS} в целом. 
			\item с теоретической (архитектурной) точки зрения говорить о \uline{фрактальном} характере не только внутренней организации \textit{ostis-систем} но и \textit{коллективов ostis-систем}, что, в свою очередь, позволяет обеспечить возможность наследования и других принципов построения \textit{индивидуальных ostis-систем} в \textit{распределенных коллективах ostis-систем}, включая, например, методику проектирования \textit{ostis-систем} и их компонентов и соответствующие средства, а также принципы синхронизации соответствующих sc-агентам параллельных \textit{информационных процессов}.
		\end{itemize}}
	\scnfileitem{В основе взаимодействия \textit{sc-агентов} в рамках \textit{индивидуальной ostis-системы} лежит уточненный принцип \scnqq{доски объявлений} при котором агенты взаимодействуют посредством общей для них sc-памяти. Для реализации той же идеи в случае \textit{распределенной коллективной ostis-системы} необходимо выбрать какую-либо sc-память для выполнения данной роли.}
	\scnfileitem{При решении задач в \textit{распределенном коллективе ostis-систем} возможны два варианта организации взаимодействия \textit{агентов} (которыми являются и сами \textit{ostis-системы})}
	\begin{scnindent}
		\begin{scnsubdividing}
			\scnfileitem{Если решаемая задача достаточно сложная и требует частого обращения к нескольким отдельным \textit{ostis-системам}, то целесообразно путем объединения раздельных ostis-систем создать \textit{временную ostis-систему}, где все sc-агенты, входившие в состав исходных \textit{ostis-систем}, становятся внутренними, и принципы организации их взаимодействия известны. В этом случае существенно снижаются затраты на решение задачи, но появляются накладные расходы на создание таких \textit{временных ostis-систем}. Таким образом, необходимо отдельно разработать критерии на основании которых будет приниматься решение о целесообразности такого объединения. Отметим, что для того, чтобы иметь возможность сохранить результат и ход решения задачи для последующего применения целесообразно осуществлять объединение \textit{ostis-систем} на базе одной из \textit{ostis-систем}, входящих в такое объединение, а не создавать совершенно новую \textit{ostis-систему}. При этом в такую систему будут копироваться знания и навыки из объединяемых систем, а сами эти объединяемые системы могут вообще никак не меняться. Тогда после решения задачи из исходной \textit{ostis-системы} необходимо будет исключить те \textit{навыки} и \textit{знания}, которые были нужны только для решения данной задачи.}
			\scnfileitem{Важно отметить, что описанная интеграция \textit{ostis-систем} благодаря особенностям их архитектуры выполняется значительно проще, чем в других компьютерных системах, поскольку принципы построения и баз знаний, и \textit{решателей задач} \textit{ostis-систем} изначально предполагают возможность неограниченного расширения имеющихся в системе знаний и навыков без необходимости внесения изменений в уже имеющуюся \textit{базу знаний} и \textit{решатель задач}. Таким образом, интеграция двух ostis-систем при условии их семантической совместимости сводится к обычному теоретико-множественному объединению их \textit{баз знаний} и \textit{решаталей задач} и последующему исключению продублированных компонентов. Благодаря этому создание таких временных ostis-систем может выполняться \uline{автоматически}, что делает применение такого подхода к организации решения задач целесообразным во многих случаях.}
			\scnfileitem{Другой возможный вариант предполагает, что в качестве среды для взаимодействия \textit{sc-агентов} (как внешних, так и внутренних, внешняя \textit{ostis-система} с точки зрения процесса решения задачи также рассматривается как \textit{sc-агент}) выбирается \textit{sc-память} одной из \textit{ostis-систем}, входящих в состав \textit{коллектива ostis-систем}. Предлагаются следующие критерии выбора этой sc-памяти:
			\begin{itemize}
				\item Если задача решается неоднократно в рамках некоторого \textit{ostis-сообщества} (\textit{сообщества ostis-систем} и их пользователей), то для координации действий \textit{sc-агентов} выбирается \textit{sc-память} \textit{корпоративной ostis-системы} для данного \textit{ostis-сообщества};
				\item Если \textit{коллектив ostis-систем} для решения данной задачи формируется временно (разово), то для координации действий \textit{sc-агентов} выбирается \textit{sc-память} той \textit{ostis-системы}, которая инициировала решение данной задач.
			\end{itemize}}
		   	\scnfileitem{Недостатком данного варианта является наличие затрат на коммуникацию между \textit{ostis-системами}. Если по каким-либо причинам эти затраты велики (например, из-за низкого качества соединения между системами), то более целесообразно использовать первый из предложенных вариантов.}
		\end{scnsubdividing}
		\begin{scnrelfromset}{смотрите}
			\scnitem{Предлагаемый подход к разработке гибридных решателей задач ostis-систем и обработке информации в ostis-системах}
			\scnitem{Иерархическая система взаимодействующих ostis-сообществ}
		\end{scnrelfromset}
	\end{scnindent}
	\scnfileitem{В любом из предложенных вариантов в конечном итоге определяется некоторая конкретная sc-память, которая становится средой для взаимодействия агентов, осуществляющих решение задачи, по изложенным принципам. Тогда можно уточнить понятие \textit{sc-агента} как компонента решателя задач в контексте распределенного решения задач \textit{коллективом ostis-систем} и считать sc-агентом не только компонент \textit{решателя задач индивидуальной ostis-системы}, но и любую ostis-систему, входящую в постоянный либо временный \textit{коллектив ostis-систем}, решающих какие-либо задачи, поскольку принципы взаимодействия \textit{ostis-систем} в таком коллективе полностью совпадают с принципами взаимодействия \textit{sc-агентов} в составе \textit{решателя задач} \textit{индивидуальной ostis-системы}.}
	\begin{scnindent}
		\begin{scnrelfromset}{смотрите}
			\scnitem{Предлагаемый подход к разработке гибридных решателей задач ostis-систем и обработке информации в ostis-системах}
		\end{scnrelfromset}
	\end{scnindent}
	\scnfileitem{Таким образом, можно говорить о фрактальной иерархической структуре распределенного \textit{гибридного решателя задач}, в рамках которой выделяется два варианта иерархии \textit{sc-агентов}.}
		\begin{scnindent}
		\begin{scnsubdividing}	
			\scnfileitem{Иерархия sc-агентов с точки зрения уровня \textit{языков представления методов}, на которых представлены соответствующие этим sc-агентам методы. В рамках этой иерархии в свою очередь можно выделить три уровня, имеющих важные отличия.}
			\begin{scnindent}
				\begin{scnsubdividing}
				\scnfileitem{Уровень \textit{sc-агентов} \textit{ostis-платформы}, обеспечивающий интерпретацию методов платформенно-независимого уровня в рамках \textit{индивидуальной ostis-системы}, в рамках которого может выделяться иерархия языков представления методов уровня ostis-платформы и соответствующих средств их интерпретации.}
				\scnfileitem{Уровень \textit{платформенно-независимых sc-агентов} в рамках \textit{индивидуальной ostis-системы}, в рамках которого может выделяться иерархия платформенно-независимых языков представления методов.}
				\scnfileitem{Уровень \textit{распределенных коллективов ostis-систем}, на котором также можно говорить о \textit{языках представления методов} и их иерархии, но при этом в общем случае даже отдельные методы могут физически храниться распределенно в разных ostis-системах. Например, можно говорить о \textit{языке представления методов} для финансовой деятельности крупных предприятий, но при этом целесообразно выделять подъязыки для описания деятельности отделов различных категорий и иметь отдельные ostis-системы для обслуживания каждого из отделов.}
				\end{scnsubdividing}
			\end{scnindent}
			\scnfileitem{Иерархия sc-агентов с точки зрения атомарности/неатомарности в рамках \uline{одного} \textit{языка представления методов}.}
			\begin{scnindent}
				\scntext{примечание}{Формирование такой иерархии может быть целесообразным на любом уровне языка \textit{языка представления методов}.}
				\begin{scnsubdividing}
				\scnfileitem{\textit{атомарные платформенно-зависимые sc-агенты} и \textit{неатомарные платформенно-зависимые sc-агенты} на уровне \textit{ostis-платформы}.}
				\scnfileitem{\textit{атомарные платформенно-независимые sc-агенты} и \textit{неатомарные платформенно-независимые sc-агенты} на платформенно-независимом уровне в рамках индивидуальной ostis-системы.}
				\scnfileitem{\textit{индивидуальные ostis-системы} и \textit{коллективны ostis-систем} на уровне решения задач в рамках \textit{Экосистемы OSTIS}.}
			   \end{scnsubdividing}
			\end{scnindent}
		\end{scnsubdividing}	
		\end{scnindent}
	\scnfileitem{Дальнейшее развитие представленных принципов решения задач распределенными коллективами ostis-систем предполагает:
		\begin{itemize}
			\item Разработку формальных критериев для оценки целесообразности или нецелесообразности формирования временных индивидуальных ostis-систем.
			\item Разработку языка и принципов обмена сообщениями между ostis-системами, входящими в коллектив ostis-систем, решающий какую-либо задачу. Несмотря на то, что с логической точки зрения каждая ostis-система трактуется как sc-агент и принципы их взаимодействия остаются теми же, реализация, например, возможности реагирования на события в базе знаний и внесения изменений в эту базу знаний для внутренних sc-агентов и внешних ostis-систем будет отличаться и требует уточнения.
		\end{itemize}}
\end{scnsubdividing}

\end{scnsubstruct}

\scntext{заключение}{В данной предметной области рассмотрены актуальные на сегодняшний день проблемы в области разработки \textit{гибридных решателей задач} и предложен общий подход к построению \textit{гибридных решателей задач}, который решает такие проблемы, как обеспечение совместимости и модифицируемости \textit{решателей задач}, а также создает предпосылки к решению других актуальных проблем.}
\begin{scnindent}
	\scnrelfrom{смотрите}{Актуальные проблемы и перспективы развития технологий разработки гибридных решателей задач}
	\begin{scnrelfromlist}{направления развития}
		\scnfileitem{Более тесно и полно интегрировать идеи ситуационного управления в предлагаемый подход.}
		\scnfileitem{Доработать предложенный механизм блокировок, в частности, минимизировать число классов блокировок, учесть и реализовать идеи реализации lock-free алгоритмов.}
		\scnfileitem{Исключить необходимость введения \textit{sc-метаагентов} и \textit{scp-метапрограмм}.}
		\scnfileitem{Доработать \textit{Язык SCP} до того, чтобы иметь возможность описывать в рамках \textit{scp-программ} рецепторное и эффекторное взаимодействие \textit{ostis-систем}.}
		\scnfileitem{При разработке \textit{Абстрактной scp-машины} учесть принципы построения волновых языков программирования и идеи инсерционного программирования и моделирования.}
		\begin{scnindent}
			\begin{scnrelfromset}{смотрите}
				\scnitem{\scncite{Letichevskij2003}}
				\scnitem{\scncite{Letichevskij2012}}
				\scnitem{\scncite{Moldovan1985}}
				\scnitem{\scncite{Sapatyj1986}}
			\end{scnrelfromset}
		\end{scnindent}
	\end{scnrelfromlist}
\end{scnindent}

\end{SCn}


\scsubsection{\S 30.1. Предметная область и онтология действий, задач, планов, протоколов и методов, реализуемых ostis-системой, а также внутренних агентов, выполняющих эти действия}
\label{sd_agents}
\begin{SCn}
\scnsectionheader{Предметная область и онтология действий, задач, планов, протоколов и методов, реализуемых ostis-системой, а также внутренних агентов, выполняющих эти действия}
\begin{scnsubstruct}

\scnheader{Предметная область и онтология действий, задач, планов, протоколов и методов, реализуемых ostis-системой в ее памяти, а также внутренних агентов, выполняющих эти действия}
\scniselement{предметная область}
\begin{scnhaselementrolelist}{максимальный класс объектов исследования}
    \scnitem{действие в sc-памяти}
    \scnitem{абстрактный sc-агент}
    \scnitem{sc-агент}
\end{scnhaselementrolelist}
\begin{scnhaselementrolelist}{класс объектов исследования}
    \scnitem{абстрактный sc-агент, не реализуемый на Языке SCP}
    \scnitem{абстрактный sc-агент, реализуемый на Языке SCP}
    \scnitem{Абстрактный программный sc-агент}
    \scnitem{неатомарный абстрактный sc-агент}
    \scnitem{атомарный абстрактный sc-агент}
    \scnitem{платформенно-независимый абстрактный sc-агент}
    \scnitem{платформенно-зависимый абстрактный sc-агент}
    \scnitem{внутренний абстрактный sc-агент}
    \scnitem{эффекторный абстрактный sc-агент}
    \scnitem{рецепторный абстрактный sc-агент}
    \scnitem{абстрактный sc-агент, не реализуемый на Языке SCP}
    \scnitem{абстрактный sc-агент, реализуемый на Языке SCP}
    \scnitem{абстрактный sc-агент интерпретации scp-программ}
    \scnitem{абстрактный программный sc-агент}
    \scnitem{абстрактный программный sc-агент, реализуемый на Языке SCP}
    \scnitem{абстрактный sc-метаагент}
    \scnitem{sc-агент}
    \scnitem{активный sc-агент}
    \scnitem{описание поведения sc-агента}
    \scnitem{тип блокировки}
    \scnitem{полная блокировка}
    \scnitem{блокировка на любое изменение}
    \scnitem{блокировка на удаление}
\end{scnhaselementrolelist}
\begin{scnhaselementrolelist}{исследуемое отношение}
    \scnitem{декомпозиция абстрактного sc-агента*}
    \scnitem{ключевые sc-элементы sc-агента*}
    \scnitem{программа sc-агента*}
    \scnitem{первичное условие инициирования*}
    \scnitem{условие инициирования и результат*}
    \scnitem{блокировка*}
\end{scnhaselementrolelist}

\scnheader{обработка информации в ostis-системах}
\begin{scnrelfromlist}{принципы, лежащие в основе}
    \scnfileitem{В основе решателя задач каждой \textit{ostis-системы} лежит многоагентная система, агенты которой взаимодействуют между собой \uline{только}(!) через общую для них \textit{sc-память} посредством спецификации в этой памяти выполняемых ими \textit{действий в sc-памяти}. При этом пользователи \textit{ostis-системы} также считаются агентами этой системы. Кроме того, \textit{sc-агенты} делятся на внутренние, рецепторные и эффекторные. Взаимодействие между агентами через общую \textit{sc-память} сводится к следующим видам действий:
        \begin{scnenumerate}
            \item К использованию общедоступной для соответствующей группы sc-агентов части хранимой базы знаний.
            \item К формированию (генерации) новых фрагментов базы знаний и/или к корректировке (редактированию) каких-либо фрагментов доступной части базы знаний.
            \item К интеграции (погружению) новых и/или обновленных фрагментов в состав доступной части базы знаний.
        \end{scnenumerate}
        Подчеркнем, что sc-агенты не общаются между собой напрямую путем отправки сообщений, как это делается в большинстве современных подходов к построению многоагентных систем. Кроме того, sc-агенты имеют доступ к общей для них базе знаний за счет чего гарантируется семантическая совместимость (взаимопонимание) между агентами, включая и пользователей ostis-систем.}
    \scnfileitem{Пользователь \textit{ostis-системы} не может сам непосредственно выполнить какое-либо действие в \mbox{sc-памяти}, но он может средствами пользовательского интерфейса инициировать построение (генерацию, формирование в \textit{sc-памяти}) \textit{sc-текста}, являющегося спецификацией \textit{действия в \mbox{sc-памяти}}, выполняемого либо одним \textit{атомарным sc-агентом} за один акт, либо одним \textit{атомарным sc-агентом} за несколько актов, либо коллективом \textit{sc-агентов} (\textit{неатомарным sc-агентом}). В спецификации каждого такого \textit{действия в sc-памяти}, инициированного пользователем, этот пользователь указывается как заказчик этого действия. Таким образом, пользователь \textit{ostis-системы} дает поручения (задания, команды) \textit{sc-агентам} этой системы на выполнение различных специфицируемых им действий в \textit{sc-памяти}.}
    \scnfileitem{Каждый \textit{sc-агент}, выполняя некоторое \textit{действие в sc-памяти}, должен помнить, что \textit{sc-память}, над которой он работает, является общим ресурсом не только для него, но и для всех остальных \textit{\mbox{sc-агентов}}, работающих над этой же \textit{sc-памятью}. Поэтому \textit{sc-агент} должен соблюдать определенную этику поведения в коллективе таких \textit{sc-агентов}, которая должна минимизировать помехи, которые он создает другим \textit{sc-агентам}.}
    \scnfileitem{Деятельность каждого агента \textit{ostis-системы} дискретна и представляет собой множество элементарных действий (актов). При этом при выполнении каждого акта агент может устанавливать блокировки нескольких типов на фрагменты базы знаний. Указанные блокировки позволяют запретить другим агентам изменять указанный фрагмент базы знаний или вообще сделать его невидимым для других агентов. Блокировки устанавливаются самим агентом при выполнении соответствующего акта и снимаются им же на последнем этапе выполнения этого акта или раньше, если это возможно.}
    \scnfileitem{Если некий \textit{sc-агент} выполняет некоторое \textit{действие в sc-памяти}, то он на время выполнения этого действия может:\\
        \begin{scnenumerate}
            \item Запретить другим \textit{sc-агентам} изменять состояние некоторых sc-элементов, хранимых в \textit{sc-памяти} --- удалять их, изменять тип.
            \item Запретить другим \textit{sc-агентам} добавлять или удалять элементы некоторых множеств, обозначаемых соответствующими \textit{sc-узлами}.
            \item Запретить другим \textit{sc-агентам} доступ на просмотр некоторых \textit{sc-элементов}, то есть эти \textit{\mbox{sc-элементы}} становятся полностью невидимыми (полностью заблокированными) для других \textit{sc-агентов}, но только на время выполнения соответствующего действия.
        \end{scnenumerate}
        Указанные блокировки должны быть полностью сняты до завершения выполнения соответствующего действия. Подчеркнем, что в число \textit{sc-элементов}, блокируемых на время выполнения некоторого действия, в основном входят атомарные и неатомарные связки, и не должны входить \textit{sc-узлы}, обозначающие бесконечные классы каких-либо сущностей, и, тем более, sc-узлы, обозначающие различные понятия (ключевые классы различных предметных областей).\\
        Этичное (неэгоистичное) поведение \textit{sc-агента}, касающееся блокировки \textit{sc-элементов} (то есть ограничения к ним доступа другим \textit{sc-агентам}) предполагает соблюдение следующих правил:\\
        \begin{scnenumerate}
            \item Не следует блокировать больше \textit{sc-элементов}, чем это необходимо для решения задачи.
            \item Как только для какого-либо \textit{sc-элемента} необходимость его блокировки отпадает до завершения выполнения соответствующего действия, этот \textit{sc-элемент} желательно сразу деблокировать (снять блокировку).
        \end{scnenumerate}
        Для того, чтобы \textit{sc-агент} имел возможность работы с каким-либо произвольным \textit{sc-элементом}, он должен либо убедиться в том, что этот \textit{sc-элемент} не входит во фрагмент базы знаний, входящий в \textit{полную блокировку}, либо убедиться в том, что эта блокировка не установлена самим этим агентом.\\
        Особой группой полностью заблокированных \textit{sc-элементов} (на время выполнения действия \textit{\mbox{sc-агентом}}) являются вспомогательные \textit{sc-элементы} (леса), создаваемые только на время выполнения этого действия. Эти sc-элементы в конце выполнения действия должны не деблокироваться, а удаляться.}
    \scnfileitem{Если \textit{действие в sc-памяти}, выполняемое \textit{sc-агентом}, завершилось (т.е. стало прошлой сущностью), то \textit{sc-агент} оформляет результат этого \textit{действия}, указывая (1) удаленные \textit{sc-элементы} и (2) сгенерированные sc-элементы. Это необходимо, если по каким-либо причинам придется сделать откат этого \textit{действия}, т.е возвратиться к состоянию базы знаний до выполнения указанного \textit{действия}.}
\end{scnrelfromlist}

\scnsegmentheader{Понятие действия в sc-памяти}
\begin{scnsubstruct}
    
\scnheader{действие в sc-памяти}
\scnidtf{внутреннее действие ostis-системы}
\scnidtf{действие, выполняемое в sc-памяти}
\scnidtf{действие, выполняемое в абстрактной унифицированной семантической памяти}
\scnidtf{действие, выполняемое машиной обработки знаний ostis-системы}
\scnidtf{действие, выполняемое агентом или коллективом агентов ostis-системы}
\scnidtf{информационный процесс над базой знаний, хранимой в sc-памяти}
\scnidtf{процесс решения информационной задачи в sc-памяти}
\scnsubset{процесс в sc-памяти}
\scntext{пояснение}{Каждое \textbf{\textit{действие в sc-памяти}} обозначает некоторое преобразование, выполняемое некоторым \textit{sc-агентом} (или коллективом \textit{sc-агентов}) и ориентированное на преобразование \textit{sc-памяти}. Спецификация действия после его выполнения может быть включена в протокол решения некоторой задачи.\\
    Преобразование состояния базы знаний включает, в том числе и информационный поиск, предполагающий (1) локализацию в базе знаний ответа на запрос, явное выделение структуры ответа и (2) трансляцию ответа на некоторый внешний язык.\\
    Во множество \textbf{\textit{действий в sc-памяти}} входят знаки действий самого различного рода, семантика каждого из которых зависит от конкретного контекста, т.е. ориентации действия на какие-либо конкретные объекты и принадлежности действия какому-либо конкретному классу действий.\\
    Следует четко отличать:
        \begin{scnitemize}
            \item каждое конкретное \textbf{\textit{действие в sc-памяти}}, представляющее собой некоторый переходный процесс, переводящий sc-память из одного состояния в другое;
            \item каждый тип \textbf{\textit{действий в sc-памяти}}, представляющий собой некоторый класс однотипных (в том или ином смысле) действий;
            \item sc-узел, обозначающий некоторое конкретное \textbf{\textit{действие в sc-памяти}};
            \item sc-узел, обозначающий структуру, которая является описанием, спецификацией, заданием, постановкой соответствующего действия.
        \end{scnitemize}}
\scnsuperset{действие в sc-памяти, инициируемое вопросом}
\scnsuperset{действие редактирования базы знаний ostis-системы}
\scnsuperset{действие установки режима ostis-системы}
\scnsuperset{действие редактирования файла, хранимого в sc-памяти}
\scnsuperset{действие интерпретации программы, хранимой в sc-памяти}

\scnheader{действие в sc-памяти, инициируемое вопросом}
\scnidtf{действие, направленное на формирование ответа на поставленный вопрос}
\scnsuperset{действие. cформировать заданный файл}
\scnsuperset{действие. cформировать заданную структуру}
    \begin{scnindent}
        \scnsuperset{действие. верифицировать заданную структуру}
        \begin{scnindent}
            \scnsuperset{действие. установить истинность или ложность указываемого логического высказывания}
            \scnsuperset{действие. установить корректность или некорректность указываемой структуры}
            \scnsuperset{действие. сформировать структуру, описывающую некорректности, имеющиеся в указываемой структуре}
        \end{scnindent}
        \scnsuperset{действие. уточнить тип заданного sc-элемента}
        \begin{scnindent}
            \scnsuperset{действие. установить позитивность/негативность указываемой sc-дуги принадлежности или непринадлежности}
        \end{scnindent}
        \scnsuperset{действие. сформировать семантическую окрестность}
        \begin{scnindent}
            \scnsuperset{действие. сформировать полную семантическую окрестность указываемой сущности}
            \scnsuperset{действие. сформировать базовую семантическую окрестность указываемой сущности}
            \scnsuperset{действие. сформировать частную семантическую окрестность указываемой сущности}
        \end{scnindent}
        \scnsuperset{действие. сформировать структуру, описывающую связи между указываемыми сущностями}
        \begin{scnindent}
            \scnsuperset{действие. сформировать структуру, описывающую сходства указываемых сущностей}
            \scnsuperset{действие. сформировать структуру, описывающую различия указываемых сущностей}
        \end{scnindent}
        \scnsuperset{действие. сформировать структуру, описывающую способ решения указываемой задачи}
        \scnsuperset{действие. сформировать план генерации ответа на указанный вопрос}
        \scnsuperset{действие. сформировать протокол выполнения указываемого действия}
        \scnsuperset{действие. сформировать обоснование корректности указываемого решения}
        \scnsuperset{действие. верифицировать обоснование корректности указываемого решения}
        \scnsuperset{действие, направленное на установление темпоральных характеристик указываемой сущности}
        \scnsuperset{действие, направленное на установление пространственных характеристик указываемой сущности}
    \end{scnindent}

\scnheader{действие редактирования базы знаний}
\scnsuperset{действие. изменить направление указанной sc-дуги}
\scnsuperset{действие. исправить ошибки в заданной структуре}
\scnsuperset{действие. преобразовать указанную структуру в соответствии с указанным правилом}
\scnsuperset{действие. отождествить два указанных sc-элемента}
\scnsuperset{действие. включить множество}
    \begin{scnindent}
        \scnidtf{сделать все элементы множества \textbf{\textit{Si}} явно принадлежащими множеству \textbf{\textit{Sj}}, то есть сгенерировать соответствующие sc-дуги принадлежности}
    \end{scnindent}
\scnsuperset{действие генерации sc-элементов}
    \begin{scnindent}
        \scnsuperset{действие генерации, одним из аргументов которого является некоторая обобщенная структура}
            \begin{scnindent}
                \scnsuperset{действие. сгенерировать структуру, изоморфную указываемому образцу}
            \end{scnindent}
        \scnsuperset{действие. сгенерировать sc-элемент указанного типа}
            \begin{scnindent}
                \scnsuperset{действие. сгенерировать sc-коннектор указанного типа}
                \scnsuperset{действие. сгенерировать sc-узел указанного типа}
            \end{scnindent}
        \scnsuperset{действие. сгенерировать файл с заданным содержимым}
        \scnsuperset{действие. установить указанный файл в качестве основного идентификатора указанного sc-элемента для указанного внешнего языка}
    \end{scnindent}
\scnsuperset{действие. обновить понятия}
    \begin{scnindent}
        \scnidtf{действие. заменить неосновные понятия на их определения через основные понятия}
    \end{scnindent}
\scnsuperset{действие. интегрировать информационную конструкцию в текущее состояние базы знаний}
    \begin{scnindent}
        \scnsuperset{действие. интегрировать содержимое указанного файла в текущее состояние базы знаний}
            \begin{scnindent}
                \scnsuperset{действие. протранслировать содержимое указанного файла в sc-память}
            \end{scnindent}
        \scnsuperset{действие. интегрировать указанную структуру в текущее состояние базы знаний}
    \end{scnindent}
\scnsuperset{действие. дополнить описание прошлого состояния ostis-системы}
    \begin{scnindent}
        \scnsuperset{действие. дополнить структуру, описывающую историю эволюции ostis-системы}
        \scnsuperset{действие. дополнить структуру, описывающую историю эксплуатации ostis-системы}
    \end{scnindent}
\scnsuperset{действие удаления sc-элементов}
    \begin{scnindent}
        \scnsuperset{действие. удалить указанные sc-элементы}
        \begin{scnindent}
            \scnsuperset{действие. удалить sc-элементы, входящие в состав указанной структуры и не являющиеся ключевыми узлами каких-либо sc-агентов}
        \end{scnindent}
    \end{scnindent}
        
\scnheader{действие. отождествить два указанных sc-элемента}
\scnidtf{действие. совместить два указанных sc-элемента}
\scnidtf{действие. склеить два указанных sc-элемента}
    \begin{scnsubdividing}
        \scnitem{действие. физически отождествить два указанных sc-элемента}
        \scnitem{действие. логически отождествить два указанных sc-элемента}
    \end{scnsubdividing}

\scnheader{действие. отождествить два указанных sc-элемента}
\scntext{пояснение}{Каждое \textbf{\textit{действие. отождествить два указанных sc-элемента}} может быть выполнено как \textit{действие. физически отождествить два указанных sc-элемента} или \textit{действие. логически отождествить два указанных sc-элемента}. В случае логического отождествления в протоколе деятельности агентов сохраняется само действие с его спецификацией, включающей обязательное указание того, какие элементы были сгенерированы, а какие удалены. В случае физического отождествления протокол действия не сохраняется.}

\scnheader{действие. обновить понятия}
\scnidtf{действие. заменить некоторое множество понятий на другое множество понятий}
\scntext{пояснение}{Каждое \textbf{\textit{действие. обновить понятия}} обозначает переход от какой-то группы понятий, использовавшихся ранее, к другой группе понятий, которые будут использоваться вместо первых и станут \textit{основными понятиями}. В общем случае \textbf{\textit{действие. обновить понятия}} состоит из следующих этапов:
    \begin{scnitemize}
        \item определить заменяемые понятия на основе заменяющих;
        \item внести соответствующие изменения в программы sc-агентов, ключевыми узлами которых являются обновляемые понятия;
        \item заменить все конструкции в базе знаний, содержащие заменяемые понятия, в соответствии с определениями этих понятий через заменяющие их понятия;
        \item при необходимости, \textit{sc-элементы}, обозначающие замененные таким образом понятия, могут быть полностью выведены из текущего состояния базы знаний.
    \end{scnitemize}
    Первым аргументом (входящим в знак \textit{действия} под атрибутом \textit{1\scnrolesign}) \textbf{\textit{действия. обновить понятия}} является знак множества \textit{sc-узлов}, обозначающих заменяемые понятия, вторым (входящим в знак \textit{действия} под атрибутом \textit{2\scnrolesign}) --- знак множества \textit{sc-узлов}, обозначающих заменяющие понятия. В общем случае любое или оба этих множества могут быть \textit{синглетонами}.
    }   
    
\scnheader{действие. удалить указанные sc-элементы}
\begin{scnsubdividing}
    \scnitem{действие. физически удалить указанные sc-элементы}
    \scnitem{действие. логически удалить указанные sc-элементы}
\end{scnsubdividing}
\scntext{пояснение}{Каждое \textbf{\textit{действие. удалить указанные sc-элементы}} может быть выполнено как \textit{действие. физически удалить указанные sc-элементы} или \textit{действие. логически удалить указанные sc-элементы}. \\
    В случае логического удаления в протоколе деятельности агентов сохраняется само действие с его спецификацией, включающей обязательное указание того, какие элементы были удалены, т.е. по сути, элементы просто исключаются из текущего состояния базы знаний.\\
    В случае физического удаления протокол действия не сохраняется. В случае удаления какого-либо \textit{sc-элемента}, инцидентные ему \textit{связки}, в том числе \textit{sc-коннекторы}, так же удаляются.}

\scnheader{действие. интегрировать указанную структуру в текущее состояние базы знаний}
\scntext{пояснение}{Для того, чтобы выполнить \textbf{\textit{действие. интегрировать указанную структуру в текущее состояние базы знаний}}, необходимо склеить \textit{sc-элементы}, входящие в интегрируемую \textit{структуру} с синонимичными им \textit{sc-элементами}, входящими в текущее состояние базы знаний, заменить неиспользуемые (например, устаревшие) понятия, входящие в интегрируемую \textit{структуру}, на используемые (т.е. заменить неиспользуемые понятия на их определения через используемые), явно включить все элементы интегрируемой \textit{структуры} в число элементов утвержденной части базы знаний и явно включить все элементы интегрируемой \textit{структуры} в число элементов одного из атомарных разделов утвержденной части базы знаний.}

\scnheader{действие интерпретации программы, хранимой в sc-памяти}
\scnsuperset{действие интерпретации scp-программы}

\scnheader{задача, решаемая в sc-памяти}
\scnsubset{задача}
\scnidtf{спецификация действия, выполняемого в sc-памяти}
\scnidtf{структура, являющая описанием (постановкой, заданием) соответствующего действия в sc-памяти, которое обладает достаточной полнотой для выполения указанного действия}
\scnidtf{семантическая окрестность некоторого действия в sc-памяти, обеспечивающая достаточно полное задание этого действия}

\scnheader{класс действий}
\scnsuperset{класс действий в sc-памяти}
\begin{scnindent}
    \scnrelto{семейство подмножеств}{действие в sc-памяти}
\end{scnindent}
\begin{scnsubdividing}
    \scnitem{класс логически атомарных действий}
    \begin{scnindent}
        \scnidtf{класс автономных действий}
    \end{scnindent}
    \scnitem{класс логически неатомарных действий}
    \begin{scnindent}
        \scnidtf{класс неавтономных действий}
    \end{scnindent}
\end{scnsubdividing}

\scnheader{класс логически атомарных действий}
\scntext{пояснение}{Каждое \textit{действие}, принадлежащее некоторому конкретному \textit{классу логически атомарных действий}, обладает двумя необходимыми свойствами:
    \begin{scnitemize}
        \item Выполнение действия не зависит от того, является ли указанное действие частью декомпозиции более общего действия. При выполнении данного действия также не должен учитываться тот факт, что данное действие предшествует каким-либо другим действиям или следует за ними (что явно указывается при помощи отношения \textit{последовательность действий*}).
        \item Указанное действие должно представлять собой логически целостный акт преобразования, например, в семантической памяти. Такое действие по сути является транзакцией, т. е. результатом такого преобразования становится новое состояние преобразуемой системы, а выполняемое действие должно быть либо выполнено полностью, либо не выполнено совсем, частичное выполнение не допускается.
    \end{scnitemize}
    В то же время логическая атомарность не запрещает декомпозировать выполняемое действие на более частные, каждое из которых, в свою очередь, также будет являться логически атомарным.}
\scnsuperset{класс логически атомарных действий в sc-памяти}
    \begin{scnindent}
        \scntext{пояснение}{На логически атомарные действия предлагается делить всю деятельность, направленную на решение каких-либо задач ostis-системой. Соответственно \textit{решатель задач ostis-системы} предлагается делить на компоненты, соответствующие таким \textit{классам логически атомарных действий в sc-памяти}, что является основой для обеспечения его \textit{модифицируемости}.}
    \end{scnindent}

\bigskip
\end{scnsubstruct}
\scnendsegmentcomment{Понятие действия в sc-памяти}

\scnsegmentheader{Понятие sc-агента и абстрактного sc-агента}
\begin{scnsubstruct}

\scnheader{sc-агент}
\scnidtf{единственный вид \textit{субъектов}, выполняющих преобразования в \textit{\textit{sc-памяти}}}
\scnidtf{\textit{субъект}, способный выполнять \textit{действия в sc-памяти}, принадлежащие некоторому определенному \textit{классу логически атомарных действий}}
\scntext{пояснение}{Логическая атомарность выполняемых sc-агентом действий предполагает, что каждый sc-агент реагирует на соответствующий ему класс ситуаций и/или событий, происходящих в sc-памяти, и осуществляет определенное преобразование sc-текста, находящегося в семантической окрестности обрабатываемой ситуации и/или события. При этом каждый sc-агент в общем случае не имеет информацию о том, какие еще sc-агенты в данный момент присутствуют в системе и осуществляет взаимодействие в другими sc-агентами исключительно посредством формирования некоторых конструкций (как правило  спецификаций действий) в общей sc-памяти. Таким сообщением может быть, например, вопрос, адресованный другим sc-агентам в системе (заранее не известно, каким конкретно), или ответ на поставленный другими sc-агентами вопрос (заранее не известно, каким конкретно). Таким образом, каждый sc-агент в каждый момент времени контролирует только фрагмент базы знаний в контексте решаемой данным агентом задачи, состояние всей остальной базы знаний в общем случае непредсказуемо для sc-агента.}

\scnheader{абстрактный sc-агент}
\scntext{примечание}{Поскольку предполагается, что копии одного и того же \textit{sc-агента} или функционально эквивалентные \textit{sc-агенты} могут работать в разных ostis-системах, будучи при этом физически разными sc-агентами, то целесообразно рассматривать свойства и классификацию не sc-агентов, а классов функционально эквивалентных sc-агентов, которые будем называть \textit{абстрактными sc-агентами}.}
\scntext{пояснение}{Под \textbf{\textit{абстрактным sc-агентом}} понимается некоторый класс функционально эквивалентных \textit{sc-агентов}, разные экземпляры (т.е. представители) которого могут быть реализованы по-разному. Каждый \textbf{\textit{абстрактный sc-агент}} имеет соответствующую ему спецификацию. В спецификацию каждого \textbf{\textit{абстрактного sc-агента}} входит:
    \begin{scnitemize}
        \item указание ключевых \textit{sc-элементов} этого \textit{sc-агента}, т.е. тех \textit{sc-элементов}, хранимых в \textit{sc-памяти}, которые для данного \textit{sc-агента} являются точками опоры;
        \item формальное описание условий инициирования данного \textit{sc-агента}, т.е. тех \textit{ситуаций} в \textit{sc-памяти}, которые инициируют деятельность данного \textit{sc-агента};
        \item формальное описание первичного условия инициирования данного \textit{sc-агента}, т.е. такой ситуации в \textit{sc-памяти}, которая побуждает \textit{sc-агента} перейти в активное состояние и начать проверку наличия своего полного условия инициирования (для \textit{внутренних абстрактных sc-агентов});
        \item строгое, полное, однозначно понимаемое описание деятельности данного \textit{sc-агента}, оформленное при помощи каких-либо понятных, общепринятых средств, не требующих специального изучения, например, на естественном языке;
        \item описание результатов выполнения данного \textit{sc-агента}.
    \end{scnitemize}
}
\begin{scnsubdividing}
    \scnitem{неатомарный абстрактный sc-агент}
    \scnitem{атомарный абстрактный sc-агент}
\end{scnsubdividing}
\begin{scnsubdividing}
    \scnitem{внутренний абстрактный sc-агент}
    \scnitem{эффекторный абстрактный sc-агент}
    \scnitem{рецепторный абстрактный sc-агент}
\end{scnsubdividing}
\begin{scnsubdividing}
    \scnitem{абстрактный sc-агент, не реализуемый на Языке SCP}
    \scnitem{абстрактный sc-агент, реализуемый на Языке SCP}
\end{scnsubdividing}
\begin{scnsubdividing}
    \scnitem{абстрактный sc-агент интерпретации scp-программ}
    \scnitem{абстрактный программный sc-агент}
    \scnitem{абстрактный sc-метаагент}
\end{scnsubdividing}
\begin{scnsubdividing}
    \scnitem{платформенно-зависимый абстрактный sc-агент}
    \begin{scnindent}
        \scnsuperset{абстрактный sc-агент, не реализуемый на Языке SCP}
    \end{scnindent}
    \scnitem{платформенно-независимый абстрактный sc-агент}
\end{scnsubdividing}

\scnheader{абстрактный sc-агент, не реализуемый на Языке SCP}
\scnidtf{абстрактный sc-агент, который не может быть реализован на платформенно-независимом уровне}
\begin{scnsubdividing}
    \scnitem{эффекторный абстрактный sc-агент}
    \scnitem{рецепторный абстрактный sc-агент}
    \scnitem{абстрактный sc-агент интерпретации scp-программ}
\end{scnsubdividing}

\scnheader{абстрактный sc-агент, реализуемый на Языке SCP}
\scnidtf{абстрактный sc-агент, который может быть реализован на платформенно-независимом уровне}
\begin{scnsubdividing}
    \scnitem{абстрактный sc-метаагент}
    \scnitem{абстрактный программный sc-агент, реализуемый на Языке SCP}
\end{scnsubdividing}
\scnheader{абстрактный программный sc-агент}
\begin{scnsubdividing}
    \scnitem{эффекторный абстрактный sc-агент}
    \scnitem{рецепторный абстрактный sc-агент}
    \scnitem{абстрактный программный sc-агент, реализуемый на Языке SCP}
\end{scnsubdividing}

\scnheader{неатомарный абстрактный sc-агент}
\scntext{пояснение}{Под \textbf{\textit{неатомарным абстрактным sc-агентом}} понимается \textit{абстрактный sc-агент}, который декомпозируется на коллектив более простых \textit{абстрактных sc-агентов}, каждый из которых в свою очередь может быть как \textit{атомарным абстрактным sc-агентом}, так и \textbf{\textit{неатомарным абстрактным sc-агентом}}. При этом в каком-либо варианте \textit{декомпозиции абстрактного sc-агента*} дочерний \textbf{\textit{неатомарный абстрактный sc-агент}} может стать \textit{атомарным абстрактным sc-агентом} и реализовываться соответствующим образом.}

\scnheader{атомарный абстрактный sc-агент}
\scntext{пояснение}{Под \textbf{\textit{атомарным абстрактным sc-агентом}} понимается \textit{абстрактный sc-агент}, для которого уточняется платформа его реализации, т.е. существует соответствующая связка отношения \textit{программа sc-агента*}.}
\begin{scnsubdividing}
    \scnitem{платформенно-независимый абстрактный sc-агент}
    \scnitem{платформенно-зависимый абстрактный sc-агент}
\end{scnsubdividing}

\scnheader{платформенно-независимый абстрактный sc-агент}
\scntext{пояснение}{К \textbf{\textit{платформенно-независимым абстрактным \mbox{sc-агентам}}} относят \textit{атомарные абстрактные sc-агенты}, реализованные на базовом языке программирования Технологии OSTIS, т.е. на \textit{Языке SCP}.\\
    При описании \textbf{\textit{платформенно-независимых абстрактных sc-агентов}} под платформенной независимостью понимается платформенная независимость с точки зрения Технологии OSTIS, т.е реализация на специализированном языке программирования, ориентированном на обработку семантических сетей (\textit{Языке SCP}), поскольку \textit{атомарные sc-агенты}, реализованные на указанном языке могут свободно переноситься с одной платформы интерпретации \textit{sc-моделей} на другую. При этом языки программирования, традиционно считающиеся платформенно-независимыми, в данном случае не могут считаться таковыми.\\
    Существуют \textit{sc-агенты}, которые принципиально не могут быть реализованы на платформенно-независимом уровне, например, собственно \textit{sc-агенты} интерпретации \textit{sc-моделей} или рецепторные и эффекторные \textit{sc-агенты}, обеспечивающие взаимодействие с внешней средой.}
    
\scnheader{платформенно-зависимый абстрактный sc-агент}
\scntext{пояснение}{К \textbf{\textit{платформенно-зависимым абстрактным sc-агентам}} относят \textit{атомарные абстрактные sc-агенты}, реализованные ниже уровня sc-моделей, т.е. не на \textit{Языке SCP}, а на каком-либо другом языке описания программ.\\
    Существуют \textit{sc-агенты}, которые принципиально должны быть реализованы на платформенно-зависимом уровне, например, собственно \textit{sc-агенты} интерпретации \textit{sc-моделей} или рецепторные и эффекторные \textit{sc-агенты}, обеспечивающие взаимодействие с внешней средой.}

\scnheader{внутренний абстрактный sc-агент}
\scntext{пояснение}{Каждый \textbf{\textit{внутренний абстрактный sc-агент}} обозначает класс \textit{sc-агентов}, которые реагируют на события в \textit{sc-памяти} и осуществляют преобразования исключительно в рамках этой же \textit{sc-памяти}.}

\scnheader{эффекторный абстрактный sc-агент}
\scntext{пояснение}{Каждый \textbf{\textit{эффекторный абстрактный sc-агент}} обозначает класс \textit{sc-агентов}, которые реагируют на события в \textit{sc-памяти} и осуществляют преобразования во внешней относительно данной \textit{ostis-системы} среде.}

\scnheader{рецепторный абстрактный sc-агент}
\scntext{пояснение}{Каждый \textbf{\textit{рецепторный абстрактный sc-агент}} обозначает класс \textit{sc-агентов}, которые реагируют на события во внешней относительно данной \textit{ostis-системы} среде и осуществляют преобразования в памяти данной системы.}

\scnheader{абстрактный sc-агент, не реализуемый на Языке SCP}
\scntext{пояснение}{Каждый \textbf{\textit{абстрактный sc-агент, не реализуемый на Языке SCP}} должен быть реализован на уровне платформы интерпретации sc-моделей, в том числе, аппаратной. К таким \textit{абстрактным sc-агентам} относятся абстрактные sc-агенты интерпретации scp-программ, а также эффекторные и рецепторные абстрактные sc-агенты.}

\scnheader{абстрактный sc-агент, реализуемый на Языке SCP}
\scntext{пояснение}{Каждый \textbf{\textit{абстрактный sc-агент, реализуемый на Языке SCP}} может быть реализован на Языке SCP, то есть на платформенно-независимом уровне, но при необходимости может реализовываться и на уровне платформы, например, с целью повышения производительности.}

\scnheader{абстрактный sc-агент интерпретации scp-программ}
\scntext{пояснение}{К \textbf{\textit{абстрактным sc-агентам интерпретации scp-программ}} относятся нереализуемые на платформенно-независимом уровне \textit{абстрактные sc-агенты}, обеспечивающие интерпретацию \textit{scp-программ} и \textit{scp-метапрограмм}, в том числе создание \textit{scp-процессов}, собственно интерпретацию \textit{scp-операторов}, а также другие вспомогательные действия. По сути, агенты данного класса обеспечивают работу sc-агентов более высоких уровней (программных sc-агентов и sc-метаагентов), реализованных на Языке SCP, в частности, обеспечивают соблюдение указанными агентами общих принципов синхронизации.}

\scnheader{абстрактный программный sc-агент}
\scntext{пояснение}{К \textbf{\textit{абстрактным программным sc-агентам}} относятся все \textit{абстрактные sc-агенты}, обеспечивающие основной функционал системы, то есть ее возможность решать те или иные задачи. Агенты данного класса должны работать в соответствии с общими принципами синхронизации деятельности субъектов в sc-памяти.}

\scnheader{абстрактный sc-метаагент}
\scntext{пояснение}{Задачей \textbf{\textit{абстрактных sc-метаагентов}} является координация деятельности \textit{абстрактных программных sc-агентов}, в частности, решение проблемы взаимоблокировок. Агенты данного класса могут быть реализованы на Языке SCP, однако для синхронизации их деятельности используются другие принципы, соответственно, для реализации таких агентов требуется Язык SCP другого уровня, типология операторов которого полностью аналогична типологии scp-операторов, однако эти операторы имеют другую операционную семантику, учитывающую отличия в принципах синхронизации (работы с \textit{блокировками*}). Программы такого языка будем называть \textit{scp-метапрограммами}, соответствующие им \mbox{\textit{процессы в sc-памяти} ---  \textit{scp-метапроцессами}}, операторы  --- \textit{scp-метаоператорами}.}

\scnheader{декомпозиция абстрактного sc-агента*}
\scniselement{отношение декомпозиции}
\scntext{пояснение}{Отношение \textbf{\textit{декомпозиции абстрактного sc-агента*}} трактует \textit{неатомарные абстрактные sc-агенты} как коллективы более простых \textit{абстрактных sc-агентов}, взаимодействующих через \textit{sc-память}.\\
    Другими словами, \textbf{\textit{декомпозиция абстрактного sc-агента*}} на \textit{абстрактные sc-агенты} более низкого уровня уточняет один из возможных подходов к реализации этого \textit{абстрактного sc-агента} путем построения коллектива более простых \textit{абстрактных sc-агентов}.}

\scnheader{sc-агент}
\scnidtf{агент над sc-памятью}
\scnsubset{субъект}
\scnrelfrom{семейство подмножеств}{абстрактный sc-агент}
\scntext{пояснение}{Под \textbf{\textit{sc-агентом}} понимается конкретный экземпляр (с теоретико-множественной точки зрения --- элемент) некоторого \textit{атомарного абстрактного sc-агента}, работающий в какой-либо конкретной интеллектуальной системе.\\
    Таким образом, каждый \textit{sc-агент} --- это субъект, способный выполнять некоторый класс однотипных действий либо только над \textit{sc-памятью}, либо над sc-памятью и внешней средой (для эффекторных \textit{sc-агентов}). Каждое такое действие инициируется либо состоянием или ситуацией в sc-памяти, либо состоянием или ситуацией во внешней среде (для рецепторных sc-агентов-датчиков),  соответствующей условию инициирования \textit{атомарного абстрактного sc-агента}, экземпляром которого является заданный \textit{sc-агент}. В данном случае можно провести аналогию между принципами объектно-ориентированного программирования, рассматривая \textit{атомарный абстрактный sc-агент} как класс, а конкретный \textit{sc-агент}  как экземпляр, конкретную имплементацию этого класса.\\
    Взаимодействие \textit{sc-агентов} осуществляется только через \textit{sc-память}. Как следствие, результатом работы любого \textit{sc-агента} является некоторое изменение состояния \textit{sc-памяти}, т.е. удаление либо генерация каких-либо \textit{sc-элементов}.\\
    В общем случае один \textit{sc-агент} может явно передать управление другому \textit{sc-агенту}, если этот \textit{sc-агент} априори известен. Для этого каждый \textit{sc-агент} в \textit{sc-памяти} имеет обозначающий его \textit{sc-узел}, с которым можно связать конкретную ситуацию в текущем состоянии базы знаний, которую инициируемый \textit{sc-агент} должен обработать.\\
    Однако далеко не всегда легко определить тот \textit{sc-агент}, который должен принять управление от заданного \textit{sc-агента}, в связи с чем описанная выше ситуация возникает крайне редко. Более того, иногда условие инициирования \textit{sc-агента} является результатом деятельности непредсказуемой группы \textit{sc-агентов}, равно как и одна и та же конструкция может являться условием инициирования целой группы \textit{sc-агентов}.\\
    При этом общаются через \textit{sc-память} не \textit{программы sc-агентов*}, а сами описываемые данными программами \textit{sc-агенты}.\\
    В процессе работы \textit{sc-агент} может сам для себя порождать вспомогательные \textit{sc-элементы}, которые сам же удаляет после завершения акта своей деятельности (это вспомогательные \textit{структуры}, которые используются в качестве информационных лесов только в ходе выполнения соответствующего акта деятельности и после завершения этого акта удаляются).}

\scnheader{активный sc-агент}
\scnsubset{sc-агент}
\scntext{пояснение}{Под \textbf{\textit{активным sc-агентом}} понимается \textit{sc-агент} ostis-системы, который реагирует на события, соответствующие его условию инициирования, и, как следствие, его \textit{первичному условию инициирования*}. Не входящие во множество \textbf{\textit{активных sc-агентов}} \textit{sc-агенты} не реагируют ни на какие события в \textit{sc-памяти}.}

\scnheader{ключевые sc-элементы sc-агента*}
\scntext{пояснение}{Связки отношения \textbf{\textit{ключевые sc-элементы sc-агента*}} связывают между собой \textit{sc-узел}, обозначающий \textit{абстрактный sc-агент} и \textit{sc-узел}, обозначающий множество \textit{sc-элементов}, которые являются ключевыми для данного \textit{абстрактного sc-агента}, то данные \textit{sc-элементы} явно упоминаются в рамках программ, реализующих данный \textit{абстрактный sc-агент}.}

\scnheader{программа sc-агента*}
\scntext{пояснение}{Связки отношения \textbf{\textit{программа sc-агента*}} связывают между собой \textit{sc-узел}, обозначающий \textit{атомарный абстрактный sc-агент} и \textit{sc-узел}, обозначающий множество программ, реализующих указанный \textit{атомарный абстрактный sc-агент}. В случае \textit{платформенно-независимого абстрактного sc-агента} каждая связка отношения \textit{программа sc-агента*} связывает \textit{sc-узел}, обозначающий указанный \textit{абстрактный sc-агент} с множеством \textit{scp-программ}, описывающих деятельность данного \textit{абстрактного sc-агента}. Данное множество содержит одну \textit{агентную scp-программу} и произвольное количество (может быть, и ни одной) \textit{scp-программ}, которые необходимы для выполнения указанной \textit{агентной scp-программы}.\\
    В случае \textit{платформенно-зависимого абстрактного sc-агента} каждая связка отношения \textit{программа \mbox{sc-агента*}} связывает \textit{sc-узел}, обозначающий указанный \textit{абстрактный sc-агент} с множеством файлов, содержащих исходные тексты программы на некотором внешнем языке программирования, реализующей деятельность данного \textit{абстрактного sc-агента}.}

\scnheader{первичное условие инициирования*}
\scntext{пояснение}{Связки отношения \textbf{\textit{первичное условие инициирования*}} связывают между собой \textit{sc-узел}, обозначающий \textit{абстрактный sc-агент} и бинарную ориентированную пару, описывающую первичное условие инициирования данного \textit{абстрактного sc-агента}, т.е. такую спецификацию \textit{ситуации} в \textit{sc-памяти}, возникновение которой побуждает \textit{sc-агента} перейти в активное состояние и начать проверку наличия своего полного условия инициирования.\\
    Первым компонентом данной ориентированной пары является знак некоторого класса \textit{элементарных событий в sc-памяти*}, например, \textit{событие добавления sc-дуги, выходящей из заданного sc-элемента*}.\\
    Вторым компонентом данной ориентированной пары является произвольный в общем случае \textit{sc-элемент}, с которым непосредственно связан указанный тип события в \textit{sc-памяти}, т.е., например, \textit{sc-элемент}, из которого выходит либо в который входит генерируемая либо удаляемая \textit{sc-дуга} либо \textit{файл}, содержимое которого было изменено.\\
    После того, как в \textit{sc-памяти} происходит некоторое событие, активизируются все \textit{активные sc-агенты}, \textbf{\textit{первичное условие инициирования*}} которых соответствует произошедшему событию.}

\scnheader{условие инициирования и результат*}
\scntext{пояснение}{Связки отношения \textbf{\textit{условие инициирования и результат*}} связывают между собой \textit{sc-узел}, обозначающий \textit{абстрактный sc-агент}, и бинарную ориентированную пару, связывающую условие инициирования данного \textit{абстрактного sc-агента} и результаты выполнения данного экземпляров данного \textit{sc-агента} в какой-либо конкретной системе.\\
    Указанную ориентированную пару можно рассматривать как логическую связку импликации, при этом на \textit{sc-переменные}, присутствующие в обеих частях связки, неявно накладывается квантор всеобщности, на \textit{sc-переменные}, присутствующие либо только в посылке, либо только в заключении неявно накладывается квантор существования.\\
    Первым компонентом указанной ориентированной пары является логическая формула, описывающая условие инициирования описываемого \textit{абстрактного sc-агента}, то есть конструкции, наличие которой в \textit{sc-памяти} побуждает \textit{sc-агент} начать работу по изменению состояния \textit{sc-памяти}. Данная логическая формула может быть как атомарной, так и неатомарной, в которой допускается использование любых связок логического языка.\\
    Вторым компонентом указанной ориентированной пары является логическая формула, описывающая возможные результаты выполнения описываемого абстрактного \textit{sc-агента}, то есть описание произведенных им изменений состояния \textit{sc-памяти}. Данная логическая формула может быть как атомарной, так и неатомарной, в которой допускается использование любых связок логического языка.}

\scnheader{описание поведения sc-агента}
\scnsubset{семантическая окрестность}
\scntext{пояснение}{\textbf{\textit{описание поведения sc-агента}} представляет собой \textit{семантическую окрестность}, описывающую деятельность \textit{sc-агента} до какой-либо степени детализации, однако такое описание должно быть строгим, полным и однозначно понимаемым. Как любая другая \textit{семантическая окрестность}, \textbf{\textit{описание поведения sc-агента}} может быть протранслировано на какие-либо понятные, общепринятые средства, не требующие специального изучения, например, на естественный язык.\\
    Описываемый \textit{абстрактный sc-агент} входит в соответствующее \textbf{\textit{описание поведения sc-агента}} под атрибутом \textit{ключевой sc-элемент\scnrolesign}.}

\bigskip
\end{scnsubstruct}
\scnendsegmentcomment{Понятие sc-агента и абстрактного sc-агента}

\scnsegmentheader{Принципы синхронизации деятельности sc-агентов}
\begin{scnsubstruct}
    
\scnheader{процесс в sc-памяти}
\scntext{примечание}{Понятия \textit{действие в sc-памяти} и \textit{процесс в sc-памяти} (информационный процесс, выполняемый агентом в семантической памяти), являются синонимичными, поскольку все процессы, протекающие в sc-памяти, являюся осознанными и выполняются каким-либо sc-агентами. Тем не менее, когда идет речь о синхронизации выполнения каких-либо преобразований в памяти компьютерной системы, в литературе принято использовать именно термины \textit{процесс}, взаимодействие процессов \cite{Dijkstra1972, Hoare1989}, в связи с чем будем использовать этот термин при описании принципов синхронизации деятельности sc-агентов при выполнении ими параллельных процессов в sc-памяти.}
\begin{scnsubdividing}
    \scnitem{процесс в sc-памяти, соответствующий платформенно-зависимому sc-агенту}
    \scnitem{scp-процесс}
\end{scnsubdividing}
\begin{scnsubdividing}
    \scnitem{scp-процесс, не являющийся scp-метапроцессом}
    \scnitem{scp-метапроцесс}
\end{scnsubdividing}

\scnheader{процесс в sc-памяти, соответствующий платформенно-зависимому sc-агенту}
\begin{scnsubdividing}
    \scnitem{процесс в sc-памяти, соответствующий платформенно-зависимому sc-агенту и не являющийся действием абстрактной scp-машины}
    \scnitem{действие абстрактной scp-машины
    \begin{scnindent}
        \scnsuperset{действие интерпретации scp-программы}
    \end{scnindent}}
\end{scnsubdividing}

\scnheader{блокировка*}
\scniselement{бинарное отношение}
\scntext{пояснение}{Для синхронизации выполнения \textit{процессов в sc-памяти} используется механизм блокировок. Отношение \textbf{\textit{блокировка*}} связывает знаки \textit{действий в sc-памяти} со знаками \textit{структур} (ситуативных), которые содержат элементы, заблокированные на время выполнения данного действия или на какую-то часть этого периода. Каждая такая \textit{структура} принадлежит какому-либо из \textit{типов блокировки}.\\
    Первым компонентом связок отношения \textbf{\textit{блокировка*}} является знак \textit{действия в sc-памяти}, вторым  знак заблокированной \textit{структуры}.}
\scnrelfrom{описание примера}{\scnfileimage[20em]{Contents/part_ps/images/sd_agents/lock.png}}

\scnheader{тип блокировки}
\scntext{пояснение}{Множество \textbf{\textit{тип блокировки}} содержит все возможные классы блокировок, т.е. \textit{структуры}, содержащие \textit{sc-элементы}, заблокированные каким-либо \textit{sc-агентом} на время выполнения им некоторого \textit{действия в sc-памяти}.}
\scnhaselement{полная блокировка}
\scnhaselement{блокировка на любое изменение}
\scnhaselement{блокировка на удаление}

\scnheader{полная блокировка}
\scntext{пояснение}{Каждая \textit{структура}, принадлежащая множеству \textbf{\textit{полная блокировка}} содержит \textit{sc-элементы}, просмотр и изменение (удаление, добавление инцидентных \textit{sc-коннекторов}, удаление самих \textit{sc-элементов}, изменение содержимого в  случае файла) которых запрещены всем \textit{sc-агентам}, кроме собственно \textit{sc-агента}, выполняющего соответствующее данной структуре \textit{действие в sc-памяти}, связанное с ней отношением \textit{блокировка*}.\\
    Для того, чтобы исключить возможность реализации \textit{sc-агентов}, которые могут внести изменения в конструкции, описывающие блокировки других \textit{sc-агентов}, все элементы этих конструкций, в том числе, сам знак \textit{структуры}, содержащей заблокированные \textit{sc-элементы} (принадлежащей как множеству \textbf{\textit{полная блокировка}}, так и любому другому \textit{типу блокировки}) и связки отношения \textit{блокировка*}, связывающие эту \textit{структуру} и конкретное \textit{действие в sc-памяти}, добавляются в \textbf{\textit{полную блокировку}}, соответствующую данному \textit{действию в sc-памяти}. Таким образом, каждой \textbf{\textit{полной блокировке}} соответствует петля принадлежности, связывающая ее знак с самим собой.}

\scnheader{блокировка на любое изменение}
\scntext{пояснение}{Каждая \textit{структура}, принадлежащая множеству \textbf{\textit{блокировка на любое изменение}} содержит \textit{sc-элементы}, изменение (физическое удаление, добавление инцидентных \textit{sc-коннекторов}, физическое удаление самих \textit{\mbox{sc-элементов}}, изменение содержимого в случае файл) которых запрещено всем \textit{sc-агентам}, кроме собственно \textit{sc-агента}, выполняющего соответствующее данной структуре \textit{действие в sc-памяти}, связанное с ней отношением \textit{блокировка*}. Однако не запрещен просмотр (чтение) этих \textit{sc-элементов} любым \textit{sc-агентом}.}

\scnheader{блокировка на удаление}
\scntext{пояснение}{Каждая \textit{структура}, принадлежащая множеству \textbf{\textit{блокировка на удаление}} содержит \textit{sc-элементы}, удаление которых запрещено всем \textit{sc-агентам}, кроме собственно \textit{sc-агента}, выполняющего соответствующее данной структуре \textit{действие в sc-памяти}, связанное с ней отношением \textit{блокировка*}. Однако не запрещен просмотр (чтение) этих \textit{sc-элементов} любым \textit{sc-агентом}, добавление инцидентных sc-коннекторов.}

\scnheader{блокировка*}
\begin{scnrelfromset}{принципы работы}
    \scnfileitem{В каждый момент времени одному процессу в sc-памяти может соответствовать только одна блокировка каждого типа.}
    \scnfileitem{В каждый момент времени одному процессу в sc-памяти может соответствовать только одна блокировка, установленная на некоторый конкретный sc-элемент.}
    \scnfileitem{При завершении выполнения любого процесса в sc-памяти все установленные им блокировки автоматически снимаются.}
    \scnfileitem{Для повышения эффективности работы системы в целом каждый процесс должен в каждый момент времени блокировать минимально необходимое множество sc-элементов, снимая блокировку с каждого sc-элемента сразу же, как это становится возможным (безопасным).}
    \scnfileitem{В случае когда в рамках \textit{процесса в sc-памяти} явно выделяются более частные подпроцессы (при помощи отношений \textit{темпоральная часть*, поддействие*, декомпозиция действия*} и т. д.), то каждый такой подпроцесс с точки зрения синхронизации выполнения рассматривается как самостоятельный процесс, которому в соответствие могт быть поставлены все необходимые блокировки.}
    \begin{scnindent}
        \begin{scnrelfromlist}{детализация}
            \scnfileitem{Все дочерние процессы в sc-памяти имеют доступ к блокировкам родительского процесса так же, как если бы это были блокировки соответствующие каждому из таких дочерних процессов.}
            \scnfileitem{В свою очередь, родительский процесс не имеет какого-либо привилегированного доступа к sc-элементам, заблокированным дочерними процессами, и работает с ними так же, как любой другой процесс в sc-памяти. Исключение составляют sc-элементы, обозначающие сами дочерние процессы, поскольку родительский процесс должен иметь возможность управления дочерним, например, приостановки или прекращения их выполнения.}
            \scnfileitem{Все дочерние процессы по отношению друг к другу работают так же, как и по отношению к любым другим процессам.}
            \scnfileitem{В случае, когда родительский процесс приостанавливает выполнение (становится \textit{отложенным действием}), \uline{все} его дочерние процессы также приостанавливают выполнение. В свою очередь, приостановка одного из дочерних процессов в общем случае не инициирует явно остановку всего родительского процесса и соответственно других дочерних.}
        \end{scnrelfromlist}
    \end{scnindent}
\end{scnrelfromset}

\scnheader{полная блокировка}
\begin{scnrelfromset}{принципы работы}
    \scnfileitem{Если sc-элемент, инцидентный некоторому sc-коннектору, попадает в какую-либо полную блокировку, то сам этот sc-коннектор по умолчанию также считается заблокированным этой же блокировкой. Обратное в общем случае неверно, т. к. часть sc-коннекторов, инцидентных некоторому sc-элементу, может быть полностью заблокирована, при этом сам этот элемент заблокирован не будет. Такая ситуация типична, например, для sc-узлов, обозначающих классы понятий.}
    \scnfileitem{Каждый процесс в sc-памяти может свободно изменять или удалять любые sc-элементы, попадающие в полную блокировку, соответствующую этому процессу.}
\end{scnrelfromset}
    \begin{scnindent}
        \scntext{примечание}{Принципы работы с \textit{полными блокировками}, с одной стороны, наиболее просты, поскольку все процессы, кроме установившего такую блокировку, не имеют доступа к заблокированным \mbox{sc-элементам} и конфликты возникнуть не могут. С другой стороны, частое использование блокировок такого типа может привести к тому, что система не сможет использовать в полной мере имеющиеся у нее знания и давать неполные или даже некорректные ответы на поставленные вопросы.}
    \end{scnindent}

\scnheader{блокировка на любое изменение}
\begin{scnrelfromset}{принципы работы}
    \scnfileitem{На один и тот же sc-элемент в один момент времени может быть установлена только одна блокировка одного типа, но разные процессы могут одновременно установить на один и тот же элемент блокировки двух разных типов. Это касается случая, когда первый процесс установил на некоторый sc-элемент блокировку на удаление, а второй процесс затем устанавливает блокировку на любое изменение. В других случаях возникает конфликт блокировок.}
    \scnfileitem{Установка блокировки любого типа также считается изменением, таким образом, если на некоторый \mbox{sc-элемент} была установлена блокировка на любое изменение, то другой процесс не сможет установить на этот же sc-элемент блокировку любого типа, пока первый процесс не снимет свою.}
    \scnfileitem{Если блокировка на удаление устанавливается на некоторый sc-коннектор, то по умолчанию та же блокировка устанавливается на инцидентные этому sc-коннектору sc-элементы, поскольку удаление этих элементов приведет к удалению этого коннектора.}
\end{scnrelfromset}
    \begin{scnindent}
        \scnrelto{принципы работы}{блокировка на удаление}
    \end{scnindent}

\scnheader{процесс в sc-памяти}
\scnidtf{действие в sc-памяти}
\scnrelfrom{разбиение}{Классификация процессов в sc-памяти с точки зрения синхронизации их выполнения}
\begin{scnindent}
    \begin{scneqtoset}
        \scnitem{действие поиска sc-элементов}
        \scnitem{действие генерации sc-элементов}
        \scnitem{действие удаления sc-элементов}
        \scnitem{действие установки блокировки некоторого типа на некоторый sc-элемент}
        \scnitem{действие снятия блокировки с некоторого sc-элемента}
    \end{scneqtoset}
\end{scnindent}

\scnheader{транзакция в sc-памяти}
\scntext{пояснение}{В некоторых случаях для того, чтобы обеспечить синхронизацию, необходимо объединять несколько элементарных действий над sc-памятью в одно неделимое действие (\textit{транзакцию в sc-памяти}), для которого гарантируется, что ни один сторонний процесс не сможет прочитать или изменить участвующие в этом действии sc-элементы, пока действие не завершится. При этом, в отличие от ситуации с полной блокировкой, процесс, пытающийся получить доступ к таким элементам, не продолжает выполнение так, как если бы этих элементов просто не было в sc-памяти, а ожидает завершения транзакции, после чего может выполнять с данными элементами любые действия согласно общим принципам синхронизации процессов. Проблема обеспечения транзакций не может быть решена на уровне SC-кода и требует реализации таких неделимых действий на уровне \textit{платформы интерпретации sc-моделей}.}

\scnheader{действие поиска sc-элементов}
\scntext{пояснение}{В случае осуществления поиска все найденные и сохраненные в рамках какого-либо процесса sc-элементы попадают в соответствующую данному процессу \textit{блокировку на любое изменение}. Таким образом, гарантируется целостность фрагмента базы знаний, с которым работает некоторый процесс в sc-памяти. При этом поиск и автоматическая установка такой блокировки должны быть реализованы как \textit{транзакция в sc-памяти}.\\
    Такой подход также позволяет избежать ситуации, когда один процесс заблокировал некоторый sc-элемент на любое изменение, а второй процесс пытается сгенерировать или удалить \textit{sc-коннектор}, инцидентный данному \textit{sc-элементу}. В таком случае второй процесс должен будет предварительно найти и заблокировать указанный \textit{sc-элемент} на любое изменение, что вызовет конфликт блокировок (\textit{взаимоблокировку*}).}

\scnheader{действие генерации sc-элементов}
\scntext{пояснение}{В случае генерации любого sc-элемента в рамках некоторого процесса он автоматически попадает в полную блокировку, соответствующую данному процессу. При этом генерация и автоматическая установка такой блокировки должны быть реализованы как \textit{транзакция в sc-памяти}. При необходимости сгенерированные элементы могут быть удалены (т. е. их временное существование вообще никак не отразится на деятельности других процессов) или разблокированы в случае, когда сгенерирована информация, которая может иметь некоторую ценность в дальнейшем.}

\scnheader{действие установки блокировки некоторого типа на некоторый sc-элемент}
\scntext{пояснение}{В случае, если какой-либо процесс пытается установить блокировку любого типа на какой-либо sc-элемент, уже заблокированный каким-либо другим процессом, то, с одной стороны, блокировка не может быть установлена, пока другой процесс не разблокирует указанный sc-элемент; с другой стороны, для того чтобы обеспечить возможность поиска и устранения \textit{взаимоблокировок}, необходимо явно указывать тот факт, что какой-либо процесс хочет получить доступ к какому-либо заблокированному другим процессом sc-элементу. Для того чтобы иметь возможность указать, какие процессы пытаются заблокировать уже заблокированный \textit{sc-элемент}, предлагается наряду с отношением \textit{блокировка*} использовать отношение \textit{планируемая блокировка*}, полностью аналогичное отношению \textit{блокировка*}.\\
    Описанный механизм регулирует также и процессы поиска, поскольку поиск и сохранение некоторого sc-элемента предполагает установку \textit{блокировки на любое изменение}. Кроме того, следует учитывать, что на один sc-элемент \textit{блокировка на любое изменение} может быть установлена после \textit{блокировки на удаление}, соответствующей другому процессу. В этом случае использовать отношение \textit{планируемые блокировки*} нет необходимости.}
\scntext{примечание}{Действие проверки наличия на некотором sc-элементе блокировки и в зависимости от результата проверки, установки блокировки или планируемой блокировки (с указанием приоритета при необходимости) должно быть реализовано как транзакция.}

\scnheader{планируемая блокировка*}
\scnsubset{блокировка*}
\scntext{пояснение}{Процесс, которому в соответствие поставлена \textit{планируемая блокировка*}, приостанавливает выполнение до тех пор, пока уже установленные блокировки не будут сняты, после чего \textit{планируемая блокировка*} становится реальной \textit{блокировкой*} и процесс продолжает выполнение в соответствии с общими правилами.}

\scnheader{приоритет блокировки*}
\scnrelfrom{область определения}{планируемая блокировка*}
\scntext{пояснение}{В случае, когда на один и тот же sc-элемент планируют установить блокировку сразу несколько процессов, используется отношение \textit{приоритет блокировки*}, связывающее между собой пары отношения \textit{планируемая блокировка*}. Как правило, приоритет блокировки определяется тем, какой из процессов раньше попытался установить блокировку на рассматриваемый sc-элемент, хотя в общем случае приоритет может устанавливаться или меняться в зависимости от дополнительных критериев.}

\scnheader{действие удаления sc-элементов}
\scntext{примечание}{В случае попытки удаления некоторого sc-элемента некоторым процессом удаление может быть осуществлено только в случае, когда на данный sc-элемент не установлена (и не планируется) ни одна блокировка каким-либо другим процессом.\\
    В других случаях необходимо обеспечить корректное завершение выполнения всех процессов, работающих с данным sc-элементом, и только потом удалить его физически.\\
    Для реализации такой возможности каждому процессу в соответствие может быть поставлено множество удаляемых данным процессом sc-элементов.}
\scntext{примечание}{Действие проверки наличия блокировок или планируемых блокировок на удаляемый sc-элемент и, собственно, его удаление или добавление во множество удаляемых sc-элементов для соответствующего процесса должно быть реализовано как транзакция.}

\scnheader{удаляемые sc-элементы*}
\scnrelfrom{первый домен}{процесс в sc-памяти}
\scntext{пояснение}{Sc-элементы, попавшие во множество удаляемых sc-элементов некоторого процесса в sc-памяти, доступны процессам, уже установившим (или планирующим установить) на эти sc-элементы блокировки ранее (до попытки его удаления), а для всех остальных процессов эти sc-элементы уже считаются удаленными. Процесс, пытающийся удалить sc-элемент, приостанавливает свое выполнение до того момента, пока все заблокировавшие и планирующие заблокировать данный sc-элемент процессы не разблокируют его. В общем случае один sc-элемент может входить во множества удаляемых элементов одновременно для нескольких процессов, в этом случае все такие процессы одновременно продолжат выполнение после снятия с этого sc-элемента всех блокировок. Если удаление пытается осуществить один из процессов, уже установивший на указанный sc-элемент блокировку, то алгоритм действий остается прежним --- sc-элемент добавляется во множество удаляемых данным процессом sc-элементов и будет физически удален, как только все остальные процессы, установившие на данный sc-элемент блокировки, снимут их.}

\scnheader{действие снятия блокировки с некоторого sc-элемента}
\begin{scnrelfromvector}{алгоритм выполнения}
    \scnfileitem{Если на данный sc-элемент установлена одна или несколько \textit{планируемых блокировок*}, то первая из них по приоритету (или единственная) становится \textit{блокировкой*}, соответствующий ей процесс продолжает выполнение (становится настоящей сущностью).}
    \scnfileitem{Связка отношения приоритет выполнения, соответствовавшая удаленной связке отношения \textit{планируемая блокировка*} также удаляется, т. е. приоритет смещается на одну позицию.}
    \scnfileitem{Если \textit{планируемых блокировок*}, установленных на данный sc-элемент, нет, но он попадает во множество удаляемых sc-элементов для одного или нескольких процессов, то рассматриваемый sc-элемент физически удаляется, а приостановленные до его удаления процессы продолжают свое выполнение (становятся настоящими сущностями).}
    \scnfileitem{Если на данный sc-элемент не установлены планируемые блокировки, и он не входит во множество удаляемых для какого-либо процесса, то блокировка просто снимается без каких-либо дополнительных изменений.}
\end{scnrelfromvector}

\scnheader{транзакция в sc-памяти}
\begin{scnsubdividing}
    \scnitem{поиск некоторой конструкции в sc-памяти и автоматическая установка блокировки на любое изменение на найденные sc-элементы}
    \scnitem{генерация некоторого sc-элемента и автоматическая установка на него полной блокировки}
    \scnitem{проверка наличия на некотором sc-элементе блокировки и в зависимости от результата проверки установка блокировки или планируемой блокировки}
    \scnitem{проверка наличия блокировок или планируемых блокировок на удаляемый sc-элемент и собственно его удаление или добавление во множество удаляемых sc-элементов для соответствующего процесса}
    \scnitem{снятие блокировки с заданного sc-элемента и при необходимости установка первой по приоритету планируемой блокировки или удаление данного sc-элемента, если он входит во множество удаляемых sc-элементов для некоторого процесса}
    \scnitem{поиск подпроцессов процесса и добавление их во множество отложенных действий в случае добавления самого процесса в данное множество}
    \scnitem{поиск подпроцессов процесса и удаление их из множества отложенных действий в случае удаления самого процесса из данного множества}
\end{scnsubdividing}

\scnheader{абстрактный программный sc-агент}
\scntext{примечание}{При реализации \textit{абстрактных программных sc-агентов} на \textit{языке SCP}, соблюдение всех принципов синхронизации соответствующих этим sc-агентам процессов обеспечивается на уровне \textit{sc-агентов интерпретации scp-программ}, т. е. средствами \textit{платформы интерпретации sc-моделей}. При реализации \textit{абстрактных программных sc-агентов} на уровне платформы, соблюдение всех принципов синхронизации возлагается, во-первых, непосредственно на разработчика агентов, во-вторых, --- на разработчика платформы. Так, например, платформа может предоставлять доступ к хранимым в sc-памяти элементам через некоторый программный интерфейс, уже учитывающий принципы работы с блокировками, что избавит разработчика агентов от необходимости учитывать все эти принципы вручную.}
\begin{scnrelfromset}{принципы работы}
    \scnfileitem{В результате появления в sc-памяти некоторой конструкции, удовлетворяющей условию инициирования какого-либо \textit{абстрактного sc-агента}, реализованного при помощи \textit{Языка SCP}, в \textit{sc-памяти} генерируется и инициируется \textit{scp-процесс}. В качестве шаблона для генерации используется \textit{агентная scp-программа}, соответствующая данному \textit{абстрактному sc-агенту}.}
    \scnfileitem{Каждый такой \textit{scp-процесс}, соответствующий некоторой \textit{агентной \mbox{scp-программе}}, может быть связан с набором структур, описывающих блокировки различных типов. Таким образом, синхронизация взаимодействия параллельно выполняемых \textit{scp-процесcов} осуществляется так же, как и в случае любых других \textit{действий в sc-памяти}.}
    \scnfileitem{Несмотря на то что каждый \textit{scp-оператор} представляет собой атомарное действие в sc-памяти, являющееся поддействием в рамках всего \textit{\mbox{scp-процесса}}, блокировки, соответствующие одному оператору, не вводятся, чтобы избежать громоздкости и избытка дополнительных системных конструкций, создаваемых при выполнении некоторого \textit{scp-процесса}. Вместо этого используются блокировки, общие для всего \textit{scp-процесса}. Таким образом, \textit{агенты интерпретации scp-программ} работают только с учетом блокировок, общих для всего интерпретируемого \textit{scp-процесса}.}
    \scnfileitem{Процессы, описывающие деятельность агентов интерпретации \textit{scp-программ}, как правило, не создаются, следовательно, и не вводятся соответствующие им блокировки. Поскольку такие агенты работают с уникальным scp-процессом и их число ограничено и известно, то использование блокировок для их синхронизации не требуется.}
    \scnfileitem{В случае приостановки \textit{scp-процесса} (добавления его во множество \textit{отложенных действий}) в соответствии с общими правилами синхронизации все его дочерние процессы также должны быть приостановлены. В связи с этим все \textit{scp-операторы}, которые в этот момент являются \textit{настоящими сущностями}, становятся \textit{отложенными действиями}.}
    \scnfileitem{Во избежание нежелательных изменений в самом теле \textit{scp-процесса}, вся конструкция, сгенерированная на основе некоторой \textit{scp-программы} (весь \textit{sc-текст}, описывающий декомпозицию \textit{scp-процесса} на \textit{scp-операторы}), должна быть добавлена в \textit{полную блокировку}, соответствующую данному \textit{scp-процессу}.}
    \scnfileitem{При необходимости разблокировать или заблокировать некоторую конструкцию каким-либо типом блокировки используются соответствующие \textit{scp-операторы} класса \textit{scp-оператор управления блокировками}.}
    \scnfileitem{После завершения выполнения некоторого scp-процесса его текст, как правило, удаляется из \textit{sc-памяти}, а все заблокированные конструкции освобождаются (разрушаются знаки структур, обозначавших блокировки).}
    \scnfileitem{Как правило, частный \textit{класс действий}, соответствующий конкретной \textit{scp-программе}, явно не вводится, а используется более общий класс \textit{scp-процесс}, за исключением тех случаев, когда введение специального \textit{класса действий} необходимо по каким-либо другим соображениям.}
\end{scnrelfromset}

\scnheader{блокировка*}
\scntext{примечание}{В общем случае весь механизм блокировок может описываться как на уровне SC-кода (для повышения уровня платформенной независимости), так и при необходимости может быть реализован на уровне \textit{платформы интерпретации sc-моделей}, например, для повышения производительности. Для этого каждому выполняемому в sc-памяти процессу на нижнем уровне может быть поставлена в соответствие некая уникальная таблица, в каждый момент времени содержащая перечень заблокированных элементов с указанием типа блокировки.}
\begin{scnrelfromvector}{пример применения}
    \scnitem{\scnfileimage[20em]{Contents/part_ps/images/sd_agents/plan_lock_1.png}}
    \begin{scnindent}
        \scntext{пояснение}{В данном примере \textit{Процесс1} непосредственно работает с sc-элементом \textit{\textbf{e1}}, \textit{Процесс2} и \textit{Процесс3} планируют установить блокировку на любое изменение и блокировку на удаление соответственно, причем \textit{Процесс2} попытался установить свою блокировку раньше, чем \textit{Процесс3}, поэтому согласно направлению связки отношения \textit{приоритет блокировки*}, его блокировка будет установлена раньше. \textit{Процесс4} и \textit{Процесс5} ожидают снятия всех блокировок и планируемых блокировок, после чего \textit{\textbf{e1}} будет удален и \textit{Процесс1} и \textit{Процесс2} продолжат свое выполнение. Никакие другие планируемые блокировки установлены быть уже не могут, поскольку \textit{\textbf{e1}} попал во множество удаляемых sc-элементов как минимум одного процесса и, в соответствии с изложенными выше правилами, все остальные процессы кроме \textit{Процесс1}-\textit{Процесс5}, уже не смогут получить доступ к этому sc-элементу. Выполняемый процесс принадлежит множеству настоящая сущность, приостановленные --- множеству отложенное действие.}
    \end{scnindent}
    \scnitem{\scnfileimage[20em]{Contents/part_ps/images/sd_agents/plan_lock_2.png}}
    \begin{scnindent}
        \scntext{пояснение}{После того как \textit{Процесс1} разблокировал sc-элемент \textit{\textbf{e1}}, этот элемент будет заблокирован \textit{Процессом2}, и \textit{Процесс2} продолжит выполнение. \textit{Планируемая блокировка*}, установленная \textit{Процессом2}, становится обычной \textit{блокировкой*}.}
    \end{scnindent}
    \scnitem{\scnfileimage[20em]{Contents/part_ps/images/sd_agents/plan_lock_3.png}}
    \begin{scnindent}
        \scntext{пояснение}{После того как \textit{Процесс2} разблокировал sc-элемент \textit{\textbf{e1}}, этот элемент будет заблокирован \textit{Процессом3}, и \textit{Процесс3} продолжит выполнение.}
    \end{scnindent}
    \scnitem{\scnfileimage[20em]{Contents/part_ps/images/sd_agents/plan_lock_4.png}}
    \begin{scnindent}
        \scntext{пояснение}{Когда все процессы снимут блокировки с sc-элемента \textit{\textbf{e1}}, он может быть физически удален и \textit{Процесс4} и \textit{Процесс5} продолжат выполнение.}
    \end{scnindent}
\end{scnrelfromvector}

\scnheader{взаимоблокировка*}
\scntext{пояснение}{В зависимости от конкретных \textit{типов блокировок}, установленных параллельно выполняемыми процессами на некоторые sc-элементы, и того, какие конкретно действия с этими \textit{sc-элементами} предполагается выполнить, далее в рамках выполнения этих процессов возможны ситуации взаимоблокировки, когда каждый из указанных процессов будет ожидать снятия блокировки вторым процессом с нужного \textit{sc-элемента}, не снимая при этом установленной им самим блокировки с \textit{sc-элемента}, доступ к которому необходим второму процессу.\\
    В случае, когда хотя бы одна из блокировок является \textit{полной блокировкой}, ситуация взаимоблокировки возникнуть не может, поскольку \textit{sc-элементы}, попавшие в \textit{полную блокировку} некоторого \textit{scp-процесса}, не доступны другим \textit{scp-процессам} даже для чтения и, таким образом, остальные \textit{scp-процессы} будут работать так, как будто заблокированные \textit{sc-элементы} просто отсутствуют в текущем состоянии \textit{sc-памяти}.\\
    В случаях, когда ни одна из установленных блокировок не является \textit{полной блокировкой}, возможно появление взаимоблокировок.}
\scntext{примечание}{Устранение \textit{взаимоблокировки} невозможно без вмешательства специализированного \textit{sc-метаагента}, который имеет право игнорировать блокировки, установленные другими процессами.\\
    В общем случае проблема конкретной взаимоблокировки может быть решена путем выполнения специализированным \textit{sc-метаагентом} следующих шагов:
        \begin{scnitemize}
            \item откат нескольких операций, выполненных одним из участвующих во взаимоблокировке процессов на столько шагов назад, насколько это необходимо для того, чтобы второй процесс получил доступ к необходимым \textit{sc-элементам} и смог продолжить выполнение;
            \item ожидание выполнения второго процесса вплоть до завершения или до снятия им всех блокировок с \textit{sc-элементов}, доступ к которым необходимо получить первому процессу;
            \item повторное выполнение в рамках первого процесса отмененных операций и продолжение его выполнения, но уже с учетом изменений в памяти, внесенных вторым процессом.
        \end{scnitemize}}

\scnheader{sc-метаагент}
\scntext{пояснение}{Для \textit{sc-метаагентов} все sc-элементы, в том числе описывающие блокировки, планируемые блокировки и т. д., полностью эквивалентны между собой с точки зрения доступа к ним, т. е. любой \textit{sc-метаагент} имеет доступ к любым sc-элементам, даже попавшим в полную блокировку для какого-либо другого процесса. Это необходимо для того, чтобы \textit{sc-метаагенты} смогли выявлять и устранять различные проблемы, например, описанную выше проблему взаимоблокировки.\\
    Таким образом, проблема синхронизации деятельности \textit{sc-метаагентов} требует введения дополнительных правил.\\
    Указанную проблему разделим на две более частные:
        \begin{scnitemize}
            \item обеспечение синхронизации деятельности \textit{sc-метаагентов} между собой;
            \item обеспечение синхронизации деятельности \textit{sc-метаагентов} и \textit{программных sc-агентов}.
        \end{scnitemize}
    Первую проблему предлагается решить за счет запрета параллельного выполнения \textit{sc-метаагентов}. Таким образом, в каждый момент времени в рамках одной \textit{ostis-системы} может существовать только один процесс, соответствующий \textit{sc-метаагенту} и являющийся \textit{настоящей сущностью}.\\
    Вторую проблему предлагается решить за счет введения дополнительных привилегий для \textit{sc-метаагентов} при обращении к какому-либо sc-элементу. Для этого достаточно одного правила:\\
    Если некоторый sc-элемент стал использоваться в рамках процесса, соответствующего \textit{sc-метаагенту} (например, стал элементом хотя бы одного scp-оператора, входящего в данный процесс), то все процессы, которым в соответствующие блокировки   попадает указанный sc-элемент, становятся отложенными действиями (приостанавливают выполнение). Как только указанный sc-элемент перестает использоваться в рамках процесса, соответствующего \textit{sc-метаагенту}, все приостановленные по этой причине процессы продолжают выполнение.\\
    Рассмотренные ограничения не ухудшают производительность ostis-системы существенно, поскольку \textit{sc-метаагенты} предназначены для решения достаточно узкого класса задач, которые, как показал опыт практической разработки прототипов различных \textit{ostis-систем}, возникают достаточно редко.}
    \scntext{примечание}{Стоит отметить, что возможна ситуация, при которой выполнение некоторого процесса в sc-памяти прервано по причине возникновения какой-либо ошибки. В таком случае существует вероятность того, что блокировка, установленная данным процессом не будет снята до тех пор, пока этого не сделает sc-метаагент, обнаруживший подобную ситуацию. Однако указанная проблема на уровне sc-модели может быть решена лишь частично. Для случаев, когда ошибка возникает при интерпретации scp-программы, проблема отслеживается scp-интепретатором и в памяти формируется соответствующая конструкция, сообщающая о проблеме sc-метаагенту. Случаи, когда возникла ошибка на уровне scp-интерпретатора или sc-хранилища, должны рассматриваться на уровне платформы интерпретации sc-моделей.}

\bigskip
\end{scnsubstruct}
\scnendsegmentcomment{Принципы синхронизации деятельности sc-агентов}

\bigskip
\end{scnsubstruct}
\scnendcurrentsectioncomment
\end{SCn}


\scsubsubsection{Пункт 30.1.1. Предметная область и онтология локальных предметных областей и онтологий действий}
\label{local_sd_actions}

\scsubsubsection{Пункт 30.1.2. Предметная область и онтология действий по управлению деятельностью многоагентных систем}
\label{local_sd_project_management}

\scsubsection{\S 30.2. Предметная область и онтология Базового языка программирования ostis-систем}
\label{sd_scp}
\begin{SCn}
\scnsectionheader{Предметная область и онтология Базового языка программирования ostis-систем}
\begin{scnsubstruct}
	\begin{scnrelfromlist}{дочерний раздел}
		\scnitem{\nameref{sd_scp_denote_sem}}
		\scnitem{\nameref{sd_scp_oper_sem}}
	\end{scnrelfromlist}
	
\scnheader{Предметная область Базового языка программирования ostis-систем (языка SCP --- Semantic Code Programming)}
\scnidtf{Предметная область Базового языка программирования ostis-систем}
\scnidtf{Предметная область Языка SCP}
\scntext{примечание}{В данную предметную область включаются все тексты программ Языка SCP. В ней исследуется типология операторов этих программ и заданные на них отношения.}
\scniselement{предметная область}
\begin{scnhaselementrole}{максимальный класс объектов исследования}
	{scp-программа}
\end{scnhaselementrole}
\begin{scnhaselementrolelist}{класс объектов исследования}
	%TODO: check by human--->
	\scnitem{агентная scp-программа}
	\scnitem{scp-процесс}
	\scnitem{scp-оператор}
	\scnitem{атомарный тип scp-оператора}
	%<---TODO: check by human
\end{scnhaselementrolelist}
\begin{scnhaselementrolelist}{исследуемое отношение}
	%TODO: check by human--->
	\scnitem{начальный оператор\scnrolesign}
	\scnitem{параметр scp-программы\scnrolesign}
	\scnitem{in-параметр\scnrolesign}
	\scnitem{out-параметр\scnrolesign}
	\scnitem{scp-операнд\scnrolesign}
	%<---TODO: check by human
\end{scnhaselementrolelist}

\scnheader{Язык SCP}
\scnidtftext{часто используемый sc-идентификатор}{scp-программа}
\scntext{пояснение}{В качестве базового языка для описания программ обработки текстов\textit{SC-кода} предлагается \textit{Язык SCP}.\\
	\textit{Язык SCP} --- это графовый язык процедурного программирования, предназначенный для эффективной обработки \textit{sc-текстов}. \textit{Язык SCP} является языком параллельного асинхронного программирования.\\
	Языком представления данных для текстов \textit{Языка SCP} (\textit{scp-программ}) является \textit{SC-код} и, соответственно, любые варианты его внешнего представления. \textit{Язык SCP} сам построен на основе \textit{SC-кода}, в следствие чего \textit{scp-программы} сами по себе могут входить в состав обрабатываемых данных для \textit{scp-программ}, в т.ч. по отношению к самим себе. Таким образом, \textit{язык SCP} предоставляет возможность построения реконфигурируемых программ. Однако для обеспечения возможности реконфигурирования программы непосредственно в процессе ее интерпретации необходимо на уровне интерпретатора \textit{Языка SCP (Aбстрактной scp-машины)} обеспечить уникальность каждой исполняемой копии исходной программы. Такую исполняемую копию, сгенерированную на основе \textit{scp-программы}, будем называть \textit{scp-процессом}. Включение знака некоторого \textit{действия в sc-памяти} во множество \textit{scp-процессов} гарантирует тот факт, что в декомпозиции данного действия будут присутствовать только знаки элементарных действий (\textit{scp-операторов}), которые может интерпретировать реализация \textit{Aбстрактной scp-машины} (интерпретатора scp-программ).\\
	\textit{Язык SCP} рассматривается как ассемблер для семантического компьютера.}

\scnheader{Базовая модель обработки sc-текстов}
\begin{scnreltoset}{объединение}
	%TODO: check by human--->
	\scnitem{Предметная область Базового языка программирования ostis-систем}
	\scnitem{Модель Абстрактной scp-машины}
	%<---TODO: check by human
\end{scnreltoset}
\begin{scnrelfromset}{особенности}
	%TODO: check by human--->
	\scnfileitem{Тексты программ \textit{Языка SCP} записываются при помощи тех же унифицированных семантических сетей, что и обрабатываемая информация, таким образом, можно сказать, что \textit{Синтаксис языка SCP} на базовом уровне совпадает с \textit{Синтаксисом SC-кода}.}
	\scnfileitem{Подход к интерпретации \textit{scp-программ} предполагает создание при каждом вызове \textit{scp-программы} уникального \textit{scp-процесса}.}
	%<---TODO: check by human
\end{scnrelfromset}
\begin{scnrelfromset}{достоинства}
	%TODO: check by human--->
	\scnfileitem{Одновременно в общей памяти могут выполняться несколько независимых\textit{sc-агентов}, при этом разные копии \textit{sc-агентов} могут выполняться на разных серверах, за счет распределенной реализации интерпретатора sc-моделей (\textit{платформы реализации sc-моделей компьютерных систем}). Более того, \textit{Язык SCP} позволяет осуществлять параллельные асинхронные вызовы подпрограмм с последующей синхронизацией, и даже параллельно выполнять операторы в рамках одной \textit{scp-программы}.}
	\scnfileitem{Перенос \textit{sc-агента} из одной системы в другую заключается в простом переносе фрагмента базы знаний, без каких-либо дополнительных операций, зависящих от платформы интерпретации.}
	\scnfileitem{Тот факт, что спецификации \textit{sc-агентов} и их программы могут быть записаны на том же языке, что и обрабатываемые знания, существенно сокращает перечень специализированных средств, предназначенных для проектирования машин обработки знаний, и упрощает их разработку за счет использования более универсальных компонентов.}
	\scnfileitem{Тот факт, что для интерпретации \textit{scp-программы} создается соответствующий ей уникальный \textit{\mbox{scp-процесс}}, позволяет по возможности оптимизировать план выполнения перед его реализацией и даже непосредственно в процессе выполнения без потенциальной опасности испортить общий универсальный алгоритм всей программы. Более того, такой подход к проектированию и интерпретации программ позволяет говорить о возможности создания самореконфигурируемых программ.}
	%<---TODO: check by human
\end{scnrelfromset}

\scnheader{Абстрактная scp-машина}
\scnrelfrom{модель}{Модель Абстрактной scp-машины}
\scntext{примечание}{\textit{Абстрактная scp-машина} представляет собой интерпретатор \textit{scp-программ}, который должен являться частью \textit{платформы интерпретации sc-моделей компьютерных систем} (хотя в общем случае могут существовать варианты платформы, не содержащие такого интерпретатора, что, однако, не позволит использовать достоинства предлагаемой базовой модели}

\scnheader{scp-программа}
\scnsubset{программа в sc-памяти}
\scntext{пояснение}{Каждая \textbf{\textit{scp-программа}} представляет собой \textit{обобщенную структуру}, описывающую один из вариантов декомпозиции действий некоторого класса, выполняемых в sc-памяти. Знак \textit{sc-переменной}, соответствующей конкретному декомпозируемому действию является в рамках \textbf{\textit{scp-программы}} \textit{ключевым sc-элементом\scnrolesign}. Также явно указывается принадлежность данного знака множеству \textit{scp-процессов}.\\
	Принадлежность некоторого действия множеству \textit{scp-процессов} гарантирует тот факт, что в декомпозиции данного действия будут присутствовать только знаки элементарных действий (\textit{scp-операторов}), которые может интерпретировать реализация абстрактной scp-машины.\\
	Таким образом, каждая \textbf{\textit{scp-программа}} описывает в обобщенном виде декомпозицию некоторого \textit{\mbox{scp-процесса}} на взаимосвязанные \textit{scp-операторы}, с указанием, при их наличии, аргументов для данного \textit{scp-процесса}.\\
	По сути каждая \textbf{\textit{scp-программа}} представляет собой описание последовательности элементарных операций, которые необходимо выполнить над семантической сетью, чтобы выполнить более сложное действие некоторого класса.}
\scnrelfrom{описание примера}{\scnfileimage[20em]{Contents/part_ps/images/sd_scp/program_example.png}}
	\begin{scnindent}
		\scntext{пояснение}{В приведенном примере показана \textit{scp-программа}, состоящая из трех \textit{scp-операторов}. Данная программа проверяет, содержится ли в заданном множестве (первый параметр) заданный элемент (второй параметр), и, если нет, то добавляет его в это множество.}
	\end{scnindent}

\begin{scnhaselementrolelist}{пример}
	\scnitem{\_scp\_process}
	\scnisvarelement{scp-процесс}
	\scnhasvarelementrole{1;in-параметр}{\_set1}
	\scnhasvarelementrole{2;in-параметр}{\_element1}
	\scnvarrelto{декомпозиция действия}{\_...}
	\begin{scnindent}
		\scnhasvarelementrole{1}{\_ operator1}
		\begin{scnindent}
			\scnisvarelement{searchElStr3}
			\scnhasvarelementrole{1; scp-операнд с заданным значением; scp-константа}{\_set1}
			\scnhasvarelementrole{2; scp-операнд со свободным значением; scp-переменная; sc-дуга основного вида}{\_arc1}
			\scnhasvarelementrole{3; scp-операнд с заданным значением; scp-константа}{\_element1}
			\scnvarrelfrom{последовательность действий при отрицательном результате}{\_operator2}
			\scnvarrelfrom{последовательность действий при положительном результате}{\_operator3}
		\end{scnindent}
		\scnhasvarelement{\_operator2}
		\begin{scnindent}
			\scnisvarelement{genElStr3}
			\scnhasvarelementrole{1:: scp-операнд с заданным значением; scp-константа}{\_set1}
			\scnhasvarelementrole{2:: scp-операнд со свободным значением; scp-переменная; sc-дуга основного вида}{\_arc1}
			\scnhasvarelementrole{3:: scp-операнд с заданным значением; scp-константа}{\_element1}
			\scnvarrelfrom{следующий оператор}{\_operator3}
		\end{scnindent}
		\scnhasvarelement{\_operator3}
		\begin{scnindent}
			\scnisvarelement{return}
		\end{scnindent}
	\end{scnindent}
\end{scnhaselementrolelist}
\end{scnsubstruct}

\scnheader{агентная scp-программа}
\scnsubset{scp-программа}
\scntext{пояснение}{\textbf{\textit{агентные scp-программы}} представляют собой частный случай \textit{scp-программ} вообще, однако заслуживают отдельного рассмотрения, поскольку используются наиболее часто. \textit{Scp-программы} данного класса представляют собой реализации программ агентов обработки знаний и имеют жестко фиксированный набор параметров. Каждая такая программа имеет ровно два \textit{in-параметра\scnrolesign}. Значение первого параметра является знаком бинарной ориентированной пары, являющейся вторым компонентом связки отношения \textit{первичное условие инициирования*} для абстрактного \textit{sc-агента}, во множество \textit{программ sc-агента*} которого входит рассматриваемая \textbf{\textit{агентная scp-программа}}, и, по сути, описывает класс событий, на которые реагирует указанный sc-агент.\\
	Значением второго параметра является \textit{sc-элемент}, с которым непосредственно связано событие, в результате возникновения которого был инициирован соответствующий \textit{sc-агент}, т.е., например, сгенерированная либо удаляемая \textit{sc-дуга} или \textit{sc-ребро}.}

\scnheader{абстрактный sc-агент, реализуемый на Языке SCP}
\begin{scnrelfromset}{принципы реализации}
	%TODO: check by human--->
	\scnfileitem{Общие принципы организации взаимодействия \textit{sc-агентов} и пользователей \textit{ostis-системы} через общую\textit{sc-память}.}
	\scnfileitem{В результате появления в sc-памяти некоторой конструкции,удовлетворяющей условию инициирования какого-либо \textit{абстрактногоsc-агента}, реализованного при помощи \textit{Языка SCP}, в \textit{sc-памяти} генерируется и инициируется \textit{scp-процесс}. В качестве шаблона для генерации используется \textit{агентная scp-программа}, указанная во множестве программ соответствующего \textit{абстрактного sc-агента}.}
	\scnfileitem{Каждый такой \textit{scp-процесс}, соответствующий некоторой \textit{агентной scp-программе}, может быть связан с набором структур, описывающих блокировки различных типов. Таким образом, синхронизация взаимодействия параллельно выполняемых \textit{scp-процесcов} осуществляется так же, как и в случае любых других \textit{действий в sc-памяти}.}
	\scnfileitem{В рамках \textit{scp-процесса} могут создаваться дочерние \textit{scp-процессы}, однако синхронизация между ними при необходимости осуществляется посредством введения дополнительных внутренних блокировок. Таким образом, каждый \textit{scp-процесс} с точки зрения \textit{процессов в sc-памяти} является атомарным и законченным актом деятельности некоторого \textit{sc-агента}.}
	\scnfileitem{Во избежание нежелательных изменений в самом теле \textit{scp-процесса}, вся конструкция, сгенерированная на основе некоторой \textit{scp-программы} (весь текст \textit{scp-процесса}), должна быть добавлена в \textit{полную блокировку}, соответствующую данному \textit{scp-процессу}.}
	\scnfileitem{Все конструкции, сгенерированные в процессе выполнения \textit{scp-процесса}, автоматически попадают в \textit{полную блокировку}, соответствующую данному \textit{scp-процессу}. Дополнительно следует отметить, что знак самой этой структуры и вся метаинформация о ней также включаются в эту структуру.}
	\scnfileitem{При необходимости можно вручную разблокировать или заблокировать некоторую конструкцию каким-либо типом блокировки, используя соответствующие \textit{scp-операторы} класса \textit{scp-оператор управления блокировками}.}
	\scnfileitem{После завершения выполнения некоторого \textit{scp-процесса} его текст как правило, удаляется из \textit{\mbox{sc-памяти}}, а все заблокированные конструкции освобождаются (разрушаются знаки структур, обозначавших блокировки).}
	\scnfileitem{Несмотря на то, что каждый \textit{scp-оператор} представляет собой атомарное \textit{действие в sc-памяти}, дополнительные блокировки, соответствующие одному оператору не вводятся, чтобы избежать громоздкости и избытка дополнительных системных конструкций, создаваемых при выполнении некоторого \textit{scp-процесса}. Вместо этого используются блокировки, общие для всего \textit{scp-процесса}. Таким образом, агенты \textit{Абстрактной scp-машины} при интерпретации \textit{scp-операторов} работают только с учетом блокировок, общих для всего интерпретируемого \textit{scp-процесса}.}
	\scnfileitem{Как правило, частный \textit{класс действий}, соответствующий конкретной \textit{scp-программе} явно не вводится, а используется более общий класс \textit{scp-процесс}, за исключением тех случаев, когда введение специального \textit{класса действий} необходимо по каким-либо другим соображениям.}
	%<---TODO: check by human
\end{scnrelfromset}

\scnheader{scp-процесс}
\scntext{пояснение}{Под \textbf{\textit{scp-процессом}} понимается некоторое \textit{действие в sc-памяти}, однозначно описывающее конкретный акт выполнения некоторой \textit{scp-программы} для заданных исходных данных. Если \textit{scp-программа} описывает алгоритм решения какой-либо задачи в общем виде, то \textit{scp-процесс} обозначает конкретное действие, реализующее данный алгоритм для заданных входных параметро
	По сути, \textbf{\textit{scp-процесс}} представляет собой уникальную копию, созданную на основе \textit{scp-программы}, в которой каждой \textit{sc-переменной}, за исключением \textit{scp-переменных\scnrolesign}, соответствует сгенерированная \textit{sc-константа}.\\
	Принадлежность некоторого действия множеству \textit{scp-процессов} гарантирует тот факт, что в декомпозиции данного действия будут присутствовать только знаки элементарных действий (\textit{scp-операторов}), которые может интерпретировать реализация \textit{Абстрактной scp-машины}.}
\begin{scnrelfromvector}{пример выполнения}
	%TODO: check by human--->
	\scnitem{\scnfileimage[20em]{Contents/part_ps/images/sd_scp/process_example.png}}
		\begin{scnindent}
			\scntext{пояснение}{Осуществляется вызов \textit{scp-программы}. Генерируется соответствующий \textit{scp-процесс}. Происходит инициирование начального оператора scp-процесса \textit{Operator1}.} 
		\end{scnindent}
	\scnitem{\scnfileimage[20em]{Contents/part_ps/images/sd_scp/process_example2.png}}
		\begin{scnindent}	
			\scntext{пояснение}{Оператор \textit{Operator1} оказался безуспешно выполненным. Производится инициирование \textit{\mbox{scp-оператора} генерации трёхэлементной конструкции} \textit{Operator2}.}
		\end{scnindent}
	\scnitem{\scnfileimage[20em]{Contents/part_ps/images/sd_scp/process_example3.png}}
		\begin{scnindent}	
			\scntext{пояснение}{Оператор \textit{Operator2} выполнился. Производится инициирование \textit{scp-оператора завершения выполнения программы} \textit{Operator3}.}
		\end{scnindent}
	\scnitem{\scnfileimage[20em]{Contents/part_ps/images/sd_scp/process_example4.png}}
		\begin{scnindent}
			\scntext{пояснение}{Оператор \textit{Operator3} выполнился. Выполнение \textit{scp-процесса} завершается.}
		\end{scnindent}
%<---TODO: check by human

\end{scnrelfromvector}
\end{SCn}


\scsubsubsection{Пункт 30.2.1. Предметная область и онтология синтаксиса Базового языка программирования ostis-систем}
\label{sd_scp_syntax}

\scsubsubsection{Пункт 30.2.2. Предметная область и онтология денотационной семантики Базового языка программирования ostis-систем}
\label{sd_scp_denote_sem}
\begin{SCn}
\scnsectionheader{Предметная область и онтология денотационной семантики Базового языка программирования ostis-систем}
\begin{scnsubstruct}
	
\scnheader{Предметная область денотационной семантики языка SCP}
\scniselement{предметная область}
\scnhaselementrole{максимальный класс объектов исследования}{scp-оператор}
\begin{scnhaselementrolelist}{исследуемое отношение}
    \scnitem{scp-операнд\scnrolesign}
    \scnitem{scp-константа\scnrolesign}
    \scnitem{scp-переменная\scnrolesign}
    \scnitem{scp-операнд с заданным значением\scnrolesign}
    \scnitem{scp-операнд со свободным значением\scnrolesign}
    \scnitem{формируемое множество\scnrolesign}
    \scnitem{удаляемый sc-элемент\scnrolesign}
\end{scnhaselementrolelist}

\scnheader{scp-оператор}
\scnrelto{включение}{действие в sc-памяти}
\scnrelto{семейство подмножеств}{атомарный тип scp-оператора}
\begin{scnsubdividing}
	%TODO: check by human--->
	\scnitem{scp-оператор генерации конструкций}
		\begin{scnindent}
			\begin{scnsubdividing}
				%TODO: check by human--->
				\scnitem{scp-оператор генерации конструкции по произвольному образцу}
				\scnitem{scp-оператор генерации пятиэлементной конструкции}
				\scnitem{scp-оператор генерации трехэлементной конструкции}
				\scnitem{scp-оператор генерации одноэлементной конструкции}
				%<---TODO: check by human
			\end{scnsubdividing}
		\end{scnindent}
	\scnitem{scp-оператор ассоциативного поиска конструкций}
		\begin{scnindent}
			\begin{scnsubdividing}
				%TODO: check by human--->
				\scnitem{scp-оператор поиска конструкции по произвольному образцу}
				\scnitem{scp-оператор поиска пятиэлементной конструкции с формированием множеств}
				\scnitem{scp-оператор поиска трехэлементной конструкции с формированием множеств}
				\scnitem{scp-оператор поиска пятиэлементной конструкции}
				\scnitem{scp-оператор поиска трехэлементной конструкции}
				%<---TODO: check by human
			\end{scnsubdividing}
		\end{scnindent}
	\scnitem{scp-оператор удаления конструкций}
		\begin{scnindent}
			\begin{scnsubdividing}
				%TODO: check by human--->
				\scnitem{scp-оператор удаления множества элементов трехэлементной конструкции}
				\scnitem{scp-оператор удаления одноэлементной конструкции}
				\scnitem{scp-оператор удаления пятиэлементной конструкции}
				\scnitem{scp-оператор удаления трехэлементной конструкции}
				%<---TODO: check by human
			\end{scnsubdividing}
		\end{scnindent}
	\scnitem{scp-оператор проверки условий}
		\begin{scnindent}
			\begin{scnsubdividing}
				%TODO: check by human--->
				\scnitem{scp-оператор сравнения числовых содержимых файлов}
				\scnitem{scp-оператор проверки равенства числовых содержимых файлов}
				\scnitem{scp-оператор проверки совпадения значений операндов}
				\scnitem{scp-оператор проверки наличия содержимого у файла}
				\scnitem{scp-оператор проверки наличия значения у переменной}
				\scnitem{scp-оператор проверки типа sc-элемента}
				%<---TODO: check by human
			\end{scnsubdividing}
		\end{scnindent}
	\scnitem{scp-оператор управления значениями операндов}
		\begin{scnindent}	
			\begin{scnsubdividing}
				%TODO: check by human--->
				\scnitem{scp-оператор удаления значения переменной}
				\scnitem{scp-оператор присваивания значения переменной}
				%<---TODO: check by human
			\end{scnsubdividing}
		\end{scnindent}
	\scnitem{scp-оператор управления scp-процессами}
		\begin{scnindent}
			\begin{scnsubdividing}
				%TODO: check by human--->
				\scnitem{scp-оператор завершения выполнения программы}
				\scnitem{конъюнкция предшествующих scp-операторов}
				\scnitem{scp-оператор ожидания завершения выполнения множества scp-программ}
				\scnitem{scp-оператор ожидания завершения выполнения scp-программы}
				\scnitem{scp-оператор асинхронного вызова подпрограммы}
				%<---TODO: check by human
			\end{scnsubdividing}
		\end{scnindent}
	\scnitem{scp-оператор управления событиями}
		\begin{scnindent}
		\begin{scnreltoset}{разбиение}
			%TODO: check by human--->
			\scnitem{scp-оператор ожидания события}
			%<---TODO: check by human
		\end{scnreltoset}
		\end{scnindent}
	\scnitem{scp-оператор обработки содержимых файлов}
		\begin{scnindent}
			\begin{scnsubdividing}
                %TODO: check by human--->
                \scnitem{scp-оператор вычисления арксинуса числового содержимого файла}
                \scnitem{scp-оператор вычисления арккосинуса числового содержимого файла}
                \scnitem{scp-оператор деления числовых содержимых файлов}
                \scnitem{scp-оператор умножения числовых содержимых файлов}
                \scnitem{scp-оператор вычитания числовых содержимых файлов}
                \scnitem{scp-оператор сложения числовых содержимых файлов}
                \scnitem{scp-оператор вычисления тангенса числового содержимого файла}
                \scnitem{scp-оператор вычисления косинуса числового содержимого файла}
                \scnitem{scp-оператор вычисления синуса числового содержимого файла}
                \scnitem{scp-оператор вычисления логарифма числового содержимого файла}
                \scnitem{scp-оператор возведения числового содержимого файла в степень}
                \scnitem{scp-оператор удаления содержимого файла}
                \scnitem{scp-оператор копирования содержимого файла}
                \scnitem{scp-оператор нахождения остатка от деления числовых содержимых файлов}
                \scnitem{scp-оператор нахождения целой части от деления числовых содержимых файлов}
                \scnitem{scp-оператор вычисления арктангенса числового содержимого файла}
                \scnitem{scp-оператор перевода в нижний регистр строкового содержимого файла}
                \scnitem{scp-оператор перевода в верхний регистр строкового содержимого файла}
                \scnitem{scp-оператор замены определенной части строкового содержимого файла на содержимое указанного файла}
                \scnitem{scp-оператор проверки совпадения конца строкового содержимого файла со строковом содержимым другого файла}
                \scnitem{scp-оператор проверки совпадения начальной части строкового содержимого файла со строковом содержимым другого файла}
                \scnitem{scp-оператор получения части строкового содержимого файла по индексам}
                \scnitem{scp-оператор поиска строкового содержимого файла в строковом содержимом другого файла}
                \scnitem{scp-оператор вычисления длины строкового содержимого файла}
                \scnitem{scp-оператор разбиения строки на подстроки}
                \scnitem{scp-оператор лексикографического сравнения строковых содержимых файлов}
                \scnitem{scp-оператор проверки равенства строковых содержимых файлов}
                %<---TODO: check by human
			\end{scnsubdividing}
		\end{scnindent}
	\scnitem{scp-оператор управления блокировками}
		\begin{scnindent}
			\begin{scnsubdividing}
                %TODO: check by human--->
                \scnitem{scp-оператор снятия всех блокировок данного scp-процесса}
                \scnitem{scp-оператор снятия блокировки с sc-элемента}
                \scnitem{scp-оператор установки полной блокировки на sc-элемент}
                \scnitem{scp-оператор установки блокировки на изменение sc-элемента}
                \scnitem{scp-оператор установки блокировки на удаление sc-элемента}
                \scnitem{scp-оператор снятия блокировки со структуры}
                \scnitem{scp-оператор установки полной блокировки на структуру}
                \scnitem{scp-оператор установки блокировки на изменение структуры}
                \scnitem{scp-оператор установки блокировки на удаление структуры}
                %<---TODO: check by human
			\end{scnsubdividing}
		\end{scnindent}
	%<---TODO: check by human
\end{scnsubdividing}
\scntext{примечание}{Каждый \textbf{\textit{scp-оператор}} представляет собой некоторое элементарное \textit{действие в sc-памяти}. Аргументы \textit{scp-оператора} будем называть операндами. Порядок операндов указывается при помощи соответствующих ролевых отношений (\textit{1\scnrolesign}, \textit{2\scnrolesign}, \textit{3\scnrolesign} и так далее). Операнд, помеченный ролевым отношением \textit{1\scnrolesign}, будем называть первым операндом, помеченный ролевым отношением \textit{2\scnrolesign} --- вторым операндом, и т.д. Тип и смысл каждого операнда также уточняется при помощи различных подклассов отношения \textit{scp-операнд\scnrolesign}. В общем случае операндом может быть любой \textit{sc-элемент}, в том числе, знак какой-либо \textit{scp-программы}, в том числе самой программы, содержащей данный оператор.}
\scntext{примечание}{Каждый \textbf{\textit{scp-оператор}} должен иметь один и более операнд, а также указание того \textbf{\textit{scp-оператора}} (или нескольких), который должен быть выполнен следующим. Исключение их данного правила составляет \textit{scp-оператор завершения выполнения программы}, который не содержит ни одного операнда и после выполнения которого никакие \textit{scp-операторы} в рамках данной программы выполняться не могут.}

\scnheader{атомарный scp-оператор}
\scntext{определение}{Каждый \textbf{\textit{атомарный тип scp-оператора}} представляет собой класс \textit{scp-операторов}, который не разбивается на более частные, и, соответственно, интерпретируется реализацией \textit{Aбстрактной scp-машины}.}

\scnheader{начальный оператор\scnrolesign}
\scnsubset{1\scnrolesign}
\scntext{примечание}{Ролевое отношение \textbf{\textit{начальный оператор\scnrolesign}} указывает в рамках декомпозиции соответствующего \textit{\mbox{scp-программе}} \textit{scp-процесса} те \textit{scp-операторы}, которые должны быть выполнены в первую очередь, то есть те, с которых собственно начинается выполнение \textit{scp-процесса}.}

\scnheader{параметр scp-программы\scnrolesign}
\scnsubset{аргумент действия\scnrolesign}
\begin{scnrelfromset}{разбиение}
	\scnitem{in-параметр\scnrolesign}
	\scnitem{out-параметр\scnrolesign}
\end{scnrelfromset}
\scntext{примечание}{Ролевое отношение \textbf{\textit{параметр scp-программы\scnrolesign}} связывает знак соответствующего \textit{scp-программе} \textit{\mbox{scp-процесса}} с его аргументами.}

\scnheader{in-параметр\scnrolesign}
\scntext{определение}{Параметры типа \textbf{\textit{in-параметр\scnrolesign}} хоть и соответствуют \textit{переменным scp-программы\scnrolesign}, не могут менять значение в процессе ее интерпретации. Фиксированное значение переменной устанавливается при создании уникальной копии \textit{scp-программы} (\textit{scp-процесса}) для ее интерпретации, и, таким образом, соответствующая \textit{scp-переменная\scnrolesign} на момент начала ее интерпретации становится \textit{scp-константой\scnrolesign} в рамках каждого \textit{scp-оператора}, в котором встречалась данная \textit{scp-переменная\scnrolesign}. Использование \textit{in-параметров} можно рассматривать по аналогии с использованием варианта механизма передачи по значению в традиционных языках программирования, с тем условием, что значение локальной переменной в рамках дочерней программы не может быть изменено.}

\scnheader{out-параметр\scnrolesign}
\scntext{определение}{Параметры типа \textbf{\textit{out-параметр\scnrolesign}} соответствуют \textit{переменным scp-программы\scnrolesign} и обладают всеми теми же соответствующими свойствами. Чаще всего предполагается, что значение данного параметра необходимо родительской \textit{scp-программе}, содержащей оператор вызова текущей \textit{scp-программы}. При этом на момент начала интерпретации в качестве параметра дочернему процессу передается непосредственно узел, обозначающий переменную (а точнее, ее уникальную копию в рамках процесса) родительского процесса. Указанная переменная может при необходимости иметь значение, либо не иметь. После завершения и во время интерпретации дочернего процесса родительский процесс по-прежнему может работать с переменной, переданной в качестве \textit{out-параметра\scnrolesign}, при необходимости просматривая или изменяя ее значение. Использование out-параметра можно рассматривать по аналогии с использованием механизма передачи по ссылке в традиционных \textit{языках программирования}.}

\scnheader{sc-конструкция}
\scnrelfrom{разбиение}{Классификация sc-конструкций с точки зрения Базовой модели обработки sc-текстов}
\begin{scnindent}
	\begin{scneqtoset}
		\scnitem{sc-конструкция нестандартного вида}
		\scnitem{sc-конструкция стандартного вида}
	\end{scneqtoset}
	\begin{scnindent}
		\begin{scnrelfromset}{разбиение}
			\scnitem{одноэлементная sc-конструкция}
			\scnitem{трехэлементная sc-конструкция}
			\scnitem{пятиэлементная sc-конструкция}	
		\end{scnrelfromset}
	\end{scnindent}
\end{scnindent}

\scnheader{sc-конструкция нестандартного вида}
\scntext{определение}{Каждая \textit{sc-конструкция нестандартного вида} состоит из произвольного количества \textit{sc-элементов} произвольного типа.}
\scnrelfrom{описание примера}{\scnfileimage[10em]{Contents/part_ps/src/images/sd_agents/pic_ps1.png}}
\begin{scnindent}
	\scnidtf{SCg-текст. Пример sc-конструкции нестандартного вида}
\end{scnindent}

\scnheader{sc-конструкция стандартного вида}
\scntext{определение}{Каждый элемент \textit{\mbox{sc-конструкции} стандартного вида} имеет свою условную строго фиксированную позицию в рамках этой \mbox{sc-конструкции} (первый элемент, второй элемент и так далее). В зависимости от указанной позиции вводятся дополнительные ограничения на тип соответствующего \textit{sc-элемента}.}

\scnheader{одноэлементная sc-конструкция}
\scntext{определение}{Каждая \textit{одноэлементная sc-конструкция} состоит из одного \textit{sc-элемента} произвольного типа.}
\scntext{определение}{Каждая \textit{sc-конструкция нестандартного вида} состоит из произвольного количества \textit{sc-элементов} произвольного типа.}
\scnrelfrom{описание примера}{\scnfileimage[20em]{Contents/part_ps/src/images/sd_agents/pic_ps2.png}}
\begin{scnindent}
	\scnidtf{SCg-текст. Пример одноэлементных sc-конструкций в SCg-коде}
\end{scnindent}

\scnheader{трехэлементная sc-конструкция}
\scntext{определение}{Каждая \textit{трехэлементная sc-конструкция} состоит из трех \textit{sc-элементов}. Второй элемент всегда является \textit{sc-коннектором}, остальные элементы могут быть произвольного типа.}
\scnrelfrom{описание примера}{\scnfileimage[20em]{Contents/part_ps/src/images/sd_agents/pic_ps3.png}}
\begin{scnindent}
	\scnidtf{SCg-текст. Пример трехэлементной sc-конструкции в SCg-коде}
\end{scnindent}

\scnheader{пятиэлементная sc-конструкция}
\scntext{определение}{Каждая \textit{пятиэлементная sc-конструкция} состоит из пяти \textit{sc-элементов}. Второй и четвертый элементы обязательно являются \textit{sc-коннекторами}, остальные элементы могут быть произвольного типа.}
\scnrelfrom{описание примера}{\scnfileimage[20em]{Contents/part_ps/src/images/sd_agents/pic_ps4.png}}
\begin{scnindent}
	\scnidtf{SCg-текст. Пример пятиэлементной sc-конструкции в SCg-коде}
\end{scnindent}

\scnheader{scp-операнд\scnrolesign}
\scnrelto{включение}{аргумент действия\scnrolesign}
\scniselement{неосновное понятие}
\scniselement{ролевое отношение}
\begin{scnsubdividing}
	%TODO: check by human--->
	\scnitem{scp-константа\scnrolesign}
	\scnitem{scp-переменная\scnrolesign}
	%<---TODO: check by human
\end{scnsubdividing}
\begin{scnsubdividing}
	%TODO: check by human--->
	\scnitem{scp-операнд с заданным значением\scnrolesign}
	\scnitem{scp-операнд со свободным значением\scnrolesign}
	%<---TODO: check by human
\end{scnsubdividing}
\begin{scnsubdividing}
	%TODO: check by human--->
	\scnitem{константный sc-элемент\scnrolesign}
	\scnitem{переменный sc-элемент\scnrolesign}
	%<---TODO: check by human
\end{scnsubdividing}
\begin{scnrelfromlist}{включение}
%TODO: check by human--->
	\scnitem{формируемое множество\scnrolesign}
		\begin{scnindent}
			\begin{scnsubdividing}
			%TODO: check by human--->
			\scnitem{формируемое множество 1\scnrolesign}
			\scnitem{формируемое множество 2\scnrolesign}
			\scnitem{формируемое множество 3\scnrolesign}
			\scnitem{формируемое множество 4\scnrolesign}
			\scnitem{формируемое множество 5\scnrolesign}
			%<---TODO: check by human
			\end{scnsubdividing}
		\end{scnindent}
	\scnitem{удаляемый sc-элемент\scnrolesign}
	\scnitem{тип sc-элемента\scnrolesign}
		\begin{scnindent}
			\begin{scnsubdividing}
				%TODO: check by human--->
				\scnitem{sc-узел\scnrolesign}
					\begin{scnindent}
						\begin{scnsubdividing}
							%TODO: check by human--->
							\scnitem{структура\scnrolesign}
							\scnitem{отношение\scnrolesign}
								\begin{scnindent}
									\scnrelfrom{включение}{ролевое отношение\scnrolesign}
								\end{scnindent}
							\scnitem{класс\scnrolesign}
							%<---TODO: check by human
						\end{scnsubdividing}
					\end{scnindent}
				\scnitem{sc-дуга\scnrolesign}
					\begin{scnindent}
						\begin{scnsubdividing}
							%TODO: check by human--->
							\scnitem{sc-дуга общего вида\scnrolesign}
							\scnitem{sc-дуга принадлежности\scnrolesign}
								\begin{scnindent}
									\scnrelfrom{включение}{sc-дуга основного вида\scnrolesign}
									\begin{scnsubdividing}
										%TODO: check by human--->
										\scnitem{позитивная sc-дуга принадлежности\scnrolesign}
										\scnitem{негативная sc-дуга принадлежности\scnrolesign}
										\scnitem{нечеткая sc-дуга принадлежности\scnrolesign}
										%<---TODO: check by human
									\end{scnsubdividing}
									\begin{scnsubdividing}
										%TODO: check by human--->
										\scnitem{временная sc-дуга принадлежности\scnrolesign}
										\scnitem{постоянная sc-дуга принадлежности\scnrolesign}
										%<---TODO: check by human
									\end{scnsubdividing}
								\end{scnindent}
							%<---TODO: check by human
						\end{scnsubdividing}
					\end{scnindent}
				\scnitem{sc-ребро\scnrolesign}
				\scnitem{файл\scnrolesign}
				%<---TODO: check by human
			\end{scnsubdividing}
		\end{scnindent}
	%<---TODO: check by human
\end{scnrelfromlist}
\scntext{пояснение}{Ролевое отношение \textit{scp-операнд\scnrolesign} является неосновным понятием и указывает на принадлежность аргументов \textit{scp-оператору}. Помимо указания какого-либо класса \textit{scp-операндов\scnrolesign} порядок аргументов \textit{scp-оператора} дополнительно уточняется \textit{ролевыми отношениями 1\scnrolesign}, \textit{2\scnrolesign} и т. д.}

\scnheader{scp-константа\scnrolesign}
\scntext{пояснение}{В рамках \textit{scp-программы} \textit{scp-константы\scnrolesign} явно участвуют в \textit{\mbox{scp-операторах}} в качестве элементов (в теоретико-множественном смысле) и напрямую обрабатываются при интерпретации \textit{scp-программы}. Константами в рамках \textit{scp-программы} могут быть \textit{sc-элементы} любого типа, как \textit{\mbox{sc-константы}}, так и \textit{\mbox{sc-переменные}}. Константа в рамках \textit{scp-программы} остается неизменной в течение всего срока интерпретации. Константа \textit{\mbox{scp-программы}} может быть рассмотрена как переменная, значение которой совпадает с самой переменной в каждый момент времени и изменено быть не может. Таким образом, далее будем считать, что \textit{scp-константа\scnrolesign} и ее значение это одно и то же. Каждый \textit{in-параметр\scnrolesign} при интерпретации каждой конкретной копии \textit{scp-программы} становится \textit{scp-константой\scnrolesign} в рамках всех ее операторов, хотя в исходном теле данной программы в каждом из этих операторов он является \textit{scp-переменной\scnrolesign}.}

\scnheader{scp-переменная\scnrolesign}
\scntext{пояснение}{В рамках \textit{scp-программы} \textit{scp-переменные\scnrolesign} не обрабатываются явно при интерпретации, обрабатываются значения переменных. Каждая переменная \textit{scp-программы} может иметь одно значение в каждый момент времени, т. е. представляет собой ситуативный \textit{синглетон}, элементом которого является текущее значение \textit{scp-переменной\scnrolesign}. Значение каждой \textit{scp-переменной\scnrolesign} может меняться в ходе интерпретации \textit{scp-программы}. При этом интерпретатор при обработке \textit{scp-оператора} работает непосредственно со значениями \textit{\mbox{scp-переменных\scnrolesign}}, а не самими \textit{scp-переменными\scnrolesign} (которые также являются узлами той же семантической сети).}

\scnheader{scp-операнд с заданным значением\scnrolesign}
\scntext{пояснение}{Значение операндов, помеченных ролевым отношением \textit{scp-операнд с заданным значением\scnrolesign}, считается заданным в рамках текущего \textit{scp-оператора}. Данное значение учитывается при выполнении \textit{scp-оператора} и остается неизменным после окончания выполнения \textit{scp-оператора}. Каждая \textit{scp-константа\scnrolesign} по умолчанию рассматривается как \textit{scp-операнд с заданным значением\scnrolesign}, в связи с чем явное использование данного ролевого отношения в таком случае является избыточным. В таком случае в качестве значения рассматривается непосредственно сам операнд. В случае, если отношением \textit{\mbox{scp-операнд} с заданным значением\scnrolesign} помечена \textit{scp-переменная\scnrolesign}, то осуществляется попытка поиска значения для данной \textit{scp-переменной\scnrolesign} (ее элемента). Если попытка оказалась безуспешной, то возникает ошибка времени выполнения, которая должна быть обработана соответствующим образом.\\
Любой \textit{scp-операнд с заданным значением\scnrolesign} независимо от конкретного типа \textit{scp-оператора} может быть \textit{scp-переменной\scnrolesign}.}

\scnheader{scp-операнд со свободным значением\scnrolesign}
\scntext{пояснение}{Значение операндов, помеченных ролевым отношением \textit{scp-операнд со свободным значением\scnrolesign}, считается свободным (не заданным заранее) в рамках текущего \textit{scp-оператора}. В начале выполнения \textit{scp-оператора} связь между \textit{scp-переменной\scnrolesign}, помеченной данным ролевым отношением, и ее элементом (значением) всегда удаляется. В результате выполнения данного оператора может быть либо сгенерировано новое значение \textit{scp-переменной\scnrolesign}, либо не сгенерировано, тогда \textit{scp-переменная\scnrolesign} будет считаться не имеющей значения. Ни одна \textit{scp-константа\scnrolesign} не может быть помечена как \textit{scp-операнд со свободным значением\scnrolesign}, поскольку константа не может изменять свое значение в ходе интерпретации \textit{scp-программы}.}

\scnheader{тип sc-элемента\scnrolesign}
\scntext{определение}{Ролевое отношение \textit{тип \mbox{sc-элемента\scnrolesign}} используется для уточнения типа \textit{sc-элемента}, выступающего в роли значения некоторого операнда. \textit{тип \mbox{sc-элемента\scnrolesign}} имеет смысл указывать только для операндов, помеченных как \textit{scp-операнд со свободным значением\scnrolesign}, тогда данное уточнение типа \textit{\mbox{sc-элемента}} будет использовано для сужения области поиска либо уточнения параметров генерации каких-либо конструкций. Значением \textit{scp-операндов с заданным значением\scnrolesign} является конкретный, известный на момент начала выполнения \textit{scp-оператора sc-элемент} с конкретным типом, не зависящим от указания \textit{типа sc-элемента\scnrolesign}, в связи с чем использование ролевого отношения \textit{тип sc-элемента\scnrolesign} в данном случае является некорректным.}
\scntext{примечание}{Допускается использование комбинаций семантически непротиворечащих друг другу подмножеств указанного отношения. Например, допускается комбинация \textit{константный sc-элемент\scnrolesign} и \textit{sc-дуга общего вида\scnrolesign}, но не допускается комбинация \textit{sc-узел\scnrolesign} и \textit{sc-дуга\scnrolesign}.}

\scnheader{формируемое множество\scnrolesign}
\scntext{определение}{Ролевое отношение \textbf{\textit{формируемое множество\scnrolesign}} используется для указания того факта, что в результате выполнения \textit{scp-оператора} должно быть сформировано либо дополнено некоторое множество \textit{sc-элементов}, являющееся значением одного из операндов данного \textit{scp-оператора}. При этом если данный операнд помечен как \textit{scp-операнд со свободным значением\scnrolesign}, то множество будет сформировано с нуля (сгенерирован новый \textit{sc-элемент}, обозначающий данное множество), в противном случае уже существующее множество может быть дополнено. Использование данного ролевого отношения предполагает, что при его отсутствии множество бы не формировалось, а значением указанного операнда стал бы произвольный \textit{sc-элемент} из данного множества.}
\scntext{примечание}{Ролевое отношение \textit{формируемое множество\scnrolesign} без уточнения порядкового номера используется только в \textit{scp-операторах обработки произвольных конструкций}. Для явного указания номера операнда, которому соответствует \textit{формируемое множество\scnrolesign}, используются подмножества данного ролевого отношения, аналогичные ролевым отношениям, задающим порядок элементов в кортеже (\textit{1\scnrolesign, 2\scnrolesign, 3\scnrolesign} и так далее), например \textit{формируемое множество 1\scnrolesign}, \textit{формируемое множество 2\scnrolesign} и так далее. Указанные ролевые отношения используются только в \textit{scp-операторах поиска конструкций с формированием множеств}.}

\scnheader{удаляемый sc-элемент\scnrolesign}
\scntext{определение}{Ролевое отношение \textbf{\textit{удаляемый sc-элемент\scnrolesign}} используется для указания тех операндов, значение которых должно быть удалено в процессе выполнения \textit{scp-операторов удаления}. Данным ролевым отношением может быть помечен как \textit{scp-операнд с заданным значением\scnrolesign}, так и \textit{scp-операнд со свободным значением\scnrolesign}. При этом удаляемым \textit{sc-элементом} может быть как \textit{scp-константа\scnrolesign}, так и \textit{scp-переменная\scnrolesign} (в случае \textit{scp-переменной\scnrolesign} удаляется не только связка принадлежности между этой \textit{scp-переменной\scnrolesign} и ее значением, но и непосредственно сам \textit{sc-элемент}, являющийся значением).}

\scnheader{следует отличать*}
\begin{scnhaselementset}
	\scnitem{scp-переменная\scnrolesign}
	\scnitem{sc-переменная}	
\end{scnhaselementset}
\begin{scnhaselementset}
	\scnitem{scp-константа\scnrolesign}
	\scnitem{sc-константа}	
\end{scnhaselementset}

\scnheader{scp-оператор генерации пятиэлементной конструкции}
\scntext{примечание}{На рисунках показан пример работы scp-оператора генерации пятиэлементной конструкции. В приведенном примере выполняется генерация пятиэлементной конструкции, которая имеет два scp-операнда с заданным значением. В примере предполагается, что рассматриваемые элементы (some\_node1 и some\_node2) изначально никак не связаны между собой.}
\begin{scnindent}
	\scnrelfrom{описание примера}{\scnfileimage[30em]{Contents/part_ps/src/images/sd_agents/genElStr5_fafaa.png}}
	\begin{scnindent}
		\scnidtf{SCg-текст. Пример выполнения scp-оператора генерации пятиэлементной конструкции (вызов scp-оператора)}
	\end{scnindent}
	\scnrelfrom{описание примера}{\scnfileimage[30em]{Contents/part_ps/src/images/sd_agents/genElStr5_fafaa_2.png}}
	\begin{scnindent}
		\scnidtf{SCg-текст. Пример выполнения scp-оператора генерации пятиэлементной конструкции (результат выполнения scp-оператора)}
	\end{scnindent}
\end{scnindent}

\scnheader{scp-оператор поиска трехэлементной конструкции}
\scntext{примечание}{На рисунках приведен пример scp-оператора поиска трехэлементной конструкции, которая имеет два scp-операнда с заданным значением. В примере предполагается, что рассматриваемые элементы (some\_node1 и some\_node2) изначально связаны между собой константной постоянной sc-дугой.}
\begin{scnindent}
	\scnrelfrom{описание примера}{\scnfileimage[30em]{Contents/part_ps/src/images/sd_agents/searchElStr3_faf.png}}
	\begin{scnindent}
		\scnidtf{SCg-текст. Пример выполнения scp-оператора поиска трехэлементной конструкции (вызов scp-оператора)}
	\end{scnindent}
	\scnrelfrom{описание примера}{\scnfileimage[30em]{Contents/part_ps/src/images/sd_agents/searchElStr3_faf_2.png}}
	\begin{scnindent}
		\scnidtf{SCg-текст. Пример выполнения scp-оператора поиска трехэлементной конструкции (результат выполнения scp-оператора)}
	\end{scnindent}
\end{scnindent}

\scnheader{scp-оператор удаления одноэлементной конструкции}
\scntext{примечание}{На рисунках приведен пример scp-оператора поиска трехэлементной конструкции, которая имеет два scp-операнда с заданным значением. В примере предполагается, что рассматриваемые элементы (some\_node1 и some\_node2) изначально связаны между собой константной постоянной sc-дугой.}
\begin{scnindent}
	\scnrelfrom{описание примера}{\scnfileimage[30em]{Contents/part_ps/src/images/sd_agents/searchElStr3_faf.png}}
	\begin{scnindent}
		\scnidtf{SCg-текст. Пример выполнения scp-оператора удаления одноэлементной конструкции (вызов scp-оператора)}
	\end{scnindent}
	\scnrelfrom{описание примера}{\scnfileimage[30em]{Contents/part_ps/src/images/sd_agents/searchElStr3_faf_2.png}}
	\begin{scnindent}
		\scnidtf{SCg-текст. Пример выполнения scp-оператора удаления одноэлементной конструкции (результат выполнения scp-оператора)}
	\end{scnindent}
\end{scnindent}

\end{scnsubstruct}
\end{SCn}

%В стандарте нет таблицы такой
%Таблица \ref{table_operands_roles} показывает возможные сочетания различных ролевых отношений, указывающих роль операнда в рамках scp-оператора:

%\begin{table}[H]
%  \caption{Роли операндов в рамках scp-оператора}\label{table_operands_roles}
%\begin{tabularx}{\hsize}{| p{43mm} | X | X |}
%  \hline
%  \textbf{Тип значения}
%  & \multirow{2}{*}{\textbf{\shortstack[l]{scp-операнд с\\ заданным значением\scnrolesign}}} & \multirow{2}{*}{\textbf{\shortstack[l]{scp-операнд со\\ свободным значением\scnrolesign}}} \\
%  \cline{0-0}
%  \textbf{Константность} & & \\
%\hline
%\textbf{scp-константа\scnrolesign} & Разрешено, может быть опущено & Запрещено \\
%\hline
%\textbf{scp-переменная\scnrolesign} & Разрешено, значение останется неизменным & Разрешено, значение переменной будет изменено либо потеряно\\
%\hline
%\end{tabularx}
%\end{table}
%}


\scsubsubsection{Пункт 30.2.3. Предметная область и онтология операционной семантики Базового языка программирования ostis-систем}
\label{sd_scp_oper_sem}
\begin{SCn}

\scnsectionheader{\currentname}

\scnstartsubstruct

\scnheader{Предметная область операционной семантики языка SCP}
\scniselement{предметная область}
\scnsdmainclasssingle{Абстрактная scp-машина}
\scnhaselementlist{ключевой объект исследования}{Абстрактный sc-агент создания scp-процессов;Абстрактный sc-агент интерпретации scp-операторов;Абстрактный sc-агент синхронизации процесса интерпретации scp-программ;Абстрактный sc-агент уничтожения scp-процессов;Абстрактный sc-агент синхронизации событий в sc-памяти и ее реализации;Абстрактный sc-агент трансляции сформированной спецификации события в sc-памяти во внутреннее представление;Абстрактный sc-агент обработки события в sc-памяти, инициирующего агентную scp-программу}

\scnheader{Абстрактная scp-машина}
\scnreltoset{декомпозиция абстрактного sc-агента}{Абстрактный sc-агент создания scp-процессов;Абстрактный sc-агент интерпретации scp-операторов;Абстрактный sc-агент синхронизации процесса интерпретации scp-программ;Абстрактный sc-агент уничтожения scp-процессов;Абстрактный sc-агент синхронизации событий в sc-памяти и ее реализации}

\scnheader{Абстрактный sc-агент создания scp-процессов}
\scnexplanation{Задачей \textit{Абстрактного} \textit{sc-агента создания scp-процессов}
является создание \textit{scp-процессов}, соответствующих заданной
\textit{scp-программе}. Данный \textit{\mbox{sc-агент}} активируется при появлении
в \textit{sc-памяти} \textit{инициированного действия}, принадлежащего
классу \textit{действие интерпретации scp-программы}.

После проверки \textit{sc-агентом} условия инициирования выполняется
создание \textit{scp-процесса} с учетов конкретных параметров
интерпретации \textit{\mbox{scp-программы}}, после чего осуществляется поиск
\textit{начального оператора' \mbox{scp-процесса}} и добавление его во множество
\textit{настоящих сущностей}.}

\scnheader{Абстрактный sc-агент интерпретации scp-операторов}
\scnexplanation{Задачей \textit{Абстрактного sc-агента интерпретации scp-операторов}
является собственно интерпретация операторов \textit{scp-программы}, то
есть выполнение в \textit{sc-памяти} действий, описываемых конкретным
\textit{\mbox{scp-оператором}}. Данный \textit{sc-агент} активируется при появлении
в \textit{sc-памяти} \textit{scp-оператора}, принадлежащего классу
\textit{настоящих сущностей}. После выполнения действия, описываемого
\textit{scp-оператором}, \textit{scp-оператор} добавляется во множество
\textit{прошлых сущностей}. В случае когда семантика действия,
описываемого \textit{\mbox{scp-оператором}}, предполагает возможность ветвления
\textit{scp-программы} после выполнения данного \textit{\mbox{scp-оператора}}, то
используется одно из подмножеств класса \textit{выполненных действий --
безуспешно выполненное действие} или \textit{успешно выполненное
действие}.}

\scnheader{Абстрактный sc-агент синхронизации процесса интерпретации scp-программ}
\scnexplanation{Задачей \textit{Абстрактного sc-агента синхронизации процесса
интерпретации scp-программ} является обеспечение переходов между
\textit{scp-операторами} в рамках одного \textit{scp-процесса}. Данный
\textit{sc-агент} активизируется при добавлении некоторого
\textit{scp-оператора} во множество \textit{прошлых сущностей}. Далее
осуществляется переход по \textit{sc-дуге}, принадлежащей отношению
\textit{последовательность действий*} (или более частным отношениям, в
случае, если \textit{\mbox{scp-оператор}} был добавлен во множество \textit{успешно
выполненных действий} или \textit{безуспешно выполненных действий}). При
этом очередной \textit{scp-оператор} становится \textit{настоящей сущностью}
(активным \textit{scp-оператором}) в том случае, если хотя бы один
\textit{scp-оператор}, связанный с ним входящими \textit{sc-дугами},
принадлежащими отношению \textit{последовательность действий*} (или более
частным отношениям), стал \textit{прошлой сущностью} (или, соответственно,
подмножеством прошлых сущностей). В случае, когда необходимо дождаться
завершения выполнения всех предыдущих операторов, для синхронизации
используется оператор класса \textit{конъюнкция предшествующих
операторов}.}

\scnheader{Абстрактный sc-агент уничтожения scp-процессов}
\scnexplanation{Задачей \textit{Абстрактного sc-агента уничтожения scp-процессов}
является уничтожение \textit{scp-процесса}, т. е. удаление из
\textit{sc-памяти} всех \textit{sc-элементов}, его составляющих. Данный
\textit{sc-агент} активируется при появлении в \textit{sc-памяти}
\textit{scp-процесса}, принадлежащего множеству \textit{прошлых сущностей}.

При этом уничтожаемый \textit{scp-процесс} необязательно должен быть
полностью сформирован. Необходимость уничтожения не до конца
сформированного \textit{scp-процесса} может возникнуть в случае, если при
создании \textit{scp-процесса} возникли проблемы, не позволяющие
продолжить создание \textit{scp-процесса} и его выполнение.}

\scnheader{Абстрактный sc-агент синхронизации событий в sc-памяти и ее реализации}
\scnexplanation{Задачей \textit{Абстрактного sc-агента синхронизации событий в
sc-памяти и ее реализации} является обеспечение работы \textit{неатомарных
sc-агентов}, реализованных на \textit{языке SCP}.}
\scnreltoset{декомпозиция абстрактного sc-агента}{Абстрактный sc-агент трансляции сформированной спецификации события в sc-памяти во внутреннее представление;Абстрактный sc-агент обработки события в sc-памяти,
инициирующего агентную scp-программу}

\scnheader{Абстрактный sc-агент трансляции сформированной спецификации события в sc-памяти во внутреннее представление}
\scnexplanation{Задачей \textit{\textbf{Абстрактного sc-агента трансляции сформированной спецификации события в sc-памяти во внутреннее представление}}
является трансляция ориентированных пар, описывающих \textit{первичное
условие инициирования*} некоторого \textit{\mbox{sc-агента}} во внутреннее
представление элементарных событий на уровне \textit{\mbox{sc-хранилища}}, при
условии, что этот \textit{sc-агент} реализован на платформенно-независимом
уровне (с использованием \textit{языка SCP}). Условием инициирования
данного \textit{sc-агента} является появление в \textit{\mbox{sc-памяти}} нового
элемента множества \textit{активных sc-агентов}, для которого будет
найдена и протранслирована соответствующая ориентированная пара.}

\scnheader{Абстрактный sc-агент обработки события в sc-памяти,
инициирующего агентную scp-программу}
\scnexplanation{Задачей \textit{Абстрактного sc-агента обработки события в sc-памяти,
инициирующего агентную \mbox{scp-программу}}, является поиск \textit{агентной
scp-программы}, входящей во множество \textit{программ sc-агента*} для
каждого \textit{sc-агента}, первичное условие инициирования которого
соответствует событию, произошедшему в \textit{sc-памяти}, а также
генерация и инициирование действия, направленного на интерпретацию этой
программы. В результате работы данного \textit{sc-агента} в
\textit{sc-памяти} появляется \textit{инициированное действие},
принадлежащее классу \textit{действие} \textit{интерпретации scp-программы.}}

\bigskip
\scnendstruct \scnendcurrentsectioncomment

\end{SCn}


\scsubsection{\S 30.3. Предметная область и онтология программ и языков программирования для ostis-систем}
\label{sd_programs}
\begin{SCn}
\scnsectionheader{Предметная область и онтология программ и языков программирования для ostis-систем}
\begin{scnsubstruct}
\scniselement{раздел базы знаний}
\scnhaselementrole{ключевой sc-элемент}{Предметная область программ и языков программирования для ostis-систем}

\scnheader{Предметная область программ и языков программирования для ostis-систем}
\scniselement{предметная область}
\begin{scnrelfromset}{автор}
    \scnitem{Зотов Н.В.}
    \scnitem{Шункевич Д.В.}
\end{scnrelfromset}

\scntext{аннотация}{Несмотря на активное развитие и использование современных технологий и языков программирования, общей семантической теории программ, на основе которой можно было бы проектировать и разрабатывать прикладные системы, на данный момент не существует. В данной предметной области предлагается семантическая теория программ для ostis-систем. Рассматриваются особенности представления и ключевые моменты процесса интерпретации программ в ostis-системах.}

\begin{scnreltovector}{конкатенация сегментов}
    \scnitem{Проблемы текущего состояния в области разработки и применения языков программирования}
    \scnitem{Существующие онтологии языков программирования}
    \scnitem{Предлагаемый подход к разработке технологий программирования для ostis-систем}
    \scnitem{Синтаксис и семантика программ в ostis-системах}
    \scnitem{Методы и средства поддержки проектирования и разработки программ в ostis-системах}
    \scnitem{Комплекс свойств, определяющих эффективность программ в ostis-системах}
\end{scnreltovector}

\begin{scnhaselementrolelist}{ключевой знак}
    \scnitem{Семантическая теория программ}
	\scnitem{Семантическая теория программ для ostis-систем}
	\begin{scnindent}
		\scnidtf{Предлагаемый вариант теории для проектирования языков программирования и программ для интеллектуальных компьютерных систем нового поколения}
	\end{scnindent}
\end{scnhaselementrolelist}

\begin{scnhaselementrolelist}{класс объектов исследования}
	\scnitem{метод}
	\scnitem{язык представления методов}
	\scnitem{эффективность метода}
\end{scnhaselementrolelist}

\begin{scnhaselementrolelist}{исследуемое отношение}
	\scnitem{спецификация метода*}
	\scnitem{синтаксис метода*}
	\scnitem{денотационная семантика метода*}
	\scnitem{операционная семантика метода*}
	\scnitem{спецификация языка представления методов*}
	\scnitem{синтаксис языка представления методов*}
	\scnitem{денотационная семантика языка представления методов*}
	\scnitem{операционная семантика языка представления методов*}
\end{scnhaselementrolelist}

\begin{scnrelfromlist}{библиографическая ссылка}
	\scnitem{\scncite{Sebesta2012}}
	\scnitem{\scncite{Zapata2010}}
	\scnitem{\scncite{Golenkov2019a}}
	\scnitem{\scncite{Penta2020}}
	\scnitem{\scncite{Scalabrino2016}}
	\scnitem{\scncite{Golenkov2012}}
	\scnitem{\scncite{Brooks2021}}
	\scnitem{\scncite{Sellitto2022}}
	\scnitem{\scncite{Turner2007}}
	\scnitem{\scncite{Chaparro2014}}
	\scnitem{\scncite{Golenkov2022a}}
	\scnitem{\scncite{Golenkov2019}}
	\scnitem{\scncite{Eden2007}}
	\scnitem{\scncite{Lando2007}}
	\scnitem{\scncite{Lando2007a}}
	\scnitem{\scncite{Turner2014}}
	\scnitem{\scncite{Deikstra1978}}
	\scnitem{\scncite{Standart2021}}
	\scnitem{\scncite{Kasyanov2003}}
	\scnitem{\scncite{Petrov1978}}
	\scnitem{\scncite{Scott2006}}
	\scnitem{\scncite{Scott1972}}
	\scnitem{\scncite{Orlov2021}}
	\scnitem{\scncite{Lu2022}}
	\scnitem{\scncite{Gulykina2012}}
	\scnitem{\scncite{Pivovarchik2016}}
	\scnitem{\scncite{Ford2019}}
	\scnitem{\scncite{IMS}}
	\scnitem{\scncite{Pivovarchik2013}}
	\scnitem{\scncite{Tin1995}}
	\scnitem{\scncite{Schiitze1991}}
	\scnitem{\scncite{Black1993}}
	\scnitem{\scncite{Zotov2022a}}
\end{scnrelfromlist}

\begin{scnrelfromvector}{введение}
    \scnfileitem{За долгий период развития компьютерных систем практически сняты аппаратные ограничения на решение различных задач. Оставшиеся ограничения отводятся на долю программного обеспечения. Прежде всего эти ограничения связаны с текущими проблемами развития программного обеспечения}
    \begin{scnindent}
        \begin{scnrelfromset}{проблемы текущего состояния}
            \scnfileitem{\uline{Аппаратная сложность опережает} умение человечества строить \uline{программные компьютерные системы}, использующее потенциальные возможности аппаратуры.}
            \scnfileitem{Навыки и \uline{технологии} разработки программ \uline{отстают от требований}, предъявляемых к разработке программ нового поколения.}
            \scnfileitem{Возможностям эксплуатировать существующие программы угрожает \uline{низкое качество их разработки}.}
        \end{scnrelfromset}
        \scntext{решение}{Ключом к решению этих проблем является глубокое понимание и грамотное использование существующих \textit{языков программирования} как основного инструмента для массового создания \textit{программных компьютерных систем нового поколения}.}
    \end{scnindent}
    \scnfileitem{В данной предметной области акцент делается на достижение следующих результатов
        \begin{itemize}
            \item (1) изложение классических основ, отражающих накопленный мировой опыт в области разработки и применения современных \textit{языков программирования};
            \item и (2) систематизация основных результатов в этой области в виде единой унифицированной \textit{Семантической теории программ для интеллектуальных компьютерных систем нового поколения}, построенных по принципам \textit{Технологии OSTIS}.
        \end{itemize}}
    \scnfileitem{В данной предметной области подробно описываются проблемы текущего состояния в области \textit{технологий} и \textit{языков программирования}. Она посвящена базовым понятиям \textit{теории языков программирования}, дается обзорная характеристика областей применения \textit{языков программирования}, достаточно востребованных современным человеческим обществом, рассматриваются способы представления и интерпретации \textit{программ} различных \textit{языков программирования}, подробно описываются формы и содержание критериев для оценки \textit{эффективности языков}.}
\end{scnrelfromvector}

\scntext{примечание}{Проектирование и реализация \textit{программы} на каком-либо \textit{языке программирования} должна сводиться к описанию ее \textit{синтаксиса} и \textit{денотационной семантики} в базе знаний ostis-системы с помощью некоторой библиотеки предметных областей и онтологий программ, описываемой в рамках этой базы знаний. Для этого нужна онтология программ, которые позволили бы в достаточном объеме описывать программы на любых языках программирования в ostis-системах. Такой подход позволяет не только описывать сложноструктурированные объекты простым и понятным языком, но и позволяет унифицировать представление различных видов знаний. Тем самым, информация о программах и сами программы представляются на одном и том же языке (имеют один синтаксис), но содержательно описываются при помощи разных онтологий. Таким образом, \uline{решением всех проблем будет являться общая теория программ}, однозначно соответствующей некоторой онтологии программ, c помощью которых можно было бы описывать синтаксис и денотационную семантику любых программ в ostis-системах.}
\scntext{примечание}{Таким образом, результатом данной предметной области является \textit{Предметная область и онтология программ} (далее --- Предметная область и онтология методов), с помощью которой можно описывать синтаксис, денотационную и операционную семантику различных методов в ostis-системах. \textit{Предметная область и онтология методов} является дочерней предметной областью по отношению к \textit{Предметной области и онтологии информационных конструкций и языков}. Это означает, что она наследует все свойства исследуемых в ней понятий и отношений.}
	
\scnsegmentheader{Проблемы текущего состояния в области разработки и применения языков программирования}
\begin{scnsubstruct}
    \begin{scnhaselementset}
        \scnfileitem{Поскольку количество \textit{языков программирования} растет с увеличением потребности в них, то растут и потребности в описании этих \textit{языков программирования} для дальнейшего использования и проектирования прикладных систем. Это в свою очередь требует высокого уровня качества спецификации конкретного языка: и описания \textit{синтаксиса} и семантики конструкций этого языка, и описания средств и методов реновации инструментальных средств, обеспечивающих интерпретацию или трансляцию этого языка. То есть, с увеличением количества \textit{языков программирования} растет не только многообразие форм представления знаний (\textit{языков программирования}), но и количество \textit{программных компьютерных систем} на различных формах представления знаний.}
        \scnfileitem{Большое многообразие форм представления знаний, как говорилось выше, предоставляет большой спектр возможностей проектирования \textit{программных компьютерных систем} на каждой из них. Получается, чтобы произвести интеграцию нескольких \textit{программных компьютерных систем}, реализованных на разных \textit{языках программирования}, необходимо сделать так, чтобы системы могли коммуницировать между собой на каждом из тех языков, на котором они реализованы. Так, стремление к использованию существующих программных компонентов затрудняется реализацией самих компонентов, поскольку чтобы объединить эти компоненты необходимо изменить их программный код. Наличие многообразия форм затрудняет реализацию \textit{совместимых интероперабельных программных компьютерных систем}.}
        \scnfileitem{С ростом сложности программного кода, уменьшается количество способных понять его смысл. Современные разработчики создают \textit{программные компьютерные системы}, не учитывая полный ее жизненный цикл. Системы должны постоянно обновляться и совершенствоваться с развитием технологий, на которых она основана. Это должно обеспечиваться хорошей документацией реализации компонентов этих систем --- это снижает не только потребности в привлечении новых ресурсов и кадров, но и способствует снижению реинжиниринга \textit{программных компьютерных систем}.}
        \scnfileitem{Полная автоматизация проектирования \textit{программных компьютерных систем} невозможна, поскольку современные языки, на которых они проектируются не имеют свойства рефлексивности --- системы не могут познавать и понимать себя и развиваться в полной мере самостоятельно. Таким образом, существующие \textit{программные компьютерные системы} не являются как таковыми интеллектуальными, потому что не имеют необходимых им свойств.}
        \scnfileitem{Ключом к легкому и глубокому освоению конкретного языка как основного профессионального инструмента программиста является понимание общих принципов построения и применения языков программирования, описываемых их общей теорией. До сегодняшнего дня, общей \textit{Семантической теории языков программирования} до сих пор не существует, что затрудняет разработку, верификацию и использование новых и существующих \textit{языков программирования}. Без общей теории каждый может разрабатывать принципиально общие методы и средства так, как хочется, а не так, как требуется.}
        \scnfileitem{Достижение максимума услуг и средств при минимуме затрат возможно только путем глубокого понимания принципов построения \textit{языков программирования} за счет простоты средств и методов представления знаний. Сложное нужно сводить к простому и изъяснять простыми понятиями, не создавая дополнительной иллюзии важности.}
    \end{scnhaselementset}
    \begin{scnrelfromset}{смотрите}
        \scnitem{\scncite{Zapata2010}}
        \scnitem{\scncite{Golenkov2019a}}
        \scnitem{\scncite{Penta2020}}
        \scnitem{\scncite{Scalabrino2016}}
        \scnitem{\scncite{Golenkov2012}}
        \scnitem{\scncite{Brooks2021}}
        \scnitem{\scncite{Sellitto2022}}
        \scnitem{\scncite{Penta2020}}
        \scnitem{\scncite{Scalabrino2016}}
        \scnitem{\scncite{Turner2007}}
        \scnitem{\scncite{Golenkov2012}}
        \scnitem{\scncite{Sellitto2022}}
        \scnitem{\scncite{Chaparro2014}}
        \scnitem{Комплекс свойств, определяющий общий уровень качества кибернетической системы}
    \end{scnrelfromset}
    \begin{scnrelfromvector}{введение}
        \scnfileitem{В современную эру развития информационных технологий существует огромное количество \textit{языков программирования}, каждый из которых имеет свое важное назначение в области проектирования \textit{программных компьютерных систем}. Многообразие \textit{языков программирования} и решений, созданных на них, настолько велико, что очень легко потеряться в море информации о всех аспектах применения и проектирования \textit{языков программирования}. Кроме этого, основная проблема заключается не в количестве существующих решений в области разработки и применения современных \textit{языков программирования}, а количестве форм (!), на которых представляются конкретные \textit{языки программирования}. Так, \textit{декларативные знания}, то есть знания, являющиеся, например, спецификацией какой-то программы, и \textit{процедурные знания}, то есть знания, которые являются программами, принадлежащими какому-то \textit{языку программирования}, представляются совершенно различными способами, методами и средствами.}
        \begin{scnindent}
        	\begin{scnrelfromset}{смотрите}
        		\scnitem{\scncite{Sebesta2012}}
        	\end{scnrelfromset}
        \end{scnindent}
        \scnfileitem{Все рассматриваемые проблемы связаны и являются проблемами текущего состояния направлений развития в области \textit{Искусственного интеллекта}.}
        \begin{scnindent}
        	\begin{scnrelfromset}{смотрите}
        		\scnitem{\scncite{Golenkov2022a}}
        	\end{scnrelfromset}
        \end{scnindent}
        \scnfileitem{Итак, для решения перечисленных проблем необходимо создавать комфортные условия для реализации \textit{программных компьютерных систем}, семантически совместимых и интероперабельных между собой. В контексте \textit{языков программирования} необходима общая \textit{Семантическая теория программ для интеллектуальных компьютерных систем нового поколения}, которая:
        \begin{itemize}
            \item \uline{позволит} без больших усилий и затрат \uline{интегрировать имеющиеся решения} в области проектирования программ компьютерных систем;
            \item \uline{объединит формы представления знаний} декларативного и процедурного вида;
            \item \uline{будет иметь широкий спектр средств} не только для описания синтаксиса и семантики существующих языков программирования, но и для проектирования новых аналогов;
            \item \uline{будет понятна} не только человеку, но и машине;
            \item \uline{обозначит принципы}, по которым необходимо проектировать \textit{языки программирования нового поколения}.
        \end{itemize}}
    	\begin{scnindent}
    		\begin{scnrelfromset}{смотрите}
    			\scnitem{\scncite{Golenkov2019}}
    			\scnitem{\scncite{Zapata2010}}
    		\end{scnrelfromset}
    	\end{scnindent}
        \scnfileitem{К проектированию таких общих теорий, строго говоря, нужно подходить с высокой степенью важности. Проектируемые \textit{компьютерные системы} должны всегда иметь возможности использовать те свойства, которые им начертаны. Для того, чтобы и эта теория могла быть использована как некоторая система знаний о том, как надо проектировать и использовать \textit{языки программирования} и программы в \textit{программных компьютерных системах}, и том, как интерпретировать их \textit{программы}, необходимо, чтобы эта теория была описана средствами и методами, которыми проектируются эти \textit{программные компьютерные системы}. Речь идет о том, что принципиально важным подходом к проектированию общей теории программ является \textit{онтологический подход}.}
        \begin{scnindent}
        	\begin{scnrelfromset}{смотрите}
        		\scnitem{\scncite{Golenkov2019}}
        		\scnitem{\scncite{Zapata2010}}
        	\end{scnrelfromset}
        \end{scnindent}
        \scnfileitem{Для воплощения данных идей необходимо изучить и интегрировать опыт, накопленный в области разработки и применения \textit{современных языков программирования}. Поэтому далее будут рассмотрены результаты других исследований в области проектирования общей теории языков программирования и программ.}
    \end{scnrelfromvector}
\end{scnsubstruct}

\scnsegmentheader{Существующие онтологии языков программирования}
\begin{scnsubstruct}
    
    \scnheader{Семантическая теория программ}
    \scntext{примечание}{В большинстве, идеи, предлагаемые в научных работах по исследованию языков программирования, безусловно являются востребованными и полезными для проектирования \textit{программных компьютерных систем}. Так, идея о том, что языки программирования и программы, реализуемые на них, должны быть организованы в общую таксономию понятий, является основополагающей, поскольку обеспечивает наиболее качественную среду для проектирования и реализации \textit{программных компьютерных систем}. Общая теория программ нужна не только для того чтобы описывать термины и понятия как некоторую спецификацию, используемую для проектирования \textit{программных компьютерных систем} (что тоже немаловажно), но и для того, чтобы определять качество языков программирования и программ по таким вопросам, как: \scnqq{Является ли данный язык языком программирования}, \scnqq{Является ли данное знание программой}, \scnqq{Насколько эффективна данная программа}, \scnqq{Какова степень интеллекта данной программной системы} и так далее. Данные идеи предложены и рассмотрены в работах Raymond Turner.}
    \begin{scnindent}
    	\begin{scnrelfromset}{смотрите}
    		\scnitem{\scncite{Eden2007}}
    		\scnitem{\scncite{Turner2007}}
    	\end{scnrelfromset}
    \end{scnindent}
    
   	\scnheader{онтология языков программирования и программ}
    \scntext{примечание}{До сегодняшнего дня существует большое количество аналогов онтологий языков программирования и программ. Также стоит отметить разработанные онтологии программ, система понятий в которых определяется строго и однозначно на формальных языках: языках логики и языках описания грамматик формальных языков. Однако ни одна из них не является таким результатом, который можно было бы использовать при проектировании \textit{программных компьютерных систем} без существенных проблем. Разработанные онтологии сосредотачивают в себе лишь краткое описание связанных между собой понятий, но общей картины того, как данные онтологии можно использовать в конкретных задачах, почти не видно.}
    \begin{scnindent}
    	\begin{scnrelfromset}{смотрите}
    		\scnitem{\scncite{Lando2007}}
    		\scnitem{\scncite{Lando2007a}}
    		\scnitem{\scncite{Turner2014}}
    		\scnitem{\scncite{Turner2007}}
    	\end{scnrelfromset}
    \end{scnindent}
    
    \scnheader{язык представления методов}
    \scntext{примечание}{Сегодня встречаются и вовсе протовоположные суждения о назначении программ и языков программирования, противоречащие формальным основам Искусственного интеллекта. \textit{Программные компьютерные системы} должны быть не только понятны человеку, но и сами должны понимать себя, свои возможности, намерения, действия и цели, и понимать себе подобные кибернетические системы. Только таким образом человечество и результаты его деятельности в виде каких-то конкретных систем смогут работать сообща, дополняя друг друга и преумножая свои результаты.}
    \begin{scnindent}
    	\begin{scnrelfromset}{смотрите}
    		\scnitem{\scncite{Golenkov2012}}
    	\end{scnrelfromset}
    \scntext{примечание}{В результате анализа приведенных работ можно сделать вывод о том, что:
        \begin{itemize}
            \item \textit{общей теории программ и языков программирования}, которая могла быть задействована при решении любой прикладной задачи и представлении и реализации средств проектирования компьютерных систем, до сих пор не существует;
            \item унификация представления средств описания и реализации по этим описании как главный аргумент к оперированию смысловому представлению знаний, к полному взаимопониманию между \textit{программными компьютерными системами} вовсе не рассматривается;
            \item \textit{программы} и совокупности этих \textit{программ} в виде \textit{программных компьютерных систем} реализуются в большинстве случаев в индивидуальном порядке и плохо документируются, что усложняет их использование, интеграцию с другими программами и \textit{программными компьютерными системами}, тестирование и совершенствование.
        \end{itemize}}
	\end{scnindent}
\end{scnsubstruct}

\scnsegmentheader{Предлагаемый подход к разработке технологий программирования для ostis-систем}
\begin{scnsubstruct}
    \begin{scnrelfromlist}{ключевой знак}
        \scnitem{Принципы программирования в интеллектуальных компьютерных системах нового поколения}
    \end{scnrelfromlist}
    
    \scnheader{язык представления методов}
    \begin{scnrelfromset}{проблемы}
        \scnfileitem{Поскольку количество \textit{языков программирования} растет с увеличением потребности в них, то растут и потребности в описании этих \textit{языков программирования} для проектирования и разработки \textit{программных компьютерных систем} на этих языках. То есть, с увеличением количества \textit{языков программирования} растет не только многообразие форм представления знаний, но и количество \textit{программных компьютерных систем} на различных формах представления знаний.}
        \scnfileitem{Многообразие форм представления знаний в свою очередь требует не только качественной спецификации конкретного \textit{языка программирования} для разработки \textit{программ} на этом языке, но и новых требований к существующим разработчикам.}
        \scnfileitem{Новые требования к существующим разработчикам влекут за собой появление барьеров и для создания семантически совместимых и интероперабельных \textit{программных компьютерных систем}, и для обеспечения благоприятной среды для взаимодействия их разработчиков.}
    \end{scnrelfromset}
    \begin{scnindent}
	    \scntext{решение}{Для преодоления данных проблем нет необходимости пересматривать уже существующие решения в области разработки программного обеспечения. Необходимо создавать принципиально новые \textit{языки программирования}, а также реализовывать \textit{программные компьютерные системы} на них, в которых будут учтены и решены существующие проблемы. Для этого следует учитывать следующие \textit{принципы программирования} этих систем.}
        \begin{scnindent}
            \begin{scnrelfromset}{принципы программировния}
                \scnfileitem{Расширение многообразия форм представления знаний происходит за счет появления новых синтаксических конструкций в \textit{языках программирования}. Поэтому разработка \textit{языков программирования} должна сводиться к уточнению \textit{синтаксиса} и \textit{семантики} уже существующих \textit{языков программирования}. При этом все \textit{языки программирования} должны являться подъязыками некоторого базового \textit{языка программирования}.}
                \scnfileitem{Нет необходимости в создании дополнительных языков, с помощью которых можно описывать семантику программ на \textit{языках программирования}. Наоборот, \textit{язык программирования}, на котором разрабатываются программы, должен позволять своими же средствами описывать \textit{семантику} \textit{программ} на этом же языке.}
                \scnfileitem{Документирование \textit{программ}, в том числе \textit{программных компьютерных систем}, должно минимизироваться за счет этапов их качественного проектирования и разработки. Смысл конструкций \textit{программ} \textit{языков программирования} должен быть настолько ясным и понятным, чтобы использование \textit{программ} на этом \textit{языке программирования} не требовало дополнительных ресурсов и инструментов как и у разработчиков этих программ и систем, таких и у новых разработчиков.}
                \scnfileitem{Появление новых программ должно влечь за собой к расширению \textit{Библиотеки многократно используемых программ} и к уменьшению количества семантически эквивалетных программ. Таким образом, программы должны быть не только максимальным образом совместимыми между собой, но и открытыми для переиспользования в других \textit{программных компьютерных системах нового поколения}.}
                \scnfileitem{Полный жизненный цикл разработки новых программ должен обеспечиваться теми же средствами и \textit{языками программирования}, на которых разрабатываются эти программы.}
                \scnfileitem{Сложность программ и \textit{программных компьютерных систем} должна сводиться к минимуму. То, что выглядит сложно, должно и может быть сделано максимально просто.}
                \scnfileitem{Построение качественного коллектива \textit{программных компьютерных систем} может быть обеспечено только совместимостью и интероперабельностью самих систем, и коллективов тех разработчиков, которые их создают.}
                \scnfileitem{Ключом к решению всех этих проблем является общая \textit{Технология проектирования компьютерных систем нового поколения}, на базе которой можно построить общую \textit{Семантическую теорию программ} (дисциплину программирования) для \textit{интеллектуальных компьютерных систем нового поколения}, построенных по принципам \textit{Технологии OSTIS}.}
                \begin{scnindent}
                    \begin{scnrelfromset}{смотрите}
                        \scnitem{\scncite{Deikstra1978}}
                    \end{scnrelfromset}
                \end{scnindent}
            \end{scnrelfromset}
        \end{scnindent}
        \scntext{примечание}{Почему \textit{Технология OSTIS} является ключом к решению описанных проблем в области проектирования и применения \textit{языков программирования}?}
        \begin{scnindent}
            \begin{scnrelfromset}{ответ}
                \scnfileitem{Стандарт Технологии OSTIS уже реализует базовые средства, необходимые для проектирования и разработки интероперабельных \textit{программных компьютерных систем}, в основе которых лежит смысловое представление знаний. Это устраняет не только необходимость создания \textit{онтологий верхнего уровня}, которые должны быть использованы в общей теории программ как базовые для описания понятий этой теории, но и помогает проектировать решения согласованно с другими онтологиями. В результате формируется общая слаженная картина мира, которая (1) непротиворечива, то есть согласована, (2) однозначно трактуема, (3) универсальна и, (4) самое главное, понятна для каждого.}
                \scnfileitem{\textit{Технология OSTIS} проектируется одним языком унифицированного представления знаний, называемым \textit{SC-кодом}. Смысл \textit{программ} и \textit{языков программирования} понятен и однозначен тогда и только тогда, когда этот смысл описывается на одном общем языке, понятному любой \textit{кибернетической системе}.}
                \scnfileitem{\textit{SC-код} синтаксически минимален. Для описания объектов и связей между ними используется минимальное количество знаков. В то же время многообразие этих связей сводится к многобразию знаковых конструкций. Все это обеспечивается за счет представления информации в виде графовых структур.}
                \scnfileitem{SC-код не просто удобен для описания и проектирования каких-то сложных объектов --- с его помощью можно проектировать и реализовывать любые \textit{языки представления знаний}, в том числе программ, компьютерные системы и, вообще, описывать реальный мир.}
                \scnfileitem{Онтологический и компонентный подходы к проектированию любых сложных объектов обеспечивают выполнение главных принципов, по которым должны проектироваться современные системы. То, что реализовано и можно использовать, нужно переиспользовать везде.}
            \end{scnrelfromset}
            \begin{scnrelfromset}{смотрите}
                \scnitem{\scncite{Kasyanov2003}}
                \scnitem{\scncite{Petrov1978}}
            \end{scnrelfromset}
        \end{scnindent}
    \end{scnindent}

    \scnheader{Предметная область и онтология информационных конструкций и языков}
    \begin{scnrelfromlist}{дочерняя предметная область и онтология}
        \scnitem{Предметная область и онтология языков}
        \begin{scnindent}
            \begin{scnrelfromlist}{дочерняя предметная область и онтология}
                \scnitem{Предметная область и онтология естественных языков}
                \scnitem{Предметная область и онтология формальных языков}
            \end{scnrelfromlist}
        \end{scnindent}
    \end{scnrelfromlist}

    \scnheader{Предметная область и онтология формальных языков}
    \begin{scnrelfromlist}{дочерняя предметная область и онтология}
        \scnitem{Предметная область и онтология языков представления знаний}
        \begin{scnindent}
            \begin{scnrelfromlist}{дочерняя предметная область и онтология}
                \scnitem{\scnkeyword{Предметная область и онтология методов}}
            \end{scnrelfromlist}
        \end{scnindent}
    \end{scnrelfromlist}

    \scnheader{Предметная область и онтология методов}
    \begin{scnrelfromlist}{дочерняя предметная область и онтология}
        \scnitem{Предметная область и онтология методов ostis-систем}
        \begin{scnindent}
            \begin{scnrelfromlist}{дочерняя предметная область и онтология}
                \scnitem{Предметная область и онтология процедурных методов ostis-систем}
            \end{scnrelfromlist}
        \end{scnindent}
    \end{scnrelfromlist}

	\scnheader{язык программирования}
    \scntext{примечание}{Каждая теория должна быть согласована понятийно. Несмотря на то, что в литературе сложилась разное трактование понятия \textit{языка программирования}, должно быть одно универсальное. Для этого вместо языков программирования далее \uline{будем говорить о языках представления методов}, а вместо программ этих языков программирования --- о методах как знаковых конструкциях языков представления методов. Такое решение обосновывается тем, что обычно язык выступает в роли инструмента какого-то знания определенного вида, а термин \textit{языка программирования} является вырожденным, поскольку стоит говорить не о языках, на которых что-то можно программировать, а о языках, на которых можно представлять знания определенного вида, в данном случае --- знания процедурного типа. Сами термины \scnqqi{языка программирования} и \scnqqi{программы} будем считать неосновными идентификаторами понятий \scnqqi{языка представления методов} и \scnqqi{метода}, соответственно. Также это правило применяется на все понятия, используемые в данной главе и содержащие термин \scnqqi{метод}.}
    
    \scnheader{Семантическая теория программ в ostis-системах}
    \scntext{примечание}{Следует отметить, что общая \textit{Семантическая теория программ в ostis-системах} не отрицает весь накопленный опыт в сфере разработки современных \textit{технологий программирования}. Наоборот, предлагаемая в данной главе идея позволяет переиспользовать те проверенные инструменты и методы для наиболее быстрой и качественной реализации программ в сложных \textit{программных компьютерных системах}.}
\end{scnsubstruct}

\scntext{заключение}{Данная предметная область является началом \textit{Семантической теории программ для компьютерных систем нового поколения}. Логичным развитием данной предметной области будут:
    \begin{itemize}
        \item уточнение и дополнение понятий \textit{Предметной области и онтологии методов} для достижения полноты теории;
        \item описание дочерних предметных областей \textit{Предметной области и онтологии методов} для конкретных видов методов, а также уточнение денотационной и операционной семантики спецификации этих методов;
        \item описание возможных путей реализации метаметодов интерпретации методов различных я.п.м.;
        \item формализация математических моделей для подсчета оценок эффективности методов.
    \end{itemize}}

\end{scnsubstruct}
\end{SCn}


\scsubsubsection{Пункт 30.3.1. Предметная область и онтология интерпретации современных языков программирования в ostis-системах}
\label{sd_program_interpreting}
\begin{SCn}
\scnsectionheader{Предметная область и онтология интерпретации современных языков программирования в ostis-системах}
\begin{scnsubstruct}

\scnheader{Предметная область интерпретации современных языков программирования в ostis-система}
\scniselement{предметная область}
\begin{scnrelfromset}{автор}
    \scnitem{Зотов Н.В.}
    \scnitem{Шункевич Д.В.}
\end{scnrelfromset}

\begin{scnreltovector}{конкатенация сегментов}
    \scnitem{Синтаксис и семантика программ в ostis-системах}
    \scnitem{Методы и средства поддержки проектирования и разработки программ в ostis-системах}
    \scnitem{Комплекс свойств, определяющих эффективность программ в ostis-системах}
\end{scnreltovector}

\scnsegmentheader{Синтаксис и семантика программ в ostis-системах}
\begin{scnsubstruct}
    \begin{scnreltovector}{конкатенация сегментов}
        \scnitem{Синтаксис программ в ostis-системах}
        \scnitem{Денотационная семантика программ в ostis-системах}
        \scnitem{Операционная семантика программ в ostis-системах}
        \scnitem{Синтаксис и семантика языков программирования в ostis-системах}
    \end{scnreltovector}
    \begin{scnhaselementrolelist}{класс объектов исследования}
        \scnitem{спецификация метода*}
        \scnitem{синтаксис метода*}
        \scnitem{денотационная семантика метода*}
        \scnitem{операционная семантика метода*}
        \scnitem{спецификация языка представления методов*}
        \scnitem{синтаксис языка представления методов*}
        \scnitem{денотационная семантика языка представления методов*}
        \scnitem{операционная семантика языка представления методов*}
    \end{scnhaselementrolelist}
    \begin{scnhaselementrolelist}{ключевой знак}
        \scnitem{Принципы описания синтаксиса и семантики программ в ostis-системах}
        \scnitem{Язык SCP}
    \end{scnhaselementrolelist}
    
    \scnheader{семантика метода $\cup$ синтаксис метода}
    \scntext{примечание}{\textit{синтаксис} и \textit{семантика метода} составляют \textit{спецификацию*} этого \textit{метода}. \textit{Семантику метода} можно рассматривать в двух аспектах: как множество знаний, связанных между собой (то есть \textit{денотационную семантику} данного \textit{метода}), и как знание, которое может быть интерпретировано другим методом (то есть \textit{операционную семантику} данного \textit{метода}).}
\end{scnsubstruct}

\scnsegmentheader{Синтаксис программ в ostis-системах}
\begin{scnsubstruct}
	
	\scnheader{метод}
    \scntext{примечание}{Любой \textit{метод} состоит из \textit{информационных конструкций}, которые задают порядок действий в базе знаний, с помощью которых нужно перейти от исходного состояния к \uline{целевому}, решив таким образом какую-то конкретную задачу. Так, например, в процедурном методе любой такой оператор представляет собой некоторую математическую функцию. Для композиции этих функций в более крупные фрагменты используются выражения и операторы. В свою очередь, линейные последовательности операторов и условные ветвления также могут быть представлены функциями, составленными из функций отдельных компонентов этих конструкций. Цикл легко описывается рекурсивной функцией, составленной из компонентов, входящих в его тело.}

	\scnheader{Синтаксис языков представления методов}
    \scntext{примечание}{\textit{Синтаксис языков представления методов} в ostis-системах может быть формально описан различными способами. Так, например, можно использовать метаязык Бэкуса-Наура для описания синтаксиса любых \textit{языков представления методов}. Другими не менее известными формами представления методов являются контекстно-свободные грамматики, расширенная форма Бэкуса-Наура, синтаксические графы.}
    \begin{scnindent}
    	\begin{scnrelfromset}{смотрите}
    		\scnitem{\scncite{Scott1972}}
    		\scnitem{\scncite{Scott2006}}
    		\scnitem{\scncite{Sebesta2012}}
    	\end{scnrelfromset}
        \scntext{примечание}{Однако значительно более логично и целесообразно описывать \textit{синтаксис} других языков на универсальном \textit{языке представления знаний} --- \textit{SC-коде}. Такой подход позволит ostis-системам самостоятельно понимать, анализировать и генерировать тексты указанных языков на основе принципов, общих для любых форм внешнего представления информации, в том числе нелинейных. Таким образом, языки, написанные на \textit{SC-коде}, имеют такой же синтаксис как и сам \textit{SC-код}.}
            \begin{scnindent}
            \begin{scnrelfromset}{смотрите}
                \scnitem{\scncite{Petrov1978}}
            \end{scnrelfromset}
        \end{scnindent}
    \end{scnindent}

\end{scnsubstruct}

\scnsegmentheader{Денотационная семантика программ в ostis-системах}
\begin{scnsubstruct}
	
	\scnheader{семантика метода}
    \scntext{примечание}{\textit{семантика метода} разъясняет смысл \textit{синтаксических конструкций метода}. Наиболее распространенными методами описания семантики \textit{языков программирования} являются: денотационной, операционный, аксиоматический, алгебраический. На базе принципов Технологии OSTIS, под семантикой метода будем подразумевать объединение \textit{денотационной} и \textit{операционной семантики метода}.}
    \begin{scnindent}
    	\begin{scnrelfromset}{смотрите}
    		\scnitem{\scncite{Orlov2021}}
    	\end{scnrelfromset}
    \end{scnindent}
    \scntext{примечание}{С помощью \textit{SC-кода} можно представлять и те языки, которые не написаны на нем. Проблема будет в том, что форма и смысл языка и его методов будут разделены, то есть будут представлены по-разному. В данном случае \textit{SC-код} выступает мощным инструментом для интеграции спецификаций различных языков внешнего представления знаний. Однако стоит отметить, что в представлении различных форм методов, принадлежащих разным \textit{языкам представления методов}, в рамках \textit{Технологии OSTIS} нет необходимости. Это объясняется тем, что:
        \begin{itemize}
            \item \textit{SC-код} является достаточно универсальным языком для представления любых видов знаний. Это означает, что различные формы алгоритма решения одной и той же задачи можно свести к минимуму. В \textit{SC-коде} фундаментом является формальная теория, что обеспечивает универсальное представление различных видов декларативных и процедурных знаний. Так, \textit{логические программы} можно представлять в виде \textit{процедурных программ}, в которых в качестве операндов операторов будут не только \textit{логические формулы} и \textit{правила вывода}, но и другие методы, обеспечивающее интерпретацию этих \textit{логических формул} при помощи правил вывода. Таким образом, \textit{SC-код} можно называть не только языком унифицированного представления знания, но и языком, на котором можно решать различные классы задачи одним и тем же способом.
            \item Различные виды знаний в \textit{ostis-системах}, проектируемые по принципам \textit{Технологии OSTIS}, глубоко интегрированы между собой. Это дает не только простоту для создания этих систем на базе имеющихся языков, которые могут быть описаны на \textit{SC-коде}, но большие возможности для создания базовых \textit{языков программирования} для \textit{программных компьютерных систем нового поколения} таких, как, например, \textit{базового языка представления процедурных методов SCP}, \textit{базового языка представления продукционных методов} и других. Современные \textit{языки представления методов} создаются с целью упрощения описания какого-то алгоритма для быстрого и качественного решения определенного класса задач. В свою очередь, предлагаемые методики и модели позволяют проектировать \textit{языки представления методов} для \textit{компьютерных систем нового поколения} с помощью базовых \textit{языков представления знаний} таким образом, чтобы сама форма представления знаний не менялась. Методы разных \textit{языков представления методов} должны иметь одну универсальную форму представления, то есть один и тот же синтаксис, но могут давать возможности описывать и представлять разными способами \textit{денотационную} и \textit{операционную семантику} своих \textit{методов} с помощью одного и того же синтаксиса.
            \item Проектирование новых \textit{языков представления методов} должно сводится к их полному описанию на минимальном семействе \textit{языков SC-кода}: \textit{SCP}, \textit{SCL} и других. Речь идет о том, что чтобы спроектировать новый \textit{язык представления методов} достаточно разработать (неатомарный) метаметод на языках \textit{SCP} и \textit{SCL}, который будет интерпретировать методы проектируемых языков, а также описать \textit{денотационную семантику} этих методов. \textit{Метаметод интерпретации методов языков представления методов} можно называть интерпретатором этих языков, то есть некоторой абстрактной sc-машиной, на которой возможно выполнение методов определенного \textit{языка представления} этих \textit{методов}.
        \end{itemize}}
\end{scnsubstruct}

\scnsegmentheader{Операционная семантика программ в ostis-системах}
\begin{scnsubstruct}
	
	\scnheader{полная спецификация метода*}
    \scntext{примечание}{Полная \textit{спецификация метода*} кроме \textit{денотационной семантики этого метода*} должна включать \textit{операционную семантику этого метода*}, то есть формальное описание интерпретатора заданного метода. \textit{Операционная семантика языка представления методов} описывает выполнение \textit{метода}, составленного на данном языке, средствами виртуального компьютера. Виртуальный компьютер определяется как абстрактный автомат. Внутренние состояния этого автомата моделируют состояния вычислительного процесса при выполнении метода. Автомат транслирует исходный текст метода в набор формально определенных операций. Этот набор задает переходы автомата из исходного состояния в последовательность промежуточных состояний, изменяя значения переменных метода. Автомат завершает свою работу, переходя в некоторое конечное состояние. Таким образом, здесь идет речь о достаточно прямой абстракции возможного использования языка представления методов. \textit{операционная семантика языка} описывает смысл метода путем выполнения его операторов на простой машине-автомате. Изменения, происходящие в состоянии машины при выполнении данного оператора, определяют смысл этого оператора.}
    
    \scnheader{операционная семантика метода}
    \scntext{примечание}{\textit{Операционная семантика} конкретного \textit{метода} сводится к описанию \textit{метаметода}, который его интерпретирует, верифицирует и так далее.}
    
    \scnheader{метаметод}
    \scnsubset{метод}
    \scnidtf{метод, значениями параметров которого являются другие методы}

    \scnheader{операционная семантика метода}
    \scnhaselement{метаметод интерпретации*}
    \scnhaselement{метаметод верификации и оценки качества*}

    \scnheader{метаметод интерпретации*}
    \scntext{определение}{Отношение \textit{метаметод интерпретации*} представляет собой \textit{класс sc-связок} между \textit{sc-связкой}, обозначающей множество \textit{методов}, и sc-узлом, обозначающим \textit{метод}, который способен произвести интерпретацию заданного множества \textit{методов}.}
     
    \scnheader{метаметод верификации и оценки качества*}
    \scntext{определение}{Отношение \textit{метаметод верификации и оценки качества*} представляет собой класс sc-связок между \textit{sc-связкой}, обозначающей множество \textit{методов}, и sc-узлом, обозначающим метод, который способен произвести верификацию и оценку качества заданного множества \textit{методов}.}
    
    \scnheader{метаметод интерпретации методов базовых языков представления методов}
    \begin{scnrelfromlist}{класс подметодов}
        \scnitem{метаметод интерпретации методов Языка SCP}
        \scnitem{метаметод интерпретации методов Языка SCL}
        \scnitem{метаметод интерпретации методов языка представления продукционных методов}
        \scnitem{метаметод интерпретации методов языка представления функциональных методов}
        \scnitem{метаметод интерпретации методов языка представления нейросетей}
        \scnitem{метаметод интерпретации методов языка представления генетических алгоритмов}
    \end{scnrelfromlist}
    \scntext{примечание}{В рамках \textit{Технологии OSTIS} таких метаметодов может быть большое разнообразие. Каждый из них может состоять из множества атомарных и неатомарных подметодов. Это могут быть как метаметоды, интерпретирующие методы определенных \textit{языков представления методов}, так и метаметоды, верифицирующие и анализирующие качество этих методов. В том числе метаметоды могут производить операции над другими метаметодами.}

    \scnheader{метаметод верификации и оценки качества методов базовых языков представления методов}
    \begin{scnrelfromlist}{класс подметодов}
        \scnitem{метаметод верификации и оценки качества методов Языка SCP}
        \scnitem{метаметод верификации и оценки качества методов Языка SCL}
        \scnitem{метаметод верификации и оценки качества методов языка представления продукционных методов}
        \scnitem{метаметод верификации и оценки качества методов языка представления функциональных методов}
        \scnitem{метаметод верификации и оценки качества методов языка представления нейросетей}
        \scnitem{метаметод верификации и оценки качества методов языка представления генетических алгоритмов}
    \end{scnrelfromlist}
    \scntext{примечание}{Так, например, при реализации методов в оstis-системах метаметодами будут являться \textit{интепретатор Языка SCP}, а также интепретаторы, реализованные непосредственно на \textit{Языке SCP}.}
    \scntext{примечание}{Понятие \textit{синтакиса}, \textit{денотационной} и операционной \textit{семантики языков представления методов} сводятся к понятию синтаксиса, денотационной и операционной семантики вообще любого языка.}
\end{scnsubstruct}

\scnsegmentheader{Синтаксис и семантика языков программирования в ostis-системах}
\begin{scnsubstruct}
	\scnheader{язык представления методов}
    \scntext{примечание}{Для использования \textit{языка представления методов} следует описать каждую конструкцию языка в отдельности, а также ее применение в совокупности с другими конструкциями. В языке существует множество различных конструкций, точное определение которых необходимо как программисту, применяющему язык, так и разработчику компилятора для этого языка. Программисту эти знания позволяют прогнозировать вычисления, производимые операторами метода. Разработчику описания конструкций необходимы для создания правильной реализации компилятора.}
    \scntext{примечание}{Описание формальной модели \textit{языка представления методов} можно задать его \textit{спецификацией}. \textit{спецификация языка представления методов*} содержит описание \textit{синтаксиса}, \textit{денотационной}, \textit{операционной} \textit{семантики языка представления методов}.}

    \scnheader{спецификация языка представления методов*}
    \scnsuperset{отношение, заданное на множестве (язык представления методов)*}
    \begin{scnrelfromset}{разбиение}
        \scnitem{синтаксис языка представления методов*}
        \begin{scnindent}
            \scnsubset{синтаксис языка*}
            \scnidtf{теория правильно построенных информационных конструкций, принадлежащих заданному языку представления методов}
        \end{scnindent}
        \scnitem{денотационная семантика языка представления методов*}
        \begin{scnindent}
            \scnsubset{денотационная семантика языка*}
            \scnidtf{обобщенная формулировка классов задач, решаемых с помощью данного языка представления методов*}
        \end{scnindent}
        \scnitem{операционная семантика языка представления методов*}
        \begin{scnindent}
            \scnsubset{операционная семантика языка*}
            \scnidtf{перечень обобщенных агентов, обеспечивающих интерпретацию методов заданного языка представления методов*}
            \scnidtf{семейство методов интерпретации текстов данного языка представления методов*}
            \scnidtf{формальное описание интерпретатора заданного языка представления методов*}
        \end{scnindent}
    \end{scnrelfromset}
\end{scnsubstruct}

\scnsegmentheader{Методы и средства поддержки проектирования и разработки программ в ostis-системах}
\begin{scnsubstruct}

    \scntext{примечание}{Текущее состояние в области проектирования и разработки программного обеспечения говорит о том, что разработчики больше стремятся автоматизировать разработку методов на конкретных языках представления методов, чем обеспечить себя инструментальными обучающими средствами их проектирования, в том числе проектирования новых \textit{языков представления методов}. Это приводит к проблемам.}
    \begin{scnindent}
        \begin{scnrelfromset}{проблемы текущего состояния}
            \scnfileitem{В то время, как количество разработчиков, понимающих код какой-то сложной программной системы, уменьшается, требования к этой системе растут все быстрее и быстрее. Зачастую, разработчики сложных программных систем сами не в состоянии объяснить логику работы этих систем. По этой причине необходимо создавать инструментальные средства, которые будут позволять автоматизировать документирование программных систем.}
            \scnfileitem{Для обучения новых разработчиков навыкам работы с программными системы и их разработки необходимо привлекать ресурсы экспертов, понимающих принципы работы этих программных систем. Проблема решается разработкой справочной системы, которая будет позволять не только обучать пользователя тому, как проектировать методы решения задачи и программные системы на основе этих методов, но и указывать на пробелы в смежных дисциплинах, необходимых для достижения качественных результатов всей своей деятельности.}
            \scnfileitem{В инженерии часто разработчики проектируют и разрабатывают решения, которые уже когда-то были созданы другими специалистами. Таким образом, получаются функционально эквивалентные методы решения задач, а то, и вовсе, программные системы, решающие схожие проблемы. Ключом к решению данной проблемы является проектирование семантически мощной \textit{библиотеки многократно используемых методов решения различных задач}.}
        \end{scnrelfromset}
		\begin{scnrelfromset}{смотрите}
			\scnitem{\scncite{Lu2022}}
		\end{scnrelfromset}
	\end{scnindent}

	\scnheader{Cемантическая теория программ}
    \scntext{примечание}{Одной \textit{Cемантической теории программ} недостаточно. Кроме нее, для перманетного и беспрепятственного проектирования и разработки \textit{методов} различного класса необходимо разрабатывать:
    \begin{itemize}
        \item интеллектуальную систему поддержки проектирования и разработки методов, которая будет не только помогать разработчику верифицировать разрабатываемый метод, но и подсказывать способы его разработки;
        \item семантически мощную библиотеку многократно используемых компонентов для быстрого поиска существующих методов решения задач и их применения для решения других более комплексных задач.
    \end{itemize}}
	\begin{scnindent}
		\begin{scnrelfromset}{смотрите}
			\scnitem{\scncite{Gulykina2012}}
			\scnitem{\scncite{Pivovarchik2016}}
			\scnitem{\scncite{Ford2019}}
		\end{scnrelfromset}
	\end{scnindent}

	\scnheader{интеллектуальная система поддержки проектирования и разработки методов}
    \scntext{примечание}{Потенциальная система должна быть частью общего инструментального средства разработки интеллектуальны компьютерных систем нового поколения --- \textit{Метасистемы OSTIS} --- и может состоять из следующих компонентов:
        \begin{itemize}
            \item интеллектуальной help-системы по семантической теории программ;
            \item интеллектуальной help-системы по библиотеке многократно используемых методов решения задач,
            \item интеллектуальной help-системы по комплексу инструментальных средств проектирования  методов решения задач,
            \item интеллектуальной help-системы по методике обучения проектированию различных методов решения задач.
        \end{itemize}}
	\begin{scnindent}
		\begin{scnrelfromset}{смотрите}
			\scnitem{\scncite{IMS}}
		\end{scnrelfromset}
	\end{scnindent}
    \scntext{примечание}{Каждый компонент должен содержать:
        \begin{itemize}
            \item справочную подсистему,
            \item подсистему мониторинга и анализа деятельности разработчика методов решения задач,
            \item подсистему управления обучением.
        \end{itemize}}
    \scntext{примечание}{Каждая из подсистем взаимодействует с другими подсистемами, а также может функционировать автономно.}
    \scntext{примечание}{Справочная подсистема является консультантом-экспертом в области \textit{Семантической теории программ}, который может ответить на любой вопрос новичка или опытного пользователя. Каждая из таких систем может становиться индивидуальным помощников в обучении новых специалистов --- персональным ostis-ассистентом.}

\end{scnsubstruct}

\scnsegmentheader{Комплекс свойств, определяющих эффективность программ в ostis-системах}
\begin{scnsubstruct}

	\scnheader{язык представления методов}
    \scntext{примечание}{\textit{язык представления методов} можно определить множеством показателей, характеризующих отдельные его свойства. Возникает задача введения меры для оценки степени приспособленности языка представления методов к выполнению возложенных на него функций --- \textit{критериев эффективности}. Критерии эффективности методов приводятся на основе частных показателей эффективности этих методов (показателей качества). Способ связи между частными показателями определяет вид критерия эффективности.}
	\begin{scnindent}
		\begin{scnrelfromset}{смотрите}
			\scnitem{\scncite{Orlov2021}}
		\end{scnrelfromset}
	\end{scnindent}
	
    \scnheader{эффективность метода}
    \begin{scnrelfromlist}{свойство-предпосылка}
        \scnitem{легкость чтения и понимания метода}
        \scnitem{легкость представления метода}
        \scnitem{стоимость метода}
        \scnitem{общий объем задач, решаемых при помощи данного класса методов}
        \scnitem{многообразие видов задач, решаемых при помощи данного класса методов}
        \scnitem{надежность метода}
    \end{scnrelfromlist}

    \scnheader{легкость чтения метода}
    \scntext{примечание}{\textbf{\textit{легкость чтения метода}} должна способствовать легкому выделению основных понятий каждой части метода без обращения к его спецификации.}

    \scnheader{легкость чтения и понимания метода}
    \begin{scnrelfromlist}{свойство-предпосылка}
        \scnitem{простота синтаксиса языка представления методов}
        \scnitem{ортогональность информационных конструкций языка представления методов}
        \scnitem{структурированность потока управления в методе}
    \end{scnrelfromlist}

	\scnheader{язык представления методов}
    \scntext{примечание}{\textit{язык представления методов} должен предоставить \textit{простой} набор \textit{информационных конструкций}, которые могут быть использованы в качестве базисных элементов при создании методов.
        \\Сильное воздействие на простоту оказывает \textit{синтаксис языка}: он должен прозрачно отражать семантику конструкций, исключать двусмысленность и неоднозначность толкования.}

	\scnheader{ортогональность информационных конструкций языка представления методов}
    \scntext{примечание}{\textbf{\textit{ортогональность информационных конструкций языка представления методов}} означает, что любые возможные комбинации различных \textit{информационных конструкций} будут осмысленными, без неожиданного поведения, возникающего в результате взаимодействия конструкций или контекста использования.}

	\scnheader{поток управления}
    \scntext{примечание}{Порядок передач управления между операторами метода, то есть \textit{поток управления}, должен быть удобен для чтения и понимания человеком.}

	\scnheader{легкость создания метода}
    \scntext{примечание}{\textbf{\textit{легкость создания метода}} отражает удобство языка для представления этого метода в конкретной предметной области.}

    \scnheader{легкость представления метода}
    \begin{scnrelfromlist}{свойство-предпосылка}
        \scnitem{простота синтаксиса языка представления методов}
        \scnitem{естественность синтаксиса языка представления методов}
        \scnitem{ортогональность информационных конструкций языка представления методов}
        \scnitem{полнота и точность спецификации языка представления методов}
        \scnitem{согласованность и целостность спецификации языка представления методов}
    \end{scnrelfromlist}

	\scnheader{синтаксис метода}
    \scntext{примечание}{\textit{синтаксис метода} должен способствовать легкому и прозрачному отображению в нем алгоритмических структур предметной области. \textit{синтаксис языка представления методов} должен быть не только \textit{простым}, но и \textit{естественным}, и поддерживать \textit{ортогональность} информационных конструкций языка.}

	\scnheader{легкость представления метода}
    \scntext{примечание}{\textbf{\textit{легкость представления нового метода}} обеспечивается \textit{полной и точной, согласованной и целостной спецификацией} соответствующего языка. То есть необходимо достаточное количество \textit{информационных конструкций} в этом языке для того чтобы представить конкретный \textit{метод}. При этом \textit{спецификация языка} должна быть согласованной и целостной чтобы представлять на ней непротиворечивые \textit{методы}.}

    \scnheader{общая стоимость метода}
    \begin{scnrelfromlist}{свойство-предпосылка}
        \scnitem{стоимость применения метода}
        \scnitem{стоимость интерпретации метода}
        \scnitem{стоимость создания, тестирования и использования метода}
        \scnitem{стоимость сопровождения метода}
        \scnitem{согласованность и целостность спецификации языка представления методов}
    \end{scnrelfromlist}
	\scntext{примечание}{Все эти критерии можно применить и касательно самих \textit{языков представления методов}.}
	
	\scnheader{стоимость применения метода}
    \scntext{примечание}{\textbf{\textit{стоимость применения метода}} во многом зависит от структуры \textit{языка представления методов}. Язык, требующий многочисленных проверок синтаксических типов во время применения метода, будет препятствовать быстрой работе программы.}

	\scnheader{размер стоимости интерпретации метода}
    \scntext{примечание}{\textbf{\textit{размер стоимости интерпретации метода}} зависит от возможностей используемого метаметода интерпретации. Чем совершеннее методы оптимизации, тем дороже стоит интерпретация.}
    
    \scnheader{общая стоимость метода}
    \scntext{примечание}{Размер стоимости создания, тестирования и использования метода зависит от используемого метаметода верификации и оценки качества этого метода.}
    \scntext{примечание}{Многочисленные исследования показывают, что значительную часть стоимости используемого метода составляет не стоимость его разработки, а \textit{стоимость его сопровождения}. Связывая сопровождение методов с другими их характеристиками, следует выделить, прежде всего, зависимость от читабельности, поскольку сопровождение обычно происходит следующим поколением разработчиков.}
    \begin{scnindent}
        \begin{scnrelfromset}{смотрите}
            \scnitem{\scncite{Brooks2021}}
        \end{scnrelfromset}
    \end{scnindent}
    
    \scnheader{язык представления методов}
    \scntext{примечание}{\textbf{\textit{общий объем задач и многообразие видов задач, решаемых при помощи данного класса методов}}, являются не менее важными характеристиками и показывают степень универсальности соответствующего языка представления методов. Чем больше задач можно решить на \textit{я.п.м.}, тем он универсальнее.}

	\scnheader{надежность методов языка представления методов}
    \scntext{примечание}{\textbf{\textit{надежность методов языка представления методов}} должна обеспечиваться минимумом ошибок при работе конкретного метода.}
  
\end{scnsubstruct}

\end{scnsubstruct}
\end{SCn}


\scsubsection{\S 30.4. Предметная область и онтология sc-языка вопросов}
\label{sd_sc_quest_lang}
\begin{SCn}

\scnsectionheader{Предметная область и онтология Языка вопросов для ostis-систем}
\begin{scnsubstruct}
    \scniselement{раздел базы знаний}
    \scnhaselementrole{ключевой sc-элемент}{Предметная область Языка вопросов для ostis-систем}

	\scnheader{Предметная область Языка вопросов для ostis-систем}
	\scniselement{предметная область}
    \begin{scnrelfromset}{автор}
        \scnitem{Самодумкин С.А.}
        \scnitem{Зотов Н.В.}
        \scnitem{Шункевич Д.В.}
        \scnitem{Ивашенко В.П.}
    \end{scnrelfromset}
    \begin{scnrelfromlist}{дочерняя предметная область}
        \scnitem{Предметная область синтаксиса Языка вопросов для ostis-систем}
        \scnitem{Предметная область денотационной семантики Языка вопросов для ostis-систем}
        \scnitem{Предметная область операционной семантики Языка вопросов для ostis-систем}
    \end{scnrelfromlist}
    
    \scntext{аннотация}{Возможности \textit{баз знаний} \textit{ostis-систем} позволяют не только представлять и структурировать знания об окружающем мире, но и быстро получать и формировать эти знания о нем, тем самым удовлетворяя информационную потребность пользователя. В данной предметной области уточнена формальная спецификация \textit{Языка вопросов для ostis-систем}, позволяющая описывать и интерпретировать любые классы \textit{вопросов} \textit{пользователей ostis-систем}.}

    \begin{scnrelfromlist}{ключевой знак}
        \scnitem{Язык вопросов для ostis-систем}
    \end{scnrelfromlist}
    
    \begin{scnhaselementrolelist}{класс объектов исследования}
        \scnitem{вопрос}
        \scnitem{ответ на вопрос}
        \scnitem{знак в рамках заданного вопроса}
        \scnitem{основной знак в рамках заданного вопроса}
        \scnitem{неосновной знак в рамках заданного вопроса}
        \scnitem{отношение в рамках заданного вопроса}
        \scnitem{базовое отношение в рамках заданного вопроса}
        \scnitem{интерпретатор Языка вопросов для ostis-систем}
    \end{scnhaselementrolelist}

    \begin{scnrelfromlist}{библиографическая ссылка}
        \scnitem{\scncite{Averyanov1993}}
        \scnitem{\scncite{Suleimanov2001}}
        \scnitem{\scncite{Suleimanov2014}}
        \scnitem{\scncite{Bukharev1990}}
        \scnitem{\scncite{Kwok2001}}
        \scnitem{\scncite{Emelyanov2007}}
        \scnitem{\scncite{Finn1976}}
        \scnitem{\scncite{Finn1981}}
        \scnitem{\scncite{Belnap1981}}
        \scnitem{\scncite{Sosnin2007}}
        \scnitem{\scncite{Zaharov2002}}
        \scnitem{\scncite{Hant1978}}
        \scnitem{\scncite{Lyubarsky1990}}
        \scnitem{\scncite{Samodumkin2009}}
        \scnitem{\scncite{Samodumkin2009a}}
    \end{scnrelfromlist}

    \begin{scnrelfromvector}{введение}
    	\scnfileitem{Одна из ключевых особенностей \textit{интеллектуальной системы} состоит в том, что \textit{пользователь} имеет возможность формулировать свою информационную потребность. Cпособом выражения такой потребности является \textit{вопрос}. В процессе общения всегда существует контекст, который определяет дополнительную информацию, способствующую правильному пониманию \textit{смысла} сообщения. Особенность представления информации в \textit{базах знаний} \textit{ostis-систем} упрощает формирование информационной потребности пользователя, так как представленная информация в \textit{базах знаний} уже структурирована и известны отношения, заданные на определенном понятии, в отношении которого разрешается вопросно-проблемная ситуация.}
    	\scnfileitem{Показано, что вопросно-проблемная ситуация не может быть решена в рамках формальной логики и природа вопроса может быть понятна в системе субъектно-объектных отношений. В связи с тем, что при формировании \textit{баз знаний} \textit{ostis-систем} происходит формирование субъектно-объектных отношений в рамках заданной \textit{предметной области}, тем самым упрощается выражение информационной потребности пользователем средствами \textit{SC-кода}.}
    	\begin{scnindent}
    		\begin{scnrelfromset}{источник}
    			\scnitem{\scncite{Averyanov1993}}
    		\end{scnrelfromset}
    	\end{scnindent}
        \scnfileitem{Целью разработки \textbf{\textit{Языка вопросов для ostis-систем}} и последующего его развития является реализация возможности понимания действий, осуществляемых \textit{ostis-системой}, при формировании ответа на поставленный \textit{вопрос}. В процессе формирования ответа на поставленный \textit{вопрос} возможны следующие варианты:
        \begin{itemize}
            \item \textit{ответ на} поставленный \textit{вопрос} существует в \textit{базе знаний} и происходит локализация \textit{фрагмента базы знаний} в контексте выраженной средствами \textit{SC-кода} информационной потребности \textit{пользователя};
            \item ответ связан с разрешением некоторой задачной ситуации, которая содержится в контексте \textit{вопроса} и формирование \textit{ответа на вопрос} возлагается на \textit{решатель задач}.
        \end{itemize}}
	    \begin{scnindent}
	    	\begin{scnrelfromset}{смотрите}
	    		\scnitem{Агентно-ориентированные модели гибридных решателей задач ostis-систем}
	    	\end{scnrelfromset}
	    \end{scnindent}
	\end{scnrelfromvector}

    \scnheader{Язык вопросов для ostis-систем}
    \scnidtf{Предлагаемый вариант языка для описания вопросов и ответов на них в ostis-системах}
    \scnidtf{sc-язык вопросов}
    \scniselement{sc-язык}
    \scnrelfrom{синтаксис языка}{Синтаксис Языка вопросов для ostis-систем}
    \begin{scnindent}
        \scnsubset{Синтаксис SC-кода}
    \end{scnindent}
    \scnrelfrom{денотационная семантика языка}{Денотационная семантика Языка вопросов для ostis-систем}
    \begin{scnindent}
        \scnidtf{Онтология классов знаков и отношений для описания формулировок вопросов на SC-коде}
        \scnsuperset{Семантическая классификация вопросов}
    \end{scnindent}
    \scnrelfrom{операционная семантика языка}{Операционная семантика Языка вопросов для ostis-систем}
    \begin{scnindent}
        \scnidtf{Коллектив sc-агентов вывода ответов на заданные вопросы пользователя ostis-системы}
    \end{scnindent}

\end{scnsubstruct}
\scnendcurrentsectioncomment
\end{SCn}


\scsubsubsection{Пункт 30.4.1. Предметная область и онтология синтаксиса sc-языка вопросов}
\label{sd_syntax_sc_quest_lang}
\begin{SCn}

\scnsectionheader{Предметная область и онтология синтаксиса Языка вопросов для ostis-систем}
\begin{scnsubstruct}

    \scnheader{Предметная область синтаксиса Языка вопросов для ostis-систем}
	\scniselement{предметная область}
    \begin{scnhaselementrole}{максимальный класс объектов исследования}
        {синтаксис Языка вопросов для ostis-систем}
    \end{scnhaselementrole}
    
    \scnheader{синтаксис Языка вопросов для ostis-систем}
    \scntext{примечание}{\textit{Язык вопросов для ostis-систем} относится к семейству семантических совместимых языков --- \textit{sc-языков}, и предназначен для формального описания поискового предписания \textit{ostis-систем} с целью удовлетворения информационной потребности \textit{пользователя}. Поэтому \textbf{\textit{Синтаксис Языка вопросов для ostis-систем}}, как и \textit{синтаксис} любого другого \textit{sc-языка}, является \textit{Синтаксисом SC-кода}. Такой подход позволяет:
        \begin{itemize}
            \item унифицировать форму представления \textit{вопросов} и \textit{знаний}, с помощью которых строятся ответы на поставленные \textit{вопросы};
            \item использовать минимум средств для интерпретации заданных \textit{вопросов пользователей};
            \item сводить формирование ответов на большую часть заданных \textit{вопросов} к поиску информации в текущем состоянии \textit{базы знаний ostis-системы}.
        \end{itemize}}

\end{scnsubstruct}
\scnendcurrentsectioncomment 
\end{SCn}


\scsubsubsection{Пункт 30.4.2. Предметная область и онтология денотационной семантики sc-языка вопросов}
\label{sd_denot_sem_sc_quest_lang}
\begin{SCn}

\scnsectionheader{Предметная область и онтология денотационной семантики Языка вопросов для ostis-систем}
\begin{scnsubstruct}

    \scnheader{Предметная область денотационной семантики Языка вопросов для ostis-систем}
	\scniselement{предметная область}
    \begin{scnhaselementrolelist}{класс объектов исследования}
        \scnitem{вопрос}
        \scnitem{ответ на вопрос}
        \scnitem{знак в рамках заданного вопроса}
        \scnitem{основной знак в рамках заданного вопроса}
        \scnitem{неосновной знак в рамках заданного вопроса}
        \scnitem{отношение в рамках заданного вопроса}
        \scnitem{базовое отношение в рамках заданного вопроса}
    \end{scnhaselementrolelist}
    \scnhaselementrole{ключевой знак}{Денотационная семантика Языка вопросов для ostis-систем}

    \scnheader{Денотационная семантика Языка вопросов для ostis-систем}
    \scntext{примечание}{\textbf{\textit{Денотационная семантика Языка вопросов для ostis-систем}} включает \textit{классы вопросов} и соответствующие \textit{классы ответов}, необходимые для спецификации формулировок \textit{вопросов} и \textit{ответов} на них, а также \textit{классы знаков} и \textit{отношений}, входящих в структуру любого \textit{вопроса}. В \textit{Семантической классификации вопросов} \textit{Языка вопросов для ostis-систем} заложена идея, описанная в работе \cite{Suleimanov2001}.}
    \begin{scnindent}
    	\begin{scnrelfromset}{источник}
    		\scnitem{\scncite{Suleimanov2001}}
    	\end{scnrelfromset}
    \end{scnindent}
    \scntext{примечание}{Любой \textbf{\textit{вопрос}} на \textit{Языке вопросов для ostis-систем} представляет собой \textit{спецификацию действия} на поиск или синтез \textit{знания}, удовлетворяющего информационную потребность \textit{пользователя}, инициирующего этот \textit{вопрос}. То есть \textit{вопрос} --- это ничто иное как \textit{задача}, с помощью которой выражается потребность пользователя в некоторой информации, возможно хранимой или выводимой в \textit{базе знаний} \textit{ostis-системы}.}
    \begin{scnindent}
    	\begin{scnrelfromset}{смотрите}
    		\scnitem{Формализация понятий действия, задачи, метода, средства, навыка и технологии}
    	\end{scnrelfromset}
    \end{scnindent}
    \scntext{примечание}{Каждому \textit{вопросу} можно взаимно однозначно сопоставить некоторое множество \textit{ответов на} этот \textit{вопрос}. Каждый \textit{ответ на вопрос} представляет собой некоторую \textit{sс-структуру} \textit{семантической окрестности основного знака}, раскрываемого в этом \textit{ответе на} заданный \textit{вопрос}.}

    \scnheader{вопрос}
    \scnidtf{запрос}
    \scnidtf{непроцедурная формулировка задачи на поиск (в текущем состоянии базы знаний) или на синтез знания, удовлетворяющего заданным требованиям}
    \scnidtf{запрос метода (способа) решения заданного (указываемого) \textit{класса задач} либо \textit{плана решения} конкретной указываемой \textit{задачи}}
    \scnidtf{задача, направленная на удовлетворение информационной потребности некоторого субъекта-заказчика}
    \scnsubset{задача}

    \scnheader{ответ на вопрос}
    \scnidtf{ответ на запрос}
    \scnidtf{результат запроса}
    \scnidtf{результат решения задачи на поиск или синтез знания, удовлетворяющий заданным требованиям}
    \scnidtf{семантическая окрестность \textit{основного знака}, знание в которой удовлетворяет информационную потребность пользователя}
    \scnidtf{знание в базе знаний ostis-системы, которое удовлетворяет информационную потребность пользователя}
    \scnsubset{знание}
    
    \scnheader{знак в рамках заданного вопроса}
    \scntext{примечание}{Среди всех классов \textit{знаков в рамках заданного вопроса} \textit{Языка вопросов для ostis-систем} можно выделить наиболее общие по иерархии классы \textit{знаков}.}
    \scnsubset{знак}
    \begin{scnrelfromset}{разбиение}
        \scnitem{основной знак в рамках заданного вопроса}
        \begin{scnindent}
            \scnidtf{ключевой sc-элемент в рамках заданного вопроса}
            \scnidtf{\textit{знак}, относительно которого задан вопрос}
        \end{scnindent}
        \scnitem{неосновной знак в рамках заданного вопроса}
        \begin{scnindent}
            \scnidtf{\textit{знак}, стоящий в некотором отношении с \textit{основным знаком в рамках заданного вопроса}}
        \end{scnindent}
    \end{scnrelfromset}
    \scntext{определение}{\textbf{\textit{знаком в рамках заданного вопроса}} является любой \textit{знак понятия} или \textit{сущности}, принадлежащий этому \textit{вопросу}.}
    \scntext{пояснение}{Между \textit{знаками в рамках заданного вопроса} задается множество связей \textit{отношений}, входящих в состав различных \textit{предметных областей}.}
    
    \scnheader{отношение в рамках заданного вопроса}
    \scnidtf{определенное отношение между знаками \textit{предметной области} в контексте \textit{вопроса}}
    \scnsubset{отношение}
    \scntext{определение}{\textbf{\textit{отношение в рамках заданного вопроса}} представляет собой \textit{отношение} между \textit{знаками} \textit{предметной области}, принадлежащих заданному \textit{вопросу}.}
    \scntext{пояснение}{Среди всех классов \textit{отношений в рамках заданного вопроса} можно выделить класс \textbf{\textit{базовых отношений в рамках заданного вопроса}} и класс \textbf{\textit{составных отношений в рамках заданного вопроса}}.}
    
    \scnheader{базовое отношение в рамках заданного вопроса}
    \scnidtf{\textit{класс отношений}, объединяющий \textit{отношения в заданном вопросе}, отражающие однотипный \textit{смысл} и раскрывающие определенный признак \textit{знаков} \textit{предметной области}}
    \scnsubset{отношение в рамках заданного вопроса}
    \begin{scnrelfromset}{декомпозиция}
        \scnitem{отношение состояния}
        \scnitem{отношение действия}
        \scnitem{отношение состава}
        \scnitem{теоретико-множественное отношение}
        \scnitem{темпоральное отношение}
        \scnitem{пространственное отношение}
        \scnitem{количественное отношение}
        \scnitem{качественное отношение}
    \end{scnrelfromset}
    \scntext{пример}{Например, \textit{отношения в рамках заданного вопроса} такие, как \scnqqi{играет*}, \scnqqi{спит*}, \scnqqi{плавает*}, объединяются в класс \textit{отношений состояния} по признаку выражать состояние знака (то есть данные отношения раскрывают признак \textit{знака} \textit{предметной области} --- \scnqqi{находиться в некотором состоянии}).}
   
    \scnheader{составное отношение в рамках заданного вопроса}
    \scnidtf{устойчивая комбинация двух \textit{отношений действия}: действия, направленного на \textit{параметр вопроса\scnrolesign}, и действия, направленного на \textit{ответ на вопрос*}}
    \scntext{пример}{Например, элемент \textit{составного отношения в рамках заданного вопроса} между \textit{знаками}: \scnqqi{\textit{Нефтеперерабатывающий завод}}, \scnqqi{\textit{нефть}} и \scnqqi{\textit{нефтепродукты}} --- может быть представлен как \scnqqi{Нефтеперерабатывающий завод перерабатывает нефть в нефтепродукты}.}
    
    \scnheader{вопрос}
    \scntext{примечание}{Смысловая классификация \textit{вопросов} дает возможность противопоставить каждому типу вопроса ограниченный набор допустимых, то есть \textit{семантически корректных информационных конструкций}, передающий правильный \textit{смысл} \textit{вопроса} в зависимости от класса \textit{вопроса}. При этом \textbf{\textit{Семантическая классификация вопросов}} позволяет разбить множество \textit{вопросов} на классы, в каждом из которых требуется раскрытие некоторого однотипного \textit{смысла}, заданного классом этого \textit{вопроса}.}
    \begin{scnrelfromset}{декомпозиция}
        \scnitem{вопрос, требующий вывода семантической окрестности \textit{основного знака}}
        \begin{scnindent}
            \begin{scnhaselementrolelist}{пример}
                \scnitem{Вопрос. Что такое \textit{Город Минск}}
            \end{scnhaselementrolelist}
        \end{scnindent}
        \scnitem{вопрос, требующий раскрытия в ответе \textit{базового отношения} \textit{основного знака}}
        \begin{scnindent}
            \begin{scnhaselementrolelist}{пример}
                \scnitem{Вопрос. Что легче: железо или дерево}
            \end{scnhaselementrolelist}
        \end{scnindent}
        \scnitem{вопрос, требующий раскрытия в ответе \textit{составного отношения} \textit{основного знака}}
        \begin{scnindent}
            \scntext{пояснение}{Такому классу \textit{вопросов} соответствуют классы \textit{ответов}, в которых \textit{главный знак} раскрывается через \textit{составное отношение}.}
            \begin{scnhaselementrolelist}{пример}
                \scnitem{Вопрос. Какие Принципы компонентного проектирования в интеллектуальных компьютерных системах нового поколения}
            \end{scnhaselementrolelist}
        \end{scnindent}
        \scnitem{вопрос, требующий раскрытия в ответе произвольной комбинации \textit{базового отношения} и/или \textit{составного отношения} \textit{основного знака}}
        \begin{scnindent}
            \begin{scnhaselementrolelist}{пример}
                \scnitem{Вопрос. Как определяется уровень интеллекта кибернетической системы}
            \end{scnhaselementrolelist}
        \end{scnindent}
        \scnitem{вопрос, требующий раскрытия в ответе более чем одного \textit{основного знака}}
        \begin{scnindent}
            \begin{scnhaselementrolelist}{пример}
                \scnitem{Вопрос. Докажите теорему Пифагора}
            \end{scnhaselementrolelist}
        \end{scnindent}
    \end{scnrelfromset}

    \scnheader{вопрос, требующий раскрытия в ответе \textit{базового отношения} \textit{основного знака}}
    \begin{scnrelfromset}{декомпозиция}
        \scnitem{вопрос, требующий раскрытия в ответе \textit{отношения состава} \textit{основного знака}}
        \begin{scnindent}
            \scnidtf{класс вопросов, в ответах на которые \textit{основной знак} \textit{S} раскрывается через его \textit{отношение состава} в связке с его составляющими знаками \textit{P} и \textit{Q}}
            \begin{scnhaselementrolelist}{пример}
                \scnitem{Вопрос. Какие административные районы входят в состав Города Витебск}
                \begin{scnindent}
                    \scneq{\scnfileimage[35em]{Contents/part_ps/src/images/sd_sc_quest_lang/question_about_vitebsk_regions.png}}
                    \scnrelfrom{ответ на вопрос}{\{Железнодорожный район Города Витебск, Октябрьский район Города Витебск, Первомайский район Города Витебск\}}
                    \begin{scnindent}
                        \scneq{\scnfileimage[35em]{Contents/part_ps/src/images/sd_sc_quest_lang/question_about_vitebsk_regions_answer.png}}
                    \end{scnindent}
                \end{scnindent}
            \end{scnhaselementrolelist}
        \end{scnindent}
        \scnitem{вопрос, требующий раскрытия в ответе \textit{теоретико-множественного отношения} \textit{основного знака}}
        \begin{scnindent}
            \scnidtf{класс вопросов, в ответах на которые \textit{основной знак} \textit{S} раскрывается через его \textit{теоретико-множественное отношение} в связке с другим знаком \textit{P}, содержащего \textit{S} как часть}
            \begin{scnhaselementrolelist}{пример}
                \scnitem{Вопрос. Частью какой области является Смолевичский район}
                \begin{scnindent}
                    \scneq{\scnfileimage[35em]{Contents/part_ps/src/images/sd_sc_quest_lang/question_about_smolevichi_inclusion.png}}
                    \scnrelfrom{ответ на вопрос}{\{Смолевичский район является частью Минской области\}}
                \end{scnindent}
            \end{scnhaselementrolelist}
        \end{scnindent}
        \scnitem{вопрос, требующий раскрытия в ответе \textit{отношения состояния} \textit{основного знака}}
        \begin{scnindent}
            \scnidtf{класс вопросов, в ответах на которые \textit{основной знак} \textit{S} раскрывается через его \textit{отношение состояния}}
            \begin{scnhaselementrolelist}{пример}
                \scnitem{Вопрос. Какие города современной территории Республики Беларусь имели Магдебургское право}
                \begin{scnindent}
                    \scneq{\scnfileimage[35em]{Contents/part_ps/src/images/sd_sc_quest_lang/question_about_minsk_district_town_with_mag_act.png}}
                    \scnrelfrom{ответ на вопрос}{\{Волковыск, Гродно, Мозырь и другие имели Магдебургское право\}}
                \end{scnindent}
            \end{scnhaselementrolelist}
        \end{scnindent}
        \scnitem{вопрос, требующий раскрытия в ответе \textit{отношения действия} \textit{основного знака}}
        \begin{scnindent}
            \scnidtf{класс вопросов, в ответах на которые \textit{основной знак} \textit{S} раскрывается через его \textit{отношение действия} в связке с другим знаком \textit{P}}
        \end{scnindent}
        \scnitem{вопрос, требующий раскрытия в ответе \textit{темпорального отношения} \textit{основного знака}}
        \begin{scnindent}
            \scnidtf{класс вопросов, в ответах на которые \textit{основной знак} \textit{S} раскрывается через его \textit{темпоральное отношение} в связке с другим знаком \textit{P} по некоторой временной шкале}
            \begin{scnhaselementrolelist}{пример}
                \scnitem{Вопрос. Какое событие произошло раньше: Первый раздел Речи Посполитой или Бородинское сражение}
                \begin{scnindent}
                    \scneq{\scnfileimage[35em]{Contents/part_ps/src/images/sd_sc_quest_lang/question_about_events.png}}
                    \scnrelfrom{ответ на вопрос}{\{Первый раздел Речи Посполитой был раньше Бородинского сражения\}}
                    \begin{scnindent}
                        \scneq{\scnfileimage[35em]{Contents/part_ps/src/images/sd_sc_quest_lang/question_about_event_answer.png}}
                    \end{scnindent}
                \end{scnindent}
            \end{scnhaselementrolelist}
        \end{scnindent}
        \scnitem{вопрос, требующий раскрытия в ответе \textit{пространственного отношения} \textit{основного знака}}
        \begin{scnindent}
            \scnidtf{класс вопросов, в ответах на которые \textit{основной знак} \textit{S} раскрывается через \textit{пространственное отношение}, отражающее его положение в пространстве относительно другого знака \textit{P}}
        \end{scnindent}
        \scnitem{вопрос, требующий раскрытия в ответе \textit{количественного отношения} \textit{основного знака}}
        \begin{scnindent}
            \scnidtf{класс вопросов, в ответах на которые раскрывается \textit{количественное отношение} \textit{основного знака}}
            \begin{scnhaselementrolelist}{пример}
                \scnitem{Вопрос. Какова высота Горы Дзержинская}
                \begin{scnindent}
                    \scneq{\scnfileimage[35em]{Contents/part_ps/src/images/sd_sc_quest_lang/question_about_mountain_length.png}}
                    \scnrelfrom{ответ на вопрос}{\{Высота Горы Дзержинская --- 345 м\}}
                \end{scnindent}
            \end{scnhaselementrolelist}
        \end{scnindent}
        \scnitem{вопрос, требующий раскрытия в ответе \textit{качественного отношения} \textit{основного знака}}
        \begin{scnindent}
            \scnidtf{класс вопросов, в ответах на которые раскрывается \textit{качественное отношение} \textit{основного знака} \textit{S} в связке с другим знаком \textit{P}}
            \begin{scnhaselementrolelist}{пример}
                \scnitem{Вопрос. Территория какой административной области больше: Минской или Брестской}
                \begin{scnindent}
                    \scneq{\scnfileimage[35em]{Contents/part_ps/src/images/sd_sc_quest_lang/question_about_district_squares.png}}
                    \scnrelfrom{ответ на вопрос}{\{Территория Минской области больше Брестской\}}
                    \begin{scnindent}
                        \scneq{\scnfileimage[35em]{Contents/part_ps/src/images/sd_sc_quest_lang/question_about_district_squares_answer.png}}
                    \end{scnindent}
                \end{scnindent}
            \end{scnhaselementrolelist}
        \end{scnindent}
    \end{scnrelfromset}

    \scnheader{вопрос, требующий раскрытия в ответе произвольной комбинации \textit{базового отношения} и/или \textit{составного отношения} \textit{основного знака}}
    \begin{scnrelfromset}{декомпозиция}
        \scnitem{вопрос, требующий раскрытия в ответе произвольной комбинации \textit{составного отношения описания} \textit{основного знака}}
        \begin{scnindent}
            \scnidtf{класс вопросов, в ответах на которые раскрываются произвольные комбинации \textit{базового отношения} и/или \textit{составного отношения} \textit{основного знака} \textit{S} в связке с другими знаками}
            \begin{scnhaselementrolelist}{пример}
                \scnitem{\{S состоит из P, Q, W. S переводит X и Y и выполняется раньше Z\}}
                \begin{scnindent}
                    \scnrelto{ответ на вопрос}{Вопрос. Что такое S}
                \end{scnindent}
            \end{scnhaselementrolelist}
        \end{scnindent}
        \scnitem{вопрос, требующий раскрытия в ответе произвольной комбинации \textit{составного отношения определения} \textit{основного знака}}
        \begin{scnindent}
            \scnidtf{класс ответов, в которых \textit{основной знак} \textit{S} раскрывается через \textit{первостепенное понятие} и его \textit{описание}}
            \begin{scnhaselementrolelist}{пример}
                \scnitem{\{Минск --- это столица, которая находится в РБ\}}
                \begin{scnindent}
                    \scnrelto{ответ на вопросы}{Вопрос. Как определяется город Минск}
                \end{scnindent}
            \end{scnhaselementrolelist}
        \end{scnindent}
        \scnitem{вопрос, требующий раскрытия в ответе произвольной комбинации \textit{составного отношения причины} \textit{основного знака}}
        \begin{scnindent}
            \scnidtf{класс вопросов, в ответах на которые раскрывается условие существования некоторых отношений \textit{основного знака} \textit{S} в связке с другими знаками}
            \begin{scnhaselementrolelist}{пример}
                \scnitem{Вопрос. Почему время в пути от города Минска до города Борисова меньше чем время в пути от города Минска до города Орша}
                \begin{scnindent}
                    \scnrelfrom{ответ на вопрос}{\{Время в пути от города Минска до города Борисова меньше чем время в пути от города Минска до города Орша, потому что расстояние от города Минска меньше до города Борисова, чем до города Орша\}}
                \end{scnindent}
            \end{scnhaselementrolelist}
        \end{scnindent}
        \scnitem{вопрос, требующий раскрытия в ответе произвольной комбинации \textit{составного отношения следствия} \textit{основного знака}}
        \scnidtf{класс вопросов, в ответах на которые раскрывается следствие от существования некоторых отношений \textit{основного знака} \textit{S} в связке с другими знаками}
        \begin{scnindent}
            \begin{scnhaselementrolelist}{пример}
                \scnitem{Вопрос. Что следует из того, что расстояние от города Минска до города Борисова меньше расстояния от города Минска до города Орша}
                \begin{scnindent}
                    \scnrelfrom{ответ на вопрос}{\{Расстояние от города Минска до города Борисова меньше расстояния от города Минска до города Орша, поэтому от города Минска до города Борисова время в пути меньше чем до города Орша\}}
                \end{scnindent}
            \end{scnhaselementrolelist}
        \end{scnindent}
    \end{scnrelfromset}

    \scnheader{вопрос, требующий раскрытия в ответе более чем одного \textit{основного знака}}
    \scnsuperset{вопрос, требующий раскрытия в ответе \textit{отношение детализации} знаков, стоящих в некоторых отношениях с \textit{основным знаком}}
    \begin{scnindent}
        \scnidtf{класс вопросов, в ответах на которые происходит детализация знаков, стоящих в некоторых отношениях с \textit{основным знаком} \textit{S}}
        \begin{scnhaselementrolelist}{пример}
            \scnitem{Вопрос. Какая связь водной сети существует между городом Минск и городом Светлогорск}
            \begin{scnindent}
                \scnrelfrom{ответ на вопрос}{\{Город Минск расположен на реке Свислочь, которая впадает в реку Березина, протекающую через город Светлогорск\}}
            \end{scnindent}
        \end{scnhaselementrolelist}
    \end{scnindent}
  
    \scnheader{вопрос}
    \scntext{примечание}{Таким образом, для каждого \textit{вопроса} \textit{пользователя ostis-системы} можно найти класс \textit{вопросов}, на котором можно реализовывать \textit{вывод ответов} на этот \textit{вопрос}. Описанная \textit{Семантическая классификация вопросов} позволяет:
    \begin{itemize}
        \item автоматически структурировать \textit{вопросы} \textit{пользователей} по описанию этих \textit{вопросов};
        \item а также формировать \textit{ответы на} эти \textit{вопросы} с учетом \textit{непроцедурных формулировок} этих \textit{вопросов}.
    \end{itemize}}

\end{scnsubstruct}

\scnendcurrentsectioncomment
    
\end{SCn}


\scsubsubsection{Пункт 30.4.3. Предметная область и онтология операционной семантики sc-языка вопросов}
\label{sd_operat_sem_sc_quest_lang}
\begin{SCn}

\scnsectionheader{Предметная область и онтология операционной семантики Языка вопросов для ostis-систем}
\begin{scnsubstruct}

    \scnheader{Предметная область операционной семантики Языка вопросов для ostis-систем}
	\scniselement{предметная область}
    \begin{scnhaselementrolelist}{класс объектов исследования}
        \scnitem{вопрос}
        \scnitem{ответ на вопрос}
        \scnitem{знак в рамках заданного вопроса}
        \scnitem{основной знак в рамках заданного вопроса}
        \scnitem{неосновной знак в рамках заданного вопроса}
        \scnitem{отношение в рамках заданного вопроса}
        \scnitem{базовое отношение в рамках заданного вопроса}
    \end{scnhaselementrolelist}
   
    \scnheader{вопрос}
    \scntext{примечание}{Каждому классу \textit{вопросов} должен соответствовать определенный \textit{коллектив sc-агентов}, реализующих поиск или синтез из \textit{базы знаний} \textit{ostis-системы} соответствующих ответов на поставленные \textit{вопросы}. Следует отметить, что в зависимости от степени наполненности \textit{базы знаний} \textit{ответы} могут содержаться в \textit{базе знаний} либо отсутствовать в текущей версии \textit{базы знаний}. В случае наличия в \textit{базе знаний} \textit{ответа на} поставленный \textit{вопрос} информационная потребность пользователя реализуется \textit{информационно-поисковыми sc-агентами}, в противном случае --- в зависимости от \textit{классов вопросов} формирование ответов осуществляется специализированными \textit{sc-агентами}, которые в процессе работы дополнительно выполняют вычислительные задачи либо осуществляют синтез на основе \textit{логического вывода} или других \textit{моделей решения задач}.} 
    \begin{scnindent}
    	\begin{scnrelfromset}{смотрите}
    		\scnitem{Смысловое представление логических формул и высказываний в различного вида логиках}
    	\end{scnrelfromset}
    \end{scnindent}
    
    \scnheader{интерпретатор Языка вопросов для ostis-систем}
    \scniselement{неатомарный sc-агент}
    \begin{scnrelfromset}{декомпозиция абстрактного sc-агента}
        \scnitem{Абстрактный sc-агент поиска ответа на заданный вопрос}
        \begin{scnindent}
            \begin{scnrelfromset}{декомпозиция абстрактного sc-агента}
                \scnitem{Абстрактный sc-агент поиска семантической окрестности \textit{основного знака}}
                \scnitem{Абстрактный sc-агент поиска ответа на вопрос, требующий раскрытия в ответе \textit{отношения состава} для \textit{основного знака}}
                \scnitem{Абстрактный sc-агент поиска ответа на вопрос, требующий раскрытия в ответе \textit{теоретико-множественного отношения} для \textit{основного знака}}
                \scnitem{Абстрактный sc-агент поиска ответа на вопрос, требующий раскрытия в ответе \textit{отношения состояния} для \textit{основного знака}}	
                \scnitem{Абстрактный sc-агент поиска ответа на вопрос, требующий раскрытия в ответе \textit{отношения действия} для \textit{основного знака}}	
                \scnitem{Абстрактный sc-агент поиска ответа на вопрос, требующий раскрытия в ответе \textit{темпорального отношения} для \textit{основного знака}}
                \scnitem{Абстрактный sc-агент поиска ответа на вопрос, требующий раскрытия в ответе \textit{пространственного отношения} для \textit{основного знака}}
                \scnitem{Абстрактный sc-агент поиска ответа на вопрос, требующий раскрытия в ответе \textit{количественного отношения} для \textit{основного знака}}
                \scnitem{Абстрактный sc-агент поиска ответа на вопрос, требующий раскрытия в ответе \textit{качественного отношения} для \textit{основного знака}}
                \scnitem{Абстрактный sc-агент поиска ответа на вопрос, требующий раскрытия в ответе \textit{отношения описания} для \textit{основного знака}}
                \scnitem{Абстрактный sc-агент поиска ответа на вопрос, требующий раскрытия в ответе \textit{отношения определения} для \textit{основного знака}}
                \scnitem{Абстрактный sc-агент поиска ответа на вопрос, требующий раскрытия в ответе \textit{отношения причины} для \textit{основного знака}}
                \scnitem{Абстрактный sc-агент поиска ответа на вопрос, требующий раскрытия в ответе \textit{отношения следствия} для \textit{основного знака}}
                \scnitem{Абстрактный sc-агент поиска ответа на вопрос, требующий раскрытия в ответе \textit{отношения детализации} для \textit{основного знака}}
            \end{scnrelfromset}
        \end{scnindent}
        \scnitem{Абстрактный sc-агент синтеза ответа на заданный вопрос}
    \end{scnrelfromset}
    \scntext{примечание}{Все \textit{sc-агенты}, выводящие \textit{ответы на} поставленные \textit{вопросы}, формируют \textit{коллектив sc-агентов} --- \textbf{\textit{интерпретатор Языка вопросов для ostis-систем}}, с помощью которого можно интерпретировать любые классы \textit{вопросов}. \textit{интерпретатор Языка вопросов для ostis-систем} может быть реализован по-разному: в виде \textit{коллектива scp-агентов} или \textit{платформенно-зависимых sc-агентов}.}

\end{scnsubstruct}

\scntext{заключение}{Перечислим основные положения:
\begin{itemize}
    \item информационная потребность \textit{пользователей ostis-системы} может быть выражена в виде \textit{вопросов}, а удовлетворение этой информационной потребности --- в виде \textit{ответов на} заданные \textit{вопросы};
    \item вывод \textit{ответов на} заданные \textit{вопросы} \textit{пользователем ostis-системы} может быть осуществлен путем поиска \textit{знаний} в текущем состоянии \textit{базы знаний} этой \textit{ostis-системы}, либо синтеза новых знаний, отсутствующих в \textit{базе знаний} этой \textit{ostis-системы};
    \item каждый \textit{вопрос} может быть представлен в виде некоторой \textit{спецификации задачи}, инициированной \textit{пользователем ostis-системы} для удовлетворения своей информационной потребности, а \textit{ответ на} этот \textit{вопрос} --- в виде \textit{семантической окрестности} \textit{основного знака в рамках заданного вопроса};
    \item каждому \textit{вопросу} может быть сопоставлен соответствующий класс \textit{вопросов} в \textit{Семантической классификации вопросов};
    \item для синтеза отсутствующих \textit{ответов на} поставленные \textit{вопросы} могут быть использованы различные \textit{модели решения задач}, в том числе \textit{логические модели решения задач};
    \item \textit{ответы на} поставленные \textit{вопросы} могут быть транслированы в \textit{естественно-языковой текст} и визуализированы при помощи соответствующих \textit{естественно-языковых интерфейсов} для удобства выдачи информации любому пользователю.
\end{itemize}}
\bigskip
\scnendcurrentsectioncomment
\end{SCn}


\scsubsection{\S 30.5. Предметная область и онтология операционной семантики логических sc-языков}
\label{sd_operat_sem_sc_logical_lang}
\begin{SCn}
\scnsectionheader{Предметная область и онтология операционной семантики логических sc-языков}
\scntext{аннотация}{Логические модели решения задач являются основой обработки знаний в интеллектуальных системах. В данной главе рассматривается интеграция различных моделей решения задач, в том числе принципы логического вывода, для решения задач на основе общей формальной модели.}
\begin{scnsubstruct}

\scnheader{Предметная область операционной семантики логических sc-языков}
\scniselement{предметная область}
\begin{scnrelfromlist}{соавтор}
    \scnitem{Ивашенко В. П.}
    \scnitem{Шункевич Д. В.}
    \scnitem{Василевская А. П.}
    \scnitem{Орлов М. К.}
\end{scnrelfromlist}

\begin{scnrelfromvector}{библиография}
    \scnitem{\scncite{Lawan2019}}
    \scnitem{\scncite{Golenkov1996_2}}
    \scnitem{\scncite{Averin2004}}
    \scnitem{\scncite{Sethy2021}}
    \scnitem{\scncite{Norton2019}}
    \scnitem{\scncite{Yuxuan2022}}
    \scnitem{\scncite{Safawi2015}}
    \scnitem{\scncite{Gungov2018}}
    \scnitem{\scncite{Geramian2017}}
    \scnitem{\scncite{Son2017}}
    \scnitem{\scncite{Uehara2017}}
    \scnitem{\scncite{Lupea2002}}
    \scnitem{\scncite{Weydert2022}}
    \scnitem{\scncite{Chen2021}}
    \scnitem{\scncite{Rybakov2020}}
    \scnitem{\scncite{Orlov2022b}}
    \scnitem{\scncite{Gavrilova2001}}
\end{scnrelfromvector}

\scnheader{логика}
\begin{scnrelfromlist}{решаемая задача}
    \scnitem{доказательство высказывания}
    \scnitem{аргументация высказывания}
    \scnitem{задача генерации гипотезы}
    \scnitem{задача опровержения гипотезы}
    \begin{scnindent}
        \scncomment{Некоторые гипотезы могут быть опровергнуты, однако извлекая причины того, почему гипотеза опровергнута, можно изменить посылку гипотезы так, чтобы создать новую гипотезу, которая впоследствии может стать теоремой.}
    \end{scnindent}
\end{scnrelfromlist}

\scnheader{современная логика}
\scnrelfrom{объект изучения}{формальный язык}
\begin{scnindent}
    \scnrelfrom{решаемая задача}{выражение логических утверждений}
\end{scnindent}
\scncomment{Логика не изучает то, как были получены знания, она позволяет представлять знания, а также из существующих знаний вывести новые (то есть из имеющихся формул логики вывести новые формулы этой же логики), установить правильность рассуждений.}

\scnheader{логический язык}
\scniselement{формальный язык}
\begin{scnrelfromlist}{цель использования}
    \scnitem{воспроизведение логических формул контекстов естественного языка}
    \scnitem{выражение логических законов}
    \scnitem{выражение способов правильных рассуждений в логических теориях, строящихся в данном языке}
\end{scnrelfromlist}

\scnheader{система логического вывода}
\begin{scnrelfromlist}{используемое средство}
    \scnitem{правило прямого заключения}
    \scnitem{правило резолюции}
\end{scnrelfromlist}
\begin{scnrelfromlist}{актуальная проблема}
    \scnitem{проблема совместимости систем логического вывода}
    \scnitem{проблема коллективного решения задач с использованием различных моделей решения задач}
\end{scnrelfromlist}

\scnheader{модель решения задач}
\begin{scnrelfromlist}{задается}
    \scnitem{язык представления методов решения задач}
    \begin{scnindent}
        \scnrelfrom{обеспечивает}{представление в памяти кибернетической системы некоторого класса методов решения задач}
    \end{scnindent}
    \scnitem{интерпретатор методов решения задач}
    \begin{scnindent}
        \scnrelfrom{определяет}{операционная семантика языка представления методов решения задач}
        \begin{scnindent}
            \scnrelto{имеет операционную семантику}{язык представления методов решения задач}
        \end{scnindent}
    \end{scnindent}
\end{scnrelfromlist}

\scnheader{логическая модель решения задач}
\scniselement{модель решения задач}
\scnrelfrom{задается}{языки логических моделей решения задач}
\begin{scnindent}
    \scnhaselement{Rule Interchange Format}
    \begin{scnindent}
        \scnidtf{RIF}
        \scnrelto{используемый язык}{Semantic Web}
        \scnrelfrom{библиографический источник}{\scncite{Lawan2019}}
    \end{scnindent}
    \scnhaselement{Semantic Web Rule Language}
    \begin{scnindent}
        \scnidtf{SWRL}
        \scnrelto{используемый язык}{Semantic Web}
        \scnrelfrom{библиографический источник}{\scncite{Lawan2019}}
        \begin{scnrelfromlist}{пример}
            \scnheader{Рисунок. Запись правил на языке SWRL}
            \scneqimage[30em]{Contents/part_ps/src/images/sd_operat_sem_sc_logical_lang/swrl_example.png}
        \end{scnrelfromlist}
    \end{scnindent}
    \scnhaselement{SHACL Rules}
    \begin{scnindent}
        \scnrelto{используемый язык}{Semantic Web}
        \scnrelfrom{библиографический источник}{\scncite{Lawan2019}}
    \end{scnindent}
    \scnhaselement{Notation3 Rules}
    \begin{scnindent}
        \scnrelto{используемый язык}{Semantic Web}
        \scnrelfrom{библиографический источник}{\scncite{Lawan2019}}
    \end{scnindent}
    \scnrelfrom{проблема}{проблема языков логических моделей решения задач}
    \begin{scnindent}
        \scnnote{Описанные языки не предусматривают возможность представления формул в различных видах логик, поэтому при помощи них невозможно решить описанные проблемы. Языки правил специально построены для вывода следствий. Синтаксис и семантика языков онтологий и языков правил довольно сильно отличаются, поэтому возникает вопрос, как их совмещать.}
    \end{scnindent}
\end{scnindent}

\scnheader{Пролог}
\scniselement{язык программирования}
\scniselement{система логического программирования}
\scnrelfrom{база знаний}{база знаний системы Пролог}
\begin{scnindent}
    \scnrelfrom{включение}{информация в виде предикатов}
\end{scnindent}
\scnrelfrom{тип программирования}{логическое программирование}
\begin{scnindent}
    \begin{scnrelfromset}{используемые правила вывода}
        \scnitem{правило резолюции}
    \end{scnrelfromset}
\end{scnindent}
\scnrelfrom{задача программы}{доказательство истинности формулы}
\begin{scnindent}
    \scnrelfrom{доказываемая формула}{
        \begin{scnstruct}
            \scnheader{вывод}
            \scnrelfrom{посылка}{имеющиеся формулы}
            \scnrelfrom{следствие}{целевое высказывание}
            \scnrelfrom{основания}{основания вывода}
        \end{scnstruct}}
\end{scnindent}
\begin{scnindent}
    \scnnote{Задача пролог-программы заключается в том, чтобы доказать, является ли заданное целевое высказывание следствием из имеющихся формул и, если является, то каким образом был получен такой вывод.}
\end{scnindent}
\scnnote{Когда пользователь задает вопрос системе Пролог, система ищет соответствующие предикаты в базе знаний и, если они найдены, сравнивает их с заданными условиями.}
\scnrelfrom{проблема}{проблема системы Пролог}
\begin{scnindent}
    \scnnote{Система Пролог хорошо справляется с нетрудными задачами, однако ограничена лишь одним принципом логического вывода и не позволяет учитывать сложноструктурированные знания в различных видах логик.}
\end{scnindent}

\scnheader{абстрактная машина логического вывода}
\scnrelfrom{задается}{логический язык на основе SC-кода}
\begin{scnrelfromset}{решаемые проблемы}
    \scnitem{проблема языков логических моделей решения задач}
    \scnitem{проблема системы Пролог}
\end{scnrelfromset}

\scnheader{Технология OSTIS}
\scnrelfrom{позволяет}{интеграция любого принципа логического вывода}
\begin{scnindent}
    \scnrelfrom{цель}{решение задач в интеллектуальных системах на основе общей формальной логике}
    \scnnote{Для того, чтобы использовать какую-либо новую или существующую модель, необходимо привести ее к предлагаемому формализму, что позволит интегрировать и синхронизировать ее с уже имеющимися в соответствующей \textit{библиотеке многократно используемых компонентов ostis-систем}. Формализм \textit{SC-кода} позволяет описывать широкий спектр понятий и отношений между ними, что делает его подходящим вариантом для реализации логического вывода в интеллектуальных компьютерных системах нового поколения. Кроме того, целесообразно воспользоваться принципом наследования, лежащим в основе иерархической структуризации баз знаний ostis-систем.}
    \begin{scnindent}
        \scnrelto{объяснение}{иерархия логических предметных областей}
    \end{scnindent}
\end{scnindent}



\scnheader{иерархия логических предметных областей}
\begin{scnstruct}
    \scnheader{Предметная область логических формул, высказываний и формальных теорий}
    \begin{scnrelfromlist}{дочерняя предметная область}
        \scnitem{Предметная область логических языков}
        \scnitem{Предметная область логического вывода}
    \end{scnrelfromlist}
    
    \scnheader{Предметная область логических языков}
    \scnrelfrom{дочерняя предметная область}{Предметная область языка логики высказываний}
    \begin{scnindent}
        \scnrelfrom{дочерняя предметная область}{Предметная область языка логики предикатов}
    \end{scnindent}
    
    \scnheader{Предметная область логических моделей решения задач}
    \begin{scnreltolist}{дочерняя предметная область}
        \scnitem{Предметная область логических языков}
        \scnitem{Предметная область логического вывода}
    \end{scnreltolist}
\end{scnstruct}
\begin{scnrelfromvector}{примечание}
    \scnfileitem{Наследование предметных областей позволяет использовать описанные логики и их компоненты при описании любых логик.}
    \scnfileitem{Базовые понятия позволяют разработчикам интеллектуальной системы добавлять новые логики.}
    \scnfileitem{Для реализации конкретной логической модели решения задач необходимо создать предметную область, которая будет дочерней по отношению к \textit{Предметной области логических моделей решения задач} и предметной области некоторого \textit{логического языка}, например, языка логики высказываний, языка логики предикатов, языка нечеткой логики и других.}
\end{scnrelfromvector}

\scnheader{Предметная область логических формул, высказываний и формальных теорий}
\scnrelfrom{задает}{денотационная семантика логических формул, высказываний и формальных теорий}
\scnrelfrom{содержит}{формальная спецификация понятий}
\begin{scnindent}
    \scnrelto{необходимое условие формирования}{логические формулы и высказывания}
\end{scnindent}

\scnheader{Предметная область логических моделей решения задач}
\scnrelfrom{описанные понятия}{понятия Предметной области логических моделей решения задач}
\begin{scnindent}
    \scnrelfrom{средство интерпретации}{логические формулы и высказывания}
\end{scnindent}
\scnsubset{модель абстрактных агентов}
\scnsubset{реализация абстрактных агентов}
\begin{scnhaselementrolelist}{ключевое понятие}
    \scnitem{логический вывод}
    \scnitem{правило вывода}
    \scnitem{равносильное преобразование}
    \scnitem{аксиомная схема}
\end{scnhaselementrolelist}

\scnheader{Язык SCL}
\scnrelto{подъязык}{SC-код}
\scnrelfrom{использование}{запись логических утверждений}
\scnrelfrom{библиографический источник}{\scncite{Golenkov1996_2}}
\scnrelfrom{предметная область с описанием}{Предметная область и онтология знаний и баз знаний ostis-систем}
\scnrelfrom{высказывания языка}{высказывания Языка SCL}
\begin{scnindent}
    \scnrelto{использует}{логический вывод}
\end{scnindent}

\scnheader{вывод в формальной системе}
\scndefinition{Любая последовательность формул такая, что любая формула либо аксиома этой формальной системы, либо непосредственное следствие каких-либо предыдущих формул по одному из правил вывода}
\scnnote{Доказательство вывода формулы равносильно доказательству противоречивости вывода отрицания этой формулы. При использовании правила резолюции это особенно удобно использовать.}

\scnheader{теорема}
\scndefinition{формула, которая выводится из аксиом}

\scnheader{правильность умозаключений}
\scnrelfrom{способ ввода}{формальный ввод}
\scnrelfrom{способ проверки}{формальная проверка}
\begin{scnindent}
    \scnrelfrom{завит от}{структура умозаключений}
    \scnrelfrom{не зависит от}{истинность посылок умозаключений}
    \begin{scnindent}
        \scnrelto{задает}{состояние базы знаний}
    \end{scnindent}
\end{scnindent}
\scnrelfrom{свойство}{свойство формальной правильности умозаключений}
\begin{scnindent}
    \scnnote{Если нам удалось доказать, пользуясь методами формальной логики, правильность рассуждения, и нам известно из опыта, что все используемые посылки истинны, то мы можем быть уверены в истинности заключения}
    \scnrelfrom{библиографический источник}{\scncite{Averin2004}}
\end{scnindent}

\scnheader{классический дедуктивный вывод}
\scnnote{Некоторые операции, необходимые в одной предметной области будут избыточными в другой. Например, в системе, решающей задачи по геометрии, химии и другим естественным наукам обоснованным будет использование дедуктивных методов вывода, поскольку решение задач в таких предметных областях основывается только на достоверных правилах. В системах же медицинской диагностики, к примеру, постоянно возникает ситуация, когда диагноз может быть поставлен только с некоторой долей уверенности и абсолютно достоверным ответ на поставленный вопрос быть не может. В связи с этим возникает необходимость использования различных \textit{решателей задач} в различных системах, при этом их состав и возможности в конкретной системе определяется не только непосредственно разработчиком, а требует консультаций с экспертами в данной предметной области. Тем не менее основой для всех видов логик является классическая логика и наиболее общие ее методы распространяются на другие логики с некоторыми модификациями, уточнениями и ограничениями.}
\scnrelfrom{библиографический источник}{\scncite{Golenkov2004}}

\scnheader{логический метод решения задач}
\scnhaselement{классический дедуктивный вывод}
\begin{scnindent}
    \scnsubset{прямой логический вывод}
    \scnsubset{обратный логический вывод}
    \scnsubset{силлогизм}
    \scntext{причина популярности}{всегда дает достоверный ответ}
    \scntext{проблема}{невозможность использования, когда отсутствуют достоверные знания}
    \scnrelfrom{библиографический источник}{\scncite{Averin2004}}
    \scnrelfrom{библиографический источник}{\scncite{Sethy2021}}
\end{scnindent}
\scnhaselement{индуктивный вывод}
\begin{scnindent}
    \scntext{причина удобства}{предоставляет возможность в процессе решения использования предположений}
    \begin{scnreltolist}{используемый метод}
        \scnitem{слабоформализованная предметная область}
        \scnitem{трудноформализуемая предметная область}
    \end{scnreltolist}
    \scnrelfrom{библиографический источник}{\scncite{Norton2019}}
    \scnrelfrom{библиографический источник}{\scncite{Yuxuan2022}}
\end{scnindent}
\scnhaselement{абдуктивный вывод}
\begin{scnindent}
    \scndefinition{В искусственном интеллекте, вывод наилучшего абдуктивного объяснения события, ставшего неожиданным для системы}
    \scntext{критерий наилучшего объяснения}{объяснение, которое удовлетворяет специальным критериям, определяемым в зависимости от решаемой задачи и используемой формализации}
    \begin{scnreltolist}{используемый метод}
        \scnitem{слабоформализованная предметная область}
        \scnitem{трудноформализуемая предметная область}
    \end{scnreltolist}
    \scnrelfrom{библиографический источник}{\scncite{Safawi2015}}
    \scnrelfrom{библиографический источник}{\scncite{Gungov2018}}
\end{scnindent}
\scnhaselement{нечеткая логика}
\begin{scnindent}
    \scntext{причина удобства}{предоставляет возможность в процессе решения использования предположений}
    \scnrelto{используемый метод}{трудноформализуемая предметная область}
    \scnnote{Импликативные высказывания могут рассматриваться как "если истинна посылка"{}, то с некоторой вероятностью (часто или редко) истинно заключение, в отличие от классической логики, где зачастую используются статические предметные области и выражение "часто или редко"{} не применимо (корректно использовать только наречие "всегда"{}).}
    \scnrelfrom{библиографический источник}{\scncite{Geramian2017}}
    \scnrelfrom{библиографический источник}{\scncite{Son2017}}
    \scnrelfrom{библиографический источник}{\scncite{Uehara2017}}
\end{scnindent}
\scnhaselement{логика умолчаний}
\begin{scnindent}
    \scntext{способ оптимизации процесса рассуждения}{процесс достоверного вывода дополняется вероятностными предположениями в тех случаях, когда вероятность ошибки крайне мала}
    \scnrelfrom{библиографический источник}{\scncite{Lupea2002}}
    \scnrelfrom{библиографический источник}{\scncite{Weydert2022}}
\end{scnindent}
\scnhaselement{темпоральная логика}
\begin{scnindent}
    \scnrelto{используемый метод}{нестатичная предметная область}
    \begin{scnindent}
        \scntext{критерий нестатичности}{истинность утверждений меняется со временем}
        \scnrelfrom{возможный язык описания}{Язык SCL}
    \end{scnindent}
    \scnrelfrom{библиографический источник}{\scncite{Chen2021}}
    \scnrelfrom{библиографический источник}{\scncite{Rybakov2020}}
\end{scnindent}

\scnheader{база знаний интеллектуальной системы}
\scnsubset{модель фактографических знаний о предметной области}
\scnsubset{модель знаний, включающая в себя логические формулы о предметной области}

\scnheader{Абстрактная scl-машина}
\scniselement{машина логического вывода}
\scniselement{абстрактная sc-машина}
\scnrelfrom{библиографический источник}{\scncite{Golenkov1996_2}}
\scnrelfrom{внутренний язык}{Язык SCL}
\scnrelfrom{множество операций}{операции Абстрактной scl-машины}
\begin{scnindent}
    \scnrelto{соответствие}{правило логического вывода}
\end{scnindent}
\scnnote{Семейство специализированных абстрактных графодинамических машин обработки знаний является формальным уточнением операционной семантики указанных выше специализированных графовых языков представления знаний, каждому из которых соответствует одна или несколько абстрактных машин. Эти абстрактные машины соответствуют различным моделям решения задач, различным логикам, различным моделям правдоподобных рассуждений. }
\scnnote{Агент из семейства агентов логического вывода может представлять собой какое-либо правило вывода, которое можно применять для решения логической задачи. Кроме того, необходимы агенты для выполнения равносильных преобразований логической формулы (например, записать формулу эквиваленции как конъюнкцию двух дизъюнкций) и другие агенты, помогающие применять правила вывода на множестве формул языка логики.}
\begin{scnindent}
    \scnrelfrom{библиографический источник}{\scncite{Orlov2022b}}
\end{scnindent}

\scnheader{Абстрактная scl-машина}
\begin{scnrelfromset}{декомпозиция абстрактного sc-агента}
    \scnitem{Абстрактный sc-агент применения правила вывода}
    \begin{scnindent}
        \scnrelfrom{использует}{Абстрактный sc-агент эквивалентных преобразований логической формулы}
        \scnrelfrom{задача}{применение заданного правила вывода с заданными логическими формулами}
        \scnrelfrom{условие инициирования}{появление в sc-памяти инициированного действия, принадлежащего классу \textit{действие применение правила вывода}}
        \begin{scnrelfromvector}{последовательность действий}
            \scnitem{проверка условия инициирования}
            \scnitem{применение правил вывода}
            \begin{scnindent}
                \scnnote{проверка, существуют ли в sc-памяти структуры, соответствующие условию применения данного правила и генерация sc-конструкций в соответствии с применяемым правилом}
            \end{scnindent}
        \end{scnrelfromvector}
        \scnrelfrom{автоматически выполняемая процедура во время работы}{унификация}
        \begin{scnindent}
            \scnnote{переменные соответствуют константам, константы соответствуют самим себе}
        \end{scnindent}
        \begin{scnrelfromlist}{пример используемого правила}
            \scnheader{SCg-текст. Формализация правила вывода Modus ponens}
            \scneqimage[30em]{Contents/part_ps/src/images/sd_operat_sem_sc_logical_lang/Modus_ponens.png}
            \scnnote{Можно привести еще целый ряд высказываний, которые описывают общие свойства всевозможных формальных теорий, каждая из которых описывает ту или иную предметную область. Свойства всевозможных формальных теорий описываются в рамках специальной метатеории для которой совокупность всевозможных формальных теорий является описываемой предметной областью.}
        \end{scnrelfromlist}
    \end{scnindent}
    \scnitem{Абстрактный sc-агент эквивалентных преобразований логической формулы}
    \begin{scnindent}
        \scnrelfrom{задача}{применение правил, которые приводят логическую формулу в определенный вид}
        \scnrelfrom{условие инициирования}{появление в sc-памяти инициированного действия, принадлежащего классу \textit{действие эквивалентное преобразование логической формулы}}
        \begin{scnrelfromvector}{последовательность действий}
            \scnitem{проверка условия инициирования}
            \scnitem{преобразование формулы из одной формы в другую}
            \begin{scnindent}
                \scnnote{При преобразовании никакие новые знания в sc-памяти с точки зрения исследуемой предметной области не генерируются.}
            \end{scnindent}
        \end{scnrelfromvector}
        \scnrelfrom{ответ}{множество формул, эквивалентных по смыслу, но различных по форме представления}
        \begin{scnindent}
            \begin{scnrelfromlist}{пример}
                \scnitem{конъюнктивная нормальная форма}
                \scnitem{дизъюнктивная нормальная форма}
            \end{scnrelfromlist}
        \end{scnindent}
        \scntext{решаемая проблема}{Логические формулы не всегда находятся в той форме, которая доступна для применения того или иного правила вывода, однако может быть приведена к нужной форме}
    \end{scnindent}
    \scnitem{Абстрактный sc-агент прямого логического вывода}
    \begin{scnindent}
        \scnrelfrom{использует}{Абстрактный sc-агент применения правила вывода}
        \scnrelfrom{задача}{генерация новых знаний на основе некоторых логических утверждений}
        \scnrelfrom{условие инициирования}{появление в sc-памяти инициированного действия, принадлежащего классу \textit{действие прямого логического вывода}}
        \begin{scnrelfromvector}{последовательность действий}
            \scnitem{проверка условия инициирования}
            \scnitem{процесс прямого логического вывода}
            \begin{scnindent}
                \begin{scnrelfromset}{декомпозиция}
                    \scnitem{применение правил вывода}
                    \begin{scnindent}
                        \scnrelfrom{основа поиска правил}{посылки формул}
                    \end{scnindent}
                    \scnitem{генерация новых знаний в sc-памяти}
                    \scnitem{проверка некоторого условия}
                    \begin{scnindent}
                        \scnrelfrom{пример}{появление в sc-памяти sc-элементов из целевой sc-структуры}
                    \end{scnindent}
                \end{scnrelfromset}
                \scnrelfrom{библиографический источник}{\scncite{Gavrilova2001}}
            \end{scnindent}
        \end{scnrelfromvector}
        \scnrelfrom{ответ}{\scnnonamednode}
        \begin{scnindent}
            \begin{scnreltoset}{объединение}
                \scnitem{целевая структура}
                \scnitem{множество формул, которые используются в ходе вывода агентом применения правил вывода}
                \scnitem{множество правил вывода}
                \scnitem{входная структура}
                \scnitem{выходная структура}
            \end{scnreltoset}
        \end{scnindent}
        \scnrelfrom{результат работы}{дерево решения}
        \begin{scnindent}
            \scnnote{Это дерево состоит из последовательности узлов, представляющих собой примененные правила, которые привели к появлению в sc-памяти требуемых знаний. Такое дерево может быть пустым в случае, если требуемую структуру не удалось сгенерировать в ходе логического вывода.}
        \end{scnindent}
        \begin{scnrelfromlist}{пример спецификации}
            \scnheader{SCg-текст. Спецификация агента прямого логического вывода}
            \scneqimage[40em]{Contents/part_ps/src/images/sd_operat_sem_sc_logical_lang/direct_inference_agent.png}
        \end{scnrelfromlist}
    \end{scnindent}
    \scnitem{Абстрактный sc-агент обратного логического вывода}
    \begin{scnindent}
        \scnrelfrom{использует}{Абстрактный sc-агент применения правила вывода}
        \scnrelfrom{задача}{проверка гипотез}
            \begin{scnindent}
                \scnnote{Некоторые гипотезы могут быть опровергнуты, однако извлекая причины того, почему гипотеза опровергнута, можно изменить посылку гипотезы так, чтобы создать новую гипотезу, которая впоследствии может стать полезной теоремой.}
            \end{scnindent}
        \scnrelfrom{условие инициирования}{появление в sc-памяти инициированного действия, принадлежащего классу \textit{действие обратного логического вывода}}
        \begin{scnrelfromvector}{последовательность действий}
            \scnitem{проверка условия инициирования}
            \scnitem{процесс обратного логического вывода}
            \begin{scnindent}
                \scnrelfrom{основа поиска правил}{следствия формул формул}
                \scnrelfrom{библиографический источник}{\scncite{Gavrilova2001}}
            \end{scnindent}
        \end{scnrelfromvector}
        \scnrelfrom{ответ}{дерево решения}
            \begin{scnindent}
                \scntext{решаемая проблема}{Дерево решения показывает, с использованием каких правил можно доказать или опровергнуть выдвинутую гипотезу.}
            \end{scnindent}
    \end{scnindent}
\end{scnrelfromset}
\scnhaselement{Реализации интерпретатора логических моделей решения задач}
\begin{scnindent}
    \scnidtf{Реализация scl-машины}
    \scntext{адрес компонента}{https://github.com/ostis-ai/scl-machine}
\end{scnindent}

\scnheader{Абстрактный sc-агент эквивалентных преобразований логической формулы}
\begin{scnrelfromset}{декомпозиция абстрактного sc-агента}
    \scnitem{Абстрактный sc-агент преобразования формулы в конъюнктивную нормальную форму}
    \scnitem{Абстрактный sc-агент преобразования формулы в дизъюнктивную нормальную форму}
    \scnitem{Абстрактный sc-агент применения законов Де Моргана}
    \scnitem{Абстрактный sc-агент эквивалентных преобразований логической формулы по определению}
    \scnitem{Абстрактный sc-агент применения свойств отрицания логических формул}
    \scnitem{Абстрактный sc-агент применения закона идемпотентности логических формул}
    \scnitem{Абстрактный sc-агент применения закона коммутативности логических формул}
    \scnitem{Абстрактный sc-агент применения закона ассоциативности логических формул}
    \scnitem{Абстрактный sc-агент применения закона поглощения логических формул}
    \scnitem{Абстрактный sc-агент применения закона противоречия логических формул}
    \scnitem{Абстрактный sc-агент применения закона двойного отрицания логических формул}
    \scnitem{Абстрактный sc-агент применения закона расщепления логических формул}
\end{scnrelfromset}

\scnheader{правило резолюции}
\begin{scnrelfromvector}{использует}
    \scnitem{формула в конъюнктивной нормальной форме}
    \begin{scnindent}
        \scnnote{Любая формула семантически эквивалентна некоторой формуле в конъюнктивной нормальной форме, в связи с этим иногда удобно применять правило резолюции.}
        \begin{scnrelfromlist}{пример}
            \scnheader{SCg-текст. Формализация конъюнктивной нормальной формы для импликации}
            \scneqimage[30em]{Contents/part_ps/src/images/sd_operat_sem_sc_logical_lang/conjunction_implication_rule.png}
        \end{scnrelfromlist}
    \end{scnindent}
    \scnitem{формула полученная в результате применения закона Де Моргана}
    \begin{scnindent}
        \scnnote{Используя законы Де Моргана можно получить формулы, пригодные для использования правила резолюции.}
    \end{scnindent}
\end{scnrelfromvector}
\scnrelto{использует}{доказательство формул Языка логики высказываний}
\scnnote{Ничего принципиально нового правило резолюции не привносит, поскольку формула $A \Rightarrow B$  равносильно $\neg A \lor B$ и из выводимости A и $A \rightarrow B$ следует выводимость B.}
\scntext{пример}{Если в любых двух дизъюнктах $C_1$ и $C_2$ имеется пара формул $A$ и $\neg A$, то можно сформировать новый дизъюнкт из оставшихся частей изначальных дизъюнктов.}
\begin{scnrelfromlist}{пример}
    \scnheader{SCg-текст. Формализация правила резолюции}
    \scneqimage[30em]{Contents/part_ps/src/images/sd_operat_sem_sc_logical_lang/resolution.png}
\end{scnrelfromlist}
\scnrelfrom{пример использования}{задача с футбольными командами}
\begin{scnindent}
    \scnrelfrom{условие}{условие задачи с футбольными командами}
    \begin{scnindent}
        \scnexplanation{Если команда A выигрывает в футбол, то город A' торжествует, а если выигрывает команда B, то торжествовать будет город B'. Выиграть может или только город A', или только город B'. Однако, если выигрывает команда A, то город B' не торжествует, а если выигрывает команда B, то не торжествует город A'. Следовательно, город B' торжествует тогда и только тогда, когда не будет торжествовать город A'.}
    \end{scnindent}
    \scnrelfrom{цель}{цель задачи с футбольными командами}
    \begin{scnindent}
        \scnexplanation{Удостовериться, что город B' торжествует тогда и только тогда, когда не будет торжествовать город A'.}
    \end{scnindent}
    \begin{scnrelfromlist}{формализация формул, соответствующих задаче}
        \scnheader{SCg-текст. Формализация правил для применения правила резолюции}
        \scneqimage[30em]{Contents/part_ps/src/images/sd_operat_sem_sc_logical_lang/resolution_formulas_example.png}
        \begin{scnindent}
            \scnnote{Каждая неатомарная формула на рисунке принадлежит некоторой формальной теории, то есть считается истинной.}
            \scnnote{Структура A представляет собой атомарную логическую формулу, которая обозначает победу команды A, структура A' представляет формулу, обозначающую торжество города A'. Соответственно, то же самое для структур B и B'.}
        \end{scnindent}
    \end{scnrelfromlist}
    \begin{scnrelfromvector}{шаги решения}
        \scnitem{выражение логических формул в булевом базисе}
        \begin{scnrelfromset}{декомпозиция}
            \scnitem{приведение импликации в конъюнктивную нормальную форму}
            \begin{scnindent}
                \scnrelfrom{используемая формула}{конъюнктивная нормальная форма для импликации}
            \end{scnindent}
            \scnitem{приведение эквиваленции в конъюнктивную нормальную форму}
            \begin{scnindent}
                \scnrelfrom{используемая формула}{определение эквиваленции}
            \end{scnindent}
        \end{scnrelfromset}
        \scnitem{применение отрицания к целевой формуле}
        \begin{scnindent}
            \begin{scnrelfromlist}{состояние после шагов решения}
                \scnheader{SCg-текст. Формализация правил для применения правила резолюции после преобразования в конъюнктивную нормальную форму}
                \scneqimage[30em]{Contents/part_ps/src/images/sd_operat_sem_sc_logical_lang/resolution_prepared_formulas_example.png}
            \end{scnrelfromlist}
        \end{scnindent}
        \scnitem{применение правила резолюции для преобразованных формул}
        \begin{scnindent}
            \scnrelfrom{результат}{пустой дизъюнкт}
            \scnnote{Пустой дизъюнкт говорит о противоречивости множества формул и доказывает формулу эквиваленции о том, что город B' торжествует тогда и только тогда, когда не будет торжествовать город A'.}
        \end{scnindent}
    \end{scnrelfromvector}
    \begin{scnrelfromlist}{примененные правила резолюции}
        \scnheader{SCg-текст. Применение принципа резолюции}
        \scneqimage[45em]{Contents/part_ps/src/images/sd_operat_sem_sc_logical_lang/resolution_inference.png}
    \end{scnrelfromlist}
\end{scnindent}


\bigskip
\end{scnsubstruct}
\scnendsegmentcomment{Предметная область и онтология операционной семантики логических sc-языков}
    
\end{SCn}


\scsubsection{\S 30.6. Предметная область и онтология sc-языков продукционного программирования}
\label{sd_sc_product_program_lang}
\begin{SCn}
\scnsectionheader{Предметная область и онтология sc-языков продукционного программирования}
\begin{scnsubstruct}
\scniselement{раздел базы знаний}
\scnhaselementrole{ключевой sc-элемент}{Предметная область sc-языков продукционного программирования}

\scnheader{Предметная область sc-языков продукционного программирования}
\scniselement{предметная область}
\begin{scnhaselementrolelist}{максимальный класс объектов исследования}
    \scnitem{язык продукционного программирования}
    \scnitem{продукция}
\end{scnhaselementrolelist}

\begin{scnrelfromlist}{соавтор}
    \scnitem{Орлов М.К.}
    \scnitem{Зотов Н.В.}
\end{scnrelfromlist}

\scnheader{продукция}
\scniselement{средство представления знаний}
\scnexplanation{Продукции наряду с фреймами являются наиболее популярными средствами представления знаний в интеллектуальных компьютерных системах.}
\begin{scnrelfromlist}{преимущества}
    \scnfileitem{Продукции близки к  логическим моделям, что позволяет организовывать на них эффективные процедуры вывода.}
    \scnfileitem{Продукции более наглядно отражают знания, чем классические логические модели.}
    \scnfileitem{В продукциях отсутствуют жесткие ограничения, характерные для логических исчислений, что дает возможность изменять интерпретацию элементов продукции.}
\end{scnrelfromlist}
\scnnote{Так как в основе \textit{Технологии OSTIS} лежит \textit{многоагентный подход}, который позволяет легко интерпретировать любую информацию, то можно сделать вывод, что продукционный подход легко интегрируется в \textit{решатели задач} \textit{ostis-систем} в виде \textit{sc-агентов}.  Интеграция \textit{Технологии OSTIS} и продукционного подхода позволяет объединить статические и динамические знания в рамках единого формализма, а также получить их графическое представление.}

\bigskip
\end{scnsubstruct}
\scnendsegmentcomment{Предметная область и онтология sc-языков продукционного программирования}

\end{SCn}


\scsubsubsection{Пункт 30.6.1. Предметная область и онтология синтаксиса sc-языков продукционного программирования}
\label{sd_sc_product_program_lang_syntax}
\begin{SCn}
\scnsectionheader{Предметная область и онтология синтаксиса sc-языков продукционного программирования}
\begin{scnsubstruct}

\scnheader{Предметная область синтаксиса sc-языков продукционного программирования}
\scniselement{предметная область}
\begin{scnrelfromlist}{соавтор}
    \scnitem{Орлов М. К.}
    \scnitem{Зотов Н. В.}
\end{scnrelfromlist}

\begin{scnrelfromvector}{библиография}
    \scnitem{\scncite{AIHandbookMM}}
\end{scnrelfromvector}

\scnheader{продукция}
\scntext{общий вид}{($i$); $Q$; $P$; $A$ $\Rightarrow$ $B$; $N$}
\begin{scnindent}
    \scnhaselement{i}
    \begin{scnindent}
        \scnnote{Имя продукции, с помощью которого данная продукция выделяется из всего множества продукций. В качестве имени может выступать некоторая лексема, отражающая суть данной продукции (например, "покупка книги"{} или "набор кода замка"{}), или порядковый номер продукции в их множестве, хранящемся в памяти системы. }
    \end{scnindent}
    \scnhaselement{Q}
    \begin{scnindent}
        \scnnote{Характеризует сферу применения продукции или же контекст.}
    \end{scnindent}
    \scnhaselement{P}
    \begin{scnindent}
        \scnnote{Условие применимости ядра продукции. Обычно $P$ представляет собой логическое выражение (как правило, предикат). Когда $P$ принимает значение «истина», ядро продукции активизируется. Если $P$ ложно, то ядро продукции не может быть использовано.}
        \scntext{пример невозможности применения продукции}{Если в продукции «НАЛИЧИЕ ДЕНЕГ; ЕСЛИ ХОЧЕШЬ КУПИТЬ ВЕЩЬ X, ТО ЗАПЛАТИ В КАССУ ЕЕ СТОИМОСТЬ И ОТДАЙ ЧЕК ПРОДАВЦУ» условие применимости ядра продукции ложно, то есть денег нет, то применить ядро продукции невозможно.}
    \end{scnindent}
    \scntext{подстрока}{$A$ $\Rightarrow$ $B$}
    \begin{scnindent}
        \scnnote{Ядро продукции. Интерпретация ядра продукции может быть различной и зависит от того, что стоит слева и справа от знака секвенции $\Rightarrow$.}
        \scntext{прочтение ядра}{ЕСЛИ $A$, ТО $B$}
        \scntext{прочтение более сложной конструкции ядра}{ЕСЛИ $A$, ТО $B_1$ ИНАЧЕ $B_2$}
        \scnnote{Секвенция может истолковываться в обычном логическом смысле как знак логического следования $B$ из истинного $A$ (если $A$ не является истинным выражением, то о $B$ ничего сказать нельзя). Возможны и другие интерпретации ядра продукции, например $A$ описывает некоторое условие, необходимое для того, чтобы можно было совершить действие $B$.}
    \end{scnindent}
    \scnhaselement{N}
    \begin{scnindent}
        \scnnote{Постусловия продукции. Они актуализируются только в том случае, если ядро продукции реализовалось. Постусловия продукции описывают действия и процедуры, которые необходимо выполнить после реализации В.}
        \scntext{пример использования}{после покупки некоторой вещи в магазине необходимо в описи товаров, имеющихся в этом магазине, уменьшить количество вещей такого типа на единицу. Выполнение $N$ может происходить не сразу после реализации ядра продукции.}
        \scnrelfrom{библиографический источник}{\scncite{AIHandbookMM}}
    \end{scnindent}
\end{scnindent}
\scnnote{Если в памяти системы хранится некоторый набор продукций, то они образуют систему продукций. В системе продукций должны быть заданы специальные процедуры управления продукциями, с помощью которых происходит актуализация продукций и выбор для выполнения той или иной продукции из числа актуализированных.}

\scnheader{структурная единица продукционного языка программирования}
\scnrelto{содержит}{продукционный язык программирования}
\scnhaselementrole{основной элемент}{продукция}
\scnhaselement{определение глобальной переменной}
\scnhaselement{оператор управления макрогенерацией}
\scnhaselement{вставка текста на алгоритмическом языке программирования}

\scnheader{системы, комбинирующие сетевые и продукционные модели}
\scnrelfrom{модель представления декларативных знаний}{сетевая модель}
\scnrelfrom{модель представления процедурных знаний}{продукционная модель}
\scntext{способ работы}{работа продукционной системы над семантической сетью}
\scnnote{Процедурные знания позволяют системе узнать, как можно использовать те или иные декларативные знания, в частности, знания о закономерностях той части действительности, в которой "живет"{} интеллектуальная система, для получения нужных системе результатов или тех результатов, которые ожидает от нее пользователь.}

\bigskip
\end{scnsubstruct}
\scnendsegmentcomment{Предметная область и онтология синтаксиса sc-языков продукционного программирования}

\end{SCn}


\scsubsubsection{Пункт 30.6.2. Предметная область и онтология денотационной семантики sc-языков продукционного программирования}
\label{sd_sc_product_program_lang_denot_sem}
\begin{SCn}
\scnsectionheader{Предметная область и онтология денотационной семантики sc-языков продукционного программирования}
\begin{scnsubstruct}

\scnheader{Предметная область денотационной семантики sc-языков продукционного программирования}
\scniselement{предметная область}
\begin{scnrelfromlist}{соавтор}
    \scnitem{Орлов М. К.}
    \scnitem{Зотов Н. В.}
\end{scnrelfromlist}

\begin{scnrelfromvector}{библиография}
    \scnitem{\scncite{Brownston1985}}
    \scnitem{\scncite{CLIPS}}
\end{scnrelfromvector}

\scnheader{OPS5}
\scniselement{язык продукционного программирования}
\scnnote{База данных в языке называется рабочей памятью (working memory) и состоит из нескольких сотен объектов, каждый из которых имеет свой набор атрибутов. Объект вместе с парами <атрибут -- значение> называется элементом рабочей памяти.}
\scnrelfrom{библиографический источник}{\scncite{Brownston1985}}
\scnnote{Основная задача, которую поставили перед собой разработчики языка \textit{OPS5}, добиться максимально высокой эффективности выполнения продукционной программы. Интерпретатор системы порождает конфликтное множество, каждый элемент которого представляет собой пару <имя продукции, список элементов рабочей памяти, которые являются означиваниями для образцов продукции>. Каждая продукция в \textit{OPS5} состоит из символа Р, имени продукции, левой части, символа --> и правой части.}
\scnnote{В \textit{OPS5} выделены три типа действий:
 \begin{itemize}
    \item{MAKE --- создает новый элемент рабочей памяти;}
    \item{MODIFY --- изменяет один или несколько значений атрибутов у существующего элемента рабочей памяти;}
    \item{REMOVE --- удаляет элемент рабочей памяти.}
 \end{itemize}}
\scnrelfrom{изображение}{Рисунок. Пример правила продукции на OPS5}
\begin{scnindent}
    \scneqimage[30em]{Contents/part_ps/src/images/sd_sc_product_program_lang_denot_sem/ops5_production_rule_example.png}
    \scnnote{Данные в рабочей памяти структурированы и переменные появляются в угловых скобках. Название структуры данных, такое как "goal"{} (цель) и "physical-object"{} (физический объект), является первым буквальным в условиях; поля структуры начинаются с "\textasciicircum"{}. На негативное состояние  указывает "{}--"{}.}
    \scnnote{Если несколько объектов подвешены к потолку, каждый с другой лестницей рядом, поддерживающей обезьяну с пустыми руками, конфликтный набор будет содержать столько же продукций правил продукции, полученных из одной и той же продукции "Holds::Object-Ceiling"{}. На этапе разрешения конфликта позже будет выбрано, какие продукции запускать.}
    \scnnote{Когда обезьяна держит подвешенный объект, статус цели устанавливается на «удовлетворено», и то же производственное правило больше не может применяться, поскольку его первое условие не выполняется.}
\end{scnindent}
\scnnote{Продукционные правила в \textit{OPS5} применяются ко всем продукциям структур данных, которые соответствуют условиям и соответствуют привязкам переменных.}

\scnheader{CLIPS}
\scnexplanation{\textbf{\textit{CLIPS}} использует продукционную модель представления знаний и поэтому содержит три основных элемента:
\begin{itemize}
    \item{база фактов (fact base);}
    \item{база правил (rule base);}
    \item{механизм логического вывода.}
\end{itemize}
База фактов представляет исходное описание задачи. База правил содержит операторы, которые преобразуют состояния проблемы, приводя его к решению --- целевому состоянию.}
\scnnote{Механизм логического вывода \textit{CLIPS} сопоставляет факты из базы фактов и правила из базы правил и выясняет, какие из правил можно активизировать. Это выполняется циклически, причем каждый цикл (так называемый продукционный цикл или цикл распознавания действия) состоит из трех основных фаз:
\begin{itemize}
    \item{сопоставление фактов и правил;}
    \item{выбор правила, подлежащего активизации;}
    \item{выполнение действий, предписанных активным («зажженным») правилом.}
\end{itemize}}
\scnnote{Факты --- это одна из основных форм представления информации в системе \textit{CLIPS}. Каждый факт представляет фрагмент информации, который был помещен в текущий список фактов, называемый fact-list. Факт представляет собой основную единицу данных, используемую правилами.

Если при добавлении нового факта к списку обнаруживается, что он полностью совпадает с одним из уже включенных в список фактов, то эта операция игнорируется.

Факт может описываться индексом или адресом. Всякий раз, когда факт добавляется (изменяется), ему присваивается уникальный целочисленный индекс. Факт также может задаваться при помощи адреса.

Идентификатор факта --- это короткая запись для отображения факта на экране. Она состоит из символа f и записанного через тире индекса факта. Существует два формата представления фактов: позиционный и непозиционный. Позиционные факты состоят из выражения символьного типа, за которым следует последовательность (возможно, пустая) из полей, разделенных пробелами. Вся запись заключается в скобки. Обычно первое поле определяет "отношение"{}, которое применяется к оставшимся полям.}
\scnrelfrom{библиографический источник}{\scncite{CLIPS}}

\bigskip
\end{scnsubstruct}
\scnendsegmentcomment{Предметная область и онтология денотационной семантики sc-языков продукционного программирования}

\end{SCn}


\scsubsubsection{Пункт 30.6.3. Предметная область и онтология операционной семантики sc-языков продукционного программирования}
\label{sd_sc_product_program_lang_oper_sem}
\begin{SCn}
\scnsectionheader{Предметная область и онтология операционной семантики sc-языков продукционного программирования}
\begin{scnsubstruct}

\scnheader{Предметная область операционной семантики sc-языков продукционного программирования}
\scniselement{предметная область}
\begin{scnhaselementrolelist}{максимальный класс объектов исследования}
    \scnitem{язык продукционного программирования}
    \scnitem{продукция}
\end{scnhaselementrolelist}

\begin{scnrelfromlist}{соавтор}
    \scnitem{Орлов М.К.}
    \scnitem{Зотов Н.В.}
\end{scnrelfromlist}

\begin{scnrelfromvector}{библиография}
    \scnitem{\scncite{Forgy1982}}
    \scnitem{\scncite{Vladimirov2010}}
\end{scnrelfromvector}

\scnheader{Алгоритм Rete}
\scnexplanation{\textbf{\textit{Алгоритм Rete}} содержит обобщение логики функционала, ответственного за связь данных (фактов) и алгоритма (продукций) в системах сопоставления с образцом (вид систем: системы основанные на правилах). Продукция состоит из одного или нескольких условий и набора действий, выполняемых если актуальный набор фактов соответствует одному из условий. Условия накладываются на атрибуты фактов, включая их типы и идентификаторы.}
\begin{scnrelfromset}{преимущества}
    \scnfileitem{\textit{Алгоритм Rete} уменьшает или исключает избыточность условий за счет объединения узлов.}
    \scnfileitem{\textit{Алгоритм Rete} сохраняет частичные соответствия между фактами при слиянии разных типов фактов. Это позволяет избежать полного вычисления всех фактов при любом изменении в рабочей памяти продукционной системы. Система работает только с самими изменениями.}
    \scnfileitem{\textit{Алгоритм Rete} позволяет эффективно высвобождать память при удалении фактов.}
\end{scnrelfromset}
\scnnote{\textit{Алгоритм Rete} широко используется для реализации сопоставления с образцом в системах с циклом сопоставление-решение-действие для генерации и логического вывода.}
\scnrelfrom{смотрите}{\cite{Forgy1982}}

\scnheader{Алгоритм Rete II}
\begin{scnrelfromset}{преимущества}
    \scnfileitem{Повышена общая производительность сети включая хешированную память для больших массивов данных.}
    \scnfileitem{Добавлен алгоритм обратного вывода, работающий на той же сети.}
    \scnfileitem{Скорость обратного вывода по сравнению с \textit{Rete I} повышена значительно.}
\end{scnrelfromset}

\scnheader{Алгоритм Rete-NT}
\begin{scnindent}
    \begin{scnrelfromset}{преимущества}
        \scnfileitem{Алгоритм был признан в 10 раз более быстрым, чем его предшественник \textit{Алгоритм Rete II}}
    \end{scnrelfromset}
\end{scnindent}

\scnheader{Миварный подход}
\scnnote{\textbf{\textit{Миварный подход}} объединяет и другие научные области компьютерных наук, информатики и дискретной математики, включая: базы данных, экспертные системы, системы логического вывода на основе развития продукций, теорию графов, матрицы, параллельное выполнение программ на кластерах, проектирование новых архитектур компьютеров, массовое суммирование чисел, техническую защиту информации и информационную безопасность, гносеологию (частично и в плане создания новой наиболее мощной модели данных на основе "тройки"{} "вещь-свойство-отношение"{}), сервисно-ориентированные архитектуры, компьютерные сети, информационные инфраструктуры, теоретическую робототехнику, многоагентные системы и некоторые другие.}
\scnrelfrom{смотрите}{\cite{Vladimirov2010}}
\scnnote{\textit{Миварный подход} объединяет две основные технологии накопления данных и обработки информации:
\begin{itemize}
    \item{миварное информационное пространство: накопление данных на основе эволюционной самоорганизующейся миварной модели данных с изменяющейся структурой в теории баз данных,}
    \item{миварные сети: обработка информации на основе развития продукционного подхода к логическому выводу с учетом включения возможности автоматического конструирования алгоритмов для "решателей задач"{} и традиционной вычислительной обработки, а также с использованием идей отношений, правил и процедур, которые теперь принято относить к сервисно-ориентированным архитектурам и многоагентным системам.}
\end{itemize}}
\scnnote{Суть \textit{миварного подхода} в объединении баз данных и систем логико-вычислительной обработки в единые эволюционно развивающиеся системы, позволяющие собрать воедино все различные научные разработки на основе сервисно-ориентированных архитектур и технологий интеллектуальных агентов --- многоагентных систем.}
\scnexplanation{Миварные сети основаны на продукционном подходе "если, то…"{} с переходом к более сложной структуре правил с предусловиями, условиями, ограничениями, действиями и последействиями. Это позволяет записывать все причинно-следственные отношения, включая и все возможные формы предикатов и подобных логических выражений. Значение предикатов и поиска истинных выражений не отрицается, а только создается возможность и для их реализации, и для реализации всех возможных других представлений правил в виде: сервисов, процедур, продукций, подпрограмм и так далее. Такой подход позволяет работать одновременно с разными описаниями предметных областей, прибавляя к предикатам и продукции, и нейросети, и генетические алгоритмы, и традиционные вычислительные процедуры, и все другие в виде универсальных миварных отношений, которые представляются и хранятся перед обработкой в миварном пространстве.}

\bigskip
\end{scnsubstruct}
\scnendsegmentcomment{Предметная область и онтология операционной семантики sc-языков продукционного программирования}

\end{SCn}


\scsubsection{\S 30.7. Предметная область и онтология sc-моделей искусственных нейронных сетей}
\label{sd_ann}
\begin{SCn}
\scnsectionheader{Предметная область и онтология sc-моделей искусственных нейронных сетей}
\begin{scnsubstruct}

\begin{scnrelfromlist}{соавтор}
	\scnitem{Головко В.А.}
	\scnitem{Ковалёв М.В.}
	\scnitem{Крощенко А.А.}
	\scnitem{Михно Е.В.}
\end{scnrelfromlist}

\begin{scnreltovector}{конкатенация сегментов}
	\scnitem{Предметная область и онтология искусственных нейронных сетей}
	\scnitem{Предметная область и онтология действий по обработке искусственных нейронных сетей}
\end{scnreltovector}

\scntext{аннотация}{Рассмотрен подход к интеграции и конвергенции искусственных нейронных сетей с базами знаний	в интеллектуальных компьютерных системах нового поколения с помощью представления и интерпретации искусственных нейронных сетей в базе знаний. Описаны \textit{Синтаксис, Денотационная и Операционная семантика Языка представления нейросетевых методов в базах знаний}. Описаны этапы построения нейросетевых методов решения задач с помощью интеллектуальной среды проектирования искусственных нейронных сетей.}

\begin{scnrelfromlist}{ключевой знак}
	\scnitem{Язык представления нейросетевых методов решения задач в базах знаний}
\end{scnrelfromlist}

\begin{scnhaselementrolelist}{класс объектов исследования}
	\scnitem{нейросетевой метод решения задач}
	\scnitem{нейросетевая модель решения задач}
	\scnitem{навык решения задач с помощью искусственных нейронных сетей}
	\scnitem{действие по построению искусственных нейронных сетей}
\end{scnhaselementrolelist}

\begin{scnrelfromset}{библиографическая ссылка}
	\scnitem{Castelvecchi2016}
	\scnitem{Ribeiro2016}
	\scnitem{Lundberg2017}
	\scnitem{Garcez2015}
	\scnitem{Besold2017}
	\scnitem{Golovko2019}
	\scnitem{Kroshchanka2022}
	\scnitem{Golovko2017}
	\scnitem{Kovalev2022}
	\scnitem{Glorot2010}
	\scnitem{He2015}
	\scnitem{Goodfellow2017}
	\scnitem{Haykin2006}
	\scnitem{Duchi2011}
	\scnitem{Kingma2014}
\end{scnrelfromset}

\begin{scnrelfromvector}{введение}
	\scnfileitem{В последнее десятилетие обозначилась устойчивая тенденция широкого применения методов машинного обучения в самых разных областях человеческой деятельности, обусловленная в первую очередь развитием теории искусственных нейронных сетей (и. н. с.), а также аппаратных возможностей.}
	\scnfileitem{Современные решатели задач интеллектуальных систем все чаще сталкиваются с необходимостью решения комплексных задач с помощью различных традиционных и интеллектуальных методов решения задач в едином информационном ресурсе (в пределе --- в единой базе знаний).}
	\scnfileitem{С другой стороны, интеллектуальные компьютерные системы нового поколения обладают, среди прочих, следующими способностями:
		\begin{itemize}
			\item способность постоянно повышать качество решения задач;
			\item способность приобретать навыки решения принципиально новых задач;
			\item способность обосновывать свои решения;
			\item способность находить и устранять ошибки в своих решения (способность к интроспекции).
		\end{itemize}}
	\scnfileitem{Представление различных методов решения задач в единой базе знаний обеспечивает семантическую совместимость этих методов. Решая задачу с помощью таких методов, система не взаимодействует с ними по принципу \scnqq{входов-выходов}. Напротив, единая память позволяет отслеживать преобразование входных знаний в реальном времени с помощью любых имеющихся методов, что обеспечивает способность к интроспекции и способность объяснять решения системы.}
	\scnfileitem{Перспективными и активно развивающимися методами решения задач являются искусственные нейронные сети (и.н.с.), что обуславливается, с одной стороны, развитием теории и.н.с., а с другой --- аппаратных возможностей машин, которые используются для их обучения.}
	\scnfileitem{Преимущество и.н.с. заключается в том, что они могут работать с неструктурированными данными.}
	\scnfileitem{Достоинствами и.н.с. можно назвать способность решения задач при неизвестных закономерностях, а так же способность решения задач без необходимости разработки проблемоориентированных подходов.}
	\scnfileitem{Главный недостаток и.н.с. --- это отсутствие понятной человеку обратной связи, которую можно было бы назвать цепочкой рассуждений, т.е. можно сказать, что и.н.с. работают как \textit{черный ящик}.
		\\Еще одним недостатком и.н.с. можно назвать эвристический характер процесса подбора архитектур моделей и 		параметров их обучения и высокие требования к объему знаний проектировщиков нейросетевых моделей.}
	\begin{scnindent}
		\begin{scnrelfromset}{источник}
			\scnitem{\scncite{gastelvecchi2016}}
		\end{scnrelfromset}
	\end{scnindent}
	\scnfileitem{Современные задачи все чаще требуют обоснования своего решения. Появилось целое направление Explainable AI, в рамках которого предпринимаются различные попытки объяснить решения и.н.с. Развиваются подходы, предлагающие интеграцию нейронных сетей с базами знаний.}
	\begin{scnindent}
		\begin{scnrelfromset}{источник}
			\scnitem{\scncite{Lundberg2017}}
			\scnitem{\scncite{Ribeiro2016}}
			\scnitem{\scncite{Lundberg2017}}
			\scnitem{\scncite{Garcez2015}}
			\scnitem{\scncite{Besold2017}}
			\scnitem{\scncite{Golovko2019}}
			\scnitem{\scncite{Kroshchanka2022}}
		\end{scnrelfromset}
	\end{scnindent}
	\scnfileitem{Сложность современных интеллектуальных систем, использующих нейросетевые модели, а также большой объём обрабатываемых ими данных обуславливают необходимость мониторинга, объяснения и понимания механизмов их работы с целью вербализации оценки и оптимизации их деятельности.}
	\scnfileitem{Исходя из перечисленных способностей, наличие которых необходимо обеспечивать в интеллектуальных компьютерных системах нового поколения, встает проблема разработки подхода к интеграции и.н.с. в базу знаний интеллектуальной системы как в качестве метода решения задач, так и в качестве объекта автоматического проектирования новых методов. Решение этой проблемы позволит преодолеть указанные выше недостатки нейросетевого	метода.}
	\scnfileitem{В связи с этим становится актуальна разработка нейросимволических подходов, в частности, подходов по интеграции и.н.с и баз знаний, использующих онтологии. Такие интегрированные системы способны сочетать:
		\begin{itemize}
			\item возможность семантической интерпретации обрабатываемых данных, используя представление решаемых и.н.с. прикладных задач, а так же спецификацию её входных и выходных данных;
			\item с представлением самой структуры и.н.с., описанием её свойств и состояний, позволяющими упростить понимание её работы.
	\end{itemize}}
	\begin{scnindent}
		\begin{scnrelfromset}{источник}
			\scnitem{\scncite{ann_ostis2018}}
			\scnitem{\scncite{nesy1}}
		\end{scnrelfromset}
	\end{scnindent}
	\scnfileitem{Можно выделить два основных направления интеграции и.н.с. с базами знаний:
		\begin{itemize}
			\item Построение интеллектуальных систем, способных использовать нейросетевые методы наравне с другими имеющимися в системе методами для решения задач или подзадач системы. Такие системы смогут учитывать семантику решаемых задач на более высоком уровне, что сделает решение этих задач более структурированными и прозрачными.
			\item Построение интеллектуальной среды по разработке, обучению и интеграции различных и.н.с., совместимых с базами знаний через представление и.н.с. с помощью онтологических структур и их интерпретацию средствами представления знаний. Такая среда предоставит возможность интроспекции и.н.с, возможность сохранения состояний и.н.с. после обучения и реконфигурации сети. Это позволит производить более глубокий анализ работы и.н.с. Так же формальное описание знаний рамках предметной области и.н.с. поможет поможет уменьшить порог вхождения разработчиков в методы решения задач с помощью и.н.с.\\
			Данный раздел посвящен предметной области и онтологии искусственных нейронных сетей и предметной области и онтологии действий по обработке искусственных нейронных сетей, которые являются основой развития обоих указанных направлений.
        \end{itemize}}
\end{scnrelfromvector}
\scniselement{раздел базы знаний}
\scnhaselementrole{ключевой sc-элемент}{Предметная область и онтология искусственных нейронных сетей}

\scnsegmentheader{Предметная область и онтология искусственных нейронных сетей}
\begin{scnsubstruct}

\scnheader{Предметная область искусственных нейронных сетей}
\scnidtf{Предметная область и.н.с.}
\scniselement{предметная область}

\begin{scnrelfromlist}{ключевой знак}
	\scnitem{Язык представления нейросетевого метода решения задач в базах знаний}
	\scnitem{Денотационная семантика Языка представления нейросетевого метода решения задач в базах знаний}
	\scnitem{Операционная семантика Языка представления нейросетевого метода в базах знаний}
\end{scnrelfromlist}

\begin{scnhaselementrolelist}{максимальный класс объектов исследования}
	\scnitem{искусственная нейронная сеть}
\end{scnhaselementrolelist}

\begin{scnhaselementrolelist}{класс объектов исследования}
    \scnitem{нейросетевой метод решения задач}
	\scnitem{формальный нейрон}
	\scnitem{навык решения задач с помощью искусственных нейронных сетей}
	\scnitem{синаптическая связь}
	\scnitem{слой и.н.с.}
	\scnitem{взвешенная сумма нейрона}
    \scnitem{искусственная нейронная сеть}
    \scnitem{искусственная нейронная сеть с прямыми связями}
    \scnitem{персептрон}
    \scnitem{персептрон Розенблатта}
    \scnitem{автоэнкодерная искусственная нейронная сеть}
    \scnitem{машина опорных векторов}
    \scnitem{искусственная нейронная сеть радиально-базисных функций}
    \scnitem{искусственная нейронная сеть с обратными связями}
    \scnitem{нейронная сеть Хопфилда}
    \scnitem{нейронная сеть Хэмминга}
    \scnitem{рекуррентная искусственная нейронная сеть}
    \scnitem{искусственная нейронная сеть Джордана}
    \scnitem{искусственная нейронная сеть Элмана}
    \scnitem{мультирекуррентная нейронная сеть}
    \scnitem{LSTM-элемент}
    \scnitem{GRU-элемент}
    \scnitem{полносвязная искусственная нейронная сеть}
    \scnitem{слабосвязная искусственная нейронная сеть}
    \scnitem{формальный нейрон}
    \scnitem{полносвязный формальный нейрон}
    \scnitem{сверточный формальный нейрон}
    \scnitem{рекуррентный формальный нейрон}
    \scnitem{синаптическая связь}
    \scnitem{параметр нейронной сети}
    \scnitem{настраиваемый параметр нейронной сети}
    \scnitem{весовой коэффициент}
    \scnitem{пороговое значение}
    \scnitem{ядро свертки}
    \scnitem{архитектурный параметр нейронной сети}
    \scnitem{количество слоев}
    \scnitem{количество формальных нейронов}
    \scnitem{количество синаптических связей}
    \scnitem{паттерн входной активности н.с.}
    \scnitem{признак}
    \scnitem{полносвязный слой и.н.с}
    \scnitem{сверточный слой и.н.с}
    \scnitem{слой и.н.с. нелинейного преобразования}
    \scnitem{dropout слой и.н.с.}
    \scnitem{pooling слой и.н.с}
\end{scnhaselementrolelist}
\begin{scnhaselementrolelist}{исследуемое отношение}
    \scnitem{формальный нейрон\scnrolesign}
    \scnitem{пороговый формальный нейрон\scnrolesign}
    \scnitem{синаптическая связь\scnrolesign}
    \scnitem{входное значение формального нейрона*}
    \scnitem{выходное значение формального нейрона*}
    \scnitem{функция активации*}
    \scnitem{взвешенная сумма*}
    \scnitem{распределяющий слой*}
    \scnitem{обрабатывающий слой*}
    \scnitem{выходной слой*}
\end{scnhaselementrolelist}

\begin{scnrelfromlist}{частная предметная область}
	\scnitem{Предметная область ИНС с заданным направлением связей}
	\begin{scnindent}
		\begin{scnrelfromlist}{частная предметная область}
			\scnitem{Предметная область ИНС с прямым связями}
			\begin{scnindent}
				\begin{scnrelfromlist}{частная предметная область}
					\scnitem{Предметная область персептронов}
					\begin{scnindent}
						\begin{scnrelfromlist}{частная предметная область}
							\scnitem{Предметная область персептронов Розенблатта}
							\scnitem{Предметная область персептронов Румельхарта}
							\scnitem{Предметная область автоэнкодерных ИНС}
						\end{scnrelfromlist}
					\end{scnindent}
					\scnitem{Предметная область ИНС радиально-базисных функций}
					\scnitem{Предметная область машин опорных векторов}
				\end{scnrelfromlist}
			\end{scnindent}
			\scnitem{Предметная область ИНС с обратными связями}
			\begin{scnindent}
				\scnidtf{Предметная область рекуррентных ИНС}
				\begin{scnrelfromlist}{частная предметная область}
					\scnitem{Предметная область ИНС Джордана}
					\scnitem{Предметная область ИНС Элмана}
					\scnitem{Предметная область LSTM-элементов}
					\scnitem{Предметная область GRU-элементов}
				\end{scnrelfromlist}
			\end{scnindent}
		\end{scnrelfromlist}
	\end{scnindent}
	\scnitem{Предметная область обучения ИНС}
	\begin{scnindent}
		\begin{scnrelfromlist}{частная предметная область}
			\scnitem{Предметная область ИНС, обучающихся с учителем}
			\scnitem{Предметная область ИНС, обучающихся без учителя}
			\begin{scnindent}
				\begin{scnrelfromlist}{частная предметная область}
					\scnitem{Предметная область обучающихся автоэнкодерных ИНС}
					\scnitem{Предметная область ИНС глубокого доверия}
					\scnitem{Предметная область генеративно-состязательных ИНС}
					\scnitem{Предметная область самоорганизующихся карт Кохонена}
					\scnitem{Предметная область ИНС Хопфилда}
					\scnitem{Предметная область подкрепляющего обучения ИНС}
				\end{scnrelfromlist}
			\end{scnindent}
		\end{scnrelfromlist}
	\end{scnindent}
	\scnitem{Предметная область топологий ИНC}
	\begin{scnindent}
		\begin{scnrelfromlist}{частная предметная область}
			\scnitem{Предметная область полносвязных ИНC}
			\scnitem{Предметная область многослойных ИНC}
			\scnitem{Предметная область слабосвязных ИНC}
		\end{scnrelfromlist}
	\end{scnindent}
	\scnitem{Предметная область задач, решаемых с помощью ИНС}
	\begin{scnindent}
		\begin{scnrelfromlist}{частная предметная область}
			\scnitem{Предметная область ИНС, решающих задачу классификации}
			\scnitem{Предметная область ИНС, решающих задачу аппроксимации}
			\scnitem{Предметная область ИНС, решающих задачу управления}
			\scnitem{Предметная область ИНС, решающих задачу фильтрации}
			\scnitem{Предметная область ИНС, решающих задачу детекции}
			\scnitem{Предметная область ИНС, решающих задачу с ассоциативной памятью}
		\end{scnrelfromlist}
	\end{scnindent}
	\scnitem{Предметная область интеграции ИНС с базой знаний}
\end{scnrelfromlist}

\scntext{введение}{Решатель задач занимается обработкой фрагментов базы знаний. На операционном уровне обработка сводится к добавлению, поиску, редактированию и удалению sc-узлов и sc-коннекторов базы знаний. На семантическом же уровне такая операция является действием, выполняемым в памяти субъекта действия, где, в общем случае, субъектом является ostis-система, а база знаний --- её памятью. действие определяется как процесс воздействия	одной сущности (или некоторого множества сущностей) на другую сущность (или на некоторое множество других сущностей) в соответствии с некоторой целью.}
\begin{scnindent}
	\begin{scnrelfromset}{смотрите}
		\scnitem{Предметная область и онтология действий, задач, планов, протоколов и методов, реализуемых ostis-системой, а также внутренних агентов, выполняющих эти действия}
	\end{scnrelfromset}
\end{scnindent}

\scnheader{искусственная нейронная сеть}
    \scnsubset{метод}
    \scntext{примечание}{Предлагается рассматривать и.н.с. как класс методов решения задач со своим языком представления.
    	\\Таким образом, искусственная нейронная сеть --- это нейросетевой метод решения задач. В соответствии с Технологией OSTIS,	спецификация класса методов решения задач сводится к спецификации соответствующего языка представления	методов, то есть к описанию его синтаксической, денотационной и операционной семантики.}
    \scntext{пояснение}{\textbf{\textit{искусственная нейронная сеть}} --- это биологически инспирированная математическая модель, обладающая обобщающей способностью после выполнения процедуры обучения. Под обобщающей способностью понимается способность модели выдавать корректные результаты для паттернов входной активности, не входящих в обучающую выборку.}
    \scnsubset{математическая модель}
    \begin{scnindent}
        \scntext{пояснение}{\textbf{\textit{математическая модель}} --- это упрощенное описание объекта реального мира, выраженное с помощью математической символики}
    \end{scnindent}
    \scnrelfrom{описание примера}{\scnfileimage[30em]{Contents/part_ps/src/images/sd_ps/sd_ann/neural_network_scg.png}}
    \begin{scnindent}
    	\scntext{примечание}{Пример формализации полносвязной двухслойной и.н.с. с двумя нейронами на входном слое и одном нейроне на обрабатывающем слоев.
    	    \\Следует отметить, что в практике авторов еще не было необходимости явно представлять и.н.с., как это показано	на рисунке. Чаще всего, представление и.н.с. сводилось к представлению ее операционной семантики в виде SCP-программы.}
    \end{scnindent}
    \scnrelfrom{решаемые задачи}{задачи, которые могут быть решены с помощью и.н.с. с приемлемой точностью}
        \begin{scnindent}
            \begin{scneqtoset}
                \scnitem{задача классификации}
                \begin{scnindent}
                    \scnsubset{задача}
                    \scntext{пояснение}{Задача построения классификатора, т.е. отображения $\tilde c: X \rightarrow C$, где $ X \in \mathbb{R}\upperscore{m}$ ---
                    признаковое пространство п.в.а., $C \eq \scnleftcurlbrace C\underscore{1}, C\underscore{2}, ...,C\underscore{k} \scnrightcurlbrace$ --- конечное и обычно небольшое множество меток классов.}
                \end{scnindent}
                \scnitem{задача регрессии}
                \begin{scnindent}
                    \scnsubset{задача}
                    \scntext{пояснение}{Задача построения оценочной функции по примерам $(x\underscore{i}, f(x\underscore{i}))$, где $f(x)$ --- неизвестная функция}
                    \scntext{пояснение}{\textbf{\textit{оценочная функция*}} --- отображение вида $\tilde{f}: X \rightarrow \mathbb{R}$, где $X \in \mathbb{R}\upperscore{m}$ --- признаковое пространство п.в.а.}
                \end{scnindent}
                \scnitem{задача кластеризации}
                \begin{scnindent}
                    \scnsubset{задача}
                    \scntext{пояснение}{Задача разбиения множества п.в.а. на группы (кластеры) по какой-либо метрике сходства.}
                    \scntext{определение}{задача кластеризации --- это задача построения функции $a : X \to Y$ , которая любому объекту $x \in X$ ставит в соответствие номер кластера $y \in Y$ в соответствие с определенной метрикой расстояния $\rho(x, x' )$, где $X$ --- множество объектов, $Y$ --- множество номеров (имен, меток) кластеров, $x$, $x' \in X$.}
                \end{scnindent}
                \scnitem{задача понижения размерности}
                \begin{scnindent}
                    \scnsubset{задача}
                    \scnidtf{задача уменьшения размерности признакового пространства}
                    \scntext{определение}{задача понижения размерности --- это задача построения функции $h : X \to Y$ , сохраняющей заданные соотношения между точками множеств $X$ и $Y$, где $X \subset \mathbb{R}\upperscore{p}$ , $Y \eq h(X) \subset \mathbb{R}\upperscore{q}$ , $q < p$.}
                \end{scnindent}
                \scnitem{задача управления}
                \begin{scnindent}
                    \scnsubset{задача}
                    \scntext{определение}{задача управления --- это задача построения модели-регулятора состояния сложного динамического объекта.}
                \end{scnindent}
                \scnitem{задача фильтрации}
                \begin{scnindent}
                    \scnsubset{задача}
                    \scntext{определение}{задача фильтрации --- это задача построения модели, которая производит очистку исходного сигнала, содержащего некоторый шум и уменьшает влияние случайных ошибок в сигнале.}
                \end{scnindent}
                \scnitem{задача детекции}
                \begin{scnindent}
                    \scnsubset{задача}
                    \scnsubset{задача классификации}
                    \scnsubset{задача регрессии}
                    \scntext{определение}{задача детекции --- это задача, которая является частным случаем задачи классификации и задачи регрессии. Задача построения модели, осуществляющей обнаружение объектов определенных типов на фото- и видеоизображениях.}
                \end{scnindent}
                \scnitem{задача с ассоциативной памятью}
                \begin{scnindent}
                    \scnsubset{задача}
                    \scntext{определение}{задача с ассоциативной памятью --- это задача построения модели, позволяющей выполнить реконструкцию исходного образа на основании сохраненных ранее образов.}
                \end{scnindent}
            \end{scneqtoset}
        \end{scnindent}
\scntext{примечане}{Схожие задачи объединены в классы, для которых заданы обобщенные формулировки задач.}
\scntext{примечание}{Для достижения семантической совместимости с другими методами решения задач Технологии OSTIS, предлагается описывать нейросетевые методы внутри семантической памяти, соответственно, Синтаксис Языка представления нейросетевых методов решения задач в базах знаний является Синтаксисом SC-кода, использующимся в Технологии OSTIS для представления знаний.}
\scntext{примечание}{Таким образом, чтобы добавить в арсенал Технологии OSTIS нейросетевые методы решения задач и тем самым расширить круг задач, решаемых ostis-системами, необходимо описать денотационную и операционную семантику Языка представления нейросетевого метода решения задач в базах знаний.}

\scnheader{навык}
\scntext{примечание}{Понятие навыка описывает метод, интерпретация которого полностью может быть осуществлена данной кибернетической системой, в памяти которой хранится указанный метод. Таким образом, формируя спецификацию в	ostis-системе для нейросетевого метода решения задач и нейросетевой модели решения задач можно говорить о 	наличии у такой системы навыка решения задач с помощью и.н.с..}

\scnheader{SCg-текст. Фрагмент теоретико-множественной онтологии и.н.с}
\scnrelfrom{изображение}{\scnfileimage[30em]{Contents/part_ps/src/images/sd_ps/sd_ann/ontology_fragment.png}}
\begin{scnindent}
\scniselement{sc.g-текст}
\scntext{примечание}{Для примера, взяты класс задач классификации и конкретная обученная сверточная и.н.с.}
\scntext{примечание}{Фрагмент теоретико-множественной онтологии и.н.с., описывающий связь таких понятий и узлов, как:
	\begin{itemize}
		\item класс задач, решаемых с помощью и.н.с.
		\item класс нейросетевых методов решения задач
		\item нейросетевая модель решения задач
		\item навык решения задач с помощью и.н.с.
		\item конкретные задачи и методы их решения
		\end{itemize}}
\end{scnindent}

\scnheader{формальный нейрон\scnrolesign}
    \scnidtf{формальный нейронный элемент\scnrolesign}
    \scnidtf{нейронный элемент\scnrolesign}
    \scnidtf{нейрон\scnrolesign}
    \scniselement{ролевое отношение}
    \scnrelfrom{первый домен}{искусственная нейронная сеть}
    \scnrelfrom{второй домен}{формальный нейрон}
    \scnrelfrom{область определения}{искусственная нейронная сеть}
    \scntext{пояснение}{\textbf{\textit{формальный нейрон\scnrolesign}} --- ролевое отношение, связывающее искусственную нейронную сеть с ее нейроном.}

\scnheader{пороговый формальный нейрон\scnrolesign}
    \scnidtf{пороговый нейронный элемент\scnrolesign}
    \scnidtf{пороговый нейрон\scnrolesign}
    \scniselement{ролевое отношение}
    \scnrelfrom{первый домен}{искусственная нейронная сеть}
    \scnrelfrom{второй домен}{формальный нейрон}
    \scnrelfrom{область определения}{искусственная нейронная сеть}
    \scntext{пояснение}{\textbf{\textit{пороговый формальный нейрон\scnrolesign}} --- ролевое отношение, связывающее искусственную нейронную сеть с таким ее нейроном, выходное значение которого всегда равно -1.}
    \scntext{пояснение}{Весовой коэффициент синаптической связи, выходящей из такого нейрона, является порогом для нейрона, в который данная синаптическая связь входит.}

\scnheader{синаптическая связь\scnrolesign}
    \scnidtf{синапс\scnrolesign}
    \scniselement{ролевое отношение}
    \scnrelfrom{первый домен}{искусственная нейронная сеть}
    \scnrelfrom{второй домен}{синаптическая связь}
	\scnrelfrom{область определения}{\scnnonamednode}
	\begin{scnindent}
		\begin{scnreltoset}{объединение}
			\scnitem{искусственная нейронная сеть}
			\scnitem{синапс}
		\end{scnreltoset}
	\end{scnindent}
   \scndefinition{\textbf{\textit{синаптическая связь\scnrolesign}} --- ролевое отношение, связывающее искусственную нейронную сеть с ее синапсом.}

\scnheader{весовой коэффициент синаптической связи}
    \scnidtf{вес синапса}
    \scnidtf{сила синаптической связи}
    \scnsubset{настраиваемый параметр}
    \scntext{пояснение}{\textbf{\textit{весовой коэффициент синаптической связи}} --- это числовой коэффициент, который ставится в соответствие каждому синапсу нейронной сети и изменяется в процессе обучения.}
    \scntext{примечание}{Если сила синаптической связи отрицательна, то она называется \textit{тормозящей}. В противном случае она является \textit{усиливающей}.}

\scnheader{входное значение формального нейрона*}
    \scnidtf{входное значение нейрона*}
    \scnidtf{входное значение*}
    \scniselement{неролевое отношение}
    \scniselement{бинарное отношение}
    \scnrelfrom{первый домен}{формальный нейрон}
    \scnrelfrom{второй домен}{число}
    \scnrelfrom{область определения}{\scnnonamednode}
    \begin{scnindent}
    	\begin{scnreltoset}{объединение}
    		\scnitem{формальный нейрон}
    		\scnitem{число}
    	\end{scnreltoset}
    \end{scnindent}
    \scntext{пояснение}{\textbf{\textit{входное значение формального нейрона*}} --- неролевое отношение, связывающее нейрон входного слоя со значением признака п.в.а., который подается на вход нейронной сети.}
    \scntext{теоретическая неточность}{Использование множества как формы представления входных данных является серьезным допущением, так как на практике входные данные структурированы более сложно --- в многомерные массивы. Самым близким теоретическим аналогом здесь выступает тензор. К сожалению, описание теории нейронных сетей с помощью тензорного исчисления в литературе как таковое отсутствует, но активно используется на практике: например, во многих разрабатываемых нейросетевых фреймворках. Формализация нейронных сетей с помощью тензоров видится авторам наиболее вероятным направлением работы в ближайших изданиях \textit{стандарта OSTIS}.}

\scnheader{паттерн входной активности и.н.с.}
	\scnidtf{п.в.а.}
    \scniselement{мультимножество}
    \scniselement{кортеж}
    \scntext{пояснение}{\textbf{\textit{паттерн входной активности и.н.с.}} --- ориентированное мультимножество численных значений признаков некоторого объекта, которые могут выступать в качестве входных значений нейронов.}
	\scntext{примечание}{В текущей версии \textit{Стандарта OSTIS} предполагается, что п.в.а. содержит только предобработанные данные, то есть данные приведенные к численному виду и, возможно, преобразованные с помощью известных статистических методов (например, нормирования).}

\scnheader{признак}
    \scnidtf{feature}
    \scnidtf{множество признаков}
    \scnsubset{ролевое отношение}
    \scntext{пояснение}{\textbf{\textit{признак}} --- множество ролевых отношений, каждое из которых связывает некоторый п.в.а. с численным значением, которое характеризует данный п.в.а. с какой-либо стороны.}

\scnheader{взвешенная сумма*}
    \scnidtf{взвешенная сумма входных значений*}
    \scnidtf{в.с.}
    \scniselement{неролевое отношение}
    \scniselement{бинарное отношение}
    \scntext{пояснение}{\textbf{\textit{взвешенная сумма*}} --- неролевое отношение, связывающее формальный нейрон с числом, являющимся суммой произведений входных сигналов на весовые коэффициенты входящих в нейрон синаптических связей.}
     \scnrelfrom{область определения}{\scnnonamednode}
    \begin{scnindent}
    	\begin{scnreltoset}{объединение}
    		\scnitem{формальный нейрон}
    		\scnitem{число}
    	\end{scnreltoset}
    \end{scnindent}
    \scnrelfrom{первый домен}{формальный нейрон}
    \scnrelfrom{второй домен}{число}
    \scnrelfrom{формула}{
        \begin{equation*}
            S \eq \sum\underscore{i \eq 1}\upperscore{\scnleftcurlbrace n\scnrightcurlbrace} w\underscore{i} x\underscore{i} - T
        \end{equation*}}
    \begin{scnindent}
        \scntext{примечание}{\textit{n} --- размерность вектора входных значений, $w\underscore{i}$ --- \textit{i}-тый элемент вектора весовых коэффициентов, $x\underscore{i}$ --- \textit{i}-тый элемент вектора входных значений, \textit{T} --- пороговое значение.}
    \end{scnindent}

\scnheader{выходное значение формального нейрона*}
    \scnidtf{выходное значение нейрона*}
    \scnidtf{выходное значение*}
    \scniselement{неролевое отношение}
    \scniselement{бинарное отношение}
    \scnrelfrom{первый домен}{формальный нейрон}
    \scnrelfrom{второй домен}{число}
    \scnrelfrom{область определения}{\scnnonamednode}
    \begin{scnindent}
    	\begin{scnreltoset}{объединение}
    		\scnitem{формальный нейрон}
    		\scnitem{число}
    	\end{scnreltoset}
    \end{scnindent}
    \scntext{пояснение}{\textbf{\textit{входное значение*}} --- неролевое отношение, связывающее нейрон с числом, являющимся результатом применения функции активации нейрона к его взвешенной сумме.}
    \scntext{примечание}{Выходное значение нейрона является одним из входных сигналов для всех нейронов, в которые ведут выходящие из данного нейрона синапсы.}

\scnheader{слой и.н.с.}
    \scntext{примечание}{функция активации слоя является функцией активации всех формальных нейронов этого слоя}
    \scntext{примечание}{конфигурация слоя задается типом, количеством формальных нейронов, функцией активации}
    \scntext{примечание}{описание последовательности слоев и.н.с. с конфигурацией каждого слоя задает архитектуру и.н.с.}

\scnheader{распределяющий слой*}
    \scnidtf{входной слой*}
    \scniselement{неролевое отношение}
    \scniselement{бинарное отношение}
    \scndefinition{\textbf{\textit{распределяющий слой*}} --- неролевое отношение, связывающее искусственную нейронную сеть с ее слоем, нейроны которого принимают входные значения всей нейронной сети.}
    \scnrelfrom{область определения}{искусственная нейронная сеть}
    \scnrelfrom{первый домен}{искусственная нейронная сеть}
    \scnrelfrom{второй домен}{слой и.н.с.}

\scnheader{обрабатывающий слой*}
    \scniselement{неролевое отношение}
    \scniselement{бинарное отношение}
    \scndefinition{\textbf{\textit{обрабатывающий слой*}} --- неролевое отношение, связывающее искусственную нейронную сеть с ее слоем, нейроны которого принимают на вход выходные значения нейронов предыдущего слоя.}
    \scnrelfrom{область определения}{искусственная нейронная сеть}
    \scnrelfrom{первый домен}{искусственная нейронная сеть}
    \scnrelfrom{второй домен}{слой и.н.с}

\scnheader{выходной слой*}
    \scniselement{неролевое отношение}
    \scniselement{бинарное отношение}
    \scntext{пояснение}{\textbf{\textit{выходной слой*}} --- неролевое отношение, связывающее искусственную нейронную сеть с ее слоем, выходные значения нейронов которого являются выходными значениями всей нейронной сети.}
    \scnrelfrom{область определения}{искусственная нейронная сеть}
    \scnrelfrom{первый домен}{искусственная нейронная сеть}
    \scnrelfrom{второй домен}{слой и.н.с}

\bigskip
\end{scnsubstruct}
\scnendsegmentcomment{Предметная область и онтология искусственных нейронных сетей}

\scnsegmentheader{Предметная область и онтология действий по обработке искусственной нейронной сети}
\begin{scnsubstruct}

\scnheader{Предметная область действий по обработке искусственных нейронных сетей}
    \scnidtf{Предметная область действий по обработке и.н.с.}
    \scniselement{предметная область}
    \begin{scnhaselementrole}{максимальный класс объектов исследования}
        {действие по обработке искусственных нейронных сетей}
    \end{scnhaselementrole}
    \begin{scnhaselementrolelist}{класс объектов исследования}
        \scnitem{действие по обработке искусственных нейронных сетей}
        \scnitem{действие конфигурации весовых коэффициентов и.н.с.}
        \scnitem{действие конфигурации и.н.с.}
        \scnitem{действие интерпретации и.н.с.}
        \scnitem{метод обучения и.н.с.}
        \scnitem{метод обучения с учителем}
        \scnitem{метод обратного распространения ошибки}
        \scnitem{метод обучения без учителя}
        \scnitem{метод оптимизации}
        \scnitem{функция потерь}
        \scnitem{параметр обучения}
        \scnitem{скорость обучения}
        \scnitem{моментный параметр}
        \scnitem{параметр регуляризации}
        \scnitem{размер группы обучения}
        \scnitem{количество эпох обучения}
        \scnitem{выборка}
    \end{scnhaselementrolelist}
    \begin{scnhaselementrolelist}{исследуемое отношение}
        \scnitem{обучающая выборка\scnrolesign}
        \scnitem{тестовая выборка\scnrolesign}
        \scnitem{валидационная выборка\scnrolesign}
        \scnitem{метод обучения\scnrolesign}
        \scnitem{метод оптимизации\scnrolesign}
        \scnitem{функция потерь\scnrolesign}
    \end{scnhaselementrolelist}

\scnheader{действие по обработке искусственной нейронной сети}
    \scnidtf{действие по обработке и.н.с.}
    \scnidtf{действие с искусственной нейронной сетью}
    \scnsubset{действие}
    \scntext{пояснение}{В зависимости от того, является ли искусственная нейронная сеть знаком внешней по отношению к памяти системы сущности, элементы множества действие по обработке и.н.с. являются либо элементами множества \textbf{\textit{действие, выполняемое кибернетической системой в своей внешней среде}}, либо элементом множества \textbf{\textit{действие, выполняемое кибернетической системой в собственной памяти.}}.}
    \begin{scnsubdividing}
        \scnitem{действие конфигурации и.н.с.}
        \begin{scnindent}
        \begin{scnsubdividing}
            \scnitem{действие создания и.н.с.}
            \scnitem{действие редактирования и.н.с.}
            \scnitem{действие удаления и.н.с.}
            \scnitem{действие конфигурации слоя и.н.с.}
            \begin{scnindent}
                \begin{scnsubdividing}
                    \scnitem{действие добавления слоя в и.н.с.}
                    \scnitem{действие редактирования слоя и.н.с.}
                    \scnitem{действие удаления слоя и.н.с.}
                    \scnitem{действие установки функции активации нейронов слоя и.н.с.}
                    \scnitem{действие конфигурации нейрона в слое и.н.с.}
                    \begin{scnindent}
                        \begin{scnsubdividing}
                            \scnitem{действие добавления нейрона в слой и.н.с.}
                            \scnitem{действие редактирования нейрона в слое и.н.с.}
                            \scnitem{действие удаления нейрона из слоя и.н.с.}
                            \scnitem{действие установки функции активации нейрона в слое и.н.с.}
                        \end{scnsubdividing}
                    \end{scnindent}
                \end{scnsubdividing}
            \end{scnindent}
        \end{scnsubdividing}
        \end{scnindent}
        \scnitem{действие конфигурации весовых коэффициентов и.н.с.}
        \begin{scnindent}
            \scnsuperset{действие обучения и.н.с.}
            \scnsuperset{действие начальной инициализации весов и.н.с.}
            \begin{scnindent}
                \scnsuperset{действие начальной инициализации весов нейронов слоя и.н.с.}
                \begin{scnindent}
                    \scnsuperset{действие начальной инициализации весов нейрона и.н.с.}
                \end{scnindent}
            \end{scnindent}
        \end{scnindent}
        \scnitem{действие интерпретации и.н.с.}
    \end{scnsubdividing}
    \scntext{примечание}{Действия по обработке и.н.с осуществляет соответствующий коллектив агентов.}
    \scntext{пояснение}{Так как в результате действий по обработке и.н.с объект этих действий, конкретная и.н.с, может существенно меняться (меняется конфигурация сети, ее весовые коэффициенты), то и.н.с представляется в базе знаний как темпоральное объединение всех ее версий. Каждая версия является и.н.с. и темпоральной сущностью. На множестве этих темпоральных сущностей задается темпоральная последовательность с указанием первой и последней версии. Для каждой версии описываются специфичные знания. Общие для всех версий знания описываются для и.н.с, являющейся темпоральным объединением всех версий.}
    \begin{scnindent}
        \scnrelfrom{пример}{\scnfileimage[20em]{Contents/part_ps/src/images/sd_ps/sd_ann/temporal_neural_network_scg.png}}
    \end{scnindent}

\scnheader{действие обучения и.н.с.}
    \scnidtf{действие обучения искусственной нейронной сети}
    \scnsubset{действие конфигурации весовых коэффициентов и.н.с.}
    \scndefinition{\textbf{\textit{действие обучения и.н.с.}} --- действие, в ходе которого реализуется определенный метод обучения и.н.с. с заданными параметрами обучения и.н.с, методом оптимизации и функцией потерь.}
    \begin{scnrelfromset}{известные проблемы}
        \scnfileitem{Переобучение --- проблема, возникающая при обучении и.н.с., заключающаяся в том, что сеть хорошо адаптируется к п.в.а. из обучающей выборки, при этом теряя способность к обобщению. Переобучение возникает из-за применения неоправданно сложной модели при обучении и.н.с. Это происходит, когда количество настраиваемых параметров и.н.с. намного больше размера обучающей выборки. Возможные варианты решения проблемы заключаются в упрощении модели, увеличении выборки, использовании регуляризации (параметр регуляризации, техника dropout и т.д.).\\
            Обнаружение переобученности сложнее, чем недообученности. Как правило, для этого применяется кросс-валидация на валидационной выборке, позволяющая оценить момент завершения процесса обучения. Идеальным вариантом является достижение баланса между переобученностью и недообученностью.}
        \scnfileitem{Недообучение --- проблема, возникающая при обучении и.н.с., заключающаяся в том, что сеть дает одинаково плохие результаты на обучающей и контрольной выборках. Чаще всего такого рода проблема возникает при недостаточном времени, затраченном на обучение модели. Однако это может быть вызвано и слишком простой архитектурой модели либо малым размером обучающей выборки. Соответственно решение, которое может быть принято ML-инженером, заключается в устранении этих недостатков: увеличение времени обучения, использование модели с большим числом настраиваемых параметров, увеличение размера обучающей выборки, а также уменьшение регуляризации и более тщательный отбор признаков для обучающих примеров.}
    \end{scnrelfromset}
    \scnrelfrom{описание примера}{\scnfileimage[30em]{Contents/part_ps/src/images/sd_ps/sd_ann/ann_trainning_scg.png}}

\scnheader{выборка}
    \scnsubset{множество}
	\scntext{пояснение}{\textbf{\textit{выборка}} --- множество п.в.а., используемых в процессе обучения, тестирования и архитектурной настройки и.н.с.}

\scnheader{обучающая выборка\scnrolesign}
    \scnidtf{training set\scnrolesign}
    \scniselement{ролевое отношение}
    \scnrelfrom{первый домен}{действие обучения и.н.с.}
    \scnrelfrom{второй домен}{выборка}
    \scnrelfrom{область определения}{\scnnonamednode}
    \begin{scnindent}
    	\begin{scnreltoset}{объединение}
    		\scnitem{действие обучения и.н.с.}
    		\scnitem{выборка}
    	\end{scnreltoset}
    \end{scnindent}
    \scntext{пояснение}{\textbf{\textit{обучающая выборка\scnrolesign}} --- ролевое отношение, связывающее действие обучения и.н.с. с выборкой, используемой для изменения настраиваемых параметров и.н.с. в процессе ее обучения.}

\scnheader{тестовая выборка\scnrolesign}
    \scnidtf{test set\scnrolesign}
    \scniselement{ролевое отношение}
    \scnrelfrom{первый домен}{действие обучения и.н.с.}
    \scnrelfrom{второй домен}{выборка}
   	\scnrelfrom{область определения}{\scnnonamednode}
   	\begin{scnindent}
   		\begin{scnreltoset}{объединение}
   			\scnitem{действие обучения и.н.с.}
   			\scnitem{выборка}
   		\end{scnreltoset}
   	\end{scnindent}
    \scntext{пояснение}{\textbf{\textit{тестовая выборка\scnrolesign}} --- ролевое отношение, связывающее действие обучения и.н.с. с выборкой, используемой для проверки обобщающей способности обученной и.н.с.}
    \scntext{примечание}{Элементы контрольной выборки не используются в процессе обучения.}

\scnheader{валидационная выборка\scnrolesign}
    \scniselement{ролевое отношение}
    \scnrelfrom{первый домен}{действие обучения и.н.с.}
    \scnrelfrom{второй домен}{выборка}
    \scnrelfrom{область определения}{\scnnonamednode}
   	\begin{scnindent}
   		\begin{scnreltoset}{объединение}
   			\scnitem{действие обучения и.н.с.}
   			\scnitem{выборка}
		\end{scnreltoset}
   	\end{scnindent}
    \scntext{пояснение}{\textbf{\textit{валидационная выборка\scnrolesign}} --- ролевое отношение, связывающее действие обучения и.н.с. с выборкой, используемой для определения (настройки) архитектурных параметров и.н.с. и параметров обучения.}
    \scntext{примечание}{Элементы валидационной выборки не используются в процессе обучения (не входят в обучающую выборку).}

\scnheader{метод обучения и.н.с.}
    \scnsubset{метод}
	\scntext{пояснение}{\textbf{\textit{метод обучения и.н.с.}} --- метод итеративного поиска оптимальных значений настраиваемых параметров и.н.с., минимизирующих некоторую заданную функцию потерь.}
	\scntext{примечание}{Стоит отметить, что хотя целью применения метода обучения является минимизация функции потерь, \scnqqi{полезность} полученной после обучения модели можно оценить только по достигнутому уровню ее обобщающей способности.}
	\scnsuperset{метод обучения с учителем}
	\begin{scnindent}
		\scntext{пояснение}{\textbf{\textit{метод обучения с учителем}} --- метод обучения с использованием заданных целевых переменных.}
		\scnsuperset{метод обратного распространения ошибки}
		\begin{scnindent}
			\scnidtf{м.о.р.о.}
			\scntext{алгоритм}{\\
				\begin{minipage}{\linewidth}
					\begin{algorithm}[H]
						\KwData{$X$ --- данные, $Et$ --- желаемый отклик (метки), $E\underscore{m}$ --- желаемая ошибка (в соответствии с выбранной функцией потерь)}
						\KwResult{обученная нейронная сеть \textit{Net}}
						инициализация весов \textit{W} и порогов \textit{T};\\
						\Repeat{$E<E\underscore{m}$}{
							\ForEach{$x \in X$ $\And$ $e \in Et$}{
								фаза прямого распространения сигнала: вычисляются активации для всех слоев и.н.с.;\\
								фаза обратного распространения ошибки: вычисляются ошибки для последнего слоя и всех предшествующих слоев;\\
								изменение настраиваемых параметров и.н.с. в соответствии с вычисленными ошибками;\\
							}
							вычисление общей ошибки E на данной эпохе;
						}
					\end{algorithm}
				\end{minipage}}
			\scntext{примечание}{м.о.р.о. использует заданный метод оптимизации и заданную функцию потерь для реализации фазы обратного распространения ошибки и изменения настраиваемых параметров и.н.с. Одним из самых распространенных методов оптимизации является метод стохастического градиентного спуска. Приведенный м.о.р.о. используется для реализации последовательного варианта обучения.}
			\scntext{примечание}{Следует также отметить, что несмотря на то, что метод отнесен к методам обучения с учителем, в случае использования м.о.р.о. для обучения автокодировщиков в классических публикациях он рассматривается как метод обучения без учителя, поскольку в данном случае размеченные данные отсутствуют.}
		\end{scnindent}
	\end{scnindent}
	\scnsuperset{метод обучения без учителя}
	\begin{scnindent}
		\scntext{пояснение}{\textbf{\textit{метод обучения без учителя}} --- метод обучения без использования заданных целевых переменных (в режиме самоорганизации)}
		\scntext{пояснение}{В ходе выполнения алгоритма метода обучения без учителя выявляются полезные структурные свойства набора. Неформально его понимают как метод для извлечения информации из распределения, выборка для которого не была вручную аннотирована человеком.}
		\begin{scnindent}
			\begin{scnrelfromset}{источник}
				\scnitem{\scncite{Goodfellow2017}}
			\end{scnrelfromset}
		\end{scnindent}
	\end{scnindent}

\scnheader{метод обучения\scnrolesign}
    \scniselement{ролевое отношение}
    \scnrelfrom{первый домен}{действие обучения и.н.с.}
    \scnrelfrom{второй домен}{метод обучения и.н.с.}
    \scnrelfrom{область определения}{\scnnonamednode}
    \begin{scnindent}
    	\begin{scnreltoset}{объединение}
    		\scnitem{действие обучения и.н.с.}
    		\scnitem{метод обучения и.н.с.}
    	\end{scnreltoset}
    \end{scnindent}
    \scntext{пояснение}{\textbf{\textit{метод обучения\scnrolesign}} --- ролевое отношение, связывающее действие обучения и.н.с. с методом обучения,  использующимся для обучения и.н.с. в рамках этого действия.}

\scnheader{метод оптимизации}
    \scnsubset{метод}
	\scndefinition{\textbf{\textit{метод оптимизации}} --- метод для минимизации целевой функции потерь при обучении и.н.с.}
	\begin{scnrelfromlist}{включение}
		\scnitem{SGD}
			\begin{scnindent}
				\scnidtf{стохастический градиентный спуск}
				\scnidtf{с.г.с.}
				\scnidtf{stochastic gradient descent}
				\scntext{примечание}{В методе стохастического градиентного спуска корректировка настраиваемых параметров и.н.с. выполняется в направлении максимального уменьшения функции стоимости, т.е. в направлении, противоположном вектору градиента функции потерь.}
				\begin{scnindent}
					\begin{scnrelfromset}{источник}
						\scnitem{\scncite{Haykin2006}}
					\end{scnrelfromset}
				\end{scnindent}
			\end{scnindent}
		\scnitem{Nesterov method}
			\begin{scnindent}
				\scnidtf{метод Нестерова}
				\scntext{примечание}{Обучение методом с.г.с. иногда происходит очень медленно. Импульсный метод позволяет ускорить обучение, особенно в условиях высокой кривизны, небольших, но устойчивых градиентов или зашумленных градиентов. В импульсном методе вычисляется экспоненциально затухающее скользящее среднее прошлых градиентов и продолжается движение в этом направлении. Метод Нестерова является вариантом импульсного алгоритма, в котором градиент вычисляется после применения текущей скорости.}
				\begin{scnindent}
					\begin{scnrelfromset}{источник}
						\scnitem{\scncite{Goodfellow2017}}
					\end{scnrelfromset}
				\end{scnindent}
			\end{scnindent}
		\scnitem{AdaGrad}
			\begin{scnindent}
				\scnidtf{adaptive gradient}
				\scntext{примечание}{Данный метод по отдельности адаптирует скорости обучения всех настраиваемых параметров и.н.с., умножая их на коэффициент, обратно пропорциональный квадратному корню из суммы всех прошлых значений квадрата градиента.}
				\begin{scnindent}
					\begin{scnrelfromset}{источник}
						\scnitem{\scncite{Duchi2011}}
					\end{scnrelfromset}
				\end{scnindent}
			\end{scnindent}
		\scnitem{RMSProp}
			\begin{scnindent}
				\scnidtf{root mean square propagation}
				\scntext{примечание}{Данный метод является модификацией AdaGrad, которая позволяет улучшить его поведение в невыпуклом случае путем изменения способа агрегирования градиента на экспоненциально взвешенное скользящее среднее. Использование экспоненциально взвешенного скользящего среднего гарантирует повышение скорости сходимости после обнаружения выпуклой впадины, как если бы внутри этой впадины алгоритм AdaGrad был инициализирован заново.}
				\begin{scnindent}
					\begin{scnrelfromset}{источник}
						\scnitem{\scncite{Goodfellow2017}}
					\end{scnrelfromset}
				\end{scnindent}
			\end{scnindent}
			
		\scnitem{Adam}
			\begin{scnindent}
				\scnidtf{adaptive moments}
				\scntext{примечание}{Данный метод можно рассматривать как комбинацию RMSProp и AdaGrad. Помимо усредненного первого момента, данный метод использует усредненное значение вторых моментов градиентов}
				\begin{scnindent}
					\begin{scnrelfromset}{источник}
						\scnitem{\scncite{Kingma2014}}
					\end{scnrelfromset}
				\end{scnindent}
			\end{scnindent}
	\end{scnrelfromlist}
	\scntext{примечание}{Успешность применения методов оптимизации зависит главным образом от знакомства пользователя с соответствующим алгоритмом.}
	\begin{scnindent}
		\begin{scnrelfromset}{источник}
			\scnitem{\scncite{Goodfellow2017}}
		\end{scnrelfromset}
	\end{scnindent}

\scnheader{метод оптимизации\scnrolesign}
    \scniselement{ролевое отношение}
    \scnrelfrom{первый домен}{метод обучения и.н.с.}
    \scnrelfrom{второй домен}{метод оптимизации}
    \scnrelfrom{область определения}{\scnnonamednode}
    \begin{scnindent}
    	\begin{scnreltoset}{объединение}
    		\scnitem{метод обучения и.н.с.}
    		\scnitem{метод оптимизации}
    	\end{scnreltoset}
    \end{scnindent}
    \scntext{пояснение}{\textbf{\textit{метод оптимизации\scnrolesign}} --- ролевое отношение, связывающее метод обучения и.н.с. с методом оптимизации, использующимся для обучения и.н.с. с помощью данного метода.}

\scnheader{функция потерь}
    \scnsubset{функция}
	\scntext{пояснение}{\textbf{\textit{функция потерь}} --- функция, используемая для вычисления ошибки, рассчитываемой как разница между фактическим эталонным значением и прогнозируемым значением, получаемым и.н.с.}
    \begin{scnrelfromlist}{включение}
		\scnitem{MSE}
		\begin{scnindent}
			\scnidtf{mean square error}
			\scnidtf{средняя квадратичная ошибка}
			\scntext{формула}{
				\begin{equation*}
                    MSE \eq \frac{1}{L} \sum\underscore{l \eq 1}\upperscore{L} \sum\underscore{i \eq 1}\upperscore{m} (y\underscore{i}\upperscore{l} - e\underscore{i}\upperscore{l})\upperscore{2}
				\end{equation*}}
            \begin{scnindent}
                \scntext{примечание}{$y\underscore{i}\upperscore{l}$ --- прогноз модели, $e\underscore{i}\upperscore{l}$ --- ожидаемый (эталонный) результат, \textit{m} --- размерность выходного вектора, \textit{L} --- объем обучающей выборки.}
            \end{scnindent}
		\end{scnindent}
		\scnitem{BCE}
		\begin{scnindent}
			\scnidtf{binary cross entropy}
			\scnidtf{бинарная кросс-энтропия}
			\scntext{формула}{
				\begin{equation*}
                    BCE \eq - \sum\underscore{l \eq 1}\upperscore{L} (e\upperscore{l} \log(y\upperscore{l}) + (1 - e\upperscore{l})\log(1 - y\upperscore{l}))
				\end{equation*}}
            \begin{scnindent}
                \scntext{примечание}{$y\upperscore{l}$ --- прогноз модели, $e\upperscore{l}$ --- ожидаемый (эталонный) результат: \textit{0} или \textit{1}, \textit{L} --- объем обучающей выборки.}
            \end{scnindent}
			\scntext{примечание}{для бинарной кросс-энтропии в выходном слое и.н.с. будет находиться один нейрон}
		\end{scnindent}
		\scnitem{MCE}
		\begin{scnindent}
			\scnidtf{multi-class cross entropy}
			\scnidtf{мультиклассовая кросс-энтропия}
			\scntext{формула}{
				\begin{equation*}
                    MCE \eq - \sum\underscore{l \eq 1}\upperscore{L} \sum\underscore{i \eq 1}\upperscore{m} e\underscore{\scnleftcurlbrace i\scnrightcurlbrace\upperscore{l}} \log(y\underscore{\scnleftcurlbrace i\scnrightcurlbrace\upperscore{l}})
                \end{equation*}}
                \begin{scnindent}
                    \scntext{примечание}{$y\underscore{\scnleftcurlbrace i\scnrightcurlbrace\upperscore{l}}$ --- прогноз модели,  $e\underscore{i\upperscore{l}}$ --- ожидаемый (эталонный результат), \textit{m} --- размерность выходного вектора.}
                \end{scnindent}
			\scntext{примечание}{для мультиклассовой кросс-энтропии количество нейронов в выходном слое и.н.с. совпадает с количеством классов}
		\end{scnindent}
	\end{scnrelfromlist}
	\scntext{примечание}{Для решения задачи классификации рекомендуется использовать бинарную или мультиклассовую кросс-энтропийную функцию потерь, для решения задачи регрессии рекомендуется использовать среднюю квадратичную ошибку.}

\scnheader{функция потерь\scnrolesign}
    \scniselement{ролевое отношение}
    \scnrelfrom{первый домен}{метод обучения и.н.с.}
    \scnrelfrom{второй домен}{функция потерь}
    \scnrelfrom{область определения}{\scnnonamednode}
    \begin{scnindent}
    	\begin{scnreltoset}{объединение}
    		\scnitem{метод обучения и.н.с.}
    		\scnitem{функция потерь}
    	\end{scnreltoset}
    \end{scnindent}
    \scntext{пояснение}{\textbf{\textit{функция потерь\scnrolesign}} --- ролевое отношение, связывающее метод обучения и.н.с. с функцией потерь, использующимся для обучения и.н.с. с помощью данного метода.}

\scnheader{параметр обучения}
   \scnidtf{группа наиболее общих параметров метода обучения и.н.с.}
   \begin{scnrelfromset}{состав группы параметров обучения}
       \scnitem{скорость обучения}
          \begin{scnindent}
              \scntext{пояснение}{\textbf{\textit{скорость обучения}} --- параметр, определяющий скорость изменения параметров и.н.с. в процессе обучения.}
          \end{scnindent}
       \scnitem{моментный параметр}
          \begin{scnindent}
              \scnidtf{момент}
              \scnidtf{momentum}
              \scntext{пояснение}{\textbf{\textit{моментный параметр}} --- параметр, используемый в процессе обучения для устранения проблемы \scnqqi{застревания} алгоритма обучения в локальных минимумах минимизируемой функции потерь.}
              \scntext{пояснение}{При обучении и.н.с. частой является ситуация остановки процесса в определенной точке локального минимума без достижения желаемого уровня обобщающей способности и.н.с. Для устранения такого нежелательного явления вводится дополнительный параметр (момент) позволяющий алгоритму обучения \scnqqi{перескочить} через локальный минимум и продолжить процесс.}
         \end{scnindent}
       \scnitem{параметр регуляризации}
         \begin{scnindent}
             \scntext{пояснение}{\textbf{\textit{параметр регуляризации}} --- параметр, применяемый для контроля уровня переобучения и.н.с.}
             \scntext{пояснение}{\textbf{\textit{регуляризация}} --- добавление дополнительных ограничений к правилам изменения настраиваемых параметров и.н.с. с целью предотвратить переобучение.}
         \end{scnindent}
       \scnitem{размер группы обучения}
         \begin{scnindent}
             \scntext{пояснение}{\textbf{\textit{размер группы обучения}} --- размер группы п.в.а., которая используется для изменения параметров и.н.с. на каждом элементарном шаге обучения.}
         \end{scnindent}
       \scnitem{количество эпох обучения}
       \begin{scnindent}
       		\scntext{пояснение}{\textbf{\textit{эпоха обучения}} --- одна итерация алгоритма обучения, в ходе которой все обучающие п.в.а. из обучающей выборки были однократно использованы.}
       \end{scnindent}
   \end{scnrelfromset}

\bigskip
\end{scnsubstruct}
\scnendsegmentcomment{Предметная область и онтология действий по обработке искусственной нейронной сети}

\bigskip
\end{scnsubstruct}
\scnendcurrentsectioncomment

\end{SCn}


\scsubsubsection{Пункт 30.7.1. Предметная область и онтология синтаксиса sc-моделей искусственных нейронных сетей}
\label{sd_syntax_sc_model_ann}
\scnsegmentheader{Предметная область и онтология синтаксиса sc-моделей искусственных нейронных сетей}
\begin{scnsubstruct}

\scnendcurrentsectioncomment
\end{scnsubstruct}

\scsubsubsection{Пункт 30.7.2. Предметная область и онтология денотационной семантики sc-моделей искусственных нейронных сетей}
\label{sd_denot_sem_sc_model_ann}
\begin{scnsubstruct}

\scnheader{Денотационная семантика Языка представления нейросетевого метода решения задач}
\scntext{примечание}{Денотационная семантика Языка представления нейросетевого метода решения задач в базах знаний описывается в рамках предметной области и соответствующей ей онтологии нейросетевого метода.}
\scntext{примечание}{Так же в \textit{Предметную область нейросетевых методов} добавлены понятия для описания метрик эффективности \textit{нейросетевых методов}. Данные метрики учитываются \textit{решателем задач} при принятии решения об использовании того или иного \textit{нейросетевого метода}.}

\scnheader{искусственная нейронная сеть}
\scnidtf{и.н.с.}
\scnidtf{множество искусственных нейронных сетей}
\scnidtf{нейронная сеть}
\scnidtf{нейросетевой метод}
\scntext{определение}{\textbf{\textit{искусственная нейронная сеть}} --- это совокупность нейронных элементов и связей между ними.}
\scntext{примечание}{Искусственная нейронная сеть состоит из \textbf{\textit{формальных нейронов}}, которые связаны между собой посредством \textbf{\textit{синаптических связей}}. Нейроны организованы в \textbf{\textit{слои}}. Каждый нейрон слоя принимает сигналы со входящих в него синаптических связей, обрабатывает их единым образом с помощью заданной ему или всему слою \textbf{\textit{функции активации}} и передает результат на выходящие из него синаптические связи.}
\begin{scnindent}
	\begin{scnrelfromset}{источник}
		\scnitem{\scncite{Golovko2017}}
	\end{scnrelfromset}
\end{scnindent}

\scnheader{архитектура и.н.с.}
\scntext{примечание}{\textit{Архитектурой и.н.с.} будем называть совокупность информации о структуре ее слоев, формальных нейронов, синаптических связей и функций активаций. То есть то, что можно обучить и использовать для решения задач.}
\scnrelfrom{пример}{\scnfileimage[30em]{Contents/part_ps/src/images/sd_ps/sd_ann/neural_network.png}}
\begin{scnindent}
	\scntext{примечание}{Пример архитектуры и.н.с.}
\end{scnindent}

\scnheader{искусственная нейронная сеть}
\scntext{примечание}{В соответствии с тем, какая у и.н.с. архитектура, можно выделить соответствующую иерархию классов и.н.с.}
\scnrelfrom{разбиение}{\scnkeyword{Типология и.н.с. по признаку направленности связей\scnsupergroupsign}}
\begin{scnindent}
	\begin{scneqtoset}
		\scnitem{искусственная нейронная сеть с прямыми связями}
		\begin{scnindent}
			\begin{scnsubdividing}
				\scnitem{персептрон}
				\begin{scnindent}
					\begin{scnsubdividing}
						\scnitem{персептрон Розенблатта}
						\scnitem{автоэнкодерная искусственная нейронная сеть}
					\end{scnsubdividing}
				\end{scnindent}
				\scnitem{машина опорных векторов}
				\scnitem{искусственная нейронная сеть радиально-базисных функций}
				\scnitem{сверточная искусственная нейронная сеть}
			\end{scnsubdividing}
		\end{scnindent}
		\scnitem{искусственная нейронная сеть с обратными связями}
		\begin{scnindent}
			\begin{scnsubdividing}
				\scnitem{нейронная сеть Хопфилда}
				\scnitem{нейронная сеть Хэмминга}
			\end{scnsubdividing}
		\end{scnindent}
		\scnitem{рекуррентная искусственная нейронная сеть}
		\begin{scnindent}
			\begin{scnsubdividing}
				\scnitem{искусственная нейронная сеть Джордана}
				\scnitem{искусственная нейронная сеть Элмана}
				\scnitem{мультирекуррентная нейронная сеть}
				\scnitem{LSTM-элемент}
				\scnitem{GRU-элемент}
			\end{scnsubdividing}
		\end{scnindent}
	\end{scneqtoset}
\end{scnindent}
\scnrelfrom{разбиение}{\scnkeyword{Типология и.н.с. по признаку полноты связей\scnsupergroupsign}}
\begin{scnindent}
	\begin{scneqtoset}
		\scnitem{полносвязная искусственная нейронная сеть}
		\scnitem{слабосвязная искусственная нейронная сеть}
	\end{scneqtoset}
\end{scnindent}

\scnheader{формальный нейрон}
\scnidtf{искусственный нейрон}
\scnidtf{нейрон}
\scnidtf{ф.н.}
\scnidtf{нейронный элемент}
\scnidtf{множество нейронов искусственных нейронных сетей}
\scnidtf{математическая модель реального биологического нейрона}
\scntext{примечание}{Отдельный формальный нейрон является искусственной нейронной сети с одним нейроном в единственном слое.}
\scnsubset{искусственная нейронная сеть}
\scntext{пояснение}{\textbf{\textit{формальный нейрон}} --- это основной элемент \textit{искусственной нейронной сети}, применяющий свою \textit{функцию активации} к сумме произведений входных сигналов на весовые коэффициенты:
	$$y = F\left(\sum_{i=1}^{n} w_ix_i - T\right) = F(WX - T)$$
	где $X = (x_1,x_2,...,x_n)^T$ --- вектор входного сигнала; $W - (w_1,w_2,...,w_n)$ --- вектор весовых коэффициентов; $T$ --- пороговое значение;
	\textit{F} --- функция активации.}
\begin{scnindent}
	\begin{scnrelfromset}{источник}
		\scnitem{\scncite{Golovko2017}}
	\end{scnrelfromset}
\end{scnindent}
\scnrelfrom{изображение}{\scnfileimage[20em]{Contents/part_ps/src/images/sd_ps/sd_ann/neuron.png}}
\begin{scnindent}
	\scntext{примечание}{Схема модели формального нейрона.}
\end{scnindent}
\scntext{примечание}{Формальные нейроны могут иметь полный набор связей с нейронами предшествующего слоя или неполный (разряженный) набор связей.}
\begin{scnsubdividing}
	\scnitem{полносвязный формальный нейрон}
	\begin{scnindent}
		\scnidtf{нейрон, у которого есть полный набор связей с нейронами предшествующего слоя}
		\scntext{пояснение}{Отдельный обрабатывающий элемент и.н.с., выполняющий функциональное преобразование взвешенной суммы элементов вектора входных значений с помощью функции активации.}
	\end{scnindent}
	\scnitem{сверточный формальный нейрон}
	\begin{scnindent}
		\scntext{пояснение}{Отдельный обрабатывающий элемент и.н.с., выполняющий функциональное преобразование результата операции свертки матрицы входных значений с помощью функции активации.}
		\scntext{примечание}{Сверточный формальный нейрон может быть представлен полносвязным формальным нейроном.}
		\scntext{примечание}{Сверточный формальный нейрон с соответствующим ему ядром свертки может быть представлен нейроном с неполным набором связей.}
	\end{scnindent}
	\scnitem{рекуррентный формальный нейрон}
	\begin{scnindent}
		\scntext{пояснение}{Формальный нейрон, имеющий обратную связь с самим собой или с другими нейронами и.н.с.}
	\end{scnindent}
\end{scnsubdividing}

\scnheader{синаптическая связь}
\scnidtf{синапс}
\scnsubset{ориентированная пара}
\scndefinition{\textbf{\textit{синаптическая связь}} --- ориентированная пара, первым компонентом которой является нейрон, из которого исходит сигнал, а вторым компонентом --- нейрон, который принимает этот сигнал.}

\scnheader{слой и.н.с.}
\scnidtf{слой}
\scnidtf{слой искусственной нейронной сети}
\scnidtf{множество слоев искусственных нейронных сетей}
\scntext{примечание}{Отдельный слой является искусственной нейронной сетью с одним слоем. Следует отметить принципиальную важность этого замечания. Один слой и.н.с. уже является нейронной сетью, поскольку над ним можно производить все основные операции, которые производятся над \scnqq{большой} и.н.с. (его можно обучить и использовать для решения определенной задачи).}
\scnsubset{искусственная нейронная сеть}
\scntext{пояснение}{\textbf{\textit{слой и.н.с}} --- это множество нейронных элементов, на которые в каждый такт времени параллельно поступает информация от других нейронных элементов сети.}
\begin{scnindent}
	\begin{scnrelfromset}{источник}
		\scnitem{\scncite{Golovko2017}}
	\end{scnrelfromset}
\end{scnindent}
\scntext{пояснение}{\textbf{\textit{слой и.н.с.}} --- это множество формальных нейронов, осуществляющих параллельную независимую обработку вектора или матрицы входных значений}
\scnrelfrom{разбиение}{\scnkeyword{Типология слоев и.н.с. по признаку операции, осуществляемой слоем\scnsupergroupsign}}
\begin{scnindent}
	\begin{scneqtoset}
	\scnitem{полносвязный слой и.н.с.}
	\begin{scnindent}
		\scnidtf{слой, в котором каждый нейрон имеет связь с каждым нейроном предшествующего слоя}
		\scnidtf{слой, в котором каждый нейрон является полносвязным}
	\end{scnindent}
	\scnitem{сверточный слой и.н.с.}
	\begin{scnindent}
		\scnidtf{слой, в котором каждый нейрон является сверточным}
	\end{scnindent}
	\scnitem{слой и.н.с. нелинейного преобразования}
	\begin{scnindent}
		\scnidtf{слой, осуществляющий нелинейное преобразование входных данных}
		\scntext{пояснение}{Как правило, выделяются в отдельные слои только в программных реализациях. Фактически рассматриваются как финальный этап расчета выходной активности любого нейрона --- применение функции активации.}
		\scntext{примечание}{не изменяет размерность входных данных}
	\end{scnindent}
	\scnitem{dropout слой и.н.с.}
	\begin{scnindent}
		\scnidtf{слой, реализующий технику регуляризации dropout}
		\scntext{примечание}{Данный тип слоя функционирует только во время обучения и.н.с.}
		\scntext{пояснение}{Поскольку полносвязные слои имеют большое количество настраиваемых параметров, они подвержены эффекту переобучения. Один из способов устранить такой негативный эффект --- выполнить частичный отсев результатов на выходе полносвязного слоя. На этапе обучения техника dropout позволяет отбросить выходную активность некоторых нейронов с определенной, заданной вероятностью. Выходная активность \scnqqi{отброшенных} нейронов полагается равной нулю.}
	\end{scnindent}
	\scnitem{pooling слой и.н.с.}
	\begin{scnindent}
		\scnidtf{подвыборочный слой}
		\scnidtf{объединяющий слой}
		\scnidtf{слой, осуществляющий уменьшение размерности входных данных}
	\end{scnindent}
	\scnitem{слой и.н.с. батч-нормализации}
	\end{scneqtoset}
	\begin{scnindent}
		\scntext{примечание}{Нужно отметить, что данный перечень неполный --- разновидности слоев и.н.с. появляются практически в каждой заслуживающей внимания публикации по нейросетевым алгоритмам и на текущий момент их существует достаточно много, однако, как правило, при построении более традиционных архитектур ограничиваются только приведенными вариантами слоев.}
	\end{scnindent}
\end{scnindent}
\scntext{примечание}{слои и.н.с. также могут быть классифицированы по исполняемой роли в рамках архитектуры (место в последовательности слоев и.н.с.).\\
	\\Так, например, слой, расположенный первым, называется распределяющим. Слои, расположенные далее, за исключением последнего, называются обрабатывающими. Наконец, последний слой носит название выходного слоя и.н.с.}


\scnheader{функция активации*}
\scnidtf{функция активации нейрона*}
\scniselement{неролевое отношение}
\scniselement{бинарное отношение}
\scntext{примечание}{функция активации* --- последний архитектурный компонент и.н.с.}
\scntext{пояснение}{\textbf{\textit{функция активации*}} --- неролевое отношение, связывающее формальный нейрон с функцией, результат применения которой к \textbf{\textit{взвешенной сумме нейрона}} определяет его \textbf{\textit{выходное значение}}.}
  \scnrelfrom{область определения}{\scnnonamednode}
  \begin{scnindent}
	  \begin{scnreltoset}{объединение}
		  \scnitem{формальный нейрон}
		  \scnitem{функция}
	  \end{scnreltoset}
  \end{scnindent}
\scnrelfrom{первый домен}{формальный нейрон}
\scnrelfrom{второй домен}{функция}
\begin{scnindent}
\begin{scnsubdividing}
	\scnitem{линейная функция}
		\begin{scnindent}
			\scntext{формула}{
				\begin{equation*}
					y = kS
				\end{equation*}}
				\begin{scnindent}
					\scntext{примечание}{\textit{k} --- коэффициент наклона прямой, \textit{S} --- в.с.}
				\end{scnindent}
		\end{scnindent}
	\scnitem{пороговая функция}
		\begin{scnindent}
			\scntext{формула}{
				\begin{equation*}
					y = sign(S) =
					\begin{cases}
						1, S > 0,\\
						0, S \leq 0
					\end{cases}
				\end{equation*}}
		\end{scnindent}
	\scnitem{сигмоидная функция}
		\begin{scnindent}
			\scntext{формула}{
				\begin{equation*}
					y = \frac{1}{1+e^{-cS}}
				\end{equation*}}
				\begin{scnindent}
					\scntext{примечание}{\textit{с} > 0 --- коэффициент, характеризующий ширину сигмоидной функции по оси абсцисс, \textit{S} --- в.с.}
				\end{scnindent}				
		\end{scnindent}
	\scnitem{функция гиперболического тангенса}
		\begin{scnindent}
			\scntext{формула}{
				\begin{equation*}
					y = \frac{e^{cS}-e^{-cS}}{e^{cs}+e^{-cS}}
				\end{equation*}}
				\begin{scnindent}
					\scntext{примечание}{\textit{с} > 0 --- коэффициент, характеризующий ширину сигмоидной функции по оси абсцисс, \textit{S} --- в.с.}
				\end{scnindent}
		\end{scnindent}
	\scnitem{функция softmax}
		\begin{scnindent}
			\scntext{формула}{
				\begin{equation*}
					y_j = softmax(S_j) = \frac{e^{S_j}}{\sum_{j} e^{S_j}}
				\end{equation*}}
				\begin{scnindent}
					\scntext{примечание}{$S_j$ --- в.с. \textit{j}-го выходного нейрона.}
				\end{scnindent}
		\end{scnindent}
	\scnitem{функция ReLU}
		\begin{scnindent}
			\scntext{формула}{
				\begin{equation*}
					y = F(S) =
					\begin{cases}
						S, S > 0,\\
						kS, S \leq 0
					\end{cases}
				\end{equation*}}
				\begin{scnindent}
					\scntext{примечание}{\textit{k} = 0 или принимает небольшое значение, например, 0.01 или 0.001.}
				\end{scnindent}
		\end{scnindent}
\end{scnsubdividing}
\end{scnindent}

\scnheader{параметр и.н.с.}
    \scnsubset{параметр}
    \begin{scnsubdividing}
        \scnitem{настраиваемый параметр и.н.с.}
        \begin{scnindent}
            \scnidtf{параметр и.н.с., значение которого изменяется в ходе обучения}
            \begin{scnsubdividing}
                \scnitem{весовой коэффициент синаптической связи}
                \scnitem{пороговое значение}
                \scnitem{ядро свертки}
                \begin{scnindent}
                    \scnidtf{квадратная матрица произвольного порядка, элементы которой изменяются в процессе обучения и.н.с.}
                    \scntext{примечание}{Если сверточный формальный нейрон представить в виде полносвязного формального нейрона, соответствующее ядро свертки преобразуется в вектор весовых коэффициентов.}
                \end{scnindent}
            \end{scnsubdividing}
        \end{scnindent}
        \scnitem{архитектурный параметр и.н.с.}
        \begin{scnindent}
            \scntext{примечание}{Параметр и.н.с., определяющий ее архитектуру.}
            \begin{scnsubdividing}
                \scnitem{количество слоев}
                \scnitem{количество нейронов}
                \scnitem{количество синапсов}
            \end{scnsubdividing}
        \end{scnindent}
    \end{scnsubdividing}

\scnheader{метрика оценки качества и.н.с.}
\scntext{примечание}{Метрики могут быть классифицированы по типу решаемой задачи.}
\scnrelfrom{разбиение}{Типология метрик по признаку решаемой задачи\scnsupergroupsign}
\begin{scnindent}
	\begin{scneqtoset}
		\scnitem{классификационные метрики}
		\begin{scnrelfromset}{декомпозиция}
			\scnitem{точность и.н.с.}
			\scnitem{полнота и.н.с.}
			\scnitem{F1-метрика}
		\end{scnrelfromset}
		\scnitem{регрессионные метрики}
		\begin{scnrelfromset}{декомпозиция}
			\scnitem{MAE}
			\scnitem{MAPE}
			\scnitem{RMSE}
		\end{scnrelfromset}
	\end{scneqtoset}
\end{scnindent}

\scnheader{точность и.н.с.}
\scnidtf{precision}
\scnidtf{доля верно идентифицированных положительных исходов в общем числе исходов, которые были идентифицированы как положительные}
\scntext{формула}{
	\begin{equation*}
		PRE = \frac{TP}{TP + FP}
	\end{equation*}}
	\begin{scnindent}
        \scntext{примечание}{\textit{TP} и \textit{FP} --- число истинно-положительных и ложно-положительных предсказаний нейронной сети соответственно}
    \end{scnindent}

\scnheader{полнота и.н.с.}
\scnidtf{recall}
\scnidtf{доля верно идентифицированных положительных исходов в общем числе положительных исходов}
\scntext{формула}{
	\begin{equation*}
		REC = \frac{TP}{TP + FN}
	\end{equation*}}
	\begin{scnindent}
        \scntext{примечание}{\textit{TP} и \textit{FN} --- число истинно-положительных и ложно-отрицательных предсказаний нейронной сети соответственно}
    \end{scnindent}

\scnheader{F1-метрика}
\scntext{формула}{
	\begin{equation*}
		F1 = 2 * \frac{PRE * REC}{PRE + REC}
	\end{equation*}}
	\begin{scnindent}
        \scntext{примечание}{\textit{PRE} и \textit{REC} --- точность и полнота и.и.с. соответственно}
    \end{scnindent}

\scnheader{MAE}
\scnidtf{mean absolute error}
\scntext{формула}{$\frac{1}{N} \sum_{i=1}^N |y_{etalon}^i - y_{predicted}^i|$}
\begin{scnindent}
	\scntext{примечание}{$y_{etalon}^i$ --- эталонное значение,\\ $y_{predicted}^i$ --- значение, полученное и.н.с.,\\ \textit{N} --- объем обучающей выборки}
\end{scnindent}

\scnheader{MAPE}
\scnidtf{mean absolute percentage error}
\scntext{формула}{$\frac{1}{N} \sum_{i=1}^N \frac{|y_{etalon}^i - y_{predicted}^i|}{y_{etalon}^i} * 100\%$}
\begin{scnindent}
	\scntext{примечание}{$y_{etalon}^i$ --- эталонное значение,\\ $y_{predicted}^i$ --- значение, полученное и.н.с.,\\ \textit{N} --- объем обучающей выборки}
\end{scnindent} 

\scnheader{RMSE}
\scnidtf{root mean squared error}
\scntext{формула}{$\sqrt{\frac{1}{N} \sum_{i=1}^N (y_{etalon}^i - y_{predicted}^i)^2}$}
\begin{scnindent}
	\scntext{примечание}{$y_{etalon}^i$ --- эталонное значение,\\ $y_{predicted}^i$ --- значение, полученное и.н.с.,\\ \textit{N} --- объем обучающей выборки}
\end{scnindent}

\scnheader{SCg-текст. Пример формализации архитектуры искусственной нейронной сети в базе знаний}
\scnrelfrom{изображение}{\scnfileimage[20em]{Contents/part_ps/src/images/sd_ps/sd_ann/neural_network_scg.png}}

\bigskip
\end{scnsubstruct}

\begin{scnrelfromvector}{заключение}
	\scnfileitem{С помощью выделенных понятий становится возможна формализация в \textit{базе знаний} архитектуры конкретных \textit{и.н.с.} В качестве примера, на рисунке \textit{SCg-текст. Пример формализации архитектуры искусственной нейронной сети в базе знаний} представлен пример формализации полносвязной двухслойной \textit{и.н.с.} с двумя нейронами на входном слое и одном нейроне на обрабатывающем слоев.}
	\scnfileitem{Следует отметить, что в практике авторов еще не было необходимости явно представлять и.н.с., как это показано на рисунке \textit{SCg-текст. Пример формализации архитектуры искусственной нейронной сети в базе знаний}. Чаще всего, представление и.н.с. сводилось к представлению ее операционной семантики в виде SCP-программы, как это будет показано далее.}
\end{scnrelfromvector}

\scnendcurrentsectioncomment


\scsubsubsection{Пункт 30.7.3. Предметная область и онтология операционной семантики sc-моделей искусственных нейронных сетей}
\label{sd_oper_sem_sc_model_ann}
\begin{scnsubstruct}

\scnheader{Операционная семантика Языка представления нейросетевого метода решения задач}
\scntext{примечание}{Операционной семантикой любого языка представления методов решения задач является спецификация семейства агентов, обеспечивающих интерпретацию любого метода, принадлежащего соответствующему классу методов. Это семейство является интерпретатором соответствующего метода решения задач. В рамках технологии OSTIS такой интерпретатор называется моделью решения задач. Так как в рамках Технологии OSTIS используется многоагентный подход, то разработка нейросетевой модели решения задач сводится к разработке агентно-ориентированной	модели интерпретации и.н.с.}
\scntext{примечание}{\textbf{\textit{Операционная семантика Языка представления нейросетевого метода в базах знаний}} задается \textit{многоагентный подход} к интерпретации \textit{искусственных нейронных сетей} и спецификацией соответствующих действий.}
\scntext{примечание}{Нейросетевой метод описан в виде программы на некотором \textit{языке программирования}, который может быть как внешним по отношению к \textit{ostis-системе}, так и внутренним (на данный момент, \textit{Язык SCP}). Каждому такому \textit{языку программирования} соответствует некоторая дочерняя \textit{предметная область} \textit{Предметная область нейросетевых методов}}
\begin{scnindent}
	\scnrelfrom{источник}{\scncite{Kovalev2022}}
\end{scnindent}
\scntext{примечание}{В случае описания \textit{нейросетевого метода} на внешнем языке, такой метод описывается в соответствующей предметной области, в рамках которой также специфицируется действие интерпретации данного метода. Данному действию соответствует агент, реализованный на соответствующем \textit{языке программирования}.
	\\Однако для достижения конвергенции и интеграции необходимо описывать нейросетевые методы на внутреннем языке ostis-системы, которым является \textit{Язык SCP}.
	\\Интерпретация \textit{scp-программы} сводится к агентно-ориентированной обработке действий в sc-памяти. Этими действиями являются \textit{scp-операторы}.}

\scnheader{Предметная область нейросетевых методов}
\scnidtf{Предметная область искусственных нейронных сетей}
\begin{scnrelfromlist}{дочерняя предметная область}
    \scnitem{Предметная область нейросетевых методов SCP}
    \scnitem{Предметная область нейросетевых методов Python}
    \scnitem{Предметная область нейросетевых методов C++}
\end{scnrelfromlist}

\scnheader{действие интерпретации слоя и.н.с.}
\begin{scnrelfromset}{декомпозиция}
	\scnitem{действие вычисления взвешенной суммы всех нейронов слоя}
	\scnitem{действие вычисления функции активации всех нейронов слоя}
	\scnitem{действие интерпретации сверточного слоя}
	\scnitem{действие интерпретации пулинг слоя}
\end{scnrelfromset}
\scntext{примечание}{При необходимости задавать различные аргументы для нейронов одного и того же слоя, можно специфицировать соответствующие действия, однако на данный момент этого не было произведено из-за слабой изученности подобного рода \textit{нейросетевых моделей решения задач}.}

\scnheader{ориентированное множество чисел}
\scnidtf{ормножество чисел}
\scnrelto{включение}{число}
\scnrelto{включение}{ориентированное множество}
\scnrelto{первый домен}{строковое представление ормножества чисел*}
\scntext{примечание}{Для описания спецификации указанных действий необходимо ввести понятия \textit{ориентированного множества чисел} и \textit{матрицы}, с помощью которых задаются входные значения \textit{и.н.с.}, выходные значения \textit{и.н.с.}, матрицы весовых коэффициентов и прочее.
	\\Каждый элемент ориентированного множества чисел является некоторым числом. Числа могут быть представлены в виде sc-узлов, либо с помощью строкового представления всего множества, для чего используется специальное отношение \textit{строковое представление ормножества чисел*}, которое введено в целях оптимизации некоторых вариантов реализации агента, интерпретирующего действие, использующее понятие ориентированного множества чисел.}
	
\scnheader{матрица}
\scntext{примечание}{\textit{матрица} является \textit{ориентированным множеством} \textit{ориентированных множеств} чисел равной мощности.}

\scnheader{Действие вычисления взвешенной суммы всех нейронов слоя}
\scntext{примечание}{Аргументы (\textit{объекты\scnrolesign}) этого действия задаются следующими отношениями: \textit{входной вектор\scnrolesign}, \textit{матрица весовых коэффициентов нейронов слоя\scnrolesign}.}
\scnhaselementrole{результат}{ориентированное множество чисел, являющихся взвешенной суммой нейронов соответствующего слоя.}

\scnheader{входной вектор\scnrolesign}
\scnrelfrom{первый домен}{действие интерпретации и.н.с.}
\scnrelfrom{второй домен}{ориентированное множество чисел}

\scnheader{матрица весовых коэффициентов нейронов слоя\scnrolesign}
\scnrelfrom{первый домен}{действие по обработке и.н.с.}
\scnrelfrom{второй домен}{матрица}

\scnheader{SCg-текст. Пример действия вычисления взвешенной суммы всех нейронов слоя}
\scnrelfrom{описание примера}{\scnfileimage[30em]{Contents/part_ps/src/images/sd_ps/sd_ann/action_weighted_sum.png}}
\begin{scnindent}
	\scntext{примечание}{Пример спецификации действия вычисления взвешенной суммы всех нейронов слоя для слоя с двумя нейронами и входным вектором размерностью 2}
\end{scnindent}

\scnheader{Действие вычисления функции активации всех нейронов слоя}
\scntext{примечание}{Аргументы этого действия задаются следующими отношениями: \textit{вектор взвешенных сумм нейронов слоя\scnrolesign}, \textit{вектор порогов нейронов слоя\scnrolesign}, \textit{функция активации\scnrolesign}.}
\scnhaselementrole{результат}{ориентированное множество чисел, являющихся выходными значениями нейронов слоя}

\scnheader{вектор взвешенных сумм нейронов слоя\scnrolesign}
\scnrelfrom{первый домен}{действие по обработке и.н.с.}
\scnrelfrom{второй домен}{ориентированное множество чисел}

\scnheader{вектор порогов нейронов слоя\scnrolesign}
\scnrelfrom{первый домен}{действие по обработке и.н.с.}
\scnrelfrom{второй домен}{ориентированное множество чисел}

\scnheader{функция активации\scnrolesign}
\scnrelfrom{первый домен}{действие по обработке и.н.с.}
\scnrelfrom{второй домен}{функция}
\scntext{примечание}{Любой агент, интерпретирующий действия с заданными с помощью отношения \textit{функция активации\scnrolesign} аргументами, должен использовать интерпретатор математических функций. использующихся в качестве функций активации.}

\scnheader{Действие интерпретации сверточного слоя}
\scntext{примечание}{Аргументы этого действия задаются следующими отношениями: \textit{входная матрица\scnrolesign}, \textit{ядро свертки\scnrolesign}, \textit{шаг свертки\scnrolesign}.}
\scnhaselementrole{результат}{Результатом действия является матрица, полученная в результате свертки входной матрицы с ядром свертки.}

\scnheader{входная матрица\scnrolesign}
\scnrelfrom{первый домен}{действие интерпретации и.н.с.}
\scnrelfrom{второй домен}{матрица}

\scnheader{ядро свертки\scnrolesign}
\scnrelfrom{первый домен}{действие интерпретации сверточного слоя}
\scnrelfrom{второй домен}{матрица}

\scnheader{шаг свертки\scnrolesign}
\scnrelfrom{первый домен}{действие интерпретации сверточного слоя}
\scnrelfrom{второй домен}{число}

\scnheader{Действие интерпретации пулинг слоя}
\scntext{примечание}{Аргументы этого действия задаются следующими отношениями: \textit{шаг окна пулинга\scnrolesign}, \textit{размер окна пулинга\scnrolesign}, \textit{входная матрица\scnrolesign}}
\scnhaselementrole{результат}{матрица, полученная в результате пулинга входной матрицы.}

\scnheader{входная матрица\scnrolesign}
\scnrelfrom{первый домен}{действие интерпретации и.н.с.}
\scnrelfrom{второй домен}{матрица}

\scnheader{размер окна пулинга\scnrolesign}
\scnrelfrom{первый домен}{действие интерпретации пулинг слоя}
\scnrelfrom{второй домен}{матрица}

\scnheader{шаг окна пулинга\scnrolesign}
\scnrelfrom{первый домен}{действие интерпретации пулинг слоя}
\scnrelfrom{второй домен}{число}

\scnheader{интерпретатор искусственных нейронных сетей}
\scntext{примечание}{Спецификация агентов, соответствующих указанным действиям, задает агентно-ориентированную модель интерпретации искусственных нейронных сетей. Реализация этой модели будет называться интерпретатором искусственных нейронных сетей}
\scntext{примечание}{Реализация интерпретатора описанных в данной главе действий по построению \textit{и.н.с.} и описания в базе знаний экспертных знаний разработчиков\textit{и.н.с.} (а значит реализация интеллектуальной среды проектирования \textit{и.н.с.}) позволит автоматически, исходя из описания задачи, генерировать нейросетевые методы в памяти \textit{ostis-системы}, что является одним из ключевых направлений дальнейшего развития конвергенции и интеграции и.н.с. с базами знаний.}

\scnheader{Рисунок. Решение задачи \scnqq{ИСКЛЮЧАЮЩЕЕ ИЛИ}}
\scntext{примечание}{Рассмотрим пример описания \textit{нейросетевого метода}, решающего задачу, которая формулируется следующим образом: вычислить результат логической операции \scnqq{ИСКЛЮЧАЮЩЕЕ ИЛИ} для значений двух логических переменных. На рисунке представлено решение этой задачи с помощью сигнальной функции.}
\scnrelfrom{описание примера}{\scnfileimage[30em]{Contents/part_ps/src/images/sd_ps/sd_ann/strong_or_graphic.png}}

\scnheader{однослойный персептрон}
\scntext{примечание}{В работе описан однослойный персептрон, решающий поставленную задачу. Персептрон состоит из двух входных нейронов и одного выходного, с заданным порогом в 0,5 и сигнальной функцией активации:
	\begin{equation*}
		F(S) =
	 	\begin{cases}
	 		1, 0 < S < 0,\\
	 		0, else
	 	\end{cases}
	 \end{equation*}}
\begin{scnindent}
	\begin{scnrelfromset}{источник}
		\scnitem{\scncite{Golovko2017}}
	\end{scnrelfromset}
\end{scnindent}

\scnheader{Рисунок. Схема однослойного персептрона, решающего задачу \scnqq{ИСКЛЮЧАЮЩЕЕ ИЛИ}}
\scnrelfrom{описание примера}{\scnfileimage[30em]{Contents/part_ps/src/images/sd_ps/sd_ann/strong_or_ann.png}}

\scnheader{Рисунок. Метод, решающий задачу \scnqq{ИСКЛЮЧАЮЩЕЕ ИЛИ}, представленный с помощью языка представления нейросетевых методов SCP}
\scnrelfrom{описание примера}{\scnfileimage[30em]{Contents/part_ps/src/images/sd_ps/sd_ann/exclusive_or_ann_scp.png}}

\scnheader{SCg-текст. Представление сигнальной функции активации в памяти ostis-системы}
\scnrelfrom{описание примера}{\scnfileimage[30em]{Contents/part_ps/src/images/sd_ps/sd_ann/signal_function_def.png}}
\scntext{примечание}{Весовые коэффициенты синапсов входного слоя равны 1. На рисунке \textit{Рисунок. Схема однослойного персептрона, решающего задачу \scnqq{ИСКЛЮЧАЮЩЕЕ ИЛИ}} представлена схема персептрона.
	\\Данному персептрону соответствует метод, представленный в базе знаний ostis-системы на описанном в этой главе языке представления нейросетевых методов SCP. Данный метод представлен на рисунке \textit{Рисунок. Метод, решающий задачу \scnqq{ИСКЛЮЧАЮЩЕЕ ИЛИ}, представленный с помощью языка представления нейросетевых методов SCP}.
	\\Описание метода состоит из последовательности двух обобщенных спецификаций действий --- действия вычисления взвешенной суммы всех нейронов слоя и действия вычисления функции активации для всех нейронов слоя.
	\\Сигнальная функция активации, использующаяся в персептроне, в памяти ostis-системы определяется логической формулой, представленной на рисунке \textit{SCg-текст. Представление сигнальной функции активации в памяти ostis-системы}.}



\scnsectionheader{Логико-семантическая модель ostis-системы автоматизации проектирования искусственных нейронных сетей, семантически совместимых с базами знаний ostis-систем}
\begin{scnsubstruct}

\scnheader{Логико-семантическая модель ostis-системы автоматизации проектирования искусственных нейронных сетей}
\begin{scnhaselementrolelist}{класс объектов исследования}
	\scnitem{действие трансляции условия задачи}
	\scnitem{действие классификации задачи}
	\scnitem{действие поиска подходящей обучающей выборки}
	\scnitem{действие формирования требований к обучающей выборке}
	\scnitem{действие очистки выборки}
	\scnitem{действие выявления содержательных признаков}
	\scnitem{действие трансформации выборки}
	\scnitem{действие разбиения выборки}
	\scnitem{действие выбора класса нейросетевых методов}
	\scnitem{действие формирования спецификации входов и выходов и.н.с.}
	\scnitem{действие выбора метода оптимизации}
	\scnitem{действие выбора минимизируемой функции ошибки}
	\scnitem{действие начальной инициализации и.н.с.}
	\scnitem{действие выбора гиперпараметров и.н.с.}
	\scnitem{метод обучения с учителем}
	\scnitem{метод обучения без учителя}
	\scnitem{действие обучения и.н.с.}
\end{scnhaselementrolelist}

\scnheader{Язык представления нейросетевых методов в базах знаний}
\scniselement{язык представления методов}
\scntext{примечание}{Наличия \textit{Языка представления нейросетевых методов в базах знаний} и его интерпретатора позволяет обеспечить интерпретацию \textit{нейросетевого метода} в памяти \textit{ostis-системы}. Наличие в единой памяти не только экземпляров методов, но и понятий, их описывающих, создает основу для автоматизации процесса построения нейросетевых методов. 
	\\В памяти \textit{ostis-системы} хранятся знания о том, методы какого класса могут решить задачу заданного класса, но экземпляров класса этого метода может не быть представлено в системе. На этот случай система должна иметь возможность сообщить пользователю о возможности решения, для которого, однако, необходимо погрузить в систему определенный метод. Так как система хранит в единой памяти задачу и требования к методу ее решения, появляется возможность спроектировать необходимый метод. Для этого необходимо наличие среды проектирования методов соответствующих классов. В случае \textit{нейросетевого метода}, речь идет об интеллектуальной среде построения \textit{нейросетевых методов}.}

\scnheader{нейросетевой метод}
\scntext{примечание}{В основе интеллектуальной среды построения \textit{нейросетевых методов} лежат соответствующие другу другу иерархии действий, задач и методов построения \textit{и.н.с.} Наличие такой иерархии позволит описать язык представления методов построения \textit{и.н.с.} и разработать интерпретатор этого языка.}
\scntext{примечание}{Построение иерархии соответствующих действий построения \textit{и.н.с.} следует начать с изучения этапов проектирования и обучения \textit{и.н.с.}, которые, в общем случае, выполняют все разработчики и.н.с.:
	\\1. Постановка задачи\\
	2. Предобработка выборки: очистка\\
	3. Предобработка выборки: выявление содержательных признаков\\
	4. Предобработка выборки: трансформация\\
	5. Разбиение выборки на обучающую, валидационную и тестовую (контрольную)\\
	6. Выбор класса нейросетевых методов в соответствии со сформулированной задачей\\
	7. Формирование спецификации на входные и выходные данные\\
	8. Выбор метода оптимизации\\
	9. Выбор минимизируемой функции ошибки\\
	10. Начальная инициализация параметров нейронной сети\\
	11. Выбора гиперпараметров и.н.с.\\
	12. Обучение модели на обучающей выборке\\
	13. Оценка эффективности и.н.с}

\scnheader{Постановка задачи}
\scntext{примечание}{Постановка задачи включает в себя описание входных данных (изображения/видео, временные ряды, текст), выходных данных и требований к методу решения (скорость, затраты по памяти и так далее). Также описывается дополнительная информация, которая может помочь в построении метода решения задачи (к примеру, спецификация обучающей выборки, если таковая имеется). Обычно, на данном этапе разработчик и.н.с. определяет класс задачи, формирует требования к обучающей выборке, если она не предоставлена.
	\\Выполнение данного этапа средой проектирования \textit{и.н.с.} подразумевает выполнение следующих действий:
	\begin{itemize}
		\item \textbf{\textit{действие трансляции условия задачи}}. Действие транслирует заданное с помощью \textit{интерфейса ostis-системы} (к примеру, естественно-языкового интерфейса) описание задачи в память ostis-системы. Действие необходимо в случае, когда условие задачи задается пользователем. Необходимо понимать, что описание задачи поступает в базу знаний не только от \textit{пользовательского интерфейса}. К примеру, задача может быть сформулирована самой системой в ходе ее жизнедеятельности.
		Данное действие является общим для всех ostis-систем, поэтому его рассмотрение выходит за рамки рассмотрения процесса построения интеллектуальной среды проектирования \textit{и.н.с.}
		\item \textbf{\textit{действие классификации задачи}}. Действие определяет класс задачи (задача регрессии, детекции, кластеризации и так далее), исходя из описания задачи в базе знаний.
		\item \textbf{\textit{действие поиска подходящей обучающей выборки}}. В базе знаний может храниться набор спецификаций выборок, к которым у ostis-системы есть доступ. Действие производит поиск выборок, которые могут быть использованы в качестве обучающей выборки.
		\item \textbf{\textit{действие формирования требований к обучающей выборке}}. Если обучающая выборка не была предоставлена и не была найдена, то необходимо сформировать описание требований к обучающей выборке, которое можно будет транслировать на язык пользовательского интерфейса и запросить необходимую выборку у пользователя.
	\end{itemize}}

\scnheader{Предобработка выборки}
\scnhaselement{очистка}
\begin{scnindent}
	\scntext{примечание}{На этом этапе обнаруживаются признаки, которые имеют в общем случае некорректные значения (например, для каких-то образов значение признака может иметь неопределенное значение, либо значение, не совпадающее по типу, либо аномально большое или очень маленькое значение, которое встречается в редком числе случаев). Для признаков, имеющих неопределенное значение, может быть применены различные методы устранения, например, такие значения могут быть заменены средним значением этого признака, рассчитанным по всем образам (для непоследовательных данных), либо они могут быть заменены средним значением по соседним образам (в случае временных рядов), либо каким-то фиксированным значением. Радикальная мера решения проблемы --- удаление образов, имеющих неопределенные значения признаков из выборки. Однако его лучше применять, если образов с отсутствующими значениями признаков немного. Для выбросов и аномалий применяются схожие стратегии (но только в том случае, если задача не состоит в прогнозировании этих аномалий).
		\\В интеллектуальной среде проектирования данный этап соответствует выполнению \textbf{\textit{действия очистки выборки}}, которое выполняется в случае обработки выборки, которая ранее не была представлена в памяти системы (к примеру, была получена от пользователя).
		\\Реализация интерпретатора (агента) данного действия требует описания в памяти классификации стратегий очистки данных и реализации методов применения этих стратегий.}
\end{scnindent}
\scnhaselement{выявление содержательных признаков}
\begin{scnindent}
	\scntext{примечание}{Осуществляется инжиниринг признаков, состоящий в отборе признаков, влияющих на результат работы модели, несодержательные признаки, которые никак не коррелируют с выходом модели, удаляются. Цель этого этапа --- уменьшение размерности пространства признаков для снижения влияния эффекта переобучения на модель.
		\\Для снижения размерности признакового пространства может применяться методы отбора признаков и выделения признаков.
		\\При отборе признаков, осуществляется формирование подмножества из исходных признаков (алгоритм последовательного обратного отбора, рекурсивный алгоритм обратного устранения признаков,  алгоритмы с использованием случайных лесов).
		\\При выделении признаков из набора признаков извлекается информация для построения нового подпространства признаков (алгоритмы с использованием автоэнкодера).
		\\В интеллектуальной среде проектирования данный этап соответствует выполнению \textbf{\textit{действия выявления содержательных признаков}}. Реализация интерпретатора (агента) данного действия требует описания в памяти классификации стратегий уменьшения размерности признакового пространства и реализации методов применения этих стратегий.}
\end{scnindent}
\scnhaselement{трансформация}
\begin{scnindent}
	\begin{scnrelfromvector}{примечание}
		\scnfileitem{На этом этапе осуществляется подготовка данных к обучению.}
		\scnfileitem{Здесь следует уделить особое внимание наличию категориальных признаков, чаще всего заданных строковыми типами. Эти признаки могут быть номинальными и порядковыми. Для кодирования порядковых признаков чаще всего применяют последовательный числовой код (1, 2, 3,...). Для кодирования номинальных такое решение неверно, так как эти признаки равноправны и не могут сравниваться по числовому коду (например, пол --- 0/1). Для номинальных признаков применяется способ прямого кодирования, заключающийся в создании и использовании фиктивных признаков по количеству значений исходного. Например, признак пол (мужской, женский) преобразуется в два новых признака мужской и женский с соответствующими значениями для имеющихся образов.}
		\scnfileitem{Масштабирование признаков предполагает приведение значений признаков к одному общему интервалу --- это особенно актуально для признаков, имеющих несоразмерные выборочные средние значения по всем образам --- например, один признак в среднем имеет значение 10.000, а другой 12. Это может проявится в выполнении минимизации только по признаку с наибольшими значениями и плохой сходимости метода обучения. Чаще всего масштабирование соответствует выполнению нормализации на отрезок (min-max нормализация)}
		\begin{scnindent}
			\scnrelfrom{формула}{
				\begin{equation*}
					x_{norm}^i = \frac{x^i - x_{min}}{x_{max} - x_{min}}
				\end{equation*}}
			\begin{scnindent}
				\scntext{примечание}{$x^i$ --- значение признака для отдельно взятого образа \textit{i}, $x_{min}$ --- наименьшее значение для признака, $x_{max}$ --- наибольшее значение для признака.}
			\end{scnindent}
		\end{scnindent}
		\scnfileitem{Другой вариант масштабирования --- применение стандартизации признаков}
		\begin{scnindent}
			\scnrelfrom{формула}{
				\begin{equation*}
					x_{std}^i = \frac{x^i - \mu(x)}{\sigma(x)}
				\end{equation*}}
			\begin{scnindent}
				\scntext{примечание}{$\mu(x)$ --- выборочное среднее отдельного признака, $\sigma(x)$ --- стандартное отклонение.}
			\end{scnindent}
		\end{scnindent}
		\scnfileitem{Стандартизация сохраняет полезную информацию о выбросах в исходных данных и делает алгоритм обучения менее чувствительным к ним.}
		\scnfileitem{Дискретизация применяется для перехода от вещественного признака к порядковому за счет кодирования интервалов одним значением (например, если признак отражает возраст человека, то может быть произведена дискретизация значений с выделением определенных возрастных групп, где каждая группа будет кодироваться одним целым числом).}
		\scnfileitem{В интеллектуальной среде проектирования данный этап соответствует выполнению \textbf{\textit{действия трансформации выборки}}. Реализация интерпретатора (агента) данного действия требует описания в памяти классификации методов масштабирования признаков и реализации методов применения этих стратегий.}
	\end{scnrelfromvector}
\end{scnindent}

\scnheader{Разбиение выборки на обучающую, валидационную и тестовую (контрольную)}
\scntext{примечание}{Производится разбиение всей выборки данных, на обучающую, тестовую и, в некоторых случаях, валидационную.
	\\Валидационная выборка используется для оценки влияния изменения гиперпараметров на результат обучения и может применяться как дополнительный инструмент для этого наравне с сеточным поиском.
	\\Разбиение проводится в соотношении 3:1:1, в процентах (60/20/20), если валидационная выборка не используется, то 80/20.
	\\В интеллектуальной среде проектирования данный этап соответствует выполнению \textbf{\textit{действия разбиения выборки}}.
	\\Все предыдущие этапы применялись к выборке, последующие этапы относятся к используемым моделям и.н.с.}

\scnheader{Выбор класса нейросетевых методов в соответствии со сформулированной задачей}
\scntext{примечание}{На этом этапе осуществляется выбор основной архитектуры и.н.с., которая будет использоваться при обучении. Однако, нужно отметить, что этот выбор относительно условный, то есть исследователь не ограничен использованием только одного типа и.н.с. для решения задачи (как, например, сверточной сети для изображений, поскольку изображения можно обрабатывать и обычным многослойным персептроном). Речь скорее идет именно о рекомендованной архитектуре, но это не исключает использование любых других вариантов архитектур и их сочетаний в рамках одной модели).
	\\Примерами таких рекомендаций являются:
	\begin{itemize}
		\item изображения/видео --- сверхточные нейронные сети;
		\item временные ряды --- многослойные персептроны или рекуррентные сети;
		\item текстовая информация --- многослойные персептроны или рекуррентные сети;
		\item наборы характеристик некоторых объектов (например, спецификации автомобилей) --- многослойный персептрон.
	\end{itemize}
	В интеллектуальной среде проектирования данный этап соответствует выполнению \textbf{\textit{действия выбора класса нейросетевых методов}}.}

\scnheader{Формирование спецификации на входные и выходные данные}
\scntext{примечание}{Выполняются дополнительные преобразования данных, связанные с изменением структур хранения (например, преобразование многомерного массива в одномерный, конвертация типов)
	\\В интеллектуальной среде проектирования данный этап соответствует выполнению \textbf{\textit{действия формирования спецификации входов и выходов и.н.с.}}.}

\scnheader{Выбор метода оптимизации}
\scntext{примечание}{В рамках ПрО и.н.с. описаны следующие методы оптимизации:
	\begin{itemize}
		\item стохастический градиентный спуск (stochastic gradient descent --- SGD);
		\item метод Нестерова;
		\item адаптивный градиент (adaptive gradient --- AdaGrad);
		\item адаптивная оценка момента (adaptive moment estimation --- Adam);
		\item среднеквадратическое распространение (root mean square propagation --- RMSProp).
	\end{itemize}
	В интеллектуальной среде проектирования данный этап соответствует выполнению \textbf{\textit{действия выбора метода оптимизации}}.}

\scnheader{Выбор минимизируемой функции ошибки}
\begin{scnrelfromvector}{примечание}
	\scnfileitem{На этом этапе задается функция ошибок, которая будет минимизироваться. К примеру, MSE лучше подходит для задач регрессии и для кластеризации, CE --- для классификационных задач. Далее приведены примеры.}
	\scnfileitem{
		\begin{equation*}
			MSE = \frac{1}{n} \sum_{i=1}^n (Y_i - \widetilde{Y_i})^2
		\end{equation*}}
		\begin{scnindent}
			\scntext{примечание}{\textit{n} --- размер обучающей выборки, $Y_i$ --- эталонное значение функции, $\widetilde{Y_i}$ --- результат, полученный НС}
		\end{scnindent}
	\scnfileitem{
		\begin{equation*}
		CE = - \frac{1}{n} \sum_{i=1}^n (Y_i\log(\widetilde{Y_i}) + (1-Y_i)\log(1 - \widetilde{Y_i}))
		\end{equation*}}
		\begin{scnindent}
			\scntext{примечание}{\textit{n} --- размер обучающей выборки, $Y_i$ --- эталонное значение функции, $\widetilde{Y_i}$ --- результат, полученный НС.
				\\(случай 2-классовой классификации)}
		\end{scnindent}
	\scnfileitem{
		\begin{equation*}
			CE = - \frac{1}{n} \sum_{i=1}^n \sum_{c=1}^M Y_i^c \log{\widetilde{Y}_i^c}
		\end{equation*}}
		\begin{scnindent}
			\scntext{примечание}{случай многоклассовой классификации}
		\end{scnindent}
	\scnfileitem{В интеллектуальной среде проектирования данный этап соответствует выполнению \textbf{\textit{действия выбора минимизируемой функции ошибки}}.}
\end{scnrelfromvector}
	
\scnheader{Начальная инициализация параметров нейронной сети}
\scntext{пример}{Наиболее часто используемые варианты инициализации весовых коэффициентов и порогов нейронной сети}
\begin{scnindent}
	\begin{scnrelfromset}{разбиение}
		\scnfileitem{Инициализация значениями из равномерного распределения на каком-то небольшом интервале, например, [-0.1, 0.1].}
		\scnfileitem{Инициализация значениями из стандартного нормального распределения.}
		\scnfileitem{Инициализация по методу Ксавье.}
		\begin{scnindent}
			\begin{scnrelfromset}{источник}
				\scnitem{\scncite{Glorot2010}}
			\end{scnrelfromset}
			\scntext{примечание}{Применяется для предотвращения резкого уменьшения или увеличения значений выхода нейронных элементов после применения функции активации при прямом прохождении образа через глубокую нейронную сеть. Фактически инициализация этим методом осуществляется посредством выбора значений из равномерного распределения на отрезке $[- \sqrt{6} / \sqrt{n_i+n_{i+1}}, \sqrt{6} / \sqrt{n_i+n_{i+1}}]$, где $n_i$ --- это число входящих связей в данный слой, а $n_i$ --- число исходящих связей из данного слоя. Таким образом, инициализация этим методом проводится для разных слоев нейронной сети из разных отрезков.}
		\end{scnindent}
		\scnfileitem{Инициализация, полученная из предобученной модели.}
		\begin{scnindent}
			\begin{scnrelfromset}{источник}
				\scnitem{\scncite{Glorot2010}}
			\end{scnrelfromset}
			\scntext{примечание}{Вариант инициализации, который предполагает использование в качестве \scnqq{стартовой} модели предобученной модели, взятой из некоторого репозитория предобученных моделей, обученную самим исследователем или в процессе работы интеллектуальной системы.}
		\end{scnindent}
		\scnfileitem{Инициализация по методу Кайминга.}
		\begin{scnindent}
			\begin{scnrelfromset}{источник}
				\scnitem{\scncite{He2015}}
			\end{scnrelfromset}
			\scntext{примечание}{Данный метод инициализации применяется для решения проблемы \scnqq{затухающего} градиента и \scnqq{взрывающегося} градиента. Производится посредством выбора значений из равномерного распределения на отрезке: $$[-\sqrt{2} / \sqrt{(1+a^2)fan}, \sqrt{2} / \sqrt{(1+a^2)fan}],$$ где \textit{a} --- угол наклона к оси абсцисс для отрицательной части области определения функции активации типа ReLU (для обычной ReLU функции этот параметр равен 0), $fan$ --- параметр режима работы, который для фазы прямого распространения равен количеству входящих связей (для устранения эффекта \scnqq{взрывающегося} градиента), а для фазы обратного распространения --- количеству выходящих (для устранения эффекта \scnqq{затухающего} градиента).
				\\В интеллектуальной среде проектирования данный этап соответствует выполнению \textbf{\textit{действия начальной инициализации и.н.с.}}.}
		\end{scnindent}
	\end{scnrelfromset}
\end{scnindent}

\scnheader{Выбора гиперпараметров и.н.с.}
\scntext{примечание}{На практике некоторые гиперпараметры (такие как количество слоев, их типы, количество нейронов в слое) часто определяются экспериментально, в процессе итеративного поиска лучшего варианта решения задачи. Хотя способы частично автоматизировать этот процесс существуют, они все же рассчитаны на наличие некоторых предусловий проведения эксперимента, в частности интервалов изменения параметра (например, скорости обучения).
	\\К гиперпараметрам, подбираемым на этом этапе, относятся:
	\begin{itemize}
		\item параметры обучения \textit{и.н.с.} (скорость обучения, моментный параметр, размер мини-батча);
		\item архитектура модели \textit{и.н.с.}, опирающаяся на ранее сформулированные спецификации входных и выходных данных (например, количество нейронов в определенном слое (слоях) или конфигурации целых слоев).
	\end{itemize}
	Нахождение оптимальных гиперпараметров может быть получено, например, использованием метода сеточного поиска, который позволяет проверить значения гиперпараметров, взятые с определенным шагом или из определенного интервала (кортежа). С помощью этого метода выбирается оптимальный набор гиперпараметров, который дает лучшие результаты, он используется для последующего дообучения. Или же, если полученные результаты являются приемлемыми, то процесс дальнейшего обучения вообще не проводится. Следует отметить затратность данного метода, так как фактически осуществляется перебор различных значений параметров обучения. Для снижения объема работы применяется метод случайного поиска.
	\\Для оптимизации архитектуры определяются типы слоев нейронной сети, количество нейронных элементов в каждом слое, их характеристики --- функция активации, для сверточных элементов --- размер ядра, а также параметры padding и шаг свертки (stride).
	\\Здесь же может осуществляться оценка не только пользовательского варианта сети, но и предобученной архитектуры. Основное правило при выборе --- количество параметров модели не должно превышать размер обучающей выборки. Для предобученных архитектур это ограничение снимается.
	\\В интеллектуальной среде проектирования данный этап соответствует выполнению \textbf{\textit{действия выбора гипперпараметров и.н.с.}}. Действие использует классификацию и спецификации гиперпараметров \textit{и.н.с.}}

\scnheader{Обучение модели на обучающей выборке}
\scntext{примечание}{Производится обучение модели до достижения выбранной точности (оценивается на тестовой выборке) или по другим заданным критериям (достижение заданного количества эпох обучения, неизменность точности на протяжении заданного количества эпох, падение точности на валидационной выборке и так далее).}
\scntext{примечание}{В интеллектуальной среде проектирования данный этап соответствует выполнению \textbf{\textit{действия обучения и.н.с.}}. Действие обучения \textit{и.н.с.} --- действие, в ходе которого реализуется определенный метод обучения \textit{и.н.с.} с заданными параметрами обучения \textit{и.н.с.}, методом оптимизации и функцией потерь.
	\\При обучении возможно возникновение следующих проблем:
	\begin{itemize}
		\item \textit{переобучение} --- проблема, возникающая при обучении \textit{и.н.с.}, заключающаяся в том,
		что сеть хорошо адаптируется к паттернам входной активности из обучающей выборки, при этом теряя способность к обобщению.
		Переобучение возникает из-за применения неоправданно сложной модели при обучении \textit{и.н.с.} Это происходит,
		когда количество настраиваемых параметров \textit{и.н.с.} намного больше размера обучающей выборки. Возможные
		варианты решения проблемы заключаются в упрощении модели, увеличении выборки, использовании регуляризации
		(параметр регуляризации, техника dropout и так далее).\\
		Обнаружение переобученности сложнее, чем недообученности. Как правило, для этого применяется
		кросс-валидация на валидационной выборке, позволяющая оценить момент завершения процесса обучения.
		Идеальным вариантом является достижение баланса между переобученностью и недообученностью.
		\item \textit{недообучение} --- проблема, возникающая при обучении  \textit{и.н.с.}, заключающаяся в том,
		что сеть дает одинаково плохие результаты на обучающей и контрольной выборках.
		Чаще всего такого рода проблема возникает при недостаточном времени, затраченном на обучение модели.
		Однако это может быть вызвано и слишком простой архитектурой модели либо малым размером обучающей
		выборки. Соответственно решение, которое может быть принято ML-инженером, заключается в устранении
		этих недостатков: увеличение времени обучения, использование модели с большим числом настраиваемых
		параметров, увеличение размера обучающей выборки, а также уменьшение регуляризации и более тщательный
		отбор признаков для обучающих примеров.
	\end{itemize}}

\scnheader{метод обучения и.н.с.}
\scnsubset{метод}
\scnrelfrom{разбиение}{Классификация алгоритмов обучения}
\begin{scnindent}
	\begin{scneqtoset}
		\scnitem{метод обучения с учителем}
		\begin{scnindent}
			\scntext{пояснение}{\textbf{\textit{метод обучения с учителем}} --- метод обучения с использованием заданных целевых переменных.}
			\scnsuperset{метод обратного распространения ошибки}
			\begin{scnindent}
				\scnidtf{м.о.р.о.}
				\scntext{пояснение}{м.о.р.о. использует заданный метод оптимизации и заданную функцию потерь для реализации фазы обратного распространения ошибки и изменения настраиваемых параметров и.н.с. Одним из самых распространенных	методов оптимизации является метод стохастического градиентного спуска.}
				\scntext{пояснение}{Следует также отметить, что несмотря на то, что метод отнесен к методам обучения с учителем, в случае	использования м.о.р.о. для обучения автокодировщиков в классических публикациях он рассматривается как	метод обучения без учителя, поскольку в данном случае размеченные данные отсутствуют.}
			\end{scnindent}
		\end{scnindent}
		\scnitem{метод обучения без учителя}
		\begin{scnindent}
			\scntext{пояснение}{\textbf{\textit{метод обучения без учителя}} --- метод обучения без использования заданных целевых переменных(в режиме самоорганизации)}
			\scntext{пояснение}{В ходе выполнения алгоритма метода обучения без учителя выявляются полезные структурные свойства набора. Неформально его понимают как метод для извлечения информации из распределения, выборка для которого	не была вручную аннотирована человеком.}
			\begin{scnindent}
				\begin{scnrelfromset}{источник}
					\scnitem{\scncite{Goodfellow2017}}
				\end{scnrelfromset}
			\end{scnindent}
		\end{scnindent}
	\end{scneqtoset}
\end{scnindent}

\scnheader{метод обучения \textit{и.н.с.}}
\scntext{определение}{метод обучения \textit{и.н.с.} --- это процесс итеративного поиска оптимальных значений настраиваемых параметров \textit{и.н.с.}, минимизирующих некоторую заданную функцию потерь.}
\scntext{примечание}{Стоит отметить, что хотя целью применения метода обучения является минимизация функции потерь, \scnqq{полезность} полученной после обучения модели можно оценить только по достигнутому уровню ее обобщающей способности.}
\scntext{примечание}{\\Методы обучения могут быть поделены на две большие группы --- \textit{\textbf{методы обучения с учителем}} и \textit{\textbf{методы обучения без учителя}} (контролируемый и неконтролируемый методы обучения).
	\\\textit{метод обучения с учителем} --- метод обучения с использованием заданных целевых переменных.
	\\Одним из методов обучения с учителем является метод обратного распространения ошибки.}

\scnheader{метод обратного распространения ошибки}
\scntext{описание}{Приведем его описание в виде алгоритма:
	\begin{algorithm}[H]
		\KwData{$X$ --- данные, $E_t$ --- желаемый отклик (метки), $E_m$ --- желаемая ошибка (в соответствии с выбранной функцией потерь)}
		\KwResult{обученная нейронная сеть \textit{Net}}
		инициализация весов \textit{W} и порогов \textit{T};\\
		\Repeat{$E<E_m$}{
			\ForEach{$x \in X$, $e \in E_t$}{
				фаза прямого распространения сигнала: вычисляются активации для всех слоев и.н.с.;\\
				фаза обратного распространения ошибки: вычисляются ошибки для последнего слоя и всех предшествующих слоев;\\
				изменение настраиваемых параметров и.н.с. в соответствии с вычисленными ошибками;\\
			}
			вычисление общей ошибки E на данной эпохе;
		}
	\end{algorithm}
	\textit{метод обратного распространения ошибки} использует заданный метод оптимизации и заданную функцию потерь для реализации фазы обратного распространения ошибки и изменения настраиваемых параметров и.н.с. Одним из самых распространенных методов оптимизации является метод стохастического градиентного спуска. Приведенный метод используется для реализации последовательного варианта обучения.}
\scntext{примечание}{Следует также отметить, что несмотря на то, что метод отнесен к методам обучения с учителем, в случае его использования для обучения автокодировщиков в классических публикациях он рассматривается как метод обучения без учителя, поскольку в данном случае размеченные данные отсутствуют.}


\scnheader{метод обучения без учителя}
\scniselement{метод обучения}
\scntext{определение}{метод обучения без учителя --- это метод обучения без использования заданных целевых переменных (в режиме самоорганизации)}
\scntext{примечание}{В ходе выполнения алгоритма метода обучения без учителя выявляются полезные структурные свойства набора. Неформально его понимают как метод для извлечения информации из распределения, выборка для которого не была вручную аннотирована человеком. Метод обучения без учителя может рассматриваться как вспомогательный метод для начальной инициализации настраиваемых параметров и.н.с. В этом случае он является методом предобучения.}
\begin{scnindent}
	\begin{scnrelfromset}{источник}
		\scnitem{\scncite{Goodfellow2017}}
	\end{scnrelfromset}
\end{scnindent}

\scnheader{целевая функция}
\begin{scnrelfromvector}{методы оптимизации}
		\scnfileitem{SGD (стохастический градиентный спуск). В данном методе корректировка настраиваемых параметров и.н.с. выполняется в направлении максимального уменьшения функции стоимости, то есть в направлении, противоположном вектору градиента функции потерь.}
		\begin{scnindent}
			\begin{scnrelfromset}{источник}
				\scnitem{\scncite{Golovko2017}}
				\scnitem{\scncite{Haykin2006}}
			\end{scnrelfromset}
		\end{scnindent}
		\scnfileitem{Метод Нестерова. Обучение методом стохастического градиентного спуска не редко происходит очень медленно. Импульсный метод позволяет ускорить обучение, особенно в условиях высокой кривизны, небольших, но устойчивых градиентов или зашумленных градиентов. В импульсном методе вычисляется экспоненциально затухающее скользящее среднее прошлых градиентов и продолжается движение в этом направлении. Метод Нестерова является вариантом импульсного алгоритма, в котором градиент вычисляется после применения текущей скорости.}
		\begin{scnindent}
			\begin{scnrelfromset}{источник}
				\scnitem{\scncite{Goodfellow2017}}
			\end{scnrelfromset}
		\end{scnindent}
		\scnfileitem{AdaGrad: данный метод по отдельности адаптирует скорости обучения всех настраиваемых параметров и.н.с., умножая их на коэффициент, обратно пропорциональный квадратному корню из суммы всех прошлых значений квадрата градиента.}
		\begin{scnindent}
			\begin{scnrelfromset}{источник}
				\scnitem{\scncite{Duchi2011}}
			\end{scnrelfromset}
		\end{scnindent}
		\scnfileitem{RMSProp. Данный метод является модификацией AdaGrad, которая позволяет улучшить его поведение в невыпуклом случае путем изменения способа агрегирования градиента на экспоненциально взвешенное скользящее среднее. Использование экспоненциально взвешенного скользящего среднего гарантирует повышение скорости сходимости после обнаружения выпуклой впадины, как если бы внутри этой впадины алгоритм AdaGrad был инициализирован заново}
		\begin{scnindent}
			\begin{scnrelfromset}{источник}
				\scnitem{\scncite{Goodfellow2017}}
			\end{scnrelfromset}
		\end{scnindent}
		\scnfileitem{Adam. Данный метод можно рассматривать как комбинацию RMSProp и AdaGrad. Помимо усредненного первого момента, данный метод использует усредненное значение вторых моментов градиентов.}
		\begin{scnindent}
			\begin{scnrelfromset}{источник}
				\scnitem{\scncite{Kingma2014}}
			\end{scnrelfromset}
		\end{scnindent}
\end{scnrelfromvector}
\scntext{примечание}{Отметим, что успешность применения методов оптимизации зависит главным образом от знакомства пользователя с соответствующим алгоритмом.}
\begin{scnindent}
	\begin{scnrelfromset}{источник}
		\scnitem{\scncite{Goodfellow2017}}
	\end{scnrelfromset}
\end{scnindent}

\scnheader{функция потерь}
\scntext{примечание}{важный компонент, влияющий на процесс обучения нейросетевой модели}
\scntext{определение}{функция потерь --- это функция, используемая для вычисления ошибки, рассчитываемой как разница между фактическим эталонным значением и прогнозируемым значением, получаемым \textit{и.н.с.}}
\begin{scnrelfromvector}{примечание}
	\scnfileitem{Среди функций потерь, используемые в качестве целевых функций для применяемого метода оптимизации, можно выделить MSE, BCE, MCE}
	\scnfileitem{MSE --- средняя квадратичная ошибка}
	\begin{scnindent}
		\scnrelfrom{формула}{
			\begin{equation*}
				MSE = \frac{1}{L} \sum_{l=1}^L \sum_{i=1}^m (y_i^l - e_i^l)^2
			\end{equation*}}
		\begin{scnindent}
			\scntext{примечание}{$y_i^l$ --- прогноз модели, $e_i^l$ --- ожидаемый (эталонный) результат, \textit{m} --- размерность выходного вектора, \textit{L} --- объем обучающей выборки}
		\end{scnindent}
	\end{scnindent}
	\scnfileitem{BCE --- бинарная кросс-энтропия (binary cross-entropy)}
	\begin{scnindent}
		\scnrelfrom{формула}{
			\begin{equation*}
				BCE = - \sum_{l=1}^L (e^l \log(y^l) + (1 - e^l)\log(1 - y^l))
			\end{equation*}}
		\begin{scnindent}
			\scntext{примечание}{$y^l$ --- прогноз модели, $e^l$ --- ожидаемый (эталонный) результат: \textit{0} или \textit{1}, \textit{L} --- объем обучающей выборки}
		\end{scnindent}
	\end{scnindent}
	\scnfileitem{MCE --- мультиклассовая кросс-энтропия (multiclass cross-entropy)}
	\begin{scnindent}
		\scnrelfrom{формула}{
			\begin{equation*}
				MCE = - \sum_{l=1}^L \sum_{i=1}^m e_{i}^l \log(y_{i}^l)
			\end{equation*}}
		\begin{scnindent}
			\scntext{примечание}{$y_{i}^l$ --- прогноз модели, $e_i^l$ --- ожидаемый (эталонный результат), \textit{m} --- размерность выходного вектора}
		\end{scnindent}
	\end{scnindent}
\end{scnrelfromvector}

\scnheader{бинарная кросс-энтропия}
\scnidtf{BCE}
\scntext{примечание}{Отметим, что для бинарной кросс-энтропии в выходном слое \textit{и.н.с.} будет находиться один нейрон, а для для мультиклассовой кросс-энтропии количество нейронов в выходном \textit{слое и.н.с.} совпадает с количеством классов.}

\scnheader{задача классификации}
\scntext{примечание}{Для решения задачи классификации рекомендуется использовать бинарную или мультиклассовую кросс-энтропийную функцию потерь, для решения задачи регрессии рекомендуется использовать среднюю квадратичную ошибку.}

\scnheader{SCg-текст. Действие обучения и.н.с.}
\scnrelfrom{описание примера}{\scnfileimage[30em]{Contents/part_ps/src/images/sd_ps/sd_ann/ann_training_nn_scg.png}}
\begin{scnindent}
	\scntext{примечание}{Пример действия обучения \textit{и.н.с.}}
\end{scnindent}

\scnheader{Оценка эффективности и.н.с}
\scntext{примечание}{После выполнения обучения осуществляется оценка полученной модели с помощью метрик оценки качества.
	\\Далее результат оценки может быть визуализирован с помощью матрицы ошибок (confusion matrix) и ROC-кривой.}
\scntext{примечание}{В интеллектуальной среде проектирования данный этап соответствует выполнению \textit{действия оценки эффективности и.н.с.}.}

\scnheader{матрица ошибок}
\scntext{определение}{матрица ошибок --- это матрица, в которую помещены сведения о числе истинно-положительных, истинно-отрицательных, ложно-положительных и ложно-отрицательных предсказаниях классификатора.}
\scnrelfrom{смотрите}{Рисунок. Матрица ошибок}

\scnheader{Рисунок. Матрица ошибок}
\scnrelfrom{описание примера}{\scnfileimage[30em]{Contents/part_ps/src/images/sd_ps/sd_ann/conf_matrix.png}}

\scnheader{ROC-кривая}
\scnidtf{receiver operating characteristic}
\scntext{определение}{ROC-кривая --- это график, в котором, основываясь на заданном пороге решения классификатора, рассчитываются доли ложноположительных и истинно положительных исходов. Основываясь на ROC-кривой, высчитывается AUC-показатель (площадь под кривой), которая используется в качестве характеристики качества модели.}

\scnheader{Задача. Классификация цифр из выборки рукописных цифр MNIST}
\begin{scnrelfromvector}{пример}
	\scnfileitem{Рассмотрим пример выполнения описанных этапов разработчиком для конкретной задачи --- \textit{классификации цифр из выборки рукописных цифр MNIST}:
		\begin{itemize}
		\item Исходными данными задачи является: выборка из 70.000 изображений, предварительно разделенная на обучающую (60.000 изображений) и контрольную (10.000 изображений) выборки. Каждое изображение представлено двумерным массивом 28x28 чисел из интервала [0, 255], числа представляют определенный оттенок серого цвета. Помимо этого каждому изображению соответствует метка класса, соответствующая конкретной цифре от 0 до 9.
		\\Ставится задача: \textit{обучить модель, которая будет принимать на вход двумерный массив данных и возвращать метку класса, соответствующей распознанной цифре.}
		\\Таким образом, тип решаемой задачи --- \textbf{классификационная}, природа данных задачи --- \textbf{изображения}.
		\item В рассматриваемой выборке отсутствуют аномалии, ошибочные данные, признаки с отсутствующими значениями.
		\item В рассматриваемой задаче отсутствуют несодержательные признаки.
		\item В качестве метода предобработки данных используем масштабирование признаков, а именно нормализацию на отрезок [0, 1].
		\item Выполним разбиение обучающей части данных на обучающую и валидационную выборки в соотношении 4:1 (48.000 в обучающей и 12.000 в валидационной).
		\item Так как выборка включает в себя изображения, будем использовать сверточную нейронную сеть.
		\item Не требуется.
		\item В качестве оптимизационного алгоритма будем использовать метод стохастического градиентного спуска (SGD).
		\item Так как решается задача классификации, выберем в качестве минимизируемой функции кросс-энтропийную функцию потерь.
		\item В качестве начальной инициализации будем использовать инициализацию по методу Кайминга.
		\item На предыдущих этапах было определено, что для решения задачи будет использоваться сверточная нейронная сеть. При использовании one-hot кодирования в последнем полносвязном слое будет 10 нейронов по числу классов в задаче.
		\end{itemize}}
	\scnfileitem{Для упрощения будем использовать архитектуру, изображенную на \textit{Рисунок. Архитектура и.н.с., решающая задачу классификации цифр}, не содержащую промежуточные слои.}
	\scnfileitem{Для нахождения оптимального набора гиперпараметров будем применять метод случайного поиска.
		\\Перечислим кортежи, из которых будут сэмплироваться гиперпараметры:
		\begin{itemize}
			\item Скорость обучения --- (0.9, 0.1, 0.01, 0.001);
			\item Количество нейронов в сверточном слое --- (5, 10, 15, 20);
			\item Размер ядра свертки --- (3, 5, 7, 9);
			\item Моментный параметр --- (0, 0.5, 0.9);
			\item Размер мини-батча --- (16, 32, 64, 128).
		\end{itemize}}
	\scnfileitem{После определения данных параметров и оценки эффективности работы алгоритма, получим следующую таблицу: \textit{Таблица. Результаты решения задачи}}
	\scnfileitem{Можно заметить, что лучший результат (acc = 0.9839) по обобщающей способности на валидационной выборке был получен при следующих параметрах: mbs = 64, ks = 7, lr = 0.01, momentum = 0.9, cnc = 15.
		\begin{itemize}
		\item В качестве критерия останова нами был выбран самый простой критерий по достижению заданного количества эпох обучения. Дообучение не проводилось, для оценки обобщающей способности использовалась модель, полученная после выполнения процедуры подбора гиперпараметров. Обобщающая способность на тестовой выборке составила \textbf{0.9853}, то есть \textbf{98.53\%}.
		\item Построив матрицу ошибок на основании обученной модели и тестовой выборки, получим результат, проиллюстрированный на рис. \textit{Рисунок. Матрица ошибок для задачи MNIST}
		\end{itemize}
		Мы получили матрицу с явно выраженным диагональным преобладанием, таким образом полученная модель делает относительно небольшое число ошибок.}
\end{scnrelfromvector}

\scnheader{Рисунок. Архитектура и.н.с., решающая задачу классификации цифр}
\scnrelfrom{описание примера}{\scnfileimage[20em]{Contents/part_ps/src/images/sd_ps/sd_ann/model.png}}

\scnheader{Таблица. Результаты решения задачи}
\scnrelfrom{описание примера}{\scnfileimage[20em]{Contents/part_ps/src/images/sd_ps/sd_ann/results_table.png}}
\begin{scnindent}
	\scntext{пояснение}{Используемые сокращения: mbs --- mini-batch size, ks --- kernel size, lr --- learning rate, cnc --- convolutional neurons count, acc --- accuracy, it --- iterations count}
\end{scnindent}

\scnheader{Рисунок. Матрица ошибок для задачи MNIST}
\scnrelfrom{описание примера}{\scnfileimage[30em]{Contents/part_ps/src/images/sd_ps/sd_ann/conf_matrix_result}}

\scnheader{действие по построению и.н.с.}
\scntext{примечание}{Исходя из анализа этапов построения и.н.с., которые выполняют разработчики, можно вывести следующую классификацию действий по построению и.н.с.}
\begin{scnrelfromset}{декомпозиция}
	\scnitem{действие по обработке выборки}
	\begin{scnrelfromset}{декомпозиция}
		\scnitem{действие поиска подходящей обучающей выборки}
		\scnitem{действие формирования требований к обучающей выборке}
		\scnitem{действие очистки выборки}
		\scnitem{действие выявления содержательных признаков}
		\scnitem{действие трансформации выборки}
		\scnitem{действие разбиения выборки}
	\end{scnrelfromset}
	\scnitem{действие по проектированию и.н.с.}
	\begin{scnrelfromset}{декомпозиция}
		\scnitem{действие выбора класса нейросетевых методов}
		\scnitem{действие формирования спецификации входов и выходов и.н.с.}
	\end{scnrelfromset}
	\scnitem{действие обучения и.н.с.}
	\begin{scnrelfromset}{декомпозиция}
		\scnitem{действие выбора метода оптимизации}
		\scnitem{действие выбора минимизируемой функции ошибки}
		\scnitem{действие начальной инициализации и.н.с.}
		\scnitem{действие выбора гиперпараметров и.н.с.}
		\scnitem{действие обучения и.н.с.}
		\scnitem{действие оценки эффективности и.н.с.}
	\end{scnrelfromset}
\end{scnrelfromset}

\scnheader{темпоральность нейронной сети}
\scntext{примечание}{Так как в результате действий по построению \textit{и.н.с.} объект этих действий, конкретная \textit{и.н.с.}, может существенно меняться (меняется конфигурация сети, ее весовые коэффициенты), то \textit{и.н.с.} представляется в базе знаний как темпоральное объединение всех ее версий. Каждая версия является \textit{и.н.с.} и темпоральной сущностью. На множестве этих темпоральных сущностей задается темпоральная последовательность с указанием первой и последней версии. Для каждой версии описываются специфичные знания.
	\\Общие для всех версий знания описываются для \textit{и.н.с.}, являющейся темпоральным объединением всех версий (рисунок \textit{SCg-текст. Темпоральность нейронной сети})}

\scnheader{SCg-текст. Темпоральность нейронной сети}
\scnrelfrom{описание примера}{\scnfileimage[20em]{Contents/part_ps/src/images/sd_ps/sd_ann/temporal_neural_network_scg.png}}

\end{scnsubstruct}

\begin{scnrelfromvector}{заключение}
	\scnfileitem{В главе описан подход к \textit{интеграции и конвергенции искусственных нейронных сетей с базами знаний} в \textit{интеллектуальных компьютерных системах нового поколения} с помощью представления и интерпретации \textit{искусственной нейронной сети} в \textit{базе знаний}.}
	\scnfileitem{Описаны \textit{Синтаксис, Денотационная и Операционная семантика Языка представления нейросетевых методов в базах знаний}, который позволяет представить и интерпретировать в памяти интеллектуальной системы любую \textit{и.н.с.} Наличие такого языка порождает семантическую совместимость нейросетевого метода с другими методами, представленными в памяти системы, что позволяет анализировать саму \textit{и.н.с.} и этапы ее работы любыми другими методами системы.}
	\scnfileitem{Так же наличие языка представления нейросетевых методов позволяет описывать в памяти системы экспертные знания разработчиков \textit{и.н.с.} В главе приведены этапы построения \textit{и.н.с.}, которые выполняют разработчики \textit{и.н.с.} На основании этих этапов, c целью проектирования интеллектуальной среды построения \textit{нейросетевых методов}, в \textit{базе знаний} были классифицированы и описаны действия по построению \textit{и.н.с.}}
	\scnfileitem{Проектирования и реализация интеллектуальной среды построения \textit{и.н.с.} в \textit{базе знаний} системы является одним из двух основных направлений дальнейшего развития работу по конвергенции и интеграции и.н.с. с базами знаний.}
	\scnfileitem{Вторым основным направлением является разработка подхода к обработке фрагментов \textit{базы знаний} с помощью \textit{и.н.с.}, для чего необходимо разработать универсальный алгоритм взаимно-однозначного соответствия фрагментов базы знаний и входных векторов \textit{и.н.с.} Язык представления знаний способен представить любое знание. Наличие в системе нейросетевого метода, способного принимать на вход фрагменты знаний, позволит решить новые, слабо изученные классы задач.}
\end{scnrelfromvector}


\bigskip
\scnendcurrentsectioncomment
\end{scnsubstruct}


%\scsectionfamily{Часть 4 Стандарта OSTIS. Онтологические модели интерфейсов интеллектуальных компьютерных систем нового поколения}
\label{part_interfaces}

\scsubsection[
    \protect\scneditor{Садовский М.Е.}
    \protect\scnmonographychapter{Глава 4.1. Структура интерфейсов интеллектуальных компьютерных систем нового поколения}
    ]{Предметная область и онтология интерфейсов ostis-систем}
\scntext{аннотация}{В предметной области рассмотрены принципы организации \textit{интерфейсов интеллектуальных компьютерных систем нового поколения}. Ключевыми свойствами \textit{интерфейсов интеллектуальных компьютерных систем нового поколения} являются адаптивность и мультимодальность, что обеспечивает переход от парадигмы грамотного пользователя к парадигме равноправного сотрудничества пользователя с интеллектуальной системой, а также позволяет повысить эффективность человеко-машинного взаимодействия. В предметной области рассматривается подход к обеспечению указанных свойств на основе онтологической модели интерфейса и онтологической модели процесса проектирования интерфейсов.}
\scntext{заключение}{В предметной области рассмотрены принципы организации \textbf{партнерского взаимодействия пользователя с интеллектуальной системой} и принципы построения \textit{интерфейсов интеллектуальных компьютерных систем нового поколения}, обеспечивающих переход к \uline{парадигме равноправного сотрудничества}. \\ \\
В результате проведенного анализа были сделаны следующие выводы:
\begin{scnitemize}
	\item{Для перехода к \uline{парадигме равноправного сотрудничества пользователя и системы} интерфейсы должны быть \uline{адаптивными, интеллектуальными и мультимодальными}. Существующие решения позволяют проектировать такие интерфейсы, однако имеют ряд недостатков.}
	\item{Структура \textit{адаптивных интеллектуальных мультимодальных интерфейсов} включает базу знаний, модуль управления взаимодействием пользователя с системой.}
	\item{При проектировании базы знаний активно применяется онтологический подход и уже реализованы некоторые онтологии, которые используются при проектировании \textit{адаптивных интеллектуальных мультимодальных интерфейсов}.}
	\item{Модуль управления взаимодействием пользователя с системой, как правило, реализуется на основе многоагентного подхода.}
\end{scnitemize}

Среди недостатков существующих решений были выделены:
\begin{scnitemize}
	\item{Существующие решения, как правило, предусматривают \uline{вопросно-ответный принцип взаимодействия}.}
	\item{Актуальной остается проблема \uline{совместимости адаптивного интеллектуального мультимодального интерфейса с интеллектуальной системой}, для которой он создается, в силу различий используемых средств и методов при проектировании и реализации.}
	\item{Актуальной остается проблема \uline{совместимости компонентов адаптивного интеллектуального мультимодального интерфейса} (база знаний и модуль управления взаимодействием) между собой.}
\end{scnitemize}

Предложен \uline{онтологический подход на основе семантической модели} к проектированию и реализации \textit{адаптивного интеллектуального мультимодального пользовательского интерфейса} на основе \textit{Технологии OSTIS}, который позволит обеспечить:
\begin{scnitemize}
	\item{\uline{совместимость адаптивного интеллектуального мультимодального интерфейса с интеллектуальной системой};}
	\item{\uline{совместимость компонентов адаптивного интеллектуального мультимодального интерфейса между собой};}
	\item{\uline{взаимодействие пользователя с системой} через \textit{адаптивный интеллектуальный мультимодальный интерфейс} на \uline{равноправной основе}.}
\end{scnitemize}

Предложена и рассмотрена структура \textit{пользовательского интерфейса ostis-системы} и составляющие ее компоненты.
}
\label{sd_interfaces}
\begin{SCn}
\scnsectionheader{Предметная область и онтология интерфейсов ostis-систем}
\begin{scnsubstruct}
\scnrelfrom{соавтор}{Садовский М.Е.}
\begin{scnrelfromlist}{дочерний раздел}
    \scnitem{Предметная область и онтология интерфейсных действий пользователей ostis-систем}
    \scnitem{Предметная область и онтология естественных языков}
\end{scnrelfromlist}

\scnheader{Предметная область интерфейсов ostis-систем}
\scniselement{предметная область}
\begin{scnhaselementrole}{максимальный класс объектов исследования}
    {пользовательский интефейс}
\end{scnhaselementrole}
\begin{scnhaselementrolelist}{класс объектов исследования}
    \scnitem{командный пользовательский интерфейс}
    \scnitem{графический пользовательский интерфейс}
    \scnitem{WIMP-интерфейс}
    \scnitem{SILK-интерфейс}
    \scnitem{естественно-языковой интерфейс}
    \scnitem{речевой интерфейс}
    \scnitem{пользовательский интерфейс ostis-системы}
    \scnitem{компонент пользовательского интерфейса}
    \scnitem{атомарный компонент пользовательского интерфейса}
    \scnitem{неатомарный компонент пользовательского интерфейса}
    \scnitem{визуальная часть пользовательского интерфейса ostis-системы}
    \scnitem{компонент пользовательского интерфейса для представления}
    \scnitem{компонент вывода}
    \scnitem{компонент выполнения}
    \scnitem{параграф}
    \scnitem{декоративный компонент пользовательского интерфейса}
    \scnitem{контейнер}
    \scnitem{меню}
    \scnitem{строка меню}
    \scnitem{панель инструментов}
    \scnitem{панель вкладок}
    \scnitem{окно}
    \scnitem{модальное окно}
    \scnitem{немодальное окно}
    \scnitem{интерактивный компонент пользовательского интерфейса}
    \scnitem{флаговая кнопка}
    \scnitem{радиокнопка}
    \scnitem{переключатель}
    \scnitem{кнопка-счетчик}
    \scnitem{полоса прокрутки}
    \scnitem{кнопка}
\end{scnhaselementrolelist}

\scnheader{пользовательский интерфейс}
\scnsuperset{командный пользовательский интерфейс}
\scnsuperset{графический пользовательский интерфейс}	
\begin{scnindent}
	\scnsuperset{WIMP-интерфейс}
	\begin{scnindent}
		\scnsuperset{пользовательский интерфейс ostis-системы}
		\begin{scnindent}
			\scnhaselement{Пользовательский интерфейс Метасистемы IMS.ostis}
			\scnhaselement{Пользовательский интерфейс ИСС по геометрии}
			\begin{scnindent}
				\scnidtf{Пользовательский интерфейс интеллектуальной справочной системы по геометрии}
			\end{scnindent}
			\scnhaselement{Пользовательский интерфейс ИСС по дискретной математике}
			\begin{scnindent}
				\scnidtf{Пользовательский интерфейс интеллектуальной справочной системы по дискретной математике}
			\end{scnindent}
			\scnhaselement{Пользовательский интерфейс ИСС по географии}
			\begin{scnindent}
				\scnidtf{Пользовательский интерфейс интеллектуальной справочной системы по 	географии}
			\end{scnindent}
			\scnhaselement{Пользовательский интерфейс ИСС по искусственным нейронным сетям}
			\begin{scnindent}
				\scnidtf{Пользовательский интерфейс интеллектуальной справочной системы по искусственным нейронным сетям}
			\end{scnindent}
			\scnhaselement{Пользовательский интерфейс ИСС по лингвистике}
			\begin{scnindent}
				\scnidtf{Пользовательский интерфейс интеллектуальной справочной системы по лингвистике}
			\end{scnindent}
		\end{scnindent}
	\end{scnindent}
	\scnsuperset{SILK-интерфейс}
	\begin{scnindent}
		\scnidtf{(Speech --- речь, Image --- образ, Language --- язык, Knowledge --- знание)}
		\scnsuperset{естественно-языковой интерфейс}
		\begin{scnindent}
			\scnsuperset{речевой интерфейс}
		\end{scnindent}
	\end{scnindent}
\end{scnindent}

\scnheader{пользовательский интерфейс}
\scntext{пояснение}{\textit{пользовательский интерфейс} --- один из наиболее важных компонентов компьютерной системы. Представляет собой совокупность аппаратных и программных средств, обеспечивающих обмен информацией между пользователем и компьютерной системой.}

\scnheader{командный пользовательский интерфейс}
\scntext{пояснение}{\textit{командный пользовательский интерфейс} --- пользовательский интерфейс, при котором обмен информацией между компьютерной системой и пользователем осуществляется путем написания текстовых инструкций или команд.}

\scnheader{графический пользовательский интерфейс}
\scntext{пояснение}{\textit{графический пользовательский интерфейс} --- пользовательский интерфейс, при котором обмен информацией между компьютерной системой и пользователем осуществляется при помощи графических компонентов компьютерной системы.}

\scnheader{WIMP-интерфейс}
\scntext{пояснение}{\textit{WIMP-интерфейс} --- пользовательский интерфейс, при котором обмен информацией между компьютерной системой и пользователем осуществляется в форме диалога при помощью окон, меню и других элементов управления.}

\scnheader{SILK-интерфейс}
\scntext{пояснение}{\textit{SILK-интерфейс} --- пользовательский интерфейс, наиболее приближенный к естественной для человека форме общения. Компьютерная система находит для себя команды, анализируя человеческую речь и находя в ней ключевые фразы. Результат выполнения команд преобразуется в понятную человеку форму, например, в естественно-языковую форму или изображение.}

\scnheader{естественно-языковой интерфейс}
\scntext{пояснение}{\textit{естественно-языковой интерфейс} --- SILK-интерфейс, обмен информацией между компьютерной системой и пользователем в котором происходит за счёт диалога. Диалог ведётся на одном из естественных языков.}

\scnheader{речевой интерфейс}
\scntext{пояснение}{\textit{речевой интерфейс} --- SILK-интерфейс, обмен информацией в котором происходит за счёт диалога, в процессе которого компьютерная система и пользователь общаются с помощью речи. Данный вид интерфейса наиболее приближен к естественному общению между людьми.}

\scnheader{пользовательский интерфейс ostis-системы}
\scnsubset{ostis-система}
\scntext{пояснение}{\textit{пользовательский интерфейс ostis-системы} представляет собой специализированную \textit{ostis-систему}, ориентированную на решение интерфейсных задач и имеющую в своем составе базу знаний и решатель задач пользовательского интерфейса ostis-системы.\\
	Для решения задачи построения пользовательского интерфейса в базе знаний \textit{пользовательского интерфейса ostis-системы} необходимо наличие sc-модели \textit{компонентов пользовательского интерфейса}, \textit{интерфейсных действий пользователей}, а также классификации \textit{пользовательских интерфейсов} в целом. При проектировании интерфейса используется компонентный подход, который предполагает представление всего интерфейса приложения в виде отдельных специфицированных компонентов, которые могут разрабатываться и совершенствоваться независимо.}

\scnheader{компонент пользовательского интерфейса}
\scntext{пояснение}{\textit{компонент пользовательского интерфейса} --- знак фрагмента базы знаний, имеющий определённую форму внешнего представления на экране.}
\begin{scnsubdividing}
	\scnitem{атомарный компонент пользовательского интерфейса}
	\scnitem{неатомарный компонент пользовательского интерфейса}
\end{scnsubdividing}

\scnheader{атомарный компонент пользовательского интерфейса}
\scntext{пояснение}{\textit{атомарный компонент пользовательского интерфейса} --- компонент пользовательского интерфейса, не содержащий в своём составе других компонентов пользовательского интерфейса.}

\scnheader{неатомарный компонент пользовательского интерфейса}
\scntext{пояснение}{\textit{неатомарный компонент пользовательского интерфейса} --- компонент пользовательского интерфейса, состоящий из других компонентов пользовательского интерфейса.}

\scnheader{визуальная часть пользовательского интерфейса ostis-системы}
\scnsubset{неатомарный компонент пользовательского интерфейса}
\scntext{пояснение}{\textit{визуальная часть пользовательского интерфейса ostis-системы} --- часть базы знаний пользовательского интерфейса ostis-системы, содержащая необходимые для отображения пользовательского интерфейса компоненты.}

\scnheader{компонент пользовательского интерфейса}
\scnidtf{user interface component}
\scnsuperset{компонент пользовательского интерфейса для отображения}
\begin{scnindent}
	\scnidtf{presentation user interface component}
	\scnsuperset{компонент вывода}
	\begin{scnindent}
		\scnidtf{output}
		\scnsuperset{компонент вывода изображения}
		\begin{scnindent}
			\scnidtf{image-output}
		\end{scnindent}
		\scnsuperset{компонент вывода графической информации}
		\begin{scnindent}
			\scnidtf{graphical-output}
			\scnsuperset{диаграмма}
			\begin{scnindent}
				\scnidtf{chart}
			\end{scnindent}
			\scnsuperset{карта}
			\begin{scnindent}
				\scnidtf{map}
			\end{scnindent}
			\scnsuperset{индикатор выполнения}
			\begin{scnindent}
				\scnidtf{progress-bar}
			\end{scnindent}
		\end{scnindent}
		\scnsuperset{компонент вывода видео}
		\begin{scnindent}
			\scnidtf{video-output}
		\end{scnindent}
		\scnsuperset{компонент вывода звука}
		\begin{scnindent}
			\scnidtf{sound-output}
		\end{scnindent}
		\scnsuperset{компонент вывода текста}
		\begin{scnindent}
			\scnidtf{text-output}
			\scnsuperset{заголовок}
			\begin{scnindent}
				\scnidtf{headline}
			\end{scnindent}
			\scnsuperset{параграф}
			\begin{scnindent}
				\scnidtf{paragraph}
			\end{scnindent}
			\scnsuperset{сообщение}
			\begin{scnindent}
				\scnidtf{message}
			\end{scnindent}
		\end{scnindent}
	\end{scnindent}
	\scnsuperset{декоративный компонент пользовательского интерфейса}
	\begin{scnindent}
		\scnidtf{decorative user interface component}
		\scnsuperset{разделитель}
		\begin{scnindent}
			\scnidtf{separator}
		\end{scnindent}
		\scnsuperset{пустое пространство}
		\begin{scnindent}
			\scnidtf{blank-space}
		\end{scnindent}
	\end{scnindent}
	\scnsuperset{контейнер}
	\begin{scnindent}
		\scnidtf{container}
		\scnsuperset{меню}
		\begin{scnindent}
			\scnidtf{menu}
		\end{scnindent}
		\scnsuperset{строка меню}
		\begin{scnindent}
			\scnidtf{menu-bar}
		\end{scnindent}
		\scnsuperset{панель инструментов}
		\begin{scnindent}
			\scnidtf{tool-bar}
		\end{scnindent}
		\scnsuperset{строка состояния}
		\begin{scnindent}
			\scnidtf{status-bar}
		\end{scnindent}
		\scnsuperset{таблично-строковый контейнер}
		\begin{scnindent}
			\scnidtf{table-row-container}
		\end{scnindent}
		\scnsuperset{списковый контейнер}
		\begin{scnindent}
			\scnidtf{list-container}
		\end{scnindent}
		\scnsuperset{таблично-клеточный контейнер}
		\begin{scnindent}
			\scnidtf{table-cell-container}
		\end{scnindent}
		\scnsuperset{древовидный контейнер}
		\begin{scnindent}
			\scnidtf{tree-container}
		\end{scnindent}
		\scnsuperset{панель вкладок}
		\begin{scnindent}
			\scnidtf{tab-pane}
		\end{scnindent}
		\scnsuperset{панель вращения}
		\begin{scnindent}
			\scnidtf{spin-pane}
		\end{scnindent}
		\scnsuperset{узловой контейнер}
		\begin{scnindent}
			\scnidtf{tree-node-container}
		\end{scnindent}
		\scnsuperset{панель прокрутки}
		\begin{scnindent}
			\scnidtf{scroll-pane}
		\end{scnindent}
		\scnsuperset{окно}
		\begin{scnindent}
			\scnidtf{window}
			\scnsuperset{модальное окно}
			\begin{scnindent}
				\scnidtf{modal-window}
			\end{scnindent}
			\scnsuperset{немодальное окно}
			\begin{scnindent}
				\scnidtf{non-modal-window}
			\end{scnindent}
		\end{scnindent}
	\end{scnindent}
\end{scnindent}	
\scnsuperset{интерактивный компонент пользовательского интерфейса}
\begin{scnindent}
	\scnidtf{interactive user interface component}
	\scnsuperset{компонент ввода данных}
	\begin{scnindent}
		\scnidtf{data-input-component}
		\scnsuperset{компонент ввода данных с прямой ответной реакцией}
		\begin{scnindent}
			\scnidtf{data-input-component-with-direct-feedback}
			\scnsuperset{компонент ввода текста с прямой ответной реакцией}
			\begin{scnindent}
                \scnidtf{text-input-component-with-direct-feedback}
				\scnsuperset{многострочное текстовое поле}
				\begin{scnindent}
					\scnidtf{multi-line-text-field}
				\end{scnindent}
				\scnsuperset{однострочное текстовое поле}
				\begin{scnindent}
					\scnidtf{single-line-text-field}
				\end{scnindent}
			\end{scnindent}
			\scnsuperset{ползунок}
			\begin{scnindent}
				\scnidtf{slider}
			\end{scnindent}
			\scnsuperset{область рисования}
			\begin{scnindent}
				\scnidtf{drawing-area}
			\end{scnindent}
			\scnsuperset{компонент выбора}
			\begin{scnindent}
				\scnidtf{selection-component}
				\scnsuperset{компонент выбора нескольких значений}
				\begin{scnindent}
					\scnidtf{selection-component-multiple-values}
				\end{scnindent}
				\scnsuperset{компонент выбора одного значения}
				\begin{scnindent}
					\scnidtf{selection-component-single-values}
				\end{scnindent}
			\end{scnindent}
			\scnsuperset{компонент выбора данных}
			\begin{scnindent}
				\scnidtf{selectable-data-representation}
				\scnsuperset{флаговая кнопка}
				\begin{scnindent}
					\scnidtf{check-box}
				\end{scnindent}
				\scnsuperset{радиокнопка}
				\begin{scnindent}
					\scnidtf{radio-button}
				\end{scnindent}
				\scnsuperset{переключатель}
				\begin{scnindent}
					\scnidtf{toggle-button}
				\end{scnindent}
				\scnsuperset{выбираемый элемент}
				\begin{scnindent}
					\scnidtf{selectable-item}
				\end{scnindent}
			\end{scnindent}
		\end{scnindent}	
		\scnsuperset{компонент ввода данных без прямой ответной реакции}
		\begin{scnindent}
			\scnidtf{data-input-component-without-direct-feedback}
			\scnsuperset{кнопка-счётчик}
			\begin{scnindent}
				\scnidtf{spin-button}
			\end{scnindent}
			\scnsuperset{компонент речевого ввода}
			\begin{scnindent}
				\scnidtf{speech-input}
			\end{scnindent}
			\scnsuperset{компонент ввода движений}
			\begin{scnindent}
				\scnidtf{motion-input}
			\end{scnindent}
		\end{scnindent}
	\end{scnindent}
	\scnsuperset{компонент для представления и взаимодействия с пользователем}
	\begin{scnindent}
		\scnidtf{presentation-manipulation-component}
		\scnsuperset{активирующий компонент}
		\begin{scnindent}
			\scnidtf{activating-component}
		\end{scnindent}
		\scnsuperset{компонент непрерывной манипуляции}
		\begin{scnindent}
			\scnidtf{continuous-manipulation-component}
			\scnsuperset{полоса прокрутки}
			\begin{scnindent}
				\scnidtf{scrollbar}
			\end{scnindent}
			\scnsuperset{компонент редактирования размера}
			\begin{scnindent}
				\scnidtf{resizer}
			\end{scnindent}
		\end{scnindent}
	\end{scnindent}
	\scnsuperset{компонент запроса действий}
	\begin{scnindent}
		\scnidtf{operation-trigger-component}
		\scnsuperset{компонент выбора команд}
		\begin{scnindent}
			\scnidtf{command-selection-component}
			\scnsuperset{кнопка}
			\begin{scnindent}
				\scnidtf{button}
			\end{scnindent}
			\scnsuperset{пункт меню}
			\begin{scnindent}
				\scnidtf{menu-item}
			\end{scnindent}
		\end{scnindent}
		\scnsuperset{компонент ввода команд}
		\begin{scnindent}
			\scnidtf{command-input-component}
		\end{scnindent}
	\end{scnindent}
\end{scnindent}

\scnheader{компонент пользовательского интерфейса для представления}
\scntext{пояснение}{\textit{компонент пользовательского интерфейса для представления} --- компонент пользовательского интерфейса, не подразумевающий взаимодействия с пользователем.}

\scnheader{компонент вывода}
\scntext{пояснение}{\textit{компонент вывода} --- компонент пользовательского интерфейса, предназначенный для представления информации.}

\scnheader{индикатор выполнения}
\scntext{пояснение}{\textit{индикатор выполнения} --- компонент пользовательского интерфейса, предназначенный для отображения процента выполнения какой-либо задачи.}

\scnheader{параграф}
\scntext{пояснение}{\textit{параграф} --- компонент пользовательского интерфейса, предназначенный для отображения блоков текста. Он отделяется от других блоков пустой строкой или первой строкой с отступом.}

\scnheader{декоративный компонент пользовательского интерфейса}
\scntext{пояснение}{\textit{декоративный компонент пользовательского интерфейса} --- компонент пользовательского интерфейса, предназначенный для стилизации интерфейса.}

\scnheader{контейнер}
\scntext{пояснение}{\textit{контейнер} --- компонент пользовательского интерфейса, задача которого состоит в размещении набора компонентов, включённых в его состав.}

\scnheader{меню}
\scntext{пояснение}{\textit{меню} --- компонент пользовательского интерфейса, содержащий несколько вариантов для выбора пользователем.}

\scnheader{строка меню}
\scntext{пояснение}{\textit{строка меню} --- горизонтальная полоса, содержащая ярлыки меню. Строка меню предоставляет пользователю место в окне, где можно найти большинство основных функций программы.}

\scnheader{панель инструментов}
\scntext{пояснение}{\textit{панель инструментов} --- компонент пользовательского интерфейса, на котором размещаются элементы ввода или вывода данных.}

\scnheader{панель вкладок}
\scntext{пояснение}{\textit{панель вкладок} --- контейнер, который может содержать несколько вкладок (секций) внутри, которые могут быть отображены, нажав на вкладке с названием в верхней части панели. Одновременно отображается только одна вкладка.}

\scnheader{окно}
\scntext{пояснение}{\textit{окно} --- обособленная область экрана, содержащая различные элементы пользовательского интерфейса. Окна могут располагаться поверх друг друга.}

\scnheader{модальное окно}
\scntext{пояснение}{\textit{модальное окно} --- окно, которое блокирует работу пользователя с системой до тех пор, пока пользователь окно не закроет.}

\scnheader{немодальное окно}
\scntext{пояснение}{\textit{немодальное окно} --- окно, которое позволяет выполнять переключение между данным окном и другим окном без необходимости закрытия окна.}

\scnheader{интерактивный компонент пользовательского интерфейса}
\scntext{пояснение}{\textit{интерактивный компонент пользовательского интерфейса} --- компонент пользовательского интерфейса, с помощью которого осуществляется взаимодействие с пользователем.}

\scnheader{флаговая кнопка}
\scntext{пояснение}{\textit{флаговая кнопка} --- компонент пользовательского интерфейса, позволяющий пользователю управлять параметром с двумя состояниями --- включено и отключено.}

\scnheader{радиокнопка}
\scntext{пояснение}{\textit{радиокнопка} --- компонент пользовательского интерфейса, который позволяет пользователю выбрать одну опцию из предопределенного набора.}

\scnheader{переключатель}
\scntext{пояснение}{\textit{переключатель} --- компонент пользовательского интерфейса, который позволяет пользователю переключаться между двумя состояниями.}

\scnheader{кнопка-счетчик}
\scntext{пояснение}{\textit{кнопка-счетчик} --- компонент пользовательского интерфейса, как правило, ориентированный вертикально, с помощью которого пользователь может изменить значение в прилегающем текстовом поле, в результате чего значение в текстовом поле увеличивается или уменьшается.}

\scnheader{полоса прокрутки}
\scntext{пояснение}{\textit{полоса прокрутки} --- компонент пользовательского интерфейса, который используется для отображения компонентов пользовательского интерфейса, больших по размеру, чем используемый для их отображения контейнер.}

\scnheader{кнопка}
\scntext{пояснение}{\textit{кнопка} --- компонент пользовательского интерфейса, при нажатии на который происходит программно связанное с этим нажатием действие либо событие.}

\scnheader{Стартовая страница пользовательского интерфейса Метасистемы IMS.ostis}
\scniselement{Визуальная часть пользовательского интерфейса Метасистемы IMS.ostis}
\begin{scnindent}
	\scnsubset{визуальная часть пользовательского интерфейса ostis-системы}
	\scnrelto{часть}{Пользовательский интерфейс Метасистемы IMS.ostis}
\end{scnindent}	
\scniselement{окно}
\scnrelfrom{иллюстрация}{\scnfileimage[40em]{Contents/part_ui/images/sd_ui/startPage.png}}
\begin{scnrelfromset}{декомпозиция}
    \scnitem{Панель навигации}
	\begin{scnindent}
		\scniselement{неатомарный компонент пользовательского интерфейса}
		\begin{scnrelfromset}{декомпозиция}
			\scnitem{Главное меню}
			\begin{scnindent}
				\scniselement{меню}
				\begin{scnrelfromset}{декомпозиция}
					\scnitem{Пункт меню для навигации по ключевым понятиям}
					\begin{scnindent}
						\scniselement{пункт меню}
					\end{scnindent}
					\scnitem{Пункт меню для выполнения команд просмотра базы знаний}
					\begin{scnindent}
						\scniselement{пункт меню}
					\end{scnindent}
					\scnitem{Компонент перехода в экспертный режим}
					\begin{scnindent}
						\scniselement{переключатель}
					\end{scnindent} 
				\end{scnrelfromset}
			\end{scnindent} 
			\scnitem{Компонент выбора языка}
			\begin{scnindent}
				\scniselement{компонент выбора одного значения}
			\end{scnindent} 
			\scnitem{Компонент авторизации}
			\begin{scnindent}
				\scniselement{кнопка}
			\end{scnindent} 
		\end{scnrelfromset}
	\end{scnindent}	 
    \scnitem{Блок истории запросов пользователя}
    	\begin{scnindent}
			\scniselement{неатомарный компонент пользовательского интерфейса}
		\end{scnindent}
    \scnitem{Основной блок}
	\begin{scnindent}
			\scniselement{неатомарный компонент пользовательского интерфейса}
		\begin{scnrelfromset}{декомпозиция}
			\scnitem{Главное}
			\begin{scnindent}
				\scniselement{окно}
			\end{scnindent}
			\scnitem{Панель инструментов}
			\begin{scnindent}
				\scniselement{неатомарный компонент пользовательского интерфейса}
				\begin{scnrelfromset}{декомпозиция}
					\scnitem{Кнопка отправки содержимого главного окна на печать}
					\begin{scnindent}
						\scniselement{кнопка}
					\end{scnindent} 
					\scnitem{Кнопка управления видимостью блока истории запросов пользователя}
					\begin{scnindent}
						\scniselement{кнопка}
					\end{scnindent} 
					\scnitem{Кнопка отображения ссылки на текущий запрос пользователя}
					\begin{scnindent}
						\scniselement{кнопка}
					\end{scnindent} 
					\scnitem{Поле поиска}
					\begin{scnindent}
						\scniselement{однострочное текстовое поле}
					\end{scnindent} 
				\end{scnrelfromset}
			\end{scnindent} 
		\end{scnrelfromset}
	\end{scnindent}
    \scnitem{Панель отображения информации об авторских правах}
	\begin{scnindent}
		\scniselement{неатомарный компонент пользовательского интерфейса}
	\end{scnindent}
\end{scnrelfromset}

\bigskip
\end{scnsubstruct}
\scnendcurrentsectioncomment
\end{SCn}


\scsubsection[
    \protect\scneditors{Садовский М.Е.;Жмырко А.В.}
    \protect\scnmonographychapter{Глава 4.1. Структура интерфейсов интеллектуальных компьютерных систем нового поколения}
    ]{Предметная область и онтология интерфейсных действий пользователей ostis-системы}
\label{sd_user_interface_actions}
\begin{SCn}
\scnsectionheader{Предметная область и онтология интерфейсных действий пользователей ostis-системы}
\begin{scnsubstruct}
\scnrelfrom{соавтор}{Садовский М.Е.}

\scnheader{Предметная область интерфейсных действий пользователей}
\scniselement{предметная область}
\begin{scnhaselementrole}{максимальный класс объектов исследования}
	{интерфейсное действие пользователя}
\end{scnhaselementrole}
\begin{scnhaselementrolelist}{класс объектов исследования}
    \scnitem{действие мышью}
    \scnitem{прокрутка мышью}
    \scnitem{прокрутка мышью вверх}
    \scnitem{прокрутка мышью вниз}
    \scnitem{наведение мышью}
    \scnitem{отпускание мышью}
    \scnitem{нажатие мыши}
    \scnitem{одиночное нажатие мыши}
    \scnitem{двойное нажатие мыши}
    \scnitem{жест мышью}
    \scnitem{отведение мышью}
    \scnitem{перетаскивание мышью}
    \scnitem{действие голосом}
    \scnitem{действие клавиатурой}
    \scnitem{нажатие функциональной клавиши}
    \scnitem{нажатие клавиши набора текста}
    \scnitem{действие осязанием}
    \scnitem{действие сенсором}
    \scnitem{нажатие сенсора}
    \scnitem{одиночное нажатие сенсора}
    \scnitem{двойное нажатие сенсора}
    \scnitem{жест по сенсору}
    \scnitem{жест по сенсору одним пальцем}
    \scnitem{жест по сенсору несколькими пальцами}
    \scnitem{отпускание сенсором}
    \scnitem{перетаскивание сенсором}
    \scnitem{действие пером}
    \scnitem{нажатие функциональной клавиши пером}
    \scnitem{рисование пером}
    \scnitem{написание текста пером}
\end{scnhaselementrolelist}
\begin{scnhaselementrole}{исследуемое отношение}
	{инициируемое пользовательским интерфейсом действие*}
\end{scnhaselementrole}
\scnrelfrom{частная предметная область}{Предметная область интерфейсных действий пользователей ostis-системы}

\scnheader{интерфейсное действие пользователя}
\scnidtf{user interface action}
\scntext{пояснение}{Действие, выполняемое пользователем над некоторым \textit{компонентом пользовательского интерфейса}. Для связи данного действия с \textit{компонентом пользовательского интерфейса} и необходимым к выполнению \textit{внутренним действием системы} используется отношение \textit{инициируемое пользовательским интерфейсом действие*}.}
\scnsuperset{действие мышью}
\begin{scnindent}
	\scnidtf{mouse-action}
	\scnsuperset{прокрутка мышью}
	\scnidtf{mouse-scroll}
	\scnsuperset{прокрутка мышью вверх}
	\begin{scnindent}
		\scnidtf{mouse-scroll-up}
		\scnsuperset{прокрутка мышью вниз}
		\begin{scnindent}
		\scnidtf{mouse-scroll-down}
		\end{scnindent}
	\end{scnindent}
	\scnsuperset{наведение мышью}
	\begin{scnindent}
		\scnidtf{mouse-hover}
	\end{scnindent}
	\scnsuperset{отпускание мышью}
	\begin{scnindent}
		\scnidtf{mouse-drop}
	\end{scnindent}
	\scnsuperset{нажатие мыши}
	\begin{scnindent}
		\scnidtf{mouse-click}
		\scnsuperset{одиночное нажатие мыши}
		\begin{scnindent}
			\scnidtf{mouse-single-click}
		\end{scnindent}
		\scnsuperset{двойное нажатие мыши}
		\begin{scnindent}
			\scnidtf{mouse-double-click}
		\end{scnindent}
	\end{scnindent}
	\scnsuperset{жест мышью}
	\begin{scnindent}
		\scnidtf{mouse-gesture}
	\end{scnindent}
	\scnsuperset{отведение мышью}
	\begin{scnindent}
		\scnidtf{mouse-unhover}
	\end{scnindent}
	\scnsuperset{перетаскивание мышью}
	\begin{scnindent}
		\scnidtf{mouse-drag}
	\end{scnindent}
\end{scnindent}
\scnsuperset{действие голосом}
\begin{scnindent}
	\scnidtf{speech-action}
\end{scnindent}
\scnsuperset{действие клавиатурой}
\begin{scnindent}
	\scnidtf{keyboard-action}
	\scnsuperset{нажатие функциональной клавиши}
	\begin{scnindent}
		\scnidtf{press-function-key}
	\end{scnindent}
	\scnsuperset{нажатие клавиши набора текста}
	\begin{scnindent}
		\scnidtf{type-text}
	\end{scnindent}
\end{scnindent}
\scnsuperset{действие осязанием}
\begin{scnindent}
	\scnidtf{tangible-action}
\end{scnindent}
\scnsuperset{действие сенсором}
\begin{scnindent}
	\scnidtf{touch-action}
	\scnsuperset{нажатие сенсора}
	\begin{scnindent}
		\scnidtf{touch-click}
		\scnsuperset{одиночное нажатие сенсора}
		\begin{scnindent}
			\scnidtf{touch-single-click}
		\end{scnindent}
		\scnsuperset{двойное нажатие сенсора}
		\begin{scnindent}
			\scnidtf{touch-double-click}
		\end{scnindent}
	\end{scnindent}
	\scnsuperset{жест по сенсору}
	\begin{scnindent}
		\scnidtf{touch-gesture}
		\scnsuperset{жест по сенсору одним пальцем}
		\begin{scnindent}
			\scnidtf{one-fingure-gesture}
		\end{scnindent}
		\scnsuperset{жест по сенсору несколькими пальцами}
		\begin{scnindent}
			\scnidtf{multiple-finger-gesture}
		\end{scnindent}
	\end{scnindent}
	\scnsuperset{отпускание сенсором}
	\begin{scnindent}
		\scnidtf{touch-drop}
	\end{scnindent}
	\scnsuperset{перетаскивание сенсором}
	\begin{scnindent}
		\scnidtf{touch-drag}
	\end{scnindent}
\end{scnindent}
\scnsuperset{действие пером}
\begin{scnindent}
	\scnidtf{pen-base-action}
	\scnsuperset{нажатие функциональной клавиши пером}
	\begin{scnindent}
		\scnidtf{touch-function-key}
	\end{scnindent}
	\scnsuperset{рисование пером}
	\begin{scnindent}
		\scnidtf{draw}
	\end{scnindent}
	\scnsuperset{написание текста пером}
	\begin{scnindent}
		\scnidtf{write-text}
	\end{scnindent}
\end{scnindent}	

\scnheader{прокрутка мышью}
\scntext{пояснение}{\textit{прокрутка мышью} --- интерфейсное действие пользователя, соответствующее прокрутке содержимого некоторого компонента пользовательского интерфейса при помощи мыши.}

\scnheader{наведение мышью}
\scntext{пояснение}{\textit{наведение мышью} --- интерфейсное действие пользователя, соответствующее появлению курсора мыши в рамках компонента пользовательского интерфейса.}

\scnheader{отпускание мышью}
\scntext{пояснение}{\textit{отпускание мышью} --- интерфейсное действие пользователя, соответствующее отпусканию некоторого компонента пользовательского интерфейса в рамках другого компонента пользовательского интерфейса при помощи мыши.}

\scnheader{нажатие мыши}
\scntext{пояснение}{\textit{нажатие мыши} --- интерфейсное действие пользователя, соответствующее выполнению нажатия мыши в рамках некоторого компонента пользовательского интерфейса.}

\scnheader{отведение мышью}
\scntext{пояснение}{\textit{отведение мышью} --- интерфейсное действие пользователя, соответствующее выходу курсора мыши за рамки компонента пользовательского интерфейса.}

\scnheader{перетаскивание мышью}
\scntext{пояснение}{\textit{перетаскивание мышью} --- интерфейсное действие пользователя, соответствующее перетаскиванию компонента пользовательского интерфейса при помощи мыши.}

\scnheader{нажатие сенсора}
\scntext{пояснение}{\textit{нажатие сенсора} --- интерфейсное действие пользователя, соответствующее выполнению нажатия сенсора в рамках некоторого компонента пользовательского интерфейса.}

\scnheader{жест по сенсору}
\scntext{пояснение}{\textit{жест по сенсору} --- интерфейсное действие пользователя, соответствующее выполнению некоторого жеста, выполняемого при помощи движения пальцев на экране сенсора.}

\scnheader{отпускание сенсором}
\scntext{пояснение}{\textit{отпускание сенсором} --- интерфейсное действие пользователя, соответствующее отпусканию некоторого компонента пользовательского интерфейса в рамках другого компонента пользовательского интерфейса при помощи сенсора.}

\scnheader{перетаскивание сенсором}
\scntext{пояснение}{\textit{перетаскивание сенсором} --- интерфейсное действие пользователя, соответствующее перетаскиванию компонента пользовательского интерфейса при помощи сенсора.}

\scnheader{действие пером}
\scntext{пояснение}{\textit{действие пером} --- интерфейсное действие пользователя, осуществляемое при помощи пера на графическом планшете.}

\scnheader{класс интерфейсных действий пользователя}
\scnsubset{класс действий}
\scnrelto{семейство подмножеств}{интерфейсное действие пользователя}
\scntext{пояснение}{\textit{класс интерфейсных действий пользователя} --- множество, элементами которого являются классы \textit{интерфейсных действий пользователя}.}

\scnheader{инициируемое пользовательским интерфейсом действие*}
\scntext{пояснение}{При взаимодействии пользователя с \textit{компонентом пользовательского интерфейса} могут быть произведены различные интерфейсные действия. В зависимости от выполненного интерфейсного действия и компонента, над которым оно было выполнено, происходит инициирование некоторого \textit{внутреннего действия системы}. Для задания такого инициируемого при взаимодействии с пользовательским интерфейсом действия и используется указанное отношение. Первым компонентом связки отношения \textit{инициируемое пользовательским интерфейсом действие*} является связка, элементами которой являются элемент множества компонентов пользовательского интерфейса и элемент множества \textit{класс интерфейсных действий пользователя}. Вторым компонентом является элемент множества \textit{класс внутренних действий системы}.}
\scniselement{квазибинарное отношение}
\scniselement{ориентированное отношение}
\scnrelfrom{первый домен}{компонент пользовательского интерфейса $\cup$ класс интерфейсных действий пользователя}
\scnrelfrom{второй домен}{класс внутренних действий системы}
\scnrelfrom{описание примера}{\scnfileimage[40em]{Contents/part_ui/images/sd_ui/ui_initiated_action.png}}

\bigskip
\end{scnsubstruct}
\scnendcurrentsectioncomment
\end{SCn}


\scsubsection[
    \protect\scneditors{Садовский М.Е.;Никифоров С.А.;Захарьев В.А.}
    \protect\scnmonographychapter{Глава 4.1. Структура интерфейсов интеллектуальных компьютерных систем нового поколения}
    ]{Предметная область и онтология сообщений, входящих в ostis-систему и выходящих из неё}
\label{sd_messages}
\begin{SCn}
\scnsectionheader{Предметная область и онтология сообщений, входящих в ostis-систему и выходящих из неё}
\begin{scnsubstruct}
\scnrelfrom{соавтор}{Садовский М.Е.}

\scnheader{сообщение}
\scntext{определение}{\scnkeyword{сообщение} --- дискретная информационная конструкция, используемая в процессе передачи от отправителя к получателю.}
\begin{scnrelfromset}{разбиение}
	\scnitem{сообщение пользователя системы}
	\begin{scnindent}
		\scnsuperset{сообщение пользователя ostis-системы}
	\end{scnindent}
	\scnitem{сообщение системы}
	\begin{scnindent}
		\scnsuperset{сообщение ostis-системы}
		\begin{scnindent}
			\scnsuperset{эффекторное сообщение ostis-системы}
			\begin{scnindent}
				\scntext{определение}{\scnkeyword{эффекторное сообщение ostis-системы} --- сообщение ostis-системы, формируемое самой ostis-системой при возникновении некоторых ситуаций.}
				\scntext{примечание}{К ситуациям, инициирующим возникновение эффекторных сообщений, можно отнести:
					\begin{scnitemize}
						\item{Ситуации, возникающие при анализе деятельности самого пользователя. Например, задание аргументов, не соответствующих типу инициируемого действия или появление подсказок при использовании компонентов пользовательского интерфейса.}
						\item{Ситуации, возникающие при анализе синтаксиса текстов внешних языков. Например, неполнота сформированного предложения на внешнем языке или использование конструкций, нехарактерных или некорректно использованных в контексте отдельно взятого внешнего языка.}
					\end{scnitemize}
				}
			\end{scnindent}
			\scnsuperset{рецепторное сообщение ostis-системы}
			\begin{scnindent}
				\scntext{определение}{\scnkeyword{рецепторное сообщение ostis-системы} --- сообщение ostis-системы, являющееся реакцией на императивное сообщение (сообщение, побуждающее к какому-либо действию).}
				\scntext{примечание}{Возможными реакциями ostis-системы на императивное сообщение пользователя являются:
					\begin{scnitemize}
						\item{указание факта завершения выполнения некоторой задачи, что, например, характерно для поведенческих действий;}
						\item{получение ответа на поставленную задачу, формируемого либо в результате анализа базы знаний пользовательского интерфейса, либо в результате анализа предметной части базы знаний самой ostis-системы.}
					\end{scnitemize}
				}
			\end{scnindent}
		\end{scnindent}
	\end{scnindent}
\end{scnrelfromset}
\begin{scnrelfromset}{разбиение}
	\scnitem{атомарное сообщение}
	\scnitem{неатомарное сообщение}
\end{scnrelfromset}
\begin{scnrelfromset}{разбиение}
	\scnitem{сообщение на естественном языке}
	\scnitem{сообщение на искусственном языке}
\end{scnrelfromset}
\scnsuperset{графическое сообщение}
\begin{scnindent}
	\scnidtf{сообщение, содержащее графическую информацию}
	\scnsuperset{видео-сообщение}
	\begin{scnindent}
		\scnidtf{сообщение, содержащее видео-информацию}
	\end{scnindent}
\end{scnindent}
\scnsuperset{аудио-сообщение}
\begin{scnindent}
	\scnidtf{сообщение, представленное в звуковом формате}
\end{scnindent}
\scnsuperset{обонятельное сообщение}
\begin{scnindent}
	\scnidtf{сообщение, содержащее информацию о запахах}
\end{scnindent}
\scnsuperset{текстовое сообщение}
\begin{scnindent}
	\scnidtf{сообщение, содержащее текстовую информацию}
\end{scnindent}
\scnsuperset{сообщение, требующее трансляции}
\begin{scnindent}
	\scnidtf{сообщение, которое необходимо сформировать системой для дальнейшей передачи его пользователю}
\end{scnindent}
\scnsuperset{протранслированное сообщение}
\begin{scnindent}
	\scnidtf{сообщение, которое было сформировано системой для дальнейшей передачи его пользователю}
\end{scnindent}

\bigskip
\end{scnsubstruct}
\scnendcurrentsectioncomment
\end{SCn}

\scsubsection[
    \protect\scneditors{Садовский М.Е.;Жмырко А.В.}
    \protect\scnmonographychapter{Глава 4.1. Структура интерфейсов интеллектуальных компьютерных систем нового поколения}
    ]{Предметная область и онтология действий и внутренних агентов пользовательского интерфейса ostis-системы}
\label{sd_actions_and_internal_agent}
\begin{SCn}
\scnsectionheader{Предметная область и онтология действий и внутренних агентов пользовательского интерфейса ostis-системы}
\begin{scnsubstruct}
\scnrelfrom{соавтор}{Садовский М.Е.}

\scnheader{решатель задач пользовательского интерфейса ostis-систем}
\scntext{примечание}{\textit{решатель задач пользовательского интерфейса ostis-систем} состоит из некоторого коллектива sc-агентов, обеспечивающих работу пользователя с компонентами пользовательского интерфейса ostis-системы.}
\scntext{примечание}{При использовании \textit{sc-агентов} стоит помнить различия в \uline{семантической} и \uline{прагматической} составляющей любого \textit{компонента пользовательского интерфейса}. \uline{Семантическая составляющая} заключается в определении того, знаком какой сущности является отображаемый на экране компонент. \uline{Прагматическая составляющая} рассматривает прикладной аспект (аспект применения) отображаемого на экране компонента.}
\scntext{примечание}{На уровне sc-памяти имеет значение только \uline{семантическая составляющая}, однако данный факт не влияет на процесс эксплуатации системы пользователем, поскольку обе составляющие отражают разные стороны одного и того же знака некоторой сущности. Например, за каждой кнопкой скрывается знак некоторого \textit{класса действия}, инициируемого при нажатии на кнопку. Таким образом, \textit{интерфейсное действие пользователя}, как правило, инициирует некоторое \textit{внутреннее действие системы}.}

\scnheader{решатель задач пользовательского интерфейса ostis-систем}
\scntext{примечание}{С точки зрения обработки модели базы знаний пользовательского интерфейса ostis-систем должны быть решены следующие задачи:
	\begin{scnitemize}
		\item{обработка пользовательских действий;}
		\item{интерпретация модели базы знаний пользовательского интерфейса ostis-системы (построение пользовательского интерфейса);}
	\end{scnitemize}
}
\scnsuperset{интерпретатор sc-моделей пользовательских интерфейсов}
\begin{scnindent}
	\scntext{примечание}{\textit{интерпретатор sc-моделей пользовательских интерфейсов} в качестве входного параметра принимает экземпляр \textit{компонента пользовательского интерфейса для отображения}. При этом компонент может быть как атомарным, так и неатомарным (например, компонент главного окна приложения). Результатом работы интерпретатора является графическое представление указанного компонента с учетом используемой реализации \textit{платформы интерпретации семантических моделей ostis-систем}.}
	\scntext{примечание}{Алгоритм работы данного интерпретатора следующий:
		\begin{scnitemize}
			\item{Проверяется тип входного компонента пользовательского интерфейса (атомарный или неатомарный).}
			\item{Если \textit{компонент пользовательского интерфейса} является атомарным, то отобразить его графическое представление на основании указанных для него свойств. В случае, если данный компонент не входит в \textit{декомпозицию} любого другого \textit{компонента пользовательского интерфейса} --- завершить выполнение. Иначе определить компонент, в \textit{декомпозицию} которого входит рассматриваемый компонент пользовательского интерфейса, применить его свойства для текущего атомарного компонента и начать обработку найденного неатомарного компонента, перейдя к первому пункту.}
			\item{Если \textit{компонент пользовательского интерфейса} является неатомарным, то проверить, были ли отображены компоненты, на которые он был декомпозирован. Если да, то завершить выполнение, иначе определить еще не отображенный компонент из декомпозиции обрабатываемого неатомарного компонента и начать обработку найденного компонента, перейдя к первому пункту.}
		\end{scnitemize}
	}
\end{scnindent}
\scnsuperset{интерпретатор пользовательских действий}
\begin{scnindent}
	\scntext{примечание}{\textit{интерпретатор пользовательских действий} является \textit{неатомарным sc-агентом}, который включает в себя множество \textit{sc-агентов}, каждый из которых обрабатывает интерфейсные действия пользователя определенного класса (например, \textit{абстрактный sc-агент обработки действия нажатия мыши}, \textit{абстрактный sc-агент обработки действия отпускания мыши} и так далее). Интерпретатор реагирует на появление в базе знаний системы экземпляра интерфейсного действия пользователя, находит связанный с ним класс внутреннего действия и генерирует экземпляр данного внутреннего действия для последующей обработки.}
\end{scnindent}

\bigskip
\end{scnsubstruct}
\scnendcurrentsectioncomment
\end{SCn}

\scsubsection[
    \protect\scneditor{Никифоров С.А.}
    \protect\scnmonographychapter{Глава 4.2. Естественно-языковые интерфейсы интеллектуальных компьютерных систем нового поколения}
    ]{Предметная область и онтология естественно-языковых интерфейсов ostis-систем}
\label{sd_natural_lang_interface}

\scsubsubsection[
    \protect\scneditors{Никифоров С.А.;Бобёр Е.С.;Захарьев В.А.}
    \protect\scnmonographychapter{Глава 4.2. Естественно-языковые интерфейсы интеллектуальных компьютерных систем нового поколения}
    ]{Предметная область и онтология синтаксического анализа естественно-языковых сообщений, входящих в ostis-систему}
\label{sd_process_syntax_message_analysis}

\scsubsubsection[
    \protect\scnmonographychapter{Глава 4.2. Естественно-языковые интерфейсы интеллектуальных компьютерных систем нового поколения}
    ]{Предметная область и онтология понимания естественно-языковых сообщений, входящих в ostis-систему}
\label{sd_process_message_understanding}

\scsubsubsection[
    \protect\scneditors{Никифоров С.А.;Бобёр Е.С.;Захарьев В.А.}
    \protect\scnmonographychapter{Глава 4.2. Естественно-языковые интерфейсы интеллектуальных компьютерных систем нового поколения}
    ]{Предметная область и онтология синтеза естественно-языковых сообщений  ostis-системы}
\label{sd_process_message_synthesis} 

%\scsectionfamily{Часть 5 Стандарта OSTIS. Методы и средства проектирования интеллектуальных компьютерных систем нового поколения}
\label{part_methods_and_tools}

\scsection[\scneditors{Шункевич Д.В.;Орлов М.К.}\protect\scnmonographychapter{Глава 5.1. Комплексная библиотека многократно используемых семантически совместимых компонентов интеллектуальных компьютерных систем нового поколения}]{Предметная область и онтология комплексной библиотеки многократно используемых семантически совместимых компонентов ostis-систем}
\label{sd_biblio_component}
\begin{SCn}
\scnsectionheader{\currentname}

\scnstartsubstruct

\scnheader{Предметная область многократно используемых компонентов ostis-систем}
\scniselement{предметная область}
\scnsdmainclasssingle{многократно используемый компонент ostis-систем}
\scnsdclass{независимый многократно используемый компонент ostis-систем;зависимый многократно используемый компонент ostis-систем;атомарный многократно используемый компонент ostis-систем;неатомарный многократно используемый компонент ostis-систем; многократно используемый компонент ostis-систем, хранящийся в виде внешних файлов;многократно используемый компонент ostis-систем, хранящийся в виде файлов исходных текстов;многократно используемый компонент ostis-систем, хранящийся в виде бинарных файлов;многократно используемый компонент, хранящийся в базе знаний ostis-системы;хранилище многократно используемого компонента ostis-систем, хранящегося в виде внешних файлов;библиотека многократно используемых компонентов ostis-систем;хранилище многократно используемого компонента ostis-систем, хранящегося в виде файлов исходных текстов;хранилище многократно используемого компонента ostis-систем, хранящегося в виде бинарных файлов; спецификация многократно используемого компонента ostis-систем;отношение, специфицирующее многократно используемый компонент ostis-систем; файл, содержащий url-адрес многократно используемого компонента ostis-систем}
\scnsdrelation{зависимый компонент*;несовместимый компонент*;адрес хранилища*;автор компонента*;установленные компоненты*; доступные к установке компоненты*}

\scnheader{многократно используемый компонент ostis-систем}
\scnexplanation{Компонент ostis-системы, который может быть использован в других ostis-системах (дочерних ostis-системах) и содержит все те (и только те) sc-элементы, которые необходимы для функционирования компонента в дочерней ostis-системе.}
\scnexplanation{Компонент некоторой материнской ostis-системы, который может быть использован в некоторой дочерней ostis-системе.}
\scnsubdividing{атомарный многократно используемый компонент ostis-систем;неатомарный многократно используемый компонент ostis-систем}
\scnsubdividing{зависимый многократно используемый компонент ostis-систем;независимый многократно используемый компонент ostis-систем}
\scnsubset{компонент ostis-системы}
\scnaddlevel{1}
\scnexplanation{Целостная часть ostis-системы, которая содержит все те (и только те) sc-элементы, которые необходимы для её функционирования в ostis-системе.}
\scnaddlevel{-1}
\scnrelfrom{разбиение}{\scnkeyword{Типология компонентов ostis-систем по типу хранения\scnsupergroupsign}}
\scnaddlevel{1} 
\scneqtoset{многократно используемый компонент ostis-систем, хранящийся в виде внешних файлов\\
	\scnaddlevel{1}
	\scnsubdividing{многократно используемый компонент ostis-систем, хранящийся в виде файлов исходных текстов;многократно используемый компонент ostis-систем, хранящийся в виде скомпилированных файлов}
	\scnaddlevel{-1}
	;многократно используемый компонент, хранящийся в базе знаний ostis-системы}
\scnaddlevel{1}
\scnnote{На данном этапе развития \textit{Технологии OSTIS} более удобным является хранение компонентов в виде исходных текстов.}
\scnaddlevel{-1}
\scnaddlevel{-1}

\scnheader{следует отличать*}
\scnhaselementset{многократно используемый компонент ostis-систем;компонент ostis-системы}
\scntext{отличие}{\textbf{\textit{многократно используемый компонент ostis-систем}} имеет спецификацию, достаточную для установки этого компонента в дочернюю ostis-систему. Спецификация является частью базы знаний \textbf{\textit{библиотеки многократно используемых компонентов}} соответствующей материнской ostis-системы.}

\scnheader{независимый многократно используемый компонент ostis-систем}
\scnexplanation{Многократно используемый компонент, который не зависит от других компонентов.}

\scnheader{зависимый многократно используемый компонент ostis-систем}
\scnexplanation{Многократно используемый компонент, который зависит от хотя бы одного другого компонента, т.е. не может быть встроен в дочернюю ostis-систему без компонентов, от которых он зависит.}

\scnheader{атомарный многократно используемый компонент ostis-систем}
\scnexplanation{Многократно используемый компонент, который в текущем состоянии библиотеки компонентов рассматривается как неделимый, то есть не содержит в своем составе других компонентов.}

\scnheader{неатомарный многократно используемый компонент ostis-систем}
\scnexplanation{Многократно используемый компонент, который в текущем состоянии библиотеки компонентов содержит в своем составе атомарные компоненты.}

\scnheader{установленные компоненты*}
\scniselement{квазибинарное отношение}
\scniselement{ориентированное отношение}
\scnexplanation{Квазибинарное отношение, связывающее некоторую ostis-систему и компоненты, которые установлены в ней.}
\scnrelfrom{первый домен}{ostis-система}
\scnrelfrom{второй домен}{многократно используемый компонент ostis-систем}
\scnnote{Данное отношение позволяет хранить сведения о системах и компонентах, которые установлены в них, тем самым предоставляя возможность анализировать функциональные возможности системы.}
\scnnote{Данное отношение позволяет оценивать частоту скачивания компонентов, то есть их использования в дочерних ostis-системах.}

\scnheader{доступные к установке компоненты*}
\scniselement{квазибинарное отношение}
\scniselement{ориентированное отношение}
\scnexplanation{Квазибинарное отношение, связывающее некоторую ostis-систему и компоненты, которые доступны для установки в данной ostis-системе.}
\scnrelfrom{первый домен}{ostis-система}
\scnrelfrom{второй домен}{многократно используемый компонент ostis-систем}
% \scnnote{Данное отношение позволяет давать рекомендации по развитию ostis-системы, в которую можно установить многократно используемые компоненты на основе уже установленных компонентов и решаемых задач ostis-системой.}
% \scnnote{Доступные к установке компоненты ostis-системы выбираются исходя из тех библиотек ostis-систем, к которым есть доступ у ostis-системы.}
% \scnnote{Смотрите \textbf{\textit{менеджер многократно используемых компонентов ostis-систем.}}}

\scnheader{хранилище многократно используемого компонента ostis-систем, хранящегося в виде внешних файлов}
% \scnexplanation{Место, предназначенное для хранения многократно используемого компонента ostis-систем.}
\scnsuperset{хранилище многократно используемого компонента ostis-систем, хранящегося в виде файлов исходных текстов}
\scnaddlevel{1}
\scnexplanation{Место хранения файлов исходных текстов многократно используемого компонента.}
\scnsuperset{хранилище на основе системы контроля версий Git}
\scnaddlevel{1}
\scnsuperset{репозиторий GitHub}
\scnnote{На данном этапе в рамках \textit{Технологии OSTIS} (в силу открытости технологии, а также хранения компонентов в виде файлов исходных текстов) для хранения компонентов чаще всего используются хранилища на основе системы контроля версий Git.}
\scnaddlevel{-1}
\scnnote{Помимо исходных текстов компонента в хранилище должна находиться его спецификация, а также набор инструкций, позволяющий интегрировать данный компонент в дочернюю ostis-систему.}
\scnaddlevel{-1}
\scnsuperset{хранилище многократно используемого компонента ostis-систем, хранящегося в виде скомпилированных файлов}
\scnaddlevel{1}
\scnexplanation{Место хранения скомпилированных файлов многократно используемого компонента.}
\scnnote{Помимо скомпилированных файлов компонента в хранилище должна находиться его спецификация, а также набор инструкций, позволяющий интегрировать данный компонент в дочернюю ostis-систему.}
\scnaddlevel{-1}

\scnheader{спецификация многократно используемого компонента ostis-систем}
\scnsubset{спецификация}
\scnidtf{описание многократно используемого компонента ostis-систем}
\scnnote{Каждый \textit{многократно используемый компонент ostis-систем} должен быть специфицирован в рамках библиотеки. Данные спецификации включают в себя основные знания о компоненте, которые позволяют обеспечить построение полной иерархии компонентов и их зависимостей, а также обеспечивают беспрепятственную интеграцию компонентов в дочерние ostis-системы.}

\scnheader{отношение, специфицирующее многократно используемый компонент ostis-систем}
\scnidtf{отношение, которое используется при спецификации многократно используемого компонента ostis-систем}
\scnhaselement{адрес хранилища*}
\scnaddlevel{1}
\scniselement{бинарное отношение}
\scniselement{ориентированное отношение}
\scnexplanation{Связки отношения \textit{адрес хранилища*} связывают многократно используемый компонент, хранящийся в виде внешних файлов и файл, содержащий url-адрес.}
\scnrelfrom{первый домен}{многократно используемый компонент ostis-систем, хранящийся в виде внешних файлов}
\scnrelfrom{второй домен}{файл, содержащий url-адрес многократно используемого компонента ostis-систем}
\scnaddlevel{1}
\scnsuperset{файл}
\scnaddlevel{-1}
\scnaddlevel{-1}
\scnhaselement{автор компонента*}
\scnaddlevel{1}
\scniselement{бинарное отношение}
\scniselement{ориентированное отношение}
\scnexplanation{Связки отношения \textit{автор компонента*} связывают многократно используемый компонент и его разработчика.}
\scnrelfrom{первый домен}{многократно используемый компонент ostis-систем}
\scnrelfrom{второй домен}{субъект}
\scnaddlevel{-1}
\scnhaselement{зависимый компонент*}
\scnaddlevel{1}
\scniselement{бинарное отношение}
\scniselement{ориентированное отношение}
\scnexplanation{Бинарное отношение, связывающее зависимый многократно используемый компонент и компонент, без которого тот не может быть встроен в дочернюю ostis-систему.}
\scnrelfrom{первый домен}{многократно используемый компонент ostis-системы}
\scnrelfrom{второй домен}{зависимый многократно используемый компонент ostis-систем}
\scnaddlevel{-1}
% \scnhaselement{несовместимый компонент*}
% \scnaddlevel{1}
% \scniselement{бинарное отношение}
% \scniselement{неориентированное отношение}
% \scnexplanation{Бинарное отношение, связывающее два компонента, которые не могут одновременно присутствовать в одной ostis-системе.}
% \scnaddlevel{1}
% \scnexplanation{Невозможность существования таких компонентов в одной ostis-системе обуславливается тем, что зачастую эти компоненты принадлежат разных формальным теориям, при объединении которых образуются противоречивые высказывания. Такими компонентами могут быть, например, компонент базы знаний по геометрии Евклида и компонент базы знаний по геометрии Лобачевского.}
% \scnaddlevel{1}
% \scnrelfrom{смотрите}{\nameref{sd_logics}}
% \scnaddlevel{-1}
% \scnaddlevel{-1}
% \scnrelfrom{первый домен}{многократно используемый компонент ostis-систем}
% \scnrelfrom{второй домен}{многократно используемый компонент ostis-систем}
% \scnaddlevel{-1}
\scnnote{Для спецификации многократно используемого компонента также необходимо указывать классы, к которым он принадлежит, дату последнего изменения, описание назначения компонента.}

\scnheader{многократно используемый компонент ostis-системы}
\scnsubdividing{многократно используемый компонент базы знаний; многократно используемый компонент решателя задач; многократно используемый компонент интерфейса}

\scnheader{библиотека многократно используемых компонентов ostis-систем}
\scnidtf{библиотека компонентов ostis-систем, многократно используемых в разных ostis-системах}
\scnidtf{библиотека многократно используемых компонентов OSTIS}
\scnhaselement{\textbf{Библиотека IMS.ostis}}
\scnaddlevel{1}
\scnidtf{библиотека многократно используемых компонентов ostis-систем в составе Метасистемы IMS.ostis}
\scnaddlevel{-1}
\scnnote{Разработчики любой ostis-системы могут включить в ее состав библиотеку, которая позволит им накапливать и распространять результаты своей деятельности среди других участников Экосистемы OSTIS в виде многократно используемых компонентов.}
\scnrelfromset{функциональные возможности}{
	\scnfileitem{Хранение многократно используемых компонентов ostis-систем и их спецификаций.}
	\scnaddlevel{1}
	\scnnote{При этом часть компонентов, специфицированных в рамках библиотеки, могут физически храниться в другом месте ввиду особенностей их  технической реализации (например, исходные тексты платформы интерпретации sc-моделей компьютерных систем могут физически храниться в каком-либо отдельном репозитории, но специфицированы как компонент будут в соответствующей библиотеке). В этом случае спецификация компонента в рамках библиотеки должна также включать описание (1) того где располагается компонент и (2) сценария его автоматической или хотя бы ручной установки в ostis-систему-потребителя. При этом спецификация компонента хранится как непосредственно рядом с компонентом (в виде исходных текстов или в той же самой базе знаний), так и дублируется в рамках библиотеки. Соответственно, существует процедура публикации спецификации компонента в библиотеке и последующая процедура синхронизации обновленной спецификации компонента с библиотекой.}
	\scnaddlevel{-1}
	;\scnfileitem{Хранение сведений о совместимости/несовместимости имеющихся в библиотеке компонентов с учетом версий.}
	;\scnfileitem{Возможность осуществлять просмотр имеющихся компонентов и их спецификаций, а также поиска компонентов по фрагментам их спецификации.}}}
\scnsubdividing{библиотека типовых подсистем ostis-систем;библиотека шаблонов типовых компонентов ostis-систем;библиотека платформ интерпретации sc-моделей компьютерных систем;библиотека многократно используемых компонентов баз знаний; библиотека многократно используемых компонентов решателей задач;библиотека многократно используемых компонентов интерфейсов}
\scnrelfromset{обобщенная декомпозиция}{база знаний библиотеки многократно используемых компонентов ostis-систем \\
\scnaddlevel{1}
\scnnote{База знаний библиотеки мнoгократно используемых компонентов ostis-систем представляет собой иерархию многократно используемых компонентов ostis-систем и их спецификаций.}
\scnaddlevel{-1}    
;решатель задач библиотеки многократно используемых компонентов ostis-систем\\
\scnaddlevel{1}
\scnrelfromset{функциональные возможности}{
	\scnfileitem{Систематизация многократно используемых компонентов ostis-систем.}
	;\scnfileitem{Обеспечение версионирования многократно используемых компонентов ostis-систем.}
	;\scnfileitem{Поиск зависимостей и конфликтов между многократно используемыми компонентами в рамках библиотеки компонентов.}
	;\scnfileitem{Формирование отдельных фрагментов многократно используемых компонентов ostis-систем.}}
\scnaddlevel{-1}
;интерфейс библиотеки многократно используемых компонентов ostis-систем\\
\scnaddlevel{1}
\scnnote{Интерфейс обеспечивает доступ к многократно используемым компонентам. Позволяет получить информацию о зависимых, конфликтующих компонентах.}
\scnrelfromset{декомпозиция}{минимальный интерфейс библиотеки многократно используемых компонентов ostis-систем\\
	\scnaddlevel{1}
	\scnnote{Данный вид интерфейса позволяет менеджеру многократно используемых компонентов ostis-систем, входящему в состав какой-либо дочерней ostis-системы, подключиться к библиотеке многократно используемых компонентов ostis-систем и использовать ее функциональные возможности, то есть, например, получить доступ к спецификации компонентов и установить выбранные компоненты в дочернюю ostis-систему, получить сведения до доступных версиях компонента, его зависимостях и т.д.}
	\scnaddlevel{-1}
	;расширенный интерфейс библиотеки многократно используемых компонентов ostis-систем
	\scnaddlevel{1}
	\scnidtf{графический интерфейс библиотеки многократно используемых компонентов ostis-систем}
	\scnnote{В частном случае у библиотеки может быть расширенный пользовательский интерфейс, который, в отличие от минимального интерфейса, позволяет не только получить доступ к компонентам для дальнейшей работы с ними, но и просматривать существующую структуру библиотеки,  а также компоненты и их элементы в удобном и интуитивно понятном для пользователя виде.}
	\scnaddlevel{-1}}
\scnaddlevel{-1}}

\scnheader{ostis-система}
\scnsuperset{материнская ostis-система}
\scnaddlevel{1}
\scnexplanation{ostis-система, имеющая в своем составе библиотеку многократно используемых компонентов.}
\scnhaselement{Метасистема IMS.ostis}
\scnnote{Материнская ostis-система в свою очередь может являться дочерней ostis-системой для какой-либо другой ostis-системы, заимствуя компоненты из библиотеки, входящей в состав этой другой ostis-системы.}
\scnaddlevel{-1}
\scnsuperset{дочерняя ostis-система}
\scnaddlevel{1}
\scnexplanation{ostis-система, в составе которой имеется компонент, заимствованный из какой-либо библиотеки многократно используемых компонентов.}
\scnaddlevel{-1}

%\scnheader{Решатель задач библиотеки многократно используемых компонентов баз знаний}
%\scnrelfromset{декомпозиция абстрактного sc-агента}{
%	Неатомарный агент поиска компонента\\
%	\scnaddlevel{1}
%	\scnexplanation{Множество агентов, обеспечивающих поиск компонентов в рамках библиотеки по определенным критериям}
%	\scnnote{Существующие критерии регламентированы спецификацией многократно используемых компонентов}
%	\scnaddlevel{-1}
%	;Неатомарный агент формирования фрагментов компонента\\
%	\scnaddlevel{1}
%	\scnexplanation{Множество агентов, позволяющих формировать фрагменты компонентов по заданным критериям, обеспечивая возможность использование только тех знаний, которые непосредственно нужны для функционирования интеллектуальной системы.}
%	\scnrelfromset{декомпозиция абстрактного sc-агента}{
	%		Агент формирования компонента по семантической окрестности заданного понятия;Агент формирования компонента по неатомарным компонентам}
%	\scnaddlevel{-1}
%	;Агент поиска зависимостей
%	\scnaddlevel{1}
%	\scnidtf{Агент поиска всех зависимостей, без которых использование запрашиваемого компонента невозможно}
%	\scnaddlevel{-1}
%	;Агент поиска конфликтов между компонентами
%	\scnaddlevel{1}
%	\scnidtf{Агент проверки отсутствия/присутствия конфликтов между установленным и устанавливаемым компонентами}
%	\scnaddlevel{-1}
%	;Неатомарный агент версионирования\\
%	\scnaddlevel{1}
%	\scnexplanation{Множество агентов, решающих задачу версионирования фрагментов БЗ. Данные агенты позволяют формировать начальное состояние многократно используемого компонента и интегрировать последующие изменения компонента в существую структуру. Затем по запросу пользователя возвращать состояние данной структуры на определенный промежуток времени, что разрешает использование в разработках интеллектуальных систем различных версий одного и того же компонента базы знаний}
%	\scnrelfromset{декомпозиция абстрактного sc-агента}{
	%		Агент формирования начального состояния в дереве
	%		состояний фрагмента БЗ
	%		;Агент интеграции изменений с текущим состоянием
	%		фрагмента БЗ;Агент воссоздания версии фрагмента БЗ по его заданному состоянию;Агент идентификации состояний в дереве состояний
	%		заданного фрагмента БЗ; Агент получения состояния по его идентификатору}
%	\scnaddlevel{-1}
%	;Агент спецификации компонента
%	\scnaddlevel{1}
%	\scnidtf{Агент, позволяющий сформировать спецификацию разрабатываемого компонента для его дальнейшей публикации}
%	\scnaddlevel{-1}}

\bigskip
\scnendstruct \scnendcurrentsectioncomment

\end{SCn}

\scsubsection[\scneditor{Банцевич К.А.}\protect\scnmonographychapter{Глава 5.1. Комплексная библиотека многократно используемых семантически совместимых компонентов интеллектуальных компьютерных систем нового поколения}]{Предметная область и онтология многократно используемых компонентов баз знаний ostis-систем}
\label{sd_know_base_component}
\begin{SCn}
    \scnsectionheader{Предметная область и онтология многократно используемых компонентов баз знаний ostis-систем}
    \begin{scnsubstruct}
        \scnheader{Предметная область многократно используемых компонентов баз знаний ostis-систем}
        \scniselement{предметная область}
        \scnrelto{частная предметная область}{Предметная область многократно используемых компонентов ostis-систем}
        \begin{scnhaselementrolelist}{класс объектов исследования}
            \scnitem{многократно используемый компонент баз знаний ostis-систем}
        \end{scnhaselementrolelist}
        \scnhaselementrole{класс объектов исследования}{отношение, специфицирующее многократно используемый компонент баз знаний ostis-систем}
        
        \scnheader{многократно используемый компонент баз знаний}
        \scnsuperset{предметная область и онтология}
        \scnsubset{раздел базы знаний}
        \scnsuperset{семантическая окрестность}
        \scnrelfrom{смотрите}{\nameref{sd_sem_neigh}}
        \scnsuperset{базовые фрагменты предметных областей и онтологий}
        \scntext{примечание}{Базовый фрагмент предметной области и онтологии включает в себя теоретико-множественную и логическую онтологию, а также фрагменты терминологической онтологии, описывающие основные идентификаторы объектов исследования предметной области.}
        \scntext{примечание}{Данный вид многократно используемых компонентов позволяет использовать только те знания, которые непосредственно необходимы для функционирования интеллектуальных систем, исключив то, что никак не влияет на работу конечной системы (пояснения, примеры, дидактический материал и т.д.).}
        \scnhaselement{Базовый фрагмент теории логических формул, высказываний и логических sc-языков}
        \scnhaselement{Базовый фрагмент теории множеств}
        \scnhaselement{Базовый фрагмент теории связок и отношений}
        \scnsuperset{база знаний прикладной ostis-системы}
        \scntext{примечание}{Целые базы знаний могут быть многократно используемыми компонентами в случае разработки интеллектуальных систем, имеющих схожие функциональные требования.}
        
        \scnheader{многократно используемый компонент базы знаний}
        \scnhaselement{Расширенное ядро базы знаний}
        \scnhaselement{Ядро базы знаний}
        \scntext{пояснение}{\textit{Ядро базы знаний} представляет собой компонент, входящий в состав каждой базы знаний, разрабатываемой по \textit{Технологии OSTIS}, и устанавливаемый в первую очередь.}
        \scntext{примечание}{Список приведенных классов многократно используемых компонентов не является окончательным. В случае, когда разработчик базы знаний интеллектуальной системы считает, что разработанный им компонент сможет стать неотъемлемой частью библиотеки, то компонент будет добавлен в библиотеку, как многократно используемый, в случае, если компонент прошел верификацию и соответствует требованиям разработчиков библиотеки.}
        
        \scnheader{отношение, специфицирующее многократно используемый компонент баз знаний ostis-систем}
        \scnsubset{отношение, специфицирующее многократно используемый компонент ostis-систем}
        \scnhaselement{максимальный класс объектов исследования\scnrolesign}
        \scnhaselement{немаксимальный класс объектов исследования\scnrolesign}
        \scnhaselement{исследуемое отношение\scnrolesign}
        \bigskip
    \end{scnsubstruct}
\end{SCn}


\scsubsection[\scneditor{Шункевич Д.В.}\protect\scnmonographychapter{Глава 5.3. Методика и средства компонентного проектирования решателей задач интеллектуальных компьютерных систем нового поколения}]{Предметная область и онтология многократно используемых компонентов решателей задач ostis-систем}
\label{sd_problem_solver_component}
\begin{SCn}
\scnsectionheader{\currentname}

\scnstartsubstruct

\scnheader{Предметная область многократно используемых компонентов решателей задач ostis-систем}
\scniselement{предметная область}
\scnrelto{частная предметная область}{Предметная область и онтология комплексной библиотеки многократно используемых семантически совместимых компонентов ostis-систем}
\scnsdmainclasssingle{многократно используемый компонент решателей задач ostis-систем}
\scnsdclass{отношение,специфицирующее многократно используемый компонент решателей задач ostis-систем}
\scnhaselementlist{понятие, используемое в предметной области}{агентная scp-программа}
\scnhaselementlist{отношение, используемое в предметной области}{программа sc-агента*}

\scnheader{библиотека многократно используемых компонентов решателей задач}
\scnrelfromset{обобщенная декомпозиция}{база знаний библиотеки многократно используемых компонентов решателей задач\\
\scnaddlevel{1}
\scnnote{База знаний представляет собой иерархию многократно используемых компонентов решателей задач ostis-систем и их спецификацию.}
\scnaddlevel{-1}    
;решатель задач библиотеки многократно используемых компонентов решателей задач\\
\scnaddlevel{1}
\scnnote{Решатель задач позволяет систематизировать, находить зависимости, конфликты.}
\scnaddlevel{-1}
;интерфейс библиотеки многократно используемых компонентов решателей задач\\
\scnaddlevel{1}
\scnnote{Интерфейс библиотеки позволяет подключиться к библиотеке и получить доступ к компонентам хранящимся в ней и к её функционалу}
\scnaddlevel{-1}}

\scnheader{многократно используемый компонент решателей задач}
\scnsuperset{программа}
\scnaddlevel{1}
\scnexplanation{Комбинация компьютерных инструкций и данных, позволяющая аппаратному обеспечению вычислительной системы выполнять вычисления или функции управления.}
\scnaddlevel{-1}
\scnsuperset{пакет программ}
\scnaddlevel{1}
\scnexplanation{Интегрированная система, позволяющая пользователю решать задачу без программирования путем описания задачи и её исходных данных.}
\scnaddlevel{-1}
\scnsuperset{абстрактный sc-агент}
\scnsuperset{решатель задач ostis-системы}
\scnaddlevel{1}
\scnnote{Целые решатели задач могут быть многократно используемыми компонентами в случае разработки интеллектуальных систем, назначение которых совпадает.}
\scnaddlevel{-1}

\scnheader{метод}
\scnidtf{программа}
\scnsuperset{программа на основе нейросетевых моделей}
\scnsuperset{программа на основе генетических алгоритмов}
\scnsuperset{императивная программа}
\scnaddlevel{1}
\scnsuperset{процедурная программа}
\scnsuperset{объектно-ориентированная программа}
\scnaddlevel{-1}
\scnsuperset{декларативная программа}
\scnaddlevel{1}
\scnsuperset{логическая программа}
\scnsuperset{функциональная программа}
\scnaddlevel{-1}
\scnsuperset{программа sc-агента}

\scnheader{абстрактный sc-агент}
\scnnote{Поскольку предполагается, что копии одного и того же \textit{sc-агента} или функционально эквивалентные \textit{sc-агенты} могут работать в разных ostis-системах, будучи при этом физически разными sc-агентами, то целесообразно рассматривать свойства и классификацию не sc-агентов, а классов функционально эквивалентных sc-агентов, которые будем называть \textit{абстрактными sc-агентами}.}
\scnexplanation{Под \textbf{\textit{абстрактным sc-агентом}} понимается некоторый класс функционально эквивалентных \textit{sc-агентов}, разные экземпляры (т.е. представители) которого могут быть реализованы по-разному.
	
Каждый \textbf{\textit{абстрактный sc-агент}} имеет соответствующую ему спецификацию. В спецификацию каждого \textbf{\textit{абстрактного sc-агента}} входит: \vspace{0.2cm}
	
\begin{scnitemize}
	\item указание ключевых \textit{sc-элементов} этого \textit{sc-агента}, т.е. тех \textit{sc-элементов}, хранимых в \textit{sc-памяти}, которые для данного \textit{sc-агента} являются «точками опоры»;
	\item формальное описание условий инициирования данного \textit{sc-агента}, т.е. тех \textit{ситуация} в \textit{sc-памяти}, которые инициируют деятельность данного \textit{sc-агента};
	\item формальное описание первичного условия инициирования данного \textit{sc-агента}, т.е. такой ситуации в \textit{sc-памяти}, которая побуждает \textit{sc-агента} перейти в активное состояние и начать проверку наличия своего полного условия инициирования (для \textit{внутренних абстрактных sc-агентов});
	\item строгое, полное, однозначно понимаемое описание деятельности данного \textit{sc-агента}, оформленное при помощи каких-либо понятных, общепринятых средств, не требующих специального изучения, например на естественном языке.
	\item описание результатов выполнения данного \textit{sc-агента}.
\end{scnitemize}
}
\scnsubdividing{неатомарный абстрактный sc-агент;атомарный абстрактный sc-агент}
\scnsubdividing{внутренний абстрактный sc-агент;эффекторный абстрактный sc-агент;рецепторный абстрактный sc-агент}
\scnsubdividing{абстрактный sc-агент, не реализуемый на Языке SCP;абстрактный sc-агент, реализуемый на Языке SCP}
\scnsubdividing{абстрактный sc-агент интерпретации scp-программ;абстрактный программный sc-агент;абстрактный sc-метаагент}
\scnsubdividing{платформенно-зависимый абстрактный sc-агент\\
\scnaddlevel{1}
\scnsuperset{абстрактный sc-агент, не реализуемый на Языке SCP}
\scnaddlevel{-1}
;платформенно-независимый абстрактный sc-агент}

\scnheader{абстрактный sc-агент, не реализуемый на Языке SCP}
\scnidtf{абстрактный sc-агент, который не может быть реализован на платформенно-независимом уровне}
\scnsubdividing{эффекторный абстрактный sc-агент;рецепторный абстрактный sc-агент
;абстрактный sc-агент интерпретации scp-программ}

\scnheader{абстрактный sc-агент, реализуемый на Языке SCP}
\scnidtf{абстрактный sc-агент, который может быть реализован на платформенно-независимом уровне}
\scnsubdividing{абстрактный sc-метаагент;абстрактный программный sc-агент, реализуемый на Языке SCP}

\scnheader{абстрактный программный sc-агент}
\scnsubdividing{эффекторный абстрактный sc-агент;рецепторный абстрактный sc-агент
;абстрактный программный sc-агент, реализуемый на Языке SCP}

\scnheader{неатомарный абстрактный sc-агент}
\scnexplanation{Под \textbf{\textit{неатомарным абстрактным sc-агентом}} понимается \textit{абстрактный sc-агент}, который декомпозируется на коллектив более простых \textit{абстрактных sc-агентов}, каждый из которых в свою очередь может быть как \textit{атомарным абстрактным sc-агентом}, так и \textbf{\textit{неатомарным абстрактным sc-агентом}}. При этом в каком либо варианте \textit{декомпозиции абстрактного sc-агента*} дочерний \textbf{\textit{неатомарный абстрактный sc-агент}} может стать \textit{атомарным абстрактным sc-агентом}, и реализовываться соответствующим образом.}

\scnheader{атомарный абстрактный sc-агент}
\scnexplanation{Под \textbf{\textit{атомарным абстрактным sc-агентом}} понимается \textit{абстрактный sc-агент}, для которого уточняется платформа его реализации, т.е. существует соответствующая связка отношения \textit{программа sc-агента*}.}
\scnsubdividing{платформенно-независимый абстрактный sc-агент;платформенно-зависимый абстрактный sc-агент}

\scnheader{платформенно-независимый абстрактный sc-агент}
\scnexplanation{К \textbf{\textit{платформенно-независимым абстрактным \mbox{sc-агентам}}} относят \textit{атомарные абстрактные sc-агенты}, реализованные на базовом языке программирования Технологии OSTIS, т.е. на \textit{Языке SCP}.
	
При описании \textbf{\textit{платформенно-независимых абстрактных sc-агентов}} под платформенной независимостью понимается платформенная независимость с точки зрения Технологии OSTIS, т.е реализация на специализированном языке программирования, ориентированном на обработку семантических сетей (\textit{Языке SCP}), поскольку \textit{атомарные sc-агенты}, реализованные на указанном языке могут свободно переноситься с одной платформы интерпретации \textit{sc-моделей} на другую. При этом языки программирования, традиционно считающиеся платформенно-независимыми в данном случае не могут считаться таковыми.
	
Существуют \textit{sc-агенты}, которые принципиально не могут быть реализованы на платформенно-независимом уровне, например, собственно \textit{sc-агенты} интерпретации \textit{sc-моделей} или рецепторные и эффекторные \textit{sc-агенты}, обеспечивающие взаимодействие с внешней средой.}

\scnheader{платформенно-зависимый абстрактный sc-агент}
\scnexplanation{К \textbf{\textit{платформенно-зависимым абстрактным sc-агентам}} относят \textit{атомарные абстрактные sc-агенты}, реализованные ниже уровня sc-моделей, т.е. не на \textit{Языке SCP}, а на каком-либо другом языке описания программ.
	
Существуют \textit{sc-агенты}, которые принципиально должны быть реализованы на платформенно-зависимом уровне, например, собственно \textit{sc-агенты} интерпретации \textit{sc-моделей} или рецепторные и эффекторные \textit{sc-агенты}, обеспечивающие взаимодействие с внешней средой.}

\scnheader{внутренний абстрактный sc-агент}
\scnexplanation{Каждый \textbf{\textit{внутренний абстрактный sc-агент}} обозначает класс \textit{sc-агентов}, которые реагируют на события в \textit{sc-памяти} и осуществляют преобразования исключительно в рамках этой же \textit{sc-памяти}.}

\scnheader{эффекторный абстрактный sc-агент}
\scnexplanation{Каждый \textbf{\textit{эффекторный абстрактный sc-агент}} обозначает класс \textit{sc-агентов}, которые реагируют на события в \textit{sc-памяти} и осуществляют преобразования во внешней относительно данной \textit{ostis-системы} среде.}

\scnheader{рецепторный абстрактный sc-агент}
\scnexplanation{Каждый \textbf{\textit{рецепторный абстрактный sc-агент}} обозначает класс \textit{sc-агентов}, которые реагируют на события во внешней относительно данной \textit{ostis-системы} среде и осуществляют преобразования в памяти данной системы.}

\scnheader{абстрактный sc-агент, не реализуемый на Языке SCP}
\scnexplanation{Каждый \textbf{\textit{абстрактный sc-агент, не реализуемый на Языке SCP}} должен быть реализован на уровне платформы интерпретации sc-моделей, в том числе, аппаратной. К таким \textit{абстрактным sc-агентам} относятся абстрактные sc-агенты интерпретации scp-программ, а также эффекторные и рецепторные абстрактные sc-агенты.}

\scnheader{абстрактный sc-агент, реализуемый на Языке SCP}
\scnexplanation{Каждый \textbf{\textit{абстрактный sc-агент, реализуемый на Языке SCP}} может быть реализован на Языке SCP, то есть платформенно-независимом уровне, но при необходимости, может реализовываться и на уровне платформы, например, с целью повышения производительности.}

\scnheader{абстрактный sc-агент интерпретации scp-программ}
\scnexplanation{К \textbf{\textit{абстрактным sc-агентам интерпретации scp-программ}} относятся не реализуемые на платформенно-независимом уровне \textit{абстрактные sc-агенты}, обеспечивающие интерпретацию \textit{scp-программ} и \textit{\mbox{scp-метапрограмм}}, в том числе создание \textit{scp-процессов}, собственно интерпретацию \textit{scp-операторов}, а также другие вспомогательные действия. По сути, агенты данного класса обеспечивают работу sc-агентов более высоких уровней (программных sc-агентов и sc-метаагентов), реализованных на Языке SCP, в частности, обеспечивают соблюдение указанными агентами общих принципов синхронизации.}

\scnheader{абстрактный программный sc-агент}
\scnexplanation{К \textbf{\textit{абстрактным программным sc-агентам}} относятся все \textit{абстрактные sc-агенты}, обеспечивающие основной функционал системы, то есть ее возможность решать те или иные задачи. Агенты данного класса должны работать в соответствии с общими принципами синхронизации деятельности субъектов в sc-памяти.}

\scnheader{абстрактный sc-метаагент}
\scnexplanation{Задачей \textbf{\textit{абстрактных sc-метаагентов}} является координация деятельности \textit{абстрактных программных sc-агентов}, в частности, решение проблемы взаимоблокировок. Агенты данного класса могут быть реализованы на Языке SCP, однако для синхронизации их деятельности используются другие принципы, соответственно, для реализации таких агентов требуется Язык SCP другого уровня, типология операторов которого полностью аналогична типологии scp-операторов, однако эти операторы имеют другую операционную семантику, учитывающую отличия в принципах синхронизации (работы с \textit{блокировками*}). Программы такого языка будем называть \textit{scp-метапрограммами}, соответствующие им \mbox{\textit{процессы в sc-памяти} – \textit{scp-метапроцессами}}, операторы – \textit{scp-метаоператорами}.}

\scnheader{агентная scp-программа}
\scnidtf{Язык SCP}
\scnexplanation{Агентные scp-программы представляют собой частный случай scp-программ вообще, однако заслуживают отдельного рассмотрения, поскольку используются наиболее часто. scp-программы данного класса представляют собой реализации программ агентов обработки знаний, и имеют жестко фиксированный набор параметров. Каждая такая программа имеет ровно два in-параметра’. Значение первого параметра является знаком бинарной ориентированной пары, являющееся вторым компонентом связки отношения первичное условие инициирования* для абстрактного sc-агента, в множество программ sc-агента* которого входит рассматриваемая агентная scp-программа.
	
Значением второго параметра является sc-элемент, с которым непосредственно связано событие, в результате возникновения которого был инициирован соответствующий sc-агент, т.е., например, сгенерированная либо удаляемая sc-дуга или sc-ребро.}

\scnheader{программа sc-агента*}
\scnexplanation{Связки отношения \textit{программа sc-агента}* связывают между собой sc-узел, обозначающий \textit{атомарный абстрактный sc-агент} и sc-узел, обозначающий множество программ, реализующих указанный \textit{атомарный абстрактный sc-агент}.}

\scnheader{решатель задач ostis-системы}
\scnsuperset{решатель задач с использованием хранимых методов}
\scnaddlevel{1}
\scnidtf{решатель, способный решать задачи тех классов, для которых на данный момент времени известен соответствующий метод решения}
\scnsuperset{решатель задач на основе нейросетевых моделей}
\scnsuperset{решатель задач на основе генетических алгоритмов}
\scnsuperset{решатель задач на основе императивных программ}
\scnaddlevel{1}
\scnsuperset{решатель задач на основе процедурных программ}
\scnsuperset{решатель задач на основе объектно-ориентированных программ}
\scnaddlevel{-1}
\scnsuperset{решатель задач на основе декларативных программ}
\scnaddlevel{1}
\scnsuperset{решатель задач на основе логических программ}
\scnsuperset{решатель задач на основе функциональных программ}
\scnaddlevel{-1}
\scnaddlevel{-1}
\scnsuperset{решатель задач в условиях, когда метод решения задач данного класса в текущий момент времени не известен}
\scnaddlevel{1}
\scnidtf{решатель, реализующий стратегии решения задач, позволяющие породить метод решения задачи, который в текущий момент времени не известен ostis-системе}
\scnidtf{решатель, использующий для решения задач метаметоды, соответствующие более общим классам задач по отношению к заданной}
\scnidtf{решатель задач, позволяющий породить метод, который является частным по отношению какому-либо известному ostis-системе методу и интерпретируется соответствующей машиной обработки знаний}
\scnsuperset{решатель, реализующий стратегию поиска путей решения задачи в глубину}
\scnsuperset{решатель, реализующий стратегию поиска путей решения задачи в ширину}
\scnsuperset{решатель, реализующий стратегию проб и ошибок}
\scnsuperset{решатель, реализующий стратегию разбиения задачи на подзадачи}
\scnsuperset{решатель, реализующий стратегию решения задач по аналогии}
\scnsuperset{решатель, реализующий концепцию интеллектуального пакета программ}
\scnaddlevel{-1}

\scnheader{отношение, специфицирующее многократно используемый компонент решателей задач ostis-систем}
\scnsubset{отношение, специфицирующее многократно используемый компонент ostis-систем}
\scnhaselement{первичное условие инициирования*}
\scnaddlevel{1}
\scnexplanation{Связки отношения первичное условие инициирования* связывают между собой sc-узел, обозначающий абстрактный sc-агент и бинарную ориентированную пару, описывающую первичное условие инициирования данного абстрактного sc-агента, т.е. такой ситуации в sc-памяти, которая побуждает sc-агента перейти в активное состояние и начать проверку наличия своего полного условия инициирования.
	
Первым компонентом данной ориентированной пары является знак некоторого подмножества понятия событие, например событие добавления выходящей sc-дуги, т.е. по сути конкретный тип события в sc-памяти.
	
Вторым компонентом данной ориентированной пары является произвольный в общем случае sc-элемент, с которым непосредственно связан указанный тип события в sc-памяти, т.е., например, sc-элемент, из которого выходит либо в который входит генерируемая либо удаляемая sc-дуга, либо sc-ссылка, содержимое которой было изменено.
	
После того, как в sc-памяти происходит некоторое событие, активизируются все активные sc-агенты, первичное условие инициирования* которых соответствует произошедшему событию.}
\scnrelfrom{первый домен}{абстрактный sc-агент}
\scnrelfrom{второй домен}{бинарная ориентированная пара}
\scnaddlevel{-1}
\scnhaselement{условие инициирования и результат*}
\scnaddlevel{1}
\scnexplanation{Связки отношения условие инициирования и результат* связывают между собой sc-узел, обозначающий абстрактный sc-агент и бинарную ориентированную пару, связывающую условие инициирования данного абстрактного sc-агента и результаты выполнения данного экземпляров данного sc-агента в какой-либо конкретной системе.
	
Указанную ориентированную пару можно рассматривать как логическую связку импликации, при этом на sc-переменные, присутствующие в обеих частях связки, неявно накладывается квантор всеобщности, на sc-переменные, присутствующие либо только в посылке, либо только в заключении неявно накладывается квантор существования.
	
Первым компонентом указанной ориентированной пары является логическая формула, описывающая условие инициирования описываемого абстрактного sc-агента, то есть конструкции, наличие которой в sc-памяти побуждает sc-агент начать работу по изменению состояния sc-памяти. Данная логическая формула может быть как атомарной, так и неатомарной, в которой допускается использование любых связок логического языка.
	
Вторым компонентом указанной ориентированной пары является логическая формула, описывающая возможные результаты выполнения описываемого абстрактного sc-агента, то есть описание произведенных им изменений состояния sc-памяти. Данная логическая формула может быть как атомарной, так и неатомарной, в которой допускается использование любых связок логического языка.}
\scnrelfrom{первый домен}{абстрактный sc-агент}
\scnrelfrom{второй домен}{бинарная ориентированная пара}
\scnaddlevel{-1}
\scnhaselement{эквивалентный компонент*}
\scnaddlevel{1}
\scnhaselement{неориентированное отношение}
\scnexplanation{Бинарное отношение связывающее функционально эквивалентные многократно используемые компоненты решателей задач.}
\scnrelfrom{первый домен}{многократно используемый компонент решателей задач}
\scnrelfrom{второй домен}{многократно используемый компонент решателей задач}
\scnaddlevel{-1}

\scnheader{Решатель задач библиотеки многократно используемых компонентов решателей задач}
\scnrelfromset{декомпозиция абстрактного sc-агента}{
Неатомарный агент поиска компонента\\
\scnaddlevel{1}
\scnexplanation{Множество агентов, обеспечивающих поиск компонентов в рамках библиотеки по определенным критериям}
\scnnote{Существующие критерии регламентированы спецификацией многократно используемых компонентов}
\scnaddlevel{-1}
;Агент поиска зависимостей
\scnaddlevel{1}
\scnidtf{Агент поиска всех зависимостей, без которых использование запрашиваемого компонента невозможно}
\scnaddlevel{-1}
;Агент поиска эквивалентных компонентов
\scnaddlevel{1}
\scnidtf{Агент поиска всех функционально эквивалентных компонентов, которые дают эквивалентный результат}
\scnaddlevel{-1}
;Агент поиска конфликтов между компонентами
\scnaddlevel{1}
\scnidtf{Агент проверки отсутствия/присутствия конфликтов между установленным и устанавливаемым компонентами}
\scnaddlevel{-1}
;Агент спецификации компонента
\scnaddlevel{1}
\scnidtf{Агент, позволяющий сформировать спецификацию разрабатываемого компонента для его дальнейшей публикации}
\scnaddlevel{-1}}


\bigskip
\scnendstruct \scnendcurrentsectioncomment

\end{SCn}

\scsubsubsection[\scneditor{Шункевич Д.В.}\protect\scnmonographychapter{Глава 5.3. Методика и средства компонентного проектирования решателей задач интеллектуальных компьютерных систем нового поколения}]{Предметная область и онтология многократно используемых методов, хранимых в памяти ostis-систем и интерпретируемых их внутренними агентами}
\label{sd_method_agent}

\scsubsubsection[\scneditors{Шункевич Д.В.;Зотов Н.В.;Орлов М.К.}\protect\scnmonographychapter{Глава 5.3. Методика и средства компонентного проектирования решателей задач интеллектуальных компьютерных систем нового поколения}]{Предметная область и онтология многократно используемых внутренних агентов ostis-систем}
\label{sd_internal_agent}

\scsubsection[\scneditor{Садовский М.Е.}\protect\scnmonographychapter{Глава 5.4. Методика и средства компонентного проектирования интерфейсов интеллектуальных компьютерных систем нового поколения}]{Предметная область и онтология многократно используемых компонентов интерфейсов ostis-систем}
\label{sd_component_interface}

\scsubsection[\scneditor{Шункевич Д.В.}\protect\scnmonographychapter{Глава 5.1. Комплексная библиотека многократно используемых семантически совместимых компонентов интеллектуальных компьютерных систем нового поколения}]{Предметная область и онтология многократно используемых встраиваемых ostis-систем}
\label{sd_embed_sys}

\scsection[\scneditor{Шункевич Д.В.}]{Предметная область и онтология действий и методик проектирования ostis-систем}
\label{sd_actions_methodology_design}

\scsubsection[\scneditors{Банцевич К.А.;Бутрин С.В.}\protect\scnmonographychapter{Глава 5.2. Методика и средства проектирования и анализа качества баз знаний интеллектуальных компьютерных систем нового поколения}]{Предметная область и онтология действий и методик проектирования баз знаний ostis-систем}
\label{sd_actions_methodology_know_base_design}

\scsubsection[\scneditor{Шункевич Д.В.}\protect\scnmonographychapter{Глава 5.3. Методика и средства компонентного проектирования решателей задач интеллектуальных компьютерных систем нового поколения}]{Предметная область и онтология действий и методик проектирования решателей задач ostis-систем}
\label{sd_actions_methodology_problem_solver_design}

\scsubsection[\scneditor{Садовский М.Е.}\protect\scnmonographychapter{Глава 5.4. Методика и средства компонентного проектирования интерфейсов интеллектуальных компьютерных систем нового поколения}]{Предметная область и онтология действий и методик проектирования интерфейсов ostis-систем}
\label{sd_actions_methodology_interface_design}

\scsection{Предметная область и онтология действий и методик \uline{обучения} проектированию ostis-систем}
\label{sd_actions_methodology_learning_design}

\scsection[\scneditor{Шункевич Д.В.}]{Предметная область и онтология средств проектирования ostis-систем}
\label{sd_fund_design}

\scsubsection[\scneditor{Бутрин С.В.}\protect\scnmonographychapter{Глава 5.2. Методика и средства проектирования и анализа качества баз знаний интеллектуальных компьютерных систем нового поколения}]{Логико-семантическая модель комплекса встраиваемых ostis-систем автоматизации проектирования баз знаний ostis-систем}
\label{logical_model_embed_automation_design}

\scsubsubsection[\scneditors{Бутрин С.В.}\protect\scnmonographychapter{Глава 5.2. Методика и средства проектирования и анализа качества баз знаний интеллектуальных компьютерных систем нового поколения}]{Логико-семантическая модель ostis-системы редактирования, сборки и ввода исходных текстов различных компонентов проектируемой базы знаний в память ostis-системы}
\label{edit_assem_logical_model}

\scsubsubsection[\scneditor{Бутрин С.В.}\protect\scnmonographychapter{Глава 5.2. Методика и средства проектирования и анализа качества баз знаний интеллектуальных компьютерных систем нового поколения}]{Логико-семантическая модель ostis-системы редактирования проектируемой базы знаний ostis-системы на уровне её внутреннего представления}
\label{edit_tools_proj_logical_model}

\scsubsubsection[\scneditor{Бутрин С.В.}\protect\scnmonographychapter{Глава 5.2. Методика и средства проектирования и анализа качества баз знаний интеллектуальных компьютерных систем нового поколения}]{Логико-семантическая модель ostis-системы обнаружения и анализа ошибок и противоречий в базе знаний ostis-системы}
\label{detec_error_logical_model}

\scsubsubsection[\scneditor{Бутрин С.В.}\protect\scnmonographychapter{Глава 5.2. Методика и средства проектирования и анализа качества баз знаний интеллектуальных компьютерных систем нового поколения}]{Логико-семантическая модель ostis-системы обнаружения и анализа информационных дыр в базе знаний ostis-системы}
\label{detec_hole_logical_model}

\scsubsubsection[\scneditors{Бутрин С.В.;Банцевич К.А.}\protect\scnmonographychapter{Глава 5.2. Методика и средства проектирования и анализа качества баз знаний интеллектуальных компьютерных систем нового поколения}]{Логико-семантическая модель ostis-системы автоматизации управления взаимодействием разработчиков различных категорий в процессе проектирования базы знаний ostis-системы}
\label{author_logical_model}

\scsubsection[\scneditor{Шункевич Д.В.}\protect\scnmonographychapter{Глава 5.3. Методика и средства компонентного проектирования решателей задач интеллектуальных компьютерных систем нового поколения}]{Логико-семантическая модель комплекса ostis-систем автоматизации проектирования решателей задач ostis-систем}
\label{problem_solver_design_tools}

\scsubsubsection[\scneditor{Шункевич Д.В.}\protect\scnmonographychapter{Глава 5.3. Методика и средства компонентного проектирования решателей задач интеллектуальных компьютерных систем нового поколения}]{Логико-семантическая модель ostis-системы автоматизации проектирования программ Базового языка программирования ostis-систем}
\label{programs_design_tools}

\scsubsubsection[\scneditor{Шункевич Д.В.}\protect\scnmonographychapter{Глава 5.3. Методика и средства компонентного проектирования решателей задач интеллектуальных компьютерных систем нового поколения}]{Логико-семантическая модель ostis-системы автоматизации проектирования внутренних агентов ostis-систем, а также коллективов таких агентов}
\label{agents_design_tools}

\scsubsubsection[\scneditors{Ковалев М.В.;Крощенко А.А.;Михно Е.В.}\protect\scnmonographychapter{Глава 3.6. Конвергенция и интеграция искусственных нейронных сетей с базами знаний в интеллектуальных компьютерных системах нового поколения}]{Логико-семантическая модель ostis-системы автоматизации проектирования искусственных нейронных сетей, семантически совместимых с базами знаний ostis-систем}
\label{autom_design_neural_network_logical_model}

\scsubsection[\scneditors{Садовский М.Е.;Жмырко А.В.}\protect\scnmonographychapter{Глава 5.4. Методика и средства компонентного проектирования интерфейсов интеллектуальных компьютерных систем нового поколения}]{Логико-семантическая модель ostis-системы автоматизации проектирования интерфейсов ostis-систем}
\label{autom_design_interface_logical_model}

\scsubsection[\scnmonographychapter{Глава 7.2. Экосистема интеллектуальных компьютерных систем нового поколения (Экосистема OSTIS) и реализация рынка знаний на ее основе}]{Предметная область и онтология ostis-систем автоматизации проектирования различных классов ostis-систем}
\label{sd_autom_design_class}

\scsection{Предметная область и онтология ostis-систем обучения проектированию ostis-систем и их компонентов}
\label{sd_learn_design_component}
\scsectionfamily{Часть 6 Стандарта OSTIS. Платформы реализации интеллектуальных компьютерных систем нового поколения}
\label{part_platforms}

\scsection[
    \protect\scneditor{Шункевич Д.В.}
    \protect\scnmonographychapter{Глава 6.1. Универсальная модель интерпретатора внутренних агентов решателя задач интеллектуальной компьютерной системы нового поколения}
    ]{Предметная область и онтология методов и средств производства ostis-систем}
\label{sd_method_prod_sys}
\begin{SCn}
\scnsectionheader{Предметная область и онтология методов и средств производства ostis-систем}
\begin{scnsubstruct}
	\begin{scnrelfromlist}{ключевое понятие}
		\scnitem{sc-модель кибернетической системы}
		\scnitem{ostis-система}
		\scnitem{ostis-платформа}
		\scnitem{sc-память}
		\scnitem{sc-модель базы знаний}
		\scnitem{sc-модель решателя задач}
		\scnitem{sc-модель интерфейса кибернетической системы}
		\scnitem{sc-машина}
		\scnitem{scp-машина}
	\end{scnrelfromlist}
	\scntext{введение}{Рассмотрим предлагаемый подход к организации реализации \textit{ostis-систем}. Одним из ключевых принципов \textit{Технологии OSTIS} является обеспечение платформенной независимости \textit{ostis-систем}.}
	
	\scnheader{платформенная независимость ostis-систем}
	\scnsuperset{платформенная независимость интеллектуальных компьютерных систем}
	\scntext{пояснение}{Строгое разделение логико-семантической модели кибернетической системы (sc-модели кибернетической системы) и платформы интерпретации sc-моделей кибернетических систем (ostis-платформы).}
	\begin{scnrelfromset}{преимущества}
		\scnfileitem{Перенос \textit{ostis-системы} с одной платформы на другую (например более новую и эффективную или ориентированную на определенный класс устройств) выполняется с минимальными накладными расходами (в идеальном случае — вообще сводится просто к загрузке sc-модели кибернетической системы на платформу).}
		\scnfileitem{Компоненты \textit{ostis-систем} становятся универсальными, то есть могут использоваться в любых ostis-системах, где их использование является целесообразным.}
		\scnfileitem{Развитие платформы и развитие sc-моделей систем может осуществляться параллельно и независимо друг от друга, в общем случае отдельными независимыми коллективами разработчиков по своим собственным правилам и методикам.}
	\end{scnrelfromset}
	
	\scnheader{логико-семантическая модель кибернетической системы}
	\scnidtf{формальная модель (формальное описание) функционирования кибернетической системы, состоящая из (1) формальной модели информации, хранимой в памяти кибернетической системы и (2) формальной модели коллектива агентов, осуществляющих обработку указанной информации.}
	\scnsuperset{sc-модель кибернетической системы}
	\begin{scnindent}
		\scnidtf{логико-семантическая модель кибернетической системы, представленная в SC-коде}
		\scnidtf{логико-семантическая модель ostis-системы, которая, в частности, может быть функционально эквивалентной моделью какой-либо кибернетической системы, не являющейся ostis-системой}
	\end{scnindent}
	
	\scnheader{кибернетическая система}
	\scnsuperset{компьютерная система}
	\begin{scnindent}
		\scnidtf{искусственная кибернетическая система}
		\scnsuperset{ostis-система}
		\begin{scnindent}
			\scnidtf{компьютерная система, построенная по Технологии OSTIS на основе интерпретации спроектированной логико-семантической sc-модели этой системы}
		\end{scnindent}
	\end{scnindent}
	
	\scnheader{ostis-система}
	\scnsubset{субъект}
	\begin{scnrelfromset}{обобщенная декомпозиция}
		\scnitem{sc-модель кибернетической системы}
		\scnitem{ostis-платформа}
	\end{scnrelfromset}
	\scntext{примечание}{При условии строгого разделения \textit{sc-модели кибернетической системы} и \textit{ostis-платформы}, а также обеспечении универсальности \textit{ostis-платформы}, то есть возможности интерпретировать \uline{любую} \textit{sc-модель кибернетической системы} на любом варианте \textit{ostis-платформы}, этап \uline{реализации} \textit{ostis-системы} фактически сводится к загрузке \textit{sc-модели кибернетической системы} на выбранный вариант \textit{ostis-платформы}.}
	
	\scnheader{sc-модель кибернетической системы}
	\begin{scnrelfromset}{обобщенная декомпозиция}
		\scnitem{sc-память}
		\scnitem{sc-модель базы знаний}
		\scnitem{sc-модель решателя задач}
		\scnitem{sc-модель интерфейса кибернетической системы}
	\end{scnrelfromset}
\begin{scnindent}
	\scntext{примечание}{Явное выделение \textit{sc-модели базы знаний}, \textit{sc-модели решателя задач} и \textit{sc-модели интерфейса кибернетической системы} в рамках \textit{sc-модели кибернетической системы} является в известной мере условным, поскольку для обеспечения платформенной независимости \textit{sc-модели кибернетической системы} и \textit{решатель задач}, и \textit{интерфейс системы} описываются средствами \textit{SC-кода} и, таким образом, тоже являются частью \textit{базы знаний}. Такое явное выделение указанных компонентов обусловлено удобством проектирования и сопровождения системы.}
\end{scnindent}
	\scntext{требование}{Ни на одном из этапов решения любой \textit{информационной задачи} в данной системе не должны учитываться особенности той платформы, на которой в дальнейшем будет интерпретироваться указанная \textit{sc-модель}. Аналогично ключевым требованием к \textit{ostis-платформе} является обеспечение интерфейса доступа (поиска и преобразования) к хранимой в \textit{sc-памяти} информации некоторым универсальным способом, не зависящим от особенностей реализации конкретной платформы. Таким образом, важнейшей задачей для обеспечения платформенной независимости \textit{ostis-систем} является четкая спецификация требований, предъявляемых к каждой реализации \textit{ostis-платформы}, то есть \uline{стандартизация} \textit{ostis-платформ}. Важно отметить, что такая стандартизация не должна зависеть от того, в каком виде реализуется \textit{ostis-платформа}, и, соответственно, подходить и для аппаратного варианта реализации.}
	
	\scnheader{sc-память}
	\scnidtf{абстрактная sc-память}
	\scnidtf{sc-хранилище}
	\scnidtf{семантическая память, хранящая конструкции SC-кода}
	\scnidtf{хранилище конструкций SC-кода}
	\scntext{примечание}{\textbf{\textit{sc-память}} представляет собой с одной стороны общую среду для хранения \textit{базы знаний}, а с другой стороны --- среду для взаимодействия \textit{sc-агентов}. При этом каждый \textit{sc-агент} опирается при работе на некоторые известные ему \textit{sc-элементы}, хранящиеся в \textit{sc-памяти} (\textit{ключевые sc-элементы} данного \textit{sc-агента}).}
	\scnrelfrom{функциональные возможности}{Функциональные возможности sc-памяти}
	\begin{scnindent}
		\begin{scneqtoset}
			\scnfileitem{Хранение конструкций \textit{SC-кода}.}
			\scnfileitem{Хранение внешних по отношению к \textit{SC-коду} информационных конструкций (файлов). В общем случае хранение файлов может быть реализовано отличным от хранения \textit{sc-конструкций} образом.}
			\scnfileitem{Доступ (чтение, создание, удаление) к конструкциям \textit{SC-кода}, реализуемый через соответствующий программный или аппаратный интерфейс. Такой интерфейс по сути представляет собой язык микропрограммирования, позволяющий реализовывать на его основе более сложные процедуры обработки хранимых конструкций, в том числе --- операторы \textit{Языка SCP}, набор которых по сути определяет перечень команд такого языка микропрограммирования. Сама \textit{sc-память} в этом плане является пассивной и просто выполняет команды, инициируемые извне какими-либо субъектами.}
		\end{scneqtoset}
		\scntext{примечание}{Отметим, что разделение функции хранения и доступа является достаточно условным, поскольку реализовать функцию хранения конструкций без возможности доступа к ним хотя бы на самом низком уровне представляется нецелесообразным, ведь пользоваться таким хранилищем будет невозможно.}
	\end{scnindent}
	\scntext{примечание}{Термины \textit{sc-память} и \textit{абстрактная sc-память} являются синонимами в том смысле, что говоря об \textit{sc-памяти} мы подразумеваем некоторую абстракцию, для которой не уточняется ее максимальный объем (максимальное количество \textit{sc-элементов}, которые могут одновременно храниться в такой памяти), конкретный способ хранения \textit{sc-элементов}, средства обеспечения надежности хранения и так далее. Все указанные особенности уточняются на уровне \textit{реализации sc-памяти} в аппаратном варианте или варианте программной модели на базе какой-либо другой архитектуры.}
	
	\scnheader{ostis-платформа}
	\scntext{примечание}{Важно отметить, что универсальность конкретного варианта реализации \textit{ostis-платформы} очевидно ограничивается физической (аппаратной) частью этой реализации. Например, если аппаратная часть выбранного варианта платформы представляет собой обычный персональный компьютер, то без добавления дополнительных аппаратных компонентов система не сможет решать задачи, связанные с физическим перемещением себя и других объектов в пространстве, даже если программная часть системы способна выполнить необходимые расчеты. Говоря другими словами, любая \textit{ostis-платформа} всегда будет ограничена в решении \textit{поведенческих задач} каких-либо классов, какими бы мощными физическими ресурсами она не обладала. Таким образом, корректнее говорить об \uline{универсальности \textit{ostis-платформы} в контексте решения \textit{информационных задач}}, то есть возможности интерпретировать любые \textit{sc-модели кибернетических систем} независимо от того, какого рода \textit{информационные задачи} решают эти системы.}
	
	\scnheader{sc-машина}
	\scnidtf{абстрактная sc-машина}
	\scniselement{абстрактная машина обработки информации}
	\begin{scnrelfromlist}{аналог}
		\scnitem{машина Поста}
		\begin{scnindent}
			\scnrelfrom{источник}{\scncite{TuringMachine}}
		\end{scnindent}
		\scnitem{машина Тьюринга}
		\begin{scnindent}
			\begin{scnrelfromlist}{источник}
				\scnitem{\scncite{Neumann1993}}
				\scnitem{\scncite{NeumanMachine}}
			\end{scnrelfromlist}
		\end{scnindent}
	\end{scnrelfromlist}
	\scnidtf{обобщение всевозможных реализаций ostis-платформ, для которого задаются общие функциональные требования}
	\scnidtf{обобщенная модель, описывающая функционирование любой ostis-платформы независимо от способа ее реализации}
	\scnidtf{обобщенная модель, определяющая общие закономерности любой ostis-платформы независимо от способа ее реализации}
	\scnidtf{обобщенный информационный образ ostis-платформы}
	\scnrelto{обобщенная модель}{ostis-платформа}
	\begin{scnrelfromset}{обобщенная декомпозиция}
		\scnitem{sc-память}
		\begin{scnindent}
			\scnrelto{обобщенная модель}{реализация sc-памяти}
		\end{scnindent}
		\scnitem{абстрактная машина обработки знаний}
		\begin{scnindent}
			\scnsubset{абстрактный sc-агент}
		\end{scnindent}
	\end{scnrelfromset}
	\scnsuperset{scp-машина}
	\begin{scnindent}
		\scnrelto{обобщенная модель}{scp-интерпретатор}
		\scnidtf{sc-машина, обеспечивающая интерпретацию базового языка программирования ostis-систем}
		\scnidtf{обобщенная модель интерпретатора базового языка программирования ostis-систем}
		\scnidtf{обобщенная модель, определяющая общие принципы интерпретации базового языка программирования ostis-систем}
		\scnidtf{обобщенная модель операционной семантики базового языка программирования ostis-систем}
		\scntext{примечание}{Потенциально можно говорить о нескольких возможных функционально эквивалентных вариантах \textit{scp-машины}, которые будут соответствовать разным вариантам базового языка программирования. В рамках текущей версии \textit{Технологии OSTIS} фиксируется как денотационная семантика \textit{Языка SCP}, так и его операционная семантика, реализуемая в виде \textit{Абстрактной scp-машины}. Более подробно об этом говорится в \textit{Предметной области и онтологии Базового языка программирования ostis-систем}.}
		\scnrelfrom{смотрите}{Предметная область и онтология Базового языка программирования ostis-систем}
	\end{scnindent}
	
	\scnheader{платформенная независимость ostis-систем}
	\scntext{примечание}{Важно подчеркнуть, что несмотря на преимущества платформенно-независимой реализации \textit{ostis-систем} иногда оказывается целесообразным реализовывать некоторые компоненты \textit{ostis-систем} (например, конкретные \textit{sc-агенты} или компоненты пользовательского интерфейса) на уровне \textit{ostis-платформы}. В случае подобной реализации программ \textit{sc-агентов} можно провести аналогию с реализацией каких-либо подпрограмм на уровне языков микропрограммирования для современных компьютеров. Чаще всего целесообразность такого решения обусловлена повышением производительности таких компонентов и системы в целом, поскольку реализация компонента с учетом особенностей платформы в общем случае является более производительной. В то же время заметим, что последнее утверждение не всегда верно, поскольку при реализации компонента на уровне логико-семантической модели могут быть реализованы, например, модели параллельной обработки информации, не всегда легко и понятно реализуемые на уровне платформы.}
	\begin{scnindent}
		\scntext{примечание}{Таким образом, при проектировании каждой конкретной \textit{ostis-системы} разработчику необходимо принимать решение о реализации тех или иных компонентов на платформенно-независимом уровне или уровне платформы. При этом очевидно, что с точки зрения развития технологии и накопления проектного опыта более приоритетной является реализация компонентов \textit{ostis-систем} на платформенно-независимом уровне.}
		\scntext{примечание}{Исходя из сказанного, можно предположить существование \textit{ostis-систем}, в которых все \textit{sc-агенты} реализованы на уровне платформы, которая в таком случае по сути \scnqqi{заточена} под конкретную \textit{ostis-систему} и может рассматриваться как аналог специализированного компьютера, ориентированного на решение задач только определенного ограниченного класса. Назовем такой вариант реализации \textit{ostis-систем} \textit{минимальной конфигурацией ostis-системы}.}
	\end{scnindent}
	
	\bigskip
\end{scnsubstruct}
\end{SCn}


\scsubsection[
    \protect\scneditor{Шункевич Д.В.}
    \protect\scnmonographychapter{Глава 6.1. Универсальная модель интерпретатора внутренних агентов решателя задач интеллектуальной компьютерной системы нового поколения}
    ]{Предметная область и онтология базовых интерпретаторов логико-семантических моделей ostis-систем}
\label{sd_interpreters}
\begin{SCn}
    \scnsectionheader{Предметная область и онтология базовых интерпретаторов логико-семантических моделей ostis-систем}
    \begin{scnsubstruct}
    	\scniselement{раздел базы знаний}
        \scnidtf{Предметная область и онтология платформ интерпретации sc-моделей компьютерных систем}
        \scnidtf{Предметная область и онтология аппаратных и программных платформ реализации интеллектуальных компьютерных систем нового поколения}
        \scnidtf{Предметная область и онтология платформ для массовой разработки семантически совместимых интероперабельных интеллектуальных компьютерных систем нового поколения}
        \scnidtf{Предметная область и онтология платформ для ostis-систем}
        \scntext{аннотация}{Уточнение понятия платформенной независимости интеллектуальных компьютерных систем нового поколения. Требования, предъявляемые к платформам интерпретации sc-моделей ostis-систем. Принципы и способы реализации аппаратной платформы для ostis-систем (ассоциативного семантического компьютера). Спецификация и существующие аналоги программного варианта реализации платформы для ostis-систем.}
        \scntext{аннотация}{В предметной области и онтологии рассматривается подход к решению проблемы платформенной независимости компьютерных систем, предполагающий унификацию принципов реализации таких систем и обеспечения их семантической совместимости на основе Технологии OSTIS. Приводится формализованная система понятий, определяющая принципы реализации данного подхода.}
        \begin{scnrelfromlist}{библиографическая ссылка}
        	\scnitem{\scncite{Komarcova2004}}
        	\scnitem{\scncite{USB_Accelerator}}
        	\scnitem{\scncite{Kolesnikov2001}}
        	\scnitem{\scncite{TuringMachine}}
        	\scnitem{\scncite{Neumann1993}}
        	\scnitem{\scncite{NeumanMachine}}
        \end{scnrelfromlist}
        \begin{scnrelfromlist}{дочерний раздел}
            \scnitem{Предметная область и онтология программных вариантов реализации базового интерпретатора логико-семантических моделей ostis-систем на современных компьютерах}
            \scnitem{Предметная область и онтология ассоциативных семантических компьютеров для ostis-систем}
        \end{scnrelfromlist}
        
        \begin{scnreltovector}{конкатенация сегментов}
        	\scnitem{Сегмент. Уточнение понятия платформенной независимости и анализ современных подходов к ее обеспечению}
        	\scnitem{Сегмент. Уточнение понятия ostis-платформы}
        \end{scnreltovector}
        
        \begin{SCn}
	\scnsectionheader{Сегмент. Уточнение понятия платформенной независимости и анализ современных подходов к ее обеспечению}
	
	\begin{scnsubstruct}
		\scnheader{производство интеллектуальных компьютерных систем}
		\scnrelfrom{этапы}{Этапы производства интеллектуальных компьютерных систем}
		\begin{scnindent}
			\begin{scneqtoset}
				\scnfileitem{Этап проектирования, то есть построения формальной модели системы, достаточной для понимания принципов ее устройства и выполнения последующего этапа ее реализации.}
				\scnfileitem{Этап реализации, то есть непосредственно воплощения разработанной модели с использованием конкретных средств (инструментов, материалов, комплектующих и так далее). В случае компьютерных систем выполнение данного этапа обычно предполагает выбор конкретных языков программирования, библиотек, сторонних средств, таких как с.у.б.д. и различные сервисы, а также собственно программирование и отладку системы с использованием выбранных средств.}
			\end{scneqtoset}
			\scntext{примечание}{Для каждого из указанных этапов могут существовать свои методики, а также средства автоматизации соответствующих процессов.}
			\scntext{отличие}{Если этап проектирования компьютерной системы как правило требует участия высококвалифицированных специалистов и экспертов в предметных областях, в которых осуществляется автоматизация, то этап реализации, с одной стороны, как правило является более простым (при условии качественного выполнения этапа проектирования), а с другой стороны требует значительных ресурсов.}
			\begin{scnindent}
				\scntext{причина}{Одной из причин этого является необходимость работы компьютерной системы на различных платформах (устройствах), каждое из которых в общем случае может иметь свои особенности и ограничения, которые необходимо учитывать на этапе реализации. Решением данной проблемы является обеспечение платформенной независимости (или кроссплатформенности) разрабатываемых компьютерных систем.}
			\end{scnindent}
		\end{scnindent}
		
		\scnheader{платформенная независимость компьютерных систем}
		\scntext{примечание}{Сама по себе идея обеспечения платформенной независимости давно и широко используется в современных компьютерных системах.}
		\scnsuperset{платформенная независимость на уровне операционных систем}
		\begin{scnindent}
			\scntext{пояснение}{Проблема обеспечения возможности работы программной системы в разных операционных системах.}
		\end{scnindent}
		\scnsuperset{платформенная независимость на уровне аппаратных архитектур}
		\begin{scnindent}
			\scntext{пояснение}{Проблема обеспечения совместимости операционной системы с различными аппаратными архитектурами. Для решения этой проблемы могут существовать разные сборки ядра операционной системы для разных аппаратных архитектур, как это делается для операционных систем семейства Linux. При этом следует отметить, что речь в подавляющем большинстве случаев идет не о принципиально разных архитектурах, а о вариантах реализации базовой архитектуры фон Неймана.}
		\end{scnindent}
	\scntext{примечание}{В случае, когда разрабатываемая компьютерная система проектируется на более низком уровне, чем операционная система как таковая (например, при программировании контроллеров управления различными устройствами), проблема обеспечения платформенной независимости значительно усугубляется и чаще всего может быть решена только для набора аппаратных средств определенного класса, для которого стандартизируется интерфейс доступа, то есть, по сути, набор низкоуровневых команд обработки информации.}
	
	\scnheader{платформенная независимость на уровне операционных систем}
	\scntext{примечание}{Большее внимание при проектировании современных компьютерных систем на данный момент уделяется платформенной независимости на уровне операционных систем.}
	\begin{scnrelfromlist}{достигается}
		\scnfileitem{Использование кроссплатформенных языков программирования, которые, в свою очередь, можно разделить на \scnqqi{полностью} интерпретируемые языки (Python, JavaScript и языки на его основе, PHP и другие) и языки, использующие компиляцию в сохраняющий независимость от платформы низкоуровневый байт-код, с его возможной последующей компиляцией в машинный код непосредственно в процессе исполнения (Just-in-time компиляция или JIT-компиляция). К языкам второго класса относятся, например, Java и C Sharp. Реализация такого подхода требует установки на целевой компьютер с операционной системой интерпретатора соответствующего языка программирования или байт-кода.}
		\begin{scnindent}
			\begin{scnrelfromlist}{ограничения}
				\scnfileitem{В среднем производительность интерпретируемых программ ниже, чем компилируемых. Одним из подходов к решению данной проблемы и является JIT-компиляция.}
				\scnfileitem{Строго говоря, кроссплатформенность при таком варианте обеспечивается не для всех операционных систем, а для класса операционных систем и соответствующего класса устройств, например, операционных систем, предназначенных для персональных компьютеров. Так, например, приложение, написанное на языке Java для персонального компьютера не может быть напрямую перенесено на мобильное устройство, поскольку при разработке мобильных приложений учитываются другие принципы работы пользователя с интерфейсом системы, отсутствие многооконности и многое другое.}
			\end{scnrelfromlist}
			\scntext{примечание}{Важно также отметить, что даже для интерпретируемых языков программирования существует проблема зависимости приложения от используемого набора библиотек и фреймворков. Так, при разработке интерфейса web-приложения могут использоваться популярные фреймворки AngularJS и ReactJS, при этом после выбора одного из них быстрый перевод приложения на другой фреймворк невозможен.}
		\end{scnindent}
		
		\scnfileitem{Реализация системы в виде web-приложения, работа с которым осуществляется через web-браузер и интерфейс которого, таким образом, реализуется на базе общепринятых стандартов Всемирной паутины (HTML, CSS, JavaScript и языки и библиотеки на его основе). Такой вариант обеспечивает возможность работы с приложением с любого устройства, имеющего web-браузер, в том числе, мобильного.}
		\begin{scnindent}
			\begin{scnrelfromlist}{недостатки}
				\scnfileitem{Как правило, высокая требовательность к производительности конечного устройства. Современный web-браузер является одним из самых ресурсоёмких приложений почти на любом устройстве.}
				\scnfileitem{Остается за кадром проблема обеспечения платформенной независимости серверной части web-приложения, которая должна решаться каким-то другим способом.}
				\scnfileitem{Несмотря на стандартизацию, разработчикам часто приходится учитывать особенности конкретных web-браузеров и тестировать работоспособность приложений для каждого из них.}
				\scnfileitem{Потенциально одним и тем же web-приложением можно пользоваться на любом устройстве, однако для обеспечения удобства и наглядности как правило приходится разрабатывать отдельные версии web-приложения, адаптированные под разные устройства, имеющие, например разные размеры экрана.}
			\end{scnrelfromlist}
		\end{scnindent}
		\scnfileitem{Виртуализация (контейнеризация, эмуляция). Перечисленные термины не являются полностью синонимичными, но в целом обозначают подход, при котором в рамках операционной системы создается некоторое изолированное локальное окружение (виртуальная машина, контейнер, среда эмуляции), содержащее все необходимые для работы приложения настройки и гарантирующее его работу на любых операционных системах и устройствах, где может интерпретироваться соответствующая виртуальная машина или контейнер. Соответственно, запуск таких окружений требует установки на конечное устройство соответствующего интерпретатора или эмулятора.}
		\begin{scnindent}
			\scntext{преимущества}{Данный подход бурно развивается и набирает популярность в настоящее время, поскольку позволяет решить не только проблему кроссплатформенности, но и избавить потребителя от установки большого числа зависимостей и выполнения настройки приложения на конечном устройстве.}
			\scntext{примеры}{Среди популярных средств, реализующих данный подход можно указать средства виртуализации (VirtualBox, DosBox, VMWare Workstation), контейнеризации (Docker), эмуляции приложений Android для настольных операционных систем (Genymotion, Bluestacks, Anbox) и многие другие.}
			\scntext{недостатки}{К недостаткам такого подхода можно отнести его ресурсоемкость и снижение производительности, а также ограниченность применения (как правило, соответствующие интерпретаторы разрабатываются только для наиболее популярных и востребованных операционных систем). Кроме того, возникает проблема следующего уровня, связанная уже с зависимостью от выбранного средства виртуализации (контейнеризации).}
		\end{scnindent}
	\end{scnrelfromlist}
	
	\scnheader{платформенная независимость компьютерных систем}
	\scntext{примечание}{Проблеме обеспечения платформенной независимости в современных компьютерных системах уделяется достаточно много внимания, однако в полной мере она не решена. В то же время, существует большое количество успешных частных решений, которые, однако, обладают серьезными ограничениями, связанными, в первую очередь, с отсутствием унификации современных подходов к разработке компьютерных систем.}
	\scntext{примечание}{Еще более актуальной проблема обеспечения платформенной независимости становится в контексте разработки \textit{интеллектуальных компьютерных систем}.}
	\begin{scnindent}
		\begin{scnrelfromlist}{обусловлено}
			\scnfileitem{Значительно более сложная по сравнению с традиционными компьютерными системами структура представляемой информации и, соответственно, многообразие форм ее представления, хранение и обработка которых на разных платформах могут быть организованы совершенно по-разному.}
			\scnfileitem{Высокие требования к производительности для некоторых классов систем, в частности, систем, использующих машинное обучение, что приводит к созданию специализированных аппаратных архитектур, таких как, например, нейрокомпьютеры.}
			\begin{scnindent}
				\begin{scnrelfromlist}{источник}
					\scnitem{\scncite{Komarcova2004}}
					\scnitem{\scncite{USB_Accelerator}}
				\end{scnrelfromlist}
			\end{scnindent}
			\scnfileitem{Многообразие моделей решения задач, которые в общем случае реализуются по-разному в разных системах.}
			\scnfileitem{Актуальность разработки гибридных интеллектуальных систем, в рамках которых интегрируются различные виды знаний и различные модели решения задач. В виду отсутствия на настоящий момент общепринятой унифицированной основы для их интеграции такие системы создаются в основном с ориентацией на какую-то определенную платформу и трудно переносимы на другие платформы.}
			\begin{scnindent}
				\begin{scnrelfromlist}{источник}
					\scnitem{\scncite{Kolesnikov2001}}
				\end{scnrelfromlist}
			\end{scnindent}
		\end{scnrelfromlist}
	\end{scnindent} 
	
	\scnheader{платформенная независимость интеллектуальных компьютерных систем}
	\scnsubset{платформенная независимость компьютерных систем}
	\scntext{примечание}{Проблема обеспечения платформенной независимости для интеллектуальных систем обусловлена во многом отсутствием семантической совместимости компонентов таких систем между собой, что, в свою очередь, создает препятствия даже для реализации подходов к обеспечению платформенной независимости, реализуемых в процессе разработки традиционных компьютерных систем. То есть, для решения проблемы обеспечения платформенной независимости интеллектуальных систем, требуется вначале обеспечить семантическую совместимость компонентов таких систем между собой.}
	\begin{scnindent}
		\scnrelfrom{предполагает}{Принципы семантической совместимости компонентов интеллектуальных компьютерных систем}
	\end{scnindent}
	
	\scnheader{Принципы семантической совместимости компонентов интеллектуальных компьютерных систем}
	\begin{scneqtoset}
		\scnfileitem{Унификация представления различного рода информации, хранимой в базах знаний таких систем.}
		\scnfileitem{Унификация базовых моделей обработки информации, хранимой в базах знаний таких систем, то есть выделение универсального низкоуровневого языка программирования, позволяющего осуществлять обработку информации, хранимой в унифицированном виде.}
		\scnfileitem{Унификация принципов реализации различных моделей решения задач и, как следствие, возможность их интеграции в рамках гибридных интеллектуальных систем.}
		\scnfileitem{Унификация принципов разработки интерфейсов компьютерных систем, которая бы позволила реализовать в рамках одной интеллектуальной системы возможность взаимодействия с другими системами и пользователями таких систем на разных внешних языках, включая естественные языки.}
	\end{scneqtoset}
	\scntext{примечание}{Указанные принципы реализуются в рамках \textit{Технологии OSTIS}, которая, таким образом, может стать основой для решения проблемы обеспечения семантической совместимости компонентов интеллектуальных компьютерных систем в целом и обеспечения платформенной независимости таких систем. С одной стороны, принципы, лежащие в основе \textit{Технологии OSTIS}, обеспечивают принципиальную возможность реализации платформенной независимости компьютерных систем, разрабатываемых на ее основе ostis-систем. С другой стороны, благодаря своей универсальности \textit{Технология OSTIS} позволяет преобразовать любую современную компьютерную систему в ostis-систему, которая будет функционально эквивалентна исходной компьютерной системе, но при этом будет обладать всеми перечисленными выше свойствами, создающими предпосылки для решения проблемы платформенной независимости.}
	\scntext{примечание}{Для реализации данного подхода в рамках Технологии OSTIS требуется разработать семейство онтологий, обеспечивающих уточнение таких понятий, как ostis-система, ostis-платформа, их структуры, типологии и предъявляемых к ним требований. Рассмотрению указанных понятий и посвящена данная онтология.}
	
	\scnheader{платформенная независимость на уровне операционных систем}
	\scntext{примечание}{Что касается обозначенной проблемы зависимости компьютерных систем от конкретных фрейморков, то аналогичная проблема может возникнуть и при дальнейшем развитии \textit{Технологии OSTIS}, в ситуации, когда соответствующие библиотеки будут содержать достаточно большое количество функционально эквивалентных компонентов. Однако, благодаря принципам, лежащим в основе \textit{Технологии OSTIS}, в частности, смысловому представлению информации и семантической совместимости компонентов, данная проблема будет значительно менее острой.}
	\begin{scnindent}
		\begin{scnrelfromset}{обоснование}
			\scnfileitem{Число функционально эквивалентных компонентов будет значительно ниже, чем в традиционных информационных технологиях, нет необходимости создавать синтаксически разные компоненты, отличия будут только на семантическом уровне.}
			\scnfileitem{Сами по себе компоненты будут являться более универсальными, то есть смогут быть использованы в значительно большем количестве систем.}
			\scnfileitem{Есть возможность автоматически выявить близкие компоненты, их сходства, различия, потенциальные конфликты и зависимости компонентов.}
			\scnfileitem{Есть возможность построения достаточно простых (по сравнению с традиционными технологиями) процедур перехода от одного фреймворка к другому, поскольку все компоненты и фреймворки имеют общую формальную смысловую основу, более высокоуровневую, чем в традиционных технологиях.}
		\end{scnrelfromset}
	\end{scnindent}
	
	\bigskip
	\end{scnsubstruct}
\scnsourcecomment{Завершили \scnqqi{Сегмент. Уточнение понятия платформенной независимости и анализ современных подходов к ее обеспечению}}
\end{SCn}
\begin{SCn}
	\scnsectionheader{Сегмент. Уточнение понятия ostis-платформы}
	\begin{scnsubstruct}
		
	\begin{scnrelfromlist}{ключевое понятие}
		\scnitem{ostis-платформа}
		\scnitem{базовая ostis-платформа}
		\scnitem{расширенная ostis-платформа}
		\scnitem{специализированная ostis-платформа}
		\scnitem{реализация sc-памяти}
		\scnitem{реализация файловой памяти sc-машины}
		\scnitem{scp-интерпретатор}
		\scnitem{базовая подсистема взаимодействия ostis-системы с внешней средой}
		\scnitem{подсистема обеспечения жизнедеятельности ostis-системы}
		\scnitem{специализированная платформенно-зависимая машина обработки знаний}
		\scnitem{минимальная конфигурация ostis-системы}
		\scnitem{однопользовательская ostis-платформа}
		\scnitem{многопользовательская ostis-платформа}
		\scnitem{программный вариант ostis-платформы}
		\scnitem{ассоциативный семантический компьютер}
	\end{scnrelfromlist}
	
	\scnheader{ostis-платформа}
	\scnidtf{платформа интерпретации sc-моделей компьютерных систем}
	\scnidtf{интерпретатор sc-моделей кибернетических систем}
	\scnidtf{интерпретатор унифицированных логико-семантических моделей компьютерных систем}
	\scnidtf{платформа реализации sc-моделей компьютерных систем}
	\scnidtf{\scnqq{пустая} ostis-система}
	\scnidtf{реализация sc-машины}
	\scnsubset{платформенно-зависимый многократно используемый компонент ostis-систем}
	\scnidtf{базовый интерпретатор логико-семантических моделей ostis-систем}
	\scnidtf{семейство платформ интерпретации sc-моделей компьютерных систем}
	\scnidtftext{часто используемый sc-идентификатор}{универсальный интерпретатор sc-моделей компьютерных систем}
	\scnidtf{универсальный интерпретатор унифицированных логико-семантических моделей компьютерных систем}
	\scnsubset{встроенная ostis-система}
	\scnidtf{встроенная \scnqq{пустая} ostis-система}
	\scnidtf{универсальный интерпретатор sc-моделей ostis-систем}
	\scnidtf{универсальная базовая ostis-система, обеспечивающая имитацию любой ostis-системы путем интерпретации sc-модели имитируемой ostis-системы}
	\scntext{примечание}{соотношение между имитируемой и универсальной ostis-системой в известной мере аналогично соотношению между машиной Тьюринга и универсальной машиной Тьюринга}
	\scntext{пояснение}{Под \textbf{\textit{ostis-платформой}} понимается реализация платформы интерпретации sc-моделей, которая в общем случае включает в себя: Реализация \textit{ostis-платформы} (\textit{универсального интерпретатора sc-моделей компьютерных систем}) может иметь большое число вариантов --- как программно, так и аппаратно реализованных. Логическая архитектура \textit{ostis-платформы} обеспечивает независимость проектируемых компьютерных систем от многообразия вариантов реализации интерпретатора их моделей и в общем случае включает в себя:
		\begin{scnitemize}
			\item хранилище \textit{sc-текстов} (\textit{sc-хранилище}, хранилище знаковых конструкций, представленных SC-коде);
			\item файловую память \textit{sc-машины};
			\item средства, обеспечивающие взаимодействие \textit{sc-агентов} над общей памятью (sc-памятью);
			\item базовые средства интерфейса для взаимодействия системы с внешним миром (пользователем или другими системами). Указанные средства включают в себя, как минимум, редактор, транслятор (в sc-память и из нее) и визуализатор для одного из базовых универсальных вариантов представления \textit{SC-кода} (\textit{SCg-код}, \textit{SCs-код}, \textit{SCn-код}), средства, позволяющие задавать системе вопросы из некоторого универсального класса (например, запрос семантической окрестности некоторого объекта);
			\item реализацию \textit{Абстрактной scp-машины}, то есть интерпретатор \textit{scp-программ} (программ Языка SCP).
		\end{scnitemize}
		При необходимости, в \textbf{\textit{ostis-платформу}} могут быть заранее на платформенно-зависимом уровне включены какие-либо компоненты машин обработки знаний или баз знаний, например, с целью упрощения создания первой версии \textit{прикладной ostis-системы}. Реализация платформы может осуществляться на основе произвольного набора существующих технологий, включая аппаратную реализацию каких-либо ее частей. С точки зрения компонентного подхода любая \textbf{\textit{ostis-платформа}} является \textbf{\textit{платформенно-зависимым многократно используемым компонентом}}.}
	\begin{scnrelfromset}{разбиение}
		\scnitem{базовая ostis-платформа}
		\scnitem{расширенная ostis-платформа}
		\scnitem{специализированная ostis-платформа}
	\end{scnrelfromset}
	
	\scnheader{базовая ostis-платформа}
	\scnidtf{базовый интерпретатор логико-семантических моделей ostis-систем}
	\scnidtf{минимальная универсальная ostis-платформа, обеспечивающая интерпретацию sc-модели любой \textit{ostis-системы} и включающая интерпретатор базового языка программирования \textit{ostis-систем} (Языка SCP)}
	\scnidtf{универсальный интерпретатор sc-моделей ostis-систем}
	\scnidtf{универсальная базовая ostis-система, обеспечивающая имитацию любой \textit{ostis-системы} путем интерпретации sc-модели имитируемой ostis-системы}
	\scntext{примечание}{Понятие \textit{базовой ostis-платформы} является ключевым с точки зрения обеспечения платформенной независимости \textit{ostis-систем}. Универсальность \textit{базовой ostis-платформы} подразумевает возможность интерпретации на ее основе любой \textit{sc-модели кибернетической системы}. Это достигается за счет наличия в рамках \textit{Технологии OSTIS} средств, позволяющих описывать на уровне sc-модели \textit{базу знаний}, \textit{решатель задач} и \textit{интерфейс кибернетической системы}, а также наличия Базового универсального языка программирования для \textit{ostis-систем} (\textit{Языка SCP}). \textit{Язык SCP} в таком случае выступает в роли базового низкоуровневого стандарта (ассемблера) обработки конструкций \textit{SC-кода}, гарантирующего полноту с точки зрения обработки, то есть, обеспечивающего возможность осуществить любое преобразование любого фрагмента \textit{SC-кода} при условии сохранения синтаксической корректности этого фрагмента. Следует отметить, что в общем случае таких функционально эквивалентных ассемблеров может быть несколько (и, как следствие, соответствующих им \textit{scp-машин}), но для обеспечения совместимости в рамках \textit{Технологии OSTIS} один из таких вариантов выбирается в качестве стандарта и описывается в \textit{Предметной области и онтологии решателей задач ostis-систем}.}
	\scntext{предъявляемое требование}{Основным и \uline{единственным требованием}, предъявляемым ко всем \textit{базовым ostis-платформам} для обеспечения их универсальности, является необходимость обеспечения интерпретации \textit{Языка SCP}, стандартизированного в рамках \textit{Технологии OSTIS}. При этом важно отметить, что все \textit{базовые ostis-платформы} обязаны быть \uline{функционально эквивалентными}, поскольку интерпретируют один и тот же стандарт \textit{Языка SCP}.}
	\begin{scnrelfromset}{обобщенная декомпозиция}
		\scnitem{реализация sc-памяти}
		\begin{scnindent}
			\scnrelfrom{обобщенная часть}{реализация файловой памяти sc-машины}
		\end{scnindent}
		\scnitem{scp-интерпретатор}
		\scnitem{базовая подсистема взаимодействия \textit{ostis-системы} с внешней средой}
		\begin{scnindent}
			\scntext{примечание}{Реализация базового набора \textit{рецепторных sc-агентов} и \textit{эффекторных sc-агентов}, обеспечивающих минимально необходимый обмен информацией между \textit{ostis-системой} и внешней средой. Конкретный перечень таких агентов требует уточнения, однако можно сказать, что в общем случае они могут быть реализованы как в составе \textit{scp-интерпретатора} (в этом случае им будут соответствовать определенные классы \textit{scp-операторов}), так и отдельно от него в составе платформы.}
		\end{scnindent}
		\scnitem{подсистема обеспечения жизнедеятельности ostis-системы}
		\begin{scnindent}
			\scntext{примечание}{Реализацию набора sc-агентов, обеспечивающих базовые функции \textit{ostis-системы}, связанные с обеспечением ее жизнедеятельности, которые принципиально не могут быть реализованы на платформенно-независимом уровне. К таким функциям относятся, например, запуск системы, загрузка базы знаний в память системы, запуск \textit{scp-интерпретатора} и так далее.}
		\end{scnindent}
	\end{scnrelfromset}
	
	\scnheader{расширенная ostis-платформа}
	\scnidtf{ostis-платформа, содержащая дополнительные компоненты, реализованные на уровне платформы}
	\scnidtf{базовая ostis-платформа и множество компонентов, реализованных на уровне платформы}
	\scntext{примечание}{\textit{расширенная ostis-платформа} представляет собой \textit{базовую ostis-платформу}, дополненную каким-либо множеством компонентов (хотя бы одним), реализованных на уровне платформы, при условии сохранения при этом всех возможностей \textit{базовой ostis-платформы}. Таким образом, \textit{расширенная ostis-платформа} по сути представляет собой \textit{базовую ostis-платформу}, адаптированную для более эффективного решения задач определенных классов в рамках конкретного класса \textit{ostis-систем}. Компонент, реализуемый на уровне платформы, становится частью этой платформы и, таким образом,  преобразует \textit{базовую ostis-платформу} в \textit{расширенную ostis-платформу}.}
	\scntext{примечание}{Введение понятия \textit{расширенной ostis-платформы} позволяет сформулировать ряд дополнительных принципов реализации \textit{ostis-систем}:
		\begin{scnitemize}
			\item{Может существовать произвольное количество ostis-систем, каждая из которых будет иметь свою уникальную \textit{расширенную ostis-платформу}, но при этом все они будут основаны на одном и том же варианте \textit{базовой ostis-платформы}.}
			\item{Для каждого варианта \textit{базовой ostis-платформы} может существовать своя \textit{библиотека многократно используемых компонентов ostis-платформ} (см. Предметная область и онтология комплексной библиотеки многократно используемых семантически совместимых компонентов ostis-систем)}.
	\end{scnitemize}}
	
	\scnheader{специализированная ostis-платформа}
	\scnidtf{ostis-платформа, не содержащая реализацию интерпретатора Языка SCP}
	\scnidtf{неуниверсальная ostis-платформа}
	\scntext{примечание}{\textbf{\textit{специализированная ostis-платформа}} представляет собой ограниченный вариант реализации \textit{ostis-платформы}, не содержащий \textit{scp-интерпретатора}. Таким образом, \uline{все} \textit{sc-агенты}, в рамках \textit{ostis-системы}, основанной на \textit{специализированной ostis-платформе} должны быть реализованы на платформенно-зависимом уровне. Такая \textit{специализированная ostis-платформа} является аналогом специализированного компьютера, реализованного для конкретной компьютерной системы. Таким образом, в общем случае каждая \textit{ostis-система}, реализуемая на \textit{специализированной ostis-платформе} будет иметь свою \uline{уникальную} \textit{специализированную ostis-платформу}.}
	\scntext{примечание}{\textbf{\textit{специализированная ostis-платформа}} может быть получена из \textit{базовой ostis-платформы} путем исключения из нее реализации  \textit{scp-интерпретатора} и реализации всех необходимых \textit{sc-агентов} на уровне платформы (или заимствования всех или части агентов из соответствующей данному варианту \textit{базовой ostis-платформы} \textit{библиотеки многократно используемых компонентов ostis-платформ}).}
	\begin{scnrelfromset}{обобщенная декомпозиция}
		\scnitem{реализация sc-памяти}
		\begin{scnindent}
			\scnrelfrom{обобщенная часть}{реализация файловой памяти sc-машины}
		\end{scnindent}
		\scnitem{базовая подсистема взаимодействия ostis-системы с внешней средой}
		\scnitem{подсистема обеспечения жизнедеятельности ostis-системы}
		\scnitem{специализированная платформенно-зависимая машина обработки знаний}
		\begin{scnindent}
			\scnidtf{sc-агент, как правило неатомарный, обеспечивающий выполнение всех функций некоторой специализированной ostis-платформы, связанных с обработкой знаний}
			\scnsubset{платформенно-зависимый sc-агент}
		\end{scnindent}
	\end{scnrelfromset}
	\scntext{примечание}{Применение \textit{специализированных ostis-платформ} может быть целесообразным на стартовом этапе развития \textit{Технологии OSTIS}, а также с целью повышения производительности конкретных наиболее высоконагруженных \textit{ostis-систем}, однако активное развитие таких \textit{специализированных ostis-платформ} и их компонентов с точки зрения \textit{Технологии OSTIS} является нецелесообразным, поскольку:
		\begin{scnitemize}
			\item если какой-либо компонент разработан с ориентацией на конкретную платформу, то нет гарантий возможности его повторного использования в других вариантах реализации \textit{ostis-платформы} (как минимум, компоненты, разработанные для \textit{программного варианта реализации ostis-платформы} не смогут быть использованы в рамках \textit{ассоциативного семантического компьютера});
			\item наличие большого числа платформенно-зависимых компонентов требует развития и сопровождения отдельной инфраструктуры библиотек для хранения и повторного использования таких компонентов. Чем больше будет вариантов \textit{ostis-платформ} и чем больше будет число платформенно-зависимых компонентов, тем более сложной и громоздкой будет такая инфраструктура. Как минимум, необходимо будет отслеживать совместимость компонентов с разными версиями разных вариантов реализации \textit{ostis-платформ};
			\item изменения в \textit{специализированной ostis-платформе}, например, связанные с переходом на более новую и эффективную версию \textit{базовой ostis-платформы}, на основе которой построена данная \textit{специализированная ostis-платформа} в общем случае могут привести к необходимости внесения изменений в компоненты, зависящие от данного варианта реализации \textit{ostis-платформы}. Чем больше таких платформенно-зависимых компонентов, тем больше потенциальных изменений может потребоваться и, соответственно, тем сложнее будет осуществляться эволюция платформы при условии сохранения работоспособности \textit{ostis-систем}, в которых она используется.
		\end{scnitemize} 
		
		Перечисленные тезисы справедливы и для \textit{расширенных ostis-платформ}, однако в случае \textit{расширенной ostis-платформы} проблемы, связанные с переходом на более новую версию платформы и изменениями в соответствующих компонентах всегда могут быть решены путем временной замены платформенно-зависимых компонентов на их платформенно-независимые версии с соответствующим снижением производительности, но зато с сохранением функциональной целостности системы.}
	
	\scnheader{минимальная конфигурация ostis-системы}
	\begin{scnrelfromset}{обобщенная декомпозиция}
		\scnitem{sc-модель базы знаний}
		\scnitem{специализированная ostis-платформа}
	\end{scnrelfromset}
	\begin{scnrelfromlist}{предъявляемые требования}
		\scnfileitem{Использование \textit{SC-кода} как базового языка кодирования информации в базе знаний, и, соответственно, наличие памяти, хранящей конструкции \textit{SC-кода}.}
		\scnfileitem{Наличие \textit{базы знаний}, определяющей денотационную семантику понятий, используемых системой.}
		\scnfileitem{Наличие хотя бы одного внутреннего sc-агента, осуществляющего обработку знаний в памяти ostis-системы. Этот sc-агент может быть реализован на уровне платформы, соответственно база знаний такой системы может не содержать процедурных знаний (методов).}
	\end{scnrelfromlist}
	\scntext{примечание}{Такой вариант \textit{минимальной конфигурации ostis-системы} обладает только \textit{внутренним sc-агентом} и, соответственно, не имеет возможности общаться с внешним миром (можно сказать, что такая \textit{ostis-система} не обладает \scnqq{органами чувств}). Для того, чтобы система имела возможность общаться с внешним миром, необходимо добавить к \textit{минимальной конфигурации ostis-системы} хотя бы один \textit{рецепторный sc-агент} и хотя бы один \textit{эффекторный sc-агент}.}
	\scntext{примечание}{Важно отметить, что, как видно из представленного описания \textit{минимальной конфигурации ostis-системы}, в общем случае \textit{ostis-система} не обязана по умолчанию быть \textit{интеллектуальной системой}. Применение \textit{Технологии OSTIS} для разработки компьютерных систем не делает их автоматически интеллектуальными, оно позволяет обеспечить возможность последующей \uline{неограниченной интеллектуализации} таких систем с минимальными накладными расходами при условии соблюдения при их разработке всех принципов \textit{Технологии OSTIS}.}
	
	\scnheader{ostis-платформа}
	\begin{scnsubdividing}
		\scnitem{однопользовательская ostis-платформа}
		\begin{scnindent}
			\scnidtf{вариант реализации ostis-платформы, рассчитанный на то, что с конкретной ostis-системой взаимодействует только один пользователь (владелец)}
			\scntext{примечание}{При таком варианте реализации платформы оказывается невозможным реализовать некоторые важные принципы \textit{Технологии OSTIS}, например, коллективную согласованную разработку базы знаний системы в процессе ее эксплуатации. При этом могут использоваться различные сторонние средства, например, для разработки базы знаний на уровне исходных текстов.}
		\end{scnindent}
		\scnitem{многопользовательская ostis-платформа}
		\begin{scnindent}
			\scnidtf{вариант реализации ostis-платформы, рассчитанный на то, что с конкретной ostis-системой одновременно или в разное время могут взаимодействовать разные пользователи, в общем случае обладающие разными правами, сферами ответственности, уровнем опыта, и имеющие свою конфиденциальную часть хранимой в базе знаний информации}
		\end{scnindent}
	\end{scnsubdividing}
	\scnheader{платформа интерпретации sc-моделей компьютерных систем}
	\begin{scnsubdividing}
		\scnitem{программный вариант реализации платформы интерпретации sc-моделей компьютерных систем}
		\begin{scnindent}
			\scnidtf{программная платформа интерпретации sc-моделей ostis-систем}
			\scnidtf{программный базовый интерпретатор sc-моделей ostis-систем}
		\end{scnindent}
		\scnitem{семантический ассоциативный компьютер}
		\begin{scnindent}
			\scnidtf{аппаратная платформа интерпретации sc-моделей ostis-систем}
			\scnidtf{аппаратно реализованный базовый интерпретатор sc-моделей ostis-систем}
		\end{scnindent}
	\end{scnsubdividing}
	
	\scnheader{ostis-платформа}
	\begin{scnrelfromset}{разбиение}
		\scnitem{программный вариант ostis-платформы}
		\begin{scnindent}
			\scnidtf{платформа интерпретации sc-моделей ostis-систем, реализованная в виде программной системы на базе традиционной компьютерной архитектуры}
			\scnidtf{программная платформа интерпретации sc-моделей ostis-систем}
			\scnidtf{программный интерпретатор sc-моделей ostis-систем}
			\scntext{примечание}{Целесообразность разработки \textit{программных вариантов ostis-платформы} на настоящий момент обусловлена очевидной распространенностью фон-неймановской архитектуры и, соответственно, необходимостью реализации \textit{ostis-систем} на современных компьютерах различного вида. В то же время очевидно, что разработка специализированных \textit{ассоциативных семантических компьютеров} позволит существенно повысить эффективность работы \textit{ostis-систем}, а четкое разделение \textit{sc-модели кибернетической системы} и платформы ее интерпретации позволит осуществить перевод уже работающих \textit{ostis-систем} с традиционных архитектур на \textit{ассоциативные семантические компьютеры} с минимальными накладными расходами.}
		\end{scnindent}
		\scnitem{ассоциативный семантический компьютер}
		\begin{scnindent}
			\scnidtf{аппаратная платформа интерпретации sc-моделей ostis-систем}
			\scnidtf{аппаратно реализованный базовый интерпретатор sc-моделей ostis-систем}
		\end{scnindent}
	\end{scnrelfromset}
	\scntext{примечание}{Важно отметить, что в любом варианте реализации \textit{ostis-платформы} всегда присутствует как программная, так и аппаратная часть. Так, любой \textit{программный вариант ostis-платформы} предполагает его последующую интерпретацию на какой-либо аппаратной основе, например, на персональном компьютере с традиционной архитектурой. В то же время, разработка \textit{ostis-платформы} в виде \textit{ассоциативного семантического компьютера} предполагает разработку набора микропрограмм, реализующих базовые операции поиска и преобразования sc-конструкций, хранящихся в \textit{sc-памяти}.}
	\begin{scnindent}
		\scntext{примечание}{Таким образом, разделение множества возможных реализаций \textit{ostis-платформы} на программный и аппаратный варианты скорее отражает вариант аппаратной архитектуры, на которую в конечном итоге ориентирован тот или иной вариант реализации платформы --- либо на традиционную фон-неймановскую архитектуру, либо на специализированную архитектуру \textit{ассоциативного семантического компьютера} со структурно-перестраиваемой (графодинамической) памятью. \textit{Программный вариант ostis-платформы} по сути является моделью (виртуальной машиной) \textit{ассоциативного семантического компьютера}, построенной на базе традиционной фон-неймановской архитектуры, а \textit{Язык SCP} выступает в роли ассемблера для \textit{ассоциативного семантического компьютера} и также может интерпретироваться как в рамках аппаратной реализации такого компьютера, так и в рамках его программной модели. }
	\end{scnindent}
	\scntext{примечание}{Каждой конкретной \textit{ostis-системе} однозначно соответствует конкретная \textit{ostis-платформа}, которая может относиться к разному набору классов \textit{ostis-платформ}. В то же время очевидно, что на этапе разработки платформы проектируется и реализуется некоторый вариант \textit{ostis-платформы}, который затем тиражируется в разные \textit{ostis-системы}. Впоследствии в каждой \textit{ostis-системе} в этот вариант \textit{ostis-платформы} могут быть внесены изменения, но в общем случае в большом количестве \textit{ostis-систем} могут использоваться полностью эквивалентные \textit{ostis-платформы}. Таким образом, целесообразно говорить о \textit{типовых ostis-платформах}, которые:
		\begin{scnitemize}
			\item{Являются объектом разработки для разработчиков \textit{ostis-платформ}.}
			\item{Являются \textit{многократно используемым компонентом ostis-систем} и специфицируются в рамках соответствующих библиотек.}
			\item{Являются образцом для тиражирования (копирования) при создании новых \textit{ostis-систем}.}
	\end{scnitemize}}
	
	\bigskip
	\end{scnsubstruct}
\scnsourcecomment{Завершили \scnqqi{Сегмент. Уточнение понятия ostis-платформы}}
\end{SCn}
        
        \bigskip
    \end{scnsubstruct}
\end{SCn}


\scsubsubsection[
    \protect\scnmonographychapter{Глава 6.1. Универсальная модель интерпретатора внутренних агентов решателя задач интеллектуальной компьютерной системы нового поколения}
    ]{Предметная область и онтология ассоциативных семантических компьютеров для ostis-систем}
\label{sd_sem_comp}
\begin{SCn}
    \scnsectionheader{Предметная область и онтология ассоциативных семантических компьютеров для ostis-систем}
    \begin{scnsubstruct}
    	\scntext{аннотация}{В главе рассмотрены принципы реализации аппаратной платформы для реализации систем, построенных на основе Технологии OSTIS, --- ассоциативного семантического компьютера.}
    	\begin{scnrelfromlist}{ключевое понятие}
    		\scnitem{машина фон-Неймана}	
    		\scnitem{архитектура вычислительной системы}
    		\scnitem{ассоциативный семантический компьютер}
    		\scnitem{scp-компьютер}
    		\scnitem{процессорный модуль}
    		\scnitem{накопительный модуль}
    		\scnitem{терминальный модуль}
    		\scnitem{процессорный элемент}
    		\scnitem{физический канал связи}
    		\scnitem{логический канал связи}
    		\scnitem{волновая микропрограмма}
    		\scnitem{волновой язык программирования}
    	\end{scnrelfromlist}
    	\begin{scnrelfromlist}{библиографическая ссылка}
    		\scnitem{\scncite{Neumann1993}}
    		\scnitem{\scncite{NeumanMachine}}
    		\scnitem{\scncite{Glushkov1974}}
    		\scnitem{\scncite{Ajlif1973}}
    		\scnitem{\scncite{Moldovan1992}}
    		\scnitem{\scncite{Chu1976}}
    		\scnitem{\scncite{Kalynychenko1990}}
    		\scnitem{\scncite{Martin1980}}
    		\scnitem{\scncite{Ozkarahan1989}}
    		\scnitem{\scncite{Kohonen1980}}
    		\scnitem{\scncite{Ignatushhenko1981}}
    		\scnitem{\scncite{Berkovich1975}}
    		\scnitem{\scncite{Ajzerman1977}}
    		\scnitem{\scncite{Marchuk1978}}
    		\scnitem{\scncite{Prangishvili1981}}
    		\scnitem{\scncite{Zatuliver1981}}
    		\scnitem{\scncite{Ackerman1979}}
    		\scnitem{\scncite{Majers1985}}
    		\scnitem{\scncite{Glushkov1980}}
    		\scnitem{\scncite{Glushkov1978}}
    		\scnitem{\scncite{Rabinovich1995}}
    		\scnitem{\scncite{Zadyhajlo1979}}
    		\scnitem{\scncite{Schuster1979}}
    		\scnitem{\scncite{Suvorov1985}}
    		\scnitem{\scncite{Brukle1978}}
    		\scnitem{\scncite{Chu1977}}
    		\scnitem{\scncite{Kohonen1982}}
    		\scnitem{\scncite{Foster1981}}
    		\scnitem{\scncite{Ershov1982}}
    		\scnitem{\scncite{Bershtejn1975}}
    		\scnitem{\scncite{Vasilev1987}}
    		\scnitem{\scncite{Sapatyj1984}}
    		\scnitem{\scncite{Popov2019}}
    		\scnitem{\scncite{Popov2020}}
    		\scnitem{\scncite{Zhang2017}}
    		\scnitem{\scncite{Hu2021}}
    		\scnitem{\scncite{Song2016}}
    		\scnitem{\scncite{Afanasyev2021}}
    		\scnitem{\scncite{Vajncvajg1980}}
    		\scnitem{\scncite{Vajncvajg1987}}
    		\scnitem{\scncite{Somsubhra2006}}
    		\scnitem{\scncite{Rabinovich1979a}}
    		\scnitem{\scncite{Rabinovich1979b}}
    		\scnitem{\scncite{Gladun1977}}
    		\scnitem{\scncite{Gladun1987}}
    		\scnitem{\scncite{Amosov1973}}
    		\scnitem{\scncite{Zolotov1982}}
    		\scnitem{\scncite{Galushkin1997}}
    		\scnitem{\scncite{Heht-Nilsen1998}}
    		\scnitem{\scncite{Komarcova2004}}
    		\scnitem{\scncite{USB_Accelerator}}
    		\scnitem{\scncite{Moussa2013}}
    		\scnitem{\scncite{Altay}}
    		\scnitem{\scncite{Allen1989}}
    		\scnitem{\scncite{CUDA}}
    		\scnitem{\scncite{OpenCL}}
    		\scnitem{\scncite{Tran2018}}
    		\scnitem{\scncite{Shi2018}}
    		\scnitem{\scncite{Lu2021}}
    		\scnitem{\scncite{Golenkov1984}}
    		\scnitem{\scncite{Golenkov1994f}}
    		\scnitem{\scncite{Golenkov1994g}}
    		\scnitem{\scncite{Golenkov1996}}
    		\scnitem{\scncite{Gaponov2000}}
    		\scnitem{\scncite{Kuzmickij2000}}
    		\scnitem{\scncite{Serdiukov2004}}
    		\scnitem{\scncite{Ivashenko2021OSTIS}}
    		\scnitem{\scncite{Ivashenko2016Tatur}}
    		\scnitem{\scncite{Ivashenko2015Tatur}}
    		\scnitem{\scncite{Rasheed2019}}
    		\scnitem{\scncite{Dubrovin2020}}
    		\scnitem{\scncite{Wolfram2002}}
    		\scnitem{\scncite{VonNeuman1971}}
    		\scnitem{\scncite{Moon1987}}
    		\scnitem{\scncite{Smith1984}}
    		\scnitem{\scncite{Steele2011}}
    		\scnitem{\scncite{McJones2015}}
    		\scnitem{\scncite{VanderLeun2017}}
    		\scnitem{\scncite{Ivashenko2020String}}
    		\scnitem{\scncite{Hewitt2009}}
    		\scnitem{\scncite{Ivashenko2020}}
    		\scnitem{\scncite{Ivashenko2016BSUIR}}
    		\scnitem{\scncite{Ivashenko2019InfiniteMemory}}
    		\scnitem{\scncite{Ivashenko2022}}
    		\scnitem{\scncite{Ivashenko2020ReductionScheme}}
    		\scnitem{\scncite{Ivashenko2021PRIP}}
    		\scnitem{\scncite{LegUp}}
    		\scnitem{\scncite{VHDPlus}}
    		\scnitem{\scncite{SystemC}}
    		\scnitem{\scncite{MyHDL}}
    		\scnitem{\scncite{Sapatyj1986}}
    		\scnitem{\scncite{Moldovan1985}}
    		\scnitem{\scncite{Letichevskij2003}}
    		\scnitem{\scncite{Letichevskij2012}}
    		\scnitem{\scncite{Backus1978}}
    		\scnitem{\scncite{Kotov1966}}
    	\end{scnrelfromlist}
    	\scntext{введение}{Применение для разработки \textit{ostis-систем} современных программно-аппаратных платформ, ориентированных на адресный доступ к хранящимся в памяти данным, не всегда оказывается эффективным, поскольку при разработке интеллектуальных систем фактически приходится моделировать нелинейную память на базе линейной. Повышение эффективности решения задач интеллектуальными системами требует разработки специализированных платформ, в том числе аппаратных, ориентированных на унифицированные семантические модели представления и обработки информации. Таким образом, основной целью создания \textit{ассоциативных семантических компьютеров} является повышение производительности ostis-систем.}
    	\begin{scnrelfromset}{заключение}
    		\scnfileitem{В предметной области и онтолгии рассмотрены недостатки доминирующей в настоящее время фон-Неймановской архитектуры компьютерных систем в качестве основы для построения интеллектуальных компьютерных систем нового поколения, проведен анализ современных подходов к разработке аппаратных архитектур, устраняющих некоторые из указанных недостатков, обоснована необходимость разработки принципиально новых аппаратных архитектур, представляющих собой аппаратный вариант реализации ostis-платформ --- \textit{ассоциативных семантических компьютеров}.}
    		\scnfileitem{Предложены общие принципы, лежащие в основе \textit{ассоциативных семантических компьютеров}, рассмотрены три возможных варианта архитектуры таких компьютеров, представлены их достоинства и недостатки.}
    		\scnfileitem{Дальнейшее развитие предложенных в главе подходов требует решения ряда задач, как технических, так и организационных:
    			\begin{scnitemize}
    				\item Разработка волнового языка для записи микропрограмм, которыми обмениваются процессорные элементы между собой, и которые исполняются этими процессорными элементами;
    				\item Разработка языка для записи программ управления процессами обмена микропрограммами и управления очередью микропрограмм;
    				\item Организация активного участия специалистов в области микроэлектроники в уточнении принципов реализации процессорных элементов и процессоро-памяти в целом, уточнение элементной базы и более низкоуровневых архитектурных особенностей \textit{ассоциативных семантических компьютеров};
    				\item Разработка алгоритмов оптимизации способов записи sc-конструкций в процессоро-память и переразмещения уже записанной sc-конструкции с целью обеспечения последующей эффективности передачи сообщений между процессорными элементами;
    				\item Уточнение типологии информационных процессов в процессоро-памяти, их свойств и соответствующей типологии меток;
    				\item Уточнение принципов реализации многоагентной обработки знаний в рамках процессоро-памяти, в частности, разработка принципов реализации событийной обработки информации в такой памяти.
    		\end{scnitemize}}
   		\end{scnrelfromset}
    	
        \scnheader{Предметная область семантических ассоциативных компьютеров для ostis-систем}
        \scniselement{предметная область}
        \begin{scnhaselementrolelist}{класс объектов исследования}
            \scnitem{семантический ассоциативный компьютер}
        \end{scnhaselementrolelist}
        \begin{scnhaselementrolelist}{класс объектов исследования}
            \scnitem{информационно-логическая задача}
            \scnitem{машина, ориентированная на решение информационно-логических задачи}
        \end{scnhaselementrolelist}
        
       	\begin{scnreltovector}{конкатенация сегментов}
        	\scnitem{Сегмент. Современное состояние работ в области разработки компьютеров для интеллектуальных систем}
        	\scnitem{Сегмент. Анализ существующих архитектур вычислительных систем}
        	\scnitem{Сегмент. Общие принципы, лежащие в основе ассоциативных семантических компьютеров для ostis-систем}
        	\scnitem{Сегмент. Архитектура ассоциативных семантических компьютеров для ostis-систем}
        \end{scnreltovector}
        
        \scnsegmentheader{Сегмент. Современное состояние работ в области разработки компьютеров для интеллектуальных систем}
\begin{scnsubstruct}
   	\scntext{введение}{Подавляющее большинство современных программно-аппаратных платформ, применяемых при разработке современных компьютерных систем, и, в частности, интеллектуальных компьютерных систем, основаны на принципах абстрактной \textit{машины фон-Неймана} или архитектуры фон-Неймана. Рассмотрим основные принципы, лежащие в основе \textit{машины фон-Неймана}.}
   	\begin{scnindent}
   		\begin{scnrelfromlist}{источник}
   			\scnitem{\scncite{Neumann1993}}
   			\scnitem{\scncite{NeumanMachine}}
   		\end{scnrelfromlist}
	\end{scnindent}
   	
	\scnheader{машина фон-Неймана}
	\scnidtf{абстрактная машина фон-Неймана}
	\scniselement{абстрактная машина обработки информации}
	\begin{scnrelfromvector}{принципы, лежащие в основе}
		\scnfileitem{Информация в памяти представляется в виде последовательности строк символов в бинарном алфавите (0 или 1).}
		\scnfileitem{Память машины представляет собой последовательность \uline{адресуемых} ячеек памяти.}
		\scnfileitem{В каждую ячейку может быть записана любая строка символов в бинарном алфавите. При этом длина строк для всех адресуемых ячеек одинакова (в текущем стандарте ячеек, называемых байтами, равна 8 бит).}
		\scnfileitem{Каждой ячейке памяти взаимно однозначно соответствует битовая строка, обозначающая эту ячейку и являющаяся ее адресом.}
		\scnfileitem{Каждому типу элементарных действий (операций), выполняемых в памяти машины фон-Неймана, взаимно однозначно ставится ее идентификатор, который в памяти представляется также в виде битовой строки.}
		\scnfileitem{Каждая конкретная операция (команда), выполняемая в памяти, представляется (специфицируется) в памяти в виде строки, состоящей
			\begin{scnitemize}
				\item{из кода соответствущего типа операции;}
				\item{из последовательности адресов фрагментов памяти, в которых находятся операнды, над которыми выполняются операции --- исходные аргументы и результаты. Любой такой фрагмент задается адресом первого байта и количеством байт. Количество операндов \uline{однозначно} задается кодом типа операции;}
			\end{scnitemize}}
		\scnfileitem{Программа, выполняемая в памяти, хранится в памяти в виде последовательности спецификаций конкретных операций (команд).}
		\scnfileitem{Таким образом, и обрабатываемые данные, и программы для обработки этих данных хранятся в одной и той же памяти (в отличие, например, от Гарвардской архитектуры) и кодируются одинаковым образом.}
	\end{scnrelfromvector}
	\scnrelfrom{особенности логической организации}{Особенности логической организации фон-Неймановской архитектуры}
	\begin{scnindent}
		\begin{scneqtoset}
			\scnfileitem{Последовательная обработка, ограничивающая эффективность компьютеров физическими возможностями элементной базы.}
			\scnfileitem{Низкий уровень доступа к памяти, то есть сложность и громоздкость выполнения процедуры ассоциативного поиска нужного фрагмента знаний.}
			\scnfileitem{Линейная организация памяти и чрезвычайно простой вид конструктивных объектов, непосредственно хранимых в памяти. Это приводит к тому, что в интеллектуальных системах, построенных на базе современных компьютеров, манипулирование знаниями осуществляется с большим трудом. Во-первых, приходится оперировать не самими структурами, а их громоздкими линейными представлениями (списками, матрицами смежности, матрицами инцидентности); во-вторых, линеаризация сложных структур разрушает локальность их преобразований.}
			\scnfileitem{Представление информации в памяти современных компьютеров имеет уровень весьма далекий от смыслового, что делает переработку знаний довольно громоздкой, требующей учета большого количества деталей, касающейся не смысла перерабатываемой информации, а способа ее представления в памяти.}
			\scnfileitem{В современных компьютерах имеет место весьма низкий уровень аппаратно реализуемых операций над нечисловыми данными и полностью отсутствует аппаратная поддержка логических операций над фрагментами знаний, имеющих сложную структуру, что делает манипулирование такими фрагментами весьма сложным.}
		\end{scneqtoset}
		\scntext{примечание}{Данные особенности архитектуры существенно затруднят эффективную реализацию \textit{ostis-систем} на ее основе.}
	\end{scnindent}
	\scnrelfrom{альтернативные подходы}{Альтернативные фон-Неймановским подходы организации ЭВМ}
	\begin{scnindent}
		\scntext{примечание}{Попытки преодоления ограничений традиционных фон-Неймановских ЭВМ привели к возникновению множества подходов, связанных с отдельными изменениями принципов логической организации ЭВМ, прежде всего в зависимости от классов задач и предметных областей, на которые ориентируется тот или иной класс ЭВМ. Все эти тенденции, рассмотренные в совокупности, позволяют очертить некоторые ключевые принципы логической организации ЭВМ, ориентированных на переработку знаний (машин переработки знаний --- МПЗ).}
		\begin{scneqtoset}
			\scnfileitem{Переход к нелинейной организации памяти и аппаратная интерпретация сложных структур данных.}
			\begin{scnindent}
				\begin{scnrelfromlist}{источник}
					\scnitem{\scncite{Glushkov1974}}
					\scnitem{\scncite{Ajlif1973}}
					\scnitem{\scncite{Moldovan1992}}
					\scnitem{\scncite{Chu1976}}
					\scnitem{\scncite{Kalynychenko1990}}
				\end{scnrelfromlist}
			\end{scnindent}
			\scnfileitem{Аппаратная реализация ассоциативного доступа к информации.}
			\begin{scnindent}
				\begin{scnrelfromlist}{источник}
					\scnitem{\scncite{Martin1980}}
					\scnitem{\scncite{Ozkarahan1989}}
					\scnitem{\scncite{Glushkov1974}}
					\scnitem{\scncite{Kohonen1980}}
					\scnitem{\scncite{Ignatushhenko1981}}
					\scnitem{\scncite{Berkovich1975}}
					\scnitem{\scncite{Ajzerman1977}}
				\end{scnrelfromlist}
			\end{scnindent}
			\scnfileitem{Реализация параллельных асинхронных процессов над памятью и, в частности, разработка вычислительных машин, управляемых потоком данных.}
			\begin{scnindent}
				\begin{scnrelfromlist}{источник}
					\scnitem{\scncite{Glushkov1974}}
					\scnitem{\scncite{Marchuk1978}}
					\scnitem{\scncite{Prangishvili1981}}
					\scnitem{\scncite{Zatuliver1981}}
					\scnitem{\scncite{Ackerman1979}}
					\scnitem{\scncite{Majers1985}}
				\end{scnrelfromlist}
			\end{scnindent}
			\scnfileitem{Аппаратная интерпретация языков высокого уровня.}
			\begin{scnindent}
				\begin{scnrelfromlist}{источник}
					\scnitem{\scncite{Glushkov1980}}
					\scnitem{\scncite{Glushkov1978}}
					\scnitem{\scncite{Rabinovich1995}}
				\end{scnrelfromlist}
			\end{scnindent}
			\scnfileitem{Разработка аппаратных средств ведения баз данных (процессоров баз данных).}
			\begin{scnindent}
				\begin{scnrelfromlist}{источник}
					\scnitem{\scncite{Zadyhajlo1979}}
					\scnitem{\scncite{Schuster1979}}
					\scnitem{\scncite{Suvorov1985}}
				\end{scnrelfromlist}
			\end{scnindent}
		\end{scneqtoset}
	\end{scnindent}
	
	\scnheader{классы вычислительных устройств}
	\scntext{примечание}{На пересечении \scnkeyword{Альтернативных фон-Неймановским подходов организации ЭВМ} в разное время возникали различные классы вычислительных устройств}
	\scnhaselement{машины с аппаратной интерпретацией сложных структур данных}
	\begin{scnindent}
		\begin{scnrelfromlist}{источник}
			\scnitem{\scncite{Brukle1978}}
			\scnitem{\scncite{Chu1976}}
			\scnitem{\scncite{Chu1977}}
		\end{scnrelfromlist}
	\end{scnindent}
	\scnhaselement{машины с развитой ассоциативной памятью}
	\begin{scnindent}
		\begin{scnrelfromlist}{источник}
			\scnitem{\scncite{Kohonen1980}}
			\scnitem{\scncite{Kohonen1982}}
			\scnitem{\scncite{Foster1981}}
		\end{scnrelfromlist}
	\end{scnindent}
	\scnhaselement{ассоциативные параллельные матричные процессоры}
	\begin{scnindent}
		\scnrelfrom{источник}{\scncite{Ershov1982}}
	\end{scnindent}
	\scnhaselement{однородные параллельные структуры для решения комбинаторно-логических задач на графах и гиперграфах}
	\begin{scnindent}
		\scnrelfrom{источник}{\scncite{Bershtejn1975}}
	\end{scnindent}
	\scnhaselement{устройства переработки графов}
	\begin{scnindent}
	\begin{scnrelfromlist}{источник}
		\scnitem{\scncite{Vasilev1987}}
		\scnitem{\scncite{Sapatyj1984}}
		\scnitem{\scncite{Popov2019}}
		\scnitem{\scncite{Popov2020}}
	\end{scnrelfromlist}
	\end{scnindent}
	\begin{scnindent}
	\scnsuperset{устройства переработки графов на основе FPGA}
	\begin{scnindent}
	\begin{scnrelfromlist}{источник}
		\scnitem{\scncite{Zhang2017}}
		\scnitem{\scncite{Hu2021}}
		\scnitem{\scncite{Song2016}}
	\end{scnrelfromlist}
	\end{scnindent}
	\scnsuperset{устройства переработки графов на основе векторных процессоров}
	\begin{scnindent}
		\scnrelfrom{источник}{\scncite{Afanasyev2021}}
	\end{scnindent}
	\end{scnindent}
	\scnhaselement{системы, осуществляющие переработку информации непосредственно в памяти путем равномерного распределения функциональных средств по памяти и, в частности, предложенная М. Н. Вайнцвайгом процессоро-память, ориентированная на решение задач Искусственного интеллекта}
	\begin{scnindent}
		\begin{scnrelfromlist}{источник}
			\scnitem{\scncite{Vajncvajg1980}}
			\scnitem{\scncite{Vajncvajg1987}}
		\end{scnrelfromlist}
	\end{scnindent}
	\scnhaselement{машины, управляемые потоком данных}
	\begin{scnindent}
		\begin{scnrelfromlist}{источник}
			\scnitem{\scncite{Ershov1982}}
			\scnitem{\scncite{Prangishvili1981}}
			\scnitem{\scncite{Majers1985}}
		\end{scnrelfromlist}
		\scnsuperset{процессоры, реконфигурируемые с учетом семантики входного потока данных}
		\begin{scnindent}
			\scnrelfrom{источник}{\scncite{Somsubhra2006}}
		\end{scnindent}
	\end{scnindent}
	\scnhaselement{рекурсивные вычислительные машины}
	\begin{scnindent}
		\scnrelfrom{источник}{\scncite{Glushkov1974}}
	\end{scnindent}
	\scnhaselement{процессоры реляционных баз данных}
	\begin{scnindent}
		\begin{scnrelfromlist}{источник}
			\scnitem{\scncite{Zadyhajlo1979}}
			\scnitem{\scncite{Majers1985}}
			\scnitem{\scncite{Ozkarahan1989}}
		\end{scnrelfromlist}
	\end{scnindent}
	\scnhaselement{вычислительные машины со структурно-перестраиваемой памятью}
	\begin{scnindent}
		\begin{scnrelfromlist}{источник}
			\scnitem{\scncite{Rabinovich1979a}}
			\scnitem{\scncite{Rabinovich1979b}}
			\scnitem{\scncite{Gladun1977}}
			\scnitem{\scncite{Gladun1987}}
		\end{scnrelfromlist}
	\end{scnindent}
	\scnhaselement{активные семантические сети}
	\begin{scnindent}
		\scnidtf{М-сети}
		\scnrelfrom{источник}{\scncite{Amosov1973}}
	\end{scnindent}
	\scnhaselement{ассоциативные однородные среды}
	\begin{scnindent}
		\scnrelfrom{источник}{\scncite{Zolotov1982}}
	\end{scnindent}
	\scnhaselement{нейроподобные структуры}
	\begin{scnindent}
		\begin{scnrelfromlist}{источник}
			\scnitem{\scncite{Galushkin1997}}
			\scnitem{\scncite{Heht-Nilsen1998}}
		\end{scnrelfromlist}
		\scntext{примечание}{В последние годы активное развитие теории искусственных нейронных сетей привело к развитию различных подходов к построению высокопроизводительных компьютеров, предназначенных для обучения и интерпретации искусственных нейронных сетей и их внедрению в различные программно-аппаратные комплексы.}
		\begin{scnindent}
			\begin{scnrelfromlist}{источник}
				\scnitem{\scncite{Komarcova2004}}
				\scnitem{\scncite{USB_Accelerator}}
				\scnitem{\scncite{Moussa2013}}
			\end{scnrelfromlist}
		\end{scnindent}
		\scntext{примечание}{В отдельное направление выделены так называемые нейроморфные процессоры, отличающиеся высокой производительностью и низким уровнем энергопотребления.}
		\begin{scnindent}
			\scnrelfrom{источник}{\scncite{Altay}}
		\end{scnindent}
	\end{scnindent}
	\scnhaselement{машины для интерпретации логических правил}
	\begin{scnindent}
		\scnrelfrom{источник}{\scncite{Allen1989}}
	\end{scnindent}
	\scntext{примечание}{Развитие графических процессоров (graphics processing unit, GPU) привело к возможности организации параллельных вычислений непосредственно на GPU, для чего разрабатываются специализированные программно-аппаратные архитектуры, например CUDA. Преимуществом GPU в данном случае выступает наличие в рамках одного GPU большого (по сравнению с центральным процессором) числа ядер, что позволяет эффективно решать на такой архитектуре задачи, обладающие естественным параллелизмом (например, операции с матрицами). Развиваются также работы, посвященные принципам обработки графовых структур на GPU.}
	\begin{scnindent}
		\begin{scnrelfromlist}{источник}
			\scnitem{\scncite{CUDA}}
			\scnitem{\scncite{OpenCL}}
			\scnitem{\scncite{Tran2018}}
			\scnitem{\scncite{Shi2018}}
			\scnitem{\scncite{Lu2021}}
		\end{scnrelfromlist}
	\end{scnindent}
	\scntext{примечание}{Благодаря повышению производительности современных ЭВМ число разработок специализированных аппаратных решений в последние десятилетия снизилось, поскольку многие сложные вычислительные задачи в настоящее время за приемлемое время могут быть решены и на традиционных универсальных архитектурах. Как показано выше, исключение составляют в основном специализированные компьютеры для обработки искусственных нейронных сетей и других графовых моделей, что обусловлено большой востребованностью таких моделей и их сложностью.}
	\scntext{примечание}{Большинство перечисленных подходов (даже если они достаточно далеко отходят от предложенных фон-Нейманом базовых принципов организации вычислительных машин) неявно сохраняют точку зрения на компьютер как на большой арифмометр и тем самым сохраняют ее ориентацию на задачи числового характера. Работы же направленные на разработку аппаратных архитектур, предназначенных для обработки информации, представленной в более сложных формах, чем в традиционных архитектурах не получили широкого распространения и применения, по причине, во-первых, частности предлагаемых решений, и во-вторых из-за отсутствия общего универсального и унифицированного языка кодирования любой информации, в роли которого в рамках \textit{Технологии OSTIS} выступает \textit{SC-код}, а также соответствующей апробированной технологии разработки программных систем для таких аппаратных архитектур. Таким образом, зачастую разработчики подобных архитектур сталкиваются необходимостью разработки специализированного программного обеспечения для этих архитектур, что в конечном итоге приводит к сильному ограничению сфер применения таких архитектур, поскольку их применение оказывается целесообразным только в случае, если трудоемкость разработки специализированного программного обеспечения оправдывает себя с учетом низкой эффективности решения соответствующих задач на более традиционных архитектурах.}
	
	\scnheader{ассоциативный семантический компьютер}
	\scntext{примечание}{\textit{SC-код}, являющийся формальной основой \textit{Технологии OSTIS} изначально разрабатывался как язык кодирования информации в памяти \textit{ассоциативных семантических компьютеров}, таким образом в нем изначально заложены такие принципы, как универсальность (возможность представить знания любого рода) и унификация (единообразие) представления, а также минимизация \textit{Алфавита SC-кода}, которая, в свою очередь, позволяет облегчить создание аппаратной платформы, позволяющей хранить и обрабатывать тексты \textit{SC-кода}.}
	\scntext{примечание}{Основная методологическая особенность предлагаемого подхода к разработке средств аппаратной реализации поддержки интеллектуальных систем заключается в том, что такие средства должны разрабатываться не до, а \uline{после того, как} основные положения соответствующей \uline{технологии} проектирования и эксплуатации интеллектуальных систем будут апробированы на современных технических средствах. Более того, в рамках \textit{\textit{Технологии OSTIS}} четко продумана методика перехода на новые аппаратные средства, которая затрагивает только самый нижний уровень технологии --- уровень реализации базовой машины обработки семантических сетей (интерпретатора \textit{Языка SCP}).}
	\begin{scnrelfromvector}{основные этапы истории}
		\scnfileitem{1984 год --- в Московском институте электронной техники В.В. Голенковым защищена кандидатского диссертация на тему \scnqq{Структурная организация и переработка информации в электронных математических машинах, управляемых потоком сложноструктурированных данных}, в которой были сформулированы и рассмотрены основные принципы семантических ассоциативных компьютеров.}
		\begin{scnindent}
			\scnrelfrom{источник}{\scncite{Golenkov1984}}
		\end{scnindent}
		\scnfileitem{1993 год --- комиссия Госкомпрома провела успешные испытания прототипа \textit{ассоциативного семантического компьютера}, разработанного на базе транспьютеров в рамках научно-исследовательского проекта \scnqq{Параллельная графовая вычислительная система, ориентированная на решение задач искусственного интеллекта}.}
			\begin{scnindent}
			\begin{scnrelfromlist}{источник}
				\scnitem{\scncite{Golenkov1994f}}
				\scnitem{\scncite{Golenkov1994g}}
			\end{scnrelfromlist}
		\end{scnindent}
		\scnfileitem{1996 год --- В.В. Голенковым защищена докторская диссертация на тему \scnqq{Графодинамические модели и методы параллельной асинхронной переработки информации в интеллектуальных системах}.}
		\begin{scnindent}
			\scnrelfrom{источник}{\scncite{Golenkov1996}}
		\end{scnindent}
		\scnfileitem{2000 год --- в Институте проблем управления РАН П.А. Гапоновым защищена кандидатская диссертация на тему \scnqq{Модели и методы параллельной асинхронной переработки информации в графодинамической ассоциативной памяти}.}
		\begin{scnindent}
			\scnrelfrom{источник}{\scncite{Gaponov2000}}
		\end{scnindent}
		\scnfileitem{2000 год --- в Институте программных систем РАН В.М. Кузьмицким защищена кандидатская диссертация на тему \scnqq{Принципы построения графодинамического параллельного компьютера, ориентированного на решение задач искусственного интеллекта}.}
		\begin{scnindent}
			\scnrelfrom{источник}{\scncite{Kuzmickij2000}}
		\end{scnindent}
		\scnfileitem{2004 год --- в Белорусском государственном университете информатики и радиоэлектроники Р.Е. Сердюковым защищена кандидатская диссертация на тему \scnqq{Базовые алгоритмы и инструментальные средства обработки информации в графодинамических ассоциативных машинах}, в которой было рассмотрено базовое программное обеспечение семантических ассоциативных компьютеров.}
		\begin{scnindent}
			\scnrelfrom{источник}{\scncite{Serdiukov2004}}
		\end{scnindent}
	\end{scnrelfromvector}
	\scntext{примечание}{В тоже время, несмотря на наличие действующего прототипа \textit{ассоциативного семантического компьютера} на базе транспьютеров, основное внимание в рамках соответствующего проекта и других перечисленных работ уделялось принципам организации распределенной параллельной обработки конструкций SC-кода, в частности, был разработан \textit{SCD-код} (Semantic Code Distributed) для распределенного хранения конструкций SC-кода и \textit{Язык SCPD} для распределенной параллельной их обработки. Однако, общие принципы хранения информации и общая архитектура каждого из процессорных элементов (транспьютеров) оставались фон-Неймановскими. В частности, для кодирования конструкций SC-кода в традиционной адресной памяти были разработаны соответствующие структуры данных.}
	\begin{scnindent}
		\scnrelfrom{смотрите}{Предметная область и онтология программных вариантов реализации базового интерпретатора логико-семантических моделей ostis-систем на современных компьютерах}
	\end{scnindent}
	\scntext{примечание}{Обоснованность и необходимость разработки \textit{ассоциативного семантического компьютера}, а также компетентность авторов в данной области подтверждается более чем 30-летним опытом работы и рядом успешных проектов в данном направлении, однако, в тоже время, в предшествующих работах в полной мере не устранены все недостатки фон-Неймановской архитектуры, рассмотренные выше, и разработка и реализация проекта \textit{ассоциативного семантического компьютера}, устраняющего перечисленные недостатки, остается актуальной.}
\bigskip
\end{scnsubstruct}
\scnsourcecomment{Завершили \scnqqi{Сегмент. Современное состояние работ в области разработки компьютеров для интеллектуальных систем}}

        \scnsegmentheader{Сегмент. Анализ существующих архитектур вычислительных систем}
\begin{scnsubstruct}
	
	\scnheader{Альтернативные фон-Неймановским подходы организации ЭВМ}
	\scntext{примечание}{Для того, чтобы преодолеть недостатки существующих архитектур вычислительных систем, включая фон-Неймановскую, было предложено множество различных подходов. При разработке новых архитектур и, в частности, архитектуры \textit{ассоциативного семантического компьютера}, целесообразно в виде соответствующей онтологии выделить основные признаки классификации и соответствующие им классы (виды) архитектур вычислительных систем.}
	\begin{scnindent}
		\begin{scnrelfromlist}{источник}
			\scnitem{\scncite{Ivashenko2021OSTIS}}
			\scnitem{\scncite{Ivashenko2016Tatur}}
			\scnitem{\scncite{Ivashenko2015Tatur}}
			\scnitem{\scncite{Rasheed2019}}
			\scnitem{\scncite{Dubrovin2020}}
			\scnitem{\scncite{Wolfram2002}}
		\end{scnrelfromlist}
	\end{scnindent}
	
	\scnheader{архитектура вычислительной системы}
	\begin{scnsubdividing}
		\scnitem{архитектура вычислительной системы с глобальной оперативной памятью}
		\begin{scnindent}
			\scnnote{Архитектура, в которой все узлы (процессоры или машины) имеют доступ к глобальной оперативной памяти.}
			\begin{scnsubdividing}
				\scnitem{архитектура вычислительной системы с глобальной оперативной памятью данных}
				\scnitem{архитектура вычислительной системы с глобальной оперативной памятью программ}
				\scnitem{архитектура вычислительной системы с глобальной оперативной памятью программ и данных}
				\begin{scnindent}
					\scnnote{Примером такой архитектуры вычислительной системы является архитектура фон-Неймана.}
				\end{scnindent}
			\end{scnsubdividing}
		\end{scnindent}
		\scnitem{архитектура вычислительной системы без глобальной оперативной памяти}
	\end{scnsubdividing}
	\begin{scnsubdividing}
		\scnitem{архитектура вычислительной системы с единственной глобальной внутренней памятью}
		\scnitem{архитектура вычислительной системы со множественной глобальной внутренней памятью}
	\end{scnsubdividing}
	\begin{scnsubdividing}
		\scnitem{архитектура вычислительной системы со структурно перестраиваемыми межпроцессорными связями}
		\scnitem{архитектура вычислительной системы без структурно перестраиваемых междпроцессорных связей}
	\end{scnsubdividing}
	\begin{scnsubdividing}
		\scnitem{архитектура вычислительной системы со структурно становящейся памятью}
		\scnitem{архитектура вычислительной системы без структурно становящейся памяти}
	\end{scnsubdividing}
	\begin{scnsubdividing}
		\scnitem{архитектура вычислительной системы с ассоциативным доступом к глобальной (внутренней) памяти}
		\begin{scnindent}
			\scnnote{Ассоциативный характер доступа важен в системах, ориентированных на хранение данных со сложной структурой и ориентированных на масштабируемые (в том числе локальные) механизмы обработки информации.}
		\end{scnindent}
		\scnitem{архитектура вычислительной системы без ассоциативного доступа к глобальной (внутренней) памяти}
	\end{scnsubdividing}
	\begin{scnsubdividing}
		\scnitem{архитектура вычислительной системы с адресным доступом к глобальной памяти с линейным адресным пространством}
		\begin{scnindent}
			\scnnote{Примерами таких архитектур является большинство используемых на настоящий момент, включая архитектуру фон-Неймана.}
			\begin{scnindent}
				\scnrelfrom{источник}{\scncite{VonNeuman1971}}
			\end{scnindent}
		\end{scnindent}
		\scnitem{архитектура вычислительной системы без адресного доступа к глобальной памяти с линейным адресным пространством}
	\end{scnsubdividing}
	\begin{scnsubdividing}
		\scnitem{архитектура вычислительной системы с системой команд регистровой обработки данных}
		\begin{scnindent}
			\scnnote{Большинство используемых на настоящий момент архитектур являются примерами архитектур данного класса, включая архитектуру фон-Неймана. Архитектуры с системой команд регистровой обработки данных удобны для задач управления данными как для систем обработки образов в задачах пользовательского интерфейса, так и для задач машинного обучения на основе аппарата линейной алгебры.}
		\end{scnindent}
		\scnitem{архитектура вычислительной системы без системы команд с регистровой обработкой данных}
	\end{scnsubdividing}
	\begin{scnsubdividing}
		\scnitem{архитектура вычислительной системы с системой команд стековой обработки данных}
		\begin{scnindent}
			\scnnote{Примерами применения такой архитектуры являются LISP-машины.}
			\begin{scnindent}
				\begin{scnrelfromlist}{источник}
					\scnitem{\scncite{Moon1987}}
					\scnitem{\scncite{Smith1984}}
					\scnitem{\scncite{Steele2011}}
					\scnitem{\scncite{McJones2015}}
					\scnitem{\scncite{VanderLeun2017}}
				\end{scnrelfromlist}
			\end{scnindent}
		\end{scnindent}
		\scnitem{архитектура вычислительной системы без системы команд стековой обработки данных}
	\end{scnsubdividing}
	\begin{scnsubdividing}
		\scnitem{архитектура вычислительной системы с поддержкой системы команд обработки обобщенных строк}
		\begin{scnindent}
			\scnnote{Примером такой архитектуры является архитектура вычислительной системы с поддержкой системы команд обработки списков и обобщенных строк. Эта модель позволяет эффективно производить операции не только над строками и списками, но и работать с отношениями вида \scnqq{ключ-значение} с целью их интеграции в системы, управляемые знаниями. Программная реализация этой модели использует B-деревья.}
			\begin{scnindent}
				\scnrelfrom{источник}{\scncite{Ivashenko2020String}}
			\end{scnindent}
		\end{scnindent}
		\scnitem{архитектура вычислительной системы без системы поддержки команд обработки обобщенных строк}
	\end{scnsubdividing}
	\begin{scnsubdividing}
		\scnitem{архитектура вычислительной системы с поддержкой системы команд обработки графовых структур}
		\begin{scnindent}
			\scnnote{Примером такой архитектуры является архитектура компьютера Leonhard. Этот компьютер ориентирован на обработку графовых и гиперграфовых структур различных видов, включая иерархические графы. Поддерживается представление в виде строк и списка смежных вершин, упорядоченных локальных списков инцидентных ребер, глобального упорядоченного списка инцидентных ребер.}
			\begin{scnindent}
				\begin{scnrelfromlist}{источник}
					\scnitem{\scncite{Rasheed2019}}
					\scnitem{\scncite{Dubrovin2020}}
				\end{scnrelfromlist}
			\end{scnindent}
		\end{scnindent}
		\scnitem{архитектура вычислительной системы без системы поддержки команд обработки графовых структур}
	\end{scnsubdividing}
	\begin{scnsubdividing}
		\scnitem{архитектура вычислительной системы с системой команд (аппаратной) обработки знаний}
		\begin{scnindent}
			\scnnote{Примером такой архитектуры является архитектура компьютера Leonhard. Компьютер Leonhard поддерживает системы команд DISC (Discrete Instruction Set Computing). Система команд DISC включает следующие команды: создание целочисленного отношения со схемой, являющегося множеством объектов формального контекста (первым доменом бинарного отношения), тогда как соответствующим множеством образов является множество неотрицательных целых чисел (второй домен бинарного отношения); добавление пары в формальный контекст, содержащей добавляемый объект (ключ), который добавляется как кортеж через добавление элементов этого кортежа, вместе с целочисленным образом (значением) для этого объекта; получение следующего или предыдущего объекта в линейно (лексикографически) упорядоченным списке объектов; получение следующего большего или предыдущего меньшего объекта в линейно (лексикографически) упорядоченном списке объектов; получение минимального или максимального объекта в линейно (лексикографически) упорядоченном списке объектов; получение количества (мощности множества) образов для заданного объекта (кортежа-ключа); поиск пар по ключу; удаление пар; удаление всех пар формального контекста, включая объекты (ключи) и образы (значения); срез (подмножество) формальных контекстов контекста; объединение,пересечение и дополнение формальных контекстов. Для представления обрабатываемых данных используются B+-деревья. Другие архитектуры рассматривают реализацию операций обработки знаний, используя логическую модель представления знаний, LISP-структуры, обобщенные формальные языки. В последнем случае для развития системы команд обработки знаний рассматривается переход от обработки знаний к обработки метазнаний (на базе семантики становления актуального и неактуального), результатом которого является система метаопераций. Рассмотрение подобных архитектур важно для создания систем, управляемых знаниями.}
			\begin{scnindent}
				\begin{scnrelfromlist}{источник}
					\scnitem{\scncite{Rasheed2019}}
					\scnitem{\scncite{Dubrovin2020}}
					\scnitem{\scncite{Hewitt2009}}
					\scnitem{\scncite{Moon1987}}
					\scnitem{\scncite{Smith1984}}
					\scnitem{\scncite{Ivashenko2020String}}
					\scnitem{\scncite{Ivashenko2020}}
					\scnitem{\scncite{Ivashenko2020}}
					\scnitem{\scncite{Ivashenko2016BSUIR}}
				\end{scnrelfromlist}
			\end{scnindent}
		\end{scnindent}
		\scnitem{архитектура вычислительной системы без системы команд (аппаратной) обработки знаний}
	\end{scnsubdividing}
	\begin{scnsubdividing}
		\scnitem{архитектура вычислительной системы с адаптивным распределением данных}
		\begin{scnindent}
			\scnnote{Адаптивное распределение данных (включая как частный случай виртуальное адресное пространство) важно для целей управления данными (и знаниями) и задач виртуализации для многозадачных и многопользовательских систем, а также тесно связано с возможностями масштабируемости системы.}	
		\end{scnindent}
		\scnitem{архитектура вычислительной системы без адаптивного распределения данных}
	\end{scnsubdividing}
	\begin{scnsubdividing}
		\scnitem{архитектура вычислительной системы с системой команд локальной обработки информации}
		\begin{scnindent}
			\scnnote{Примером такой архитектуры является клеточный автомат. Элементарные клеточные (двоичные) автоматы разделяются на: быстро переходящие в однородное состояние (состояние только из нулей или единиц); быстро переходящие в устойчивое или циклическое состояние; остающиеся в хаотическом (случайном) состоянии; образующие как области с устойчивым или циклическим состоянием, так и области, в которых проявляются сложные взаимодействия элементов состояний, вплоть до Тьюринг-полных.\\
			Обработка информации с помощью клеточных автоматов позволяет строить вычислительные системы в том числе с перестраиваемой (в том числе фрактало-подобной) структурой на основе локальных параллельно (конкурентно) выполняемых несложных правил. Существуют разновидности клеточных автоматов, поддерживающих необратимые, обратимые, детерминированные, недетерминированные, специализированные, универсальные (в том числе Тьюринг-полные) вычисления. Работа клеточных автоматов напоминает волновые процессы распространяющиеся в среде элементов состояния клеточного автомата.}
			\begin{scnindent}
				\scnrelfrom{источник}{\scncite{Wolfram2002}}
			\end{scnindent}
		\end{scnindent}
		\scnitem{архитектура вычислительной системы без системы команд локальной обработки информации}
	\end{scnsubdividing}
	\begin{scnsubdividing}
		\scnitem{архитектура вычислительной системы исключительно с двоичным представлением данных в оперативной памяти}
		\begin{scnindent}
			\scnnote{Большинство современных архитектур цифровых вычислительных систем, включая реализации архитектуры фон-Неймана, использует именно двоичное представление.}
		\end{scnindent}
		\scnitem{архитектура вычислительной системы не исключительно с двоичным представлением данных в оперативной памяти}
	\end{scnsubdividing}
	\begin{scnsubdividing}
		\scnitem{архитектура вычислительной системы с исключительно дискретным представлением данных}
		\begin{scnindent}
			\scnnote{Примером такой архитектуры является архитектура фон-Неймана.}
		\end{scnindent}
		\scnitem{архитектура вычислительной системы без исключительно дискретного представления данных}
	\end{scnsubdividing}
	\begin{scnsubdividing}
		\scnitem{архитектура вычислительной системы с дискретным представлением данных}
		\begin{scnindent}
			\scnsuperset{архитектура вычислительной системы с исключительно дискретным представлением данных}
		\end{scnindent}
		\scnitem{архитектура вычислительной системы без дискретного представления данных}
	\end{scnsubdividing}
	\begin{scnsubdividing}
		\scnitem{архитектура вычислительной системы с управлением от потока данных}
		\begin{scnindent}
			\scnnote{Архитектуры вычислительных систем с управлением от потока данных видятся более естественными при решении многих задач Искусственного интеллекта. Варианты таких архитектур рассмотрены в работах. Архитектуру клеточных автоматов можно рассматривать как архитектуру вычислительной системы с управлением от потока данных.}
		\end{scnindent}
		\scnitem{архитектура вычислительной системы без управления от потока данных}
	\end{scnsubdividing}
	\begin{scnsubdividing}
		\scnitem{архитектура вычислительной системы с управлением от потока команд}
		\begin{scnindent}
			\scnnote{Примером такой архитектуры является архитектура фон-Неймана.}
		\end{scnindent}
		\scnitem{архитектура вычислительной системы без управления от потока команд}
	\end{scnsubdividing}
	\begin{scnsubdividing}
		\scnitem{архитектура вычислительной системы с процессором с арифметико-логическим устройством}
		\begin{scnindent}
			\scnnote{Примером такой архитектуры является архитектура фон-Неймана.}
		\end{scnindent}
		\scnitem{архитектура вычислительной системы без процессора с арифметико-логическим устройством}
	\end{scnsubdividing}
	\begin{scnsubdividing}
		\scnitem{архитектура вычислительной системы с управляющим блоком со счетчиком инструкций}
		\begin{scnindent}
			\scnnote{Примером такой архитектуры является архитектура фон-Неймана.}
		\end{scnindent}
		\scnitem{архитектура вычислительной системы без управляющего блока со счетчиком иструкций}
	\end{scnsubdividing}
	\begin{scnsubdividing}
		\scnitem{архитектура вычислительной системы с управляющим блоком с регистром команд}
		\begin{scnindent}
			\scnnote{Примером такой архитектуры является архитектура фон-Неймана.}
		\end{scnindent}
		\scnitem{архитектура вычислительной системы без управляющего блока с регистром команд}
	\end{scnsubdividing}
	\begin{scnsubdividing}
		\scnitem{архитектура вычислительной системы с устройством ввода-вывода}
		\begin{scnindent}
			\scnnote{Примером такой архитектуры является архитектура фон-Неймана.}
		\end{scnindent}
		\scnitem{архитектура вычислительной системы без устройства ввода-вывода}
	\end{scnsubdividing}
	\begin{scnsubdividing}
		\scnitem{архитектура вычислительной системы с доступом к внешнему (оперативному) запоминающему устройству}
		\begin{scnindent}
			\scnnote{Примером такой архитектуры является архитектура фон-Неймана.}
		\end{scnindent}
		\scnitem{архитектура вычислительной системы без доступа к внешнему (оперативному) запоминающему устройству}
	\end{scnsubdividing}
	\begin{scnsubdividing}
		\scnitem{архитектура вычислительной системы с масштабируемой (модульной) глобальной памятью}
		\begin{scnindent}
			\scnnote{Масштабируемость как свойство архитектуры важно для систем, ориентированных на обучение (самообучение) с целью решения широкого класса задач. Подобные архитектуры могут быть ориентированы на обработку структур знаний, интегрированных в единое смысловое пространство.}
			\begin{scnindent}
				\begin{scnrelfromlist}{источник}
					\scnitem{\scncite{Ivashenko2019InfiniteMemory}}
					\scnitem{\scncite{Ivashenko2016BSUIR}}
				\end{scnrelfromlist}
			\end{scnindent}
		\end{scnindent}
		\scnitem{архитектура вычислительной системы без масштабируемой глобальной памяти}
	\end{scnsubdividing}
	\begin{scnsubdividing}
		\scnitem{архитектура вычислительной системы с поддержкой модели активной графовой памяти}
		\begin{scnindent}
			\scnnote{Модель активной графовой памяти в архитектурах вычисленных систем важна для эффективной и согласованной (конвергентной) реализации параллельных процессов обработки знаний, включая механизмы возбуждения и торможения процессов обработки знаний. Модель активной графовой памяти ориентирована на реализацию представления знаний в смысловом пространстве и реализацию систем, управляемых знаниями.}
			\begin{scnindent}
				\begin{scnrelfromlist}{источник}
					\scnitem{\scncite{Ivashenko2021OSTIS}}
					\scnitem{\scncite{Ivashenko2022}}
					\scnitem{\scncite{Ivashenko2016BSUIR}}
				\end{scnrelfromlist}
			\end{scnindent}
		\end{scnindent}
		\scnitem{архитектура вычислительной системы без поддержки модели активной графовой памяти}
	\end{scnsubdividing}
	\begin{scnsubdividing}
		\scnitem{архитектура вычислительной системы с поддержкой параллельной обработки информации}		
		\begin{scnindent}
			\scnnote{Архитектуры вычислительных систем с поддержкой параллельной обработки знаний важны для эффективной реализации процессов обработки знаний, повышения производительности и масштабируемости систем обработки знаний, включая многоагентные системы в виде интеллектуальных компьютерных систем и коллективов интеллектуальных компьютерных систем.}
			\begin{scnindent}
				\begin{scnrelfromlist}{источник}
					\scnitem{\scncite{Ivashenko2020ReductionScheme}}
					\scnitem{\scncite{Ivashenko2020}}
					\scnitem{\scncite{Kuzmickij2000}}
				\end{scnrelfromlist}
			\end{scnindent}
		\end{scnindent}
		\scnitem{архитектура вычислительной системы с последовательной обработкой информации}
	\end{scnsubdividing}
	\begin{scnsubdividing}
		\scnitem{архитектура вычислительной системы с поддержкой последовательной модели консистентности глобальной оперативной памяти}
		\begin{scnindent}
			\scnnote{Архитектуры с поддержкой моделей консистентности ориентированы на решение задач управления взаимодействующими процессами, включая их синхронизацию и синхронные и асинхронные механизмы исполнения алгоритмов обработки знаний. Целью поддержки последовательной модели консистентности является обеспечение существования глобальных состояний базы знаний как структур единого смыслового пространства в интеллектуальных компьютерных системах.}
			\begin{scnindent}
				\scnrelfrom{смотрите}{Предметная область и онтология решателей задач ostis-систем}
				\begin{scnrelfromlist}{источник}
					\scnitem{\scncite{Ivashenko2020}}
					\scnitem{\scncite{Ivashenko2021PRIP}}
					\scnitem{\scncite{Gaponov2000}}
					\scnitem{\scncite{Serdiukov2004}}
				\end{scnrelfromlist}
			\end{scnindent}
		\end{scnindent}
		\scnitem{архитектура вычислительной системы без поддержки последовательной модели консистентности глобальной оперативной памяти}
	\end{scnsubdividing}
	\begin{scnsubdividing}
		\scnitem{архитектура с поддержкой причинной модели консистентности памяти}
		\begin{scnindent}
			\scnnote{Архитектуры с поддержкой моделей консистентности ориентированы на решение задач управления взаимодействующими процессами, включая их синхронизацию и синхронные и асинхронные механизмы исполнения алгоритмов обработки знаний. Целью поддержки причинной модели консистентности является обеспечение интероперабельности и конвергенции в едином смысловом пространстве структур знаний агентов коллективов интеллектуальных компьютерных системах. Для обеспечение той или иной модели консистентности могут использоваться различные механизмы.}
			\begin{scnindent}
				\scnrelfrom{смотрите}{Предметная область и онтология решателей задач ostis-систем}
				\begin{scnrelfromlist}{источник}
					\scnitem{\scncite{Ivashenko2020}}
					\scnitem{\scncite{Gaponov2000}}
					\scnitem{\scncite{Serdiukov2004}}
				\end{scnrelfromlist}
			\end{scnindent}
		\end{scnindent}
		\scnitem{архитектура без поддержки причинной модели консистентности памяти}
	\end{scnsubdividing}
	\begin{scnsubdividing}
		\scnitem{архитектура вычислительной системы, обладающая асимметрией}
		\begin{scnindent}
			\scnnote{Архитектуры вычислительных систем, обладающие асимметрией важны для эволюции многоагентных систем, интеллектуальных компьютерных систем и их коллективов, в которых асимметрия рассматривается в широком смысле, в том числе и как неоднородность таких систем или коллективов. Частным случаем случаем неоднородности является разнородность и гетерогенность архитектуры, которая позволяет реализовывать как интегрированное, так и гибридные модели обработки знаний, в рамках интеллектуальных компьютерных систем и их коллективов.}
		\end{scnindent}
		\scnitem{архитектура вычислительной системы, обладающая симметрией}
	\end{scnsubdividing}
	
	\scnheader{ассоциативный семантический компьютер}
	\scntext{примечание}{Для определения архитектуры \textit{ассоциативных семантических компьютеров}, в соответствии с выявленными классами и признаками, а также выработанными общими принципами, лежащими в основе таких архитектур, необходимо в рамках соответствующего признакового пространства рассмотреть конкретные множества архитектур и провести сравнительный анализ элементов этих множеств с целью обоснования выбора (оптимальных) вариантов архитектуры \textit{ассоциативных семантических компьютеров}.}
\bigskip
\end{scnsubstruct}
\scnsourcecomment{Завершили \scnqqi{Сегмент. Анализ существующих архитектур вычислительных систем}}


        \scnsegmentheader{Сегмент. Общие принципы, лежащие в основе ассоциативных семантических компьютеров для ostis-систем}
\begin{scnsubstruct}
	
	\scnheader{информационно-логическая задача}
	\scnidtf{информационно-логическая или комбинаторная задачи по переработке сложноструктурированных баз данных}
	\scnidtf{класс задач, предполагающих переработку нечисловой сложноструктурированной информации и допускающих при этом отсутствие точного алгоритма их решения}
	\scntext{примечание}{Понятие \textit{информационно-логических задач} фактически совпадает по смыслу с широко используемым в последнее время понятием задач искусственного интеллекта (\textit{интеллектуальных задач}), что позволяет использовать оба эти термина.}
	\scntext{примечание}{Разработка средств решения задач того или иного класса в настоящее время обычно осуществляется путем создания \textit{языка программирования высокого уровня}, ориентированного на этот класс задач, и путем реализации такого языка на современных компьютерах, т.е. путем создания транслятора. Поскольку логика решения задач искусственного интеллекта плохо согласуется с современными языками программирования, более целесообразной является разработка принципиально новых языков, отражающих на уровне их элементарных операций основные логические механизмы решения задач рассматриваемого класса. Такие языки программирования обычно называют языками сверх-высокого уровня или непроцедурными языками. Реализация языков сверх-высокого уровня на современных компьютерах представляется весьма сложной в силу большого разрыва между языками этого класса и внутренними языками современных компьютеров, для преодоления которого создание эффективного транслятора оказывается практически невозможным.}
	\scntext{пояснение}{Таким образом, состояние проблемы автоматизации решения задач искусственного интеллекта (информационно-логических задач) в настоящее время можно охарактеризовать тем, что эта проблема входит в противоречие c принципами логической организации современных компьютеров и, в первую очередь, с используемыми в современных компьютерах внутренними языками. Причины плохой приспособленности современных компьютеров к решению информационно-логических задач:
		\begin{scnitemize}
			\item в современных компьютерах при работе со сложноструктурированными базами данных время, затрачиваемое на информационный поиск, на 2-3 порядка превышает время, затрачиваемое на собственно переработку;
			\item в современных компьютерах имеет место весьма низкий уровень аппаратно реализуемых операций над нечисловыми данными;
			\item представление информации в памяти современных компьютеров имеет уровень весьма далекий от семантического, что делает переработку информации довольно громоздкой, требующей учета большого количества деталей, касающихся не смысла перерабатываемой информации, a способа ее представления в памяти;
			\item в современных компьютерах громоздко реализуются даже простейшие процедуры логического вывода.
		\end{scnitemize}
		Перечисленные причины, по существу, не устраняются также и в развиваемых в настоящее время подходах к построению нетрадиционных высокопроизводительных компьтеров, ибо, в основном, все эти подходы (даже если они достаточно далеко отходят от предложенных фон Нейманом принципов организации вычислительных машин) неявно сохраняют точку зрения на компьютер как на большой арифмометр и тем самым сохраняют ее ориентацию на задачи числового характера. Очевидно, что эффективность этих машин будет прежде всего определяться степенью близости их внутреннего языка к языкам непроцецурного типа (к языкам сверх-высокого уровня). Учитывая указанное назначение таких машин, их естественно назвать не вычислительными машинами, a математическими машинами или даже мыслящими машинами.}
	\scnheader{машина, ориентированная на решение информационно-логических задач}
	\scnrelfrom{класс решаемых задач}{информационно-логическая задача}
	\scnidtf{машина, реализующая стремление так организовать процесс переработки информации, чтобы он был наиболее близок к семантическому (содержательному) уровню}
	\begin{scnrelfromset}{задачи разработки}
		\scnfileitem{разработка семантического способа представления перерабатываемой информации в памяти машины}
		\scnfileitem{разработка такого внутреннего языка, запись программ на котором была бы максимально близка к тому, что называют записью алгоритма на содержательном уровне}
		\scnfileitem{разработка и исследование способов семантического представления информации различного вида}
		\scnfileitem{разработка и исследование принципов организации развитой ассоциативной памяти для непосредственного хранения семантического представления информации}
		\scnfileitem{разработка и исследование языка программирования высокого уровня, который (I) ориентирован на решение информационно-логических задач; (2) обеспечивает непосредственную реализацию простых процедур логического вывода. (3) согласован c выбранным способом семантического представления перерабатываемой информации, (4) приспособлен к использованию в качестве внутреннего языкё параллельной однородной структуры, имеющей распределенную ассоциативную память для хранения сложноструктурированных данных}
		\scnfileitem{разработка и исследование принципов построения и принципов параллельного взаимодействия функциональных средств, обеспечивающих непосредственную переработку семантического представления информации в распределенной ассоциативной памяти и реализующих управление потоком словноструктурированных данных}
		\scnfileitem{экспериментальная проверка полученных результатов}
	\end{scnrelfromset}
	\scntext{примечание}{Исследования по системам искусственного интеллекта убедительно показали, что способ представления знаний в их памяти, точнее степень его близости к семантическому, является фактором, во многом определяющим эффективность таких систем.}
	\begin{scnrelfromset}{принципы построения}
		\scnfileitem{такие машины целесообразно строить как машины, манипулирующие графовыми структурами непосредственно на физическом уровне}
		\begin{scnindent}
			\scntext{пояснение}{Предполагается создание структурно-перестраиваемой (графовой) памяти. Такая память состоит из ячеек, однозначно соответствующих вершинам хранимого в памяти графа, но, в отличие от обычной памяти, эти ячейки связываются не фиксированными условными связями, задающими фиксированную последовательность (порядок) ячеек в памяти, a реально (физически) проводимыми связями произвольной конфигурации. Эти связи соответствуют дугам, ребрам, гиперребрам записанного в памяти графа. Очевидно, что в ходе переработки информации в структурно-перестраиваемой памяти меняются на только и даже не столько состояния ячеек памяти, как это имеет место в обычной памяти, сколько конфигурация связей между этими ячейками. Т.е. в структурно-перестраиваемой памяти в ходе переработки информации не только перераспределяются метки на вершинах записанного в памяти графа, но и меняется структура самого этого графа.}
		\end{scnindent}
		\scnfileitem{в качестве внутреннего языка таких машин целесообразно использовать язык типа PROLOG}
		\begin{scnindent}
			\scntext{пояснение}{Разработка языка типа PROLOG, предназначенного к использованию в качестве внутреннего языка программирования для машин co структурно-перестраиваемой памятью требует решения нетривиальной задачи согласования графового способа представления данных в структурно-перестраиваемой памяти и способа записи в этой же памяти самих программ, описывающих переработку этих данных. Переход на графовый способ кодирования программ и данных в структурно-перестраиваемой памяти обеспечивает компактность их представления и существенно упрощает аппаратурную реализацию операций над сложными структурами. Говоря об аппаратурной интерпретации языка типа PROLOG, необходимо подчеркнуть следующее. На уровне любого языка типа PROLOG, т.е. на уровне абстрактной PROLOG --- машины, естественным образом реализуется эффективное распараллеливание процесса переработки сложных структур, организованное по принципу управления потоком запросов или управления потоком перерабатываемых сложноструктурированных данных. Управление потоком сложноструктурированных данных при этом основывается на использовании развитой формы ассоциативного доступа, a именно, доступа к произвольному фрагменту перерабатываемых данных (фрагменту, имеющему произвольный размер и произвольную структуру). Из вышесказанного следует, что создание машины, аппаратурно интерпретирующей язык типа PROLOG, есть не что иное, как создание параллельной машины, управляемой потоком сложноструктурированных данных и имеющей развитую ассоциативную память.}
		\end{scnindent}
		\scnfileitem{принцип организации переработки информации непосредственно в памяти}
		\begin{scnindent}
			\scntext{пояснение}{Этот принцип обеспечивает значительное ускорение переработки информации благодаря исключению этапов передачи информации из памяти в процессор и обратно, но оплачивается ценой большой избыточности функциональных (процессорных) средств, равномерно распределяемых по памяти. При распределении функциональных средств по структурно-перестраиваемой памяти каждая ячейка дополняется функциональным (процессорным) элементом, a перестраиваемые связи между ячейками становятся коммутируемыми каналами связи между функциональными элементами. Каждый функциональный элемент при этом имеет свою специальную внутреннюю регистровую память, отражающую важные для данного функционального элемента аспекты текущего состояния процесса выполнения элементарных операций внутреннего языка.}
		\end{scnindent}
	\end{scnrelfromset}
	\begin{scnrelfromset}{формальная основа и направления исследований}
		\scnfileitem{сочетание теории множеств, теории алгебраических систем, теории графов, математической логики, теории вычислений, методов машинного моделирования}
		\scnfileitem{теория наиболее общего вида структур данных (квази-графов), используемых в задачах искусственного интеллекта и являющихся обобщением классических алгебраических моделей}
		\scnfileitem{ориентированный на аппаратурную интерпретацию способ кодирования сложноструктурированных данных (квази-графв) в виде однородных семантических сетей специального вида}
		\scnfileitem{новый логический язык, являющийся модификацией классического и обеспечивающий описание квази-графов}
		\scnfileitem{графовые варианты логического языка описания квази-графов и, в частности, ориентированный на аппаратурную интерпретацию способ кодирования логических формул в виде однородных семантических сетей}
		\scnfileitem{непроцецурный язык программирования типа PROLOG, отличающийся удобством работы со сложными структурами данных (квази-графами) и развитостью средств управления вычислительным процессом}
		\scnfileitem{предлагаемый в качестве внутреннего аппаратурно интерпретируемого языка программирования способ записи непроцедурных программ в виде однородных семантических сетей}
		\scnfileitem{архитектура и принципы организации однородной ассоциативной параллельной структуры, ориентированной на переработку семантических сетей и обеспечивающей аппаратурную интерпретацию непроцецурного языка программирования типа PROLOG}
	\end{scnrelfromset}
	\scntext{задачи исследований}{Исследовать пути построения параллельных машин, управляемых потоком сложноструктурированных данных. В качестве памяти таких машин рассмотреть структурно-перестраиваемые запоминающие среды, обеспечивающие непосредственное хранение графовых структур и манипулирование ими, a также обеспечивающие ассоциативный доступ к произвольным фрагментам перерабатываемых графовых структур (фрагментам, имеющим произвольный вид и произвольный размер). Исследовать пути и принципы аппаратной интерпретации непроцецурных языков программирования типа PROLOG. B качестве интерпретируемого (внутреннего) языка исследовать язык программирования графового типа, являющийся способом записи программ в виде однородных семантических сетей.}
	\scntext{достоинства}{Разработка \textit{машин, ориентированных на решение информационно-логических задач} позволит:
		\begin{scnitemize}
			\item существенно расширить класс аппаратурно интерпретируемых (непосредственно перарабатываемых) структур данных;
			\item обеспечить высокую скорость переработки сложноструктурированных данных, благодаря (1) укрупнению аппаратно реализуемых операций преобразования структур данных, (2) глубокому распараллеливанию процесса переработки сложных структур как на программном, так и на микропрограммном уровне, (3) организации переработки информации непосредственно в памяти;
			\item существенно расширить логические возможности компьютеров благодаря использованию логического языка в качестве основы внутреннего языка программирования;
			\item обеспечить достаточно высокую технологичность компьютеров рассматриваемого класса благодаря их организации как однородных структур.
		\end{scnitemize}
	}
	
	\scnheader{ассоциативный семантический компьютер}
	\scntext{примечание}{На данном этапе работы были выявлены несколько основных вариантов вариантов архитектуры \textit{ассоциативных семантических компьютеров}. В основу предлагаемого подхода к разработке \textit{ассоциативного семантического компьютера} положены идеи, предложенные в работах В.В. Голенкова и получившие развитие в работе В.М. Кузьмицкого.}
	\begin{scnindent}
		\begin{scnrelfromlist}{источник}
			\scnitem{\scncite{Golenkov1996}}
			\scnitem{\scncite{Kuzmickij2000}}
		\end{scnrelfromlist}
	\end{scnindent}
	\scntext{предпосылка создания}{При формализации предметных областей, имеющих достаточно сложную семантическую организацию, перерабатываемые данные естественным образом группируются в некоторые сложные структуры. Эффективность решения задач, связанных с переработкой сложноструктурированных данных, на многопроцессорных вычислительных системах значительно возрастает в том случае, когда структура связей между процессорными элементами вычислительной системы, решающей эту задачу, совпадает со структурой данных, перерабатываемых в ходе ее решения (или, в более общем случае --- отображается в структуру перерабатываемых данных простым и естественным образом). При переходе к переработке данных все более сложной структурной и семантической организации (а затем и к переработке знаний) сохранение высокой эффективности вычислительной системы обеспечивается главным образом путем увеличения числа одновременно работающих процессорных элементов и усложнения структуры связей между ними.}
	\begin{scnindent}
		\scnrelfrom{источник}{\scncite{Kuzmickij2000}}
		\scnnote{Такую тенденцию развития технических средств ЭВМ мы и рассмотрим в качестве основной линии эволюции, создающей предпосылки для появления \textit{ассоциативных семантических компьютеров}. К ней относятся параллельные регулярные спецпроцессоры (векторные, матричные), спецвычислители для решения задач на графах и средства аппаратной поддержки семантических и нейронных сетей. К этой линии примыкают также и ассоциативные процессоры (в которых в роли процессорных элементов выступают ячейки ассоциативной памяти), процессоры баз данных и вычислительные системы, эффективно решающие те или иные классы задач за счет совпадения структуры связей между процессорными элементами со структурой информационного графа алгоритма (систолические вычислители, машины потоков данных).}
		\begin{scnindent}
			\scnrelfrom{источник}{\scncite{Kuzmickij2000}}
		\end{scnindent}
	\end{scnindent}
	
	\scnheader{архитектура вычислительной системы}
	\scnnote{Закономерным результатом развития вычислительных систем является переход к системам, изменяющим структуру связей между процессорными элементами в процессе функционирования. Такие системы настраивают свою внутреннюю структуру на структуру перерабатываемых данных и информационные графы алгоритмов решаемых задач и могут решать разные классы задач, сохраняя при этом высокую эффективность.}
	
	\scnheader{ЭВМ, ориентированная на переработку знаний}
	\scnnote{ЭВМ, ориентированная на переработку знаний, должна представлять собой в общем случае коллектив спецпроцессоров, ориентированных на максимально эффективное решение тех или иных классов задач.}
	\begin{scnrelfromset}{свойства}
		\scnfileitem{Спецпроцессоры представляют собой многопроцессорную вычислительную систему.}
		\begin{scnindent}
			\scnnote{В качестве семантического спецпроцессора можно предложить нелинейную (графовую) структурно перестраиваемую (динамическую) процессоро-память, аппаратно реализующую некоторый язык переработки семантических сетей, а саму ЭВМ такого рода можно, таким образом, назвать графодинамическим параллельным ассоциативным компьютером или \textit{ассоциативным семантическим компьютером}.}
		\end{scnindent}
		\scnfileitem{Структура связей между процессорными элементами спецпроцессоров совпадает со структурой данных или (реже) со структурой информационного графа алгоритма решения задачи.}
		\scnfileitem{Связи между процессорными элементами спецпроцессоров имеют перестраиваемую структуру.}
		\scnfileitem{Набор и функции спецпроцессоров определяются для каждой машины переработки знаний конкретно в зависимости от набора предметных областей, на которые эта машина ориентирована, и специфики задач, решаемых в этих областях.}
		\scnfileitem{Выявленный для некоторого семантического процессора набор механизмов переработки знаний должен быть \scnqq{погружен} в язык представления и переработки знаний. При этом наиболее удобными для этой цели представляются языки семантических сетей.}
		\scnfileitem{Процессорные элементы соответствуют вершинам или фрагментам семантической сети.}
		\scnfileitem{Переработка информации сводится к изменению структуры связей между процессорными элементами, соответствующему изменению конфигурации семантической сети.}
	\end{scnrelfromset}
	
	\scnheader{ассоциативный семантический компьютер}
	\begin{scnrelfromset}{принципы, лежащие в основе}
		\scnfileitem{Нелинейная память --- каждый элементарный фрагмент хранимого в памяти текста может быть логически инцидентен неограниченному числу других элементарных фрагментов этого текста. Таким образом, ячейки памяти, в отличие от обычной памяти, связываются не фиксированными условными связями, задающими фиксированную последовательность (порядок) ячеек в памяти, a логически или даже физически (с использованием технических средств коммутации) проводимыми связями произвольной конфигурации. Эти связи соответствуют дугам, ребрам, гиперребрам записанного в памяти графа (sc-текста).}
		\scnfileitem{Структурно-перестраиваемая (реконфигурируемая) память --- процесс отработки хранимой в памяти информации сводится не только к изменению состояния элементов, но и к реконфигурации связей между ними. То есть, в ходе переработки информации в структурно-перестраиваемой памяти меняются на только и даже не столько состояния ячеек памяти, как это имеет место в обычной памяти, сколько конфигурация связей между этими ячейками. Т.е. в структурно-перестраиваемой памяти в ходе переработки информации не только перераспределяются метки на вершинах записанного в памяти графа, но и меняется структура самого этого графа.}
		\scnfileitem{В качестве внутреннего способа кодирования знаний, хранимых в памяти \textit{ассоциативного семантического компьютера}, используется универсальный (!) способ нелинейного (графоподобного) смыслового представления знаний --- SC-код.}
		\scnfileitem{Обработка информации осуществляется коллективом агентов, работающих над общей памятью. Каждый из них реагирует на соответствующую ему ситуацию или событие в памяти (компьютер, управляемый хранимыми знаниями).}
		\scnfileitem{Есть программно реализуемые агенты, поведение которых описывается хранимыми в памяти агентно-ориентированными программами, которые интерпретируются соответствующими коллективами агентов.}
		\scnfileitem{Есть базовые агенты, которые не могут быть реализованы программно (в частности, это агенты интерпретации агентных программ, базовые рецепторные агенты-датчики, базовые эффекторные агенты).}
		\scnfileitem{Все агенты работают над общей памятью одновременно. Более того, если для какого-либо агента в некоторый момент времени в различных частях памяти возникает сразу несколько условий его применения, разные информационные процессы, соответствующие указанному агенту в разных частях памяти могут выполняться одновременно.}
		\scnfileitem{Для того, чтобы информационные процессы агентов, параллельно выполняемые в общей памяти не \scnqq{мешали} друг другу, для каждого информационного процесса в памяти фиксируется и постоянно актуализируется его текущее состояние. То есть каждый информационный процесс сообщает всем остальным о своих намерениях и пожеланиях, которым остальные информационные процессы не должны препятствовать. Реализация такого подхода может выполняться, например, на основе механизма блокировок элементов семантической памяти.}
		\begin{scnindent}
			\scnrelfrom{смотрите}{Предметная область и онтология решателей задач ostis-систем}
		\end{scnindent}
		\scnfileitem{Процессор и память \textit{ассоциативного семантического компьютера} глубоко интегрированы и составляют единую процессоро-память. Процессор \textit{ассоциативного семантического компьютера} равномерно \scnqq{распределен} по его памяти так, что процессорные элементы одновременно являются и элементами памяти компьютера. То есть каждая ячейка дополняется функциональным (процессорным) элементом, a перестраиваемые связи между ячейками становятся коммутируемыми каналами связи между процессорными элементами. Каждый процессорный элемент при этом имеет свою специальную внутреннюю регистровую память, отражающую важные для данного процессорного элемента аспекты текущего состояния процесса выполнения элементарных операций языка микропрограмм, обеспечивающих интерпретацию языка более высокого уровня (Языка SCP).}
		\scnfileitem{Обработка информации в \textit{ассоциативном семантическом компьютере} сводится к реконфигурации каналов связи между процессорными элементами, следовательно память такого компьютера есть не что иное, как \uline{коммутатор} (!) указанных каналов связи. Таким образом, текущее состояние конфигурации этих каналов связи и есть текущее состояние обрабатываемой информации. Этот принцип обеспечивает значительное ускорение переработки информации благодаря исключению этапов передачи информации из памяти в процессор и обратно, но оплачивается ценой большой избыточности процессорных (функциональных) средств, равномерно распределяемых по памяти.}
	\end{scnrelfromset}
	\bigskip
\end{scnsubstruct}
\scnsourcecomment{Завершили \scnqqi{Сегмент. Общие принципы, лежащие в основе ассоциативных семантических компьютеров для ostis-систем}}


        \scnsegmentheader{Сегмент. Архитектура ассоциативных семантических компьютеров для ostis-систем}
\begin{scnsubstruct}
	
	
	\bigskip
\end{scnsubstruct}
\scnsourcecomment{Завершили \scnqqi{Сегмент. Архитектура ассоциативных семантических компьютеров для ostis-систем}}


        
        \bigskip
    \end{scnsubstruct}
    \scnendcurrentsectioncomment
\end{SCn}


\scsubsubsection[
    \protect\scneditors{Шункевич Д.В.;Корончик Д.Н.;Марковец В.С.;Зотов Н.В.;Орлов М.К.}
    \protect\scnmonographychapter{Глава 6.2. Программная платформа интеллектуальных компьютерных систем нового поколения}
    ]{Предметная область и онтология программных вариантов реализации базового интерпретатора логико-семантических моделей ostis-систем на современных компьютерах}
\label{sd_program_interp}
\begin{SCn}
\scnsectionheader{Предметная область и онтология программных вариантов реализации базового интерпретатора sc-моделей ostis-систем на современных компьютерах}
\begin{scnsubstruct}
    \begin{scnrelfromlist}{соавтор}
    	\scnitem{Зотов Н.В.}
    	\scnitem{Шункевич Д.В.}
   	\end{scnrelfromlist}
    \scnheader{Предметная область программных вариантов реализации базового интерпретатора  sc-моделей ostis-систем на современных компьютерах}
    \scniselement{предметная область}
    \begin{scnhaselementrolelist}{класс объектов исследования}
        \scnitem{ostis-платформа}
    \end{scnhaselementrolelist}
    \begin{scnhaselementrolelist}{ключевой объект исследования}
        \scnitem{Программная платформа ostis-систем}
    \end{scnhaselementrolelist}

    \scnsegmentheader{Спецификация Программной платформы ostis-систем}
	\begin{scnsubstruct}
	\scntext{часто используемый sc-идентификатор}{Документация Программной платформы ostis-систем}
	\scntext{часто используемый sc-идентификатор}{Документация Программного варианта реализации ostis-платформы}
	\scnidtf{База знаний Программного варианта реализации ostis-платформы}
	\scnidtf{Семейство sc-языков, описывающих реализацию компонентов, входящих в состав Программного варианта реализации ostis-платформы (в том числе их программных интерфейсов)}
	\scnidtf{Подробная документация всех компонентов базового Программного варианта реализации ostis-платформы}
	\scnidtf{\textbf{Это пример того, как надо описывать современные программные компьютерные системы}}
	\scniselement{спецификация компонента}
	\scnsubset{База знаний Метасистемы OSTIS}
	\scnrelfrom{автор}{Зотов Н.В.}
	\scntext{идея}{
		Спецификация (документация) такого сложного программного объекта, как \textit{Программная платформа ostis-систем}, может и должна быть представлена на \uline{формальном} \textit{языке представления знаний}, в данном случае на \textit{SC-коде}, тексты которого она хранит и интерпретирует. В таком случае такой язык, который описывает \textit{Программный вариант реализации ostis-платформы}, должен являться \textit{подъязыком} \textit{SC-кода}, то есть должен наследовать все свойства \textit{Синтаксиса} и \textit{Денотационной семантики SC-кода}. Такой способ представления спецификации \textit{программных компьютерных систем} даёт \uline{безусловно} сильные преимущества по сравнению с другими возможными вариантами представления спецификаций \cite{Dillon2008}:
		\begin{itemize}
			\item Язык, тексты которого система хранит и обрабатывает, и язык спецификации того, как система представляет тексты первого языка в памяти самой себя, являются подмножествами \uline{одного и того же} языка. Это упрощает не только становление понимания разработчика, который разрабатывает сложную \textit{программную компьютерную систему}, за счет того, что форма представления обрабатываемого этой системой языка и языка ее спецификации, \uline{унифицирована}, но и позволяет открыть для этой системы новые функциональные возможности в \uline{познании} самой себя. Таким образом, такой подход позволяет полностью реализовывать свойства \textit{интеллектуальной компьютерной системы}, например, \textit{рефлексивности}.
			\item Нельзя проектировать и реализовывать \textit{интеллектуальные компьютерные системы} на \textit{программной компьютерной системе}, которая сама таковой не является. Представление спецификации системы в такой форме позволяет \uline{существенно} повысить уровень ее \textit{интеллекта} \cite{Zagorskiy2022b}.
			\item Нет необходимости в создании дополнительных средств для верификации и анализа работы всей системы, поскольку форма представления языка описания системы \uline{унифицирована} с языком, тексты которого она хранит и интерпретирует. Это позволяет не только уменьшить количество используемых средств при проектировании и реализации такой программной системы, но и позволяет \uline{унифицировать} информацию, хранимую в этой программной системе и описывающую эту программную систему, с целью использования этой информации в процессе эволюции её компонентов. При этом спецификация программной системы остаётся \textit{платформенно-независимой}, поэтому при смене одного варианта реализации \textit{ostis-платформы} на другой подход к описанию таких платформ остаётся \uline{одним и тем же}.
		\end{itemize}
	}
	\scnhaselementrole{ключевой sc-элемент}{\textbf{Программная платформа ostis-систем}}
	\begin{scnindent}
		\begin{scnrelfromset}{общие принципы документирования}
			\scnfileitem{Вне зависимости от языка реализации каждого компонента \textit{Программного варианта реализации ostis-платформы}, спецификация каждого компонента включает спецификацию, непосредственно описанную в \uline{исходных файлах} самого компонента, описывающую программный интерфейс этого компонента, а также спецификацию как \uline{части} базы знаний ostis-платформы, детально описывающую реализацию этого компонента, в том числе используемые алгоритмы.}
			\scnfileitem{\uline{Каждый} компонент \textit{Программного варианта реализации ostis-платформы} описывается средствами \textit{Технологии OSTIS}, то есть на \textit{SC-коде}, тексты которого она обрабатывает и хранит. Таким образом, это дает возможности платформе анализировать свое состояние и способствовать поддерживать свой жизненный цикл без участия ее разработчиков. \textbf{\textit{Программный вариант реализации ostis-платформы} выступает полноценным \uline{субъектом}}, принимающим непосредственное участие в собственной разработке.}
			\scnfileitem{\textit{Спецификация Программного варианта реализации ostis-платформы} представляет собой \uline{\textit{sc-язык}}, то есть подъязык \textit{SC-кода}, для которого уточнены \textit{Cинтаксис} и \textit{Денотационная семантика SC-кода}. Этот \textit{sc-язык} можно представить в виде некоторого семейства более частных sc-языков. Такой подход позволяет без особых препятствий интегрировать описания различных компонентов, входящих в состав \textit{Программного варианта реализации ostis-платформы}, поскольку вся \textit{Спецификация Программного варианта реализации ostis-платформы} является ее базой знаний.}
			\scnfileitem{\uline{Каждый} разработчик \textit{Программного варианта реализации ostis-} \textit{платформы} \uline{заботится} о перманентной поддержки не только состояния ее компонентов, но и спецификации этих компонентов. \uline{Гарантом} качественной \textit{Спецификации Программного варианта реализации ostis-платформы} является ее \uline{коллектив разработчиков}, способных не только понимать детали реализации \textit{ostis-платформы}, но (!) и способствовать к созданию взаимовыгодного сотрудничества для достижения поставленных целей.}
		\end{scnrelfromset}
		\begin{scnindent}
			\scnnote{Данные принципы можно использовать при описании любых (!) других \textit{программных компьютерных систем}, в том числе тех \textit{программных компьютерных систем}, которые не реализуются на данной \textit{ostis-платформе}.}
		\end{scnindent}
		\begin{scnrelfromset}{аналоги}
			\scnitem{Платформа CYC}
			\scnitem{Платформа SMILA}
			\scnitem{Графовая СУБД Neo4j}
			\scnitem{Графовая СУБД OrientDB}
			\scnitem{Графовая СУБД ArangoDB}
			\scnitem{Графовая СУБД Amazon Neptune}
		\end{scnrelfromset}
	\end{scnindent}
	
		
    \scnheader{Программная платформа ostis-систем}
    \scntext{часто используемый sc-идентификатор}{Программный вариант реализации ostis-платформы}
    \scnidtf{Программный вариант реализации базового интерпретатора sc-моделей ostis-систем на традиционных компьютерах}
    \scnidtf{Базовая программная платформа для массового создания интеллектуальных компьютерных систем нового поколения}
    \scnidtf{Программная реализация платформы интерпретации sc-моделей компьютерных систем}
    \scnidtf{Предлагаемый нами программный вариант реализации ассоциативного семантического компьютера}
    \begin{scnindent}
	    \scntext{пояснение}{Одним из путей, позволяющих осуществлять апробацию, развитие, а в ряде случаев и внедрение новых моделей и технологий вне зависимости от наличия соответствующих аппаратных средств является разработка программных моделей этих аппаратных средств, которые были бы функционально эквивалентны этим аппаратным средствам, но при этом интерпретировались на базе традиционной аппаратной архитектуры (в данной работе традиционной архитектурой будем считать архитектуру фон Неймана, как доминирующую в настоящее время). Очевидно, что производительность таких программных моделей в общем случае будет ниже, чем самих аппаратных решений, однако в большинстве случаев она оказывается достаточной для того, чтобы развивать соответствующую технологию параллельно с разработкой аппаратных средств и осуществления постепенного перевода уже работающих систем с программной модели на аппаратные средства.}
    \end{scnindent}
    \scnidtf{Реализация sc-машины}
    \scnidtf{sc-machine}
    \scniselement{базовая ostis-платформа}
    \scniselement{web-ориентированный вариант реализации ostis-платформы}
    \begin{scnindent}
    	\scnidtf{вариант реализации платформы интерпретации sc-моделей компьютерных систем, предполагающий взаимодействие пользователей с системой \uline{посредством} сети Интернет}
    \end{scnindent}
    \scniselement{многопользовательский вариант реализации ostis-платформы}
    \scniselement{многократно используемый компонент ostis-систем, хранящийся в виде файлов исходных текстов}
   	\scniselement{платформенно-зависимый многократно используемый компонент ostis-систем}
	\scniselement{неатомарный многократно используемый компонент ostis-систем}
    \scniselement{зависимый многократно используемый компонент ostis-систем}
    \scntext{адрес компонента}{https://github.com/ostis-ai/sc-machine}
    \begin{scnrelfromset}{декомпозиция программной системы}
    	\scnitem{Реализация sc-памяти}
    	\scnitem{Реализация scp-интерпретатора}
		\begin{scnindent}
			\scnsuperset{Реализация интерпретатора сетевого взаимодействия между ostis-системами}
    		\scnsuperset{Реализация интерпретатора sc-моделей пользовательских интерфейсов ostis-систем}
    	\end{scnindent}
    	\scnitem{Реализация менеджера многократно используемых компонентов ostis-систем}
    \end{scnrelfromset}
    \begin{scnrelfromset}{зависимости компонента}
    	\scnitem{Реализация sc-памяти в ostis-платформе}
    \end{scnrelfromset}
	\scntext{принципы реализации}{Поскольку sc-тексты представляют собой семантические сети, то есть, по сути, графовые конструкции определенного вида, то на нижнем уровне задача разработки программного варианта реализации платформы интерпретации sc-моделей сводится к разработке средств хранения и обработки таких графовых конструкций.К настоящему времени разработано большое количество простейших моделей представления графовых конструкций в линейной памяти, таких как матрицы смежности, списки смежности и другие (\cite{Diskrete_Math}). Однако, при разработке сложных систем как правило приходится использовать более эффективные модели, как с точки зрения объема информации, требуемого для представления, так и с точки зрения эффективности обработки графовых конструкций, хранимых в той или иной форме. К наиболее распространенным программным средствам, ориентированным на хранение и обработку графовых конструкций относятся графовые СУБД (Neo4j \cite{Neo4j}, ArangoDB \cite{ArangoDB}, OrientDB \cite{OrientDB}, Grakn \cite{Grakn} и др.), а также так называемые rdf-хранилища (Virtuoso \cite{Virtuoso}, Sesame \cite{Sesame} и др.), предназначенные для хранения конструкций, представленных в модели RDF. Для доступа к информации, хранимой в рамках таких средств, могут использоваться как языки, реализуемые в рамках конкретного средства (например, язык Cypher в Neo4j), так и языки, являющиеся стандартами для большого числа систем такого класса (например, SPARQL для rdf-хранилищ).Популярность и развитость такого рода средств приводит к тому, что на первый взгляд целесообразным и эффективным кажется вариант реализации \textit{программного варианта реализации платформы интерпретации sc-моделей} на базе одного из таких средств. Однако, существует ряд причин, по которым было принято решение о реализации \textit{программного варианта реализации платформы интерпретации sc-моделей} с нуля. К ним относятся следующие:
	\begin{scnitemize}
		\item Для обеспечения эффективности хранения и обработки \textit{информационных конструкций} определенного вида (в данном случае --- \textit{sc-конструкций}), должна учитываться специфика этих конструкций. В частности, описанные в работе \cite{Koronchik2013a} эксперименты показали значительный прирост эффективности собственного решения по сравнению с существующими на тот момент.   	
		\item В отличие от классических \textit{графовых конструкций}, где \textit{дуга} или \textit{ребро} могут быть инцидентны только \textit{узлу} \textit{графа} (это справедливо и для \textit{rdf-графов}) в \textit{SC-коде} вполне типичной является ситуация, когда \textit{sc-коннектор} инцидентен другому \textit{sc-коннектору} или даже двум \textit{sc-коннекторам} \cite{Ivashenko2022}. В связи с этим существующие средства хранения \textit{графовых конструкций} не позволяют в явном виде хранить \textit{sc-конструкции} (\textit{sc-графы}). Данная проблема также решаема при переходе от \textit{неориентированного графа} к \textit{орграфу} \cite{Ivashenko2015}.
		\item В основе обработки информации в рамках \textit{Технологии OSTIS} лежит \textit{многоагентный подход} \cite{Shunkevich2022a}, в рамках которого агенты обработки информации, хранимой в sc-памяти (sc-агенты) реагируют на события, происходящие в sc-памяти и обмениваются информацией посредством спецификации выполняемых ими действий в sc-памяти \cite{Shunkevich2018}. В связи с этим одной из важнейших задач является реализация в рамках \textit{Программного варианта реализации ostis-платформы} возможности подписки на события, происходящие в программной модели sc-памяти, которая на данный момент практически не поддерживается в рамках современных средств хранения и обработки графовых конструкций.
		\item \textit{SC-код} позволяет описывать также внешние \textit{информационные конструкции} любого рода (изображения, текстовые файла, аудио- и видеофайлы и так далее \cite{Ivashenko2022}, которые формально трактуются как содержимое \textit{sc-элементов}, являющихся знаками \textit{внешних файлов ostis-системы}. Таким образом, компонентом \textit{Программного варианта ostis-платформы} должна быть реализация файловой памяти, которая позволяет хранить указанные конструкции в каких-либо общепринятых форматах. Реализация такого компонента в рамках современных средств хранения и обработки \textit{графовых конструкций} также не всегда представляется возможной.
	\end{scnitemize}
	По совокупности перечисленных причин было принято решение о реализации \textit{программного варианта реализации платформы интерпретации sc-моделей} \scnqq{с нуля} с учетом особенностей хранения и обработки информации в рамках Технологии OSTIS.}
    \begin{scnrelfromset}{особенности реализации}
    	\scnfileitem{Текущий \textit{Программный вариант реализации ostis-платформы} является \uline{web-ориентированным}, поэтому с этой точки зрения каждая \mbox{ostis-система} представляет собой web-сайт, доступный онлайн посредством обычного браузера. Такой вариант реализации обладает очевидным преимуществом --- доступ к системе возможен из любой точки мира, где есть Интернет, при этом для работы с системой не требуется никакого специализированного программного обеспечения. С другой стороны, такой вариант реализации обеспечивает возможность параллельной работы нескольких пользователей с системой.}
    	\scnfileitem{Реализация является \uline{\textit{кроссплатформенной}} и может быть собрана из исходных текстов в различных \textit{операционных системах}. В то же время, взаимодействие клиентской и серверной части организовано таким образом, что \mbox{web-интерфейс} может быть легко заменен на настольный или мобильный интерфейс, как универсальный, так и специализированный.}
    	\scnfileitem{Текущий вариант реализации ostis-платформы является \uline{базовым}, то есть включает \textit{Реализацию scp-интерпретатора}).}
    	\scnfileitem{\uline{Ядром} этой платформы является \textit{Реализация sc-памяти}. Текущая \textit{Реализация sc-памяти} \uline{функционально полна}, то есть позволяет хранить \textit{sc-конструкции}, с помощью которых описывается любая \textit{sc-модель ostis-системы}, внешние \textit{информационные конструкции}, не принадлежащие \textit{SC-коду}, а также предоставлять различные уровни доступа для обработки этих конструкций. В контексте текущего \textit{Программного варианта реализации ostis-платформы} \textit{Реализация sc-памяти} включает \textit{Реализацию файловой памяти}, предназначенной для хранения внешних \textit{информационных конструкций}, не принадлежащих \textit{SC-коду}, то есть содержимое \textit{внутренних файлов ostis-системы}, но дополнительно описывающие, поясняющие и детализирующие \textit{sc-конструкции} \textit{sc-моделей ostis-систем}.}
    	\scnfileitem{Текущий \textit{Программный вариант реализации ostis-платформы} включает \textit{Реализацию менеджера многократно используемых компонентов ostis-систем}. В первую очередь это связано с тем, что текущая \textit{Реализация менеджера многократно используемых компонентов ostis-систем} использует \textit{Реализацию sc-памяти ostis-платформы} для хранения и обработки спецификаций устанавливаемых компонентов вне зависимости от языка их реализации, а также является платформенно-зависимым многократно используемым компонентом ostis-систем.}
    \end{scnrelfromset}
	
	\scnsegmentheader{Спецификация Реализации sc-памяти в Программной платформе ostis-систем}
	\begin{scnsubstruct}
	\scnidtf{Документация Реализации sc-памяти в Программной платформе ostis-систем}
	\scnhaselementrole{ключевой sc-элемент}{\textbf{Реализация sc-памяти}}

	\scnheader{Реализация sc-памяти}
	\scnidtf{Программный вариант реализации графодинамической ассоциативной памяти в Программной платформе ostis-систем}
 	\scnidtf{Программная модель семантической памяти, реализованная на основе традиционной линейной памяти и включающая средства хранения sc-конструкций и базовые средства для обработки доступа к этим конструкциям, в том числе удаленного доступа к ним посредством соответствующих сетевых языков (протоколов)}
	\scnidtf{Предлагаемый нами вариант реализации графодинамической ассоциативной памяти для ostis-систем}
	\scniselement{реализация sc-памяти}
	\scniselement{многократно используемый компонент ostis-систем, хранящийся в виде файлов исходных текстов}
	\scniselement{платформенно-зависимый многократно используемый компонент ostis-систем}
	\scniselement{неатомарный многократно используемый компонент ostis-систем}
	\scniselement{зависимый многократно используемый компонент ostis-систем}
	\scnrelto{программная модель}{sc-память}
	\begin{scnindent}
		\scnrelto{семейство подмножеств}{сегмент sc-памяти}
		\begin{scnindent}
			\scnidtf{страница sc-памяти}
	 		\scntext{пояснение}{В рамках данного \textit{Программного варианта реализации ostis-платформы} \textit{sc-память} моделируется в виде набора \textit{сегментов}, каждый из которых представляет собой фиксированного размера упорядоченную последовательность \textit{элементов sc-памяти}, каждый из которых соответствует конкретному \textit{sc-элементу}.}
			\scnrelto{семейство подмножеств}{элемент sc-памяти}
			\begin{scnindent}
				\scntext{пояснение}{В настоящее время каждый сегмент состоит из $2^{16}-1=65535$ \textit{элементов sc-памяти}. Каждый сегмент состоит из набора структур данных, описывающих конкретные \textit{sc-элементы} (элементов sc-памяти). Независимо от типа описываемого sc-элемента каждый \textit{элемент sc-памяти} имеет фиксированный размер (в текущий момент --- 44 байт), что обеспечивает удобство их хранения. Таким образом, максимальный размер базы знаний в текущей программной модели sc-памяти может достигнуть 223 Гб (без учета содержимого \textit{внутренних файлов ostis-системы}, хранимого в файловой памяти).}
			\end{scnindent}
			\scntext{примечание}{Выделение \textit{сегментов sc-памяти} позволяет, с одной стороны, упростить адресный доступ к \textit{элементам sc-памяти}, с другой стороны --- реализовать возможность выгрузки части \textit{sc-памяти} из \textit{оперативной памяти} на \textit{файловую систему} при необходимости. Во втором случае сегмент \textit{sc-памяти} становится минимальной (атомарной) выгружаемой частью sc-памяти. Механизм выгрузки сегментов реализуется в соответствии с существующими принципами организации виртуальной памяти в современных \textit{операционных системах}.}
			\scntext{примечание}{По умолчанию все сегменты физически располагаются в оперативной памяти, если объема памяти не хватает, то предусмотрен механизм выгрузки части сегментов на жесткий диск (механизм виртуальной памяти).}
		\end{scnindent}
	\end{scnindent}
	\begin{scnrelfromset}{зависимости компонента}
		\scnitem{Библиотека методов и структур данных GLib}
		\scnitem{Библиотека методов и структур данных C++ Standard Library}
	\end{scnrelfromset}
	\begin{scnrelfromlist}{используемый язык программирования}
		\scnitem{C}
		\scnitem{C++}
	\end{scnrelfromlist}
	\scnrelfrom{внутренний язык}{SCin-код}
	\begin{scnindent}
  		\scntext{примечание}{Такую модель sc-памяти достаточно просто описывать на \textit{sc-языке}, то есть на подъязыке \textit{SC-кода}. Такой язык позволяет описывать то, как внутри sc-памяти \textit{ostis-платформы} представляются тексты языка, на этом же языке. При этом соблюдается не только унификация представления информации, обрабатываемой \textit{ostis-платформой}, и информации, описывающей саму \textit{ostis-платформу}, но и даются возможности для расширения и использования языка в процессе эволюции \textit{ostis-платформы} и ее компонентов, в том числе в процессе эволюции \textit{Реализации sc-памяти}.}
	\end{scnindent}
    \scntext{адрес компонента}{https://github.com/ostis-ai/sc-machine}
    \begin{scnrelfromset}{декомпозиция программной системы}
        \scnitem{Реализация подсистемы управления доступом к sc-памяти}
        \scnitem{Реализация подсистемы управления процессами и событиями в sc-памяти}
        \scnitem{Реализация файловой памяти}
    \end{scnrelfromset}
	\scntext{примечание}{Текущий вариант \textit{Реализации sc-памяти в ostis-платформе} предполагает возможность сохранения состояния (слепка) \textit{sc-памяти} на жесткий диск и последующей загрузки из ранее сохраненного состояния. Такая возможность необходима для перезапуска системы, в случае возможных сбоев, а также при работе с исходными текстами \textit{базы знаний}, когда сборка из исходных текстов сводится к формированию слепка состояния памяти, который затем помещается в \textit{Реализации sc-памяти}.}
    \scntext{примечание}{В общем случае \textit{sc-память} может быть реализована по-разному. Так, например, другой вариант \textit{sc-памяти ostis-платформы} можно реализовать при помощи программной реализации \textit{Графовой СУБД Neo4j}. Отличие такого возможного варианта реализации \textit{sc-памяти} от текущего состоит в том, что хранение \textit{графовых конструкций} и управление потоком действий над ними должно осуществляться в большей мере средствами, предоставляемыми \textit{Графовой СУБД Neo4j}, в то же время представление \textit{графовых конструкций} должно реализовываться по-своему, поскольку зависит от \textit{Cинтаксиса SC-кода}.}
    
    \scnsegmentheader{Спецификация Метаязыка описания представления sc-конструкций в текущей Реализации sc-памяти Программной платформы ostis-систем}
    \begin{scnsubstruct}
    \scnidtf{Документация Метаязыка описания представления sc-конструкций в текущей Реализации sc-памяти Программной платформы ostis-систем}
    \scnhaselementrole{ключевой sc-элемент}{\textbf{SCin-код}}
  	\scntext{идея}{Раннее упоминалось об \textit{sc-языках}, которые используются при описании компонентов текущего \textit{Программного варианта реализации ostis-платформы}. Один из таких \textit{sc-языков} описывает то, как \textit{sc-конструкции} хранятся внутри \textit{sc-памяти ostis-платформы}. Такой язык называется \textbf{\textit{Метаязыком описания представления sc-конструкций в текущей Реализации sc-памяти Программной платформы ostis-систем}}, или, кратко, \textit{\textbf{SCin-кодом} (Semantic Сode interior)}. \textit{sc-память} текстов \textit{SC-кода} (как некоторую абстрактную модель, по которой можно реализовать \textit{sc-память ostis-платформы}) можно рассматривать как подмножество \textit{sc.in-текста}.}
    
   	\scnheader{SCin-код}
   	\scnidtf{Semantic Code interior}
   	\scnidtf{Язык описания представления SC-кода внутри sc-памяти ostis-платформы}
   	\scnidtf{Метаязык описания представления sc-конструкций в текущей Реализации sc-памяти Программной платформы ostis-систем}
   	\scntext{часто используемый sc-идентификатор}{sc.in-текст}
   	\begin{scnindent}
   		\scniselement{имя нарицательное}
   	\end{scnindent}
   	\scniselement{абстрактный язык}
   	\scniselement{метаязык}
   	\scniselement{sc-язык}
   	\scnsubset{SC-код}
   	\scnsuperset{sc-память}
   	
   	\scnheader{следует отличать*}
   	\begin{scnhaselementset}
   		\scnitem{SC-код}
   		\begin{scnindent}
   			\scnidtf{Универсальный язык внутреннего смыслового представления знаний в памяти ostis-систем}
   		\end{scnindent}
   		\scnitem{SCin-код}
   		\begin{scnindent}
   			\scnidtf{Метаязык описания представления SC-кода в sc-памяти ostis-платформы}
   			\scnsubset{SC-код}
   		\end{scnindent}
   	\end{scnhaselementset}
   	\scntext{пояснение}{Одним из замечательных достоинств \textit{SC-кода} является то, что синтаксис любого его \textit{подъязыка} наследует свойства \textit{Синтаксиса SC-кода}. То есть форма представления различных знаний, описываемых его подъязыками, остается одна и та же. В данном случае, \textit{Синтаксис SCin-кода} не является исключением и уточняет семантику \textit{Синтаксиса SC-кода}. При этом \textit{Синтаксис SCin-кода}, как и синтаксис любого другого sc-языка, является частью \textit{Денотационной семантики SCin-кода}. То есть \textit{Синтаксис SCin-кода} описывает семантику того, как формируются sc.in-тексты при помощи \textit{Синтаксических правил SCin-кода}. \textbf{\textit{Синтаксис SCin-кода}} задается: (1) \textit{Алфавитом SCin-кода}, (2) отношением инцидентности \textit{sc-адрес элемента sc-памяти*}. \textit{Алфавит SCin-кода\scnsupergroupsign}, а также отношение инцидентности \textit{sc-адрес элемента sc-памяти*} являются подмножествами \textit{Алфавит SC-кода\scnsupergroupsign} и отношений инцидентности \textit{SC-кода} соответственно, при этом правила \textit{Синтаксиса SCin-кода} включают правила \textit{Синтаксиса SC-кода} и правила, которые уточняют нюансы \textit{Синтаксиса SCin-кода}.}
    
   	\scnheader{Алфавит SCin-кода\scnsupergroupsign}
   	\scnidtf{синтаксический класс элемента sc-памяти}
   	\scnidtf{синтаксический тип элемента sc-памяти}
   	\scnidtf{Множество классов элементов sc-памяти}
   	\scnidtf{Множество типов элементов sc-памяти}
   	\scnrelto{алфавит}{SCin-код}
   	\begin{scneqtoset}
   		\scnitem{элемент sc-памяти, соответствующий sc-узлу}
   		\scnitem{элемент sc-памяти, соответствующий sc-коннектору}
   		\scnitem{элемент sc-памяти, имеющий нулевой sc-адрес}
   		\begin{scnindent}
   			\scniselement{синглетон}
   		\end{scnindent}
   	\end{scneqtoset}
    \scntext{пояснение}{\textbf{\textit{Алфавит SCin-кода\scnsupergroupsign}} состоит из трех синтаксически выделяемых классов элементов sc-памяти: \textit{элемента sc-памяти, соответствующего sc-узлу}, \textit{элемента sc-памяти, соответствующего sc-коннектору}, и \textit{элемента sc-памяти, имеющего нулевой sc-адрес}. Такой алфавит не только позволяет задавать в \textit{sc-памяти} минимальный набор объектов, с которым можно производить вычислительные операции, но и, при необходимости, удобен для расширения. Так, например, даннный алфавит языка можно расширить, добавив в него \textit{элемент sc-памяти, соответствующий внутреннему файлу ostis-системы}, либо \textit{элемент sc-памяти, соответствующий sc-ребру}.}
   
   	\scnstructheader{Синтаксис SCin-кода}
   	\begin{scnsubstruct}
   
   	\scnheader{элемент sc-памяти}
   	\scnidtf{элемент sc-памяти, соответствующий sc-элементу}
   	\scnidtf{ячейка sc-памяти}
   	\scnidtf{образ sc-элемента в рамках sc-памяти}
   	\scnidtf{структура данных, каждый экземпляр которой в рамках sc-памяти соответствует одному sc-элементу}
   	\scnidtf{sc\_element}
   	\begin{scnindent}
   		\scniselement{C}
   	\end{scnindent}
   	\scniselement{sc-элемент}
   	\scnsuperset{sc-элемент}
   	\scnrelfrom{разбиение}{Алфавит SCin-кода\scnsupergroupsign}
    
    \scnheader{sc-адрес}
    \scnidtf{адрес элемента sc-памяти, соответствующего заданному sc-элементу, в рамках текущего варианта реализации sc-памяти}
    \scnidtf{sc\_addr}
    \begin{scnindent}
    	\scniselement{C}
    \end{scnindent}
    \scnidtf{ScAddr}
    \begin{scnindent}
    	\scniselement{C++}
    \end{scnindent}
    \scntext{пояснение}{Каждый элемент sc-хранилища в текущей реализации может быть однозначно задан его адресом (sc-адресом), состоящим из номера сегмента и номера \textit{элемента sc-памяти} в рамках сегмента. Таким образом, \textit{sc-адрес} служит уникальными координатами \textit{элемента sc-памяти} в рамках \textit{Реализации sc-памяти}.}
    \scntext{примечание}{sc-адрес никак не учитывается при обработке базы знаний на семантическом уровне и необходим только для обеспечения доступа к соответствующей структуре данных, хранящейся в линейной памяти на уровне \textit{Реализации sc-памяти}.}
    \scntext{примечание}{В общем случае sc-адрес элемента sc-памяти, соответствующего заданному sc-элементу, может меняться, например, при пересборке базы знаний из исходных текстов и последующем перезапуске системы. При этом sc-адрес элемента sc-памяти, соответствующего заданному sc-элементу, непосредственно в процессе работы системы в текущей реализации меняться не может.}
    \begin{scnrelfromlist}{семейство отношений, однозначно задающих структуру заданной сущности}
    	\scnitem{номер сегмента sc-памяти*}
    	\scnitem{номер элемента sc-памяти в рамках сегмента*}
    \end{scnrelfromlist}
    \scntext{пояснение}{Для каждого \textit{sc-адреса элемента sc-памяти} можно \textit{взаимно однозначно} поставить в соответствие некоторый \textit{хэш}, полученный в результате применения специальной \textit{хэш-функции*} над этим \textit{sc-адресом элемента sc-памяти}. \textit{хэш} является \textit{неотрицательным целым числом} и является результатом преобразования \textit{номера sc-сегмента sc-памяти} \textit{si}, в котором располагается \textit{элемент sc-памяти}, и \textit{номера} этого \textit{элемента sc-памяти} \textit{ei} \textit{в рамках} этого {sc-сегмента} \textit{si}. В рамках \textit{sc-памяти} используется единственная \textit{хеш-функция*} для получения \textit{хеша sc-адреса элемента sc-памяти} и задается как $f(si, ei) = si << 16 | ei \& 0xffff$, где операция $<<$ --- операция \textit{битового сдвига влево*} левого аргумента на количество единиц, заданное правым аргументом, относительно этой операции, операция $|$ --- операция \textit{битового сложения*}, операция $\&$ --- операция \textit{битового умножения*}, число $0xffff$ --- \textit{Число 65535}, представленное в шестнадцатеричном виде и обозначающее максимальное количество элементов в одном \textit{sc-сегменте sc-памяти}. Числовое выражение \textit{sc-адреса} является \textit{32-битовым целым числом}.}
    \scntext{пояснение}{Отношение \textbf{\textit{sc-адрес элемента sc-памяти*}} определяется как \textit{взаимно однозначное соответствие}, первым компонентом каждой ориентированной пары которого является некоторый элемент sc-памяти, соответствующей некоторому sc-элементу, а вторым компонентом является sc-адрес этого \textit{элемента sc-памяти}. То есть каждый \textit{элемент sc-памяти} имеет уникальный \textit{sc-адрес} как некоторый идентификатор, с помощью которого определяется уникальность \textit{элемента sc-памяти}.}
    
   	\scnheader{класс элемента sc-памяти\scnsupergroupsign}
   	\scnidtf{класс всех синтаксических и семантических классов элементов sc-памяти}
   	\scnidtf{битовая маска, обозначающая синтаксический и семантический класс элемента sc-памяти}
   	\scnidtf{sc\_type}
   	\begin{scnindent}
   		\scniselement{C}
   	\end{scnindent}
   	\scnidtf{ScType}
   	\begin{scnindent}
   		\scniselement{C++}
   	\end{scnindent}
   	\begin{scnrelfromset}{разбиение}
   		\scnitem{синтаксический класс элемента sc-памяти\scnsupergroupsign}
   		\scnitem{семантический класс элемента sc-памяти\scnsupergroupsign}
   	\end{scnrelfromset}
   	\scnrelto{следствие}{\scnfilelong{В рамках любой \textit{реализации sc-памяти} должен существовать набор \textit{синтаксических} и \textit{семантических классов элементов sc-памяти\scnsupergroupsign} (меток), которые:
   		\begin{itemize}
   			\item задают тип элемента на уровне \textit{ostis-платформы} и не имеют соответствующей \textit{sc-дуги принадлежности} (а точнее --- \textit{базовой sc-дуги}), явно хранимой в рамках sc-памяти (ее наличие подразумевается, однако она не хранится явно, поскольку это приведет к бесконечному увеличению числа элементов, которые необходимо хранить в \textit{sc-памяти});
   			\item могут быть представлены в виде параметров соответствующих \textit{элементов sc-памяти}, то есть множеством таких элементов, каждый из которых имеет \scnqq{метку}, выраженную некоторым числовым значением;
   			\item могут уточнять \textit{класс элементов sc-памяти} с той степенью детализации, которая необходима чтобы, например, при совершении операции поиска с помощью таких \textit{классов элементов sc-памяти\scnsupergroupsign} можно было легко определить класс конкретного из них.
   		\end{itemize}
   	С этой целью, в \textit{SCin-коде} выделяется базовая \textbf{\textit{Cинтаксическая классификация элементов SCin-кода}}.
    }}
    
   	\scnstructheader{Синтаксическая классификация элементов SCin-кода}
   	\begin{scnsubstruct}
   		\scnheader{элемент sc-памяти}
   		\begin{scnrelfromset}{разбиение}
   			\scnitem{элемент sc-памяти, соответствующий sc-узлу}
   			\begin{scnindent}
   				\scniselement{синтаксический класс элемента sc-памяти\scnsupergroupsign}
   			\end{scnindent}
   			\scnitem{элемент sc-памяти, соответствующий sc-коннектору}
   			\begin{scnindent}
   				\scniselement{синтаксический класс элемента sc-памяти\scnsupergroupsign}
   			\end{scnindent}
   			\scnitem{элемент sc-памяти, имеющий нулевой sc-адрес}
   			\begin{scnindent}
   				\scniselement{синтаксический класс элемента sc-памяти\scnsupergroupsign}
   			\end{scnindent}
   		\end{scnrelfromset}
   	\end{scnsubstruct}
    \scnnote{Для того чтобы представлять и хранить любые \textit{sc-конструкции} достаточно иметь только два базовых \textit{класса элементов sc-памяти} (\textit{элемент sc-памяти, соответствующий sc-узлу}, и \textit{элемент sc-памяти, соответствующий sc-коннектору}), при этом остальные \textit{классы элементов sc-памяти} можно добавить в расширенной версии \textit{SCin-кода} и тем самым дореализовать необходимую логику на уровне конкретной реализации \textit{sc-памяти}.}
    \scnnote{Стоит отметить, что все \textbf{\textit{классы элементов sc-памяти\scnsupergroupsign}}, входящие в состав \textit{Cинтаксической классификации элементов SCin-кода}, являются синтаксически выделяемыми \textit{классами элементов SCin-кода}, то есть на уровне \textit{Реализации sc-памяти в ostis-платформе} такие программные модели этих элементов представляются по-разному.}
    
   	\scnheader{sc-адрес элемента sc-памяти*}
   	\begin{scnrelfromset}{разбиение}
   		\scnitem{sc-адрес элемента sc-памяти, соответствующего выходящему sc-коннектору из заданного sc-элемента*}
   		\scnitem{sc-адрес элемента sc-памяти, соответствующего входящему sc-коннектору в заданный sc-элемент*}
   		\scnitem{sc-адрес элемента sc-памяти, соответствующего инцидентному sc-элементу sc-коннектора*}
   	\end{scnrelfromset}
    \scnnote{Несмотря на то, что отношение \textit{sc-адрес элемента sc-памяти*} позволяет полностью описать связи \textit{элементов sc-памяти}, для спецификации представления конструкций SC-кода внутри sc-памяти только одного отношения \textit{sc-адрес элемента sc-памяти*} не всегда достаточно, чтобы полностью точно и ясно указывать связи между \textit{элементами sc-памяти}, соответствующими \textit{sc-элементам} этих конструкций. Поэтому на практике при описании представления \textit{sc-конструкций} внутри \textit{sc-памяти} необходимо использовать более частные отношения этого базового отношения, например, такие, как \textit{sc-адрес элемента sc-памяти, соответствующего выходящему sc-коннектору из заданного sc-элемента*}, \textit{sc-адрес элемента sc-памяти, соответствующего входящему sc-коннектору в заданный sc-элемент*} и \textit{sc-адрес элемента sc-памяти, соответствующего инцидентному sc-элементу sc-коннектора*}.}
    
   	\scnheader{sc-адрес элемента sc-памяти, соответствующего выходящему sc-коннектору из заданного sc-элемента*}
   	\begin{scnrelfromset}{разбиение}
   		\scnitem{sc-адрес элемента sc-памяти, соответствующего начальному выходящему sc-коннектору из заданного sc-элемента*}
   		\scnitem{sc-адрес элемента sc-памяти, соответствующего следующему выходящему sc-коннектору из заданного sc-элемента*}
   		\scnitem{sc-адрес элемента sc-памяти, соответствующего предыдущему выходящему sc-коннектору из заданного sc-элемента*}
   	\end{scnrelfromset}
    \scndefinition{Отношение \textbf{\textit{sc-адрес элемента sc-памяти, соответствующего выходящему sc-коннектору из заданного sc-элемента*}} определяется как \textit{бинарное ориентированное отношение}, первым компонентом каждой \textit{ориентированной пары} которого является некоторый \textit{элемент sc-памяти, соответствующий некоторому sc-элементу}, из которого выходит заданный \textit{sc-коннектор}, а вторым компонентом этой пары является \textit{sc-адрес элемента sc-памяти, соответствующего} этому выходящему \textit{sc-коннектору}. Частными видами этого отношения являются отношение \textit{sc-адрес элемента sc-памяти, соответствующего начальному выходящему sc-коннектору из заданного sc-элемента*}, отношение \textit{sc-адрес элемента sc-памяти, соответствующего следующему выходящему sc-коннектору из заданного sc-элемента*} и отношение \textit{sc-адрес элемента sc-памяти, соответствующего предыдущему выходящему sc-коннектору из заданного sc-элемента*}.}
 
   	\scnheader{sc-адрес элемента sc-памяти, соответствующего входящему sc-коннектору в заданный sc-элемент*}
   	\begin{scnrelfromset}{разбиение}
   		\scnitem{sc-адрес элемента sc-памяти, соответствующего начальному входящему sc-коннектору в заданный sc-элемент*}
   		\scnitem{sc-адрес элемента sc-памяти, соответствующего следующему входящему sc-коннектору в заданный sc-элемент*}
   		\scnitem{sc-адрес элемента sc-памяти, соответствующего предыдущему входящему sc-коннектору в заданный sc-элемент*}
   	\end{scnrelfromset}
    \scndefinition{Отношение \textbf{\textit{sc-адрес элемента sc-памяти, соответствующего входящему sc-коннектору в заданный sc-элемент*}} определяется как \textit{бинарное ориентированное отношение}, первым компонентом каждой \textit{ориентированной пары} которого является \textit{некоторый элемент sc-памяти, соответствующий некоторому sc-элементу}, в который входит заданный \textit{sc-коннектор}, а вторым компонентом этой пары является \textit{sc-адрес элемента sc-памяти, соответствующего} этому входящему {sc-коннектору}. Частными видами этого отношения являются отношение \textit{sc-адрес элемента sc-памяти, соответствующего начальному входящему sc-коннектору в заданный sc-элемент*}, отношение \textit{sc-адрес элемента sc-памяти, соответствующего следующему входящему sc-коннектору в заданный sc-элемент*} и отношение \textit{sc-адрес элемента sc-памяти, соответствующего предыдущему входящему sc-коннектору в заданный sc-элемент*}.}
    
   	\scnheader{sc-адрес элемента sc-памяти, соответствующего инцидентному sc-элементу sc-коннектора*}
   	\begin{scnrelfromset}{разбиение}
   		\scnitem{sc-адрес элемента sc-памяти, соответствующего начальному sc-элементу sc-коннектора*}
   		\scnitem{sc-адрес элемента sc-памяти, соответствующего конечному sc-элементу sc-коннектора*}
   	\end{scnrelfromset}
   	\scndefinition{Отношение \textbf{\textit{sc-адрес элемента sc-памяти, соответствующего инцидентному sc-элементу sc-дуги*}} определяется как \textit{бинарное ориентированное отношение}, первым компонентом каждой \textit{ориентированной пары} которого является некоторый \textit{элемент sc-памяти, соответствующий некоторому sc-коннектору}, а вторым компонентом является \textit{sc-адрес элемента sc-памяти, соответствующего некоторому} инцидентному этому sc-коннектору {sc-элементу}. Частными видами этого отношения являются отношение \textit{sc-адрес элемента sc-памяти, соответствующего начальному sc-элементу sc-коннектора*} и отношение \textit{sc-адрес элемента sc-памяти, соответствующего конечному sc-элементу sc-коннектора*}.}
    
    \scnheader{Синтаксис SCin-кода}
    \scntext{синтаксические правила}{На синтаксические \textit{конструкции SCin-кода}, кроме ограничения самого \textit{SC-кода}, накладываются дополнительные ограничения:
    	\begin{itemize}
    		\item Для каждого \textit{элемента sc-памяти} взаимно однозначно ставится в соответствие \textit{sc-адрес} этого \textit{элемента sc-памяти}.
    		\item Для каждого \textit{элемента sc-памяти, соответствующего sc-узлу}, существует одна и только одна пара отношения \textit{sc-адрес элемента sc-памяти, соответствующего начальному выходящему sc-коннектору из заданного sc-элемента*} и одна и только одна пара отношения \textit{sc-адрес элемента sc-памяти, соответствующего начальному входящему sc-коннектору в заданный sc-элемент*}.
    		\item Для каждого \textit{элемента sc-памяти, соответствующего выходящему sc-коннектору из заданного sc-элемента} (\textit{элемента sc-памяти, соответствующего входящему sc-коннектору в заданный sc-элемент}), существует не более чем одна пара отношения \textit{sc-адрес элемента sc-памяти, соответствующего следующему выходящему sc-коннектору из заданного sc-элемента*} (\textit{sc-адрес элемента sc-памяти, соответствующего следующему входящему sc-коннектору в заданный sc-элемент*}) и не более чем одна пара отношения \textit{sc-адрес элемента sc-памяти, соответствующего предыдущему выходящему sc-коннектору из заданного sc-элемента*} (\textit{sc-адрес элемента sc-памяти, соответствующего предыдущему входящему sc-коннектору в заданный sc-элемент*}).
    		\item Для каждого \textit{элемента sc-памяти, соответствующего sc-коннектору}, который является вторым компонентом каждой пары отношения \textit{sc-адрес элемента sc-памяти, соответствующего начальному выходящему sc-коннектору из заданного sc-элемента*} (\textit{sc-адрес элемента sc-памяти, соответствующего начальному входящему sc-коннектору в заданный sc-элемент*}) существует только и только одна пара отношения \textit{sc-адрес элемента sc-памяти, соответствующего следующему выходящему sc-коннектору из заданного sc-элемента*} (\textit{sc-адрес элемента sc-памяти, соответствующего следующему входящему sc-коннектору в заданный sc-элемент*}).
    	\end{itemize}
    }
   	\end{scnsubstruct}
    
    \scnstructheader{Денотационная семантика SCin-кода}
    \begin{scnsubstruct}
    \scnnote{Для каждого \textit{класса sc-элементов} должна существовать программная модель \textit{класса элементов sc-памяти\scnsupergroupsign}, которая удовлетворяет всем перечисленным требованиям. Поэтому важно, чтобы \textit{Алфавит SCin-кода} изначально был полон, чтобы погрузить не только \textit{sc-конструкции} \textit{Ядра SC-кода}, но и его расширенных версий. Для этого разработаны \textbf{\textit{семантические классы элементов sc-памяти\scnsupergroupsign}}, спецификация которых представляется в виде \textbf{\textit{Семантической классификации элементов SCin-кода}}.}
    
   	\scnstructheader{Семантическая классификация элементов SCin-кода}
   	\begin{scnsubstruct}
   		
		\scnheader{элемент sc-памяти}
		\scnrelfrom{разбиение}{Типология элементов sc-памяти по признаку константности\scnsupergroupsign}
		\begin{scnindent}
			\begin{scneqtoset}
				\scnitem{элемент sc-памяти, соответствующий sc-константе}
				\begin{scnindent}
					\scniselement{семантический класс элемента sc-памяти\scnsupergroupsign}
				\end{scnindent}
				\scnitem{элемент sc-памяти, соответствующий sc-переменной}
				\begin{scnindent}
					\scniselement{семантический класс элемента sc-памяти\scnsupergroupsign}
				\end{scnindent}
				\scnitem{элемент sc-памяти, соответствующий sc-метапеременной}
				\begin{scnindent}
					\scniselement{семантический класс элемента sc-памяти\scnsupergroupsign}
				\end{scnindent}
			\end{scneqtoset}
		\end{scnindent}
		\scnrelfrom{разбиение}{Типология элементов sc-памяти по признаку постоянности\scnsupergroupsign}
		\begin{scnindent}
			\begin{scneqtoset}
				\scnitem{элемент sc-памяти, соответствующий}
				\begin{scnindent}
					\scniselement{семантический класс элемента sc-памяти\scnsupergroupsign}
				\end{scnindent}
				\scnitem{элемент sc-памяти, соответствующий временному sc-элементу}
				\begin{scnindent}
					\scniselement{семантический класс элемента sc-памяти\scnsupergroupsign}
				\end{scnindent}
			\end{scneqtoset}
		\end{scnindent}
		\scnrelfrom{разбиение}{Типология элементов sc-памяти по признаку доступности\scnsupergroupsign}
		\begin{scnindent}
			\scnidtf{класс уровня доступа к элементу sc-памяти}
			\begin{scneqtoset}
				\scnitem{элемент sc-памяти, соответствующий sc-элементу, на котором разрешено право чтения}
				\begin{scnindent}
					\scniselement{семантический класс элемента sc-памяти\scnsupergroupsign}
				\end{scnindent}
				\scnitem{элемент sc-памяти, соответствующий sc-элементу, на котором разрешено право записи}
				\begin{scnindent}
					\scniselement{семантический класс элемента sc-памяти\scnsupergroupsign}
				\end{scnindent}
			\end{scneqtoset}
		\end{scnindent}
		\scnrelfrom{включение}{элемент sc-памяти, соответствующий внутреннему файлу ostis-системы}
		
		\scnheader{элемент sc-памяти, соответствующий sc-узлу общего вида}
		\scnrelfrom{разбиение}{Структурная типология элементов sc-памяти, соответствующих sc-узлам\scnsupergroupsign}
		\begin{scnindent}
			\begin{scneqtoset}
				\scnitem{элемент sc-памяти, соответствующий sc-узлу, обозначающему небинарную sc-связку}
				\begin{scnindent}
					\scniselement{семантический класс элемента sc-памяти\scnsupergroupsign}
				\end{scnindent}
				\scnitem{элемент sc-памяти, соответствующий sc-классу}
				\begin{scnindent}
					\scniselement{семантический класс элемента sc-памяти\scnsupergroupsign}
				\end{scnindent}
				\scnitem{элемент sc-памяти, соответствующий sc-узлу, обозначающему класс классов}
				\begin{scnindent}
					\scniselement{семантический класс элемента sc-памяти\scnsupergroupsign}
				\end{scnindent}
				\scnitem{элемент sc-памяти, соответствующий sc-структуре}
				\begin{scnindent}
					\scniselement{семантический класс элемента sc-памяти\scnsupergroupsign}
				\end{scnindent}
				\scnitem{элемент sc-памяти, соответствующий sc-узлу, обозначающему ролевое отношение}
				\begin{scnindent}
					\scniselement{семантический класс элемента sc-памяти\scnsupergroupsign}
				\end{scnindent}
				\scnitem{элемент sc-памяти, соответствующий sc-узлу, обозначающему неролевое отношение}
				\begin{scnindent}
					\scniselement{семантический класс элемента sc-памяти\scnsupergroupsign}
				\end{scnindent}
				\scnitem{элемент sc-памяти, соответствующий sc-узлу, обозначающему первичную сущность}
				\begin{scnindent}
					\scniselement{семантический класс элемента sc-памяти\scnsupergroupsign}
				\end{scnindent}
			\end{scneqtoset}
		\end{scnindent}
		\scnrelfrom{разбиение}{Структурная типология элементов sc-памяти, соответствующих sc-дугам\scnsupergroupsign}
		\begin{scnindent}
			\begin{scneqtoset}
				\scnitem{элемент sc-памяти, соответствующий sc-дуге принадлежности}
				\begin{scnindent}
					\scniselement{семантический класс элемента sc-памяти\scnsupergroupsign}
				\end{scnindent}
				\scnitem{элемент sc-памяти, соответствующий sc-дуге общего вида}
				\begin{scnindent}
					\scniselement{семантический класс элемента sc-памяти\scnsupergroupsign}
				\end{scnindent}
			\end{scneqtoset}
		\end{scnindent}
		\scnrelfrom{разбиение}{Типология элементов sc-памяти, соответствующих sc-дугам принадлежности, по типу обозначаемой принадлежности\scnsupergroupsign}
		\begin{scnindent}
			\begin{scneqtoset}
				\scnitem{элемент sc-памяти, соответствующий sc-дуге позитивной принадлежности}
				\begin{scnindent}
					\scniselement{семантический класс элемента sc-памяти\scnsupergroupsign}
				\end{scnindent}
				\scnitem{элемент sc-памяти, соответствующий sc-дуге нечеткой принадлежности}
				\begin{scnindent}
					\scniselement{семантический класс элемента sc-памяти\scnsupergroupsign}
				\end{scnindent}
				\scnitem{элемент sc-памяти, соответствующий sc-дуге негативной принадлежности}
				\begin{scnindent}
					\scniselement{семантический класс элемента sc-памяти\scnsupergroupsign}
				\end{scnindent}
			\end{scneqtoset}
		\end{scnindent}
   		
   	\end{scnsubstruct}
    \scnexplanation{Все семантически и синтаксически выделяемые \textit{классы элементов sc-памяти\scnsupergroupsign}, а также всевозможные подклассы этих классов являются экземплярами (элементами) sc-класса. Все перечисленные классы на уровне программной \textit{Реализации sc-памяти в ostis-платформе} выражаются в виде битов в битовой строке, описываемой в каждом \textit{элементе sc-памяти}.}
    \scnnote{В текущий момент \textit{sc-ребра} хранятся так же, как \textit{sc-дуги}, то есть имеют начальный и конечный \textit{sc-элементы}, отличие заключается только в \textit{семантическом классе элемента sc-памяти\scnsupergroupsign}. Это приводит к ряду неудобств при обработке, но \textit{sc-ребра} используются в настоящее время достаточно редко.}
    \scnnote{\textbf{\textit{Спецификация SCin-кода}} является объединением спецификацией его \textit{элементов sc-памяти}.}
    
   	\scnheader{элемент sc-памяти}
   	\scnrelfrom{понятие, специфицирующее заданную сущность}{\scnnonamednode}
   	\begin{scnindent}
   		\scnsuperset{сужение отношения по первому домену(спецификация знака*, элемент sc-памяти, соответствующий sc-узлу)*}
   		\scnsuperset{сужение отношения по первому домену(понятие, специфицирующее заданную сущность знака*, элемент sc-памяти, соответствующий sc-коннектору)*}
   	\end{scnindent}
    \scnnote{Каждый \textit{элемент sc-памяти} описывается его синтаксическим типом (меткой), а также независимо от \textit{класса элемента sc-памяти\scnsupergroupsign} указывается \textit{sc-адрес первого входящего в данный sc-элемент sc-коннектора} и \textit{sc-адрес первого выходящего из данного sc-элемента sc-коннектора} (могут быть пустыми, если таких sc-коннекторов нет). Оставшиеся байты в зависимости от \textit{класса элемента sc-памяти} (sc-узел или sc-коннектор) могут использоваться для хранения спецификации \textit{элемента sc-памяти, соответствующего sc-коннектору}. Также \textit{sc-адрес первой sc-дуги, выходящей из данного sc-элемента*} и \textit{sc-адрес первой sc-дуги, входящей в данный sc-элемент*} в общем случае могут отсутствовать (быть нулевыми, \scnqq{пустыми}), но размер \textit{элемента sc-памяти} в байтах останется тем же.}
    
   	\scnstructheader{Спецификация элемента sc-памяти, соответствующего sc-узлу}
   	\begin{scnsubstruct}
   		
   		\scnheader{элемент sc-памяти, соответствующий sc-узлу}
   		\begin{scnrelfromset}{понятие, специфицирующее заданную сущность}
   			\scnitem{класс элемента sc-памяти\scnsupergroupsign}
   			\scnitem{класс уровня доступа к элементу sc-памяти\scnsupergroupsign}
   			\scnitem{sc-адрес элемента sc-памяти*}
   			\scnitem{sc-адрес первого sc-коннектора, выходящего из данного sc-элемента*}
   			\scnitem{sc-адрес первого sc-коннектора, входящего в данный sc-элемент*}
   		\end{scnrelfromset}
   		
   	\end{scnsubstruct}
   	
   	\scnstructheader{Спецификация элемента sc-памяти, соответствующего sc-коннектору}
   	\begin{scnsubstruct}
   		
   		\scnheader{элемент sc-памяти, соответствующий sc-коннектору}
   		\begin{scnrelfromset}{понятие, специфицирующее заданную сущность}
   			\scnitem{класс элемента sc-памяти\scnsupergroupsign}
   			\scnitem{класс уровня доступа к элементу sc-памяти\scnsupergroupsign}
   			\scnitem{sc-адрес элемента sc-памяти*}
   			\scnitem{sc-адрес элемента sc-памяти, соответствующего начальному sc-элементу sc-коннектора*}
   			\scnitem{sc-адрес элемента sc-памяти, соответствующего конечному sc-элементу sc-коннектора*}
   			\scnitem{sc-адрес элемента sc-памяти, соответствующего начальному выходящему sc-коннектору из заданного sc-элемента*}
   			\scnitem{sc-адрес элемента sc-памяти, соответствующего начальному входящему sc-коннектору в заданный sc-элемент*}
   			\scnitem{sc-адрес элемента sc-памяти, соответствующего следующему выходящему sc-коннектору из заданного sc-элемента*}
   			\scnitem{sc-адрес элемента sc-памяти, соответствующего следующему входящему sc-коннектору в заданный sc-элемент*}
   			\scnitem{sc-адрес элемента sc-памяти, соответствующего предыдущему выходящему sc-коннектору из заданного sc-элемента*}
   			\scnitem{sc-адрес элемента sc-памяти, соответствующего предыдущему входящему sc-коннектору в заданный sc-элемент*}
   		\end{scnrelfromset}
   		
   	\end{scnsubstruct}
    \scnnote{В текущей \textit{Реализации sc-памяти в ostis-платформе} \textbf{\textit{классы уровня доступа\scnsupergroupsign}} используются для того, чтобы обеспечить возможность ограничения доступа некоторых \textit{процессов в sc-памяти} к некоторым \textit{элементам sc-памяти}. Каждый \textit{элемент sc-памяти} принадлежит одному из двух классов: классу \textit{элементов sc-памяти, соответствующих sc-элементам, на которых разрешено право чтения} и классу \textit{элементов sc-памяти, соответствующих sc-элементам, на которых разрешено право записи}, каждый из которых выражается \textit{целым числом} от 0 до 255.
    	
   	Таким образом нулевое значение числовых выражений класса \textit{элементов sc-памяти, соответствующих sc-элементам, на которых разрешено право чтения} и класса \textit{элементов sc-памяти, соответствующих sc-элементам, на которых разрешено право записи} означает, что любой процесс может получить неограниченный доступ к данному \textit{элементу sc-памяти}.
   	
   	В качестве примера на рисунке \nameref{fig:sc_code_in_memory_representation} представлены пятиэлементная sc-конструкция (слева) и конструкция в sc-памяти, представленная на SCin-коде (справа).
    	
    	%\begin{figure*}[htbp]
    	%	\caption{SCg-текст. Пример трансляции sc-текста в sc-память ostis-платформы}
    	%	\includegraphics[scale=0.65]{author/part6/figures/sc_code_in_memory_representation.png}
    	%	\label{fig:sc_code_in_memory_representation}
    	%\end{figure*}
	}
    
    \end{scnsubstruct}
    
    \scnheader{SCin-код}
    \scntext{достоинства}{Описанная модель представления в текущей \textit{Реализации sc-памяти в ostis-платформы} \textit{синтаксических} и \textit{семантических классов sc-элементов} в виде \textit{синтаксических} и \textit{семантических классов элементов sc-памяти\scnsupergroupsign}, которые соответствуют первым, обладает рядом преимуществ:
    	\begin{itemize}
    		\item \textit{cинтаксические} и {семантические классы элементов sc-памяти\scnsupergroupsign} могут \uline{комбинироваться} между собой для получения более частных классов. С точки зрения программной реализации такая комбинация может быть представлена операцией \textit{битовое сложение*} \textit{классов элементов sc-памяти\scnsupergroupsign} (здесь, в спецификации на \textit{SC-коде} это можно сделать с помощью пересечения соответствующих классов). Так, например, \textit{битовое сложение*} классов \textit{элементов sc-памяти, соответствующих sc-узл}у и \textit{sc-константе} в результате образуют новый \textit{класс элементов sc-памяти\scnsupergroupsign} --- \textit{элемент sc-памяти, соответствующий константному sc-узлу}.
    		\item Числовые выражения некоторых классов могут совпадать. Это сделано для уменьшения размера \textit{элемента sc-памяти} за счет уменьшения максимального размера числового выражения класса этих \textit{элементов sc-памяти}. Конфликт в данном случае не возникает, поскольку такие классы не могут комбинироваться, например \textit{элемент sc-памяти, соответствующий sc-узлу ролевого отношения} и \textit{элемент sc-памяти, соответствующий sc-дуге нечеткой принадлежности}.
    		\item Важно отметить, что каждому из выделенных \textit{классов элементов sc-памяти} (кроме классов, получаемых путем комбинации других классов) однозначно соответствует порядковый номер бита в линейной памяти, что можно заметить, глядя на соответствующие числовые выражения этих классов. Это означает, что классы элементов не включаются друг в друга (хоть в спецификации это и не так), например, указание принадлежности к классу \textit{элементов sc-памяти, соответствующих sc-дуге позитивной принадлежности} не означает автоматическое указание принадлежности \textit{элементов sc-памяти, соответствующих sc-дуге принадлежности}. На уровне реализации это позволяет сделать операции комбинирования и сравнения меток более эффективными.
    	\end{itemize}
	}
    
    \scnheader{Реализация sc-памяти}
    \scntext{достоинства}{\textit{Реализация sc-памяти} учитывает технические аспекты реализации современных \textit{операционных систем}, что дает следующие достоинства:
    \begin{itemize}
    	\item Используемый подход адресации \textit{sc-элементов} позволяет загружать сегменты с диска в память, а также выгружать их обратно на диск в любой момент и при этом данная операция не требует дополнительных преобразований. Все содержимое из \textit{оперативной памяти} без изменений попадает на диск. Это дает возможность выгружать неиспользуемые сегменты на диск, что позволяет \textit{sc-памяти} абстрагироваться от имеющихся ресурсов \textit{оперативной памяти} и работать на любых ее объемах;
    	\item Максимальное количество хранимых \textit{sc-элементов} можно увеличивать путем расширения \textit{sc-адреса элемента sc-памяти}.
    \end{itemize}
    
    С точки зрения программной \textit{Реализации sc-памяти в ostis-платформе}, структура данных для хранения \textit{sc-узла} и \textit{sc-коннектора} остается остается та же, но в ней меняется список полей (компонентов). Кроме того, как можно заметить каждый \textit{элемент sc-памяти} (в том числе, \textit{элемент sc-памяти, соответствующий sc-дуге}) не хранит список \textit{sc-адресов} связанных с ним \textit{элементов sc-памяти}, а хранит \textit{sc-адреса} одного \textit{выходящего} и одного \textit{входящего элементов sc-памяти, соответствующих sc-коннекторам}, каждый из которых в свою очередь хранит \textit{sc-адреса следующего} и \textit{предыдущего элементов, соответствующих sc-коннекторам}, в списке выходящих и входящих \textit{элементов sc-памяти.} Все перечисленное позволяет:
    \begin{itemize}
    	\item сделать размер такой структуры фиксированным (в настоящее время 36 байт) и не зависящим от \textit{синтаксического класса} хранимого \textit{элемента в sc-памяти\scnsupergroupsign};
    	\item с минимальными временными затратами добавлять и удалять инцидентные элементы в и из программной структуры \textit{элемента sc-памяти} соответственно;
    	\item обеспечить возможность работы с sc-элементами без учета их синтаксического класса в случаях, когда это необходимо (например, при реализации поисковых запросов вида \scnqqi{Какие sc-элементы являются элементами данного множества}, \scnqqi{Какие sc-элементы непосредственно связаны с данным sc-элементом} и так далее);
    	\item обеспечить возможность доступа к \textit{элементу sc-памяти} за константное время;
    	\item обеспечить возможность помещения \textit{элемента sc-памяти} в процессорный кэш, что в свою очередь, позволяет ускорить обработку \textit{sc-конструкций}.
    \end{itemize}
	}
	\end{scnsubstruct}

    \scnheader{контекст процесса в рамках программной модели sc-памяти}
    \scnidtf{ScContext}
    \scnidtf{контекст процесса, выполняемого на уровне программной модели sc-памяти}
    \scnidtf{метаописание процесса в sc-памяти, выполняемого на уровне программной модели sc-памяти}
    \scnidtf{структура данных, содержащая метаинформацию о процессе, выполняемом в sc-памяти на уровне платформы}
    \scnrelto{класс компонентов}{Реализация sc-хранилища}
    \scntext{пояснение}{Каждому процессу, выполняемому в sc-памяти на уровне \textit{платформы интерпретации sc-моделей компьютерных систем} (и чаще всего соответствующего некоторому \textit{sc-агенту}, реализованному на уровне платформы) ставится в соответствие \textit{контекст процесса}, который является структурой данных, описывающей метаинформацию о данном процессе. На текущий момент контекст процесса содержит сведения об уровне доступа на чтение и запись для данного процесса (См. \textit{метка уровня доступа sc-элемента}).При вызове в рамках процесса любых функций (методов), связанных с доступом к хранимым в sc-памяти конструкциям одним из параметров обязательно является \textit{контекст процесса}.}
    \scnheader{блокировка sc-элемента в рамках программной модели sc-памяти}
    \scnidtf{ScLock}
    \scnrelto{класс компонентов}{Реализация sc-хранилища}
    \scnheader{подписка на событие в sc-памяти в рамках программной модели sc-памяти}
    \scnidtf{ScEvent}
    \scnidtf{структура данных, описывающая в рамках программной модели sc-памяти соответствие между классом событий в sc-памяти и действиями, которые должно быть совершены при возникновении в sc-памяти событий данного класса}
    \scnrelto{класс компонентов}{Реализация sc-хранилища}
    \scntext{пояснение}{Для того, чтобы обеспечить возможность создания sc-агентов в рамках \textit{платформы интерпретации sc-моделей компьютерных систем} реализована возможность создать подписку на событие, принадлежащее одному из классов \textit{элементарных событий в sc-памяти*} (см. Раздел \scnqqi{\textit{Предметная область и онтология темпоральных сущностей базы знаний ostis-системы}}), уточнив при этом sc-элемент, с которым должно быть связано событие данного класса (например, sc-элемент, для которого должна появиться входящая или исходящая sc-дуга). Подписка на событие представляет собой структуру данных, описывающую класс ожидаемых событий и функцию в программном коде, которая должна быть вызвана при возникновении данного события.Все подписки на события регистрируются в рамках таблицы событий. При любом изменении в sc-памяти происходит просмотр данной таблицы и запуск функций, соответствующих произошедшему событию.В текущей реализации обработка каждого события осуществляется в отдельном потоке операционной системы, при этом на уровне реализации задается параметр, описывающий число максимальных потоков, которые могут выполняться параллельно.Таким образом оказывается возможным реализовать sc-агенты, реагирующие на события в sc-памяти, а также при выполнении некоторого процесса в sc-памяти приостановить его работу и дождаться возникновения некоторого события (например, создать подзадачу некоторому коллективу sc-агентов и дождаться ее решения).}
    \scnheader{sc-итератор}
    \scnidtf{ScIterator}
    \scnrelto{класс компонентов}{Реализация sc-хранилища}
    \scntext{пояснение}{С функциональной точки зрения \textit{sc-итераторы} как часть \textit{Реализации sc-хранилища} представляют собой базовое средство доступа к конструкциям, хранимым в sc-памяти, которое позволяет осуществить чтение (просмотр) конструкций, изоморфных простейшим шаблонам --- \textit{трехэлементным sc-конструкциям} и \textit{пятиэлементным sc-конструкциям}.С точки зрения реализации \textit{sc-итератор} представляет собой структуру данных, которая соответствует определенному дополнительно уточняемому классу sc-конструкций и позволяет при помощи соответствующего набора функций последовательно осуществлять просмотр всех sc-конструкций данного класса, представленных в текущем состоянии sc-памяти (итерацию по sc-конструкциям).Каждому классу \textit{sc-итераторов} соответствует некоторый известный класс (шаблон, образец) \mbox{sc-конструкций}. При создании sc-итератора данный шаблон уточняется, то есть некоторым (как минимум одному) элементам шаблона ставится в соответствие конкретный заранее известный \textit{sc-элемент} (отправная точка при поиске), а другим элементам шаблона (тем, которые нужно найти) ставится в соответствие некоторый тип sc-элемента из числа типов, соответствующих \textit{меткам синтаксического типа sc-элемента}.Далее путем вызова соответствующей функции (или метода класса в ООП) осуществляется последовательный просмотр всех sc-конструкций, соответствующих полученному шаблону (с учетом указанных типов sc-элементов и заранее заданных известных sc-элементов), то есть \textit{sc-итератор} последовательно \scnqq{переключается} с одной конструкции на другую до тех пор, пока такие конструкции существуют. Проверка существования следующей конструкции проверяется непосредственно перед переключением. В общем случае конструкций, соответствующих указанному шаблону, может не существовать, в этом случае итерирование происходить не будет (будет 0 итераций).На каждой итерации в sc-итератор записываются sc-адреса sc-элементов, входящих в соответствующую sc-конструкцию, таким образом найденные элементы могут быть обработаны нужным образом в зависимости от задачи.}
    \scnsuperset{трехэлементный sc-итератор}
    \begin{scnindent}
        \scnrelfrom{класс sc-конструкций}{трехэлементная sc-конструкция}
    \end{scnindent}
    \scnsuperset{пятиэлементный sc-итератор}
    \begin{scnindent}
        \scnrelfrom{класс sc-конструкций}{пятиэлементная sc-конструкция}
        \scntext{примечание}{В настоящее время \textit{пятиэлементный sc-итератор} реализуется на основе \textit{трехэлементных sc-итераторов} и в этом смысле не является атомарным. Однако, введение \textit{пятиэлементных sc-итераторов} целесообразно с точки зрения удобства разработчика программ обработки \mbox{sc-конструкций}.}
    \end{scnindent}
    \scnheader{sc-шаблон}
    \scnidtf{ScTemplate}
    \scnidtf{структура данных в линейной памяти, описывающая обобщенную sc-структуру, которая в свою очередь может быть либо явно представлена sc-памяти, либо не представлена в ее текущем состоянии, но может быть представлена при необходимости}
    \scnrelto{класс компонентов}{Реализация sc-хранилища}
    \scntext{пояснение}{\textit{Sc-итераторы} позволяют осуществлять поиск только sc-конструкций простейшей конфигурации. Для реализации поиска sc-конструкций более сложной конфигурации, а также генерации сложных sc-конструкций используются \textit{sc-шаблоны}, на основе которых затем осуществляется поиск или генерация конструкций. \textit{Sc-шаблон} представляет собой структуру данных, соответствующую некоторой \textit{обобщенной структуре}, т.е. \textit{структуре}, содержащей \textit{sc-переменные}. При помощи соответствующего набора функций можно осуществлять
        \begin{scnitemize}
            \item поиск в текущем состоянии sc-памяти всех sc-конструкций, изоморфных заданному шаблону. В качестве параметров поиска можно указать значения для каких-либо из sc-переменных в составе шаблона. После осуществления поиска будет сформировано множество результатов поиска, каждый из которых представляет собой множество пар вида \scnqqi{sc-переменная из шаблона --- соответствующая ей sc-константа}. Данное множество может быть пустым (в текущем состоянии sc-памяти нет конструкций, изоморфных заданному образцу) или содержать один или более элементов. Подстановка значений sc-переменных может осуществляться как по sc-адресу, так и по системному sc-идентификатору;
            \item генерацию sc-конструкции, изоморфной заданному шаблону. Параметры и результаты генерации формируются так же, как в случае поиска, за исключением того, что в случае генерации результат всегда один и множество результатов не формируется.
        \end{scnitemize}
        Таким образом, каждый \textit{sc-шаблон} фактически задает множество шаблонов, формируемых путем указания значений для sc-переменных, входящих в исходный шаблон.Важно отметить, что \textit{sc-шаблон} представляет собой структуру данных в линейной памяти, соответствующую некоторой \textit{обобщенной структуре} в sc-памяти, но не саму эту \textit{обобщенную структуру}. Это означает, что sc-шаблон может быть автоматически сформирован на основе \textit{обобщенной структуры}, явно представленной в sc-памяти, а также сформирован на уровне программного кода путем вызова соответствующих функций (методов). Во втором случае \textit{sc-шаблон} будет существовать только в линейной памяти и соответствующая \textit{обобщенная структура} не будет явно представлена в sc-памяти. В этом случае подстановка значений sc-переменных будет возможна только по системному sc-идентификатору, поскольку sc-адресов у соответствующих элементов шаблона существовать не будет.}
    \scntext{примечание}{При поиске sc-конструкций, изоморфных заданному шаблону, крайне важно с точки зрения производительности с какого sc-элемента начинать поиск. Как известно, в общем случае задача поиска в графе представляет собой NP-полную задачу, однако поиск в sc-графе позволяет учитывать семантику обрабатываемой информации, что, в свою очередь, позволяет существенно снизить время поиска.Одним из возможных вариантов оптимизации алгоритма поиска, реализованным на данный момент, является упорядочение трехэлементных sc-конструкций, входящих в состав sc-шаблона, по очередности поиска по этим sc-конструкциям по критерию снижения числа возможных вариантов поиска, которые порождает та или иная трехэлементная sc-конструкция, содержащая sc-переменные. Так, в первую очередь при поиске выбираются те трехэлементные sc-конструкции, которые изначально содержат две sc-константы, затем те, которые изначально содержат одну sc-константу. После выполнения шага поиска приоритет sc-конструкций изменяется с учетом результатов, полученных на предыдущем шаге.Другой вариант оптимизации основывается на той особенности формализации в SC-коде, что в общем случае число sc-дуг, входящих в некоторый sc-элемент, как правило значительно меньше числа выходящих из него sc-дуг. Таким образом, целесообразным оказывается осуществлять поиск вначале по входящим sc-дугам.}
    \scntext{примечание}{Можно предположить, что возможности, предоставляемые \textit{sc-шаблонами} позволяют полностью исключить использование \textit{sc-итераторов}. Однако это не совсем так по следующим причинам:
        \begin{scnitemize}
            \item функции поиска и генерации по шаблону реализуются на основе sc-итераторов, как базового средства поиска sc-конструкций в рамках \textit{Реализации sc-хранилища}.
            \item \textit{sc-итераторы} дают возможность более гибко организовать процесс поиска с учетом семантики конкретных sc-элементов, участвующих в поиске. Так например, можно учесть тот факт, что для некоторых sc-элементов число входящих sc-дуг значительно меньше, чем выходящих (или наоборот) таким образом, при поиске конструкций, содержащих такие sc-элементы более эффективно начать перебор с тех участков, где дуг потенциально меньше.
        \end{scnitemize}
    }
	\end{scnsubstruct}

    \scnsegmentheader{Описание реализации подсистемы взаимодействия с внешней средой с использованием сетевых языков}
    \begin{scnsubstruct}
        \scnheader{Реализация подсистемы взаимодействия с внешней средой с использованием сетевых языков}
        \begin{scnrelfromlist}{компонент программной системы}
            \scnitem{Реализация подсистемы взаимодействия с внешней средой с использованием сетевых языков на основе языка JSON}
        \end{scnrelfromlist}
        \scntext{пояснение}{Взаимодействие программной модели sc-памяти с внешними ресурсами может осуществляться посредством специализированного программного интерфейса (API), однако этот вариант неудобен в большинстве случае, поскольку:
            \begin{scnitemize}
                \item поддерживается только для очень ограниченного набора языков программирования (С, С++);
                \item требует того, чтобы клиентское приложение, обращающееся к программной модели sc-памяти, фактически составляло с ней единое целое, таким образом исключается возможность построения распределенного коллектива ostis-систем;
                \item как следствие предыдущего пункта, исключается возможность параллельной работы с sc-памятью нескольких клиентских приложений.
            \end{scnitemize}
            Для того, чтобы обеспечить возможность удаленного доступа к sc-памяти не учитывая при этом языки программирования, с помощью которых реализовано конкретное клиентское приложение, было принято решение о реализации возможности доступа к sc-памяти с использованием универсального языка, не зависящего от средств реализации того или иного компонента или системы. В качестве такого языка был разработан строковый язык на базе языка JSON.}
        \scnstructheader{Описание подсистемы взаимодействия c sc-памятью на основе языка JSON}
        \begin{scnsubstruct}
            \scnheader{Реализация подсистемы взаимодействия c sc-памятью на основе языка JSON}
            \scntext{пояснение}{Реализация подсистемы взаимодействия c sc-памятью на основе языка JSON позволяет ostis-системам взаимодействовать с системами из внешней среды на основе общепринятого транспортного формата передачи данных JSON и предоставляет API для доступа к sc-памяти платформы интерпретации sc-моделей.}
            \begin{scnrelfromlist}{используемый язык программирования}
                \scnitem{C}
                \scnitem{C++}
                \scnitem{Python}
                \scnitem{TypeScript}
                \scnitem{C\#}
                \scnitem{Java}
            \end{scnrelfromlist}
            \begin{scnrelfromlist}{используемый язык}
                \scnitem{SC-JSON-код}
            \end{scnrelfromlist}
            \scnrelfrom{архитектура}{Клиент-серверная архитектура}
            \scnrelto{реализация}{Подсистема взаимодействия с sc-памятью на основе языка JSON}
            \begin{scnindent}
                \scnidtf{Подсистема взаимодействия с sc-памятью на основе формата JSON}
                \scnidtf{Подсистема взаимодействия с sc-памятью на основе транспортного формата передачи данных JSON}
                \scniselement{многократно используемый компонент ostis-систем}
                \scniselement{неатомарный многократно используемый компонент ostis-систем}
                \scniselement{зависимый многократно используемый компонент ostis-систем}
                \begin{scnrelfromlist}{автор}
                    \scnitem{Корончик Д. Н.}
                    \scnitem{Шункевич Д. В.}
                    \scnitem{Зотов Н. В.}
                    \scnitem{Загорский А. Г.}
                \end{scnrelfromlist}
                \scntext{пояснение}{Взаимодействие c sc-памятью обеспечивается с помощью передачи информации на \textit{\textbf{SC-JSON-коде}} и ведётся, с одной стороны, между сервером, являющегося частью ostis-системы, написанным на том же языке реализации этой ostis-системы и имеющим доступ к её sc-памяти, и с другой стороны множеством клиентом, которым известно о наличии сервера в пределах сети их использования.}
                \scntext{примечание}{Осмысленные фрагменты текстов \textit{\textbf{SC-JSON-кода}} представляют семантически корректную структуру сущностей и связей между ними.}
                \scntext{примечание}{С помощью подсистемы взаимодействия с sc-памятью на основе языка JSON можно взаимодействовать с ostis-системой на таком же множестве возможных операций, как и в случае, если бы взаимодействие происходило (непосредственно) напрямую, на том же языке реализации платформы интерпретации sc-моделей компьютерных систем. При этом результат работы отличается только скоростью обработки информации.}
                \begin{scnrelfromset}{декомпозиция программной системы}
                    \scnitem{Серверная система на основе Websocket, обеспечивающая доступ к sc-памяти платформы интерпретации sc-моделей при помощи команд SC-JSON-кода}
                    \scnitem{Множество клиентских систем, подключаемых и взаимодействующих с \textit{Серверной системой на основе Websocket, обеспечивающей доступ к sc-памяти платформы интерпретации sc-моделей при помощи команд SC-JSON-кода}}
                    \begin{scnindent}
                        \begin{scnrelfromset}{декомпозиция программной системы}
                            \scnitem{Клиентская система, подключаемая и взаимодействующая с \textit{SC-сервером}, реализованная на языке программирования Python}
                            \scnitem{Клиентская система, подключаемая и взаимодействующая с \textit{SC-сервером}, реализованная на языке программирования TypeScript}
                            \scnitem{Клиентская система, подключаемая и взаимодействующая с \textit{SC-сервером}, реализованная на языке программирования C\#}
                            \scnitem{Клиентская система, подключаемая и взаимодействующая с \textit{SC-сервером}, реализованная на языке программирования Java}
                        \end{scnrelfromset}
                    \end{scnindent}
                \end{scnrelfromset}
            \end{scnindent}
            \scnheader{SC-JSON-код}
            \scnidtf{Semantic JSON-code}
            \scnidtf{Semantic JavaScript Object Notation code}
            \scnidtf{Язык внешнего смыслового представления знаний для взаимодействия с ostis-системами на основе языка JSON}
            \scnidtf{Метаязык, являющийся подмножеством языка JSON и обеспечивающий внешнее представление и структуризацию \textit{sc-текстов}, используемых ostis-системой в процессе своего функционирования и взаимодействия со внешней средой.}
            \scntext{часто используемый неосновной внешний идентификатор sc-элемента}{sc-json-текст}
            \begin{scnindent}
                \scniselement{имя нарицательное}
            \end{scnindent}
            \scniselement{абстрактный язык}
            \scniselement{линейный язык}
            \scnsubset{JSON}
            \begin{scnrelfromlist}{автор}
                \scnitem{Зотов Н. В.}
                \scnitem{Корончик Д. Н.}
            \end{scnrelfromlist}
            \begin{scnrelfromvector}{принципы, лежащие в основе}
                \scnitem{Тексты, описываемые на языке внешнего представления знаний \textit{\textbf{SC-JSON-код}} представляют собой линейную структуру, представляемую в виде линейного строкового текста и состоящую из набора корректных осмысленных команд, записанных в виде \textit{sc-json-пар} вида \{отношение: объект\}, где отношением выступает знак квазибинарного отношения, состоящего из пар вида \{субъект: объект\}, где объектом выступает знак, обозначаемый предложением, включающее такие пары, а субъектом - sc-json-объекты: sc-json-литерал, sc-json-списки sc-json-объектов, sc-json-предложения, состоящие из sc-json-списков sc-json-объектов.}
                \scnitem{Тексты \textit{\textbf{SC-JSON-кода}} представляют собой sc-json-команды. Каждая команда представляет собой json-объект, в котором указываются уникальный идентификатор команды, тип этой команды и ее аргументы. C каждой командой ассоциируется ответ на эту команду. Ответ на команду представляет собой команду, в котором указываются идентификатор команды, ее статус (выполнена успешно/безуспешно) и результаты. Структура аргументов и результатов команды определяется типом команды. Для каждого ответа существует запрос.}
            \end{scnrelfromvector}
            \begin{scnrelfromlist}{достоинство}
                \scnfileitem{Язык JSON является общепринятым открытым форматом, для работы с которым существует большое количество библиотек для популярных языков программирования. Это, в свою очередь, упрощает реализацию клиента и сервера для протокола, построенного на базе \textit{\textbf{SC-JSON-код}}.}
                \scnfileitem{Реализация подсистемы взаимодействия со внешней средой на базе \textit{\textbf{SC-JSON-код}} не накладывает принципиальных ограничений на объем (длину) каждой команды, в отличие от других бинарных протоколов. Таким образом, появляется возможность использования неатомарных команд, позволяющих, например, за один акт пересылки такой команды по сети создать сразу несколько sc-элементов. Важными примерами таких команд являются \textit{команда создания sc-конструкции, изоморфной заданному sc-шаблону}, и \textit{команда поиска sc-конструкций, изоморфных заданному sc-шаблону}.}
            \end{scnrelfromlist}
            \scntext{примечание}{Можно сказать, что язык на базе JSON является следующим шагом на пути к созданию мощного и универсального языка запросов, аналогичного языку SQL для реляционных баз данных и предназначенному для работы с sc-памятью. Следующий шагом станет реализация такого протокола на основе одного из стандартов внешнего отображения sc-конструкций, например, \textit{SCs-кода}, что, в свою очередь, позволит передавать в качестве команд целые программы обработки sc-конструкций, например на языке SCP.}
            \scnstructheader{Синтаксис SC-JSON-кода}
            \begin{scnsubstruct}
                \scnheader{Синтаксис SC-JSON-кода}
                \scntext{примечание}{\textit{Синтаксис SC-JSON-кода} задается: (1) \textit{Алфавитом SC-JSON-кода}, (2) Грамматикой SC-JSON-кода}
                \scnrelto{синтаксис}{SC-JSON-код}
                \scnstructheader{Синтаксическая классификация элементов SC-JSON-кода}
                \begin{scnsubstruct}
                    \scnstructheader{SC-JSON-код}
                    \scnrelto{семейство подмножеств}{sc-json-предложение}
                    \begin{scnindent}
                        \scnsubset{json-список json-пар}
                        \scnrelto{семейство подмножеств}{sc-json-пара*}
                        \begin{scnindent}
                            \begin{scnreltovector}{декартово произведение}
                                \scnitem{sc-json-строка}
                                \scnitem{sc-json-объект}
                                \begin{scnindent}
                                    \begin{scnrelfromset}{разбиение}
                                        \scnitem{sc-json-cписок}
                                        \scnitem{sc-json-пара}
                                        \scnitem{sc-json-литерал}
                                        \begin{scnindent}
                                            \begin{scnrelfromset}{разбиение}
                                                \scnitem{sc-json-строка}
                                                \scnitem{sc-json-число}
                                            \end{scnrelfromset}
                                        \end{scnindent}
                                    \end{scnrelfromset}
                                \end{scnindent}
                            \end{scnreltovector}
                        \end{scnindent}
                        \begin{scnrelfromset}{разбиение}
                            \scnitem{команда на SC-JSON-коде}
                            \scnitem{ответ на команду на SC-JSON-коде}
                        \end{scnrelfromset}
                    \end{scnindent}
                \end{scnsubstruct}
                \scnsourcecomment{Завершили представление \textit{Синтаксической классификации элементов SC-JSON-кода}}
                \scnheader{Алфавит SC-JSON-кода\scnsupergroupsign}
                \scnidtf{Множество всех возможных символов в SC-JSON-коде}
                \scntext{пояснение}{Поскольку SC-JSON-код является линейным строковым языком представления знаний, то его алфавит включает объединение алфавитов всех языков, тексты на которых могут представлять внешние идентификаторы и/или содержимое файлов ostis-системы, множество всех цифр и множество всех других специальных символов.}
                \scnrelto{алфавит}{SC-JSON-код}
                \scntext{примечание}{Последовательности знаков алфавита могут образовывать sc-json ключевые слова, sc-json-пары, sc-json-предложения из sc-json-пар и sc-json-тексты из sc-json-предложений.}
                \scnheader{SC-JSON-код}
                \begin{scnrelfromlist}{синтаксические правила}
                    \scnfileitem{Каждое правило \textit{Грамматики SC-JSON-кода} описывает корректный с точки зрения \textit{Синтаксиса SC-JSON-кода} порядок sc-json-объектов в sc-json-предложении. Совокупность правил \textit{Грамматики SC-JSON-кода} описывает корректный с точки зрения \textit{Синтаксиса SC-JSON-кода} порядок sc-json-предложений в sc-json-тексте.}
                    \scnfileitem{Каждое sc-json-предложение является sc-json-списком, состоящим из sc-json-пар и представляет собой команду или ответ на эту команду.}
                    \scnfileitem{Каждое \textit{команда (ответ на команду) на SC-JSON-коде} состоит из заголовка, включающего sc-json-пары описания самой команды (ответа на команду), и сообщения, различного для каждого класса команд (ответов на команды). Сообщение \textit{команды (ответа на команду) на SC-JSON-коде} обычно представляет собой список sc-json-объектов и может не ограничиваться по мощности.}
                    \scnfileitem{Каждая sc-json-пара состоит из двух элементов: ключевого слова и некоторого другого sc-json-объекта, ассоциируемого с этим ключевым словом. Набор ключевых слов в sc-json-парах определяется конкретным классом \textit{команд (ответов на команды) на SC-JSON-коде}. Sc-json-пара начинается знаком открывающейся фигурной скобки \scnqq{\{} и заканчивается знаком закрывающейся фигурной скобки \scnqq{\}}. Ключевое слово и sc-json-объект, ассоциируемый с ним, разделяются при помощи знака двоеточия \scnqq{:}.}
                    \scnfileitem{Sc-json-строки, записанные в sc-json-текстах, начинаются и заканчиваются знаком двух ковычек \textquotedblleft.}
                    \scnfileitem{Sc-json-списки, состоящие не из sc-json-пар, начинаются знаком открывающейся квадратной скобки \scnqq{} и заканчиваются знаком закрывающейся квадратной скобки \scnqq{}. Sc-json-объекты в sc-json-списках разделяются запятыми \scnqq{,}.}
                \end{scnrelfromlist}
                \scnheader{Грамматика SC-JSON-кода}
                \scnidtf{Множество всех возможных правил, используемых при построении команд и ответов на них на SC-JSON-коде}
                \scntext{пояснение}{Каждой команде \textit{SC-JSON-кода} однозначно соответствует правило грамматики \textit{SC-JSON-кода}.}
                \scnrelto{грамматика}{SC-JSON-код}
                \scntext{пояснение}{Правила \textit{Грамматики SC-JSON-кода} позволяют правильно составить команду на SC-JSON-коде. Каждое правило грамматики \textit{SC-JSON-кода} представляется в виде правила на \textit{Языке описания грамматик ANTLR} и его интерпретации на естественном языке.}
                \scnhaselementrole{ключевой sc-элемент}{Правило, задающее синтаксис \textit{команд на SC-JSON-коде}}
                \begin{scnindent}
                    \scnrelboth{семантическая эквивалентность}{\scnfileimage[20em]{Contents/part_platform/images/command.png}}
                    \begin{scnindent}
                        \scniselement{Язык описания грамматики языков ANTLR}
                        \scntext{интерпретация}{Класс \textit{команд на SC-JSON-коде} включает \textit{команду создания sc-элементов}, \textit{команду получения соответствующих типов sc-элементов}, \textit{команду удаления sc-элементов}, \textit{команду обработки ключевых sc-элементов}, \textit{команду обработки содержимого файлов ostis-системы}, \textit{команду поиска sc-конструкций, изоморфных заданному sc-шаблону}, \textit{команду генерации sc-конструкции, изоморфной заданному sc-шаблону}, и \textit{команду обработки sc-событий}. В \textit{команду на SC-JSON-коде} включаются идентификатор этой команды, тип и сообщение.}
                    \end{scnindent}
                    \scnrelto{синтаксическое правило}{команда на SC-JSON-коде}
                \end{scnindent}
                \scnhaselementrole{ключевой sc-элемент}{Правило, задающее синтаксис \textit{ответа на команду на SC-JSON-коде}}
                \begin{scnindent}
                    \scnrelboth{семантическая эквивалентность}{\scnfileimage[20em]{Contents/part_platform/images/command_answer.png}}
                    \begin{scnindent}
                        \scniselement{Язык описания грамматики языков ANTLR}
                        \scntext{интерпретация}{Класс \textit{ответов на команды на SC-JSON-коде} включает \textit{ответ на команду создания sc-элементов}, \textit{ответ на команду получения соответствующих типов sc-элементов}, \textit{ответ на команду удаления sc-элементов}, \textit{ответ на команду обработки ключевых sc-элементов}, \textit{ответ на команду обработки содержимого файлов ostis-системы}, \textit{ответ на команду поиска sc-конструкций, изоморфных заданному sc-шаблону}, \textit{ответ на команду генерации sc-конструкции, изоморфной заданному sc-шаблону}, и \textit{ответ на команду обработки sc-событий}. В \textit{ответ на команду на SC-JSON-коде} включаются идентификатор соответствующей команды, статус обработки ответа и ответное сообщение.}
                    \end{scnindent}
                    \scnrelto{синтаксическое правило}{ответ на команду на SC-JSON-коде}
                \end{scnindent}
                \scnhaselement{Правило, задающее синтаксис \textit{команды создания sc-элементов}}
                \begin{scnindent}
                    \scnrelboth{семантическая эквивалентность}{\scnfileimage[20em]{Contents/part_platform/images/create_elements_command.pdf}}
                    \begin{scnindent}
                        \scniselement{Язык описания грамматики языков ANTLR}
                        \scntext{интерпретация}{В сообщении \textit{команды создания sc-элементов} представляется список описаний создаваемых sc-элементов. Такими sc-элементами могут быть sc-узел, sc-дуга, sc-ребро или файл ostis-системы. Тип sc-элемента указывается в паре с ключевым словом \scnqq{el}: для sc-узла sc-json-тип элемент представляется как \scnqq{node}, для sc-дуги и sc-ребра - \scnqq{edge}, для файла ostis-системы - \scnqq{link}. Метки типов sc-элементов уточняются в соответствующих им описаниях в сообщении команды в паре с ключевым словом \scnqq{type}. Если создаваемым sc-элементом является файл ostis-системы, то дополнительно указывается содержимое этого файла ostis-системы в паре с ключевым словом \scnqq{content}, если создаваемым sc-элементом является sc-дуга или sc-ребро, то указываются описания sc-элементов, из которых они выходят, и sc-элементов, в которые они входят. Описание таких sc-элементов состоят из двух пар: первая пара указывает на способ ассоциации с sc-элементом и представляется как \scnqq{addr} или \scnqq{idtf} или \scnqq{ref} в паре с ключевым словом \scnqq{type}, вторая пара - то, по чему происходит ассоциация с этим sc-элементом: его хэшу, системному идентификатору или номеру в массиве создаваемых sc-элементов - в паре с ключевым словом \scnqq{value}.}
                    \end{scnindent}
                    \scnrelto{синтаксическое правило}{команда создания sc-элементов}
                \end{scnindent}
                \scnhaselement{Правило, задающее синтаксис \textit{ответа на команду создания sc-элементов}}
                \begin{scnindent}
                    \scnrelboth{семантическая эквивалентность}{\scnfileimage[20em]{Contents/part_platform/images/create_elements_command_answer.png}}
                    \begin{scnindent}
                        \scniselement{Язык описания грамматики языков ANTLR}
                        \scntext{интерпретация}{Сообщением \textit{ответа на команду создания sc-элементов} является список хэшей созданных sc-элементов, соответствующих описаниям \textit{команды создания sc-элементов} со статусом 1, в случае успешной обработки команды.}
                    \end{scnindent}
                    \scnrelto{синтаксическое правило}{ответ на команду создания sc-элементов}
                \end{scnindent}
                \scnhaselement{Правило, задающее синтаксис \textit{команды создания sc-элементов по фрагменту SCs-текста}}
                \begin{scnindent}
                    \scnrelboth{семантическая эквивалентность}{\scnfileimage[20em]{Contents/part_platform/images/create_elements_by_scs_command.png}
                    }
                    \begin{scnindent}
                        \scniselement{Язык описания грамматики языков ANTLR}
                        \scntext{интерпретация}{В списке описаний создаваемых sc-элементов сообщения этой команды вместо описания создаваемого отдельного sc-элемента указывается фрагмент SCs-текста.}
                    \end{scnindent}
                    \scnrelto{синтаксическое правило}{команда создания sc-элементов по фрагменту SCs-текста}
                \end{scnindent}
                \scnhaselement{Правило, задающее синтаксис \textit{ответа на команду создания sc-элементов по фрагменту SCs-текста}}
                \begin{scnindent}
                    \scnrelboth{семантическая эквивалентность}{\scnfileimage[10em]{Contents/part_platform/images/create_elements_by_scs_command_answer.png}}
                    \begin{scnindent}
                        \scniselement{Язык описания грамматики языков ANTLR}
                        \scntext{интерпретация}{Сообщением \textit{ответа на команду создания sc-элементов} является список результатов обработки переданных SCs-текстов. Нулевой статус говорит о том, что обработка соотвествующего SCs-текста завершилась безуспешно.}
                    \end{scnindent}
                    \scnrelto{синтаксическое правило}{ответ на команду создания sc-элементов по фрагменту SCs-текста}
                \end{scnindent}
                \scnhaselement{Правило, задающее синтаксис \textit{команды получения соответствующих типов sc-элементов}}
                \begin{scnindent}
                    \scnrelboth{семантическая эквивалентность}{\scnfileimage[20em]{Contents/part_platform/images/check_elements_command.png}}
                    \begin{scnindent}
                        \scniselement{Язык описания грамматики языков ANTLR}
                        \scntext{интерпретация}{Сообщением \textit{команды получения соответствующих типов sc-элементов} является списком хэшей sc-элементов, у которых необходимо получить синтаксические типы.}
                    \end{scnindent}
                    \scnrelto{синтаксическое правило}{команда получения соответствующих типов sc-элементов}
                \end{scnindent}
                \scnhaselement{Правило, задающее синтаксис \textit{ответа на команду получения соответствующих типов sc-элементов}}
                \begin{scnindent}
                    \scnrelboth{семантическая эквивалентность}{\scnfileimage[20em]{Contents/part_platform/images/check_elements_command_answer.png}}
                    \begin{scnindent}
                        \scniselement{Язык описания грамматики языков ANTLR}
                        \scntext{интерпретация}{Сообщением \textit{ответа на команду получения соответствующих типов sc-элементов} является список типов проверенных sc-элементов, соответствующих описаниям \textit{команды получения соответствующих типов sc-элементов} со статусом 1, в случае успешной обработки команды.}
                    \end{scnindent}
                    \scnrelto{синтаксическое правило}{ответ на команду получения соответствующих типов sc-элементов}
                \end{scnindent}
                \scnhaselement{Правило, задающее синтаксис \textit{команды удаления sc-элементов}}
                \begin{scnindent}
                    \scnrelboth{семантическая эквивалентность}{\scnfileimage[20em]{Contents/part_platform/images/delete_elements_command.png}
                    }
                    \begin{scnindent}
                        \scniselement{Язык описания грамматики языков ANTLR}
                        \scntext{интерпретация}{Сообщением \textit{команды удаления sc-элементов} является список хэшей sc-элементов, которые необходимо удалить из sc-памяти.}
                    \end{scnindent}
                    \scnrelto{синтаксическое правило}{команда удаления sc-элементов}
                \end{scnindent}
                \scnhaselement{Правило, задающее синтаксис \textit{ответа на команду удаления sc-элементов}}
                \begin{scnindent}
                    \scnrelboth{семантическая эквивалентность}{\scnfileimage[20em]{Contents/part_platform/images/delete_elements_command_answer.png}}
                    \begin{scnindent}
                        \scniselement{Язык описания грамматики языков ANTLR}
                        \scntext{интерпретация}{Сообщение \textit{ответа на команду удаления sc-элементов} является пустым со статусом 1, в случае успешной обработки команды.}
                    \end{scnindent}
                    \scnrelto{синтаксическое правило}{ответ на команду удаления sc-элементов}
                \end{scnindent}
                \scnhaselement{Правило, задающее синтаксис \textit{команды обработки ключевых sc-элементов}}
                \begin{scnindent}
                    \scnrelboth{семантическая эквивалентность}{\scnfileimage[20em]{Contents/part_platform/images/handle_keynodes_command.png}}
                    \begin{scnindent}
                        \scniselement{Язык описания грамматики языков ANTLR}
                        \scntext{интерпретация}{Сообщение \textit{команды обработки ключевых sc-элементов} может включать описание ключевых sc-элементов, которые необходимо найти и/или разрешить по их идентификаторам. Такое деление осуществляется с помощью подкоманд, содержащихся в сообщении команды. Идентификаторами подкоманд могут быть \scnqq{find} и \scnqq{resolve} соответственно, стоящие в паре с ключевым словом \scnqq{command}. Описание искомого sc-элемента команды \scnqq{find} включает системный идентификатор sc-элемента, по которому необходимо найти этот sc-элемент, стоящий в паре с ключевым словом \scnqq{idtf}. Описание разрешаемого sc-элемента команды \scnqq{resolve} включает системный идентификатор sc-элемента, по которому необходимо найти этот sc-элемент, либо в случае безуспешного поиска создать sc-элемент некоторого типа, указанного в его описании в паре с ключевым словом \scnqq{elType}, и установить для него системный идентификатор, по которому была произведена попытка найти другой sc-элемент.}
                    \end{scnindent}
                    \scnrelto{синтаксическое правило}{команда обработки ключевых sc-элементов}
                \end{scnindent}
                \scnhaselement{Правило, задающее синтаксис \textit{ответа на команду обработки ключевых sc-элементов}}
                \begin{scnindent}
                    \scnrelboth{семантическая эквивалентность}{\scnfileimage[20em]{Contents/part_platform/images/handle_keynodes_command_answer.png}}
                    \begin{scnindent}
                        \scniselement{Язык описания грамматики языков ANTLR}
                        \scntext{интерпретация}{Сообщением \textit{ответа на команду обработки ключевых sc-элементов} является список хэшей sc-элементов, соответствующих описаниям \textit{команды обработки ключевых sc-элементов} со статусом 1, в случае успешной обработки команды.}
                    \end{scnindent}
                    \scnrelto{синтаксическое правило}{ответ на команду обработки ключевых sc-элементов}
                \end{scnindent}
                \scnhaselement{Правило, задающее синтаксис \textit{команды обработки содержимого файлов ostis-системы}}
                \begin{scnindent}
                    \scnrelboth{семантическая эквивалентность}{\scnfileimage[10em]{Contents/part_platform/images/handle_link_contents_command.png}}
                    \begin{scnindent}
                        \scniselement{Язык описания грамматики языков ANTLR}
                        \scntext{интерпретация}{Сообщение \textit{команды обработки содержимого файлов ostis-системы} может включать описание ключевых файлов ostis-системы, которые необходимо найти по их содержимому или части этого содержимого, для которых необходимо установить содержимое разрешить и/или у которых необходимо получить содержимое. Как и в \textit{Правиле, задающее синтаксис команды обработки ключевых sc-элементов} деление осуществляется с помощью подкоманд, содержащихся в сообщении команды. Идентификаторами подкоманд могут быть \scnqq{find}, \scnqq{find\_by\_substr}, \scnqq{set} и \scnqq{get} соответственно, стоящие в паре с ключевым словом \scnqq{command}. В описаниях команд \scnqq{set} и \scnqq{get} указывается хэш файла ostis-системы в паре с ключевым словом \scnqq{addr}. В описаниях команд \scnqq{set}, \scnqq{find} и \scnqq{find\_by\_substr} указывается содержимое файла ostis-системы в паре с ключевым словом \scnqq{data}. Дополнительно в описании подкоманды \scnqq{set} указывается тип устанавливаемого содержимого файла ostis-системы.}
                    \end{scnindent}
                    \scnrelto{синтаксическое правило}{команда обработки содержимого файлов ostis-системы}
                \end{scnindent}
                \scnhaselement{Правило, задающее синтаксис \textit{ответа на команду обработки содержимого файлов ostis-системы}}
                \begin{scnindent}
                    \scnrelboth{семантическая эквивалентность}{\scnfileimage[60em]{Contents/part_platform/images/handle_link_contents_command_answer.png}
                    }
                    \begin{scnindent}
                        \scniselement{Язык описания грамматики языков ANTLR}
                        \scntext{интерпретация}{Сообщением \textit{ответа на команду обработки содержимого файлов ostis-системы} является список, состоящий из булевого результата установки содержимого в файл ostis-системы и/или найденных файлов ostis-системы по их содержимому и/или описания полученного содержимого файлов ostis-системы, соответствующих описаниям \textit{команды обработки содержимого файлов ostis-системы} со статусом 1, в случае успешной обработки команды.}
                    \end{scnindent}
                    \scnrelto{синтаксическое правило}{ответ на команду обработки содержимого файлов ostis-системы}
                \end{scnindent}
                \scnhaselement{Правило, задающее синтаксис \textit{команды поиска sc-конструкций, изоморфных заданному sc-шаблону}}
                \begin{scnindent}
                    \scnrelboth{семантическая эквивалентность}{\scnfileimage[10em]{Contents/part_platform/images/search_template_command.png}}
                    \begin{scnindent}
                        \scniselement{Язык описания грамматики языков ANTLR}
                        \scnsubset{Правило, задающее синтаксис сообщения \textit{команды поиска sc-конструкций, изоморфных заданному sc-шаблону}, и \textit{команды генерации sc-конструкции, изоморфной заданному sc-шаблону}}
                        \scntext{интерпретация}{\textit{Правило, задающее синтаксис команды поиска sc-конструкций, изоморфных заданному sc-шаблону} включает \textit{Правило, задающее синтаксис сообщения \textit{команды поиска sc-конструкций, изоморфных заданному sc-шаблону,} и \textit{команды генерации sc-конструкции, изоморфной заданному sc-шаблону}} и описывает команду поиска sc-конструкций по сформированному этой командой sc-шаблону (см. \textit{Правило, задающее синтаксис сообщения \textit{команды поиска sc-конструкций, изоморфных заданному sc-шаблону,} и \textit{команды генерации sc-конструкции, изоморфной заданному sc-шаблону}}).}
                    \end{scnindent}
                    \scnrelto{синтаксическое правило}{команда поиска sc-конструкций, изоморфных заданному sc-шаблону}
                \end{scnindent}
                \scnhaselement{Правило, задающее синтаксис \textit{ответа на команду поиска sc-конструкций, изоморфных заданному sc-шаблону}}
                \begin{scnindent}
                    \scnrelboth{семантическая эквивалентность}{\scnfileimage[15em]{Contents/part_platform/images/search_template_command_answer.png}}
                    \begin{scnindent}
                        \scniselement{Язык описания грамматики языков ANTLR}
                        \scntext{интерпретация}{Сообщение \textit{ответа на команду поиска sc-конструкций, изоморфных заданному sc-шаблону} состоит из списка найденных sc-конструкций и отображения псевдонимов sc-элементов на их позиции в тройках sc-шаблона. Ответ имеет статус 1, в случае успешной обработки команды.}
                    \end{scnindent}
                    \scnrelto{синтаксическое правило}{ответ на команду поиска sc-конструкций, изоморфных заданному sc-шаблону}
                \end{scnindent}
                \scnhaselement{Правило, задающее синтаксис \textit{команды создания sc-конструкции, изоморфной заданному sc-шаблону}}
                \begin{scnindent}
                    \scnrelboth{семантическая эквивалентность}{\scnfileimage[20em]{Contents/part_platform/images/generate_template_command.png}}
                    \begin{scnindent}
                        \scniselement{Язык описания грамматики языков ANTLR}
                        \scnsubset{Правило, задающее синтаксис сообщения \textit{команды поиска sc-конструкций, изоморфных заданному sc-шаблону,} и \textit{команды генерации sc-конструкции, изоморфной заданному sc-шаблону}}
                        \scntext{интерпретация}{\textit{Правило, задающее синтаксис команды создания sc-конструкции, изоморфной заданному sc-шаблону} включает \textit{Правило, задающее синтаксис сообщения \textit{команды поиска sc-конструкций, изоморфных заданному sc-шаблону,} и \textit{команды генерации sc-конструкции, изоморфной заданному sc-шаблону}} и описывает команду создания sc-конструкции по сформированному этой командой sc-шаблону (см. \textit{Правило, задающее синтаксис сообщения \textit{команды поиска sc-конструкций, изоморфных заданному sc-шаблону,} и \textit{команды генерации sc-конструкции, изоморфной заданному sc-шаблону}}).}
                    \end{scnindent}
                    \scnrelto{синтаксическое правило}{команда создания sc-конструкции, изоморфной заданному sc-шаблону}
                \end{scnindent}
                \scnhaselement{Правило, задающее синтаксис \textit{ответа на команду создания sc-конструкции, изоморфной заданному sc-шаблону}}
                \begin{scnindent}
                    \scnrelboth{семантическая эквивалентность}{\scnfileimage[20em]{Contents/part_platform/images/generate_template_command_answer.png}}
                    \begin{scnindent}
                        \scniselement{Язык описания грамматики языков ANTLR}
                        \scntext{интерпретация}{Сообщение \textit{ответа на команду создания sc-конструкции, изоморфной заданному sc-шаблону} состоит из списка найденных sc-конструкций и отображения псевдонимов sc-элементов на их позиции в тройках sc-шаблона. Ответ имеет статус 1, в случае успешной обработки команды.}
                    \end{scnindent}
                    \scnrelto{синтаксическое правило}{ответ на команду создания sc-конструкции, изоморфной заданному sc-шаблону}
                \end{scnindent}
                \scnhaselement{Правило, задающее синтаксис сообщения \textit{команды поиска sc-конструкций, изоморфных заданному sc-шаблону,} и \textit{команды создания sc-конструкции, изоморфной заданному sc-шаблону}}
                \begin{scnindent}
                    \scnrelboth{семантическая эквивалентность}{\scnfileimage[20em]{Contents/part_platform/images/template_message_command.png}}
                    \begin{scnindent}
                        \scniselement{Язык описания грамматики языков ANTLR}
                        \scntext{интерпретация}{Сообщения \textit{команды поиска sc-конструкций, изоморфных заданному sc-шаблону,} и \textit{команды создания sc-конструкции, изоморфной заданному sc-шаблону} включают описание троек, необходимых для создания sc-шаблона поиска или генерации изоморфных sc-конструкций. Описание каждой тройки sc-шаблона включает описание sc-элементов этой тройки. Описания первого и третьего sc-элементов тройки должны всегда содержать хэш или тип в паре с ключевым словом \scnqq{value}. Если выбран тип, то в паре с ключевым словом \scnqq{type} указывается \scnqq{type}, если - хэш, то - \scnqq{addr}. Вторым sc-элементом тройки должна быть дуга, для которой всегда указывается тип в паре с ключевым словом \scnqq{value}. Для каждого sc-элемента тройки может указываться псевдоним в паре с \scnqq{alias}. Псевдоним представляет собой строку и может быть использован для ассоциации с sc-элементами в других тройках sc-шаблона, либо ассоциации со значениями переменных sc-шаблона, которые указываются в списке под ключевым словом \scnqq{params} и могут представлять собой либо хэш sc-элемента, либо его системный идентификатор. Таким образом, в некоторых случаях может отсутствовать необходимость указания хэша или типа sc-элемента. Также вместо списка описаний троек sc-шаблона, может указываться хэш или системный идентификатор sc-структуры, хранящейся в sc-памяти. хэш и системный идентификатор указываются в паре с ключевым словом \scnqq{value}.}
                    \end{scnindent}
                \end{scnindent}
                \scnhaselement{Правило, задающее синтаксис \textit{команды обработки sc-событий}}
                \begin{scnindent}
                    \scnrelboth{семантическая эквивалентность}{\scnfileimage[20em]{Contents/part_platform/images/handle_events_command.png}}
                    \begin{scnindent}
                        \scniselement{Язык описания грамматики языков ANTLR}
                        \scntext{интерпретация}{Сообщение \textit{команды обработки sc-событий} может включать описание sc-элементов, по котором необходимо зарегистрировать или разрегистрировать sc-события. Идентификаторами подкоманд в описании команды могут быть \scnqq{create} и \scnqq{delete} соответственно, стоящие в паре с ключевым словом \scnqq{command}. Описание команды регистрации sс-cобытий \scnqq{create} представляет собой список описаний типов sc-событий и sc-элементов, по которым необходимо зарегистрировать sc-события. Описания sc-элементов включают хэши этих sc-элементов в парах с ключевым словом \scnqq{addr}. Описание команды разрегистрации sc-событий представляет собой список позиций sc-событий в очереди sc-событий, которые необходимо удалить из этой очереди sc-событий.}
                    \end{scnindent}
                    \scnsuperset{Правило, задающее синтаксис типов sc-событий}
                    \begin{scnindent}
                        \scnrelboth{семантическая эквивалентность}{\scnfileimage[10em]{Contents/part_platform/images/sc_event_types.png}}
                        \begin{scnindent}
                            \scniselement{Язык описания грамматики языков ANTLR}
                            \scntext{интерпретация}{Sc-событиями могут быть \textit{sc-события добавления выходящей дуги из sc-элемента (add\_outgoing\_edge)}, \textit{sc-события добавления входящей дуги в sc-элемент (add\_ingoing\_edge)}, \textit{sc-события удаления выходящей дуги из sc-элемента (remove\_outgoing\_edge)}, \textit{sc-события удаления входящей дуги в sc-элемент (remove\_ingoing\_edge)}, \textit{sc-события изменения содержимого файла ostis-системы (content\_change)} и \textit{sc-события удаления sc-элемента (delete\_element)}.}
                        \end{scnindent}
                    \end{scnindent}
                    \scnrelto{синтаксическое правило}{команда обработки sc-событий}
                \end{scnindent}
                \scnhaselement{Правило, задающее синтаксис \textit{ответа на команду обработки sc-событий}}
                \begin{scnindent}
                    \scnrelboth{семантическая эквивалентность}{\scnfileimage[30em]{Contents/part_platform/images/handle_events_command_answer.png}}
                    \begin{scnindent}
                        \scniselement{Язык описания грамматики языков ANTLR}
                        \scntext{интерпретация}{Сообщение \textit{ответа на команду обработки sc-событий} состоит из позиций зарегистрированных sc-событий в очереди. Успешным результатом \textit{команды обработки sc-событий} является статус 1.}
                    \end{scnindent}
                    \scnrelto{синтаксическое правило}{ответ на команду обработки sc-событий}
                \end{scnindent}
                \scnhaselement{Правило, задающее синтаксис \textit{ответа инициализации sc-события}}
                \begin{scnindent}
                    \scnrelboth{семантическая эквивалентность}{\scnfileimage[20em]{Contents/part_platform/images/init_event_command_answer.png}}
                    \begin{scnindent}
                        \scniselement{Язык описания грамматики языков ANTLR}
                        \scntext{интерпретация}{\textit{Ответ инициализации sc-события} возникает тогда и только тогда, когда в sc-памяти инициализируется соответствующее sc-событие. \textit{Ответ инициализации sc-события} всегда отсылается той клиентской системе, которая зарегистрировала это sc-событие. В сообщение \textit{ответа инициализации sc-события} включаются хэши тех sc-элементов, которые связаны с зарегистрированным sc-событием. Таким образом, если было зарегистрировано sc-событие выходящей sc-дуги, то в списке сообщения \textit{ответа инициализации sc-события} будут находится хэши трёх sc-элементов: хэш sc-элемента, который был подписан на sc-событие, хэш добавленной выходящей из него sc-дуги и хэш sc-элемента, являющегося концом этой sc-дуги.}
                    \end{scnindent}
                    \scnrelto{синтаксическое правило}{ответ инициализации sc-события}
                \end{scnindent}
                \scnhaselement{Правило, задающее синтаксис \textit{синтаксических типов sc-элементов}}
                \begin{scnindent}
                    \scnrelboth{семантическая эквивалентность}{\scnfileimage[10em]{Contents/part_platform/images/sc_addr_types.png}}
                    \begin{scnindent}
                        \scniselement{Язык описания грамматики языков ANTLR}
                        \scntext{интерпретация}{\textit{Правило, задающее синтаксис синтаксических типов sc-элементов} включает \textit{Правило, задающее синтаксис синтаксических типов sc-узлов}, \textit{Правило, задающее синтаксис синтаксических типов sc-дуг}, \textit{Правило, задающее синтаксис синтаксических типов файлов ostis-системы}. Синтаксические типы sc-элементов представляются в виде целого числа и соответствуют программным синтаксическим типам, представляемым в sc-памяти.}
                    \end{scnindent}
                    \scntext{примечание}{На данный момент форма представления синтаксического типа sc-элемента зависит от того, как располагаются биты в маске sc-элемента. Следующим шагом в развитии \textit{SC-JSON-кода} и его грамматики могли быть стать устранение такой зависимости и переход к представлению синтаксических типов в виде строковых литералов, интерпретируемых \textit{Серверной системы на основе Websocket, обеспечивающей доступ к sc-памяти платформы интерпретации sc-моделей при помощи команд SC-JSON-кода}.}
                \end{scnindent}
                \scnhaselement{Правило, задающее синтаксис \textit{синтаксических типов sc-узлов}}
                \begin{scnindent}
                    \scnrelboth{семантическая эквивалентность}{\scnfileimage[20em]{Contents/part_platform/images/sc_node_types.png}}
                    \begin{scnindent}
                        \scniselement{Язык описания грамматики языков ANTLR}
                        \scntext{интерпретация}{\textit{Правило, задающее синтаксис синтаксических типов sc-узлов} описывает возможные синтаксические типы sc-узлов, интерпретируемые на стороне \textit{Серверной системы на основе Websocket, обеспечивающей доступ к sc-памяти платформы интерпретации sc-моделей при помощи команд SC-JSON-кода}.}
                    \end{scnindent}
                \end{scnindent}
                \scnhaselement{Правило, задающее синтаксис \textit{синтаксических типов sc-дуг}}
                \begin{scnindent}
                    \scnrelboth{семантическая эквивалентность}{\scnfileimage[20em]{Contents/part_platform/images/sc_edge_types.png}}
                    \begin{scnindent}
                        \scniselement{Язык описания грамматики языков ANTLR}
                        \scntext{интерпретация}{\textit{Правило, задающее синтаксис синтаксических типов sc-дуг} описывает возможные синтаксические типы sc-дуг, в том числе и sc-рёбер, интерпретируемые на стороне \textit{Серверной системы на основе Websocket, обеспечивающей доступ к sc-памяти платформы интерпретации sc-моделей при помощи команд SC-JSON-кода}.}
                    \end{scnindent}
                \end{scnindent}
                \scnhaselement{Правило, задающее синтаксис \textit{синтаксических типов файлов ostis-системы}}
                \begin{scnindent}
                    \scnrelboth{семантическая эквивалентность}{\scnfileimage[10em]{Contents/part_platform/images/sc_link_types.png}}
                    \begin{scnindent}
                        \scniselement{Язык описания грамматики языков ANTLR}
                        \scntext{интерпретация}{\textit{Правило, задающее синтаксис синтаксических типов файлов ostis-системы} описывает возможные синтаксические типы файлов ostis-системы, интерпретируемые на стороне \textit{Серверной системы на основе Websocket, обеспечивающей доступ к sc-памяти платформы интерпретации sc-моделей при помощи команд SC-JSON-кода}.}
                    \end{scnindent}
                \end{scnindent}
                \scnheader{команда на SC-JSON-коде}
                \scnidtf{sc-json-code command}
                \scnsubset{SC-JSON-код}
                \scntext{примечание}{Множество \textit{команд на SC-JSON-коде} легко расширяемо засчёт гибкости и функциональности языка JSON.}
                \scnheader{ответ на команду на SC-JSON-коде}
                \scnidtf{sc-json-code command answer}
                \scnsubset{SC-JSON-код}
                \scntext{примечание}{Множество \textit{ответов на команды на SC-JSON-коде} легко расширяемо вместе с расширением \textit{команд на SC-JSON-коде}.}
                \scnheader{команда создания sc-элементов}
                \scnidtf{create elements command}
                \scnsubset{команда на SC-JSON-коде}
                \scnrelfrom{пример}{Пример команды создания sc-элементов}
                \begin{scnindent}
                    \scneqimage[10em]{Contents/part_platform/images/create_elements_command_example.png}
                    \scniselement{команда создания sc-элементов}
                    \scnrelfrom{ответ}{Пример ответа на команду создания sc-элементов}
                    \scntext{интерпретация}{Обработать команду создания sc-элементов: sc-узла с типом 1 (неуточняемого типа), файла ostis-системы с типом 2 (неуточняемого типа) и содержимым в виде числа с плавающей точкой 45.4 и sc-дуги типа 32 (константного типа) между sc-элементом, находящимся на нулевой позиции в массиве создаваемых sc-элементов, и sc-элементом, находящимся на первой позиции в том же самом массиве.}
                \end{scnindent}
                \scnrelfrom{класс команд}{ответ на команду создания sc-элементов}
                \scntext{примечание}{Стоит отметить, что на уровне интерфейса sc-памяти команда интерпретируется быстро за счёт того, что не используются шаблоны создания изоморфных им конструкций. Также содержимое сообщения \textit{команды создания sc-элементов} может быть пустым.}
                \scnheader{ответ на команду создания sc-элементов}
                \scnidtf{create elements command answer}
                \scnsubset{ответ на команду на SC-JSON-коде}
                \scnrelfrom{пример}{Пример ответа на команду создания sc-элементов}
                \begin{scnindent}
                    \scneqimage[7em]{Contents/part_platform/images/create_elements_command_answer_example.png}
                    \scniselement{ответ на команду создания sc-элементов}
                    \scntext{интерпретация}{Созданы sc-элементы с хэшами 323, 534 и 342 соответственно. Команда обработана успешно.}
                \end{scnindent}
                \scnheader{команда получения соответствующих типов sc-элементов}
                \scnidtf{check elements command}
                \scnsubset{команда на SC-JSON-коде}
                \scnrelfrom{пример}{Пример команды получения соответствующих типов sc-элементов}
                \begin{scnindent}
                    \scneqimage[10em]{Contents/part_platform/images/check_elements_command_example.png}
                    \scniselement{команда получения соответствующих типов sc-элементов}
                    \scnrelfrom{ответ}{Пример ответа на команду получения соответствующих типов sc-элементов}
                    \scntext{интерпретация}{Получить синтаксические типы sc-элементов с хэшами 885 и 1025.}
                \end{scnindent}
                \scnrelfrom{класс команд}{ответ на команду получения соответствующих типов sc-элементов}
                \scntext{примечание}{Содержимое сообщения \textit{команды получения соответствующих типов sc-элементов} может быть пустым.}
                \scnheader{ответ на команду получения соответствующих типов sc-элементов}
                \scnidtf{check elements command answer}
                \scnsubset{ответ на команду на SC-JSON-коде}
                \scnrelfrom{пример}{Пример ответа на команду получения соответствующих типов sc-элементов}
                \begin{scnindent}
                    \scneqimage[10em]{Contents/part_platform/images/check_elements_command_answer_example.png}
                    \scniselement{ответ на команду получения соответствующих типов sc-элементов}
                    \scntext{интерпретация}{Типы sc-элементов 32 и 0 соответственно. Команда обработана успешно.}
                \end{scnindent}
                \scntext{примечание}{Если sc-элемент с указанным хэшем не существует, то его тип будет равен 0.}
                \scnheader{команда удаления sc-элементов}
                \scnidtf{delete elements command}
                \scnsubset{команда на SC-JSON-коде}
                \scnrelfrom{пример}{Пример команды удаления sc-элементов}
                \begin{scnindent}
                    \scneqimage[15em]{Contents/part_platform/images/delete_elements_command_example.png}
                    \scniselement{команда удаления sc-элементов}
                    \scnrelfrom{ответ}{Пример ответа на команду удаления sc-элементов}
                    \scntext{интерпретация}{Удалить sc-элементы с хэшами 885 и 1025.}
                \end{scnindent}
                \scnrelfrom{класс команд}{ответ на команду удаления sc-элементов}
                \scntext{примечание}{Содержимое сообщения \textit{команды удаления sc-элементов} может быть пустым.}
                \scnheader{ответ на команду удаления sc-элементов}
                \scnidtf{delete elements command answer}
                \scnsubset{ответ на команду на SC-JSON-коде}
                \scnrelfrom{пример}{Пример ответа на команду удаления sc-элементов}
                \begin{scnindent}
                    \scneqimage[10em]{Contents/part_platform/images/delete_elements_command_answer_example.png}
                    \scniselement{ответ на команду удаления sc-элементов}
                    \scntext{интерпретация}{Sc-элементы были удалены из sc-памяти успешно.}
                \end{scnindent}
                \scntext{примечание}{Если sc-элемент с указанным хэшем не существует, ответ на команду будет безуспешным.}
                \scnheader{команда обработки ключевых sc-элементов}
                \scnidtf{handle keynodes command}
                \scnsubset{команда на SC-JSON-коде}
                \scnrelfrom{пример}{Пример команды обработки ключевых sc-элементов}
                \begin{scnindent}
                    \scneqimage[10em]{Contents/part_platform/images/handle_keynodes_command_example.png}
                    \scniselement{команда обработки ключевых sc-элементов}
                    \scnrelfrom{ответ}{Пример ответа на команду обработки ключевых sc-элементов}
                    \scntext{интерпретация}{(1) Найти sc-элемент по системному идентификатору \scnqq{any\_system\_identifier}; (2) Разрешить sc-элемент с типом 1 (неуточняемого типа) по системному идентификатору \scnqq{any\_system\_identifier}.}
                \end{scnindent}
                \scnrelfrom{класс команд}{ответ на команду обработки ключевых sc-элементов}
                \scntext{примечание}{Данный класс команд позволяет быстро обращаться к sc-элементам по их системным идентификаторам, поскольку ключевые sc-элементы кэшируются на уровне интерфейса.}
                \scnheader{ответ на команду обработки ключевых sc-элементов}
                \scnidtf{handle keynodes command answer}
                \scnsubset{ответ на команду на SC-JSON-коде}
                \scnrelfrom{пример}{Пример ответа на команду обработки ключевых sc-элементов}
                \begin{scnindent}
                    \scneqimage[10em]{Contents/part_platform/images/handle_keynodes_command_answer_example.png}
                    \scniselement{ответ на команду обработки ключевых sc-элементов}
                    \scntext{интерпретация}{Ключевый sc-элемент с системным идентификатором \scnqq{any\_system\_identifier} не был найден, поэтому был создан. хэш нового ключевого sc-элемента - 128. Команда выполнена успешно.}
                \end{scnindent}
                \scnheader{команда обработки содержимого файлов ostis-системы}
                \scnidtf{handle link contents command}
                \scnsubset{команда на SC-JSON-коде}
                \scnrelfrom{пример}{Пример команды обработки содержимого файлов ostis-системы}
                \begin{scnindent}
                    \scneqimage[10em]{Contents/part_platform/images/handle_link_contents_command_example.png}
                    \scniselement{команда обработки содержимого файлов ostis-системы}
                    \scnrelfrom{ответ}{Пример ответа на команду обработки содержимого файлов ostis-системы}
                    \scntext{интерпретация}{(1) Установить содержимое 67 типа \scnqq{int} в файл ostis-системы с хэшем 3123; (2) Получить содержимое файла ostis-системы с хэшем 232; (3) Найти файлы ostis-системы с содержимым \scnqq{exist}.}
                \end{scnindent}
                \scnrelfrom{класс команд}{ответ на команду обработки содержимого файлов ostis-системы}
                \scntext{примечание}{Стоит отметить, что в случае, если файл ostis-системы уже имеет содержимое, то при установке нового содержимого старое содержимое будет удалено из памяти. Содержимое файла ostis-системы может быть установлено как пустое.}
                \scnheader{ответ на команду обработки содержимого файлов ostis-системы}
                \scnidtf{handle link contents command answer}
                \scnsubset{ответ на команду на SC-JSON-коде}
                \scnrelfrom{пример}{Пример ответа на команду обработки содержимого файлов ostis-системы}
                \begin{scnindent}
                    \scneqimage[10em]{Contents/part_platform/images/handle_link_contents_command_answer_example.png}
                    \scniselement{ответ на команду обработки содержимого файлов ostis-системы}
                    \scntext{интерпретация}{(1) Содержимое 67 типа \scnqq{int} было установлено успешно в файл ostis-системы с хэшем 3123; (2) Содержимое файла ostis-системы с хэшем 232 - число 67 целочисленного типа; (3) Файлы ostis-системы с содержимым \scnqq{exist}: 324 и 423.}
                \end{scnindent}
                \scnheader{команда поиска sc-конструкций, изоморфных заданному sc-шаблону}
                \scnidtf{search template command}
                \scnsubset{команда на SC-JSON-коде}
                \scnrelfrom{пример}{Пример команды поиска sc-конструкций, изоморфных заданному sc-шаблону}
                \begin{scnindent}
                    \scneqimage[10em]{Contents/part_platform/images/search_template_command_example.png}
                    \scniselement{команда поиска sc-конструкций, изоморфных заданному sc-шаблону}
                    \scnrelfrom{ответ}{Пример ответа на команду поиска sc-конструкций, изоморфных заданному sc-шаблону}
                    \scntext{интерпретация}{Найти все такие тройки, в которых первым элементом является sc-элемент c хэшем 23123, третьим sc-элементом является файл ostis-системы неуточняемого константного типа c псевдонимом \scnqq{\_trg}, а вторым элементом - sc-дуга типа sc\_edge\_d\_common c псевдонимом \scnqq{\_edge1}, исходящая от sc-элемента c хэшем 23123 и входящая в файл ostis-системы с псевдонимом \scnqq{\_trg}, и найти все такие тройки, в которых первым элементом является sc-элемент c хэшем 231342, третьим элементов является sc-дуга под псевдонимом \scnqq{\_edge1}, а вторым элементом - sc-дуга типа sc\_edge\_access\_const\_pos\_perm, исходящая от sc-элемента c хэшем 231342 и входящая в sc-дугу \scnqq{\_edge1}. На место переменной с псевдонимом \scnqq{\_trg} подставить sc-элемент с хэшем 564.}
                \end{scnindent}
                \newpage\scnrelfrom{класс команд}{ответ на команду поиска sc-конструкций, изоморфных заданному sc-шаблону}
                \scntext{примечание}{Поиск sс-конструкций по сформированному sc-шаблону осуществляется специализированным модулем, являющимся частью sc-памяти.}
                \scnheader{ответ на команду поиска sc-конструкций, изоморфных заданному sc-шаблону}
                \scnidtf{search template command answer}
                \scnsubset{ответ на команду на SC-JSON-коде}
                \scnrelfrom{пример}{Пример ответа на команду поиска sc-конструкций, изоморфных заданному sc-шаблону}
                \begin{scnindent}
                    \scneqimage[10em]{Contents/part_platform/images/search_template_command_answer_example.png}
                    \scniselement{ответ на команду поиска sc-конструкций, изоморфных заданному sc-шаблону}
                    \scntext{интерпретация}{Найдена одна sc-конструкция, состоящая из двух троек. хэши sc-элементов в тройках: 23123, 4953, 564 и 231342, 533, 4953 соответственно их расположению в тройках. Команда выполнена успешно.}
                \end{scnindent}
                \scntext{примечание}{Важно отметить, что sc-шаблон поиска sc-конструкций не может быть пустым.}
                \scnheader{команда создания sc-конструкции, изоморфной заданному sc-шаблону}
                \scnidtf{generate template command}
                \scnsubset{команда на SC-JSON-коде}
                \scnrelfrom{пример}{Пример команды создания sc-конструкции, изоморфной заданному sc-шаблону}
                \begin{scnindent}
                    \scneqimage[10em]{Contents/part_platform/images/generate_template_command_example.png}
                    \scniselement{команда создания sc-конструкции, изоморфной заданному sc-шаблону}
                    \scnrelfrom{ответ}{Пример ответа на команду создания sc-конструкции, изоморфной заданному sc-шаблону}
                    \scntext{интерпретация}{Создать такую тройку, в которой первым элементом является sc-элемент c хэшем 589, третьим sc-элементом является sc-узел неуточняемого типа c псевдонимом \scnqq{\_trg}, а вторым элементом - sc-дуга типа sc\_edge\_d\_common c псевдонимом \scnqq{\_edge1}, исходящая от sc-элемента c хэшем 589 и входящая в sc-узел с псевдонимом \scnqq{\_trg}. На место переменной с псевдонимом \scnqq{\_trg} подставить sc-элемент с хэшем 332.}
                \end{scnindent}
                \scnrelfrom{класс команд}{ответ на команду создания sc-конструкции, изоморфной заданному sc-шаблону}
                \scntext{примечание}{Создание sс-конструкции по сформированному sc-шаблону осуществляется специализированным модулем, являющимся частью sc-памяти.}
                \scnheader{ответ на команду создания sc-конструкции, изоморфной заданному sc-шаблону}
                \scnidtf{search template command answer}
                \scnsubset{ответ на команду на SC-JSON-коде}
                \scnrelfrom{пример}{Пример ответа на команду создания sc-конструкции, изоморфной заданному sc-шаблону}
                \begin{scnindent}
                    \scneqimage[10em]{Contents/part_platform/images/generate_template_command_answer_example.png}
                    \scniselement{ответ на команду создания sc-конструкции, изоморфной заданному sc-шаблону}
                    \scntext{интерпретация}{Создана одна sc-конструкция, состоящая из одной тройки. хэши sc-элементов в тройке: 128, 589, 332 соответственно их расположению в тройках. Команда выполнена успешно.}
                \end{scnindent}
                \scntext{примечание}{Важно отметить, что sc-шаблон создания sc-конструкции не может быть пустым.}
                \scnheader{команда обработки sc-событий}
                \scnidtf{handle events command}
                \scnsubset{команда на SC-JSON-коде}
                \scnrelfrom{пример}{Пример команды обработки sc-событий}
                \begin{scnindent}
                    \scneqimage[15em]{Contents/part_platform/images/handle_events_command_example.png}
                    \scniselement{команда обработки sc-событий}
                    \scnrelfrom{ответ}{Пример ответа на команду обработки sc-событий}
                    \scntext{интерпретация}{(1) Зарегистрировать sc-событие типа \scnqq{add\_outgoing\_edge} по sc-элементу с хэшем 324; (2) Разрегистрировать sc-события с позициями sc-элементов 2, 4 и 5 соответственно.}
                \end{scnindent}
                \scnrelfrom{класс команд}{ответ на команду обработки sc-событий}
                \scnrelfrom{класс команд}{ответ инициализации sc-события}
                \scntext{примечание}{\textit{Ответ инициализации sc-события} может производиться несколько раз за разные промежутки времени.}
                \scnheader{ответ на команду обработки sc-событий}
                \scnidtf{handle events command answer}
                \scnsubset{ответ на команду на SC-JSON-коде}
                \scnsuperset{SC-JSON-код}
                \scnrelfrom{пример}{Пример ответа на команду обработки sc-событий}
                \begin{scnindent}
                    \scneqimage[10em]{Contents/part_platform/images/handle_events_command_answer_example.png}
                    \scniselement{ответ на команду обработки sc-событий}
                    \scntext{интерпретация}{(1) Sc-событие типа \scnqq{add\_outgoing\_edge} по sc-элементу с хэшем 324 было зарегистрировано успешно на 7-ой позиции очереди sc-событий; (2) Sc-события под позициями 2, 4, 5 удалены успешно.}
                \end{scnindent}
                \scnheader{ответ инициализации sc-события}
                \scnidtf{init event command answer}
                \scnsubset{ответ на команду на SC-JSON-коде}
                \scnrelfrom{пример}{Пример ответа инициализации sc-события}
                \begin{scnindent}
                    \scneqimage[10em]{Contents/part_platform/images/init_event_command_answer_example.png}
                    \scniselement{ответ инициализации sc-события}
                    \scntext{интерпретация}{Sc-событие было инициализировано успешно: добавлена выходящая sc-дуга с хэшем 328 из зарегистрированного sc-элемента с хэшем 324 в sc-элемент c хэшем 35. Статус sc-события - 1.}
                \end{scnindent}
            \end{scnsubstruct}
            \scnsourcecomment{Завершили представление \textit{Синтаксиса SC-JSON-кода}}
            \scnheader{Серверная система на основе Websocket, обеспечивающая доступ к sc-памяти платформы интерпретации sc-моделей при помощи команд SC-JSON-кода}
            \scnidtf{Система, работающая по принципам Websocket и предоставляющая параллельно-асинхронный многоклиентский доступ к sc-памяти платформы интерпретации sc-моделей при помощи SC-JSON-кода}
            \scnidtf{SC-JSON-сервер}
            \scntext{часто используемый неосновной внешний идентификатор sc-элемента}{SC-сервер}
            \scniselement{многократно используемый компонент ostis-систем}
            \scniselement{атомарный многократно используемый компонент ostis-систем}
            \scniselement{зависимый многократно используемый компонент ostis-систем}
            \scnrelto{компонент системы}{Программный вариант реализации платформы интерпретации sc-моделей компьютерных систем}
            \begin{scnrelfromlist}{автор}
                \scnitem{Зотов Н. В.}
            \end{scnrelfromlist}
            \begin{scnrelfromlist}{используемый язык программирования}
                \scnitem{С}
                \scnitem{C++}
            \end{scnrelfromlist}
            \begin{scnrelfromlist}{используемый язык}
                \scnitem{SC-JSON-код}
            \end{scnrelfromlist}
            \begin{scnrelfromlist}{зависимости компонента}
                \scnitem{Библиотека программных компонентов для обработки и, задающее синтаксис json-текстов JSON for Modern C++ версии 3.10.5}
                \begin{scnindent}
                    \scnidtf{nlohmann-json 3.10.5}
                    \scnrelto{версия компонента}{Библиотека программных компонентов для обработки и, задающее синтаксис json-текстов JSON for Modern C++}
                    \begin{scnindent}
                        \scnidtf{nlohmann-json}
                        \scniselement{многократно используемый компонент ostis-систем}
                        \scniselement{неатомарный многократно используемый компонент ostis-систем}
                        \scniselement{зависимый многократно используемый компонент ostis-систем}
                        \scntext{адрес хранилища}{https://github.com/nlohmann/json}
                        \begin{scnindent}
                            \scniselement{адрес хранилища на GitHub}
                        \end{scnindent}
                        \scntext{скрипт установки}{sudo add-apt-repository universe
                            \\sudo apt-get update
                            \\sudo apt-get install -y nlohmann-json3-dev}
                        \begin{scnindent}
                            \scniselement{скрипт на языке Bash}
                            \scniselement{скрипт на языке Bash, поддерживаемый семейством операционных систем UNIX}
                        \end{scnindent}
                        \scntext{скрипт установки}{brew install nlohmann-json}
                        \begin{scnindent}
                            \scniselement{скрипт на языке Bash}
                            \scniselement{скрипт на языке Bash, поддерживаемый семейством операционных систем MaсOS}
                        \end{scnindent}
                    \end{scnindent}
                \end{scnindent}
                \scnitem{Библиотека кросс-платформенных программных компонентов для реализации серверных приложений на основе Websocket WebSocket++ версии 0.8.2}
                \begin{scnindent}
                    \scnidtf{websocketcpp 0.8.2}
                    \scnrelto{версия компонента}{Библиотека кросс-платформенных программных компонентов для реализации серверных приложений на основе Websocket WebSocket++}
                    \begin{scnindent}
                        \scnidtf{websocketcpp}
                        \scniselement{многократно используемый компонент ostis-систем}
                        \scniselement{неатомарный многократно используемый компонент ostis-систем}
                        \scniselement{зависимый многократно используемый компонент ostis-систем}
                        \scntext{адрес хранилища}{https://github.com/zaphoyd/websocketpp}
                        \begin{scnindent}
                            \scniselement{адрес хранилища на GitHub}
                        \end{scnindent}
                        \scntext{скрипт установки}{sudo apt-get update
                            \\sudo apt-get install -y libwebsocketpp-dev}
                        \begin{scnindent}
                            \scniselement{скрипт на языке Bash}
                            \scniselement{скрипт на языке Bash, поддерживаемый семейством операционных систем UNIX}
                        \end{scnindent}
                        \scntext{скрипт установки}{brew install libwebsocketpp}
                        \begin{scnindent}
                            \scniselement{скрипт на языке Bash}
                            \scniselement{скрипт на языке Bash, поддерживаемый семейством операционных систем MaсOS}
                        \end{scnindent}
                    \end{scnindent}
                \end{scnindent}
                \scnitem{Программный компонент настройки программных компонентов ostis-систем}
                \begin{scnindent}
                    \scnidtf{sc-config-utils}
                    \scnrelto{версия компонента}{Программный компонент настройки программных компонентов ostis-систем}
                    \begin{scnindent}
                        \scnidtf{sc-config-utils}
                        \scniselement{многократно используемый компонент ostis-систем}
                        \scniselement{неатомарный многократно используемый компонент ostis-систем}
                        \scniselement{зависимый многократно используемый компонент ostis-систем}
                        \begin{scnrelfromlist}{автор}
                            \scnitem{Зотов Н. В.}
                            \scnitem{Насевич П. Е.}
                            \scnitem{Хорошавин В. Д.}
                        \end{scnrelfromlist}
                        \scntext{адрес хранилища}{https://github.com/ostis-ai/sc-machine/tree/main/sc-tools/sc-config-utils}
                        \begin{scnindent}
                            \scniselement{адрес хранилища на GitHub}
                        \end{scnindent}
                    \end{scnindent}
                \end{scnindent}
                \scnitem{Программная модель sc-памяти версии 0.6.1}
                \scnidtf{sc-machine 0.6.1}
                \begin{scnindent}
                    \scnrelto{версия компонента}{Программная модель sc-памяти}
                \end{scnindent}
            \end{scnrelfromlist}
            \scntext{адрес хранилища}{https://github.com/ostis-ai/sc-machine/tree/main/sc-tools/sc-server}
            \begin{scnindent}
                \scniselement{адрес хранилища на GitHub}
            \end{scnindent}
            \scntext{пояснение}{\textit{Серверная система на основе Websocket, обеспечивающая доступ к sc-памяти платформы интерпретации sc-моделей при помощи команд SC-JSON-кода}, представляет собой интерпретатор команд и ответов на них \textit{SC-JSON-кода} на программное представление sc-конструкций в sc-памяти при помощи Библиотеки программных компонентов для обработки и, задающее синтаксис json-текстов JSON for Modern C++ и Библиотека кросс-платформенных программных компонентов для реализации серверных приложений на основе Websocket WebSocket++, а также обеспечивается комплексным тестовым покрытием посредством программных фреймворков Google Tests и Google Benchmark Tests. Библиотека программных компонентов для обработки и, задающее синтаксис json-текстов JSON for Modern C++ имеет богатый, удобный и быстродействующий функционал, необходимый для реализации подобных компонентов ostis-систем, а Библиотеки кросс-платформенных программных компонентов для реализации серверных приложений на основе Websocket WebSocket++ позволяет элегантно проектировать серверные системы без использовании избыточных зависимостей и решение. Настройка программного компонента осуществляется с помощью \textit{Программного компонента настройки программных компонентов ostis-систем} и скриптов языков CMake и Bash.}
            \scntext{пояснение}{Стоит отметить, что текущая реализация \textit{Серверной системы на основе Websocket, обеспечивающая доступ к sc-памяти платформы интерпретации sc-моделей при помощи команд SC-JSON-кода} не является первой в своём роде и заменяет предыдущую её реализацию, написанную на языке Python. Причиной такой замены состоит в следующем:
                \begin{scnitemize}
                    \item предыдущая реализация \textit{Серверной системы на основе Websocket, обеспечивающая доступ к sc-памяти платформы интерпретации sc-моделей при помощи команд SC-JSON-кода}, реализованная на языке программирования Python, зависит от библиотеки Boost Python, предоставляемой сообществом по развитию и коллаборации языков С++ и Python. Дело в том, что такое решение требует поддержки механизма интерпретации программного кода на языке Python на язык С++, что является избыточным и необоснованным, поскольку большая часть программного кода \textit{\textbf{Программного варианта реализации платформы интерпретации sc-моделей компьютерных систем}} реализована на языках С и С++. Новая реализация описываемой программной системы позволяет избавиться от использования ёмких и ресурсозатратных библиотек (например, boost-python-lib, llvm) и языка Python;
                    \item при реализации распределённых подсистем важную роль играет скорость обработки знаний, то есть возможность быстро и срочно отвечать на запросы пользователя. Качество доступа к sc-памяти посредством реализованной \textit{Подсистемы взаимодействия с sc-памятью на основе языка JSON} должно быть соизмеримо с качеством доступа к sc-памяти при помощи специализированного программного интерфейса API, реализованного на том же языке программирования, что и сама система. Новая реализация позволяет повысить скорость обработки запросов \textit{Подсистемой взаимодействия с sc-памятью на основе языка JSON}, в том числе и обработка знаний, не менее чем в 1,5 раза по сравнению с предыдущим вариантом реализации этой подсистемы.
                \end{scnitemize}
            }
        \end{scnsubstruct}
        \scnsourcecomment{Завершили описание \textit{Подсистемы взаимодействия c sc-памятью на основе языка JSON}}
        \bigskip
    \end{scnsubstruct}
    \scnendsegmentcomment{\textit{Описание реализации подсистемы взаимодействия с внешней средой с использованием сетевых языков}}
    \scnheader{Реализация вспомогательных инструментальных средств для работы с sc-памятью}
    \scnrelfrom{компонент программной системы}{Реализация сборщика базы знаний из исходных текстов, записанных в SCs-коде}
    \begin{scnindent}
        \scnidtf{sc-builder}
        \scnrelfrom{используемый язык}{SCs-код}
        \scntext{пояснение}{Сборщик базы знаний из исходных текстов позволяет осуществить сборку базы знаний из набора исходных текстов, записанных в SCs-коде с ограничениями (см. \textit{Раздел **про исходные тексты**}) в бинарный формат, воспринимаемый \textit{Программной моделью sc-памяти}. При этом возможна как сборка \scnqq{с нуля} (с уничтожением ранее созданного слепка памяти), так и аддитивная сборка, когда информация, содержащаяся в заданном множестве файлов, добавляется к уже имеющемуся слепку состояния памяти.В текущей реализации сборщик осуществляет \scnqq{склеивание} (\scnqq{слияние}) sc-элементов, имеющих на уровне исходных текстов одинаковые \textit{системные sc-идентификаторы}.}
    \end{scnindent}
    \scnheader{Реализация scp-интерпретатора}
    \scnrelto{программная реализация}{Абстрактная scp-машина}
    \scntext{примечание}{Важнейшей особенностью Языка SCP является тот факт, что его программы записываются таким же образом, что и обрабатываемые ими знания, то есть в SC-коде. Это, с одной стороны, дает возможность сделать ostis-системы платформенно-независимыми (четко разделить \textit{sc-модель компьютерной системы} и платформу интерпретации таких моделей), а с другой стороны требует наличия в рамках платформы \textit{Реализации scp-интерпретатора}, то есть интерпретатора программ Языка SCP.}
    \begin{scnrelfromlist}{используемый язык программирования}
        \scnitem{C++}
    \end{scnrelfromlist}
    \begin{scnrelfromlist}{компонент программной системы}
        \scnitem{Реализация Абстрактного sc-агента создания scp-процессов}
        \scnitem{Реализация Абстрактного sc-агента интерпретации scp-операторов}
        \begin{scnindent}
            \begin{scnrelfromlist}{компонент программной системы}
                \scnitem{Реализация Абстрактного sc-агента интерпретации scp-операторов генерации конструкций}
                \scnitem{Реализация Абстрактного sc-агента интерпретации scp-операторов ассоциативного поиска конструкций}
                \scnitem{Реализация Абстрактного sc-агента интерпретации scp-операторов удаления конструкций}
                \scnitem{Реализация Абстрактного sc-агента интерпретации scp-операторов проверки условий}
                \scnitem{Реализация Абстрактного sc-агента интерпретации scp-операторов управления значениями операндов}
                \scnitem{Реализация Абстрактного sc-агента интерпретации scp-операторов управления scp-процессами}
                \scnitem{Реализация Абстрактного sc-агента интерпретации scp-операторов управления событиями}
                \scnitem{Реализация Абстрактного sc-агента интерпретации scp-операторов обработки содержимых числовых файлов}
                \scnitem{Реализация Абстрактного sc-агента интерпретации scp-операторов обработки содержимых строковых файлов}
            \end{scnrelfromlist}
        \end{scnindent}
        \scnitem{Реализация Абстрактного sc-агента синхронизации процесса интерпретации scp-программ}
        \scnitem{Реализация Абстрактного sc-агента уничтожения scp-процессов}
        \scnitem{Реализация Абстрактного sc-агента синхронизации событий в sc-памяти и ее реализации}
    \end{scnrelfromlist}
    \scntext{примечание}{Текущая \textit{Реализация scp-интерпретатора} не включает в себя специализированных средств для работы с блокировками, поскольку механизм блокировок элементов sc-памяти реализован на более низком уровне в рамках \textit{Реализация sc-хранилища и механизма доступа к нему}}
    \scnheader{Реализация интерпретатора sc-моделей пользовательских интерфейсов}
    \scnidtf{sc-web}
    \scntext{пояснение}{Наряду с реализацией \textit{Программной модели sc-памяти} важной частью \textit{Программного варианта реализации платформы интерпретации sc-моделей компьютерных систем} является \textit{Реализация интерпретатора sc-моделей пользовательских интерфейсов}, которая предоставляет базовые средства просмотра и редактирования базы знаний пользователем, средства для навигации по базе знаний (задания вопросов к базе знаний) и может дополняться новыми компонентами в зависимости от задач, решаемых каждой конкретной ostis-системой.}
    \begin{scnrelfromlist}{используемый язык программирования}
        \scnitem{JavaScript}
        \scnitem{TypeScript}
        \scnitem{Python}
    \end{scnrelfromlist}
    \begin{scnindent}
        \scntext{пояснение}{На данной иллюстрации показан планируемый вариант архитектуры \textit{Реализация интерпретатора sc-моделей пользовательских интерфейсов}, важным принципом которой является простота и однотипность подключения любых компонентов пользовательского интерфейса (редакторов, визуализаторов, переключателей, команд меню и т.д.). Для этого реализуется программная прослойка Sandbox, в рамках которой реализуются низкоуровневые операции взаимодействия с серверной частью и которая обеспечивает более удобный программный интерфейс для разработчиков компонентов.}
    \end{scnindent}
    \begin{scnrelfromlist}{недостатки текущей реализации}
        \scnfileitem{Отсутствие единого унифицированного механизма клиент-серверного взаимодействия. Часть компонентов (визуализатор sc-текстов в SCn-коде, команды меню и др.) работают по протоколу HTTP, часть по протоколу SCTP с использованием технологии WebSocket, это приводит к значительным трудностям при развитии платформы.}
        \scnfileitem{Протокол HTTP предполагает четкое разделение активного клиента и пассивного сервера, который отвечает на запросы клиентов. Таким образом, сервер (в данном случае --- sc-память) практически не имеет возможности по своей инициативе отправить сообщение клиенту, что повышает безопасность системы, но значительно снижает ее интерактивность. Кроме того, такой вариант реализации затрудняет реализацию принятого в Технологии OSTIS многоагентного подхода, в частности, затрудняет реализацию sc-агентов на стороне клиента. Указанные проблемы могут быть решены путем постоянного мониторинга определенных событий со стороны клиента, однако такой вариант неэффективен.Кроме того, часть интерфейса фактически работает напрямую с sc-памятью с использованием технологии WebSocket, а часть --- через прослойку на базе библиотеки tornado для языка программирования Python, что приводит к дополнительным зависимостям от сторонних библиотек.}
        \scnfileitem{Часть компонентов (например, поле поиска по идентификатору) реализована сторонними средствами и практически никак не связана с sc-памятью. Это затрудняет развитие платформы.}
        \scnfileitem{Текущая \textit{Реализация интерпретатора sc-моделей пользовательских интерфейсов} ориентирована только на ведение диалога с пользователем (в стиле вопрос пользователя --- ответ системы). Не поддерживаются такие очевидно необходимые ситуации, как выполнение команды, не предполагающей ответа;~возникновение ошибки или отсутствие ответа;~необходимость задания вопроса системой пользователю и т.д.}
        \scnfileitem{Ограничена возможность взаимодействия пользователя с системой без использования специальных элементов управления. Например, можно задать вопрос системе, нарисовав его в SCg-коде, но ответ пользователь не увидит, хотя в памяти он будет сформирован соответствующим агентом.;Большая часть технологий, использованных при реализации платформы, к настоящему моменту устарела, что затрудняет развитие платформы.}
        \scnfileitem{Идея платформенной независимости пользовательского интерфейса (построения sc-модели пользовательского интерфейса) реализована не в полной мере. Полностью описать sc-модель пользовательского интерфейса (включая точное размещение, размеры, дизайн компонентов, их поведение и др.) в настоящее время скорее всего окажется затруднительно из-за ограничений производительности, однако вполне возможно реализовать возможность задания вопросов ко всем компонентам интерфейса, изменить их расположение и т.д., однако эти возможности нельзя реализовать в текущей версии реализации платформы.}
        \scnfileitem{Интерфейсная часть работает медленно из-за некоторых недостатков реализации серверной части на языке Python.}
        \scnfileitem{Не реализован механизм наследования при добавлении новых внешних языков. Например, добавление нового языка даже очень близкого к SCg-коду требует физического копирования кода компонента и внесение соответствующих изменений, при этом получаются два никак не связанных между собой компонента, которые начинают развиваться независимо друг от друга.}
        \scnfileitem{Слабый уровень задокументированности текущей \textit{Реализации интерпретатора sc-моделей пользовательских интерфейсов}.}
    \end{scnrelfromlist}
    \begin{scnrelfromlist}{требования к будущей реализации}
        \scnfileitem{Унифицировать принципы взаимодействия всех компонентов интерфейса с \textit{Программной моделью sc-памяти}, независимо от того, к какому типу относится компонент. Например, список команд меню должен формироваться через тот же механизм, что и ответ на запрос пользователя, и команда редактирования, сформированная пользователем, и команда добавления нового фрагмента в базу знаний и т.д.}
        \scnfileitem{Унифицировать принципы взаимодействия пользователей с системой независимо от способа взаимодействия и внешнего языка. Например, должна быть возможность задания вопросов и выполнения других команд прямо через SCg/SCn интерфейс. При этом необходимо учитывать принципы редактирования базы знаний, чтобы пользователя не мог под видом задания вопроса внести новую информацию в согласованную часть базы знаний.}
        \scnfileitem{Унифицировать принципы обработки событий, происходящих при взаимодействии пользователя с компонентами интерфейса --- поведение кнопок и других интерактивных компонентов должно задаваться не статически сторонними средствами, а реализовываться в виде агента, который, тем не менее, может быть реализован произвольным образом (не обязательно на платформенно-независимом уровне). Любое действие, совершаемое пользователем, на логическом уровне должно трактоваться и обрабатываться как инициирование агента.}
        \scnfileitem{Обеспечить возможность выполнять команды (в частности, задавать вопросы) с произвольным количеством аргументов, в том числе --- без аргументов.}
        \scnfileitem{Обеспечить возможность отображения ответа на вопрос по частям, если ответ очень большой и для отображения требуется много времени.}
        \scnfileitem{Каждый отображаемый компонент интерфейса должен трактоваться как изображение некоторого sc-узла, описанного в базе знаний. Таким образом, пользователь должен иметь возможность задания произвольных вопросов к любым компонентам интерфейса.}
        \scnfileitem{Максимально упростить и задокументировать механизм добавления новых компонентов.}
        \scnfileitem{Обеспечить возможность добавления новых компонентов на основе имеющихся без создания независимых копий. Например, должна быть возможность создать компонент для языка, расширяющего язык SCg новыми примитивами, переопределять принципы размещения sc-текстов и т.д.}
        \scnfileitem{Свести к минимуму зависимость от сторонних библиотек.}
        \scnfileitem{Свести к минимуму использование протокола HTTP (начальная загрузка общей структуры интерфейса), обеспечить возможность равноправного двустороннего взаимодействия серверной и клиентской части.}
    \end{scnrelfromlist}
    \scntext{примечание}{Очевидно, что реализация большинства из приведенных требований связана не только с собственно вариантом реализации платформы, но и требует развития теории логико-семантических моделей пользовательских интерфейсов и уточнения в рамках нее общих принципов организации пользовательских интерфейсов ostis-систем. Однако, принципиальная возможность реализации таких моделей должна быть учтена в рамках реализации платформы.}
    \begin{scnrelfromlist}{компоненты программной системы}
        \scnitem{Панель меню команд пользовательского интерфейса}
        \begin{scnindent}
            \scntext{пояснение}{\textit{Панель меню команд пользовательского интерфейса} содержит изображения классов команд (как атомарных, так и неатомарных), имеющихся на данный момент в базе знаний и входящих в декомпозицию \textit{Главного меню пользовательского интерфейса} (имеется в виду полная декомпозиция, которая в общем случае может включать несколько уровней неатомарных классов команд).
                \\Взаимодействие с изображением неатомарного класса команд инициирует команду изображения классов команд, входящих в декомпозицию данного неатомарного класса команд.
                \\Взаимодействие с изображением атомарного класса команд инициирует генерацию команды данного класса с ранее выбранными аргументами на основе соответствующей \textit{обобщенной формулировки класса команд} (шаблона класса команд).}
        \end{scnindent}
        \scnitem{Компонент переключения языка идентификации отображаемых sc-элементов}
        \begin{scnindent}
            \scntext{пояснение}{\textit{Компонент переключения языка идентификации отображаемых sc-элементов} является изображением множества имеющихся в системе естественных языков. Взаимодействие пользователя с данным компонентом переключает пользовательский интерфейс в режим общения с конкретным пользователем с использованием \textit{основных sc-идентификаторов}, принадлежащих данному \textit{естественному языку}. Это значит, что при изображении sc-идентификаторов sc-элементов на каком-либо языке, например, SCg-коде или SCn-коде будут использоваться \textit{основные sc-идентификаторы}, принадлежащие данному \textit{естественному языку}. Это касается как sc-элементов, отображаемых в рамках \textit{Панели визуализации и редактирования знаний}, так и любых других sc-элементов, например, классов команд и даже самих \textit{естественных языков}, изображаемых в рамках самого \textit{Компонента переключения языка идентификации отображаемых sc-элементов}.}
        \end{scnindent}
        \scnitem{Компонент переключения внешнего языка визуализации знаний}
        \begin{scnindent}
            \scntext{пояснение}{\textit{Компонент переключения внешнего языка визуализации знаний} служит для переключения языка визуализации знаний в текущем окне, отображаемом на \textit{Панели визуализации и редактирования знаний}. В текущей реализации в качестве таких языков по умолчанию поддерживаются SCg-код и SCn-код, а также любые другие языки, входящие во множество \textit{внешних языков визуализации SC-кода}.}
        \end{scnindent}
        \scnitem{Поле поиска sc-элементов по идентификатору}
        \begin{scnindent}
            \scntext{пояснение}{\textit{Поле поиска sc-элементов по идентификатору} позволяет осуществлять поиск \mbox{sc-идентификаторов}, содержащих подстроку, введенную в данное поле (с учетом регистра). В результате поиска отображается список sc-идентификаторов, содержащих указанную подстроку, при взаимодействии с которыми осуществляется автоматическое задание вопроса\scnqqi{Что это такое?}, аргументом которого является либо для сам sc-элемент, имеющий данный sc-идентификатор (в случае, если указанный sc-идентификатор является основным или системным, и, таким образом, указанный sc-элемент может быть определен однозначно), либо для самого внутреннего файла ostis-системы, являющегося sc-идентификатором (в случае, если данный sc-идентификатор является неосновным).}
        \end{scnindent}
        \scnitem{Панель отображения диалога пользователя с ostis-системой}
        \begin{scnindent}
            \scntext{пояснение}{\textit{Панель отображения диалога пользователя с ostis-системой} отображает упорядоченный по времени список sc-элементов, являющихся знаками действий, которые инициировал пользователь в рамках диалога с ostis-системой путем взаимодействия с изображениями соответствующих классов команд (то есть, если действие было инициировано другим способом, например, путем его явного инициирования через создание дуги принадлежности множеству \textit{инициированных действий} в sc.g-редакторе, то на данной панели оно отображено не будет). При взаимодействии пользователя с любым из изображенных знаков действий на \textit{Панели визуализации и редактирования знаний} отображается окно, содержащее результат выполнения данного \textit{действия} на том языке визуализации, на котором он был отображен, когда пользователь просматривал его в последний (предыдущий) раз. Таким образом, в текущей реализации данная панель может работать только в том случае, если инициированное пользователем действие предполагает явно представленный в памяти результат данного действия. В свою очередь, из этого следует, что в настоящее время данная панель, как и в целом \textit{Реализация интерпретатора sc-моделей пользовательских интерфейсов}, позволяет работать с системой только в режиме диалога \scnqqi{вопрос-ответ}.}
        \end{scnindent}
        \scnitem{Панель визуализации и редактирования знаний}
        \begin{scnindent}
            \scntext{пояснение}{\textit{Панель визуализации и редактирования знаний} отображает окна, содержащие sc-текст, представленный на некотором языке из множества \textit{внешних языков визуализации SC-кода} и, как правило, являющийся результатом некоторого действия, инициированного пользователем. Если соответствующий визуализатор поддерживает возможность редактирования текстов соответствующего естественного языка, то он одновременно является также и редактором.}
            \begin{scnrelfromlist}{компонент программной системы}
                \scnitem{Визуализатор sc.n-текстов}
                \scnitem{Визуализатор и редактор sc.g-текстов}
            \end{scnrelfromlist}
            \scntext{примечание}{При необходимости пользовательский интерфейс каждой конкретной ostis-системы может быть дополнен визуализаторами и редакторами различных внешних языков, которые в текущей версии \textit{Реализации интерпретатора sc-моделей пользовательских интерфейсов} будут также располагаться на \textit{Панели визуализации и редактирования знаний}.}
        \end{scnindent}
    \end{scnrelfromlist}
	\end{scnsubstruct}
    \bigskip
\end{scnsubstruct}
\scnendcurrentsectioncomment
\end{SCn}


%\scsectionfamily{Часть 7 Стандарта OSTIS. Методы и средства реинжиниринга и эксплуатации интеллектуальных компьютерных систем нового поколения}
\label{part_reengineering}

\scsection{Предметная область и онтология методов и средств поддержки жизненного цикла ostis-систем}
\label{sd_method_means_operation}

\scsubsection{Предметная область и онтология методов и средств реинжиниринга ostis-систем в ходе эксплуатации}
\label{sd_method_means_operation_reengineering}

\scsubsection{Предметная область и онтология встроенных ostis-систем поддержки использования ostis-систем конечными пользователями}
\label{sd_embedded_sys_support_operation}

%\scsectionfamily{Часть 8 Стандарта OSTIS. Экосистема интеллектуальных компьютерных систем нового поколения и их пользователей}
\label{part_ecosystem}

\scsection[
    \protect\scneditor{Загорский А.Г.}
    \protect\scnmonographychapter{Глава 7.2. Экосистема интеллектуальных компьютерных систем нового поколения (Экосистема OSTIS) и реализация рынка знаний на ее основе}
    \protect\scnmonographychapter{Глава 7.3. Структура Экосистемы OSTIS}
    \protect\scnmonographychapter{Глава 7.4. Интеграция Экосистемы OSTIS с современными сервисами и информационными ресурсами}
    ]{Предметная область и онтология Экосистемы OSTIS}
\label{sd_ostis_ecosystem}
\begin{SCn}
    \scnsectionheader{Предметная область и онтология Экосистемы OSTIS}
    \begin{scnsubstruct}
        \begin{scnrelfromlist}{дочерний раздел}
            \scnitem{\nameref{sd_learning}}
            \scnitem{\nameref{sd_assistants}}
            \scnitem{\nameref{sd_portals}}
            \scnitem{\nameref{sd_ecosys_enterprise}}
        \end{scnrelfromlist}
    
    \begin{scnrelfromlist}{соавторы}
        \scnitem{Загорский А.~С.}
        \scnitem{Голенков В.~В.}
        \scnitem{Шункевич Д.~В.}
    \end{scnrelfromlist}
        
    \begin{scnrelfromlist}{библиографическая ссылка}
        \scnitem{\scncite{Zagorskiy2022a}}
        \scnitem{\scncite{Van2005}}
        \scnitem{\scncite{Mack2001}}
        \scnitem{\scncite{Ameri2005}}
        \scnitem{\scncite{Gerhard2017}}
        \scnitem{\scncite{Meurisch2017}}
        \scnitem{\scncite{Meurisch2020}}
        \scnitem{\scncite{Jeni2022}}
        \scnitem{\scncite{Akbar2022}}
        \scnitem{\scncite{Briscoe2008}}
        \scnitem{\scncite{Boley2007}}
        \scnitem{\scncite{Masahary2018}}
        \scnitem{\scncite{Mohseni2021}}
        \scnitem{\scncite{Berners2001}}
        \scnitem{\scncite{Kiranne2018}}
        \scnitem{\scncite{Mccooi2006}}
        \scnitem{\scncite{Nacer2014}}
        \scnitem{\scncite{Burstrom2022}}
    \end{scnrelfromlist}

    \scnheader{Предметная область Экосистемы OSTIS}
    \scniselement{предметная область}
    \begin{scnhaselementrole}{максимальный класс объектов исследования}
        {Экосистема OSTIS}
    \end{scnhaselementrole}
    \begin{scnhaselementrolelist}{класс объектов исследования}
        \scnitem{ostis-система}
        \scnitem{самостоятельная ostis-система}
        \scnitem{поддержка совместимости между компьютерными системами и их пользователями в Экосистеме OSTIS}
        \scnitem{распределенная система}
	    \scnitem{цифровая экосистема}
        \scnitem{встроенная ostis-система}
        \scnitem{коллектив ostis-систем}
        \scnitem{Корпоративная система Экосистемы OSTIS}
        \scnitem{агент Экосистемы OSTIS}
        \scnitem{пользователь Экосистемы OSTIS}
        \scnitem{ostis-сообщество}
    \end{scnhaselementrolelist}
    \scntext{аннотация}{Рассмотрена структура \textit{цифровой экосистемы} \textit{интеллектуальных компьютерных систем} на основе \textit{Технологии OSTIS}. Уточнена формальная трактовка таких понятий как \textit{ostis-система}, \textit{ostis-сообщество}, выделена типология \textit{ostis-систем}, что в совокупности позволяет определить структуру \textit{Экосистемы OSTIS}.}    
    \scntext{аннотация}{Рассмотрена архитектура \textit{Экосистемы OSTIS} и ее основных компонентов, методы разработки коллективов \textit{ostis-систем}, а также типология \textit{ostis-систем}, входящих в состав \textit{Экосистемы OSTIS}. }
    \scntext{цель}{Предоставить полное представление о возможностях создания \textit{цифровых экосистем} на примере \textit{Экосистемы OSTIS}.}

    \scnheader{Проект OSTIS}
    \scnidtf{Проект, направленный на создание \textit{Технологии OSTIS} и, в частности, на разработку \textit{Стандарта OSTIS}}
    \begin{scnrelfromlist}{продукт}
        \scnitem{Технология OSTIS}
        \scnitem{Метасистема OSTIS}
            \begin{scnindent}
                \scnidtf{Метасистема OSTIS}
            \end{scnindent}
        \scnitem{Стандарт OSTIS}
        \scnitem{Экосистема OSTIS}
    \end{scnrelfromlist}
    \begin{scnrelfromlist}{подпроект}
        \scnitem{Проект разработки Технологии OSTIS}
        \scnitem{Проект разработки Метасистемы OSTIS}
        \scnitem{Проект разработки Стандарта OSTIS}
        \scnitem{Проект разработки Экосистемы OSTIS}
    \end{scnrelfromlist}
    \begin{scnrelfromlist}{библиографический источник}
        \scnitem{\cite{DeNicola2021}}
        \scnitem{\cite{Alrehaili2021}}
        \scnitem{\cite{Alrehaili2017}}
        \scnitem{\cite{Shahzad2021}}
    \end{scnrelfromlist}
    
    \scnheader{Экосистема OSTIS}
    \scnidtf{Социотехническая экосистема, представляющая собой коллектив взаимодействующих семантических компьютерных систем и осуществляющая перманентную поддержку эволюции и семантической совместимости всех входящих в нее систем, на протяжении всего их жизненного цикла.}
    \scnidtf{Неограниченно расширяемый коллектив постоянно эволюционируемых семантических компьютерных систем, которые взаимодействуют между собой и с пользователями для корпоративного решения сложных задач и для постоянной поддержки высокого уровня совместимости и взаимопонимания во взаимодействии как между собой, так и с пользователями.}
    \scntext{пояснение}{Поскольку \textit{Технология OSTIS} ориентирована на разработку \textit{семантических компьютерных систем}, обладающих высоким уровнем \textit{обучаемости} и, в частности, высоким уровнем семантической \textit{совместимости}, и поскольку обучаемость и совместимость есть только \uline{способность} к обучению (т.е. к высоким темпам расширения и совершенствования своих знаний и навыков), а также \uline{способность} к обеспечению высокого уровня взаимопонимания (согласованности), необходима некая среда, социотехническая инфраструктура, в рамках которой были бы созданы максимально комфортные условия для реализации указанных выше способностей. Такая среда названа нами \textit{\textbf{Экосистемой OSTIS}}, которая представляет собой коллектив взаимодействующих (через сеть Интернет):
        \begin{scnitemize}
            \item самих \textit{ostis-систем};
            \item пользователей указанных \textit{ostis-систем} (как конечных пользователей, так и разработчиков);
            \item некоторых компьютерных систем, не являющихся \textit{ostis-системами}, но рассматриваемых ими в качестве дополнительных информационных ресурсов или сервисов.
        \end{scnitemize}}
    \scntext{цель}{Обеспечить постоянную поддержку совместимости компьютерных систем, входящих в \textit{Экосистему OSTIS} как на этапе их разработки, так и в ходе их эксплуатации. Проблема здесь заключается в том, что в ходе эксплуатации систем, входящих в \textit{Экосистему OSTIS}, они могут изменяться из-за чего совместимость может нарушаться.}
    \begin{scnrelfromlist}{задачи}
        \scnitem{оперативное внедрение всех согласованных изменений стандарта \textit{ostis-систем} (в том числе, и изменений систем используемых понятий и соответствующих им терминов)}
        \scnitem{перманентная поддержка высокого уровня взаимопонимания всех систем, входящих в \textit{Экосистему OSTIS}, и всех их пользователей}
        \scnitem{корпоративное решение различных сложных задач, требующих координации деятельности нескольких (чаще всего, априори неизвестных) \textit{ostis-систем}, а также, возможно, некоторых пользователей}
    \end{scnrelfromlist}
    \scntext{примечание}{\textit{Экосистема OSTIS} --- это переход от самостоятельных (автономных, отдельных, целостных) \textit{ostis-систем} к коллективам самостоятельных \textit{ostis-систем}, т.е. к распределенным \textit{ostis-системам}.}
    
    \scnheader{цифровая экосистема}
    \begin{scnrelfromlist}{примечание}
        \scnfileitem{Понятие \textit{цифровой экосистемы} представляет собой сложную и динамичную систему, которая состоит из множества компонентов, включая технологии, процессы, пользователей, предприятия и многое другое. В контексте цифровых технологий переход к \textit{цифровой экосистеме} является ключевым аспектом для достижения целей бизнеса и общества.}
        \scnfileitem{Понятие \textit{цифровой экосистемы} можно определить как совокупность цифровых продуктов и сервисов, которые взаимодействуют друг с другом и с внешней средой, образуя единую среду обитания. Реализация \textit{цифровой экосистемы} сильно связано с формированием \textit{распределенной системы}. Такой принцип реализации имеет как преимущества (высокий уровень адаптивности, устойчивости, связности), так и недостатки (неоптимальность, неуправляемость, непредсказуемость поведения) (см. \scncite{Briscoe2008}, \scncite{Boley2007}, \scncite{Burstrom2022}). В отличие от полностью иерархически контролируемых систем, \textit{цифровая экосистема} представляет собой децентрализованную структуру, которая обеспечивает более гибкое и устойчивое управление (см. \scncite{Masahary2018}).}
        \begin{scnindent}
            \begin{scnrelfromset}{источник}
                \scnitem{\scncite{Briscoe2008}}
                \scnitem{\scncite{Boley2007}}
                \scnitem{\scncite{Burstrom2022}}
                \scnitem{\scncite{Masahary2018}}
            \end{scnrelfromset}
        \end{scnindent}
        \scnfileitem{При традиционных подходах к решению проблемы формирования \textit{цифровой экосистемы} возникают проблемы, связанные с низким уровнем \textit{интероперабельности} таких систем (см. \scncite{Li2012a}). Традиционные подходы к решению данной проблемы зачастую неэффективны, поскольку каждая из систем имеет свой специализированный программный интерфейс и формат данных для взаимодействия. Это приводит к дополнительным расходам на устранение недостатков таких проблем. Поддержка жизненного цикла и модификация уже существующих систем может также потребовать дополнительных временных и ресурсных затрат.}
        \begin{scnindent}
            \begin{scnrelfromset}{источник}
                \scnitem{\scncite{Li2012a}}
            \end{scnrelfromset}
        \end{scnindent}
        \scnfileitem{Использование современных подходов к формированию \textit{цифровой экосистемы}, таких как открытые стандарты и протоколы взаимодействия, может значительно упростить задачу обеспечения \textit{интероперабельности} между различными системами. Это позволяет повысить эффективность и экономическую целесообразность проектов цифровой трансформации, снизить временные и финансовые затраты на разработку и поддержку \textit{цифровой экосистемы} (см. \scncite{Mohseni2021}). Однако стоит отметить, что даже при принятии идей семантического веба (см. \scncite{Berners2001}) могут возникнуть некоторые проблемы или ограничения, которые необходимо учитывать (см. \scncite{Kiranne2018}, \scncite{Mccooi2006}, \scncite{Nacer2014}).}
        \begin{scnindent}
            \begin{scnrelfromset}{источник}
                \scnitem{\scncite{Mohseni2021}}
                \scnitem{\scncite{Berners2001}}
                \scnitem{\scncite{Kiranne2018}}
                \scnitem{\scncite{Mccooi2006}}
                \scnitem{\scncite{Nacer2014}}
            \end{scnrelfromset}
        \end{scnindent}
        \scnfileitem{\textit{Технология OSTIS} предоставляет возможности для создания \textit{цифровых экосистем}. Она обеспечивает эффективное управление данными и знаниями, обеспечивает автоматическую обработку информации и позволяет создавать интеллектуальные системы, способные обмениваться данными и знаниями между собой.}       
        \scnfileitem{Для создания успешной \textit{цифровой экосистемы} необходимо решать множество проблем, связанных с обеспечением высокого уровня интероперабельности между самостоятельно действующими системами. Одним из возможных решений является переход к универсальным сообществам индивидуальных \textit{интеллектуальных кибернетических систем}, которые объединяются в \textit{многоагентные системы}. Реализация такого универсального сообщества интероперабельных интеллектуальных кибернетических систем может осуществляться в виде глобальной Экосистемы OSTIS.}
        \begin{scnindent}
            \begin{scnrelfromset}{источник}
                \scnitem{\scncite{Zagorskiy2022a}}
            \end{scnrelfromset}
        \end{scnindent}
    \end{scnrelfromlist}

    \scnheader{ostis-система}
    \scntext{примечание}{Система, построенная в соответствии с требованиями и стандартами Технологии OSTIS, определяется как ostis-система.}
    \begin{scnsubdividing}
        \scnitem{самостоятельная ostis-система}
            \begin{scnindent}
                \scnidtf{целостная \textit{ostis-система}, которая должна самостоятельно решать соответствующее множество задач и, в частности, взаимодействовать с внешней средой (как вербально --- с пользователями и другими компьютерными системами, так и невербально)}
            \end{scnindent}
        \scnitem{встроенная ostis-система}
            \begin{scnindent}
                \scnidtf{интеллектуальная компьютерная подсистема, разработанная по \textit{Технологии OSTIS} и реализующая часть функционала \textit{ostis-системы} более высокого уровня иерархии}
                \scnidtf{\textit{ostis-система}, интегрированная в состав \textit{самостоятельной ostis-системы}}
                \begin{scnsubdividing}
                    \scnitem{атомарная встроенная ostis-система}
                        \begin{scnindent}
                        \scnidtf{\textit{встроенная ostis-система}, не включающая в себя какие-либо другие \textit{встроенные ostis-системы}}
                        \end{scnindent}
                    \scnitem{неатомарная встроенная ostis-система}
                        \begin{scnindent}
                            \scnsuperset{интерфейс ostis-системы}
                        \end{scnindent}
                \end{scnsubdividing}
            \end{scnindent}
        \scnitem{коллектив ostis-систем}
            \begin{scnindent}
                \scnidtf{группа общающихся ostis-систем, в состав которой могут входить не только самостоятельные ostis-системы, но и коллективы ostis-систем}
                \scnidtf{распределенная ostis-система}
            \end{scnindent}
    \end{scnsubdividing}
    \scntext{примечание}{В рамках \textit{Экосистемы OSTIS} \textit{ostis-системы} способны коммуницировать друг с другом и формировать специализированные коллективы для коллективного решения сложных задач. Такой подход не только повышает уровень интеллекта каждой \textit{индивидуальной кибернетической системы}, но и обеспечивает более эффективное взаимодействие между ними в рамках единой \textit{цифровой экосистемы}. Это обеспечивает существенное развитие целого ряда свойств каждой \textit{компьютерной системы}, позволяющих значительно повысить \textit{уровень интеллекта} (и, прежде всего, их \textit{уровень обучаемости} и \textit{уровень социализации}).}
  
    \scnheader{самостоятельная ostis-система}
    \scntext{пояснение}{Подчеркнем, что к \textit{\textbf{самостоятельным ostis-системам}}, входящим в состав \textit{Экосистемы OSTIS}, предъявляются особые требования:
        \begin{scnitemize}
            \item Они должны обладать всеми необходимыми знаниями и навыками для обмена сообщениями и целенаправленной организации взаимодействия с другими \textit{ostis-системам}и, входящими в \textit{Экосистему OSTIS}.
            \item В условиях постоянного изменения и эволюции \textit{ostis-систем}, входящих в \textit{Экосистему OSTIS}, каждая из них должна \uline{сама следить за состоянием своей совместимости} (согласованности) со всеми остальными \textit{ostis-системами},  т.е. должна самостоятельно поддерживать эту совместимость, согласовывая с другими ostis-системами все требующие согласования изменения, происходящие у себя и в других системах.
            \item Каждая система, входящая в состав \textit{Экосистемы OSTIS}, должна:
            \begin{scnitemizeii}
                \item Интенсивно, активно и целенаправленно обучаться (как с помощью учителей-разработчиков, так и самостоятельно).
                \item Сообщать всем другим системам о предлагаемых или окончательно утвержденных изменениях в \textit{онтологиях} и, в частности, в наборе используемых \textit{понятий}.
                \item Принимать от других \textit{ostis-систем} предложения об изменениях в \textit{онтологиях} (в том числе в наборе используемых понятий) для согласования или утверждения этих предложений.
                \item Реализовывать утвержденные изменения в \textit{онтологиях}, хранимых в ее базе знаний.
                \item Способствовать поддержанию высокого уровня семантической совместимости не только с другими \textit{ostis-системами}, входящими в \textit{Экосистему OSTIS}, но и со своими \textit{пользователями} ( т.е. обучать их, информировать их об изменениях в онтологиях).
            \end{scnitemizeii}
        \end{scnitemize}}
    
    \scnheader{Экосистема OSTIS}
    \scntext{пояснение}{\textit{Экосистема OSTIS} является формой реализации, совершенствования и применения \textit{Технологии OSTIS} и, следовательно, является формой создания, развития, самоорганизации рынка семантически совместимых компьютерных систем  и включает в себя все необходимые для этого ресурсы ---  информационные, технологические, кадровые, организационные, инфраструктурные. \textit{Экосистеме OSTIS} ставится в соответствие ее \textit{\textbf{объединенная база знаний}}, которая представляет собой \textbf{виртуальное объединение} \textit{баз знаний} всех \textit{ostis-систем}, входящих в состав \textit{Экосистемы OSTIS}. Качество этой \textit{базы знаний} (полнота, непротиворечивость, чистота) является постоянной заботой всех самостоятельных \textit{ostis-систем}, входящих в состав \textit{Экосистемы OSTIS}. Соответственно этому каждой указанной \textit{ostis-системе} ставится в соответствие своя \textit{база знаний} и своя иерархическая система \textit{sc-агентов}. По назначению \textit{ostis-системы}, входящие в \textit{Экосистему OSTIS}, могут быть:
        \begin{scnitemize}
            \item ассистентами конкретных пользователей или конкретных пользовательских коллективов;
            \item типовыми встраиваемыми подсистемами \textit{ostis-систем};
            \item системами информационной и инструментальной поддержки проектирования различных компонентов и различных классов \textit{ostis-систем};
            \item системами информационной и инструментальной поддержки проектирования или производства различных классов технических и других искусственно создаваемых систем;
            \item порталами знаний по самым различным научным дисциплинам;
            \item системами автоматизации управления различными сложными объектами (производственными предприятиями, учебными заведениями, кафедрами вузов, конкретными обучаемыми);
            \item интеллектуальными справочными и help-системами;
            \item интеллектуальными обучающими системами, семантическими электронными учебными пособиями;
            \item интеллектуальными робототехническими системами.
        \end{scnitemize}}
    \scntext{примечание}{\textit{Экосистема OSTIS} является максимальным \textit{иерархическим ostis-сообществом}, обеспечивающим комплексную автоматизацию всех видов человеческой деятельности. Оно не может входить в состав какого-либо другого \textit{ostis-сообщества}.}
    \scntext{примечание}{Качество \textit{Экосистемы OSTIS} во многом определяется эффективностью взаимодействия каждой \textit{ostis-системы} (в том числе каждого \textit{ostis-сообщества}), каждого человека со своей внешней средой, а также качеством и чистотой самой внешней среды. Потому основной целью \textit{Экосистемы OSTIS} является повышение качества информационной внешней среды для всех субъектов, входящих в состав \textit{Экосистемы OSTIS}. Иными словами, \textit{Экосистема OSTIS} обеспечивает поддержку информационной экологии человеческого общества.}
    \begin{scnrelfromvector}{уровни иерархии}
        \scnitem{индивидуальные \textit{кибернетические системы} (индивидуальные \textit{ostis-системы} и люди, являющиеся конечными пользователями \textit{ostis-систем})}
        \scnitem{иерархическая система \textit{ostis-сообществ}, членами каждого из которых могут быть \textit{индивидуальные ostis-системы}, люди, а также другие \textit{ostis-сообщества}}
        \scnitem{\textit{Максимальное ostis-сообщество} \textit{Экосистемы OSTIS}, не являющееся членом никакого другого \textit{ostis-сообщества}, входящего в состав \textit{Экосистемы OSTIS}}
    \end{scnrelfromvector}
    \scntext{примечание}{\textit{Технология OSTIS} позволяет создавать семантически совместимые системы, которые способны обрабатывать запросы и задачи пользователей, учитывая их контекст и смысл. Это достигается за счет использования семантических сетей, которые позволяют описывать знания и связи между ними. Кроме того, \textit{технология OSTIS} обеспечивает масштабируемость и гибкость системы, что позволяет ей адаптироваться к изменениям в поведении пользователей и изменениям в их потребностях.}

    \scnheader{участники коллектива Экосистемы OSTIS}
    \begin{scnrelfromlist}{характеристики}
        \scnitem{\textit{семантически совместимые}}
        \scnitem{постоянно эволюционирующие индивидуально}
        \scnitem{постоянно поддерживающие свою совместимость с другими участниками в ходе своей индивидуальной эволюции}
        \scnitem{способные децентрализованно координировать свою деятельность}
    \end{scnrelfromlist}
    
    \scnheader{поддержка совместимости между компьютерными системами и их пользователями в Экосистеме OSTIS}
    \scntext{пояснение}{Есть три аспекта поддержки совместимости и взаимопонимания в \textit{Экосистеме OSTIS}:
        \begin{scnitemize}
            \item поддержка совместимости между самими \textit{ostis-системами}, входящими в \textit{Экосистему OSTIS} в процессе их эволюции;
            \item поддержка совместимости между каждой ostis-системой и текущим состоянием Технологии OSTIS в процессе эволюции этой технологии;
            \item поддержка совместимости и взаимопонимания между \textit{ostis-системами}, входящими в \textit{Экосистему OSTIS}, и их пользователями при активном стимулировании со стороны \textit{Экосистемы OSTIS} того, чтобы каждый пользователь \textit{Экосистемы OSTIS} был одновременно не только активным ее конечным пользователем, но и активным ее разработчиком.
        \end{scnitemize}
        Таким образом, для обеспечения высокой эффективности эксплуатации и высоких темпов эволюции \textit{Экосистемы OSTIS}, необходимо постоянно повышать уровень информационной совместимости (уровень взаимопонимания) не только между компьютерными системами, входящими в состав \textit{Экосистемы OSTIS}, но также между этими системами и их пользователями. Одним из направлений обеспечения такой совместимости является стремление к тому, чтобы \textit{база знаний} (картина мира) каждого пользователя стала частью (фрагментом) \textbf{\textit{Объединенной базы знаний Экосистемы OSTIS}}. Это значит, что каждый пользователь должен знать, как устроена структура каждой научно-технической дисциплины (объекты исследования, предметы исследования, определения, закономерности и т.д.), как могут быть связаны между собой различные дисциплины.\\
        Формирование таких навыков системного построения картины Мира необходимо начинать со средней школы. Для этой цели необходимо создать комплекс совместимых интеллектуальных обучающих систем по всем дисциплинам среднего образования с четко описанными междисциплинарными связями (\cite{Bashmakov}, \cite{Taranchuk2015}). Благодаря этому можно предотвратить формирование у пользователей мозаичной картины Мира как множества слабо связанных между собой дисциплин. А это, в свою очередь, означает существенное повышение качества образования, которое абсолютно необходимо для качественной эксплуатации компьютерных систем следующего поколения --- \textit{семантических компьютерных систем}.\\
        Пользователи и, первую очередь, разработчики \textit{Экосистемы OSTIS} должны иметь высокий уровень:
        \begin{scnitemize}
            \item Математической культуры (культуры формализации) при построении формальной модели среды, в которой функционирует интеллектуальная система, формальных моделей решаемых ею задач и формальных моделей различных используемых ею способов решения задач.
            \item Системной культуры, позволяющей адекватно оценивать качество разрабатываемых систем с точки зрения общей теории систем и, в частности, оценивать общий уровень автоматизации, реализуемый с помощью этих систем. Системная культура предполагает стремление и умение избегать эклектики, стремление и умение обеспечить качественную стратифицированность, гибкость, рефлексивность, а также качественное сопровождение, высокий уровень обучаемости и комфортный пользовательский интерфейс разрабатываемых систем.
            \item Технологической культуры, обеспечивающей совместимость разрабатываемых систем и их компонентов, а также постоянное расширение библиотеки многократно используемых компонентов создаваемых систем и предполагающей высокий уровень проектной дисциплины.
            \item Умения работать в команде разработчиков наукоемких систем, что предполагает высокий уровень умения работать на междисциплинарных стыках, высокий уровень коммуникабельности и \uline{договороспособности}, т.е. способности не столько отстаивать свою точку зрения, сколько согласовывать ее с точками зрения других разработчиков в интересах развития \textit{Экосистемы OSTIS}.
            \item Активности и ответственности за общий результат --- высокие темпы эволюции \textit{Экосистемы OSTIS} в целом.
        \end{scnitemize}
        Таким образом высокие темпы эволюции \textit{Экосистемы OSTIS} обеспечиваются не только профессиональной квалификацией пользователей (знаниями о \textit{Технологии OSTIS}, о текущем состоянии и проблемах \textit{Экосистемы OSTIS} и навыками использования \textit{Технологии OSTIS} и интеллектуальных систем, входящих в \textit{Экосистему OSTIS}), но и соответствующими человеческими качествами. Очевидно, что современный уровень \uline{договороспособности, активности и ответственности} не может быть основой для эволюции таких систем, как \textit{Экосистема OSTIS}.\\
        Поддержка совместимости \textit{Экосистемы OSTIS} с ее пользователями осуществляется следующим образом:
        \begin{scnitemize}
            \item в каждую \textit{ostis-систему} включаются встроенные ostis-системы, ориентированные
            \begin{scnitemizeii}
                \item на перманентный мониторинг деятельности конечных пользователей и разработчиков этой \textit{\mbox{ostis-системы}},
                \item на анализ качества и, в первую очередь, корректности этой деятельности,
                \item на перманентное ненавязчивое персонифицированное обучение, направленное на повышение качества деятельности пользователей, т.е. на повышение их квалификации;
            \end{scnitemizeii}
            \item в состав \textit{Экосистемы OSTIS} включаются \textit{ostis-системы}, специально предназначенные для обучения пользователей \textit{Экосистемы OSTIS} базовым общепризнанным знаниям и навыкам решения соответствующих классов задач. Сюда входят и знания, соответствующие уровню среднего образования, и знания соответствующие базовым дисциплинам высшего образования в области информатики (и, в том числе, в области искусственного интеллекта), и базовые знания по \textit{Технологии OSTIS} и об \textit{Экосистеме OSTIS}.
        \end{scnitemize}}
    
    \scnheader{Экосистема OSTIS}
    \scntext{обоснование}{Проблема создания рынка совместимых компьютерных систем --- \textbf{вызов современной науке и технике}. От ученых, работающих в области искусственного интеллекта требуется умение коллективно работать над решением междисциплинарных проблем и доводить эти решения до общей интегрированной теории интеллектуальных систем, предполагающей интеграцию всех направлений искусственного интеллекта, и до технологий, доступных широкому кругу инженеров. От инженеров интеллектуальных систем требуется активное участие в развитии соответствующих технологий и существенное повышение уровня математической, системный, технологической и организационно-психологической культуры.\\
        Но главной задачей здесь является снижение барьера между научными исследованиями в области искусственного интеллекта и инженерией в области разработки интеллектуальных систем. Для этого наука должна стать конструктивной и ориентированной на интеграцию своих результатов в форме комплексной технологии разработки интеллектуальных систем, а инженерия, осознав наукоемкость своей деятельности, должна активно участвовать в разработке технологий.\\
        Особый акцент в \textit{Экосистеме OSTIS} делается на постоянный процесс согласования \textit{онтологий} (и, в первую очередь, на согласование семейства всех используемых понятий и терминов, соответствующих этим понятиям) между \uline{всеми} (!) активными субъектами \textit{Экосистемы OSTIS} --- между всеми \textit{ostis-системами} и всеми пользователями.\\
        При наличии \textit{ostis-систем}, являющихся персональными ассистентами пользователей во взаимодействии с \textit{Экосистемой OSTIS}, вся эта Экосистема будет восприниматься пользователями как единая интеллектуальная система, объединяющая все имеющиеся в \textit{Экосистеме OSTIS} информационные ресурсы и сервисы.\\
        Принципы организации \textit{Экосистемы OSTIS} создают все необходимые условия для привлечения к разработке и совершенствованию \textit{Технологии OSTIS} научные, организационные и финансовые ресурсы, которые будут направлены на развитие методов и средств искусственного интеллекта и на формирование рынка семантически совместимых интеллектуальных систем.}
    
    \scnheader{агент Экосистемы OSTIS}
    \scnidtf{субъект, входящий в состав \textit{Экосистемы OSTIS}}
    \begin{scnrelfromset}{разбиение}
        \scnitem{индивидуальная ostis-система Экосистемы OSTIS}
        \begin{scnindent}
        \begin{scnrelfromset}{разбиение}
            \scnitem{самостоятельная ostis-система Экосистемы OSTIS}
            \scnitem{встроенная ostis-система Экосистемы OSTIS}
        \end{scnrelfromset}
        \end{scnindent}
        \scnitem{пользователь Экосистемы OSTIS}
        \scnitem{ostis-сообщество}
        \begin{scnindent}
        \begin{scnrelfromset}{разбиение}
            \scnitem{простое ostis-сообщество}
            \scnitem{иерархическое ostis-сообщество}
        \end{scnrelfromset}
        \end{scnindent}
    \end{scnrelfromset}
    \begin{scnrelfromlist}{правила поведения}
        \scnfileitem{Согласовывать денотационную семантику всех используемых знаков (в первую очередь понятий).}
        \scnfileitem{Согласовывать терминологию, устранять противоречия и информационные дыры.}
        \scnfileitem{Постоянно бороться с синонимией и омонимией как на уровне sc-элементов (внутренних знаков), так и на уровне соответствующих им терминов и прочих внешних идентификаторов (внешних обозначений).}
        \scnfileitem{Каждый \textit{агент Экосистемы OSTIS} по своей инициативе может стать членом любого ostis-сообщества Экосистемы OSTIS после соответствующей регистрации.}
    \end{scnrelfromlist}
    \begin{scnindent}
        \scntext{примечание}{Все правила поведения \textit{агентов Экосистемы OSTIS} должны соблюдаться не только \textit{ostis-системами}, которые являются агентами \textit{Экосистемы OSTIS}, но и людьми, являющиеся ими. Корректное поведение \textit{ostis-систем} как \textit{агентов Экосистемы OSTIS} значительно проще обеспечить, чем корректное поведение людей в качестве таких агентов. Поведение пользователей (естественных агентов) \textit{Экосистемы OSTIS} необходимо внимательно мониторить и контролировать, постоянно способствуя повышению уровня их квалификации как \textit{агентов Экосистемы OSTIS}, а также повышению уровня их мотивации, целенаправленности и самореализации.}
    \end{scnindent}

    \scnheader{ostis-система, являющаяся агентом Экосистемы OSTIS}
    \scnsuperset{персональный ostis-ассистент}
    \scnsuperset{корпоративная ostis-система}
    \scnsuperset{ostis-портал знаний}
    \scnsuperset{ostis-система автоматизации проектирования}
    \scnsuperset{ostis-система автоматизации производства}
    \scnsuperset{ostis-система автоматизации образовательной деятельности}
    \begin{scnindent}
        \scnsuperset{обучающаяся ostis-система}
        \scnsuperset{корпоративная ostis-система виртуальной кафедры}
    \end{scnindent}
    \scnsuperset{ostis-система автоматизации бизнес-деятельности}
    \scnsuperset{ostis-система автоматизации управления}
    \begin{scnindent}
        \scnsuperset{ostis-система управления проектами соответствующего вида}
        \scnsuperset{ostis-система сенсомоторной координации при выполнении определенного вида сложных действий во внешней среде}
        \begin{scnindent}
            \scnsuperset{ostis-система управления самостоятельным перемещением} 
            \scnsuperset{робота по пересеченной местности}
        \end{scnindent}
    \end{scnindent}

    \scnheader{Экосистема OSTIS}
    \begin{scnrelfromlist}{заключение}
        \scnfileitem{\textit{Экосистема OSTIS} представляет собой саморазвивающуюся сеть \textit{ostis-систем}, которая обеспечивает комплексную автоматизацию всевозможных видов и областей человеческой деятельности.}
        \scnfileitem{\textit{Экосистема OSTIS} является следующим этапом развития человеческого общества, обеспечивающий существенное повышение уровня общественного, коллективного интеллекта путем преобразования человеческого общества в экосистему, состоящую из людей и семантически совместимых интеллектуальных систем. \textit{Экосистема OSTIS} --- предлагаемый подход к реализации \textit{smart-общества} или Общества 5.0, построенного на основе \textit{Технологии OSTIS}.}
        \scnfileitem{Сверхзадачей \textit{Экосистемы OSTIS} является не просто комплексная автоматизация всех видов человеческой деятельности (только тех видов деятельности, автоматизация которых целесообразна), но и существенное повышение уровня интеллекта различных человеко-машинных сообществ и всего человеческого общества в целом.}
    \end{scnrelfromlist}

    \bigskip
    \end{scnsubstruct}
    \scnendcurrentsectioncomment
\end{SCn}


\scsubsection[
    \protect\scnmonographychapter{Глава 7.1. Проблемы и перспективы автоматизации различных видов и областей человеческой деятельности с помощью интеллектуальных компьютерных систем нового поколения}
    ]{Предметная область и онтология автоматизируемых видов и областей человеческой деятельности}
\label{tech_human_activity_types}

\scsubsection[
    \protect\scnidtf{{Характеристики технологий \textit{автоматизации человеческой деятельности}, определяющие качество этих \textit{технологий}}}
    \protect\scnmonographychapter{Глава 7.1. Проблемы и перспективы автоматизации различных видов и областей человеческой деятельности с помощью интеллектуальных компьютерных систем нового поколения}
    ]{Предметная область и онтология технологий автоматизации различных видов и областей человеческой деятельности}
\label{tech_human_activity}

\scsubsubsection[
    \protect\scnidtf{История эволюции и современное состояние \textit{технологий} проектирования, реализации, сопровождения, реинжиниринга и использования \textit{компьютерных систем} и, в том числе, \textit{интеллектуальных компьютерных систем} различного назначения. История эволюции \textit{традиционных информационных технологий} и \textit{технологий Искусственного интеллекта}}
    \protect\scnmonographychapter{Глава 7.1. Проблемы и перспективы автоматизации различных видов и областей человеческой деятельности с помощью интеллектуальных компьютерных систем нового поколения}
    ]{Предметная область и онтология технологий компьютеризации различных видов и областей человеческой деятельности}
\label{trad_comp_tech}

\scsubsection[
    \protect\scnidtf{Современное состояние и направления \textit{конвергенции} работ в области \textit{Искусственного интеллекта}}
    \protect\scnidtf{Анализ методологических проблем современного состояния работ в области \textit{Искусственного интеллекта}}
    \protect\scnmonographychapter{Глава 7.1. Проблемы и перспективы автоматизации различных видов и областей человеческой деятельности с помощью интеллектуальных компьютерных систем нового поколения}
    ]{Предметная область и онтология деятельности в области Искусственного интеллекта}
\label{intro_ostis}
\begin{SCn}
    \scnsectionheader{Предметная область и онтология деятельности в области Искусственного интеллекта}
    \begin{scnsubstruct}
        \begin{scnreltovector}{конкатенация сегментов}
    \scnitem{Структура деятельности в области Искусственного интеллекта}
    \scnitem{Текущее состояние и проблемы дальнейшего развития деятельности в области Искусственного интеллекта}
    \scnitem{Понятие Технологии OSTIS}
    \scnitem{Использование Технологии OSTIS для повышения качества человеческой деятельности в области Искусственного интеллекта}
    \scnitem{Понятие Экосистемы OSTIS}
\end{scnreltovector}
\begin{scnrelfromset}{рассматриваемые вопросы}
    \scnfileitem{Каковы основные стратегические цели (сверхзадачи) научно-технической деятельности в области \textit{Искусственного интеллекта}.}
    \scnfileitem{Какие проблемы являются на сегодняшний день актуальными для дальнейшего развития различных направлений \textit{Искусственного интеллекта} и для развития \textit{Искусственного интеллекта} в целом как общей (объединённой) \textit{научно-технической дисциплины}, а также для развития различных форм деятельности в этой области (научно-исследовательской деятельности создания технологий разработки интеллектуальных компьютерных систем, образовательной деятельности, бизнеса).}
    \scnfileitem{Какие проблемы являются на сегодняшний день актуальными для развития других \textit{научно-технических дисциплин} и являются ли эти проблемы аналогичными тем, которые актуальны для развития \textit{Искусственного интеллекта}.}
    \scnfileitem{Какие можно предложить подходы к решению указанных выше проблем и как для этого можно использовать создаваемый сейчас новый технологический уклад в области \textit{Искусственного интеллекта} (следующий уровень технологий искусственного интеллекта).}
    \scnfileitem{Как будет выглядеть на основе следующего уровня \textit{технологий Искусственного интеллекта} комплексная автоматизация всех \textit{видов человеческой деятельности}, а также взаимодействие различных \textit{видов человеческой деятельности}, т.е. как будет выглядеть архитектура \textit{smart-общества}.}
    \scnfileitem{Устраивает ли нас уровень семантической совместимости взаимопонимания между современными виртуальными компьютерными системами и что необходимо сделать для повышения этого уровня.}
    \scnfileitem{Устраивает ли нас уровень семантической совместимости взаимопонимания между современными интеллектуальными компьютерными системами их пользователями и что необходимо сделать для повышения этого уровня.}
\end{scnrelfromset}
\scntext{аннотация}{Предлагаемое вашему вниманию рассмотрение методологических проблем современного состояния работ в области \textit{Искусственного интеллекта} состоит из следующих частей:
    \begin{scnitemize}
        \item Анализ актуальных проблем, препятствующих дальнейшему развитию  \textit{Искусственного интеллекта} как \textit{научно-технической дисциплины}:
        \begin{scnitemizeii}
            \item Проблемы развития научных исследований в области \textit{Искусственного интеллекта}.
            \item Проблемы разработки технологий проектирования и реализации \textit{интеллектуальных компьютерных систем}.
            \item Проблемы формирования рынка \textit{интеллектуальных компьютерных систем}.
            \item Образовательные проблемы в области \textit{Искусственного интеллекта}.
            \item Проблемы развития бизнеса в области \textit{Искусственного интеллекта}.
        \end{scnitemizeii}
        \item Анализ проблем автоматизации сложных видов деятельности:
        \begin{scnitemizeii}
            \item научно-исследовательской деятельности в рамках различных научных дисциплин;
            \item создание \textit{технологий проектирования} и производства (реализации) сложных технических систем;
            \item \textit{инженерной деятельности} по разработке сложных технических систем;
            \item \textit{образовательной деятельности} по наукоёмким техническим специальностям.
        \end{scnitemizeii}
        \item Формулировка принципов, лежащих в основе \textit{Технологии OSTIS}, предназначенной для решения указанных выше проблем.
        \item Рассмотрение структуры \textit{Экосистемы OSTIS}, построенной по \textit{Технологии OSTIS} и обеспечивающей комплексную автоматизацию всех видов человеческой деятельности.
    \end{scnitemize}}
\begin{scnrelfromset}{используемые знаки общих понятий и иных сущностей}
    \scnitem{деятельность}
	    \begin{scnindent}
	    	\scnidtf{область деятельности}
	    	\scnsuperset{человеческая деятельность}
	    \end{scnindent}
    \scnitem{вид деятельности}
    \begin{scnindent}
        \scnhaselement{проектирование}
        \begin{scnindent}
            \scnidtf{проектная деятельность}
        \end{scnindent}
        \scnhaselement{производство}
        \begin{scnindent}
            \scnidtf{производственная деятельность}
        \end{scnindent}
        \scnhaselement{наука}
        \begin{scnindent}
            \scnidtf{научная деятельность}
        \end{scnindent}
    \end{scnindent}
    \scnitem{проект}
    \begin{scnindent}
        \scnsuperset{открытый проект}
    \end{scnindent}
    \scnitem{консорциум}
    \scnitem{технология}
    	\begin{scnindent}
        \scnsuperset{информационная технология}
        \begin{scnindent}
            \scnsuperset{технология искусственного интеллекта}
        \end{scnindent}
    \end{scnindent}
    \scnitem{кибернетическая система}
    \begin{scnindent}
        \scnsuperset{интеллектуальная система}
        \begin{scnindent}
            \scnsuperset{интеллектуальная компьютерная система}
            \begin{scnindent}
                \scnidtf{искусственная интеллектуальная система}
            \end{scnindent}
        \end{scnindent}
    \end{scnindent}
    \scnitem{конвергенция\scnsupergroupsign}
    \begin{scnindent}
        \scnidtf{уровень конвергенции (близости)}
        \scnsuperset{конвергенция кибернетических систем\scnsupergroupsign}
        %Ключевого знака в стандарте не было
        \begin{scnreltolist}{ключевой знак}
            \scnitem{\cite{Yankovskaya2017}}
            \scnitem{\cite{Palagin2013}}
            \scnitem{\cite{Yankovskaya2010}}
            \scnitem{\cite{Kovalchuk2011}}
        \end{scnreltolist}
    \end{scnindent}
    \scnitem{интеграция*}
    \begin{scnindent}
        \scnsuperset{интеграция кибернетических систем*}
        \scnsuperset{эклектичная интеграция*}
        \scnsuperset{глубокая интеграция*}
    \end{scnindent}
    \scnitem{интегрированная система}
    \begin{scnindent}
        \scnsuperset{эклектичная система}
        \scnsuperset{гибридная система}
    \end{scnindent}
    \scnitem{экосистема интеллектуальных компьютерных систем}
    \scnitem{рынок знаний}
    \begin{scnindent}
        \scnidtf{рыночная организация порождения эволюции и применения знаний}
    \end{scnindent}
    \scnitem{smart-общество}
    \begin{scnindent}
        \scnidtf{общество,в основе которого лежит экосистема интеллектуальных компьютерных систем и рынок знаний}
    \end{scnindent}
\end{scnrelfromset}
\begin{scnrelfromset}{ключевые знаки}
    \scnitem{Искусственный интеллект}
    \begin{scnindent}
        \scniselement{научно-техническая дисциплина}
        \begin{scnindent}
            \scnsubset{научно-техническая деятельность}
        \end{scnindent}
    \end{scnindent}
    \scnitem{интеллектуальная система}
    	\begin{scnindent}
        	\scnsuperset{интеллектуальная компьютерная система}
        \end{scnindent}
    \scnitem{Общая теория интеллектуальных систем}
    \scnitem{Базовая комплексная технология проектирования интеллектуальных компьютерных систем}
    \scnitem{Технология производства спроектированных интеллектуальных компьютерных систем}
    \scnitem{Специализированная инженерия в области Искусственного интеллекта}
    \scnitem{Образовательная деятельность в области Искусственного интеллекта}
    \scnitem{Бизнес-деятельность в области Искусственного интеллекта}

    \bigskip

    \scnitem{\scnkeyword{Технология OSTIS}}
    \scnitem{\scnkeyword{ostis-система}}
    \scnitem{смысловое преставление информации}
    \scnitem{агентно-ориентированная модель обработки информации в памяти}
    \scnitem{стандартизация ostis-систем}
    \scnitem{\scnkeyword{SC-код}}
    \scnitem{абстрактная sc-машина}
    \scnitem{конвергенция знаний в памяти}
    \scnitem{ostis-систем}
    \scnitem{конвергенция моделей решения задач в  ostis-системе}
    \scnitem{интеграция знаний в памяти  ostis-системы}
    \scnitem{интеграция моделей решения задач в  ostis-системе}
    \scnitem{ostis-сообщество}
    \scnitem{ostis-технология}
    \begin{scnindent}
        \scnsuperset{ostis-технология проектирования}
        \scnsuperset{ostis-технология производства}
        \scnsuperset{технология эксплуатации ostis-систем}
        \scnsuperset{технология реинжиниринга ostis-систем} 
    \end{scnindent}
    \scnitem{\scnkeyword{Ядро Технологии OSTIS}}  

    \bigskip

    \scnitem{OSTIS-портал научных знаний в области Искусственного интеллекта}
    \scnitem{Проект IMS.ostis}
    \scnitem{\scnkeyword{Метасистема IMS.ostis}}
    \scnitem{Проект Программной реализации универсальной абстрактной sc-машины}
    \scnitem{Проект разработки Универсального sc-компьютера}
    \scnitem{Специализированная инженерия, осуществляемая на основе Технологии OSTIS}
    \scnitem{Образовательная деятельность в области Искусственного интеллекта, осуществляемая на основе технологии OSTIS}
    \scnitem{\scnkeyword{Консорциум OSTIS}}

    \bigskip

    \scnitem{\scnkeyword{Экосистема OSTIS}}
    \begin{scnindent}
        \scnidtf{Симбиоз семантически совместимых и координирующих свою деятельность \textit{ostis-систем} и людей, направленный на существенное, качественное повышение уровня автоматизации всех \textit{видов человеческой деятельности}.}
        \scntext{примечание}{Семантически совместимая (понятийно согласованная) формализация всех(!) видов человеческой деятельности является органической частью \textit{Технологии OSTIS}(!). То есть формализуемые отраслевые стандарты \textit{всех видов человеческой деятельности} должны строго наследовать свойства всей системы базовых понятий и знаний, лежащих в основе Технологии OSTIS. Таким образом речь идет о строгой формальной модели \textit{Экосистемы OSTIS} как единого целого и здесь есть место всем приложениям, но приведенным в комплексную систему. Если к построению такой комплексной формальной модели \textit{всех видов человеческой деятельности} подходить системно, то все не так страшно, так как многие модели можно и нужно строить на основе аналогий, стратификации, наследования и свойств. Это придаст существенную динамику эволюции этих формальных моделей.\\
            К сожалению, современная наука психологически ориентирована на поиск отличий, на выявление принципиальной (научной) новизны своих результатов (что является необходимым фактором оценки этих решений). В этом ничего плохого нет, но для решения сиситемых проблем (в частности, для вывода \textit{Искусственного интеллекта} из кризисного состояния) необходимо существенно активизировать поиск сходств, аналогий, реализацию конвергентных процессов по построению комплексных интегрированных моделей. Это не менее значимые научные результаты, чем выявление принципиально новых свойств и закономерностей.}
    \end{scnindent}
    \scnitem{человеческая деятельность}
    \scnitem{вид человеческой деятельности}
    \scnitem{автоматизация человеческой деятельности}
    \scnitem{качество человеческой деятельности}
    \scnitem{субъект Экосистемы OSTIS}
    \scnitem{Рынок знаний, реализованный в рамках Экосистемы OSTIS}
    \scnitem{smart-общество}
\end{scnrelfromset}

\scntext{предисловие}{Анализ современного состояния работы в области \textit{Искусственного интеллекта} показывает то, что указанная область \textit{человеческой деятельности} находится в глубоком фундаментальным методологическом и трудновидимом кризисе. Поэтому основными целями данного раздела являются:
    \begin{scnitemize}
        \item выявление основных причин возникновения указанного кризиса;
        \item уточнение основных мер, направленных на устранение этого кризиса.
    \end{scnitemize}}
\scntext{основная цель}{Сформировать мотивацию и инфраструктуру для создания эволюции информационных технологий принципиально нового уровня, в основе которого лежат семантические совместимые \textit{интеллектуальные компьютерные системы}, способные согласовывать свои действия в заранее непредсказуемых обстоятельствах.}
\begin{scnindent}
    \scntext{примечание}{Сейчас актуально не столько обсуждать различные вопросы \textit{Искусственного интеллекта}, а обсуждать \uline{проблемы} и пути решения этих проблем. Нельзя делать вид, что всё хорошо.}
\end{scnindent}
	
\scnheader{Искусственный интеллект}
\scntext{примечание}{Современное кризисное состояние \textit{Искусственного интеллекта} вполне логично --- это естественный этап эволюции любых сложных систем и технологий:
    \begin{scnitemize}
        \item Сначала накопление большого количества конкретных решений;
        \item Анализ полученного многообразия и превращение его в стройную систему качественно более высокого уровня.
    \end{scnitemize}
    Кризисов не надо бояться --- их надо преодолевать. Диалектику и, в частности, переход количества в качество ещё никто не отменял.}
    
\scnheader{Современное состояние технологии Искусственного интеллекта}
\scntext{пояснение}{К настоящему моменту мы научились разрабатывать \textit{интеллектуальные компьютерные системы} самого различного назначения. Но для повышения уровня автоматизации всё более и более широких видов человеческой деятельности необходим \uline{качественный} переход к разработке не отдельных \textit{интеллектуальных компьютерных систем}, а целых комплексов самостоятельно взаимодействующих между собой \textit{интеллектуальных компьютерных систем}.Это требует фундаментального переосмысления теории технологии проектирования \textit{интеллектуальных компьютерных систем}. Эффективный переход количества в новое качество требует серьезных усилий.}
\scntext{эпиграф}{Необходим переход от зоопарка локальных идей, сервисов и информационных ресурсов к их системе.}
\scntext{эпиграф}{From data science to knowledge science.}
\scntext{эпиграф}{Одно дело --- создавать локальные шедевры и совсем другое дело --- двигаться ко всеобщей гармонии.}

\scnheader{Современное состояние информационных технологии}
\scntext{примечание}{Экспертам в процессе обсуждения инновационных вопросов приходится тратить много времени на формирование нового \uline{понятийного аппарата}. <...> Мировому сообществу есть смысл задуматься над созданием нового искусственного языка, <...> чтобы с учётом возможностей современного мира сформулировать новую среду экспертного общения.}
\scnrelfrom{автор}{Курбацкий А.Н.}

\scnheader{будущие технологии Искусственного интеллекта}
\scntext{примечание}{\uline{\textit{Смысл}} той \textit{информации}, которой оперирует \uline{каждая}(!) \textit{интеллектуальная компьютерная система}, а также \uline{\textit{смысл}} того, что она делает (какие \textit{задачи} она решает) должен быть четко формализован и является частью её \textit{базы знаний}. Формализация этого \textit{смысла} (в состав которой входит экспертное согласование системы используемых \textit{понятий}) представляет собой первый этап проектирования каждой \textit{интеллектуальной компьютерной системы}, обеспечивающий \textit{семантическую совместимость} (взаимопонимание) \textit{интеллектуальных компьютерных систем} и эффективное их взаимодействие (самостоятельную организацию коллективной деятельности).Таким образом, необходим \textit{\uline{универсальный} формальный язык}, который используется как экспертами, разработчиками, так и непосредственно самими \textit{интеллектуальными компьютерными системами}.}

\scnheader{Технология OSTIS}
\begin{scnrelfromset}{теоретический компонент}
    \scnfileitem{Комплексная семантическая теория интеллектуальных компьютерных систем (ostis-систем).}
    \scnfileitem{Комплексная семантическая теория человеческой деятельности как Экосистемы (симбиоза) с иерархическим комплексом интеллектуальных компьютерных систем.}
    \scnfileitem{Теория перманентной эволюции (реинжиниринга) указанной Экосистемы (с минимизацией этапов локальной приостановки).}
    \begin{scnindent}
        \scntext{примечание}{Cтандарты должны меняться быстро, а Экосистема должна быстро приводиться в соответствие с новыми стандартами.}
    \end{scnindent}
\end{scnrelfromset}

\scnheader{Подготовка специалистов в области Искусственного интеллекта}
\scntext{примечание}{Массовая подготовка высококвалифицированных \textit{специалистов в области Искусственного интеллекта}, способных преодолеть современное кризисное состояние \textit{Искусственного интеллекта}, фактически и является самым главным фактом преодоления указанного кризиса.\\
    Необходимым условием и эпицентром вывода \textit{Искусственного интеллекта} из кризисного состояния и повышения темпов эволюции технологий \textit{Искусственного интеллекта} является организация \textit{подготовки специалистов в области Искусственного интеллекта} на основе активного привлечения студентов, магистрантов и аспирантов к \uline{перманентному} процессу эволюции \textit{Технологии OSTIS}.\\
    Очевидно, этому должно способствовать объединение соответствующей учебно-методической базы для разных кафедр, осуществляющих такую подготовку.\\
    На современном этапе развития \textit{Искусственного интеллекта} требуется не просто подготовка специалистов в этой области --- а подготовка специалистов \uline{принципиально новой формации}, способных:
    \begin{scnitemize}
        \item рассматривать область \textit{Искусственного интеллекта} не просто как многообразие \textit{интеллектуальных компьютерных систем}, а как постоянно эволюционируемую \uline{\textit{Экосистему}} таких систем;
        \item эффективно участвовать в решении как фундаментальных, системных, технологических проблем, так и практических, прикладных проблем эволюции указанной \textit{Экосистемы}.
    \end{scnitemize}
    Все это требует существенного переосмысления организации учебного процесса и учебно-методического обеспечения.\\
    <<Часто, например, совершенствование программных систем сводится к программным заплаткам. Через какое-то время мы имеем программу со множеством заплаток, как правило уже громоздкую и малоэффективную. В итоге --- иногда её проще выбросить и создать новую.>> (\cite{Kurbatski})\\
    Современная разработка каждой сложной программной системы требует построения \uline{качественной} формальной (цифровой) модели объекта управления, объекта автоматизации, причем \textit{семантически совместимой} с соответствующими моделями в смежных системах.\\
    Здесь важна общая математическая культура и унификация такой формализации.\\
    В настоящее время методологический подход к инженерной деятельности при разработке компьютерных систем часто выглядит следующим образом: <<Поставьте мне четкую инженерную задачу и я ее выполню. Но ответственность за ее постановку я с себя снимаю и не хочу учитывать критерии качества постановки задачи на более высоком уровне>>.\\
    Для наукоемких проектов, реализуемых в рамках развивающихся технологий, это недопустимо.\\
    Каждый инженер должен \uline{понимать}, что он делает и каковы истинные более глубокие критерии качества его результата.\\
    Нужна принципиально новая психологическая установка.\\
    Необходимо учитывать не только желание заказчика, но и общие принципы и стандарты разрабатываемых \textit{интеллектуальных компьютерных систем}.\\
    В основе организации образовательной деятельности на современном этапе развития \textit{Искусственного интеллекта} лежит:
    \begin{scnitemize}
        \item четкое формальное описание того, чему мы учим (каким знаниям и навыкам) --- в нашем случае это описание текущей версии \textit{Стандарта OSTIS} и направлений эволюции этого стандарта;
        \item уточнение того, что должен делать студент, магистрант и любой специалист для быстрого и качественного приобретения этих знаний и навыков.
    \end{scnitemize}
    Нужна \uline{комплексная} учебная программа по специальности \textit{Искусственный интеллект}, а не мозаика отдельных учебных дисциплин. И, соответственно этому необходимо \uline{комплексное} учебно-методическое пособие, достаточно полно отражающее текущее состояние теории и технологии проектирования \textit{интеллектуальных компьютерных систем}.\\
    Использование проектного метода при подготовке специалистов в области Искусственного интеллекта предполагает составление систематизированного сборника упражнений и задач, в частности, направленных на эволюцию Технологии OSTIS и посильных для студентов специальности \textit{Искусственный интеллект}:
    \begin{scnitemize}
        \item представление конкретных фрагментов различных предметных областей и онтологий;
        \item представление конкретных специфицированных методов (пополнение библиотек используемых методов из разных предметных областей, например, из теории графов);
        \item спецификация библиографических источников (в контексте \textit{Базы знаний IMS.ostis});
        \item выявление синонимии, омонимии, противоречий;
        \item сравнительный анализ и обзор близких внешних публикаций.
    \end{scnitemize}
    Таким образом, фронт самостоятельных, весьма полезных и посильных для студентов работ весьма широк. Главное сформировать у студентов профессиональный интерес, познавательную активность, инициативность и самостоятельность.}
\scntext{проектный метод}{Для того, чтобы научиться разрабатывать \textit{интеллектуальные компьютерные системы}, необходимо приобрести достаточно большой опыт участия и \uline{завершения} разработки реально востребованных \textit{интеллектуальных компьютерных систем}.}
\scntext{проектный метод}{Для того, чтобы научиться разрабатывать и совершенствовать \textit{технологии искусственного интеллекта}, необходимо приобрести достаточно большой опыт успешного (!) участия в создании различных компонентов комплексной \textit{технологии Искусственного интеллекта}.}
\scntext{пояснение}{Итак, современного специалиста в области \textit{Искусственного интеллекта} необходимо учить:
    \begin{scnitemize}
        \item не только тому, как следует разрабатывать \textit{интеллектуальные компьютерные системы} с помощью имеющихся (существующих) \textit{методов} и \textit{средств}, т.е. с помощью имеющихся \textit{технологий},
        \item но и тому, как надо развивать (совершенствовать) имеющиеся \textit{технологии}.
    \end{scnitemize}
    \textit{Технология OSTIS} рассматривается нами не столько, как предлагаемая технология разработки \textit{интеллектуальных компьютерных систем}, а как предлагаемые \uline{принципы} построения технологии разработки \textit{интеллектуальных компьютерных систем} следующего поколения. Т.е. фактически мы предлагаем не саму технологию (\textit{Технологию OSTIS}), а участие в её создании и развитии, которое может привести даже к радикальным изменениям текущего состояния (текущей версии) этой \textit{технологии}. Это психологически снимает ощущение навязывания предлагаемой технологии и заменяет его на атмосферу партнерства, направленного на перманентную эволюцию указанной технологии. Такой подход создаст также условия для существенного повышения качества \textit{подготовки специалистов в области Искусственного интеллекта}, поскольку дает возможность осуществлять обучение путём непосредственного вовлечения студентов и магистрантов в реальные, практически значимые процессы разработки \textit{интеллектуальных компьютерных систем}, а также в процессы совершенствования (эволюции) соответствующих \textit{технологий}.}
\scnheader{Анализ методологических проблем современного состояния работ в области Искусственного интеллекта}
\begin{scnrelfromvector}{примечания}
    \scnfileitem{Самые тяжелые кризисные ситуации в научно-технической сфере --- это те, которые носят фундаментальный и не совсем очевидный характер. Развитие кибернетики, информатики и искусственного интеллекта подтверждает это. За впечатляющими практическими и теоретическими достижениями незаметно возрастает огромный вал накладных расходов при разработке сложных больших систем --- возрастает дублирование, нестыковки, несогласованности.}
    \scnfileitem{Нет ничего более грустного, чем созерцать активную творческую деятельность большого количества умных людей, которые по инерции, не отдавая себе отчета, накапливают проблемы, препятствующие дальнейшему качественному развитию этой деятельности. Вместо того, чтобы разгребать эти проблемы на благо всем.}
    \scnfileitem{Как только мы начнем серьезно относиться к \uline{формальному} уточнению и согласованию всего многообразия понятий, используемых в области Искусственного интеллекта и различных его приложений, как только мы начнем \uline{реальную}(!) \uline{совместную} работу по общей комплексной формальной теории интеллектуальных компьютерных систем и по созданию комплексной технологии их проектирования, многие современные проблемы \textit{Искусственного интеллекта} начнут решаться. Нет ничего практичнее хорошей теории.}
    \scnfileitem{Основной лейтмотив развития технологий \textit{Искусственного интеллекта} --- это не только создание компьютерной технологии разработки сей совместной \textit{интеллектуальной компьютерной системы}, но и создание \uline{Метатехнологии} перманентной \uline{эволюции}(!) такой технологии. Иначе --- эклектика, усугубляющая современный кризис. Для создания эффективно и самостоятельно взаимодействующих \textit{интеллектуальных компьютерных систем} несущественных мелочей не бывает --- дьявол кроется в деталях и тонкостях. Важен не столько инжиниринг, сколько реинжиниринг интеллектуальных компьютерных систем и человеческой деятельности в целом.}
    \scnfileitem{Решение рассматриваемых кризисных проблем требует:\\
	        \begin{scnitemize}
	            \item Существенного фундаментального общесистемного переосмысления всего того, что мы творим.
	            \item Осознания того, что кибернетика, информатика и искусственный интеллект --- это общая фундаментальная наука, требующая единого серьезного математического аппарата.
	            \item Осознания того, что сейчас требуется не расширяемость многообразия точек зрения, а учиться их согласовывать, совершенствуя соответствующие методы.
	        \end{scnitemize}}
    \scnfileitem{Нам необходимо переходить от автоматизации отдельных видов \textit{человеческой деятельности} к интегрированной автоматизации всего комплекса человеческой деятельности, к созданию и постоянной эволюции всей общечеловеческой \textit{экосистемы}, состоящей из самостоятельно взаимодействующих \textit{интеллектуальных компьютерных систем} как между собой, так и между людьми, автоматизацию деятельности которой они осуществляют. При этом надо помнить, что основные накладные расходы, основные проблемы, возникают на стыках при интеграции различных технических решений. Разработчик каждой подсистемы должен гарантировать отсутствие указанных накладных расходов.}
    \scnfileitem{Самое главное --- надо ориентироваться не на создание идеальной информационной \textit{экосистемы}, а на создание эффективной технологии, направленной на перманентную эволюцию(!) указанной экосистемы.}
    \scnfileitem{Уникальность современного кризиса в области кибернетики, информатики и искусственного интеллекта заключается в том, что, несмотря на глобальность этого кризиса, абсолютно реально создать локальный эпицентр по разрешению этого кризиса --- в частности, в Республике Беларусь. Для этого есть все предпосылки --- специальность \textit{Искусственный интеллект}, опыт разработки компьютерных систем, достаточный научный уровень.}
\end{scnrelfromvector}

        \\
\scnaddlevel{1}
\scnrelfromset{подвопрос}{
	\scnfileitem{Недостатки современных интеллектуальных компьютерных систем};
	\scnfileitem{Недостатки современной технологии Искусственного интеллекта};
	\scnfileitem{Каким требованиям должна удовлетворять качественная технология разработки интеллектуальных компьютерных систем}\\
	\scnaddlevel{1}
		\scnrelfromset{подвопрос}{
			\scnfileitem{уточнить требования, представляемые к интеллектуальным компьютерным системам (что такое интеллектуальная компьютерная система)};
			\scnfileitem{уточнить, почему этого нет};
			\scnfileitem{как эти требования удовлетворить в рамках интеллектуальных компьютерных систем (принципы)};
			\scnfileitem{уточнить требования к технологии};
			\scnfileitem{понять, уточнить, почему, что мешает созданию технологии}
			\scnaddlevel{1}
				\scnrelfromset{причина}{
						\scnfileitem{сложность объекта};
						\scnfileitem{отсутствие понимания того, что задача такой сложности требует создания принципиально нового творческого коллектива с принципиально новой организацией взаимодействия}}
			\scnaddlevel{-1};
	\scnfileitem{как это сделать (принципы, лежащие в основе создания технологии интеллектуальных компьютерных систем)}
	\scnaddlevel{-1}};
	\scnfileitem{Что такое ИИ (как наука)};
%	\scniselement{научно-техническая дисциплина}
	;Что такое интеллектуальная кибернетическая  система\\
%	\scnsubset{кибернетическая система}
	;Что такое технология проектирования и реализации интеллектуальная кибернетическая система
	\scnaddlevel{1}
	;проблемы создания технологии проектирования;
	технология реализации от традиционных компьютеров к компьютерам, ориентированным на реализацию интеллектуальных кибернетических систем
	\scnaddlevel{-1}
	;Результат использования технологии проектирования и реализации
	это не отдельные интеллектуальные компьютерные системы и Экосистема из интеллектуальных компьютерных систем и людей
	\scnaddlevel{1}
	;структура Экосистемы -- иерархическая система специализированных сообществ;
	Чем нас не устраивают те, интеллектуальные компьютерные системы, которые мы разрабатываем сейчас;
	Чем нас не устраивают современные технологии ИИ;
	Какие интеллектуальные компьютерные системы нам нужны;
	Какими свойствами и способностями мы хотели бы их наделить\scnaddlevel{1}
	;высокая степень обучаемости в разных направлениях\scnaddlevel{1};
	расширение знаний без введения новых понятий;
	введение новых понятий без расширения многообразия видов знаний;
	расширение многообразия видов знаний;
	расширение моделей решения задач(новый вид методов + их интерпретация)\scnaddlevel{-2}
	;Какие технологии нам нужны;
	Почему таких икс и технологий ещё нет;
	Что мешает?;
	Что делать?;
	Какие недостатки имеют современные интеллектуальные системы;
	\scnaddlevel{1}недостаточно высокий уровень интеллектуальности;
	нет эффективного взаимодействия(координации);
	высокая степень обучаемости в разных направлениях;
	\scnaddlevel{-1}
	Какие недостатки имеют современные технологии Искусственного интеллекта
	;Какова трудоёмкость разработки выбранных
	икс
	;Какова трудоёмкость системной интеграции икс и их компонентов;
	Обеспечивается ли совместимость компонентов
	икс, разрабатываемых с помощью различных
}
\scnaddlevel{-1}
}
        \scnsegmentheader{Анализ структуры Деятельности в области Искусственного интеллекта}
\begin{scnsubstruct}
    \scntext{аннотация}{Для того, чтобы рассмотреть проблемы дальнейшего развития \textit{деятельности} в области \textit{Искусственного интеллекта} как \textit{научно-технической дисциплины} и, в частности, проблемы комплексной автоматизации этой \textit{деятельности}, необходимо уточнить структуру указанной \textit{деятельности}.}
    
	\scnheader{Человеческая деятельность в области Искусственного интеллекта}
	\scntext{примечание}{Человеческая деятельность в области \textit{Искусственного интеллекта} направлена на исследование и создание \textit{интеллектуальных компьютерных систем} различного вида и различного назначения.}
	\begin{scnhaselementrolelist}{объект исследования}
		\scnitem{индивидуальные интеллектуальные компьютерные системы}
		\begin{scnindent}
			\scnhaselement{когнитивные агенты}
		\end{scnindent}
		\scnitem{многоагентные интеллектуальные компьютерные системы}
		\begin{scnindent}
			\scnhaselement{сообщества, состоящие из индивидуальных интеллектуальных компьютерных систем}
		\end{scnindent}
		\scnitem{человеко-машинные сообщества, состоящие из интеллектуальных компьютерных систем и их пользователей}
	\end{scnhaselementrolelist}

	\scnheader{Искусственный интеллект}
    \scniselement{область человеческой деятельности}
    \scniselement{научно-техническая дисциплина}
    \begin{scnindent}
    	\scniselement{вид человеческой деятельности}
    \end{scnindent}
    \begin{scnrelfromlist}{цель}
		\scnfileitem{построение теории интеллектуальных систем}
		\scnfileitem{создание технологии разработки интеллектуальных компьютерных систем (искусственных интеллектуальных систем)}
		\scnfileitem{переход на принципиально новый уровень комплексной автоматизации всех \textit{видов человеческой деятельности}, который основан на массовом применение \textit{интеллектуальных компьютерных систем}}
		\begin{scnindent}
			\begin{scnrelfromlist}{детализация}
				\scnfileitem{наличие \textit{интеллектуальных компьютерных систем}, способных понимать друг друга и согласовывать свою деятельность}
				\scnfileitem{построение \textit{Общей теории человеческой деятельности}, осуществляемый в условиях нового уровня её автоматизации, (теории деятельности \textit{smart-общества}), которая должна быть понятна \textit{используемым интеллектуальным компьютерным системам} и которая потребует существенного переосмысления современной организации \textit{человеческой деятельности}}
			\end{scnrelfromlist}
		\end{scnindent}
	\end{scnrelfromlist}
	\begin{scnrelfromlist}{определение}
		\scnfileitem{Научно-техническая деятельность, направленная на построение теории интеллектуальных систем, а также на создание технологии проектирования и производства искусственных интеллектуальных систем (\textit{интеллектуальных компьютерных систем}).}
		\scnfileitem{Научно-техническая деятельность в области \textit{Искусственного интеллекта}.}
		\scnfileitem{Деятельность в области \textit{Искусственного интеллекта}}
		\scnfileitem{Научно-техническая деятельность, направленная на исследование феномена \textit{интеллекта}, а также на создание искусственных интеллектуальных систем (\textit{интеллектуальных компьютерных систем}) и включающая в себя соответствующую научно-исследовательскую деятельность, инженерно-технологическую, инженерно-прикладную, образовательную и организационную деятельность.}
		\scnfileitem{Междисциплинарная (трансдисциплинарная) область \textit{научно-технической деятельности}, направленная на разработку и эксплуатацию \textit{интеллектуальных компьютерных систем}, обеспечивающих автоматизацию различных сфер \textit{человеческой деятельности}.}
		\scnfileitem{Научно-техническая дисциплина направленная на разработку теории индивидуальных \textit{интеллектуальных компьютерных систем} и \textit{интеллектуальных сообществ} (коллективов) таких систем, а также средств поддержки их проектирования и реализации).}
		\scnfileitem{\textit{Научно-техническая дисциплина}, являющаяся частью \textit{кибернетики} (теория кибернетических систем), объектом исследования которой являются \textit{интеллектуальные компьютерные системы} (искусственные \textit{интеллектуальные системы}) и целями которой являются (1) разработка \textit{теории интеллектуальных компьютерных систем}, (2) разработка \textit{технологии(методов и средств)} \textit{проектирования и производства компьютерных систем}, а также, (3) разработка конкретных интеллектуальных компьютерных систем различного назначения.}
    \end{scnrelfromlist}
	\scnrelfrom{декомпозиция}{Декомпозиция Искусственного интеллекта по формам деятельности}
    \begin{scnindent}
		\begin{scneqtoset}
			\scnitem{Научно-исследовательская деятельность в области Искусственного интеллекта}
			\begin{scnindent}
				\scnrelfrom{продукт}{Общая теория интеллектуальных систем}
				\begin{scnindent}
					\scnidtf{Теория, уточняющая структуру и принципы функционирования \textit{интеллектуальных систем}, а также акцентирующая внимание на причинах (предпосылках) возникновение свойства интеллектуальности (феномены \textit{интеллекта})}
				\end{scnindent}
				\scniselement{коллективная научно-техническая деятельность}
				\begin{scnindent}
					\scniselement{вид человеческой деятельности}
					\scnidtf{научно-исследовательская дисциплина или направление}
					\scnrelboth{следует отличать}{продукт научно-исследовательской деятельности}
				\end{scnindent}
				\scntext{примечание}{В процессе \textit{Научно-исследовательской деятельности в области Искусственного интеллекта} осуществляется конкуренция различных точек зрения и подходов к построению формальных моделей различных компонентов \textit{интеллектуальных компьютерных систем}. Конечной целью такой деятельности является постоянно развиваемая \textit{Общая теория} \textit{интеллектуальных} \textit{компьютерных систем}, объектами исследования которой являются \textit{интеллектуальные компьютерные системы} и их формальные \textit{логико-семантические модели}, включающие в себя формальные модели различного вида \textit{знаний}, входящих в состав \textit{баз знаний} интеллектуальных компьютерных систем, а также различные \textit{модели решения задач} (логические модели различного вида, нейросетевые, генетические, продукционные, функциональные и другие).}
			\end{scnindent}
			\scnitem{Разработка Стандарта интеллектуальных компьютерных систем}
			\begin{scnindent}
				\scntext{примечание}{\textit{Разработка Стандарта интеллектуальных компьютерных систем} включает в себя перманентную эволюцию этого стандарта и поддержку целостности каждой его версии. Текущая версия \textit{Стандарта интеллектуальных компьютерных систем} --- это \uline{согласованная} (общепризнанная) \uline{на текущий момент} часть \textit{Общей теории интеллектуальных компьютерных систем.}}
			\end{scnindent}
			\scnitem{Разработка Базовой комплексной технологии проектирования интеллектуальных компьютерных систем}
			\begin{scnindent}
				\scnrelfrom{продукт}{Базовая комплексная технология проектирования интеллектуальных компьютерных систем}
				\begin{scnindent}
					\scntext{примечание}{Комплексность данной \textit{технологии} заключается в том, что она ориентирована (1) на проектирование \textit{интеллектуальных компьютерных систем} в целом, а не только отдельных их компонентов и (2) на создание объединённой \textit{технологии}, объединяющей самые различные технологические подходы на основе их \textit{конвергенции} и глубокой \textit{интеграции}}
				\end{scnindent}
				\scniselement{разработка технологии проектирования искусственных объектов заданного класса}
				\begin{scnindent}
					\scniselement{вид человеческой деятельности}
				\end{scnindent}
				\scntext{примечание}{\textit{Разработка технологии проектирования интеллектуальных компьютерных систем} включает в себя семейство методик проектирования, а также методов и средств автоматизации \textit{проектирования} различных \textit{компонентов} \textit{интеллектуальных компьютерных систем} и \textit{интеллектуальных компьютерных систем} в целом. Результатом проектирования \textit{интеллектуальных компьютерных систем} является полная формальная логико-семантическая модель этой системы.}
			\end{scnindent}
			\scnitem{Разработка Технологии производства спроектированных интеллектуальных компьютерных систем}
			\begin{scnindent}
				\scnhaselement{Разработка технологии реализации спроектированных интеллектуальных компьютерных систем}
				\scnhaselement{Разработка технологий эксплуатации и сопровождения интеллектуальных компьютерных систем}
				\scnrelfrom{продукт}{Технология производства спроектированных интеллектуальных компьютерных систем}
				\begin{scnindent}
					\scnidtf{технология реализации (сборки и установки) спроектированных интеллектуальных компьютерных систем}
					\scntext{примечание}{Очевидно, что данная \textit{технология} должна быть самым тесным образом связана с \textit{Базовой комплексной технологией проектирования интеллектуальных компьютерных систем} (по крайней мере данная \textit{технология} должна знать в какой форме ей на вход передаётся результат проектирования). Поэтому имеет смысл говорить об объединённой технологии проектирования и производства \textit{интеллектуальных компьютерных систем}}
				\end{scnindent}
				\scniselement{производства спроектированных искусственных объектов заданного класса}
				\begin{scnindent}
					\scniselement{вид человеческой деятельности}
				\end{scnindent}
				\scntext{примечание}{В основе технологии реализации (производства) спроектированных \textit{интеллектуальных компьютерных систем} лежит \textit{универсальный интерпретатор формальных логико-семантических моделей интеллектуальных компьютерных систем}, являющихся результатом проектирования указанных систем. Указанный универсальный интерпретатор может быть реализован либо в виде \textit{программной системы} на современных компьютерах, либо в виде \textit{универсального компьютера нового поколения}, ориентированного на интерпретацию формальных \textit{логико-семантических моделей интеллектуальных компьютерных систем}.}
			\end{scnindent}
			\scnitem{Специализированная инженерия в области Искусственного интеллекта}
			\begin{scnindent}
				\scnidtf{Прикладная инженерная деятельность в области Искусственного интеллекта}
				\scnidtf{Множество Процессов разработки (проектирования и производства) \textit{интеллектуальных компьютерных систем} различного назначения, кроме \textit{интеллектуальных компьютерных систем автоматизации проектирования} и автоматизации производства \textit{интеллектуальных компьютерных систем}}
				\scnrelfrom{продукт}{множество специализированных интеллектуальных компьютерных систем}
				\begin{scnindent}
					\scnrelfrom{основной sc-идентификатор}{прикладная интеллектуальная компьютерная система}
				\end{scnindent}
				\scnsubset{проектирование и производство искусственного объекта заданного класса на основе заданной технологии}
				\begin{scnindent}
					\scniselement{вид человеческой деятельности}
				\end{scnindent}
				\scntext{примечание}{Прикладная инженерная деятельность в области Искусственного интеллекта, то есть непосредственное проектирование, реализация и сопровождение включает в себя обновление (реинжиниринг), осуществляемое в ходе эксплуатации, конкретных \textit{интеллектуальных компьютерных систем.}}
			\end{scnindent}
			\scnitem{Образовательная деятельность в области Искусственного интеллекта}
			\begin{scnindent}
				\scnidtf{Учебная деятельность в области Искусственного интеллекта}
				\scnidtf{Деятельность, направленная на подготовку молодых специалистов области \textit{Искусственного интеллекта} на перманентное повышение квалификации действующих специалистов в этой области}
				\scntext{примечание}{Сложность и высокая степень наукоемкости задач, больших своего решения на текущем этапе развития \textit{Искусственного интеллекта}, добавляют к специалистам, работающим в этой области высокие требования к уровню их:
					\begin{itemize}
						\item математической культуры (культуры формализации),
						\item системной культуры,
						\item технологической культуры,
						\item инженерная культура,
						\item умения работать в коллективных наукоемких проектах.
					\end{itemize}}
				\scnsubset{образовательная деятельность}
				\begin{scnindent}
					\scniselement{вид человеческой деятельности}
				\end{scnindent}
				\scntext{примечание}{\textit{Учебная деятельность в области Искусственного интеллекта} направлена на подготовку специалистов области \textit{Искусственного интеллекта} и на перманентное повышение квалификации действующих специалистов в этой области. Без эффективной организации учебной деятельности в области \textit{Искусственного интеллекта} быстрый прогресс в этой области невозможен. Непосредственное включение учебной деятельности в общую структуру человеческой деятельности в области \textit{Искусственного интеллекта} обусловлено следующими обстоятельствами:
					\begin{itemize}
						\item необходимостью глубокой \textit{конвергенции} между различными направлениями и видами деятельности в области \textit{Искусственного интеллекта} и соответствующей спецификой требований, предъявляемых к специалистам в этой области --- каждый такой специалист должен быть достаточно компетентен и в научно-исследовательской деятельности в области \textit{Искусственного интеллекта}, и в разработке технологий (методик и средств) \textit{проектирования интеллектуальных компьютерных систем}, и в разработке технологий \textit{воспроизводства} (реализации) спроектированных \textit{интеллектуальных компьютерных систем}, а также технологий их \textit{эксплуатации} и \textit{сопровождения}, и в прикладной \textit{инженерной деятельности в области} \textit{Искусственного интеллекта};
						\item высокими темпами эволюции результатов в области \textit{Искусственного интеллекта}, что делает необходимой организацию обучения соответствующих специалистов путем их непосредственного подключения не к учебным (упрощенным) проектам, а к реальным проектам, реализуемым в текущий момент. Иначе подготовленные специалисты будут иметь квалификацию \scnqq{вчерашнего дня};
						\item существенным расширением объемов работ в области \textit{Искусственного интеллекта} и острой необходимостью массовой подготовки соответствующих специалистов.
					\end{itemize}}
			\end{scnindent}
			\scnitem{Бизнес-деятельность в области искусственного интеллекта}
			\begin{scnindent}
				\scntext{пояснение}{Речь идет о бизнес-деятельности в широком смысле как деятельности, направленный на создание инфраструктурных условий для качественного выполнения всех \textit{видов деятельности} в области \textit{Искусственного интеллекта}}
				\begin{scnindent}
					\begin{scnrelfromlist}{пример}
						\scnfileitem{разработка и реализация грамотной научно-технической политики, связывающие как тактические, так и стратегические цели}
						\scnfileitem{глубокая \textit{конвергенцию} всех форм и \textit{видов деятельности} в области \textit{Искусственного интеллекта}}
						\scnfileitem{организация взаимовыгодного сотрудничество различных школ и коллективов, работающих в области \textit{Искусственного интеллекта}}
						\scnfileitem{финансовое обеспечение}
						\scnfileitem{кадровое обеспечение}
						\scnfileitem{материально-техническое обеспечение}
						\scnfileitem{организация проведения различных мероприятий (конференций, выставок, семинаров)}
						\scnfileitem{публикационная деятельность и защита интеллектуальной собственности}
						\scnfileitem{материально-техническое обеспечение}
					\end{scnrelfromlist}
				\end{scnindent}
				\scnsubset{бизнес-деятельность научно-технической области}
				\begin{scnindent}
					\scniselement{вид человеческой деятельности}
				\end{scnindent}
			\end{scnindent}
			\scnitem{Организационная деятельность в области Искусственного интеллекта}
			\begin{scnindent}
				\scntext{примечание}{\textit{Организационная деятельность в области Искусственного интеллекта} направлена на создание инфраструктуры для качественного выполнения всех остальных видов деятельности в области \textit{Искусственного интеллекта}, а именно:
				\begin{itemize}
					\item для обеспечения глубокой \textit{конвергенции} между различными направлениями и видами деятельности в области \textit{Искусственного интеллекта} и, в частности, между теорией, технологиями и инженерной практикой в этой области;
					\item для обеспечения баланса между тактикой и стратегией в развитии деятельности в области \textit{Искусственного интеллекта} как ключевой основы существенного повышения уровня автоматизации всевозможных видов \textit{человеческой деятельности} и перехода к \textit{smart-обществу}.
				\end{itemize}}
			\end{scnindent}
		\end{scneqtoset}
    \end{scnindent}
    \scntext{оценка}{Декомпозиция \textit{человеческой} \textit{деятельности} в области \textit{Искусственного интеллекта} по \textit{видам} \textit{деятельности} не является традиционным признаком декомпозиции \textit{научно-технических дисциплин}. Обычно декомпозиция \textit{научно-технических дисциплин} осуществляется по содержательным направлениям, которые соответствуют декомпозиции \textit{технических систем}, исследуемых и разрабатываемых в рамках этих \textit{научно-технических дисциплин}, то есть соответствуют выделению в этих \textit{технических системах} различного вида компонентов.}
    
    \scnheader{Искусственный интеллект}
    \scnrelfrom{разбиение}{Декомпозиция Искусственного интеллекта по видам деятельности}
    \begin{scnindent}
	    \begin{scneqtoset} 
	    	\scnfileitem{исследование и разработка формальных моделей и языков представления знаний}
	    	\scnfileitem{исследование и разработка баз знаний}
	    	\scnfileitem{исследование и разработка логических моделей обработки знаний}
	   		\scnfileitem{исследование и разработка искусственных нейронных сетей}
	   		\scnfileitem{исследование и разработка подсистем компьютерного зрения}
	   		\scnfileitem{исследование и разработка подсистем обработки естественно-языковых текстов (синтаксический анализ, понимание, синтез)}
	    \end{scneqtoset}
    \end{scnindent}
    \scntext{примечание}{Важность декомпозиции \textit{Искусственного интеллекта} по \textit{видам} \textit{деятельности} определяется тем, что выделение различных \textit{видов деятельности} позволяет четко ставить задачу на разработку средств автоматизации этих \textit{видов деятельности}.}
    
   \scnheader{Cтруктура Человеческой деятельности в области Искусственного интеллекта}
   \begin{scnstruct}
    	\scnheader{Человеческая деятельность в области Искусственного интеллекта}
    	\scnidtf{Искусственный интеллект (как научно-техническая дисциплина)}
    	\scniselement{научно-техническая дисциплина}
    	\scnidtf{Искусственный интеллект (как научно-техническая дисциплина)}
    	\scnidtf{Человеческая деятельность в Предметной области интеллектуальных компьютерных систем}
    	\scniselement{деятельность}
    	\begin{scnrelfromset}{декомпозиция}
    		\scnitem{Интегральная деятельность по поддержке жизненного цикла всевозможных интеллектуальных компьютерных систем}
    		\begin{scnindent}
    			\scnrelfrom{декомпозиция}{поддержка жизненного цикла интеллектуальных компьютерных систем}
    			\begin{scnindent}
    				\scniselement{вид деятельности}
    				\scnsuperset{поддержка жизненного цикла ostis-систем}
    				\begin{scnrelfromset}{обобщенная декомпозиция}
    					\scnitem{проектирование интеллектуальных компьютерных систем}
    					\scnitem{производство интеллектуальных компьютерных систем}
    					\scnitem{начальное обучение интеллектуальных компьютерных систем}
    					\scnitem{мониторинг качества интеллектуальных компьютерных систем}
    					\scnitem{восстановление требуемого уровня качества интеллектуальных компьютерных систем}
    					\scnitem{реинжиниринг интеллектуальных компьютерных систем}
    					\scnitem{обеспечение безопасности интеллектуальных компьютерных систем}
    					\scnitem{эксплуатация интеллектуальных компьютерных систем конечными пользователями}
    				\end{scnrelfromset}
    			\end{scnindent}
    		\end{scnindent}
    		\scnitem{Поддержка жизненного цикла Общей теории интеллектуальных компьютерных систем}
    		\begin{scnindent}
    			\scniselement{научно-исследовательская деятельность}
    		\end{scnindent}        
    		\scnitem{Поддержка жизненного цикла Стандарта интеллектуальных компьютерных систем}
    		\begin{scnindent}
    			\scniselement{стандартизация}
    			\scnrelfrom{часть}{Поддержка жизненного цикла Стандарта ostis-систем}
    		\end{scnindent}        
    		\scnitem{Поддержка жизненного цикла Технологии комплексной поддержки жизненного цикла интеллектуальных компьютерных систем}
    		\begin{scnindent}
    			\scniselement{поддержка жизненного цикла технологий}
    			\begin{scnindent}
    				\scnidtf{создание и сопровождение технологий}
    			\end{scnindent}
    			\scnrelfrom{часть}{Поддержка жизненного цикла Технологии OSTIS}
    		\end{scnindent}
    		\scnitem{Поддержка жизненного цикла кадровых ресурсов для Человеческой деятельности в области Искусственного интеллекта}
    		\scnitem{Поддержка жизненного цикла системы комплексной организации взаимодействия между всеми направлениями Человеческой деятельности в области Искусственного интеллекта}
    		\begin{scnindent}
    			\scniselement{поддержка жизненного цикла метасистем комплексного управления поддержкой и обеспечением поддержки жизненного цикла сущностей соответствующего класса}
    		\end{scnindent}        
    	\end{scnrelfromset}
    \end{scnstruct}
    
    \scnheader{Человеческая деятельность в области Искусственного интеллекта}
    \begin{scnrelfromset}{практический результат}
    	\scnfileitem{Реорганизация и комплексная автоматизация \textit{человеческой деятельности в области Искусственного интеллекта} с помощью \textit{интеллектуальных компьютерных систем нового поколения}.}
    	\scnfileitem{\uline{Поэтапное} создание глобальной сети эффективно взаимодействующих \textbf{\textit{интеллектуальных компьютерных систем нового поколения}}, обеспечивающих \uline{комплексную} автоматизацию всевозможных видов и областей \textit{человеческой деятельности}.}
    \end{scnrelfromset}
    	
    \scnheader{Технология поддержки жизненного цикла интеллектуальных компьютерных систем}
    \scnrelfrom{вид деятельности}{поддержка жизненного цикла интеллектуальных компьютерных систем}
    \begin{scnrelfromset}{декомпозиция}
    	\scnitem{Технология проектирования интеллектуальных компьютерных систем}
    	\begin{scnindent}
    		\scnrelfrom{вид деятельности}{проектирование интеллектуальных компьютерных систем}
    	\end{scnindent}
    	\scnitem{Технология производства интеллектуальных компьютерных систем}
    	\begin{scnindent}
    		\scnrelfrom{вид деятельности}{производство интеллектуальных компьютерных систем}
    	\end{scnindent}
    	\scnitem{Технология начального обучения интеллектуальных компьютерных систем (адаптации к конкретной деятельности)}
    	\begin{scnindent}
    		\scnrelfrom{вид деятельности}{начальное обучение интеллектуальных компьютерных систем}
    	\end{scnindent}
    	\scnitem{Технология мониторинга качества интеллектуальных компьютерных систем}
    	\begin{scnindent}
    		\scnrelfrom{вид деятельности}{мониторинг качества интеллектуальных компьютерных систем}
    		\begin{scnindent}
    			\scnidtf{плановое тестирование и диагностика интеллектуальных компьютерных систем}
    		\end{scnindent}
    	\end{scnindent}
    	\scnitem{Технология восстановления требуемого уровня качества интеллектуальных компьютерных систем в ходе их эксплуатации}
    	\begin{scnindent}
    		\scnidtf{Технология выявления и исправления потенциально опасных ситуаций и событий в деятельности интеллектуальных компьютерных систем (ошибок, противоречий, и так далее)}
    		\scnrelfrom{вид деятельности}{восстановление требуемого уровня качества интеллектуальных компьютерных систем}
    	\end{scnindent}
    	\scnitem{Технология реинжиниринга  интеллектуальных компьютерных систем}
    	\begin{scnindent}
    		\scnidtf{Технология совершенствования, модернизации, обновления интеллектуальных компьютерных систем}
    		\scnrelfrom{вид деятельности}{реинжиниринг интеллектуальных компьютерных систем}
    	\end{scnindent}
    	\scnitem{Технология обеспечения безопасности интеллектуальных компьютерных систем}
    	\begin{scnindent}
    		\scnrelfrom{вид деятельности}{обеспечение безопасности интеллектуальных компьютерных систем}
    	\end{scnindent}
    	\scnitem{Технология эксплуатации интеллектуальных компьютерных систем конечными пользователями}
    	\begin{scnindent}
    		\scnrelfrom{вид деятельности}{эксплуатация интеллектуальных компьютерных систем конечными пользователями}
    	\end{scnindent}
    \end{scnrelfromset}
    
    \scnheader{Разработка базовой комплексной технологии проектирования интеллектуальных компьютерных систем}
    \begin{scnrelfromset}{декомпозиция}
        \scnitem{Разработка общей теории интеллектуальных компьютерных систем}
		\begin{scnindent}
			\scnrelfrom{продукт}{Общая теория интеллектуальных компьютерных систем}
			\scniselement{разработка теории искусственных объектов заданного класса}
			\begin{scnindent}
				\scniselement{вид человеческой деятельности}
			\end{scnindent}
		\end{scnindent}
        \scnitem{Разработка общей теории интеллектуальных компьютерных систем}
		\begin{scnindent}
			\scnrelfrom{продукт}{Теория проектирования интеллектуальных компьютерных систем}
			\begin{scnindent}
				\scntext{примечание}{В состав этой теории входят методы проектирования, библиотеки проектирования и спецификация используемых индустриальных средств.}
			\end{scnindent}
			\scnidtf{Разработка \textit{Теории проектной деятельности} по построению формальных моделей \textit{интеллектуальных компьютерных систем}}
			\scniselement{теории проектирования интеллектуальных объектов заданного класса}
			\begin{scnindent}
				\scniselement{вид человеческой деятельности}
			\end{scnindent}
		\end{scnindent}
        \scnitem{Разработка комплекса средств автоматизации проектирования интеллектуальных компьютерных систем}
		\begin{scnindent}
			\scniselement{разработка комплекса средств автоматизации проектирования искусственных объектов заданного класса}
			\begin{scnindent}
				\scniselement{вид человеческой деятельности}
			\end{scnindent}
		\end{scnindent}
    \end{scnrelfromset}
    
    \scnheader{Разработка Технологии производства спроектированных интеллектуальных компьютерных систем}
    \begin{scnrelfromset}{декомпозиция}
        \scnitem{Разработка Теории производства спроектированных интеллектуальных компьютерных систем}
		\begin{scnindent}
			\scnidtf{Разработка \textit{Теории производственной деятельности} по реализации (сборке и установке) \textit{интеллектуальных компьютерных систем}}
			\scnrelfrom{продукт}{Теория производства спроектированных интеллектуальных компьютерных систем}
			\begin{scnindent}
				\scntext{примечание}{В состав этой \textit{теории} входят \textit{методы производства} (сборки и установки) \textit{интеллектуальных компьютерных систем}, а также спецификация \textit{используемых инструментальных средств}.}
			\end{scnindent}
		\end{scnindent}
        \scnitem{Разработка комплекса средств автоматизации производства интеллектуальных компьютерных систем}
		\begin{scnindent}
			\scniselement{разработка комплекса средств автоматизации производства искусственных объектов заданного класса}
			\begin{scnindent}
				\scniselement{вид человеческой деятельности}
			\end{scnindent}
		\end{scnindent}
    \end{scnrelfromset}
    
    \scnheader{Специализированная инженерия в области Искусственного интеллекта}
    \scnidtf{проектирование и производство конкретной \textit{интеллектуальной компьютерной системы} по заданной технологии}
    \begin{scnrelfromset}{обобщенная декомпозиция}
        \scnitem{проектирование конкретных интеллектуальных компьютерные системы}
		\begin{scnindent}
			\begin{scnrelfromset}{обобщенное разбиение}
				\scnitem{проектирование интеллектуальной компьютерной системы автоматизации проектирования соответствующего класса интеллектуальных компьютерных систем}
				\scnitem{проектирование интеллектуальной компьютерной системы автоматизации проектирования соответствующего класса объектов, не являющихся интеллектуальными компьютерными системами}
				\scnitem{проектирование интеллектуальной компьютерной системы,не являющейся системой автоматизации проектирования}
			\end{scnrelfromset}
		\end{scnindent}
        \scnitem{производство конкретной спроектированной интеллектуальной компьютерной системы}
    \end{scnrelfromset}
    
    \scnheader{производство конкретной спроектированной интеллектуальной компьютерной системы}
    \scntext{примечание}{Производственная деятельность, направленная на \textit{производство} (реализацию) спроектированной \textit{интеллектуальной компьютерной системы} значительно уступает по уровню сложности деятельности по проектированию этой \textit{интеллектуальной компьютерной системы}, так как это производство сводится к сборке результата \textit{проектирования} (формальной логико-семантической модели разрабатываемой \textit{интеллектуальной компьютерной системы}) и загрузки этой модели в память компьютера или программного \textit{универсального интерпретатора логико-семантических моделей интеллектуальных компьютерных систем}, в качестве которого может быть использован:
        \begin{itemize}
            \item либо специально разработанный для этого \textit{компьютер}, ориентированные на обработку \textit{баз знаний} и интерпретацию различных \textit{интеллектуальных моделей решения задач},
            \item либо программная эмуляция такого \textit{компьютера}, реализованная на современных \textit{компьютерах} фон-неймановской архитектуры.
        \end{itemize}
        Простота производства спроектированных систем характерна для производства не только интеллектуальных, но и любых других \textit{компьютерных систем}.
        \\Мы выделяем производственный этап реализации \textit{интеллектуальных компьютерных систем} для того, чтобы по аналогии рассматривать этап производства (массового, мелкосерийного, разового производства) спроектированных искусственных объектов любого другого вида(микросхем, автомобилей, зданий, компьютеров).
        \\Очевидно, что массовое производство некоторых видов продукции может иметь весьма большой уровень сложности, но при этом суть \textit{производственной деятельности} как процесса перехода от проекта (спецификации) некоторого объекта к его реализации остаётся одной и той же независимо от уровня сложности реализуемого объекта (производимой продукции).}
        
    \scnheader{следует отличать*}
    \begin{scnhaselementset}
        \scnitem{специализированная интеллектуальная компьютерная система}
        \scnitem{интеллектуальная компьютерная система автоматизации проектирования интеллектуальных компьютерных систем}
        \scnitem{интеллектуальная компьютерная система автоматизации производства спроектированных интеллектуальных компьютерных систем}
    \end{scnhaselementset}
    \begin{scnhaselementset}
        \scnitem{человеческая деятельность}
		\begin{scnindent}
			\scnsuperset{научно-исследовательская деятельность}
			\scnsuperset{научно-техническая деятельность}
		\end{scnindent}
        \scnitem{продукт человеческой деятельности}
		\begin{scnindent}
			\scnsuperset{продукт научно-исследовательской деятельности}
			\scnsuperset{продукты научно-технической деятельности стакан}
		\end{scnindent}
    \end{scnhaselementset}
    
    \scnheader{Искусственный интеллект}
    \scnrelfrom{декомпозиция}{Декомпозиция Искусственного интеллекта по направлениям}
    \begin{scnindent}
	    \begin{scneqtoset}
	        \scnitem{Разработка теории представления знаний и технологии проектирования баз знаний актуальных компьютерных систем}
	        \scnitem{Разработка теории решения задач и технологии проектирования решателей задач интеллектуальных компьютерных систем}
			\begin{scnindent}
				\begin{scnrelfromset}{декомпозиция}
					\scnitem{Разработка теории решения интерфейсных задач и технологии проектирования соответствующих решателей}
					\begin{scnindent}
						\scnrelfrom{часть}{Разработка теории естественно языковых интерфейсов интеллектуальных компьютерных систем и технологии их проектирования}
					\end{scnindent}
					\scnitem{Разработка теории решения информационных задач в базах знаний интеллектуальных компьютерных систем и технологии проектирования соответствующих решателей}
					\scnitem{Разработка теории решения поведенческих задач во внешней среде интеллектуальных компьютерных систем и технологии проектирования соответствующих решателей}
				\end{scnrelfromset}
				\begin{scnrelfromset}{декомпозиция}
					\scnitem{Разработка логических моделей решения задач и технологии проектирования соответствующих решателей}
					\scnitem{Разработка нейросетевых моделей решение задач и технологии проектирования соответствующих решателей}
				\end{scnrelfromset}
			\end{scnindent}
	        \scnitem{Разработка универсальных интерпретаторов базовых моделей обработки баз знаний интеллектуальных компьютерных систем}
	        \scnitem{Разработка общей теории человеческой деятельности, автоматизируемой с помощью комплекса взаимодействующих интеллектуальных компьютерных систем}
	    \end{scneqtoset}
    \end{scnindent}
    \scntext{примечание}{Переход от современных интеллектуальных компьютерных систем к \textit{интеллектуальным компьютерным системам нового поколения} и к соответствующей комплексной технологии не требует от специалистов в области Искусственного интеллекта изменения сферы их научных интересов. От них требуется только преодолеть синдром \scnqqi{Вавилонского столпотворения}, оформляя свои научные результаты как часть общего коллективного продукта.}
    \scntext{примечание}{Проблемы текущего этапа развития \textit{Искусственного интеллекта}, направленного на создание Общей теории и технологии \textit{интеллектуальных компьютерных систем нового поколения}, требуют \uline{фундаментального} комплексного междисциплинарного подхода и принципиально новой организации научно-технической деятельности.}  

\bigskip
\end{scnsubstruct}
\scnendcurrentsectioncomment

        \scnsegmentheader{Анализ текущего состояния и проблем дальнейшего развития деятельности в области Искусственного интеллекта}
\begin{scnsubstruct}
    \scntext{аннотация}{Рассмотрим в каких направлениях должна происходить эволюция повышенного качества деятельности в области \textit{Искусственного интеллекта}, а также эволюция продуктов этой деятельности}
    \bigskip
    \scnheader{Научно-исследовательская деятельность в области Искусственного интеллекта}
    \begin{scnrelfromset}{проблемы текущего состояния}
        \scnfileitem{Отсутствует согласованность систем \textit{понятий} в разных направлениях \textit{Искусственного интеллекта} и, как следствие, отсутствует \textit{семантическая совместимость} и \textit{конвергенция} этих направлений, в результате чего ни о каком движении в направлении построения \textit{общей теории интеллектуальных систем} с высоким уровнем формализации и речи быть не может. Существование и продолжающееся увеличение высоты барьеров между различными направлениями исследований в области \textit{Искусственного интеллекта} проявляется в том, что специалист, работающий в рамках какого-либо направления \textit{Искусственного интеллекта}, посещая заседания не своей секции на конференции по \textit{Искусственному интеллекту}, мало что там может понять и, соответственно, извлечь полезного для себя.}
        \scnfileitem{Отсутствует мотивация и осознание острой необходимости в указанной \textit{конвергенции} между различными направлениями \textit{Искусственного интеллекта}.}
        \scnfileitem{Отсутствует реальное движение в направлении построения \textit{Общей теории интеллектуальных систем}, поскольку отсутствует соответствующая мотивация и осознание острой практической необходимости в этом.}
    \end{scnrelfromset}
    \bigskip
    
    \scnheader{Разработка базовой комплексной технологии проектирования интеллектуальных компьютерных систем}
    \scntext{текущее состояние}{Современная технология \textit{Искусственного интеллекта} представляет собой целое семейство всевозможных частных технологий, ориентированных на разработку и сопровождение различного вида компонентов \textit{интеллектуальных компьютерных систем}, реализующих самые различные модели представления и обработки информации, различные модели решения задач, ориентированных на разработку различных классов \textit{интеллектуальных компьютерных систем}.}
    \begin{scnrelfromset}{проблемы текущего состояния}
        \scnfileitem{Высокая трудоемкость разработки интеллектуальных компьютерных систем.}
        \scnfileitem{Необходимая высокая квалификация разработчиков.}
        \scnfileitem{Современные технологии \textit{Искусственного интеллекта} принципиально не обеспечивают разработки таких \textit{интеллектуальных компьютерных систем}, в которых устраняются недостатки современных \textit{интеллектуальных компьютерных систем}.}
        \scnfileitem{Совместимость частных технологий \textit{Искусственного интеллекта} практически отсутствует и, как следствие, отсутствует \textit{семантическая совместимость} разрабатываемых \textit{интеллектуальных компьютерных систем}, поэтому их системная интеграция осуществляется \uline{вручную}.}
        \scnfileitem{Разрабатываемые \textit{интеллектуальные компьютерные системы} не способны \uline{самостоятельно} координировать свою деятельность друг с другом следовательно:\\
            \begin{scnitemize}
                \item нет общей комплексной технологии проектирования интеллектуальных компьютерных систем;
                \item не обеспечивается совместимость и взаимодействие разрабатываемых систем (синтаксическая и семантическая совместимость);
                \item нет совместимости между существующими частными технологиями проектирования различных компонентов интеллектуальных компьютерных систем (базы знаний, нейросетевые модели, интеллектуальные интерфейсы и т.д.);
                \item есть инструментальные средства по разработке компонентов, но склеивать (соединять, интегрировать) разработанные компоненты надо вручную, то есть нет комплексных инструментальных средств, позволяющих разрабатывать интеллектуальные системы в целом.
            \end{scnitemize}}
    \end{scnrelfromset}
    \bigskip
    
    \scnheader{Разработка технологии производства спроектированных интеллектуальных компьютерных систем}
    \scntext{текущее состояние}{Был сделан целый ряд попыток разработки \textit{компьютеров} нового поколения, ориентированных на использование в \textit{интеллектуальных компьютерных системах}. Но все они оказались неудачными, так как не были ориентированы на всё многообразие моделей решения задач в \textit{интеллектуальных компьютерных системах}. В этом смысле они не были \textit{\uline{универсальными} компьютерами} для \textit{интеллектуальных компьютерных систем}.}
    \begin{scnrelfromset}{проблемы текущего состояния}
        \scnfileitem{Разрабатываемые \textit{интеллектуальные компьютерные системы} могут использовать самые различные комбинации \textit{моделей решения интеллектуальных задач} (логических моделей, соответствующих различного вида логикам, нейросетевых моделей различного вида, моделей целеполагания, синтеза планов, моделей управления сложными объектами, моделей понимания и синтеза текстов естественного языка и т.д.). Современные (традиционные, фон-неймановские) \textit{компьютеры} не в состоянии достаточно производительно интерпретировать всё многообразие указанных моделей решения задач. При этом разработка специализированных \textit{компьютеров}, ориентированных на интерпретацию какой-либо одной модели решения задач (нейросетевой модели или какой-либо логической модели) проблему не решает, так как в \textit{интеллектуальной компьютерной системе} необходимо использовать сразу несколько разных моделей решения задач, причём в различных сочетаниях.}
    \end{scnrelfromset}
    \bigskip
    
    \scnheader{Специализированная инженерия в области Искусственного интеллекта}
    \scnidtf{Деятельность, направленная на разработку \textit{интеллектуальных компьютерных систем} различного назначения с использованием имеющихся для этого моделей, методов и средств}
    \scnidtf{Деятельность по проектированию и производству \textit{интеллектуальных компьютерных систем}}
    \scnidtf{Деятельность, направленная на формирование рынка \textit{интеллектуальных компьютерных систем}}
    \scnrelfrom{в перспективе}{Специализированная инженерия в области \textit{Искусственного интеллекта}, осуществляемая специальной частью Экосистемы OSTIS}
	    \begin{scnindent}
		    \scnrelfrom{продукт}{Экосистема OSTIS}
		    \scnrelfrom{субъект действия}{часть Экосистемы OSTIS, осуществляющая специализированную инженерию в области \textit{Искусственного интеллекта}}
	    \end{scnindent}
    \begin{scnrelfromset}{проблемы текущего состояния}
        \scnfileitem{Отсутствует четкая систематизация многообразия \textit{интеллектуальных компьютерных систем}, соответствующая систематизации автоматизируемых \textit{видов человеческой деятельности}.}
        \scnfileitem{Отсутствует \textit{конвергенция} \textit{интеллектуальных компьютерных систем}, обеспечивающих автоматизацию \textit{областей человеческой деятельности}, принадлежащих одному и тому же \textit{виду человеческой деятельности}.}
        \scnfileitem{Отсутствует \textit{семантическая совместимость}(семантическая унификация, взаимопонимание) между \textit{интеллектуальными компьютерными системами}, основной причиной чего является отсутствие согласованной системы общих используемых \textit{понятий}.}
        \scnfileitem{Семантическая недружественность \textit{пользовательского интерфейса} и отсутствие встроенной справочной системы, позволяющей запрашивать информацию об элементах интерфейса и возможностях системы, приводят к низкой эффективности эксплуатации всех возможностей \textit{интеллектуальной компьютерной системы}.}
        \scnfileitem{Анализ проблем автоматизации всех \textit{видов человеческой деятельности} убеждает в том, что дальнейшая автоматизация \textit{человеческой деятельности} требует не только повышения уровня \textit{интеллекта} соответствующих \textit{интеллектуальных компьютерных систем}, но и реализации их способности:\\
            \begin{scnitemize}
                \item устанавливать свою \textit{семантическую совместимость} (взаимопонимание) как с другими \textit{компьютерными системами}, так и со своими пользователями;
                \item поддерживать эту \textit{семантическую совместимость} в процессе собственной эволюции, а также эволюции пользователей и других \textit{компьютерных систем};
                \item координировать свою деятельность с пользователями и другими \textit{компьютерными системами} при коллективно решении различных задач;
                \item участвовать в распределении работ (подзадач) при коллективном решении различных задач.
            \end{scnitemize}
            Важно подчеркнуть то, что реализация вышеперечисленных способностей создаст возможность для существенной и даже полной автоматизации \textit{системной интеграции} \textit{компьютерных систем} в комплексы взаимодействующих систем и автоматизации реинжиниринга таких комплексов. Такая автоматизация системной интеграции и её реинжиниринга:\\
            \begin{scnitemize}
                \item даст возможность комплексам кибернетических систем \uline{самостоятельно} адаптироваться к решению новых задач;
                \item существенно повысит эффективность эксплуатации таких комплексов компьютерных систем, так как реинжиниринг системной интеграции компьютерных систем, входящих в такой комплекс, часто востребован (например, при реконструкции предприятия);
                \item существенно сокращает число ошибок по сравнению с ручным (неавтоматизированным) выполнением \textit{системной интеграции} и её \textit{реинжиниринга}, которые, к тому же, требует высокой квалификации.
            \end{scnitemize}
            Таким образом следующий этап повышения уровня автоматизации \textit{человеческой деятельности} настоятельно требует создания таких \textit{интеллектуальных компьютерных систем}, которые могли бы легко сами (без системного интегратора) объединяться для совместного решения сложных задач. }
    \end{scnrelfromset}
    \bigskip
    
    \scnheader{Образовательная деятельность в области искусственного интеллекта}
    \scntext{текущее состояние}{Целенаправленная подготовка специалистов в области Искусственного интеллекта имеет богатую историю и осуществляется во многих ведущих университетах (Stanford University, MIT, МГУ (Москва), НИУ МЭИ (Москва), РГГУ (Москва), СПбГУ (Санкт-Петербург), ДВФУ (Владивосток), НГТУ (Новосибирск), НТУУ КПИ (Киев), БГУИР (Минск), БГУ (Минск), БрГТУ (Брест) и других).}
    \begin{scnrelfromset}{проблемы текущего состояния}
        \scnfileitem{Поскольку деятельность в области \textit{Искусственного интеллекта} сочетает в себе и высокую степень наукоемкости и высокую степень сложности инженерных работ, подготовка специалистов в этой области требует одновременного формирования у них как научно-исследовательских навыков, культуры и стиля мышления, так и инженерно-практических навыков, культуры и стиля мышления. С точки зрения методики и психологии обучения сочетание фундаментальной научной и инженерно-практической подготовки специалистов является весьма сложный образовательной педагогической задачей.}
        \scnfileitem{Отсутствует \textit{семантическая совместимость} между различными учебными дисциплинами, что приводит к мозаичности восприятия информации}
        \scnfileitem{Отсутствует системный подход к подготовке молодых специалистов в области \textit{Искусственного интеллекта}}
        \scnfileitem{Нет персонификации обучения, а также установки на выявление, раскрытие и развитие индивидуальных способностей}
        \scnfileitem{Отсутствует целенаправленное формирование мотивации к творчеству}
        \scnfileitem{Нет формирования навыков работы в реальных коллективах разработчиков}
        \scnfileitem{Отсутствует адаптация к реальной практической деятельности}
        \scnfileitem{Любая современная технология (в том числе и Технология OSTIS) должна иметь высокие темпы своего развития, поскольку без этого невозможно поддерживать высокий уровень её конкурентоспособности. Но для быстро развиваемой технологии требуется:\\
            \begin{scnitemize}
                \item не просто высокая квалификация кадров, использующих и развивающих технологию,
                \item но и высокие \uline{темпы} повышения уровня этой квалификации, так как без этого невозможно эффективно использовать и развивать \uline{быстро меняющуюся} технологию.
            \end{scnitemize}
            \bigskip 
            Из этого следует, что образовательная деятельность в области \textit{Искусственного интеллекта} и соответствующая ей технология должна быть не просто важной частью деятельности в области \textit{Искусственного интеллекта}, а частью, глубоко интегрированной во все остальные виды деятельности в области \textit{Искусственного интеллекта}. Так, например, каждая \textit{интеллектуальная компьютерная система} должная быть ориентирована не только на обслуживание своих конечных пользователей, не только на организацию целенаправленного взаимодействия со своими разработчиками, которые постоянно совершенствуют эту систему, и не только на обеспечение минимального порога вхождения для новых конечных пользователей и разработчиков, но и на организацию постоянного и персонифицированного повышения квалификации каждого своего конечного пользователя и разработчика в условиях постоянных изменений, вносимых в указанную \textit{интеллектуальную компьютерную систему}. Для этого эксплуатируемая \textit{интеллектуальная компьютерная система} должна знать, что в ней изменилось, на что она способна и как эти способности инициировать (содержание и форма, соответствующих пользовательских команд)}
    \end{scnrelfromset}
    \bigskip
    
    \scnheader{Бизнес-деятельность в области Искусственного интеллекта}
    \scntext{текущее состояние}{Острая потребность в существенном повышении уровня автоматизации в самых различных областях человеческой деятельности (в промышленности, медицине, транспорте, образовании, строительстве и во многих других), а также современные результаты в развитии \textit{технологий Искусственного интеллекта} привели к существенному расширению работ по созданию \textit{прикладных интеллектуальных компьютерных систем} и к появлению большого количества коммерческих организаций, ориентированных на разработку таких приложений.}
    \begin{scnrelfromset}{проблемы текущего состояния}
        \scnfileitem{Не так просто обеспечить баланс тактических и стратегических направлений развития всех форм деятельности в области \textit{Искусственного интеллекта} (научно-исследовательской деятельности, разработки технологии проектирования и производства интеллектуальных компьютерных систем, разработки прикладных систем, образовательной деятельности), а также баланс между всеми перечисленными формами деятельности.}
        \scnfileitem{В настоящее время отсутствует глубокая конвергенция различных форм деятельности в области \textit{Искусственного интеллекта} (в первую очередь, конвергенция развития технологий \textit{Искусственного интеллекта} и разработки различных прикладных интеллектуальных компьютерных систем), что существенно затрудняет развитие каждой из этих форм.}
        \scnfileitem{Высокий уровень наукоемкости работ в области \textit{Искусственного интеллекта} предъявляет особые требования к квалификации сотрудников и к их способности работать в составе творческих коллективов.}
        \scnfileitem{Для повышения квалификации своих сотрудников и для обеспечения высокого уровня своих разработок необходимо активное сотрудничество с различными научными школами, с кафедрами, осуществляющими подготовку молодых специалистов в области \textbf{\textit{Искусственного интеллекта}}, активное участие в подготовке и проведении соответствующих конференций, семинаров, выставок.}
    \end{scnrelfromset}
    \bigskip
    
    \scnheader{Искусственный интеллект}
    \begin{scnrelfromset}{\scnkeyword{сверхзадачи текущего состояния}}
        \scnfileitem{Построение и перманентное развитие \textit{общей формальной теории интеллектуальных систем}}
	        \begin{scnindent}
		        \begin{scnrelfromset}{подзадачи}
		            \scnfileitem{Уточнение требований, предъявляемых к интеллектуальным компьютерным системам уточнение свойств интеллектуальных компьютерных систем, определяющих высокий уровень их интеллекта.}
		            \scnfileitem{Конвергенция и интеграция всевозможных видов знаний и всевозможных моделей решения задач в рамках каждой интеллектуальной компьютерной системы.}
		            \scnfileitem{Ориентация на последующую разработку унифицированных семантически совместимых формальных моделей интеллектуальных систем.}
		            \scnfileitem{Ориентация на разработку различного вида универсальных интерпретаторов формальных моделей интеллектуальных систем (и в том числе компьютеров нового поколения ) и обеспечение четкой стратификации между формальными моделями интеллектуальных систем и различными вариантами построения их интерпретаторов, обеспечивающей высокую степень независимости эволюции формальных моделей интеллектуальных систем и эволюции их интерпретаторов. Это требует особой детализации формальных моделей интеллектуальных систем.}
		            \scnfileitem{Обеспечение коммуникационной (социальной) совместимости (договороспособности) интеллектуальных компьютерных систем, позволяющей им самостоятельно формировать коллективы интеллектуальных компьютерных систем и их пользователей, а также самостоятельно согласовывать (координировать) деятельность в рамках этих коллективов при решении сложных задач в непредсказуемых условиях. Без этого невозможна реализация таких проектов, как умный дом, умный город, умное предприятие, умная больница и т.д.}
		        \end{scnrelfromset}
		    \end{scnindent}
        \scnfileitem{Создание и перманентное развитие \textit{общей комплексной технологии} проектирования и производства \textit{семантически совместимых} \textit{интеллектуальных компьютерных систем}, способных координировать свою деятельность с себе подобными}
        	\begin{scnindent}
		        \begin{scnrelfromset}{подзадачи}
		            \scnfileitem{Четкое описание стандарта интеллектуальных компьютерных систем, обеспечивающего семантическую совместимость разрабатываемых систем}
		            \scnfileitem{Разработка мощных библиотек семантически совместимых и многократно (повторно) используемых компонентов разрабатываемых интеллектуальных компьютерных систем}
		            \scnfileitem{Обеспечение низкого порога вхождения в технологию проектирования интеллектуальных компьютерных систем как для пользователей технологии (т.е. разработчиков прикладных или специализированных интеллектуальных компьютерных систем), так и для разработчиков самой технологии}
		            \scnfileitem{Обеспечение высоких темпов развития технологии за счет учета опыта разработки различных приложений путем активного привлечения авторов приложений к участию в развитии (совершенствовании) технологии}
		        \end{scnrelfromset}
		 	\end{scnindent}
        \scnfileitem{Разработка компьютеров нового поколения, ориентированных на производство высокопроизводительных \textit{интеллектуальных компьютерных систем} самого различного назначения и высокого качества}
        \scnfileitem{Создание глобальной \textit{экосистемы} взаимодействующих между собой \textit{интеллектуальных компьютерных систем}, обеспечивающих комплексную автоматизацию всех \textit{видов человеческой деятельности}}
	        \begin{scnindent}
	        	\scntext{подзадача}{Построение формальной модели человеческой деятельности в контексте теории smart-общества}
	        \end{scnindent}
        \scnfileitem{Создание и перманентное развитие глобальной \textit{социотехнической экосистемы}, которая состоит из \textit{интеллектуальных компьютерных систем}, а также всех пользователей этих систем, которая обеспечивает комплексную автоматизацию всех \textit{видов человеческой деятельности}}
        \scnfileitem{Необходим переход от эклектичного построения сложных \textit{интеллектуальных компьютерных систем}, использующих различные виды \textit{знаний} и различные виды \textit{моделей решения задач}, к их глубокой \textit{\textbf{интеграции}} и унификации, когда одинаковые модели представления и модели обработки знаний реализуется в разных системах и подсистемах одинаково}
        \scnfileitem{Необходимо сократить дистанцию между современным уровнем \textbf{\textit{теории интеллектуальных компьютерных систем}} и практики их разработки.}
    \end{scnrelfromset}
    \scnidtf{Деятельность в области Искусственного интеллекта (как совокупность всех форм и направлений этой деятельности)}
    \scntext{проблема текущего состояния}{Эпицентром современных проблем развития деятельности в области \textit{Искусственного интеллекта} является \textit{конвергенция} и \textit{глубокая интеграция} всех форм, направлений и результатов этой деятельности. Уровень взаимосвязи, взаимодействия и \textit{конвергенции} между различными формами и направлениями деятельности в области \textit{Искусственного интеллекта} явно недостаточен. Это приводит к тому, что каждая из них развивается обособленно, независимо от других.  Речь идет о \textit{конвергенции} между такими направлениями \textit{Искусственного интеллекта}, как представление знаний, решение интеллектуальных задач, интеллектуальное поведение, понимание и др., а также между такими формами \textit{человеческой деятельности в области Искусственного интеллекта}, как научные исследования, разработка технологий, разработка приложений, образование, бизнес. Почему на фоне уже достаточно длительного интенсивного развития научных исследований в области \textit{Искусственного интеллекта} до сих пор не создан рынок интеллектуальных компьютерных систем и комплексная технология \textit{Искусственного интеллекта}, обеспечивающая разработку широкого спектра \textit{интеллектуальных компьютерных систем} самого различного назначения и доступной широкому контингенту инженеров. Потому что сочетание высокого уровня наукоемкости и прагматизма этой проблемы требует для ее решения принципиально нового подхода к организации взаимодействия \textit{\uline{ученых}}, работающих в области \textit{Искусственного интеллекта}, \textit{\uline{разработчиков}} средств автоматизации проектирования \textit{интеллектуальных компьютерных систем}, \uline{\textit{разработчиков}} средств реализации интеллектуальных компьютерных систем, включая средства аппаратной поддержки интеллектуальных компьютерных систем, \uline{\textit{разработчиков}} прикладных интеллектуальных компьютерных систем. Такое \uline{целенаправленное} взаимодействие должно осуществляться как в рамках каждой из этих форм деятельности в области \textit{Искусственного интеллекта}, так и между ними. Таким образом, основной тенденцией дальнейшего развития теоретических и практических работ в области \textit{Искусственного интеллекта} является конвергенция как самых разных видов (форм и направлений) человеческой деятельности в области \textit{Искусственного интеллекта}, так и самых разных продуктов (результатов) этой деятельности. Необходимо ликвидировать барьеры между различными видами и продуктами деятельности в области \textit{Искусственного интеллекта} в целях обеспечения их совместимости и интегрируемости.Проблема создания быстро развивающегося рынка семантически совместимых интеллектуальных систем  это вызов, адресованный специалистам в области \textit{Искусственного интеллекта}, требующий преодоления вавилонского столпотворения во всех его проявлениях, формирование высокой культуры договороспособности и унифицированной, согласованной формы представления коллективно накапливаемых, совершенствуемых и используемых знаний.Ученые, работающие в области \textit{Искусственного интеллекта}, должны обеспечить конвергенцию результатов различных направлений \textit{Искусственного интеллекта} и построить:
        \begin{scnitemize}
            \item общую теорию интеллектуальных компьютерных систем;
            \item общую технологию проектирования семантически совместимых интеллектуальных компьютерных систем, включающую соответствующие стандарты интеллектуальных компьютерных систем и их компонентов. Инженеры, разрабатывающие интеллектуальные компьютерные системы, должны сотрудничать с учеными и участвовать в развитии технологии проектирования интеллектуальных компьютерных систем.
        \end{scnitemize}
    }
    
    \scnheader{конвергенция в области Искусственного интеллекта}
    \scnrelfrom{разбиение}{Направления конвергенции в области Искусственного интеллекта}
    \begin{scnindent}
	    \scnhaselement{конвергенция Искусственного интеллекта со смежными научными дисциплинами}
	    \begin{scnindent}
            \begin{scnrelfromset}{примечание}
                \scnitem{Искусственный интеллект}
                \begin{scnindent}
                    \begin{scnrelbothlist}{смежная дисциплина}
                    \scnitem{Логика}
                    \scnitem{Психология человека}
                    \scnitem{Зоопсихология}
                    \scnitem{Нейропсихология}
                    \scnitem{Этология}
                    \scnitem{Кибернетика}
                    \scnitem{Общая теория систем}
                    \scnitem{Семиотика}
                    \scnitem{Лингвистика}
                    \end{scnrelbothlist}
                \end{scnindent}
            \end{scnrelfromset}
        \end{scnindent}
	    \scnhaselement{конвергенция различных направлений Искусственного интеллекта}
	    \begin{scnindent}
		    \scnidtf{Конвергенция различных направлений исследований в области Искусственного интеллекта, результатом которой должна быть формализованная практически ориентированная общая теория интеллектуальных систем и, в частности, интеллектуальных компьютерных систем}
		    \scnidtf{Конвергенция между различными направлениями и продуктами научных исследований в области искусственного интеллекта, результатом (целевым продуктом) которой должна стать общая формальная теория интеллектуальных компьютерных систем}
		    \scntext{примечание}{Разобщенность различных направлений исследований в области искусственного интеллекта является главным препятствием создания общей комплексной технологии проектирования интеллектуальных компьютерных систем}
	    \end{scnindent}
	    \scnhaselement{конвергенция различного вида знаний в памяти интеллектуальной компьютерной системы}
	    \begin{scnindent}
	    	\scnidtf{Конвергенция и интеграция внутреннего представления в памяти интеллектуальной компьютерной системы различного вида знаний}
	    \end{scnindent}
	    \scnhaselement{конвергенция различных моделей решения задач в памяти интеллектуальной компьютерной системы}
	    \begin{scnindent}
	    	\scnidtf{Конвергенция и интеграция различных моделей решения задач, которая включает логико-семантическую типологию задач и типологию моделей решения задач и требует уточнения семантики таких понятий как задача, класс задач, метод, класс методов, модель решения задач (иерархический метод интерпретации класса методов)}
	    \end{scnindent}
	    \scnhaselement{конвергенция интеллектуальных компьютерных систем}
	    \begin{scnindent}
		    \scnidtf{Обеспечение семантической совместимости (взаимопонимания) интеллектуальных систем, согласование используемых онтологий}
		    \scnidtf{Конвергенция между различными прикладными компьютерными системами, результатом (целевым продуктом) которой должна стать экосистема, состоящая из перманентно эволюционирующих, семантически совместимых и взаимодействующих интеллектуальных компьютерных систем, а также их пользователей}
		    \scntext{пояснение}{Конвергенция (семантическая совместимость) всех разрабатываемых интеллектуальных компьютерных систем (в том числе прикладных), преобразующая набор индивидуальных (самостоятельных) интеллектуальных компьютерных систем различного назначения в коллектив активно взаимодействущих интеллектуальных компьютерных систем для совместного (коллективного) решения сложных (комплексных) задач и для перманентной поддержки семантической совместимости в ходе индивидуальной эволюции каждой интеллектуальной компьютерной системы.}
	    \end{scnindent}
	    \scnhaselement{конвергенция средств автоматизации проектирования различного вида компонентов интеллектуальных компьютерных систем}
	    \begin{scnindent}
		    \scnidtf{Конвергенция (семантическая совместимость) средств автоматизации проектирования различного вида компонентов интеллектуальных компьютерных систем, результатом которой должен быть общий комплекс средств автоматизации проектирования всех компонентов интеллектуальных компьютерных систем}
		    \scnidtf{Конвергенция между инструментальными средствами, обеспечивающими автоматизацию проектирования различных компонентов или различных классов интеллектуальных компьютерных систем, результатом (целевым продуктом) которой должен стать единый комплекс методологических и инструментальных средств, ориентированный на поддержку комплексного проектирования любых интеллектуальных компьютерных систем}
	    \end{scnindent}
	    \scnhaselement{конвергенция логико-семантических моделей интеллектуальных компьютерных систем}
	    \begin{scnindent}
	    	\scntext{примечание}{\textit{логико-семантические модели интеллектуальных компьютерных систем} являются результатом (сухим остатком) \textit{проектирования} этих систем и представляют собой формальное представления исходного (начального) состояния \textit{баз знаний} разрабатываемых \textit{интеллектуальных компьютерных систем}}
	    \end{scnindent}
	    \scnhaselement{конвергенция средств интерпретации логико-семантических моделей разрабатываемых интеллектуальных компьютерных систем}
	    \begin{scnindent}
	    	\scntext{пояснение}{Конвергенция (совместимость) средств реализации (производства) интеллектуальных компьютерных систем на основе спроектированных формальных моделей создаваемых интеллектуальных компьютерных систем (средств интерпретации спроектированных моделей интеллектуальных компьютерных систем). Такая интерпретация может осуществляться либо программным путем на современных компьютерах, либо путем создания принципиально новых компьютеров, специально ориентированных на интерпретацию формальных моделей интеллектуальных компьютерных систем, помещаемых в память указанных компьютеров}
	    \end{scnindent}
	    \scnhaselement{конвергенция между информационно-программным и аппаратным обеспечением интеллектуальных компьютерных систем}
	    \begin{scnindent}
	    	\scnidtf{Конвергенция между Software и Hardware интеллектуальных компьютерных систем}
	    \end{scnindent}
	    \scnhaselement{Конвергенция различных форм деятельности в области Искусственного интеллекта}
	    \begin{scnindent}
	    \scnidtftext{пояснение}{Конвергенция между:
	        \begin{scnitemize}
	            \item научными исследованиями по созданию общей теории интеллектуальных компьютерных систем;
	            \item разработкой средств автоматизации проектирования интеллектуальных компьютерных систем;
	            \item разработкой средств интерпретации спроектированных формальных моделей интеллектуальных компьютерных систем;
	            \item разработкой прикладных интеллектуальных компьютерных систем различного назначения;
	            \item подготовкой и перманентным повышением квалификации кадров, способных эффективно участвовать во всех перечисленных направлениях деятельности.
	        \end{scnitemize} }
	    \scnidtf{Конвергенция между:
	        \begin{scnitemize}
	            \item научно-исследовательской деятельностью в области искусственного интеллекта;
	            \item инженерно-технологической деятельностью, которая направлена на разработку комплексной технологии проектирования интеллектуальных компьютерных систем и которая имеет высокий уровень наукоемкости;
	            \item инженерно-прикладной деятельностью, которая направлена на разработку прикладных интеллектуальных систем и которая также имеет высокий уровень наукоемкости, обусловленной необходимостью качественной формализации соответствующих предметных областей и, в частности, методов решения задач в этих областях;
	            \item образованием (образовательной деятельностью) в области искусственного интеллекта, повышение эффективности которого настоятельно требует раннего и поэтапного вовлечения студентов в реальные, а не учебные проекты --- сначала в инженерно-прикладные, потом в инженерно- исследовательские проекты;
	            \item деятельностью, направленной на создание инфраструктуры, обеспечивающей поддержку открытого массового активного международного сотрудничества по консолидации усилий, направленных на решение современных проблем в области искусственного интеллекта;
	            \item бизнесом в области искусственного интеллекта, который не просто должен обеспечить финансовую поддержку перечисленных видов деятельности, но и обеспечить грамотный баланс между ними, грамотное сочетание тактических и стратегических целей
	        \end{scnitemize} }
	    \scntext{примечание}{Глубокая конвергенция между всеми этими формами деятельности возможна только тогда, когда \uline{каждый} участник создания комплексной технологии искусственного интеллекта является участником \uline{каждой} из перечисленных форм деятельности.}
	    \end{scnindent}
    \end{scnindent}
    	
    \scnheader{Искусственный интеллект}
    \begin{scnrelfromset}{\scnkeyword{методологические проблемы текущего состояния}}
        \scnfileitem{Далеко не всеми учеными, работающими в области искусственного интеллекта принимается прагматичность практической направленности этой науки}
        \scnfileitem{Не всеми принимается необходимость конвергенции различных направлений искусственного интеллекта и необходимость их интеграции в целях построения общей теории интеллектуальных систем}
        \scnfileitem{Нет движения к построению общей компьютерной технологии интеллектуальных компьютерных систем}
        \scnfileitem{Нет движения к построению экосистем интеллектуальных компьютерных систем}
        \scnfileitem{Не всеми принимается необходимость конвергенции различных форм деятельности в области Искусственного интеллекта}
    \end{scnrelfromset}
    \scntext{примечание}{Современная трактовка целей и задач \textit{Искусственного интеллекта} как научно-технической дисциплины требует переосмысления, так как, к сожалению, носит несогласованный, а часто и значительно более узкий характер, чем этого требует текущее положение}
    \bigskip
    
    \scnheader{следует отличать*}
    \begin{scnhaselementset}
        \scnitem{конвергенция}
	        \begin{scnindent}
	        	\scnidtf{Процесс сближения структурных и/или функциональных характеристик нескольких (как минимум двух) заданных сущностей}
	            \scnidtf{Процесс конвергенции заданных сущностей в ходе их изменения, совершенствование, эволюции}
	            \scnsubset{процесс}
	         \end{scnindent}
        \scnitem{конвергенция\scnsupergroupsign}
        	\begin{scnindent}
	            \scnidtf{Степень близости (сходство) заданных сущностей}
	            \scniselement{свойство}
        	\end{scnindent}
    \end{scnhaselementset}
    
    \scnheader{конвергенция}
    \scntext{примечание}{\textit{Конвергенция} пар конкретных искусственных сущностей (например, технических систем) есть стремление их унификацию (в частности, к стандартизации), т.е. стремление к минимизации многообразия форм решения аналогичных практических задач --- стремление к тому, чтобы все, что можно сделать одинаково, сделалось одинаково, но без ущерба требуемого качества. Последнее очень важно, так как безграмотная стандартизация может привести к существенному торможению прогресса. Ограничение многообразия форм не должно приводить к ограничению содержания, возможностей. Образно говоря, словам должно быть тесно, а мыслям --- свободно.}
    \scntext{примечание}{Методологически конвергенция искусственно создаваемых сущностей (артефактов) сводится (1) к выявлению (обнаружению) принципиальных сходств между этими сущностями, которые часто весьма закамуфлированы и их трудно увидеть, и (2) к реализации обнаруженных сходств одинаковым образом (в одинаковой форме, в одинаковом синтаксисе). Образно говоря, от семантической (смысловой) эквивалентности требуется перейти и к синтаксической эквивалентности. Кстати, в этом как раз и заключается суть (идея) смыслового представления информации (знаний), целью которого является создание такой языковой среды (\textit{смыслового пространства}), в рамках которого (1) семантически эквивалентные информационные конструкции полностью совпадали, а (2) конвергенция информационных конструкций сводилась бы к выявлению изоморфных фрагментов этих конструкций.}
    \scntext{примечание}{Очень важно уточнить, формализовать понятие конвергенции (конвергенции знаний, методов, модели решения задач, конвергенции интеллектуальных компьютерных систем в целом)}
    \scnsuperset{конвергенция информационных конструкций}
	    \begin{scnindent}
	    	\scnidtf{конвергенция синтаксических и семантических свойств информационных конструкций}
	    \end{scnindent}
    \scnsuperset{конвергенция языков}
    \scnsuperset{конвергенция научных дисциплин}
    	\begin{scnindent}
    		\scnidtf{конвергенция различных научных дисциплин или различных направлений одной и той же и дисциплины}
    	\end{scnindent}
    \scnsuperset{конвергенция баз знаний}
    \scnsuperset{конвергенция моделей решения задач}
    \scnsuperset{конвергенция гибридных решателей задач}
    \scnsuperset{конвергенция кибернетических систем}
    \scnsuperset{конвергенция интеллектуальных систем}
    	\begin{scnindent}
    		\scnsuperset{конвергенция интеллектуальных систем, направленная на обеспечение их \uline{семантической совместимости}}
    	\end{scnindent}
    
    \scnheader{конвергенция результатов научно-технической деятельности}
    \scntext{примечание}{Важным препятствием для конвергенции результатов научно-технической деятельности является сформировавшийся в науке и технике акцент на выявлении не сходств, а отличий. Чтобы убедиться в этом достаточно обратить внимание на то, что уровень научных результатов оценивается научной \uline{новизной}, которая может имитироваться новизной не по существу, а по форме представления (например, с помощью новых понятий или даже новых терминов). Результаты в технике, например, в патентах также оцениваются \uline{отличиями} от предшествующих технических решений. Но для конвергенции нужны другие акценты --- ни поиск отличий, а выявление неочевидных сходств и превращения их в очевидные сходства, представленные в одинаковой \uline{форме}.}
    
    \scnheader{совместимость\scnsupergroupsign}
    \scnidtf{совместимость заданных двух или более сущностей\scnsupergroupsign}
    \scnidtf{простота интеграции заданной группы сущностей\scnsupergroupsign}
    \scnidtf{интегрируемость\scnsupergroupsign}
    \scntext{примечание}{Степень (уровень) совместимости заданных сущностей может рассматриваться как оценка результата их конвергенции. Чем качественнее (основательнее, глубже) проведена конвергенция заданных сущностей, тем выше уровень их совместимости и, собственно, тем легче их интегрировать.}
    \scnsuperset{cовместимость информационных конструкций\scnsupergroupsign}
    	\begin{scnindent}
    		\scnsuperset{семантическая совместимость информационных конструкций\scnsupergroupsign}
    	\end{scnindent}
    \scnsuperset{совместимость языков\scnsupergroupsign}
    	\begin{scnindent}
    		\scnsuperset{семантическая совместимость языков\scnsupergroupsign}
    	\end{scnindent}
    \scnsuperset{семантическая совместимость научных дисциплин\scnsupergroupsign}
    \scnsuperset{совместимость баз знаний\scnsupergroupsign}
    \scnsuperset{совместимость моделей решения задач\scnsupergroupsign}
    \scnsuperset{совместимость кибернетических систем\scnsupergroupsign}
    	\begin{scnindent}
    		\scnsuperset{семантическая совместимость кибернетических систем\scnsupergroupsign}
    	\end{scnindent}
    \scnsuperset{семантическая совместимость\scnsupergroupsign}
    
    \scnheader{интеграция*}
    \scnidtf{объединение нескольких разных сущностей, в результате чего возникает некоторая объединённая целостная сущность*}
    \scnsuperset{эклектичная интеграция*}
    	\begin{scnindent}
    		\scnidtf{Интеграция разнородных (гетерогенных) сущностей, которой не предшествует конвергенция (сближение) этих сущностей*}
    	\end{scnindent}
    \scnsuperset{глубокая интеграция*}
    \scntext{примечание}{Понятие \textit{интеграции*} и особенно понятие \textit{глубокой интеграции*} имеет тесную связь с понятием \textit{конвергенции\scnsupergroupsign}. Чем выше степень конвергенции (степень сближения) интегрируемых объектов, тем выше качество результата интеграции. Особенно, если речь идёт о глубокой интеграции.}
    
    \scnheader{глубокая интеграция*}
    \scnidtf{бесшовная интеграция*}
    %TODO ссылка на Грибову
    \scnidtf{интеграция однородных сущностей, предполагающая глубокую взаимную диффузию (сращивание) соединяемых сущностей, которая не обязательно должна осуществляться физически}
    \scntext{примечание}{Примером виртуальной глубокой интеграции является формирование коллектива \uline{семантический совместимых} индивидуальный кибернетических систем}
    \scnidtf{бесшовная интеграция*}
    \scnidtf{гибридизация*}
    \scnidtf{интеграция, результатом которой являются гибридные объекты*}
    \scnidtf{интеграция, которой предшествует высокий уровень конвергенции интегрируемых объектов*}
    \scnidtf{(конвергенция + интеграция)*}
    \scnidtf{бесшовная интеграция}
    \scnidtf{интеграция, в результате которой возникает гибридная система*}
    \scnidtf{интеграция, которой предшествует конвергенция (в частности, унификация) интегрируемых систем, приведение этих систем к максимально похожему виду (общему знаменателю)*}
    %TODO сложно при чтении воспринимать, конвергенция и приведение как-то сливаются, становится не совсем понятно, к чему относится приведение к конвергенции или к интеграции, может как-то более явно указать, что конвергенция это то приведение?
    \scnidtf{интеграция с диффузией , взаимопроникновением на основе унификации того, что можно сделать одинаковым*}
    
    \scnheader{интеграция*}
    \scnsuperset{интеграция информационных конструкций}
    \scnsuperset{интеграция языков}
    \scnsuperset{интеграция научных дисциплин}
    \scnsuperset{интеграция баз знаний}
    \scnsuperset{интеграция моделей решения задач}
    \scnsuperset{интеграции индивидуальных кибернетических систем}
    \begin{scnindent}
    	\scnsuperset{слияние индивидуальных кибернетических систем}
    		\begin{scnindent}
    			\scnidtf{преобразование нескольких \uline{искусственных} индивидуальных кибернетических систем в интегрированную индивидуальную кибернетическую систему, которая способна решать все задачи, каждая из которых могла бы быть решена в рамках какой-либо из интегрируемых систем}
    		\end{scnindent}
    	\scnsuperset{формирование коллектива индивидуальных кибернетических систем}
    		\begin{scnindent}
    			\scnidtf{формирования многоагентной системы, состоящей из индивидуальных кибернетических систем}
    		\end{scnindent}
    	\scntext{примечание}{Эффективность интеграции индивидуальных кибернетических систем определяется тем, насколько объем задач, решаемых коллективом индивидуальных кибернетических систем, превысит объединение объёмов задач, решаемых членами коллектива в отдельности.}	
    \end{scnindent}
    
    \bigskip
\end{scnsubstruct}
\scnendcurrentsectioncomment

        \scnsegmentheader{Уточнение Понятия Технологии OSTIS}
\begin{scnsubstruct}
	
    \scnheader{Технология OSTIS}
    \scnidtf{Комплекс (семейство) технологий, обеспечивающих проектирование, производство, эксплуатацию и реинжиниринг интеллектуальных \textit{компьютерных систем} (\textit{ostis-систем}), предназначенных для автоматизации самых различных видов человеческой деятельности и в основе которых лежит смысловое представление и онтологическая систематизация знаний, а также агентно-ориентированная обработка знаний}
    \scnidtf{Open Semantic Technology for Intelligent Systems}
	    \begin{scnindent}
	    	\scntext{сокращение}{OSTIS}
	    \end{scnindent}
    \scnidtf{Семейство (комплекс) \textit{ostis-технологий}}
    \scnidtf{Комплексная открытая семантическая технология проектирования, производства, эксплуатации и реинжиниринга гибридных, семантически совместимых, активных и договороспособных \textit{интеллектуальных компьютерных систем}}
    \begin{scnrelfromset}{принципы, лежащие в основе}
        \scnfileitem{Ориентация на разработку \textit{интеллектуальных компьютерных систем}, имеющих высокий уровень \textit{интеллекта} и, в частности, высокий уровень \textit{социализации}. Указанные системы, разработанные по \textit{Технологии OSTIS}, будем называть \textbf{\textit{ostis-системами}}}
        \scnfileitem{Ориентация на \uline{комплексную} автоматизацию всех видов и областей \textit{человеческой деятельности} путем создания сети взаимодействующих и координирующих свою деятельность \textit{ostis-систем}. Указанную сеть \textit{ostis-систем} вместе с их пользователями будем называть \textbf{\textit{Экосистемой OSTIS}}}
        \scnfileitem{Поддержка перманентной эволюции \textit{ostis-систем} в ходе их эксплуатации.}
        \scnfileitem{\textit{Технология OSTIS} реализуется в виде сети \textit{ostis-систем}, которая является частью \textit{Экосистемы OSTIS}.Ключевой \textit{ostis-системой} указанной сети является \textbf{\textit{Метасистема IMS.ostis}} (Intelligent MetaSystem for ostis-systems), реализующая \textbf{\textit{Ядро Технологии OSTIS}}, которое включает в себя базовые (предметно независимые) методы и средства проектирования и производства \textit{ostis-систем} с интеграцией в их состав типовых встроенных подсистем поддержки эксплуатации и реинжиниринга \textit{ostis-систем}. Остальные \textit{ostis-системы}, входящие в состав рассматриваемой сети, реализуют различные специализированные \textit{ostis-технологии} проектирования различных классов \textit{ostis-систем}, обеспечивающих автоматизацию любых областей и \textit{видов человеческой деятельности}, кроме \textit{проектирования ostis-систем}.}
        \scnfileitem{Конвергенция и интеграция на основе \textit{смыслового представления знаний} всевозможных научных направлений \textit{Искусственного интеллекта} (в частности, всевозможных базовых знаний и навыков решения \textit{интеллектуальных задач}) в рамках \textit{Общей формальной семантической теории \mbox{ostis-систем}}.}
        \scnfileitem{Ориентация на разработку компьютеров нового поколения, обеспечивающих эффективную (в том числе производительную) интерпретацию логико-семантических моделей \textit{ostis-систем}, которые представлены \textit{базами знаний} этих систем, имеющими \textit{смысловое представление}.}
    \end{scnrelfromset}
\end{scnsubstruct}

\bigskip\scnheader{Понятие ostis-системы}
\begin{scnsubstruct}
    \scnheader{ostis-система}
    \scnidtf{\textit{интеллектуальная компьютерная система}, спроектированная и реализованная по требованиям и стандартам \textit{Технологии OSTIS}, которые задокументированы в \textit{Общей теории ostis-систем}}
    \scnidtf{Множество \textit{ostis-систем} различного назначения}
	    \begin{scnindent}
	    	\scniselement{имя собственное}	
	    \end{scnindent}
    \scnidtf{Множество всевозможных \textit{интеллектуальных компьютерных систем}, построенных по \textit{Технологии OSTIS}}
    \scnsubset{интеллектуальная компьютерная система}
    \scnidtf{\textit{интеллектуальная компьютерная система}, которая построена в соответствии с требованиями и стандартами \textit{Технологии OSTIS}, что обеспечивает существенное развитие целого ряда \textit{свойств} (способностей) этой \textit{компьютерной системы}, позволяющих значительно повысить \textit{уровень интеллекта} этой системы (и, прежде всего, ее \textit{уровень обучаемости} и \textit{уровень социализации})}
    \begin{scnsubdividing}
        \scnitem{индивидуальная ostis-система}
        \scnitem{коллективная ostis-система}
        	\begin{scnindent}
	            \begin{scnsubdividing}
	                \scnitem{простой коллектив ostis-систем}
	                \scnitem{иерархический коллектив ostis-систем}
	            \end{scnsubdividing}
            \end{scnindent}
    \end{scnsubdividing}
    \scntext{примечание}{Когда речь идет о таком компоненте \textit{Технологии OSTIS}, как \textit{Общая теория ostis-систем}, имеется в виду строгое формальное уточнение того, как устроена \textit{ostis-система}, какова ее архитектура, принципы организации памяти, принципы организации представления и обработки информации, принципы организации интерфейса с внешней средой (в том числе, с пользователями)}
    \begin{scnrelfromset}{принципы, лежащие в основе}
        \scnfileitem{Хранение информации в памяти \textit{ostis-системы} ориентируется на \textit{\uline{смысловое} представление информации} --- без синонимии и омонимии знаков, без семантической эквивалентности информационных конструкций, т.е. без дублирования информации.}
        \begin{scnindent}
	        \scnrelfrom{ключевой знак}{\scnkeyword{смысловое представление информации}}
	        	\begin{scnindent}
		            \scnidtf{смысл представленной информационной конструкции}
		            \begin{scnrelfromvector}{принципы, лежащие в основе}
		                \scnfileitem{В рамках смыслового представления информационной конструкции все \textit{знаки}, входящие в эту \textit{информационную конструкцию} уникальны, т.е. обозначают \uline{разные} описываемые \textit{сущности}. Другими словами, в рамках \textit{смыслового представления информационной конструкции} запрещено присутствие \textit{синонимичных знаков}.}
		                \scnfileitem{В рамках \textit{смыслового представления информационной конструкции} \uline{все} \textit{сущности}, описываемые этой \textit{информационной конструкцией}, должны быть \uline{явно} представлены своим \textit{знаком}.}
		                \scnfileitem{Каждый \textit{знак}, входящий в \textit{смысловое представление информационной конструкции} является \textit{синтаксически элементарным} (атомарным) \textit{фрагментом} этой конструкции, внутреняя структура которого несущественна (существенен только алфавит таких фрагментов).}
		                \scnfileitem{Поскольку любая описываемая \textit{сущность} может быть связана неограниченным числом \textit{связей} с другими \textit{сущностями} (при этом указанные связи также считаются описываемыми сущностями), \textit{смысловое представление инофрмационной конструкции} является \textit{графоподобной конструкцией}}
		                \scnfileitem{Интеграция (объединение, соединение) \textit{информационной конструкции}, представленных в смысловой форме сводится к \textit{склеиванию} (отождествлению) \textit{синонимичных знаков}.}
		                \scnfileitem{Смысл представленной информации содержится не в самих \textit{знаках}, а в конфигурации \textit{связей} между ними, которая отражает (является информационной моделью) описываемой конфигурации связей между описываемыми \textit{сущностями}. Суть смыслового представления информационной конструкции заключается в том, что конфигурация \textit{связей} между \textit{знаками}, входящими в эту \textit{информационную конструкцию}, становится \uline{\textit{изоморфной}} конфигурации \textit{связей} между описываемыми \textit{сущностями}, которые обозначаются этими \textit{знаками}.}
		                \scnfileitem{Способ (язык) \textit{смыслового представления информации}, должен быть универсальным, т.е. должно быть обеспечено описание (и, прежде всего, обозначение) \uline{любых} \textit{связей} между описываемыми сущностями. При этом, если описываемые \textit{связи} считать одним из видов описываемых \textit{сущностей}, то можно описывать \textit{связи} между \textit{связями}, \textit{связи}, связывающие \textit{связи} с описываемыми \textit{сущностями} иных видов.}
		            \end{scnrelfromvector}
		     \end{scnindent}
        \end{scnindent}
        \scnfileitem{Абстрактная память \textit{ostis-системы} является графодинамической (т.е. нелинейной (графовой) и структурно перестраиваемой). Переработка информации в памяти \textit{ostis-системы} сводится не столько к изменению состояния элементов памяти (это происходит только при изменении синтаксического типа элементов и при изменении содержимого тех элементов, которые обозначают файлы), сколько к изменению \uline{конфигурации связей} между ними.}
        \scnfileitem{Ориентация на компьютеры нового поколения.}
        \scnfileitem{В основе организации решения задач в памяти \textit{ostis-системы} лежит \textit{агентно-ориентированная модель обработки информации}, управляемая ситуациями и событиями, возникающими в обрабатываемой информации (точнее, в обрабатываемой \textit{базе знаний}). С точки зрения архитектуры \textit{ostis-система} представляет собой \uline{иерархическую} многоагентную систему с общедоступной памятью (т.е. с памятью, общедоступной \uline{всем} агентам \textit{ostis-системы}).
            \\Заметим при этом, что общая память большинства исследуемых в настоящее время \textit{многоагентных систем} является не общедоступной, а распределенной, т.е. представляет собой абстрактное (виртуальное) объединение, в состав которого входит память каждого агента многоагентной системы. Координация деятельности агентов \textit{ostis-системы} при выполнении сложных \textit{действий в памяти} \textit{ostis-системы} реализуется также через \textit{память ostis-системы} с помощью хранимых в памяти \textit{методов} решения различных \textit{классов задач}, а также с помощью хранимых в памяти \textit{планов} решения конкретных задач.
            \\На основании этого можно строить неограниченную иерархическую систему \textit{агентов ostis-системы} --- от элементарных агентов, обеспечивающих выполнение базовых действий в памяти \textit{ostis-системы}, до неэлементарных агентов, представляющих собой коллективы (группы) элементарных и/или неэлементарных агентов, обеспечивающих решение различных типовых задач с помощью соответствующих методов и планов.}
        \scnfileitem{Реализация децентрализованного ситуационного управления деятельностью \textit{ostis-систем} не только на уровне внутренних информационных процессов, но также на уровне организации индивидуальной деятельности во внешней среде и даже на уровне участия в коллективной деятельности в рамках различных коллективов \textit{ostis-систем}. Организация выполнения \textit{ostis-системой действий во внешней среде} осуществляется следующим образом:\\
            \begin{scnitemize}
                \item Выделяются классы \textit{элементарных действий во внешней среде}, для реализации каждого из которых вводятся \textit{эффекторные агенты} \textit{ostis-системы}.
                \item Координация деятельности \textit{эффекторных агентов} \textit{ostis-системы} при выполнении \textit{сложных действий во внешней среде} осуществляется через \textit{память ostis-системы} с помощью хранимых в памяти \textit{методов} и \textit{планов} решения различных задач во \textit{внешней среде}, а также с помощью \textit{рецепторных агентов} \textit{ostis-системы}, обеспечивающих обратную связь и, соответственно, сенсомоторную координацию.
            \end{scnitemize}}
        \scnfileitem{Унификация базового набора (базовой системы) используемых \textit{понятий}, что является основой обеспечения \textit{семантической совместимости} всех \textit{ostis-систем}.}
        \scnfileitem{В основе структуризации информации (\textit{базы знаний}), хранимой в памяти \textit{ostis-системы}, лежит иерархическая система \textit{предметных областей} и соответствующих им \textit{формальных онтологий}.}
        \scnfileitem{Переход от исследования обработки данных (data science) к исследованию обработки знаний (knowledge science), что предполагает при разработке различных классов задач \uline{учет семантики обрабатываемой информации}. В этом смысле традиционное программирование хромает на одну ногу}
        \scnfileitem{Способность к пониманию (к семантическому погружению, к семантической интеграции) новых приобретаемых знаний (и, в том числе, новых навыков) в состав текущего состояния \textit{базы знаний}.}
        \scnfileitem{Способность к \textit{семантической конвергенции} (к обнаружению сходств) новых приобретаемых знаний (и, в частности, навыков) со знаниями, входящими в состав текущего состояния \textit{базы знаний} \textit{ostis-системы}.}
        \scnfileitem{Способность к интеграции различных видов \textit{знаний}.}
        \scnfileitem{Способность к интеграции различных \textit{моделей решения задач}.}
        \scnfileitem{Способность \textit{ostis-систем} понимать друг друга, а также любого своего пользователя путем согласования системы используемых понятий (по терминам и по денотационной семантике). Способность \textit{ostis-системы} обеспечивать и поддерживать высокий уровень своей \textit{семантической совместимости} (высокий уровень взаимопонимания) с другими \textit{ostis-системами} в процессе собственной эволюции, а также эволюции других ostis-систем, которая приводит к расширению и/или корректировке системы используемых \textit{понятий}.}
        \scnfileitem{Способность \textit{ostis-системы} согласовывать, координировать свою деятельность с другими системами при решении задач, которые усилиями одной (индивидуальной) интеллектуальной компьютерной системы не могут быть решены либо принципиально, либо за разумное время.}
        \scnfileitem{Высокая степень индивидуальной обучаемости \textit{ostis-систем}, обеспечиваемая высокой степенью их гибкости, стратифицированности, рефлексивности, а также универсальностью средств представления и образования \textit{знаний}.}
        \scnfileitem{Высокая степень коллективной обучаемости \textit{ostis-систем}, обеспечиваемая высокой степенью их \textit{семантической совместимости}.}
    \end{scnrelfromset}
	\begin{scnindent}
	   	\scntext{следовательно}{Перечисленные свойства \textit{ostis-систем} свидетельствуют о том, что они имеют существенно более высокий \textit{уровень интеллекта} и, в частности, более высокий \textit{уровень социализации} по сравнению с современными \textit{интеллектуальными компьютерными системами}.}
	\end{scnindent}
    \bigskip
    %trouble
    \scnheaderlocal{\scnnonamednode}
	\begin{scneqtoset}
		\scnitem{память*}
		\scnitem{ostis-система}
	\end{scneqtoset}
    \scnrelfrom{сужение второго домена заданного отношения для заданного первого домена}{память ostis-системы}
	\begin{scnindent}
    	\scnsubset{смысловая память}
	\end{scnindent}
    \bigskip
    \scnheaderlocal{\scnnonamednode}
	\begin{scneqtoset}
		\scnitem{информация, хранимая в памяти кибернетической системы*}
		\scnitem{ostis-система}
	\end{scneqtoset}
    \scnrelfrom{сужение второго домена заданного отношения для заданного первого домена}{база знаний ostis-системы}
   	\begin{scnindent}
   		\scnsubset{смысловое представление информации}
   	\end{scnindent}
    
    \scnheader{решатель задач ostis-системы }
    \scnsubset{агентно-ориентированная модель обработки информации в памяти}
    \scnheader{смысловое представление информации}
    \begin{scnrelfromset}{принципы, лежащие в основе}
        \scnfileitem{Каждый синтаксически элементарный (атомарный) фрагмент представленной информации является обозначением некоторой сущности, которая может быть реальной или абстрактной, конкретной (фиксированной, константной) или произвольной (переменной), постоянной или временной, четкой (достоверной) или нечеткой (недостоверной с возможным дополнительным уточнением степени правдоподобности).}
	        \begin{scnindent}
	        	\scntext{следовательно}{В состав смыслового представления информации не могут входить буквы (не являются обозначениями сущностей), слова, словосочетания (не являются элементарными фрагментами), разделители, ограничители (не являются обозначениями сущностей)}
	        \end{scnindent}
        \scnfileitem{В рамках смыслового представления информации отсутствует синонимия (пары синонимичных знаков), омонимия (омонимичные знаки), семантическая эквивалентность (пары семантически эквивалентных информационных конструкций), т.е. отсутствует любая форма дублирования информации, а также отсутствует неоднозначность соотношения между знаками и их денотатами.}
    \end{scnrelfromset}
	    \begin{scnindent}
	    	\scntext{следовательно}{Смысловое представление информации не может выглядеть как цепочка (строка, последовательность) синтаксически элементарных фрагментов, поскольку каждая описываемая сущность и взаимно однозначно соответствующий ей ее знак может быть связана не с двумя, а с любым количеством описываемых сущностей. Другими словами, смысловое представление информации является нелинейной (графовой) информационной конструкцией.}
		    	\begin{scnindent}
		    		\scntext{следовательно}{Если внутреннее представление информации в памяти компьютерной системы является смысловым представлением, то обработка информации в такой памяти носит графодинамический характер и сводится не к изменению состояния элементов памяти, а к изменению конфигурации связей между ними.}
		    	\end{scnindent}
	    \end{scnindent}
    \scntext{примечание}{Ключевая проблема современного этапа развития общей теории интеллектуальных компьютерных систем и технологии их разработки это проблема обеспечения \textbf{\textit{семантической совместимости}}
        \begin{scnitemize}
            \item различных видов знаний, входящих в состав баз знаний интеллектуальных компьютерных систем;
            \item различных видов моделей решателей задач;
            \item различных интеллектуальных компьютерных систем в целом;
        \end{scnitemize}
        Для решения этой проблемы очевидно необходима унификация (стандартизация) формы представления знаний в памяти интеллектуальных компьютерных систем. Предлагаемым нами подходом для такой унификации и является ориентация на \textbf{\textit{смысловое представление информации}} (знаний) в памяти интеллектуальных компьютерных систем. Основой предполагаемого нами подхода к обеспечению высокого уровня обучаемости и семантической совместимости интеллектуальных компьютерных систем, а также к разработке стандарта интеллектуальных компьютерных систем является унификация \textbf{\textit{смыслового представления информации}} (знаний) в памяти интеллектуальных компьютерных систем и построение глобального \textbf{\textit{смыслового пространства}} знаний.}
        \begin{scnindent}
    		\scntext{примечание}{Информация в знаковой конструкции в основном содержится не в самих знаках (в их структуре), а в связях между знаками. При этом существенно, чтобы эти связи (синтаксические связи) имели четкую смысловую (семантическую) интерпретацию. Если структура знаков содержит информацию об обозначаемой сущности всегда можно заменить на бесструктурные знаки, которые имеют семантическую окрестность}
    	\end{scnindent}
    
    \scnheader{семантическая сеть}
    \scnsubset{смысловое представление информации}
    \scntext{пояснение}{Семантическая сеть нами рассматривается не как красивая метафора сложноструктурированных знаковых конструкций, а как формальное уточнение понятия смыслового представления информации, как принцип представления информации, лежащей в основе нового поколения компьютерных языков и самих компьютерных систем --- графовых языков и графовых компьютеров.}
    \scnsubset{знаковая конструкция}
    \scntext{пояснение}{Семантическая сеть --- это знаковая конструкция, обладающая следующими свойствами:
        \begin{scnitemize}
            \item внутренюю структуру (строение) знаков, входящих в семантическую сеть не требуется учитывать при ее семантическом анализе (понимании)
            \item Смысл семантической сети определяется денотационной семантикой всех входящих в нее знаков и конфигурацией связей инцидентности этих знаков
            \item Из двух инцидентных знаков, входящих в семантическую сеть, один является знаком связи
            \item Отсутствие синонимии, омонимии
        \end{scnitemize}
    }
    \scnrelfrom{предлагаемый подход}{\scnkeyword{SC-код}}
	    \begin{scnindent}
		    \scnidtf{Предлагаемое в рамках \textit{Технологии OSTIS} уточнение понятия \textit{семантической сети}}
		    \scnsubset{семантическая сеть}
		    \scnidtf{Semantic Computer Code}
		    \scnrelfrom{смотрите}{\nameref{intro_sc_code}}
	    \end{scnindent}
    
    \scnheader{многоагентная система}
    \scnsubset{кибернетическая система}
    \scntext{пояснение}{Кибернетическая система, представляющая собой множество кибернетических систем, способных коммуницировать, т.е. обмениваться информацией друг с другом (причем не обязательно каждый с каждым)}
    
    \scnheader{агент*}
    \scnidtf{агент многоагентной системы*}
    
    \scnheader{внешняя среда*}
    \scnidtf{внешняя среда кибернетической системы}
    
    \scnheader{память*}
    \scnidtf{внутренняя (информационная) среда кибернетической системы}
    \scntext{примечание}{Не каждая кибернетическая система (в том числе многоагентная система) имеет явно выделенную память, являющуюся хранилищем накапливаемой информации, накапливаемого опыта.}
    
    \scnheader{многоагентная система}
    \begin{scnsubdividing}
        \scnitem{многоагентная система без общей памяти}
        \scnitem{многоагентная система с общей памятью}
    \end{scnsubdividing}
    \begin{scnsubdividing}
        \scnitem{многоагентная система, в которой управление агентами осуществляется только путем обмена сообщениями между ними}
        \scnitem{многоагентная система, в которой управление агентами осуществляется через общую для них память}
    \end{scnsubdividing}
    \begin{scnsubdividing}
        \scnitem{многоагентная система с централизованным управлением агентами}
        \scnitem{многоагентная система с децентрализованным управлением агентами}
    \end{scnsubdividing}
    \begin{scnsubdividing}
        \scnitem{многоагентная система, в которой областью деятельности всех ее агентов является только внешняя среда этой системы}
        \scnitem{многоагентная система, в которой областью деятельности ее агентов является как внешняя среда, так и память этой системы}
        	\begin{scnindent}
        		\scntext{примечание}{некоторые агенты такой системы могут работать только в памяти}
        	\end{scnindent}
    \end{scnsubdividing}
    
    \scnheader{агентно-ориентированная модель обработки информации в памяти}
    \scnidtf{агентно-ориентированная модель решения задач}
    \scnidtf{агентно-ориентированная архитектура решателя задач, представляющая собой многоагентную систему, в которой управление ее агентами осуществляется общей для них памятью и областью деятельности агентов является та же самая общая для них память}
	    \begin{scnindent}
	    	\scntext{следовательно}{условием инициирования каждого указанного агента является возникновение в указанной памяти соответствующего вида ситуации или события}
	    \end{scnindent}
    \begin{scnreltoset}{пересечение}
        \scnitem{многоагентная система, в которой управление агентами осуществляется через общую для них память}
        \scnitem{многоагентная система с децентрализованным управлением агентами}
        \scnitem{многоагентная система, в которой областью деятельности ее агентов является как внешняя среда, так и память этой системы}
    \end{scnreltoset}
    
    \scnheader{агентно-ориентированная модель обработки информации в памяти}
    \begin{scnrelfromset}{принципы, лежащие в основе}
        \scnfileitem{Распределение целенаправленной деятельности между агентами, выполняющими различные действия в памяти, осуществляется на основе генерируемой в \textit{базе знаний} иерархической системы, описывающей связь (сведение) инициированных целей (задач) с подцелями (подзадачами).}
        \scnfileitem{Условием инициирования агента является появление в базе знаний формулировки той цели (задачи), которая, во-первых, инициирована, а, во-вторых, либо может быть полностью достигнута (решена) этим агентом, либо может быть этим агентом достигнута (решена) частично.}
        \scnfileitem{В результате частичного достижения (решения) некоторой цели (задачи) агент может сгенерировать новые подцели (подзадачи).}
        \scnfileitem{Таким образом, условием инициирования агента обработки информации (базы знаний) является появление соответствующей этому агенту ситуации или соответствия.}
    \end{scnrelfromset}
    \scnrelfrom{предлагаемый подход}{\scnkeyword{абстрактная sc-машина}}
	    \begin{scnindent}
	    	\scnidtf{Предлагаемое в рамках \textit{Технологии OSTIS} уточнение понятия агентно-ориентированной модели обработки информации в памяти}
	    \end{scnindent}
    \scnsuperset{абстрактная sc-машина}
    \scntext{примечание}{Децентрализованное (агентно-ориентированное) управление процессом решения задач в ostis-системах реализуется как на внутреннем уровне (на уровне решателя задач ostis-системы), так и на внешнем уровне (на уровне взаимодействия между ostis-системами)}
    
    \scnheader{стандартизация ostis-систем}
    \scnidtf{унификация \textit{ostis-систем}}
    \scntext{пояснение}{Стандартизация \textit{ostis-систем} включает в себя:
        \begin{scnitemize}
            \item cтандартизацию языка внутреннего представления информации в памяти \textit{ostis-систем}
            \item cтандартизацию принципов децентрализованного управления обработкой информации в памяти \mbox{\textit{ostis-систем}}
            \item cтандартизацию языка описания ситуаций и событий (в памяти \textit{ostis-систем}), которые являются условиями инициирования различных информационных процессов в памяти \textit{ostis-систем}
            \item стандартизацию базового языка спецификации (описания, программирования) агентов, выполняющих соответствующие информационные процессы в памяти \textit{ostis-систем}
            \item стандартизацию базовых языков ввода/вывода информации в/из памяти \textit{ostis-систем}.
        \end{scnitemize}
    }
    
    \scnheader{SC-код}
    \scnidtf{Стандарт \textit{смыслового представления информации} в памяти \textit{ostis-системы}, а, точнее, \textit{стандарт семантических сетей}}
    
    \scnheader{абстрактная sc-машина}
    \scnidtf{Стандарт \textit{агентно-ориентированной модели обработки информации в памяти ostis-системы}}
    
    \scnheader{стандартизация}
    \scnidtf{унификация}
    \begin{scnrelfromset}{проблемы текущего состояния}
        \scnfileitem{Разработка и совершенствование стандартов происходит очень медленно}
        \scnfileitem{В разработке и совершенствовании стандартов принимает участие явно недостаточное число профессионалов --- не учитываются все мнения}
        \scnfileitem{В разработке и совершенствовании стандарта отсутствует четкая методика формирования консенсуса}
        \scnfileitem{При введении новой версии стандарта отсутствует четкая методика перевода на новую версию стандарта всех систем, разработанных по предыдущей версии}
    \end{scnrelfromset}
    \scntext{предлагаемый подход}{Стандарт --- это перманентно совершенствуемая \textit{база знаний}, поддержку эволюции которой осуществляет соответствующий портал}
    
    \scnheader{конвергенция знаний в памяти ostis-системы}
    \begin{scnrelfromset}{принципы, лежащие в основе}
        \scnfileitem{Вводится \uline{универсальный} базовый язык внутреннего \uline{смыслового} представления знаний в памяти \mbox{ostis-систем} (\mbox{\textit{SC-код}}), по строению к которому все внутренние языки, ориентированные на представление знаний различного вида (логические языки, языки представления методов решения задач (в частности, программ), язык формулировки задач, онтологические языки и многие другие) являются подъязыками \mbox{\textit{SC-кода}}, синтаксис которых полностью совпадает с синтаксисом \mbox{\textit{SC-кода}}.}
        \scnfileitem{Конвергенция различных знаний сводится к согласованию систем понятий, используемых для представления знаний различного вида. Такое согласование направлено на увеличение числа общих понятий, используемых при представлении различных знаний.}
    \end{scnrelfromset}
    
    \scnheader{конвергенция моделей решения задач в \mbox{ostis-системе}}
    \begin{scnrelfromset}{принципы, лежащие в основе}
        \scnfileitem{Синтаксис языка представления соответствующего класса методов решения задач в памяти --- синтаксис \mbox{SC-кода}}
        \scnfileitem{Денотационная семантика описывается в виде соответствующей онтологии и представляется в виде текста \mbox{SC-кода}}
        \scnfileitem{Операционная семантика каждой модели решения задач --- коллектив \uline{агентов}. Он может быть иерархическим на основе различных моделей решателей, но есть базовая модель интерпретации \uline{любых} методов:\\
            \begin{scnitemize}
                \item Язык SCP
                \begin{scnitemizeii}
                    \item cинтаксис совпадает с синтаксисом SC-кода
                    \item денотационная семантика --- процедурный язык программирования в графодинамической памяти
                    \item операционная семантика реализуется на уровне программной или аппаратной платформы
                \end{scnitemizeii}
                \item sc-агенты работают в общей среде --- (sc-памяти) параллельно, асинхронно на основе ряда правил, позволяющих им не мешать друг другу
            \end{scnitemize}}
    \end{scnrelfromset}
    
    \scnheader{интеграция знаний в памяти ostis-системы*}
    \scntext{пояснение}{Интеграция знаний в памяти \textit{ostis-систем} сводится к склеиванию (отождествлению) синонимичных знаков}
    \scnheader{интеграция моделей решения задач в ostis-системе*}
    \scntext{пояснение}{Поскольку модель решения задач, используемая ostis-системой, представлена в памяти ostis-системы как соответствующий вид знаний, интеграция различных моделей решения задач может происходить в ostis-системе точно так же, как и интеграция любых других видов знаний. Кроме того, когда речь идет об интеграции различных моделей решения задач, имеется в виду возможность одновременного использования различных моделей решения задач при обработке одних и тех же знаний и, в частности, при решении одной и той же задачи. Такая возможность в ostis-системе обеспечивается \textit{агентно-ориентированной моделью обработки информации} в памяти ostis-системы. Таким образом, такого рода интеграция различных моделей решения задач для ostis-систем является тривиальной.}
    
    \scnheader{ostis-система}
    \begin{scnrelfromset}{достоинства}
        \scnfileitem{Высокий уровень способности \textit{ostis-системы} осуществлять семантическую интеграцию знаний в своей памяти (в частности, при погружении новых знаний в текущее состояние базы знаний) \uline{обеспечивается} смысловым характером внутреннего кодирования информации,  хранимой в памяти ostis-системы и, в частности, тем, что во внутреннем коде базы знаний \textit{ostis-системы} запрещены омонимичные знаки и пары синонимичных знаков.}
        \scnfileitem{Высокий уровень способности интегрировать различные виды знаний в \textit{ostis-системах} обеспечивается тем, что каждый язык, ориентированный на представление знаний соответствующего вида является \uline{подъязыком} одного и того же базового языка \textit{SC-кода}.}
	        \begin{scnindent}
	        	\scntext{примечание}{Кроме того можно говорить об иерархии sc-языков}
	        \end{scnindent}
        \scnfileitem{Высокий уровень способности интегрировать различные модели решения задач в \textit{ostis-системах} \uline{обеспечивается}:\\
            \begin{scnitemize}
                \item тем, что все эти модели ориентированы на обработку информации, представленной в \textit{SC-коде};
                \item один и тот же фрагмент базы знаний ostis-системы (т.е. одна и та же конструкция SC-кода) может одновременно обрабатываться несколькими \uline{разными} моделями решения задач;
                \item все модели решения задач в ostis-системах интегрируются с помощью одной и той же базовой модели решения задач --- \textit{scp-модели решения задач}.
            \end{scnitemize}}
        \scnfileitem{Высокий уровень обучаемости \textit{ostis-систем} \uline{обеспечивается}:\\
            \begin{scnitemize}
                \item высоким уровнем семантической гибкости информации, хранимой в памяти ostis-системы, поскольку каждое удаление или добавление синтаксически элементарного фрагмента хранимой информации, а также удаление или добавление каждой связи инцидентности между такими элементами имеет четкую семантическую интерпретацию;
                \item высоким уровнем стратифицированности хранимой информации, что обеспечивается онтологически ориентированной структуризацией базы знаний ostis-системы;
                \item высоким уровнем рефлексии ostis-системы, что обеспечивается мощными метаязыковыми возможностями языка внутреннего представления информации (знаний) в памяти \textit{ostis-систем}.
            \end{scnitemize}}
        \scnfileitem{Каждая \textit{ostis-система} имеет высокий \textit{уровень обучаемости} (способности к быстрому расширению своих \textit{знаний} и \textit{навыков}) и высокий \textit{уровень социализации} (способности к эффективному участию в деятельности различных коллективов  коллективов, состоящих из \textit{ostis-систем}, и сообществ, состоящих из \textit{ostis-систем} и людей.}
        \begin{scnindent}
	        \begin{scnrelfromset}{детализация достоинства}
	            \scnfileitem{Существуют четкие формальные критерии, определяющие \textit{уровень семантической совместимости} (уровень семантической конвергенции) различных знаний, навыков, целых \textit{ostis-систем} (точнее, баз знаний этих систем). Очевидно, что \textit{уровень семантической совместимости} прежде всего определяется количеством точек соприкосновения в сравниваемых \textit{знаниях}, \textit{навыках} и \textit{базах знаний}  это \textit{знаки}, присутствующие \uline{в разных} сравниваемых объектах, но имеющие одинаковые денотаты (т.е. обозначающие одинаковые сущности). При этом среди таких знаков, обозначающих одинаковые сущности и присутствующих в разных сравниваемых объектах особенно важны знаки, обозначающие \textit{понятия}.Количество таких общих понятий в сравниваемых знаниях, навыках, базах знаний определяет уровень семантической совместимости (уровень согласованности) систем используемых понятий в сравниваемых указанных объектах. Увеличение количества знаков, обозначающих одинаковые сущности и присутствующих в разных сравниваемых объектах, может привести к тому, что в разных указанных сравниваемых объектах будут присутствовать не только семантически эквивалентные знаки, но и семантически эквивалентные целые фрагменты (целые информационные конструкции).Существенно при этом подчеркнуть, что семантически эквивалентные знаковые конструкции, представленные на внутреннем языке ostis-систем (в SC-коде), в памяти разных ostis-систем всегда являются синтаксически изоморфными графовыми конструкциями, в которых соответствие изоморфизма связывает знаки, хранимые в памяти разных ostis-систем, но обозначающие одинаковые сущности (точнее, одну и ту же сущность). Заметим также, что в рамках памяти каждой индивидуальной \textit{ostis-системы} синонимия знаков и, соответственно, семантическая эквивалентность знаковых конструкций запрещены.}
	            \scnfileitem{Благодаря постоянно развиваемым семантическим стандартам \textit{Технологии OSTIS} , которые представлены системой формальных онтологий для самых различных предметных областей, разрабатываемые \textit{ostis-системы} \uline{изначально} имеют достаточно высокий \textit{уровень семантической совместимости} со всеми остальными \textit{ostis-системами}. Более того, в \textit{Технологии OSTIS} выделяется целое ядро всех ostis-систем, содержащее фундаментальные базовые знания и базовые навыки, одинаковые для всех ostis-систем и позволяющее каждой копии этого ядра развиваться (общаться, специализироваться) в любом направлении.}
	            \scnfileitem{Каждая ostis-система, взаимодействуя с людьми (пользователями) или с другими \mbox{ostis-системами}, обладает способностью повышать уровень семантической совместимости (взаимопонимания) с ними, а также поддерживать (сохранять) высокий уровень такой совместимости в условиях (1) собственной эволюции, (2) эволюции других ostis-систем и пользователей, (3) эволюции семантических стандартов Технологии OSTIS. Указанное взаимодействие, в основном, направлено на согласование изменений в системе используемых понятий, т.е. корректировки соответствующих фрагментов онтологий.}
	            \scnfileitem{Благодаря высокому уровню семантической совместимости ostis-систем и смысловому представлению знаний в памяти ostis-систем существенно снижается сложность и повышается качество семантического анализа и понимания информации, поступающей (сообщаемой, передаваемой) ostis-системе от других ostis-систем или пользователей.}
	            \scnfileitem{Каждая ostis-система способна:\\
	                \begin{scnitemize}
	                    \item самостоятельно или по приглашению войти в состав ostis-коллектива (коллектива ostis-систем) или в состав ostis-сообщества, состоящего из ostis-систем и людей. Такие коллективы и сообщества создаются на временной (разовой) или постоянной основе для коллективного решения сложных задач
	                    \item участвовать в распределении (в т.ч. в согласовании распределения) задач --- как разовых задач, так и долгосрочных задач (обязанностей)
	                    \item мониторить состояние всего процесса коллективной деятельности и координировать свою деятельность с деятельностью других членов коллектива при возможных непредсказуемых изменениях условий (состояния) соответствующей среды.
	                \end{scnitemize}}
	        \end{scnrelfromset}
        \end{scnindent}
        \scnfileitem{Высокий уровень интеллекта ostis-систем и, соответственно, высокий уровень их самостоятельности и целенаправленности позволяет ostis-системам быть полноправными членами самых различных сообществ, в рамках которых ostis-системы получают права самостоятельно инициировать (на основе детального анализа текущего положения дел и, в том числе, текущего состояния плана действий сообщества) широкий спектр действий (задач), выполняемых другими членами сообщества, и тем самым участвовать в согласовании и координации деятельности членов сообщества.}
        \scnfileitem{Способность ostis-системы согласовывать свою деятельность с другими ostis-системами, а также корректировать деятельность всего коллектива ostis-систем, адаптируясь к различного вида изменениям среды (условий), в которой эта деятельность осуществляется, позволяет существенно автоматизировать деятельность системного интегратора как на этапе сборки коллектива ostis-систем, так и на этапе его обновления (реинжиниринга).}
    \end{scnrelfromset}
    \scntext{примечание}{Достоинства \textit{ostis-систем} обеспечиваются:
        \begin{scnitemize}
            \item достоинствами \textit{SC-кода} --- языка внутреннего кодирования информации, хранимой в памяти \textit{ostis-систем}
            \item достоинствами организации \textit{sc-памяти} --- памяти \textit{ostis-систем}
            \item достоинствами \textit{sc-моделей баз знаний} ostis\textit{}систем средствами структуризации таких \textit{баз знаний}
            \item достоинствами \textit{sc-моделей решения задач} --- агентно-ориентированных моделей решения задач, используемых в \textit{ostis-системах}.
        \end{scnitemize}}
    
\end{scnsubstruct}
\scnsourcecommentinline{Завершили рассмотрение понятия ostis-системы}
\bigskip

\scnheader{Понятие ostis-сообщества}
\begin{scnsubstruct}
	
    \scnheader{ostis-сообщество}
    \scnidtf{Человеко-машинный симбиоз, представляющий собой коллектив, состоящий из людей и ostis-систем и обеспечивающий высокий уровень автоматизации определённого (соответствующего) вида человеческой деятельности.}
    \scntext{примечание}{В состав каждого ostis-сообщества входит корпоративная ostis-система, которая в рамках этого \mbox{ostis-сообщества} выполняет:
        \begin{scnitemize}
            \item роль координатора деятельности членов данного ostis-сообщества;
            \item роль памяти ostis-сообщества, т.е. хранителя общих (обобществляемых, общедоступных) знаний для всех членов данного ostis-сообщества, которое несет ответственность за совершенствование этих знаний, а также для всех членов всех тех ostis-сообществ, в состав которых данное ostis-сообщество входит (указанные субъекты являются пользователями рассматриваемых общих знаний). Таким образом, корпоративная ostis-система некоторого ostis-сообщества является официальным представителем этого ostis-сообщества во всех ostis-сообществах, в состав которых входит, и, следовательно, является координатором деятельности даного ostis-сообщества (как единого целого) в рамках всех ostis-сообществ, в состав которых оно входит;
        \end{scnitemize}}
    
    \scnheader{есть сходства*}
    \begin{scnhaselementset}
        \scnitem{ostis-сообщество}
        \scnitem{решатель задач ostis-системы}
    \end{scnhaselementset}
    \begin{scnindent}
    	\begin{scnrelfromset}{пояснение}
   			\scnitem{ostis-сообщество}
		    \begin{scnindent}
	   			\scnsuperset{многоагентная система, в которой управление агентами осуществляется через общую для них память}
	   			\scnsuperset{многоагентная система, с децентрализованным управлением агентами}
	   			\scnsuperset{многоагентная система, в которой областью деятельности её агентов является как внешняя среда, так и память этой системы}
  			 \end{scnindent}
   			\scnitem{решатель задач ostis-системы}
   			\begin{scnindent}
	   			\scnsuperset{многоагентная система, в которой управление агентами осуществляется через общую для них память}
	   			\scnsuperset{многоагентная система, с децентрализованным управлением агентами}
	   			\scnsuperset{многоагентная система, в которой областью деятельности её агентов является как внешняя среда, так и память этой системы}
  			 \end{scnindent}
   			\scnitem{агентно-ориентированная модель обработки информации в памяти}
    	\end{scnrelfromset}
    \end{scnindent}
    
    \scnheader{многоагентная система с децентрализованным управлением агентами}
    \begin{scnrelfromlist}{включение;пример}
        \scnitem{оркестр, играющий без дирижера или даже без композитора}
	        \begin{scnindent}
	        	\scntext{необходимое требование}{каждый участник оркестра должен иметь квалификацию дирижера или композитора}
	        \end{scnindent}
       \scnitem{комплексная строительная бригада, работающая без прораба}
       		\begin{scnindent}
            	\scntext{необходимое требование}{каждый участник строительной бригады должен иметь квалификацию прораба}
            \end{scnindent}
        \scnitem{научно-исследовательская лаборатория, работающая без заведующего и научного руководителя}
        	\begin{scnindent}
            	\scntext{необходимое требование}{каждый участник научно-исследовательской лаборатории должен иметь квалификацию заведующего или научного руководителя}
            \end{scnindent}
        \scnitem{кафедра, работающая без заведующего и ученого секретаря}
            \begin{scnindent}
            	\scntext{необходимое требование}{каждый участник кафедры должен иметь квалификацию заведующего и ученого секретаря}
            \end{scnindent}
    \end{scnrelfromlist}
    \bigskip
    \end{scnsubstruct}
\scnsourcecommentinline{Завершили рассмотрение понятия ostis-сообщества}
\bigskip

\scnstructheader{Понятие ostis-технологии}
\begin{scnsubstruct}
	
    \scnheader{ostis-технология}
    \begin{scnreltoset}{объединение}
        \scnitem{ostis-технология проектирования}
        \begin{scnindent}
            \begin{scnsubdividing}
                \scnitem{ostis-технология проектирования ostis-систем соответствующего класса}
                \begin{scnindent}
                    \scnhaselement{Базовая ostis-технология проектирования ostis-систем}
                \end{scnindent}
                \scnitem{ostis-технология проектирования соответствующего класса компонентов ostis-систем}
                \begin{scnindent}
                    \scnhaselement{Базовая ostis-технология проектирования баз знаний ostis-систем}
                    \scnhaselement{Базовая ostis-технология проектирования решателей задач ostis-систем}
                    \scnhaselement{Базовая ostis-технология проектирования интерфейсов ostis-систем}
                \end{scnindent}
                \scnitem{ostis-технология проектирования объектов заданного класса, не являющихся ostis-системами}
            \end{scnsubdividing}
        \end{scnindent}
        \scnitem{ostis-технология производства}
        \begin{scnindent}
            \scnsuperset{технология производства спроектированных ostis-систем}
            \scnsuperset{ostis-технология управления производством спроектированных продуктов заданного класса, не являющихся ostis-системами}
        \end{scnindent}
        \scnitem{технология эксплуатации ostis-систем}
        \begin{scnindent}
            \scnhaselement{Базовая технология эксплуатации ostis-систем}
	        \scnsuperset{технология эксплуатации ostis-систем соответствующего класса}
	            \begin{scnindent}
		            \scnsuperset{ostis-технология управления производством спроектированных продуктов заданного класса, не являющихся ostis-системами}
		            \begin{scnindent}
		            	\scnidtf{технология эксплуатации ostis-систем управления производством спроектированных продуктов заданного класса, не являющихся ostis-системами}
		            \end{scnindent}
	            \end{scnindent}
        \end{scnindent}
        \scnitem{технология реинжиниринга ostis-систем}
        \begin{scnindent}
            \scnhaselement{Базовая технология реинжиниринга ostis-систем}
            \scnsuperset{технология реинжиниринга ostis-систем соответствующего класса}
        \end{scnindent}
    \end{scnreltoset}
    
    \scnheader{ostis-технология}
    \scnidtf{компонент Технологии OSTIS}
    \scnhaselement{Ядро Технологии OSTIS}
	    \begin{scnindent}
	    	\scnidtf{Базовая ostis-технология}
	    \end{scnindent}
    \scnsuperset{частная ostis-технология}
    \begin{scnindent}
	    \scnsuperset{ostis-технология проектирования соответствующего класса компонентов ostis-систем}
	    \begin{scnindent}
		    \scnhaselement{Технология проектирования баз знаний ostis-систем}
		    \scnhaselement{Технология проектирования решателей задач ostis-систем}
		    \scnhaselement{Технология проектирования невербальных интерфейсов ostis-систем с внешней средой}
		    \scnhaselement{Технология проектирования интерфейсов ostis-систем с другими техническими системами}
		    \scnhaselement{Технология проектирования пользовательских интерфейсов ostis-систем}
		 \end{scnindent}
	\end{scnindent}
    \scnsuperset{специализированная ostis-технология проектирования ostis-систем соответствующего класса}
    \begin{scnindent}
	    \scnhaselement{Технология проектирования ostis-систем управления предприятиями рецептурного производства}
	    \scnhaselement{Технология проектирования ostis-систем управления предприятиями производства молочной продукции}
	    \scnhaselement{Технология проектирования интеллектуальных обучающих ostis-систем}
	    \scnhaselement{Технология проектирования интеллектуальных обучающих ostis-систем для школьников}
	    \scnhaselement{Технология проектирования интеллектуальных обучающих ostis-систем для подготовки специалистов в области Математики}
	    \scnhaselement{Технология проектирования интеллектуальных обучающих ostis-систем для подготовки специалистов в области Искуственного интеллекта}
	\end{scnindent}
    
    \scnheader{ostis-технология проектирования}
    \scntext{примечание}{Каждой ostis-технологии проектирования соответсвует своя ostis-система автоматизации проектирования соответствующего класса объектов}
    \scnrelfrom{соответствующее семейство средств автоматизации}{ostis-система автоматизации проектирования}
    \scnrelfrom{соответствующее семейство классов проектируемых объектов}{(ostis-система автоматизации проектирования ostis-систем $\cup$ ostis-система автоматизации проектирования объектов, не являющихся ostis-системами)}
    \scnsuperset{ostis-технология проектирования ostis-систем соответствующего класса}
    
    \scnheader{ostis-технология проектирования ostis-систем соответствующего класса}
    \scnidtf{технология проектирования \textit{ostis-систем} соответствующего (заданного) класса, который, в свою очередь, соответствует определенному \textit{виду человеческой деятельности}, подвиды которого автоматизируются с помощью указанных выше проектируемых \textit{ostis-систем}}
    
    \scnheader{ostis-технология}
    \begin{scnrelfromlist}{отношение, заданное на данном множестве}
        \scnitem{частная технология*}
        \scnitem{специализированная технология*}
        \scnitem{комплекс специализированных технологий*}
    \end{scnrelfromlist}
    \scntext{пояснение}{Базовая частная или специализированная технология, входящая в состав комплексной \textit{Технологии OSTIS}, которая:
        \begin{scnitemize}
            \item направлена на автоматизацию конкретного вида человеческой деятельности;
            \item ориентирована на использование ostis-систем (как индивидуальных, так и коллективных) в качестве самостоятельных субъектов или активных интеллектуальных инструментов, либо на использование человеко-машинных ostis-сообществ при решении:
            \begin{scnitemizeii}
                \item как задач, выполняемых в памяти ostis-систем (в т.ч. в памяти коллективов ostis-систем);
                \item так и задач, выполняемых во внешней среде ostis-систем, в процессе решения которых субъектами соответствующих действий либо ostis-системы (индивидуальные или коллективные), либо конкретные персоны, либо ostis-сообщества.
            \end{scnitemizeii}
        \end{scnitemize}}
    \scnidtf{Множество всевозможных технологий, соответствующих стандартам технологии OSTIS и направленных на автоматизацию различных конкретных видов человеческой деятельности}
    \scnrelboth{следует отличать}{Технология OSTIS}
    \begin{scnindent}
    	\scntext{примечание}{\textit{Технология OSTIS} в отличие от понятия \textit{ostis-технологии} представляет собой не множество технологий, а комплекс взаимосвязанных между собой самых различных технологий, превращающий указанное множество технологий в единую объединенную технологию, в сумму взаимосвязанных глубоко интегрированных технологий. В этом смысле Технология OSTIS является максимальной ostis-технологией, в состав которой входят все ostis-технологии.}
    \end{scnindent}
    \begin{scnrelfromlist}{включение;пример}
        \scnitem{ostis-технология проектирования и перепроектирования}
        \scnitem{ostis-технология производства}
        \scnitem{ostis-технология образования}
    \end{scnrelfromlist}
    \begin{scnhaselementrolelist}{пример}
        \scnitem{Технология OSTIS}
        \scnitem{OSTIS-технология публикации и согласования результатов научно-технической деятельности (в широком смысле)}
        \scnitem{OSTIS-технология проектирования, реализации и реинжиниринга ostis-систем}
        \scnitem{OSTIS-технология разработки стандартов Технологии OSTIS}
    \end{scnhaselementrolelist}
    
    \scnheader{ostis-технология коллективной разработки информационных ресурсов}
    \scnsuperset{ostis-технология коллективного проектирования}
    \scnsuperset{ostis-технология коллективной разработки планов}
    \scnsuperset{ostis-технология публикации и согласования результатов научно-технической деятельности}
    \scnsubset{ostis-технология}
    
    \scnheader{ostis-технология эксплуатации ostis-систем}
    \scnidtf{Общие методы и средства (языковые и интерфейсные) организации взаимодействия ostis-систем со своими конечными пользователями}
    \scnsubset{ostis-технология}
    \scntext{примечание}{Поскольку в рамках Экосистемы OSTIS каждому человеку придется взаимодействовать с больщим числом ostis-систем разного назначения, принципы организации взаимодействия всех ostis-систем со своими пользователями должны быть абсолютно одинаковыми. Удобство (usability) пользовательских интерфейсов должно быть направлено не только на синтаксическую красоту, но и на простую семантическую интерпретацию (понятность).}
    
    \scnheader{ostis-технология проектирования ostis-систем}
    \scnidtf{Технология построения (разработки) логико-семантических моделей (sc-моделей) ostis-систем}
    \scniselement{ostis-технология}
    \scntext{примечание}{Продуктом каждого завершенного (целостного) коллективного проекта, реализованного в рамках этой технологии, является полная \textit{логико-семантическая модель ostis-системы}.}
    \scnrelfrom{класс продуктов}{логико-семантическая модель ostis-системы}
    \scnrelfrom{средство}{Метасистема OSTIS}
    \scnrelfrom{класс субъектов}{коллектив разработчиков ostis-системы}
    \scnrelfrom{класс исходных данных}{исходная спецификация ostis-системы}
    
    \scnheader{ostis-технология производства ostis-систем}
    \scnidtf{Технология сборки и установки ostis-систем}
    \scniselement{ostis-технология}
    \scnrelfrom{исходная информация}{логико-семантическая модель ostis-системы}
    \scnrelfrom{комплектация}{универсальный интерпретатор логико-семантических моделей ostis-систем}
    \begin{scnindent}
    	\scntext{примечание}{Это, своего рода, мотор, движок ostis-систем}
    \end{scnindent}
    \scnrelfrom{методы}{Методика производства ostis-систем}
    \scnrelfrom{активный инструмент}{Метасистема OSTIS}
    \scnrelfrom{продукты}{ostis-система}
    
    \scnheader{ostis-технология реинжиниринга ostis-систем}
    \scnidtf{Технология обновления (перепроектирования) ostis-систем в ходе их эксплуатации}
    \scniselement{ostis-технология}
    
    \scnheader{следует отличать*}
    \begin{scnhaselementset}
        \scnitem{Технология реинжиниринга ostis-систем}
        \scnitem{Технология проектирования ostis-систем}
    \end{scnhaselementset}
    \begin{scnindent}
    	\scntext{примечание}{Эти технологии сходны. Их методы и средства совпадают. Не совпадают только исходные данные и результаты, которыми в \textit{Технологии обновления ostis-систем} являются предшествующие и последующие состояния ostis-систем. В \textit{Технологии проектирования ostis-систем} исходными данными являются исходные спецификации (замыслы) проектируемых ostis-систем, и результатами --- полные логико-семантические модели этих систем}
    \end{scnindent}
    	
    \scnheader{Технология OSTIS}
    \scnidtf{Совокупность (интеграция, объединение) всех \textit{ostis-технологий}}
    \scnrelto{интеграция}{ostis-технология}
    \scnidtf{Комплекс (множество) семантически совместимых \textit{технологий}, в состав которого входит \textit{Ядро Технологии OSTIS} и иерархическая система \textit{ostis-технологий}, каждая из которых ориентирована на \textit{проектирование}, \textit{производство}, \textit{эксплуатацию} или \textit{реинжиниринг} соответствующего \textit{класса ostis-систем}, обеспечивающих автоматизацию соответствующего \textit{вида человеческой деятельности}. При этом каждая такая проектируемая \textit{ostis-система} автоматизирует либо область, либо \textit{вид человеческой деятельности}, которая (который) является соответственно либо экземпляром (элементом), либо подвидом (подклассом) указанного выше \textit{вида человеческой деятельности}, соответствующего используемой \textit{специализированной \textit{ostis-технологии}}.}
    
    \scnheader{Ядро Технологии OSTIS}
    \scnidtf{Универсальная базовая \textit{ostis-технология}}
    \scnidtf{Универсальный компонент Технологии OSTIS}
    \scniselementrole{ключевой элемент}{ostis-технология}
    \scnrelto{ядро}{Технология OSTIS}
    \scnhaselement{технология}
    \scnrelfrom{вид деятельности, выполняемой с помощью технологии}{проектирование, производство, эксплуатация и реинжиниринг ostis-системы}
    \begin{scnindent}
	    \begin{scnreltoset}{объединение}
	        \scnitem{проектирование ostis-системы}
		        \begin{scnindent}
		            \scnidtf{построение логико-семантической модели \textit{ostis-системы}}
		        \end{scnindent}
	        \scnitem{производство ostis-системы}
		        \begin{scnindent}
		            \scnidtf{сборка логико-семантической модели ostis-системы и загрузка этой модели в память универсального интерпретатора таких моделей}
		        \end{scnindent}
	        \scnitem{эксплуатация ostis-системы}
		        \begin{scnindent}
		            \scnidtf{базовый (предметно-независимый) уровень организации деятельности конечного пользователя ostis-системы с помощью соответствующих методови средств}
		        \end{scnindent}
	        \scnitem{реинжиниринг ostis-системы}
			     \begin{scnindent}
			         \scnidtf{совершенствование \textit{ostis-системы} в процессе её эксплуатации}
			     \end{scnindent}
	    \end{scnreltoset}
	    \scnrelfrom{создаваемые продукты}{ostis-система}
	    \begin{scnindent}
	        \scnidtf{\textit{интеллектуальная компьютерная система}, построенная в соответствии со стандартом \textit{Технологии OSTIS}, предъявляемым к продуктам, создаваемым с помощью этой технологии}
	        \begin{scnindent}
	        	\scntext{примечание}{Указанный стандарт продуктов, создаваемых с помощью технологии OSTIS есть не что иное, как \textit{общая формальная семантическая теория интеллектуальных компьютерных систем}}
	        \end{scnindent}
	    \end{scnindent}
	\end{scnindent}    
   
    \begin{scnrelfromlist}{частная технология}
        \scnitem{Базовая Технология Проектирования ostis-систем}
        \begin{scnindent}
	        \begin{scnrelfromlist}{частная технология}
	            \scnitem{Технология проектирования баз знаний ostis-систем}
	            \scnitem{Технология проектирования решателей задач ostis-систем}
	            \scnitem{Технология проектирования интерфейсов ostis-систем}
		             \begin{scnrelfromlist}{частная технология}
		                 \scnitem{Технология проектирования невербальных интерфейсов ostis-систем с внешней средой}
		                 \scnitem{Технология проектирования интерфейсов ostis-систем с другими техническими системами}
		                 \scnitem{Технология проектирования пользовательских интерфейсов ostis-систем}
		             \end{scnrelfromlist}
	        \end{scnrelfromlist}
	        \scnrelfrom{реализация}{Метасистема IMS.ostis}
	        \begin{scnindent}
		        \scnidtf{Intelligent MetaSystem for ostis-systems design}
		        \scnidtf{OSTIS-система автоматизации проектирования ostis-систем}
	        \end{scnindent}
	  \end{scnindent}
	  \bigskip
        \scnitem{Технология производства ostis-систем}
        \begin{scnindent}
            \scntext{пояснение}{Основным компонентом, точнее, инструментальным средством \textit{технологии производства \mbox{ostis-систем}} является \textit{универсальный интерпретатор логико-семантических моделей \mbox{ostis-систем}}. Указанные \textit{логико-семантические модели ostis-систем} являются результатом \textit{проектирования ostis-систем} и представляют собой начальные (исходные) состояния \textit{баз знаний} разрабатываемых \textit{ostis-систем}. В отличие от \textit{инструмента производства ostis-систем}, методика их производства весьма проста и сводится к сборке разработанных логико-семантических моделей (начального состояния \textit{баз знаний}) разрабатываемых \textit{ostis-систем} и загрузке этих моделей в память \textit{универсального интерпретатора логико-семантических моделей \mbox{ostis-систем}}.}
            \scnrelfrom{реализация}{универсальный интерпретатор логико-семантических моделей ostis-систем}
            \begin{scnindent}
	            \scntext{пояснение}{Такой интерпретатор логико-семантических моделей ostis-систем может быть реализован либо программно на \textit{современных компьютерах}, либо аппаратно в виде компьютеров нового поколения, ориентированных на реализацию интеллектуальных компьютерных систем.}
	            \scntext{пояснение}{С формальной точки зрения универсальный интерпретатор логико-семантических моделей ostis-систем является пустой ostis-системой, которая способна приобретать и записывать формализованную информацию в свою память.}
            \end{scnindent}
        \end{scnindent}
        \scnitem{Базовая технология эксплуатации ostis-систем}
        \begin{scnindent}
            \scnidtf{Общая технология эксплуатации ostis-систем, включающая в себя общие методы и средства, используемые в процессе эксплуатации любых ostis-систем}
	         \scnrelfrom{реализация}{встраиваемая ostis-система поддержки эксплуатации ostis-систем}
	         \begin{scnindent}
	            	\scntext{пояснение}{Данная ostis-система входит (интегрирована) в состав каждой ostis-системы.}
	         \end{scnindent}
        \end{scnindent}
        \scnitem{Базовая технология реинжиниринга ostis-систем}
        \begin{scnindent}
            \scnrelfrom{реализация}{встраиваемая ostis-система поддержки реинжиниринга ostis-систем}
            \begin{scnindent}
            	\scntext{пояснение}{Данная ostis-система входит (интегрирована) в состав каждой ostis-системы и обеспечивает внесение изменений руками инженеров, сопровождающих эксплуатацию ostis-системы, или авторов базы знаний этой ostis-системы в текущее состояние базы знаний ostis-системы в ходе её экспуатации}
            \end{scnindent}
       \end{scnindent}
    \end{scnrelfromlist}
    \begin{scnrelfromlist}{специализированная технология}
        \scnitem{Общая технология проектирования ostis-систем автоматизации проектирования}
        \begin{scnindent}
            \begin{scnrelfromlist}{специализированная технология}
                \scnitem{Технология проектирования ostis-систем автоматизации проектирования строительных объектов}
                \scnitem{Технология проектирования ostis-систем автоматизации проектирования автомобилей}
                \scnitem{Технология проектирования ostis-систем автоматизации проектирования интегральных микросхем}
            \end{scnrelfromlist}
        \end{scnindent}
        \scnitem{Технология проектирования ostis-систем управления производством}
        \begin{scnindent}
            \begin{scnrelfromlist}{специализированная технология}
                \scnitem{Технология проектирования ostis-систем управления строительством различных объектов}
                \scnitem{Технология проектирования ostis-систем управления производством автомобилей}
                \scnitem{Технология проектирования ostis-систем управления производством микросхем}
                \scnitem{Технология проектирования ostis-систем управления предприятиями рецептурного производства}
                \begin{scnindent}
                    \scnrelfrom{специализированная технология}{Технология проектирования ostis-систем управления предприятиями производства молочной продукции}
                \end{scnindent}
            \end{scnrelfromlist}
        \end{scnindent}
        \scnitem{Технология проектирования интеллектуальных обучающих ostis-систем}
        \begin{scnindent}
            \begin{scnrelfromset}{комплекс специализированных технологий}
                \scnitem{Технология проектирования интеллектуальных обучающих ostis-систем для школьников}
                \scnitem{Технология проектирования интеллектуальных обучающих ostis-систем для студентов по общеобразовательным дисциплинам}
                \scnitem{Технология проектирования интеллектуальных обучающих ostis-систем для студентов по профильным дисциплинам}
                \scnitem{Технология проектирования интеллектуальных обучающих ostis-систем для магистрантов}
            \end{scnrelfromset}
            \begin{scnrelfromset}{комплекс специализированных технологий}
                \scnitem{Технология проектирования интеллектуальных обучающих ostis-систем по Математике}
                \scnitem{Технология проектирования интеллектуальных обучающих ostis-систем по Искусственному интеллекту}
            \end{scnrelfromset}
		\end{scnindent}
    \end{scnrelfromlist}
    
    \scnheader{специализированная ostis-технология}
    \scntext{примечание}{Приведённый нами перечень \textit{специализированных ostis-технологий} охватывает только некоторые области (фрагменты) \textit{человеческой деятельности}, подлежащие автоматизации с помощью \textit{ostis-технологий} в рамках \textit{Экосистемы OSTIS}.}
    \scnheader{Ядро Технологии OSTIS}
    \scntext{примечание}{Форма реализации \textit{Ядра Технологии OSTIS} (в виде ostis-системы \textit{IMS.ostis}) позволяет:
        \begin{scnitemize}
            \item использовать достоинства \textit{Технологии OSTIS} для повышения уровня автоматизации развития самой \textit{Технологии OSTIS} и для существенного повышения темпов такого развития;
            \item приобрести очень важный опыт применения \textit{Технологии OSTIS}
            \item создать центрально ядро \textit{Экосистемы OSTIS}, обеспечивающее поддержку семантической совместимости всех \textit{ostis-систем} и \textit{ostis-сообществ}, входящих в состав \textit{Экосистемы OSTIS}.
        \end{scnitemize}}
\end{scnsubstruct}
\scnsourcecommentinline{Завершили рассмотрение \textit{понятия ostis-технологии}}
\bigskip
\begin{scnsubstruct}
    \scnheader{Технология OSTIS}
    \scntext{пояснение}{\textit{Технология OSTIS} рассматривается нами как один из вариантов комплексного решения всех перечисленных выше сверхзадач, которые направлены на развитие деятельности в области искусственного интеллекта и которые, очевидно, сильно связаны друг с другом. Таким образом, \textit{Технология OSTIS} включает в себя:
        \begin{scnitemize}
            \item и постоянно развивающуюся общую формальную теорию интеллектуальных компьютерных систем, представленную в виде базы знаний соответствующего портала научно-технических знаний;
            \item и постоянно развивающийся комплекс моделей, методов и средств, используемых при проектировании интеллектуальных компьютерных систем и оформленных в виде интеллектуальной системы информационной и инструментальной поддержки (автоматизации) проектирования семантически совместимых интеллектуальных компьютерных систем;
            \item и постоянно развивающуюся глобальную экосистему, состоящую из семантически совместимых взаимодействующих интеллектуальных компьютерных систем, ориентированных на комплексную автоматизацию всевозможных видов человеческой деятельности;
        \end{scnitemize}}
    \scntext{пояснение}{Целью создания \textit{Технологии OSTIS} является не только построение методики, обеспечивающей четкую организацию коллективной \textit{человеческой деятельности} по проектированию, производству, эксплуатации и реинжинирингу \textit{интеллектуальных компьютерных систем}, но и построение мощных средств автоматизации (компьютерной поддержки) этой деятельности. Здесь важно подчеркнуть то, что \textit{интеллектуальные компьютерные системы}, разрабатываемые с помощью \textit{Технологии OSTIS (ostis-системы)} могут быть использованы для автоматизации \uline{любых} видов \textit{человеческой деятельности} и, в том числе, для автоматизации коллективной \textit{человеческой деятельности} по проектированию, производству, эксплуатации и реинжинирингу \textit{ostis-систем}. В рамках \textit{Технологии OSTIS} так и происходит - автоматизация проектирования, производства, эксплуатации и реинжиниринга \textit{ostis-систем} осуществляется с помощью специально предназначенных для этого \textit{ostis-систем}, некоторые из которых (например, для поддержки эксплуатации и реинжиниринга \textit{ostis-систем}) являются \textit{ostis-системами}, встроенными (интегрированными) в те \textit{ostis-системы}, поддержку эксплуатации и реинжиниринга которых они осуществляют.}
    \scnidtf{Комплексная технология, обеспечивающая автоматизацию самых различных действий (в том числе, и всевозможных видов человеческой деятельности) на основе семантически совместимых интеллектуальных компьютерных систем, способных координировать (согласовывать) свои действия как с себе подобными, так и с людьми}
    \scnidtf{Сумма (интеграция) всевозможных \textit{ostis-технологий}}
    \begin{scnrelfromlist}{достоинство}
        \scnfileitem{\textit{Технология OSTIS} представляет собой принципиально новый уровень развития \textit{информационных технологий}, в основе которого лежит переход от (from) data science к (to) knowledge science}
        \scnfileitem{Открытый характер \textit{Технологии OSTIS} как для тех, кто желает участвовать в её развитии, так и для пользователей \textit{Технологии OSTIS} --- для разработчиков прикладных \textit{ostis-систем}}
        \scnfileitem{Низкий порог вхождения для желающих развивать и желающих использовать имеющиеся в текущий момент методы и средства \textit{Технологии OSTIS}, что обеспечивается поддержкой качественного состояния документации по текущей версии \textit{Технологии OSTIS} с дополнительным описанием эволюции (развития) Технологии OSTIS, а также плана дальнейшего её развития}
        \scnfileitem{Децентрализованный характер управления проектами разработки \textit{ostis-систем}, основанный на четком согласовании коллективом разработчиков проектных задач}
        \scnfileitem{Ориентация на новое поколение компьютеров, без появления которых дальнейшее развитие \textit{технологий искусственного интеллекта} невозможно. При этом \textit{Технология OSTIS} позволяет достаточно конструктивно сформулировать требования, предъявляемые к таким компьютерам}
        \scnfileitem{\textit{Технология OSTIS} не только обеспечивает автоматизацию широкого многообразия видов человеческой деятельности, но и существенно повышает уровень (качество) этой автоматизации, благодаря (1) широкому применению методов и средств \textit{искусственного интеллекта} и (2) создание условий для \textit{конвергенции}, семантической совместимости и \textit{глубокой интеграции} как автоматизируемых видов человеческой деятельности, так и продуктов этой деятельности. В частности, это касается и автоматизации человеческой деятельности в области \textit{искусственного интеллекта}. \textit{Технология OSTIS} рассматривается как предлагаемый подход к конвергенции и интеграции как различных видов деятельности в области искусственного интеллекта, так и результатов этой деятельности (частных теорий различных компонентов и различных видов интеллектуальных систем, частных методов и средств проектирования различных видов и различных компонентов интеллектуальных компьютерных систем)}
        \scnfileitem{Ориентация на разработку компьютерных систем и коллективов таких систем, имеющих высокий уровень \textit{интеллекта} Ориентация на разработку глобальной сети \textit{интеллектуальных компьютерных систем}, обеспечивающей комплексную автоматизацию всех видов и областей \textit{человеческой деятельности}}
        \scnfileitem{Создание условий для формирования \textit{рынка знаний} на основе иерархической системы семантически совместимых \textit{порталов знаний}, соответствующих самым различным областям и \textit{видам человеческой деятельности}}
        \scnfileitem{Создание условий для перехода от традиционной формы публикации статей, монографий, отчетов и прочих документов к их публикации как фрагментов \textit{баз знаний} соответствующих \textit{порталов знаний}, что полностью исключает дублирование информации в публикуемых документах и обеспечивает непосредственное использование этой информации в \textit{интеллектуальных компьютерных системах}}
    \end{scnrelfromlist}
    \scntext{ближайшая задача}{Обеспечить низкий порог входа в Технологию OSTIS:
        \begin{scnitemize}
            \item для желающих участвовать в развитии \textit{Технологии OSTIS}, т.е. в совершенствовании \textit{Метасистемы IMS.ostis} (системы информационной поддержки и автоматизации проектирования \textit{ostis-систем}), которая сама также является \textit{ostis-системой}
            \item для разработчиков \textit{интеллектуальных компьютерных систем}, желающих использовать для этого \textit{Технологию OSTIS} (эти разработчики являются конечными пользователями \textit{Метасистем IMS.ostis});
            \item для конечных пользователей всевозможных иных \textit{ostis-систем}, т.е. компьютерных систем, разработанных по \textit{Технологии OSTIS} с непосредственным использованием в качестве инструмента \textit{Метасистемы IMS.ostis} (подчеркнем при этом, что базовы принципы организации взаимодействия \textit{Метасистемы IMS.ostis} с конечными пользователями полностью совпадают с базовыми принципами организации взаимодействия всех остальных \textit{ostis-систем}, разработанных с помощью \textit{Метасистемы IMS.ostis}, со своими конечными пользователями. Это обусловлено тем, что \textit{Метасистема IMS.ostis} сама также является \textit{ostis-системой} --- материнской \textit{ostis-системой}).
        \end{scnitemize}}
    \scntext{примечание}{Для решения указанной задачи необходимо создать инфраструктуру коллективного перманентного обновления (совершенствования) комплексной документации по \textit{Технологии OSTIS}, которая:
        \begin{scnitemize}
            \item обеспечила бы достаточную полноту и четкость фиксации текущего состояния \textit{Технологии OSTIS} и удовлетворяла бы как разработчиков \textit{Технологии OSTIS} (т.е. разработчиков \textit{Метасистемы IMS.ostis}), так и разработчиков \textit{ostis-систем}, не являющихся \textit{Метасистемой IMS.ostis} (т.е. конечных пользователей \textit{Метасистемы IMS.ostis}), и также конечных пользователей любых \textit{ostis-систем}
            \item обеспечила бы высокие темпы совершенствования данной документации на основании (1) четких правил согласования и утверждения различного рода предложений, (2) максимально возможной автоматизации процессов анализа, согласования и утверждения указанных предложений, (3) постоянного расширения числа авторов и (4) четких правил защиты авторских прав;
            \item обеспечила бы четкую фиксацию границ между текущим состоянием \textit{Технологии OSTIS} и разрабатываемыми, тестируемыми фрагментами её будущих версий с обоснованием таких нововведений и с планом их включения в соответствующую версию \textit{Технологии OSTIS}
            \item обеспечила бы четкую семантическую интеграцию документации той части \textit{Технологии OSTIS}, которая касается проектирования семантических моделей \textit{ostis-систем} и которая фактически сводится к проектированию \textit{баз знаний ostis-систем}, а также документации той части \textit{Технологии OSTIS}, которая описывает различные варианты программной или аппаратной реализации универсального интерпретатора логико-семантических моделей \textit{ostis-систем}. Подчеркнем при этом, что универсальность используемого в \textit{Технологии OSTIS} языка представления знаний дает возможность описывать на нем все, что угодно, в том числе и интерпретаторы семантических моделей \textit{ostis-систем}. Но делать это нужно с разумной степенью детализации.
        \end{scnitemize}}
    \scntext{ближайшая задача}{Осуществить конвергенцию и интеграцию всевозможных частных технологий проектирования и реализации различных видов компонентов интеллектуальных компонентов систем (в частности, баз знаний, различного вида логических моделей, искусственных нейронных сетей и т.п.)}
    \begin{scnrelfromlist}{класс создаваемых продуктов}
        \scnitem{ostis-система}
        \begin{scnindent}
            \scnidtf{индивидуальная ostis-система}
            \scntext{примечание}{Существенно подчеркнуть, что \textit{Технология OSTIS} порождает не просто множество \textit{ostis-систем}, а множество семантически совместимых и взаимодействующих \textit{ostis-систем}, образующих экосистему, которую будем называть \textit{Экосистемой OSTIS} (Экосистемой ostis-систем и их пользователей). Таким образом, можно считать что интегрированным продуктом \textit{Технологии OSTIS} является не множество ostis-систем, а  система (экосистема) \textit{ostis-систем}.}
        \end{scnindent}
        \scnitem{коллектив ostis-систем}
        \begin{scnindent}
            \scnsuperset{простой коллектив ostis-систем}
            \begin{scnindent}
            	\scnidtf{коллектив ostis-систем, членами которого являются только индивидуальные \mbox{ostis-системы}}
            \end{scnindent}
            \scnsuperset{иерархический коллектив ostis-систем}
            \begin{scnindent}
            	\scnidtf{коллектив ostis-систем, по крайней мере одним членом которого является коллектив ostis-систем}
           	\end{scnindent}
        \end{scnindent}
        \scnitem{ostis-сообщество}
        \begin{scnindent}
            \scnsuperset{простое ostis-сообщество}
            \scnsuperset{иерархическое ostis-сообщество}
            \begin{scnindent}
            	\scnhaselement{Экосистема OSTIS}
            \end{scnindent}
        \end{scnindent}
    \end{scnrelfromlist}
    \begin{scnrelfromlist}{основной продукт}
        \scnitem{\textit{Экосистема OSTIS}}
        \begin{scnindent}
            \scnidtf{Максимальное \textit{ostis-сообщество}, направленное на автоматизацию всех видов человеческой деятельности}
        \end{scnindent}
        \scnitem{\textit{Консорциум OSTIS}}
        \begin{scnindent}
            \scnidtf{\textit{ostis-сообщество}, направленное на развитие \textit{Технологии OSTIS}}
        \end{scnindent}
        \scnitem{\textit{Метасистема IMS.ostis}}
        \begin{scnindent}
            \scnidtf{Метасистема, являющаяся
                \begin{scnitemize}
                    \item \textit{корпоративной ostis-системой}, обеспечивающей организацию (координацию) деятельности \textit{Консорциума OSTIS}
                    \item формой представления реализации и фиксации текущего состояния \textit{Ядра Технологии OSTIS}
                    \item корпоративной \textit{ostis-системой}, взаимодействующей со всеми корпоративными \mbox{ostis-системами}, каждая из которых координирует развитие соответствующей \textit{специализированной ostis-технологии}.
                \end{scnitemize}}
       \end{scnindent}
    \end{scnrelfromlist}
    
    \scnheader{Экосистема OSTIS}
    \scntext{примечание}{Важной особенностью и достоинством \textit{Технологии OSTIS} является то, что все остальные продукты её использования (конкретные \textit{ostis-системы}) объединяются в сеть, т.е. становятся единым целостным продуктом использования \textit{Технологии OSTIS} --- \textit{Экосистема OSTIS}}
    \scntext{пояснение}{Социально-техническая сеть, состоящая из людей и \textit{ostis-систем}, которые являются
        \begin{scnitemize}
            \item семантически совместимыми;
            \item постоянно эволюционирующими индивидуально;
            \item постоянно поддерживающими свою совместимость с другими агентами в ходе своей индивидуальной эволюции;
            \item способными децентрализованно координировать свою деятельность.
        \end{scnitemize}}
    
    \scnheader{Технология OSTIS}
    \begin{scnrelfromset}{решаемая проблема}
        \scnitem{обеспечение семантической совместимости \uline{разрабатываемых} компьютерных систем}
        \begin{scnindent}
            \begin{scnrelfromset}{подход к решению}
                \scnitem{применение смыслового представления информации в памяти компьютерных систем}
                \scnitem{согласование и унификация системы используемых понятий и соответствующей иерархической системы формальных онтологий}
            \end{scnrelfromset}
        \end{scnindent}
        \scnitem{обеспечение \uline{поддержки} семантической совместимости компьютерных систем в ходе их эксплуатации и эволюции}
        \begin{scnindent}
            \scnrelfrom{подход к решению}{создание самоорганизованной экосистемы компьтерных систем}
        \end{scnindent}
        
    \end{scnrelfromset}
    \bigskip
\end{scnsubstruct}

        \scnsegmentheader{Перспективы использования Технологии OSTIS для повышения качества человеческой деятельности в области Искусственного интеллекта}
\begin{scnsubstruct}
    \begin{scnrelfromlist}{рассматриваемые вопросы}
        \scnfileitem{Могут ли \uline{практические} результаты работ в области \textit{Искусственного интеллекта}, существенно повысить эффективность развития \textit{Искусственного интеллекта} как научно-технической дисциплины, включая \uline{все формы} деятельности в области \textit{Искусственного интеллекта}}
        \scnfileitem{Какова перспектива использования \textit{Технологии OSTIS} для автоматизации других областей и видов \textit{человеческой деятельности}}
    \end{scnrelfromlist}
    \bigskip
    
    \scnheader{Научно-исследовательская деятельность в области Искусственного интеллекта}
    \scnrelfrom{субъект деятельности}{OSTIS-сообщество научно-исследовательской деятельности в области Искусственного интеллекта}
    \begin{scnindent}
    	\scntext{примечание}{Имеются ввиду специалисты разных стран и разных направлений \textit{Искусственного интеллекта}}
    \end{scnindent}
    \begin{scnrelfromset}{направления деятельности}
        \scnitem{Конвергенция и интеграция различных направлений Искусственного интеллекта}
        \scnitem{Конвергенция Искусственного интеллекта как отдельной научно-технической дисциплины с другими смежными научными дисциплинами}
        \begin{scnindent}
            \scntext{примечание}{Конвергенция с математикой, кибернетикой, информатикой, общей теорией систем, психологией, семиотикой, лингвистикой, гносеологией, логикой, методологией и др.}
        \end{scnindent}
        \scnitem{Разработка Общей теории интеллектуальных систем}
        \begin{scnindent}
            \scntext{примечание}{Речь идет как о естественных, так и об искусственных \textit{интеллектуальных системах}.}
        \end{scnindent}
    \end{scnrelfromset}
    \scnrelfrom{средство автоматизации}{OSTIS-портал научных знаний в области Искусственного интеллекта}
    \scnrelfrom{технология}{OSTIS-технология организации коллективной научно-теоретической деятельности }
    \begin{scnindent}
	    \scnrelto{частная технология}{Технология \textbf{\textit{реинжиниринга}} ostis-систем}
	    \begin{scnindent}
		    \scnidtf{Технология коллективного \textbf{\textit{реинжиниринга}} \textit{баз знаний ostis-систем}, обеспечивающая конвергенцию, интеграцию и согласование различных точек зрения и реализуемая абсолютно одинаковыми \textit{ostis-системами}, которые встраиваются (интегрируются) в состав каждой \textit{ostis-системы}}
		    \scnrelfrom{реализация}{Встраиваемая ostis-система поддержки реижиниринга ostis-систем}
	    \end{scnindent}
    \end{scnindent}
    \scnrelfrom{продукт}{Общая формальная теория интеллектуальных систем}
    \begin{scnindent}
    	\scntext{примечание}{В рамках \textit{Технологии OSTIS} \textit{Общая формальная теория интеллектуальных систем} представляется в виде \textit{базы знаний} \textit{OSTIS-портала научных знаний в области Искусственного интеллекта}.}
    \end{scnindent}
    
    \scnheader{Общая формальная теория интеллектуальных систем}
    \scntext{примечание}{Зачем нужна \textit{Общая теория интеллектуальных систем}?
        \\Очевидно, что без этой теории невозможно построить набор методов и средств, обеспечивающий комплексную поддержку разработки \textit{интеллектуальных компьютерных систем} различного назначения и с различным набором навыков (способностей, возможностей), которыми могут обладать \textit{интеллектуальные компьютерные системы}, но необязательно каждая из них.При этом важно не просто построить \textit{Общую теорию интеллектуальных систем} и довести ее до строгого формального уровня, но также довести представление такой формальной теории до уровня базы знаний соответствующего \textit{портала научных знаний}.}
        
    \scnheader{OSTIS-технология организации коллективной научно-теоретической деятельности}
    \scntext{пояснение}{Подчеркнем, что \textit{Технология OSTIS} создает достаточно удобные (конструктивные) условия для решения таких проблем, как
        \begin{scnitemize}
            \item cогласование систем понятий разных научных дисциплин (в частности, разных направлений \textit{Искуственного интеллекта}) и, как следствие, возможность реализациидостаточно качественной семантической совместимости;
            \item конвергенция разных научных дисциплин, важным механизмом которой является увеличение числа общих понятий, используемых этими дисциплинами (в частности, этого можно добиться путем введения таких понятий, каждое из которых является обобщением, например, двух понятий, одно из которых относится к одной дисциплине, а другое - к другой)
            \item интеграция научных дисциплин.
        \end{scnitemize}
        Удобство (конструктивность, формализованность) решения указанных проблем обусловлено тем, что каждая научная дисциплина представляется постоянно развивающейся базой знаний, которая в \textit{Технологии OSTIS} представляется в виде специальнойсемантической сети (в виде текста \textit{SC-кода}), которой соответствуют достаточно простые синтаксические и семантические правила.Важной проблемой организации научно-теоретической деятельностиявляется реализация эффективной процедуры согласования различных точек зрения и обеспечения их конвергенции и глубокой (бесшовной) интеграции. В рамках коллективного развития базы знаний портала научных знаний можно обеспечить:
        \begin{scnitemize}
            \item существенное сокращение времени, затрачиваемого на согласование используемых понятий;
            \item существенное повышение эффективности рецензирования самых различных предложений;
            \item существенное сокращение времени, затрачиваемого на публикацию научных результатов, так как меняется форма публикаций-публикации. Эти результаты оформляются в смысловом виде как фрагменты соответствующей базы знаний, что предполагает отсутствие дублированиянаучных текстов, т.е. отсутствие возможности представления одного и того же результата во многих формах в разных статьях и монографиях;
            \item автоматизацию анализа качества новых знаний, предлагаемых в состав совершенствуемой базы знаний;
            \item автоматизацию мониторинга общего качества всей базы знаний.
        \end{scnitemize}
        Очевидно, что качество накапливаемых человечеством \textit{научно-технических знаний} во многом определяет качество (уровень развития) всего человечества как коллектива интеллектуальных систем. Качество указанных знаний определяется
        \begin{scnitemize}
            \item трудоемкостью их накопления и систематизации
            \item уровнем конвергенции различных научно-технических дисциплин (уровнем целостности всего комплекса знаний)
            \item четкостью фиксации текущего (согласованного) состояния накопленных знаний
            \item четкостью фиксации истории эволюции накапливаемых знаний
            \item трудоемкостью согласования различных техн ..
            \item четкостью фиксации противоречий и разногласий
        \end{scnitemize}}
    
    \scnheader{язык научно-технической информации}
    \scntext{примечание}{Современные научно-технические тексты не являются естественно-языковыми --- это смесь формальных и естественно языковых текстов. Но именно в таком виде осуществляется накопление человеческих знаний в Internete. Необходим универсальный формальный язык, в который был бы в достаточной степени удобным (привычным) и для человека, и для интеллектуальных компьютерных систем. Перспективным подходом к решению этой проблемы является разработка универсального языка семантических сетей.\\
    Переход к \textit{интегрированному} семантическому пространству научно-технических знаний, в основе которой лежит интернациональная семантическая формализация знаний.\\
    Например, научно техническая статья должна быть \textit{завершенной} нек матем знаний, возволяющей, например, полностью автоматизировать анализ (рецензирование, верификацию) поступившей статьи (например верификацию локазательства теорем).\\
    А научно-технический журнал превращается в портал знаний по заданной научно-технической области.\\
    Перманентное развитие баз знаний такого портала становится открытым проектоом, в рамках которого
        \begin{scnitemize}
            \item каждый желающий может высказать свое предложение
            \item каждый может высказать \uline{замечание} по поводу какого-либо предложения (быть рецензентом)
            \item внести исправления в свое предложение (по замечаниям)
            \item признать исправления своих замечаний
            \item проголосовать за/против предложение или исправленное предложение
            \item признание каждого предложения осуществляется \uline{автоматически} на основании высказанных мнений с учетом квалификации каждого высказывшегося по отношению к соответствующему разделу базы знаний.
            \item текущая квалификация каждого специалиста постоянно уточняется (повышается) на основании вклада этого специалиста в развитие базы знаний портала
            \item вклад любого сепциалиста персонифицируется (защита интеллектуальной собственности) и его ценность (значимость) автоматически оценивается (на основе анализа того, как используется знания, автором или рецензентом --- соавтором которых специалист является, в коммерческих проектах!! При этом ссылка на использованные знания является обязательной)
        \end{scnitemize}}
    
    \scnheader{OSTIS-портал научных знаний в области Искусственного интеллекта}
    \scntext{в перспективе}{В перспективе каждый \textit{ostis-портал научных знаний} может преобразоватьсяв сеть семантически совместимых \textit{ostis-порталов научных знаний}, соответствующих различным направлениям заданной \textit{научной дисциплиной} (например, различным направлениям \textit{Искусственного интеллекта})}
    \bigskip
    
    \scnheader{Коллектив специалистов в области Искусственного интеллекта}
    \scntext{в перспективе}{\textit{OSTIS-сообщество} субъектов \textit{научно-исследовательской деятельности} в области \textit{Искусственного интеллекта}}
    \begin{scnindent}
	    \scntext{уточнение}{указанными субъектами являются объединенные в сеть специалисты в области \textit{Искусственного интеллекта}, неформальные группы таких специалистов, организации или подразделения различных организаций, работающих в области \textit{Искусственного интеллекта} и \textit{ostis-порталы научных знаний} вобласти Искусственного интеллекта}
	    \scnrelto{часть}{Экосистема OSTIS}
    \end{scnindent}
    \bigskip
    
    \scnheader{Разработка Базовой Комплексной технологии проектирования интеллектуальных компьютерных систем}
    \scnrelfrom{субъект деятельности}{Коллектив разработчиков Базовой Комплексной технологии проектирования интеллектуальных компьютерных систем}
    \begin{scnindent}
	    \scntext{примечание}{Речь идет об открытом проекте разработки указанной технологии и, соответственно, об открытом международном коллективе разработчиков, формируемом на добровольной основе}
	    \begin{scnrelfromset}{направления деятельности}
	        \scnitem{Разработка Общей теории интеллектуальных компьютерных систем}
	        \begin{scnindent}
	            \scntext{примечание}{Речь идет об \uline{искусственных} (компьютерных) интеллектуальных системах и о разработке \uline{стандарта} таких технологических систем.}
	        \end{scnindent}
	        \scnitem{Разработка Теории проектирования интеллектуальных компьютерных систем}
	        \begin{scnindent}
	            \scntext{примечание}{Имеются в виду интеллектуальные компьютерные системы, соответствующие стандарту, разработанному в виде общей теории таких систем, имеется в виду рассмотрение самого процесса проектирования таких систем, т.е. рассмотрение методов их проектирования и проектных библиотек.}
	        \end{scnindent}
	        \scnitem{Разработка комплекса средств автоматизации проектированияинтеллектуальных компьютерных систем}
	        \begin{scnindent}
	            \scntext{примечание}{Данные средства автоматизации проектирования (средства решения проектных задач) при их реализации с помощью \textit{Технологии OSTIS} входят в состав решателя задач \textit{Метасистемы IMS.оstis}.}
	        \end{scnindent}
	        \scnitem{Конвергенция и интеграция различного вида знаний, хранимых в памяти проектируемых интеллектуальных компьютерных систем}
	        \scnitem{Конвергенция и интеграция различных моделей решения задач,используемых проектируемыми интеллектуальными компьютернымисистемами}
	    \end{scnrelfromset}
    \end{scnindent}
    \scnrelfrom{предлагаемый подход}{\textbf{Проект IMS.ostis}}
    \begin{scnindent}
    \scnidtf{Разработка Базовой Комплексной оstis-технологии проектирования оstis-систем}
    \scnidtf{Разработка Базовой Комплексной технологии проектирования оstis-систем с помощью специально предназначенной для этого \textit{оstis-системы}, котораяназвана нами \textit{Метасистемой IMS.ostis}}
    \scnidtf{Проект разработки \textit{Метасистемы IMS.ostis}}
    \begin{scnrelfromset}{принципы, лежащие в основе}
        \scnfileitem{Речь идет о проектировании не просто интеллектуальных компьютерных систем, а \mbox{ostis-систем}, в виде которых можно построить любую интеллектуальную компьютерную систему. Соблюдение этого принципа является важнейшейцелью эволюции \textit{Технологии ОSTIS}}
        \scnfileitem{Система автоматизации проектирования ostis-систем реализуется также в виде ostis-системы --- \textit{Метасистемы IMS.ostis}}
        \scnfileitem{Эволюция технологии проектирования ostis-систем сводится к эволюции (реинжинирингу) базы знаний \textit{Метасистемы IMS.ostis}.}
    \end{scnrelfromset}
    \bigskip
    \scntext{примечание}{Если речь вести о \textit{Технологии ОSTIS}, то следует говорить не только о самой данной технологии, но и о проекте, направленном на создание и перманентное совершенствование этой технологии, так как важнейшей особенностью и достоинством \textit{Технологии ОSTIS} являются высокие темпы ее эволюции. Указанное достоинство обеспечивается прежде всего тем, что Технология ОSTIS реализуется в виде ostis-системы (\textit{Метасистемы IMS.ostis}).}
    \scnrelfrom{средство автоматизации}{Метасистема IMS.ostis}
    \begin{scnindent}
	    \scnidtf{ОSTIS-система автоматизации комплексного проектирования ostis-систем}
	    \scntext{примечание}{При \textit{Разработке Базовой Комплексной технологии проектирования интеллектуальных компьютерных систем} (точнее ostis-систем) средством автоматизации этой деятельности является не вся \textit{Метасистема IMS.ostis}, а только ее часть --- входящая в состав \textit{Метасистемы  IMS.ostis} типовая \textit{Встраиваемая ostis-система поддержки реижиниринга ostis-систем}, которая поддерживает деятельность разработчиков базы знаний Метасистемы IMS.оstis. Это обусловлено тем, что вся деятельность по \textit{Разработке Базовой Комплексной технологии проектирования интеллектуальных компьютерных систем} (ostis-систем) сводится к разработке (инженирингу) и обновлению (совершенствованию, реинжинирингу) \textit{Базы знаний Метасистемы IMS.ostis}).}
    \end{scnindent}
    \scnrelfrom{технология}{Технология реинжиниринга ostis-систем}
    \begin{scnindent}
    	\scnrelfrom{реализация}{Встраиваемая ostis-система поддержки реинжиниринга ostis-систем}
    \end{scnindent}
    \scnrelfrom{продукт}{Комплексная ostis-технология проектирования ostis-систем}
    \begin{scnindent}
    	\scnrelfrom{реализация}{Метасистема IMS.оstis}
    \end{scnindent}
    \scnidtf{Человеко-машинная деятельность, осуществляемая в рамках \textit{Экосистемы OSTIS} и направленная на разработку и перманентное совершенствование \textit{Метасистемы IMS.ostis}, которая является формой представления (отображения) (1) текущего состояния \textit{Технологии OSTIS}, как комплекса методов и средств автоматизации (поддержки) разработки\textit{ostis-систем} и (2) текущего состояния самого \textit{Проекта IMS.ostis}.}
    \scntext{примечание}{Принципы (правила) организации деятельности в рамках \textit{Проекта IMS.ostis} полностью совпадают с принципами (правилами) организации деятельности в рамках любого другого проекта, направленного на разработку и совершенствование любой другой ostis-системы.}
    \scnrelto{ключевой подпроект}{Проект Экосистемы OSTIS}
    \begin{scnindent}
    	\scnidtf{Совместная деятельность ученых, инженеров и ostis-систем, входящих в \textit{Экосистему OSTIS}, направленная на перманентное совершенствование \textit{Экосистемы OSTIS} --- на совершенствование (реинжиниринг) входящих в неё  \textit{ostis-систем} и на создание новых ostis-систем и их включение в состав \textit{Экосистемы OSTIS.}}
    \end{scnindent}
    \scntext{пояснение}{\textit{ostis-система}, являющаяся:
        \begin{scnitemize}
            \item ostis-порталом научно-технических знаний по \textit{Технологии OSTIS}, база знаний которого включает в себя:
            \begin{scnitemizeii}
                \item формальную теорию \textit{ostis-систем}
                \item формальную теорию (методику) проектирования  \textit{ostis-систем}
                \item формальную спецификацию средств автоматизации проектирования \textit{ostis-систем}
                \item библиотеку проектирования \textit{ostis-систем}
                \item формальную спецификацию средств производства спроектированных \textit{ostis-систем}
            \end{scnitemizeii}
            \item \textit{ostis-системой} автоматизации (поддержки) проектирования \textit{ostis-систем}
            \item \textit{ostis-системой} поддержки производства (сборки, синтеза, генерации) спроектированных \textit{ostis-систем}
            \item \textit{ostis-системой} поддержки реинжиниринга \textit{ostis-систем} в ходе их эксплуатации
        \end{scnitemize}}
    \end{scnindent}
    \bigskip
    
    \scnheader{Метасистема IMS.ostis}
    \scnidtf{Универсальная базовая (предметно-независимая) ostis-система автоматизации проектирования ostis-систем (любых ostis-систем)}
    \scnrelboth{следует отличать}{специализированная ostis-система автоматизации проектирования ostis-систем}
    \scniselement{ostis-система}
    \scnrelto{корпоративная ostis-система}{Консорциум OSTIS}
    \scnidtf{IMS.ostis}
    \scnidtf{Интеллектуальная метасистема, построенная по стандартам \textit{технологии OSTIS} и предназначенная (1) для инженеров \textit{ostis-систем} --- для поддержки проектирования. Реализации и обновления (реинжиниринга) \textit{ostis-систем} и для разработчиков \textit{Технологии OSTIS} --- для поддержки коллективной деятельности по развитию стандартов и библиотек \textit{Технологии OSTIS.}}
    \scnrelto{форма реализации}{Технология OSTIS}
    \scnrelto{продукт}{Проект IMS.ostis}
    \scnidtf{Интеллектуальная Метасистема, являющаяся формой (вариантом) реализации (представления, оформления) \textit{Технологии OSTIS} в виде \textit{ostis-системы}}
    \scntext{примечание}{Тот факт, что Технология OSTIS реализуется в виде ostis-системы, является весьма важным для эволюции Технологии OSTIS, поскольку методы и средства эволюции (перманентного совершенствования) Технологии OSTIS становятся фактически совпадающими с методами и средствами разработки любой (!) ostis-системы на всех этапах их жизненного цикла.
        \\Другими словами, эволюция Технологии OSTIS осуществляется методами и средствами самой этой технологии.}
    \scnidtf{Система комплексной автоматизации (информационной и инструментальной поддержки) проектирования и реализации ostis-систем, которая сама реализована также в виде ostis-системы.}
    \scnidtf{Портал знаний по Технологии OSTIS, интегрированный с САПРом ostis-систем и реализованный в виде ostis-системы.}
    \scniselement{портал научно-технических знаний}
    \bigskip
    \bigskip
    \begin{scnset}
        \scnitem{Метасистема IMS.ostis}
        \begin{scnindent}
            \scniselement{система автоматизации проектирования}
            \begin{scnindent}
                \scnidtf{CAD-система}
                \begin{scnindent}
                    \scnrelto{аббревиатура}{\scnfilelong{Computer Aided Design system}}
                \end{scnindent}
            \end{scnindent}
            \scniselement{интеллектуальная обучающая система}
        \end{scnindent}
    \end{scnset}
    \scnrelboth{семантическая эквивалентность}{\scnfilelong{Метасистема IMS.ostis является одновременно и системой автоматизации проектирования ostis-систем, и интеллектуальной системой, обучающей методам  и средствам проектирования ostis-систем.}}
    \begin{scnindent}
    	\scntext{следовательно}{этот факт существенно повышает качество проектирования прикладных ostis-систем, расширяет контингент разработчиков ostis-систем и интегрирует проектную (инженерную) деятельность в области искусственного интеллекта с образовательной деятельностью в этой области.}
    \end{scnindent}
    \bigskip
    
    \scnheader{Разработка технологии производства спроектированных интеллектуальных компьютерных систем}
    \scnrelfrom{предлагаемый подход}{Проект разработки универсальных интерпретаторов логико-семантических моделей ostis-систем}
    \scnrelfrom{класс продуктов}{универсальный интерпретатор логико-семантических моделей ostis-систем}
    \begin{scnindent}
        \scnidtf{пустая ostis-система --- ostis-система, на базе которой можно построить любую ostis-систему, если логико-семантическую модель этой системы, загрузить в память указанной выше пустой ostis-системы}
        \scnidtf{Базовый интерпретатор логико-семантических моделей ostis-систем}
        \scnidtf{Интерпретатор Универсальной абстрактной sc-машины}
	\end{scnindent}
	
    \scnheader{Проект разработки универсальных интерпретаторов логико-семантических моделей ostis-систем.}
    \scnidtf{Проект реализации универсальной абстрактной sc-машины}
    \scnrelfrom{альтернативный подпроект}{Проект Программной реализации интерпретаторов Универсальной абстрактной sc-машины}
    \scnrelfrom{альтернативный подпроект}{Проект разработки универсальных sc-компьютеров}
    \scntext{применение}{Подчеркнём, что разные альтернативные варианты реализации универсального интерпретатора логико-семантических моделей ostis-систем (универсальной абстрактной sc-машины) никоим образом не влияет на процесс и результат проектирования ostis-систем, то есть на процесс и результат построения логико-семантических моделей разрабатываемых ostis-систем.\\
    Другими словами, принципы представления и структуризации логико-семантических моделей ostis-систем и архитектура универсального интерпретатора этих моделей чётко стратифицированы и, следовательно, могут эволюционировать в достаточной степени независимо друг от друга. Тем не менее некоторая зависимость всё же есть --- согласованная трактовка понятия универсальной sc-машины и согласованная форма (язык) передача логико-семантической модели разрабатываемой ostis-системы из \textit{Метасистемы IMS.ostis} в пустую ostis-систему.}
    
    
    \scnheader{Универсальная абстрактная sc-машина}
    \scntext{пояснение}{Абстрактная машина, которая задается:
        \begin{scnitemize}
            \item \textit{SC-кодом} --- внутренним языком представления знаний в памяти \textit{ostis-системы}
            \item абстрактной \textit{sc-памятью}, которая уточняет динамику обрабатываемых текстов \textit{SC-кода}
            \item универсальным набором (семейством) \textit{sc-агентов}, осуществляющих обработку текстов \textit{SC-кода}.
        \end{scnitemize}}
    \begin{scnindent}
    	\scntext{примечание}{В основе Универсальной абстрактной \textit{sc-машины} лежит интерпретатор \textit{Языка SCP} --- Базового языка программирования \textit{ostis-систем}.}
    \end{scnindent}
    
    \scnheader{Язык SCP}
    \scnidtf{Базовый язык программирования ostis-систем с его синтаксисом, денотационной семантикой и операционной семантикой.}
    \scnidtf{Язык программирования SCP (Semantic Code Programming)}
    \scnrelfrom{смотрите}{\nameref{sd_scp}}
    
    \scnheader{Проект программной реализации интерпретаторов Универсальной абстрактной sc-машины.}
    \scnidtf{Проект разработки программной реализации Универсальной абстрактной sc-машины на современных компьютерах}
    \scnrelfrom{класс продуктов}{программно реализованный интерпретатор Универсальной абстрактной sc-машины}
    \begin{scnindent}
    	\scnrelfrom{смотрите}{\nameref{sd_program_interp}}
    \end{scnindent}
    
    \scnheader{Проект разработки универсальных sc-компьютеров}
    \scnidtf{Проект разработки аппаратной реализации универсальной абстрактной sc-машины в виде компьютера нового поколения, ориентированного на использование в интеллектуальных компьютерных системах( в нашем случае --- в ostis-системах)}
    \scnrelfrom{класс продуктов}{универсальный sc-компьютер}
    \scnidtf{универсальный ostis-компьютер}
    \scnidtf{cемантический ассоциативный компьютер для ostis-систем}
    \scnidtf{аппаратно реализованный интерпретатор абстрактной sc-машины}
    \begin{scnindent}
    	\scnrelfrom{смотрите}{\nameref{sd_sem_comp}}
    \end{scnindent}
    \scntext{примечание}{Тот факт, что универсальный sc-компьютер, разрабатывается под конкретную технологию проектирование интеллектуальных компьютерных систем (Технологию OSTIS), которая развивается, накапливает опыт разработки и внедрения самых различных прикладных интеллектуальных систем независимо от наличия универсальных sc-компьютеров, имеет принципиальное значение. Опыт создания компьютеров, имеющих принципиально новую архитектуру, показывает, что разработка компьютеров нового поколения без серьезной подготовки технологий их применения, без подготовки соответствующий инфраструктуры приводит к неэффективному использованию результатов разработки и к их быстрому моральному старению.}
    \scntext{примечание}{Разработка универсального sc-компьютера является важнейшим следующим этапом развития технологии OSTIS, который обеспечит существенное повышение производительности (быстродействия) ostis-систем.\\
    Развитие технологий искусственного интеллекта неизбежно приведёт к необходимости создания компьютеров принципиально нового поколения, предназначенных для использования в интеллектуальных компьютерных системах. Поэтому изначально ориентация Технологии OSTIS на компьютеры нового поколения является принципиальной и весьма перспективной особенностью Технологии OSTIS, обеспечивающей её высокую конкурентоспособность.}
    
    \scnheader{Проект разработки универсальных интерпретаторов логико-семантических моделей ostis-систем}
    \scntext{примечание}{При построении любого интерпретатора любой информационной машины (в нашем случае --- абстрактной sc-машины) должны быть чётко полно, а самое главное на формальном языке (в нашем случае --- SC-коде) описано следующее:
        \begin{scnitemize}
            \item синтаксис, денотационная семантика и операционная семантика интерпретируемой машины (в нашем случае для абстрактной sc-машины это синтаксис и денотационная семантика SC-кода и языка SCP, а также операционная семантика языка SCP);
            \item синтаксис, денотационная семантика и операционная семантика интерпретирующей информационный машины;
            \item соотношение между указанными формальными моделями интерпретируемой информационной машины и интерпретирующей информационной машины, определяющее семантическую и операционную (функциональную) эквивалентность.
        \end{scnitemize}
        Подчеркнем, что без построения указанной строгой формальной модели соответствия (эквивалентности) интегрируемой и интерпретирующей информационной машины организовать качественную коллективную разработку интерпретаторов сложной информационной машины (например, абстрактной sc-машины) невозможно, так как будет совершаться большое количество поздно обнаруживаемых ошибок.}
    \scntext{примечание}{Разрабатываемые сейчас варианты реализации \textit{универсального интерпретатора логико-семантических моделей ostis-систем} (программный и аппаратный вариант) являются в известной мере привычными объектами проектирования для современных технологий проектирования программных систем и технологий проектирования интегрированных микросхем и их комплексов.\\
    Тем не менее, повышение уровня сложности указанных объектов проектирования и указанных характеристик проектирования требует существенного повышения уровня интеллекта у соответствующих систем автоматизации (поддержки) проектирования). \textit{Технология OSTIS} уже имеет достаточный опыт разработки \textit{ostis-систем автоматизации проектирования} (достаточно указать Метасистему IMS.ostis, обеспечивающую автоматизацию проектирования ostis-систем). Таким образом для повышения качества разработки \textit{Программной реализации универсальной абстрактной sc-машины} и разработки \textit{универсального sc-компьютера} целесообразно разработать, соответственно, \textit{OSTIS-систему поддержки проектирования сложных программных систем}, а также \textit{OSTIS-систему поддержки проектирования интегральных микросхем и их комплексов}. Здесь речь может идти об интеллектуальных надстройках над существующими средствами автоматизации проектирования и управления проектами.\\
    При проектировании \textit{Программной реализации универсальной абстрактной sc-машины}, а также \textit{универсального sc-компьютера} такая интеллектуальная надстройка абсолютно необходима, поскольку при проектировании указанных объектов необходимо четко отслеживать соответствия между компонентами этих объектов и компонентами интерпретируемой ими \textit{универсальной абстрактной sc-машины}. Актуальность указанной интеллектуальной надстройки обусловлена также тем, что \textit{универсальная абстрактная sc-машина} может корректироваться.\\
    Следует отметить возможную связь между процессом проектирования \textit{Программной реализации универсальной абстрактной sс-машины} и проектированием \textit{универсального sc-компьютера}. Дело в том, что \textit{Программную реализацию универсальной абстрактной sc-машины} можно и нужно рассматривать как программную модель не только интегрируемой \textit{универсальной абстрактной sc-машины}, но и проектируемого \textit{универсального sc-компьютера}. Таким образом, реализацию универсальной sc-машины можно развивать в двух направлениях:
        \begin{scnitemize}
            \item в направлении повышения её производительности;
            \item в направлении более детальной эмуляции универсального sc-компьютера на уровне взаимодействия всё более мелких компонентов этого компьютера.
        \end{scnitemize}}
    
    \bigskip
    
    \scnheader{Специализированная инженерия в области Искусственного интеллекта}
    \scnrelfrom{предлагаемый подход}{Специализированная инженерия, осуществляемая на основе Технологии OSTIS}
	\begin{scnindent}
	    \begin{scnrelfromset}{декомпозиция}
	        \scnitem{Разработка ostis-систем автоматизации проектирования различных классов ostis-систем}
	        \begin{scnindent}
	            \scnidtf{Разработка специализированных ostis-технологий}
	            \begin{scnrelfromlist}{часть}
	                \scnitem{Разработка ostis-систем автоматизации проектирования ostis-систем автоматизации проектирования}
	                \begin{scnindent}
	                    \scnidtf{Разработка ostis-технологий проектирования}
	                \end{scnindent}
	                \scnitem{Разработка ostis-систем автоматизации проектирования ostis-систем автоматизации производства}
	                \begin{scnindent}
	                    \scnidtf{Разработка ostis-технологий управления производством}
	                \end{scnindent}
	                \scnitem{Разработка ostis-систем автоматизации проектирования ostis-систем управления транспортными системами}
	                \begin{scnindent}
	                    \scnidtf{Разработка ostis-технологий управления транспортными системами}
	                \end{scnindent}
	                \scnitem{Разработка ostis-систем автоматизации проектирования диагностических ostis-систем}
	                \begin{scnindent}
	                    \scnidtf{Разработка ostis-технологий диагностики (технической, медицинской)}
	                \end{scnindent}
	                \scnitem{Разработка ostis-систем автоматизации проектирования обучающих ostis-систем}
	                \begin{scnindent}
	                    \scnidtf{Разработка ostis-технологий обучения людей}
	                \end{scnindent}
	                \scnitem{Разработка ostis-систем автоматизации проектирования ostis-систем управления умными домами}
	                \begin{scnindent}
	                    \scnidtf{Разработка ostis-технологий управления умными домами}
	                \end{scnindent}
	                \scnitem{Разработка ostis-систем автоматизации проектирования ostis-систем управления умными больницами}
	                \begin{scnindent}
	                    \scnidtf{Разработка ostis-технологий управления умными больницами}
	                \end{scnindent}
	                \scnitem{Разработка ostis-систем автоматизации проектирования ostis-систем управления умными поликлиниками}
	                \begin{scnindent}
	                    \scnidtf{Разработка ostis-технологий управления умными поликлиниками}
	                \end{scnindent}
	                \scnitem{Разработка ostis-систем автоматизации проектирования ostis-систем управления умными городскими районами}
	                \begin{scnindent}
	                    \scnidtf{Разработка ostis-технологий управления умными городскими районами}
	                \end{scnindent}
	                \scnitem{Разработка ostis-систем автоматизации проектирования ostis-систем управления умными городами}
	                \begin{scnindent}
	                    \scnidtf{Разработка ostis-технологий управления умными городами}
	                \end{scnindent}
	            \end{scnrelfromlist}
	        \end{scnindent}
	        \scnitem{Разработка (на основе соответствующих ostis-технологий проектирования) ostis-систем автоматизации проектирования различных классов объектов, не являющихся ostis-системами}
	        \begin{scnindent}
	            \begin{scnrelfromlist}{часть}
	                \scnitem{Разработка семейства ostis-систем автоматизации проектирования различных видов интегральных микросхем}
	                \scnitem{Разработка семейства ostis-систем автоматизации проектирования различных видов автомобилей}
	                \scnitem{Разработка семейства ostis-систем автоматизации проектирования различных видов строительных объектов}
	            \end{scnrelfromlist}
	        \end{scnindent}
	        \scnitem{Разработка ostis-систем автоматизации производства}
	        \begin{scnindent}
	            \scnidtf{Разработка интеллектуальных систем управления производственными предприятиями}
	        \end{scnindent}
	        \scnitem{Разработка ostis-систем управления транспортными средствами}
	        \scnitem{Разработка диагностических ostis-систем}
	        \scnitem{Разработка обучающих ostis-систем}
	        \scnitem{Разработка ostis-систем управления умными домами}
	        \scnitem{Разработка ostis-систем управления умными больницами}
	        \scnitem{Разработка ostis-систем управления умными поликлиниками}
	        \scnitem{Разработка ostis-систем управления умными городскими районами}
	        \scnitem{Разработка ostis-систем управления умными городами}
	    \end{scnrelfromset}
	\end{scnindent}
    \bigskip
    
    \scnheader{Образовательная деятельность в области Искусственного интеллекта}
    \scnrelfrom{предлагаемый подход}{Образовательная деятельность в области Искусственного интеллекта, осуществляемая на основе Технологии OSTIS}
	\begin{scnindent}
	    \scniselement{образовательная деятельность}
	    \scniselement{человеческая деятельность, осуществляемая на основе Технологии OSTIS}
	    \begin{scnindent}
	    	\scnidtf{человеческая деятельность, комплексная автоматизация которой осуществляется либо индивидуальной \textit{ostis-системой}, либо \textit{коллективом ostis-систем} (сетью ostis-систем)}
	    \end{scnindent}
	    \scnrelfrom{субъект}{OSTIS-сообщество Образовательной деятельности в области Искусственного интеллекта}
	    \begin{scnindent}
		    \scnidtf{глобальное (максимальное) OSTIS-сообщество, осуществляющее Образовательную деятельность в области Искусственного интеллекта и обеспечивающее активное и взаимовыгодное сотрудничество между всеми заинтересованными в этом субъектами и, в первую очередь, с соответствующими кафедрами различных вузов}
		    \scnrelto{часть}{Экосистема OSTIS}
		    \begin{scnindent}
			    \scnidtf{глобальная сеть ostis-систем вместе с их пользователями}
			    \scnidtf{глобальное ostis-сообщество}
			\end{scnindent}
		    \scniselement{ostis-сообщество}
		    \begin{scnindent}
		    	\scnidtf{локальная сеть \textit{ostis-систем} вместе с их пользователями}
		    \end{scnindent}
		    \scntext{пояснение}{Данное \textit{ostis-сообщество} включает в себя:
		        \begin{scnitemize}
		            \item все кафедры, которые готовят молодых специалистов в области \textit{Искусственного интеллекта} и которые могут входить в состав самых различных вузов;
		            \item все те организации, которые разрабатывают или эксплуатируют интеллектуальные компьютерные системы и которые готовы сотрудничать с вузами для повышения квалификации поступающих к ним молодых специалистов в области \textit{Искусственного интеллекта}
		            \item студентов, магистрантов и аспирантов, обучающихся в области \textit{Искусственного интеллекта} в разных вузах;
		            \item их преподавателей;
		            \item семейство интеллектуальных обучающих ostis-систем по различным дисциплинам (направлениям) Искусственного интеллекта, которые семантически совместимы и тесно связаны с \textit{OSTIS-порталом научных знаний по Искусственному интеллекту} и с \textit{Метасистемой IMS.ostis}
		            \item \textit{OSTIS-портал научных знаний по Искусственному интеллекту}, осуществляющий поддержку развития Общей теории интеллектуальных систем как естественного, так и искусственного происхождения;
		            \item \textit{Метасистема IMS.ostis}, осуществляющая поддержку развития \textit{Общей теории интеллектуальных компьютерных систем} (искусственных интеллектуальных систем) и поддержку развития \textit{Базовой универсальной комплексной технологии проектирования интеллектуальных компьютерных систем}
		            \item семейство персональных ostis-ассистентов студентов, магистрантов и аспирантов, обучающихся в области \textit{Искусственного интеллекта}
		            \item семейство персональных ostis-ассистентов преподавателей, осуществляющих подготовку молодых специалистов в области \textit{Искусственного интеллекта}
		            \item семейство кафедральных корпоративных \textit{ostis-систем}, осуществляющих управление учебным процессом на уровне кафедр, обеспечивающих подготовку молодых специалистов в области Искусственного интеллекта. В рамках таких корпоративных \textit{ostis-систем} осуществляется:
		            \begin{scnitemizeii}
		                \item составление кафедрального расписания занятий на следующий семестр и его согласование с расписанием других кафедр этого же вуза;
		                \item распределение учебной нагрузки на очередной семестр и учебный год;
		                \item мониторинг проведения различного вида занятий (лекций, консультаций, семинаров, практических занятий, зачетов/экзаменов);
		                \item мониторинг самостоятельной деятельности обучаемых (курсовых и дипломных проектов, рефератов, диссертаций, тестов и др.);
		                \item фиксация текущего соответствия между учебными дисциплинами и разделами \textit{Общей теории интеллектуальных систем} и \textit{Базовой универсальной комплексной технологии проектирования интеллектуальных компьютерных систем} (речь идет не только о дисциплинах, непосредственно относящихся к \textit{Искусственному интеллекту}, но и о различных общеобразовательных и смежных дисциплинах, таких, как теория познания, методология, иностранные языки, современные компьютерные системы и сети, компьютеры нового поколения, теория алгоритмов и программ, ориентированных на современные компьютеры, семантическая теория алгоритмов и программ, ориентированных на обработку баз знаний и др.). Принципиально важно сформировать у студентов, магистрантов и аспирантов целостную картину проблематики \textit{Искусственного интеллекта} и место \textit{Искусственного интеллекта} в общей Картине Мира. Барьеров между учебными дисциплинами быть не должно.
		            \end{scnitemizeii}
		            \item Корпоративная \textit{ostis-система} OSTIS-сообщества, являющегося субъектом \textit{Образовательной деятельности в области Искусственного интеллекта}. Через эту корпоративную \textit{ostis-систему} осуществляется взаимодействие между всеми членами указанного \textit{\mbox{ostis-сообщества}} и, прежде всего между кафедрами, осуществляющими подготовку молодых специалистов в области \textit{Искусственного интеллекта}.
		        \end{scnitemize}}
        \end{scnindent}
    \end{scnindent}
    \begin{scnrelfromvector}{принципы, лежащие в основе}
        \scnfileitem{Подготовка молодых специалистов в области \textit{Искусственного интеллекта} должна осуществляться путем поэтапного и непосредственного их подключения к реальным коллективным проектам:\\
            \begin{scnitemize}
                \item к развитию базы знаний по \textit{Общей теории интеллектуальных систем}, хранимой в памяти соответствующего интеллектуального портала знаний
                \item к развитию базы знаний по \textit{Общей теории интеллектуальных компьютерных систем}, хранимой в памяти соответствующего интеллектуального портала знаний (в памяти \textit{Метасистемы IMS.ostis})
                \item к развитию базы знаний по \textit{Базовой комплексной технологии проектирования интеллектуальных компьютерных систем}, хранимой в памяти интеллектуальной компьютерной системы автоматизации проектирования интеллектуальных компьютерных систем (в памяти \textit{Метасистемы IMS.ostis})
                \item к развитию различных методов и средств проектирования различных компонентов \textit{интеллектуальных компьютерных систем}
                \item к развитию различных специализированных технологий проектирования различных классов интеллектуальных компьютерных систем
                \item к разработке различных прикладных интеллектуальных компьютерных систем на основе развиваемой Базовой (универсальной) комплексной технологии проектирования интеллектуальных компьютерных систем.
            \end{scnitemize}}
        \scnfileitem{Каждый студент и магистрант в процессе обучения привлекается к нескольким разным формам деятельности в области \textit{Искусственного интеллекта} и, в частности, обязательно и к разработке приложений, и к развитию технологий. Специалист, занимающийся автоматизацией какой-либо деятельности должен на себе прочувствовать проблемы и трудности этой автоматизируемой деятельности}
        \scnfileitem{Все студенты, магистранты и преподаватели должны активно участвовать в анализе эффективности своей образовательной деятельности и активно способствовать повышению эффективности и повышению уровня автоматизации этой деятельности с помощью развиваемой технологии проектирования и производства интеллектуальных компьютерных систем. Данный принцип можно условно назвать устранением синдрома сапожника без сапог.}
        \scnfileitem{Результаты самостоятельной работы студентов и магистрантов (лабораторных работ, практических занятий, рефератов, курсовых работ и проектов, дипломных работ и проектов, магистерских диссертаций) должны быть востребованы в тех проектах, к которым они подключены и должны быть доведены до уровня внедрения в эти проекты, т. е. должны быть по соответствующей процедуре согласованы и одобрены. При этом приветствуется и соответствующим образом поощряется любая такого рода инициатива студентов и магистрантов. Указанная востребованность (полезность) результатов самостоятельной работы студентов и магистрантов предполагает то, что отчеты по этим результатам оформляются в формализованном виде --- в виде исходных текстов соответствующих фрагментов баз знаний. При этом указанные результаты могут требовать как весьма высокой квалификации, так и не очень высокой (например, квалификации первокурсника). К таким несложным, но весьма полезным работам относятся:\\
            \begin{scnitemize}
                \item введение в \textit{базы знаний} полезных библиографических ссылок и цитат
                \item сравнительный анализ различных положений, представленных в некоторой разрабатываемой базе знаний
                \item различные пояснения, примечания и комментарии, вводимые в \textit{базу знаний}
                \item спецификация выявленных в разрабатываемой базе знаний ошибок, противоречий, информационных дыр и информационного мусора
                \item примеры, иллюстрирующие различные понятия
                \item упражнения к различным разделам разрабатываемых \textit{баз знаний}, которые особенно актуальны для интеллектуальных компьютерных систем, используемых в учебном процессе (это не только интеллектуальные обучающие системы).
            \end{scnitemize}}
        \scnfileitem{Вклад каждого студента и магистранта в развитие всех проектов, в которых он принимает участие, фиксируется и при подведении итогов по каждому семестру соответствующим образом оценивается. Это своего рода предтеча будущего рынка знаний.}
        \scnfileitem{Учебным пособием по каждой учебной дисциплине должна быть база знаний или некоторый раздел базы знаний некоторой интеллектуальной компьютерной системы. Такой может быть либо интеллектуальная обучающая система, либо, например, \textit{Метасистема IMS.ostis}. Условием максимально эффективного проведения лекционного занятия является предварительное прочтение студентами или магистрантами материала предстоящей лекции (соответствующего раздела базы знаний). Тогда на лекции можно акцентировать внимание не на изложение материала, опубликованного в виде базы знаний, а на обсуждение непонятных фрагментов этого материала, на обсуждение проблем, касающихся содержания (принципиальных положений) этого материала. Все это формирует культуру взаимопонимания и согласования различных точек зрения, а также способствует повышению качества базы знаний, представляющей материал соответствующей учебной дисциплины.}
        \scnfileitem{Важнейшей задачей подготовки молодых специалистов является формирование у них:\\
            \begin{scnitemize}
                \item высокой математической культуры (культуры формализации)
                \item высокой системной культуры (понимания того, что количество далеко не всегда переходит в ожидаемое качество)
                \item высокого уровня технологической культуры, технологической дисциплины, четкого соблюдения текущих стандартов и способности участвовать в эволюции стандартов
                \item способности работать в наукоемких проектах в составе творческих коллективов с децентрализованным управлением
                \item способности к достижению семантической совместимости (взаимопонимания) со своими коллегами
                \item договороспособности (способности к согласованию различных точек зрения).
            \end{scnitemize}}
        \scnfileitem{Подготовку молодых специалистов в области \textit{Искусственного интеллекта} можно осуществлять с ориентацией на следующие условно выделенные уровни их квалификации:\\
            \begin{scnitemize}
                \item инженерия прикладных \textit{интеллектуальных компьютерных систем} по заданной технологии
                \item инженерия специализированных технологий проектирования различных классов прикладных интеллектуальных компьютерных систем (на основе базовой универсальной комплексной технологии проектирования интеллектуальных компьютерных систем)
                \item инженерия базовой универсальной комплексной технологии проектирования интеллектуальных компьютерных систем
                \item инженерия программных и аппаратных средст, интерпретации логико-семантических моделей интеллектуальных компьютерных систем
                \item инженерия комплексов интеллектуальных компьютерных систем
                \item научно-исследовательская деятельность по развитию \textit{Общей формальной теории интеллектуальных компьютерных систем}.
            \end{scnitemize}}
    \end{scnrelfromvector}
    \bigskip
    
    \scnheader{Бизнес-деятельность в области Искусственного интеллекта}
    \scnrelfrom{предлагаемый подход}{Бизнес-деятельность в области Искусственного интеллекта, осуществляемая на основе \textit{Технологии OSTIS}}
	\begin{scnindent}
		\scnrelfrom{субъект}{OSTIS-сообщество Бизнес-деятельности в области Искусственного интеллекта, осуществляемой на основе \textit{Технологии OSTIS}}
		\begin{scnindent}
		    \scnidtf{Глобальное (максимальное) OSTIS-сообщество, осуществляющее Бизнес-деятельность в области Искусственного интеллекта}
		    \scnrelto{часть}{Экосистема OSTIS}
		    \scniselement{ostis-сообщество}
		    \scntext{пояснение}{Речь идет об ostis-сообществе, которое включает в себя все компании и лаборатории, работающие в области Искусственного интеллекта и желающие на взаимовыгодных условиях сотрудничать в направлении совместного, перманентного и интенсивного развития стандартов, методов и средств комплексного проектирования и производства семантически совместимых и договороспособных интеллектуальных компьютерных систем, способных самостоятельно и целенаправленно взаимодействовать друг с другом. Кроме указанных компаний и лабораторий в состав рассматриваемого ostis-сообщества входят:
		        \begin{scnitemize}
		            \item семейство корпоративных ostis-систем, которые представляют интересы указанных компаний и лабораторий в рамках рассматриваемого ostis-сообщества и которые обеспечивают автоматизацию внутренней деятельности (бизнес-процессов) этих компаний и лабораторий, включая делопроизводство, юридический мониторинг, бухгалтерскую деятельность, административно-хозяйственную деятельность, управление персоналом, управление выполняемыми проектами и т.д.;
		            \item Корпоративная ostis-система OSTIS-сообщества, являющегося субъектом Бизнес-де\-я\-тель\-ности в области Искусственного интеллекта. Через эту корпоративную ostis-систему осуществляется взаимодействие между членами рассматриваемого ostis-сообщества --- между компаниями и лабораториями, работающими в области искусственного интеллекта.
		        \end{scnitemize}}
        \end{scnindent}
    \end{scnindent}
    
    \scnheader{Консорциум OSTIS}
    \scniselement{ostis-сообщество}
    \scntext{пояснение}{Весь комплекс деятельности в области \textit{Искусственного интеллекта} мы декомпозировали на шесть форм (частей). Для каждой из этих форм деятельности создается свое \textit{ostis-сообщество}, каждому из которых, в свою очередь, соответствует своя \textit{корпоративная ostis-система}. \textit{Консорциум OSTIS} объединяет все указанные \textit{ostis-сообщества}, включая в свой состав (в состав \textit{Консорциума OSTIS}) прежде всего все \textit{корпоративные ostis-системы} указанных \textit{ostis-сообществ}. Кроме того, для координации деятельности членов самого \textit{Консорциума OSTIS} создается \textit{Корпоративная ostis-система Консорциума OSTIS}. Напомним, что для каждого \textit{ostis-сообщества}, создается соответствующая ему \textit{корпоративная ostis-система}, являющаяся ключевым членом этого \textit{ostis-сообщества} и осуществляющая координацию всех остальных его членов.}
    \begin{scnrelfromlist}{член ostis-сообщества}
        \scnitem{Корпоративная система Консорциума OSTIS}
    	\begin{scnindent}
            \scnrelto{корпоративная ostis-система}{Консорциум OSTIS}
        	\begin{scnindent}
            	\scnrelto{субъект}{Деятельность в области Искусственного интеллекта, осуществляемая на основе Технологии OSTIS}
            \end{scnindent}
        \end{scnindent}
        \scnitem{OSTIS-портал научных знаний в области Искусственного интеллекта}
        \begin{scnindent}
            \scnrelto{корпоративная ostis-система}{OSTIS-сообщество научно-исследовательской деятельности в области Искусственного интеллекта}
            \begin{scnindent}
            	\scnrelto{субъект}{научно-исследовательская деятельность в области Искусственного интеллекта, осуществляемая на основе Технологии OSTIS}
            \end{scnindent}
        \end{scnindent}
        \scnitem{Метасистема IMS.ostis}
        \begin{scnindent}
            \scnrelto{корпоративная ostis-система}{OSTIS-сообщество Проекта IMS.ostis}
            \begin{scnindent}
            	\scnrelto{субъект}{Проект IMS.ostis}
        	\end{scnindent}
        \end{scnindent}
        \scnitem{Корпоративная система OSTIS-сообщества Проекта разработки универсального интерпретатора логико-семантических моделей ostis-систем}
        \begin{scnindent}
            \scnrelto{корпоративная ostis-система}{Корпоративная ostis-система OSTIS-сообщество Проекта разработки универсального интерпретатора логико-семантических моделей ostis-систем}
            \begin{scnindent}
            	\scnrelto{субъект}{Проект разработки универсального интерпретатора логико-семантических моделей ostis-систем}
        	\end{scnindent}
        \end{scnindent}
        \scnitem{Корпоративная система OSTIS-сообщества специализированной инженерии в области Искусственного интеллекта, осуществляемой на основе Технологии OSTIS}
        \begin{scnindent}
            \scnrelto{корпоративная ostis-система}{OSTIS-сообщество Специализированной инженерии в области Искусственного интеллекта, осуществляемой на основе Технологии OSTIS}
            \begin{scnindent}
            	\scnrelto{субъект}{Специализированная инженерия в области Искусственного интеллекта, осуществляемая на основе Технологии OSTIS}
            \end{scnindent}
        \end{scnindent}
        \scnitem{Корпоративная система OSTIS-сообщества образовательной деятельности в области Искусственного интеллекта, осуществляемой на основе Технологии OSTIS}
        \begin{scnindent}
            \scnrelto{корпоративная ostis-система}{OSTIS-сообщество Образовательной деятельности в области Искусственного интеллекта, осуществляемой на основе Технологии OSTIS}
            \begin{scnindent}
            	\scnrelto{субъект}{Образовательная деятельность в области Искусственного интеллекта, осуществляемая на основе Технологии OSTIS}
            \end{scnindent}
        \end{scnindent}
        \scnitem{Корпоративная система OSTIS-сообщества Бизнес-деятельности в области Искусственного интеллекта, осуществляемой на основе Технологии OSTIS}
        \begin{scnindent}
            \scnrelto{корпоративная ostis-система}{OSTIS-сообщество Бизнес-деятельности в области Искусственного интеллекта, осуществляемой на основе Технологии OSTIS}
            \begin{scnindent}
            	\scnrelto{субъект}{Бизнес-деятельность в области Искусственного интеллекта, осуществляемая на основе Технологии OSTIS}
            \end{scnindent}
        \end{scnindent}
    \end{scnrelfromlist}
    \scntext{примечание}{Конвергенция и интеграция различных форм и направлений деятельности в области \textit{Искусственного интеллекта} должна проходить через каждого персонального члена \textit{Консорциума OSTIS} --- желательно, чтобы большинство из них были одновременно:
        \begin{scnitemize}
            \item и участниками научно-исследовательской деятельности в области \textit{Искусственного интеллекта} (аспирантами, докторантами и т.д.);
            \item и участниками совершенствования (развития) целостного комплекса методов и средств проектирования и реализации \textit{интеллектуальных компьютерных систем}
            \item и разработчиками различных прикладных \textit{интеллектуальных компьютерных систем}
            \item и преподавателями, участвующими в подготовке молодых специалистов в области \textit{Искусственного интеллекта}.
        \end{scnitemize}}
    \scnidtf{OSTIS-сообщество субъектов всех форм и направлений деятельности в области Искусственного интеллекта}
    \scnrelto{часть}{Экосистема OSTIS}
    \scnidtf{Научно-техническое и учебное объединение специалистов и организаций, работающих в области Искусственного интеллекта}
    \scntext{перспективы}{Создание Консорциума OSTIS на основе широкого применения Технологии OSTIS может и должно осуществляться с поэтапным расширением состава участников и поэтапным повышением уровня автоматизации деятельности Консорциума OSTIS. Ключевыми направлениями деятельности Консорциума OSTIS являются:
        \begin{scnitemize}
            \item Существенное повышение темпов эволюции Ядра Технологии OSTIS, темпов перехода на все более совершенные версии стандартов интеллектуальных компьютерных систем, проектных библиотек и средств автоматизации проектирования интеллектуальных компьютерных систем;
            \item Разработка компьютеров нового поколения, ориентированных на интерпретацию логико-семантических моделей интеллектуальных компьютерных систем;
            \item Разработка иерархического семейства семантически совместимых специализированных технологий проектирования различных классов интеллектуальных компьютерных систем;
            \item Создание условий для развития технологий искусственного интеллекта в направлении унификации интеллектуальных компьютерных систем для обеспечения их конвергенции и семантической совместимости.
        \end{scnitemize}}
    \bigskip
\end{scnsubstruct}

        \scnsegmentheader{Уточнение Понятия Экосистемы OSTIS}
\begin{scnsubstruct}
    \scnidtf{Использование \textit{Технологии OSTIS} для повышения качества и, в частности, уровня автоматизации всех \textit{областей человеческой деятельности}}
    \scnidtf{Понятие \textit{Экосистемы OSTIS} как формы реализации \textit{smart-общества}, представляющего собой сеть взаимодействующих людей, интеллектуальных компьютерных систем, умных домов, умных предприятий, умных больниц, умных учебных заведений, умных городов, умных транспортных систем и т.п.}
    \begin{scnrelfromset}{рассматриваемые вопросы}
        \scnfileitem{Какова архитектура \textit{Экосистемы OSTIS}}
        \scnfileitem{Какова архитектура \textit{ostis-сообщества}, входящего в состав \textit{Экосистемы OSTIS}}
        \scnfileitem{Как взаимодействуют между собой различные \textit{ostis-сообщества} в рамках \textit{Экосистемы OSTIS}}
        \scnfileitem{Как интегрируется \textit{деятельность} различных \textit{ostis-сообществ} и результаты этой \textit{деятельности}}
        \scnfileitem{Какова типология \textit{ostis-сообществ} и по каким признакам классификации можно эту типологию проводить}
        \scnfileitem{Можно ли опыт автоматизации деятельности в области \textit{Искусственного интеллекта} с помощью \textit{Технологии OSTIS} расширить на все многообразие областей и видов человеческой деятельности}
        \scnfileitem{Как выглядит систематизация областей и видов человеческой деятельности}
        \scnfileitem{Как осуществляется конвергенция и интеграция различных областей и видов человеческой деятельности}
        \scnfileitem{Как взаимодействуют \textit{ostis-системы}, осуществляющие автоматизацию различных областей видов человеческой деятельности}
        \scnfileitem{Как может выглядеть \uline{комплексная} автоматизация всех областей и видов \textit{человеческой деятельности} с помощью \textit{Технологии OSTIS}}
    \end{scnrelfromset}
    \begin{scnrelfromvector}{план изложения}
        \scnfileitem{Что такое \textit{Экосистема OSTIS}}
        \scnfileitem{Структура \textit{Экосистемы OSTIS}}
        \scnfileitem{Что такое \textit{ostis-система}, являющаяся агентом \textit{Экосистемы OSTIS}}
        \scnfileitem{Что такое \textit{ostis-сообщество} с точки зрения \textit{Экосистемы OSTIS}}
        \scnfileitem{Что такое Проект создания \textit{Экосистемы OSTIS}}
        \scnfileitem{Цель создания и основные свойства \textit{Экосистемы OSTIS}}
        \scnfileitem{Как структурируется \textit{человеческая деятельность}}
        \scnfileitem{Как выглядит \textit{рынок знаний}, реализуемый в рамках \textit{Экосистемы OSTIS}}
        \scnfileitem{Чем определяется качество \textit{человеческой деятельности}}
        \scnfileitem{Что такое эффективная автоматизация \textit{человеческой деятельности}}
        \scnfileitem{Почему повышение эффективности \textit{человеческой деятельности} невозможно без \textit{интеллектуальных компьютерных систем}}
        \scnfileitem{Какие достоинства имеет \textit{Экосистема OSTIS}}
    \end{scnrelfromvector}

    \scnheader{Экосистема OSTIS}
    \scntext{вопрос}{Какова структура Экосистемы OSTIS}
    \scntext{пояснение}{Популяция
        \begin{itemize}
            \item семантически совместимых
            \item эволюционируемых
            \item активно взаимодействующих
            \item способных координировать (согласовывать) свою деятельность с другими субъектами
        \end{itemize}
        интеллектуальных компьютерных систем (\textit{ostis-систем}). При этом указанная популяция \textit{ostis-систем} поддерживает децентрализованное управление собственной деятельностью, а также деятельностью людей (пользователей \textit{ostis-систем}) и человеко-машинных сообществ (\textit{ostis-сообществ}), обеспечивая тем самым автоматизацию системной интеграции любых новых субъектов (\textit{ostis-систем}, людей, \textit{ostis-сообществ}) в состав \textit{Экосистемы OSTIS}.}
    \begin{scnrelfromvector}{принципы, лежащие в основе}
        \scnfileitem{\textit{Экосистема OSTIS} представляет собой сеть \textit{ostis-сообществ}}
        \scnfileitem{Каждому \textit{ostis-сообществу} взаимно однозначно соответствует \textit{корпоративная ostis-система} этого \textit{ostis-сообщества}, которая:
            \begin{itemize}
                \item обеспечивает координацию деятельности членов соответствующего \textit{ostis-сообщества}
                \item является представителем этого \textit{ostis-сообщества} в других \textit{ostis-сообществах}, членом которых указанное \textit{ostis-сообщество} является.
            \end{itemize}}
        \scnfileitem{Каждое \textit{ostis-сообщество} может входить в состав любого другого \textit{ostis-сообщества} по своей инициативе. Формально это означает, что \textit{корпоративная ostis-система} первого \textit{ostis-сообщества} является членом другого \textit{ostis-сообщества}.}
        \scnfileitem{Каждому специалисту, входящему в состав \textit{Экосистемы OSTIS} ставится во взаимнооднозначное соответствие его \textit{персональный ostis-ассистент}, который трактуется как \textit{корпоративная \mbox{ostis-система}} вырожденного \textit{ostis-сообщества}, состоящего из одного человека.}
    \end{scnrelfromvector}
    
    \scnheader{следует отличать*}
    \begin{scnhaselementset}
        \scnitem{корпоративная ostis-система*}
        \begin{scnindent}
        	\scnidtf{корпоративная ostis-система данного ostis-сообщества*}
        \end{scnindent}
        \scnitem{корпоративная ostis-система}
        \begin{scnindent}
            \scnrelto{второй домен}{корпоративная ostis-система*}
        \end{scnindent}
        \scnitem{член ostis-сообщества*}
        \scnitem{персональный ostis-ассистент*}
        \begin{scnindent}
        	\scnidtf{персональный ostis-ассистент данного специалиста*}
            \scnsubset{корпоративная ostis-система*}
        \end{scnindent}
        \scnitem{персональный ostis-ассистент}
        \begin{scnindent}
            \scnrelto{второй домен}{персональный ostis-ассистент*}
            \scnsubset{корпоративная ostis-система}
        \end{scnindent}
    \end{scnhaselementset}

    \scnheader{есть сходства*}
    \begin{scnhaselementset}
        \scnitem{Экосистема OSTIS}
        \scnitem{ostis-сообщество}
        \begin{scnindent}
            \scnhaselement{Экосистема OSTIS}
        \end{scnindent}
    \end{scnhaselementset}
	\begin{scnindent}
    	\scntext{пояснение}{\textit{Экосистема OSTIS} является максимальным \textit{ostis-сообществом}, включающим в себя все существующие \textit{ostis-сообщества}}
	\end{scnindent}
	
    \scnheader{Экосистема OSTIS}
    \scnidtf{Максимальное \textit{иерархическое ostis-сообщество}, обеспечивающее комплексную автоматизацию \uline{всех} видов \textit{человеческой деятельности}}
    \scnidtf{Глобальное \textit{ostis-сообщество}, которое не может входить в состав какого-либо другого \textit{ostis-сообщества}}
    \scniselement{иерархическое ostis-сообщество}
    \begin{scnindent}
	    \scnidtf{такое \textit{ostis-сообщество}, по крайней мере одним из членов которого является некоторое другое \textit{ostis-сообщество}}
	    \scnsubset{ostis-сообщество}
	    \begin{scnindent}
	    	\scnrelboth{следует отличать}{коллектив ostis-систем}
	    \end{scnindent}
    \end{scnindent}
    \scntext{пояснение}{Понятие \textit{ostis-сообщества} представляет собой не только \textit{коллектив ostis-систем}, но также определенный \textit{коллектив людей} (пользователей и разработчиков соответствующих \textit{ostis-систем})}
    \scnsuperset{ostis-система автоматизации проектирования}
    \scnsuperset{ostis-система автоматизации производства}
    \begin{scnindent}
    	\scnidtf{ostis-система управления производством}
    \end{scnindent}
    \scnsuperset{ostis-система автоматизации образовательной деятельности}
    \scnsuperset{обучающаяся ostis-система}
    \scnsuperset{корпоративная ostis-система виртуальной кафедры}
    \begin{scnindent}
    	\scnidtf{корпоративная ostis-система, обеспечивающая интеграцию деятельности кафедр одинакового профиля и, возможно, различных вузов}
    \end{scnindent}
    \scnsuperset{ostis-система автоматизации бизнес-деятельности}
    \scnsuperset{ostis-система автоматизации управления}
    \scnsuperset{ostis-система управления проектами соответствующего вида}
    \scnsuperset{ostis-система сенсомоторной координации при выполнении определенного вида сложных действий во внешней среде}
    \scnsuperset{ostis-система управления самостоятельным перемещением робота по пересеченной местности}
    
    \scnheader{обучающая ostis-система}
    \scntext{примечание}{Поскольку качество эксплуатации каждой \textit{ostis-системы} зависит не только от нее, но и от квалификации пользователя (семантическая совместимость, знания о возможностях системы), каждая \textit{ostis-система} должна быть способна обучать пользователя знаниям и навыкам эффективного её использования.}

    \scnheader{ostis-сообщество}
    \scnidtf{устойчивый фрагмент \textit{Экосистемы OSTIS}, обеспечивающий комплексную автоматизацию определенной части коллективной человеческой деятельности и перманентное повышение ее эффективности (в т.ч. уровня автоматизации)}
    \scntext{примечание}{Наряду с \textit{Экосистемой OSTIS}, идея \textit{ostis-сообщества} представляет собой естественный этап перехода (эволюции) творчески ориентированных коллективов людей в принципиально новое существенно более интеллектуальное и, соответственно, более позитивное качество}
    \begin{scnreltolist}{перманентно-решаемая задача}
        \scnitem{перманентная поддержка семантической совместимости членов ostis-сообщества}
        \scnitem{перманентная поддержка высокого качества базы знаний, доступной всем членам ostis-сообщества}
        \begin{scnindent}
            \scntext{пояснение}{Имеется в виду поддержка непротиворечивости (корректности отсутствия синонимов, омонимов и противоречий), полноты (отсутствия информационных дыр) и чистоты (отсутствия информационного мусора)}
        \end{scnindent}
        \scnitem{перманентная поддержка мониторинга эффективности распределения работ между членами ostis-сообщества и контроля исполнительской дисциплины}
        \scnitem{перманентная поддержка мониторинга динамики роста квалификации каждого члена ostis-сообщества}
    \end{scnreltolist}
    
    \scnheader{коллектив людей}
    \scnidtf{человеческое общество}
    \scntext{примечание}{Низкий уровень интеллекта современных коллективов людей определяется (1) низким уровнем качества организации общей памяти каждого такого коллектива (общей памяти всех его членов) (2) низкий уровень качества знаний, хранимых в этой памяти (как минимум это наличие большого количества синонимов, омонимов и противоречий). Поэтому переход от современных коллективов людей к соответствующим ostis-сообществам существенно повышает уровень интеллекта этих коллективов}

    \scnheader{Экосистема OSTIS}
    \scniselement{ostis-сообщество}
    \scnrelto{субъект}{Объединенная человеческая деятельность, осуществляемая на основе Технологии OSTIS}
    \scnrelfrom{корпоративная ostis-система}{Корпоративная система Экосистемы OSTIS}
    \begin{scnindent}
    	\scntext{пояснение}{Основным назначением \textit{Корпоративной системы Экосистемы OSTIS} является организация общего взаимодействия при выполнении самых различных видов и \textit{областей человеческой деятельности}, которые могут быть либо полностью автоматизированными, либо частично автоматизированными, либо вообще неавтоматизированными. Из этого следует, что база знаний \textit{Корпоративной системы Экосистемы OSTIS} должна содержать \textit{Общую формальную теорию человеческой деятельности}, включающей в себя типологию видов и областей \textit{человеческой деятельности}, а также общую \textit{методологию} этой \textit{деятельности}.}
    \end{scnindent}
    	
    \scnheader{Деятельность в области Искусственного интеллекта, осуществляемая на основе Технологии OSTIS}
    \scnrelfrom{субъект}{Консорциум OSTIS}
    \begin{scnindent}
    	\scnrelfrom{корпоративная ostis-система}{Корпоративная ostis-система Консорциума OSTIS}
    \end{scnindent}
    \scnrelfrom{основной продукт}{Экосистема OSTIS}
    \begin{scnindent}
	    \scnrelfrom{частное ostis-сообщество}{Консорциум OSTIS}
	    \scnrelfrom{член ostis-сообщества}{Консорциум OSTIS}
	    \begin{scnindent}
            \begin{scnrelfromset}{примечание}
                \scnitem{член ostis-сообщества*}
                \begin{scnindent}
                    \scnsubset{частное ostis-сообщество*}
                    \begin{scnindent}
                        \scnidtf{\textit{ostis-сообщество}, входящее в состав заданного \textit{ostis-сообщества}}
                    \end{scnindent}
                \end{scnindent}
            \end{scnrelfromset}
	    \end{scnindent}
	\end{scnindent}
    \scnidtf{Проект, основной целью (продуктом) которого является создание \textit{Экосистемы OSTIS}}
    \scnrelfrom{часто используемый sc-идентификатор}{Проект Экосистемы OSTIS}
    \scnidtf{Деятельность, направленная на создание и перманентное развитие \textit{Экосистемы OSTIS}}
    \scnidtf{Проект, направленный на проектирование, производство и реинжиниринг \textit{ostis-систем}, входящих в сеть \textit{Экосистемы OSTIS}, а также на проектирование и реинжиниринг Экосистемы OSTIS в целом (как сети \textit{ostis-систем} и их пользователей)}
    \begin{scnrelfromlist}{подпроект}
        \scnitem{Проект Метасистемы OSTIS}
        \scnitem{Проект программной реализации абстрактной sc-машины}
        \scnitem{Проект разработки универсального sc-компьютера}
    \end{scnrelfromlist}
    \scntext{примечание}{В состав \textit{Проекта Экосистемы OSTIS} входит большое количество \textit{проектов} (\textit{подпроектов}*), направленных на \textit{проектирование} и \textit{производство ostis-систем} самого различного назначения}
    \scntext{пояснение}{Распространим предлагаемый нами подход к повышению эффективной и человеческой \textit{Деятельности в области Искусственного интеллекта} на всю \textit{Объединенную человеческую деятельность} в целом, т.е. рассмотрим структуру Глобального \textit{ostis-сообщества} (\textit{Экосистемы OSTIS})}
    \scntext{эпиграф}{От \textit{Консорциума OSTIS} к \textit{Экосистеме OSTIS}}
    
    \scnheader{Экосистема OSTIS}
    \scntext{примечание}{Подчеркнем, что \textit{Экосистема OSTIS} является:
        \begin{itemize}
            \item с одной стороны, \textit{основным продуктом*} человеческой \textit{Деятельности в области Искусственного интеллекта, осуществляемой на основе Технологии OSTIS} (эту Деятельность мы также будем называть Проектом Экосистемы OSTIS),
            \item а, с другой стороны, \textit{субъектом* Объединенной человеческой деятельности, осуществляемой на основе Технологии OSTIS}.
        \end{itemize}
        Особо подчеркнем то, что продуктом человеческой \textit{Деятельности в области Искусственного интеллекта, осуществляемой на основе Технологии OSTIS}, является не просто множество \textit{ostis-систем} различного назначения, а Экосистема, состоящая из \underline{взаимодействующих} \textit{ostis-систем} и их пользователей}
    \scntext{пояснение}{Принципиальным является то, что продуктом (результатом применения) \textit{Технологии OSTIS} является не просто множество \textit{ostis-систем}, а целая система, состоящая из \textit{ostis-систем} и их пользователей, взаимодействующих между собой и осуществляющих комплексную автоматизацию всех \textit{видов человеческой деятельности}, а также комплексное повышение уровня эффективности организации человеческой деятельности (и, в частности, повышение уровня автоматизации этой деятельности).}

    \scnheader{Экосистема OSTIS}
    \begin{scnrelfromlist}{\scnkeyword{вопрос}}
        \scnitem{Каковы основные свойства Экосистемы OSTIS}
        \scnitem{Какова основная цель создания Экосистемы OSTIS}
    \end{scnrelfromlist}
    \scnidtf{Экосистема ostis-систем и их пользователей}
    \scnrelto{общий создаваемый продукт}{Технология OSTIS}
    \scnidtf{Расширяемый коллектив эволюционируемых, семантически совместимых и взаимодействующих ostis-систем и их пользователей}
    \scniselement{многоагентная система}
    \scnidtf{Многоагентная система, агентами которой являются ostis-системы, а также их конечные пользователи и разработчики}
    
    \scnheader{Экосистема интеллектуальных компьютерных систем}
    \scnidtf{Smart-сообщество}
    \scnidtf{Smart-сообщество интеллектуальных компьютерных систем и людей}
    \scnidtf{Интеллектуальная многоагентная система, состоящая из интеллектуальных компьтерных систем и людей}
    \scntext{примечание}{Многоагентная система может состоять из кибернетических систем, не являющихся интеллектуальными.}
    \scntext{примечание}{Многоагентная система может состоять из интеллектуальных систем, но сама не быть интеллектуальной. Количество далеко не всегда переходит в нужное качество.}
    \scnidtf{Экосистема ostis-систем, а также их разработчиков и пользователей}
    \scnidtf{Эволюционирумая сеть ostis-систем, обеспечивающая конвергенцию и интеграцию всех видов человеческой деятельности}
    
    \scnheader{Экосистема OSTIS}
    \scnidtf{Глобальная компьютерная сеть ostis-систем, обеспечивающая комплексную автоматизацию всевозможных видов и областей человеческой деятельности и отражающая иерархию уровней этой деятельности}
    \scnidtf{Глобальная \textit{многоагентная система}, состоящая из людей и семантически совместимых \textit{интеллектуальных компьютерных систем}, построенных по \textit{Технологии OSTIS}, которые, взаимодействуя между собой и с людьми, обеспечивают существенное повышение уровня автоматизации всех видов \textit{человеческой деятельности} и существенное повышение эффективности человеческого взаимодействия}
    \scnidtf{Предлагаемых нами подход к реализации smart-общества}
    \scnidtf{Smart-общество, построенное на основе Технологии OSTIS}
    \scnidtf{следующий этап развития человеческого общества, обеспечивающий существенное повышение уровня общественного (коллективного) интеллекта путем преобразования человеческого общества в экосистему, состоящую из людей и семантически совместимых интеллектуальных систем}
    \scntext{обоснование}{Необходимо обеспечить не только повышение уровня автоматизации человеческой деятельности (как информационной (умственной), так и физической), но и существенное повышение уровня интеллекта человеческого общества как социальной кибернетической системы путем создания многоагентной кибернетической системы, сосотоящей из \textit{интеллектуальных компьютерных систем} и людей и имеющей \textit{высокий уровень интеллекта}. Человечества пока не умеет создавать инетеллектуальные сообщества (коллективы) людей и, тем более, интеллектуальные общества людей и интеллектуальных компьютерных системы.Уровень интеллекта каждого такого сообщества обычно определяется уровнем интеллекта его руководителя (лица, принимающего решение).А надо, чтобы уровень интеллекта сообщества был результатом интеграции интеллектуального потенциала всех его членов. При этом следует помнить, что интеллект определяется не только и не столько множеством решаемых задач, а \uline{скоростью} расширения этого множества.}
    \scntext{примечание}{Предметом инженерной деятельности в области \textit{искусственного интеллекта} следует считать не множество \textit{интеллектуальных компьютерных систем} (например, \textit{ostis-систем}), а весь комплекс взаимодействующих между собой \textit{интеллектуальных компьютерных систем}. Назовём такой комплекс \textit{Экосистемой интеллектуальных компьютерных систем} (в нашем случае это Экосистема OSTIS Экосистема взаимодействующих \textit{ostis-систем}) здесь важно построить архитектуру таковой экосистемы, в основе которой должна лежать комплексная формальная модель всевозможных видов человеческой деятельности, автоматизируемых с помощью интеллектуальных компьютерных систем (ostis-систем). Указання комплексная модель человеческой деятельности является необходимой основой создания smart-общества (общества 5.0)}
    \scnrelto{общий создаваемый продукт}{Технология OSTIS}
    \scnidtf{Общий (объединенный, интегрированный) продукт использования \textit{Технологии OSTIS}, представляющий собой глобальную сеть \textit{ostis-систем}, обеспечивающий комплексную автоматизацию и интеграцию всевозможных \textit{видов человеческой деятельности} и, в частности, включающий в себя (в виде соответствующего \mbox{\textit{ostis-сообщества}}) \textit{консорциум OSTIS}, т.е. инфраструктуру, направленную на перманентное развитие \textit{Технологии OSTIS} (как Ядра Технологии OSTIS, так и иерархического семейства \textit{специализированных ostis-технологий})}
    \scntext{следовательно}{\textit{Экосистема OSTIS} представляет собой саморазвивающуюся сеть ostis-систем}
    \scntext{пояснение}{Сверхзадачей \textit{Экосистемы OSTIS} является не просто комплексная автоматизация всех \textit{видов человеческой деятельности} (разумеется, только тех видов деятельности, автоматизация которых целесообразна), но и существенное повышение уровня интеллекта различных человеческих (точнее человеко-машинных) сообществ и всего человеческого общества в целом. Это потребует соблюдения ряда требований, предъявляемых не только к \textit{интеллектуальным компьютерным системам}, но и к людям, входящим в состав \textit{Экосистемы OSTIS}}
    \scntext{пояснение}{\textit{Экосистема OSTIS} представляет собой открытый коллектив взаимодействующих интеллектуальных систем, состав которого входят \textit{ostis-системы} и их пользователи (конечные пользователи и разработчики,участвующие в совершенствование этих \textit{ostis-систем}). Особое место среди \textit{ostis-систем}, входящих в состав \textit{экосистемы OSTIS}, занимают \textit{корпоративные ostis-системы}, через которое осуществляется координация и эволюция деятельности некоторых групп \textit{ostis-систем} и их пользователей. Основная цель корпоративных \textit{ostis-систем} --- локализовать базы знаний указанных групп ostis-систем, перевести их из статуса виртуальных в статус реальных и автоматизировать их эволюцию.}
    \scnidtf{Сообщество ostis-систем или людей, обеспечивающее принципиально новый уровень автоматизации человеческой деятельности и принципиально \uline{новый уровень интеллекта человеческого общества}}
    
    \scnheader{Экосистема OSTIS}
    \scntext{примечание}{Очень важно проектировать не только саму \textit{Экосистему OSTIS}, ну и процесс \uline{поэтапного перехода} от современной глобальной сети \textit{компьютерных систем} к глобальной сети \textit{ostis-систем} (т.е. к \textit{Экосистеме OSTIS}). В рамках такого переходного периода \textit{ostis-системы} могут выполнять роль системных интеграторов различных ресурсов и сервисов, реализованных современными \textit{компьютерными системами}, поскольку уровень интеллекта \textit{ostis-систем} позволяет им с любой степенью детализации специфицировать интегрируемые \textit{компьютерные системы} и, следовательно, достаточно адекватно понимать, что знает и/или умеет каждая из них, а также достаточно качественно координировать их деятельность и обеспечивать релевантный поиск нужного ресурса и сервиса. Кроме того системы могут выполнять роль интеллектуальных help-систем --- помощников и консультантов по вопросам эффективной эксплуатации различных \textit{компьютерных систем} со сложными функциональными возможностями, имеющими пользовательский интерфейс с нетривиальной семантикой и использующимися в сложных предметных областях. Такие интеллектуальные help-системы можно сделать интеллектуальными посредниками между соответствующими компьютерными системами их пользователями. При этом пользователь может работать одновременно и с help-системой и с соответствующей эксплуатируемой компьютерной системой, консультируясь с help-системой в затруднительных для него ситуациях. Основными недостатками такого варианта является то,что: (1) пользователь должен использовать два разных интерфейса и (2) help-система не может мониторить деятельность пользователя и, следовательно, пользователь сам должен сообщать системе о своих затруднительных ситуациях. Указанные недостатки можно устранить, если компьютерную систему, которая построена по современным технологиям и эксплуатация которой нуждается в качественной консультационной (help-овой) поддержке, интегрировать с соответствующей ей help-системой, построенной по стандартам технологии OSTIS, так, чтобы пользовательским интерфейсом такой интегрированной системы стал пользовательский интерфейс, соответствующий стандартам технологии OSTIS и важнейшим достоинством которого является чёткая формализация семантики всех элементов управления пользовательским интерфейсом. Благодаря этому взаимодействие пользователя с пользовательским интерфейсом ostis-систем становится, во-первых, осмысленным и, во-вторых, позволяющим легко переносить опыт интерфейсного взаимодействия с одной ostis-системы на другую ostis-систему.}

    \scnheader{Объединенная человеческая деятельность}
    \scnidtf{максимальная область человеческой деятельности}
    \scnidtf{вся человеческая деятельность}
    \scnidtf{человеческая деятельность в целом}
    \scnidtf{объединение всевозможных областей человеческой деятельности}
    \scniselement{человеческая деятельность}
    \begin{scnindent}
	    \scnidtf{область человеческой деятельности}
	    \scnidtf{система целенаправленных действий некоторого количества(возможно одного) людей над некоторыми объектами с помощью некоторых инструментов}
	    \scnsubset{деятельность}
	    \begin{scnindent}
		    \scnidtf{трудно выполнимое сложное действие}
		    \scnidtf{область деятельности}
		    \scnidtf{система целенаправленных действий некоторых (возможно одного) субъектов над некоторыми объектами с помощью некоторых инструментов}
	    \end{scnindent}
    \end{scnindent}
    
    \scnheader{следует отличать*}
    \begin{scnhaselementset}
        \scnitem{Объединенная человеческая деятельность}
        \begin{scnindent}
            \scnidtf{человеческая деятельность в целом}
            \scnidtf{максимальная область человеческой деятельности}
        \end{scnindent}
        \scnitem{область человеческой деятельности}
        \begin{scnindent}
            \scnidtf{фрагмент (часть, раздел) человеческой деятельности}
            \scnidtf{человеческая деятельность}
            \scnidtf{деятельность, осуществляемая либо одним человеком (индивидуальная человеческая деятельность), либо коллективом людей}
            \scnsubset{деятельность}
            \begin{scnindent}
            	\scnsubset{действие}
            \end{scnindent}
            \scnsuperset{индивидуальная человеческая деятельность}
            \scnsuperset{коллективная человеческая деятельность}
            \scnidtf{множество всевозможных областей человеческой деятельности}
        \end{scnindent}
        \scnitem{вид человеческой деятельности}
        \begin{scnindent}
            \scnidtf{класс однотипных областей человеческой деятельности, которому можно поставить в соответствие некоторую технологию}
            \scnsubset{вид деятельности}
            \begin{scnindent}
            	\scnsubset{класс действий}
            \end{scnindent}
        \end{scnindent}
    \end{scnhaselementset}
    
    \scnheader{следует отличать*}
    \begin{scnhaselementset}
        \scnitem{Объединенная человеческая деятельность}
        \begin{scnindent}
            \scnidtf{максимальный процесс человеческой деятельности, включающий в себя деятельность всех людей и всех сообществ}
        \end{scnindent}
        \scnitem{человеческая деятельность}
        \begin{scnindent}
            \scnidtf{множество всевозможных целостных, целенаправленных фрагментов \textit{Объединенной человеческой деятельности}}
            \scntext{примечание}{На данном множестве заданы такие отношения, как \textit{часть*}, \textit{декомпозиция*}. Т.е. конкретный экземпляр (элемент) данного множества может быть \textit{частью*} (входить в состав) другой конкретной человеческой деятельности. Более того, целесообразно рассматривать достаточно сложную иерархию процессов человеческой деятельности.}
            \scnidtf{конкретный процесс человеческой деятельности}
            \scnidtf{бизнес-процесс}
            \scnidtf{деятельность, основными субъектами которой являются люди и различные сообщества людей}
            \scnsubset{деятельность}
            \begin{scnindent}
            	\scnsubset{действие}
            \end{scnindent}
            \scntext{примечание}{Если для автоматизации человеческой деятельности используются интеллектуальные компьютерные системы, то эти системы также становятся достаточно самостоятельными полноценными субъектами этой деятельности, мнение которых обязательно принимается во внимание, но при этом интеллектуальные компьютерные системы не становятся основными субъектами человеческой деятельности.}
            \scnhaselement{объединенная человеческая деятельность}
            \begin{scnindent}
            	\scnidtf{максимальная человеческая деятельность, для которой не существует никакой другой конкретной человеческой деятельности, частью* которой указанная Максимальная человеческая деятельность является.}
            \end{scnindent}
            \scntext{примечание}{каждая конкретная человеческая деятельность (каждый бизнес-процесс) может быть:
                \begin{itemize}
                    \item либо полностью автоматизирована от человека требуется только корректно сформулировать соответствующую команду (цель инициируемого действия)
                    \item либо автоматизирована, но требующая от человека управления функционированием соответствующего одного инструментального средства
                    \item либо состоящая из фрагментов (подпроцессов, частных бизнес-процессов), некоторые из которых автоматизированы, а некоторые нет
                    \item либо полностью неавтоматизирована (т.е. выполняется вручную)
                \end{itemize}}
            \scntext{примечание}{Когда речь идет о спецификации конкретной человеческой деятельности (конкретного бизнес-процесса), важно провести четкую грань между теми действиями, которые выполняются автоматически (в том числе интеллектуальными компьютерными системами), и действиям, которые выполняются людьми вручную --- это как минимум действия по формулировке команд, которые адресуются соответствующим инструментальным средствам (язык и, соответственно, интерфейс формулировки таких команд для разных инструментальных средств может сильно отличаться).Отсутствие унификации языка взаимодействия (интерфейсы) между людьми и различными инструментальными средствами (автомобилями, станками, холодильниками, газовыми плитами, микроволновками, компьютерными системами различного назначения) существенно снижает комплексную эффективность автоматизации человеческой деятельности, т.к. вынуждает людей тратить много времени на усвоение не сути (смысла) автоматизации, а формы (синтаксиса) своей деятельности по организации использования различных средств автоматизации.}
        \end{scnindent}
        \scnitem{вид человеческой деятельности}
        \begin{scnindent}
            \scnidtf{класс (множество однотипных) процессов человеческой деятельности}
            \scnidtf{класс бизнес-процессов}
            \scnidtf{множество всевозможных классов бизнес-процессов}
            \scnsubset{вид деятельности}
            \scntext{примечание}{Каждый конкретный вид человеческой деятельности (т. е. каждый элемент множества \textit{вид человеческой деятельности}) является \textit{подмножеством*} множества человеческая деятельность.Каждому виду человеческой деятельности соответствует своя \textit{технология человеческой деятельности}, т.е. свой набор \textit{методов} и \textit{средств}, обеспечивающих выполнение каждой конкретной деятельности, принадлежащей этому виду.}
        \end{scnindent}
        \scnitem{область человеческой деятельности}
        \begin{scnindent}
            \scnsubset{человеческая деятельность}
            \scnidtf{достаточно крупный фрагмент человеческой деятельности}
            \scnidtf{раздел человеческой деятельности}
        \end{scnindent}
    \end{scnhaselementset}
    
    \scnheader{Экосистема OSTIS}
    \scntext{примечание}{Содержательную типологию \textit{ostis-систем}, входящих в состав \textit{Экосистемы OSTIS} следует проводить на основе глубокого анализа содержательной структуры человеческой деятельности, требующей взаимодействия человека с другими людьми и даже с организациями. Очевидно, что эффективность такого взаимодействия во многом определяется качеством организации информационного взаимодействия, уровнем взаимопонимания, уровнем квалификации участников, оперативностью получения качественной консультативной помощи по любому (!) вопросу.}
    
    \scnheader{вид человеческой деятельности, продуктом которой является информационная модель некоторого объекта или класса объектов}
    \scnidtf{вид человеческой деятельности, направленной на построение описания (спецификации) некоторого объекта исследования или класса таких объектов}
    \scnsubset{вид человеческой деятельности}
    \scnhaselement{научно-исследовательская деятельность}
    \begin{scnindent}
	    \scnhaselement{Научно-исследовательская деятельность в области Искусственного интеллекта}
	    \begin{scnindent}
	    	\scnidtf{разработка Общей теории интеллектуальных систем}
	    \end{scnindent}
	\end{scnindent}
    \scnhaselement{разработка теории искусственных объектов заданного класса}
    \begin{scnindent}
	    \scnhaselement{Разработка Общей теории интеллектуальных компьютерных систем}
		\begin{scnindent}
		    \scnidtf{разработка стандарта интеллектуальных компьютерных систем}
		\end{scnindent}
   	\end{scnindent}
    \scnhaselement{разработка теории проектирования искусственных объектов заданного класса}
    \begin{scnindent}
	    \scnidtf{разработка системы проектных действий для искусственных объектов (артефактов) заданного класса}
	    \scnhaselement{разработка теории проектирования интеллектуальных компьютерных систем}
	    \begin{scnindent}
	    	\scnidtf{разработка стандарта организации коллективных проектных действий для проектирования интеллектуальных компьютерных систем}
	    \end{scnindent}
    \end{scnindent}
    \scnhaselement{разработка теории производства спроектированных искусственных объектов заданного класса}
    \begin{scnindent}
	    \scnidtf{разработка стандарта системы производственных действий, методов и инструментов, обеспечивающих производство спроектированных артефактов заданного класса}
	    \scnhaselement{Разработка Теории производства спроектированных интеллектуальных компьютерных систем}
    \end{scnindent}
    \scnhaselement{проектирование искусственного объекта заданного класса}
    \begin{scnindent}
	    \scnidtf{проектная деятельность, направленная на построение такой информационной модели (спецификации) искусственно создаваемого объекта (артефакта) заданного класса, которой достаточно для производства этого объекта}
	    \scnsuperset{проектирование конкретной интеллектуальной компьютерной системы}
	    \begin{scnindent}
	    	\scnidtf{процесс проектирования некоторой компьютерной системы по заданной технологии проектирования}
	    \end{scnindent}
    \end{scnindent}
    \scntext{пояснение}{Данный вид человеческой деятельности характерен следующими особенностями:
        \begin{itemize}
            \item очень часто продукт этой деятельности (создаваемая информационная конструкция) имеет высокую степень сложности и, следовательно, указанная деятельность не может быть индивидуальной, а несет коллективный характер;
            \item основными факторами качественного коллективного построения сложной информационной конструкции являются семантическая совместимость (взаимопонимание) авторов, а также согласованность их действий;
            \item важнейшим направлением автоматизации коллективной деятельности, объект и продукт которой представляет собой сложную информационную конструкцию, является автоматизация редактирования коллективно создаваемого информационного объекта, а также автоматизация обеспечения семантической совместимости и согласованности продуктов индивидуальной деятельности всех соавторов;
            \item указанную автоматизацию легко реализовать с помощью корпоративной интеллектуальной компьютерной системы, объединяющей всех соавторов создаваемого информационного объекта и снабженной мощными средствами поддержки коллективного проектирования различных разделов базы знаний этой системы. Примерами таких систем являются интеллектуальные порталы различного вида знаний.
        \end{itemize}}
    
    \scnheader{вид человеческой деятельности, продуктом которой является информационная модель некоторого объекта или класса объектов}
    \scntext{пояснение}{Здесь речь идет о коллективной человеческой деятельности, которая принципиально не может быть полностью автоматизирована (исследовательская, проектная), то основной проблемой ее автоматизации являются
        \begin{itemize}
            \item недостаточный уровень семантической совместимости и взаимопонимания между людьми и отсутствие сознания серьезности этой проблемы;
            \item недостаточный уровень договоренности и отсутствия понимания серьезности этой проблемы;
            \item отсутствие четкой методики согласования точек зрения и отсутствие понимая серьезности этой проблемы.
        \end{itemize}
        Интеллектуальные компьютерные системы могут и должны создать корпоративную среду для решения этих проблем.По сути это не что иное, как поддержка коллективного проектирования соответствующих разделов баз знаний интеллектуальной компьютерной системы, реализуемая на \uline{семантическом уровне}, когда интеллектуальная компьютерная система становится самостоятельным полноправным участником деятельности, в обязанности которого входит:
        \begin{itemize}
            \item анализ семантической совместимости точек уровня различных участников,
            \begin{itemize}
                \item выявление противоречий и альтернатив
            \end{itemize}
            \item фиксация авторства
            \item отмена современной формы представления интеллектуального продукта (статьи, книги, документы)
        \end{itemize}
        Недостаточно высокий уровень семантической согласованности используемых понятий приводит к огромному количеству искусственно создаваемых противоречий.При этом следует отличать семантические противоречия (например, синонимию вводимых знаков) и, соответственно, методику их устранения или разногласия по поводу системы вводимых понятий от терминологических разногласий, методика устранения которых может и должна быть максимально простой и лишенной эмоциональной окраски. Излишнее увлечение терминологическими спорами существенно тормозит творческий процесс, но и несерьезное отношение к постоянному совершенствованию и соблюдению \uline{правил} построения терминов также недопустимо.}
        
    \scnheader{Рынок знаний, реализуемый в рамках Экосистемы OSTIS}
    \scntext{пояснение}{Важнейшим видом предметно-независимой человеческой деятельности, осуществляемой в рамках \textit{Экосистемы OSTIS} является перманентный реинжиниринг всех \textit{ostis-систем}, входящих в \textit{Экосистему OSTIS}. Указанная деятельность должна быть направлена на перманентную и быструю эволюцию всех ostis-систем и, самое важное, на эволюцию \textit{Экосистемы OSTIS} в целом. Особо следует подчеркнуть, что эволюция \textit{ostis-систем} и \textit{Экосистемы OSTIS} в целом представляет собой весьма сложный творческий, коллективный процесс, который принципиально может быть автоматизирован \uline{только частично}. При этом от людей, участвующих в этом процессе требуется высокая квалификация, высочайшая системная культура на уровне глубокого знания общей теории систем, высокая математическая культура --- культура формализации, высокая культура конвергенции (обнаружения сходств, доведение их до формальных аналогий), высокая культура глубокой интеграции, высокий уровень договороспособности.\\
        Кроме указанных требований необходим высочайший уровень мотивации к тому, чтобы эволюция отдельных компонентов \textit{Экосистемы OSTIS} (в частности, отдельных \textit{ostis-систем}) не осуществлялась в ущерб эволюции \textit{Экосистемы OSTIS} в целом, например, путём привнесения эклектичности, многообразия форм решения похожих проблем, путем ослабления фундаментального требования \uline{максимально возможной простоты} и логичности принципов, лежащих в основе Экосистемы OSTIS.\\
        Существенно подчеркнуть, что эволюция \textit{ostis-систем} и \textit{Экосистем OSTIS} в целом сводится к коллективному реинжинирингу \textit{баз знаний ostis-систем}, что, в свою очередь сводится к:
        \begin{itemize}
            \item ручной генерации предлагаемых дополнительных (новых) знаний в базу знаний указываемой \mbox{ostis-системы}
            \item ручной генерации предлагаемых изменений текущего состояниябазы знаний указываемой ostis-системы;
            \item автоматическому назначению компетентных и заинтересованныхрецензентов;
            \item ручному рецензированию каждого поступившего предложения,результатом чего является:
            \begin{scnitemizeii}
                \item либо полное одобрение;
                \item либо полное неодобрения с предлагаемой аргументацией;
                \item либо детальная рекомендация доработки, предположения;
            \end{scnitemizeii}
            \item автоматическому назначению достаточно широкого круга компетентных и заинтересованных специалистов для утвержденияпоступившего предложения (после получения одобрения от всех назначенных экспертов);
            \item автоматическому принятию решения по одобрению поступившегопредложения на основании мнения всех привлечённых экспертов испециалистов.
        \end{itemize}
        Таким образом в \textit{базе знаний} каждой \textit{ostis-системы} можно (и нужно!)фиксировать весь процесс обсуждения каждого поступившего предложения суказанием (1) моментов времени всех привлечённых событий; (2) участников каждого события (авторов предложений, авторов рецензий участников голосования).\\
        Кроме того, каждая \textit{ostis-система}, анализируя процесс использованияхранимых ею знаний в процессе эксплуатации, может оценивать частотунепосредственного и опосредованного использования этих знаний, т.е. может оценить степень востребованности этих знаний.\\
        Следовательно, в перспективе \textit{Экосистема OSTIS} может с достаточновысокой степенью \uline{объективности} может оценивать объем и значимость вклада каждого специалиста в развитие распределенной базы знаний \textit{Экосистемы OSTIS}. Это является фундаментальной основой дляформирования достаточно объективного (честного) \textit{рынка знаний}.}
        
    \scnheader{Рынок знаний, реализуемый в рамках Экосистемы OSTIS}
    \scntext{правило для авторов в рамках Экосистемы OSTIS}{знания, \uline{предлагаемые} для рецензирования, согласования,утверждения и публикации в базе знаний соответствующей ostis-системы должны быть специфицированы (указана ostis-система, атомарный раздел базы знаний, дата и время, автор,новый вид публикации, рынок знаний,защита авторского права не на уровне документов, а на уровне смысла.}
    \scntext{коллективное совершенствование базы знаний}{Абсолютно идеальных решений (в том числе проектных) не бывает. Поэтому (1) не надо бояться ошибок и (2) надо минимизировать степень ошибочности за счёт (2.1) \uline{оперативности} исправления ошибок и (2.2) повышения качества (уровня) анализа при принятии решения путем (2.2.1) \uline{коллективного} характера экспертизы,(2.2.2) достаточного количества привлекаемыхэкспертов и (2.2.3) учёта уровня осведомленности(квалифицированности и  погруженности всоответствующую предметную область ионтологию). Для каждого эксперта, привлекаемого к принятию решения нужен постоянно уточняемый, по объективным критериям коэффициент осведомленности-авторитетности каждого эксперта к каждой конкретной предметной области.}
    \scntext{правила редактирования Общей базы знаний коллектива интеллектуальной системы}{
        \begin{itemize}
            \item Если Вы в рамках базы знаний разрабатываемой Вами ostis-системы хотите ввести знак новой ранее не описываемой сущности, то Вы должны проверить, что эта сущностьдействительно не описывалась в рамках виртуальной базы знаний всей Экосистемы OSTIS
            \begin{scnitemizeii}
                \item Если в результате такой проверки выяснилось, что указанная сущность уже рассматривалась, то Вы должны использоватьвведенный ранее основной внешний идентификатор этой сущности (Если он Вам не нравится, можете предложить, но пока не использовать, свой)
                \item Если сущность не рассматриваласть, нужно специфицировать, связать с семантическиблизким (особенно для понятий)
            \end{scnitemizeii}
        \end{itemize}
        От толковых словарей и энциклопедий --- к стройнойсемантической сети таких спецификации \uline{всех} описываемых сущностей, которые позволяют установить (желательно автоматически) наличие или отсутствие в рамках технического состояния базызнаний синонимичного знака для любого нового знака, вводимого в базу знаний.}
    \scntext{cтруктура качественной спецификации}{Нужно стремиться:
        \begin{itemize}
            \item к однозначности такой спецификации;
            \item координаты в пространстве декомпозиций
            \item к семантической близости;
            \item сходства, отличия;
        \end{itemize}}
    
    \scnheader{качество человеческой деятельности}
    \scnidtf{качество деятельности человеческого общества}
    \scntext{пояснение}{Поскольку человеческого общество в целом является кибернетической системой, (которая принадлежит классу иерархических многоагентных систем, качество деятельности человеческого общества можно оценивать по критериям качества кибернетических систем.
        \\На основании этих критериев можно оценивать:
        \begin{itemize}
            \item качество информационной среды, формируемой человеческим обществом, т.е. качество накапливаемой и общедоступной информации;
            \item качество текущего состояния общечеловеческих знаний;
            \item качество методов и технологий, используемых для решения задач как в рамках накопленных человечеством знаний, так и врамках внешней среды человеческого общества;
            \item качество организаций человеческой деятельности в целом;
            \item обучаемость (темпы эволюции) человеческого общества в целом.
        \end{itemize}
        Современный этап развития науки и техники характерен тем, что при оценке качества научно-технических результатов акцентируется внимание на новизне результатов, на их \uline{отличиях} от текущего положения дел. Это создает почву и дляимитации этой новизны и для увеличения барьеров между различными дисциплинами, что существенно препятствует конвергенции и интеграции различных дисциплин. Указанная конвергенция и инеграция, в частности, необходима для \uline{комплексной} автоматизации \uline{всех} видов человеческойдеятельности в рамках smart-общества. Очевидно, что основной такой комплексной автоматизации должна быть \textbf{\textit{Общая формальная теория человеческой деятельности}}.}
        
    \scnheader{уровень конвергенции и интерации человеческой деятельности и её результатов}
    \scnrelto{свойство-предпосылка}{Качество человеческой деятельности}
    \scntext{пояснение}{Повышение уровня конвергенции и интеграции различных видов человеческой деятельности и, соответственно, результатовэтой деятельности является важнейшим фактором (важнейшим направлением) повышения качества(эффективности) человеческой деятельности, а, следовательно, и качества самого человеческого общества как сложной распределенной социотехнической кибернетической системы.}
    \scntext{вопрос}{Что является главным препятствием существенному повышению уровня конвергенции и интеграции человеческой деятельности.}
    \begin{scnindent}
    	\scntext{ответ}{Главным препятствием повышению уровня конвергенции и интеграции человеческой деятельности является то, что на текущем этапе эволюции человеческого общества основным механизмом эволюции является конкуренция. Конкуренция предполагает противопоставление результатов своей деятельности результатам конкурентов. т.е. акцентирует внимание на отличиях, новизне, преимуществах своих результатов по отношению к результатам своих конкурентов. При этом мысль о целесообразности объединения усилий со своими конкурентами чаще обусловлена стремлением повысить уровень конкурентоспособности и прибыли по отношению к другим более сильным конкурентам и значительно реже обусловлена искренним стремлением получить более качественный результат.
            \\Таким образом, повышение уровня конвергенции и интеграции всех видов человеческой деятельности требует весьма сложного перехода от использования механизма конкуренции в её современном виде к созданию мощной технологической основы, обеспечивающей широкое взаимовыгодное сотрудничество и гарантированные возможности самореализации каждого человека и каждого коллектива. Фундаментом указанной технологической основы может и должен стать общечеловеческий рынок знаний, который построен на базе сети интеллектуальных компьютерных систем и в рамках которого фиксируется и объективно оценивается значимость вклада каждого человека и каждого коллектива.}
    \end{scnindent}
    
    \scnheader{автоматизация человеческой деятельности}
    \begin{scnrelfromlist}{вопрос}
        \scnfileitem{В чем заключаются проблемы комплексной автоматизации человеческой деятельности}
        \scnfileitem{Как автоматизировать участие человеческая одновременно в нескольких разных действиях (разных областях деятельности), принадлежащих в общем случае разным видам деятельности}
    \end{scnrelfromlist}

    \scnheader{Экосистема OSTIS}
    \scntext{примечание}{Во многом разработка принципов организации взаимодействия интеллектуальных компьютерных систем (ostis-систем) и людей, входящих в состав \textit{Экосистемы OSTIS} должна опираться на анализ того, как взаимодействие осуществляется между людьми, когда основные проблемы возникают из-за:
        \begin{itemize}
            \item отсутствия взаимопонимания (семантической совместимости),
            \item противоречий между целями различных субъектов,
            \item имитации целенаправленных действий,
            \item нарушений каких-либо соглашений, договоренностей и даже законов (правил поведения и обязанностей).
        \end{itemize}
        Общая (общедоступная) \textit{база знаний} всей \textit{Экосистемы OSTIS}, а также корпоративная \textit{база знаний} каждого \textit{ostis-сообщества}, входящего в состав \textit{Экосистемы OSTIS}, является распределенной, но при этом обязательно целостной. Она поддерживается группой специальных \textit{ostis-систем}, являющихся \textit{порталами знаний} по самым различным областям. Для \textit{Технологии OSTIS} роль такого \textit{портала знаний} выполняет \textit{Метасистема OSTIS}. \textit{Экосистема OSTIS} представляет собой многоагентную социотехническую систему, в которой каждая \textit{индивидуальная ostis-система}, входящая в состав \textit{Экосистемы OSTIS}, каждый пользователь указанных \textit{ostis-систем}, а также каждое \textit{ostis-сообщество}, входящее в Экосистему, является её самостоятельным \textit{субъектом*} (когнитивным агентом). При этом каждый субъект \textit{Экосистемы OSTIS} должен соблюдать определенные правила, обеспечивающие качественную (эффективную) эксплуатацию и эволюцию \textit{Экосистемы OSTIS}.}
    \scntext{резюме}{Сама идея комплексной автоматизации всех видов человеческой деятельности предполагает необходимость:
        \begin{itemize}
            \item разработки достаточно детальных формальных теорий всех видов человеческой деятельности, причем, теорий, доведенных до уровня разделов баз знаний соответствующих корпоративных компьютерных систем --- это, фактически, строгое описание стандартов различных видов человеческой деятельности, доведенное до такого уровня, чтобы соответствующая корпоративная система \underline{понимала} , в какой деятельности она участвует, и могла быть активным и полноценным субъектом (участником) этой деятельности;
            \item серьезного отношения и научного подхода к формализации различных видов человеческой деятельности, к разработке самых различных стандартов;
            \item глубокой конвергенции различных областей (разделов) человеческой деятельности и, соответственно, различных видов человеческой деятельности, осуществляемой в условиях достигнутого уровня автоматизации этой деятельности. Это предполагает необходимость рассмотрения каждого вида человеческой деятельности в контексте \textbf{\textit{Общей теории человеческой деятельности}} в условиях \underline{текущего} состояния уровня автоматизации этой деятельности;
            \item обеспечения высоких темпов эволюции и, следовательно, высокого уровня \underline{гибкости} \textit{Общей теории человеческой деятельности} и теорий (стандартов) каждого \textit{вида деятельности} в силу их большой зависимости от текущего уровня автоматизации;
            \item автоматизации взаимодействия субъектов не только внутри каждой области (раздела) человеческой деятельности, но и между этими областями (разделами), что предполагает автоматизацию \underline{представительства} каждой области (раздела) человеческой деятельности во множестве всех таких областей;
            \item понимания того, что эффективность человеческой деятельности во многом определяется скоординированностью, адекватностью, грамотностью поведения каждого субъекта. Поэтому автоматизация человеческой деятельности должна быть направлена на более глубокую координацию этой деятельности на основе учета смысла и целей этой деятельности. А это превращает средства автоматизации в полноценных субъектов коллективной деятельности
        \end{itemize}}
    
    \scnheader{автоматизация человеческой деятельности}
    \scntext{примечание}{Рассмотрение комплексной автоматизации человеческой \textit{деятельности в области Искусственного интеллекта} естественным образом можно расширить (обобщить) до рассмотрения комплексной автоматизации человеческой деятельности в целом.}
    \scntext{примечание}{Для того, чтобы обеспечить качественную автоматизацию любой \textit{области человеческой деятельности} с помощью \textit{интеллектуальных компьютерных систем}, необходимо построить \textit{формальную модель} этой области деятельности и довести эту модель до такого уровня формализации, чтобы она могла стать частью \textit{базы знаний интеллектуальной компьютерной системы}, используемой для автоматизации указанной \textit{области человеческой деятельности}. Очевидно, что, чем субъекты, участвующие в какой-либо коллективной деятельности, (люди и интеллектуальные компьютерные системы) лучше понимают суть, цели, критерии качества указанной коллективной деятельности, тем выше качество выполнения этой деятельности.}
    
    \scnheader{формальная модели автоматизируемой области человеческой деятельности}
    \scnhaselement{объект деятельности}
    \scnhaselement{среда деятельности}
    \scnhaselement{инструменты (инструментальные средства) деятельности}
    \scnhaselement{субъект деятельности}
    \scnhaselement{текущее состояния деятельности (как процесса) --- в том числе, спецификация действий (целей, задач), выполняемых в текущий момент}
    \scnhaselement{формулировка закономерностей --- в том числе, правил поведения субъектов деятельности}
    \scnhaselement{спецификация всех используемых субъектами деятельности методов выполнения сложных действий (решения задач)}
    
   \scnheader{область человеческой деятельности}  
   \scnhaselement{процесс взаимодействия умного дома с его жильцами и посетителями}
   \scnhaselement{процесс взаимодействия умного предприятия, выпускающего определенного вида продукцию, с его сотрудниками}
   \scnhaselement{процесс взаимодействия студентов и преподавателей в рамках умной кафедры, осуществляющей подготовку молодых специалистов по какой-либо инженерной специальности}
   \scnhaselement{процесс взаимодействия постояльцев, посетителей и сотрудников умного отеля}
   \scnhaselement{процесс взаимодействия посетителей и сотрудников умного музея}
   \scnhaselement{процесс взаимодействия пациентов и медицинского персонала умной поликлиники, умной больницы}
   \scnhaselement{процесс взаимодействия граждан и чиновников в рамках умной администрации некоторого региона}
   \scnhaselement{процесс взаимодействия жителей и гостей в рамках умного города}
   \scntext{примечание}{Для комплексной автоматизации человеческой деятельности в целом (Объединенной человеческой деятельности) автоматизации отдельных областей человеческой деятельности явно не достаточно, поскольку тесные связи между различными областями человеческой деятельности требуют автоматизации не только деятельности внутри каждой из этих областей, но и внешней деятельности, обусловленной необходимостью взаимодействия между различными областями деятельности, например, в рамках более крупных областей деятельности. Так, например, каждое предприятие взаимодействует со своими поставщиками и потребителями. Очевидно, что автоматизация такой внешней деятельности и, тем более, автоматизация с использованием интеллектуальных компьютерных систем существенно упрощается, если будут совпадать (будут унифицированы) принципы, лежащие в основе автоматизации каждой области деятельности, а также принципы автоматизации крупных областей деятельности, в состав которых входит некоторое количество более мелких областей человеческой деятельности.
        \\Таким образом, для комплексной автоматизации человеческой деятельности в целом с применением интеллектуальных компьютерных систем и для обеспечения эффективной интеграции различных областей человеческой деятельности необходима разработка \textbf{\textit{Общей формальной теории человеческой деятельности}}, которая объединила бы формальные модели всевозможных областей человеческой деятельности.}
        
    \scnheader{Общая формальная теория человеческой деятельности}
    \scnhaselement{формальная теория видов человеческой деятельности}
     \begin{scnindent}
     	\scntext{примечание}{Поскольку каждый \textbf{\textit{вид человеческой деятельности}} --- это класс однотипных \textit{областей человеческой деятельности}, формальная теория каждого вида человеческой деятельности --- это формальное представление \underline{стандарта} соответствующего класса областей человеческой деятельности. Так, например, можно говорить о формальной модели конкретного предприятия рецептурного производства (например, предприятие Савушкин продукт , выпускающего молочную продукцию), но можно говорить и о формальной теории всего класса предприятий рецептурного производства --- о формальном представлении стандарта ISA-88. Формальная теория каждого вида человеческой деятельности включает в себя формальное описание технологии, обеспечивающей осуществление каждой области (фрагмента) человеческой деятельности, принадлежащей указанному виду деятельности. В описание технологии входит описание используемых методов, средств и основных объектов и субъектов деятельности.}
      \end{scnindent}
     \scnhaselement{иерархическая декомпозицию Объединенной человеческой деятельности по нескольким признакам}
        \begin{scnindent}
     	\scntext{примечание}{Основными признаками такой декомпозиции являются региональный признак и целевая направленность деятельности. По региональному признаку на высшем уровне иерархии выделяются такие области человеческой деятельности, как Деятельность Франции, Деятельность Германии и далее деятельность всех стран. По признаку целевой направленности на высшем уровне иерархии выделяются: Научно-исследовательская деятельность человечества, Проектная деятельность человечества, Производственная деятельность человечества, Образовательная деятельность человечества, Здравоохранительная деятельность человечества, Природоохранная деятельность человечества, Административная деятельность человечества и др. Дальнейшая декомпозиция областей человеческой деятельности по признаку целевой направленности выделяет такие области деятельности, как
            \begin{scnitemizeii}
                \item \textit{Научно-исследовательская деятельность человечества в области Математики}
                \item \textit{Научно-исследовательская деятельность человечества в области Лингвистики}
                \item и др.
            \end{scnitemizeii}
            Заметим при этом, что, в отличие от чисто научных дисциплин, дисциплины научно-технического типа (например, дисциплина \textit{Искусственный Интеллект}) представляют собой симбиоз фрагментов (областей) деятельности, принадлежащих разным видам деятельности:
            \begin{scnitemizeii}
                \item научно-исследовательской деятельности;
                \item деятельности по разработке технологии проектирования (CAD);
                \item деятельности по разработке технологии производства (CAM);
                \item проектная деятельность;
                \item производство спроектированного объекта;
                \item образовательной деятельности;
                \item бизнес-деятельности.
            \end{scnitemizeii}}
        	\begin{scnindent}
        		\scntext{уточнение}{Кроме указанных областей человеческой деятельности выделяются области, соответствующие различным сочетаниям значений указанных признаков декомпозиции областей человеческой деятельности. Примерами таких областей являются: Научно-исследовательская деятельность Франции, Образовательная деятельность Германии. Подчеркнем то, что количество областей человеческой деятельности, выделенных в результате указанной иерархической декомпозиции \textit{Объединенной человеческой деятельности}, является, хоть и не очень большим, но конечным в каждый момент времени.}
        	\end{scnindent}
         \end{scnindent}
         \begin{scnrelfromlist}{задача}
            \scnfileitem{унификация формального описания самых различных технологий для самых различных областей человеческой деятельности}
            \scnfileitem{унификация формального описания всевозможных видов человеческой деятельности}
            \scnfileitem{унификация формального описания связей между различными областями и видами человеческой деятельности, различными субъектами деятельности, объектами, средствами (инструментами)}
            \scnfileitem{глубокая конвергенция всех видов человеческой деятельности, областей человеческой деятельности, используемых методов}
     	\end{scnrelfromlist}
    \begin{scnrelfromvector}{что делать}
        \scnfileitem{Необходим переход от локальной автоматизации различных областей и видов человеческой деятельности путем независимой друг от друга разработки систем автоматизации бизнес-процессов даже близких по виду деятельности предприятий к комплексной автоматизации человеческой деятельности в целом прежде всего для обеспечения совместимости различных областей деятельности и исключения ужасающего и никому не нужного дублирования (многообразия форм) автоматизации аналогичных бизнес-процессов}
        \scnfileitem{Все многообразие человеческой деятельности необходимо четко стратифицировать, доведя эту стратификацию до строгого формального представления}
        \scnfileitem{Необходимо
            \begin{itemize}
                \item четко выделить все виды человеческой деятельности, соответствующие текущему уровню развития человеческого общества
                \item построить четкую иерархию этих видов на основании отношения, связывающего виды человеческой деятельности с их подвидами
                \item унифицировать человеческую деятельность в рамках каждого выделенного вида, разработав соответствующие стандарты, для каждого из которых построить четкую систему используемых понятий
                \item довести указанные стандарты до такого уровня формализации, чтобы они стали частью базы знаний интеллектуальной системы автоматизации соответствующего вида человеческой деятельности.
            \end{itemize}}
        \scnfileitem{Необходимо обеспечить конвергенцию, семантическую совместимость и глубокую интеграцию различных видов и областей человеческой деятельности путем:\\
            \begin{itemize}
                \item согласования систем понятий, соответствующих стандартам разных видов человеческой деятельности, и особенно согласования систем понятий между стандартами видов и подвидов человеческой деятельности
                \item представления стандарта каждого вида человеческой деятельности в виде формальной онтологии
                \item построения такой иерархической системы формальных онтологий, соответствующих всевозможным видам человеческой деятельности, в которой обеспечивалась бы конвергенция и \underline{семантическая совместимость} онтологий, входящих в эту систему, а также \underline{наследование свойств} от онтологии каждого вида человеческой деятельности к онтологии каждого подвида этого вида человеческой деятельности.
            \end{itemize}}
    \end{scnrelfromvector}
    \begin{scnindent}
    	\scntext{следовательно}{Таким образом, в целях повышения эффективности автоматизации человеческой деятельности и, в первую очередь, в целях существенного снижения трудозатрат на такую автоматизацию необходимо с точки зрения общей теории систем фундаментально переосмыслить современную организацию человеческой деятельности, поскольку автоматизация беспорядка приводит к ещё большему беспорядку. На этом пути имеется только одно препятствие --- противодействие лени с высоким уровнем эгоизма, которым современный беспорядок организации человеческой деятельности выгоден.}
    \end{scnindent}
    
\scnheader{Комплексная автоматизация человеческой деятельности в области Искусственного интеллекта с помощью интеллектуальных компьютерных систем нового поколения}
\begin{scnsubstruct}
	\scnheader{Поддержка жизненного цикла интеллектуальных компьютерных систем нового поколения}
    \scntext{примечание}{В рамках \textit{Технологии OSTIS} поддержка жизненного цикла интеллектуальных компьютерных систем нового поколения (\textit{ostis-систем}) осуществляется на основе \textit{Метасистемы OSTIS}, которая относится к классу \textit{ostis-систем} и фактически является формой реализации указанной Технологии.}
    \scntext{принципы, лежащие в основе}{Автоматизация поддержки жизненного цикла \textit{ostis-систем} осуществляется как в форме инструментального обслуживания инженерной деятельности (в частности, Метасистема OSTIS является системой автоматизации проектирования ostis-систем), так и в форме информационного обслуживания указанной деятельности. Для этого база знаний \textit{Метасистемы OSTIS} содержит: 
	\begin{itemize}
		\item текущее состояние полного текста \textit{Стандарта ostis-систем};
		\item Текущее состояние Библиотеки многократно используемых компонентов ostis-систем;
		\item Используемые и реализуемые инженерами методики поддержки жизненного цикла ostis-систем;
		\item Документацию инструментальных средств, инженерами для поддержки жизненного цикла  ostis-систем.
	\end{itemize}}
	
	\scnheader{Метасистема OSTIS}
	\begin{scnrelfromlist}{задача}
		\scnfileitem{Обеспечить автоматизацию \textit{Поддержки жизненного цикла Стандарта ostis-систем}, то есть обеспечивает организацию взаимодействия между авторами этого Стандарта, направленного на перманентное его развитие.}
		\scnfileitem{Обеспечить автоматизацию \textit{Поддержки жизненного цикла Технологии OSTIS}, которая сводится к поддержке жизненного цикла основной части базы знаний \textit{Метасистемы OSTIS}, которая является полной документацией текущего состояния \textit{Технологии OSTIS}.}
	\end{scnrelfromlist}
	
	\scnheader{Человеческая деятельность в области Искусственного интеллекта}
	\scntext{примечание}{Автоматизация направлений \textit{Человеческой деятельности в области Искусственного интеллекта} также может осуществляться с помощью \textit{ostis-систем}, семантически совместимых и взаимодействующих с \textit{Метасистемой OSTIS} в рамках \textit{Экосистемы OSTIS}.}
\end{scnsubstruct}

    \scnheader{следует отличать}
    \begin{scnhaselementset}
        \scnitem{вид человеческой деятельности}
        \begin{scnindent}
            \begin{scnhaselementrolelist}{пример}
                \scnitem{научно-исследовательская деятельность}
                \scnitem{проектирование}
                \begin{scnindent}
                    \scnidtf{проектная деятельность}
                    \scnsuperset{проектирование интеллектуальной компьютерной системы}
                    \begin{scnindent}
                    	\scnsuperset{проектирование ostis-системы}
               		\end{scnindent}
                 \end{scnindent}
            \end{scnhaselementrolelist}
        \end{scnindent}
        \scnitem{область человеческой деятельности}
        \begin{scnindent}
            \begin{scnhaselementrolelist}{пример}
                \scnitem{Научно-исследовательская деятельность в области Искусственного интеллекта}
                \begin{scnindent}
                    \scniselement{научно-исследовательская деятельность}
                    \scnrelfrom{часть}{Научно-исследовательская деятельность РАИИ}
                \end{scnindent}
                \scnitem{Проектирование Метасистемы OSTIS}
                \begin{scnindent}
                    \scniselement{проектирование ostis-системы}
                \end{scnindent}
            \end{scnhaselementrolelist}
        \end{scnindent}
        \scnitem{подвид человеческой деятельности*}
        \begin{scnindent}
            \scnsubset{включение*}
            \begin{scnindent}
            	\scnidtf{подмножество*}
            \end{scnindent}
        \end{scnindent}
        \scnitem{подобласть человеческой деятельности*}
        \begin{scnindent}
            \scnsubset{часть*}
        \end{scnindent}
    \end{scnhaselementset}

    \scnheader{информационная технология}
    \scntext{пояснение}{Множество технологий, связанных с проектированием и производством компьютерных систем и их компонентов, с эксплуатацией компьютерных систем, а также с их использованием в качестве инструмента в составе самых различных технологий. В рамках различных информационных технологий компьютерные системы рассматриваются как инструментальные средства, как вспомогательные субъекты, обеспечивающие автоматизацию соответствующих видов деятельности. Но в некоторых информационных технологиях компьютерные системы являются также и \underline{объектами} автоматизируемых видов деятельности. Примерами таких технологий являются:
        \begin{itemize}
            \item технология проектирования компьютерных систем;
            \item технология реализации (сборки) компьютерных систем;
            \item технология обновления компьютерных систем.
        \end{itemize}}
    \scnhaselement{Комплекс современных информационных технологий}
    \scnhaselement{Комплекс современных технологий искусственного интеллекта}
    \scnhaselement{Технология OSTIS}

    \scnheader{автоматизация человеческой деятельности}
    \scnidtf{человеческая деятельность, направленная на повышения уровня автоматизации человеческой деятельности, а также на повышение качества (в том числе, уровня интеллекта) человеческого общества как многоагентной кибернетической системы}
    \scnsubset{вид человеческой деятельности}
    \scntext{примечание}{Важнейшим этапом автоматизации человеческой деятельности в перспективе должен стать переход к существенно более высокому уровню \textit{интеллекта человеческого общества} как целостной кибернетической системы путем преобразования современного человеческого общества в сообщество взаимодействующих между собой людей и интеллектуальных компьютерных систем. Такое сообщество иногда называют smart-обществом, обществом 5.0.\\
        Особо подчеркнем то, что переход к такому интеллектуальному обществу требует существенного переосмысления современной организации различных видов человеческой деятельности. Прежде всего, следует подчеркнуть, что эффективность (коэффициент полезного действия) современной организации человеческой деятельности в целом ужасающе низка, а, как известно, автоматизация беспорядка (даже с помощью интеллектуальных компьютерных систем) приводит к ещё большему беспорядку.}
    
    \scnheader{вид человеческой деятельности}
    \scnidtf{Множество всевозможных видов человеческой деятельности}
    \scnrelfrom{разбиение}{Разбиение Множества видов человеческой деятельности по степени их автоматизируемости}
    \begin{scnindent}
	    \begin{scneqtoset}
	        \scnitem{вид человеческой деятельности, который принципиально может быть автоматизирован полностью}
	        \begin{scnindent}
	            \scnidtf{Множество полностью автоматизируемых видов человеческой деятельности}
	        \end{scnindent}
	        \scnitem{вид человеческой деятельности, который может быть автоматизирован только частично}
	        \begin{scnindent}
	            \scnidtf{Множество частично автоматизируемых видов человеческой деятельности}
	        \end{scnindent}
	        \scnitem{вид человеческой деятельности, который принципиально никак не может быть автоматизирован}
	        \begin{scnindent}
	            \scnidtf{Множество неавтоматизируемых видов человеческой деятельности, которые могут быть выполнены только вручную\ (точнее самими людьми с возможным использованием каких-либо пассивных\ инструментов --- топора, лопаты и т. п.)}
	        \end{scnindent}
	    \end{scneqtoset}
    \end{scnindent}
    
    \scnheader{вид человеческой деятельности, который принципиально может быть автоматизирован полностью}
    \scntext{примечание}{Есть виды человеческой деятельности, которые принципиально могут быть автоматизированы \underline{полностью}, но в текущий момент эта автоматизация не полна. Это ,например, частично автоматизированная деятельность по производству спроектированных искусственных объектов. Здесь важна \underline{четкость} распределения обязанностей между различными средствами автоматизации и поэтапное исключение неавтоматизированных действий, вручную выполняемых людьми (например, сотрудниками производственных предприятий), т. е. поэтапная автоматизация этих действий.}

    \scnheader{автоматизация человеческой деятельности}
    \scntext{вопрос}{Почему для комплексной автоматизации человеческой деятельности целесообразно использовать семантически совместимые, договороспособные, самостоятельные интеллектуальные компьютерные системы, которым можно делегировать права на принятие некоторых решений.}
    \scntext{вопрос}{Почему для повышения уровня комплексной \textit{автоматизации человеческой деятельности} необходим переход от современных (традиционных) \textit{компьютерных систем} и соответствующих им информационных технологий, а также от \textit{современных интеллектуальных компьютерных систем} и соответствующих им современных технологий искусственного интеллекта к \textit{интеллектуальным компьютерным системам} \uline{нового поколения} и к соответствующей им Комплексной технологии проектирования таких систем.}
    \begin{scnindent}
    \scntext{ответ}{В силу отсутствия унификации представления обрабатываемой информации в традиционных компьютерных системах и, как следствие, отсутствия совместимости этих систем как на синтаксическом уровне, так и на семантическом уровне, принциально не существует универсального метода системной интеграции традиционных компьютерных систем и, следовательно, невозможна полная автоматизация решения этой задачи. Системная интеграция традиционных компьютерных систем практически всегда осуществляется вручную с учетом индивидуальной специфики каждой интегрируемой системы и, следовательно, является весьма трудоемкой и требующей высокой квалификации разработчиков.
        \\Но на данном этапе эволюции компьютерных систем крайне актуальной является полная автоматизация их интеграции без какого бы то ни было участия разработчиков и тем более конечных пользователей. Если компьютерные системы не приобретут способность \uline{самостоятельно} взаимодействовать между собой в целях решения сложных комплексных задач, то эффективность использования человечеством интенсивно расширяемого многообразия весьма полезных и качественно реализованных информационных ресурсов и сервисов будет весьма низкой. Традиционно компьютерные технологии позволяют реализовать \uline{любую} модель обработки информации (в том числе и \uline{любую} интеллектуальную модель --- нейросетевую, логическую и т.д.). Однако актуальным является не реализация самих этих моделей, а их интеграция, что требует обоспечения синтаксически и семантически совместимых компьютерных систем и полной автоматизации их системной интеграции. Следовательно необходим переход на принципиально новое поколение компьютерных технологий и, в частности, на принципиально новое поколение самих компьютеров, ориентированных на решение проблем совместимости компьютерных систем и полной автоматизации их системной интеграции.
        \\Таким образом, дальнейшее повышение уровня автоматизации различных видов человеческой деятельности потребует перехода на принципиально новый уровень информационных технологий --- от современных (традиционных) \textit{компьютерных систем} к компьютерным системам, имеющим существенно более \textit{высокий уровень интеллекта} и способным не только индивидуально решать достаточно сложные (в том числе, интеллектуальные) задачи, но и эфффективно \uline{самостоятельно} взаимодействовать между собой, координируя свою деятельность при решении задач, принадлежащих априори неизвестным (заранее не предусмотренным) классам задач и требующих коллективного (корпоративного) решения.
        \\Основные проблемы автоматизации \textit{человеческой деятельности} в настоящее время лежат не в области разработки средств автоматизации решения различных конкретных классов задач (в том числе и весьма сложных, интеллектуальных, труднорешаемых задач), а в области системной интеграции этих средств в комплексы, компоненты которых способны самостоятельно кооперироваться для совместного (коллективного) решения сложных задач. Но для этого указанные компоненты должны уметь согласовывать, координировать свои действия, должны понимать друг друга, должны быть семантически совместимы.}
    \end{scnindent}
    
    \scnheader{интеллектуальная компьютерная система}
    \scntext{примечание}{Различные интеллектуальные компьютерные системы могут быть эффективно использованы в качестве средств автоматизации самых различных видов человеческой деятельности. Но, поскольку все виды человеческой деятельности взаимосвязаны (как минимум потому, что каждый человек может одновременно участвовать сразу в нескольких видах деятельности, причем в разные моменты времени этот набор видов деятельности для каждого человека может быть различным), интеллектуальная компьютерная система автоматизации каждого вида человеческой деятельности должна эффективно взаимодействовать с другими интеллектуальными компьютерными системами, осуществляющими автоматизацию других видов человеческой деятельности. Другими словами, необходимо переходить от автоматизации отдельных видов человеческой деятельности к автоматизации комплекса всех видов человеческой деятельности. Для этого необходимо:
        \begin{itemize}
            \item не просто достаточно детально разработать \uline{теорию каждого вида деятельности}, выделив (1) все классы автоматизируемых действий, (2) все классы неавтоматизируемых действий, (3) соответствующие этим классам методы выполнения действий (в частности, это могут быть обобщенные бизнес-процессы), которые по сравнению с используемыми в настоящий момент могут потребовать существенного реинжиниринга бизнес-процессов;
            \item но и представить эти теории в формализованном и унифицированном виде в качестве фрагментов баз знаний соответствующих интеллектуальных компьютерных систем, обеспечив при этом высокую степень конвергенции этих теорий.
        \end{itemize}}
    \begin{scnrelfromlist}{возможное амплуа}
        \scnitem{средство автоматизации проектирования}
        \begin{scnindent}
            \scnsuperset{средство автоматизации проектирования интеллектуальных компьютерных систем}
            \begin{scnindent}
            	\scntext{примечание}{В силу большой сложности процесса проектирования интеллектуальных компьютерных систем для автоматизации этого процесса необходимо использовать именно интеллектуальные компьютерные системы.}
            \end{scnindent}
        \end{scnindent}
        \scnitem{средство автоматизации производства}
        \scnitem{средство повышения качества эксплуатации сложного объекта}
        \begin{scnindent}
            \scnsuperset{средство help-поддержки конечных пользователей}
            \scnsuperset{средство управления процессом повышения качества деятельности конечных пользователей}
            \scnsuperset{средство поддержки оптимальных эксплуатационных свойств эксплуатируемого объекта}
            \begin{scnindent}
            	\scntext{пояснение}{Здесь имеется в виду мониторинг состояния эксплуатируемого объекта, контроль условий эксплуатации, своевременная профилактика и ремонт.}
            \end{scnindent}
            \scnsuperset{средство поддержки совершенствования эксплуатируемого объекта в ходе его эксплуатации}
            \scntext{примечание}{Для интеллектуальных компьютерных систем все средства повышения качества их эксплуатации целесообразно встраивать в эти системы. Имеется в виду слияние нескольких интеллектуальных компьютерных систем в одну интегрированную.}
        \end{scnindent}
        \scnitem{средство автоматизации научно-исследовательской деятельности в рамках заданной научной дисциплины}
        \scnitem{средство автоматизации образовательной деятельности}
        \begin{scnindent}
            \scntext{примечание}{Автоматизация образовательной деятельности может осуществляться в рамках:
                \begin{itemize}
                    \item заданной учебной дисциплины
                    \item заданной учебной специальности
                    \item заданного учебного заведения
                    \item заданного государства.
                \end{itemize}}
        \end{scnindent}
        \scnitem{средство автоматизации бизнес-деятельности в заданной научно-технической области}
        \begin{scnindent}
        	\scntext{примечание}{Здесь важна автоматизация контроля за реализацией всех направлений организационной деятельности с учетом разработанных и постоянно уточняемых и корректируемых планов, а также с учетом согласованных приоритетов.}
        \end{scnindent}
        \scnitem{средство автоматизации деятельности в области здравоохранения}
        \scnitem{средство автоматизации административной деятельности}
        \scnitem{средство автоматизации деятельности жилищно-коммунального хозяйства}
        \scnitem{юриспруденция}
        \scnitem{правоохранительная деятельность}
        \scnitem{транспорт}
    \end{scnrelfromlist}
    
    \scnheader{Экосистема OSTIS}
    \scntext{примечание}{\textit{Экосистема OSTIS} является основой для перевода уровня информатизации различных областей человеческой деятельности на принципиально новый уровень, а также для интеграции соответствующих проектов --- Общество 5.0, Industry 4.0, University 3.0, Умный дом, Умный город и других (без интеллектуальных компьютерных систем все эти проекты невозможны).
        \\Все эти проекты должны быть приведены в единую стройную иерархическую систему взаимосвязанных проектов, охватывающих весь объем и многообразие человеческой деятельности.}
  
        \scnheader{Экосистема OSTIS}
        \scntext{вопрос}{Какие достоинства имеет Экосистема OSTIS}
        \scntext{достоинство}{Важнейшей особенностью Экосистемы OSTIS является то, что входящие в нее \textit{ostis-системы} благодаря высокому уровню их интеллекта и, в частности, высокому уровню их социализации, становятся самостоятельными, активными и полноправными субъектами, участвующими в реализации самых различных видов человеческой деятельности, что существенно повышает уровень её автоматизации.}
        \begin{scnrelfromset}{Что такое интеллектуальная система}
            \scnitem{\uline{система}(!)свойств}
            \scnitem{В.К. Финн}
            \scnitem{требования, предъявленные к интеллектуальным компьютерным системам}
        \end{scnrelfromset}
        \begin{scnrelfromset}{достоинства}
            \scnfileitem{семантическая совместимость
                \begin{itemize}
                    \item интеллектуальных компьютерных систем между собой
                    \item интеллектуальных компьютерных систем с их пользователями и разработчиками
                    \item семантическая совместимость = взаимопонимание
                \end{itemize}}
            \scnfileitem{Перманентная поддержка семантической совместимости}
            \scnfileitem{Способность координировать свои действия (договороспособность, координация(!)) при коллективном решении задач автоматизации \textbf{системной интеграции} интеллектуальных компьютерных систем, ручная реализация системной интеграции --- главный тормоз комплексной автоматизации. Многоагентная система из интеллектуальных компьютерных систем + \textbf{людей}.
                \begin{itemize}
                    \item Каждая ostis-система является, кроме всего прочего, способной обучать(повышать квалификацию) своих пользователей т.е. повышать эффективность своей эксплуатации
                    \\ostis-система
                    \\\scnsubset{интеллектуальная обучающая система}
                \end{itemize}}
            \scnfileitem{Экосистема интеллектуальных компьютерных систем
                \\\scneq{комплексная автоматизация человеческой деятельности}
                Достоинства Экосистемы(преимущества) и перспективы создания и развития Технологии OSTIS
                \begin{itemize}
                    \item Технология OSTIS как основа эволюции человеческого общества => переход к smart-обществу, к более интеллектуальному обществу
                    \item Экосистема OSTIS как продукт Технологии, т.е продукт технологии --- не отдельные интеллектуальные компьютерные системы, а целая Экосистема
                \end{itemize}
                это вариант smart-общества
                \\smart-предприятие, smart-город
                \\Требования к технологии разработки интеллектуальных компьютерных систем
                \begin{itemize}
                    \item smart-сообщество разработчиков интеллектуальных компьютерных систем и разработчиков самой технологии
                    \item консорциум!!
                    \item стандарты интеллектуальных компьютерных систем
                    \item стандарты процесса разработки различных интеллектуальных компьютерных систем
                    \item стандарты процесса совершенствования самой технологии
                    \item ориентация на новые компьютеры
                \end{itemize}
                Экосистема OSTIS -> цель
                \\\scneq{smart-общество = общество 5.0 как интеграция всевозможных специализированных smart-сообществ(...)}}
            \scnfileitem{конвергенция в области Искусственного интеллекта и не только(!!)(это необходимо для Экосистемы интеллектуальных компьютерны систем)}
            \scnfileitem{глубокая интеграция(совместимость)}
            \scnfileitem{новое поколение компьютеров}
            \scnfileitem{Консорциум OSTIS по разработке глобального комплекса семантически совместимых технологий, обеспечивающих комплексную автоматизацию всевозможных видов человеческой деятельности(для Экосистемы интеллектуальных компьютерных систем).Достоинства эволюции интеллектуальных компьютерных систем автоматизации проектирования интеллектуальных компьютерных систем распространяется на все технические дисциплины и соответственно сообщества.}
            \scnfileitem{система Проектов OSTIS, реализуемых консорциумом OSTIS --- как прообраз project-management нового типа, ориентированного на реализацию \uline{наукоемких} проектов с децентрализованным управлением и с перманентным коллективным уточнением и детализацией целей}
            \scnfileitem{\textbf{Рынок знаний и его реализация}
                \begin{itemize}
                    \item Смысловые представления знаний и глобальный характер минимизирует субъективизм, предвзятость, более эффективно защищает авторские права
                    \item Тормозит научно-техническое развитие(прогресс) становиться труднее
                    \item Выигрывает тот, кто действительно способствует прогрессу, а не тормозит его
                \end{itemize}
                Нет центральных и периферийных публикаций --- есть общая база знаний, в которой нет семантической эквивалентности(и, следовательно, нет плагиата). Монография OSTIS + \textbf{Метасистема OSTIS} как новый уровень автоматизации создания и эволюции научно-технического каталога статей и монографий к семантически совместимым базам и порталам научно-технических знаний! Достоинства эволюции портала научных знаний по Искусственному интеллекту и соответственно сообщества ученых распространяется на все научные дисциплины}
        \end{scnrelfromset}
    
\scnheader{Комплексная автоматизация всевозможных видов и областей человеческой деятельности с помощью и.к.с.}
\begin{scnsubstruct}
\scntext{примечание}{В предметной области рассмотрено то, как осуществляется и автоматизируется с помощью интеллектуальных компьютерных систем нового поколения \uline{весь комплекс} \textit{Человеческой деятельности в области Искусственного интеллекта}. Сейчас обобщим это и рассмотрим принципы организации и комплексной автоматизации \textit{человеческой деятельности} в целом, то есть автоматизации самых различных видов и областей человеческой деятельности.}

\scnheader{Общие принципы систематизации человеческой деятельности и ее комплексной автоматизации с помощью интеллектуальных компьютерных систем нового поколения}
\begin{scnsubstruct}
\scnheader{Почему можно обобщить опыт комплексной организации, структуризации и автоматизации \textit{человеческой} деятельности в области \textit{Искусственного интеллекта}?}
\scnidtf{Почему можно обобщить опыт комплексной организации, структуризации и автоматизации \textit{человеческой} деятельности в области создания и сопровождения интеллектуальных компьютерных систем?}
\scniselement{вопрос}
\begin{scnrelfromvector}{ответ}
	\scnfileitem{Во-первых, потому, что человеческая деятельность, направленная на поддержку всего \textit{жизненного цикла интеллектуальных компьютерных систем} нового поколения, является \uline{частной} \uline{областью деятельности} по отношению к виду человеческой деятельности, направленному на (обеспечивающему) поддержку всего жизненного цикла \uline{любой искусственной} (искусственно создаваемой) сущности (любого артефакта). В зависимости от сложности искусственно создаваемой сущности, уровень сложности человеческой деятельности, направленной на поддержку жизненного цикла этой сущности, может быть самым различным, но общая структура этой деятельности, соответствующая различным этапам жизненного цикла искусственно создаваемых сущностей, а также необходимым направлениям \uline{обеспечения} этой инженерной деятельности является одинаковой для искусственных сущностей различных классов.}
	\begin{scnindent}
        \begin{scnrelfromvector}{направления обеспечения поддержки жизненного цикла искусственных сущностей}
            \scnfileitem{Научно-исследовательская деятельность, направленная на изучение искусственных сущностей соответствующего класса.}
            \scnfileitem{Разработка стандарта искусственных сущностей указанного класса.}
            \scnfileitem{Разработка технологии поддержки искусственных сущностей указанного класса.}
            \scnfileitem{Подготовка кадров, способных осуществлять поддержку жизненного цикла искусственных сущностей указанного класса, то есть способных эффективно использовать указанную выше технологию.}
            \scnfileitem{Подготовка кадров, способных участвовать в указанной выше научно-исследовательской деятельности.}
            \scnfileitem{Подготовка кадров, способных участвовать в разработке стандарта искусственных сущностей заданного класса.}
            \scnfileitem{Подготовка кадров, способных участвовать в разработке и развитии указанной выше технологии.}
            \scnfileitem{Организационное обеспечение всего комплекса работ по развитию и использованию указанной технологии.}
        \end{scnrelfromvector}
    \end{scnindent}
	\scnfileitem{Во-вторых, потому, что многие сложные технические системы фактически становятся \textit{интеллектуальными компьютерными системами} (в том числе распределенными) с различными наборами сенсорных и эффекторных подсистем --- интеллектуальными автомобилями с автопилотом и автоштурманом, интеллектуальными заводами-автоматами, умными домами, умными городами и тому подобными.}
	\scnfileitem{В-третьих, потому, что характер деятельности \textit{интеллектуальных компьютерных систем нового поколения} и характер деятельности каждого \textit{человека} и каждой организации по сути мало чем отличаются, поскольку \textit{интеллектуальные компьютерные системы нового поколения} становятся \uline{равноправными} партнерами (субъектами) \textit{человеческой деятельности}, так как уровень их самостоятельности, ответственности, интероперабельности и интеллектуальности приближается к соответствующим качествам \textit{естественных} субъектов человеческой деятельности (физических лиц, юридических лиц, подразделений крупных организаций, неформальных организаций).}
\end{scnrelfromvector}

\scnheader{Человеческая деятельность в области Искусстенного интеллекта}
\scntext{примечание}{Итак, структуризацию \textit{человеческой деятельности} в области \textit{Искусственного интеллекта} на основе понятий \textit{вида деятельности}, \textit{области деятельности}, \textit{продукта деятельности} (объекта деятельности) можно легко обобщить для всех \textit{научно-технических дисциплин}, что дает возможность рассматривать автоматизацию деятельности в рамках всех \textit{научно-технических дисциплин} с общих позиций, так как автоматизация различных \textit{видов деятельности} в рамках различных \textit{научно-технических дисциплин} может выглядеть аналогичным образом, а иногда может быть реализована с помощью одной и той же \textit{интеллектуальной компьютерной системы}. Так например, любая \textit{интеллектуальная компьютерная система автоматизации проектирования} технических систем заданного вида может быть построена на основе \textit{интеллектуальной компьютерной системы автоматизации проектирования и реинжиниринга баз знаний}, поскольку результатом проектирования любой \textit{технической системы} является формальная модель (описание, спецификация, документация) этой \textit{технической системы}, обладающая достаточно полнотой для воспроизводства (реализации) этой системы.}

\scnheader{Искусственного интеллекта}
\scntext{задача}{На текущем этапе развития \textit{Искусственного интеллекта} необходимо переходить от автоматизации отдельных \textit{видов человеческой деятельности} к интегрированной автоматизации всего комплекса \textit{человеческой деятельности}, к созданию и постоянной эволюции всей \textbf{\textit{Глобальной экосистемы интеллектуальных компьютерных систем}}, самостоятельно взаимодействующих как между собой, так и с людьми, автоматизацию деятельности которых они осуществляют, а также с современными компьютерными системами, не являющимися интеллектуальными системами. При этом надо помнить, что основные \scnqqi{накладные} расходы, основные проблемы, возникают на \scnqqi{стыках} при интеграции различных технических решений. Разработчик каждой подсистемы должен гарантировать отсутствие указанных \scnqqi{накладных} расходов. При этом необходимо подчеркнуть, что следует ориентироваться не столько на создание эффективной \textit{Глобальной экосистемы интеллектуальных компьютерных систем}, сколько на создание эффективных методик и средств, направленных на \textit{перманентную эволюцию} такой \textit{экосистемы}.}

\scnheader{Методика автоматизации человеческой деятельности в области Искусстенного интеллекта}
\begin{scnrelfromvector}{этапы}
	\scnfileitem{Построение общей \textbf{\textit{структуры человеческой деятельности}}, в основе которой лежит иерархия \textit{человеческой деятельности} по видам деятельности и продуктам деятельности с четкой фиксацией различного вида связей между различными компонентами этой структуры.}
	\scnfileitem{Формализация различных \textit{видов человеческой деятельности}.}
	\scnfileitem{Разработка \textbf{\textit{технологии}}, обеспечивающей максимально возможную автоматизацию этой деятельности с помощью \textit{интеллектуальных компьютерных систем нового поколения}.}
	\scnfileitem{Обеспечение максимально возможной \textbf{\textit{конвергенции}} различных \textit{видов деятельности}, что позволит сократить многообразие средств автоматизации (то есть соответствующих \textit{интеллектуальных компьютерных систем нового поколения}).}
	\scnfileitem{Обеспечение максимально возможной \textbf{\textit{конвергенции технологий}} выполнения одного и того же \textit{вида деятельности} для разных объектов деятельности (конвергенции технологий проектирования объектов различных классов, конвергенции технологий мониторинга, профилактики и диагностики для агентов различных классов и так далее) и, тем самым, обеспечить \textbf{\textit{конвергенцию}} соответствующих средств автоматизации, построенных на основе \textit{интеллектуальных компьютерных систем нового поколения}.}
\end{scnrelfromvector}

\end{scnsubstruct}

\scnheader{Многообразие видов человеческой деятельности и связей между ними}
\begin{scnsubstruct}

    \scnheader{вид человеческой деятельности}
    \scniselement{вид деятельности}
    \scnhaselementrole{класс объектов}{класс всевозможных социально значимых объектов, на которые имеет смысл воздействовать}
    \begin{scnindent}
    	\scnidtf{класс всевозможных социально значимых объектов, поддержку жизненного цикла которых целесообразно осуществлять}
    \end{scnindent}

	\scnheader{поддержка жизненного цикла}
	\scnidtf{поддержка жизненного цикла социально значимых сущностей}
    \scntext{примечание}{Базовым видом человеческой деятельности можно считать \textbf{\textit{поддержку жизненного цикла}} различных сущностей.}
	\scniselement{вид деятельности}
	\begin{scnrelfromlist}{частный вид деятельности, выполняемой на некотором этапе}
		\scnitem{проектирование}
		\scnitem{производство}    
		\scnitem{начальное обучение}  
		\begin{scnindent}
			\scnidtf{настройка}
		\end{scnindent}
		\scnitem{мониторинг качества}
		\begin{scnindent}
			\scnidtf{плановое обследование и диагностика}
		\end{scnindent}
		\scnitem{восстановление требуемого уровня качества}
		\begin{scnindent}
			\scnidtf{ремонт, лечение}
		\end{scnindent}
		\scnitem{реинжиниринг}
		\begin{scnindent}
			\scnidtf{обновление, совершенствование}
		\end{scnindent}
		\scnitem{обеспечение безопасности}
		\scnitem{использование}
		\begin{scnindent}
			\scnidtf{эксплуатация, употребление}
		\end{scnindent}
	\end{scnrelfromlist}
	\begin{scnrelfromlist}{частный вид деятельности над подклассом объектов деятельности}
		\scnitem{научно-исследовательская деятельность}
		\begin{scnindent}
			\scnidtf{поддержка жизненного цикла научных теорий}
			\scnrelfrom{класс объектов деятельности}{научная теория}
		\end{scnindent}
		\scnitem{стандартизация}
		\begin{scnindent}
			\scnidtf{поддержка жизненного цикла стандартов}
			\scnrelfrom{класс объектов деятельности}{стандарт}
		\end{scnindent}
		\scnitem{поддержка жизненного цикла технологий}
		\begin{scnindent}
			\scnrelfrom{класс объектов деятельности}{технология}
		\end{scnindent}
		\scnitem{образовательная деятельность}
		\begin{scnindent}
			\scnidtf{учебная деятельность}
			\scnidtf{поддержка жизненного цикла кадровых ресурсов}
			\scnrelfrom{класс объектов деятельности}{кадровый ресурс}
		\end{scnindent}
		\scnitem{поддержка жизненного цикла метасистем комплексного управления поддержкой и обеспечение поддержки жизненного цикла сущностей соответствующих классов}
		\begin{scnindent}
			\scnrelfrom{класс объектов деятельности}{метасистема комплексного управления поддержкой и обеспечением поддержки жизненного цикла сущностей соответствующих классов}
		\end{scnindent}
	\end{scnrelfromlist}

\scnheader{Человеческая деятельность в области Искусственного интеллекта}
\scntext{примечание}{Общая структура \textit{человеческой деятельности} рассматривается путём обобщения структуры \textit{Человеческой деятельности в области Искусственного интеллекта}}

\scnheader{поддержка жизненного цикла}
\scntext{примечание}{\textit{поддержка жизненного цикла} различных социально значимых объектов является особым видом \textit{человеческой деятельности}. 
	\\Во-первых, эффективность \textit{Человеческой деятельности} в целом зависит (1) от длительности социально полезной (активной) фазы жизненного цикла используемых объектов и (2) от объема затрат общества на поддержание необходимых социально полезных свойств используемых объектов. 
	\\Во-вторых, характер и \textit{технология} поддержки жизненного цикла разных видов социально значимых объектов могут существенно отличаться друг от друга. Так, например, существенно отличается организация поддержки жизненного цикла автомобилей, традиционных компьютерных систем различного назначения, современных интеллектуальных компьютерных систем, интероперабельных интеллектуальных компьютерных систем, людей, предприятий, домов, различных юридических лиц, населенных пунктов и других. При этом типология социально значимых объектов, жизненный цикл которых должен поддерживаться, включает в себя самые разнообразные классы объектов - искусственно создаваемые материальные информационные продукты человеческой деятельности, всех людей, всевозможные социальные сообщества и предприятия. Многообразие типов социально значимых объектов порождает многообразие соответствующих им технологий, что усложняет комплексную автоматизацию человеческой деятельности в целом.}
\scntext{примечание}{Заметим, что \textit{видов человеческой деятельности} значительно меньше, чем \textit{областей человеческой деятельности}. Это в определенной степени обусловлено тем, что видов связей между сущностями (относительных понятий) значительно меньше, чем классов различных сущностей. Данное обстоятельство указывает на то, что в основе движения в направление глобальной автоматизации деятельности \textit{общества} должна лежать ориентация на грамотную систематизацию \textit{видов человеческой деятельности}, и на их максимально глубокую \textit{конвергенцию} (как внутри каждого вида деятельности, так и между различными видами). Благодаря этому искусственно привносимое многообразие средств автоматизации \textit{человеческой деятельности} может быть сведено к минимуму.}

\scnheader{следует отличать}
\begin{scnhaselementset}
	\scnitem{научно-исследовательская деятельность}
	\begin{scnindent}
		\scnidtf{поддержка жизненного цикла научных теорий}
	\end{scnindent}
	\scnitem{стандартизация}
	\begin{scnindent}
		\scnidtf{разработка и развитие стандартов}
		\scnidtf{поддержка жизненного цикла стандартов}
	\end{scnindent}
	\scnitem{поддержка жизненного цикла технологий}
\end{scnhaselementset}

\scnheader{научно-исследовательская деятельность}
\scnrelfrom{результат}{Общая теория сущностей заданного класса}
\scntext{определение}{\textit{Научно-исследовательская деятельность} направлена на \textbf{\textit{изучение сущностей заданного класса}}, на изучение принципов, лежащих в основе их структуры и функционирования. В рамках этого вида деятельности важна новизна и конкуренция идей и подходов, важно соотношение между структурой (архитектурой), организацией функционирования исследуемых \textit{сущностей} и общими характеристиками (параметрами) качества этих сущностей, общими предъявляемыми к ним требованиями.
	\\Продуктом рассматриваемой деятельности является \textit{Общая теория сущностей заданного класса}, которая отражает множественность и даже \uline{противоречивость} разных точек зрения и важнейшим направлением развития (эволюции) которой является сближение (\textit{конвергенция}) различных точек зрения и обеспечение совместимости и непротиворечивости между ними.
	\\В основе \textit{научно-исследовательской деятельности} лежит конкуренция точек зрения, принципиальная новизна идей и верифицированных результатов, направленных на выявление и обоснование неочевидных свойств и закономерностей соответствующей \textit{предметной области}, на разработку методов решения различных \textit{классов задач}, решаемых в рамках этой \textit{предметной области}. Цель \textit{научно-исследовательской деятельности} --- требуемая детализация вырабатываемых знаний об объектах исследований соответствующих \textit{предметных областей}.}

\scnheader{стандартизация}
\scntext{определение}{В отличие от \textit{научно-исследовательской деятельности} в основе разработки \textit{стандарта} создаваемых сущностей и разработки соответствующей \textit{технологии} поддержки их жизненного цикла лежит \uline{согласование} различных точек зрения (поиск консенсуса) и максимально возможное их \uline{упрощение} (соблюдение \textit{Принципа Бритвы Оккама}). Необходимость такой методологической установки обусловлена массовым характером \textit{человеческой деятельности} по созданию и \textit{поддержке жизненного цикла} соответствующего класса сущностей и необходимостью вовлечения в эту деятельность людей с \textit{разной} (в том числе и достаточно низкой) квалификацией.
	\\В процессе \textit{разработки стандарта сущностей заданного класса} важна не конкуренция различных точек зрения, а их \textit{конвергенция}, \textit{семантическая совместимость} и глубокая интеграция. Каждый \textit{стандарт} \textit{искусственных сущностей заданного класса} --- это согласованная \uline{на текущий момент} точка зрения (консенсус) о структуре, функционировании, свойствах и закономерностях искусственных сущностей заданного класса, согласованная (общепризнанная) часть \textit{Общей теория искусственных сущностей заданного класса}, доступная для понимания широкому контингенту практиков (инженеров), которые проектируют, производят и поддерживают весь жизненный цикл конкретных \textit{искусственных сущностей заданного класса}.}

\scnheader{жизненный цикл технологий}
\scntext{примечание}{При создании и \textit{поддержке жизненного цикла технологий} должны учитываться ряд требований, предъявляемых к \uline{любым} \textit{технологиям}.}
\begin{scnindent}
	\begin{scnrelfromlist}{требование}
		\scnitem{комплексность}
		\begin{scnindent}
			\scntext{пояснение}{максимально возможное покрытие всех задач, которые должны решаться с помощью \textit{технологии} (как минимум всех этапов жизненного цикла)}
		\end{scnindent}
		\scnitem{простота}
		\begin{scnindent}
			\scntext{пояснение}{максимально возможная простота в использовании \textit{технологии} (необходимая полнота документации, интеллектуальная help-поддержка, отсутствие лишней информации, которая не является необходимой для использования \textit{технологии}, наличие богатой и систематизированной библиотеки типовых многократно используемых решений)}
		\end{scnindent}
	\end{scnrelfromlist}
\end{scnindent}

\scnheader{общество}
\scntext{определение}{общество --- это иерархическая система взаимодействующих индивидуальных и коллективных субъектов, каждый из которых:
\begin{itemize}
	\item Производит либо часть социально значимой продукции, производимой коллективным субъектом, в состав которого входит данный субъект, либо целостный социально-значимый продукт (производимый товар), потребляемый другими внешними субъектами или оказывает некоторую услугу другому субъекту, направленную на обеспечение жизнедеятельности и совершенствование этого другого субъекта.
	\item Потребляет продукцию, произведенную другими субъектами, необходимую для производства собственной продукции (сырье и оборудование), а также необходимую для обеспечения своей жизнедеятельности.
	\item Потребляет услуги, оказываемые другими субъектами необходимые для производства собственной продукции или услуг, а также необходимые для совершенствования своей деятельности.
\end{itemize}}

\scnheader{автоматизация человеческой деятельности}
\begin{scnrelfromlist}{направление}
	\scnfileitem{автоматизация социально полезной профессиональной деятельности всех субъектов деятельности (как индивидуальных субъектов --- всех физических лиц, так и всевозможных коллективных --- корпоративных субъектов, в том числе юридических лиц).}
	\scnfileitem{автоматизация обеспечения (создания) комфортных условий для всех субъектов деятельности общества на основе мониторинга деятельности и конкретного (адаптированного) содействия эволюции каждого субъекта с учетом его непосредственных потребностей и проблем.}
\end{scnrelfromlist}
\scntext{текущее состояние}{Организация взаимодействий каждого субъекта с внешней средой должна осуществляться как со стороны этого субъекта, так и со стороны указанной внешней среды (то есть со стороны общества). \textit{общество} должно повернуться \scnqqi{лицом} к каждому субъекту и не бросать его на произвол судьбы. В настоящее время создание (обеспечение) условий субъектов деятельности общества отдано на откуп каждого такого субъекта. Общество в лице специально предназначенных для этого других субъектов оказывает услуги и снабжает товарами \uline{по заказу} (по инициативе) нуждающегося в этом субъекта. Таким образом, ответственность за развитие каждого субъекта деятельности ложится исключительно на \scnqqi{плечи} этого субъекта. Поддержка общества носит общий характер и никак не учитывает особенности текущего положения каждого субъекта.
    \\Важнейшей причиной, препятствующей дальнейшему повышению общего уровня автоматизации человеческой деятельности является то, что автоматизация различных областей человеческой деятельности осуществляется \uline{локально}.
    \\На современном этапе применения интеллектуальных компьютерных систем основной проблемой является не автоматизация локальных видов и областей человеческой деятельности, а автоматизация комплексных процессов человеческой деятельности, требующая \textit{интеграции} в априори \uline{непредсказуемых} комбинациях самых различных информационных ресурсов и самых различных автоматизированных сервисов, реализуемых в виде специализированных интеллектуальных компьютерных систем.
    \\Локальность автоматизации человеческой деятельности приводит к тому, что вся человеческая деятельность приобретает облик \scnqqi{архипелага}, состоящего из хорошо автоматизированных \scnqqi{островов}, но соединяемых между собой \scnqqi{вручную}. Это \scnqqi{ручное} не автоматизируемое соединение указанных \scnqqi{островов} полностью зависит от человеческого фактора и квалификации соответствующих исполнителей.
    \\Указанное \scnqqi{ручное} соединение некоторого множества семантически близких автоматизированных областей человеческой деятельности можно автоматизировать, но делать это надо очень грамотно на высоком уровне системной культуры и на фундаментальной основе общей теории человеческой деятельности.}
    \scntext{проблема}{Важная причина, препятствующая дальнейшему повышению общего уровня автоматизации общества заключается в том, что автоматизация различных областей человеческой деятельности осуществляется без выявления и глубокого анализа сходства некоторых видов деятельности в разных областях и соответственно без сближения, \textbf{\textit{конвергенции}} и \textbf{\textit{унификации}} этих \textit{видов деятельности}.}
    \scntext{примечание}{Важнейшим направлением повышения уровня автоматизации человеческой деятельности является переход к автоматизации все более и более \uline{комплексных} (крупных) видов и областей человеческой деятельности например, от автоматизации деятельности различных предприятий, организаций, хозяйственных служб к автоматизации деятельности города в целом).
    \\Автоматизации комплексных видов человеческой деятельности требует создания комплекса активно взаимодействующих компьютерных систем, каждая из которых обеспечивает автоматизацию соответствующего частного вида человеческой деятельности, входящего в состав автоматизируемого комплексного вида деятельности. При этом число уровней иерархии автоматизируемых видов человеческой деятельности ничем не ограничивается. Очевидно, что уровень автоматизации комплексных видов человеческой деятельности определяется:
    \begin{itemize}
        \item уровнем конвергенции (сближения, совместимости) соответствующих частных видов деятельности;
        \item качеством интеграции этих частных видов деятельности;
        \item уровнем конвергенции компьютерных систем, обеспечивающих автоматизацию указанных частных видов деятельности;
        \item качеством взаимодействия этих компьютерных систем то есть уровнем интероперабельности этих систем).
    \end{itemize}}
\scntext{примечание}{Уровень эволюции общества во многом зависит от уровня автоматизации человеческой деятельности, от уровня развития соответствующих технологий такой автоматизации. Но эта зависимость выглядит значительно сложнее чем, кажется на первый взгляд, особенно, если речь идет об автоматизации не физической, интеллектуальной человеческой деятельности (как индивидуальной, так и коллективной). Безграмотная, а тем более социально безответственная или злонамеренная автоматизация информационной деятельности общества способны нанести огромный ущерб его развитию. Такая безграмотность и безответственность, например, приводит к таким побочным факторам, как компьютерная зависимость, виртуализация окружающей среды, поверхностный характер мышления, снижение познавательной мотивации и активности и многое другое.}
\begin{scnindent}
	\begin{scnrelfromlist}{следовательно}
		\scnfileitem{необходимо существенно повысить уровень социальной ответственности у разработчиков компьютерных систем и соответствующих технологий.}
		\scnfileitem{Опасность от безграмотного, социально безответственного и тем более злонамеренного внедрения интеллектуальных компьютерных систем нового поколения может иметь для человечества летальный характер.}
	\end{scnrelfromlist}
\end{scnindent}
	
\scnheader{человеческое общество}
\scntext{направления развития}{Если рассматривать \textit{общество} как \textit{многоагентную систему}, состоящую из самостоятельных интеллектуальных агентов, то, очевидно, что важнейшими факторами, определяющими повышение качества (уровня развития) \textit{общества} являются:
    \begin{scnitemize}
        \item повышение эффективности использования опыта, накопленного \textit{обществом}, эффективности использования человечеством \textit{знаний} и \textit{навыков}
        \item повышение темпов приобретения, накопления и систематизации эффективно используемым человечеством \textit{знаний} и \textit{навыков}.
    \end{scnitemize}
    Решение указанных проблем становится вполне возможным, если для этого использовать \textit{интеллектуальные компьютерные системы нового поколения}, с помощью которых накапливаемые человечеством \textit{знания} и \textit{навыки} будут организованы как систематизированная распределенная библиотека многократно используемых информационных ресурсов (\textit{знаний} и \textit{навыков}).}
\begin{scnindent}
    \scntext{следовательно}{Систематизация и автоматизация многократного использования накапливаемых человечеством информационных ресурсов требует их конвергенции, глубокой интеграции и формализации. Особое место в этом процессе занимает математика, как основа систематизации и формализации знаний и навыков на уровне формальных \textit{онтологий верхнего уровня}.}
\end{scnindent}
	
\end{scnsubstruct}

\end{scnsubstruct}

\end{scnsubstruct}

    \end{scnsubstruct}
    \begin{scnrelfromvector}{заключение}
        \scnfileitem{Благодаря тому, что \textit{интеллектуальные компьютерные системы нового поколения} становятся самостоятельными и активными субъектами \textit{человеческой деятельности} в достаточной степени равноправными людям (естественным индивидуальным субъектам человеческой деятельности), характер и, соответственно, уровень автоматизации \textit{человеческой деятельности} существенно меняется --- снимается необходимость \uline{управлять} средствами автоматизации, поскольку такое "ручное"{} управление заменяется распределением обязанностей и ответственности между людьми и \textit{интеллектуальными компьютерными системами нового поколения}.}
        \scnfileitem{Если автоматизация \uline{любого} вида в любой области \textit{человеческой деятельности} будет осуществляться с помощью \textit{интеллектуальных компьютерных систем нового поколения} и если \textit{интеллектуальные компьютерные системы нового поколения}, обеспечивающие автоматизацию \uline{разных} видов и областей \textit{человеческой деятельности}, будут содержательно взаимодействовать между собой, то общий уровень автоматизации \textit{человеческой деятельности} существенно возрастет благодаря тому, что отпадет необходимость \uline{вручную} координировать использование различных средств автоматизации.}
        \scnfileitem{Эффективность и трудоемкость автоматизации различных видов и областей \textit{человеческой деятельности} будет существенно определяться степенью \textbf{\textit{конвергенции}} между различными видами и областями \textit{человеческой деятельности}. Необходимо построить иерархическую модель \textit{человеческой деятельности,} в рамках которой должна быть проведена грамотная систематизация и стратификация всех видов и областей \textit{человеческой деятельности}, направленная против излишнего эклектического многообразия. 
        	\\Таким образом, прежде, чем осуществлять комплексную автоматизации \textit{человеческой деятельности} с помощью \textit{интеллектуальных компьютерных систем} \textit{нового поколения}, необходимо с позиций общей теории систем переосмыслить организацию этой деятельности. В противном случае автоматизация беспорядка приведет к еще большему беспорядку.}
        \scnfileitem{Особо подчеркнем то, что многие из рассмотренных нами проблем текущего состояния и направлений дальнейшего развития \textit{Человеческой деятельности в области Искусственного интеллекта} аналогичны проблемам и тенденциям развития многих других научно-технических дисциплин. Следовательно, подходы к решению этих проблем могут носить междисциплинарный характер.}
        \scnfileitem{Время каждого человека является главным невосполнимым ресурсом общества и тратить его надо не на рутинную поддержку жизненного цикла всевозможных социально значимых объектов, а на комплексное развитие соответствующих \textit{технологий}. Автоматизация человеческой деятельности с помощью глобальной системы интероперабельных семантически совместимых и активно взаимодействующих \textit{интеллектуальных компьютерных систем} в самых разных областях \textit{человеческой деятельности} позволит существенно сократить время каждого человека на выполнение рутинной, легко автоматизируемой деятельности. Человеческая деятельность должна стать ориентированной на максимально возможную самореализацию, раскрытие \uline{творческого} потенциала каждого человека, направленного на ускорение темпов повышения уровня интеллекта всего общества.}
        \scnfileitem{Необходимо создать \textit{Глобальную экосистему интеллектуальных компьютерных систем нового поколения}.}
        \begin{scnindent}
            \begin{scnrelfromlist}{требование}
                \scnfileitem{Построение формальной модели \textit{человеческой деятельности}.}
                \scnfileitem{Переход от эклектичного построения сложных \textit{интеллектуальных компьютерных систем}, использующих различные виды \textit{знаний} и различные виды \textit{моделей решения задач}, к их глубокой \textit{интеграции} и \textit{унификации}, когда одинаковые модели представления и модели обработки знаний реализуется в разных системах и подсистемах одинаково.}
                \scnfileitem{Сокращение дистанции между современным уровнем \textit{теории интеллектуальных компьютерных систем} и практики их разработки.}
                \scnfileitem{Разработку грамотной тактики и стратегии переходного периода, в рамках которого современные \textit{интеллектуальные компьютерные системы} должны постепенно заменяться на \textit{интеллектуальные компьютерные системы нового поколения}, которые должны эффективно взаимодействовать не только между собой, но и с хорошо зарекомендовавшими себя современными информационными ресурсами и сервисами.}
            \end{scnrelfromlist}
        \end{scnindent}
\end{scnrelfromvector}
\bigskip
\scnendcurrentsectioncomment
\end{SCn}


\scsubsection[
    \protect\scneditor{Загорский А.Г.}
    \protect\scnmonographychapter{Глава 7.2. Экосистема интеллектуальных компьютерных систем нового поколения (Экосистема OSTIS) и реализация рынка знаний на ее основе}
    ]{Логико-семантическая модель интеграции разнородных информационных ресурсов и сервисов в Экосистеме OSTIS в процессе ее расширения}
\label{services_integr_logical_model}

\scsubsubsection[
    \protect\scneditor{Банцевич К.А.}
    \protect\scnmonographychapter{Глава 7.2. Экосистема интеллектуальных компьютерных систем нового поколения (Экосистема OSTIS) и реализация рынка знаний на ее основе}
    ]{Предметная область и онтология библиографических источников и других информационных ресурсов}
\label{sd_bibliography}

\scsubsection[
    \protect\scnmonographychapter{Глава 7.2. Экосистема интеллектуальных компьютерных систем нового поколения (Экосистема OSTIS) и реализация рынка знаний на ее основе}
    ]{Предметная область и онтология семантически совместимых интеллектуальных ostis-порталов научных знаний}
\label{sd_portals}
\begin{SCn}
    \scnsectionheader{Предметная область и онтология семантически совместимых интеллектуальных ostis-порталов научных знаний}
    \begin{scnsubstruct}
    \scnrelfrom{дочерний раздел}{\nameref{ims_ostis_model}}
    \scniselement{раздел базы знаний}
    \scnhaselementrole{ключевой sc-элемент}{Предметная область семантически совместимых интеллектуальных порталов научно-технических знаний}

    \begin{scnrelfromlist}{библиографическая ссылка}
        \scnitem{\scncite{Van2005}}
        \scnitem{\scncite{Mack2001}}
    \end{scnrelfromlist}
    
    \scnheader{Предметная область семантически совместимых интеллектуальных порталов научно-технических знаний}
    \scniselement{предметная область}
    \begin{scnhaselementrolelist}{максимальный класс объектов исследования}
        \scnitem{портал научных знаний}
    \end{scnhaselementrolelist}
    \begin{scnhaselementrolelist}{класс объектов исследования}
        \scnitem{портал знаний}
        \scnitem{ostis-портал знаний}
    \end{scnhaselementrolelist}
    
    \scnheader{портал научных знаний}
    \scntext{примечание}{Понятие \textit{портала знаний} представляет собой один из способов создания централизованного доступа к информации, которая может быть необходима для решения задач, связанных с работой в организации. Такие порталы могут содержать информацию, связанную с процессами, документацией, процедурами, обучающими материалами, а также ответы на часто задаваемые вопросы.}
    \begin{scnindent}
        \begin{scnrelfromset}{смотрите}
            \scnitem{\scncite{Van2005}}
            \scnitem{\scncite{Mack2001}}
        \end{scnrelfromset}
    \end{scnindent}
    \scntext{примечание}{Одним из ключевых преимуществ \textit{порталов знаний} является их способность к сбору и хранению информации из различных источников, таких как базы данных, системы управления документами, системы управления проектами и так далее. Это позволяет пользователям получать полную и актуальную информацию в одном месте.}
    \scntext{примечание}{На основе \textit{портала знаний} обеспечивается возможность взаимодействия между пользователями путем создания форумов, обсуждений и коллективного редактирования документов. Это может способствовать обмену знаниями и опытом между сотрудниками организации и повысить эффективность их работы.}
    \scntext{примечание}{При создании \textit{порталов знаний} возникают проблемы, связанные с организацией и управлением информацией. Например, необходимо обеспечить корректное и структурированное хранение информации, ее поиск и обновление. Также необходимо учитывать потребности пользователей и обеспечить удобный и интуитивно понятный интерфейс.}
    \begin{scnrelfromlist}{цель}
        \scnfileitem{Ускорение погружения каждого человека в новые для него научные области при постоянном сохранении общей целостной картины Мира (образовательная цель).}
        \scnfileitem{Фиксация в систематизированном виде новых научных результатов так, чтобы все основные связи новых результатов с известными были четко обозначены.}
        \scnfileitem{Автоматизация координации работ по рецензированию новых результатов.}
        \scnfileitem{Автоматизация анализа текущего состояния базы знаний.}
    \end{scnrelfromlist}
    \scntext{пояснение}{Создание интеллектуальных \textbf{\textit{порталов научных знаний}}, обеспечивающих повышение темпов интеграции и согласования различных точек зрения, --- это способ существенного повышения темпов эволюции научно-технической деятельности.\\
        Совместимые \textbf{\textit{порталы научных знаний}}, реализованные в виде \textit{ostis-систем}, входящих в \textit{Экосистему OSTIS}, являются основой новых принципов организации научной деятельности, в которой
        \begin{scnitemize}
            \item результатами этой деятельности являются не статьи, монографии, отчеты и другие научно-технические документы, а фрагменты глобальной базы знаний, разработчиками которых являются свободно формируемые научные коллективы, состоящие из специалистов в соответствующих научных дисциплинах,
            \item с помощью \textbf{\textit{порталов научных знаний}} осуществляется
            \begin{scnitemizeii}
                \item координация процесса рецензирования новой научно-технической информации, поступающей от научных работников в базы знаний этих порталов,
                \item процесс согласования различных точек зрения ученых (в частности, введению и семантической корректировке понятий, а также введению и корректировке терминов, соответствующих различным сущностям).
            \end{scnitemizeii}
        \end{scnitemize}
        Реализация семейства семантически совместимых порталов научных знаний в виде совместимых \textit{\mbox{ostis-систем}}, входящих в состав \textit{Экосистемы OSTIS}, предполагает разработку иерархической системы семантически согласованных формальных онтологий, соответствующих различным научно-техническим дисциплинам, с четко заданным наследованием свойств описываемых сущностей и с четко заданными междисциплинарными связями, которые описываются связями между соответствующими формальными онтологиями и специфицируемыми ими предметными областями.\\
        Реализация \textbf{\textit{порталов научных знаний}} в виде семейства семантически совместимых \textit{ostis-систем} означает также попытку преодолеть вавилонское столпотворение\ многообразия научно-технических языков, не меняя сути научно-технических знаний, а сводя эти знания к единой универсальной форме смыслового представления знаний в памяти порталов научных знаний, т.е. к форме которая в достаточной степени понятна как \textit{ostis-системам}, так и любым потенциальным их пользователям.}
    \scntext{пример}{Примером \textbf{\textit{портала научных знаний}}, построенного в виде \textit{ostis-системы} является \textit{Метасистема OSTIS}, содержащая все известные на текущий момент знания и навыки, входящие в состав \textit{Технологии OSTIS}.}
    \begin{scnrelfromlist}{преимущества}
        \scnfileitem{Использование методов семантической обработки информации, что позволяет более точно и эффективно организовывать и искать информацию на портале знаний.}
        \scnfileitem{Высокий уровень гибкости и расширяемости, что позволяет адаптировать \textit{ostis-порталы знаний} под различные нужды и требования пользователей.}
        \scnfileitem{Автоматическая интеграция \textit{ostis-порталов знаний} с другими \textit{ostis-системами} в рамках \textit{Экосистемы OSTIS}, что позволяет создать централизованный доступ к информации из различных источников.}
        \scnfileitem{Возможность создания персонализированного \textit{ostis-портала знаний}, который учитывает интересы и потребности каждого пользователя, что позволяет более эффективно использовать знания \textit{ostis-систем}.}
        \scnfileitem{Возможность производить \textit{ostis-порталы знаний} быстро и с минимальными затратами благодаря использованию существующих компонентов и инструментов.}
    \end{scnrelfromlist}
        \scntext{примечание}{Реализация \textit{порталов знаний} на основе \textit{Технологии OSTIS} позволяет создать более эффективную и гибкую систему для хранения, организации и поиска знаний, которая может быть адаптирована под различные требования пользователей и организаций.}

    \end{scnsubstruct}
    \scnendcurrentsectioncomment
\end{SCn}


\scsubsubsection[
    \protect\scneditors{Шункевич Д.В.;Банцевич К.А.;Садовский М.Е.;Бутрин С.В.}
    \protect\scnmonographychapter{Глава 7.3. Метасистема OSTIS и Стандарт OSTIS}
    ]{Логико-семантическая модель Метасистемы OSTIS}
\label{ims_ostis_model}
\begin{SCn}
    \scnsectionheader{Логико-семантическая модель Метасистемы OSTIS}
    \begin{scnsubstruct}
    	\scntext{аннотация}{Данный раздел посвящен рассмотрению подхода к автоматизации процессов создания, развития и применения стандартов на основе Технологии OSTIS. Также в разделе сформулированы основные принципы стандартизации интеллектуальных компьютерных систем, методов и средств их проектирования в рамках предлагаемого подхода.}
    	
    	\begin{scnreltovector}{конкатенация сегментов}
    		\scnitem{Сегмент. Введение в Логико-семантическую модель Метасистемы OSTIS}
    		\scnitem{Сегмент. Структура, назначение, особенности и достоинства Метасистемы OSTIS}
    	\end{scnreltovector}
    	
    	\begin{SCn}
	\scnsectionheader{Сегмент. Ввдение в Логико-семантическую модель Метасистемы OSTIS}
	
	\begin{scnsubstruct}
		\scnheader{Логико-семантическая модель Метасистемы OSTIS}
		\begin{scnrelfromlist}{введение}
		\scnfileitem{В основе каждой развитой сферы человеческой деятельности лежит ряд стандартов, формально описывающих различные ее аспекты --- систему понятий (включая терминологию), типологию и последовательность действий, выполняемых в процессе применения соответствующих методов и средств.}
		\begin{scnindent}
			\scnrelfrom{источник}{\scncite{Golenkov2019}}
		\end{scnindent}
		\scnfileitem{Стандарты в самых различных областях являются важнейшим видом знаний, главной целью которых является обеспечение совместимости различных видов деятельности. Несмотря на развитие информационных технологий, в настоящее время подавляющее большинство стандартов представлено либо в виде традиционных линейных документов, либо в виде web-ресурсов, содержащих набор статических страниц, связанных гиперссылками. Для того чтобы стандарты выполняли свою главную функцию, они должны постоянно совершенствоваться.}
		\scnfileitem{Текущее оформление стандартов имеет ряд недостатков, которые мешают эффективному и грамотному использованию стандартов в различных областях:
		\begin{scnitemize}
			\item{дублирование информации в рамках документа, описывающего стандарт;}
			\item{трудоемкость сопровождения самого стандарта, обусловленная в том числе дублированием информации, в частности, трудоемкость изменения терминологии;}
			\item{проблема интернационализации стандарта --- фактически перевод стандарта на несколько языков приводит к необходимости поддержки и согласования независимых версий стандарта на разных языках;}
			\item{неудобство применения стандарта, в частности, трудоемкость поиска необходимой информации и, как следствие, трудоемкость изучения стандарта;}
			\item{несогласованность формы различных стандартов между собой, как следствие --- трудоемкость автоматизации процессов развития и применения стандартов;}
			\item{трудоемкость автоматизации проверки соответствия объектов или процессов требования того или иного стандарта;}
			\item{и другие.}
		\end{scnitemize}
		Перечисленные проблемы связаны в основном с формой представления стандартов.}
		\begin{scnindent}
			\begin{scnrelfromlist}{источник}
				\scnitem{\scncite{Serenkov2004}}
				\scnitem{\scncite{Uglev2012}}
			\end{scnrelfromlist}
		\end{scnindent}
		\scnfileitem{Задачей любого стандарта в общем случае является описание согласованной системы понятий (и соответствующих терминов), бизнес-процессов, правил и других закономерностей, способов решения определенных классов задач и так далее. Для формального описания информации такого рода с успехом применяются онтологии. Более того, в настоящее время в ряде областей вместо разработки стандарта в виде традиционного документа разрабатывается соответствующая онтология. Такой подход дает очевидные преимущества в плане автоматизации процессов согласования и использования стандартов.}
		\scnfileitem{Актуальной остается проблема, связанная не с формой, а с сутью (семантикой) стандартов --- проблема несогласованности системы понятий и терминов между различными стандартами, которая актуальна даже для стандартов в рамках одной и той же сферы деятельности.}
		\scnfileitem{В настоящее время Информатика преодолевает важнейший этап своего развития — переход от информатики данных (data science) к информатике знаний (knowledge science), где акцентируется внимание на семантических аспектах представления и обработки знаний. Без фундаментального анализа такого перехода невозможно решить многие проблемы, связанные с управлением знаниями, экономикой знаний, с семантической совместимостью интеллектуальных компьютерных систем.}
		\scnfileitem{С семантической точки зрения каждый стандарт есть иерархическая онтология, уточняющих структуру и систем понятий соответствующих им предметных областей, которая описывает структуру и функционирование либо некоторого класса технических или иных искусственных систем, либо некоторого класса организаций, либо некоторого вида деятельности.}
		\scnfileitem{Наиболее перспективным подходом к решению перечисленных проблем является преобразование каждого конкретного стандарта в базу знаний, в основе которой лежит набор онтологий, соответствующих данному стандарту. Такой подход позволяет в значительной мере автоматизировать процессы развития стандарта и его применения. В рамках Технологии OSTIS данный подход используется при построении Стандарта OSTIS.}
		\scnfileitem{Предлагаемый Стандарт OSTIS оформлен в виде семейства разделов базы знаний специальной интеллектуальной компьютерной Метасистемы OSTIS (Intelligent MetaSystem for ostis-systems), которая построена по Технологии OSTIS и представляет собой постоянно совершенствуемый интеллектуальный портал научно-технических знаний, который поддерживает перманентную эволюцию Стандарта OSTIS, а также разработку различных ostis-систем (интеллектуальных компьютерных систем, построенных по Технологии OSTIS).}
		\begin{scnindent}
			\scnrelfrom{источник}{\scncite{IMS2017}}
		\end{scnindent}
		\end{scnrelfromlist}
		
		
		\bigskip
	\end{scnsubstruct}
	\scnsourcecomment{Завершили \scnqqi{Сегмент. Ввдение в Логико-семантическую модель Метасистемы OSTIS}}
\end{SCn}
    	\begin{SCn}
	\scnsectionheader{Сегмент. Структура, назначение, особенности и достоинства Метасистемы OSTIS}
	
	\begin{scnsubstruct}
		
	\scnheader{Метасистема OSTIS}
	\scniselement{ostis-система}
	\scntext{назначение}{Эффективность любой технологии, в том числе и \textit{\textbf{Технологии OSTIS}} определяется не только сроками создания искусственных систем соответствующего класса, но и темпами совершенствования самой технологии (темпами совершенствования средств автоматизации и темпами совершенствования системы стандартов, лежащих в основе технологии).\\
	Для фиксации текущего состояния \textit{Технологии OSTIS}, а также для организации ее эффективного использования и ее перманентного совершенствования с участием ученых, работающих в области искусственного интеллекта, и инженеров, разрабатывающих семантические компьютерные системы различного назначения, в состав \textit{Экосистемы OSTIS} вводится \textit{Метасистема OSTIS} (\cite{IMS}), назначение которой делает ее \uline{ключевой} \textit{ostis-системой} в рамках \textit{Экосистемы OSTIS}.}
	\scnrelto{форма реализации}{Технология OSTIS}
	\scnidtf{Интеллектуальная Метасистема, являющаяся формой (вариантом) реализации (представления, оформления) \textit{Технологии OSTIS} в виде \textit{ostis-системы}}
	\scntext{примечание}{Тот факт, что \textit{Технология OSTIS} реализуется в виде \textit{ostis-системы}, является весьма важным для эволюции \textit{Технологии OSTIS}, поскольку методы и средства эволюции (перманентного совершенствования) \textit{Технологии OSTIS} становятся фактически совпадающими с методами и средствами разработки любой (!) \textit{ostis-системы} на всех этапах их жизненного цикла. Другими словами, эволюция Технологии OSTIS осуществляется методами и средствами самой этой технологии.}
	\scnidtf{Система комплексной автоматизации (информационной и инструментальной поддержки) проектирования и реализации ostis-систем, которая сама реализована также в виде ostis-системы}
	\scnidtf{Портал знаний по \textit{Технологии OSTIS}, интегрированный с с.а.п.р. ostis-систем и реализованный в виде ostis-системы}
	\scniselement{портал научно-технических знаний}
	\scnidtf{Интеллектуальная метасистема комплексной информационной и инструментальной поддержки проектирования совместимых семантических компьютерных систем, которая является формой реализации общей теории и технологии проектирования семантических компьютерных систем и которая поддерживает высокий темп эволюции указанной теории и технологии}
	\scnidtf{Интеллектуальная метасистема, построенная по стандартам Технологии OSTIS и предназначенная (1) для инженеров \textit{ostis-систем} --- для поддержки проектирования, реализации и обновления (реинжиниринга) \textit{ostis-систем}, и для разработчиков \textit{Технологии OSTIS} --- для поддержки коллективной деятельности по развитию стандартов и библиотек \textit{Технологии OSTIS}}
	\scnidtf{интеллектуальная система, предназначенная для комплексной информационной и инструментальной поддержки проектирования семантически совместимых компьютерных систем, на назначение которых не накладывается никаких ограничений}
	\scnidtf{Универсальная базовая (предметно-независимая) ostis-система автоматизации проектирования ostis-систем любых ostis-систем}
	\scnidtf{Intelligent MetaSystem for intelligent systems design}
	\scnidtf{МетасOSTIS}
	\scnidtf{ostis-система автоматизации проектирования ostis-систем}
	\scnidtf{Фреймворк интеллектуальных систем}
	\scnidtf{Интеллектуальная метасистема комплексной поддержки проектирования совместимых семантических компьютерных систем по Технологии OSTIS}
	\scnidtf{Фреймворк ostis-систем}
	\scnidtf{Фреймворк OSTIS}
	\scntext{url}{http://OSTIS.net}
	\scntext{назначение}{\textit{Метасистема OSTIS} является в \textit{Экосистеме OSTIS} ключевой интеллектуальной системой, которая поддерживает не только проектирование новых интеллектуальных систем и не только замену устаревших компонентов в интеллектуальных системах, входящих в состав \textit{Экосистемы OSTIS}, но и включение (интеграцию) в состав \textit{Экосистемы OSTIS} новых создаваемых интеллектуальных систем.\\
		\textit{Метасистема OSTIS} ориентирована на разработку и практическое внедрение методов и средств \textbf{компонентного проектирования} семантически совместимых интеллектуальных систем, которая предоставляет возможность быстрого создания интеллектуальных приложений различного назначения. Подчеркнем при этом, что сферы практического применения методики компонентного проектирования семантически совместимых интеллектуальных систем ничем не ограничены.}
	\scnidtf{реализация технологии проектирования семантически совместимых компьютерных систем в виде метасистемы, построенной по той же технологии и обеспечивающей комплексную информационную и инструментальную поддержку проектирования семантически совместимых компьютерных систем}
	\scntext{примечание}{При разработке \textit{Teхнологии OSTIS} средством автоматизации этой деятельности является не вся \textit{Метасистема OSTIS}, а только ее часть --- входящая в состав \textit{Метасистемы OSTIS}, \textit{Встраиваемая ostis-система поддержки реижиниринга ostis-систем}, которая поддерживает деятельность разработчиков базы знаний \textit{Метасистемы OSTIS}. Это обусловлено тем, что вся деятельность по разработке \textit{Teхнологии OSTIS} сводится к разработке, инжинирингу) и обновлению (совершенствованию, реинжинирингу) \textit{Базы знаний Метасистемы OSTIS}).}
	\scntext{новизна}{Новизна \textit{Метасистемы OSTIS} заключается в унификации представления различного вида информации в памяти компьютерных систем на основе смыслового (семантического) представления этой информации, что обеспечивает:
		\begin{scnitemize}
			\item устранение дублирования одной и той же информации в разных интеллектуальных системах и в разных компонентах одной и той же системы;
			\item семантическую совместимость различных компонентов интеллектуальных систем и различных интеллектуальных систем в целом;
			\item существенное расширение библиотек совместимых многократно используемых компонентов компьютерных систем за счет крупных\ компонентов и, в частности, типовых подсистем.
	\end{scnitemize}}
	\begin{scnrelfromlist}{обеспечивает}
		\scnfileitem{Комплексную информационную поддержку всех этапов \textit{жизненного цикла интеллектуальных компьютерных систем нового поколения}.}
		\scnfileitem{Автоматизацию проектирования всех компонентов \textit{интеллектуальных компьютерных систем нового поколения}.}
		\scnfileitem{Комплексную автоматизацию всех этапов жизненного цикла \textit{интеллектуальных компьютерных систем нового поколения}.}
	\end{scnrelfromlist}
	\scntext{примечание}{\textit{Метасистема OSTIS} --- метасистема, являющаяся:
		\begin{scnitemize}
			\item{\textit{корпоративной ostis-системой}, обеспечивающей организацию (координацию) деятельности \textit{Консорциума OSTIS}.}
			\item{формой представления реализации и фиксации текущего состояния \textit{Ядра Технологии OSTIS}.}
			\item{\textit{корпоративной ostis-системой}, взаимодействующей со всеми корпоративными ostis-системами, каждая из которых координирует развитие соответствующей специализированной \textit{ostis-технологии}.}
	\end{scnitemize}}
	\scnrelto{продукт}{Проект OSTIS}
	
	\scnheader{Проект OSTIS}
	\begin{scnrelfromset}{подзадачи}
		\scnfileitem{Разработать \textit{Метасистему OSTIS}, обеспечивающую быстрое компонентное проектирование семантически совместимых компьютерных систем различного назначения.}
		\scnfileitem{Разработать методы и средства, обеспечивающие интенсивное развитие рынка семантически совместимых прикладных интеллектуальных систем, созданных на основе \textit{Метасистемы OSTIS}.}
		\scnfileitem{Разработать методы и средства, обеспечивающие стимулирование интенсивного развития самой \textit{Метасистемы OSTIS}.}
	\end{scnrelfromset}
	\scntext{принципы организации}{Организация \textit{Проекта OSTIS} реализуется в форме взаимодействия \textit{Метасистемы OSTIS} с его пользователями и основана на следующих принципах:\begin{scnitemize}
			\item Решатель задач и пользовательский интерфейс \textit{Метасистемы OSTIS} обеспечивают поддержку всего комплекса проектных задач, решаемых разработчиками прикладных интеллектуальных систем, а также разработчиками самой \textit{Метасистемы OSTIS}.
			\item Для стимулирования развития рынка совместимых прикладных интеллектуальных систем, разработанных с помощью \textit{Метасистемы OSTIS}, и развития самой этой метасистемы используются технические средства анализа и оценки объекта и значимости персонального вклада каждого разработчика в специальных условных единицах.
			\item Для стимулирования развития рынка совместимых прикладных интеллектуальных систем, разработанных с помощью \textit{Метасистемы OSTIS}, за каждую такую интеллектуальную систему, зарегистрированную и специфицированную в рамках \textit{Метасистемы OSTIS}, разработчикам выделяется вознаграждение в используемых условных единицах после того, как эта прикладная система будет протестирована на предмет семантической совместимости с другими системами, разработанными с помощью \textit{Метасистемы OSTIS}. При этом \textit{Метасистема OSTIS} становится площадкой для рекламы и распространения интеллектуальных систем, разработанных с ее помощью.
			\item Стимулирование развития самой \textit{Метасистемы OSTIS} осуществляется следующим образом. Участие в развитии \textit{Метасистемы OSTIS} носит открытый характер, для чего достаточно соответствующим образом зарегистрироваться. Авторские права каждого разработчика \textit{Метасистемы OSTIS} защищаются и каждый его вклад в зависимости от его ценности автоматически измеряется и фиксируется в используемых условных единицах.\item Участие в развитии \textit{Метасистемы OSTIS} может иметь самые различные формы (в простейшем случае, это может быть указание на конкретные ошибки, на конкретные трудности, с которыми пользователь столкнулся, формулировка конкретных пожеланий; более сложным вкладом является добавление в базу знаний метасистемы новых знаний, новых компонентов в библиотеку многократно используемых компонентов). При этом автор нового многократно используемого компонента, включенного в библиотеку \textit{Метасистемы OSTIS}, может выбрать любую лицензию для его распространения и, в том числе, назначить ему любую цену.
			\item Ознакомление зарегистрированными пользователями с \textit{Метасистемой OSTIS} носит бесплатный открытый характер. При коммерческой разработке прикладных интеллектуальных систем стоимость каждого обращения к библиотекам \textit{Метасистемы OSTIS} вполне доступна, но существенно снижается в зависимости от степени активности пользователя в развитии \textit{Метасистемы OSTIS}. Это еще один механизм стимулирования участия в развитии \textit{Метасистемы OSTIS}.
		\end{scnitemize}
		Таким образом, указанные принципы организации \textit{Метасистемы OSTIS} обеспечивают на постоянной основе привлечение к разработке \textit{Метасистемы OSTIS} и к формированию рынка семантически совместимых прикладных интеллектуальных систем неограниченные научные, технические и финансовые ресурсы и, в частности, привлечение любых специалистов, желающих участвовать в этом открытом проекте.}
	
	\scnheader{Метасистема OSTIS}
	\scntext{принципы реализации}{Принципы технической реализации \textit{Метасистемы OSTIS} полностью совпадают с принципами технической реализации прикладных интеллектуальных систем, разрабатываемых с помощью этой метасистемы.}
	\begin{scnrelfromset}{декомпозиция}
		\scnitem{полное описание самой Технологии OSTIS}
		\scnitem{история эволюции Технологии OSTIS}
		\scnitem{описание правил использования Технологии OSTIS}
		\scnitem{описание организационной инфраструктуры, направленной на развитие Технологии OSTIS}
		\scnitem{библиотека многократно используемых компонентов ostis-систем}
		\scnitem{методы и инструментальные средства проектирования различного вида компонентов ostis-систем}
		\scnitem{технические средства координации деятельности участников проекта, направленные на постоянное совершенствование Технологии OSTIS}
	\end{scnrelfromset}
	\begin{scnrelfromset}{декомпозиция ostis-системы}
		\scnitem{SC-модель Метасистемы OSTIS}
		\begin{scnindent}
			\begin{scnrelfromset}{декомпозиция sc-модели ostis-системы}
				\scnitem{База знаний Метасистемы OSTIS}
				\scnitem{Решатель задач Метасистемы OSTIS}
				\scnitem{Пользовательский интерфейс Метасистемы OSTIS}
			\end{scnrelfromset}
		\end{scnindent}
		\scnitem{Программный вариант реализации ostis-платформы}
	\end{scnrelfromset}
	\begin{scnrelfromlist}{подсистема}
		\scnitem{Библиотека Метасистемы OSTIS}
		\scnitem{Средства разработки компонентов ostis-систем}
		\begin{scnindent}
			\begin{scnrelfromset}{декомпозиция}
				\scnitem{Средства поддержки проектирования баз знаний ostis-систем}
				\scnitem{Средства поддержки проектирования решателей задач ostis-систем}
				\scnitem{Средства поддержки проектирования пользовательских интерфейсов ostis-систем}
			\end{scnrelfromset}
		\end{scnindent}
	\end{scnrelfromlist}
	
	\scnheader{База знаний Метасистемы OSTIS}
	\begin{scnrelfromset}{декомпозиция}
		\scnitem{Стандарт OSTIS}
		\scnitem{Раздел Проект OSTIS. История, текущее состояние и перспективы эволюции и применения Технологии OSTIS}
		\scnitem{Документация Метасистемы OSTIS}
		\scnitem{История и текущие процессы эксплуатации Метасистемы OSTIS}
		\scnitem{Раздел Проект OSTIS. История, текущие процессы и план развития Метасистемы OSTIS}
	\end{scnrelfromset}
	
	\scnheader{Решатель задач Метасистемы OSTIS}
	\scntext{примечание}{Решатель задач собственно \textit{Метасистемы OSTIS}, без учета подсистем, включает в себя набор агентов информационного поиска, реализующих базовые механизмы навигации по базе знаний.}
	\begin{scnrelfromset}{декомпозиция}
		\scnitem{Абстрактный sc-агент поиска всех входящих константных позитивных стационарных sc-дуг принадлежности}
		\scnitem{Абстрактный sc-агент поиска всех идентификаторов заданного sc-элемента}
		\scnitem{Абстрактный sc-агент поиска полной семантической окрестности заданного элемента}
		\scnitem{Абстрактный sc-агент поиска связок декомпозиции для заданного sc-элемента}
		\scnitem{Абстрактный sc-агент поиска всех известных сущностей, являющихся общими по отношению к заданной}
		\scnitem{Абстрактный sc-агент поиска определения или пояснения для заданного объекта}
		\scnitem{Абстрактный sc-агент поиска всех известных сущностей, являющихся частными по отношению к заданной}
		\scnitem{Абстрактный sc-агент поиска всех выходящих константных позитивных стационарных sc-дуг принадлежности с их ролевыми отношениями}
		\scnitem{Абстрактный sc-агент поиска всех выходящих константных позитивных стационарных sc-дуг принадлежности}
		\scnitem{Абстрактный sc-агент поиска всех входящих константных позитивных стационарных sc-дуг принадлежности с их ролевыми отношениями}
	\end{scnrelfromset}
	
	\scnheader{Средства поддержки проектирования баз знаний ostis-систем}
	\scnidtf{Встраиваемая ostis-система комплексной поддержки проектирования баз знаний ostis-систем}
	\scnidtf{Встраиваемая типовая интеллектуальная система комплексной автоматизации проектирования, а также управления процессом коллективного проектирования и совершенствования баз знаний интеллектуальных систем на всех этапах их жизненного цикла}
	\scnidtf{Интеллектуальная система автоматизированного проектирования баз знаний}
	\scnidtf{Встраиваемая интеллектуальная система поддержки проектирования и совершенствования баз знаний интеллектуальных систем на всех этапах их жизненного цикла}
	\scnidtf{Интеллектуальный компьютерный фреймворк баз знаний интеллектуальных систем, разрабатываемых по Технологии OSTIS}
	\scntext{пояснение}{Известно, что разработка базы знаний интеллектуальной системы является весьма трудоемким процессом, во много определяющим качество интеллектуальной системы. Очевидно также, что сокращение сроков разработки базы знаний возможно путем организации коллективной разработки, но при условии решения ряда задач, например:
		\begin{scnitemize}
			\item Как в рамках коллектива разработчиков одной и той же базы знаний предотвратить синдром лебедя, рака и щуки или синдром семи нянек и как снизить накладные расходы на согласование их деятельности по созданию качественной базы знаний.
			\item Как обеспечить возможность включения любых уже формализованных знаний в базу знаний любой интеллектуальной системы (если они там необходимы) без какой-либо ручной корректировки этих знаний и тем самым полностью исключить повторную разработку и адаптацию этих знаний.
	\end{scnitemize}}
	\scntext{назначение}{\textit{Средства поддержки проектирования баз знаний ostis-систем} осуществляют:
		\begin{scnitemize}
			\item мониторинг деятельности каждого участника процесса проектирования баз знаний, что необходимо для защиты его авторских прав, для оценки объема и значимости его вклада в проектную деятельность, для оценки его профессиональной квалификации, для качественного распределения новых проектных работ с учетом его текущей квалификации и планируемого направления ее повышения, для реализации откатов, то есть отмены ошибочных решений, принятых администраторами или менеджерами проектируемой базы знаний;
			\item контроль версий проектируемой базы знаний, реализацию необходимых возвратов к предшествующим версиям;
			\item контроль исполнительской дисциплины;
			\item анализ текущего состояния и динамики процесса проектирования, выявление критических ситуаций;
			\item семантический анализ корректности результатов проектных работ всех участников;
			\item оценку объема и значимости деятельности каждого участника проекта;
			\item оценку текущего состояния и динамики развития квалификационного портрета каждого участника проекта;
			\item формирование рекомендаций по повышению квалификации каждого участника проекта;
			\item контроль качества (непротиворечивости, целостности, полноты, чистоты) текущего состояния проектируемой и совершенствуемой базы знаний.
	\end{scnitemize}}
	\scntext{принципы функционирования}{Каждый участник процесса проектирования базы знаний может выполнять различные виды проектных работ:
		\begin{scnitemize}
			\item предложить новый фрагмент в согласованную часть базы знаний или некоторую корректировку (удаление, изменение) в этой части базы знаний;
			\item высказать согласие или несогласие с предложенной кем-то корректировкой или добавлением в согласованную часть базы знаний;
			\item провести верификацию, тестирование, рецензирование предложенной кем-то корректировки или добавления в согласованную часть базы знаний и написать замечания к доработке этого предложения;\item предложить формулировку нового проектного задания, например, на устранение указываемого противоречия (ошибки), на заполнение указываемой информационной дыры;
			\item высказать конструктивные критические замечания к формулировке нового проектного задания;
			\item предложить исполнителя или группу исполнителей для выполнения пока не исполняемого проектного задания;
			\item высказать конструктивные критические замечания к предложенным исполнителям некоторого свободного проектного задания.
	\end{scnitemize}}
	\begin{scnrelfromset}{декомпозиция}
		\scnitem{База знаний средств поддержки проектирования баз знаний ostis-систем}
		\scnitem{Решатель задач средств поддержки проектирования баз знаний ostis-систем}
		\begin{scnindent}
			\begin{scnrelfromset}{декомпозиция}
				\scnitem{Абстрактный sc-агент верификации баз знаний}
				\scnitem{Абстрактный sc-агент редактирования баз знаний}
				\scnitem{Абстрактный sc-агент автоматизации деятельности разработчиков баз знаний}
				\scnitem{Абстрактный sc-агент автоматизации деятельности администраторов баз знаний}
				\scnitem{Абстрактный sc-агент автоматизации деятельности менеджеров баз знаний}
				\scnitem{Абстрактный sc-агент автоматизации деятельности экспертов базы знаний}
				\scnitem{Абстрактный sc-агент оценки качества базы знаний}
			\end{scnrelfromset}
		\end{scnindent}
		\scnitem{Пользовательский интерфейс средств поддержки проектирования баз знаний ostis-систем}
	\end{scnrelfromset}
	\begin{scnindent}
		\scnrelfrom{источник}{\cite{Davydenko2018}}
	\end{scnindent}
	
	\scnheader{Средства поддержки проектирования решателей задач ostis-систем}
	\begin{scnrelfromset}{декомпозиция}
		\scnitem{Средства поддержки проектирования программ Базового языка программирования ostis-систем}
		\scnitem{Средства поддержки проектирования коллективов внутренних агентов ostis-систем}
		\scnitem{Интеллектуальная среда проектирования искусственных нейронных сетей, семантически совместимых с базами знаний ostis-систем}
	\end{scnrelfromset}
	
	\scnheader{Средства поддержки проектирования коллективов внутренних агентов ostis-систем}
	\begin{scnrelfromset}{декомпозиция}
		\scnitem{База знаний средств поддержки проектирования коллективов внутренних агентов ostis-систем}
		\scnitem{Решатель задач средств поддержки проектирования коллективов внутренних агентов ostis-систем}
		\begin{scnindent}
			\begin{scnrelfromset}{декомпозиция}
				\scnitem{Абстрактный sc-агент верификации sc-агентов}
				\begin{scnindent}
					\begin{scnrelfromset}{декомпозиция}
						\scnitem{Абстрактный sc-агент верификации спецификации sc-агента}
						\scnitem{Абстрактный sc-агент проверки неатомарного sc-агента на непротиворечивость его спецификации спецификациям более частных sc-агентов в его составе}
					\end{scnrelfromset}
				\end{scnindent}
				\scnitem{Абстрактный sc-агент отладки коллективов sc-агентов}
				\begin{scnindent}
					\begin{scnrelfromset}{декомпозиция}
						\scnitem{Абстрактный sc-агент поиска всех выполняющихся процессов, соответствующих заданному sc-агенту}
						\scnitem{Абстрактный sc-агент инициирования заданного sc-агента на заданных аргументах}
						\scnitem{Абстрактный sc-агент активации заданного sc-агента}
						\scnitem{Абстрактный sc-агент деактивации заданного sc-агента}
						\scnitem{Абстрактный sc-агент установки блокировки заданного типа для заданного процесса на заданный sc-элемент}
						\scnitem{Абстрактный sc-агент снятия всех блокировок заданного процесса}
						\scnitem{Абстрактный sc-агент снятия всех блокировок с заданного sc-элемента}
					\end{scnrelfromset}
				\end{scnindent}
			\end{scnrelfromset}
		\end{scnindent}
		\scnitem{Пользовательский интерфейс средств поддержки проектирования коллективов внутренних агентов ostis-систем}
	\end{scnrelfromset}
	\begin{scnindent}
		\scnrelfrom{источник}{\cite{Shunkevich2018}}
	\end{scnindent}
	
	\scnheader{Средства поддержки проектирования программ Базового языка программирования ostis-систем}
	\begin{scnrelfromset}{декомпозиция}
		\scnitem{База знаний средств поддержки проектирования программ Базового языка программирования ostis-систем}
		\begin{scnindent}
			\scntext{пояснение}{\textit{База знаний средств поддержки проектирования программ Базового языка программирования ostis-систем} включает в себя, в частности, типологию некорректностей в scp-программах, способов их выявления и устранения.}
		\end{scnindent}
		\scnitem{Решатель задач средств поддержки проектирования программ Базового языка программирования ostis-систем}
		\begin{scnindent}
			\begin{scnrelfromset}{декомпозиция}
				\scnitem{Абстрактный sc-агент верификации scp-программ}
				\scnitem{Абстрактный sc-агент отладки scp-программ}
				\begin{scnindent}
					\begin{scnrelfromset}{декомпозиция}
						\scnitem{Абстрактный sc-агент запуска заданной scp-программы для заданного множества входных данных}
						\scnitem{Абстрактный sc-агент запуска заданной scp-программы для заданного множества входных данных в режиме пошагового выполнения}
						\scnitem{Абстрактный sc-агент поиска всех scp-операторов в рамках scp-программы}
						\scnitem{Абстрактный sc-агент поиска всех точек останова в рамках scp-процесса}
						\scnitem{Абстрактный sc-агент добавления точки останова в scp-программу}
						\scnitem{Абстрактный sc-агент удаления точки останова из scp-программы}
						\scnitem{Абстрактный sc-агент добавления точки останова в scp-процесс}
						\scnitem{Абстрактный sc-агент удаления точки останова из scp-процесса}
						\scnitem{Абстрактный sc-агент продолжения выполнения scp-процесса на один шаг}
						\scnitem{Абстрактный sc-агент продолжения выполнения scp-процесса до точки останова или завершения}
						\scnitem{Абстрактный sc-агент просмотра информации об scp-процессе}
						\scnitem{Абстрактный sc-агент просмотра информации об scp-операторе}
					\end{scnrelfromset}
				\end{scnindent}
			\end{scnrelfromset}
		\end{scnindent}
		\scnitem{Пользовательский интерфейс средств поддержки проектирования программ Базового языка программирования ostis-систем}
		
		\scnheader{Метасистема OSTIS}
		\scntext{назначение}{\textit{Метасистема OSTIS} является в \textit{Экосистеме OSTIS} ключевой интеллектуальной системой, которая поддерживает не только проектирование новых интеллектуальных систем и не только замену устаревших компонентов в интеллектуальных системах, входящих в состав \textit{Экосистемы OSTIS}, но и включение (интеграцию) в состав \textit{Экосистемы OSTIS} новых создаваемых интеллектуальных систем.}
		\scntext{примечание}{Описываемая \textit{Метасистема OSTIS} является:
		\begin{scnitemize}
			\item{Системой информационной и инструментальной поддержки всех этапов жизненного цикла и.к.с. нового поколения (\textit{ostis-систем}) самого различного назначения.}
			\item{Порталом знаний по \textit{Технологии OSTIS}, обеспечивающим координацию работ по развитию \textit{Технологии OSTIS} и автоматизацию анализа качества \textit{Стандарта OSTIS}.}
		\end{scnitemize}
	То есть \textit{Метасистема OSTIS} является системой управления \textit{Проектом создания и развития Стандарта OSTIS}.}
	\end{scnrelfromset}
	\scntext{примечание}{Важнейшим направлением \textit{Метасистемы OSTIS} и, соответственно, важнейшим направлением применения \textit{Стандарта OSTIS} является использование их в качестве комплексного интегрированного компьютерного учебного пособия по специальности \scnqqi{Искуственный интеллект}. Для этого устанавливается связь между разделами Стандарта OSTIS и программами различных учебных дисциплин указанной специальности. Важно подчеркнуть при этом: \textit{Стандарт OSTIS} содержит достаточно полный сравнительный анализ с различными альтернативными подходами, то есть ни в коем случае не ограничивается рассмотрением только \textit{Технологией OSTIS}.}
	\scntext{примечание}{\textit{Метасистема OSTIS} взаимодействует не только со своими разработчиками и конечными пользователями, но и с другими ostis-системами, которые созданы с помощью \textit{Технологии OSTIS} и представляют собой ее \textit{дочерние системы*}.}
	\scntext{примечание}{\textit{Метасистема OSTIS} для своих дочерних систем может:
	\begin{scnitemize}
		\item{Осуществлять автоматическую сборку \textit{дочерних ostis-систем} стартовых версий по инструкциям. Таким образом, генерировать новые \textit{дочерние ostis-системы}.}
		\item{Включать в \textit{дочерние ostis-системы} новые многократно используемые компоненты из постоянно пополняемой \textit{библиотеки многократно используемых семантически совместимых компонентов ostis-систем}.}
		\item{Заменять в \textit{дочерних ostis-системах} устаревшие версии многократно используемых компонентов на новые версии из \textit{Библиотеки Метасистемы OSTIS}.}
		\item{Включать в \textit{дочерние ostis-системы} подсистему совершенствования своей расширенной базы знаний и, при необходимости, подсистему улучшения ее интегрированной машины обработки знаний и пользовательского интерфейса.}
	\end{scnitemize}
	Таким образом, после появления \textit{дочерней ostis-системы} ее связь с \textit{Метасистемой OSTIS} не прерывается и она становится постоянным участником процесса совершенствования всех \textit{дочерних ostis-систем}.}
	\scntext{примечание}{\textit{Метасистема OSTIS} является одновременно и системой автоматизации проектирования ostis-систем, и интеллектуальной системой, обучающей методам и средствам проектирования \textit{ostis-систем}. Этот факт существенно повышает качество проектирования прикладных \textit{ostis-систем}, расширяет контингент разработчиков ostis-систем и интегрирует проектную (инженерную) деятельность в области искусственного интеллекта с образовательной деятельностью в этой области.}
	\scntext{примечание}{Все опубликованные материалы о \textit{Технологии OSTIS} в формализованном виде входят в \textit{Базу знаний Метасистемы OSTIS}.}
	
		\bigskip
	\end{scnsubstruct}
	\scnsourcecomment{Завершили \scnqqi{Сегмент. Структура, назначение, особенности и достоинства Метасистемы OSTIS}}
\end{SCn}    	
        
        \bigskip
    \end{scnsubstruct}
    \scnendcurrentsectioncomment
\end{SCn}


\scsubsection{Предметная область и онтология семантически совместимых информационно-справочных ostis-систем и интеллектуальных help-систем}
\label{sd_help_semantic_comp_sys}

\scsubsection[
    \protect\scnmonographychapter{Глава 7.2. Экосистема интеллектуальных компьютерных систем нового поколения (Экосистема OSTIS) и реализация рынка знаний на ее основе}
    ]{Предметная область и онтология семантически совместимых интеллектуальных корпоративных ostis-систем различного назначения}
\label{sd_purpos_semantic_comp_sys}
\begin{SCn}
    \scnsectionheader{Предметная область и онтология семантически совместимых интеллектуальных корпоративных ostis-систем различного назначения}
    \begin{scnsubstruct}

        \begin{scnrelfromlist}{библиографическая ссылка}
            \scnitem{\scncite{Ameri2005}}
            \scnitem{\scncite{Gerhard2017}}
        \end{scnrelfromlist}

        \scnheader{Предметная область и онтология семантически совместимых интеллектуальный корпоративных ostis-систем различного назначения}
        \scniselement{предметная область}
        \begin{scnhaselementrole}{максимальный класс объектов исследования}
            {интеллектуальная комната данных}
        \end{scnhaselementrole}
        \begin{scnhaselementrolelist}{класс объектов исследования}
            \scnitem{корпоративная система}
            \scnitem{корпоративная ostis-система}
        \end{scnhaselementrolelist}

        \scnheader{интеллектуальная комната данных}
        \scnidtf{Intelligent Data Room}
        \scnidtf{IDR}
        \scnidtf{система, которая позволяет отслеживать, анализировать и постепенно автоматизировать все процессы обработки данных в компании}
        \begin{scnrelfromset}{принципы работы}
            \scnfileitem{Интеллектуальные подсистемы (агенты) упорядочивают структуру ваших данных таким образом, что актуальная информация всегда доступна, а устаревшая информация автоматически архивируется или удаляется в соответствии с законами о хранении и защите данных.}
            \scnfileitem{Запросы к системе выполняются в виде простых инструкций, система помогает менеджерам вводить необходимую информацию, осуществляет частичную или полную автоматизацию обновления информации из баз данных, доступных через Интернет.}
            \scnfileitem{Искусственный интеллект (ИИ) выполняет структуризацию и классификацию документов и информации для принятия быстрых и правильных решений, автоматически обрабатывает документы и доступные базы данных для отбора ключевой информации, необходимой в данный момент и в будущем.}
            \scnfileitem{Существующее системное окружение на предприятии может быть легко подключено к искусственному интеллекту через открытые интерфейсы, вся информация остается доступной. Все ключевые системы данных синхронизируются с искусственным интеллектом, данные постоянно сравниваются друг с другом, чтобы избежать потерь.}
            \scnfileitem{Вся информация доступна в базе знаний, которая является источником данных для рабочих процессов, отчетности и комплексных проверок.}
            \scnfileitem{Таким образом, предлагаемая платформа на основе искусственного интеллекта позволяет представить всю компанию единым целостным образом.}
        \end{scnrelfromset}
        \begin{scnrelfromset}{достоинства внедрения}
            \scnfileitem{\textit{IDR} помогает собирать и оценивать информацию без преднамеренных искажений или ошибок, связанных с человеческим фактором.}
            \scnfileitem{Компания с \textit{IDR} полностью контролирует свои данные.}
            \scnfileitem{Система предоставляет только высококачественные, достоверные и актуальные данные.}
            \scnfileitem{Цифровое представление всех процессов компании обеспечивает интегрированную обработку информации внутри компании, что дает полную прозрачность управления, облегчает доступ ко всей информации и ее анализ.}
            \scnfileitem{Благодаря поддержке подсистем искусственного интеллекта все необходимые данные из документов, процессов и внешних источников могут быть извлечены, структурированы и грамотно оценены.}
            \scnfileitem{Искусственный интеллект предоставляет инструмент для интеллектуальной оцифровки и интеграции знаний вашей компании и взаимодействия между всеми заинтересованными сторонами в рамках вашей компании, как следствие --- обеспечивает автоматическую поддержку соответствующих бизнес-процессов и устраняет локальные изолированные решения внутри компании, превращая ее в единую согласованную систему.}
        \end{scnrelfromset}
        \scntext{примечание}{Идея \textit{интеллектуальной комнаты данных} в общем случае может реализовываться двумя путями. Любой из них может реализовываться постепенно, с подключением к \textit{IDR} все новых и новых информационных ресурсов.}
        \scnsuperset{цифровой сотрудник}
        \scntext{пояснение}{Надстройка над уже существующими информационными ресурсами предприятия (различные базы данных, облачные и физические хранилища документов и т.д.). Для реализации \textit{цифрового сотрудника} необходимо в базе знаний \textit{IDR} в виде семейства онтологий описать метаинформацию об имеющихся информационных ресурсах (схемы баз данных, структуру и расположение документов и т.д.), а также механизмы доступа к этим ресурсам (например, их физическое расположение, языки запросов). На основе такого описания \textit{IDR} сможет автоматически построить необходимый набор запросов к нужным информационным ресурсам, интегрировать полученные ответы и выдать ответ пользователю в удобной ему форме. При этом пользователю системы не нужно знать, где именно и в какой форме хранится нужная ему информация, запрос к системе делается на языке, близком к естественному.}
        \begin{scnrelfromset}{достоинства}
            \scnfileitem{Нет необходимости в рамках базы знаний \textit{IDR} дублировать информацию, которая уже содержится в использовавшихся ранее информационных ресурсах, она может и дальше храниться в тех же местах и в той же форме. При этом данные могут быстро меняться, это никак не повлияет на работу системы.}
            \scnfileitem{Нет необходимости вносить изменения в уже налаженные процессы и используемое на предприятии ПО, резко переобучать людей и менять отлаженные схемы работы.}
            \scnfileitem{В общем случае реализация \textit{цифрового сотрудника } проще и дешевле в реализации и позволяет экспериментальным путем выявить наиболее больные места предприятия, где автоматизация информационных процессов и обеспечение их прозрачности позволит сэкономить наибольшее количество средств и минимизировать число ошибок.}
        \end{scnrelfromset}
        \begin{scnrelfromset}{недостатки}
            \scnfileitem{\textit{Цифровой сотрудник} не решает проблемы, связанные с дублированием информации, представленной в разной форме в разных местах, и только частично решает проблемы, связанные с ошибками при внесении новой или редактированием имеющейся информации в разнородные информационные ресурсы.}
            \scnfileitem{Из-за отсутствия унификации представления информации система \textit{IDR} ограничена в своих возможностях, в частности при верификации информации. Проверить корректность, непротиворечивость, полноту информации намного проще, если вся информация представлена в унифицированном виде (как по форме, так и по смыслу).}
        \end{scnrelfromset}
        \scnsuperset{полноценная комната данных}
        \scnidtf{полный цифровой двойник предприятия}
        \scntext{пояснение}{Реализация \textit{полноценной комнаты данных} предполагает полный перенос в базу знаний \textit{IDR} части или всей информации, хранящейся в электронном виде в информационных ресурсах предприятия. Для этого необходимо описывать в базе знаний как онтологии (системы понятий) предметной области предприятия, так и конкретные экземпляры и связи между ними. При этом очевидно, что если онтологии изменяются относительно редко и могут обновляться вручную, то конкретные экземпляры и их описание должно формироваться автоматически.}
        \begin{scnrelfromset}{достоинства}
            \scnfileitem{Полностью исключается необходимость дублирования информации в разных местах и в разных формах, таким образом снижается объем хранимой информации и минимизируется количество ошибок.}
            \scnfileitem{В онтологиях описывается только смысл информации, нет необходимости описывать существующую структуру информационных ресурсов предприятия и ориентироваться на нее.}
            \scnfileitem{Полностью задействуются возможности \textit{IDR} по верификации информации предприятия на предмет корректности, непротиворечивости и полноты.}
            \scnfileitem{Наращивание функциональных возможностей такой реализации значительно упрощается за счет гибкости подходов, лежащих в основе \textit{IDR}, в частности, подхода к разработке и структуризации базы знаний и многоагентного подхода к обработке информации.}
        \end{scnrelfromset}
        \begin{scnrelfromset}{недостатки}
            \scnfileitem{Требуются изменения в уже налаженных процессах на предприятии, отказ от части используемого в настоящий момент ПО, переобучение сотрудников, обслуживающих и развивающих информационные системы предприятия.}
            \scnfileitem{В общем случае \textit{полноценная комната данных} может оказаться более трудоемкой при первоначальной реализации, поскольку предполагает больший объем работ по формализации информации, обеспечению надежности ее хранения и доступа к ней, эффективности ее редактирования и дополнения.}
        \end{scnrelfromset}
        \scntext{примечание}{Оба варианта реализации интеллектуальной комнаты данных (\textit{полноценная комната данных} и \textit{цифровой сотрудник}) дают возможность описать в базе знаний \textit{IDR} не только информацию, которая уже активно используется на предприятии, но и дополнительные знания, которые могут оказаться полезными, например:
            \begin{itemize}
                \item Различные стандарты и документы, регламентирующие работу предприятия;
                \item Описание структуры предприятия, правил работы, основных контактов;
                \item Учебные материалы и руководства для новых сотрудников;
                \item и многое другое
            \end{itemize}}
    
    \scnheader{корпоративная система}
    \scntext{пояснение}{\textit{корпоративные системы} представляют собой программные решения, предназначенные для автоматизации бизнес-процессов и управления ресурсами и данными внутри организации. Они могут включать в себя различные подсистемы, такие как управление отношениями с клиентами, управление контентом, управление проектами, управление ресурсами предприятия, управление документами и многое другое.}
    \scntext{роль}{Обеспечение эффективного управления бизнес-процессами и ресурсами, повышении производительности и качества работы, а также обеспечении прозрачности и оперативности принятия решений на основе актуальных данных.}
    \begin{scnindent}
        \begin{scnrelfromset}{смотрите}
            \scnitem{\scncite{Ameri2005}}
            \scnitem{\scncite{Gerhard2017}}
        \end{scnrelfromset}
    \end{scnindent}
    \begin{scnrelfromset}{цель}
        \scnfileitem{Автоматизация многих рутинных задач, таких как обработка заказов, управление складом, учет финансовых операций и так далее. Это позволяет сократить время на выполнение задач и уменьшить количество ошибок.}
        \scnfileitem{Сбор, хранение и обработка данных о бизнес-процессах и ресурсах организации. Это позволяет увеличить точность и оперативность принятия решений, а также обеспечить прозрачность в управлении организацией.}
        \scnfileitem{Эффективное управление ресурсами организации, такими как финансы, трудовые ресурсы, материальные и технические ресурсы и так далее. Это позволяет сократить затраты на управление ресурсами и повысить эффективность их использования.}
        \scnfileitem{Управление отношениями с клиентами, автоматизация процессов продаж и обслуживания, а также анализ данных о клиентах. Это позволяет повысить удовлетворенность клиентов и увеличить объемы продаж.}
        \scnfileitem{Управление проектами, планирование и отслеживание выполнения работ, управление ресурсами и расписание проектов. Это позволяет повысить эффективность выполнения проектов, уменьшить сроки выполнения работ и снизить затраты на проекты.}
        \scnfileitem{Управление документами, контроль версиями, автоматизация процессов редактирования и утверждения документов. Это позволяет повысить эффективность работы с документами и обеспечить безопасность их хранения и передачи.}
    \end{scnrelfromset}
    \begin{scnrelfromset}{недостатки}
        \scnfileitem{Внедрение корпоративных систем может быть дорогостоящим и трудоемким процессом, который требует значительных ресурсов и экспертизы. Кроме того, многие системы могут потребовать изменения бизнес-процессов и требовать адаптации культуры организации.}
        \scnfileitem{Корпоративные системы могут столкнуться с проблемами совместимости с другими системами, используемыми в организации. Это может привести к проблемам с обменом данными и снижению эффективности работы.}
        \scnfileitem{Корпоративные системы могут стать мишенью для кибератак, поэтому важно обеспечить безопасность хранения и передачи данных, используемых в системах.}
        \scnfileitem{Корпоративные системы могут потребовать значительных затрат на обслуживание и поддержку, включая установку обновлений, устранение ошибок и техническую поддержку.}
        \scnfileitem{Внедрение новых корпоративных систем может потребовать обучения персонала, что может быть трудоемким и затратным процессом.}
        \scnfileitem{Внедрение корпоративных систем может потребовать изменения бизнес-процессов, что может быть сложным и вызвать сопротивление со стороны сотрудников.}
    \end{scnrelfromset}
    \scntext{примечание}{С точки зрения структуры Экосистемы OSTIS \textit{корпоративная ostis-система} осуществляет координации и эволюцию деятельности некоторых групп ostis-систем и их пользователей.}
    \scntext{примечание}{Для создания семантически совместимых интеллектуальных \textit{корпоративных систем} необходимо обеспечить высокую степень гибкости, масштабируемости, автоматизации и интеграции. Это позволит организациям более эффективно управлять ресурсами и данными и повысить их конкурентоспособность на рынке. Для достижения этих целей необходимо использовать современные технологии, такие как Аналитика данных, Машинное обучение, Искусственный интеллект и технологии распределенных вычислений. Кроме того, необходимо учитывать особенности организации и ее бизнес-процессов, чтобы обеспечить максимальную эффективность использования системы.}
    \scntext{примечание}{\textit{корпоративные ostis-системы} могут быть применены в различных областях: медицина и здравоохранение, образовательная деятельность широкого профиля, страховой бизнес, промышленная деятельность, административная деятельность, недвижимость, транспорт и так далее.}
    \scntext{определение}{\textit{корпоративная ostis-система} --- центральная \textit{ostis-система}, осуществляющая координацию, организацию, а также поддержку эволюции деятельности членов соответствующего \textit{ostis-сообщества}. \textit{корпоративная ostis-система} является представителем соответствующего \textit{ostis-сообщества} в других \textit{ostis-сообществах}, членом которых оно является.}
    \begin{scnindent}
        \scnrelfrom{пример}{SCg-текст. Пример корпоративной ostis-системы ostis-сообщества}
        \begin{scnindent}
            \scnidtf{SCg-текст. Пример корпоративной ostis-системы ostis-сообщества}
        \end{scnindent}
    \end{scnindent}

    \end{scnsubstruct}
    \scnendcurrentsectioncomment
\end{SCn}


\scsubsubsection{Предметная область и онтология организаций}
\label{sd_organiztion}

\scsubsection[
    \protect\scnmonographychapter{Глава 7.2. Экосистема интеллектуальных компьютерных систем нового поколения (Экосистема OSTIS) и реализация рынка знаний на ее основе}
    ]{Предметная область и онтология ostis-систем, являющихся персональными ассистентами пользователей, обеспечивающими организацию эффективного взаимодействия каждого пользователя с другими ostis-системами и пользователями, входящими в состав Экосистемы OSTIS}
\label{sd_assistants}
\begin{SCn}
    \scnsectionheader{Предметная область и онтология ostis-систем, являющихся персональными ассистентами пользователей, обеспечивающими организацию эффективного взаимодействия каждого пользователя с другими ostis-системами и пользователями, входящими в состав Экосистемы OSTIS}
    \begin{scnsubstruct}
        \begin{scnrelfromlist}{библиографическая ссылка}
            \scnitem{\scncite{Meurisch2017}}
            \scnitem{\scncite{Meurisch2020}}
            \scnitem{\scncite{Jeni2022}}
            \scnitem{\scncite{Akbar2022}}
        \end{scnrelfromlist}

        \scnheader{Предметная область ostis-систем, являющихся персональными ассистентами пользователей в рамках Экосистемы OSTIS}
        \scniselement{предметная область}
        \begin{scnhaselementrole}{максимальный класс объектов исследования}
            {персональный ostis-ассистент}
        \end{scnhaselementrole}
        \begin{scnhaselementrolelist}{класс объектов исследования}
            \scnitem{персональный ассистент}
        \end{scnhaselementrolelist}

        \scnheader{персональный ассистент}
        \begin{scnrelfromlist}{пояснение}
            \scnfileitem{Общество должно обеспечивать персональную поддержку каждому человеку, учитывая его индивидуальные особенности, с целью достижения следующих целей:
                \begin{itemize}
                    \item максимального уровня физического здоровья, активности и долголетия;
                    \item максимального уровня физического комфорта, личного пространства и материального благосостояния;
                    \item максимального уровня социального комфорта и защиты прав и свобод.
                \end{itemize}}
            \scnfileitem{Необходимо перейти от оказания услуг в решении различных проблем по инициативе самих лиц, столкнувшихся с этими проблемами, к своевременному обнаружению возможности возникновения этих проблем и к соответствующей профилактике. Это возможно только при наличии четкой системной организации персонального мониторинга.}
            \scnfileitem{Цифровые \textit{персональные ассистенты} --- это программы, основанные на технологиях искусственного интеллекта и машинного обучения, которые помогают пользователям в выполнении повседневных задач, таких как составление расписания, управление контактами, поиск информации, напоминание о важных событиях и так далее.}
            \begin{scnindent}
                \begin{scnrelfromset}{смотрите}
                    \scnitem{\scncite{Meurisch2017}}
                    \scnitem{\scncite{Meurisch2020}}
                    \scnitem{\scncite{Jeni2022}}
                    \scnitem{\scncite{Akbar2022}}
                \end{scnrelfromset}
            \end{scnindent}
            \scnfileitem{\textit{Персональный ассистент} должен учитывать, что роли пользователя в обществе могут меняться, расширяться, также как и его интересы и цели. При этом, все \textit{персональные ассистенты} должны быть семантически совместимыми с целью понимания друг друга, а также обладать способностью самостоятельно взаимодействовать в рамках различных \textit{корпоративных систем}, представляя интересы своих пользователей.}
            \scnfileitem{Пользователь не обязан знать множество сервисов, из которых он должен выбирать подходящий ему функционал. Комплекс семантически совместимых сервисов должен располагаться \scnqq{за кадром}. Следовательно, все используемые информационные ресурсы и сервисы должны быть семантически совместимы. Выбор подходящего для пользователя ресурса или сервиса должен производить его \textit{персональный ассистент}.}
            \scnfileitem{При реализации цифровых \textit{персональных ассистентов} необходимо обеспечить их масштабируемость и адаптивность к потребностям пользователей. Это означает, что система должна быть способна автоматически адаптироваться к изменениям в поведении пользователя, учитывая его предпочтения, особенности работы и другие факторы.}
        \end{scnrelfromlist}
        \begin{scnrelfromlist}{требования}
            \scnfileitem{Персональный мониторинг каждой личности по всем направлениям.}
            \scnfileitem{Диагностика и устранение нежелательных отклонений.}
            \scnfileitem{Оказание своевременной персональной помощи в уточнении направлений дальнейшей эволюции каждой личности.}
        \end{scnrelfromlist}
        \scntext{проблема}{Одной из основных проблем, связанных с реализацией цифровых \textit{персональных ассистентов}, является необходимость точного понимания запросов и задач, поступающих от пользователя. Это может быть вызвано различными факторами, такими как нечеткость и неоднозначность формулировок, использование аббревиатур и сленга, а также многозначность некоторых слов.}
            

        \scnheader{персональный ostis-ассистент}
        \scnidtf{ostis-система, являющаяся персональным ассистентом пользователя в рамках Экосистемы OSTIS}
        \begin{scnrelfromset}{возможности}
            \scnfileitem{Возможность анализа деятельности пользователя и формирования рекомендаций по ее оптимизации.}
            \scnfileitem{Возможность адаптации под настроение пользователя, его личностные качества, общую окружающую обстановку, задачи, которые чаще всего решает пользователь.}
            \scnfileitem{Перманентное обучение самого ассистента в процессе решения новых задач, при этом обучаемость потенциально не ограничена.}
            \scnfileitem{Возможность вести диалог с пользователем на естественном языке, в том числе в речевой форме.}
            \scnfileitem{Возможность отвечать на вопросы различных классов, при этом если системе что-то не понятно, то она сама может задавать встречные вопросы.}
            \scnfileitem{Возможность автономного получения информации от всей окружающей среды, а не только от пользователя (в текстовой или речевой форме). При этом система может как анализировать доступные информационные источники (например, в интернете), так и анализировать окружающий ее физический мир, например, окружающие предметы или внешний вид пользователя.}
        \end{scnrelfromset}
        \begin{scnindent}
        	\scntext{примечание}{При этом система может как анализировать доступные информационные источники (например, в интернете), так и анализировать окружающий ее физический мир, например, окружающие предметы или внешний вид пользователя.}
        \end{scnindent}
        \begin{scnrelfromset}{достоинства}
            \scnfileitem{Пользователю нет необходимости хранить разную информацию в разной форме в разных местах, вся информация хранится в единой базе знаний компактно и без дублирований.}
            \scnfileitem{Благодаря неограниченной обучаемости ассистенты могут потенциально автоматизировать практически любую деятельность, а не только самую рутинную.}
            \scnfileitem{Благодаря базе знаний, ее структуризации и средствам поиска информации в базе знаний пользователь может получить более точную информацию более быстро.}
        \end{scnrelfromset}
        \scnsuperset{персональный ostis-ассистент учебного назначения}
        \begin{scnindent}
        	\scnidtf{персональный ассистент-учитель}
        \end{scnindent}
        \scnsuperset{персональный ostis-ассистент по здоровому образу жизни и здоровому питанию}
        \begin{scnindent}
        	\scnidtf{персональный фитнесс-тренер}
        \end{scnindent}
        \scnsuperset{персональный ostis-ассистент для ухода за пациентом}
        \scnsuperset{персональный секретарь-референт}
        \scntext{примечание}{\textit{Персональные ассистенты} имеют самое различное назначение и могут быть использованы для самых различных категорий пользователей (пациент, юридическое обслуживание, административное обслуживание, покупатель, потребитель различных услуг). \textit{Персональный ostis-ассистент} может использовать знания и данные, хранящиеся в других \textit{ostis-системах}, таких как \textit{корпоративные ostis-системы}, чтобы предоставлять пользователю более полную и актуальную информацию. Это может быть особенно полезно для пользователей, которые работают с большим количеством данных и информации. \textit{Персональный ostis-ассистент} автоматически интегрируется с другими \textit{ostis-системами}, что позволяет ему более эффективно работать с данными и информацией. Он может использовать технологии машинного обучения и искусственного интеллекта для адаптации к поведению пользователя и улучшения его производительности и эффективности. \textit{Персональный ostis-ассистент} может быть создан и настроен с учетом конкретных потребностей организации и ее процессов, что может привести к значительным экономическим и производственным преимуществам.}
        \scntext{примечание}{\textit{Персональные ostis-ассистенты} обладают рядом преимуществ по сравнению с другими реализациями цифровых \textit{персональных ассистентов}, таких как более точное понимание запросов и задач пользователей, доступ к актуальным данным и информации, автоматическая интеграция с другими \textit{ostis-системами} в рамках \textit{Экосистемы OSTIS} и адаптация к потребностям организации и ее процессов.}
        \scntext{примечание}{Каждой персоне, входящей в состав \textit{Экосистемы OSTIS} взаимно однозначно соответствует его личный (персональный) ассистент в виде \textit{персонального ostis-ассистента}.
        Таким образом, количество \textit{персональных ostis-ассистентов}, входящих в состав \textit{Экосистемы OSTIS}, совпадает с числом персон, входящих в состав \textit{Экосистемы OSTIS}.}
        \begin{scnindent}
            \scnrelfrom{пример}{\scnfileimage[20em]{Contents/part_ecosystem/src/images/personal_ostis_assistant_example_ru.png}}
            \begin{scnindent}
                \scnidtf{SCg-текст. Пример персоны и соответствующего ему персонального ostis-ассистента}
            \end{scnindent}
        \end{scnindent}

    \end{scnsubstruct}
    \scnendcurrentsectioncomment
\end{SCn}


\scsubsubsection{Предметная область и онтология персон}
\label{sd_person}

\scsubsection[
    \protect\scneditor{Гулякина Н.А.}
    \protect\scnmonographychapter{Глава 7.4. Интеллектуальные обучающие системы нового поколения}
    ]{Предметная область и онтология семантически совместимых ostis-систем автоматизации образовательной деятельности}
\label{sd_learning}
\begin{SCn}
\bigskip
\scnsectionheader{\currentname}

\scnstartsubstruct

\scnheader{Предметная область и онтология методов и средств реализации целенаправленного и персонифицированного процесса обучения пользователей для каждой ostis-системы, входящей в состав Экосистемы OSTIS}
\scnsdmainclasssingle{***}
\scnsdclass{***}
\scnsdrelation{***}

\scnheader{обучаемость}
\scnidtf{способность системы приобретать новые знания и навыки}
\scnrelfrom{включение}{неограниченная обучаемость}

\scnheader{неограниченная обучаемость}
\scnidtf{степень обучаемости, при которой не
накладывается никаких ограничений на типологию приобретенных знаний и навыков}
\scnexplanation{Говоря другими словами, система,
обладающая неограниченной обучаемостью, при необходимости может с течением времени приобрести любое знание и способность решать любую задачу}
	\scnaddlevel{1}
	\scnnote{Уточним, что это не означает, что одна конкретная система будет уметь решать любую задачу, это означает, что система может приобрести способность решать нужную ей задачу, при этом нет принципиальных ограничений на класс таких задач.}
	\scnaddlevel{-1}
	
\scnheader{интеллектуальная компьютерная система}
\scnidtf{сложная техническая система, разработка и даже использование которой требует высоких профессиональных качеств}
\scnrelfromset{проблемы текущего состояния}{\scnfileitem{недостаточная эффективность использования современных интеллектуальных систем, трудоемкость их внедрения и сопровождения, которые в значительной мере определяются высоким порогом вхождения конечных пользователей в интеллектуальные системы}
;\scnfileitem{пользователь часто не использует значительную часть функций даже традиционных компьютерных
систем просто по той причине, что не знает об их наличии и не имеет простого механизма, позволяющего о них узнать. Для интеллектуальных систем данная проблема стоит еще более остро}
;\scnfileitem{высоки затраты на обучение разработчиков интеллектуальных систем, на их адаптацию под особенности устройства конкретной интеллектуальной системы}}
	\scnaddlevel{1}
	\scnnote{Перечисленные трудности связаны не только с естественной сложностью интеллектуальных компьютерных
систем по сравнению с традиционными компьютерными системами, но с низким уровнем документации
для таких систем, неудобством использования такой
документации, трудоемкостью локализации средств
и области решения той или иной задачи, как для
конечного пользователя, так и для разработчика.}
	\scnaddlevel{-1}
\scnrelfrom{предлагаемый подход}{Подход к решению проблемы обучения конечных пользователей и разработчиков интеллектуальных систем}

\scnheader{Подход к решению проблемы обучения конечных пользователей и разработчиков интеллектуальных систем}
\scnidtf{подход к решению указанных проблем, предполагающий дополнение каждой интеллектуальной системы модулем, представляющим собой интеллектуальную обучающую подсистему}
	\scnaddlevel{1}
	\scnnote{Целью данной подсистемы является обучение конечного пользователя и разработчика основной системы принципам работы с ней, принципам ее функционирования и развития.}
	\scnaddlevel{-1}
\scntext{основная идея}{Независимо от того, для решения каких задач разрабатывается интеллектуальная система, она должна обладать некоторыми функциями обучающей системы, даже если система изначально не является обучающей.}
	\scnaddlevel{1}
	\scnexplanation{Следовательно,
	\begin{scnitemize}
	\item пользователь должен иметь возможность
обучаться как принципам работы с интеллектуальной системой, так и иметь возможность получать
новые знания о той предметной области, для которой создается интеллектуальная система;
	\item разработчик интеллектуальных систем должен иметь возможность обучаться принципам внутреннего устройства системы, принципам ее функционирования,
назначению конкретных компонентов системы, иметь возможность локализовать ту часть системы, в которой он должен разобраться для внесения изменений в функциональные возможности системы.	
	\end{scnitemize}}
	\scnnote{Для реализации данной идеи интеллектуальная система должна содержать не только знания о той предметной области, для которой она разработана, но и:
\begin{scnitemize}
\item знания о самой себе, своей архитектуре, компонентах, функциях, принципах работы и т.д.;
\item знания о пользователе, его опыте, навыках, предпочтениях, интересах;
\item знания о задачах, которые решает сама система в
текущий момент и задачах которые планируются к решению в будущем;
\item знания об актуальных задачах по развитию системы и ее сопровождению. 
\end{scnitemize}}
	\scnaddlevel{-1}
\scnrelfrom{технологическая основа}{модель представления знаний в виде унифицированных семантических сетей с теоретико-множественной интерпретацией}
	\scnaddlevel{1}
	\scnrelfromset{возможности}{
	\scnfileitem{Указанная модель является универсальной, то есть позволяет представлять в виде однородных семантических сетей знания любого рода, в том числе конкретные факты, логические утверждения (аксиомы, теоремы, определения), текстовые и мультимедийные иллюстрации и комментарии, примеры конкретных задач с решениями, в том числе доказательства и т.д.}
	;\scnfileitem{Подобная модель представления знаний позволяет рассматривать базу знаний любой системы как иерархию предметных областей, то есть позволяет произвести семантическую структуризацию предлагаемого учащемуся материала, что существенно облегчает процесс обучения за счет систематизации знаний на основе именно их семантики, а не каких-либо других сторонних факторов. Кроме этого, знания в базе могут делиться на логические разделы, каждый из которых соответствует какому-либо фрагменту излагаемого материала. Представление знаний в виде семантической сети позволяет осуществлять свободную навигацию по любым ассоциативным связям, изучая таким образом материал в той последовательности, какая кажется более логичной для самого обучаемого. С другой стороны, такой подход позволяет указать рекомендуемую последовательность изучения материала. При необходимости структура предметных областей может быть легко перестроена.}
	;\scnfileitem{Модель представления является унифицированной, то есть знания из различных областей представляются в сходном виде, что позволяет говорить не о семействе не связанных между собой обучающих систем по различным предметным областям, а о глобальном смысловом пространстве, объединяющем в себе знания всего семейства разрабатываемых систем. В свою очередь, наличие такого смыслового пространства обеспечивает ряд дополнительных возможностей:
	\begin{scnitemize}
	\item каждая система при необходимости может использовать знания, относящиеся к другим системам, что позволяется задавать не только вопросы, касающиеся конкретной предметной области, но и вопросы, носящие междисциплинарный характер;
	\item в рамках глобального смыслового пространства можно выделить часть знаний, которые имеют отношение ко многим системам из всего комплекса, например базовые знания из области математики, логики и т.д. Концепция глобального смыслового пространства позволяет записывать такие фрагменты знаний только в одной из систем, а затем использовать их во всех остальных, что существенно уменьшает количество дублирований, сокращает сроки разработки систем и снижает накладные расходы.
	\end{scnitemize}}
	;\scnfileitem{Рассматриваемый подход к представлению знаний позволяет унифицировать не только модель представления знаний, но и модели обработки знаний, в том числе модели информационного поиска и решения задач. Данный факт позволяется говорить о возможности реализации универсального набора поисковых операций, а также о реализации универсального решателя задач, позволяющего решать различные задачи в рамках каждой из рассматриваемых предметных областей, что существенно сокращает количество реализуемых операций обработки знаний при сохранении функциональных возможностей каждой системы.}
	;\scnfileitem{Как уже было сказано выше, унифицированная модель представления знаний позволяет не только вносить в каждую систему примеры условий типовых для заданной предметной области задач с решениями, но и говорить об интеллектуальном решателе, который позволит системе генерировать ответы на поставленный вопрос за счет знаний уже имеющихся в базе знаний, например с использованием правил логического вывода.}
	;\scnfileitem{Унифицированное представление знаний позволяет не ограничивать номенклатуру пользовательских запросов только специально выделенными для этого командами, а задавать произвольный запрос системе с использованием универсального языка отображения знаний, что делает перечень возможных запросов зависящим только от количества и разнообразия знаний, внесенных в базу знаний системы.}
	;\scnfileitem{Унификация моделей пользовательских интерфейсов позволяет отображать знания различного рода в унифицированном виде независимо от предметной области, к которой эти знания относятся. Таким образом, все разрабатываемые системы будут обладать пользовательским интерфейсом, построенным по одним и тем же принципам, что позволит существенно сократить срок ознакомления учащегося со всем семейством систем. Данный факт не отрицает возможность и необходимость разработки отдельных компонентов интерфейса, ориентированных на конкретную предметную область, например, редактора геометрических чертежей, виртуальной лаборатории для проведения химических опытов и т.д.}
	;\scnfileitem{Каждый компонент пользовательского интерфейса также является отображением определенного элемента из базы знаний, что позволяет, во-первых, легко менять интерфейс системы даже во время ее работы, а, во-вторых, позволяет пользователю задавать системе вопросы не только касательно предметной области, которой посвящена данная система, но и касательно любого из компонентов интерфейса и других частей системы. Таким образом, пользователю достаточно научиться задавать системе несколько простейших вопросов, чтобы в дальнейшем изучить все тонкости работы системой уже в процессе общения с ней.}
	;\scnfileitem{Предлагаемые модели представления и обработки знаний позволяют физически отделить смысл хранимой информации от вариантов ее внешнего отображения, в частности, от идентификаторов тех или иных сущностей в рамках какого-либо естественного языка. Это дает возможность легко интернационализировать любую из разработанных систем, поскольку для перевода системы на какой-либо другой язык необходимо перевести только фрагменты текстов на естественном языке, явно хранимые в базе знаний, не затрагивая при этом сами семантические связи, то есть смысл представленной информации.}}
	\scnaddlevel{-1}
\scnrelfrom{модель обработки знаний}{модель, основанная на многоагентном подходе}
	\scnaddlevel{1}
	\scnrelfromset{преимущества}{
	\scnfileitem{Работа агентов осуществляется независимо друг от друга, что позволяет легко расширять функционал той или иной системы при необходимости, а также позволяет интегрировать в одной и той системе различные модели информационного поиска, решения задач и т.д.}
	;\scnfileitem{Работа агентов осуществляется параллельно, что существенно улучшает производительность всей системы в целом.}}
	\scnaddlevel{-1}
\scnrelfrom{предлагаемый фундамент}{\scnkeyword{Технология OSTIS}}
	\scnaddlevel{1}
	\scnidtf{открытая семантическая технология проектирования интеллектуальных систем}
	\scnnote{\textit{Технология OSTIS} позволяет интегрировать любые виды знаний и любые модели решения задач}
	\scnrelfromset{преимущества}{
\scnfileitem{В основе технологии лежит \textit{SC-код} -- универсальный и унифицированный язык кодирования информации в графодинамической памяти компьютерных систем. \textit{SC-код} позволяет в унифицированном (одинаковом) виде представить любую информацию, что позволит сделать предлагаемый подход универсальным и подходящим для любого класса интеллектуальных систем}
;\scnfileitem{\textit{Технология OSTIS} и, в частности, \textit{SC-код}, легко интегрируется с любыми современными технологиями, что позволит применить предлагаемый подход для большого числа уже разработанных интеллектуальных систем}
;\scnfileitem{\textit{SC-код} позволяет хранить и описывать в базе знаний ostis-системы любую внешнюю (инородную)
по отношению к \textit{SC-коду} информацию в виде внутренних файлов \textit{ostis-систем}. Таким образом, база знаний обучающей подсистемы может содержать в явном виде фрагменты уже имеющейся документации к системе, представленной в любой форме}
;\scnfileitem{В рамках \textit{Технологии OSTIS} уже разработаны модели баз знаний \textit{ostis-систем}, решателей задач \textit{ostis-систем} и пользовательских интерфейсов \textit{ostis-систем}, предполагающие полное их описание в базе знаний системы. Таким образом для \textit{ostis-систем} предлагаемый подход реализуется значительно проще и дает дополнительные преимущества, подробнее рассмотренные
в работе}
;\scnfileitem{одним из основных принципов \textit{Технологии OSTIS} является обеспечение гибкости (модифицируемости) систем, разрабатываемых на ее основе. Таким образом, использование \textit{Технологии OSTIS} обеспечит возможность эволюции самой интеллектуальной обучающей подсистемы}}
	\scnaddlevel{-1}

\scnheader{интеллектуальная обучающая система}
\scnsuperset{интеллектуальная система}
\scnnote{Такого рода системы по сравнению с традиционными системами электронного обучения (например, электронными учебниками) предоставляют обладают рядом существенных преимуществ.}
\scnnote{В случае реализации \textit{интеллектуальной обучающей системы} на основе \textit{Технологии OSTIS}, появляются дополнительные возможности, к числу которых можно отнести следующие:
	\begin{scnitemize}
	\item пользователю в явном виде представляется семантическая структура изучаемого учебного материала и изучаемой предметной области. При этом
обеспечивается наглядная визуализация любого
уровня указанной семантической структуры;
	\item пользователю становятся доступны достаточно полные сведения об изучаемой предметной области, отражены все ее аспекты, благодаря явному
помещению в базу знаний всех предметных закономерностей и взаимосвязей понятий;
	\item помимо возможности чтения текстов и иллюстративных материалов учебника предоставляется возможность навигации по семантическому пространству предметной области;
	\item пользователю предоставляется возможность задавать системе любые вопросы и задачи по изучаемой предметной области;
	\item пользователю предоставляется возможность под контролем системы тренироваться (приобретать практические навыки) в решении самых различных
задач по изучаемой предметной области.
 При этом система:
	\begin{scnitemizeii}
	\item осуществляет семантический анализ правильности решения задач как по свободно конструируемым ответам (результатам), так и по протоколам решения;
	\item локализует допущенные пользователем ошибки
в решении задач, определяет их причину и выдает соответствующие рекомендации пользователю.
	\end{scnitemizeii}
	\item при общении с системой пользователю предоставляется свобода в выборе любого из множества синонимичных терминов (идентификаторов), зарегистрированных в базе знаний системы;
	\item появляется принципиальная возможность реализации естественно-языкового интерфейса с пользователем (благодаря широким возможностям семантического анализа пользовательских сообщений и возможностям синтеза на семантическом уровне сообщений, адресуемых пользователям);
	\item пользователю предоставляется полная свобода в выборе последовательности изучения учебного материала (маршрута навигации по учебному материалу), в выборе решаемых им задач (в сборнике задач и лабораторных работ), но соответствующие рекомендации выдаются.
	\end{scnitemize}}
	\scnaddlevel{1}
	\scnnote{Часть из перечисленных возможностей (а в предельном случае и все их них) могут быть реализованы в рамках подсистемы обучения пользователей интеллектуальной системы.}
	\scnaddlevel{-1}


\scnheader{подсистема обучения пользователей интеллектуальных систем}
\scnexplanation{Для реализации взаимодействия \textit{подсистемы обучения пользователей интеллектуальных систем}, реализуемой на основе \textit{Технологии OSTIS} с основной \textit{интеллектуальной системой} предполагается разработка интерфейсного компонента, который также является частью подсистемы. Важно отметить, что для разных \textit{интеллектуальных систем} такие компоненты будут в значительной степени пересекаться, что, в свою очередь, позволит снизить затраты на интеграцию подсистемы обучения и основной \textit{интеллектуальной системы}}
\scnrelfrom{иллюстрация}{\scnfileimage{\includegraphics[width=0.7\linewidth]{figures/sd_learning/system_arch.png}}}

\scnnote{В случае, если рассматриваемая интеллектуальная система является \textit{ostis-системой}, ее интеграция с подсистемой обучения пользователей интеллектуальных систем осуществляется более глубоко и архитектуру полученной интегрированной системы можно изобразить следующим образом (см. Рис.\textit{ Архитектура подсистемы обучения пользователей интеллектуальных систем в составе другой ostis-системы}). Как видно из рисунка, компоненты \textit{подсистемы обучения пользователей интеллектуальных систем} просто дополняют уже существующие в основной ostis-системе компоненты, что позволяет максимально снизить затраты на интеграцию \textit{подсистемы обучения пользователей интеллектуальных систем} и основной \textit{ostis-системы}.}

\scnheader{Рис. Архитектура подсистемы обучения пользователей интеллектуальных систем в составе другой ostis-системы}
\scneqfile{\\\includegraphics[width=0.5\linewidth]{figures/sd_learning/subsystem_arch.png}\\}

\scnheader{Подход к разработке баз знаний в подсистеме обучения пользователей интеллектуальных систем}

\scnrelfromlist{пример}{
\scnfileitem{\includegraphics[width=0.3\linewidth]{figures/sd_learning/example_1.png}}
\scnaddlevel{1}
\scnnote{В данном примере показано, как используя средства структуризации баз знаний, разработанные в рамках Технологии OSTIS, можно описать в базе знаний различные виды информации об одной и той же сущности, в частности, текущую занятость и профессиональные навыки пользователя. Аналогичным образом можно описать любую другую информацию о пользователе.}
\scnaddlevel{-1};
\scnitem{
\scnaddlevel{1}
\textbf{\textit{семантическая модель базы знаний}}\\
\scnrelfromset{абстрактная базовая декомпозиция}{
история и текущие процессы эксплуатации компьютерной системы\\
	\scnrelfromset{абстрактная базовая декомпозиция}{
	\scnitem{история эксплуатации компьютерной системы}
	;\scnitem{текущие процессы эксплуатации компьютерной системы}}\\
;документация компьютерной системы
;контекст предметной части базы знаний в рамках Глобальной базы знаний
;предметная часть базы знаний
;история, текущие процессы и план развития компьютерной системы\\
	\scnrelfromset{абстрактная базовая декомпозиция}{
	\scnitem{текущие процессы развития компьютерной системы}
	;\scnitem{история развития компьютерной системы}
	;\scnitem{структура и организация проекта компьютерной системы}
	;\scnitem{план развития компьютерной системы}}}}
\scnnote{База знаний ostis-системы может быть структурирована по различным признакам. В данном примере наибольший интерес представляет структуризация базы знаний с точки зрения процесса ее разработки.}
\scnaddlevel{-1};
\scnfileitem{\includegraphics[width=0.3\linewidth]{figures/sd_learning/example_2.png}}
\scnaddlevel{1}
\scnnote{Выше показан пример описания информации об исполнителях некоторого проекта, выполняющих в нем различные роли. С точки зрения структуры базы знаний эта информация является частью раздела структура и организация проекта компьютерной системы.}
\scnaddlevel{-1};\scnfileitem{\includegraphics[width=0.6\linewidth]{figures/sd_learning/example_3.png}}
\scnaddlevel{1}
\scnnote{Выше показан пример описания проектных задач и их исполнителей с учетом квалификации каждого исполнителя. С точки зрения структуры базы знаний эта информация является частью раздела текущие процессы развития компьютерной системы.}
\scnaddlevel{-1};
\scnitem{
\scnaddlevel{1}
\textbf{\textit{Решатель задач системы контроля качества нанесения маркировки}}\\
\scnrelfromset{декомпозиция абстрактного sc-агента}{
Атомарный абстрактный sc-агент распознавания маркировки на основе нейронной сети
;Неатомарный абстрактный sc-агент принятия решений\\
	\scnrelfromset{декомпозиция абстрактного sc-агента}{
	Атомарный абстрактный sc-агент, реализующий концепцию пакета программ\\
	;Неатомарный абстрактный sc-агент достоверного вывода
	;Неатомарный абстрактный sc-агент правдоподобного вывода}\\
;Неатомарный абстрактный sc-агент ассоциативного поиска\\
;Неатомарный абстрактный sc-агент интерпретации программ управления роботизированной установкой\\
	\scnrelfromset{декомпозиция абстрактного sc-агента}{
	Атомарный абстрактный sc-агент интерпретации действия перемещения\\
	;Атомарный абстрактный sc-агент интерпретации действия захвата}}
\scnheader{Неатомарный абстрактный sc-агент достоверного вывода}
\scnrelfromset{декомпозиция абстрактного sc-агента}{
Атомарный абстрактный sc-агент, реализующий стратегию правдоподобного вывода\\
;Неатомарный абстрактный sc-агент интерпретации логических правил}
\scnheader{Неатомарный абстрактный sc-агент правдоподобного вывода}
\scnrelfromset{декомпозиция абстрактного sc-агента}{
Атомарный абстрактный sc-агент, реализующий стратегию правдоподобного вывода\\
;Неатомарный абстрактный sc-агент интерпретации логических правил}
\scnheader{Неатомарный абстрактный sc-агент интерпретации логических правил}
\scnrelfromset{декомпозиция абстрактного sc-агента}{
Атомарный абстрактный sc-агент применения импликативных правил\\
;Атомарный абстрактный sc-агент применения правил об эквиваленции}
}
\scnnote{Согласно предложенному в рамках Технологии OSTIS подходу к разработке решателей задач основу решателя составляет иерархическая система агентов над семантической памятью (sc-агентов). Структура решателя также может быть описана в базе знаний ostis-системе. В данном примере представлена структура решателя задач системы контроля качества нанесения маркировки для предприятия рецептурного производства}
\scnaddlevel{-1}}
\scnendstruct \scnendcurrentsectioncomment
\end{SCn}


\scsubsubsection[
    \protect\scnmonographychapter{Глава 7.4. Интеллектуальные обучающие системы нового поколения}
    ]{Предметная область и онтология дидактических знаний}
\label{sd_didactic}

\scsubsection{Предметная область и онтология семантически совместимых ostis-систем автоматизации проектирования и управления проектированием различных объектов}
\label{sd_management_semantic_comp_sys}

\scsubsection[
    \protect\scnmonographychapter{Глава 7.6. Умное предприятие и интеллектуальные компьютерные системы нового поколения. Опыт автоматизации предприятия \scnqqi{Савушкин продукт}}
    ]{Предметная область и онтология семантически совместимых ostis-систем автоматизации производственной деятельности}
\label{sd_activity_semantic_comp_sys}

\scsubsubsection[
    \protect\scneditors{Крощенко А.А.;Иванюк Д.С.;Пупена А.Н.;Зотов Н.В.;Орлов М.К.}
    \protect\scnmonographychapter{Глава 7.6. Умное предприятие и интеллектуальные компьютерные системы нового поколения. Опыт автоматизации предприятия \scnqqi{Савушкин продукт}}
    ]{Предметная область и онтология семантически совместимых ostis-систем управления рецептурным производством}
\label{sd_ecosys_enterprise}
\begin{SCn}

\scnsectionheader{\currentname}
\scnstartsubstruct

\scniselement{раздел базы знаний}

\scnheader{Предметная область семантически совместимых ostis-систем управления рецептурным производством}
\scnsdmainclasssingle{***}
\scnsdclass{***}
\scnsdrelation{***}

\bigskip

\scnheader{ostis-система управления рецептурным производством}

\scnendstruct \scnendcurrentsectioncomment

\end{SCn}

\scsubsection[
    \protect\scneditor{Самодумкин С.А.}
    \protect\scnmonographychapter{Глава 7.8. Интеллектуальные геоинформационные системы нового поколения}
    ]{Предметная область и онтология геоинформационных ostis-систем}
\label{sd_geosystems}

\scsubsubsection[
    \protect\scnmonographychapter{Глава 7.8. Интеллектуальные геоинформационные системы нового поколения}
    ]{Предметная область и онтология географических объектов}
\label{sd_geograph_obj}

\scsubsection[
    \protect\scnmonographychapter{Глава 7.9. Информационная безопасность интеллектуальных компьютерных систем нового поколения}
    ]{Предметная область и онтология средств обеспечения информационной безопасности ostis-систем в рамках Экосистемы OSTIS}
\label{sd_inf_security}

%
\scchapter{Библиография к описанию Технологии OSTIS}

\begin{SCn}

\scnsectionheader{\currentname}

\scnstartstruct

\scnendstruct

\end{SCn}


% \backmatter%%%%%%%%%%%%%%%%%%%%%%%%%%%%%%%%%%%%%%%%%%%%%%%%%%%%%%%
%\printindex

%%%%%%%%%%%%%%%%%%%%%%%%%%%%%%%%%%%%%%%%%%%%%%%%%%%%%%%%%%%%%%%%%%%%%%

\end{document}